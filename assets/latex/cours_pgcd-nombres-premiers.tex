\documentclass[a4paper]{article}

%================================================================================================================================
%
% Packages
%
%================================================================================================================================

\usepackage[T1]{fontenc} 	% pour caractères accentués
\usepackage[utf8]{inputenc}  % encodage utf8
\usepackage[french]{babel}	% langue : français
\usepackage{fourier}			% caractères plus lisibles
\usepackage[dvipsnames]{xcolor} % couleurs
\usepackage{fancyhdr}		% réglage header footer
\usepackage{needspace}		% empêcher sauts de page mal placés
\usepackage{graphicx}		% pour inclure des graphiques
\usepackage{enumitem,cprotect}		% personnalise les listes d'items (nécessaire pour ol, al ...)
\usepackage{hyperref}		% Liens hypertexte
\usepackage{pstricks,pst-all,pst-node,pstricks-add,pst-math,pst-plot,pst-tree,pst-eucl} % pstricks
\usepackage[a4paper,includeheadfoot,top=2cm,left=3cm, bottom=2cm,right=3cm]{geometry} % marges etc.
\usepackage{comment}			% commentaires multilignes
\usepackage{amsmath,environ} % maths (matrices, etc.)
\usepackage{amssymb,makeidx}
\usepackage{bm}				% bold maths
\usepackage{tabularx}		% tableaux
\usepackage{colortbl}		% tableaux en couleur
\usepackage{fontawesome}		% Fontawesome
\usepackage{environ}			% environment with command
\usepackage{fp}				% calculs pour ps-tricks
\usepackage{multido}			% pour ps tricks
\usepackage[np]{numprint}	% formattage nombre
\usepackage{tikz,tkz-tab} 			% package principal TikZ
\usepackage{pgfplots}   % axes
\usepackage{mathrsfs}    % cursives
\usepackage{calc}			% calcul taille boites
\usepackage[scaled=0.875]{helvet} % font sans serif
\usepackage{svg} % svg
\usepackage{scrextend} % local margin
\usepackage{scratch} %scratch
\usepackage{multicol} % colonnes
%\usepackage{infix-RPN,pst-func} % formule en notation polanaise inversée
\usepackage{listings}

%================================================================================================================================
%
% Réglages de base
%
%================================================================================================================================

\lstset{
language=Python,   % R code
literate=
{á}{{\'a}}1
{à}{{\`a}}1
{ã}{{\~a}}1
{é}{{\'e}}1
{è}{{\`e}}1
{ê}{{\^e}}1
{í}{{\'i}}1
{ó}{{\'o}}1
{õ}{{\~o}}1
{ú}{{\'u}}1
{ü}{{\"u}}1
{ç}{{\c{c}}}1
{~}{{ }}1
}


\definecolor{codegreen}{rgb}{0,0.6,0}
\definecolor{codegray}{rgb}{0.5,0.5,0.5}
\definecolor{codepurple}{rgb}{0.58,0,0.82}
\definecolor{backcolour}{rgb}{0.95,0.95,0.92}

\lstdefinestyle{mystyle}{
    backgroundcolor=\color{backcolour},   
    commentstyle=\color{codegreen},
    keywordstyle=\color{magenta},
    numberstyle=\tiny\color{codegray},
    stringstyle=\color{codepurple},
    basicstyle=\ttfamily\footnotesize,
    breakatwhitespace=false,         
    breaklines=true,                 
    captionpos=b,                    
    keepspaces=true,                 
    numbers=left,                    
xleftmargin=2em,
framexleftmargin=2em,            
    showspaces=false,                
    showstringspaces=false,
    showtabs=false,                  
    tabsize=2,
    upquote=true
}

\lstset{style=mystyle}


\lstset{style=mystyle}
\newcommand{\imgdir}{C:/laragon/www/newmc/assets/imgsvg/}
\newcommand{\imgsvgdir}{C:/laragon/www/newmc/assets/imgsvg/}

\definecolor{mcgris}{RGB}{220, 220, 220}% ancien~; pour compatibilité
\definecolor{mcbleu}{RGB}{52, 152, 219}
\definecolor{mcvert}{RGB}{125, 194, 70}
\definecolor{mcmauve}{RGB}{154, 0, 215}
\definecolor{mcorange}{RGB}{255, 96, 0}
\definecolor{mcturquoise}{RGB}{0, 153, 153}
\definecolor{mcrouge}{RGB}{255, 0, 0}
\definecolor{mclightvert}{RGB}{205, 234, 190}

\definecolor{gris}{RGB}{220, 220, 220}
\definecolor{bleu}{RGB}{52, 152, 219}
\definecolor{vert}{RGB}{125, 194, 70}
\definecolor{mauve}{RGB}{154, 0, 215}
\definecolor{orange}{RGB}{255, 96, 0}
\definecolor{turquoise}{RGB}{0, 153, 153}
\definecolor{rouge}{RGB}{255, 0, 0}
\definecolor{lightvert}{RGB}{205, 234, 190}
\setitemize[0]{label=\color{lightvert}  $\bullet$}

\pagestyle{fancy}
\renewcommand{\headrulewidth}{0.2pt}
\fancyhead[L]{maths-cours.fr}
\fancyhead[R]{\thepage}
\renewcommand{\footrulewidth}{0.2pt}
\fancyfoot[C]{}

\newcolumntype{C}{>{\centering\arraybackslash}X}
\newcolumntype{s}{>{\hsize=.35\hsize\arraybackslash}X}

\setlength{\parindent}{0pt}		 
\setlength{\parskip}{3mm}
\setlength{\headheight}{1cm}

\def\ebook{ebook}
\def\book{book}
\def\web{web}
\def\type{web}

\newcommand{\vect}[1]{\overrightarrow{\,\mathstrut#1\,}}

\def\Oij{$\left(\text{O}~;~\vect{\imath},~\vect{\jmath}\right)$}
\def\Oijk{$\left(\text{O}~;~\vect{\imath},~\vect{\jmath},~\vect{k}\right)$}
\def\Ouv{$\left(\text{O}~;~\vect{u},~\vect{v}\right)$}

\hypersetup{breaklinks=true, colorlinks = true, linkcolor = OliveGreen, urlcolor = OliveGreen, citecolor = OliveGreen, pdfauthor={Didier BONNEL - https://www.maths-cours.fr} } % supprime les bordures autour des liens

\renewcommand{\arg}[0]{\text{arg}}

\everymath{\displaystyle}

%================================================================================================================================
%
% Macros - Commandes
%
%================================================================================================================================

\newcommand\meta[2]{    			% Utilisé pour créer le post HTML.
	\def\titre{titre}
	\def\url{url}
	\def\arg{#1}
	\ifx\titre\arg
		\newcommand\maintitle{#2}
		\fancyhead[L]{#2}
		{\Large\sffamily \MakeUppercase{#2}}
		\vspace{1mm}\textcolor{mcvert}{\hrule}
	\fi 
	\ifx\url\arg
		\fancyfoot[L]{\href{https://www.maths-cours.fr#2}{\black \footnotesize{https://www.maths-cours.fr#2}}}
	\fi 
}


\newcommand\TitreC[1]{    		% Titre centré
     \needspace{3\baselineskip}
     \begin{center}\textbf{#1}\end{center}
}

\newcommand\newpar{    		% paragraphe
     \par
}

\newcommand\nosp {    		% commande vide (pas d'espace)
}
\newcommand{\id}[1]{} %ignore

\newcommand\boite[2]{				% Boite simple sans titre
	\vspace{5mm}
	\setlength{\fboxrule}{0.2mm}
	\setlength{\fboxsep}{5mm}	
	\fcolorbox{#1}{#1!3}{\makebox[\linewidth-2\fboxrule-2\fboxsep]{
  		\begin{minipage}[t]{\linewidth-2\fboxrule-4\fboxsep}\setlength{\parskip}{3mm}
  			 #2
  		\end{minipage}
	}}
	\vspace{5mm}
}

\newcommand\CBox[4]{				% Boites
	\vspace{5mm}
	\setlength{\fboxrule}{0.2mm}
	\setlength{\fboxsep}{5mm}
	
	\fcolorbox{#1}{#1!3}{\makebox[\linewidth-2\fboxrule-2\fboxsep]{
		\begin{minipage}[t]{1cm}\setlength{\parskip}{3mm}
	  		\textcolor{#1}{\LARGE{#2}}    
 	 	\end{minipage}  
  		\begin{minipage}[t]{\linewidth-2\fboxrule-4\fboxsep}\setlength{\parskip}{3mm}
			\raisebox{1.2mm}{\normalsize\sffamily{\textcolor{#1}{#3}}}						
  			 #4
  		\end{minipage}
	}}
	\vspace{5mm}
}

\newcommand\cadre[3]{				% Boites convertible html
	\par
	\vspace{2mm}
	\setlength{\fboxrule}{0.1mm}
	\setlength{\fboxsep}{5mm}
	\fcolorbox{#1}{white}{\makebox[\linewidth-2\fboxrule-2\fboxsep]{
  		\begin{minipage}[t]{\linewidth-2\fboxrule-4\fboxsep}\setlength{\parskip}{3mm}
			\raisebox{-2.5mm}{\sffamily \small{\textcolor{#1}{\MakeUppercase{#2}}}}		
			\par		
  			 #3
 	 		\end{minipage}
	}}
		\vspace{2mm}
	\par
}

\newcommand\bloc[3]{				% Boites convertible html sans bordure
     \needspace{2\baselineskip}
     {\sffamily \small{\textcolor{#1}{\MakeUppercase{#2}}}}    
		\par		
  			 #3
		\par
}

\newcommand\CHelp[1]{
     \CBox{Plum}{\faInfoCircle}{À RETENIR}{#1}
}

\newcommand\CUp[1]{
     \CBox{NavyBlue}{\faThumbsOUp}{EN PRATIQUE}{#1}
}

\newcommand\CInfo[1]{
     \CBox{Sepia}{\faArrowCircleRight}{REMARQUE}{#1}
}

\newcommand\CRedac[1]{
     \CBox{PineGreen}{\faEdit}{BIEN R\'EDIGER}{#1}
}

\newcommand\CError[1]{
     \CBox{Red}{\faExclamationTriangle}{ATTENTION}{#1}
}

\newcommand\TitreExo[2]{
\needspace{4\baselineskip}
 {\sffamily\large EXERCICE #1\ (\emph{#2 points})}
\vspace{5mm}
}

\newcommand\img[2]{
          \includegraphics[width=#2\paperwidth]{\imgdir#1}
}

\newcommand\imgsvg[2]{
       \begin{center}   \includegraphics[width=#2\paperwidth]{\imgsvgdir#1} \end{center}
}


\newcommand\Lien[2]{
     \href{#1}{#2 \tiny \faExternalLink}
}
\newcommand\mcLien[2]{
     \href{https~://www.maths-cours.fr/#1}{#2 \tiny \faExternalLink}
}

\newcommand{\euro}{\eurologo{}}

%================================================================================================================================
%
% Macros - Environement
%
%================================================================================================================================

\newenvironment{tex}{ %
}
{%
}

\newenvironment{indente}{ %
	\setlength\parindent{10mm}
}

{
	\setlength\parindent{0mm}
}

\newenvironment{corrige}{%
     \needspace{3\baselineskip}
     \medskip
     \textbf{\textsc{Corrigé}}
     \medskip
}
{
}

\newenvironment{extern}{%
     \begin{center}
     }
     {
     \end{center}
}

\NewEnviron{code}{%
	\par
     \boite{gray}{\texttt{%
     \BODY
     }}
     \par
}

\newenvironment{vbloc}{% boite sans cadre empeche saut de page
     \begin{minipage}[t]{\linewidth}
     }
     {
     \end{minipage}
}
\NewEnviron{h2}{%
    \needspace{3\baselineskip}
    \vspace{0.6cm}
	\noindent \MakeUppercase{\sffamily \large \BODY}
	\vspace{1mm}\textcolor{mcgris}{\hrule}\vspace{0.4cm}
	\par
}{}

\NewEnviron{h3}{%
    \needspace{3\baselineskip}
	\vspace{5mm}
	\textsc{\BODY}
	\par
}

\NewEnviron{margeneg}{ %
\begin{addmargin}[-1cm]{0cm}
\BODY
\end{addmargin}
}

\NewEnviron{html}{%
}

\begin{document}
\meta{url}{/cours/pgcd-nombres-premiers/}
\meta{pid}{567}
\meta{titre}{PGCD et nombres premiers (Spécialité)}
\meta{type}{cours}
\begin{h2}1. PGCD\end{h2}
\cadre{bleu}{Définition}{% id="d10"
     Soient $a$ et $b$ deux entiers naturels tels que $a\neq 0$ ou $b\neq 0$.
     \par
     Le PGCD de $a$ et de $b$ est le plus grand diviseur commun à $a$ et à $b$.
}
\bloc{orange}{Exemple}{% id="e10"
     On cherche le PGCD de 60 et de 45.
     \par
     Les diviseurs de 60 sont : 1; 2; 3; 4; 5; 6; 10; 12; 15; 20; 30 et 60.
     \par
     Les diviseurs de 45 sont : 1; 3;  5; 9; 15 et 45.
     \par
     Les diviseurs communs à 60 et 45 sont : 1; 3; 5 et 15.
     \par
     Donc le PGCD de 60 et 45 est 15.
}
\bloc{cyan}{Remarques}{% id="r10"
     \begin{itemize}
          \item Si $b$ divise $a$, PGCD($a ; b$) $= b$. En effet, $b$ divise alors $a$ et $b$, et $b$ est le plus grand diviseur de $b$.
          \par
          En particulier, PGCD($a ; 1$) $= 1$ et PGCD(0 ; $b$) $= b$
          \item On prolonge la notion de PGCD à des entiers \textbf{relatifs} $a$ et $b$ par PGCD($a ; b$)=PGCD($|a| ; |b|$).
     \end{itemize}
}
\cadre{rouge}{Théorème}{% id="t20"
     Soient $a$ et  $b$ deux entiers naturels non nuls et $r$ le reste de la division euclidienne de $a$ par $b$.
     \par
     Alors : PGCD($a ; b$) = PGCD($b ; r$).
}
\bloc{orange}{Exemple}{% id="e20"
     Le reste de la division euclidienne de 60 par 45 est 15. donc PGCD(60 ; 45) = PGCD(45 ; 15).
     \par
     Si l'on réitère le processus, le reste de la division euclidienne de 45 par 15 est 0 donc PGCD(45 ; 15) = PGCD(15 ; 0) = 15.
}
\cadre{rouge}{Algorithme d'Euclide}{% id="t30"
     \begin{itemize}
          \item On effectue la division euclidienne de $a$ par $b$ et on note $r_{1}$ le reste de cette division.
          \item Puis si $r_{1}\neq 0$, on effectue la division euclidienne de $b$ par $r_{1}$ et on note $r_{2}$ le reste de cette division.
          \item Puis si $r_{2}\neq 0$, on effectue la division euclidienne de $r_{1}$ par $r_{2}$, et ainsi de suite...
     \end{itemize}
     La suite $r_{0}=b$, $r_{1}$, $r_{2}$, ... est strictement décroissante, et pour un certain rang $n$ on aura $r_{n}=0$.
     \par
     Par conséquent :
     \par
     PGCD($a$ ; $b$) = PGCD($b$ ; $r_{0}$) = PGCD($r_{0}$ ; $r_{1}$) = ...
     \par
     $                   $= PGCD($r_{n-1}$ ; $r_{n}$) = PGCD($r_{n-1}$ ; 0) = $r_{n-1}$
     \par
     Le PGCD de $a$ et $b$ est donc \textbf{le dernier reste non nul} dans cette suite.
}
\bloc{orange}{Exemple}{% id="e30"
     On cherche à déterminer le PGCD de 2691 et de 1404.
     \begin{itemize}
          \item le reste de la division euclidienne de 2691 par 1404 est 1287,
          \item le reste de la division euclidienne de 1404 par 1287 est 117,
          \item le reste de la division euclidienne de 1287 par 117 est 0.
     \end{itemize}
     Par conséquent PGCD(2691 ; 1404) = 117.
}
\cadre{vert}{Propriété}{% id="p40"
     Soient $a$ et $b$ deux entiers naturels non nuls.
     \par
     L'ensemble des diviseurs communs à $a$ et à $b$ est l'ensemble des diviseurs de leur PGCD.
}
\cadre{bleu}{Définition}{% id="d50"
     On dit que deux entiers naturels non nuls sont \textbf{premiers entre eux} si leur PGCD est égal à 1.
}
\bloc{cyan}{Remarque}{% id="r50"
     On peut généraliser cette notion à plus de deux entiers de deux façons différentes.
     \par
     Si $a$, $b$ et $c$ sont trois entiers non nuls :
     \begin{itemize}
          \item on dit que $a$, $b$ et $c$ sont premiers entre eux \textbf{dans leur ensemble} lorsque le seul diviseur commun à $a$, $b$ et $c$ est 1 ;
          \item on dit que $a$, $b$ et $c$ sont premiers entre eux \textbf{deux à deux} lorsque PGCD($a$ ; $b$) = 1, PGCD($b$ ; $c$) = 1 et PGCD($a$ ; $c$) = 1.
     \end{itemize}
     Par exemple 4 , 6 et 9 sont premiers entre eux dans leur ensemble (pas de diviseur commun à ces trois nombres autre que 1)  mais ne sont pas premiers entre eux deux à deux puisque PGCD(4 ; 6) = 2 et PGCD(6 ; 9) = 3.
}
\cadre{vert}{Propriété}{% id="p55"
     Soient $a$ et $b$ deux entiers naturels non nuls.
     \par
     $d$ est le PGCD de $a$ et de $b$ si et seulement si il esiste deux entiers $a^{\prime}$ et $b^{\prime}$ \textbf{premiers entre eux} tels que $a=a^{\prime}d$ et $b=b^{\prime}d$.
}
\bloc{orange}{Exemple}{% id="e55"
     Le  PGCD de 60 et de 45 est 15. On a :
     \par
     60 = 4×15 et 45 = 3×15 et 4 et 3 sont premiers entre eux.
}
\cadre{rouge}{Théorème (de Bézout)}{% id="t60"
     Deux entiers naturels $a$ et $b$ non nuls sont premiers entre eux si et seulement si il existe deux entiers relatifs $u$ et $v$ tels que :
     \begin{center}$au+bv = 1$.\end{center}
}
\bloc{cyan}{Remarque}{% id="r60"
     Les valeurs de $u$ et de $v$ peuvent être obtenues à l'aide de l'algorithme d'Euclide (fiche méthode à venir...)
}
\bloc{orange}{Exemple}{% id="e60"
     Pour tout entier naturel $n$, $2n+1$ et $n$ sont premiers entre eux.
     \par
     En effet $1\times \left(2n+1\right)-2\times n=1$. Donc d'après le théorème de Bézout (avec $u=1$ et $v=-2$), $n$ et $2n+1$ sont premiers entre eux.
}
\cadre{vert}{Propriété}{% id="p70"
     Soient $a$ et $b$ deux entiers naturels non nuls et $d$ leur PGCD.
     \par
     Alors, il existe deux entiers relatifs $u$ et $v$ tels que :
     \begin{center}$au+bv = d$.\end{center}
}
\bloc{cyan}{Remarque}{% id="r70"
     Attention, la réciproque est fausse.
     \par
     Si $au+bv = d$ on peut seulement en déduire que le PGCD de $a$ et de $b$ divise $d$ (d'après une \mcLien{/cours/terminale-s/divisibilite-congruences\#p20}{propriété du chapitre précédent}). Par exemple $3\times 4+2\times \left(-5\right)=2$ mais le PGCD de 3 et de 2 est 1 (ils sont premiers entre eux) et non 2.
}
\cadre{rouge}{Théorème (de Gauss)}{% id="t90"
     Soient $a$, $b$ et $c$ trois entiers naturels non nuls.
     \begin{itemize}
          \item Si $a$ divise le produit $bc$
          \item et si $a$ est premier avec $b$,
     \end{itemize}
     alors, $a$ divise $c$.
}
\bloc{orange}{Exemple}{% id="e90"
     On cherche tous les couples d'entiers naturels $\left(m ; n\right)$ tels que $5m=3n$.
     \par
     L'égalité $5m=3n$ signifie que $5$ divise $3n$. Comme $5$ et $3$ sont premiers entre eux, d'après le théorème de Gauss $5$ divise $n$. Donc il existe un entier naturel $k$ tel que $n=5k$. On a alors $5m=3n=15k$ soit $m=3k$.
     \par
     Réciproquement, on vérifie aisément que tout couple de la forme $\left(3k ; 5k\right)$ (où $k \in  \mathbb{N}$) est solution de l'équation proposée.
}
\cadre{vert}{Propriété}{% id="p100"
     Si $a$ et $b$ divisent $c$ et sont premiers entre eux, alors le produit $ab$ divise $c$.
}
\bloc{orange}{Exemples}{% id="e100"
     D'après cette propriété :
     \begin{itemize}
          \item $n$ est divisible par 6 si et seulement si il est divisible par 2 et par 3 (car 2 et 3 sont premiers entre eux).
          \item $n$ est divisible par 15 si et seulement si il est divisible par 3 et par 5 (car 3 et 5 sont premiers entre eux).
     \end{itemize}
}
\bloc{cyan}{Remarque}{% id="r100"
     L'hypothèse « $a$ et $b$ sont premiers entre eux » est essentielle. Par exemple 90 est divisible par 6 et par 10 mais n'est pas divisible par 6×10 = 60.
}
\begin{h2}2. Nombres premiers\end{h2}
\cadre{bleu}{Définition}{% id="d150"
     Un entier naturel est premier s'il admet exactement deux diviseurs (dans $\mathbb{N}$) : 1 et lui-même.
}
\bloc{cyan}{Remarque}{% id="r150"
     1 n'est pas un nombre premier (il possède un seul diviseur).
}
\cadre{vert}{Propriétés}{% id="p160"
     \begin{itemize}
          \item Tout entier naturel $n > 1$ admet au moins un diviseur premier.
          \item Tout entier naturel $n > 1$ \textbf{non premier} admet au moins un diviseur premier inférieur ou égal à $\sqrt{n}$.
\end{itemize}}
\bloc{cyan}{Remarque}{% id="r160"
     La seconde propriété est souvent utilisée pour démontrer (par l'absurde) qu'un entier naturel $n$ est premier. Il suffit, en effet, de montrer que $n$ n'est divisible par aucun nombre premier $p$ inférieur ou égal à $\sqrt{n}$. On peut donc arrêter la recherche de diviseurs premiers $p$ dès que $p^{2} > n$.
}
\bloc{orange}{Exemple}{% id="e160"
     41 est-il premier ?
     \begin{itemize}
          \item 41 n'est pas divisible par 2 (dernier chiffre impair),
          \item 41 n'est pas divisible par 3 (somme des chiffres 4+1=5),
          \item 41 n'est pas divisible par 5 (dernier chiffre différent de 0 et de 5),
          \item 7²=49 > 41 (donc $7 > \sqrt{41}$) : inutile de chercher plus loin...
     \end{itemize}
     Conclusion : 41 est un nombre premier.
}
\cadre{vert}{Propriété}{% id="p170"
     Il existe une infinité de nombres premiers.
}
\bloc{orange}{Démonstration}{% id="d170"
     On raisonne par l'absurde en supposant que l'ensemble des nombres premiers n'est pas infini. Il existe alors un plus grand nombre premier $p$.
     \par
     On pose $N=2\times 3\times 5\times \cdots \times p$ (produit de tous les nombres premiers).
     \par
     Comme tout entier naturel supérieur à 1 admet au moins un diviseur premier, $N+1$ admet un diviseur premier $d$.
     \par
     Or $d$ divise aussi le nombre $N$ puisque $N$ est le produit de \textbf{tous} les nombres premiers.
     \par
     $d$ divise $N+1$ et $N$, donc il divise leur différence 1, ce qui est impossible.
}
\cadre{vert}{Propriété}{% id="p180"
     Si $p$ est un nombre premier et $a$ un entier naturel non nul non divisible par $p$, alors $p$ et $a$ sont premier entre eux.
}
\cadre{vert}{Propriété}{% id="p185"
     Soient $a$ et $b$ deux entiers naturels non nuls.
     \par
     Si un nombre premier $p$ divise le produit $ab$, alors $p$ divise $a$ ou $b$.
}
\bloc{cyan}{Remarque}{% id="r185"
     Cette propriété résulte immédiatement de la propriété précédente et du théorème de Gauss.
}
\cadre{rouge}{Théorème (théorème fondamental de l'arithmétique)}{% id="t190"
     Tout entier naturel $n > 1$ se décompose en produit de nombres premiers.
     \par
     Cette décomposition peut s'écrire :
     \begin{center}$n=p_{1}^{a_{1}}p_{2}^{a_{2}}\cdots p_{k}^{a_{k}}$\end{center}
     où les $p_{i}$ sont des nombres premiers distincts et les $a_{i}$ des entiers naturels non nuls.
     \par
     Cette décomposition est unique à l'ordre près des facteurs.
}
\bloc{orange}{Exemple}{% id="e190"
     Cherchons la décomposition de 60 en facteurs premiers.
     \begin{itemize}
          \item 60 est divisible par 2 et le quotient de cette division est 30.
          \item 30 est divisible par 2 et le quotient est 15.
          \item 15 est divisible par 3 et le quotient est 5.
          \item Enfin, 5 est premier.
     \end{itemize}
     Donc $60=2^{2}\times 3\times 5$.
}
\cadre{vert}{Propriété}{% id="p200"
     Soit $n$ un entier naturel supérieur à 1 dont la décomposition en facteurs premiers s'écrit $n=p_{1}^{a_{1}}p_{2}^{a_{2}}\cdots p_{k}^{a_{k}}$.
     \par
     Alors, les diviseurs de $n$ sont les entiers de la forme :
     \begin{center}$n=p_{1}^{b_{1}}p_{2}^{b_{2}}\cdots p_{k}^{b_{k}}$\end{center}
     avec $0 \leqslant  b_{i} \leqslant  a_{i}$ pour tout $0 \leqslant  i \leqslant  k$.
}
\bloc{orange}{Exemple}{% id="e200"
     $60=2^{2}\times 3\times 5$ admet comme diviseurs les nombres de la forme $2^{b_{1}}\times 3^{b_{2}}\times 5^{b_{3}}$ avec $0 \leqslant  b_{1} \leqslant  2$, $0 \leqslant  b_{2} \leqslant  1$ et $0 \leqslant  b_{3} \leqslant  1$.
     \par
     Il y a trois valeurs possibles pour $b_{1}$ et deux valeurs possibles pour $b_{2}$ et pour $b_{3}$. Au total, $60$ possède donc $3\times 2\times 2=12$ diviseurs (en comptant $1$ et lui-même).
}

\end{document}