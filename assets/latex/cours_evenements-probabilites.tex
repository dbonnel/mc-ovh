\documentclass[a4paper]{article}

%================================================================================================================================
%
% Packages
%
%================================================================================================================================

\usepackage[T1]{fontenc} 	% pour caractères accentués
\usepackage[utf8]{inputenc}  % encodage utf8
\usepackage[french]{babel}	% langue : français
\usepackage{fourier}			% caractères plus lisibles
\usepackage[dvipsnames]{xcolor} % couleurs
\usepackage{fancyhdr}		% réglage header footer
\usepackage{needspace}		% empêcher sauts de page mal placés
\usepackage{graphicx}		% pour inclure des graphiques
\usepackage{enumitem,cprotect}		% personnalise les listes d'items (nécessaire pour ol, al ...)
\usepackage{hyperref}		% Liens hypertexte
\usepackage{pstricks,pst-all,pst-node,pstricks-add,pst-math,pst-plot,pst-tree,pst-eucl} % pstricks
\usepackage[a4paper,includeheadfoot,top=2cm,left=3cm, bottom=2cm,right=3cm]{geometry} % marges etc.
\usepackage{comment}			% commentaires multilignes
\usepackage{amsmath,environ} % maths (matrices, etc.)
\usepackage{amssymb,makeidx}
\usepackage{bm}				% bold maths
\usepackage{tabularx}		% tableaux
\usepackage{colortbl}		% tableaux en couleur
\usepackage{fontawesome}		% Fontawesome
\usepackage{environ}			% environment with command
\usepackage{fp}				% calculs pour ps-tricks
\usepackage{multido}			% pour ps tricks
\usepackage[np]{numprint}	% formattage nombre
\usepackage{tikz,tkz-tab} 			% package principal TikZ
\usepackage{pgfplots}   % axes
\usepackage{mathrsfs}    % cursives
\usepackage{calc}			% calcul taille boites
\usepackage[scaled=0.875]{helvet} % font sans serif
\usepackage{svg} % svg
\usepackage{scrextend} % local margin
\usepackage{scratch} %scratch
\usepackage{multicol} % colonnes
%\usepackage{infix-RPN,pst-func} % formule en notation polanaise inversée
\usepackage{listings}

%================================================================================================================================
%
% Réglages de base
%
%================================================================================================================================

\lstset{
language=Python,   % R code
literate=
{á}{{\'a}}1
{à}{{\`a}}1
{ã}{{\~a}}1
{é}{{\'e}}1
{è}{{\`e}}1
{ê}{{\^e}}1
{í}{{\'i}}1
{ó}{{\'o}}1
{õ}{{\~o}}1
{ú}{{\'u}}1
{ü}{{\"u}}1
{ç}{{\c{c}}}1
{~}{{ }}1
}


\definecolor{codegreen}{rgb}{0,0.6,0}
\definecolor{codegray}{rgb}{0.5,0.5,0.5}
\definecolor{codepurple}{rgb}{0.58,0,0.82}
\definecolor{backcolour}{rgb}{0.95,0.95,0.92}

\lstdefinestyle{mystyle}{
    backgroundcolor=\color{backcolour},   
    commentstyle=\color{codegreen},
    keywordstyle=\color{magenta},
    numberstyle=\tiny\color{codegray},
    stringstyle=\color{codepurple},
    basicstyle=\ttfamily\footnotesize,
    breakatwhitespace=false,         
    breaklines=true,                 
    captionpos=b,                    
    keepspaces=true,                 
    numbers=left,                    
xleftmargin=2em,
framexleftmargin=2em,            
    showspaces=false,                
    showstringspaces=false,
    showtabs=false,                  
    tabsize=2,
    upquote=true
}

\lstset{style=mystyle}


\lstset{style=mystyle}
\newcommand{\imgdir}{C:/laragon/www/newmc/assets/imgsvg/}
\newcommand{\imgsvgdir}{C:/laragon/www/newmc/assets/imgsvg/}

\definecolor{mcgris}{RGB}{220, 220, 220}% ancien~; pour compatibilité
\definecolor{mcbleu}{RGB}{52, 152, 219}
\definecolor{mcvert}{RGB}{125, 194, 70}
\definecolor{mcmauve}{RGB}{154, 0, 215}
\definecolor{mcorange}{RGB}{255, 96, 0}
\definecolor{mcturquoise}{RGB}{0, 153, 153}
\definecolor{mcrouge}{RGB}{255, 0, 0}
\definecolor{mclightvert}{RGB}{205, 234, 190}

\definecolor{gris}{RGB}{220, 220, 220}
\definecolor{bleu}{RGB}{52, 152, 219}
\definecolor{vert}{RGB}{125, 194, 70}
\definecolor{mauve}{RGB}{154, 0, 215}
\definecolor{orange}{RGB}{255, 96, 0}
\definecolor{turquoise}{RGB}{0, 153, 153}
\definecolor{rouge}{RGB}{255, 0, 0}
\definecolor{lightvert}{RGB}{205, 234, 190}
\setitemize[0]{label=\color{lightvert}  $\bullet$}

\pagestyle{fancy}
\renewcommand{\headrulewidth}{0.2pt}
\fancyhead[L]{maths-cours.fr}
\fancyhead[R]{\thepage}
\renewcommand{\footrulewidth}{0.2pt}
\fancyfoot[C]{}

\newcolumntype{C}{>{\centering\arraybackslash}X}
\newcolumntype{s}{>{\hsize=.35\hsize\arraybackslash}X}

\setlength{\parindent}{0pt}		 
\setlength{\parskip}{3mm}
\setlength{\headheight}{1cm}

\def\ebook{ebook}
\def\book{book}
\def\web{web}
\def\type{web}

\newcommand{\vect}[1]{\overrightarrow{\,\mathstrut#1\,}}

\def\Oij{$\left(\text{O}~;~\vect{\imath},~\vect{\jmath}\right)$}
\def\Oijk{$\left(\text{O}~;~\vect{\imath},~\vect{\jmath},~\vect{k}\right)$}
\def\Ouv{$\left(\text{O}~;~\vect{u},~\vect{v}\right)$}

\hypersetup{breaklinks=true, colorlinks = true, linkcolor = OliveGreen, urlcolor = OliveGreen, citecolor = OliveGreen, pdfauthor={Didier BONNEL - https://www.maths-cours.fr} } % supprime les bordures autour des liens

\renewcommand{\arg}[0]{\text{arg}}

\everymath{\displaystyle}

%================================================================================================================================
%
% Macros - Commandes
%
%================================================================================================================================

\newcommand\meta[2]{    			% Utilisé pour créer le post HTML.
	\def\titre{titre}
	\def\url{url}
	\def\arg{#1}
	\ifx\titre\arg
		\newcommand\maintitle{#2}
		\fancyhead[L]{#2}
		{\Large\sffamily \MakeUppercase{#2}}
		\vspace{1mm}\textcolor{mcvert}{\hrule}
	\fi 
	\ifx\url\arg
		\fancyfoot[L]{\href{https://www.maths-cours.fr#2}{\black \footnotesize{https://www.maths-cours.fr#2}}}
	\fi 
}


\newcommand\TitreC[1]{    		% Titre centré
     \needspace{3\baselineskip}
     \begin{center}\textbf{#1}\end{center}
}

\newcommand\newpar{    		% paragraphe
     \par
}

\newcommand\nosp {    		% commande vide (pas d'espace)
}
\newcommand{\id}[1]{} %ignore

\newcommand\boite[2]{				% Boite simple sans titre
	\vspace{5mm}
	\setlength{\fboxrule}{0.2mm}
	\setlength{\fboxsep}{5mm}	
	\fcolorbox{#1}{#1!3}{\makebox[\linewidth-2\fboxrule-2\fboxsep]{
  		\begin{minipage}[t]{\linewidth-2\fboxrule-4\fboxsep}\setlength{\parskip}{3mm}
  			 #2
  		\end{minipage}
	}}
	\vspace{5mm}
}

\newcommand\CBox[4]{				% Boites
	\vspace{5mm}
	\setlength{\fboxrule}{0.2mm}
	\setlength{\fboxsep}{5mm}
	
	\fcolorbox{#1}{#1!3}{\makebox[\linewidth-2\fboxrule-2\fboxsep]{
		\begin{minipage}[t]{1cm}\setlength{\parskip}{3mm}
	  		\textcolor{#1}{\LARGE{#2}}    
 	 	\end{minipage}  
  		\begin{minipage}[t]{\linewidth-2\fboxrule-4\fboxsep}\setlength{\parskip}{3mm}
			\raisebox{1.2mm}{\normalsize\sffamily{\textcolor{#1}{#3}}}						
  			 #4
  		\end{minipage}
	}}
	\vspace{5mm}
}

\newcommand\cadre[3]{				% Boites convertible html
	\par
	\vspace{2mm}
	\setlength{\fboxrule}{0.1mm}
	\setlength{\fboxsep}{5mm}
	\fcolorbox{#1}{white}{\makebox[\linewidth-2\fboxrule-2\fboxsep]{
  		\begin{minipage}[t]{\linewidth-2\fboxrule-4\fboxsep}\setlength{\parskip}{3mm}
			\raisebox{-2.5mm}{\sffamily \small{\textcolor{#1}{\MakeUppercase{#2}}}}		
			\par		
  			 #3
 	 		\end{minipage}
	}}
		\vspace{2mm}
	\par
}

\newcommand\bloc[3]{				% Boites convertible html sans bordure
     \needspace{2\baselineskip}
     {\sffamily \small{\textcolor{#1}{\MakeUppercase{#2}}}}    
		\par		
  			 #3
		\par
}

\newcommand\CHelp[1]{
     \CBox{Plum}{\faInfoCircle}{À RETENIR}{#1}
}

\newcommand\CUp[1]{
     \CBox{NavyBlue}{\faThumbsOUp}{EN PRATIQUE}{#1}
}

\newcommand\CInfo[1]{
     \CBox{Sepia}{\faArrowCircleRight}{REMARQUE}{#1}
}

\newcommand\CRedac[1]{
     \CBox{PineGreen}{\faEdit}{BIEN R\'EDIGER}{#1}
}

\newcommand\CError[1]{
     \CBox{Red}{\faExclamationTriangle}{ATTENTION}{#1}
}

\newcommand\TitreExo[2]{
\needspace{4\baselineskip}
 {\sffamily\large EXERCICE #1\ (\emph{#2 points})}
\vspace{5mm}
}

\newcommand\img[2]{
          \includegraphics[width=#2\paperwidth]{\imgdir#1}
}

\newcommand\imgsvg[2]{
       \begin{center}   \includegraphics[width=#2\paperwidth]{\imgsvgdir#1} \end{center}
}


\newcommand\Lien[2]{
     \href{#1}{#2 \tiny \faExternalLink}
}
\newcommand\mcLien[2]{
     \href{https~://www.maths-cours.fr/#1}{#2 \tiny \faExternalLink}
}

\newcommand{\euro}{\eurologo{}}

%================================================================================================================================
%
% Macros - Environement
%
%================================================================================================================================

\newenvironment{tex}{ %
}
{%
}

\newenvironment{indente}{ %
	\setlength\parindent{10mm}
}

{
	\setlength\parindent{0mm}
}

\newenvironment{corrige}{%
     \needspace{3\baselineskip}
     \medskip
     \textbf{\textsc{Corrigé}}
     \medskip
}
{
}

\newenvironment{extern}{%
     \begin{center}
     }
     {
     \end{center}
}

\NewEnviron{code}{%
	\par
     \boite{gray}{\texttt{%
     \BODY
     }}
     \par
}

\newenvironment{vbloc}{% boite sans cadre empeche saut de page
     \begin{minipage}[t]{\linewidth}
     }
     {
     \end{minipage}
}
\NewEnviron{h2}{%
    \needspace{3\baselineskip}
    \vspace{0.6cm}
	\noindent \MakeUppercase{\sffamily \large \BODY}
	\vspace{1mm}\textcolor{mcgris}{\hrule}\vspace{0.4cm}
	\par
}{}

\NewEnviron{h3}{%
    \needspace{3\baselineskip}
	\vspace{5mm}
	\textsc{\BODY}
	\par
}

\NewEnviron{margeneg}{ %
\begin{addmargin}[-1cm]{0cm}
\BODY
\end{addmargin}
}

\NewEnviron{html}{%
}

\begin{document}
\meta{url}{/cours/evenements-probabilites/}
\meta{pid}{485}
\meta{titre}{Probabilités conditionnelles - Indépendance}
\meta{type}{cours}
%
\begin{h2}1.Rappels\end{h2}
%
\cadre{bleu}{Rappels de définitions}{
     \begin{itemize}
          \item Une expérience \textbf{aléatoire} est une expérience dont le résultat dépend du hasard.
          \item Chacun des résultats possibles s'appelle une \textbf{éventualité} (ou une \textbf{issue}).
          \item L'ensemble $\Omega $ de tous les résultats possibles d'une expérience aléatoire s'appelle l'\textbf{univers} de l'expérience.
          \item On définit une \textbf{loi de probabilité} sur $\Omega $ en associant, à chaque éventualité $x_{i}$, un réel $p_{i}$ compris entre $0$ et $1$ tel que la somme de tous les $p_{i}$ soit égale à $1$.
          \item Un événement  est un sous-ensemble de $\Omega $.
     \end{itemize}
}
\bloc{orange}{Exemples}{
     Le lancer d'un dé à six faces est une expérience aléatoire d'univers comportant 6 \textbf{éventualités}:
     $\Omega =\left\{1; 2; 3; 4; 5; 6\right\}$
     \begin{itemize}
          \item L'ensemble $E_{1}=\left\{2; 4; 6\right\}$ est un \textbf{événement}. En français, cet événement peut se traduire par la phrase : \og\textit{le résultat du dé est un nombre pair}\fg{}
          \item L'ensemble $E_{2}=\left\{1; 2; 3\right\}$ est un autre événement. Ce second événement peut se traduire par la phrase : \og\textit{le résultat du dé est strictement inférieur à 4}\fg{}.
     \end{itemize}
     Ces événements peuvent être représentés par un diagramme de Venn :

\begin{center}
     \imgsvg{Venn}{0.33}% alt="Diagramme de Venn" style="width:30rem"
\end{center}

}
\cadre{bleu}{Définitions}{
     \begin{itemize}
          \item l'événement contraire de $A$ noté $\bar{A}$ est l'ensemble des éventualités de $\Omega $ qui n'appartiennent pas à $A$.
          \item l'événement $A \cup  B$ (lire \og A union B \fg{} ou \og A ou B \fg{} est constitué des éventualités qui appartiennent soit à A, soit à B, soit aux deux ensembles.
          \item l'événement $A \cap  B$ (lire \og A inter B \fg{} ou \og A et B \fg{} est constitué des éventualités qui appartiennent à la fois à A et à B.
     \end{itemize}
}
\bloc{orange}{Exemple}{
     On reprend l'exemple précédent :
     $E_{1}=\left\{2; 4; 6\right\}$
     $E_{2}=\left\{1; 2; 3\right\}$
     \begin{itemize}
          \item $\overline{E}_{1}=\left\{1; 3; 5\right\}$ : cet événement peut se traduire par \og le résultat est un nombre impair \fg{}
\begin{center}
     \imgsvg{Venn-complementaire}{0.33}% alt="Diagramme de Venn - Complémentaire" style="width:30rem"
\end{center}
          \item $E_{1} \cup  E_{2}=\left\{1; 2; 3; 4; 6\right\}$ : cet événement peut se traduire par \og le résultat est pair \textbf{ou} strictement inférieur à 4 \fg{}.
\begin{center}
     \imgsvg{Venn-union}{0.33}% alt="Diagramme de Venn - Union" style="width:30rem"
\end{center}
          \item $E_{1} \cap  E_{2}=\left\{2\right\}$ : cet événement peut se traduire par \og le résultat est pair \textbf{et} strictement inférieur à 4 \fg{}.
\begin{center}
     \imgsvg{Venn-inter}{0.33}% alt="Diagramme de Venn - Intersection" style="width:30rem"
\end{center}
     \end{itemize}
}
\cadre{bleu}{Définition}{
     On dit que A et B sont \textbf{incompatibles} si et seulement si $A \cap  B=\varnothing$
}
\bloc{mauve}{Remarque}{
     Deux événements contraires sont incompatibles mais deux événements peuvent être incompatibles sans être contraires.
}
\bloc{orange}{Exemple}{
     \og Obtenir un chiffre inférieur à 2 \fg{} et \og obtenir un chiffre supérieur à 4 \fg{} sont deux événements incompatibles.
}
\cadre{vert}{Propriétés}{
     \begin{itemize}
          \item $p\left(\varnothing\right)=0$
          \item $p\left(\Omega \right)=1$
          \item $p\left(\overline{A}\right)=1-p\left(A\right)$
          \item $p\left(A \cup  B\right)=p\left(A\right)+p\left(B\right)-p\left(A \cap  B\right)$.
     \end{itemize}
     Si A et B sont incompatibles, la dernière égalité devient :
     \begin{itemize}
          \item $p\left(A \cup  B\right)=p\left(A\right)+p\left(B\right)$.
     \end{itemize}
}
\begin{h2}2. Arbre\end{h2}
Lorsqu'une expérience aléatoire comporte plusieurs étapes, on utilise souvent un arbre pondéré pour la représenter.
\bloc{orange}{Exemple}{
     Dans une classe de Terminale, 52\% de garçons et 48\% de filles étaient candidats au baccalauréat.
     \par
     80\% des garçons et 85\% des filles ont obtenu leur diplôme.
     \par
     On choisit un élève au hasard et on note :
     \begin{itemize}
          \item $G$ : l'événement \og l'élève choisi est un garçon \fg{};
          \item $F$ : l'événement \og l'élève choisie est une fille \fg{};
          \item $B$ : l'événement \og l'élève choisi(e) a obtenu son baccalauréat \fg{}.
     \end{itemize}
     \medskip
     On peut représenter la situation à l'aide de l'arbre pondéré ci-dessous :
}
\begin{center}
\begin{extern} %width="300" alt="arbre pondéré" class="aligncenter"
     % Racine à Gauche, développement vers la droite
     \begin{tikzpicture}[xscale=1,yscale=1]
          % Styles (MODIFIABLES)
          \tikzstyle{fleche}=[-,>=latex,thick]
          \tikzstyle{noeud}=[fill=white,circle,draw]
          \tikzstyle{feuille}=[fill=white,circle,draw]
          \tikzstyle{etiquette}=[midway,fill=white]
          % Dimensions (MODIFIABLES)
          \def\DistanceInterNiveaux{3}
          \def\DistanceInterFeuilles{2}
          % Dimensions calculées (NON MODIFIABLES)
          \def\NiveauA{(0)*\DistanceInterNiveaux}
          \def\NiveauB{(1.5)*\DistanceInterNiveaux}
          \def\NiveauC{(2.5)*\DistanceInterNiveaux}
          \def\InterFeuilles{(-1)*\DistanceInterFeuilles}
          % Noeuds (MODIFIABLES : Styles et Coefficients d'InterFeuilles)
          \node[noeud] (R) at ({\NiveauA},{(1.5)*\InterFeuilles}) {$\ $};
          \node[noeud] (Ra) at ({\NiveauB},{(0.5)*\InterFeuilles}) {$G$};
          \node[feuille] (Raa) at ({\NiveauC},{(0)*\InterFeuilles}) {$B$};
          \node[feuille] (Rab) at ({\NiveauC},{(1)*\InterFeuilles}) {$\overline{B}$};
          \node[noeud] (Rb) at ({\NiveauB},{(2.5)*\InterFeuilles}) {$F$};
          \node[feuille] (Rba) at ({\NiveauC},{(2)*\InterFeuilles}) {$B$};
          \node[feuille] (Rbb) at ({\NiveauC},{(3)*\InterFeuilles}) {$\overline{B}$};
          % Arcs (MODIFIABLES : Styles)
          \draw[fleche] (R)--(Ra) node[etiquette] {$0,52$};
          \draw[fleche] (Ra)--(Raa) node[etiquette] {$0,8$};
          \draw[fleche] (Ra)--(Rab) node[etiquette] {$0,2$};
          \draw[fleche] (R)--(Rb) node[etiquette] {$0,48$};
          \draw[fleche] (Rb)--(Rba) node[etiquette] {$0,85$};
          \draw[fleche] (Rb)--(Rbb) node[etiquette] {$0,15$};
     \end{tikzpicture}
\end{extern}
\end{center}
Le premier niveau indique le genre de l'élève ($G$ ou $F$) et le second indique l'obtention du diplôme ($B$ ou $\overline{B}$).
\par
On inscrit les probabilités sur chacune des branches.
\par
La \textbf{somme} des probabilités inscrites sur les branches partant d’un même nœud \textbf{est toujours égale à 1}.
%
\begin{h2}3. Probabilités conditionnelles\end{h2}
\cadre{bleu}{Définition}{
     Soit A et B deux événements tels que $p\left(A\right)\neq 0$, \textbf{la probabilité de B sachant A} est le nombre :
     \[  p_{A}\left(B\right)=\frac{p\left(A \cap  B\right)}{p\left(A\right)}. \]
     On peut aussi noter cette probabilité $p\left(B/A\right)$.
}
\bloc{orange}{Exemple}{
     On reprend l'exemple du lancer d'un dé.
     La probabilité d'obtenir un chiffre pair sachant que le chiffre obtenu est strictement inférieur à 4 est (en cas d'équiprobabilité) :
     \[ p_{E_{2}}\left(E_{1}\right)=\frac{p\left(E_{1} \cap  E_{2}\right)}{p\left(E_{2}\right)}=\frac{1}{3}. \]
}
\bloc{mauve}{Remarques}{
     \begin{itemize}
          \item L'égalité précédente s'emploie souvent sous la forme :
          \[p\left(A \cap  B\right)=p\left(A\right)\times p_{A}\left(B\right)\]
          pour calculer la probabilité de $A \cap  B$.
          \item \textbf{Attention} à ne pas confondre $p_{A}\left(B\right)$ et $p\left(A \cap  B\right)$ dans les exercices.\\
          On doit calculer  $p_{A}\left(B\right)$ lorsque l'\textbf{on sait que $A$ est réalisé.}
          \item Avec un arbre pondéré, les probabilités conditionnelles figurent sur les branches du second niveau et des niveaux supérieurs (s'il y en a).\\
          La probabilité inscrite sur la branche reliant $A$ à $B$ est $p_A(B)$.\\
          Typiquement, un arbre binaire à deux niveaux se présentera ainsi :
         \begin{center}
          \begin{extern} %width="300" alt="arbre pondéré" class="aligncenter"
               % Racine à Gauche, développement vers la droite
               \begin{tikzpicture}[xscale=1,yscale=1]
                    % Styles (MODIFIABLES)
                    \tikzstyle{fleche}=[-,>=latex,thick]
                    \tikzstyle{noeud}=[fill=white,circle,draw]
                    \tikzstyle{feuille}=[fill=white,circle,draw]
                    \tikzstyle{etiquette}=[midway,fill=white]
                    % Dimensions (MODIFIABLES)
                    \def\DistanceInterNiveaux{3}
                    \def\DistanceInterFeuilles{2}
                    % Dimensions calculées (NON MODIFIABLES)
                    \def\NiveauA{(0)*\DistanceInterNiveaux}
                    \def\NiveauB{(1.5)*\DistanceInterNiveaux}
                    \def\NiveauC{(2.5)*\DistanceInterNiveaux}
                    \def\InterFeuilles{(-1)*\DistanceInterFeuilles}
                    % Noeuds (MODIFIABLES : Styles et Coefficients d'InterFeuilles)
                    \node[noeud] (R) at ({\NiveauA},{(1.5)*\InterFeuilles}) {$\ $};
                    \node[noeud] (Ra) at ({\NiveauB},{(0.5)*\InterFeuilles}) {$A$};
                    \node[feuille] (Raa) at ({\NiveauC},{(0)*\InterFeuilles}) {$B$};
                    \node[feuille] (Rab) at ({\NiveauC},{(1)*\InterFeuilles}) {$\overline{B}$};
                    \node[noeud] (Rb) at ({\NiveauB},{(2.5)*\InterFeuilles}) {$\overline{A}$};
                    \node[feuille] (Rba) at ({\NiveauC},{(2)*\InterFeuilles}) {$B$};
                    \node[feuille] (Rbb) at ({\NiveauC},{(3)*\InterFeuilles}) {$\overline{B}$};
                    % Arcs (MODIFIABLES : Styles)
                    \draw[fleche] (R)--(Ra) node[etiquette] {$p(A)$};
                    \draw[fleche] (Ra)--(Raa) node[etiquette] {$p_A(B)$};
                    \draw[fleche] (Ra)--(Rab) node[etiquette] {$p_A(\overline{B})$};
                    \draw[fleche] (R)--(Rb) node[etiquette] {$p(\overline{A})$};
                    \draw[fleche] (Rb)--(Rba) node[etiquette] {$p_{\overline{A}}(B)$};
                    \draw[fleche] (Rb)--(Rbb) node[etiquette] {$p_{\overline{A}}(\overline{B})$};
               \end{tikzpicture}
          \end{extern}
        \end{center}
          \item La formule $p\left(A \cap  B\right)=p\left(A\right)\times p_{A}\left(B\right)$ s'interprète alors de la façon suivante : \\
          \og La probabilité de l'événement  $A \cap  B$ s'obtient en faisant \textbf{le produit} des probabilités inscrites sur le chemin passant par $A$ et $B$\fg{}.
          %
         \begin{center}
          \begin{extern} %width="300" alt="arbre pondéré" class="aligncenter"
               % Racine à Gauche, développement vers la droite
               \begin{tikzpicture}[xscale=1,yscale=1]
                    % Styles (MODIFIABLES)
                    \tikzstyle{fleche}=[-,>=latex,thick]
                    \tikzstyle{rfleche}=[-,>=latex,color=red,thick]
                    \tikzstyle{noeud}=[fill=white,circle,draw]
                    \tikzstyle{rnoeud}=[fill=white,circle,draw=red]
                    \tikzstyle{feuille}=[fill=white,circle,draw]
                    \tikzstyle{rfeuille}=[fill=white,circle,draw=red]
                    \tikzstyle{etiquette}=[midway,fill=white]
                    % Dimensions (MODIFIABLES)
                    \def\DistanceInterNiveaux{3}
                    \def\DistanceInterFeuilles{2}
                    % Dimensions calculées (NON MODIFIABLES)
                    \def\NiveauA{(0)*\DistanceInterNiveaux}
                    \def\NiveauB{(1.5)*\DistanceInterNiveaux}
                    \def\NiveauC{(2.5)*\DistanceInterNiveaux}
                    \def\InterFeuilles{(-1)*\DistanceInterFeuilles}
                    % Noeuds (MODIFIABLES : Styles et Coefficients d'InterFeuilles)
                    \node[rnoeud] (R) at ({\NiveauA},{(1.5)*\InterFeuilles}) {$\ $};
                    \node[rnoeud] (Ra) at ({\NiveauB},{(0.5)*\InterFeuilles}) {$\red A$};
                    \node[rfeuille] (Raa) at ({\NiveauC},{(0)*\InterFeuilles}) {$\red B$};
                    \node[feuille] (Rab) at ({\NiveauC},{(1)*\InterFeuilles}) {$\overline{B}$};
                    \node[noeud] (Rb) at ({\NiveauB},{(2.5)*\InterFeuilles}) {$\overline{A}$};
                    \node[feuille] (Rba) at ({\NiveauC},{(2)*\InterFeuilles}) {$B$};
                    \node[feuille] (Rbb) at ({\NiveauC},{(3)*\InterFeuilles}) {$\overline{B}$};
                    % Arcs (MODIFIABLES : Styles)
                    \draw[rfleche] (R)--(Ra) node[etiquette] {\red $p(A)$};
                    \draw[rfleche] (Ra)--(Raa) node[etiquette] {\red $p_A(B)$};
                    \draw[fleche] (Ra)--(Rab) node[etiquette] {$p_A(\overline{B})$};
                    \draw[fleche] (R)--(Rb) node[etiquette] {$p(\overline{A})$};
                    \draw[fleche] (Rb)--(Rba) node[etiquette] {$p_{\overline{A}}(B)$};
                    \draw[fleche] (Rb)--(Rbb) node[etiquette] {$p_{\overline{A}}(\overline{B})$};
               \end{tikzpicture}
          \end{extern}
        \end{center}
     \end{itemize}
}
\begin{h2}4. Événements indépendants\end{h2}
\cadre{bleu}{Définition}{
     Deux événements A et B sont \textbf{indépendants} si et seulement si :
     \[ p\left(A \cap  B\right)=p\left(A\right)\times p\left(B\right). \]
}
\cadre{vert}{Propriété}{
     $A$ et $B$ sont indépendants si et seulement si :
     \[ p_{A}\left(B\right)=p\left(B\right). \]
}
\bloc{orange}{Démonstration}{
     Elle résulte directement du fait que pour deux événements quelconques :
     \[ p\left(A \cap  B\right)=p\left(A\right)\times p_{A}\left(B\right). \]
}
\bloc{mauve}{Remarque}{
     Comme $A \cap  B=B \cap  A$, $A$ et $B$ sont interchangeables dans cette formule et on a également :
     \begin{center}
          $A$ et $B$ sont indépendants $ \Leftrightarrow  $ $p_{B}\left(A\right)=p\left(A\right)$.
     \end{center}
}
\begin{h2}5. Formule des probabilités totales\end{h2}
\cadre{bleu}{Définition}{
     $A_{1}$, $A_{2}$, ... , $A_{n}$ forment une partition de $\Omega $ si et seulement si  $A_{1} \cup  A_{2} . . . \cup  A_{n}=\Omega $ et $A_{i} \cap  A_{j}=\varnothing$ pour $i\neq j$.
}
\bloc{mauve}{Cas particulier fréquent}{
     Pour toute partie $A\subset\Omega $, $A$ et $\overline{A}$ forment une partition de $\Omega$.
}
\cadre{vert}{Propriété (Formule des probabilités totales)}{
     Si $A_{1}$, $A_{2}$,... $A_{n}$ forment une partition de $\Omega $, pour tout événement $B$, on a :
     \begin{center}
          $ p\left(B\right)=p\left(A_{1} \cap  B\right)+p\left(A_{2} \cap  B\right)+ \cdots $\nosp$ +p\left(A_{n} \cap  B\right). $
     \end{center}
     \par
     Cette formule peut également s'écrire à l'aide de probabilités conditionnelles :
     \begin{center}
          $p\left(B\right)=p\left(A_{1} \right)\times p_{A_{1} }\left(B\right)$\nosp$+p\left(A_{2} \right)\times p_{A_{2}}\left(B\right)+\cdots$\nosp$+p\left(A_{n}\right)\times p_{A_{n}}\left(B\right)$.
     \end{center}
}
\bloc{mauve}{Cas particulier fréquent}{
     En utilisant la partition  $\left\{A, \overline{A}\right\}$, quels que soient les événements $A$ et $B$ :
     \begin{center}
          $p\left(B\right)=p\left(A \cap  B\right)+p\left(\overline{A} \cap  B\right)$\\
          $p\left(B\right)=p\left(A\right)\times p_{A}\left(B\right)+p\left(\overline{A}\right)\times p_{\overline{A}}\left(B\right)$.
\end{center}}
\bloc{mauve}{Remarque}{
     \`A l'aide d'un arbre pondéré, ce résultat s'interprète de la façon suivante :\\
     \og La probabilité de l'événement $B$ est égale à la somme des probabilités des trajets menant à $B$ \fg{}.

\begin{center}
\begin{extern} %width="300" alt="arbre pondéré" 
          % Racine à Gauche, développement vers la droite
          \begin{tikzpicture}[xscale=1,yscale=1]
               % Styles (MODIFIABLES)
               \tikzstyle{fleche}=[-,>=latex,thick]
               \tikzstyle{rfleche}=[-,>=latex,color=red,thick]
               \tikzstyle{noeud}=[fill=white,circle,draw]
               \tikzstyle{rnoeud}=[fill=white,circle,draw=red]
               \tikzstyle{feuille}=[fill=white,circle,draw]
               \tikzstyle{rfeuille}=[fill=white,circle,draw=red]
               \tikzstyle{etiquette}=[midway,fill=white]
               % Dimensions (MODIFIABLES)
               \def\DistanceInterNiveaux{3}
               \def\DistanceInterFeuilles{2}
               % Dimensions calculées (NON MODIFIABLES)
               \def\NiveauA{(0)*\DistanceInterNiveaux}
               \def\NiveauB{(1.5)*\DistanceInterNiveaux}
               \def\NiveauC{(2.5)*\DistanceInterNiveaux}
               \def\InterFeuilles{(-1)*\DistanceInterFeuilles}
               % Noeuds (MODIFIABLES : Styles et Coefficients d'InterFeuilles)
               \node[rnoeud] (R) at ({\NiveauA},{(1.5)*\InterFeuilles}) {$\ $};
               \node[rnoeud] (Ra) at ({\NiveauB},{(0.5)*\InterFeuilles}) {$\red A$};
               \node[rfeuille] (Raa) at ({\NiveauC},{(0)*\InterFeuilles}) {$\red B$};
               \node[feuille] (Rab) at ({\NiveauC},{(1)*\InterFeuilles}) {$\overline{B}$};
               \node[rnoeud] (Rb) at ({\NiveauB},{(2.5)*\InterFeuilles}) {$\red \overline{A}$};
               \node[rfeuille] (Rba) at ({\NiveauC},{(2)*\InterFeuilles}) {$\red B$};
               \node[feuille] (Rbb) at ({\NiveauC},{(3)*\InterFeuilles}) {$\overline{B}$};
               % Arcs (MODIFIABLES : Styles)
               \draw[rfleche] (R)--(Ra) node[etiquette] {\red $p(A)$};
               \draw[rfleche] (Ra)--(Raa) node[etiquette] {\red $p_A(B)$};
               \draw[fleche] (Ra)--(Rab) node[etiquette] {$p_A(\overline{B})$};
               \draw[rfleche] (R)--(Rb) node[etiquette] {$p(\overline{A})$};
               \draw[rfleche] (Rb)--(Rba) node[etiquette] {$p_{\overline{A}}(B)$};
               \draw[fleche] (Rb)--(Rbb) node[etiquette] {$p_{\overline{A}}(\overline{B})$};
          \end{tikzpicture}
     \end{extern}
   \end{center}
}

\end{document}