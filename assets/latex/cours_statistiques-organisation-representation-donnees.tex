\documentclass[a4paper]{article}

%================================================================================================================================
%
% Packages
%
%================================================================================================================================

\usepackage[T1]{fontenc} 	% pour caractères accentués
\usepackage[utf8]{inputenc}  % encodage utf8
\usepackage[french]{babel}	% langue : français
\usepackage{fourier}			% caractères plus lisibles
\usepackage[dvipsnames]{xcolor} % couleurs
\usepackage{fancyhdr}		% réglage header footer
\usepackage{needspace}		% empêcher sauts de page mal placés
\usepackage{graphicx}		% pour inclure des graphiques
\usepackage{enumitem,cprotect}		% personnalise les listes d'items (nécessaire pour ol, al ...)
\usepackage{hyperref}		% Liens hypertexte
\usepackage{pstricks,pst-all,pst-node,pstricks-add,pst-math,pst-plot,pst-tree,pst-eucl} % pstricks
\usepackage[a4paper,includeheadfoot,top=2cm,left=3cm, bottom=2cm,right=3cm]{geometry} % marges etc.
\usepackage{comment}			% commentaires multilignes
\usepackage{amsmath,environ} % maths (matrices, etc.)
\usepackage{amssymb,makeidx}
\usepackage{bm}				% bold maths
\usepackage{tabularx}		% tableaux
\usepackage{colortbl}		% tableaux en couleur
\usepackage{fontawesome}		% Fontawesome
\usepackage{environ}			% environment with command
\usepackage{fp}				% calculs pour ps-tricks
\usepackage{multido}			% pour ps tricks
\usepackage[np]{numprint}	% formattage nombre
\usepackage{tikz,tkz-tab} 			% package principal TikZ
\usepackage{pgfplots}   % axes
\usepackage{mathrsfs}    % cursives
\usepackage{calc}			% calcul taille boites
\usepackage[scaled=0.875]{helvet} % font sans serif
\usepackage{svg} % svg
\usepackage{scrextend} % local margin
\usepackage{scratch} %scratch
\usepackage{multicol} % colonnes
%\usepackage{infix-RPN,pst-func} % formule en notation polanaise inversée
\usepackage{listings}

%================================================================================================================================
%
% Réglages de base
%
%================================================================================================================================

\lstset{
language=Python,   % R code
literate=
{á}{{\'a}}1
{à}{{\`a}}1
{ã}{{\~a}}1
{é}{{\'e}}1
{è}{{\`e}}1
{ê}{{\^e}}1
{í}{{\'i}}1
{ó}{{\'o}}1
{õ}{{\~o}}1
{ú}{{\'u}}1
{ü}{{\"u}}1
{ç}{{\c{c}}}1
{~}{{ }}1
}


\definecolor{codegreen}{rgb}{0,0.6,0}
\definecolor{codegray}{rgb}{0.5,0.5,0.5}
\definecolor{codepurple}{rgb}{0.58,0,0.82}
\definecolor{backcolour}{rgb}{0.95,0.95,0.92}

\lstdefinestyle{mystyle}{
    backgroundcolor=\color{backcolour},   
    commentstyle=\color{codegreen},
    keywordstyle=\color{magenta},
    numberstyle=\tiny\color{codegray},
    stringstyle=\color{codepurple},
    basicstyle=\ttfamily\footnotesize,
    breakatwhitespace=false,         
    breaklines=true,                 
    captionpos=b,                    
    keepspaces=true,                 
    numbers=left,                    
xleftmargin=2em,
framexleftmargin=2em,            
    showspaces=false,                
    showstringspaces=false,
    showtabs=false,                  
    tabsize=2,
    upquote=true
}

\lstset{style=mystyle}


\lstset{style=mystyle}
\newcommand{\imgdir}{C:/laragon/www/newmc/assets/imgsvg/}
\newcommand{\imgsvgdir}{C:/laragon/www/newmc/assets/imgsvg/}

\definecolor{mcgris}{RGB}{220, 220, 220}% ancien~; pour compatibilité
\definecolor{mcbleu}{RGB}{52, 152, 219}
\definecolor{mcvert}{RGB}{125, 194, 70}
\definecolor{mcmauve}{RGB}{154, 0, 215}
\definecolor{mcorange}{RGB}{255, 96, 0}
\definecolor{mcturquoise}{RGB}{0, 153, 153}
\definecolor{mcrouge}{RGB}{255, 0, 0}
\definecolor{mclightvert}{RGB}{205, 234, 190}

\definecolor{gris}{RGB}{220, 220, 220}
\definecolor{bleu}{RGB}{52, 152, 219}
\definecolor{vert}{RGB}{125, 194, 70}
\definecolor{mauve}{RGB}{154, 0, 215}
\definecolor{orange}{RGB}{255, 96, 0}
\definecolor{turquoise}{RGB}{0, 153, 153}
\definecolor{rouge}{RGB}{255, 0, 0}
\definecolor{lightvert}{RGB}{205, 234, 190}
\setitemize[0]{label=\color{lightvert}  $\bullet$}

\pagestyle{fancy}
\renewcommand{\headrulewidth}{0.2pt}
\fancyhead[L]{maths-cours.fr}
\fancyhead[R]{\thepage}
\renewcommand{\footrulewidth}{0.2pt}
\fancyfoot[C]{}

\newcolumntype{C}{>{\centering\arraybackslash}X}
\newcolumntype{s}{>{\hsize=.35\hsize\arraybackslash}X}

\setlength{\parindent}{0pt}		 
\setlength{\parskip}{3mm}
\setlength{\headheight}{1cm}

\def\ebook{ebook}
\def\book{book}
\def\web{web}
\def\type{web}

\newcommand{\vect}[1]{\overrightarrow{\,\mathstrut#1\,}}

\def\Oij{$\left(\text{O}~;~\vect{\imath},~\vect{\jmath}\right)$}
\def\Oijk{$\left(\text{O}~;~\vect{\imath},~\vect{\jmath},~\vect{k}\right)$}
\def\Ouv{$\left(\text{O}~;~\vect{u},~\vect{v}\right)$}

\hypersetup{breaklinks=true, colorlinks = true, linkcolor = OliveGreen, urlcolor = OliveGreen, citecolor = OliveGreen, pdfauthor={Didier BONNEL - https://www.maths-cours.fr} } % supprime les bordures autour des liens

\renewcommand{\arg}[0]{\text{arg}}

\everymath{\displaystyle}

%================================================================================================================================
%
% Macros - Commandes
%
%================================================================================================================================

\newcommand\meta[2]{    			% Utilisé pour créer le post HTML.
	\def\titre{titre}
	\def\url{url}
	\def\arg{#1}
	\ifx\titre\arg
		\newcommand\maintitle{#2}
		\fancyhead[L]{#2}
		{\Large\sffamily \MakeUppercase{#2}}
		\vspace{1mm}\textcolor{mcvert}{\hrule}
	\fi 
	\ifx\url\arg
		\fancyfoot[L]{\href{https://www.maths-cours.fr#2}{\black \footnotesize{https://www.maths-cours.fr#2}}}
	\fi 
}


\newcommand\TitreC[1]{    		% Titre centré
     \needspace{3\baselineskip}
     \begin{center}\textbf{#1}\end{center}
}

\newcommand\newpar{    		% paragraphe
     \par
}

\newcommand\nosp {    		% commande vide (pas d'espace)
}
\newcommand{\id}[1]{} %ignore

\newcommand\boite[2]{				% Boite simple sans titre
	\vspace{5mm}
	\setlength{\fboxrule}{0.2mm}
	\setlength{\fboxsep}{5mm}	
	\fcolorbox{#1}{#1!3}{\makebox[\linewidth-2\fboxrule-2\fboxsep]{
  		\begin{minipage}[t]{\linewidth-2\fboxrule-4\fboxsep}\setlength{\parskip}{3mm}
  			 #2
  		\end{minipage}
	}}
	\vspace{5mm}
}

\newcommand\CBox[4]{				% Boites
	\vspace{5mm}
	\setlength{\fboxrule}{0.2mm}
	\setlength{\fboxsep}{5mm}
	
	\fcolorbox{#1}{#1!3}{\makebox[\linewidth-2\fboxrule-2\fboxsep]{
		\begin{minipage}[t]{1cm}\setlength{\parskip}{3mm}
	  		\textcolor{#1}{\LARGE{#2}}    
 	 	\end{minipage}  
  		\begin{minipage}[t]{\linewidth-2\fboxrule-4\fboxsep}\setlength{\parskip}{3mm}
			\raisebox{1.2mm}{\normalsize\sffamily{\textcolor{#1}{#3}}}						
  			 #4
  		\end{minipage}
	}}
	\vspace{5mm}
}

\newcommand\cadre[3]{				% Boites convertible html
	\par
	\vspace{2mm}
	\setlength{\fboxrule}{0.1mm}
	\setlength{\fboxsep}{5mm}
	\fcolorbox{#1}{white}{\makebox[\linewidth-2\fboxrule-2\fboxsep]{
  		\begin{minipage}[t]{\linewidth-2\fboxrule-4\fboxsep}\setlength{\parskip}{3mm}
			\raisebox{-2.5mm}{\sffamily \small{\textcolor{#1}{\MakeUppercase{#2}}}}		
			\par		
  			 #3
 	 		\end{minipage}
	}}
		\vspace{2mm}
	\par
}

\newcommand\bloc[3]{				% Boites convertible html sans bordure
     \needspace{2\baselineskip}
     {\sffamily \small{\textcolor{#1}{\MakeUppercase{#2}}}}    
		\par		
  			 #3
		\par
}

\newcommand\CHelp[1]{
     \CBox{Plum}{\faInfoCircle}{À RETENIR}{#1}
}

\newcommand\CUp[1]{
     \CBox{NavyBlue}{\faThumbsOUp}{EN PRATIQUE}{#1}
}

\newcommand\CInfo[1]{
     \CBox{Sepia}{\faArrowCircleRight}{REMARQUE}{#1}
}

\newcommand\CRedac[1]{
     \CBox{PineGreen}{\faEdit}{BIEN R\'EDIGER}{#1}
}

\newcommand\CError[1]{
     \CBox{Red}{\faExclamationTriangle}{ATTENTION}{#1}
}

\newcommand\TitreExo[2]{
\needspace{4\baselineskip}
 {\sffamily\large EXERCICE #1\ (\emph{#2 points})}
\vspace{5mm}
}

\newcommand\img[2]{
          \includegraphics[width=#2\paperwidth]{\imgdir#1}
}

\newcommand\imgsvg[2]{
       \begin{center}   \includegraphics[width=#2\paperwidth]{\imgsvgdir#1} \end{center}
}


\newcommand\Lien[2]{
     \href{#1}{#2 \tiny \faExternalLink}
}
\newcommand\mcLien[2]{
     \href{https~://www.maths-cours.fr/#1}{#2 \tiny \faExternalLink}
}

\newcommand{\euro}{\eurologo{}}

%================================================================================================================================
%
% Macros - Environement
%
%================================================================================================================================

\newenvironment{tex}{ %
}
{%
}

\newenvironment{indente}{ %
	\setlength\parindent{10mm}
}

{
	\setlength\parindent{0mm}
}

\newenvironment{corrige}{%
     \needspace{3\baselineskip}
     \medskip
     \textbf{\textsc{Corrigé}}
     \medskip
}
{
}

\newenvironment{extern}{%
     \begin{center}
     }
     {
     \end{center}
}

\NewEnviron{code}{%
	\par
     \boite{gray}{\texttt{%
     \BODY
     }}
     \par
}

\newenvironment{vbloc}{% boite sans cadre empeche saut de page
     \begin{minipage}[t]{\linewidth}
     }
     {
     \end{minipage}
}
\NewEnviron{h2}{%
    \needspace{3\baselineskip}
    \vspace{0.6cm}
	\noindent \MakeUppercase{\sffamily \large \BODY}
	\vspace{1mm}\textcolor{mcgris}{\hrule}\vspace{0.4cm}
	\par
}{}

\NewEnviron{h3}{%
    \needspace{3\baselineskip}
	\vspace{5mm}
	\textsc{\BODY}
	\par
}

\NewEnviron{margeneg}{ %
\begin{addmargin}[-1cm]{0cm}
\BODY
\end{addmargin}
}

\NewEnviron{html}{%
}

\begin{document}
\meta{url}{/cours/statistiques-organisation-representation-donnees/}
\meta{pid}{144}
\meta{titre}{Statistiques en Seconde}
\meta{type}{cours}
\begin{h2}I. Organisation et représentation des données\end{h2}
\cadre{bleu}{Définitions}{%
     \begin{itemize}
          \item Les statistiques permettent d'étudier un \textbf{caractère} d'une \textbf{population}.
          \item Le nombre d'éléments de la population s'appelle l'\textbf{effectif global} (ou l'\textbf{effectif total}).
          \item Pour une valeur de caractère donnée, l'\textbf{effectif} est le nombre d'éléments correspondant à cette valeur.
          \item Une \textbf{série statistique} est un tableau donnant les effectifs pour chacune des valeurs possibles du caractère.
     \end{itemize}
}
\bloc{orange}{Exemple 1}{%
     On fait une étude portant sur l'âge des élèves d'un lycée.
     \begin{itemize}
          \item le \textbf{caractère} étudié est l'âge
          \item la \textbf{population} est l'ensemble des élèves du lycée
          \item l'\textbf{effectif global} est le nombre d'élèves du lycée
          \item le tableau ci-dessous est la \textbf{série statistique} pour ce caractère dans un lycée donné~:
          \begin{center}
               \begin{tabular}{|c|c|c|c|c|c|c|c|c|} %class="compact" width="600"
                    \hline
                    âges (en années)  &  14  &   15  &  16  &  17  &  18  &  19  &  20
                    \\ \hline
                    effectifs         &  3   &   22  &  65  &  82  &  59  &  35  &  2
                    \\ \hline
               \end{tabular}
          \end{center}
     \end{itemize}
}
\bloc{orange}{Exemple 2 : création d'un tableau pour une série statistique}{%
     On suppose que les notes à un contrôle dans une classe de 21 élèves sont les suivantes~:
     \begin{center}
          5 ; 14 ; 13 ; 16 ; 9 ; 8 ; 18 ; 2 ; 13 ; 12 ; 15 ; 12 ; 8 ; 6 ; 5 ; 17 ; 3 ; 19 ; 9 ; 13 ; 14
     \end{center}
     Ces données brutes sont assez peu pratiques à utiliser sous cette forme (notamment lorsqu'il y a beaucoup de valeurs).
     \par
     Pour commencer on commence à trier les notes de la plus petite à la plus grande :
     \begin{center}
          2 ; 3 ; 5 ; 5 ; 6 ; 8 ; 8 ; 9 ; 9 ; 12 ; 12 ; 13 ; 13 ; 13 ; 14 ; 14 ; 15 ; 16 ; 17 ; 18 ; 19
     \end{center}
     Ensuite, on va créer le tableau de cette série en indiquant pour chaque note son effectif c'est à dire le nombre d'élèves ayant obtenu cette note :
     \begin{center}
          \begin{tabular}{|c|c|c|c|c|c|c|c|c|c|c|c|c|c|c|}%class="compact" width="600"
               \hline
               notes       &  2  &   3  &   5  &  6  &  8  &  9  &  12  &  13  &  14  &  15  &  16  &  17  &  18  &  19
               \\ \hline
               effectifs   &  1   &  1  &  2   &  1  &  2  &  2  &  2   &  3   &  2  &  1   &  1  &  1  &  1  &  1
               \\ \hline
          \end{tabular}
     \end{center}
}
\begin{h2}II - Médiane - Quartiles\end{h2}
\cadre{bleu}{Définition}{%
     La \textbf{médiane} d'une série statistique est la valeur du caractère qui partage la population en deux classes de même effectif.
}
\bloc{cyan}{Remarque}{%
     En pratique pour trouver la médiane d'une série statistique d'effectif global $n$ :
     \begin{itemize}
          \item On ordonne les valeurs du caractère dans l'ordre croissant.
          \item Si $n$ est pair, la médiane sera la moyenne des valeurs du terme de rang $\frac{n}{2}$ et du terme de rang $\frac{n}{2}+1$.
          \item Si $n$ est impair, la médiane sera la valeur du terme de rang $\frac{n+1}{2}$.
          \item Lorsque l'effectif global est élevé, il est souvent utile de calculer les effectifs cumulés pour trouver cette valeur.
     \end{itemize}
}
\bloc{orange}{Exemple}{%
     Reprenons l'exemple 2 ci-dessus.
     \par
     Dans cet exemple, c'est la 11ème note ($11=\frac{21+1}{2}$) qui est la médiane. En effet, il y a 10 notes au dessous et 10 notes au dessus :
     \begin{center}
          2 ; 3 ; 5 ; 5 ; 6 ; 8 ; 8 ; 9 ; 9 ; 12 ; \textbf{$12$} ; 13 ; 13 ; 13 ; 14 ; 14 ; 15 ; 16 ; 17 ; 18 ; 19
     \end{center}
     \textbf{La médiane est donc 12.}
     \par
     Supposons qu'il n'y ait que 20 élèves (on enlève l'élève qui a eu 2) :
     \begin{center}
          3 ; 5 ; 5 ; 6 ; 8 ; 8 ; 9 ; 9 ; 12 ; 12 ; 13 ; 13 ; 13 ; 14 ; 14 ; 15 ; 16 ; 17 ; 18 ; 19
     \end{center}
     Il n'y a plus ici de note située "juste au milieu".
     \par
     Si on choisit la 10ème note (qui est 12) il y a 9 notes en dessous et 10 notes au dessus.
     \par
     Si on choisit la 11ème note (qui est 13) il y a 10 notes en dessous et 9 notes au dessus.
     \begin{center}
          3 ; 5 ; 5 ; 6 ; 8 ; 8 ; 9 ; 9 ; 12 ; \textbf{$12 ; 13 $}; 13 ; 13 ; 14 ; 14 ; 15 ; 16 ; 17 ; 18 ; 19
     \end{center}
     Dans ce cas, on prend comme médiane la moyenne de 12 et de 13 c'est à dire 12,5.
     \par
     \textbf{La médiane est donc 12,5.}
}
\cadre{bleu}{Définitions}{%
     \begin{itemize}
          \item Le \textbf{premier quartile} Q1 d'une série statistique est la plus petite valeur des termes de la série pour laquelle au moins un quart des données sont inférieures ou égales à Q1.
          \item Le \textbf{troisième quartile} Q3  d'une série statistique est la plus petite valeur des termes de la série pour laquelle au moins trois quarts des données sont inférieures ou égales à Q3.
     \end{itemize}
}
\bloc{orange}{Exemple}{%
     Reprenons l'exemple des notes ci-dessus (avec 21 élèves).
     \par
     Pour le \textbf{premier quartile} il faut qu'il y ait au moins 1/4 des notes qui soient inférieures ou égales. 1/4$\times $21=5,25. Le premier quartile est donc la 6ème note.
     \begin{center}
          2 ; 3 ; 5 ; 5 ; 6 ; \textbf{$8$} ; 8 ; 9 ; 9 ; 12 ; 12 ; 13 ; 13 ; 13 ; 14 ; 14 ; 15 ; 16 ; 17 ; 18 ; 19
     \end{center}
     \textbf{le premier quartile est 8.}
     \par
     Pour le \textbf{troisième quartile} il faut qu'il y ait au moins 3/4 des notes qui soient inférieures ou égales.3/4$\times $21=15,75.
     \par
     Le troisième quartile est donc la 16ème note.
     \begin{center}
          2 ; 3 ; 5 ; 5 ; 6 ; 8 ; 8 ; 9 ; 9 ; 12 ; 12 ; 13 ; 13 ; 13 ; 14 ; \textbf{$14$} ; 15 ; 16 ; 17 ; 18 ; 19
     \end{center}
     \textbf{le troisième quartile est 14.}
}

\end{document}