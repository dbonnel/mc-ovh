\documentclass[a4paper]{article}

%================================================================================================================================
%
% Packages
%
%================================================================================================================================

\usepackage[T1]{fontenc} 	% pour caractères accentués
\usepackage[utf8]{inputenc}  % encodage utf8
\usepackage[french]{babel}	% langue : français
\usepackage{fourier}			% caractères plus lisibles
\usepackage[dvipsnames]{xcolor} % couleurs
\usepackage{fancyhdr}		% réglage header footer
\usepackage{needspace}		% empêcher sauts de page mal placés
\usepackage{graphicx}		% pour inclure des graphiques
\usepackage{enumitem,cprotect}		% personnalise les listes d'items (nécessaire pour ol, al ...)
\usepackage{hyperref}		% Liens hypertexte
\usepackage{pstricks,pst-all,pst-node,pstricks-add,pst-math,pst-plot,pst-tree,pst-eucl} % pstricks
\usepackage[a4paper,includeheadfoot,top=2cm,left=3cm, bottom=2cm,right=3cm]{geometry} % marges etc.
\usepackage{comment}			% commentaires multilignes
\usepackage{amsmath,environ} % maths (matrices, etc.)
\usepackage{amssymb,makeidx}
\usepackage{bm}				% bold maths
\usepackage{tabularx}		% tableaux
\usepackage{colortbl}		% tableaux en couleur
\usepackage{fontawesome}		% Fontawesome
\usepackage{environ}			% environment with command
\usepackage{fp}				% calculs pour ps-tricks
\usepackage{multido}			% pour ps tricks
\usepackage[np]{numprint}	% formattage nombre
\usepackage{tikz,tkz-tab} 			% package principal TikZ
\usepackage{pgfplots}   % axes
\usepackage{mathrsfs}    % cursives
\usepackage{calc}			% calcul taille boites
\usepackage[scaled=0.875]{helvet} % font sans serif
\usepackage{svg} % svg
\usepackage{scrextend} % local margin
\usepackage{scratch} %scratch
\usepackage{multicol} % colonnes
%\usepackage{infix-RPN,pst-func} % formule en notation polanaise inversée
\usepackage{listings}

%================================================================================================================================
%
% Réglages de base
%
%================================================================================================================================

\lstset{
language=Python,   % R code
literate=
{á}{{\'a}}1
{à}{{\`a}}1
{ã}{{\~a}}1
{é}{{\'e}}1
{è}{{\`e}}1
{ê}{{\^e}}1
{í}{{\'i}}1
{ó}{{\'o}}1
{õ}{{\~o}}1
{ú}{{\'u}}1
{ü}{{\"u}}1
{ç}{{\c{c}}}1
{~}{{ }}1
}


\definecolor{codegreen}{rgb}{0,0.6,0}
\definecolor{codegray}{rgb}{0.5,0.5,0.5}
\definecolor{codepurple}{rgb}{0.58,0,0.82}
\definecolor{backcolour}{rgb}{0.95,0.95,0.92}

\lstdefinestyle{mystyle}{
    backgroundcolor=\color{backcolour},   
    commentstyle=\color{codegreen},
    keywordstyle=\color{magenta},
    numberstyle=\tiny\color{codegray},
    stringstyle=\color{codepurple},
    basicstyle=\ttfamily\footnotesize,
    breakatwhitespace=false,         
    breaklines=true,                 
    captionpos=b,                    
    keepspaces=true,                 
    numbers=left,                    
xleftmargin=2em,
framexleftmargin=2em,            
    showspaces=false,                
    showstringspaces=false,
    showtabs=false,                  
    tabsize=2,
    upquote=true
}

\lstset{style=mystyle}


\lstset{style=mystyle}
\newcommand{\imgdir}{C:/laragon/www/newmc/assets/imgsvg/}
\newcommand{\imgsvgdir}{C:/laragon/www/newmc/assets/imgsvg/}

\definecolor{mcgris}{RGB}{220, 220, 220}% ancien~; pour compatibilité
\definecolor{mcbleu}{RGB}{52, 152, 219}
\definecolor{mcvert}{RGB}{125, 194, 70}
\definecolor{mcmauve}{RGB}{154, 0, 215}
\definecolor{mcorange}{RGB}{255, 96, 0}
\definecolor{mcturquoise}{RGB}{0, 153, 153}
\definecolor{mcrouge}{RGB}{255, 0, 0}
\definecolor{mclightvert}{RGB}{205, 234, 190}

\definecolor{gris}{RGB}{220, 220, 220}
\definecolor{bleu}{RGB}{52, 152, 219}
\definecolor{vert}{RGB}{125, 194, 70}
\definecolor{mauve}{RGB}{154, 0, 215}
\definecolor{orange}{RGB}{255, 96, 0}
\definecolor{turquoise}{RGB}{0, 153, 153}
\definecolor{rouge}{RGB}{255, 0, 0}
\definecolor{lightvert}{RGB}{205, 234, 190}
\setitemize[0]{label=\color{lightvert}  $\bullet$}

\pagestyle{fancy}
\renewcommand{\headrulewidth}{0.2pt}
\fancyhead[L]{maths-cours.fr}
\fancyhead[R]{\thepage}
\renewcommand{\footrulewidth}{0.2pt}
\fancyfoot[C]{}

\newcolumntype{C}{>{\centering\arraybackslash}X}
\newcolumntype{s}{>{\hsize=.35\hsize\arraybackslash}X}

\setlength{\parindent}{0pt}		 
\setlength{\parskip}{3mm}
\setlength{\headheight}{1cm}

\def\ebook{ebook}
\def\book{book}
\def\web{web}
\def\type{web}

\newcommand{\vect}[1]{\overrightarrow{\,\mathstrut#1\,}}

\def\Oij{$\left(\text{O}~;~\vect{\imath},~\vect{\jmath}\right)$}
\def\Oijk{$\left(\text{O}~;~\vect{\imath},~\vect{\jmath},~\vect{k}\right)$}
\def\Ouv{$\left(\text{O}~;~\vect{u},~\vect{v}\right)$}

\hypersetup{breaklinks=true, colorlinks = true, linkcolor = OliveGreen, urlcolor = OliveGreen, citecolor = OliveGreen, pdfauthor={Didier BONNEL - https://www.maths-cours.fr} } % supprime les bordures autour des liens

\renewcommand{\arg}[0]{\text{arg}}

\everymath{\displaystyle}

%================================================================================================================================
%
% Macros - Commandes
%
%================================================================================================================================

\newcommand\meta[2]{    			% Utilisé pour créer le post HTML.
	\def\titre{titre}
	\def\url{url}
	\def\arg{#1}
	\ifx\titre\arg
		\newcommand\maintitle{#2}
		\fancyhead[L]{#2}
		{\Large\sffamily \MakeUppercase{#2}}
		\vspace{1mm}\textcolor{mcvert}{\hrule}
	\fi 
	\ifx\url\arg
		\fancyfoot[L]{\href{https://www.maths-cours.fr#2}{\black \footnotesize{https://www.maths-cours.fr#2}}}
	\fi 
}


\newcommand\TitreC[1]{    		% Titre centré
     \needspace{3\baselineskip}
     \begin{center}\textbf{#1}\end{center}
}

\newcommand\newpar{    		% paragraphe
     \par
}

\newcommand\nosp {    		% commande vide (pas d'espace)
}
\newcommand{\id}[1]{} %ignore

\newcommand\boite[2]{				% Boite simple sans titre
	\vspace{5mm}
	\setlength{\fboxrule}{0.2mm}
	\setlength{\fboxsep}{5mm}	
	\fcolorbox{#1}{#1!3}{\makebox[\linewidth-2\fboxrule-2\fboxsep]{
  		\begin{minipage}[t]{\linewidth-2\fboxrule-4\fboxsep}\setlength{\parskip}{3mm}
  			 #2
  		\end{minipage}
	}}
	\vspace{5mm}
}

\newcommand\CBox[4]{				% Boites
	\vspace{5mm}
	\setlength{\fboxrule}{0.2mm}
	\setlength{\fboxsep}{5mm}
	
	\fcolorbox{#1}{#1!3}{\makebox[\linewidth-2\fboxrule-2\fboxsep]{
		\begin{minipage}[t]{1cm}\setlength{\parskip}{3mm}
	  		\textcolor{#1}{\LARGE{#2}}    
 	 	\end{minipage}  
  		\begin{minipage}[t]{\linewidth-2\fboxrule-4\fboxsep}\setlength{\parskip}{3mm}
			\raisebox{1.2mm}{\normalsize\sffamily{\textcolor{#1}{#3}}}						
  			 #4
  		\end{minipage}
	}}
	\vspace{5mm}
}

\newcommand\cadre[3]{				% Boites convertible html
	\par
	\vspace{2mm}
	\setlength{\fboxrule}{0.1mm}
	\setlength{\fboxsep}{5mm}
	\fcolorbox{#1}{white}{\makebox[\linewidth-2\fboxrule-2\fboxsep]{
  		\begin{minipage}[t]{\linewidth-2\fboxrule-4\fboxsep}\setlength{\parskip}{3mm}
			\raisebox{-2.5mm}{\sffamily \small{\textcolor{#1}{\MakeUppercase{#2}}}}		
			\par		
  			 #3
 	 		\end{minipage}
	}}
		\vspace{2mm}
	\par
}

\newcommand\bloc[3]{				% Boites convertible html sans bordure
     \needspace{2\baselineskip}
     {\sffamily \small{\textcolor{#1}{\MakeUppercase{#2}}}}    
		\par		
  			 #3
		\par
}

\newcommand\CHelp[1]{
     \CBox{Plum}{\faInfoCircle}{À RETENIR}{#1}
}

\newcommand\CUp[1]{
     \CBox{NavyBlue}{\faThumbsOUp}{EN PRATIQUE}{#1}
}

\newcommand\CInfo[1]{
     \CBox{Sepia}{\faArrowCircleRight}{REMARQUE}{#1}
}

\newcommand\CRedac[1]{
     \CBox{PineGreen}{\faEdit}{BIEN R\'EDIGER}{#1}
}

\newcommand\CError[1]{
     \CBox{Red}{\faExclamationTriangle}{ATTENTION}{#1}
}

\newcommand\TitreExo[2]{
\needspace{4\baselineskip}
 {\sffamily\large EXERCICE #1\ (\emph{#2 points})}
\vspace{5mm}
}

\newcommand\img[2]{
          \includegraphics[width=#2\paperwidth]{\imgdir#1}
}

\newcommand\imgsvg[2]{
       \begin{center}   \includegraphics[width=#2\paperwidth]{\imgsvgdir#1} \end{center}
}


\newcommand\Lien[2]{
     \href{#1}{#2 \tiny \faExternalLink}
}
\newcommand\mcLien[2]{
     \href{https~://www.maths-cours.fr/#1}{#2 \tiny \faExternalLink}
}

\newcommand{\euro}{\eurologo{}}

%================================================================================================================================
%
% Macros - Environement
%
%================================================================================================================================

\newenvironment{tex}{ %
}
{%
}

\newenvironment{indente}{ %
	\setlength\parindent{10mm}
}

{
	\setlength\parindent{0mm}
}

\newenvironment{corrige}{%
     \needspace{3\baselineskip}
     \medskip
     \textbf{\textsc{Corrigé}}
     \medskip
}
{
}

\newenvironment{extern}{%
     \begin{center}
     }
     {
     \end{center}
}

\NewEnviron{code}{%
	\par
     \boite{gray}{\texttt{%
     \BODY
     }}
     \par
}

\newenvironment{vbloc}{% boite sans cadre empeche saut de page
     \begin{minipage}[t]{\linewidth}
     }
     {
     \end{minipage}
}
\NewEnviron{h2}{%
    \needspace{3\baselineskip}
    \vspace{0.6cm}
	\noindent \MakeUppercase{\sffamily \large \BODY}
	\vspace{1mm}\textcolor{mcgris}{\hrule}\vspace{0.4cm}
	\par
}{}

\NewEnviron{h3}{%
    \needspace{3\baselineskip}
	\vspace{5mm}
	\textsc{\BODY}
	\par
}

\NewEnviron{margeneg}{ %
\begin{addmargin}[-1cm]{0cm}
\BODY
\end{addmargin}
}

\NewEnviron{html}{%
}

\begin{document}
\meta{url}{/cours/vecteurs-droites/}
\meta{pid}{338}
\meta{titre}{Vecteurs et droites}
\meta{type}{cours}
\begin{h2}1. Vecteurs et repère cartésien\end{h2}
\cadre{bleu}{Définition (Vecteurs colinéaires)}{% id="d10"
On dit que deux vecteurs non nuls $\vec{u}$ et $\vec{v}$ sont \textbf{colinéaires} s'il existe un réel $k$ tel que $\vec{v} = k\vec{u}$}
\begin{center}
     \begin{extern}%width="220" alt="Vecteurs colinéaires"
          \psset{xunit=0.5cm,yunit=0.5cm,algebraic=true,dimen=middle,linewidth=1pt}
          \begin{pspicture*}(-1,0.)(11,8)
               \psline[linecolor=blue]{->}(3,3)(6,4)
               \psline[linecolor=red]{->}(0,4)(9,7)
               \psline[linecolor=mcvert]{->}(10,3)(4,1)
               \rput[b](4.5,5.8){$\red 3\vec{u}$}
               \rput[b](4.5,3.8){$\blue \vec{u}$}
               \rput[b](7,2.3){$\color{mcvert}-2\vec{u}$}
          \end{pspicture*}
     \end{extern}
\end{center}
\begin{center}
     \textit{Vecteurs colinéaires}
\end{center}
\bloc{cyan}{Remarques}{% id="r10"
     \begin{itemize}
          \item Par convention, on considère que le vecteur nul est colinéaire est tout vecteur du plan
          \item Deux vecteurs colinéaires ont la même «direction»~;~ils ont le même sens si $k > 0$ et sont de sens contraire si $k < 0$.
     \end{itemize}
}
\cadre{bleu}{Définition}{% id="d20"
     On dit que le vecteur non nul $\vec{u}$ est un \textbf{vecteur directeur} de la droite $d$ si et seulement si il existe deux points $A$ et $B$ de $d$ tels que $\vec{u}=\overrightarrow{AB}$.
}
\begin{center}
     \begin{extern}%width="250" alt="vecteur directeur"
          \psset{xunit=0.5cm,yunit=0.5cm,algebraic=true,dimen=middle,dotstyle=o,dotsize=5pt 0,linewidth=1.2pt}
          \begin{pspicture*}(-4.,-1.84)(7.36,3.8)
               \psplot[linewidth=0.8pt]{-4.}{7.36}{(-1.--2.*x)/5.}
               \psline[linewidth=0.8pt,linecolor=red]{->}(-1.,1.)(4.,3.)
               \psline[linewidth=0.8pt,linecolor=red]{->}(-2.,-1.)(3.,1.)
               \rput[tl](0.9,3){$\red{\vec{u}}$}
               \rput[tl](5.9,3){$d$}
               \psdots[dotsize=2pt 0,dotstyle=*,linecolor=blue](-2.,-1.)
               \rput[bl](-2.5,-0.8){\blue{$A$}}
               \psdots[dotsize=2pt 0,dotstyle=*,linecolor=blue](3.,1.)
               \rput[bl](2.6,1.3){\blue{$B$}}
          \end{pspicture*}
     \end{extern}
\end{center}
\begin{center}
     \textit{Vecteur directeur}
\end{center}
\cadre{vert}{Propriété}{% id="p25"
     Trois points distincts $A, B$ et $C$ sont alignés si et seulement si les vecteurs $\overrightarrow{AB}$ et $\overrightarrow{AC}$ sont colinéaires.
}
\cadre{vert}{Propriété}{% id="p30"
     Deux droites sont parallèles si et seulement si elles ont des vecteurs directeurs colinéaires.
}
\cadre{rouge}{Théorème et définitions}{% id="t40"
     Soient $O$ un point et $\vec{i}$ et $\vec{j}$ deux vecteurs \textbf{non colinéaires} du plan.
     \par
     Le triplet $\left(O ; \vec{i}, \vec{j}\right)$ s'appelle un \textbf{repère cartésien} du plan.
     \begin{itemize}
          \item Pour tout point $M$ du plan, il existe deux réels $x$ et $y$ tels que :
          \begin{center}$\overrightarrow{OM}=x\vec{i}+y\vec{j}$\end{center}
          \item Pour tout vecteur $\vec{u}$ du plan, il existe deux réels $x$ et $y$ tels que :
          \begin{center}$\vec{u}=x\vec{i}+y\vec{j}$\end{center}
     \end{itemize}
     Le couple $\left(x ; y\right)$ s'appelle le couple de \textbf{coordonnées} du point $M$ (ou du vecteur $\vec{u}$) dans le repère $\left(O ; \vec{i}, \vec{j}\right)$
}
\begin{center}
     \begin{extern}%width="330" alt="Coordonnées dans un repère cartésien"
          \psset{xunit=1.0cm,yunit=1.0cm,algebraic=true,dimen=middle,linewidth=1pt}
          \begin{pspicture*}(-2.5,0.)(6,5)
               \psplot[linewidth=0.8pt]{-2.5}{6}{3.+2.*x}
               \psplot[linewidth=0.8pt]{-2.5}{6}{1-0.*x}
               \psline[linewidth=0.8pt,linecolor=red]{->}(-1.,1.)(5.,4.)
               \psplot[linewidth=0.8pt,linecolor=lightgray]{-2.5}{6}{(-6.--2.*x)/1.}
               \psplot[linewidth=0.8pt,linecolor=lightgray]{-2.5}{6}{(--12.-0.*x)/3.}
               \psline[linewidth=0.8pt,linecolor=blue]{->}(-1.,1.)(1.,1.)
               \psline[linewidth=0.8pt,linecolor=blue]{->}(-1.,1.)(-0.5,2)
               \rput[tl](0,0.8){$\blue{\vec{i}}$}
               \rput[tl](-1.3,1.8){$\blue{\vec{j}}$}
               \rput[tl](3.5,0.8){$x$}
               \rput[tl](0.3,4.4){$y$}
               \rput[tl](1.8,3){$\red{\vec{u}}$}
               \psdots[dotsize=1pt 0,dotstyle=*,linecolor=blue](-1.,1.)
               \rput[bl](-1.6,0.6){\blue{$O$}}
               \psdots[dotsize=1pt 0,dotstyle=*,linecolor=blue](5.,4.)
               \rput[bl](4.7,4.2){\blue{$M$}}
          \end{pspicture*}
     \end{extern}
\end{center}
\begin{center}
     \textit{Coordonnées dans un repère cartésien}
\end{center}
\bloc{cyan}{Remarque}{% id="r40"
     Dans ce chapitre, les repères utilisés ne seront pas nécessairement orthonormés.
     \par
     L'étude spécifique des repères orthonormés sera détaillée dans le chapitre «produit scalaire»
}
\cadre{vert}{Propriétés}{% id="p50"
     On se place dans un repère $\left(O ; \vec{i}, \vec{j}\right)$.
     \par
     Soient deux points $A\left(x_{A} ; y_{A}\right)$ et $B\left(x_{B} ; y_{B}\right)$, alors :
     \begin{itemize}
          \item Le vecteur $\overrightarrow{AB}$ a pour coordonnées $\left(x_{B}-x_{A} ; y_{B}-y_{A}\right)$
          \item Le milieu $M$ de $\left[AB\right]$ a pour coordonnées $M \left(\frac{x_{A}+x_{B}}{2} ; \frac{y_{A}+y_{B}}{2}\right)$
     \end{itemize}
}
\cadre{rouge}{Théorème}{% id="t50"
     Soient $\vec{u}$ et $\vec{v}$ deux vecteurs de coordonnées respectives $\left(x ; y\right)$ et $\left(x^{\prime} ; y^{\prime}\right)$ dans un repère $\left(O ; \vec{i}, \vec{j}\right)$. Les vecteurs $\vec{u}$ et $\vec{v}$ sont colinéaires si et seulement si leurs coordonnées sont proportionnelles, c'est à dire si et seulement si :
     \begin{center}$xy^{\prime}-x^{\prime}y=0$\end{center}
}
\begin{h2}2. Équations de droites\end{h2}
Dans cette partie, on se place dans un repère $\left(O ; \vec{i}, \vec{j}\right)$ (non nécessairement orthonormé).
\cadre{rouge}{Théorème}{% id="t60"
     Soit $d$ une droite passant par un point $A$ et de vecteur directeur $\vec{u}$.
     \par
     Un point $M$ appartient à la droite $d$ si et seulement si les vecteurs $\overrightarrow{AM}$ et $\vec{u}$ sont colinéaires.
}
\bloc{orange}{Exemple}{% id="e60"
     Soient le point $A\left(0;1\right)$ et le vecteur $\vec{u}\left(1;-1\right)$. Le point $M\left(x ; y\right)$ appartient à la droite passant par $A$ et de vecteur directeur $\vec{u}$ si et seulement si $\overrightarrow{AM}$ et $\vec{u}$ sont colinéaires. Or les coordonnées de $\overrightarrow{AM}$ sont $\left(x ; y-1\right)$ donc :
     \par
     $M \in  d  \Leftrightarrow x\times \left(-1\right)-\left(y-1\right)\times 1=0 \Leftrightarrow -x-y+1=0$
     \par
     Cette dernière égalité s'appelle une équation cartésienne de la droite $d$.
}
\cadre{rouge}{Théorème}{% id="t70"
     Toute droite du plan possède une \textbf{équation cartésienne} du type :
     \[  ax+by+c=0 \]
     où $a, b$ et $c$ sont trois réels.
     \par
     \textbf{Réciproquement, }l'ensemble des points $M\left(x ; y\right)$ tels que $ax+by+c=0$ où $a, b$ et $c$ sont trois réels avec $a\neq 0$ ou $b\neq 0$ est une droite.
}
\bloc{cyan}{Remarques}{% id="r70"
     \begin{itemize}
          \item Une droite possède une infinité d'équation cartésienne (il suffit de multiplier une équation par un facteur non nul pour obtenir une équation équivalente).
          \item Si $b\neq 0$ l'équation peut s'écrire :
          \par
          $ax+by+c= 0 \Leftrightarrow by=-ax-c \Leftrightarrow y=-\frac{a}{b}x-\frac{c}{b}$
          \par
          qui est de la forme $y=mx+p$ (en posant $m=-\frac{a}{b}$ et $p=-\frac{c}{b}$).
          \par
          Cette forme est appelée \textbf{équation réduite} de la droite.
          \par
          Ce cas correspond à une droite qui n'est pas parallèle. à l'axe des ordonnées.
          \item Si $b=0$ et $a\neq 0$ l'équation peut s'écrire :
          \par
          $ax+c= 0 \Leftrightarrow ax=-c \Leftrightarrow x=-\frac{c}{a}$
          \par
          qui est du type $x=k$ (en posant $k=-\frac{c}{a}$)
          \par
          Ce cas correspond à une droite qui est parallèle. à l'axe des ordonnées.
     \end{itemize}
}
\cadre{vert}{Propriété}{% id="p80"
     Soit $d$ une droite d'équation $ax+by+c=0$.
     \par
     Le vecteur $\vec{u}$ de coordonnées $\left(-b ; a\right)$ est un vecteur directeur de la droite $d$.
}
\bloc{cyan}{Démonstration}{% id="r80"
     Voir exercice~: \mcLien{/exercices/droites-equations-vecteur-directeur-140818}{«~Equation cartésienne - Vecteur directeur~»}.
}

\end{document}