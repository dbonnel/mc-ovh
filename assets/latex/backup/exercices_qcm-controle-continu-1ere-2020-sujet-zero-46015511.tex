\documentclass[a4paper]{article}

%================================================================================================================================
%
% Packages
%
%================================================================================================================================

\usepackage[T1]{fontenc} 	% pour caractères accentués
\usepackage[utf8]{inputenc}  % encodage utf8
\usepackage[french]{babel}	% langue : français
\usepackage{fourier}			% caractères plus lisibles
\usepackage[dvipsnames]{xcolor} % couleurs
\usepackage{fancyhdr}		% réglage header footer
\usepackage{needspace}		% empêcher sauts de page mal placés
\usepackage{graphicx}		% pour inclure des graphiques
\usepackage{enumitem,cprotect}		% personnalise les listes d'items (nécessaire pour ol, al ...)
\usepackage{hyperref}		% Liens hypertexte
\usepackage{pstricks,pst-all,pst-node,pstricks-add,pst-math,pst-plot,pst-tree,pst-eucl} % pstricks
\usepackage[a4paper,includeheadfoot,top=2cm,left=3cm, bottom=2cm,right=3cm]{geometry} % marges etc.
\usepackage{comment}			% commentaires multilignes
\usepackage{amsmath,environ} % maths (matrices, etc.)
\usepackage{amssymb,makeidx}
\usepackage{bm}				% bold maths
\usepackage{tabularx}		% tableaux
\usepackage{colortbl}		% tableaux en couleur
\usepackage{fontawesome}		% Fontawesome
\usepackage{environ}			% environment with command
\usepackage{fp}				% calculs pour ps-tricks
\usepackage{multido}			% pour ps tricks
\usepackage[np]{numprint}	% formattage nombre
\usepackage{tikz,tkz-tab} 			% package principal TikZ
\usepackage{pgfplots}   % axes
\usepackage{mathrsfs}    % cursives
\usepackage{calc}			% calcul taille boites
\usepackage[scaled=0.875]{helvet} % font sans serif
\usepackage{svg} % svg
\usepackage{scrextend} % local margin
\usepackage{scratch} %scratch
\usepackage{multicol} % colonnes
%\usepackage{infix-RPN,pst-func} % formule en notation polanaise inversée
\usepackage{listings}

%================================================================================================================================
%
% Réglages de base
%
%================================================================================================================================

\lstset{
language=Python,   % R code
literate=
{á}{{\'a}}1
{à}{{\`a}}1
{ã}{{\~a}}1
{é}{{\'e}}1
{è}{{\`e}}1
{ê}{{\^e}}1
{í}{{\'i}}1
{ó}{{\'o}}1
{õ}{{\~o}}1
{ú}{{\'u}}1
{ü}{{\"u}}1
{ç}{{\c{c}}}1
{~}{{ }}1
}


\definecolor{codegreen}{rgb}{0,0.6,0}
\definecolor{codegray}{rgb}{0.5,0.5,0.5}
\definecolor{codepurple}{rgb}{0.58,0,0.82}
\definecolor{backcolour}{rgb}{0.95,0.95,0.92}

\lstdefinestyle{mystyle}{
    backgroundcolor=\color{backcolour},   
    commentstyle=\color{codegreen},
    keywordstyle=\color{magenta},
    numberstyle=\tiny\color{codegray},
    stringstyle=\color{codepurple},
    basicstyle=\ttfamily\footnotesize,
    breakatwhitespace=false,         
    breaklines=true,                 
    captionpos=b,                    
    keepspaces=true,                 
    numbers=left,                    
xleftmargin=2em,
framexleftmargin=2em,            
    showspaces=false,                
    showstringspaces=false,
    showtabs=false,                  
    tabsize=2,
    upquote=true
}

\lstset{style=mystyle}


\lstset{style=mystyle}
\newcommand{\imgdir}{C:/laragon/www/newmc/assets/imgsvg/}
\newcommand{\imgsvgdir}{C:/laragon/www/newmc/assets/imgsvg/}

\definecolor{mcgris}{RGB}{220, 220, 220}% ancien~; pour compatibilité
\definecolor{mcbleu}{RGB}{52, 152, 219}
\definecolor{mcvert}{RGB}{125, 194, 70}
\definecolor{mcmauve}{RGB}{154, 0, 215}
\definecolor{mcorange}{RGB}{255, 96, 0}
\definecolor{mcturquoise}{RGB}{0, 153, 153}
\definecolor{mcrouge}{RGB}{255, 0, 0}
\definecolor{mclightvert}{RGB}{205, 234, 190}

\definecolor{gris}{RGB}{220, 220, 220}
\definecolor{bleu}{RGB}{52, 152, 219}
\definecolor{vert}{RGB}{125, 194, 70}
\definecolor{mauve}{RGB}{154, 0, 215}
\definecolor{orange}{RGB}{255, 96, 0}
\definecolor{turquoise}{RGB}{0, 153, 153}
\definecolor{rouge}{RGB}{255, 0, 0}
\definecolor{lightvert}{RGB}{205, 234, 190}
\setitemize[0]{label=\color{lightvert}  $\bullet$}

\pagestyle{fancy}
\renewcommand{\headrulewidth}{0.2pt}
\fancyhead[L]{maths-cours.fr}
\fancyhead[R]{\thepage}
\renewcommand{\footrulewidth}{0.2pt}
\fancyfoot[C]{}

\newcolumntype{C}{>{\centering\arraybackslash}X}
\newcolumntype{s}{>{\hsize=.35\hsize\arraybackslash}X}

\setlength{\parindent}{0pt}		 
\setlength{\parskip}{3mm}
\setlength{\headheight}{1cm}

\def\ebook{ebook}
\def\book{book}
\def\web{web}
\def\type{web}

\newcommand{\vect}[1]{\overrightarrow{\,\mathstrut#1\,}}

\def\Oij{$\left(\text{O}~;~\vect{\imath},~\vect{\jmath}\right)$}
\def\Oijk{$\left(\text{O}~;~\vect{\imath},~\vect{\jmath},~\vect{k}\right)$}
\def\Ouv{$\left(\text{O}~;~\vect{u},~\vect{v}\right)$}

\hypersetup{breaklinks=true, colorlinks = true, linkcolor = OliveGreen, urlcolor = OliveGreen, citecolor = OliveGreen, pdfauthor={Didier BONNEL - https://www.maths-cours.fr} } % supprime les bordures autour des liens

\renewcommand{\arg}[0]{\text{arg}}

\everymath{\displaystyle}

%================================================================================================================================
%
% Macros - Commandes
%
%================================================================================================================================

\newcommand\meta[2]{    			% Utilisé pour créer le post HTML.
	\def\titre{titre}
	\def\url{url}
	\def\arg{#1}
	\ifx\titre\arg
		\newcommand\maintitle{#2}
		\fancyhead[L]{#2}
		{\Large\sffamily \MakeUppercase{#2}}
		\vspace{1mm}\textcolor{mcvert}{\hrule}
	\fi 
	\ifx\url\arg
		\fancyfoot[L]{\href{https://www.maths-cours.fr#2}{\black \footnotesize{https://www.maths-cours.fr#2}}}
	\fi 
}


\newcommand\TitreC[1]{    		% Titre centré
     \needspace{3\baselineskip}
     \begin{center}\textbf{#1}\end{center}
}

\newcommand\newpar{    		% paragraphe
     \par
}

\newcommand\nosp {    		% commande vide (pas d'espace)
}
\newcommand{\id}[1]{} %ignore

\newcommand\boite[2]{				% Boite simple sans titre
	\vspace{5mm}
	\setlength{\fboxrule}{0.2mm}
	\setlength{\fboxsep}{5mm}	
	\fcolorbox{#1}{#1!3}{\makebox[\linewidth-2\fboxrule-2\fboxsep]{
  		\begin{minipage}[t]{\linewidth-2\fboxrule-4\fboxsep}\setlength{\parskip}{3mm}
  			 #2
  		\end{minipage}
	}}
	\vspace{5mm}
}

\newcommand\CBox[4]{				% Boites
	\vspace{5mm}
	\setlength{\fboxrule}{0.2mm}
	\setlength{\fboxsep}{5mm}
	
	\fcolorbox{#1}{#1!3}{\makebox[\linewidth-2\fboxrule-2\fboxsep]{
		\begin{minipage}[t]{1cm}\setlength{\parskip}{3mm}
	  		\textcolor{#1}{\LARGE{#2}}    
 	 	\end{minipage}  
  		\begin{minipage}[t]{\linewidth-2\fboxrule-4\fboxsep}\setlength{\parskip}{3mm}
			\raisebox{1.2mm}{\normalsize\sffamily{\textcolor{#1}{#3}}}						
  			 #4
  		\end{minipage}
	}}
	\vspace{5mm}
}

\newcommand\cadre[3]{				% Boites convertible html
	\par
	\vspace{2mm}
	\setlength{\fboxrule}{0.1mm}
	\setlength{\fboxsep}{5mm}
	\fcolorbox{#1}{white}{\makebox[\linewidth-2\fboxrule-2\fboxsep]{
  		\begin{minipage}[t]{\linewidth-2\fboxrule-4\fboxsep}\setlength{\parskip}{3mm}
			\raisebox{-2.5mm}{\sffamily \small{\textcolor{#1}{\MakeUppercase{#2}}}}		
			\par		
  			 #3
 	 		\end{minipage}
	}}
		\vspace{2mm}
	\par
}

\newcommand\bloc[3]{				% Boites convertible html sans bordure
     \needspace{2\baselineskip}
     {\sffamily \small{\textcolor{#1}{\MakeUppercase{#2}}}}    
		\par		
  			 #3
		\par
}

\newcommand\CHelp[1]{
     \CBox{Plum}{\faInfoCircle}{À RETENIR}{#1}
}

\newcommand\CUp[1]{
     \CBox{NavyBlue}{\faThumbsOUp}{EN PRATIQUE}{#1}
}

\newcommand\CInfo[1]{
     \CBox{Sepia}{\faArrowCircleRight}{REMARQUE}{#1}
}

\newcommand\CRedac[1]{
     \CBox{PineGreen}{\faEdit}{BIEN R\'EDIGER}{#1}
}

\newcommand\CError[1]{
     \CBox{Red}{\faExclamationTriangle}{ATTENTION}{#1}
}

\newcommand\TitreExo[2]{
\needspace{4\baselineskip}
 {\sffamily\large EXERCICE #1\ (\emph{#2 points})}
\vspace{5mm}
}

\newcommand\img[2]{
          \includegraphics[width=#2\paperwidth]{\imgdir#1}
}

\newcommand\imgsvg[2]{
       \begin{center}   \includegraphics[width=#2\paperwidth]{\imgsvgdir#1} \end{center}
}


\newcommand\Lien[2]{
     \href{#1}{#2 \tiny \faExternalLink}
}
\newcommand\mcLien[2]{
     \href{https~://www.maths-cours.fr/#1}{#2 \tiny \faExternalLink}
}

\newcommand{\euro}{\eurologo{}}

%================================================================================================================================
%
% Macros - Environement
%
%================================================================================================================================

\newenvironment{tex}{ %
}
{%
}

\newenvironment{indente}{ %
	\setlength\parindent{10mm}
}

{
	\setlength\parindent{0mm}
}

\newenvironment{corrige}{%
     \needspace{3\baselineskip}
     \medskip
     \textbf{\textsc{Corrigé}}
     \medskip
}
{
}

\newenvironment{extern}{%
     \begin{center}
     }
     {
     \end{center}
}

\NewEnviron{code}{%
	\par
     \boite{gray}{\texttt{%
     \BODY
     }}
     \par
}

\newenvironment{vbloc}{% boite sans cadre empeche saut de page
     \begin{minipage}[t]{\linewidth}
     }
     {
     \end{minipage}
}
\NewEnviron{h2}{%
    \needspace{3\baselineskip}
    \vspace{0.6cm}
	\noindent \MakeUppercase{\sffamily \large \BODY}
	\vspace{1mm}\textcolor{mcgris}{\hrule}\vspace{0.4cm}
	\par
}{}

\NewEnviron{h3}{%
    \needspace{3\baselineskip}
	\vspace{5mm}
	\textsc{\BODY}
	\par
}

\NewEnviron{margeneg}{ %
\begin{addmargin}[-1cm]{0cm}
\BODY
\end{addmargin}
}

\NewEnviron{html}{%
}

\begin{document}
\meta{url}{/exercices/qcm-controle-continu-1ere-2020-sujet-zero/}
\meta{pid}{11189}
\meta{titre}{QCM - Contrôle continu 1ère - 2020 - Sujet zéro}
\meta{type}{exercices}
%
Ce QCM comprend 5 questions.
\newpar
Pour chacune des questions, une seule des 4 réponses proposées est correcte.
\newpar
Les questions sont indépendantes.
\newpar
Pour chaque question, indiquer le numéro de la question et recopier sur la copie la lettre correspondant à la réponse choisie.
\newpar
Aucune justification n'est demandée mais il peut être nécessaire d'effectuer des recherches au brouillon pour aider à déterminer votre réponse.
\newpar
Chaque réponse correcte rapporte 1 point. Une réponse incorrecte ou une question sans réponse n'apporte ni ne retire de point.
\newpar
\begin{h2}Question 1\end{h2}
Pour tout réel $x$,  $\left( \text{e}^{ x } \right)^{ 3 }$ est égale à :
\newpar
\begin{enumerate}[label=\alph*.]
     \item
     $\text{e}^{ x } \times \text{e}^{ 3 }$
     \item
     $\text{e}^{ x+3 }$
     \item
     $\text{e}^{ 3x }$
     \item
     $\text{e}^{ x{}^{ 3 } }$
\end{enumerate}
\begin{h2}Question 2\end{h2}
Pour tout réel $x$,  $\cos{ ( x+ \pi  ) }$ est égale à :
\newpar
\begin{enumerate}[label=\alph*.]
     \item
     $\sin{ x }$
     \item
     $ - \cos{ x }$
     \item
     $\cos{ x }$
     \item
     $ - \sin{ x }$
\end{enumerate}
\begin{h2}Question 3\end{h2}
On souhaite modéliser le niveau de la mer par une suite $\left( U_{ n } \right)$ de façon que $U_{ 0 }$ représente le niveau de la mer, en millimètres, en 2003 et que $U_{ n }$ représente le niveau de la mer, en millimètres,  $n$  années après 2003.
\newpar
Selon le site \textit{notre-planete.info}, on constate une hausse assez rapide du niveau de la mer, qu'on estime à 3,3 mm par an depuis 2003.
\newpar
Pour traduire ce constat, la suite $\left( U_{ n } \right)$ doit être~:
\newpar
\begin{enumerate}[label=\alph*.]
     \item
     Une suite géométrique de raison 3,3.
     \item
     Une suite géométrique de raison 1,033.
     \item
     Une suite arithmétique de raison 1,033.
     \item
     Une suite arithmétique de raison 3,3.
\end{enumerate}
\begin{h2}Question 4\end{h2}
Les figures ci-dessous représentent 4 polynômes du second degré dans un repère orthonormé et le signe de leur discriminant  $ \Delta $ .
\newpar
Parmi ces propositions, laquelle est juste ?
\newpar
\begin{enumerate}[label=\alph*.]
     \item
     \begin{center}
          \begin{extern} %width="400" alt="Parabole 1"
               \resizebox{8cm}{!}{
                    %
                    \begin{tikzpicture}[line cap=round,line join=round,>=triangle 45,x=1.0cm,y=1.0cm]
                         \begin{axis}[
                              x=1.0cm,y=1.0cm,
                              axis lines=middle,
                              ymajorgrids=true,
                              xmajorgrids=true,
                              xmin=-3.0,
                              xmax=5.0,
                              ymin=-3.0,
                              ymax=3.0,
                              xtick={-3.0,-2.0,...,5.0},
                              ytick={-3.0,-2.0,...,3.0},]
                              \clip(-3.,-3.) rectangle (5.,3.);
                              \draw[line width=2.pt,color=blue,smooth,samples=100,domain=-3.0:5.0] plot(\x,{0-0.45*((\x)-1.5)^(2.0)+2.0});
                         \end{axis}
                    \end{tikzpicture}
               }
          \end{extern}
     \end{center}
     \begin{center}
          $  \Delta >0$
     \end{center}
     \medbreak
     \item
     \begin{center}
          \begin{extern} %width="400" alt=""
               \resizebox{8cm}{!}{
                    %
                    \begin{tikzpicture}[line cap=round,line join=round,>=triangle 45,x=1.0cm,y=1.0cm]
                         \begin{axis}[
                              x=1.0cm,y=1.0cm,
                              axis lines=middle,
                              ymajorgrids=true,
                              xmajorgrids=true,
                              xmin=-5.0,
                              xmax=5.0,
                              ymin=-5.0,
                              ymax=2.0,
                              xtick={-5.0,-4.0,...,5.0},
                              ytick={-5.0,-4.0,...,2.0},]
                              \clip(-5.,-5.) rectangle (5.,2.);
                              \draw[line width=1.pt,color=blue,smooth,samples=100,domain=-5.0:5.0] plot(\x,{(\x)^(2.0)+4.0*(\x)});
                         \end{axis}
                    \end{tikzpicture}
               }
          \end{extern}
     \end{center}
     \begin{center}
          $ \Delta =0$
     \end{center}
     \medbreak
     \item
     \begin{center}
          \begin{extern} %width="400" alt=""
               \resizebox{8cm}{!}{
                    %
                    \begin{tikzpicture}[line cap=round,line join=round,>=triangle 45,x=1.0cm,y=1.0cm]
                         \begin{axis}[
                              x=1.0cm,y=1.0cm,
                              axis lines=middle,
                              ymajorgrids=true,
                              xmajorgrids=true,
                              xmin=-4.0,
                              xmax=4.0,
                              ymin=-0.5,
                              ymax=6.0,
                              xtick={-4.0,-3.0,...,4.0},
                              ytick={-0.0,1.0,...,6.0},]
                              \clip(-4.,-0.5) rectangle (4.,6.);
                              \draw[line width=1.pt,color=blue,smooth,samples=100,domain=-4.0:4.0] plot(\x,{(\x)^(2.0)+1.0});
                         \end{axis}
                    \end{tikzpicture}
               }
          \end{extern}
     \end{center}
     \begin{center}
          $ \Delta >0$
     \end{center}
     \medbreak
     \item
     \begin{center}
          \begin{extern} %width="400" alt=""
               \resizebox{8cm}{!}{
                    %
                    \begin{tikzpicture}[line cap=round,line join=round,>=triangle 45,x=1.0cm,y=0.1cm]
                         \begin{axis}[
                              x=1.0cm,y=0.1cm,
                              axis lines=middle,
                              ymajorgrids=true,
                              xmajorgrids=true,
                              xmin=-2.0,
                              xmax=6.0,
                              ymin=-40.0,
                              ymax=10.0,
                              xtick={-2.0,-1.0,...,6.0},
                              ytick={-40.0,-30.0,...,10.0},]
                              \clip(-2.,-40.) rectangle (6.,10.);
                              \draw[line width=1.pt,color=blue,smooth,samples=100,domain=-2.0:6.0] plot(\x,{0-5.0*((\x)-2.0)^(2.0)});
                         \end{axis}
                    \end{tikzpicture}
               }
          \end{extern}
     \end{center}
     \begin{center}
          $ \Delta <0$
     \end{center}
     \newpar
\end{enumerate}
\begin{h2}Question 5\end{h2}
Le plan est rapporté à un repère orthonormé.
\newpar
$D$ est une droite dont une équation cartésienne est $2x-y+3=0$.
\newpar
Parmi ces propositions, laquelle est juste ?
\newpar
\begin{enumerate}[label=\alph*.]
     \item
     La droite $D$ passe par le point $A$ de coordonnées $( 2;1 )$
     \item
     La droite $D$ est dirigée par le vecteur de coordonnées $(  - 1;2 )$
     \item
     Le vecteur de coordonnées $( 2; - 1 )$ est normal à la droite $D$
     \item
     Le point d'intersection de la droite  $D$ avec l'axe des abscisses a comme coordonnées $( 0;3 ).$
\end{enumerate}
\begin{corrige}
     \cadre{vert}{Remarque}{ % id=r010
     Bien qu'aucun justificatif ne soit demandé, une explication est fournie pour aider les élèves.}
     % fin propriété
     \begin{h2}Question 1\end{h2}
     \textbf{Réponse correcte~: c.}
     \newpar
     En effet, on utilise la formule~:
     \[
     \left( \text{e}^{ a } \right){}^{ b }=\text{e}^{ ab }
     \]
     \begin{h2}Question 2\end{h2}
     \textbf{Réponse correcte~: b.}
     \newpar
     C'est une formule du cours que l'on peut retrouver à l'aide d'un cercle trigonométrique.
     \begin{h2}Question 3\end{h2}
     \textbf{Réponse correcte~: d.}
     \newpar
     Puisque l'on \textbf{ajoute} 3,3 mm par an, la suite $( U_{ n } )$ vérifie la relation de récurrence~:
     \newpar
     $U_{ n+1 }=U_{ n }+3,3$
     \newpar
     ce qui est la \mcLien{https://www.maths-cours.fr/cours/suites-geometriques/\#d10}{formule caractéristique d'une suite arithmétique} de raison $r=3,3.$
     \begin{h2}Question 4\end{h2}
     \textbf{Réponse correcte~: a.}
     \newpar
     \begin{enumerate}[label=\alph*.]
          \item
          La courbe possède deux points d'intersection avec l'axe des abscisses. Le polynôme admet donc  2 racines et son discriminant est donc bien strictement positif.
          \item
          Là encore, la courbe possède deux points d'intersection avec l'axe des abscisses~; le discriminant est strictement positif donc, il n'est pas nul.
          \item
          La courbe ne possède pas de point d'intersection avec l'axe des abscisses~; le discriminant du polynôme est donc strictement négatif.
          \item
          La courbe est tangente à l'axe des abscisses~; le polynôme admet donc une unique racine~; son discriminant est donc égal à zéro.
     \end{enumerate}
     \begin{h2}Question 5\end{h2}
     \textbf{Réponse correcte~: c.}
     \begin{enumerate}[label=\alph*.]
          \item
          Le couple $\left( 2~;~1 \right)$ ne vérifie pas l'équation  $2x - y+3=0$~; en effet~:
          \newpar
          $2 \times 2 - 1+3 \neq 0$
          \item
          La droite $D$ est dirigé par un vecteur de coordonnées $(  1~;~2 )$ qui n'est pas colinéaire au vecteur de coordonnées  $(  - 1~;~2 ).$
          \item
          Cette réponse est exacte. En effet, le vecteur de coordonnées  $( a~;~b )$ est normal à la droite d'équation $ax+by+c=0.$
          \item
          Le point de coordonnées $( 0~;~3 )$ appartient bien à la droite $D$~; toutefois, ce point n'est pas situé sur l'axe des abscisses mais sur l'axe des ordonnées.
     \end{enumerate}
\end{corrige}

\end{document}