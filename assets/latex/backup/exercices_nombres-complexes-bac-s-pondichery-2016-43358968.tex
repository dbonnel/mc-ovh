\documentclass[a4paper]{article}

%================================================================================================================================
%
% Packages
%
%================================================================================================================================

\usepackage[T1]{fontenc} 	% pour caractères accentués
\usepackage[utf8]{inputenc}  % encodage utf8
\usepackage[french]{babel}	% langue : français
\usepackage{fourier}			% caractères plus lisibles
\usepackage[dvipsnames]{xcolor} % couleurs
\usepackage{fancyhdr}		% réglage header footer
\usepackage{needspace}		% empêcher sauts de page mal placés
\usepackage{graphicx}		% pour inclure des graphiques
\usepackage{enumitem,cprotect}		% personnalise les listes d'items (nécessaire pour ol, al ...)
\usepackage{hyperref}		% Liens hypertexte
\usepackage{pstricks,pst-all,pst-node,pstricks-add,pst-math,pst-plot,pst-tree,pst-eucl} % pstricks
\usepackage[a4paper,includeheadfoot,top=2cm,left=3cm, bottom=2cm,right=3cm]{geometry} % marges etc.
\usepackage{comment}			% commentaires multilignes
\usepackage{amsmath,environ} % maths (matrices, etc.)
\usepackage{amssymb,makeidx}
\usepackage{bm}				% bold maths
\usepackage{tabularx}		% tableaux
\usepackage{colortbl}		% tableaux en couleur
\usepackage{fontawesome}		% Fontawesome
\usepackage{environ}			% environment with command
\usepackage{fp}				% calculs pour ps-tricks
\usepackage{multido}			% pour ps tricks
\usepackage[np]{numprint}	% formattage nombre
\usepackage{tikz,tkz-tab} 			% package principal TikZ
\usepackage{pgfplots}   % axes
\usepackage{mathrsfs}    % cursives
\usepackage{calc}			% calcul taille boites
\usepackage[scaled=0.875]{helvet} % font sans serif
\usepackage{svg} % svg
\usepackage{scrextend} % local margin
\usepackage{scratch} %scratch
\usepackage{multicol} % colonnes
%\usepackage{infix-RPN,pst-func} % formule en notation polanaise inversée
\usepackage{listings}

%================================================================================================================================
%
% Réglages de base
%
%================================================================================================================================

\lstset{
language=Python,   % R code
literate=
{á}{{\'a}}1
{à}{{\`a}}1
{ã}{{\~a}}1
{é}{{\'e}}1
{è}{{\`e}}1
{ê}{{\^e}}1
{í}{{\'i}}1
{ó}{{\'o}}1
{õ}{{\~o}}1
{ú}{{\'u}}1
{ü}{{\"u}}1
{ç}{{\c{c}}}1
{~}{{ }}1
}


\definecolor{codegreen}{rgb}{0,0.6,0}
\definecolor{codegray}{rgb}{0.5,0.5,0.5}
\definecolor{codepurple}{rgb}{0.58,0,0.82}
\definecolor{backcolour}{rgb}{0.95,0.95,0.92}

\lstdefinestyle{mystyle}{
    backgroundcolor=\color{backcolour},   
    commentstyle=\color{codegreen},
    keywordstyle=\color{magenta},
    numberstyle=\tiny\color{codegray},
    stringstyle=\color{codepurple},
    basicstyle=\ttfamily\footnotesize,
    breakatwhitespace=false,         
    breaklines=true,                 
    captionpos=b,                    
    keepspaces=true,                 
    numbers=left,                    
xleftmargin=2em,
framexleftmargin=2em,            
    showspaces=false,                
    showstringspaces=false,
    showtabs=false,                  
    tabsize=2,
    upquote=true
}

\lstset{style=mystyle}


\lstset{style=mystyle}
\newcommand{\imgdir}{C:/laragon/www/newmc/assets/imgsvg/}
\newcommand{\imgsvgdir}{C:/laragon/www/newmc/assets/imgsvg/}

\definecolor{mcgris}{RGB}{220, 220, 220}% ancien~; pour compatibilité
\definecolor{mcbleu}{RGB}{52, 152, 219}
\definecolor{mcvert}{RGB}{125, 194, 70}
\definecolor{mcmauve}{RGB}{154, 0, 215}
\definecolor{mcorange}{RGB}{255, 96, 0}
\definecolor{mcturquoise}{RGB}{0, 153, 153}
\definecolor{mcrouge}{RGB}{255, 0, 0}
\definecolor{mclightvert}{RGB}{205, 234, 190}

\definecolor{gris}{RGB}{220, 220, 220}
\definecolor{bleu}{RGB}{52, 152, 219}
\definecolor{vert}{RGB}{125, 194, 70}
\definecolor{mauve}{RGB}{154, 0, 215}
\definecolor{orange}{RGB}{255, 96, 0}
\definecolor{turquoise}{RGB}{0, 153, 153}
\definecolor{rouge}{RGB}{255, 0, 0}
\definecolor{lightvert}{RGB}{205, 234, 190}
\setitemize[0]{label=\color{lightvert}  $\bullet$}

\pagestyle{fancy}
\renewcommand{\headrulewidth}{0.2pt}
\fancyhead[L]{maths-cours.fr}
\fancyhead[R]{\thepage}
\renewcommand{\footrulewidth}{0.2pt}
\fancyfoot[C]{}

\newcolumntype{C}{>{\centering\arraybackslash}X}
\newcolumntype{s}{>{\hsize=.35\hsize\arraybackslash}X}

\setlength{\parindent}{0pt}		 
\setlength{\parskip}{3mm}
\setlength{\headheight}{1cm}

\def\ebook{ebook}
\def\book{book}
\def\web{web}
\def\type{web}

\newcommand{\vect}[1]{\overrightarrow{\,\mathstrut#1\,}}

\def\Oij{$\left(\text{O}~;~\vect{\imath},~\vect{\jmath}\right)$}
\def\Oijk{$\left(\text{O}~;~\vect{\imath},~\vect{\jmath},~\vect{k}\right)$}
\def\Ouv{$\left(\text{O}~;~\vect{u},~\vect{v}\right)$}

\hypersetup{breaklinks=true, colorlinks = true, linkcolor = OliveGreen, urlcolor = OliveGreen, citecolor = OliveGreen, pdfauthor={Didier BONNEL - https://www.maths-cours.fr} } % supprime les bordures autour des liens

\renewcommand{\arg}[0]{\text{arg}}

\everymath{\displaystyle}

%================================================================================================================================
%
% Macros - Commandes
%
%================================================================================================================================

\newcommand\meta[2]{    			% Utilisé pour créer le post HTML.
	\def\titre{titre}
	\def\url{url}
	\def\arg{#1}
	\ifx\titre\arg
		\newcommand\maintitle{#2}
		\fancyhead[L]{#2}
		{\Large\sffamily \MakeUppercase{#2}}
		\vspace{1mm}\textcolor{mcvert}{\hrule}
	\fi 
	\ifx\url\arg
		\fancyfoot[L]{\href{https://www.maths-cours.fr#2}{\black \footnotesize{https://www.maths-cours.fr#2}}}
	\fi 
}


\newcommand\TitreC[1]{    		% Titre centré
     \needspace{3\baselineskip}
     \begin{center}\textbf{#1}\end{center}
}

\newcommand\newpar{    		% paragraphe
     \par
}

\newcommand\nosp {    		% commande vide (pas d'espace)
}
\newcommand{\id}[1]{} %ignore

\newcommand\boite[2]{				% Boite simple sans titre
	\vspace{5mm}
	\setlength{\fboxrule}{0.2mm}
	\setlength{\fboxsep}{5mm}	
	\fcolorbox{#1}{#1!3}{\makebox[\linewidth-2\fboxrule-2\fboxsep]{
  		\begin{minipage}[t]{\linewidth-2\fboxrule-4\fboxsep}\setlength{\parskip}{3mm}
  			 #2
  		\end{minipage}
	}}
	\vspace{5mm}
}

\newcommand\CBox[4]{				% Boites
	\vspace{5mm}
	\setlength{\fboxrule}{0.2mm}
	\setlength{\fboxsep}{5mm}
	
	\fcolorbox{#1}{#1!3}{\makebox[\linewidth-2\fboxrule-2\fboxsep]{
		\begin{minipage}[t]{1cm}\setlength{\parskip}{3mm}
	  		\textcolor{#1}{\LARGE{#2}}    
 	 	\end{minipage}  
  		\begin{minipage}[t]{\linewidth-2\fboxrule-4\fboxsep}\setlength{\parskip}{3mm}
			\raisebox{1.2mm}{\normalsize\sffamily{\textcolor{#1}{#3}}}						
  			 #4
  		\end{minipage}
	}}
	\vspace{5mm}
}

\newcommand\cadre[3]{				% Boites convertible html
	\par
	\vspace{2mm}
	\setlength{\fboxrule}{0.1mm}
	\setlength{\fboxsep}{5mm}
	\fcolorbox{#1}{white}{\makebox[\linewidth-2\fboxrule-2\fboxsep]{
  		\begin{minipage}[t]{\linewidth-2\fboxrule-4\fboxsep}\setlength{\parskip}{3mm}
			\raisebox{-2.5mm}{\sffamily \small{\textcolor{#1}{\MakeUppercase{#2}}}}		
			\par		
  			 #3
 	 		\end{minipage}
	}}
		\vspace{2mm}
	\par
}

\newcommand\bloc[3]{				% Boites convertible html sans bordure
     \needspace{2\baselineskip}
     {\sffamily \small{\textcolor{#1}{\MakeUppercase{#2}}}}    
		\par		
  			 #3
		\par
}

\newcommand\CHelp[1]{
     \CBox{Plum}{\faInfoCircle}{À RETENIR}{#1}
}

\newcommand\CUp[1]{
     \CBox{NavyBlue}{\faThumbsOUp}{EN PRATIQUE}{#1}
}

\newcommand\CInfo[1]{
     \CBox{Sepia}{\faArrowCircleRight}{REMARQUE}{#1}
}

\newcommand\CRedac[1]{
     \CBox{PineGreen}{\faEdit}{BIEN R\'EDIGER}{#1}
}

\newcommand\CError[1]{
     \CBox{Red}{\faExclamationTriangle}{ATTENTION}{#1}
}

\newcommand\TitreExo[2]{
\needspace{4\baselineskip}
 {\sffamily\large EXERCICE #1\ (\emph{#2 points})}
\vspace{5mm}
}

\newcommand\img[2]{
          \includegraphics[width=#2\paperwidth]{\imgdir#1}
}

\newcommand\imgsvg[2]{
       \begin{center}   \includegraphics[width=#2\paperwidth]{\imgsvgdir#1} \end{center}
}


\newcommand\Lien[2]{
     \href{#1}{#2 \tiny \faExternalLink}
}
\newcommand\mcLien[2]{
     \href{https~://www.maths-cours.fr/#1}{#2 \tiny \faExternalLink}
}

\newcommand{\euro}{\eurologo{}}

%================================================================================================================================
%
% Macros - Environement
%
%================================================================================================================================

\newenvironment{tex}{ %
}
{%
}

\newenvironment{indente}{ %
	\setlength\parindent{10mm}
}

{
	\setlength\parindent{0mm}
}

\newenvironment{corrige}{%
     \needspace{3\baselineskip}
     \medskip
     \textbf{\textsc{Corrigé}}
     \medskip
}
{
}

\newenvironment{extern}{%
     \begin{center}
     }
     {
     \end{center}
}

\NewEnviron{code}{%
	\par
     \boite{gray}{\texttt{%
     \BODY
     }}
     \par
}

\newenvironment{vbloc}{% boite sans cadre empeche saut de page
     \begin{minipage}[t]{\linewidth}
     }
     {
     \end{minipage}
}
\NewEnviron{h2}{%
    \needspace{3\baselineskip}
    \vspace{0.6cm}
	\noindent \MakeUppercase{\sffamily \large \BODY}
	\vspace{1mm}\textcolor{mcgris}{\hrule}\vspace{0.4cm}
	\par
}{}

\NewEnviron{h3}{%
    \needspace{3\baselineskip}
	\vspace{5mm}
	\textsc{\BODY}
	\par
}

\NewEnviron{margeneg}{ %
\begin{addmargin}[-1cm]{0cm}
\BODY
\end{addmargin}
}

\NewEnviron{html}{%
}

\begin{document}
\meta{url}{/exercices/nombres-complexes-bac-s-pondichery-2016/}
\meta{pid}{4047}
\meta{titre}{Nombres complexes – Bac S Pondichéry 2016}
\meta{type}{exercices}
%
\begin{h2}Exercice 2 - 3 points\end{h2}
\par
\textbf{Commun  à tous les candidats}
\par
L'objectif de cet exercice est de trouver une méthode pour construire à la règle et au compas un pentagone régulier.
\par
Dans le plan complexe muni d'un repère orthonormé direct $(O~;~\vec{u},\vec{v})$, on considère le pentagone régulier $A_0A_1A_2A_3A_4$, de centre $O$ tel que $\overrightarrow{OA_0} = \vec{u}$.

\begin{center}
\imgsvg{nombres-complexes-bac-s-pondichery-2016-1}{0.3}% alt="pentagone régulier" style="width:40rem"
\end{center}
On rappelle que dans le pentagone régulier $A_0A_1A_2A_3A_4$, ci-dessus :
\begin{itemize}
     \item
     $A_0,\:A_1,\:A_2,\:A_3$ et $A_4$appartiennent au cercle trigonométrique ;\item
     $k$ appartenant à $\{0~;~1~;~2~;~3\}$ on a $\left(\overrightarrow{OA_k}~;~\overrightarrow{OA}_{k+1}\right) = \frac{2\pi}{5} $.
\end{itemize}
\begin{enumerate}
     \item
     On considère les points $B$ d'affixe $-1$ et $J$ d'affixe $\frac{i}{2}$.
     \par
     Le cercle $(\mathscr{C})$ de centre $J$ et de rayon $\dfrac{1}{2}$ coupe le segment $[BJ]$ en un point $K$.
     \par
     Calculer $BJ$, puis en déduire $BK$. \item
     \begin{enumerate}[label=\alph*.]

          \item
          Donner sous forme exponentielle l'affixe du point $A_2$. Justifier brièvement.
          \item
          Démontrer que $B{A_2}^{2} = 2+ 2\cos \left(\dfrac{4\pi}{5}\right)$.
          \item
          Un logiciel de calcul formel affiche les résultats ci-dessous, que l'on pourra utiliser sans justification :
\begin{tabularx}{0.8\linewidth}{|*{3}{>{\centering \arraybackslash }X|}}%class="singleborder" width="600"
     \hline
 \textbf{1.}&\textbf{cos(4*pi/5)}& 
     \\ \hline
     &    &  $ \to \dfrac{1}{4}\left(- \sqrt{5}-1\right)  $
     \\ \hline
     \textbf{2.}& \textbf{sqrt((3-sqrt(5))/2)}  $  $
     \\ \hline
     &    &   $\to \dfrac{1}{2}\left(\sqrt{5}-1\right)$
     \\ \hline
\end{tabularx}
        
          \par
          "sqrt" signifie "racine carrée"
          \par
          En   déduire, grâce à ces résultats, que $BA_2 = BK$.
     \end{enumerate}
     \item
     Dans le repère $(O~;~\vec{u},\vec{v})$ donné ci-dessous, construire à la règle et au compas un pentagone régulier. N'utiliser ni le rapporteur ni les graduations de la règle et laisser apparents les traits de construction.
\begin{center}
\imgsvg{nombres-complexes-bac-s-pondichery-2016-2}{0.3}% alt="pentagone régulier" style="width:40rem"
\end{center}

\end{enumerate}
\begin{corrige}
     \begin{enumerate}
          \item

\cadre{bleu}{Rappel}{ % id=r010  
Pour deux points $A(z_A)$ et $B(z_B)$, la longueur $AB$ est égale à $\left|z_B-z_A\right|$
} % fin Rappel
     $BJ = \left|z_J-z_B\right|$
     \par
     $\phantom{BJ }= \left|\frac{i}{2}+1\right|$
     \par
     $\phantom{BJ }= \sqrt{1+\frac{1}{4}}$
     \par
     $\phantom{BJ }= \sqrt{\frac{5}{4}}=\frac{\sqrt{5}}{2}$
\begin{center}
\imgsvg{nombres-complexes-bac-s-pondichery-2016-3}{0.3}% alt="pentagone régulier" style="width:40rem"
\end{center}
     Les points $B, K$ et $J$ étant alignés dans cet ordre :
     \par
     $BK=BJ-KJ$
     \par
     $\phantom{BK}=\frac{\sqrt{5}}{2}-\frac{1}{2}$ car $KJ$ est un rayon du cercle $\mathscr C$
     \par
     $\phantom{BK}=\frac{\sqrt{5}-1}{2}$
     \item
     \begin{enumerate}[label=\alph*.]
          \item
          Notons $z_2$ l'affixe du point $A_2$.
          \par
          Comme $A_2$ est situé sur le cercle trigonométrique, $|z_2|=1 $.
          \par
          Un argument de $z_2$ est une mesure de l'angle orienté $(\vec{u}, \overrightarrow{OA_2})$.
          \par
          D'après la relation de Chasles sur les angles orientés :
          \par
          $(\vec{u}, \overrightarrow{OA_2})=(\vec{u}, \overrightarrow{OA_1})+(\overrightarrow{OA_1}, \overrightarrow{OA_2}) $
          \par
          $(\vec{u}, \overrightarrow{OA_2})=\frac{2\pi}{5}+\frac{2\pi}{5} \quad [2\pi] $
          \par
          $(\vec{u}, \overrightarrow{OA_2})=\frac{4\pi}{5} \quad [2\pi] $
          \par
          La forme exponentielle de $z_2$ est donc $z_2=e^{i \frac{4\pi}{5}}$
          \item
          Par conséquent :
          \par
          $z_2=\cos\left(\frac{4\pi}{5}\right)+i\sin\left(\frac{4\pi}{5}\right)$
          \par
          $B{A_2}^2=\left|z_2-(-1) \right|^2$
          \par
          $\phantom{B{A_2}^2}=\left|1+\cos\left(\frac{4\pi}{5}\right)+i\sin\left(\frac{4\pi}{5}\right) \right|^2$
          \par
          $\phantom{B{A_2}^2}=\left(1+\cos\left(\frac{4\pi}{5}\right)\right)^2+\left(\sin\left(\frac{4\pi}{5}\right) \right)^2$
          \par
          $\phantom{B{A_2}^2}=1+2\cos\left(\frac{4\pi}{5}\right)+\cos^2\left(\frac{4\pi}{5}\right)+\sin^2\left(\frac{4\pi}{5}\right)$
          \par
          $\phantom{B{A_2}^2}=2+2\cos\left(\frac{4\pi}{5}\right)\quad$ car $\cos^2\left(\frac{4\pi}{5}\right)+\sin^2\left(\frac{4\pi}{5}\right)=1 $
          \item
          D'après le logiciel de calcul formel (\textit{ligne 1}) : $\cos\left(\frac{4\pi}{5}\right)=  \frac{1}{4}\left(- \sqrt{5}-1\right)$.
          \par
          Donc :
          \par
          $B{A_2}^2=2+2 \times \frac{1}{4}\left(-\sqrt{5}-1\right)$
          \par
          $\phantom{B{A_2}^2}=\frac{4}{2}+\frac{-\sqrt{5}-1}{2}$
          \par
          $\phantom{B{A_2}^2}=\frac{3-\sqrt{5}}{2}$
          \par
          et d'après le logiciel de calcul formel (\textit{ligne 2}) : $\sqrt{\frac{3-\sqrt{5}}{2}} = \frac{\sqrt{5}-1}{2} $.
          \par
          Par conséquent $BA_2=\frac{\sqrt{5}-1}{2}=BK$.
     \end{enumerate}
     \item
     \textit{Dans la suite, on note $I$ le point d'affixe $i$.}
     \textbf{1ère étape : } Construction du milieu $J$ de $[OI]$.
     \par
     On construit au compas la médiatrice du segment $[OI]$. Cette droite coupe l'axe des ordonnées en $J$.
\begin{center}
\imgsvg{nombres-complexes-bac-s-pondichery-2016-4}{0.3}% alt="pentagone régulier" style="width:40rem"
\end{center}
     \textbf{2ème étape : } Construction du point $K$
     \par
     On trace le cercle de centre $J$ et de rayon $[OJ]$.
     \par
     Ce cercle coupe le segment $[BJ]$ en $K$.
\begin{center}
\imgsvg{nombres-complexes-bac-s-pondichery-2016-5}{0.3}% alt="pentagone régulier" style="width:40rem"
\end{center}
     \textbf{3ème étape : } Construction des points $A_2$ et $A_3$
     \par
     On reporte la longueur $[BK]$ de part et d'autre du point $B$ sur le cercle trigonométrique.
     \par
     On obtient alors les points $A_2$ et $A_3$ (car d'après la question précédente $BA_2=BK$ et $A_2$ et $A_3$ sont symétriques par rapport à l'axe des abscisses).
\begin{center}
\imgsvg{nombres-complexes-bac-s-pondichery-2016-6}{0.3}% alt="pentagone régulier" style="width:40rem"
\end{center}
     \textbf{4ème étape : } Tracé du pentagone
     \par
     Le segment $[A_2A_3]$ est un côté du pentagone. On complète la construction en reportant plusieurs fois la longueur $A_2A_3$ sur le cercle trigonométrique.
\begin{center}
\imgsvg{nombres-complexes-bac-s-pondichery-2016-7}{0.3}% alt="pentagone régulier" style="width:40rem"
\end{center}
\end{enumerate}
\end{corrige}

\end{document}