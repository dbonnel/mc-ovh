\documentclass[a4paper]{article}

%================================================================================================================================
%
% Packages
%
%================================================================================================================================

\usepackage[T1]{fontenc} 	% pour caractères accentués
\usepackage[utf8]{inputenc}  % encodage utf8
\usepackage[french]{babel}	% langue : français
\usepackage{fourier}			% caractères plus lisibles
\usepackage[dvipsnames]{xcolor} % couleurs
\usepackage{fancyhdr}		% réglage header footer
\usepackage{needspace}		% empêcher sauts de page mal placés
\usepackage{graphicx}		% pour inclure des graphiques
\usepackage{enumitem,cprotect}		% personnalise les listes d'items (nécessaire pour ol, al ...)
\usepackage{hyperref}		% Liens hypertexte
\usepackage{pstricks,pst-all,pst-node,pstricks-add,pst-math,pst-plot,pst-tree,pst-eucl} % pstricks
\usepackage[a4paper,includeheadfoot,top=2cm,left=3cm, bottom=2cm,right=3cm]{geometry} % marges etc.
\usepackage{comment}			% commentaires multilignes
\usepackage{amsmath,environ} % maths (matrices, etc.)
\usepackage{amssymb,makeidx}
\usepackage{bm}				% bold maths
\usepackage{tabularx}		% tableaux
\usepackage{colortbl}		% tableaux en couleur
\usepackage{fontawesome}		% Fontawesome
\usepackage{environ}			% environment with command
\usepackage{fp}				% calculs pour ps-tricks
\usepackage{multido}			% pour ps tricks
\usepackage[np]{numprint}	% formattage nombre
\usepackage{tikz,tkz-tab} 			% package principal TikZ
\usepackage{pgfplots}   % axes
\usepackage{mathrsfs}    % cursives
\usepackage{calc}			% calcul taille boites
\usepackage[scaled=0.875]{helvet} % font sans serif
\usepackage{svg} % svg
\usepackage{scrextend} % local margin
\usepackage{scratch} %scratch
\usepackage{multicol} % colonnes
%\usepackage{infix-RPN,pst-func} % formule en notation polanaise inversée
\usepackage{listings}

%================================================================================================================================
%
% Réglages de base
%
%================================================================================================================================

\lstset{
language=Python,   % R code
literate=
{á}{{\'a}}1
{à}{{\`a}}1
{ã}{{\~a}}1
{é}{{\'e}}1
{è}{{\`e}}1
{ê}{{\^e}}1
{í}{{\'i}}1
{ó}{{\'o}}1
{õ}{{\~o}}1
{ú}{{\'u}}1
{ü}{{\"u}}1
{ç}{{\c{c}}}1
{~}{{ }}1
}


\definecolor{codegreen}{rgb}{0,0.6,0}
\definecolor{codegray}{rgb}{0.5,0.5,0.5}
\definecolor{codepurple}{rgb}{0.58,0,0.82}
\definecolor{backcolour}{rgb}{0.95,0.95,0.92}

\lstdefinestyle{mystyle}{
    backgroundcolor=\color{backcolour},   
    commentstyle=\color{codegreen},
    keywordstyle=\color{magenta},
    numberstyle=\tiny\color{codegray},
    stringstyle=\color{codepurple},
    basicstyle=\ttfamily\footnotesize,
    breakatwhitespace=false,         
    breaklines=true,                 
    captionpos=b,                    
    keepspaces=true,                 
    numbers=left,                    
xleftmargin=2em,
framexleftmargin=2em,            
    showspaces=false,                
    showstringspaces=false,
    showtabs=false,                  
    tabsize=2,
    upquote=true
}

\lstset{style=mystyle}


\lstset{style=mystyle}
\newcommand{\imgdir}{C:/laragon/www/newmc/assets/imgsvg/}
\newcommand{\imgsvgdir}{C:/laragon/www/newmc/assets/imgsvg/}

\definecolor{mcgris}{RGB}{220, 220, 220}% ancien~; pour compatibilité
\definecolor{mcbleu}{RGB}{52, 152, 219}
\definecolor{mcvert}{RGB}{125, 194, 70}
\definecolor{mcmauve}{RGB}{154, 0, 215}
\definecolor{mcorange}{RGB}{255, 96, 0}
\definecolor{mcturquoise}{RGB}{0, 153, 153}
\definecolor{mcrouge}{RGB}{255, 0, 0}
\definecolor{mclightvert}{RGB}{205, 234, 190}

\definecolor{gris}{RGB}{220, 220, 220}
\definecolor{bleu}{RGB}{52, 152, 219}
\definecolor{vert}{RGB}{125, 194, 70}
\definecolor{mauve}{RGB}{154, 0, 215}
\definecolor{orange}{RGB}{255, 96, 0}
\definecolor{turquoise}{RGB}{0, 153, 153}
\definecolor{rouge}{RGB}{255, 0, 0}
\definecolor{lightvert}{RGB}{205, 234, 190}
\setitemize[0]{label=\color{lightvert}  $\bullet$}

\pagestyle{fancy}
\renewcommand{\headrulewidth}{0.2pt}
\fancyhead[L]{maths-cours.fr}
\fancyhead[R]{\thepage}
\renewcommand{\footrulewidth}{0.2pt}
\fancyfoot[C]{}

\newcolumntype{C}{>{\centering\arraybackslash}X}
\newcolumntype{s}{>{\hsize=.35\hsize\arraybackslash}X}

\setlength{\parindent}{0pt}		 
\setlength{\parskip}{3mm}
\setlength{\headheight}{1cm}

\def\ebook{ebook}
\def\book{book}
\def\web{web}
\def\type{web}

\newcommand{\vect}[1]{\overrightarrow{\,\mathstrut#1\,}}

\def\Oij{$\left(\text{O}~;~\vect{\imath},~\vect{\jmath}\right)$}
\def\Oijk{$\left(\text{O}~;~\vect{\imath},~\vect{\jmath},~\vect{k}\right)$}
\def\Ouv{$\left(\text{O}~;~\vect{u},~\vect{v}\right)$}

\hypersetup{breaklinks=true, colorlinks = true, linkcolor = OliveGreen, urlcolor = OliveGreen, citecolor = OliveGreen, pdfauthor={Didier BONNEL - https://www.maths-cours.fr} } % supprime les bordures autour des liens

\renewcommand{\arg}[0]{\text{arg}}

\everymath{\displaystyle}

%================================================================================================================================
%
% Macros - Commandes
%
%================================================================================================================================

\newcommand\meta[2]{    			% Utilisé pour créer le post HTML.
	\def\titre{titre}
	\def\url{url}
	\def\arg{#1}
	\ifx\titre\arg
		\newcommand\maintitle{#2}
		\fancyhead[L]{#2}
		{\Large\sffamily \MakeUppercase{#2}}
		\vspace{1mm}\textcolor{mcvert}{\hrule}
	\fi 
	\ifx\url\arg
		\fancyfoot[L]{\href{https://www.maths-cours.fr#2}{\black \footnotesize{https://www.maths-cours.fr#2}}}
	\fi 
}


\newcommand\TitreC[1]{    		% Titre centré
     \needspace{3\baselineskip}
     \begin{center}\textbf{#1}\end{center}
}

\newcommand\newpar{    		% paragraphe
     \par
}

\newcommand\nosp {    		% commande vide (pas d'espace)
}
\newcommand{\id}[1]{} %ignore

\newcommand\boite[2]{				% Boite simple sans titre
	\vspace{5mm}
	\setlength{\fboxrule}{0.2mm}
	\setlength{\fboxsep}{5mm}	
	\fcolorbox{#1}{#1!3}{\makebox[\linewidth-2\fboxrule-2\fboxsep]{
  		\begin{minipage}[t]{\linewidth-2\fboxrule-4\fboxsep}\setlength{\parskip}{3mm}
  			 #2
  		\end{minipage}
	}}
	\vspace{5mm}
}

\newcommand\CBox[4]{				% Boites
	\vspace{5mm}
	\setlength{\fboxrule}{0.2mm}
	\setlength{\fboxsep}{5mm}
	
	\fcolorbox{#1}{#1!3}{\makebox[\linewidth-2\fboxrule-2\fboxsep]{
		\begin{minipage}[t]{1cm}\setlength{\parskip}{3mm}
	  		\textcolor{#1}{\LARGE{#2}}    
 	 	\end{minipage}  
  		\begin{minipage}[t]{\linewidth-2\fboxrule-4\fboxsep}\setlength{\parskip}{3mm}
			\raisebox{1.2mm}{\normalsize\sffamily{\textcolor{#1}{#3}}}						
  			 #4
  		\end{minipage}
	}}
	\vspace{5mm}
}

\newcommand\cadre[3]{				% Boites convertible html
	\par
	\vspace{2mm}
	\setlength{\fboxrule}{0.1mm}
	\setlength{\fboxsep}{5mm}
	\fcolorbox{#1}{white}{\makebox[\linewidth-2\fboxrule-2\fboxsep]{
  		\begin{minipage}[t]{\linewidth-2\fboxrule-4\fboxsep}\setlength{\parskip}{3mm}
			\raisebox{-2.5mm}{\sffamily \small{\textcolor{#1}{\MakeUppercase{#2}}}}		
			\par		
  			 #3
 	 		\end{minipage}
	}}
		\vspace{2mm}
	\par
}

\newcommand\bloc[3]{				% Boites convertible html sans bordure
     \needspace{2\baselineskip}
     {\sffamily \small{\textcolor{#1}{\MakeUppercase{#2}}}}    
		\par		
  			 #3
		\par
}

\newcommand\CHelp[1]{
     \CBox{Plum}{\faInfoCircle}{À RETENIR}{#1}
}

\newcommand\CUp[1]{
     \CBox{NavyBlue}{\faThumbsOUp}{EN PRATIQUE}{#1}
}

\newcommand\CInfo[1]{
     \CBox{Sepia}{\faArrowCircleRight}{REMARQUE}{#1}
}

\newcommand\CRedac[1]{
     \CBox{PineGreen}{\faEdit}{BIEN R\'EDIGER}{#1}
}

\newcommand\CError[1]{
     \CBox{Red}{\faExclamationTriangle}{ATTENTION}{#1}
}

\newcommand\TitreExo[2]{
\needspace{4\baselineskip}
 {\sffamily\large EXERCICE #1\ (\emph{#2 points})}
\vspace{5mm}
}

\newcommand\img[2]{
          \includegraphics[width=#2\paperwidth]{\imgdir#1}
}

\newcommand\imgsvg[2]{
       \begin{center}   \includegraphics[width=#2\paperwidth]{\imgsvgdir#1} \end{center}
}


\newcommand\Lien[2]{
     \href{#1}{#2 \tiny \faExternalLink}
}
\newcommand\mcLien[2]{
     \href{https~://www.maths-cours.fr/#1}{#2 \tiny \faExternalLink}
}

\newcommand{\euro}{\eurologo{}}

%================================================================================================================================
%
% Macros - Environement
%
%================================================================================================================================

\newenvironment{tex}{ %
}
{%
}

\newenvironment{indente}{ %
	\setlength\parindent{10mm}
}

{
	\setlength\parindent{0mm}
}

\newenvironment{corrige}{%
     \needspace{3\baselineskip}
     \medskip
     \textbf{\textsc{Corrigé}}
     \medskip
}
{
}

\newenvironment{extern}{%
     \begin{center}
     }
     {
     \end{center}
}

\NewEnviron{code}{%
	\par
     \boite{gray}{\texttt{%
     \BODY
     }}
     \par
}

\newenvironment{vbloc}{% boite sans cadre empeche saut de page
     \begin{minipage}[t]{\linewidth}
     }
     {
     \end{minipage}
}
\NewEnviron{h2}{%
    \needspace{3\baselineskip}
    \vspace{0.6cm}
	\noindent \MakeUppercase{\sffamily \large \BODY}
	\vspace{1mm}\textcolor{mcgris}{\hrule}\vspace{0.4cm}
	\par
}{}

\NewEnviron{h3}{%
    \needspace{3\baselineskip}
	\vspace{5mm}
	\textsc{\BODY}
	\par
}

\NewEnviron{margeneg}{ %
\begin{addmargin}[-1cm]{0cm}
\BODY
\end{addmargin}
}

\NewEnviron{html}{%
}

\begin{document}
\meta{url}{/exercices/nombres-complexes-bac-s-asie-2018/}
\meta{pid}{9446}
\meta{titre}{Nombres complexes – Bac S Asie 2018}
\meta{type}{exercices}
%
\begin{h2}Exercice 4 (5 points)\end{h2}
\textbf{Candidats n'ayant pas choisi la spécialité \og mathématiques \fg{}}
\bigbreak
Dans cet exercice, $x$ et $y$ sont des nombres réels supérieurs à 1.
\medbreak
Dans le plan complexe muni d'un repère orthonormé direct $(O~;~\overrightarrow{i},~\overrightarrow{j}~,~\overrightarrow{k})$, on considère les points A, B
et C d'affixes respectives
\[z_{\text{A}} = 1 + \text{i}, \:  z_{\text{B}} = x + \text{i}\: \text{ et }\: z_{\text{C}} = y + \text{i}.\]
\begin{center}
     \begin{extern}%width="550" alt="Nombres complexes Bac S Asie 2018"
          \psset{unit=4cm}
          \begin{pspicture}(-0.2,-0.2)(3,1.1)
               \psaxes[linewidth=1pt]{->}(0,0)(1,1)
               \psframe[linecolor=lightgray](0,0)(3,1)
               \psline[linecolor=lightgray](1,0)(1,1)
               \psline[linecolor=lightgray](1.6,0)(1.6,1)
               \uput[d](0.5,0){$\overrightarrow{u}$}
               \uput[l](0,0.5){$\overrightarrow{v}$}
               \uput[u](1,1){A$(1 + \text{i})$}
               \uput[u](1.6,1){B$(x + \text{i})$}
               \uput[u](3,1){C$(y + \text{i})$}
               \uput[l](0,0){O}
               \psline(3,1)(0,0)(1.6,1)
               \psline(1,1)
          \end{pspicture}
     \end{extern}
\end{center}
\medbreak
\textbf{Problème~:} on cherche les valeurs éventuelles des réels $x$ et $y$, supérieures à 1, pour lesquelles~:
$\text{OC} = \text{OA} \times \text{OB} $ et $\left(\overrightarrow{u},~\overrightarrow{\text{OB}}\right) + \left(\overrightarrow{u},~\overrightarrow{\text{OC}}\right) = \left(\overrightarrow{u},~\overrightarrow{\text{OA}}\right).$
\medbreak
\begin{enumerate}
     \item Démontrer que si $\text{OC} = \text{OA} \times \text{OB}$, alors $y^2 = 2x^2 + 1$.
     \item Reproduire sur la copie et compléter l'algorithme ci-après pour qu'il affiche tous les couples $(x,~y)$ tels que~:
     \begin{center}
          $\left\{\begin{array}{l}
                    y^2 = 2x^2 + 1\\
                    x~\text{et}~y~\text{sont des nombres entiers} \\
                    1  \leqslant x \leqslant 10 ~\text{et}~ 1 \leqslant y \leqslant 10
          \end{array}\right.$
     \end{center}
     \begin{center}
          \begin{extern}%width="250" alt="algorithme Bac S Asie 2018"
               \begin{tabular}{|l|}\hline
                    Pour $x$ allant de 1 à \ldots faire\\
                    \hspace{0.5cm}Pour \ldots\\
                    \hspace{1cm}Si \ldots\\
                    \hspace{1.5cm}Afficher $x$ et $y$\\
                    \hspace{1cm}Fin Si\\
                    \hspace{0.5cm}Fin Pour\\
                    Fin Pour\\ \hline
               \end{tabular}
          \end{extern}
     \end{center}
     \emph{Lorsque l'on exécute cet algorithme, il affiche la valeur $2$ pour la variable $x$ et la valeur $3$ pour la variable $y$.}
     \smallbreak
     \item Étude d'un cas particulier~: dans cette question seulement, on prend $x = 2$ et $y = 3$.
     \begin{enumerate}[label=\alph*.]
          \item Donner le module et un argument de $z_{\text{A}}$.
          \item Montrer que $\text{OC} = \text{OA} \times \text{OB}$.
          \item Montrer que $z_{\text{B}}z_{\text{C}} = 5 z_{\text{A}}$ et en déduire que :
          $\left(\overrightarrow{u},~\overrightarrow{\text{OB}}\right) + \left(\overrightarrow{u},~\overrightarrow{\text{OC}}\right)$\nosp$ = \left(\overrightarrow{u},~\overrightarrow{\text{OA}}\right)$.
     \end{enumerate}
     \item On revient au cas général, et on cherche s'il existe d'autres valeurs des réels $x$ et $y$ telles que les points A, B et C vérifient les deux conditions~:
     \par
     $\text{OC} = \text{OA} \times \text{OB} \quad$ et $\:  \left(\overrightarrow{u},~\overrightarrow{\text{OB}}\right) + \left(\overrightarrow{u},~\overrightarrow{\text{OC}}\right)$\nosp$ = \left(\overrightarrow{u},~\overrightarrow{\text{OA}}\right)$.
     \par
     On rappelle que si $\text{OC} = \text{OA} \times \text{OB}$, alors $y^2 = 2x^2 + 1$ (question 1.).
     \begin{enumerate}[label=\alph*.]
          \item Démontrer que si $\left(\overrightarrow{u},~\overrightarrow{\text{OB}}\right) + \left(\overrightarrow{u},~\overrightarrow{\text{OC}}\right)$\nosp$ = \left(\overrightarrow{u},~\overrightarrow{\text{OA}}\right)$, alors arg$\left[\dfrac{(x + \text{i})(y + \text{i})}{1 + \text{i}}\right] = 0$\nosp$ \:\text{mod }\: 2\pi$.
          \par
          En déduire que sous cette condition~: $x + y - xy + 1 = 0$.
          \item Démontrer que si les deux conditions sont vérifiées et que de plus $x \neq 1$, alors~:
          \begin{center}
               $y= \sqrt{2x^2 + 1}\quad $ et $\: y = \dfrac{x + 1}{x - 1}.$
          \end{center}
     \end{enumerate}
     \item On définit les fonctions $f$ et $g$ sur l'intervalle $]1~;~+ \infty[$ par~:
     \begin{center}
          $f(x) = \sqrt{2x^2 + 1}\quad$ et $\: g(x) = \dfrac{x + 1}{x - 1}.$
     \end{center}
     Déterminer le nombre de solutions du problème initial.
     \par
     On pourra utiliser la fonction $h$ définie sur l'intervalle $]1~;~+ \infty[$ par $h(x) = f(x) - g(x)$ et s'appuyer sur la copie d'écran d'un logiciel de calcul formel donnée ci-dessous.
     \begin{center}
          \begin{extern}%width="230" alt="Calcul formel Bac S Asie 2018"
               \renewcommand{\arraystretch}{1.5}
               \begin{tabular}{|l|}
                    \hline
                    $f(x)~:= \text{sqrt}(2*x\verb+^+2+1)$	\\
                    \hspace{1.5cm}$x \to  \sqrt{2*x^2+1}$	\\ \hline
                    deriver$(f)$							\\
                    \hspace{1.5cm}$x \to \dfrac{2*x}{\sqrt{2*x^2+ 1}}$ \\[0.3cm] \hline
                    $g(x)~:=(x+1)/(x-1)$					\\
                    \hspace{1.5cm}$x \to \dfrac{x + 1}{x - 1}$ \\[0.2cm] \hline
                    deriver$(g)$							\\
                    \hspace{1.5cm}$x \to  - \dfrac{2}{(x - 1)^2}$ \\[0.2cm] \hline
               \end{tabular}
          \end{extern}
     \end{center}
\end{enumerate}

\end{document}