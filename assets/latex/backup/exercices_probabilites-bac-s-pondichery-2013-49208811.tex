\documentclass[a4paper]{article}

%================================================================================================================================
%
% Packages
%
%================================================================================================================================

\usepackage[T1]{fontenc} 	% pour caractères accentués
\usepackage[utf8]{inputenc}  % encodage utf8
\usepackage[french]{babel}	% langue : français
\usepackage{fourier}			% caractères plus lisibles
\usepackage[dvipsnames]{xcolor} % couleurs
\usepackage{fancyhdr}		% réglage header footer
\usepackage{needspace}		% empêcher sauts de page mal placés
\usepackage{graphicx}		% pour inclure des graphiques
\usepackage{enumitem,cprotect}		% personnalise les listes d'items (nécessaire pour ol, al ...)
\usepackage{hyperref}		% Liens hypertexte
\usepackage{pstricks,pst-all,pst-node,pstricks-add,pst-math,pst-plot,pst-tree,pst-eucl} % pstricks
\usepackage[a4paper,includeheadfoot,top=2cm,left=3cm, bottom=2cm,right=3cm]{geometry} % marges etc.
\usepackage{comment}			% commentaires multilignes
\usepackage{amsmath,environ} % maths (matrices, etc.)
\usepackage{amssymb,makeidx}
\usepackage{bm}				% bold maths
\usepackage{tabularx}		% tableaux
\usepackage{colortbl}		% tableaux en couleur
\usepackage{fontawesome}		% Fontawesome
\usepackage{environ}			% environment with command
\usepackage{fp}				% calculs pour ps-tricks
\usepackage{multido}			% pour ps tricks
\usepackage[np]{numprint}	% formattage nombre
\usepackage{tikz,tkz-tab} 			% package principal TikZ
\usepackage{pgfplots}   % axes
\usepackage{mathrsfs}    % cursives
\usepackage{calc}			% calcul taille boites
\usepackage[scaled=0.875]{helvet} % font sans serif
\usepackage{svg} % svg
\usepackage{scrextend} % local margin
\usepackage{scratch} %scratch
\usepackage{multicol} % colonnes
%\usepackage{infix-RPN,pst-func} % formule en notation polanaise inversée
\usepackage{listings}

%================================================================================================================================
%
% Réglages de base
%
%================================================================================================================================

\lstset{
language=Python,   % R code
literate=
{á}{{\'a}}1
{à}{{\`a}}1
{ã}{{\~a}}1
{é}{{\'e}}1
{è}{{\`e}}1
{ê}{{\^e}}1
{í}{{\'i}}1
{ó}{{\'o}}1
{õ}{{\~o}}1
{ú}{{\'u}}1
{ü}{{\"u}}1
{ç}{{\c{c}}}1
{~}{{ }}1
}


\definecolor{codegreen}{rgb}{0,0.6,0}
\definecolor{codegray}{rgb}{0.5,0.5,0.5}
\definecolor{codepurple}{rgb}{0.58,0,0.82}
\definecolor{backcolour}{rgb}{0.95,0.95,0.92}

\lstdefinestyle{mystyle}{
    backgroundcolor=\color{backcolour},   
    commentstyle=\color{codegreen},
    keywordstyle=\color{magenta},
    numberstyle=\tiny\color{codegray},
    stringstyle=\color{codepurple},
    basicstyle=\ttfamily\footnotesize,
    breakatwhitespace=false,         
    breaklines=true,                 
    captionpos=b,                    
    keepspaces=true,                 
    numbers=left,                    
xleftmargin=2em,
framexleftmargin=2em,            
    showspaces=false,                
    showstringspaces=false,
    showtabs=false,                  
    tabsize=2,
    upquote=true
}

\lstset{style=mystyle}


\lstset{style=mystyle}
\newcommand{\imgdir}{C:/laragon/www/newmc/assets/imgsvg/}
\newcommand{\imgsvgdir}{C:/laragon/www/newmc/assets/imgsvg/}

\definecolor{mcgris}{RGB}{220, 220, 220}% ancien~; pour compatibilité
\definecolor{mcbleu}{RGB}{52, 152, 219}
\definecolor{mcvert}{RGB}{125, 194, 70}
\definecolor{mcmauve}{RGB}{154, 0, 215}
\definecolor{mcorange}{RGB}{255, 96, 0}
\definecolor{mcturquoise}{RGB}{0, 153, 153}
\definecolor{mcrouge}{RGB}{255, 0, 0}
\definecolor{mclightvert}{RGB}{205, 234, 190}

\definecolor{gris}{RGB}{220, 220, 220}
\definecolor{bleu}{RGB}{52, 152, 219}
\definecolor{vert}{RGB}{125, 194, 70}
\definecolor{mauve}{RGB}{154, 0, 215}
\definecolor{orange}{RGB}{255, 96, 0}
\definecolor{turquoise}{RGB}{0, 153, 153}
\definecolor{rouge}{RGB}{255, 0, 0}
\definecolor{lightvert}{RGB}{205, 234, 190}
\setitemize[0]{label=\color{lightvert}  $\bullet$}

\pagestyle{fancy}
\renewcommand{\headrulewidth}{0.2pt}
\fancyhead[L]{maths-cours.fr}
\fancyhead[R]{\thepage}
\renewcommand{\footrulewidth}{0.2pt}
\fancyfoot[C]{}

\newcolumntype{C}{>{\centering\arraybackslash}X}
\newcolumntype{s}{>{\hsize=.35\hsize\arraybackslash}X}

\setlength{\parindent}{0pt}		 
\setlength{\parskip}{3mm}
\setlength{\headheight}{1cm}

\def\ebook{ebook}
\def\book{book}
\def\web{web}
\def\type{web}

\newcommand{\vect}[1]{\overrightarrow{\,\mathstrut#1\,}}

\def\Oij{$\left(\text{O}~;~\vect{\imath},~\vect{\jmath}\right)$}
\def\Oijk{$\left(\text{O}~;~\vect{\imath},~\vect{\jmath},~\vect{k}\right)$}
\def\Ouv{$\left(\text{O}~;~\vect{u},~\vect{v}\right)$}

\hypersetup{breaklinks=true, colorlinks = true, linkcolor = OliveGreen, urlcolor = OliveGreen, citecolor = OliveGreen, pdfauthor={Didier BONNEL - https://www.maths-cours.fr} } % supprime les bordures autour des liens

\renewcommand{\arg}[0]{\text{arg}}

\everymath{\displaystyle}

%================================================================================================================================
%
% Macros - Commandes
%
%================================================================================================================================

\newcommand\meta[2]{    			% Utilisé pour créer le post HTML.
	\def\titre{titre}
	\def\url{url}
	\def\arg{#1}
	\ifx\titre\arg
		\newcommand\maintitle{#2}
		\fancyhead[L]{#2}
		{\Large\sffamily \MakeUppercase{#2}}
		\vspace{1mm}\textcolor{mcvert}{\hrule}
	\fi 
	\ifx\url\arg
		\fancyfoot[L]{\href{https://www.maths-cours.fr#2}{\black \footnotesize{https://www.maths-cours.fr#2}}}
	\fi 
}


\newcommand\TitreC[1]{    		% Titre centré
     \needspace{3\baselineskip}
     \begin{center}\textbf{#1}\end{center}
}

\newcommand\newpar{    		% paragraphe
     \par
}

\newcommand\nosp {    		% commande vide (pas d'espace)
}
\newcommand{\id}[1]{} %ignore

\newcommand\boite[2]{				% Boite simple sans titre
	\vspace{5mm}
	\setlength{\fboxrule}{0.2mm}
	\setlength{\fboxsep}{5mm}	
	\fcolorbox{#1}{#1!3}{\makebox[\linewidth-2\fboxrule-2\fboxsep]{
  		\begin{minipage}[t]{\linewidth-2\fboxrule-4\fboxsep}\setlength{\parskip}{3mm}
  			 #2
  		\end{minipage}
	}}
	\vspace{5mm}
}

\newcommand\CBox[4]{				% Boites
	\vspace{5mm}
	\setlength{\fboxrule}{0.2mm}
	\setlength{\fboxsep}{5mm}
	
	\fcolorbox{#1}{#1!3}{\makebox[\linewidth-2\fboxrule-2\fboxsep]{
		\begin{minipage}[t]{1cm}\setlength{\parskip}{3mm}
	  		\textcolor{#1}{\LARGE{#2}}    
 	 	\end{minipage}  
  		\begin{minipage}[t]{\linewidth-2\fboxrule-4\fboxsep}\setlength{\parskip}{3mm}
			\raisebox{1.2mm}{\normalsize\sffamily{\textcolor{#1}{#3}}}						
  			 #4
  		\end{minipage}
	}}
	\vspace{5mm}
}

\newcommand\cadre[3]{				% Boites convertible html
	\par
	\vspace{2mm}
	\setlength{\fboxrule}{0.1mm}
	\setlength{\fboxsep}{5mm}
	\fcolorbox{#1}{white}{\makebox[\linewidth-2\fboxrule-2\fboxsep]{
  		\begin{minipage}[t]{\linewidth-2\fboxrule-4\fboxsep}\setlength{\parskip}{3mm}
			\raisebox{-2.5mm}{\sffamily \small{\textcolor{#1}{\MakeUppercase{#2}}}}		
			\par		
  			 #3
 	 		\end{minipage}
	}}
		\vspace{2mm}
	\par
}

\newcommand\bloc[3]{				% Boites convertible html sans bordure
     \needspace{2\baselineskip}
     {\sffamily \small{\textcolor{#1}{\MakeUppercase{#2}}}}    
		\par		
  			 #3
		\par
}

\newcommand\CHelp[1]{
     \CBox{Plum}{\faInfoCircle}{À RETENIR}{#1}
}

\newcommand\CUp[1]{
     \CBox{NavyBlue}{\faThumbsOUp}{EN PRATIQUE}{#1}
}

\newcommand\CInfo[1]{
     \CBox{Sepia}{\faArrowCircleRight}{REMARQUE}{#1}
}

\newcommand\CRedac[1]{
     \CBox{PineGreen}{\faEdit}{BIEN R\'EDIGER}{#1}
}

\newcommand\CError[1]{
     \CBox{Red}{\faExclamationTriangle}{ATTENTION}{#1}
}

\newcommand\TitreExo[2]{
\needspace{4\baselineskip}
 {\sffamily\large EXERCICE #1\ (\emph{#2 points})}
\vspace{5mm}
}

\newcommand\img[2]{
          \includegraphics[width=#2\paperwidth]{\imgdir#1}
}

\newcommand\imgsvg[2]{
       \begin{center}   \includegraphics[width=#2\paperwidth]{\imgsvgdir#1} \end{center}
}


\newcommand\Lien[2]{
     \href{#1}{#2 \tiny \faExternalLink}
}
\newcommand\mcLien[2]{
     \href{https~://www.maths-cours.fr/#1}{#2 \tiny \faExternalLink}
}

\newcommand{\euro}{\eurologo{}}

%================================================================================================================================
%
% Macros - Environement
%
%================================================================================================================================

\newenvironment{tex}{ %
}
{%
}

\newenvironment{indente}{ %
	\setlength\parindent{10mm}
}

{
	\setlength\parindent{0mm}
}

\newenvironment{corrige}{%
     \needspace{3\baselineskip}
     \medskip
     \textbf{\textsc{Corrigé}}
     \medskip
}
{
}

\newenvironment{extern}{%
     \begin{center}
     }
     {
     \end{center}
}

\NewEnviron{code}{%
	\par
     \boite{gray}{\texttt{%
     \BODY
     }}
     \par
}

\newenvironment{vbloc}{% boite sans cadre empeche saut de page
     \begin{minipage}[t]{\linewidth}
     }
     {
     \end{minipage}
}
\NewEnviron{h2}{%
    \needspace{3\baselineskip}
    \vspace{0.6cm}
	\noindent \MakeUppercase{\sffamily \large \BODY}
	\vspace{1mm}\textcolor{mcgris}{\hrule}\vspace{0.4cm}
	\par
}{}

\NewEnviron{h3}{%
    \needspace{3\baselineskip}
	\vspace{5mm}
	\textsc{\BODY}
	\par
}

\NewEnviron{margeneg}{ %
\begin{addmargin}[-1cm]{0cm}
\BODY
\end{addmargin}
}

\NewEnviron{html}{%
}

\begin{document}
\meta{url}{/exercices/probabilites-bac-s-pondichery-2013/}
\meta{pid}{2514}
\meta{titre}{Probabilités - Suites - Bac S Pondichéry 2013}
\meta{type}{exercices}
%
\begin{h3}Exercice 4   (6 points)\end{h3}
\textbf{Commun  à tous les candidats}
\par
Dans une entreprise, on s'intéresse à la probabilité qu'un salarié soit absent durant une période d'épidémie de grippe.
\begin{itemize}
     \item
     Un salarié malade est absent
     \item
     La première semaine de travail, le salarié n'est pas malade.
     \item
     Si la semaine $n$ le salarié n'est pas malade, il tombe malade la semaine $n+1$ avec une probabilité égale à $0,04$.
     \item
     Si la semaine $n$ le salarié est malade, il reste malade la semaine $n+1$ avec une probabilité égale à $0,24$.
\end{itemize}
On désigne, pour tout entier naturel $n$ supérieur ou égal à 1, par $E_{n}$ l'évènement \textit{"le salarié est absent pour cause de maladie la $n$-ième semaine"}. On note $p_{n}$ la probabilité de l'évènement $E_{n}$.
\par
On a ainsi : $p_{1}=0$ et, pour tout entier naturel $n$ supérieur ou égal à 1 : $0\leqslant  p_{n} < 1$.
\begin{enumerate}
     \item
     \begin{enumerate}[label=\alph*.]
          \item
          Déterminer la valeur de $p_{3}$ à l'aide d'un arbre de probabilité.
          \item
          Sachant que le salarié a été absent pour cause de maladie la troisième semaine, déterminer la probabilité qu'il ait été aussi absent pour cause de maladie la deuxième semaine.
     \end{enumerate}
     \item
     \begin{enumerate}[label=\alph*.]
          \item
          Recopier sur la copie et compléter l'arbre de probabilité donné ci-dessous
          <img src="/assets/imgsvg/mc-0037.png" alt="" class="aligncenter size-full  img-pc" />
          \item
          Montrer que, pour tout entier naturel $n$ supérieur ou égal à 1,
          \par
          $p_{n+1}=0,2p_{n}+0,04$.
          \item
          Montrer que la suite $\left(u_{n}\right)$ définie pour tout entier naturel $n$ supérieur ou égal à 1 par $u_{n}=p_{n}-0,05$ est une suite géométrique dont on donnera le premier terme et la raison $r$.
          \par
          En déduire l'expression de $u_{n}$ puis de $p_{n}$ en fonction de $n$ et $r$.
          \item
          En déduire la limite de la suite $\left(p_{n}\right)$.
          \item
          On admet dans cette question que la suite $\left(p_{n}\right)$ est croissante. On considère l'algorithme  suivant :
          \begin{tabularx}{0.8\linewidth}{|*{3}{>{\centering \arraybackslash }X|}}%class="singleborder" width="600"
               \hline
               Variables		 &  K et J sont des entiers naturels,
             	 &  P est un nombre réel
               \\ \hline
               Initialisation     &  P prend la valeur $0$
               \\ \hline
               & J prend la valeur $1$
               \\ \hline
               Entrée			 &  Saisir la valeur de K
               \\ \hline
               Traitement		 & Tant que $P < 0,05-10^{- K}$
               \\ \hline
               &$ \quad \quad $ P prend la valeur $0,2\times P+0,04$
               \\ \hline
               & $ \quad \quad $ J prend la valeur J + 1
               \\ \hline
               & Fin tant que
               \\ \hline
               Sortie			 & Afficher J
               \\ \hline
          \end{tabularx}

     A quoi correspond l'affichage final J ?
     \par
     Pourquoi est-on sûr que cet algorithme s'arrête ?
\end{enumerate}
\item
Cette entreprise emploie 220 salariés. Pour la suite on admet que la probabilité pour qu'un salarié soit malade une semaine donnée durant cette période d'épidémie est égale à $p=0,05$.
\par
On suppose que l'état de santé d'un salarié ne dépend pas de l'état de santé de ses collègues.
\par
On désigne par $X$ la variable aléatoire qui donne le nombre de salariés malades une semaine donnée.
\begin{enumerate}[label=\alph*.]
     \item
     Justifier que la variable aléatoire $X$ suit une loi binomiale dont on donnera les paramètres.
     \par
     Calculer l'espérance mathématique $\mu $ et l'écart type $\sigma $ de la variable aléatoire $X$.
     \item
     On admet que l'on peut approcher la loi de la variable aléatoire $\frac{X-\mu }{\sigma }$ par la loi normale  centrée réduite c'est-à-dire de paramètres $0$ et $1$.
     \par
     On note $Z$ une variable aléatoire suivant la loi normale centrée réduite.
     \par
     Le tableau suivant donne les probabilités de l'évènement $Z < x$ pour quelques valeurs du nombre réel $x$.
     \begin{tabularx}{0.8\linewidth}{|*{3}{>{\centering \arraybackslash }X|}}%class="compact" width="600"
          \hline
          $x$ &  -1,55  & -1,24  & -0,93  & - 0,62  & - 0,31
          \\ \hline
          $P\left(Z < x\right)$ &  0,061  & 0,108  & 0,177  & 0,268  & 0,379
          \\ \hline
     \end{tabularx}
     \begin{tabularx}{0.8\linewidth}{|*{3}{>{\centering \arraybackslash }X|}}%class="compact" width="600"
          \hline
          $x$ & 0,00  & 0,31  & 0,62  & 0,93  & 1,24  & 1,55
          \\ \hline
          $P\left(Z < x\right)$  & 0,500  & 0,621  & 0,732  & 0,823  & 0,892  & 0,939
          \\ \hline
     \end{tabularx}
     Calculer, au moyen de l'approximation proposée en question b., une valeur approchée à $10^{-2}$ près de la probabilité de l'évènement : \textit{"le nombre de salariés absents dans l'entreprise au cours d'une semaine donnée est supérieur ou égal  à 7 et inférieur ou égal à 15"}.
\end{enumerate}
\end{enumerate}
\begin{corrige}
     \begin{enumerate}
          \item
          \begin{enumerate}[label=\alph*.]
               \item
               <img src="/assets/imgsvg/mc-0040.png" alt="" class="aligncenter size-full  img-pc" />
               $p_{3}=p\left(E_{3}\right)=0,04\times 0,24+0,96\times 0,04=0,048$
               \item
               On recherche $p_{E_{3}}\left(E_{2}\right)$
               \par
               $p_{E_{3}}\left(E_{2}\right)=\frac{p\left(E_{2} \cap  E_{3}\right)}{p\left(E_{3}\right)}=\frac{0,04\times 0,24}{0,048}=0,2$
          \end{enumerate}
          \item
          \begin{enumerate}[label=\alph*.]
               \item
               <img src="/assets/imgsvg/mc-0041.png" alt="" class="aligncenter size-full  img-pc" />
%##
% type=arbre; width=25; wcell=3.5; hcell=2
%--
% >E_n:p_n
% >>E_{n+1}:...
% >>\overline{E_{n+1}}:...
% >\overline{E_n}:...
% >>E_{n+1}:...
% >>\overline{E_{n+1}}:...
%--
\begin{center}
 \begin{extern}%style="width:25rem" alt="Arbre pondéré"
    \resizebox{11cm}{!}{
       \definecolor{dark}{gray}{0.1}
       \begin{tikzpicture}[scale=.8, line width=.5pt, dark]
       \def\width{3.5}
       \def\height{2}
       \tikzstyle{noeud}=[fill=white,circle,draw]
       \tikzstyle{poids}=[fill=white,font=\footnotesize,midway]
    \node[noeud] (r) at ({1*\width},{-1.5*\height}) {$$};
    \node[noeud] (ra) at ({2*\width},{-0.5*\height}) {$E_n$};
     \draw (r) -- (ra) node [poids] {$p_n$};
    \node[noeud] (raa) at ({3*\width},{0*\height}) {$E_{n+1}$};
     \draw (ra) -- (raa) node [poids] {$...$};
    \node[noeud] (rab) at ({3*\width},{-1*\height}) {$\overline{E_{n+1}}$};
     \draw (ra) -- (rab) node [poids] {$...$};
    \node[noeud] (rb) at ({2*\width},{-2.5*\height}) {$\overline{E_n}$};
     \draw (r) -- (rb) node [poids] {$...$};
    \node[noeud] (rba) at ({3*\width},{-2*\height}) {$E_{n+1}$};
     \draw (rb) -- (rba) node [poids] {$...$};
    \node[noeud] (rbb) at ({3*\width},{-3*\height}) {$\overline{E_{n+1}}$};
     \draw (rb) -- (rbb) node [poids] {$...$};
       \end{tikzpicture}
      }
   \end{extern}
\end{center}
%##
\item
               D'après la formule des probabilités totales :
               \par
               $p_{n+1}=p\left(E_{n+1}\right)=0,24p_{n}+0,04\left(1-p_{n}\right)=0,2p_{n}+0,04$
               \item
               $u_{n+1}=p_{n+1}-0,05=0,2p_{n}+0,04-0,05=0,2\left(u_{n}+0,05\right)-0,01=0,2u_{n}$
               \par
               Donc la suite $\left(u_{n}\right)$ est une suite géométrique de raison $q=0,2$. Son premier terme est $u_{1}=p_{1}-0,05=-0,05$.
               \par
               On a donc :
               \par
               $u_{n}=u_1\ q^{n-1}=-0,05\times 0,2^{n-1}$
               \par
               et :
               \par
               $p_{n}=-0,05\times 0,2^{n-1}+0,05$
               \item
               Comme $-1 < 0,2 < 1$, $\lim\limits_{n\rightarrow +\infty }0,2^{n-1}=0$ et donc $\lim\limits_{n\rightarrow +\infty }p_{n}=0,05$
               \item
               Le nombre $J$ affiché est le rang à partir duquel $p_{J}\geqslant  0,05-10^{- \text{K}}$.
               \par
               L'algorithme se termine toujours car la suite $p_{n}$ est croissante et $\lim\limits_{n\rightarrow \infty }p_{n}=0,05$, donc, quelque soit $K$ on trouvera toujours un rang $J$ tel que $p_{J}\geqslant  0,05-10^{- \text{K}}$
          \end{enumerate}
          \item
          \begin{enumerate}[label=\alph*.]
               \item
               Pour un salarié donné, l'évènement $S$ : \textit{"Le salarié est malade"}  correspond à une épreuve de Bernouilli de probabilité $p\left(S\right)=0,05$
               \par
               Si l'on s'intéresse à l'état de santé des 220 employés, on répète 220 épreuves de Bernouilli identiques et indépendantes puisque par hypothèse : \textit{l'état de santé d'un salarié ne dépend pas de l'état de santé de ses collègues}.
               \par
               La variable aléatoire $X$ suit donc une loi binomiale de paramètres $n=220$ et $p=0,05$.
               \par
               L'espérance mathématique de $X$ est :
               \par
               $\mu =np=220\times 0,05=11$
               \par
               Son écart-type est :
               \par
               $\sigma =\sqrt{np\left(1-p\right)}=\sqrt{10,45}\approx 3,23$ à $10^{-2}$ près
               \item
               La probabilité cherchée est $p\left(7\leqslant X\leqslant 15\right)$. Or :
               \par
               $p\left(7\leqslant X\leqslant 15\right)=p\left(\frac{7-\mu }{\sigma }\leqslant \frac{X-\mu }{\sigma }\leqslant \frac{15-\mu }{\sigma }\right)\approx p\left(-1,24\leqslant \frac{X-\mu }{\sigma }\leqslant 1,24\right) $
               \par
               $p\left(7\leqslant X\leqslant 15\right)\approx p\left(\frac{X-\mu }{\sigma }\leqslant 1,24\right)-p\left(\frac{X-\mu }{\sigma }\leqslant -1,24\right)\approx  0,892-0,108$
               \par
               $p\left(7\leqslant X\leqslant 15\right)\approx 0,78$ à $10^{-2}$ près.
          \end{enumerate}
     \end{enumerate}
\end{corrige}

\end{document}