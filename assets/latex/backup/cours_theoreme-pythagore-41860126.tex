\documentclass[a4paper]{article}

%================================================================================================================================
%
% Packages
%
%================================================================================================================================

\usepackage[T1]{fontenc} 	% pour caractères accentués
\usepackage[utf8]{inputenc}  % encodage utf8
\usepackage[french]{babel}	% langue : français
\usepackage{fourier}			% caractères plus lisibles
\usepackage[dvipsnames]{xcolor} % couleurs
\usepackage{fancyhdr}		% réglage header footer
\usepackage{needspace}		% empêcher sauts de page mal placés
\usepackage{graphicx}		% pour inclure des graphiques
\usepackage{enumitem,cprotect}		% personnalise les listes d'items (nécessaire pour ol, al ...)
\usepackage{hyperref}		% Liens hypertexte
\usepackage{pstricks,pst-all,pst-node,pstricks-add,pst-math,pst-plot,pst-tree,pst-eucl} % pstricks
\usepackage[a4paper,includeheadfoot,top=2cm,left=3cm, bottom=2cm,right=3cm]{geometry} % marges etc.
\usepackage{comment}			% commentaires multilignes
\usepackage{amsmath,environ} % maths (matrices, etc.)
\usepackage{amssymb,makeidx}
\usepackage{bm}				% bold maths
\usepackage{tabularx}		% tableaux
\usepackage{colortbl}		% tableaux en couleur
\usepackage{fontawesome}		% Fontawesome
\usepackage{environ}			% environment with command
\usepackage{fp}				% calculs pour ps-tricks
\usepackage{multido}			% pour ps tricks
\usepackage[np]{numprint}	% formattage nombre
\usepackage{tikz,tkz-tab} 			% package principal TikZ
\usepackage{pgfplots}   % axes
\usepackage{mathrsfs}    % cursives
\usepackage{calc}			% calcul taille boites
\usepackage[scaled=0.875]{helvet} % font sans serif
\usepackage{svg} % svg
\usepackage{scrextend} % local margin
\usepackage{scratch} %scratch
\usepackage{multicol} % colonnes
%\usepackage{infix-RPN,pst-func} % formule en notation polanaise inversée
\usepackage{listings}

%================================================================================================================================
%
% Réglages de base
%
%================================================================================================================================

\lstset{
language=Python,   % R code
literate=
{á}{{\'a}}1
{à}{{\`a}}1
{ã}{{\~a}}1
{é}{{\'e}}1
{è}{{\`e}}1
{ê}{{\^e}}1
{í}{{\'i}}1
{ó}{{\'o}}1
{õ}{{\~o}}1
{ú}{{\'u}}1
{ü}{{\"u}}1
{ç}{{\c{c}}}1
{~}{{ }}1
}


\definecolor{codegreen}{rgb}{0,0.6,0}
\definecolor{codegray}{rgb}{0.5,0.5,0.5}
\definecolor{codepurple}{rgb}{0.58,0,0.82}
\definecolor{backcolour}{rgb}{0.95,0.95,0.92}

\lstdefinestyle{mystyle}{
    backgroundcolor=\color{backcolour},   
    commentstyle=\color{codegreen},
    keywordstyle=\color{magenta},
    numberstyle=\tiny\color{codegray},
    stringstyle=\color{codepurple},
    basicstyle=\ttfamily\footnotesize,
    breakatwhitespace=false,         
    breaklines=true,                 
    captionpos=b,                    
    keepspaces=true,                 
    numbers=left,                    
xleftmargin=2em,
framexleftmargin=2em,            
    showspaces=false,                
    showstringspaces=false,
    showtabs=false,                  
    tabsize=2,
    upquote=true
}

\lstset{style=mystyle}


\lstset{style=mystyle}
\newcommand{\imgdir}{C:/laragon/www/newmc/assets/imgsvg/}
\newcommand{\imgsvgdir}{C:/laragon/www/newmc/assets/imgsvg/}

\definecolor{mcgris}{RGB}{220, 220, 220}% ancien~; pour compatibilité
\definecolor{mcbleu}{RGB}{52, 152, 219}
\definecolor{mcvert}{RGB}{125, 194, 70}
\definecolor{mcmauve}{RGB}{154, 0, 215}
\definecolor{mcorange}{RGB}{255, 96, 0}
\definecolor{mcturquoise}{RGB}{0, 153, 153}
\definecolor{mcrouge}{RGB}{255, 0, 0}
\definecolor{mclightvert}{RGB}{205, 234, 190}

\definecolor{gris}{RGB}{220, 220, 220}
\definecolor{bleu}{RGB}{52, 152, 219}
\definecolor{vert}{RGB}{125, 194, 70}
\definecolor{mauve}{RGB}{154, 0, 215}
\definecolor{orange}{RGB}{255, 96, 0}
\definecolor{turquoise}{RGB}{0, 153, 153}
\definecolor{rouge}{RGB}{255, 0, 0}
\definecolor{lightvert}{RGB}{205, 234, 190}
\setitemize[0]{label=\color{lightvert}  $\bullet$}

\pagestyle{fancy}
\renewcommand{\headrulewidth}{0.2pt}
\fancyhead[L]{maths-cours.fr}
\fancyhead[R]{\thepage}
\renewcommand{\footrulewidth}{0.2pt}
\fancyfoot[C]{}

\newcolumntype{C}{>{\centering\arraybackslash}X}
\newcolumntype{s}{>{\hsize=.35\hsize\arraybackslash}X}

\setlength{\parindent}{0pt}		 
\setlength{\parskip}{3mm}
\setlength{\headheight}{1cm}

\def\ebook{ebook}
\def\book{book}
\def\web{web}
\def\type{web}

\newcommand{\vect}[1]{\overrightarrow{\,\mathstrut#1\,}}

\def\Oij{$\left(\text{O}~;~\vect{\imath},~\vect{\jmath}\right)$}
\def\Oijk{$\left(\text{O}~;~\vect{\imath},~\vect{\jmath},~\vect{k}\right)$}
\def\Ouv{$\left(\text{O}~;~\vect{u},~\vect{v}\right)$}

\hypersetup{breaklinks=true, colorlinks = true, linkcolor = OliveGreen, urlcolor = OliveGreen, citecolor = OliveGreen, pdfauthor={Didier BONNEL - https://www.maths-cours.fr} } % supprime les bordures autour des liens

\renewcommand{\arg}[0]{\text{arg}}

\everymath{\displaystyle}

%================================================================================================================================
%
% Macros - Commandes
%
%================================================================================================================================

\newcommand\meta[2]{    			% Utilisé pour créer le post HTML.
	\def\titre{titre}
	\def\url{url}
	\def\arg{#1}
	\ifx\titre\arg
		\newcommand\maintitle{#2}
		\fancyhead[L]{#2}
		{\Large\sffamily \MakeUppercase{#2}}
		\vspace{1mm}\textcolor{mcvert}{\hrule}
	\fi 
	\ifx\url\arg
		\fancyfoot[L]{\href{https://www.maths-cours.fr#2}{\black \footnotesize{https://www.maths-cours.fr#2}}}
	\fi 
}


\newcommand\TitreC[1]{    		% Titre centré
     \needspace{3\baselineskip}
     \begin{center}\textbf{#1}\end{center}
}

\newcommand\newpar{    		% paragraphe
     \par
}

\newcommand\nosp {    		% commande vide (pas d'espace)
}
\newcommand{\id}[1]{} %ignore

\newcommand\boite[2]{				% Boite simple sans titre
	\vspace{5mm}
	\setlength{\fboxrule}{0.2mm}
	\setlength{\fboxsep}{5mm}	
	\fcolorbox{#1}{#1!3}{\makebox[\linewidth-2\fboxrule-2\fboxsep]{
  		\begin{minipage}[t]{\linewidth-2\fboxrule-4\fboxsep}\setlength{\parskip}{3mm}
  			 #2
  		\end{minipage}
	}}
	\vspace{5mm}
}

\newcommand\CBox[4]{				% Boites
	\vspace{5mm}
	\setlength{\fboxrule}{0.2mm}
	\setlength{\fboxsep}{5mm}
	
	\fcolorbox{#1}{#1!3}{\makebox[\linewidth-2\fboxrule-2\fboxsep]{
		\begin{minipage}[t]{1cm}\setlength{\parskip}{3mm}
	  		\textcolor{#1}{\LARGE{#2}}    
 	 	\end{minipage}  
  		\begin{minipage}[t]{\linewidth-2\fboxrule-4\fboxsep}\setlength{\parskip}{3mm}
			\raisebox{1.2mm}{\normalsize\sffamily{\textcolor{#1}{#3}}}						
  			 #4
  		\end{minipage}
	}}
	\vspace{5mm}
}

\newcommand\cadre[3]{				% Boites convertible html
	\par
	\vspace{2mm}
	\setlength{\fboxrule}{0.1mm}
	\setlength{\fboxsep}{5mm}
	\fcolorbox{#1}{white}{\makebox[\linewidth-2\fboxrule-2\fboxsep]{
  		\begin{minipage}[t]{\linewidth-2\fboxrule-4\fboxsep}\setlength{\parskip}{3mm}
			\raisebox{-2.5mm}{\sffamily \small{\textcolor{#1}{\MakeUppercase{#2}}}}		
			\par		
  			 #3
 	 		\end{minipage}
	}}
		\vspace{2mm}
	\par
}

\newcommand\bloc[3]{				% Boites convertible html sans bordure
     \needspace{2\baselineskip}
     {\sffamily \small{\textcolor{#1}{\MakeUppercase{#2}}}}    
		\par		
  			 #3
		\par
}

\newcommand\CHelp[1]{
     \CBox{Plum}{\faInfoCircle}{À RETENIR}{#1}
}

\newcommand\CUp[1]{
     \CBox{NavyBlue}{\faThumbsOUp}{EN PRATIQUE}{#1}
}

\newcommand\CInfo[1]{
     \CBox{Sepia}{\faArrowCircleRight}{REMARQUE}{#1}
}

\newcommand\CRedac[1]{
     \CBox{PineGreen}{\faEdit}{BIEN R\'EDIGER}{#1}
}

\newcommand\CError[1]{
     \CBox{Red}{\faExclamationTriangle}{ATTENTION}{#1}
}

\newcommand\TitreExo[2]{
\needspace{4\baselineskip}
 {\sffamily\large EXERCICE #1\ (\emph{#2 points})}
\vspace{5mm}
}

\newcommand\img[2]{
          \includegraphics[width=#2\paperwidth]{\imgdir#1}
}

\newcommand\imgsvg[2]{
       \begin{center}   \includegraphics[width=#2\paperwidth]{\imgsvgdir#1} \end{center}
}


\newcommand\Lien[2]{
     \href{#1}{#2 \tiny \faExternalLink}
}
\newcommand\mcLien[2]{
     \href{https~://www.maths-cours.fr/#1}{#2 \tiny \faExternalLink}
}

\newcommand{\euro}{\eurologo{}}

%================================================================================================================================
%
% Macros - Environement
%
%================================================================================================================================

\newenvironment{tex}{ %
}
{%
}

\newenvironment{indente}{ %
	\setlength\parindent{10mm}
}

{
	\setlength\parindent{0mm}
}

\newenvironment{corrige}{%
     \needspace{3\baselineskip}
     \medskip
     \textbf{\textsc{Corrigé}}
     \medskip
}
{
}

\newenvironment{extern}{%
     \begin{center}
     }
     {
     \end{center}
}

\NewEnviron{code}{%
	\par
     \boite{gray}{\texttt{%
     \BODY
     }}
     \par
}

\newenvironment{vbloc}{% boite sans cadre empeche saut de page
     \begin{minipage}[t]{\linewidth}
     }
     {
     \end{minipage}
}
\NewEnviron{h2}{%
    \needspace{3\baselineskip}
    \vspace{0.6cm}
	\noindent \MakeUppercase{\sffamily \large \BODY}
	\vspace{1mm}\textcolor{mcgris}{\hrule}\vspace{0.4cm}
	\par
}{}

\NewEnviron{h3}{%
    \needspace{3\baselineskip}
	\vspace{5mm}
	\textsc{\BODY}
	\par
}

\NewEnviron{margeneg}{ %
\begin{addmargin}[-1cm]{0cm}
\BODY
\end{addmargin}
}

\NewEnviron{html}{%
}

\begin{document}
\meta{url}{/cours/theoreme-pythagore/}
\meta{pid}{1571}
\meta{titre}{Théorème de Pythagore -Trigonométrie}
\meta{type}{cours}
\begin{h2}1. Théorème de Pythagore (rappels de 4ème)\end{h2}
\cadre{rouge}{Théorème de Pythagore}{% id="d10"
     Si un triangle est rectangle alors le carré de la longueur de l'hypoténuse est égal à la somme des carrés des longueurs des côtés de l'angle droit.
}
\bloc{cyan}{Remarque}{% id="r10"
     \begin{itemize}
          \item On rappelle que l'hypoténuse est le côté opposé à l'angle droit et le côté ayant la plus grande longueur.
          \item Ce théorème sert à calculer la longueur d'un côté connaissant les longueurs des deux autres lorsque l'on \textbf{sait} que le triangle est rectangle
     \end{itemize}
}
\bloc{orange}{Exemple}{% id="e10"
     Soit $ABC$ un triangle rectangle en $A$ tel que $AB=4$cm et $AC=3$cm
     \begin{center}
          \begin{extern}%width="200" alt="sinus cosinus tangente"
               \psset{xunit=1.0cm,yunit=1.0cm,algebraic=true,dimen=middle,dotstyle=*,dotsize=5pt 0,linewidth=1.pt,arrowsize=3pt 2,arrowinset=0.25}
               \begin{pspicture*}(0.,0.5)(6.,5)
                    \psframe[linewidth=0.6pt](1,1)(1.2,1.2)
                    \psline(1.,1.)(5.,1.)
                    \psline(1.,4.)(5.,1.)
                    \psline(1.,4.)(1.,1.)
                    \rput[t](3,0.9){$4$}
                    \rput[r](0.9,2.5){$3$}
                    \rput[bl](0.6,0.8){$A$}
                    \rput[bl](5.1,0.84){$B$}
                    \rput[bl](0.6,4.07){$C$}
               \end{pspicture*}
          \end{extern}
     \end{center}
     D'après le théorème de Pythagore :
     \par
     $ BC^{2}=AB^{2}+AC^{2}=4^{2}+3^{2}=16+9=25$
     \par
     Donc $BC=\sqrt{25}=5$cm.
}
\cadre{rouge}{Théorème (Réciproque du théorème de Pythagore)}{% id="t20"
     Un triangle est rectangle si et seulement si le carré de la longueur du plus grand coté est égal à la somme des carrés des longueurs des deux autres côtés.
}
\bloc{cyan}{Remarques}{% id="r20"
     Ce théorème sert à \textbf{démontrer} qu'un triangle est un triangle rectangle lorsqu'on connait les longueurs de ses trois côtés.
}
\bloc{orange}{Exemple}{% id="e20"
     Soit $ABC$ un triangle tel que $AB=12$cm,  $AC=5$cm et $BC=13$cm.
     \par
     $ABC$ est-il rectangle ?
     \par
     On calcule séparément $BC^{2}$ (carré de la longueur du plus grand coté) et $AB^{2}+AC^{2}$ (somme des carrés des longueurs des deux autres cotés) :
     \par
     $BC^{2}=13^{2}=169$
     \par
     $AB^{2}+AC^{2}=12^{2}+5^{2}=144+25=169$
     \par
     $BC^{2} = AB^{2}+AC^{2}$ donc le triangle $ABC$ est rectangle en $A$ d'après la réciproque du théorème de Pythagore.
}
\begin{h2}2. Trigonométrie\end{h2}
\cadre{bleu}{Définitions}{% id="d50"
     Soit $ABC$ un triangle rectangle en $A$ :
     \begin{itemize}
          \item le \textbf{sinus} de $\widehat{ABC}$ est le nombre :
          \par
          $\sin\left(\widehat{ABC}\right)=$\nosp$\frac{\text{longueur\ du\ côté\ opposé\ à\ B}}{\text{longueur\ de\ l \prime hypoténuse}}$
          \item le \textbf{cosinus} de $\widehat{ABC}$ est le nombre :
          \par
          $\cos\left(\widehat{ABC}\right)=$\nosp$\frac{\text{longueur\ du\ côté\ adjacent\ à\ B}}{\text{longueur\ de\ l hypoténuse}}$
          \item la \textbf{tangente} de $\widehat{ABC}$ est le nombre :
          \par
          $\tan\left(\widehat{ABC}\right)=$\nosp$\frac{\text{longueur\ du\ côté\ opposé\ à\ B}}{\text{longueur\ du\ côté\ adjacent\ à\ B}}$
     \end{itemize}
}
\bloc{orange}{Exemple}{% id="e50"
     \begin{center}
          \begin{extern}%width="200" alt="sinus cosinus tangente"
               \psset{xunit=1.0cm,yunit=1.0cm,algebraic=true,dimen=middle,dotstyle=*,dotsize=5pt 0,linewidth=1.pt,arrowsize=3pt 2,arrowinset=0.25}
               \begin{pspicture*}(0.,0.5)(6.,5)
                    \psframe[linewidth=0.6pt](1,1)(1.2,1.2)
                    \pscustom[linewidth=0.4pt,linecolor=mcvert,fillcolor=mcvert,fillstyle=solid,opacity=0.1]{
                         \parametricplot{2.5}{3.1416}{0.66*cos(t)+5.|0.66*sin(t)+1.}
                    \lineto(5.,1.)\closepath}
                    \psline(1.,1.)(5.,1.)
                    \psline(1.,4.)(5.,1.)
                    \psline(1.,4.)(1.,1.)
                    \rput[t](3,0.9){$4$}
                    \rput[l](3.2,2.6){$5$}
                    \rput[r](0.9,2.5){$3$}
                    \rput[bl](0.6,0.8){$A$}
                    \rput[bl](5.1,0.84){$B$}
                    \rput[bl](0.6,4.07){$C$}
               \end{pspicture*}
          \end{extern}
     \end{center}
     Dans le triangle rectangle $ABC$ ci-dessus :
     \begin{itemize}
          \item $\sin\left(\widehat{ABC}\right)=\frac{AC}{BC}=\frac{3}{5}=0,6$
          \item $\cos\left(\widehat{ABC}\right)=\frac{AB}{BC}=\frac{4}{5}=0,8$
          \item $\tan\left(\widehat{ABC}\right)=\frac{AC}{AB}=\frac{3}{4}=0,75$
     \end{itemize}
}
\bloc{cyan}{Remarques}{% id="r50"
     \begin{itemize}
          \item Les sinus, cosinus et tangente n'ont pas d'unité !
          \item Les sinus et cosinus d'un angle aigu sont compris entre 0 et 1. Par contre, la tangente peut être supérieure à 1.
          \item Connaissant le sinus, il est possible de calculer la mesure de l'angle en degré à la calculatrice à l'aide de la touche \textbf{$\sin^{-1}$} (ou \textbf{Arcsin} ou \textbf{asin} suivant le modèle de la calculatrice). Vérifiez bien que la calculatrice est en mode \textbf{degré} !
     \end{itemize}
}
\cadre{vert}{Propriétés}{% id="p60"
     Pour tout angle aigu $\widehat{a}$ d'un triangle rectangle :
     \begin{center}$\left(\cos \widehat{a}\right)^{2}+\left(\sin \widehat{a}\right)^{2}=1$\end{center}
     \begin{center}$\tan \widehat{a}=\frac{\sin \widehat{a}}{\cos \widehat{a}}$\end{center}
}
\bloc{cyan}{Remarque}{% id="r60"
     Pour simplifier les notations, on écrit en général $\cos^{2} \widehat{a}$ pour $\left(\cos \widehat{a}\right)^{2}$. La première formule s'écrit alors :
     \par
     $\cos^{2} \widehat{a}+\sin^{2} \widehat{a}=1$
}
\bloc{cyan}{Démonstrations}{% id="m60"
     \begin{center}
          \begin{extern}%width="200" alt=""
               \psset{xunit=1.0cm,yunit=1.0cm,algebraic=true,dimen=middle,dotstyle=*,dotsize=5pt 0,linewidth=1.pt,arrowsize=3pt 2,arrowinset=0.25}
               \begin{pspicture*}(0.,0.5)(6.,5)
                    \psframe[linewidth=0.6pt](1,1)(1.2,1.2)
                    \pscustom[linewidth=0.4pt,linecolor=mcvert,fillcolor=mcvert,fillstyle=solid,opacity=0.1]{
                         \parametricplot{2.5}{3.1416}{0.66*cos(t)+5.|0.66*sin(t)+1.}
                    \lineto(5.,1.)\closepath}
                    \rput[tl](4.05,1.45){\color{mcvert}{$\hat a$}}
                    \psline(1.,1.)(5.,1.)
                    \psline(1.,4.)(5.,1.)
                    \psline(1.,4.)(1.,1.)
                    \rput[bl](0.6,0.8){$A$}
                    \rput[bl](5.1,0.84){$B$}
                    \rput[bl](0.6,4.07){$C$}
               \end{pspicture*}
          \end{extern}
     \end{center}
     \begin{itemize}
          \item $\cos \widehat{a}=\frac{AB}{BC}$ donc $\left(\cos \widehat{a}\right)^{2}=\frac{AB^{2}}{BC^{2}}$
          \par
          $\sin \widehat{a}=\frac{AC}{BC}$ donc $\left(\sin \widehat{a}\right)^{2}=\frac{AC^{2}}{BC^{2}}$
          \par
          Par conséquent :
          \par
          $\left(\cos \widehat{a}\right)^{2}+\left(\sin \widehat{a}\right)^{2}=\frac{AB^{2}}{BC^{2}}+\frac{AC^{2}}{BC^{2}}=\frac{AB^{2}+AC^{2}}{BC^{2}}$
          \par
          Or d'après le théorème de Pythagore $AB^{2}+AC^{2}=BC^{2}$ donc :
          \par
          $\left(\cos \widehat{a}\right)^{2}+\left(\sin \widehat{a}\right)^{2}=\frac{BC^{2}}{BC^{2}}=1$ après simplification par $BC^{2}$
          \par
          \item $\frac{\sin \widehat{a}}{\cos \widehat{a}}=\frac{\frac{AC}{BC}}{\frac{AB}{BC}}=\frac{AC}{BC}\times \frac{BC}{AB}=\frac{AC}{AB}$ après simplification par $BC$.
          \par
          Or, $\frac{AC}{AB}=\tan \widehat{a}$, par conséquent :
          \par
          $\tan \widehat{a}=\frac{\sin \widehat{a}}{\cos \widehat{a}}.$
     \end{itemize}
}
\bloc{orange}{Exemple}{% id="e60"
     On sait que le cosinus d'un angle $\widehat{a}$ vaut $0,5$. Calculer une valeur approchée à $10^{-2}$ du sinus puis de la tangente de cet angle.
     \par
     $\cos^{2} \widehat{a}+\sin^{2} \widehat{a}=1$
     \par
     $\sin^{2} \widehat{a}=1-\cos^{2}\widehat{a}=1-0,5^{2}=0,75$
     \par
     $\sin \widehat{a}=\sqrt{0,75}\approx 0,87$ à $10^{-2}$ près
     \par
     $\tan \widehat{a}=\frac{\sin \widehat{a}}{\cos \widehat{a}}=\frac{\sqrt{0,75}}{0,5}\approx 1,73$ à $10^{-2}$ près.
}

\end{document}