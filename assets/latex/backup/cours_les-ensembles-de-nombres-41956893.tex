\documentclass[a4paper]{article}

%================================================================================================================================
%
% Packages
%
%================================================================================================================================

\usepackage[T1]{fontenc} 	% pour caractères accentués
\usepackage[utf8]{inputenc}  % encodage utf8
\usepackage[french]{babel}	% langue : français
\usepackage{fourier}			% caractères plus lisibles
\usepackage[dvipsnames]{xcolor} % couleurs
\usepackage{fancyhdr}		% réglage header footer
\usepackage{needspace}		% empêcher sauts de page mal placés
\usepackage{graphicx}		% pour inclure des graphiques
\usepackage{enumitem,cprotect}		% personnalise les listes d'items (nécessaire pour ol, al ...)
\usepackage{hyperref}		% Liens hypertexte
\usepackage{pstricks,pst-all,pst-node,pstricks-add,pst-math,pst-plot,pst-tree,pst-eucl} % pstricks
\usepackage[a4paper,includeheadfoot,top=2cm,left=3cm, bottom=2cm,right=3cm]{geometry} % marges etc.
\usepackage{comment}			% commentaires multilignes
\usepackage{amsmath,environ} % maths (matrices, etc.)
\usepackage{amssymb,makeidx}
\usepackage{bm}				% bold maths
\usepackage{tabularx}		% tableaux
\usepackage{colortbl}		% tableaux en couleur
\usepackage{fontawesome}		% Fontawesome
\usepackage{environ}			% environment with command
\usepackage{fp}				% calculs pour ps-tricks
\usepackage{multido}			% pour ps tricks
\usepackage[np]{numprint}	% formattage nombre
\usepackage{tikz,tkz-tab} 			% package principal TikZ
\usepackage{pgfplots}   % axes
\usepackage{mathrsfs}    % cursives
\usepackage{calc}			% calcul taille boites
\usepackage[scaled=0.875]{helvet} % font sans serif
\usepackage{svg} % svg
\usepackage{scrextend} % local margin
\usepackage{scratch} %scratch
\usepackage{multicol} % colonnes
%\usepackage{infix-RPN,pst-func} % formule en notation polanaise inversée
\usepackage{listings}

%================================================================================================================================
%
% Réglages de base
%
%================================================================================================================================

\lstset{
language=Python,   % R code
literate=
{á}{{\'a}}1
{à}{{\`a}}1
{ã}{{\~a}}1
{é}{{\'e}}1
{è}{{\`e}}1
{ê}{{\^e}}1
{í}{{\'i}}1
{ó}{{\'o}}1
{õ}{{\~o}}1
{ú}{{\'u}}1
{ü}{{\"u}}1
{ç}{{\c{c}}}1
{~}{{ }}1
}


\definecolor{codegreen}{rgb}{0,0.6,0}
\definecolor{codegray}{rgb}{0.5,0.5,0.5}
\definecolor{codepurple}{rgb}{0.58,0,0.82}
\definecolor{backcolour}{rgb}{0.95,0.95,0.92}

\lstdefinestyle{mystyle}{
    backgroundcolor=\color{backcolour},   
    commentstyle=\color{codegreen},
    keywordstyle=\color{magenta},
    numberstyle=\tiny\color{codegray},
    stringstyle=\color{codepurple},
    basicstyle=\ttfamily\footnotesize,
    breakatwhitespace=false,         
    breaklines=true,                 
    captionpos=b,                    
    keepspaces=true,                 
    numbers=left,                    
xleftmargin=2em,
framexleftmargin=2em,            
    showspaces=false,                
    showstringspaces=false,
    showtabs=false,                  
    tabsize=2,
    upquote=true
}

\lstset{style=mystyle}


\lstset{style=mystyle}
\newcommand{\imgdir}{C:/laragon/www/newmc/assets/imgsvg/}
\newcommand{\imgsvgdir}{C:/laragon/www/newmc/assets/imgsvg/}

\definecolor{mcgris}{RGB}{220, 220, 220}% ancien~; pour compatibilité
\definecolor{mcbleu}{RGB}{52, 152, 219}
\definecolor{mcvert}{RGB}{125, 194, 70}
\definecolor{mcmauve}{RGB}{154, 0, 215}
\definecolor{mcorange}{RGB}{255, 96, 0}
\definecolor{mcturquoise}{RGB}{0, 153, 153}
\definecolor{mcrouge}{RGB}{255, 0, 0}
\definecolor{mclightvert}{RGB}{205, 234, 190}

\definecolor{gris}{RGB}{220, 220, 220}
\definecolor{bleu}{RGB}{52, 152, 219}
\definecolor{vert}{RGB}{125, 194, 70}
\definecolor{mauve}{RGB}{154, 0, 215}
\definecolor{orange}{RGB}{255, 96, 0}
\definecolor{turquoise}{RGB}{0, 153, 153}
\definecolor{rouge}{RGB}{255, 0, 0}
\definecolor{lightvert}{RGB}{205, 234, 190}
\setitemize[0]{label=\color{lightvert}  $\bullet$}

\pagestyle{fancy}
\renewcommand{\headrulewidth}{0.2pt}
\fancyhead[L]{maths-cours.fr}
\fancyhead[R]{\thepage}
\renewcommand{\footrulewidth}{0.2pt}
\fancyfoot[C]{}

\newcolumntype{C}{>{\centering\arraybackslash}X}
\newcolumntype{s}{>{\hsize=.35\hsize\arraybackslash}X}

\setlength{\parindent}{0pt}		 
\setlength{\parskip}{3mm}
\setlength{\headheight}{1cm}

\def\ebook{ebook}
\def\book{book}
\def\web{web}
\def\type{web}

\newcommand{\vect}[1]{\overrightarrow{\,\mathstrut#1\,}}

\def\Oij{$\left(\text{O}~;~\vect{\imath},~\vect{\jmath}\right)$}
\def\Oijk{$\left(\text{O}~;~\vect{\imath},~\vect{\jmath},~\vect{k}\right)$}
\def\Ouv{$\left(\text{O}~;~\vect{u},~\vect{v}\right)$}

\hypersetup{breaklinks=true, colorlinks = true, linkcolor = OliveGreen, urlcolor = OliveGreen, citecolor = OliveGreen, pdfauthor={Didier BONNEL - https://www.maths-cours.fr} } % supprime les bordures autour des liens

\renewcommand{\arg}[0]{\text{arg}}

\everymath{\displaystyle}

%================================================================================================================================
%
% Macros - Commandes
%
%================================================================================================================================

\newcommand\meta[2]{    			% Utilisé pour créer le post HTML.
	\def\titre{titre}
	\def\url{url}
	\def\arg{#1}
	\ifx\titre\arg
		\newcommand\maintitle{#2}
		\fancyhead[L]{#2}
		{\Large\sffamily \MakeUppercase{#2}}
		\vspace{1mm}\textcolor{mcvert}{\hrule}
	\fi 
	\ifx\url\arg
		\fancyfoot[L]{\href{https://www.maths-cours.fr#2}{\black \footnotesize{https://www.maths-cours.fr#2}}}
	\fi 
}


\newcommand\TitreC[1]{    		% Titre centré
     \needspace{3\baselineskip}
     \begin{center}\textbf{#1}\end{center}
}

\newcommand\newpar{    		% paragraphe
     \par
}

\newcommand\nosp {    		% commande vide (pas d'espace)
}
\newcommand{\id}[1]{} %ignore

\newcommand\boite[2]{				% Boite simple sans titre
	\vspace{5mm}
	\setlength{\fboxrule}{0.2mm}
	\setlength{\fboxsep}{5mm}	
	\fcolorbox{#1}{#1!3}{\makebox[\linewidth-2\fboxrule-2\fboxsep]{
  		\begin{minipage}[t]{\linewidth-2\fboxrule-4\fboxsep}\setlength{\parskip}{3mm}
  			 #2
  		\end{minipage}
	}}
	\vspace{5mm}
}

\newcommand\CBox[4]{				% Boites
	\vspace{5mm}
	\setlength{\fboxrule}{0.2mm}
	\setlength{\fboxsep}{5mm}
	
	\fcolorbox{#1}{#1!3}{\makebox[\linewidth-2\fboxrule-2\fboxsep]{
		\begin{minipage}[t]{1cm}\setlength{\parskip}{3mm}
	  		\textcolor{#1}{\LARGE{#2}}    
 	 	\end{minipage}  
  		\begin{minipage}[t]{\linewidth-2\fboxrule-4\fboxsep}\setlength{\parskip}{3mm}
			\raisebox{1.2mm}{\normalsize\sffamily{\textcolor{#1}{#3}}}						
  			 #4
  		\end{minipage}
	}}
	\vspace{5mm}
}

\newcommand\cadre[3]{				% Boites convertible html
	\par
	\vspace{2mm}
	\setlength{\fboxrule}{0.1mm}
	\setlength{\fboxsep}{5mm}
	\fcolorbox{#1}{white}{\makebox[\linewidth-2\fboxrule-2\fboxsep]{
  		\begin{minipage}[t]{\linewidth-2\fboxrule-4\fboxsep}\setlength{\parskip}{3mm}
			\raisebox{-2.5mm}{\sffamily \small{\textcolor{#1}{\MakeUppercase{#2}}}}		
			\par		
  			 #3
 	 		\end{minipage}
	}}
		\vspace{2mm}
	\par
}

\newcommand\bloc[3]{				% Boites convertible html sans bordure
     \needspace{2\baselineskip}
     {\sffamily \small{\textcolor{#1}{\MakeUppercase{#2}}}}    
		\par		
  			 #3
		\par
}

\newcommand\CHelp[1]{
     \CBox{Plum}{\faInfoCircle}{À RETENIR}{#1}
}

\newcommand\CUp[1]{
     \CBox{NavyBlue}{\faThumbsOUp}{EN PRATIQUE}{#1}
}

\newcommand\CInfo[1]{
     \CBox{Sepia}{\faArrowCircleRight}{REMARQUE}{#1}
}

\newcommand\CRedac[1]{
     \CBox{PineGreen}{\faEdit}{BIEN R\'EDIGER}{#1}
}

\newcommand\CError[1]{
     \CBox{Red}{\faExclamationTriangle}{ATTENTION}{#1}
}

\newcommand\TitreExo[2]{
\needspace{4\baselineskip}
 {\sffamily\large EXERCICE #1\ (\emph{#2 points})}
\vspace{5mm}
}

\newcommand\img[2]{
          \includegraphics[width=#2\paperwidth]{\imgdir#1}
}

\newcommand\imgsvg[2]{
       \begin{center}   \includegraphics[width=#2\paperwidth]{\imgsvgdir#1} \end{center}
}


\newcommand\Lien[2]{
     \href{#1}{#2 \tiny \faExternalLink}
}
\newcommand\mcLien[2]{
     \href{https~://www.maths-cours.fr/#1}{#2 \tiny \faExternalLink}
}

\newcommand{\euro}{\eurologo{}}

%================================================================================================================================
%
% Macros - Environement
%
%================================================================================================================================

\newenvironment{tex}{ %
}
{%
}

\newenvironment{indente}{ %
	\setlength\parindent{10mm}
}

{
	\setlength\parindent{0mm}
}

\newenvironment{corrige}{%
     \needspace{3\baselineskip}
     \medskip
     \textbf{\textsc{Corrigé}}
     \medskip
}
{
}

\newenvironment{extern}{%
     \begin{center}
     }
     {
     \end{center}
}

\NewEnviron{code}{%
	\par
     \boite{gray}{\texttt{%
     \BODY
     }}
     \par
}

\newenvironment{vbloc}{% boite sans cadre empeche saut de page
     \begin{minipage}[t]{\linewidth}
     }
     {
     \end{minipage}
}
\NewEnviron{h2}{%
    \needspace{3\baselineskip}
    \vspace{0.6cm}
	\noindent \MakeUppercase{\sffamily \large \BODY}
	\vspace{1mm}\textcolor{mcgris}{\hrule}\vspace{0.4cm}
	\par
}{}

\NewEnviron{h3}{%
    \needspace{3\baselineskip}
	\vspace{5mm}
	\textsc{\BODY}
	\par
}

\NewEnviron{margeneg}{ %
\begin{addmargin}[-1cm]{0cm}
\BODY
\end{addmargin}
}

\NewEnviron{html}{%
}

\begin{document}
\meta{url}{/cours/les-ensembles-de-nombres/}
\meta{pid}{10887}
\meta{titre}{Ensembles de nombres - Intervalles - Valeurs absolues}
\meta{type}{cours}
\begin{h2}I - Les ensembles de nombres \end{h2}
\cadre{bleu}{Définition}{% 
     $\mathbb{N}=\left\{0; 1; 2; 3; 4; 5; \cdots \right\}$ est l'ensemble des \textbf{entiers naturels}.
} 
\bloc{cyan}{Remarque}{% 
     On emploie le signe $ \in $ pour indiquer qu'un nombre appartient à un ensemble. On écrira par exemple: $2\in \mathbb{N}$ et $\frac{2}{3} \notin \mathbb{N}$. 
}
\cadre{bleu}{Définition}{%
     $\mathbb{Z}= \left\{\cdots;  -3; -2 ; -1; 0; 1; 2; 3; \cdots \right\}$ est l'ensemble des \textbf{entiers relatifs}. 
}
\cadre{bleu}{Définition}{%
     $\mathbb{D}$ est l'ensemble des \textbf{nombres décimaux}. Les nombres décimaux peuvent s'écrire sous la forme d'une fraction dont le dénominateur est une puissance de 10 (1; 10; 100; 1 000; ...). \\
     Ils peuvent aussi s'écrire sous forme décimale dont le nombre de chiffres après la virgule est finie. 
}

\bloc{cyan}{Remarques}{%
     \begin{itemize}
          \item \textit{"fini"} signifie ici  \og qui n'est pas infini \fg{}. 
          \item Les calculatrices les plus simples ne manipulent que des nombres décimaux pour effectuer les calculs. Certaines permettent des opérations sur les fractions. Quelques modèles plus avancés (effectuant du "calcul formel") peuvent également effectuer des calculs avec des nombres irrationnels. 
     \end{itemize}
}
\cadre{bleu}{Définition}{%
     $\mathbb{Q}$ est l'ensemble des \textbf{nombres rationnels}. Les nombres rationnels peuvent s'écrire sous la forme d'une fraction dont le numérateur et le dénominateur sont des entiers relatifs.
}
\cadre{bleu}{Définition}{%
     $\mathbb{R}$ est l'ensemble des \textbf{nombres réels}. Les nombres réels sont tous les nombres connus (en Seconde...). 

}
\bloc{cyan}{Remarque}{%
     Les nombres réels qui ne sont pas rationnels (comme $\pi $ ou $\sqrt{2}$ ) sont appelés des nombres \textbf{irrationnels}.
}
\cadre{vert}{Propriété}{%
     $\mathbb{N} \subset \mathbb{Z} \subset \mathbb{D} \subset \mathbb{Q} \subset \mathbb{R}$.
}
\begin{center}
     \begin{extern} %width="500" alt=" représentation graphique d'une fonction"
          \resizebox{8cm}{!}{%
               \newrgbcolor{wwqqcc}{0.4 0. 0.8}
               \newrgbcolor{yqqqqq}{0.50 0. 0.}
               \newrgbcolor{qqwuqq}{0. 0.39 0.}
               \newrgbcolor{qqqqcc}{0. 0. 0.8}
               \newrgbcolor{ffqqtt}{1. 0. 0.2}
               \psset{xunit=1.0cm,yunit=1.0cm,algebraic=true,dimen=middle}
               \begin{pspicture*}(0.58,-17.18)(23.54,-3.1)
                    \fontsize{16pt}{16.1pt}\selectfont
                    \rput{0.}(10.,-10.){\psellipse[linewidth=2.pt,linecolor=wwqqcc](0,0)(3.,2.23)}
                    \rput{0.}(10.5,-10.){\psellipse[linewidth=2.pt,linecolor=yqqqqq](0,0)(4.54,2.90)}
                    \rput{0.}(10.97,-10.){\psellipse[linewidth=2.pt,linecolor=qqwuqq](0,0)(6.24,3.71)}
                    \rput{-0.17}(11.45,-10.02){\psellipse[linewidth=2.pt,linecolor=qqqqcc](0,0)(8.32,4.90)}
                    \rput{0}(12.064353510453511,-10.0){\psellipse[linewidth=2.pt,linecolor=ffqqtt](0,0)(11.03,6.46)}
                    \rput[tl](9.68,-8.8){$\wwqqcc{\mathbb{N}}$}
                    \rput[tl](13.64,-9.56){$\yqqqqq{\mathbb{Z}}$}
                    \rput[tl](15.76,-8.84){$\qqwuqq{\mathbb{D}}$}
                    \rput[tl](18.28,-9.52){$\qqqqcc{\mathbb{Q}}$}
                    \rput[tl](20.92,-9.52){$\ffqqtt{\mathbb{R}}$}
                    \rput[tl](8.28,-10.06){$0$}
                    \rput[tl](11.24,-10.7){$6$}
                    \rput[tl](12.36,-7.8){$-1$}
                    \rput[tl](13.2,-10.66){$-23$}
                    \rput[tl](15.32,-8.1){$1,5$}
                    \rput[tl](15,-10.86){$-\frac{3}{4}$}
                    \rput[tl](17,-11.82){$\frac{4}{3}$}
                    \rput[tl](16.88,-7.02){$-\frac{1}{7}$}
                    \rput[tl](20.04,-6.62){$\sqrt{2}$}
                    \rput[tl](19.2,-12.82){$\pi $}
               \end{pspicture*}
          }
     \end{extern}
\end{center}
\bloc{cyan}{Remarques}{%
     \begin{itemize}
          \item Le symbole $ \subset $ se lit \textit{"inclus dans"}.
          \item La proposition précédente signifie que tous les entiers naturels sont aussi des entiers relatifs qui sont eux-même des nombres décimaux qui sont des nombres rationnels qui sont des nombres réels.
          \item Un même nombre admet plusieurs écritures différentes. Par exemple le nombre 2 peut aussi s'écrire 2,0 (écriture décimale) $\frac{2}{1}$ ou $\frac{4}{2}$ etc. (écriture fractionnaire) $\sqrt{4}$ (écriture avec un radical) et même (aussi curieux que cela puisse vous paraitre) 1,999999.... (écriture décimale illimitée).
     \end{itemize}
}

\begin{h2}II - Intervalles \end{h2}

\begin{h3}Intervalles bornés\end{h3}
\cadre{bleu}{Définition}{ % id=d50 
Soient $a$ et $b$ deux nombres réels tels que $ a < b $.
\begin{itemize}
\item
L'intervalle  \textbf{fermé} $  \left[ a~;~b \right]  $ est l'ensemble des nombres réels $x$ tels que  $ a  \leqslant x  \leqslant b. $
\item
L'intervalle  \textbf{ouvert} $  \left] a~;~b \right[  $ est l'ensemble des nombres réels $x$ tels que  $ a  < x  < b. $
\item
L'intervalle  $  \left[ a~;~b \right[  $ (fermé en $a$, ouvert en $b$) est l'ensemble des nombres réels $x$ tels que  $ a   \leqslant  x  < b. $
\item
L'intervalle  $  \left] a~;~b \right]  $ (ouvert en $a$, fermé  en $b$) est l'ensemble des nombres réels $x$ tels que  $ a   < x  \leqslant  b.$
\end{itemize}
} % fin définition

\bloc{orange}{Exemple}{ % id=e55
Par exemple, l'intervalle $  \left[ -2~;~3 \right[  $ est constitué des nombres réels qui sont à la fois supérieur ou égal à $  -2$ et strictement inférieur à $3$.
\par
On pourra, par exemple, écrire~:

\begin{itemize}
\item
 $ -3  \notin \left[ -2~;~3 \right[  $
\item
 $ -2  \in \left[ -2~;~3 \right[  $
\item
 $ 0  \in \left[ -2~;~3 \right[  $
\item
 $ 3  \notin \left[ -2~;~3 \right[  $
\item
 $ 4  \notin \left[ -2~;~3 \right[  $
\end{itemize}
\medskip
On peut représenter l’intervalle  $  \left[ -2~;~3 \right[  $ de la façon suivante~:
\begin{center} 
\begin{extern}%alt="Intervalle ouvert-fermé" model="modeles-graphiques"
\psset{xunit=1.0cm,yunit=1.0cm,algebraic=true,dimen=middle,dotstyle=*,dotsize=4pt 0,linewidth=.5pt,arrowsize=3pt 2,arrowinset=0.25}
\begin{pspicture*}(-4.5,-0.75)(4.5,0.75)
\psaxes[labelFontSize=\small,xAxis=true,yAxis=false,Dx=1.,Dy=1.,ticksize=-2pt 0,subticks=2]{->}(0,0)(-4.5,-0.75)(4.5,0.75)
\rput[tl](-0.1,-0.25){$\small 0$}  
\psline[linewidth=1.2pt,linecolor=red](-2.,0.)(3.,0.)
\rput[tl](-2.07,0.2){$\red\Large\textbf{[}$}
\rput[tl](2.93,0.2){$\red\Large\textbf{[}$} 
\end{pspicture*}        
 \end{extern}
\end{center}
          
} % fin exemple

\begin{h3}Intervalles non bornés\end{h3}
\cadre{bleu}{Définition}{ % id=d60 
Soit $a$ un nombre réel.
\begin{itemize}
\item
L'intervalle  $\left[ a~;~+\infty   \right[ $ est l'ensemble des nombres réels $x$ tels que  $  x   \geqslant a. $
\item
L'intervalle $\left] a~;~+\infty   \right[ $  est l'ensemble des nombres réels $x$ tels que  $ x >  a. $
\item
L'intervalle $\left] -\infty~;~a \right] $  est l'ensemble des nombres réels $x$ tels que  $ x  \leqslant   a. $
\item
L'intervalle $\left] -\infty~;~a \right] $  est l'ensemble des nombres réels $x$ tels que  $ x  <   a. $
\end{itemize}
} % fin définition

\bloc{cyan}{Remarque}{ % id=r65 
En $  +\infty  $ et en $  -\infty  $, le crochet est toujours  \textbf{ouvert}.
} % fin remarque
\bloc{orange}{Exemple}{ % id=e67

\begin{itemize}
\item
$ 0  \notin  \left[ 1~;~ +\infty  \right[  $
\item
$ 1 \in  \left[ 1~;~ +\infty  \right[  $
\item
$ 100  \in  \left[ 1~;~ +\infty  \right[  $
\end{itemize}
\medskip
On représente l’intervalle  $ \left[ 1~;~ +\infty  \right[  $ ainsi~:
\begin{center} 
\begin{extern}%alt="Intervalle non borné" model="modeles-graphiques"
\psset{xunit=1.0cm,yunit=1.0cm,algebraic=true,dimen=middle,dotstyle=*,dotsize=4pt 0,linewidth=.5pt,arrowsize=3pt 2,arrowinset=0.25}
\begin{pspicture*}(-4.5,-0.75)(4.5,0.75)
\psaxes[labelFontSize=\small,xAxis=true,yAxis=false,Dx=1.,Dy=1.,ticksize=-2pt 0,subticks=2]{->}(0,0)(-4.5,-0.75)(4.5,0.75)
\rput[tl](-0.1,-0.25){$\small 0$}  
\psline[linewidth=1.2pt,linecolor=red](1.,0.)(4.5,0.)
\rput[tl](0.95,0.2){$\red\Large\textbf{[}$}
\end{pspicture*}        
 \end{extern}
\end{center}
} % fin exemple

\begin{h3}Union et intersection\end{h3}

\cadre{bleu}{Définition}{ % id=d70
Soient $I$ et $J$ deux intervalles.

\begin{itemize}
\item
L'\textbf{intersection} de $I$ et de $J$ notée $ I  \cap J $ (lire  \og $I$ inter $J$  \fg{} ) est l'ensemble des nombres appartenant à la fois à $I$  \textbf{et} à $J$.

\item
L'\textbf{union} (ou la  \textbf{réunion}  de $I$ et de $J$ notée $ I  \cup J $ (lire  \og $I$ union $J$  \fg{} ) est l'ensemble des nombres appartenant à $I$  \textbf{ou} à $J$ \textbf{ou} aux deux intervalles.
\end{itemize} 
} % fin définition

\bloc{cyan}{Remarque}{ % id=r73
Retenir que l'\textbf{intersection} correspond au mot  \og  \textbf{et} \fg{} et que la  \textbf{réunion} correspond au mot  \og  \textbf{ou}\fg{}. 
} % fin remarque

\bloc{orange}{Exemple}{ % id=e75 
Si $ I =  \left[ -3~;~1 \right[  $ et $ J=  \left[ 0~;~3 \right] $, $I$ est représenté en bleu et $J$ en rouge sur la figure suivante~:
\\
\begin{center} 
\begin{extern}%alt="Intervalle union intersection" model="modeles-graphiques"
\psset{xunit=1.0cm,yunit=1.0cm,algebraic=true,dimen=middle,dotstyle=*,dotsize=4pt 0,linewidth=.5pt,arrowsize=3pt 2,arrowinset=0.25}
\begin{pspicture*}(-4.5,-0.75)(4.5,0.75)
\psaxes[labelFontSize=\small,xAxis=true,yAxis=false,Dx=1.,Dy=1.,ticksize=-2pt 0,subticks=2]{->}(0,0)(-4.5,-0.75)(4.5,0.75)
\rput[tl](-0.1,-0.25){$\small 0$}  
\psline[linewidth=1.2pt,linecolor=blue](-3.,-0.03)(1.,-0.03)
\rput[tl](-3.07,0.2){$\blue\Large\textbf{[}$}
\rput[tl](0.93,0.2){$\blue\Large\textbf{[}$} 
\psline[linewidth=1.2pt,linecolor=red](0.,0.03)(3.,0.03)
\rput[tl](-0.07,0.2){$\red\Large\textbf{[}$}
\rput[tl](2.93,0.2){$\red\Large\textbf{]}$} 
\end{pspicture*}        
 \end{extern}
\end{center}
} % fin exemple

\begin{center}
$ I  \cap J =  \left[ 0~;~1 \right[  $ et $ I  \cup J =  \left[ -3~;~3 \right] $ 
\end{center}

\begin{h2}III - Valeurs absolues \end{h2}
Intuitivement, la valeur absolue d'un nombre c'est  \og le nombre sans son signe. \fg{}. Par exemple, la valeur absolue de $ -5 $ est $5$ et la valeur absolue de $ 1,12 $ est $ 1,12 $. Toutefois, lors de calculs littéraux, le signe peut être  \og caché \fg{} à l'intérieur de la lettre~; par exemple, on ne peut pas dire que la valeur absolue de $ -x $ est égale à $ x $ car c'est faux si $x$ est négatif. D'où la définition suivante~:
\cadre{bleu}{Définition}{% id=d100 
Soit $x$ un nombre réel
     On appelle \textbf{valeur absolue} de $x$ et on note $ |x|$ le nombre réel positif ou nul défini par
     \begin{itemize}
          \item $|x|$ = $x$ si $x$ est positif ou nul,
          \item $|x|$ = $-x$ si $x$ est négatif ou nul.
     \end{itemize}
}

\bloc{orange}{Exemples}{ % id=e105 

\begin{itemize}
\item
$| -1 | = -(-1) = 1$ 
\item
$| \sqrt{ 2 } - 1 | = \sqrt{ 2 } - 1$ car $ \sqrt{ 2 } > 1 $ donc $  \sqrt{ 2 } - 1  $ est positif.
\end{itemize}
} % fin exemple
\cadre{vert}{Propriété}{% id=p110
     La distance entre les nombres réels $x$ et $y$ est égale à $|y-x|$ (ou aussi à $|x-y|$).
\par
En particulier, $  \left| x \right|  $ est la distance de $x$ à $0$.
}
\bloc{orange}{Exemple}{% id=e115 
Les nombres $ -3 $ et $2$ sont représentés sur l'axe ci-dessous~:
 \begin{center} 
\begin{extern}% alt="distance et valeur absolue"
\psset{xunit=1.0cm,yunit=1.0cm,algebraic=true,dimen=middle,dotstyle=*,dotsize=3pt 0,linewidth=.5pt,arrowsize=3pt 2,arrowinset=0.25}
\begin{pspicture*}(-4.5,-0.75)(4.5,0.75)
\psaxes[labelFontSize=\small,xAxis=true,yAxis=false,Dx=1.,Dy=1.,ticksize=-2pt 0,subticks=2]{->}(0,0)(-4.5,-0.75)(4.5,0.75)
\rput[tl](-0.1,-0.25){$\small 0$}  
\psline[linewidth=1.2pt,linecolor=blue](-3,0)(2.,0.)
\psdots[linecolor=blue](-3,0)(2.,0.)
\end{pspicture*}        
 \end{extern}
\end{center}

La distance entre $ -3 $ et $ 2 $ est égale à~:
\begin{center}
     $|-3-2|=|-5|=5$
\end{center}
\par
La distance entre $ -3 $ et $ 0 $ est égale à~:
\begin{center}
     $|-3|=3$
\end{center}
}

\end{document}