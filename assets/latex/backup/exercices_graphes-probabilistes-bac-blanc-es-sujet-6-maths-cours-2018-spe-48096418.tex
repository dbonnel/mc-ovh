\documentclass[a4paper]{article}

%================================================================================================================================
%
% Packages
%
%================================================================================================================================

\usepackage[T1]{fontenc} 	% pour caractères accentués
\usepackage[utf8]{inputenc}  % encodage utf8
\usepackage[french]{babel}	% langue : français
\usepackage{fourier}			% caractères plus lisibles
\usepackage[dvipsnames]{xcolor} % couleurs
\usepackage{fancyhdr}		% réglage header footer
\usepackage{needspace}		% empêcher sauts de page mal placés
\usepackage{graphicx}		% pour inclure des graphiques
\usepackage{enumitem,cprotect}		% personnalise les listes d'items (nécessaire pour ol, al ...)
\usepackage{hyperref}		% Liens hypertexte
\usepackage{pstricks,pst-all,pst-node,pstricks-add,pst-math,pst-plot,pst-tree,pst-eucl} % pstricks
\usepackage[a4paper,includeheadfoot,top=2cm,left=3cm, bottom=2cm,right=3cm]{geometry} % marges etc.
\usepackage{comment}			% commentaires multilignes
\usepackage{amsmath,environ} % maths (matrices, etc.)
\usepackage{amssymb,makeidx}
\usepackage{bm}				% bold maths
\usepackage{tabularx}		% tableaux
\usepackage{colortbl}		% tableaux en couleur
\usepackage{fontawesome}		% Fontawesome
\usepackage{environ}			% environment with command
\usepackage{fp}				% calculs pour ps-tricks
\usepackage{multido}			% pour ps tricks
\usepackage[np]{numprint}	% formattage nombre
\usepackage{tikz,tkz-tab} 			% package principal TikZ
\usepackage{pgfplots}   % axes
\usepackage{mathrsfs}    % cursives
\usepackage{calc}			% calcul taille boites
\usepackage[scaled=0.875]{helvet} % font sans serif
\usepackage{svg} % svg
\usepackage{scrextend} % local margin
\usepackage{scratch} %scratch
\usepackage{multicol} % colonnes
%\usepackage{infix-RPN,pst-func} % formule en notation polanaise inversée
\usepackage{listings}

%================================================================================================================================
%
% Réglages de base
%
%================================================================================================================================

\lstset{
language=Python,   % R code
literate=
{á}{{\'a}}1
{à}{{\`a}}1
{ã}{{\~a}}1
{é}{{\'e}}1
{è}{{\`e}}1
{ê}{{\^e}}1
{í}{{\'i}}1
{ó}{{\'o}}1
{õ}{{\~o}}1
{ú}{{\'u}}1
{ü}{{\"u}}1
{ç}{{\c{c}}}1
{~}{{ }}1
}


\definecolor{codegreen}{rgb}{0,0.6,0}
\definecolor{codegray}{rgb}{0.5,0.5,0.5}
\definecolor{codepurple}{rgb}{0.58,0,0.82}
\definecolor{backcolour}{rgb}{0.95,0.95,0.92}

\lstdefinestyle{mystyle}{
    backgroundcolor=\color{backcolour},   
    commentstyle=\color{codegreen},
    keywordstyle=\color{magenta},
    numberstyle=\tiny\color{codegray},
    stringstyle=\color{codepurple},
    basicstyle=\ttfamily\footnotesize,
    breakatwhitespace=false,         
    breaklines=true,                 
    captionpos=b,                    
    keepspaces=true,                 
    numbers=left,                    
xleftmargin=2em,
framexleftmargin=2em,            
    showspaces=false,                
    showstringspaces=false,
    showtabs=false,                  
    tabsize=2,
    upquote=true
}

\lstset{style=mystyle}


\lstset{style=mystyle}
\newcommand{\imgdir}{C:/laragon/www/newmc/assets/imgsvg/}
\newcommand{\imgsvgdir}{C:/laragon/www/newmc/assets/imgsvg/}

\definecolor{mcgris}{RGB}{220, 220, 220}% ancien~; pour compatibilité
\definecolor{mcbleu}{RGB}{52, 152, 219}
\definecolor{mcvert}{RGB}{125, 194, 70}
\definecolor{mcmauve}{RGB}{154, 0, 215}
\definecolor{mcorange}{RGB}{255, 96, 0}
\definecolor{mcturquoise}{RGB}{0, 153, 153}
\definecolor{mcrouge}{RGB}{255, 0, 0}
\definecolor{mclightvert}{RGB}{205, 234, 190}

\definecolor{gris}{RGB}{220, 220, 220}
\definecolor{bleu}{RGB}{52, 152, 219}
\definecolor{vert}{RGB}{125, 194, 70}
\definecolor{mauve}{RGB}{154, 0, 215}
\definecolor{orange}{RGB}{255, 96, 0}
\definecolor{turquoise}{RGB}{0, 153, 153}
\definecolor{rouge}{RGB}{255, 0, 0}
\definecolor{lightvert}{RGB}{205, 234, 190}
\setitemize[0]{label=\color{lightvert}  $\bullet$}

\pagestyle{fancy}
\renewcommand{\headrulewidth}{0.2pt}
\fancyhead[L]{maths-cours.fr}
\fancyhead[R]{\thepage}
\renewcommand{\footrulewidth}{0.2pt}
\fancyfoot[C]{}

\newcolumntype{C}{>{\centering\arraybackslash}X}
\newcolumntype{s}{>{\hsize=.35\hsize\arraybackslash}X}

\setlength{\parindent}{0pt}		 
\setlength{\parskip}{3mm}
\setlength{\headheight}{1cm}

\def\ebook{ebook}
\def\book{book}
\def\web{web}
\def\type{web}

\newcommand{\vect}[1]{\overrightarrow{\,\mathstrut#1\,}}

\def\Oij{$\left(\text{O}~;~\vect{\imath},~\vect{\jmath}\right)$}
\def\Oijk{$\left(\text{O}~;~\vect{\imath},~\vect{\jmath},~\vect{k}\right)$}
\def\Ouv{$\left(\text{O}~;~\vect{u},~\vect{v}\right)$}

\hypersetup{breaklinks=true, colorlinks = true, linkcolor = OliveGreen, urlcolor = OliveGreen, citecolor = OliveGreen, pdfauthor={Didier BONNEL - https://www.maths-cours.fr} } % supprime les bordures autour des liens

\renewcommand{\arg}[0]{\text{arg}}

\everymath{\displaystyle}

%================================================================================================================================
%
% Macros - Commandes
%
%================================================================================================================================

\newcommand\meta[2]{    			% Utilisé pour créer le post HTML.
	\def\titre{titre}
	\def\url{url}
	\def\arg{#1}
	\ifx\titre\arg
		\newcommand\maintitle{#2}
		\fancyhead[L]{#2}
		{\Large\sffamily \MakeUppercase{#2}}
		\vspace{1mm}\textcolor{mcvert}{\hrule}
	\fi 
	\ifx\url\arg
		\fancyfoot[L]{\href{https://www.maths-cours.fr#2}{\black \footnotesize{https://www.maths-cours.fr#2}}}
	\fi 
}


\newcommand\TitreC[1]{    		% Titre centré
     \needspace{3\baselineskip}
     \begin{center}\textbf{#1}\end{center}
}

\newcommand\newpar{    		% paragraphe
     \par
}

\newcommand\nosp {    		% commande vide (pas d'espace)
}
\newcommand{\id}[1]{} %ignore

\newcommand\boite[2]{				% Boite simple sans titre
	\vspace{5mm}
	\setlength{\fboxrule}{0.2mm}
	\setlength{\fboxsep}{5mm}	
	\fcolorbox{#1}{#1!3}{\makebox[\linewidth-2\fboxrule-2\fboxsep]{
  		\begin{minipage}[t]{\linewidth-2\fboxrule-4\fboxsep}\setlength{\parskip}{3mm}
  			 #2
  		\end{minipage}
	}}
	\vspace{5mm}
}

\newcommand\CBox[4]{				% Boites
	\vspace{5mm}
	\setlength{\fboxrule}{0.2mm}
	\setlength{\fboxsep}{5mm}
	
	\fcolorbox{#1}{#1!3}{\makebox[\linewidth-2\fboxrule-2\fboxsep]{
		\begin{minipage}[t]{1cm}\setlength{\parskip}{3mm}
	  		\textcolor{#1}{\LARGE{#2}}    
 	 	\end{minipage}  
  		\begin{minipage}[t]{\linewidth-2\fboxrule-4\fboxsep}\setlength{\parskip}{3mm}
			\raisebox{1.2mm}{\normalsize\sffamily{\textcolor{#1}{#3}}}						
  			 #4
  		\end{minipage}
	}}
	\vspace{5mm}
}

\newcommand\cadre[3]{				% Boites convertible html
	\par
	\vspace{2mm}
	\setlength{\fboxrule}{0.1mm}
	\setlength{\fboxsep}{5mm}
	\fcolorbox{#1}{white}{\makebox[\linewidth-2\fboxrule-2\fboxsep]{
  		\begin{minipage}[t]{\linewidth-2\fboxrule-4\fboxsep}\setlength{\parskip}{3mm}
			\raisebox{-2.5mm}{\sffamily \small{\textcolor{#1}{\MakeUppercase{#2}}}}		
			\par		
  			 #3
 	 		\end{minipage}
	}}
		\vspace{2mm}
	\par
}

\newcommand\bloc[3]{				% Boites convertible html sans bordure
     \needspace{2\baselineskip}
     {\sffamily \small{\textcolor{#1}{\MakeUppercase{#2}}}}    
		\par		
  			 #3
		\par
}

\newcommand\CHelp[1]{
     \CBox{Plum}{\faInfoCircle}{À RETENIR}{#1}
}

\newcommand\CUp[1]{
     \CBox{NavyBlue}{\faThumbsOUp}{EN PRATIQUE}{#1}
}

\newcommand\CInfo[1]{
     \CBox{Sepia}{\faArrowCircleRight}{REMARQUE}{#1}
}

\newcommand\CRedac[1]{
     \CBox{PineGreen}{\faEdit}{BIEN R\'EDIGER}{#1}
}

\newcommand\CError[1]{
     \CBox{Red}{\faExclamationTriangle}{ATTENTION}{#1}
}

\newcommand\TitreExo[2]{
\needspace{4\baselineskip}
 {\sffamily\large EXERCICE #1\ (\emph{#2 points})}
\vspace{5mm}
}

\newcommand\img[2]{
          \includegraphics[width=#2\paperwidth]{\imgdir#1}
}

\newcommand\imgsvg[2]{
       \begin{center}   \includegraphics[width=#2\paperwidth]{\imgsvgdir#1} \end{center}
}


\newcommand\Lien[2]{
     \href{#1}{#2 \tiny \faExternalLink}
}
\newcommand\mcLien[2]{
     \href{https~://www.maths-cours.fr/#1}{#2 \tiny \faExternalLink}
}

\newcommand{\euro}{\eurologo{}}

%================================================================================================================================
%
% Macros - Environement
%
%================================================================================================================================

\newenvironment{tex}{ %
}
{%
}

\newenvironment{indente}{ %
	\setlength\parindent{10mm}
}

{
	\setlength\parindent{0mm}
}

\newenvironment{corrige}{%
     \needspace{3\baselineskip}
     \medskip
     \textbf{\textsc{Corrigé}}
     \medskip
}
{
}

\newenvironment{extern}{%
     \begin{center}
     }
     {
     \end{center}
}

\NewEnviron{code}{%
	\par
     \boite{gray}{\texttt{%
     \BODY
     }}
     \par
}

\newenvironment{vbloc}{% boite sans cadre empeche saut de page
     \begin{minipage}[t]{\linewidth}
     }
     {
     \end{minipage}
}
\NewEnviron{h2}{%
    \needspace{3\baselineskip}
    \vspace{0.6cm}
	\noindent \MakeUppercase{\sffamily \large \BODY}
	\vspace{1mm}\textcolor{mcgris}{\hrule}\vspace{0.4cm}
	\par
}{}

\NewEnviron{h3}{%
    \needspace{3\baselineskip}
	\vspace{5mm}
	\textsc{\BODY}
	\par
}

\NewEnviron{margeneg}{ %
\begin{addmargin}[-1cm]{0cm}
\BODY
\end{addmargin}
}

\NewEnviron{html}{%
}

\begin{document}
\meta{url}{/exercices/graphes-probabilistes-bac-blanc-es-sujet-6-maths-cours-2018-spe/}
\meta{pid}{10600}
\meta{titre}{Graphes probabilistes - Bac blanc ES Sujet 6 - Maths-cours 2018 (spé)}
\meta{type}{exercices}
%
\begin{h2}Exercice 5 (5 points)\end{h2}
\par
\textbf{Candidats ayant suivi l'enseignement de spécialité}
\par
Pour le paiement des cotisations, une société d'assurance propose à ces clients le choix entre deux types de règlement :
\par
\begin{itemize}
     \item
     le prélèvement automatique mensuel ;
     \item
     le règlement annuel par chèque.
\end{itemize}
\par
En 2016, 50\% des clients avaient opté pour le prélèvement mensuel.
\par
Chaque année :
\begin{itemize}
     \item
     90\% des clients payant par prélèvement mensuel conservent ce mode de paiement l'année suivante ;
     \item
     25\% des clients réglant leur cotisation par chèque annuel choisissent le prélèvement mensuel l'année suivante ;
\end{itemize}

\par
Pour tout entier naturel $n$, on note :
\begin{itemize}
     \item
     $m_{n}$, la proportion de clients ayant choisi le prélèvement mensuel pour l'année $2016 + n$ ;
     \item
     $a_{n}$, la proportion de clients ayant choisi le règlement annuel pour l'année $2016 + n$ ;
     \item
     $P_{n}$, la matrice-ligne $\left(m_{n} \quad a_{n}\right)$ donnant l'état probabiliste de l'année $2016 + n$.
\end{itemize}
\par
On choisit au hasard un client de cet assureur.
\par
On note :
\begin{itemize}
     \item %
     $M$ l'état \og le client a opté pour le prélèvement mensuel \fg{} ;
     \item %
     $A$ l'état \og le client a opté pour le règlement annuel \fg{} .
\end{itemize}
\par
%============================================================================================================================
%
\TitreC{Partie A}
%
%============================================================================================================================
\par
\begin{enumerate}
     \item %1
     Traduire la situation par un graphe probabiliste.
     \item %2
     Déterminer la matrice de transition $T$ associée à ce graphe, les sommets étant classés dans l'ordre $M, A$.
     \item %3
     Déterminer les matrices-ligne $P_0, P_1$ et $P_2$ (\textit{Si nécessaire, on arrondira les résultats au millième}).\\
     Quel est le pourcentage de clients ayant choisi le prélèvement mensuel en 2018 ?
     \item %4
     Déterminer l'état stable de ce graphe. \\
     Interpréter ce résultat.
     \par
\end{enumerate}
\par
%============================================================================================================================
%
\TitreC{Partie B}
%
%============================================================================================================================
\par
\begin{enumerate}
     \item %1
     Montrer que pour tout entier naturel $n$ :
     \[ m_{n+1}=0,65m_n + 0,25. \]
     \item %2
     Le directeur d'agence souhaite connaître la proportion des clients qui optera pour le prélèvement mensuel pour l'année 2016+n où $n$ est un entier naturel non nul.
     \par
     On lui a proposé les trois algorithmes suivants :
     \par
     \begin{center}
          \begin{extern}%width="400" alt="algorithme n°1"
               \begin{tabular}{|l l|}\hline
                    \textbf{Entrée :}	& 	Saisir $n$\\
                    \textbf{Traitement :}	& Affecter à $i$ la valeur 1 \\
                    & Affecter à $m$ la valeur 0.5\\
                    & Tant que $i < n$\\
                    &\qquad Affecter à $m$ la valeur $0,65m + 0,25$ \\
                    &Fin Tant que\\
                    \textbf{Sortie :}		&Afficher $m$ \\ \hline
               \end{tabular}
          \end{extern}
          \par
          \textbf{Algorithme 1}
     \end{center}
     \begin{center}
          \begin{extern}%width="400" alt="algorithme n°2"
               \begin{tabular}{|l l|}\hline
                    \textbf{Entrée :}	& 	Saisir $n$\\
                    \textbf{Traitement :} & Affecter à $m$ la valeur 0,5\\
                    & Pour $i$ allant de $1$ à $n$\\
                    &\qquad Affecter à $m$ la valeur $0,65m + 0,25$ \\
                    &Fin Pour\\
                    \textbf{Sortie :}		&Afficher $m$ \\ \hline
               \end{tabular}
          \end{extern}
          \par
          \textbf{Algorithme 2}
     \end{center}
     \begin{center}
          \begin{extern}%width="400" alt="algorithme n°3"
               \begin{tabular}{|l l|}\hline
                    \textbf{Entrée :}	& 	Saisir $n$\\
                    \textbf{Traitement :} & Affecter à $m$ la valeur 0,5\\
                    & Pour $i$ allant de $1$ à $n$\\
                    &\qquad Affecter à $m$ la valeur $0,65m + 0,25$ \\
                    &Fin Pour\\
                    \textbf{Sortie :}		&Afficher $n$ \\ \hline
               \end{tabular}
          \end{extern}
          \par
          \textbf{Algorithme 3}
     \end{center}
     \par
     Parmi ces trois algorithmes, un seul répond correctement à la demande du directeur. Lequel ?\\
     Justifier votre réponse en indiquant les erreurs présentes dans les deux autres algorithmes.
     \item %3
     Pour tout entier naturel $n$, on pose $u_n=m_n-\dfrac{5}{7}$.
     \par
     \begin{enumerate}[label=\alph*.]
          \item %3a
          Montrer que la suite $(u_n)$ est une suite géométrique dont on précisera le premier terme et la raison.
          \item %3b
          En déduire que pour tout entier naturel $n$ :
          \[ m_n=\dfrac{5}{7}-\dfrac{3}{14} \times 0,65^n. \]
     \end{enumerate}
     \item %4
     Quelle est la limite de la suite $(m_n)$ ? \\
     Interpréter et comparer ce résultat à celui de la question \textbf{4.} de la \textbf{Partie A}.
     \par
\end{enumerate}
\begin{corrige}
     %============================================================================================================================
     %
     \TitreC{Partie A}
     %
     %============================================================================================================================
     \begin{enumerate}
          \item On traduit les données de l'énoncé par un graphe probabiliste de sommets $M$ et $A$:
          \begin{center}
               \begin{extern}%width="400" alt="Graphe probabiliste à deux états"
                    \begin{pspicture}(-2,-0.5)(4,1)
                         \circlenode{M}{$M$} \hskip 4cm \circlenode{A}{$A$}% définition des sommets
                         \psset{arcangle=15,arrowsize=2pt 3}%  différents paramètres
                         \ncarc{->}{M}{A} \Aput{0,1}%
                         \ncarc{->}{A}{M} \Aput{0,25}%
                         \nccircle[angleA=90]{->}{M}{4mm}   \Bput{0,9}%    boucle autour de T
                         \nccircle[angleA=-90]{->}{A}{.4cm} \Bput{0,75}%    boucle autour de B
                    \end{pspicture}
               \end{extern}
          \end{center}
          \item  La matrice de transition de ce graphe en plaçant les sommets dans l'ordre $M$, $A$ est :
          \[ T=
          \begin{pmatrix}
               0,9 & 0,1\\
               0,25 & 0,75
          \end{pmatrix}. \]
          \item
          D'après l'énoncé, en 2016, 50\% des clients avaient opté pour le prélèvement mensuel donc $m_0=0,5$ et $p_0=1-m_0=0,5$.
          \par
          Par conséquent $P_0=(0,5 \quad 0,5)$.
          \par
          On en déduit :
          \par
          $P_1 = P_0 \times T = (0,5 \quad 0,5) \times \begin{pmatrix}
               0,9 & 0,1\\
               0,25 & 0,75
          \end{pmatrix} = (0,575 \quad 0,425)$ ;
          \par
          $P_2 = P_1 \times T = (0,575 \quad 0,425) \times \begin{pmatrix}
               0,9 & 0,1\\
               0,25 & 0,75
          \end{pmatrix} = (0,624 \quad 0,376)$ \\(arrondi au millième).
          \item
          Une matrice $P = (a\quad b)$ est un état stable si et seulement si ${a + b = 1}$ et $PT = P$.
          \par
          $PT=P \Leftrightarrow \begin{pmatrix} a&b\end{pmatrix}
          \times \begin{pmatrix} 0,9 & 0,1\\0,25 & 0,75 \end{pmatrix}
          =\begin{pmatrix} a&b\end{pmatrix}$
          \par
          $\phantom{PT=P} \Leftrightarrow \begin{pmatrix} 0,9a+0,25b & 0,1a+0,75b\end{pmatrix}
          =\begin{pmatrix} a&b\end{pmatrix}$
          \par
          $\phantom{PT=P} \Leftrightarrow
          \left\lbrace
          \begin{array}{r c l}
               0,9a+0,25b  &=& a\\
               0,1a+0,75b  &=& b
          \end{array}
     \right.$
     \par
     $\phantom{PM=P} \Leftrightarrow
     \left\lbrace
     \begin{array}{r c l}
          -0,1a+0,25b  &=& 0\\
          0,1a-0,25b  &=& 0
     \end{array}
\right.$
\par
$\phantom{PM=P}
\Leftrightarrow
0,1a-0,25b = 0$
\par
Comme $a+b=1$ : $b=1-a$. Par conséquent :
\par
$0,1a-0,25(1-a) = 0$
\par
$0,35a-0,25 = 0$
\par
$a = \dfrac{0,25}{0.35}=\dfrac{5}{7}$
\par
et $b=1-a=1-\dfrac{5}{7}=\dfrac{2}{7}$.
\par
On peut donc en déduire que
\[ P=\begin{pmatrix} \dfrac{5}{7} & \dfrac{2}{7} \end{pmatrix} \]
est l'état stable du graphe.
\par
Au cours du temps, la proportion des clients qui opteront pour le prélèvement mensuel se rapprochera de cinq septièmes.
\end{enumerate}
\par
%============================================================================================================================
%
\TitreC{Partie B}
%
%============================================================================================================================
\par
\begin{enumerate}
     \item %B1
     Pour tout entier naturel $n$ :
     \par
     Par conséquent $P_0=(0,5 \quad 0,5)$.
     \par
     $T$ étant la matrice de transition du graphe, pour tout entier naturel $n$ :
     \par
     $P_{n+1} = P_n \times T$.
     \par
     Donc :
     \par
     $(m_{n+1} \quad a_{n+1}) = (m_n \quad a_n) \times \begin{pmatrix}
          0,9 & 0,1\\
          0,25 & 0,75
     \end{pmatrix} = (0,575 \quad 0,425)$
     \par
     $\phantom{(m_{n+1} \quad a_{n+1})} = \begin{pmatrix} 0,9m_n+0,25a_n & 0,1m_n+0,75a_n\end{pmatrix} $.
     \par
     En comparant les termes situés en première colonne, on obtient :
     \par
     $m_{n+1} = 0,9m_n+0,25a_n$.
     \par
     Or, pour tout entier naturel $n$, $m_n$ et $a_n$ sont les probabilités de deux événements contraires donc ${a_n=1-m_n}$.
     \par
     Par conséquent :
     \par
     \par
     $m_{n+1}  = 0,9m_n+0,25(1-m_n)$\\
     $\phantom{m_{n+1}	}	 = 0,9m_n+0,25-0,25m_n$\\
     $\phantom{m_{n+1}	}	 		 = 0,65m_n+0,25.$
     \par
     \item %B2
     L'algorithme correct est l'\textbf{algorithme 2}.
     \par
     L'algorithme 1 est incorrect car il manque l'instruction \og Affecter à $i$ la valeur $i+1$ \fg{} (incrémentation de $i$) à l'intérieur de la boucle \og Tant que \fg{}.
     \par
     L'algorithme 3 est incorrect car il affiche, en sortie, la valeur de $n$ (qui correspond au nombre saisi par l'utilisateur) et non le résultat souhaité qui est stocké dans la variable $m$.
     \item %B3
   
     \begin{enumerate}[label=\alph*.]
          \item %3a
          Pour tout entier naturel $n$, $u_{n}= m_{n}-\dfrac{5}{7}$, donc :
          \par
          $u_{n+1}= m_{n+1}-\dfrac{5}{7}$.
          \par
          Or, pour tout entier naturel $n$, $m_{n+1}=0,65m_n+0,25$ ; par conséquent :
          \par
          $u_{n+1}= 0,65m_n+0,25-\dfrac{5}{7}$\\
          $\phantom{u_{n+1}}=0,65m_n+\dfrac{1}{4}-\dfrac{5}{7}$\\
          $\phantom{u_{n+1}}=0,65m_n+\dfrac{1}{4}-\dfrac{13}{28}.$
          \par
          \par
          Or, puisque $u_{n}= m_{n}-\dfrac{5}{7}$ : $m_{n}= u_{n}+\dfrac{5}{7}$.
          \par
          On en déduit :
          \par
          $u_{n+1}= 0,65\left(u_{n}+\dfrac{5}{7}\right)-\dfrac{13}{28}$\\
          $\phantom{u_{n+1}}= 0,65u_{n}+\dfrac{3,25}{7}-\dfrac{13}{28}$\\
          $\phantom{u_{n+1}}= 0,65u_{n}+\dfrac{13}{28}-\dfrac{13}{28}$\\
          $\phantom{u_{n+1}}= 0,65u_{n}.$
          \par
          \par
          Comme $u_{0}= m_{0}-\dfrac{5}{7}=\dfrac{1}{2}-\dfrac{5}{7}=-\dfrac{3}{14}$, la suite $(u_n)$ est une suite géométrique de premier terme ${u_0=-\dfrac{3}{14}}$ et de raison $0,65$.
          \item %3b
          La suite $(u_n)$ étant une suite géométrique de premier terme ${u_0=-\dfrac{3}{14}}$ et de raison $0,65$, pour tout entier naturel $n$ :
          \par
          $u_n=u_0q^n=-\dfrac{3}{14} \times 0,65^n$.
          \par
          Par conséquent :
          \par
          $m_{n}= u_n + \dfrac{5}{7}=\dfrac{5}{7}-\dfrac{3}{14} \times 0,65^n$.
          \par
     \end{enumerate}
     \item %B4
     ${0 \leqslant 0,65 < 1}\ $ donc $\ \lim\limits_{n \rightarrow +\infty } 0,65^n = 0$.
     \par
     On en déduit que :
     \par
     $\lim\limits_{n \rightarrow +\infty}\dfrac{3}{14} \times 0,65^n = 0\ $ et $\ \lim\limits_{n \rightarrow +\infty}\dfrac{5}{7}-\dfrac{3}{14} \times 0,65^n = \dfrac{5}{7}$.
     \par
     La suite $(m_n)$ converge donc vers $\dfrac{5}{7}$.
     \par
     On retrouve le résultat de la \textbf{partie A}, à savoir que la proportion de clients choisissant le prélèvement mensuel tendra vers $\dfrac{5}{7}$.
\end{enumerate}
\end{corrige}

\end{document}