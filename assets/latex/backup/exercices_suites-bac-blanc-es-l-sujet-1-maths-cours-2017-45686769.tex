\documentclass[a4paper]{article}

%================================================================================================================================
%
% Packages
%
%================================================================================================================================

\usepackage[T1]{fontenc} 	% pour caractères accentués
\usepackage[utf8]{inputenc}  % encodage utf8
\usepackage[french]{babel}	% langue : français
\usepackage{fourier}			% caractères plus lisibles
\usepackage[dvipsnames]{xcolor} % couleurs
\usepackage{fancyhdr}		% réglage header footer
\usepackage{needspace}		% empêcher sauts de page mal placés
\usepackage{graphicx}		% pour inclure des graphiques
\usepackage{enumitem,cprotect}		% personnalise les listes d'items (nécessaire pour ol, al ...)
\usepackage{hyperref}		% Liens hypertexte
\usepackage{pstricks,pst-all,pst-node,pstricks-add,pst-math,pst-plot,pst-tree,pst-eucl} % pstricks
\usepackage[a4paper,includeheadfoot,top=2cm,left=3cm, bottom=2cm,right=3cm]{geometry} % marges etc.
\usepackage{comment}			% commentaires multilignes
\usepackage{amsmath,environ} % maths (matrices, etc.)
\usepackage{amssymb,makeidx}
\usepackage{bm}				% bold maths
\usepackage{tabularx}		% tableaux
\usepackage{colortbl}		% tableaux en couleur
\usepackage{fontawesome}		% Fontawesome
\usepackage{environ}			% environment with command
\usepackage{fp}				% calculs pour ps-tricks
\usepackage{multido}			% pour ps tricks
\usepackage[np]{numprint}	% formattage nombre
\usepackage{tikz,tkz-tab} 			% package principal TikZ
\usepackage{pgfplots}   % axes
\usepackage{mathrsfs}    % cursives
\usepackage{calc}			% calcul taille boites
\usepackage[scaled=0.875]{helvet} % font sans serif
\usepackage{svg} % svg
\usepackage{scrextend} % local margin
\usepackage{scratch} %scratch
\usepackage{multicol} % colonnes
%\usepackage{infix-RPN,pst-func} % formule en notation polanaise inversée
\usepackage{listings}

%================================================================================================================================
%
% Réglages de base
%
%================================================================================================================================

\lstset{
language=Python,   % R code
literate=
{á}{{\'a}}1
{à}{{\`a}}1
{ã}{{\~a}}1
{é}{{\'e}}1
{è}{{\`e}}1
{ê}{{\^e}}1
{í}{{\'i}}1
{ó}{{\'o}}1
{õ}{{\~o}}1
{ú}{{\'u}}1
{ü}{{\"u}}1
{ç}{{\c{c}}}1
{~}{{ }}1
}


\definecolor{codegreen}{rgb}{0,0.6,0}
\definecolor{codegray}{rgb}{0.5,0.5,0.5}
\definecolor{codepurple}{rgb}{0.58,0,0.82}
\definecolor{backcolour}{rgb}{0.95,0.95,0.92}

\lstdefinestyle{mystyle}{
    backgroundcolor=\color{backcolour},   
    commentstyle=\color{codegreen},
    keywordstyle=\color{magenta},
    numberstyle=\tiny\color{codegray},
    stringstyle=\color{codepurple},
    basicstyle=\ttfamily\footnotesize,
    breakatwhitespace=false,         
    breaklines=true,                 
    captionpos=b,                    
    keepspaces=true,                 
    numbers=left,                    
xleftmargin=2em,
framexleftmargin=2em,            
    showspaces=false,                
    showstringspaces=false,
    showtabs=false,                  
    tabsize=2,
    upquote=true
}

\lstset{style=mystyle}


\lstset{style=mystyle}
\newcommand{\imgdir}{C:/laragon/www/newmc/assets/imgsvg/}
\newcommand{\imgsvgdir}{C:/laragon/www/newmc/assets/imgsvg/}

\definecolor{mcgris}{RGB}{220, 220, 220}% ancien~; pour compatibilité
\definecolor{mcbleu}{RGB}{52, 152, 219}
\definecolor{mcvert}{RGB}{125, 194, 70}
\definecolor{mcmauve}{RGB}{154, 0, 215}
\definecolor{mcorange}{RGB}{255, 96, 0}
\definecolor{mcturquoise}{RGB}{0, 153, 153}
\definecolor{mcrouge}{RGB}{255, 0, 0}
\definecolor{mclightvert}{RGB}{205, 234, 190}

\definecolor{gris}{RGB}{220, 220, 220}
\definecolor{bleu}{RGB}{52, 152, 219}
\definecolor{vert}{RGB}{125, 194, 70}
\definecolor{mauve}{RGB}{154, 0, 215}
\definecolor{orange}{RGB}{255, 96, 0}
\definecolor{turquoise}{RGB}{0, 153, 153}
\definecolor{rouge}{RGB}{255, 0, 0}
\definecolor{lightvert}{RGB}{205, 234, 190}
\setitemize[0]{label=\color{lightvert}  $\bullet$}

\pagestyle{fancy}
\renewcommand{\headrulewidth}{0.2pt}
\fancyhead[L]{maths-cours.fr}
\fancyhead[R]{\thepage}
\renewcommand{\footrulewidth}{0.2pt}
\fancyfoot[C]{}

\newcolumntype{C}{>{\centering\arraybackslash}X}
\newcolumntype{s}{>{\hsize=.35\hsize\arraybackslash}X}

\setlength{\parindent}{0pt}		 
\setlength{\parskip}{3mm}
\setlength{\headheight}{1cm}

\def\ebook{ebook}
\def\book{book}
\def\web{web}
\def\type{web}

\newcommand{\vect}[1]{\overrightarrow{\,\mathstrut#1\,}}

\def\Oij{$\left(\text{O}~;~\vect{\imath},~\vect{\jmath}\right)$}
\def\Oijk{$\left(\text{O}~;~\vect{\imath},~\vect{\jmath},~\vect{k}\right)$}
\def\Ouv{$\left(\text{O}~;~\vect{u},~\vect{v}\right)$}

\hypersetup{breaklinks=true, colorlinks = true, linkcolor = OliveGreen, urlcolor = OliveGreen, citecolor = OliveGreen, pdfauthor={Didier BONNEL - https://www.maths-cours.fr} } % supprime les bordures autour des liens

\renewcommand{\arg}[0]{\text{arg}}

\everymath{\displaystyle}

%================================================================================================================================
%
% Macros - Commandes
%
%================================================================================================================================

\newcommand\meta[2]{    			% Utilisé pour créer le post HTML.
	\def\titre{titre}
	\def\url{url}
	\def\arg{#1}
	\ifx\titre\arg
		\newcommand\maintitle{#2}
		\fancyhead[L]{#2}
		{\Large\sffamily \MakeUppercase{#2}}
		\vspace{1mm}\textcolor{mcvert}{\hrule}
	\fi 
	\ifx\url\arg
		\fancyfoot[L]{\href{https://www.maths-cours.fr#2}{\black \footnotesize{https://www.maths-cours.fr#2}}}
	\fi 
}


\newcommand\TitreC[1]{    		% Titre centré
     \needspace{3\baselineskip}
     \begin{center}\textbf{#1}\end{center}
}

\newcommand\newpar{    		% paragraphe
     \par
}

\newcommand\nosp {    		% commande vide (pas d'espace)
}
\newcommand{\id}[1]{} %ignore

\newcommand\boite[2]{				% Boite simple sans titre
	\vspace{5mm}
	\setlength{\fboxrule}{0.2mm}
	\setlength{\fboxsep}{5mm}	
	\fcolorbox{#1}{#1!3}{\makebox[\linewidth-2\fboxrule-2\fboxsep]{
  		\begin{minipage}[t]{\linewidth-2\fboxrule-4\fboxsep}\setlength{\parskip}{3mm}
  			 #2
  		\end{minipage}
	}}
	\vspace{5mm}
}

\newcommand\CBox[4]{				% Boites
	\vspace{5mm}
	\setlength{\fboxrule}{0.2mm}
	\setlength{\fboxsep}{5mm}
	
	\fcolorbox{#1}{#1!3}{\makebox[\linewidth-2\fboxrule-2\fboxsep]{
		\begin{minipage}[t]{1cm}\setlength{\parskip}{3mm}
	  		\textcolor{#1}{\LARGE{#2}}    
 	 	\end{minipage}  
  		\begin{minipage}[t]{\linewidth-2\fboxrule-4\fboxsep}\setlength{\parskip}{3mm}
			\raisebox{1.2mm}{\normalsize\sffamily{\textcolor{#1}{#3}}}						
  			 #4
  		\end{minipage}
	}}
	\vspace{5mm}
}

\newcommand\cadre[3]{				% Boites convertible html
	\par
	\vspace{2mm}
	\setlength{\fboxrule}{0.1mm}
	\setlength{\fboxsep}{5mm}
	\fcolorbox{#1}{white}{\makebox[\linewidth-2\fboxrule-2\fboxsep]{
  		\begin{minipage}[t]{\linewidth-2\fboxrule-4\fboxsep}\setlength{\parskip}{3mm}
			\raisebox{-2.5mm}{\sffamily \small{\textcolor{#1}{\MakeUppercase{#2}}}}		
			\par		
  			 #3
 	 		\end{minipage}
	}}
		\vspace{2mm}
	\par
}

\newcommand\bloc[3]{				% Boites convertible html sans bordure
     \needspace{2\baselineskip}
     {\sffamily \small{\textcolor{#1}{\MakeUppercase{#2}}}}    
		\par		
  			 #3
		\par
}

\newcommand\CHelp[1]{
     \CBox{Plum}{\faInfoCircle}{À RETENIR}{#1}
}

\newcommand\CUp[1]{
     \CBox{NavyBlue}{\faThumbsOUp}{EN PRATIQUE}{#1}
}

\newcommand\CInfo[1]{
     \CBox{Sepia}{\faArrowCircleRight}{REMARQUE}{#1}
}

\newcommand\CRedac[1]{
     \CBox{PineGreen}{\faEdit}{BIEN R\'EDIGER}{#1}
}

\newcommand\CError[1]{
     \CBox{Red}{\faExclamationTriangle}{ATTENTION}{#1}
}

\newcommand\TitreExo[2]{
\needspace{4\baselineskip}
 {\sffamily\large EXERCICE #1\ (\emph{#2 points})}
\vspace{5mm}
}

\newcommand\img[2]{
          \includegraphics[width=#2\paperwidth]{\imgdir#1}
}

\newcommand\imgsvg[2]{
       \begin{center}   \includegraphics[width=#2\paperwidth]{\imgsvgdir#1} \end{center}
}


\newcommand\Lien[2]{
     \href{#1}{#2 \tiny \faExternalLink}
}
\newcommand\mcLien[2]{
     \href{https~://www.maths-cours.fr/#1}{#2 \tiny \faExternalLink}
}

\newcommand{\euro}{\eurologo{}}

%================================================================================================================================
%
% Macros - Environement
%
%================================================================================================================================

\newenvironment{tex}{ %
}
{%
}

\newenvironment{indente}{ %
	\setlength\parindent{10mm}
}

{
	\setlength\parindent{0mm}
}

\newenvironment{corrige}{%
     \needspace{3\baselineskip}
     \medskip
     \textbf{\textsc{Corrigé}}
     \medskip
}
{
}

\newenvironment{extern}{%
     \begin{center}
     }
     {
     \end{center}
}

\NewEnviron{code}{%
	\par
     \boite{gray}{\texttt{%
     \BODY
     }}
     \par
}

\newenvironment{vbloc}{% boite sans cadre empeche saut de page
     \begin{minipage}[t]{\linewidth}
     }
     {
     \end{minipage}
}
\NewEnviron{h2}{%
    \needspace{3\baselineskip}
    \vspace{0.6cm}
	\noindent \MakeUppercase{\sffamily \large \BODY}
	\vspace{1mm}\textcolor{mcgris}{\hrule}\vspace{0.4cm}
	\par
}{}

\NewEnviron{h3}{%
    \needspace{3\baselineskip}
	\vspace{5mm}
	\textsc{\BODY}
	\par
}

\NewEnviron{margeneg}{ %
\begin{addmargin}[-1cm]{0cm}
\BODY
\end{addmargin}
}

\NewEnviron{html}{%
}

\begin{document}
\meta{url}{/exercices/suites-bac-blanc-es-l-sujet-1-maths-cours-2017/}
\meta{pid}{10405}
\meta{titre}{Suites - Bac blanc ES/L Sujet 1 - Maths-cours 2017}
\meta{type}{exercices}
%
\begin{h2}Exercice 3 (5 points)\end{h2}
\par
Antoine et Bruno travaillent dans deux entreprises différentes depuis le premier janvier 2015.
\par
En 2015, leurs salaires annuels s'élevaient à  $19\ 500$~euros pour Antoine et à $21\ 000$~euros pour Bruno.
\par
Chaque année, leurs salaires sont réévalués de la façon suivante :\nopagebreak
\par
\begin{itemize}
     \item le salaire d'Antoine augmente de 3\% par an ;
     \item le salaire de Bruno augmente de 500~euros par an.
\end{itemize}
\par
On note $a_n$ et $b_n$ les salaires respectifs d'Antoine et de Bruno (en~euros) pour l'année $(2015+n)$.
\par
On a donc $a_0=19\ 500$ et $b_0=21\ 000$.
\par
%============================================================================================================================
%
\TitreC{Partie A}
%
%============================================================================================================================
\par
\begin{enumerate}
     \par
     \item %1
     \par
     \begin{enumerate}[label=\alph*.]
          \par
          \item %1a
          Calculer $a_1$.
          \par
          \item %1b
          \'Etablir une relation entre $a_{n+1}$ et $a_n$.
          \par
          \item %1c
          Quelle est la nature de la suite $(a_n)$ ?
          \par
     \end{enumerate}
     \par
     \item %2
     \par
     \begin{enumerate}[label=\alph*.]
          \par
          \item %2a
          Exprimer $a_n$ en fonction de $n$.
          \par
          \item %2b
          Quel sera le salaire d'Antoine en 2030 ?
          \par
     \end{enumerate}
     \par
\end{enumerate}
\par
%============================================================================================================================
%
\TitreC{Partie B}
%
%============================================================================================================================
\par
\begin{enumerate}
     \par
     \item %1
     \par
     \begin{enumerate}[label=\alph*.]
          \item %1a
          Calculer $b_1$.
          \par
          \item %1b
          \'Etablir une relation entre $b_{n+1}$ et $b_n$.
          \par
          \item %1c
          Quelle est la nature de la suite $(b_n)$ ?
          \par
     \end{enumerate}
     \par
     \item %2
     \par
     \begin{enumerate}[label=\alph*.]
          \item %2a
          Exprimer $b_n$ en fonction de $n$.
          \par
          \item %2b
          D'Antoine ou de Bruno, qui percevra le salaire le plus élevé en 2030 ? Justifier la réponse.
          \par
     \end{enumerate}
     \par
\end{enumerate}
\par
%============================================================================================================================
%
\TitreC{Partie C}
%
%============================================================================================================================
\par
On considère l'algorithme suivant :
\par
\begin{center}
     \begin{extern}%width="400" alt="algorithme suites"
          \begin{tabularx}{0.60\linewidth}{|l|X|}\hline
               Variables :	& $n$ est un entier naturel\\
               &$a$ et $b$ sont des nombres réels\\
               & \\
               Initialisation: &Affecter à $n$ la valeur 0\\
               &Affecter à $a$ la valeur $19\ 500$\\
               &Affecter à $b$ la valeur $21\ 000$\\
               & \\
               Traitement: &Tant que $a \leqslant  b$ faire\\
               &\qquad$n$ prend la valeur $n + 1$\\
               &\qquad$a$ prend la valeur $1,03a$\\
               &\qquad$b$ prend la valeur $b+500$\\
               &Fin Tant que\\
               & \\
               Sortie :	&Afficher 2015+n \\
               \hline
          \end{tabularx}
     \end{extern}
\end{center}
\par
\begin{enumerate}
     \item Recopier et compléter le tableau ci-après, en ajoutant autant de colonnes que nécessaire. On arrondira les résultats à l'unité près.
     \begin{center}
          \begin{tabular}{|l|c|c|c|}\hline %class="compact"
               Valeur de $n$	&0	&	1 &	 $\quad \cdots \quad$  \\ \hline
               Valeur de $a$	&$19\ 500$	& $\quad \cdots \quad$ & $\quad \cdots \quad$ 	 \\ \hline
               Valeur de $b$	&$21\ 000$	& $\quad \cdots \quad$ & $\quad \cdots \quad$  \\ \hline
               Condition $a \leqslant  b$	&vraie	& $\quad \cdots \quad$ & $\quad \cdots \quad$ 	\\ \hline
          \end{tabular}
     \end{center}
     \item Quelle valeur affichera cet algorithme en sortie ?\\
     Interpréter cette valeur dans le contexte de l'exercice.
\end{enumerate}
\begin{corrige}
     %============================================================================================================================
     %
     \TitreC{Partie A}
     %
     %============================================================================================================================
     \par
     \begin{enumerate}
          \item %1
          \begin{enumerate}[label=\alph*.]
               \item %1a
               \`A une augmentation de $3\%$ correspond un coefficient multiplicateur :
               \[CM=1+\frac{3}{100}=1,03. \]
               \par
               $a_1$ représente le salaire d'Antoine en 2016. On a donc :
               \par
               $a_1=a_0 \times 1,03=19\ 500 \times 1,03 = 20\ 085.$
               \par
               \cadre{rouge}{À retenir}{
                    Pour augmenter une valeur de $t\%$, on multiplie cette valeur par le coefficient multiplicateur :
                    \[CM=1+\frac{t}{100} \]
                    \par
                    Pour diminuer une valeur de $t\%$, on multiplie cette valeur par le coefficient multiplicateur :
                    \[CM=1-\frac{t}{100} \]
               }
               \par
               \item %1b
               \par
               Le salaire d'Antoine pour l'année $n+1$ est égal à son salaire de l'année $n$ augmenté de $3\%$ ; par conséquent :
               \[a_{n+1}=1,03a_n.\]
               \par
               \item %1c
               \par
               La suite $(a_n)$ est donc une suite géométrique de premier terme $a_0=19\ 500$ et de raison $q=1,03$.
               \cadre{rouge}{Attention}{
                    \textbf{Ne pas écrire} \og la suite $a_n$ \fg{} (sans parenthèses) mais : \og la suite $(a_n)$ \fg{} (entre parenthèses).
                    \par
                    $(a_n)$ représente la \textbf{suite}, tandis que $a_n$ représente un \textbf{terme} de la suite, c'est à dire un \textbf{nombre} réel.
               }
               \par
               \cadre{rouge}{À retenir}{
                    Une \textbf{suite géométrique} $(u_n)$ est définie par une relation de récurrence de la forme :
                    \[u_{n+1}=q \times u_n\]
                    où $q$ s'appelle la \textbf{raison} de la suite.
               }
               \par
          \end{enumerate}
          \par
          \item %2
          \par
          \begin{enumerate}[label=\alph*.]
               \par
               \item %2a
               La suite $(a_n)$ étant une suite géométrique de premier terme $a_0=19\ 500$ et de raison $q=1,03$ :
               \par
               $a_n = a_0q^n=19\ 500 \times 1,03^n$.
               \cadre{rouge}{À retenir}{
                    Pour une suite \textbf{géométrique} $(u_n)$ de premier terme $u_0$ et de raison $q$, le $n$-ième terme vaut :
                    \[u_{n}=u_0 \times q^n.\]
               }
               \item %2b
               ${2030=2015+15}$. Le salaire d'Antoine en 2030 sera donc égal à $a_{15}$ :
               \par
               $a_{15}=19\ 500 \times 1,03^{15}\approx 30\ 380$ (arrondi à l'euro).
               \par
               Le salaire d'Antoine en 2030 sera 30~380~euros.
               \par
          \end{enumerate}
          \par
     \end{enumerate}
     \par
     %============================================================================================================================
     %
     \TitreC{Partie B}
     %
     %============================================================================================================================
     \par
     \begin{enumerate}
          \par
          \item %1
          \par
          \begin{enumerate}[label=\alph*.]
               \par
               \item %1a
               \par
               Le salaire de Bruno augmente de 500~euros par an, donc :
               \par
               $b_1=b_0+500=21\ 000+500=21\ 500$.
               \par
               \item %1b
               \par
               Pour la même raison :
               \par
               \[b_{n+1}=b_{n}+500.\]
               \par
               \item %1c
               \par
               La suite $(b_n)$ est une suite arithmétique de premier terme ${b_0=21\ 000}$ et de raison ${r=500}$.
               \cadre{rouge}{À retenir}{
                    Une \textbf{suite arithmétique} $(u_n)$ est définie par une relation de récurrence de la forme :
                    \[u_{n+1}= u_n + r\]
                    où $r$ s'appelle la \textbf{raison} de la suite.
               }
          \end{enumerate}
          \par
          \item %2
          \par
          \begin{enumerate}[label=\alph*.]
               \par
               \item %2a
               La suite $(b_n)$ étant une suite arithmétique de premier terme ${b_0=21\ 000}$ et de raison ${r=500}$ :
               \par
               $b_n=b_0+nr=21\ 000+500n$.
               \cadre{rouge}{À retenir}{
                    Pour une suite \textbf{arithmétique} $(u_n)$ de premier terme $u_0$ et de raison $r$, le $n$-ième terme vaut :
                    \[u_{n}=u_0 +nr \]
               }
               \item %2b
               En 2030 le salaire de Bruno sera :
               \par
               $b_{15}=21~000 + 500 \times 15 = 28~500$.
               \par
               Par conséquent, en 2030, \textbf{le salaire d'Antoine sera supérieur au salaire de Bruno}.
               \par
          \end{enumerate}
          \par
     \end{enumerate}
     \par
     %============================================================================================================================
     %
     \TitreC{Partie C}
     %
     %============================================================================================================================
     \par
     \begin{enumerate}
          \item \`A l'aide de la calculatrice on obtient le tableau suivant :
          \par
          \begin{tabular}{|c|c|c|c|c|c|c|}\hline  %class="compact"
               $n$	& 0	&	1 &	2 &	3 &	4 &	5  \\ \hline
               $a$	& $19\ 500$	& $20\ 085$  	& $20\ 688$  	&$21\ 308$  	& $21\ 947$  	& $22\ 606$   \\ \hline
               $b$	& $21\ 000$	& $21\ 500$  	& $22\ 000$  	& $22\ 500$  	& $23\ 000$  	& $23\ 500$ 	\\ \hline
               $a \leqslant  b$	& vraie	& 	vraie	& vraie	& vraie	& vraie	& vraie	 \\ \hline
          \end{tabular}
          \par
          \begin{tabular}{|c|c|c|c|c|c|}\hline %class="compact"
               $n$	& 6 &	7 &	8 &	9 &	10 \\ \hline
               $a$	& $23\ 284$  	& $23\ 983$  	& $24\ 702$  	& $25\ 443$  	& $26\ 206$  	\\ \hline
               $b$	& $24\ 000$  	& $24\ 500$  	& $25\ 000$  	& $25\ 500$  	& $26\ 000$  	\\ \hline
               $a \leqslant  b$	& vraie	& vraie	& vraie	& vraie	& fausse	  \\ \hline
          \end{tabular}
          \par
          \cadre{vert}{En pratique}{
               Pour calculer rapidement les différentes valeurs de $a$ et de $b$, on peut utiliser l'écran \og suite \fg{} ou l'écran \og fonction \fg{} de la calculatrice.
               \par
               Par contre, le jour du bac, saisir l'algorithme complet dans la calculatrice est long et peu utile.
          }
          \par
          \item Le tableau précédent montre que lorsque l'algorithme se termine $n$ vaut $10$.
          \par
          L'algorithme affiche donc la valeur ${2015+10}$ soit $2025$. Cette valeur correspond à l'année à partir de laquelle le salaire d'Antoine dépassera celui de Bruno.
     \end{enumerate}
\end{corrige}

\end{document}