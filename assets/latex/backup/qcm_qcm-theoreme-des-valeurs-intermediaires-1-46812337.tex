\documentclass[a4paper]{article}

%================================================================================================================================
%
% Packages
%
%================================================================================================================================

\usepackage[T1]{fontenc} 	% pour caractères accentués
\usepackage[utf8]{inputenc}  % encodage utf8
\usepackage[french]{babel}	% langue : français
\usepackage{fourier}			% caractères plus lisibles
\usepackage[dvipsnames]{xcolor} % couleurs
\usepackage{fancyhdr}		% réglage header footer
\usepackage{needspace}		% empêcher sauts de page mal placés
\usepackage{graphicx}		% pour inclure des graphiques
\usepackage{enumitem,cprotect}		% personnalise les listes d'items (nécessaire pour ol, al ...)
\usepackage{hyperref}		% Liens hypertexte
\usepackage{pstricks,pst-all,pst-node,pstricks-add,pst-math,pst-plot,pst-tree,pst-eucl} % pstricks
\usepackage[a4paper,includeheadfoot,top=2cm,left=3cm, bottom=2cm,right=3cm]{geometry} % marges etc.
\usepackage{comment}			% commentaires multilignes
\usepackage{amsmath,environ} % maths (matrices, etc.)
\usepackage{amssymb,makeidx}
\usepackage{bm}				% bold maths
\usepackage{tabularx}		% tableaux
\usepackage{colortbl}		% tableaux en couleur
\usepackage{fontawesome}		% Fontawesome
\usepackage{environ}			% environment with command
\usepackage{fp}				% calculs pour ps-tricks
\usepackage{multido}			% pour ps tricks
\usepackage[np]{numprint}	% formattage nombre
\usepackage{tikz,tkz-tab} 			% package principal TikZ
\usepackage{pgfplots}   % axes
\usepackage{mathrsfs}    % cursives
\usepackage{calc}			% calcul taille boites
\usepackage[scaled=0.875]{helvet} % font sans serif
\usepackage{svg} % svg
\usepackage{scrextend} % local margin
\usepackage{scratch} %scratch
\usepackage{multicol} % colonnes
%\usepackage{infix-RPN,pst-func} % formule en notation polanaise inversée
\usepackage{listings}

%================================================================================================================================
%
% Réglages de base
%
%================================================================================================================================

\lstset{
language=Python,   % R code
literate=
{á}{{\'a}}1
{à}{{\`a}}1
{ã}{{\~a}}1
{é}{{\'e}}1
{è}{{\`e}}1
{ê}{{\^e}}1
{í}{{\'i}}1
{ó}{{\'o}}1
{õ}{{\~o}}1
{ú}{{\'u}}1
{ü}{{\"u}}1
{ç}{{\c{c}}}1
{~}{{ }}1
}


\definecolor{codegreen}{rgb}{0,0.6,0}
\definecolor{codegray}{rgb}{0.5,0.5,0.5}
\definecolor{codepurple}{rgb}{0.58,0,0.82}
\definecolor{backcolour}{rgb}{0.95,0.95,0.92}

\lstdefinestyle{mystyle}{
    backgroundcolor=\color{backcolour},   
    commentstyle=\color{codegreen},
    keywordstyle=\color{magenta},
    numberstyle=\tiny\color{codegray},
    stringstyle=\color{codepurple},
    basicstyle=\ttfamily\footnotesize,
    breakatwhitespace=false,         
    breaklines=true,                 
    captionpos=b,                    
    keepspaces=true,                 
    numbers=left,                    
xleftmargin=2em,
framexleftmargin=2em,            
    showspaces=false,                
    showstringspaces=false,
    showtabs=false,                  
    tabsize=2,
    upquote=true
}

\lstset{style=mystyle}


\lstset{style=mystyle}
\newcommand{\imgdir}{C:/laragon/www/newmc/assets/imgsvg/}
\newcommand{\imgsvgdir}{C:/laragon/www/newmc/assets/imgsvg/}

\definecolor{mcgris}{RGB}{220, 220, 220}% ancien~; pour compatibilité
\definecolor{mcbleu}{RGB}{52, 152, 219}
\definecolor{mcvert}{RGB}{125, 194, 70}
\definecolor{mcmauve}{RGB}{154, 0, 215}
\definecolor{mcorange}{RGB}{255, 96, 0}
\definecolor{mcturquoise}{RGB}{0, 153, 153}
\definecolor{mcrouge}{RGB}{255, 0, 0}
\definecolor{mclightvert}{RGB}{205, 234, 190}

\definecolor{gris}{RGB}{220, 220, 220}
\definecolor{bleu}{RGB}{52, 152, 219}
\definecolor{vert}{RGB}{125, 194, 70}
\definecolor{mauve}{RGB}{154, 0, 215}
\definecolor{orange}{RGB}{255, 96, 0}
\definecolor{turquoise}{RGB}{0, 153, 153}
\definecolor{rouge}{RGB}{255, 0, 0}
\definecolor{lightvert}{RGB}{205, 234, 190}
\setitemize[0]{label=\color{lightvert}  $\bullet$}

\pagestyle{fancy}
\renewcommand{\headrulewidth}{0.2pt}
\fancyhead[L]{maths-cours.fr}
\fancyhead[R]{\thepage}
\renewcommand{\footrulewidth}{0.2pt}
\fancyfoot[C]{}

\newcolumntype{C}{>{\centering\arraybackslash}X}
\newcolumntype{s}{>{\hsize=.35\hsize\arraybackslash}X}

\setlength{\parindent}{0pt}		 
\setlength{\parskip}{3mm}
\setlength{\headheight}{1cm}

\def\ebook{ebook}
\def\book{book}
\def\web{web}
\def\type{web}

\newcommand{\vect}[1]{\overrightarrow{\,\mathstrut#1\,}}

\def\Oij{$\left(\text{O}~;~\vect{\imath},~\vect{\jmath}\right)$}
\def\Oijk{$\left(\text{O}~;~\vect{\imath},~\vect{\jmath},~\vect{k}\right)$}
\def\Ouv{$\left(\text{O}~;~\vect{u},~\vect{v}\right)$}

\hypersetup{breaklinks=true, colorlinks = true, linkcolor = OliveGreen, urlcolor = OliveGreen, citecolor = OliveGreen, pdfauthor={Didier BONNEL - https://www.maths-cours.fr} } % supprime les bordures autour des liens

\renewcommand{\arg}[0]{\text{arg}}

\everymath{\displaystyle}

%================================================================================================================================
%
% Macros - Commandes
%
%================================================================================================================================

\newcommand\meta[2]{    			% Utilisé pour créer le post HTML.
	\def\titre{titre}
	\def\url{url}
	\def\arg{#1}
	\ifx\titre\arg
		\newcommand\maintitle{#2}
		\fancyhead[L]{#2}
		{\Large\sffamily \MakeUppercase{#2}}
		\vspace{1mm}\textcolor{mcvert}{\hrule}
	\fi 
	\ifx\url\arg
		\fancyfoot[L]{\href{https://www.maths-cours.fr#2}{\black \footnotesize{https://www.maths-cours.fr#2}}}
	\fi 
}


\newcommand\TitreC[1]{    		% Titre centré
     \needspace{3\baselineskip}
     \begin{center}\textbf{#1}\end{center}
}

\newcommand\newpar{    		% paragraphe
     \par
}

\newcommand\nosp {    		% commande vide (pas d'espace)
}
\newcommand{\id}[1]{} %ignore

\newcommand\boite[2]{				% Boite simple sans titre
	\vspace{5mm}
	\setlength{\fboxrule}{0.2mm}
	\setlength{\fboxsep}{5mm}	
	\fcolorbox{#1}{#1!3}{\makebox[\linewidth-2\fboxrule-2\fboxsep]{
  		\begin{minipage}[t]{\linewidth-2\fboxrule-4\fboxsep}\setlength{\parskip}{3mm}
  			 #2
  		\end{minipage}
	}}
	\vspace{5mm}
}

\newcommand\CBox[4]{				% Boites
	\vspace{5mm}
	\setlength{\fboxrule}{0.2mm}
	\setlength{\fboxsep}{5mm}
	
	\fcolorbox{#1}{#1!3}{\makebox[\linewidth-2\fboxrule-2\fboxsep]{
		\begin{minipage}[t]{1cm}\setlength{\parskip}{3mm}
	  		\textcolor{#1}{\LARGE{#2}}    
 	 	\end{minipage}  
  		\begin{minipage}[t]{\linewidth-2\fboxrule-4\fboxsep}\setlength{\parskip}{3mm}
			\raisebox{1.2mm}{\normalsize\sffamily{\textcolor{#1}{#3}}}						
  			 #4
  		\end{minipage}
	}}
	\vspace{5mm}
}

\newcommand\cadre[3]{				% Boites convertible html
	\par
	\vspace{2mm}
	\setlength{\fboxrule}{0.1mm}
	\setlength{\fboxsep}{5mm}
	\fcolorbox{#1}{white}{\makebox[\linewidth-2\fboxrule-2\fboxsep]{
  		\begin{minipage}[t]{\linewidth-2\fboxrule-4\fboxsep}\setlength{\parskip}{3mm}
			\raisebox{-2.5mm}{\sffamily \small{\textcolor{#1}{\MakeUppercase{#2}}}}		
			\par		
  			 #3
 	 		\end{minipage}
	}}
		\vspace{2mm}
	\par
}

\newcommand\bloc[3]{				% Boites convertible html sans bordure
     \needspace{2\baselineskip}
     {\sffamily \small{\textcolor{#1}{\MakeUppercase{#2}}}}    
		\par		
  			 #3
		\par
}

\newcommand\CHelp[1]{
     \CBox{Plum}{\faInfoCircle}{À RETENIR}{#1}
}

\newcommand\CUp[1]{
     \CBox{NavyBlue}{\faThumbsOUp}{EN PRATIQUE}{#1}
}

\newcommand\CInfo[1]{
     \CBox{Sepia}{\faArrowCircleRight}{REMARQUE}{#1}
}

\newcommand\CRedac[1]{
     \CBox{PineGreen}{\faEdit}{BIEN R\'EDIGER}{#1}
}

\newcommand\CError[1]{
     \CBox{Red}{\faExclamationTriangle}{ATTENTION}{#1}
}

\newcommand\TitreExo[2]{
\needspace{4\baselineskip}
 {\sffamily\large EXERCICE #1\ (\emph{#2 points})}
\vspace{5mm}
}

\newcommand\img[2]{
          \includegraphics[width=#2\paperwidth]{\imgdir#1}
}

\newcommand\imgsvg[2]{
       \begin{center}   \includegraphics[width=#2\paperwidth]{\imgsvgdir#1} \end{center}
}


\newcommand\Lien[2]{
     \href{#1}{#2 \tiny \faExternalLink}
}
\newcommand\mcLien[2]{
     \href{https~://www.maths-cours.fr/#1}{#2 \tiny \faExternalLink}
}

\newcommand{\euro}{\eurologo{}}

%================================================================================================================================
%
% Macros - Environement
%
%================================================================================================================================

\newenvironment{tex}{ %
}
{%
}

\newenvironment{indente}{ %
	\setlength\parindent{10mm}
}

{
	\setlength\parindent{0mm}
}

\newenvironment{corrige}{%
     \needspace{3\baselineskip}
     \medskip
     \textbf{\textsc{Corrigé}}
     \medskip
}
{
}

\newenvironment{extern}{%
     \begin{center}
     }
     {
     \end{center}
}

\NewEnviron{code}{%
	\par
     \boite{gray}{\texttt{%
     \BODY
     }}
     \par
}

\newenvironment{vbloc}{% boite sans cadre empeche saut de page
     \begin{minipage}[t]{\linewidth}
     }
     {
     \end{minipage}
}
\NewEnviron{h2}{%
    \needspace{3\baselineskip}
    \vspace{0.6cm}
	\noindent \MakeUppercase{\sffamily \large \BODY}
	\vspace{1mm}\textcolor{mcgris}{\hrule}\vspace{0.4cm}
	\par
}{}

\NewEnviron{h3}{%
    \needspace{3\baselineskip}
	\vspace{5mm}
	\textsc{\BODY}
	\par
}

\NewEnviron{margeneg}{ %
\begin{addmargin}[-1cm]{0cm}
\BODY
\end{addmargin}
}

\NewEnviron{html}{%
}

\begin{document}
\meta{url}{/qcm/qcm-theoreme-des-valeurs-intermediaires-1/}
\meta{pid}{1657}
\meta{titre}{QCM : Théorème des valeurs intermédiaires}
\meta{type}{qcm}
%
§Q:Soit une fonction $f$ définie sur $\mathbb{R}$ dont le tableau de variations est le suivant :
%##
% type=table; width=25; l2=20
%--
% x|   -\infty   ~    0  ~   +\infty 
% f(x)|  +\infty    \          :-1              /   +\infty
%--
\begin{center}
 \begin{extern}%style="width:25rem" alt="Exercice"
    \resizebox{11cm}{!}{
       \definecolor{dark}{gray}{0.1}
       \definecolor{light}{gray}{0.8}
       \tikzstyle{fleche}=[->,>=latex]
       \begin{tikzpicture}[scale=.8, line width=.5pt, dark]
       \def\width{.15}
       \def\height{.10}
       \draw (0, -10*\height) -- (54*\width, -10*\height);
       \draw (10*\width, 0*\height) -- (10*\width, -10*\height);
       \node (l0c0) at (5*\width,-5*\height) {$ x $};
       \node (l0c1) at (14*\width,-5*\height) {$ -\infty $};
       \node (l0c2) at (23*\width,-5*\height) {$ ~ $};
       \node (l0c3) at (32*\width,-5*\height) {$ 0 $};
       \node (l0c4) at (41*\width,-5*\height) {$ ~ $};
       \node (l0c5) at (50*\width,-5*\height) {$ +\infty $};
       \draw (0, -30*\height) -- (54*\width, -30*\height);
       \draw (10*\width, -10*\height) -- (10*\width, -30*\height);
       \node (l1c0) at (5*\width,-20*\height) {$ f(x) $};
       \node (l1c1) at (14*\width,-15*\height) {$ +\infty $};
       \node (l1c2) at (23*\width,-20*\height) {$ ~ $};
       \draw[light] (32*\width, -10*\height) -- (32*\width, -30*\height);
       \node (l1c3) at (32*\width,-25*\height) {$ -1 $};
       \node (l1c4) at (41*\width,-20*\height) {$ ~ $};
       \node (l1c5) at (50*\width,-15*\height) {$ +\infty $};
       \draw (0, 0) rectangle (54*\width, -30*\height);
       \draw[fleche] (l1c1) -- (l1c3);
       \draw[fleche] (l1c3) -- (l1c5);
       \end{tikzpicture}
      }
   \end{extern}
\end{center}
%##
L'équation $f\left(x\right)=1$ :
§S-:n'admet pas de solution sur $\mathbb{R}$
§S-:admet une solution unique sur $\mathbb{R}$
§S+:admet 2 solutions sur $\mathbb{R}$
§S-:admet 3 solutions sur $\mathbb{R}$
§R: Sur l'intervalle $\left]-\infty ; 0\right[$ $f$ est continue est strictement décroissante de $+\infty $ à -1. Comme $1\in \left]-1,+\infty \right[$, l'équation $f\left(x\right)=1$ admet une unique solution sur l'intervalle $\left]-\infty ; 0\right[$
\par
Sur l'intervalle $\left[0; +\infty \right[$ $f$ est continue est strictement croissante de $-1$ à $+\infty $. Comme $1\in \left[-1,+\infty \right[$, l'équation $f\left(x\right)=1$ admet une unique solution sur l'intervalle $\left[0; +\infty \right[$
\par
Finalement, l'équation $f\left(x\right)=1$ possède 2 solutions sur $\mathbb{R}$
§Q:Soit une fonction $f$ définie sur $\mathbb{R}$ dont le tableau de variations est le suivant :
%##
% type=table; width=35; l2=20
%--
% x|   -\infty   ~    1   ~   3  ~   +\infty 
% f(x)|  +\infty    \          :-1            /   :2   \   -\infty
%--
\begin{center}
 \begin{extern}%style="width:35rem" alt="Exercice"
    \resizebox{11cm}{!}{
       \definecolor{dark}{gray}{0.1}
       \definecolor{light}{gray}{0.8}
       \tikzstyle{fleche}=[->,>=latex]
       \begin{tikzpicture}[scale=.8, line width=.5pt, dark]
       \def\width{.15}
       \def\height{.10}
       \draw (0, -10*\height) -- (72*\width, -10*\height);
       \draw (10*\width, 0*\height) -- (10*\width, -10*\height);
       \node (l0c0) at (5*\width,-5*\height) {$ x $};
       \node (l0c1) at (14*\width,-5*\height) {$ -\infty $};
       \node (l0c2) at (23*\width,-5*\height) {$ ~ $};
       \node (l0c3) at (32*\width,-5*\height) {$ 1 $};
       \node (l0c4) at (41*\width,-5*\height) {$ ~ $};
       \node (l0c5) at (50*\width,-5*\height) {$ 3 $};
       \node (l0c6) at (59*\width,-5*\height) {$ ~ $};
       \node (l0c7) at (68*\width,-5*\height) {$ +\infty $};
       \draw (0, -30*\height) -- (72*\width, -30*\height);
       \draw (10*\width, -10*\height) -- (10*\width, -30*\height);
       \node (l1c0) at (5*\width,-20*\height) {$ f(x) $};
       \node (l1c1) at (14*\width,-15*\height) {$ +\infty $};
       \node (l1c2) at (23*\width,-20*\height) {$ ~ $};
       \draw[light] (32*\width, -10*\height) -- (32*\width, -30*\height);
       \node (l1c3) at (32*\width,-25*\height) {$ -1 $};
       \node (l1c4) at (41*\width,-20*\height) {$ ~ $};
       \draw[light] (50*\width, -10*\height) -- (50*\width, -30*\height);
       \node (l1c5) at (50*\width,-15*\height) {$ 2 $};
       \node (l1c6) at (59*\width,-20*\height) {$ ~ $};
       \node (l1c7) at (68*\width,-25*\height) {$ -\infty $};
       \draw (0, 0) rectangle (72*\width, -30*\height);
       \draw[fleche] (l1c1) -- (l1c3);
       \draw[fleche] (l1c3) -- (l1c5);
       \draw[fleche] (l1c5) -- (l1c7);
       \end{tikzpicture}
      }
   \end{extern}
\end{center}
%##
L'équation $f\left(x\right)=0$ : 
§S-:n'admet pas de solution sur $\mathbb{R}$
§S-:admet une solution unique sur $\mathbb{R}$
§S-:admet 2 solutions sur $\mathbb{R}$
§S+:admet 3 solutions sur $\mathbb{R}$
§R:Raisonnement analogue à la question précédente.
\par
Une solution sur $\left]-\infty ;1\right[$; une solution sur $\left]1;3\right[$; une solution sur $\left]3;+\infty \right[$
§Q:Soit une fonction $f$ définie sur $I=\left[-1; 5\right]$ dont le tableau de variations est le suivant :
<img src="/wp-content/uploads/fvar1.png" alt="" class="aligncenter size-full  img-pc" />
%##
% type=table; width=25; l2=20
%--
% x|   -1   ~   4  ~   5 
% f(x)|  0    \          :-1              /   1
%--
\begin{center}
 \begin{extern}%style="width:25rem" alt="Exercice"
    \resizebox{11cm}{!}{
       \definecolor{dark}{gray}{0.1}
       \definecolor{light}{gray}{0.8}
       \tikzstyle{fleche}=[->,>=latex]
       \begin{tikzpicture}[scale=.8, line width=.5pt, dark]
       \def\width{.15}
       \def\height{.10}
       \draw (0, -10*\height) -- (54*\width, -10*\height);
       \draw (10*\width, 0*\height) -- (10*\width, -10*\height);
       \node (l0c0) at (5*\width,-5*\height) {$ x $};
       \node (l0c1) at (14*\width,-5*\height) {$ -1 $};
       \node (l0c2) at (23*\width,-5*\height) {$ ~ $};
       \node (l0c3) at (32*\width,-5*\height) {$ 4 $};
       \node (l0c4) at (41*\width,-5*\height) {$ ~ $};
       \node (l0c5) at (50*\width,-5*\height) {$ 5 $};
       \draw (0, -30*\height) -- (54*\width, -30*\height);
       \draw (10*\width, -10*\height) -- (10*\width, -30*\height);
       \node (l1c0) at (5*\width,-20*\height) {$ f(x) $};
       \node (l1c1) at (14*\width,-15*\height) {$ 0 $};
       \node (l1c2) at (23*\width,-20*\height) {$ ~ $};
       \draw[light] (32*\width, -10*\height) -- (32*\width, -30*\height);
       \node (l1c3) at (32*\width,-25*\height) {$ -1 $};
       \node (l1c4) at (41*\width,-20*\height) {$ ~ $};
       \node (l1c5) at (50*\width,-15*\height) {$ 1 $};
       \draw (0, 0) rectangle (54*\width, -30*\height);
       \draw[fleche] (l1c1) -- (l1c3);
       \draw[fleche] (l1c3) -- (l1c5);
       \end{tikzpicture}
      }
   \end{extern}
\end{center}
%##
L'équation $f\left(x\right)=2$ :
§S+:n'admet pas de solution sur $I$
§S-:admet une solution unique sur $I$
§S-:admet 2 solutions sur $I$
§S-:admet 3 solutions sur $I$
§R: Sur l'intervalle $\left]-1; 4\right[$, $f\left(x\right)$ est inférieur à 0 donc ne peut pas valoir 2. L'équation $f\left(x\right)=2$ n'admet pas de solution sur cet intervalle.
\par
De même, sur l'intervalle $\left]4 ; 5\right[$ $f\left(x\right)$ est inférieur à 1 donc ne peut pas valoir 2. L'équation $f\left(x\right)=2$ n'admet pas de solution sur cet intervalle.
\par
Pour $x=4$: $f\left(4\right)=-1\neq 2$
\par
Finalement, l'équation $f\left(x\right)=2$ n'admet pas de solution sur $I$
§Q:Soit une fonction $f$ définie sur $I=\left[0; 4\right]$ dont le tableau de variations est le suivant :
%##
% type=table; width=35; l2=20
%--
% x|   -\infty   ~    1   ~   3  ~   +\infty 
% f(x)|  +\infty    \          :-1            /   :2   \   -\infty
%--
\begin{center}
 \begin{extern}%style="width:35rem" alt="Exercice"
    \resizebox{11cm}{!}{
       \definecolor{dark}{gray}{0.1}
       \definecolor{light}{gray}{0.8}
       \tikzstyle{fleche}=[->,>=latex]
       \begin{tikzpicture}[scale=.8, line width=.5pt, dark]
       \def\width{.15}
       \def\height{.10}
       \draw (0, -10*\height) -- (72*\width, -10*\height);
       \draw (10*\width, 0*\height) -- (10*\width, -10*\height);
       \node (l0c0) at (5*\width,-5*\height) {$ x $};
       \node (l0c1) at (14*\width,-5*\height) {$ -\infty $};
       \node (l0c2) at (23*\width,-5*\height) {$ ~ $};
       \node (l0c3) at (32*\width,-5*\height) {$ 1 $};
       \node (l0c4) at (41*\width,-5*\height) {$ ~ $};
       \node (l0c5) at (50*\width,-5*\height) {$ 3 $};
       \node (l0c6) at (59*\width,-5*\height) {$ ~ $};
       \node (l0c7) at (68*\width,-5*\height) {$ +\infty $};
       \draw (0, -30*\height) -- (72*\width, -30*\height);
       \draw (10*\width, -10*\height) -- (10*\width, -30*\height);
       \node (l1c0) at (5*\width,-20*\height) {$ f(x) $};
       \node (l1c1) at (14*\width,-15*\height) {$ +\infty $};
       \node (l1c2) at (23*\width,-20*\height) {$ ~ $};
       \draw[light] (32*\width, -10*\height) -- (32*\width, -30*\height);
       \node (l1c3) at (32*\width,-25*\height) {$ -1 $};
       \node (l1c4) at (41*\width,-20*\height) {$ ~ $};
       \draw[light] (50*\width, -10*\height) -- (50*\width, -30*\height);
       \node (l1c5) at (50*\width,-15*\height) {$ 2 $};
       \node (l1c6) at (59*\width,-20*\height) {$ ~ $};
       \node (l1c7) at (68*\width,-25*\height) {$ -\infty $};
       \draw (0, 0) rectangle (72*\width, -30*\height);
       \draw[fleche] (l1c1) -- (l1c3);
       \draw[fleche] (l1c3) -- (l1c5);
       \draw[fleche] (l1c5) -- (l1c7);
       \end{tikzpicture}
      }
   \end{extern}
\end{center}
%##
L'équation $f\left(x\right)=\frac{1}{2}$ :
§S-:n'admet pas de solution sur $I$
§S-:admet une solution unique sur $I$
§S+:admet 2 solutions sur $I$
§S-:admet 3 solutions sur $I$
§R:Raisonnement analogue aux questions précédentes.
\par
Pas de solution sur $\left]0;1\right[$; une solution sur $\left]1;2\right[$; une solution sur $\left]2;4\right[$
§Q:Soit une fonction $f$ définie sur $I=\left[-5; 5\right]$ dont le tableau de variations est le suivant :
%##
% type=table; width=25; l2=20
%--
% x|   0   ~    0  ~   +\infty 
% f(x)|  +\infty    \          :-1              /   +\infty
%--
\begin{center}
 \begin{extern}%style="width:25rem" alt="Exercice"
    \resizebox{11cm}{!}{
       \definecolor{dark}{gray}{0.1}
       \definecolor{light}{gray}{0.8}
       \tikzstyle{fleche}=[->,>=latex]
       \begin{tikzpicture}[scale=.8, line width=.5pt, dark]
       \def\width{.15}
       \def\height{.10}
       \draw (0, -10*\height) -- (54*\width, -10*\height);
       \draw (10*\width, 0*\height) -- (10*\width, -10*\height);
       \node (l0c0) at (5*\width,-5*\height) {$ x $};
       \node (l0c1) at (14*\width,-5*\height) {$ 0 $};
       \node (l0c2) at (23*\width,-5*\height) {$ ~ $};
       \node (l0c3) at (32*\width,-5*\height) {$ 0 $};
       \node (l0c4) at (41*\width,-5*\height) {$ ~ $};
       \node (l0c5) at (50*\width,-5*\height) {$ +\infty $};
       \draw (0, -30*\height) -- (54*\width, -30*\height);
       \draw (10*\width, -10*\height) -- (10*\width, -30*\height);
       \node (l1c0) at (5*\width,-20*\height) {$ f(x) $};
       \node (l1c1) at (14*\width,-15*\height) {$ +\infty $};
       \node (l1c2) at (23*\width,-20*\height) {$ ~ $};
       \draw[light] (32*\width, -10*\height) -- (32*\width, -30*\height);
       \node (l1c3) at (32*\width,-25*\height) {$ -1 $};
       \node (l1c4) at (41*\width,-20*\height) {$ ~ $};
       \node (l1c5) at (50*\width,-15*\height) {$ +\infty $};
       \draw (0, 0) rectangle (54*\width, -30*\height);
       \draw[fleche] (l1c1) -- (l1c3);
       \draw[fleche] (l1c3) -- (l1c5);
       \end{tikzpicture}
      }
   \end{extern}
\end{center}
%##
<img src="/wp-content/uploads/fvar3.png" alt="" class="aligncenter size-full  img-pc" />
L'équation $f\left(x\right)=1$ :
§S-:n'admet pas de solution sur $I$
§S-:admet une solution unique sur $I$
§S+:admet 2 solutions sur $I$
§S-:admet 3 solutions sur $I$
§R:Raisonnement analogue aux questions précédentes.
\par
Une solution sur $\left]-5;-2\right[$; une solution sur $\left]-2;5\right[$
\par

\end{document}