\documentclass[a4paper]{article}

%================================================================================================================================
%
% Packages
%
%================================================================================================================================

\usepackage[T1]{fontenc} 	% pour caractères accentués
\usepackage[utf8]{inputenc}  % encodage utf8
\usepackage[french]{babel}	% langue : français
\usepackage{fourier}			% caractères plus lisibles
\usepackage[dvipsnames]{xcolor} % couleurs
\usepackage{fancyhdr}		% réglage header footer
\usepackage{needspace}		% empêcher sauts de page mal placés
\usepackage{graphicx}		% pour inclure des graphiques
\usepackage{enumitem,cprotect}		% personnalise les listes d'items (nécessaire pour ol, al ...)
\usepackage{hyperref}		% Liens hypertexte
\usepackage{pstricks,pst-all,pst-node,pstricks-add,pst-math,pst-plot,pst-tree,pst-eucl} % pstricks
\usepackage[a4paper,includeheadfoot,top=2cm,left=3cm, bottom=2cm,right=3cm]{geometry} % marges etc.
\usepackage{comment}			% commentaires multilignes
\usepackage{amsmath,environ} % maths (matrices, etc.)
\usepackage{amssymb,makeidx}
\usepackage{bm}				% bold maths
\usepackage{tabularx}		% tableaux
\usepackage{colortbl}		% tableaux en couleur
\usepackage{fontawesome}		% Fontawesome
\usepackage{environ}			% environment with command
\usepackage{fp}				% calculs pour ps-tricks
\usepackage{multido}			% pour ps tricks
\usepackage[np]{numprint}	% formattage nombre
\usepackage{tikz,tkz-tab} 			% package principal TikZ
\usepackage{pgfplots}   % axes
\usepackage{mathrsfs}    % cursives
\usepackage{calc}			% calcul taille boites
\usepackage[scaled=0.875]{helvet} % font sans serif
\usepackage{svg} % svg
\usepackage{scrextend} % local margin
\usepackage{scratch} %scratch
\usepackage{multicol} % colonnes
%\usepackage{infix-RPN,pst-func} % formule en notation polanaise inversée
\usepackage{listings}

%================================================================================================================================
%
% Réglages de base
%
%================================================================================================================================

\lstset{
language=Python,   % R code
literate=
{á}{{\'a}}1
{à}{{\`a}}1
{ã}{{\~a}}1
{é}{{\'e}}1
{è}{{\`e}}1
{ê}{{\^e}}1
{í}{{\'i}}1
{ó}{{\'o}}1
{õ}{{\~o}}1
{ú}{{\'u}}1
{ü}{{\"u}}1
{ç}{{\c{c}}}1
{~}{{ }}1
}


\definecolor{codegreen}{rgb}{0,0.6,0}
\definecolor{codegray}{rgb}{0.5,0.5,0.5}
\definecolor{codepurple}{rgb}{0.58,0,0.82}
\definecolor{backcolour}{rgb}{0.95,0.95,0.92}

\lstdefinestyle{mystyle}{
    backgroundcolor=\color{backcolour},   
    commentstyle=\color{codegreen},
    keywordstyle=\color{magenta},
    numberstyle=\tiny\color{codegray},
    stringstyle=\color{codepurple},
    basicstyle=\ttfamily\footnotesize,
    breakatwhitespace=false,         
    breaklines=true,                 
    captionpos=b,                    
    keepspaces=true,                 
    numbers=left,                    
xleftmargin=2em,
framexleftmargin=2em,            
    showspaces=false,                
    showstringspaces=false,
    showtabs=false,                  
    tabsize=2,
    upquote=true
}

\lstset{style=mystyle}


\lstset{style=mystyle}
\newcommand{\imgdir}{C:/laragon/www/newmc/assets/imgsvg/}
\newcommand{\imgsvgdir}{C:/laragon/www/newmc/assets/imgsvg/}

\definecolor{mcgris}{RGB}{220, 220, 220}% ancien~; pour compatibilité
\definecolor{mcbleu}{RGB}{52, 152, 219}
\definecolor{mcvert}{RGB}{125, 194, 70}
\definecolor{mcmauve}{RGB}{154, 0, 215}
\definecolor{mcorange}{RGB}{255, 96, 0}
\definecolor{mcturquoise}{RGB}{0, 153, 153}
\definecolor{mcrouge}{RGB}{255, 0, 0}
\definecolor{mclightvert}{RGB}{205, 234, 190}

\definecolor{gris}{RGB}{220, 220, 220}
\definecolor{bleu}{RGB}{52, 152, 219}
\definecolor{vert}{RGB}{125, 194, 70}
\definecolor{mauve}{RGB}{154, 0, 215}
\definecolor{orange}{RGB}{255, 96, 0}
\definecolor{turquoise}{RGB}{0, 153, 153}
\definecolor{rouge}{RGB}{255, 0, 0}
\definecolor{lightvert}{RGB}{205, 234, 190}
\setitemize[0]{label=\color{lightvert}  $\bullet$}

\pagestyle{fancy}
\renewcommand{\headrulewidth}{0.2pt}
\fancyhead[L]{maths-cours.fr}
\fancyhead[R]{\thepage}
\renewcommand{\footrulewidth}{0.2pt}
\fancyfoot[C]{}

\newcolumntype{C}{>{\centering\arraybackslash}X}
\newcolumntype{s}{>{\hsize=.35\hsize\arraybackslash}X}

\setlength{\parindent}{0pt}		 
\setlength{\parskip}{3mm}
\setlength{\headheight}{1cm}

\def\ebook{ebook}
\def\book{book}
\def\web{web}
\def\type{web}

\newcommand{\vect}[1]{\overrightarrow{\,\mathstrut#1\,}}

\def\Oij{$\left(\text{O}~;~\vect{\imath},~\vect{\jmath}\right)$}
\def\Oijk{$\left(\text{O}~;~\vect{\imath},~\vect{\jmath},~\vect{k}\right)$}
\def\Ouv{$\left(\text{O}~;~\vect{u},~\vect{v}\right)$}

\hypersetup{breaklinks=true, colorlinks = true, linkcolor = OliveGreen, urlcolor = OliveGreen, citecolor = OliveGreen, pdfauthor={Didier BONNEL - https://www.maths-cours.fr} } % supprime les bordures autour des liens

\renewcommand{\arg}[0]{\text{arg}}

\everymath{\displaystyle}

%================================================================================================================================
%
% Macros - Commandes
%
%================================================================================================================================

\newcommand\meta[2]{    			% Utilisé pour créer le post HTML.
	\def\titre{titre}
	\def\url{url}
	\def\arg{#1}
	\ifx\titre\arg
		\newcommand\maintitle{#2}
		\fancyhead[L]{#2}
		{\Large\sffamily \MakeUppercase{#2}}
		\vspace{1mm}\textcolor{mcvert}{\hrule}
	\fi 
	\ifx\url\arg
		\fancyfoot[L]{\href{https://www.maths-cours.fr#2}{\black \footnotesize{https://www.maths-cours.fr#2}}}
	\fi 
}


\newcommand\TitreC[1]{    		% Titre centré
     \needspace{3\baselineskip}
     \begin{center}\textbf{#1}\end{center}
}

\newcommand\newpar{    		% paragraphe
     \par
}

\newcommand\nosp {    		% commande vide (pas d'espace)
}
\newcommand{\id}[1]{} %ignore

\newcommand\boite[2]{				% Boite simple sans titre
	\vspace{5mm}
	\setlength{\fboxrule}{0.2mm}
	\setlength{\fboxsep}{5mm}	
	\fcolorbox{#1}{#1!3}{\makebox[\linewidth-2\fboxrule-2\fboxsep]{
  		\begin{minipage}[t]{\linewidth-2\fboxrule-4\fboxsep}\setlength{\parskip}{3mm}
  			 #2
  		\end{minipage}
	}}
	\vspace{5mm}
}

\newcommand\CBox[4]{				% Boites
	\vspace{5mm}
	\setlength{\fboxrule}{0.2mm}
	\setlength{\fboxsep}{5mm}
	
	\fcolorbox{#1}{#1!3}{\makebox[\linewidth-2\fboxrule-2\fboxsep]{
		\begin{minipage}[t]{1cm}\setlength{\parskip}{3mm}
	  		\textcolor{#1}{\LARGE{#2}}    
 	 	\end{minipage}  
  		\begin{minipage}[t]{\linewidth-2\fboxrule-4\fboxsep}\setlength{\parskip}{3mm}
			\raisebox{1.2mm}{\normalsize\sffamily{\textcolor{#1}{#3}}}						
  			 #4
  		\end{minipage}
	}}
	\vspace{5mm}
}

\newcommand\cadre[3]{				% Boites convertible html
	\par
	\vspace{2mm}
	\setlength{\fboxrule}{0.1mm}
	\setlength{\fboxsep}{5mm}
	\fcolorbox{#1}{white}{\makebox[\linewidth-2\fboxrule-2\fboxsep]{
  		\begin{minipage}[t]{\linewidth-2\fboxrule-4\fboxsep}\setlength{\parskip}{3mm}
			\raisebox{-2.5mm}{\sffamily \small{\textcolor{#1}{\MakeUppercase{#2}}}}		
			\par		
  			 #3
 	 		\end{minipage}
	}}
		\vspace{2mm}
	\par
}

\newcommand\bloc[3]{				% Boites convertible html sans bordure
     \needspace{2\baselineskip}
     {\sffamily \small{\textcolor{#1}{\MakeUppercase{#2}}}}    
		\par		
  			 #3
		\par
}

\newcommand\CHelp[1]{
     \CBox{Plum}{\faInfoCircle}{À RETENIR}{#1}
}

\newcommand\CUp[1]{
     \CBox{NavyBlue}{\faThumbsOUp}{EN PRATIQUE}{#1}
}

\newcommand\CInfo[1]{
     \CBox{Sepia}{\faArrowCircleRight}{REMARQUE}{#1}
}

\newcommand\CRedac[1]{
     \CBox{PineGreen}{\faEdit}{BIEN R\'EDIGER}{#1}
}

\newcommand\CError[1]{
     \CBox{Red}{\faExclamationTriangle}{ATTENTION}{#1}
}

\newcommand\TitreExo[2]{
\needspace{4\baselineskip}
 {\sffamily\large EXERCICE #1\ (\emph{#2 points})}
\vspace{5mm}
}

\newcommand\img[2]{
          \includegraphics[width=#2\paperwidth]{\imgdir#1}
}

\newcommand\imgsvg[2]{
       \begin{center}   \includegraphics[width=#2\paperwidth]{\imgsvgdir#1} \end{center}
}


\newcommand\Lien[2]{
     \href{#1}{#2 \tiny \faExternalLink}
}
\newcommand\mcLien[2]{
     \href{https~://www.maths-cours.fr/#1}{#2 \tiny \faExternalLink}
}

\newcommand{\euro}{\eurologo{}}

%================================================================================================================================
%
% Macros - Environement
%
%================================================================================================================================

\newenvironment{tex}{ %
}
{%
}

\newenvironment{indente}{ %
	\setlength\parindent{10mm}
}

{
	\setlength\parindent{0mm}
}

\newenvironment{corrige}{%
     \needspace{3\baselineskip}
     \medskip
     \textbf{\textsc{Corrigé}}
     \medskip
}
{
}

\newenvironment{extern}{%
     \begin{center}
     }
     {
     \end{center}
}

\NewEnviron{code}{%
	\par
     \boite{gray}{\texttt{%
     \BODY
     }}
     \par
}

\newenvironment{vbloc}{% boite sans cadre empeche saut de page
     \begin{minipage}[t]{\linewidth}
     }
     {
     \end{minipage}
}
\NewEnviron{h2}{%
    \needspace{3\baselineskip}
    \vspace{0.6cm}
	\noindent \MakeUppercase{\sffamily \large \BODY}
	\vspace{1mm}\textcolor{mcgris}{\hrule}\vspace{0.4cm}
	\par
}{}

\NewEnviron{h3}{%
    \needspace{3\baselineskip}
	\vspace{5mm}
	\textsc{\BODY}
	\par
}

\NewEnviron{margeneg}{ %
\begin{addmargin}[-1cm]{0cm}
\BODY
\end{addmargin}
}

\NewEnviron{html}{%
}

\begin{document}
\meta{url}{/exercices/medianes-tetraedre-bac-s-pondichery-2011/}
\meta{pid}{2451}
\meta{titre}{Médianes d'un tétraèdre - Bac S Pondichéry 2011}
\meta{type}{exercices}
%
\begin{h2}Exercice 2\end{h2}
\textbf{Candidats n'ayant pas suivi l'enseignement de spécialité}
\begin{h3}Partie 1\end{h3}
Dans cette partie, ABCD est un tétraèdre régulier, c'est-à-dire un solide dont les quatre faces sont des triangles équilatéraux.
<img src="/wp-content/uploads/bac-s-pondichery-2011-1.png" alt="" class="aligncenter size-full  img-pc" />

\begin{center}
\imgsvg{bac-s-pondichery-2011-1}{0.3}% alt="@" style="width:30rem"
\end{center}
A' est le centre de gravité du triangle BCD.
\par
Dans un tétraèdre, le segment joignant un sommet au centre de gravité de la face opposée est appelé médiane. Ainsi, le segment [AA'] est une médiane du tétraèdre ABCD.
\begin{enumerate}
     \item
     On souhaite démontrer la propriété suivante :
     \textbf{P<sub>1</sub> : Dans un tétraèdre régulier, chaque médiane est orthogonale à la face opposée.}
     \begin{enumerate}
          \item
          Montrer que $\overrightarrow{AA^{\prime}} . \overrightarrow{BD}=0$ et que $\overrightarrow{AA^{\prime}} . \overrightarrow{BC}=0$.
          \par
          (On pourra utiliser le milieu I du segment [BD] et le milieu J du segment [BC]).
          \item
          En déduire que la médiane (AA') est orthogonale à la face BCD.
          \par
          Un raisonnement analogue montre que les autres médianes du tétraèdre régulier ABCD sont également orthogonales à leur face opposée.
     \end{enumerate}
     \item
     G est l'isobarycentre des points A, B, C et D.
     \par
     On souhaite démontrer la propriété suivante :
     \textbf{P<sub>2</sub> : Les médianes d'un tétraèdre régulier sont concourantes en G.}
     En utilisant l'associativité du barycentre, montrer que G appartient à la droite (AA'), puis conclure.
\end{enumerate}
\begin{h3}Partie II\end{h3}
On munit l'espace d'un repère orthonormal $\left(O ; \vec{i},\vec{j},\vec{k}\right)$.
\par
On considère les points $P\left(1 ; 2 ; 3\right), Q\left(4 ;  2 ; -1\right)$ et $R\left(-2 ; 3 ; 0\right)$.
\begin{enumerate}
     \item
     Montrer que le tétraèdre $OPQR$ n'est pas régulier.
     \item
     Calculer les coordonnées de $P^{\prime}$, centre de gravité du triangle $OQR$.
     \item
     Vérifier qu'une équation cartésienne du plan $\left(OQR\right)$ est : $3x+2y+16z=0$.
     \item
     La propriété \textbf{P<sub>1</sub>} de la \textbf{partie 1} est-elle vraie dans un tétraèdre quelconque ?
\end{enumerate}
\begin{corrige}
     \begin{h3} Partie I \end{h3}
     \begin{enumerate}
          \item
          \begin{enumerate}
               \item
               Soit I le milieu du segment [BD]. La droite (CI) est une médiane et donc une hauteur du triangle équilatéral BCD. Comme le point A' appartient à cette hauteur, $\overrightarrow{A^{\prime}I}.\overrightarrow{BD}=0$.
               \par
               De même (AI) est une hauteur du triangle ABD donc $\overrightarrow{AI}.\overrightarrow{BD}=0$.
               \par
               D'après la relation de Chasles :
               \par
               $\overrightarrow{AA^{\prime}}.\overrightarrow{BD}=\left(\overrightarrow{AI}+\overrightarrow{IA^{\prime}}\right).\overrightarrow{BD}=\overrightarrow{AI}.\overrightarrow{BD}+\overrightarrow{IA^{\prime}}.\overrightarrow{BD}=0$
               \par
               On démontre de la même manière que $\overrightarrow{AA^{\prime}}.\overrightarrow{BC}=0$.
               \item
               La médiane (AA') est orthogonale à deux droites sécantes du plan (BCD) donc elle est orthogonale au plan (BCD).
          \end{enumerate}
          \item
          G est la barycentre du système {(A;1),(B;1),(C;1),(D;1)} donc d'après l'associativité du barycentre, G est le barycentre de {(A;1),(A';3)} donc G appartient à la médiane (AA').
          \par
          La démonstration est analogue pour les autres médianes.
          \par
          Donc les médianes du tétraèdre ABCD sont concourantes en G.
     \end{enumerate}
     \begin{h3} Partie II \end{h3}
     \begin{enumerate}
          \item
          $OP=\sqrt{1^{2}+2^{2}+3^{2}}=\sqrt{14}$
          \par
          $OQ=\sqrt{4^{2}+2^{2}+\left(-1\right)^{2}}=\sqrt{21}$
          \par
          $OP\neq OQ$ donc le tétraèdre OPQR n'est pas régulier.
          \item
          $x_{P}=\frac{1}{3}\times \left(x_{O}+x_{Q}+x_{R}\right)=\frac{1+4-2}{3}=1$
          \par
          $y_{P}=\frac{2+2-3}{3}=\frac{1}{3}$
          \par
          $z_{P}=\frac{3-1}{3}=\frac{2}{3}$
          \item
          Les coordonnées de O, Q, R vérifient l'équation $3x+2y+16z=0$.
          \par
          En effet :
          \par
          $3\times 0+2\times 0+16\times 0=0$
          \par
          $3\times 4+2\times 2+16\times \left(-1\right)=0$
          \par
          $3\times \left(-2\right)+2\times 3+16\times 0=0$
          \par
          Donc $3x+2y+16z=0$ est bien une équation du plan (OQR).
          \item
          La propriété P<sub>1</sub> n'est pas vraie dans ce tétraèdre car la droite (PP') n'est pas orthogonale au plan (OQR). En effet, le vecteur $\overrightarrow{PP^{\prime}}$ a pour coordonnées $\left(\frac{2}{3}-1;\frac{5}{3}-2;-\frac{1}{3}-3\right)$ c'est-à-dire
          \par
          $\left(-\frac{1}{3};-\frac{1}{3};-\frac{10}{3}\right)$.
          \par
          Or, un vecteur normal au plan (OQR) est $n\left(3;2;16\right)$. Ce vecteur n'est pas colinéaire au vecteur $\overrightarrow{PP^{\prime}}$.
     \end{enumerate}
}\end{corrige}

\end{document}