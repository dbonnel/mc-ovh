\documentclass[a4paper]{article}

%================================================================================================================================
%
% Packages
%
%================================================================================================================================

\usepackage[T1]{fontenc} 	% pour caractères accentués
\usepackage[utf8]{inputenc}  % encodage utf8
\usepackage[french]{babel}	% langue : français
\usepackage{fourier}			% caractères plus lisibles
\usepackage[dvipsnames]{xcolor} % couleurs
\usepackage{fancyhdr}		% réglage header footer
\usepackage{needspace}		% empêcher sauts de page mal placés
\usepackage{graphicx}		% pour inclure des graphiques
\usepackage{enumitem,cprotect}		% personnalise les listes d'items (nécessaire pour ol, al ...)
\usepackage{hyperref}		% Liens hypertexte
\usepackage{pstricks,pst-all,pst-node,pstricks-add,pst-math,pst-plot,pst-tree,pst-eucl} % pstricks
\usepackage[a4paper,includeheadfoot,top=2cm,left=3cm, bottom=2cm,right=3cm]{geometry} % marges etc.
\usepackage{comment}			% commentaires multilignes
\usepackage{amsmath,environ} % maths (matrices, etc.)
\usepackage{amssymb,makeidx}
\usepackage{bm}				% bold maths
\usepackage{tabularx}		% tableaux
\usepackage{colortbl}		% tableaux en couleur
\usepackage{fontawesome}		% Fontawesome
\usepackage{environ}			% environment with command
\usepackage{fp}				% calculs pour ps-tricks
\usepackage{multido}			% pour ps tricks
\usepackage[np]{numprint}	% formattage nombre
\usepackage{tikz,tkz-tab} 			% package principal TikZ
\usepackage{pgfplots}   % axes
\usepackage{mathrsfs}    % cursives
\usepackage{calc}			% calcul taille boites
\usepackage[scaled=0.875]{helvet} % font sans serif
\usepackage{svg} % svg
\usepackage{scrextend} % local margin
\usepackage{scratch} %scratch
\usepackage{multicol} % colonnes
%\usepackage{infix-RPN,pst-func} % formule en notation polanaise inversée
\usepackage{listings}

%================================================================================================================================
%
% Réglages de base
%
%================================================================================================================================

\lstset{
language=Python,   % R code
literate=
{á}{{\'a}}1
{à}{{\`a}}1
{ã}{{\~a}}1
{é}{{\'e}}1
{è}{{\`e}}1
{ê}{{\^e}}1
{í}{{\'i}}1
{ó}{{\'o}}1
{õ}{{\~o}}1
{ú}{{\'u}}1
{ü}{{\"u}}1
{ç}{{\c{c}}}1
{~}{{ }}1
}


\definecolor{codegreen}{rgb}{0,0.6,0}
\definecolor{codegray}{rgb}{0.5,0.5,0.5}
\definecolor{codepurple}{rgb}{0.58,0,0.82}
\definecolor{backcolour}{rgb}{0.95,0.95,0.92}

\lstdefinestyle{mystyle}{
    backgroundcolor=\color{backcolour},   
    commentstyle=\color{codegreen},
    keywordstyle=\color{magenta},
    numberstyle=\tiny\color{codegray},
    stringstyle=\color{codepurple},
    basicstyle=\ttfamily\footnotesize,
    breakatwhitespace=false,         
    breaklines=true,                 
    captionpos=b,                    
    keepspaces=true,                 
    numbers=left,                    
xleftmargin=2em,
framexleftmargin=2em,            
    showspaces=false,                
    showstringspaces=false,
    showtabs=false,                  
    tabsize=2,
    upquote=true
}

\lstset{style=mystyle}


\lstset{style=mystyle}
\newcommand{\imgdir}{C:/laragon/www/newmc/assets/imgsvg/}
\newcommand{\imgsvgdir}{C:/laragon/www/newmc/assets/imgsvg/}

\definecolor{mcgris}{RGB}{220, 220, 220}% ancien~; pour compatibilité
\definecolor{mcbleu}{RGB}{52, 152, 219}
\definecolor{mcvert}{RGB}{125, 194, 70}
\definecolor{mcmauve}{RGB}{154, 0, 215}
\definecolor{mcorange}{RGB}{255, 96, 0}
\definecolor{mcturquoise}{RGB}{0, 153, 153}
\definecolor{mcrouge}{RGB}{255, 0, 0}
\definecolor{mclightvert}{RGB}{205, 234, 190}

\definecolor{gris}{RGB}{220, 220, 220}
\definecolor{bleu}{RGB}{52, 152, 219}
\definecolor{vert}{RGB}{125, 194, 70}
\definecolor{mauve}{RGB}{154, 0, 215}
\definecolor{orange}{RGB}{255, 96, 0}
\definecolor{turquoise}{RGB}{0, 153, 153}
\definecolor{rouge}{RGB}{255, 0, 0}
\definecolor{lightvert}{RGB}{205, 234, 190}
\setitemize[0]{label=\color{lightvert}  $\bullet$}

\pagestyle{fancy}
\renewcommand{\headrulewidth}{0.2pt}
\fancyhead[L]{maths-cours.fr}
\fancyhead[R]{\thepage}
\renewcommand{\footrulewidth}{0.2pt}
\fancyfoot[C]{}

\newcolumntype{C}{>{\centering\arraybackslash}X}
\newcolumntype{s}{>{\hsize=.35\hsize\arraybackslash}X}

\setlength{\parindent}{0pt}		 
\setlength{\parskip}{3mm}
\setlength{\headheight}{1cm}

\def\ebook{ebook}
\def\book{book}
\def\web{web}
\def\type{web}

\newcommand{\vect}[1]{\overrightarrow{\,\mathstrut#1\,}}

\def\Oij{$\left(\text{O}~;~\vect{\imath},~\vect{\jmath}\right)$}
\def\Oijk{$\left(\text{O}~;~\vect{\imath},~\vect{\jmath},~\vect{k}\right)$}
\def\Ouv{$\left(\text{O}~;~\vect{u},~\vect{v}\right)$}

\hypersetup{breaklinks=true, colorlinks = true, linkcolor = OliveGreen, urlcolor = OliveGreen, citecolor = OliveGreen, pdfauthor={Didier BONNEL - https://www.maths-cours.fr} } % supprime les bordures autour des liens

\renewcommand{\arg}[0]{\text{arg}}

\everymath{\displaystyle}

%================================================================================================================================
%
% Macros - Commandes
%
%================================================================================================================================

\newcommand\meta[2]{    			% Utilisé pour créer le post HTML.
	\def\titre{titre}
	\def\url{url}
	\def\arg{#1}
	\ifx\titre\arg
		\newcommand\maintitle{#2}
		\fancyhead[L]{#2}
		{\Large\sffamily \MakeUppercase{#2}}
		\vspace{1mm}\textcolor{mcvert}{\hrule}
	\fi 
	\ifx\url\arg
		\fancyfoot[L]{\href{https://www.maths-cours.fr#2}{\black \footnotesize{https://www.maths-cours.fr#2}}}
	\fi 
}


\newcommand\TitreC[1]{    		% Titre centré
     \needspace{3\baselineskip}
     \begin{center}\textbf{#1}\end{center}
}

\newcommand\newpar{    		% paragraphe
     \par
}

\newcommand\nosp {    		% commande vide (pas d'espace)
}
\newcommand{\id}[1]{} %ignore

\newcommand\boite[2]{				% Boite simple sans titre
	\vspace{5mm}
	\setlength{\fboxrule}{0.2mm}
	\setlength{\fboxsep}{5mm}	
	\fcolorbox{#1}{#1!3}{\makebox[\linewidth-2\fboxrule-2\fboxsep]{
  		\begin{minipage}[t]{\linewidth-2\fboxrule-4\fboxsep}\setlength{\parskip}{3mm}
  			 #2
  		\end{minipage}
	}}
	\vspace{5mm}
}

\newcommand\CBox[4]{				% Boites
	\vspace{5mm}
	\setlength{\fboxrule}{0.2mm}
	\setlength{\fboxsep}{5mm}
	
	\fcolorbox{#1}{#1!3}{\makebox[\linewidth-2\fboxrule-2\fboxsep]{
		\begin{minipage}[t]{1cm}\setlength{\parskip}{3mm}
	  		\textcolor{#1}{\LARGE{#2}}    
 	 	\end{minipage}  
  		\begin{minipage}[t]{\linewidth-2\fboxrule-4\fboxsep}\setlength{\parskip}{3mm}
			\raisebox{1.2mm}{\normalsize\sffamily{\textcolor{#1}{#3}}}						
  			 #4
  		\end{minipage}
	}}
	\vspace{5mm}
}

\newcommand\cadre[3]{				% Boites convertible html
	\par
	\vspace{2mm}
	\setlength{\fboxrule}{0.1mm}
	\setlength{\fboxsep}{5mm}
	\fcolorbox{#1}{white}{\makebox[\linewidth-2\fboxrule-2\fboxsep]{
  		\begin{minipage}[t]{\linewidth-2\fboxrule-4\fboxsep}\setlength{\parskip}{3mm}
			\raisebox{-2.5mm}{\sffamily \small{\textcolor{#1}{\MakeUppercase{#2}}}}		
			\par		
  			 #3
 	 		\end{minipage}
	}}
		\vspace{2mm}
	\par
}

\newcommand\bloc[3]{				% Boites convertible html sans bordure
     \needspace{2\baselineskip}
     {\sffamily \small{\textcolor{#1}{\MakeUppercase{#2}}}}    
		\par		
  			 #3
		\par
}

\newcommand\CHelp[1]{
     \CBox{Plum}{\faInfoCircle}{À RETENIR}{#1}
}

\newcommand\CUp[1]{
     \CBox{NavyBlue}{\faThumbsOUp}{EN PRATIQUE}{#1}
}

\newcommand\CInfo[1]{
     \CBox{Sepia}{\faArrowCircleRight}{REMARQUE}{#1}
}

\newcommand\CRedac[1]{
     \CBox{PineGreen}{\faEdit}{BIEN R\'EDIGER}{#1}
}

\newcommand\CError[1]{
     \CBox{Red}{\faExclamationTriangle}{ATTENTION}{#1}
}

\newcommand\TitreExo[2]{
\needspace{4\baselineskip}
 {\sffamily\large EXERCICE #1\ (\emph{#2 points})}
\vspace{5mm}
}

\newcommand\img[2]{
          \includegraphics[width=#2\paperwidth]{\imgdir#1}
}

\newcommand\imgsvg[2]{
       \begin{center}   \includegraphics[width=#2\paperwidth]{\imgsvgdir#1} \end{center}
}


\newcommand\Lien[2]{
     \href{#1}{#2 \tiny \faExternalLink}
}
\newcommand\mcLien[2]{
     \href{https~://www.maths-cours.fr/#1}{#2 \tiny \faExternalLink}
}

\newcommand{\euro}{\eurologo{}}

%================================================================================================================================
%
% Macros - Environement
%
%================================================================================================================================

\newenvironment{tex}{ %
}
{%
}

\newenvironment{indente}{ %
	\setlength\parindent{10mm}
}

{
	\setlength\parindent{0mm}
}

\newenvironment{corrige}{%
     \needspace{3\baselineskip}
     \medskip
     \textbf{\textsc{Corrigé}}
     \medskip
}
{
}

\newenvironment{extern}{%
     \begin{center}
     }
     {
     \end{center}
}

\NewEnviron{code}{%
	\par
     \boite{gray}{\texttt{%
     \BODY
     }}
     \par
}

\newenvironment{vbloc}{% boite sans cadre empeche saut de page
     \begin{minipage}[t]{\linewidth}
     }
     {
     \end{minipage}
}
\NewEnviron{h2}{%
    \needspace{3\baselineskip}
    \vspace{0.6cm}
	\noindent \MakeUppercase{\sffamily \large \BODY}
	\vspace{1mm}\textcolor{mcgris}{\hrule}\vspace{0.4cm}
	\par
}{}

\NewEnviron{h3}{%
    \needspace{3\baselineskip}
	\vspace{5mm}
	\textsc{\BODY}
	\par
}

\NewEnviron{margeneg}{ %
\begin{addmargin}[-1cm]{0cm}
\BODY
\end{addmargin}
}

\NewEnviron{html}{%
}

\begin{document}
\meta{url}{/cours/les-statistiques/}
\meta{pid}{284}
\meta{titre}{Les statistiques en Première}
\meta{type}{cours}
Dans tout ce chapitre, on considère une série statistique représentée par le tableau :
\begin{center}
     \begin{tabularx}{0.7\linewidth}{|*{6}{>{\centering \arraybackslash }X|}}%class="compact" width="500"
          \hline
          \textbf{Valeurs} & $x_{1}$ & $x_{2}$ & ... & $x_{p}$ & \textbf{Total}
          \\ \hline
          \textbf{Effectifs} & $n_{1}$ & $n_{2}$ & ... & $n_{p}$ & $N$
          \\ \hline
     \end{tabularx}
\end{center}
\begin{h2}1. Paramètres de position\end{h2}
\cadre{bleu}{Définition}{% id="d10"
     La \textbf{moyenne} d'une série statistique est le nombre :
     \begin{center}
          $\overline x=\frac{n_{1}x_{1}+n_{2}x_{2}+. . .+n_{p}x_{p}}{N}$\nosp$=\frac{1}{N}\sum_{k=1}^{p}n_{k}x_{k}$
\end{center}}
\bloc{orange}{Exemple}{% id="e10"
     Les âges des élèves d'un lycée sont donnés par le tableau :
     \begin{center}
          \begin{tabularx}{0.95\linewidth}{|*{9}{>{\centering \arraybackslash }X|}}%class="compact" width="700"
               \hline
               \textbf{Ages} & 14 & 15 & 16 & 17 & 18 & 19 & 20 & \textbf{Total}
               \\ \hline
               \textbf{Effectifs} & 2 & 52 & 78 & 75 & 81 & 25 & 2 & 315
               \\ \hline
          \end{tabularx}
     \end{center}
     La moyenne des âges vaut:
     \par
$\overline x=\frac{1}{315}\left(2\times 14+52\times 15\right.$\nosp$\left.+78\times 16+75\times 17+81\times 18+25\times 19+2\times 20\right)$
\par
$\overline x=\frac{5304}{315} \approx  16,84 $ à $ 10^{-2} $ près.
}
\cadre{bleu}{Définition}{% id="d20"
     La \textbf{médiane} d'une série statistique est la valeur du caractère qui partage la population en deux classes de même effectif.
}
\bloc{cyan}{Remarque}{% id="r20"
     En pratique pour trouver la médiane d'une série statistique d'effectif global $N$ :
     \begin{itemize}
          \item On ordonne les valeurs du caractère dans l'ordre croissant.
          \item Si $N$ est pair, la médiane sera la moyenne des valeurs du terme de rang $\frac{N}{2}$ et du terme de rang $\frac{N}{2}+1$.
          \item Si $N$ est impair, la médiane sera la valeur du terme de rang $\frac{N+1}{2}$.
          \item Lorsque l'effectif global est élevé, il est souvent utile de calculer les effectifs cumulés pour trouver cette valeur.
     \end{itemize}
}
\bloc{orange}{Exemple}{% id="e20"
     On lance 10 fois un dé à six faces. Les résultats obtenus sont : 1~;~5~;~6~;~6~;~3~;~2~;~3~;~1~;~4~;~1
     \par
     On trie ces valeurs par ordre croissant : 1~;~1~;~1~;~2~;~3~;~3~;~4~;~5~;~6~;~6
     \par
     N=10 étant pair on effectue la moyenne du cinquième et du sixième terme (3 et 3) et on obtient donc 3.
}
\bloc{cyan}{Remarque}{% id="r21"
     Voir la fiche de \mcLien{/cours/seconde/statistiques-organisation-representation-donnees}{Statistiques en seconde} pour un exemple plus détaillé.
}
\begin{h2}2. Paramètres de dispersion\end{h2}
\cadre{bleu}{Définitions}{% id="d30"
     La \textbf{variance} d'une série statistique est le nombre :
     \begin{center}$V=\dfrac{1}{N}$\nosp$\left(n_{1}\left(x_{1}-\overline x\right)^{2}+n_{2}\left(x_{2}-\overline x\right)^{2}+. . .\right.$\nosp$\left.+n_{p}\left(x_{p}-\overline x\right)^{2}\right)$\\
$\phantom{V}=\frac{1}{N}\sum_{k=1}^{p}n_{k}\left(x_{k}-\overline x\right)^{2}$\end{center}
L'\textbf{écart-type} est la racine carrée de la variance :
\begin{center}$\sigma =\sqrt{V}$\end{center}
}
\cadre{vert}{Propriété}{% id="p40"
     La \textbf{variance} d'une série statistique est égale à :
     \begin{center}$V=\dfrac{n_{1}x_{1}^{2}+n_{2}x_{2}^{2}+. . .+n_{p}x_{p}^{2}}{N}-\overline x^{2}$\nosp$=\overline{x^{2}}-\overline x^{2}$\end{center}
}
\cadre{bleu}{Définitions}{% id="d50"
     \begin{itemize}
          \item Le \textbf{premier quartile} Q1 d'une série statistique est la plus petite valeur des termes de la série pour laquelle au moins un quart des données sont inférieures ou égales à Q1.
          \item Le \textbf{troisième quartile} Q3  d'une série statistique est la plus petite valeur des termes de la série pour laquelle au moins trois quarts des données sont inférieures ou égales à Q3.
          \item Le \textbf{premier décile} D1 d'une série statistique est la plus petite valeur  des termes de la série pour laquelle au moins 10\% des données sont inférieures ou égales à D1.
          \item Le \textbf{neuvième décile} D9 d'une série statistique est la plus petite valeur des termes de la série pour laquelle au moins 90\% des données sont inférieures ou égales à D9
     \end{itemize}
}
\cadre{bleu}{Définition}{% id="d60"
     L'\textbf{écart interquartile} est la différence entre le troisième et le premier quartile $Q_{3}-Q_{1}$.
}
\bloc{cyan}{Remarque}{% id="r60"
     L'écart interquartile mesure la dispersion autour de la médiane.
}
\begin{h2}3. Diagramme en boîte\end{h2}
\begin{center}
     \img{diag-boite-1}{0.4}%width="400" alt="boite à moustaches"
\end{center}
On peut résumer un certain nombre d'informations relatives à une série statistique grâce à un \textbf{diagramme en boîte} (aussi appelé \textit{boîte à moustache}) qui fait apparaître (voir figure ci-dessus) :
\begin{itemize}
     \item les valeurs minimum et maximum
     \item le premier et le troisième quartile (Q1 et Q3)
     \item la médiane
\end{itemize}
\bloc{orange}{Exemple}{% id="e70"
     \begin{center}
          \img{diag-boite-2}{0.6}%width="480" alt="exemple diagramme boite à moustaches"
     \end{center}
     Le figure ci-dessus représente une série statistique de valeurs extrêmes 3 et 20, de premier quartile 6, de troisième quartile 14 et de médiane 9,5.
}
\bloc{cyan}{Remarque}{% id="r70"
     Parfois, notamment lorsqu'on étudie des séries dont certaines valeurs peuvent être erronées, on remplace les valeurs minimum et maximum par les premier et neuvième déciles afin d'éliminer les valeurs aberrantes.
}

\end{document}