\documentclass[a4paper]{article}

%================================================================================================================================
%
% Packages
%
%================================================================================================================================

\usepackage[T1]{fontenc} 	% pour caractères accentués
\usepackage[utf8]{inputenc}  % encodage utf8
\usepackage[french]{babel}	% langue : français
\usepackage{fourier}			% caractères plus lisibles
\usepackage[dvipsnames]{xcolor} % couleurs
\usepackage{fancyhdr}		% réglage header footer
\usepackage{needspace}		% empêcher sauts de page mal placés
\usepackage{graphicx}		% pour inclure des graphiques
\usepackage{enumitem,cprotect}		% personnalise les listes d'items (nécessaire pour ol, al ...)
\usepackage{hyperref}		% Liens hypertexte
\usepackage{pstricks,pst-all,pst-node,pstricks-add,pst-math,pst-plot,pst-tree,pst-eucl} % pstricks
\usepackage[a4paper,includeheadfoot,top=2cm,left=3cm, bottom=2cm,right=3cm]{geometry} % marges etc.
\usepackage{comment}			% commentaires multilignes
\usepackage{amsmath,environ} % maths (matrices, etc.)
\usepackage{amssymb,makeidx}
\usepackage{bm}				% bold maths
\usepackage{tabularx}		% tableaux
\usepackage{colortbl}		% tableaux en couleur
\usepackage{fontawesome}		% Fontawesome
\usepackage{environ}			% environment with command
\usepackage{fp}				% calculs pour ps-tricks
\usepackage{multido}			% pour ps tricks
\usepackage[np]{numprint}	% formattage nombre
\usepackage{tikz,tkz-tab} 			% package principal TikZ
\usepackage{pgfplots}   % axes
\usepackage{mathrsfs}    % cursives
\usepackage{calc}			% calcul taille boites
\usepackage[scaled=0.875]{helvet} % font sans serif
\usepackage{svg} % svg
\usepackage{scrextend} % local margin
\usepackage{scratch} %scratch
\usepackage{multicol} % colonnes
%\usepackage{infix-RPN,pst-func} % formule en notation polanaise inversée
\usepackage{listings}

%================================================================================================================================
%
% Réglages de base
%
%================================================================================================================================

\lstset{
language=Python,   % R code
literate=
{á}{{\'a}}1
{à}{{\`a}}1
{ã}{{\~a}}1
{é}{{\'e}}1
{è}{{\`e}}1
{ê}{{\^e}}1
{í}{{\'i}}1
{ó}{{\'o}}1
{õ}{{\~o}}1
{ú}{{\'u}}1
{ü}{{\"u}}1
{ç}{{\c{c}}}1
{~}{{ }}1
}


\definecolor{codegreen}{rgb}{0,0.6,0}
\definecolor{codegray}{rgb}{0.5,0.5,0.5}
\definecolor{codepurple}{rgb}{0.58,0,0.82}
\definecolor{backcolour}{rgb}{0.95,0.95,0.92}

\lstdefinestyle{mystyle}{
    backgroundcolor=\color{backcolour},   
    commentstyle=\color{codegreen},
    keywordstyle=\color{magenta},
    numberstyle=\tiny\color{codegray},
    stringstyle=\color{codepurple},
    basicstyle=\ttfamily\footnotesize,
    breakatwhitespace=false,         
    breaklines=true,                 
    captionpos=b,                    
    keepspaces=true,                 
    numbers=left,                    
xleftmargin=2em,
framexleftmargin=2em,            
    showspaces=false,                
    showstringspaces=false,
    showtabs=false,                  
    tabsize=2,
    upquote=true
}

\lstset{style=mystyle}


\lstset{style=mystyle}
\newcommand{\imgdir}{C:/laragon/www/newmc/assets/imgsvg/}
\newcommand{\imgsvgdir}{C:/laragon/www/newmc/assets/imgsvg/}

\definecolor{mcgris}{RGB}{220, 220, 220}% ancien~; pour compatibilité
\definecolor{mcbleu}{RGB}{52, 152, 219}
\definecolor{mcvert}{RGB}{125, 194, 70}
\definecolor{mcmauve}{RGB}{154, 0, 215}
\definecolor{mcorange}{RGB}{255, 96, 0}
\definecolor{mcturquoise}{RGB}{0, 153, 153}
\definecolor{mcrouge}{RGB}{255, 0, 0}
\definecolor{mclightvert}{RGB}{205, 234, 190}

\definecolor{gris}{RGB}{220, 220, 220}
\definecolor{bleu}{RGB}{52, 152, 219}
\definecolor{vert}{RGB}{125, 194, 70}
\definecolor{mauve}{RGB}{154, 0, 215}
\definecolor{orange}{RGB}{255, 96, 0}
\definecolor{turquoise}{RGB}{0, 153, 153}
\definecolor{rouge}{RGB}{255, 0, 0}
\definecolor{lightvert}{RGB}{205, 234, 190}
\setitemize[0]{label=\color{lightvert}  $\bullet$}

\pagestyle{fancy}
\renewcommand{\headrulewidth}{0.2pt}
\fancyhead[L]{maths-cours.fr}
\fancyhead[R]{\thepage}
\renewcommand{\footrulewidth}{0.2pt}
\fancyfoot[C]{}

\newcolumntype{C}{>{\centering\arraybackslash}X}
\newcolumntype{s}{>{\hsize=.35\hsize\arraybackslash}X}

\setlength{\parindent}{0pt}		 
\setlength{\parskip}{3mm}
\setlength{\headheight}{1cm}

\def\ebook{ebook}
\def\book{book}
\def\web{web}
\def\type{web}

\newcommand{\vect}[1]{\overrightarrow{\,\mathstrut#1\,}}

\def\Oij{$\left(\text{O}~;~\vect{\imath},~\vect{\jmath}\right)$}
\def\Oijk{$\left(\text{O}~;~\vect{\imath},~\vect{\jmath},~\vect{k}\right)$}
\def\Ouv{$\left(\text{O}~;~\vect{u},~\vect{v}\right)$}

\hypersetup{breaklinks=true, colorlinks = true, linkcolor = OliveGreen, urlcolor = OliveGreen, citecolor = OliveGreen, pdfauthor={Didier BONNEL - https://www.maths-cours.fr} } % supprime les bordures autour des liens

\renewcommand{\arg}[0]{\text{arg}}

\everymath{\displaystyle}

%================================================================================================================================
%
% Macros - Commandes
%
%================================================================================================================================

\newcommand\meta[2]{    			% Utilisé pour créer le post HTML.
	\def\titre{titre}
	\def\url{url}
	\def\arg{#1}
	\ifx\titre\arg
		\newcommand\maintitle{#2}
		\fancyhead[L]{#2}
		{\Large\sffamily \MakeUppercase{#2}}
		\vspace{1mm}\textcolor{mcvert}{\hrule}
	\fi 
	\ifx\url\arg
		\fancyfoot[L]{\href{https://www.maths-cours.fr#2}{\black \footnotesize{https://www.maths-cours.fr#2}}}
	\fi 
}


\newcommand\TitreC[1]{    		% Titre centré
     \needspace{3\baselineskip}
     \begin{center}\textbf{#1}\end{center}
}

\newcommand\newpar{    		% paragraphe
     \par
}

\newcommand\nosp {    		% commande vide (pas d'espace)
}
\newcommand{\id}[1]{} %ignore

\newcommand\boite[2]{				% Boite simple sans titre
	\vspace{5mm}
	\setlength{\fboxrule}{0.2mm}
	\setlength{\fboxsep}{5mm}	
	\fcolorbox{#1}{#1!3}{\makebox[\linewidth-2\fboxrule-2\fboxsep]{
  		\begin{minipage}[t]{\linewidth-2\fboxrule-4\fboxsep}\setlength{\parskip}{3mm}
  			 #2
  		\end{minipage}
	}}
	\vspace{5mm}
}

\newcommand\CBox[4]{				% Boites
	\vspace{5mm}
	\setlength{\fboxrule}{0.2mm}
	\setlength{\fboxsep}{5mm}
	
	\fcolorbox{#1}{#1!3}{\makebox[\linewidth-2\fboxrule-2\fboxsep]{
		\begin{minipage}[t]{1cm}\setlength{\parskip}{3mm}
	  		\textcolor{#1}{\LARGE{#2}}    
 	 	\end{minipage}  
  		\begin{minipage}[t]{\linewidth-2\fboxrule-4\fboxsep}\setlength{\parskip}{3mm}
			\raisebox{1.2mm}{\normalsize\sffamily{\textcolor{#1}{#3}}}						
  			 #4
  		\end{minipage}
	}}
	\vspace{5mm}
}

\newcommand\cadre[3]{				% Boites convertible html
	\par
	\vspace{2mm}
	\setlength{\fboxrule}{0.1mm}
	\setlength{\fboxsep}{5mm}
	\fcolorbox{#1}{white}{\makebox[\linewidth-2\fboxrule-2\fboxsep]{
  		\begin{minipage}[t]{\linewidth-2\fboxrule-4\fboxsep}\setlength{\parskip}{3mm}
			\raisebox{-2.5mm}{\sffamily \small{\textcolor{#1}{\MakeUppercase{#2}}}}		
			\par		
  			 #3
 	 		\end{minipage}
	}}
		\vspace{2mm}
	\par
}

\newcommand\bloc[3]{				% Boites convertible html sans bordure
     \needspace{2\baselineskip}
     {\sffamily \small{\textcolor{#1}{\MakeUppercase{#2}}}}    
		\par		
  			 #3
		\par
}

\newcommand\CHelp[1]{
     \CBox{Plum}{\faInfoCircle}{À RETENIR}{#1}
}

\newcommand\CUp[1]{
     \CBox{NavyBlue}{\faThumbsOUp}{EN PRATIQUE}{#1}
}

\newcommand\CInfo[1]{
     \CBox{Sepia}{\faArrowCircleRight}{REMARQUE}{#1}
}

\newcommand\CRedac[1]{
     \CBox{PineGreen}{\faEdit}{BIEN R\'EDIGER}{#1}
}

\newcommand\CError[1]{
     \CBox{Red}{\faExclamationTriangle}{ATTENTION}{#1}
}

\newcommand\TitreExo[2]{
\needspace{4\baselineskip}
 {\sffamily\large EXERCICE #1\ (\emph{#2 points})}
\vspace{5mm}
}

\newcommand\img[2]{
          \includegraphics[width=#2\paperwidth]{\imgdir#1}
}

\newcommand\imgsvg[2]{
       \begin{center}   \includegraphics[width=#2\paperwidth]{\imgsvgdir#1} \end{center}
}


\newcommand\Lien[2]{
     \href{#1}{#2 \tiny \faExternalLink}
}
\newcommand\mcLien[2]{
     \href{https~://www.maths-cours.fr/#1}{#2 \tiny \faExternalLink}
}

\newcommand{\euro}{\eurologo{}}

%================================================================================================================================
%
% Macros - Environement
%
%================================================================================================================================

\newenvironment{tex}{ %
}
{%
}

\newenvironment{indente}{ %
	\setlength\parindent{10mm}
}

{
	\setlength\parindent{0mm}
}

\newenvironment{corrige}{%
     \needspace{3\baselineskip}
     \medskip
     \textbf{\textsc{Corrigé}}
     \medskip
}
{
}

\newenvironment{extern}{%
     \begin{center}
     }
     {
     \end{center}
}

\NewEnviron{code}{%
	\par
     \boite{gray}{\texttt{%
     \BODY
     }}
     \par
}

\newenvironment{vbloc}{% boite sans cadre empeche saut de page
     \begin{minipage}[t]{\linewidth}
     }
     {
     \end{minipage}
}
\NewEnviron{h2}{%
    \needspace{3\baselineskip}
    \vspace{0.6cm}
	\noindent \MakeUppercase{\sffamily \large \BODY}
	\vspace{1mm}\textcolor{mcgris}{\hrule}\vspace{0.4cm}
	\par
}{}

\NewEnviron{h3}{%
    \needspace{3\baselineskip}
	\vspace{5mm}
	\textsc{\BODY}
	\par
}

\NewEnviron{margeneg}{ %
\begin{addmargin}[-1cm]{0cm}
\BODY
\end{addmargin}
}

\NewEnviron{html}{%
}

\begin{document}
\meta{url}{/cours/algorithmes-tests-boucles/}
\meta{pid}{237}
\meta{titre}{Algorithmes : Tests et boucles}
\meta{type}{cours}
Les algorithmes que nous avons utilisés dans le chapitre précédent exécutent toujours la même tâche ce qui limite leur intérêt. Les tests et les boucles vont enrichir nos algorithmes leur permettant d'agir différemment en fonction des données entrées par l'utilisateur.
\begin{h2}1. Conditions\end{h2}
Une condition est une expression qui peut prendre l'une des deux valeurs suivantes \textbf{vrai} ou \textbf{faux}. On dit également que c'est une valeur de type \textbf{"logique"} ou \textbf{"booléen"}.
\par
Les principaux opérateurs de comparaison que vous rencontrerez sont les suivants :
\begin{itemize}
     \item égal à ( = en pseudo code)
     \item différent de ( != en pseudo code)
     \item strictement supérieur (\textbf{  > } en pseudo code)
     \item strictement inférieur ( \textbf{ < } en pseudo code)
     \item supérieur ou égal ( \textbf{ > =} en pseudo code)
     \item inférieur ou égal (\textbf{  < =} en pseudo code)
\end{itemize}
Ces comparaisons n'ont un sens que si les variables que l'on compare sont de même type.
\bigbreak
\begin{h3}Conditions composées\end{h3}
On peut écrire des conditions plus complexes en reliant des comparaisons à l'aide des opérateurs logiques \textbf{ET},\textbf{ OU} et \textbf{NON}.
\begin{itemize}
     \item \textbf{Condition 1 ET condition 2} sera vraie si les deux conditions sont \textbf{toutes les deux vraies}.
     \par
     Par exemple, la condition : "\textbf{âge supérieur à 5 ET âge inférieur à 10}" sera vraie si la variable âge est \textbf{strictement comprise entre 5 et 10}.
     \item \textbf{Condition 1 OU condition 2} sera vraie si \textbf{l'une au moins} des deux conditions est vraie.
     \par
     Par exemple, la condition "\textbf{prénom=Jean OU nom=Dupont}" sera vraie pour :
     \begin{itemize}
          \item Jean Dupont (conditions 1 et 2 vraies)
          \item Jean Durand (condition 1 vraie)
          \item Pierre Dupont (condition 2 vraie)
     \end{itemize}
     mais fausse pour
     \begin{itemize}
          \item Pierre Durand (conditions 1 et 2 fausses)
     \end{itemize}
     \item \textbf{NON (condition 1)} sera vraie si et seulement si \textbf{condition 1 est fausse}.
     \par
     Par exemple : "\textbf{NON (x < 3)}" sera vraie si \textbf{x  > = 3}
\end{itemize}
\begin{h2}2. Tests\end{h2}
\cadre{bleu}{Définition}{%
     Un \textbf{test} est une instruction qui permet d'effectuer un traitement différent selon qu'une condition est vérifiée ou non.
}
\begin{h3}Première forme\end{h3}
La première forme possible est la suivante :
\begin{code}
si \textit{condition} alors 
     \textit{instructions}
fin si
\end{code}
Les instructions ne seront exécutées que \textbf{si la condition est vérifiée}. Par exemple :
\begin{code}
\textbf{variable}
     x : entier
\textbf{début algorithme}
     lire x
     si x > 10 alors
        x prend la valeur 10
     fin si
     afficher x
\textbf{fin algorithme}
\end{code}
Si l'utilisateur entre un entier supérieur à 10 l'algorithme affichera 10 sinon il affichera le nombre saisi par l'utilisateur.
\begin{h3}Seconde forme\end{h3}
La seconde forme est légèrement plus complexe :
\begin{code}
si \textit{condition} alors
     \textit{instructions 1}
sinon
     \textit{instructions 2}
fin si
\end{code}
\textbf{Si la condition est vraie}, l'algorithme effectuera les "instructions 1" puis passera aux instructions situées après le "fin si".
\textbf{Si la condition est fausse}, l'algorithme effectuera les "instructions 2" puis passera aux instructions situées après le "fin si".
\bloc{orange}{Exemple}{%
     \begin{code}
     \textbf{variables}
          âge, prix : entier
          \textbf{début algorithme}
          afficher "entrez votre âge :"
          lire âge
          si âge < 16 alors
             afficher "vous bénéficiez du tarif réduit"
             prix prend la valeur 10
          sinon
             afficher "vous ne bénéficiez pas du tarif réduit"
             prix prend la valeur 15
          fin si
          afficher "vous devez payer", prix, "euros"
          \textbf{fin algorithme}
     \end{code}
     Si vous entrez \textbf{15} comme âge, vous obtiendrez le résultat suivant :
     \par
     \textit{Vous bénéficiez du tarif réduit
     Vous devez payer 10 euros}
     \par
     Si vous entrez \textbf{16} comme âge, vous obtiendrez :
     \par
     \textit{Vous ne bénéficiez pas du tarif réduit
     Vous devez payer 15 euros}
     \par
}
\begin{h2}3. Boucle\end{h2}
\cadre{bleu}{Définition}{%
     Une \textbf{boucle} permet de répéter un traitement un certain nombre de fois.
}
\begin{h3}Première forme\end{h3}
\textbf{ Boucle "Tant que" }
\begin{code}
tant que \textit{condition}
     \textit{instructions}
fin tant que
\end{code}
L'algorithme ci-dessus effectuera les instructions tant que la condition sera vraie. Dès que la condition devient fausse, on se branchera sur l'instruction suivant le fin tant que.
\bloc{orange}{Exemple}{%
     \begin{code}
     \textbf{variables}
          nombre, somme: nombres
          continuer: texte
          \textbf{début algorithme}
          continuer prend la valeur "oui" // initialisation
          afficher 'entrez un nombre :'
          lire nombre
          somme prend la valeur nombre
          tant que continuer="oui"
             afficher "entrez le nombre suivant"
             lire nombre
             somme prend la valeur somme+nombre
             afficher "voulez-vous continuer (oui/non)"
             lire continuer
          fin tant que
          afficher "la somme des nombres entrés est" somme
          \textbf{fin algorithme}
     \end{code}
     L'algorithme précédent demande à l'utilisateur d'\textbf{entrer un premier nombre}.
     \par
     Puis il lui demande \textbf{s'il veut entrer un autre nombre}.
     \par
     Tant que l'utilisateur répond \textit{"oui"}, l'algorithme lui \textbf{demande un nouveau nombre} qu'il \textbf{additionne} au contenu de la variable "somme".
     \par
     Dès que l'utilisateur répond autre chose que \textit{"oui"}, l'algorithme \textbf{sort de la boucle}, \textbf{affiche le total} et se \textbf{termine}.
}
\begin{h3}Deuxième forme\end{h3}
\textbf{ Boucle "Pour" }
\bloc{orange}{Exemple}{%
     \begin{code}
     \textbf{variables}
          i : nombre
          ...
          \textbf{début algorithme}
          ...
          pour i variant de 1 à 10
          \textit{instructions}
          fin pour
          ...
          \textbf{fin algorithme}
     \end{code}
     L'algorithme ci-dessus va exécuter \textbf{dix fois} les \textit{instructions} situées dans la boucle.
     \par
     Plus précisément :
     \begin{itemize}
          \item \textbf{La première fois} que l'algorithme va rencontrer l'instruction \textit{"pour i variant de 1 à 10"}, il va affecter la valeur 1 à i; comme i est strictement inférieur à 10, il passe ensuite aux \textit{instructions} situées \textbf{à l'intérieur} de la boucle
          \item après les avoir exécutées, la ligne \textit{"fin pour"} va faire boucler l'algorithme et le faire revenir à l'instruction \textit{"pour i variant de 1 à 10"}
          \item \textbf{La seconde fois (et les fois suivantes...)} que l'algorithme va exécuter l'instruction \textit{"pour i variant de 1 à 10"}, il va :
          \begin{itemize}
               \item ajouter 1 à i (on dit \textbf{incrémenter i})
               \item si i est inférieur ou égal à 10, il passe aux \textit{instructions} situées \textbf{à l'intérieur} de la boucle
               \item si i est supérieur à 10, il passe aux instructions situées \textbf{après} la ligne \textit{"fin tant que"}
          \end{itemize}
     \end{itemize}
}
\bloc{cyan}{Remarque}{%
     Si l'on souhaite incrémenter l'indice avec une valeur différente de 1 on utilise l'instruction :
     \begin{code}
     pour i variant de ... à ... \textbf{avec un pas de} ...
     \end{code}
     Par exemple :
     \begin{code}
     pour i variant de 2 à 8 \textbf{avec un pas de 2}
     \end{code}
     \textit{i} va prendre successivement les valeurs : 2; 4 ; 6; 8 (et il quittera la boucle lorsqu'il vaudra 10)
}
\bloc{orange}{Exemple}{%
     L'algorithme ci-dessous affiche les carrés des 21 premiers nombres entiers naturels (de 0 à 20)
     \begin{code}
     \textbf{variables}
          n : nombre
          c : nombre
          \textbf{début algorithme}
          pour n variant de 0 à 20
          c prend la valeur n*n
          afficher "Le carré de ", n, " est ", c
          fin pour
          \textbf{fin algorithme}
     \end{code}
}
\bloc{cyan}{Remarque}{%
     On utilise généralement une instruction \textit{"pour"} lorsqu'on connaît le nombre d'itérations à réaliser dès le début de la boucle et une instruction \textit{"Tant que"} lorsque ce nombre est inconnu ou difficile à déterminer.
}

\end{document}