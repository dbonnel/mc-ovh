\documentclass[a4paper]{article}

%================================================================================================================================
%
% Packages
%
%================================================================================================================================

\usepackage[T1]{fontenc} 	% pour caractères accentués
\usepackage[utf8]{inputenc}  % encodage utf8
\usepackage[french]{babel}	% langue : français
\usepackage{fourier}			% caractères plus lisibles
\usepackage[dvipsnames]{xcolor} % couleurs
\usepackage{fancyhdr}		% réglage header footer
\usepackage{needspace}		% empêcher sauts de page mal placés
\usepackage{graphicx}		% pour inclure des graphiques
\usepackage{enumitem,cprotect}		% personnalise les listes d'items (nécessaire pour ol, al ...)
\usepackage{hyperref}		% Liens hypertexte
\usepackage{pstricks,pst-all,pst-node,pstricks-add,pst-math,pst-plot,pst-tree,pst-eucl} % pstricks
\usepackage[a4paper,includeheadfoot,top=2cm,left=3cm, bottom=2cm,right=3cm]{geometry} % marges etc.
\usepackage{comment}			% commentaires multilignes
\usepackage{amsmath,environ} % maths (matrices, etc.)
\usepackage{amssymb,makeidx}
\usepackage{bm}				% bold maths
\usepackage{tabularx}		% tableaux
\usepackage{colortbl}		% tableaux en couleur
\usepackage{fontawesome}		% Fontawesome
\usepackage{environ}			% environment with command
\usepackage{fp}				% calculs pour ps-tricks
\usepackage{multido}			% pour ps tricks
\usepackage[np]{numprint}	% formattage nombre
\usepackage{tikz,tkz-tab} 			% package principal TikZ
\usepackage{pgfplots}   % axes
\usepackage{mathrsfs}    % cursives
\usepackage{calc}			% calcul taille boites
\usepackage[scaled=0.875]{helvet} % font sans serif
\usepackage{svg} % svg
\usepackage{scrextend} % local margin
\usepackage{scratch} %scratch
\usepackage{multicol} % colonnes
%\usepackage{infix-RPN,pst-func} % formule en notation polanaise inversée
\usepackage{listings}

%================================================================================================================================
%
% Réglages de base
%
%================================================================================================================================

\lstset{
language=Python,   % R code
literate=
{á}{{\'a}}1
{à}{{\`a}}1
{ã}{{\~a}}1
{é}{{\'e}}1
{è}{{\`e}}1
{ê}{{\^e}}1
{í}{{\'i}}1
{ó}{{\'o}}1
{õ}{{\~o}}1
{ú}{{\'u}}1
{ü}{{\"u}}1
{ç}{{\c{c}}}1
{~}{{ }}1
}


\definecolor{codegreen}{rgb}{0,0.6,0}
\definecolor{codegray}{rgb}{0.5,0.5,0.5}
\definecolor{codepurple}{rgb}{0.58,0,0.82}
\definecolor{backcolour}{rgb}{0.95,0.95,0.92}

\lstdefinestyle{mystyle}{
    backgroundcolor=\color{backcolour},   
    commentstyle=\color{codegreen},
    keywordstyle=\color{magenta},
    numberstyle=\tiny\color{codegray},
    stringstyle=\color{codepurple},
    basicstyle=\ttfamily\footnotesize,
    breakatwhitespace=false,         
    breaklines=true,                 
    captionpos=b,                    
    keepspaces=true,                 
    numbers=left,                    
xleftmargin=2em,
framexleftmargin=2em,            
    showspaces=false,                
    showstringspaces=false,
    showtabs=false,                  
    tabsize=2,
    upquote=true
}

\lstset{style=mystyle}


\lstset{style=mystyle}
\newcommand{\imgdir}{C:/laragon/www/newmc/assets/imgsvg/}
\newcommand{\imgsvgdir}{C:/laragon/www/newmc/assets/imgsvg/}

\definecolor{mcgris}{RGB}{220, 220, 220}% ancien~; pour compatibilité
\definecolor{mcbleu}{RGB}{52, 152, 219}
\definecolor{mcvert}{RGB}{125, 194, 70}
\definecolor{mcmauve}{RGB}{154, 0, 215}
\definecolor{mcorange}{RGB}{255, 96, 0}
\definecolor{mcturquoise}{RGB}{0, 153, 153}
\definecolor{mcrouge}{RGB}{255, 0, 0}
\definecolor{mclightvert}{RGB}{205, 234, 190}

\definecolor{gris}{RGB}{220, 220, 220}
\definecolor{bleu}{RGB}{52, 152, 219}
\definecolor{vert}{RGB}{125, 194, 70}
\definecolor{mauve}{RGB}{154, 0, 215}
\definecolor{orange}{RGB}{255, 96, 0}
\definecolor{turquoise}{RGB}{0, 153, 153}
\definecolor{rouge}{RGB}{255, 0, 0}
\definecolor{lightvert}{RGB}{205, 234, 190}
\setitemize[0]{label=\color{lightvert}  $\bullet$}

\pagestyle{fancy}
\renewcommand{\headrulewidth}{0.2pt}
\fancyhead[L]{maths-cours.fr}
\fancyhead[R]{\thepage}
\renewcommand{\footrulewidth}{0.2pt}
\fancyfoot[C]{}

\newcolumntype{C}{>{\centering\arraybackslash}X}
\newcolumntype{s}{>{\hsize=.35\hsize\arraybackslash}X}

\setlength{\parindent}{0pt}		 
\setlength{\parskip}{3mm}
\setlength{\headheight}{1cm}

\def\ebook{ebook}
\def\book{book}
\def\web{web}
\def\type{web}

\newcommand{\vect}[1]{\overrightarrow{\,\mathstrut#1\,}}

\def\Oij{$\left(\text{O}~;~\vect{\imath},~\vect{\jmath}\right)$}
\def\Oijk{$\left(\text{O}~;~\vect{\imath},~\vect{\jmath},~\vect{k}\right)$}
\def\Ouv{$\left(\text{O}~;~\vect{u},~\vect{v}\right)$}

\hypersetup{breaklinks=true, colorlinks = true, linkcolor = OliveGreen, urlcolor = OliveGreen, citecolor = OliveGreen, pdfauthor={Didier BONNEL - https://www.maths-cours.fr} } % supprime les bordures autour des liens

\renewcommand{\arg}[0]{\text{arg}}

\everymath{\displaystyle}

%================================================================================================================================
%
% Macros - Commandes
%
%================================================================================================================================

\newcommand\meta[2]{    			% Utilisé pour créer le post HTML.
	\def\titre{titre}
	\def\url{url}
	\def\arg{#1}
	\ifx\titre\arg
		\newcommand\maintitle{#2}
		\fancyhead[L]{#2}
		{\Large\sffamily \MakeUppercase{#2}}
		\vspace{1mm}\textcolor{mcvert}{\hrule}
	\fi 
	\ifx\url\arg
		\fancyfoot[L]{\href{https://www.maths-cours.fr#2}{\black \footnotesize{https://www.maths-cours.fr#2}}}
	\fi 
}


\newcommand\TitreC[1]{    		% Titre centré
     \needspace{3\baselineskip}
     \begin{center}\textbf{#1}\end{center}
}

\newcommand\newpar{    		% paragraphe
     \par
}

\newcommand\nosp {    		% commande vide (pas d'espace)
}
\newcommand{\id}[1]{} %ignore

\newcommand\boite[2]{				% Boite simple sans titre
	\vspace{5mm}
	\setlength{\fboxrule}{0.2mm}
	\setlength{\fboxsep}{5mm}	
	\fcolorbox{#1}{#1!3}{\makebox[\linewidth-2\fboxrule-2\fboxsep]{
  		\begin{minipage}[t]{\linewidth-2\fboxrule-4\fboxsep}\setlength{\parskip}{3mm}
  			 #2
  		\end{minipage}
	}}
	\vspace{5mm}
}

\newcommand\CBox[4]{				% Boites
	\vspace{5mm}
	\setlength{\fboxrule}{0.2mm}
	\setlength{\fboxsep}{5mm}
	
	\fcolorbox{#1}{#1!3}{\makebox[\linewidth-2\fboxrule-2\fboxsep]{
		\begin{minipage}[t]{1cm}\setlength{\parskip}{3mm}
	  		\textcolor{#1}{\LARGE{#2}}    
 	 	\end{minipage}  
  		\begin{minipage}[t]{\linewidth-2\fboxrule-4\fboxsep}\setlength{\parskip}{3mm}
			\raisebox{1.2mm}{\normalsize\sffamily{\textcolor{#1}{#3}}}						
  			 #4
  		\end{minipage}
	}}
	\vspace{5mm}
}

\newcommand\cadre[3]{				% Boites convertible html
	\par
	\vspace{2mm}
	\setlength{\fboxrule}{0.1mm}
	\setlength{\fboxsep}{5mm}
	\fcolorbox{#1}{white}{\makebox[\linewidth-2\fboxrule-2\fboxsep]{
  		\begin{minipage}[t]{\linewidth-2\fboxrule-4\fboxsep}\setlength{\parskip}{3mm}
			\raisebox{-2.5mm}{\sffamily \small{\textcolor{#1}{\MakeUppercase{#2}}}}		
			\par		
  			 #3
 	 		\end{minipage}
	}}
		\vspace{2mm}
	\par
}

\newcommand\bloc[3]{				% Boites convertible html sans bordure
     \needspace{2\baselineskip}
     {\sffamily \small{\textcolor{#1}{\MakeUppercase{#2}}}}    
		\par		
  			 #3
		\par
}

\newcommand\CHelp[1]{
     \CBox{Plum}{\faInfoCircle}{À RETENIR}{#1}
}

\newcommand\CUp[1]{
     \CBox{NavyBlue}{\faThumbsOUp}{EN PRATIQUE}{#1}
}

\newcommand\CInfo[1]{
     \CBox{Sepia}{\faArrowCircleRight}{REMARQUE}{#1}
}

\newcommand\CRedac[1]{
     \CBox{PineGreen}{\faEdit}{BIEN R\'EDIGER}{#1}
}

\newcommand\CError[1]{
     \CBox{Red}{\faExclamationTriangle}{ATTENTION}{#1}
}

\newcommand\TitreExo[2]{
\needspace{4\baselineskip}
 {\sffamily\large EXERCICE #1\ (\emph{#2 points})}
\vspace{5mm}
}

\newcommand\img[2]{
          \includegraphics[width=#2\paperwidth]{\imgdir#1}
}

\newcommand\imgsvg[2]{
       \begin{center}   \includegraphics[width=#2\paperwidth]{\imgsvgdir#1} \end{center}
}


\newcommand\Lien[2]{
     \href{#1}{#2 \tiny \faExternalLink}
}
\newcommand\mcLien[2]{
     \href{https~://www.maths-cours.fr/#1}{#2 \tiny \faExternalLink}
}

\newcommand{\euro}{\eurologo{}}

%================================================================================================================================
%
% Macros - Environement
%
%================================================================================================================================

\newenvironment{tex}{ %
}
{%
}

\newenvironment{indente}{ %
	\setlength\parindent{10mm}
}

{
	\setlength\parindent{0mm}
}

\newenvironment{corrige}{%
     \needspace{3\baselineskip}
     \medskip
     \textbf{\textsc{Corrigé}}
     \medskip
}
{
}

\newenvironment{extern}{%
     \begin{center}
     }
     {
     \end{center}
}

\NewEnviron{code}{%
	\par
     \boite{gray}{\texttt{%
     \BODY
     }}
     \par
}

\newenvironment{vbloc}{% boite sans cadre empeche saut de page
     \begin{minipage}[t]{\linewidth}
     }
     {
     \end{minipage}
}
\NewEnviron{h2}{%
    \needspace{3\baselineskip}
    \vspace{0.6cm}
	\noindent \MakeUppercase{\sffamily \large \BODY}
	\vspace{1mm}\textcolor{mcgris}{\hrule}\vspace{0.4cm}
	\par
}{}

\NewEnviron{h3}{%
    \needspace{3\baselineskip}
	\vspace{5mm}
	\textsc{\BODY}
	\par
}

\NewEnviron{margeneg}{ %
\begin{addmargin}[-1cm]{0cm}
\BODY
\end{addmargin}
}

\NewEnviron{html}{%
}

\begin{document}
\meta{url}{/cours/nombres-complexes-geometrie/}
\meta{pid}{19238}
\meta{titre}{Nombres complexes et géométrie}
\meta{type}{cours}
\begin{h2}1. Représentation géométrique d'un nombre complexe
\end{h2}
Le plan $\left(P\right)$ est muni d'un repère orthonormé $\left(O; \vec{u}, \vec{v}\right)$
\cadre{bleu}{Définitions}{% id="d70"
     A tout nombre complexe $z=a+ib$, on associe le point $M$ de coordonnées $\left(a ; b\right)$
     \par
     On dit que $M$ est l'\textbf{image} de $z$ et que $z$ est l'\textbf{affixe} du point $M$.
     \par
     A tout vecteur $\vec{k}$ de coordonnées $\left(a ; b\right)$  on associe le nombre complexe $z=a+ib$.
     \par
     On dit que $z$ est l'\textbf{affixe} du vecteur $\vec{k}$.
}
\begin{center}
     \begin{extern} %width="450" alt="représentation graphique des nombres complexes"
          \newrgbcolor{dblue}{0. 0. 0.7}
          \newrgbcolor{dvert}{0. 0.4 0.}
          \newrgbcolor{dmauve}{0.5 0. 0.5}
          \resizebox{8cm}{!}{%
               % -+-+-+ variables modifiables
               \def\xmin{-1.2}
               \def\xmax{6.5}
               \def\ymin{-0.8}
               \def\ymax{5.5}
               \def\xunit{1.5}  % unités en cm
               \def\yunit{1.5}
               \psset{xunit=\xunit,yunit=\yunit,algebraic=true,arrowsize=3pt 2,arrowinset=0.25}
               \fontsize{15pt}{15pt}\selectfont
               \begin{pspicture*}[linewidth=1pt](\xmin,\ymin)(\xmax,\ymax)
                    %      \psgrid[gridcolor=mcgris, subgriddiv=5, gridlabels=0pt](\xmin,\ymin)(\xmax,\ymax)
                    %           \psaxes[labels=none,linewidth=0.75pt]{->}(0,0)(\xmin,\ymin)(\xmax,\ymax)
                    \psline[linewidth=0.75pt,linecolor=dmauve]{->}(0,0)(\xmin,0)(\xmax,0)
                    \psline[linewidth=0.75pt,linecolor=dvert]{->}(0,0)(0,\ymin)(0,\ymax)
                    \rput[tr](-0.1,-0.1){$O$}
                    \psdots[dotsize=2pt 0,dotstyle=*,linecolor=dblue](3,2)
                    \rput[bl](3,2){$\color{dblue} M(z=a+\text{i}b)$}
                    \psline[linewidth=0.75pt,linecolor=dblue]{->}(0,0)(3,2)
                    \psline[linestyle=dotted,linewidth=0.75pt,linecolor=dblue](3,0)(3,2)(0,2)
                    \rput[t](3,-0.1){\color{dblue} $a$}
                    \rput[r](-0.1,2){\color{dblue} $b$}
                    \rput[tr](6.4,-0.1){\color{dmauve} axe des réels}
                    \rput[tr]{90}(-0.8,5.3){\color{dvert} axe des}
                    \rput[tr]{90}(-0.45,5.3){\color{dvert} imaginaires purs}
                    \psline[linewidth=1.25pt,linecolor=dvert]{->}(0,0)(0,1)
                    \psline[linewidth=1.25pt,linecolor=dmauve]{->}(0,0)(1,0)
                    \rput[t](0.5,-0.03){\color{dmauve} $\vect{u}$}
                    \rput[r](-0.03,0.5){\color{dvert} $\vect{v}$}
               \end{pspicture*}
          }
     \end{extern}
\end{center}
\cadre{vert}{Propriétés}{% id="p80"
     \begin{itemize}
          \item $M$ appartient à l'axe des abscisses si et seulement si son affixe $z$ est un nombre réel
          \item $M$ appartient à l'axe des ordonnées si et seulement si son affixe $z$ est un nombre imaginaire pur
          \item Deux nombres complexes conjugués ont des affixes symétriques par rapport à l'axe des abscisses
     \end{itemize}
}
\begin{center}
     \begin{extern} %width="450" alt="nombres complexes conjugués"
          \newrgbcolor{dblue}{0. 0. 0.7}
          \newrgbcolor{dvert}{0. 0.4 0.}
          \newrgbcolor{dmauve}{0.5 0. 0.5}
          \resizebox{8cm}{!}{%
               % -+-+-+ variables modifiables
               \def\xmin{-1.2}
               \def\xmax{6.5}
               \def\ymin{-2.5}
               \def\ymax{3.2}
               \def\xunit{1.5}  % unités en cm
               \def\yunit{1.5}
               \psset{xunit=\xunit,yunit=\yunit,algebraic=true,arrowsize=3pt 2,arrowinset=0.25}
               \fontsize{15pt}{15pt}\selectfont
               \begin{pspicture*}[linewidth=1pt](\xmin,\ymin)(\xmax,\ymax)
                    %      \psgrid[gridcolor=mcgris, subgriddiv=5, gridlabels=0pt](\xmin,\ymin)(\xmax,\ymax)
                    \psaxes[ticks=none,labels=none,linewidth=0.75pt]{->}(0,0)(\xmin,\ymin)(\xmax,\ymax)
                    \rput[tr](-0.1,-0.1){$O$}
                    \psdots[dotsize=2pt 0,dotstyle=*,linecolor=dblue](3,2)
                    \rput[bl](3,2){$\color{dblue} M(z)$}
                    \psline[linewidth=0.75pt,linecolor=dvert]{->}(0,0)(3,-2)
                    \rput[bl](3,-2){$\color{dvert} M'(\bar{z})$}
                    \psline[linewidth=0.75pt,linecolor=dblue]{->}(0,0)(3,2)
                    \psline[linestyle=dotted,linewidth=0.75pt,linecolor=dblue](3,0)(3,2)(0,2)
                    \psline[linestyle=dotted,linewidth=0.75pt,linecolor=dvert](3,0)(3,-2)(0,-2)
                    \rput[t](3,-0.1){\color{dblue} $a$}
                    \rput[r](-0.1,2){\color{dblue} $b$}
                    \rput[r](-0.1,-2){\color{dvert} $-b$}
                    \psline[linewidth=1.25pt]{->}(0,0)(0,1)
                    \psline[linewidth=1.25pt]{->}(0,0)(1,0)
                    \rput[t](0.5,-0.03){$\vect{u}$}
                    \rput[r](-0.03,0.5){ $\vect{v}$}
               \end{pspicture*}
          }
     \end{extern}
\end{center}
\cadre{vert}{Propriétés}{% id="p90"
     Soient $A$ et $B$ deux points d'affixes respectives $z_{A}$ et $z_{B}$.
     \begin{itemize}
          \item  l'affixe du vecteur $\overrightarrow{AB}$ est égale à~:
          \begin{center}$z_{\overrightarrow{AB}}= z_{B}-z_{A}$\end{center}
          \item  l'affixe du milieu $M$ du segment $\left[AB\right]$ est égale à~:
          \begin{center}$z_{M}= \frac{z_{A}+z_{B}}{2}$\end{center}
     \end{itemize}
}
\cadre{vert}{Propriétés}{% id="p100"
     Soient $\vec{w}\left(z\right)$ et $\overrightarrow{w^{\prime}}\left(z^{\prime}\right)$ deux vecteurs du plan et $k$ un nombre réel.
     \begin{itemize}
          \item Le vecteur $\vec{w}+\overrightarrow{w^{\prime}}$ a pour affixe $z+z^{\prime}$~;
          \item Le vecteur $k\vec{w}$ a pour affixe $kz$.
     \end{itemize}
}
\begin{h2}2. Forme trigonométrique\end{h2}
\cadre{bleu}{Définition}{% id="d110"
     Soit $z$ un nombre complexe \textbf{non nul} d'image $M$ dans le repère $\left(O; \vec{u}, \vec{v}\right)$.
     \par
     On appelle module de $z$, et on note $|z|$ le nombre \textbf{réel} positif ou nul $|z|=\sqrt{a^{2}+b^{2}}$.
     \par
     On appelle argument de $z$ et on note $\text{arg}\left(z\right)$ une mesure, exprimée en radians, de l'angle
     \par
     $\left(\vec{u}; \overrightarrow{OM}\right)$.
}
\begin{center}
     \begin{extern} %width="450" alt="forme trigonométrique des nombres complexes"
          \newrgbcolor{dblue}{0. 0. 0.7}
          \newrgbcolor{dvert}{0. 0.4 0.}
          \newrgbcolor{dmauve}{0.5 0. 0.5}
          \resizebox{8cm}{!}{%
               % -+-+-+ variables modifiables
               \def\xmin{-1.2}
               \def\xmax{6.5}
               \def\ymin{-0.8}
               \def\ymax{3.2}
               \def\xunit{1.5}  % unités en cm
               \def\yunit{1.5}
               \psset{xunit=\xunit,yunit=\yunit,algebraic=true,arrowsize=3pt 2,arrowinset=0.25}
               \fontsize{15pt}{15pt}\selectfont
               \begin{pspicture*}[linewidth=1pt](\xmin,\ymin)(\xmax,\ymax)
                    %      \psgrid[gridcolor=mcgris, subgriddiv=5, gridlabels=0pt](\xmin,\ymin)(\xmax,\ymax)
                    \psaxes[ticks=none,labels=none,linewidth=0.75pt]{->}(0,0)(\xmin,\ymin)(\xmax,\ymax)
                    \rput[tr](-0.1,-0.1){$O$}
                    \psdots[dotsize=2pt 0,dotstyle=*,linecolor=dblue](3,2)
                    \rput[bl](3,2){$\color{dblue} M(z)$}
                    \psline[linewidth=0.75pt,linecolor=dvert](0,0)(3,2)
                    \rput[t]{33.7}(1.3,1.3){$\color{dvert} |z|=OM$}
                    \psline[linestyle=dotted,linewidth=0.75pt,linecolor=dblue](3,0)(3,2)(0,2)
                    \rput[t](3,-0.1){\color{dblue} $a$}
                    \rput[r](-0.1,2){\color{dblue} $b$}
                    \pscustom[linewidth=0.8pt,linecolor=dmauve,fillcolor=dmauve,fillstyle=solid,opacity=0.1]{ % color angle
                         \parametricplot{0.0}{0.588}{0.6*cos(t)|0.6*sin(t)}
                         \lineto(0.,0.)
                    \closepath}
                    \psellipticarc[linewidth=0.8pt,linecolor=dmauve,arrows=->](0.,0.)(0.6,0.6){0.}{33.7} % fleche angle
                    \rput[t](1.4,0.38){$\color{dmauve} \theta=\text{arg}(z)$}
                    \psline[linewidth=1.25pt]{->}(0,0)(0,1)
                    \psline[linewidth=1.25pt]{->}(0,0)(1,0)
                    \rput[t](0.5,-0.03){$\vect{u}$}
                    \rput[r](-0.03,0.5){$\vect{v}$}
               \end{pspicture*}
          }
     \end{extern}
\end{center}
\cadre{vert}{Propriétés des modules}{% id="p130"
     Pour tous nombres complexes $z$ et $z^{\prime}$~:
     \begin{itemize}
          \item $|z|^{2} = z\times \overline{z}$
          \item $|zz^{\prime}| = |z|\times |z^{\prime}|$
          \item $|\frac{z}{z^{\prime}}| = \frac{|z|}{|z^{\prime}|}  $ pour $z^{\prime}\neq 0$
     \end{itemize}
}
\cadre{vert}{Propriétés des arguments}{% id="p140"
     Pour tous nombres complexes $z$ et $z^{\prime}$ \textbf{non nuls} et tout entier $n\in \mathbb{Z}$~:
     \begin{itemize}
          \item $\text{arg}\left(\overline{z}\right)=-\text{arg}\left(z\right)$
          \item $\text{arg}\left(zz^{\prime}\right)=\text{arg}\left(z\right)+\text{arg}\left(z^{\prime}\right)$
          \item $\text{arg}\left(z^{n}\right)=n\times \text{arg}\left(z\right)$
          \item $\text{arg}\left(\frac{z}{z^{\prime}}\right)=\text{arg}\left(z\right)-\text{arg}\left(z^{\prime}\right)$
     \end{itemize}
}
\bloc{cyan}{Remarque}{% id="r140"
     En particulier~:
     \begin{itemize}
          \item $\text{arg}\left(-z\right)=\text{arg}\left(z\right)+\text{arg}\left(-1\right) = \text{arg}\left(z\right)+\pi $
          \item $\text{arg}\left(\frac{1}{z}\right)=\text{arg}\left(1\right)-\text{arg}\left(z\right) = -\text{arg}\left(z\right)$.
     \end{itemize}
}
\cadre{rouge}{Théorème et définition}{% id="t170"
     Soit $z$ un nombre complexe non nul de module $r$ et d'argument $\theta $~:
     \begin{center}$z=r\left(\cos\theta  + i \sin\theta \right)$\end{center}
     Cette écriture s'appelle \textbf{forme trigonométrique} du nombre $z$.
}
\cadre{vert}{Passage de la forme algébrique à la forme trigonométrique}{% id="p180"
     Soit $z=a+ib$ un nombre complexe non nul.
     \begin{itemize}
          \item $r=|z|=\sqrt{a^{2}+b^{2}}$
          \item $\theta =\text{arg}\left(z\right)$ est défini par~:
          \par
          $\cos \theta  = \frac{a}{\sqrt{a^{2}+b^{2}}}$ et $ \sin \theta  = \frac{b}{\sqrt{a^{2}+b^{2}}}$.
     \end{itemize}
}
\bloc{orange}{Exemple}{% id="e180"
     Soit $z=\sqrt{3}+i$.
     \par
     $|z|=\sqrt{3+1}=2$
     \par
     Si $\theta $ est un argument de $z$~:
     \par
     $\cos \theta =\frac{\sqrt{3}}{2} $ et $ \sin \theta =\frac{1}{2}$ donc $\theta =\frac{\pi }{6}   \left(\text{mod. } 2\pi \right)$
     \par
     La forme trigonométrique de $z$ est donc~:
     \par
     $z=2\left(\cos \frac{\pi }{6} + i \sin \frac{\pi }{6}\right) $.
     \begin{center}
          \begin{extern} %width="380" alt="représentation graphique du nombre complexe racine de 3 + i"
               \newrgbcolor{dblue}{0. 0. 0.7}
               \newrgbcolor{dvert}{0. 0.4 0.}
               \newrgbcolor{dmauve}{0.5 0. 0.5}
               \resizebox{7cm}{!}{%
                    % -+-+-+ variables modifiables
                    \def\xmin{-1.2}
                    \def\xmax{3}
                    \def\ymin{-0.8}
                    \def\ymax{2}
                    \def\xunit{2.5}  % unités en cm
                    \def\yunit{2.5}
                    \psset{xunit=\xunit,yunit=\yunit,algebraic=true,arrowsize=3pt 2,arrowinset=0.25}
                    \fontsize{15pt}{15pt}\selectfont
                    \begin{pspicture*}[linewidth=1pt](\xmin,\ymin)(\xmax,\ymax)
                         %      \psgrid[gridcolor=mcgris, subgriddiv=5, gridlabels=0pt](\xmin,\ymin)(\xmax,\ymax)
                         \psaxes[ticks=none,labels=none,linewidth=0.75pt]{->}(0,0)(\xmin,\ymin)(\xmax,\ymax)
                         \rput[tr](-0.1,-0.1){$O$}
                         \psdots[dotsize=2pt 0,dotstyle=*,linecolor=dblue](1.732,1)
                         \rput[bl](1.732,1){$\color{dblue} M(\sqrt{3}+\text{i})$}
                         \psline[linewidth=0.75pt,linecolor=dvert](0,0)(1.732,1)
                         \rput[t]{30}(0.8,0.7){$\color{dvert} |z|=2$}
                         \psline[linestyle=dotted,linewidth=0.75pt,linecolor=dblue](1.732,0)(1.732,1)(0,1)
                         \rput[t](1.732,-0.1){\color{dblue} $\sqrt{3}$}
                         \rput[r](-0.1,1){\color{dblue} $1$}
                         \pscustom[linewidth=0.8pt,linecolor=dmauve,fillcolor=dmauve,fillstyle=solid,opacity=0.1]{ % color angle
                              \parametricplot{0.0}{0.524}{0.5*cos(t)|0.5*sin(t)}
                              \lineto(0.,0.)
                         \closepath}
                         \psellipticarc[linewidth=0.8pt,linecolor=dmauve,arrows=->](0.,0.)(0.5,0.5){0.}{30} % fleche angle
                         \rput[t](0.9,0.42){$\color{dmauve} \theta=\dfrac{\pi}{6}$}
                         \psline[linewidth=1.25pt]{->}(0,0)(0,1)
                         \psline[linewidth=1.25pt]{->}(0,0)(1,0)
                         \rput[t](0.5,-0.03){$\vect{u}$}
                         \rput[r](-0.03,0.5){$\vect{v}$}
                    \end{pspicture*}
               }
          \end{extern}
     \end{center}
}
\cadre{vert}{Angle de vecteurs et arguments}{% id="p190"
     Soit $A, B$ et $C$ trois points du plan d'afixes respectives $z_{A}$,$z_{B}$, $z_{C}$ avec $A\neq B$ et $A\neq C$~:
     \par
     $\left(\overrightarrow{AB};\overrightarrow{AC}\right)= \text{arg}\left(\frac{z_{C}-z_{A}}{z_{B}-z_{A}}\right)$.
}
\begin{center}
     \begin{extern} %width="450" alt="forme trigonométrique des nombres complexes"
          \newrgbcolor{dblue}{0. 0. 0.7}
          \newrgbcolor{dvert}{0. 0.4 0.}
          \newrgbcolor{dmauve}{0.5 0. 0.5}
          \resizebox{8cm}{!}{%
               % -+-+-+ variables modifiables
               \def\xmin{-1.2}
               \def\xmax{6.5}
               \def\ymin{-0.8}
               \def\ymax{4.2}
               \def\xunit{1.5}  % unités en cm
               \def\yunit{1.5}
               \psset{xunit=\xunit,yunit=\yunit,algebraic=true,arrowsize=3pt 2,arrowinset=0.25}
               \fontsize{15pt}{15pt}\selectfont
               \begin{pspicture*}[linewidth=1pt](\xmin,\ymin)(\xmax,\ymax)
                    %      \psgrid[gridcolor=mcgris, subgriddiv=5, gridlabels=0pt](\xmin,\ymin)(\xmax,\ymax)
                    \psaxes[ticks=none,labels=none,linewidth=0.75pt]{->}(0,0)(\xmin,\ymin)(\xmax,\ymax)
                    \rput[tr](-0.1,-0.1){$O$}
                    \psdots[dotsize=2pt 0,dotstyle=*,linecolor=dblue](1,1)
                    \rput[tr](0.9,0.9){$\color{dblue} A$}
                    \psdots[dotsize=2pt 0,dotstyle=*,linecolor=dblue](5,2)
                    \rput[bl](5,2.1){$\color{dblue} B$}
                    \psdots[dotsize=2pt 0,dotstyle=*,linecolor=dblue](2,3)
                    \rput[bl](2,3.1){$\color{dblue} C$}
                    \psline[linewidth=0.75pt,linecolor=dblue]{->}(1,1)(5,2)
                    \psline[linewidth=0.75pt,linecolor=dblue]{->}(1,1)(2,3)
                    \pscustom[linewidth=0.8pt,linecolor=dmauve,fillcolor=dmauve,fillstyle=solid,opacity=0.1]{ % color angle
                         \parametricplot{0.25}{1.11}{0.6*cos(t)+1|0.6*sin(t)+1}
                         \lineto(1,1)
                    \closepath}
                    \psellipticarc[linewidth=0.8pt,linecolor=dmauve,arrows=->](1.,1.)(0.6,0.6){14.}{63.4} % fleche angle
                    \rput[t](2.9,2.6){$\color{dmauve} \theta=\text{arg}\left(\frac{z_{C}-z_{A}}{z_{B}-z_{A}}\right)$}
                    \psline[linewidth=1.25pt]{->}(0,0)(0,1)
                    \psline[linewidth=1.25pt]{->}(0,0)(1,0)
                    \rput[t](0.5,-0.03){$\vect{u}$}
                    \rput[r](-0.03,0.5){$\vect{v}$}
               \end{pspicture*}
          }
     \end{extern}
\end{center}
\bloc{cyan}{Remarques}{% id="r190"
     \begin{itemize}
          \item Notez bien l'\textbf{ordre des affixes} (inverse de l'ordre des points dans l'écriture de l'angle).
          \item \textbf{Premier cas particulier important~:}
          \par
          $A, B$ et $C$ sont alignés \\
          $\phantom{A, B} \Leftrightarrow  \text{arg}\left(\frac{z_{C}-z_{A}}{z_{B}-z_{A}}\right) = 0~\text{ou}~\pi~\left[\text{mod. } 2\pi \right] $ \\
          $\phantom{A, B} \Leftrightarrow  \frac{z_{C}-z_{A}}{z_{B}-z_{A}} \in  \mathbb{R}$.
          \item \textbf{Second cas particulier important~:}
          \par
          $\widehat{BAC}$ est un angle droit \\
          $\phantom{A, B} \Leftrightarrow  \text{arg}\left(\frac{z_{C}-z_{A}}{z_{B}-z_{A}}\right) = \pm \frac{\pi }{2} ~  \left[\text{mod. } 2\pi \right] $\\
          $\phantom{A, B} \Leftrightarrow  \frac{z_{C}-z_{A}}{z_{B}-z_{A}}$ est un \textbf{imaginaire pur}.
     \end{itemize}
}
\begin{h2}3. Forme exponentielle\end{h2}
\cadre{bleu}{Notation}{% id="d200"
     Si $z$ est un nombre complexe de module $r$ et d'argument $\theta $, la notation exponentielle du nombre $z$ est~:
     \begin{center}$z=re^{i\theta }$\end{center}
}
\bloc{cyan}{Remarque}{% id="r200"
     Ce sont les propriétés des arguments~:
     \begin{itemize}
          \item $\text{arg}\left(zz^{\prime}\right)=\text{arg}\left(z\right)+\text{arg}\left(z^{\prime}\right)$
          \item $\text{arg}\left(z^{n}\right)=n\times \text{arg}\left(z\right)$
          \item $\text{arg}\left(\frac{z}{z^{\prime}}\right)=\text{arg}\left(z\right)-\text{arg}\left(z^{\prime}\right)$
     \end{itemize}
     similaires aux propriétés de l'exponentielle qui justifient cette notation.
}
\bloc{orange}{Exemple}{% id="e200"
     Le nombre $-1$ a pour module $1$ et pour argument $\pi   \left(\text{mod. } 2\pi \right)$. On peut donc écrire~:
     \par
     $-1=e^{i\pi }$ ou encore $e^{i\pi }+1=0$.
     \par
     C'est la célèbre \Lien{http://fr.wikipedia.org/wiki/Identit\%C3\%A9_d\%27Euler}{identité d'Euler} qui relie $0$, $1$, $e$, $i$ et $\pi $.
}
Les propriétés des arguments vues précédemment s'écrivent alors~:
\cadre{vert}{Propriétés}{% id="p210"
     Pour tous réels $\theta $ et $\theta ^{\prime}$~:
     \begin{itemize}
          \item $e^{i\theta }\times e^{i\theta ^{\prime}}=e^{i\left(\theta +\theta ^{\prime}\right)}$
          \item $\left(e^{i\theta }\right)^{n}=e^{in\theta }$
          \item $\frac{e^{i\theta }}{e^{i\theta ^{\prime}}}=e^{i\left(\theta -\theta ^{\prime}\right)}$.
     \end{itemize}
}

\end{document}