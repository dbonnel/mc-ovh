\documentclass[a4paper]{article}

%================================================================================================================================
%
% Packages
%
%================================================================================================================================

\usepackage[T1]{fontenc} 	% pour caractères accentués
\usepackage[utf8]{inputenc}  % encodage utf8
\usepackage[french]{babel}	% langue : français
\usepackage{fourier}			% caractères plus lisibles
\usepackage[dvipsnames]{xcolor} % couleurs
\usepackage{fancyhdr}		% réglage header footer
\usepackage{needspace}		% empêcher sauts de page mal placés
\usepackage{graphicx}		% pour inclure des graphiques
\usepackage{enumitem,cprotect}		% personnalise les listes d'items (nécessaire pour ol, al ...)
\usepackage{hyperref}		% Liens hypertexte
\usepackage{pstricks,pst-all,pst-node,pstricks-add,pst-math,pst-plot,pst-tree,pst-eucl} % pstricks
\usepackage[a4paper,includeheadfoot,top=2cm,left=3cm, bottom=2cm,right=3cm]{geometry} % marges etc.
\usepackage{comment}			% commentaires multilignes
\usepackage{amsmath,environ} % maths (matrices, etc.)
\usepackage{amssymb,makeidx}
\usepackage{bm}				% bold maths
\usepackage{tabularx}		% tableaux
\usepackage{colortbl}		% tableaux en couleur
\usepackage{fontawesome}		% Fontawesome
\usepackage{environ}			% environment with command
\usepackage{fp}				% calculs pour ps-tricks
\usepackage{multido}			% pour ps tricks
\usepackage[np]{numprint}	% formattage nombre
\usepackage{tikz,tkz-tab} 			% package principal TikZ
\usepackage{pgfplots}   % axes
\usepackage{mathrsfs}    % cursives
\usepackage{calc}			% calcul taille boites
\usepackage[scaled=0.875]{helvet} % font sans serif
\usepackage{svg} % svg
\usepackage{scrextend} % local margin
\usepackage{scratch} %scratch
\usepackage{multicol} % colonnes
%\usepackage{infix-RPN,pst-func} % formule en notation polanaise inversée
\usepackage{listings}

%================================================================================================================================
%
% Réglages de base
%
%================================================================================================================================

\lstset{
language=Python,   % R code
literate=
{á}{{\'a}}1
{à}{{\`a}}1
{ã}{{\~a}}1
{é}{{\'e}}1
{è}{{\`e}}1
{ê}{{\^e}}1
{í}{{\'i}}1
{ó}{{\'o}}1
{õ}{{\~o}}1
{ú}{{\'u}}1
{ü}{{\"u}}1
{ç}{{\c{c}}}1
{~}{{ }}1
}


\definecolor{codegreen}{rgb}{0,0.6,0}
\definecolor{codegray}{rgb}{0.5,0.5,0.5}
\definecolor{codepurple}{rgb}{0.58,0,0.82}
\definecolor{backcolour}{rgb}{0.95,0.95,0.92}

\lstdefinestyle{mystyle}{
    backgroundcolor=\color{backcolour},   
    commentstyle=\color{codegreen},
    keywordstyle=\color{magenta},
    numberstyle=\tiny\color{codegray},
    stringstyle=\color{codepurple},
    basicstyle=\ttfamily\footnotesize,
    breakatwhitespace=false,         
    breaklines=true,                 
    captionpos=b,                    
    keepspaces=true,                 
    numbers=left,                    
xleftmargin=2em,
framexleftmargin=2em,            
    showspaces=false,                
    showstringspaces=false,
    showtabs=false,                  
    tabsize=2,
    upquote=true
}

\lstset{style=mystyle}


\lstset{style=mystyle}
\newcommand{\imgdir}{C:/laragon/www/newmc/assets/imgsvg/}
\newcommand{\imgsvgdir}{C:/laragon/www/newmc/assets/imgsvg/}

\definecolor{mcgris}{RGB}{220, 220, 220}% ancien~; pour compatibilité
\definecolor{mcbleu}{RGB}{52, 152, 219}
\definecolor{mcvert}{RGB}{125, 194, 70}
\definecolor{mcmauve}{RGB}{154, 0, 215}
\definecolor{mcorange}{RGB}{255, 96, 0}
\definecolor{mcturquoise}{RGB}{0, 153, 153}
\definecolor{mcrouge}{RGB}{255, 0, 0}
\definecolor{mclightvert}{RGB}{205, 234, 190}

\definecolor{gris}{RGB}{220, 220, 220}
\definecolor{bleu}{RGB}{52, 152, 219}
\definecolor{vert}{RGB}{125, 194, 70}
\definecolor{mauve}{RGB}{154, 0, 215}
\definecolor{orange}{RGB}{255, 96, 0}
\definecolor{turquoise}{RGB}{0, 153, 153}
\definecolor{rouge}{RGB}{255, 0, 0}
\definecolor{lightvert}{RGB}{205, 234, 190}
\setitemize[0]{label=\color{lightvert}  $\bullet$}

\pagestyle{fancy}
\renewcommand{\headrulewidth}{0.2pt}
\fancyhead[L]{maths-cours.fr}
\fancyhead[R]{\thepage}
\renewcommand{\footrulewidth}{0.2pt}
\fancyfoot[C]{}

\newcolumntype{C}{>{\centering\arraybackslash}X}
\newcolumntype{s}{>{\hsize=.35\hsize\arraybackslash}X}

\setlength{\parindent}{0pt}		 
\setlength{\parskip}{3mm}
\setlength{\headheight}{1cm}

\def\ebook{ebook}
\def\book{book}
\def\web{web}
\def\type{web}

\newcommand{\vect}[1]{\overrightarrow{\,\mathstrut#1\,}}

\def\Oij{$\left(\text{O}~;~\vect{\imath},~\vect{\jmath}\right)$}
\def\Oijk{$\left(\text{O}~;~\vect{\imath},~\vect{\jmath},~\vect{k}\right)$}
\def\Ouv{$\left(\text{O}~;~\vect{u},~\vect{v}\right)$}

\hypersetup{breaklinks=true, colorlinks = true, linkcolor = OliveGreen, urlcolor = OliveGreen, citecolor = OliveGreen, pdfauthor={Didier BONNEL - https://www.maths-cours.fr} } % supprime les bordures autour des liens

\renewcommand{\arg}[0]{\text{arg}}

\everymath{\displaystyle}

%================================================================================================================================
%
% Macros - Commandes
%
%================================================================================================================================

\newcommand\meta[2]{    			% Utilisé pour créer le post HTML.
	\def\titre{titre}
	\def\url{url}
	\def\arg{#1}
	\ifx\titre\arg
		\newcommand\maintitle{#2}
		\fancyhead[L]{#2}
		{\Large\sffamily \MakeUppercase{#2}}
		\vspace{1mm}\textcolor{mcvert}{\hrule}
	\fi 
	\ifx\url\arg
		\fancyfoot[L]{\href{https://www.maths-cours.fr#2}{\black \footnotesize{https://www.maths-cours.fr#2}}}
	\fi 
}


\newcommand\TitreC[1]{    		% Titre centré
     \needspace{3\baselineskip}
     \begin{center}\textbf{#1}\end{center}
}

\newcommand\newpar{    		% paragraphe
     \par
}

\newcommand\nosp {    		% commande vide (pas d'espace)
}
\newcommand{\id}[1]{} %ignore

\newcommand\boite[2]{				% Boite simple sans titre
	\vspace{5mm}
	\setlength{\fboxrule}{0.2mm}
	\setlength{\fboxsep}{5mm}	
	\fcolorbox{#1}{#1!3}{\makebox[\linewidth-2\fboxrule-2\fboxsep]{
  		\begin{minipage}[t]{\linewidth-2\fboxrule-4\fboxsep}\setlength{\parskip}{3mm}
  			 #2
  		\end{minipage}
	}}
	\vspace{5mm}
}

\newcommand\CBox[4]{				% Boites
	\vspace{5mm}
	\setlength{\fboxrule}{0.2mm}
	\setlength{\fboxsep}{5mm}
	
	\fcolorbox{#1}{#1!3}{\makebox[\linewidth-2\fboxrule-2\fboxsep]{
		\begin{minipage}[t]{1cm}\setlength{\parskip}{3mm}
	  		\textcolor{#1}{\LARGE{#2}}    
 	 	\end{minipage}  
  		\begin{minipage}[t]{\linewidth-2\fboxrule-4\fboxsep}\setlength{\parskip}{3mm}
			\raisebox{1.2mm}{\normalsize\sffamily{\textcolor{#1}{#3}}}						
  			 #4
  		\end{minipage}
	}}
	\vspace{5mm}
}

\newcommand\cadre[3]{				% Boites convertible html
	\par
	\vspace{2mm}
	\setlength{\fboxrule}{0.1mm}
	\setlength{\fboxsep}{5mm}
	\fcolorbox{#1}{white}{\makebox[\linewidth-2\fboxrule-2\fboxsep]{
  		\begin{minipage}[t]{\linewidth-2\fboxrule-4\fboxsep}\setlength{\parskip}{3mm}
			\raisebox{-2.5mm}{\sffamily \small{\textcolor{#1}{\MakeUppercase{#2}}}}		
			\par		
  			 #3
 	 		\end{minipage}
	}}
		\vspace{2mm}
	\par
}

\newcommand\bloc[3]{				% Boites convertible html sans bordure
     \needspace{2\baselineskip}
     {\sffamily \small{\textcolor{#1}{\MakeUppercase{#2}}}}    
		\par		
  			 #3
		\par
}

\newcommand\CHelp[1]{
     \CBox{Plum}{\faInfoCircle}{À RETENIR}{#1}
}

\newcommand\CUp[1]{
     \CBox{NavyBlue}{\faThumbsOUp}{EN PRATIQUE}{#1}
}

\newcommand\CInfo[1]{
     \CBox{Sepia}{\faArrowCircleRight}{REMARQUE}{#1}
}

\newcommand\CRedac[1]{
     \CBox{PineGreen}{\faEdit}{BIEN R\'EDIGER}{#1}
}

\newcommand\CError[1]{
     \CBox{Red}{\faExclamationTriangle}{ATTENTION}{#1}
}

\newcommand\TitreExo[2]{
\needspace{4\baselineskip}
 {\sffamily\large EXERCICE #1\ (\emph{#2 points})}
\vspace{5mm}
}

\newcommand\img[2]{
          \includegraphics[width=#2\paperwidth]{\imgdir#1}
}

\newcommand\imgsvg[2]{
       \begin{center}   \includegraphics[width=#2\paperwidth]{\imgsvgdir#1} \end{center}
}


\newcommand\Lien[2]{
     \href{#1}{#2 \tiny \faExternalLink}
}
\newcommand\mcLien[2]{
     \href{https~://www.maths-cours.fr/#1}{#2 \tiny \faExternalLink}
}

\newcommand{\euro}{\eurologo{}}

%================================================================================================================================
%
% Macros - Environement
%
%================================================================================================================================

\newenvironment{tex}{ %
}
{%
}

\newenvironment{indente}{ %
	\setlength\parindent{10mm}
}

{
	\setlength\parindent{0mm}
}

\newenvironment{corrige}{%
     \needspace{3\baselineskip}
     \medskip
     \textbf{\textsc{Corrigé}}
     \medskip
}
{
}

\newenvironment{extern}{%
     \begin{center}
     }
     {
     \end{center}
}

\NewEnviron{code}{%
	\par
     \boite{gray}{\texttt{%
     \BODY
     }}
     \par
}

\newenvironment{vbloc}{% boite sans cadre empeche saut de page
     \begin{minipage}[t]{\linewidth}
     }
     {
     \end{minipage}
}
\NewEnviron{h2}{%
    \needspace{3\baselineskip}
    \vspace{0.6cm}
	\noindent \MakeUppercase{\sffamily \large \BODY}
	\vspace{1mm}\textcolor{mcgris}{\hrule}\vspace{0.4cm}
	\par
}{}

\NewEnviron{h3}{%
    \needspace{3\baselineskip}
	\vspace{5mm}
	\textsc{\BODY}
	\par
}

\NewEnviron{margeneg}{ %
\begin{addmargin}[-1cm]{0cm}
\BODY
\end{addmargin}
}

\NewEnviron{html}{%
}

\begin{document}
\meta{url}{/exercices/qcm-general-bac-s-polynesie-francaise-2008/}
\meta{pid}{2287}
\meta{titre}{QCM général - Bac S Polynésie Francaise 2008}
\meta{type}{exercices}
%
\begin{h2}Exercice 3\end{h2}
\textit{5 points-Candidats n'ayant pas suivi l'enseignement de spécialité}
Pour chacune des propositions suivantes, indiquer si elle est vraie ou fausse et donner une justification de la réponse choisie.
\par
Une réponse non justifiée ne rapporte aucun point. Toutefois, toute trace de recherche, même incomplète, ou d'initiative, même non fructueuse, sera prise en compte dans l'évaluation.
\begin{enumerate}
     \item
     Soit $f$ la fonction solution sur $\mathbb{R}$ de l'équation différentielle $y^{\prime}=-y+2$ telle que $f\left(\ln 2\right)=1$
\par
     \textbf{Proposition 1} : « La courbe représentative de $f$ admet au point d'abscisse 0, une tangente d'équation $y=2x$ ».
     \item
     Soient $f$ et g deux fonctions définies sur un intervalle [A,$+\infty $[ où A est un réel strictement positif.
\par
     \textbf{Proposition 2} : « Si $\lim\limits_{x\rightarrow +\infty }f\left(x\right)=0$ alors $\lim\limits_{x\rightarrow +\infty }f\left(x\right)g\left(x\right)=0$ ».
     \item
     On admet qu'un bloc de glace fond en perdant 10\% de sa masse par minute.
     \par
     Sa masse initiale est de 10 kg.
\par
     \textbf{Proposition 3} : « A partir de la soixante-dixième minute, sa masse devient inférieure à 1 g ».
     \item
     Soient A et B deux évènements d'un même univers $\Omega $ muni d'une probabilité p.
\par
     \textbf{Proposition 4} : « Si A et B sont indépendants et si p(A)=p(B)=0,4 alors p(A$\cup $B)=0,8 ».
     \item
     Une usine fabrique des pièces. Une étude statistique a montré que 2\% de la production est défectueuse. Chaque pièce est soumise à un contrôle de fabrication. Ce contrôle refuse 99\% des pièces défectueuses et accepte 97\% des pièces non défectueuses.
     \par
     On choisit au hasard une pièce avant son passage au contrôle.
\par
     \textbf{Proposition 5 }: « La probabilité que la pièce soit acceptée est égale à 0,9508 ».
\end{enumerate}
\begin{corrige}
     \begin{enumerate}
          \item
          \textbf{Proposition 1} : « La courbe représentative de $ > f$ admet au point d'abscisse 0, une tangente d'équation $y=2x$ ».
          \par
          Réponse exacte : VRAI
          \par
          $f\left(x\right)$ est de la forme $f\left(x\right)=Ce^{-x}+2$
          \par
          La condition $f\left(\ln2\right)=1$ donne $\frac{1}{2}C+2=1$ soit $C=-2$
          \par
          Donc $f\left(x\right)=-2e^{-x}+2$.
          \par
          $f\left(0\right)=-2+2=0$ et $f^{\prime}\left(0\right)=-f\left(0\right)+2=2$
          \par
          L'équation de la tangente au point d'abscisse 0 est donc
          \par
          $y=2\left(x-0\right)+0$
          \par
          $y=2x$
          \item
          \textbf{Proposition 2} : « Si $\lim\limits_{x\rightarrow +\infty }f\left(x\right)=0$ alors $\lim\limits_{x\rightarrow +\infty }f\left(x\right)g\left(x\right)=0$ ».
          \par
          Réponse exacte : FAUX
          \par
          On prend par exemple $f\left(x\right)=\frac{1}{x}$ et $g\left(x\right)=x$ sur $\left[1;+\infty \right[$
          \par
          $\lim\limits_{x\rightarrow +\infty }f\left(x\right)=0$ mais $\lim\limits_{x\rightarrow +\infty }f\left(x\right)g\left(x\right)=1$
          \item
          \textbf{Proposition 3} : « A partir de la soixante-dixième minute, sa masse devient inférieure à 1 g ».
          \par
          Réponse exacte : FAUX
          \par
          Le coefficient multiplicateur correspondant à une diminution de 10\% est $C=1-\frac{10}{100}=0,9$
          \par
          La masse à la soixante-dixième minute est :
          \par
          $m=10 000\times 0,9^{70}\approx 6,3$
          \item
          \textbf{Proposition 4} : « Si A et B sont indépendants et si p(A)=p(B)=0,4 alors p(A$\cup $B)=0,8 ».
          \par
          Réponse exacte : FAUX
          \par
          Comme A et B sont indépendants p(A$\cap $B)=p(A)$\times $p(B)=0,16.
          \par
          p(A$\cap $B=p(A)+p(B)-p(A$\cap $B)=0,64
          \item
          \textbf{Proposition 5 }: « La probabilité que la pièce soit acceptée est égale à 0,9508 ».
          \par
          Réponse exacte : VRAI
          \par
          Notons :
          \par
          A l'évènement : "la pièce est accepté";
          \par
          D l'évènement : "la pièce est défectueuse".
          \par
          Les données de l'énoncé permettent de tracer l'arbre suivant:
%##
% type=arbre; width=25; wcell=3.5; hcell=1.5
%--
% >D:0,02
% >>A:0,01
% >>\overline{A}:0,99
% >\overline{D}:0,98
% >>A:0,97
% >>\overline{A}:0,03
%--
\begin{center}
 \begin{extern}%style="width:25rem" alt="Arbre pondéré"
    \resizebox{11cm}{!}{
       \definecolor{dark}{gray}{0.1}
       \begin{tikzpicture}[scale=.8, line width=.5pt, dark]
       \def\width{3.5}
       \def\height{1.5}
       \tikzstyle{noeud}=[fill=white,circle,draw]
       \tikzstyle{poids}=[fill=white,font=\footnotesize,midway]
    \node[noeud] (r) at ({1*\width},{-1.5*\height}) {$$};
    \node[noeud] (ra) at ({2*\width},{-0.5*\height}) {$D$};
     \draw (r) -- (ra) node [poids] {$0,02$};
    \node[noeud] (raa) at ({3*\width},{0*\height}) {$A$};
     \draw (ra) -- (raa) node [poids] {$0,01$};
    \node[noeud] (rab) at ({3*\width},{-1*\height}) {$\overline{A}$};
     \draw (ra) -- (rab) node [poids] {$0,99$};
    \node[noeud] (rb) at ({2*\width},{-2.5*\height}) {$\overline{D}$};
     \draw (r) -- (rb) node [poids] {$0,98$};
    \node[noeud] (rba) at ({3*\width},{-2*\height}) {$A$};
     \draw (rb) -- (rba) node [poids] {$0,97$};
    \node[noeud] (rbb) at ({3*\width},{-3*\height}) {$\overline{A}$};
     \draw (rb) -- (rbb) node [poids] {$0,03$};
       \end{tikzpicture}
      }
   \end{extern}
\end{center}
%##
$p\left(A\right)=p\left(A \cap D\right)+p\left(A \cap \overline{D}\right)=p_{D}\left(A\right)\times p\left(D\right)+p_{\overline{D}}\left(A\right)\times p\left(\overline{D}\right)$
          \par
          $p\left(A\right)=0,01\times 0,02+0,97\times 0,98=95,08$
     \end{enumerate}
\end{corrige}

\end{document}