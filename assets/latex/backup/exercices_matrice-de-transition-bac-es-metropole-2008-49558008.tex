\documentclass[a4paper]{article}

%================================================================================================================================
%
% Packages
%
%================================================================================================================================

\usepackage[T1]{fontenc} 	% pour caractères accentués
\usepackage[utf8]{inputenc}  % encodage utf8
\usepackage[french]{babel}	% langue : français
\usepackage{fourier}			% caractères plus lisibles
\usepackage[dvipsnames]{xcolor} % couleurs
\usepackage{fancyhdr}		% réglage header footer
\usepackage{needspace}		% empêcher sauts de page mal placés
\usepackage{graphicx}		% pour inclure des graphiques
\usepackage{enumitem,cprotect}		% personnalise les listes d'items (nécessaire pour ol, al ...)
\usepackage{hyperref}		% Liens hypertexte
\usepackage{pstricks,pst-all,pst-node,pstricks-add,pst-math,pst-plot,pst-tree,pst-eucl} % pstricks
\usepackage[a4paper,includeheadfoot,top=2cm,left=3cm, bottom=2cm,right=3cm]{geometry} % marges etc.
\usepackage{comment}			% commentaires multilignes
\usepackage{amsmath,environ} % maths (matrices, etc.)
\usepackage{amssymb,makeidx}
\usepackage{bm}				% bold maths
\usepackage{tabularx}		% tableaux
\usepackage{colortbl}		% tableaux en couleur
\usepackage{fontawesome}		% Fontawesome
\usepackage{environ}			% environment with command
\usepackage{fp}				% calculs pour ps-tricks
\usepackage{multido}			% pour ps tricks
\usepackage[np]{numprint}	% formattage nombre
\usepackage{tikz,tkz-tab} 			% package principal TikZ
\usepackage{pgfplots}   % axes
\usepackage{mathrsfs}    % cursives
\usepackage{calc}			% calcul taille boites
\usepackage[scaled=0.875]{helvet} % font sans serif
\usepackage{svg} % svg
\usepackage{scrextend} % local margin
\usepackage{scratch} %scratch
\usepackage{multicol} % colonnes
%\usepackage{infix-RPN,pst-func} % formule en notation polanaise inversée
\usepackage{listings}

%================================================================================================================================
%
% Réglages de base
%
%================================================================================================================================

\lstset{
language=Python,   % R code
literate=
{á}{{\'a}}1
{à}{{\`a}}1
{ã}{{\~a}}1
{é}{{\'e}}1
{è}{{\`e}}1
{ê}{{\^e}}1
{í}{{\'i}}1
{ó}{{\'o}}1
{õ}{{\~o}}1
{ú}{{\'u}}1
{ü}{{\"u}}1
{ç}{{\c{c}}}1
{~}{{ }}1
}


\definecolor{codegreen}{rgb}{0,0.6,0}
\definecolor{codegray}{rgb}{0.5,0.5,0.5}
\definecolor{codepurple}{rgb}{0.58,0,0.82}
\definecolor{backcolour}{rgb}{0.95,0.95,0.92}

\lstdefinestyle{mystyle}{
    backgroundcolor=\color{backcolour},   
    commentstyle=\color{codegreen},
    keywordstyle=\color{magenta},
    numberstyle=\tiny\color{codegray},
    stringstyle=\color{codepurple},
    basicstyle=\ttfamily\footnotesize,
    breakatwhitespace=false,         
    breaklines=true,                 
    captionpos=b,                    
    keepspaces=true,                 
    numbers=left,                    
xleftmargin=2em,
framexleftmargin=2em,            
    showspaces=false,                
    showstringspaces=false,
    showtabs=false,                  
    tabsize=2,
    upquote=true
}

\lstset{style=mystyle}


\lstset{style=mystyle}
\newcommand{\imgdir}{C:/laragon/www/newmc/assets/imgsvg/}
\newcommand{\imgsvgdir}{C:/laragon/www/newmc/assets/imgsvg/}

\definecolor{mcgris}{RGB}{220, 220, 220}% ancien~; pour compatibilité
\definecolor{mcbleu}{RGB}{52, 152, 219}
\definecolor{mcvert}{RGB}{125, 194, 70}
\definecolor{mcmauve}{RGB}{154, 0, 215}
\definecolor{mcorange}{RGB}{255, 96, 0}
\definecolor{mcturquoise}{RGB}{0, 153, 153}
\definecolor{mcrouge}{RGB}{255, 0, 0}
\definecolor{mclightvert}{RGB}{205, 234, 190}

\definecolor{gris}{RGB}{220, 220, 220}
\definecolor{bleu}{RGB}{52, 152, 219}
\definecolor{vert}{RGB}{125, 194, 70}
\definecolor{mauve}{RGB}{154, 0, 215}
\definecolor{orange}{RGB}{255, 96, 0}
\definecolor{turquoise}{RGB}{0, 153, 153}
\definecolor{rouge}{RGB}{255, 0, 0}
\definecolor{lightvert}{RGB}{205, 234, 190}
\setitemize[0]{label=\color{lightvert}  $\bullet$}

\pagestyle{fancy}
\renewcommand{\headrulewidth}{0.2pt}
\fancyhead[L]{maths-cours.fr}
\fancyhead[R]{\thepage}
\renewcommand{\footrulewidth}{0.2pt}
\fancyfoot[C]{}

\newcolumntype{C}{>{\centering\arraybackslash}X}
\newcolumntype{s}{>{\hsize=.35\hsize\arraybackslash}X}

\setlength{\parindent}{0pt}		 
\setlength{\parskip}{3mm}
\setlength{\headheight}{1cm}

\def\ebook{ebook}
\def\book{book}
\def\web{web}
\def\type{web}

\newcommand{\vect}[1]{\overrightarrow{\,\mathstrut#1\,}}

\def\Oij{$\left(\text{O}~;~\vect{\imath},~\vect{\jmath}\right)$}
\def\Oijk{$\left(\text{O}~;~\vect{\imath},~\vect{\jmath},~\vect{k}\right)$}
\def\Ouv{$\left(\text{O}~;~\vect{u},~\vect{v}\right)$}

\hypersetup{breaklinks=true, colorlinks = true, linkcolor = OliveGreen, urlcolor = OliveGreen, citecolor = OliveGreen, pdfauthor={Didier BONNEL - https://www.maths-cours.fr} } % supprime les bordures autour des liens

\renewcommand{\arg}[0]{\text{arg}}

\everymath{\displaystyle}

%================================================================================================================================
%
% Macros - Commandes
%
%================================================================================================================================

\newcommand\meta[2]{    			% Utilisé pour créer le post HTML.
	\def\titre{titre}
	\def\url{url}
	\def\arg{#1}
	\ifx\titre\arg
		\newcommand\maintitle{#2}
		\fancyhead[L]{#2}
		{\Large\sffamily \MakeUppercase{#2}}
		\vspace{1mm}\textcolor{mcvert}{\hrule}
	\fi 
	\ifx\url\arg
		\fancyfoot[L]{\href{https://www.maths-cours.fr#2}{\black \footnotesize{https://www.maths-cours.fr#2}}}
	\fi 
}


\newcommand\TitreC[1]{    		% Titre centré
     \needspace{3\baselineskip}
     \begin{center}\textbf{#1}\end{center}
}

\newcommand\newpar{    		% paragraphe
     \par
}

\newcommand\nosp {    		% commande vide (pas d'espace)
}
\newcommand{\id}[1]{} %ignore

\newcommand\boite[2]{				% Boite simple sans titre
	\vspace{5mm}
	\setlength{\fboxrule}{0.2mm}
	\setlength{\fboxsep}{5mm}	
	\fcolorbox{#1}{#1!3}{\makebox[\linewidth-2\fboxrule-2\fboxsep]{
  		\begin{minipage}[t]{\linewidth-2\fboxrule-4\fboxsep}\setlength{\parskip}{3mm}
  			 #2
  		\end{minipage}
	}}
	\vspace{5mm}
}

\newcommand\CBox[4]{				% Boites
	\vspace{5mm}
	\setlength{\fboxrule}{0.2mm}
	\setlength{\fboxsep}{5mm}
	
	\fcolorbox{#1}{#1!3}{\makebox[\linewidth-2\fboxrule-2\fboxsep]{
		\begin{minipage}[t]{1cm}\setlength{\parskip}{3mm}
	  		\textcolor{#1}{\LARGE{#2}}    
 	 	\end{minipage}  
  		\begin{minipage}[t]{\linewidth-2\fboxrule-4\fboxsep}\setlength{\parskip}{3mm}
			\raisebox{1.2mm}{\normalsize\sffamily{\textcolor{#1}{#3}}}						
  			 #4
  		\end{minipage}
	}}
	\vspace{5mm}
}

\newcommand\cadre[3]{				% Boites convertible html
	\par
	\vspace{2mm}
	\setlength{\fboxrule}{0.1mm}
	\setlength{\fboxsep}{5mm}
	\fcolorbox{#1}{white}{\makebox[\linewidth-2\fboxrule-2\fboxsep]{
  		\begin{minipage}[t]{\linewidth-2\fboxrule-4\fboxsep}\setlength{\parskip}{3mm}
			\raisebox{-2.5mm}{\sffamily \small{\textcolor{#1}{\MakeUppercase{#2}}}}		
			\par		
  			 #3
 	 		\end{minipage}
	}}
		\vspace{2mm}
	\par
}

\newcommand\bloc[3]{				% Boites convertible html sans bordure
     \needspace{2\baselineskip}
     {\sffamily \small{\textcolor{#1}{\MakeUppercase{#2}}}}    
		\par		
  			 #3
		\par
}

\newcommand\CHelp[1]{
     \CBox{Plum}{\faInfoCircle}{À RETENIR}{#1}
}

\newcommand\CUp[1]{
     \CBox{NavyBlue}{\faThumbsOUp}{EN PRATIQUE}{#1}
}

\newcommand\CInfo[1]{
     \CBox{Sepia}{\faArrowCircleRight}{REMARQUE}{#1}
}

\newcommand\CRedac[1]{
     \CBox{PineGreen}{\faEdit}{BIEN R\'EDIGER}{#1}
}

\newcommand\CError[1]{
     \CBox{Red}{\faExclamationTriangle}{ATTENTION}{#1}
}

\newcommand\TitreExo[2]{
\needspace{4\baselineskip}
 {\sffamily\large EXERCICE #1\ (\emph{#2 points})}
\vspace{5mm}
}

\newcommand\img[2]{
          \includegraphics[width=#2\paperwidth]{\imgdir#1}
}

\newcommand\imgsvg[2]{
       \begin{center}   \includegraphics[width=#2\paperwidth]{\imgsvgdir#1} \end{center}
}


\newcommand\Lien[2]{
     \href{#1}{#2 \tiny \faExternalLink}
}
\newcommand\mcLien[2]{
     \href{https~://www.maths-cours.fr/#1}{#2 \tiny \faExternalLink}
}

\newcommand{\euro}{\eurologo{}}

%================================================================================================================================
%
% Macros - Environement
%
%================================================================================================================================

\newenvironment{tex}{ %
}
{%
}

\newenvironment{indente}{ %
	\setlength\parindent{10mm}
}

{
	\setlength\parindent{0mm}
}

\newenvironment{corrige}{%
     \needspace{3\baselineskip}
     \medskip
     \textbf{\textsc{Corrigé}}
     \medskip
}
{
}

\newenvironment{extern}{%
     \begin{center}
     }
     {
     \end{center}
}

\NewEnviron{code}{%
	\par
     \boite{gray}{\texttt{%
     \BODY
     }}
     \par
}

\newenvironment{vbloc}{% boite sans cadre empeche saut de page
     \begin{minipage}[t]{\linewidth}
     }
     {
     \end{minipage}
}
\NewEnviron{h2}{%
    \needspace{3\baselineskip}
    \vspace{0.6cm}
	\noindent \MakeUppercase{\sffamily \large \BODY}
	\vspace{1mm}\textcolor{mcgris}{\hrule}\vspace{0.4cm}
	\par
}{}

\NewEnviron{h3}{%
    \needspace{3\baselineskip}
	\vspace{5mm}
	\textsc{\BODY}
	\par
}

\NewEnviron{margeneg}{ %
\begin{addmargin}[-1cm]{0cm}
\BODY
\end{addmargin}
}

\NewEnviron{html}{%
}

\begin{document}
\meta{url}{/exercices/matrice-de-transition-bac-es-metropole-2008/}
\meta{pid}{1152}
\meta{titre}{Matrice de transition - Bac ES Métropole 2008}
\meta{type}{exercices}
%
\begin{h2}Exercice 2 (5 points)\end{h2}
\textit{(Pour les candidats \textbf{ayant suivi l'enseignement de spécialité})}
\par
Deux fabricants de parfum lancent simultanément leur nouveau produit qu'ils nomment respectivement Aurore et Boréale.
\par
Afin de promouvoir celui-ci, chacun organise une campagne de publicité.
\par
L'un d'eux contrôle l'efficacité de sa campagne par des sondages hebdomadaires.
\par
Chaque semaine, il interroge les mêmes personnes qui toutes se prononcent en faveur de l'un de ces deux produits.
\par
Au début de la campagne, 20 \% des personnes interrogées préfèrent Aurore et les autres préfèrent Boréale.
\par
Les arguments publicitaires font évoluer cette répartition : 10 \% des personnes préférant Aurore et 15 \% des personnes préférant Boréale changent d'avis d'une semaine sur l'autre.
\par
La semaine du début de la campagne est notée semaine 0.
\par
Pour tout entier naturel n, l'état probabiliste de la semaine n est défini par la matrice ligne $P_{n}=\begin{pmatrix}a_{n} & b_{n}\end{pmatrix}$ où $a_{n}$ désigne la probabilité qu'une personne interrogée au hasard préfère Aurore la semaine n et $b_{n}$ la probabilité que cette personne préfère Boréale la semaine n.
\begin{enumerate}
     \item
     Déterminer la matrice ligne $P_{0}$ de l'état probabiliste initial.
     \item
     Représenter la situation par un graphe probabiliste de sommets A et B, A pour Aurore et B pour Boréale.
     \item
     \begin{enumerate}[label=\alph*.]
          \item
          Ecrire la matrice de transition $M$ de ce graphe en respectant l'ordre alphabétique des sommets.
          \item
          Montrer que la matrice ligne $P$, est égale à $\begin{pmatrix}0,3 & 0,7 \end{pmatrix}$.
     \end{enumerate}
     \item
     \begin{enumerate}[label=\alph*.]
          \item
          Exprimer, pour tout entier naturel $n$, $P_{n}$ en fonction de $P_{0}$ et de $n$.
          \item
          En déduire la matrice ligne $P_{3}$. Interpréter ce résultat.
     \end{enumerate}
     \item
     \textit{Dans la question suivante, toute trace de recherche même incomplète ou d'initiative même non fructueuse sera prise en compte dans l'évaluation.}
\par
     Soit $P=\begin{pmatrix}a & b \end{pmatrix}$ la matrice ligne de l'état probabiliste stable.
\begin{enumerate}[label=\alph*.]
          \item
          Déterminer $a$ et $b$.
          \item
          Le parfum Aurore finira-t-îl par être préféré au parfum Boréale ? Justifier.
     \end{enumerate}
\end{enumerate}
\begin{corrige}
     \begin{enumerate}
          \item
          Au début de la campagne 20\% des personnes interrogées préfèrent Aurore donc 80\% préfèrent Boréale.
          \par
          La matrice de l'état initial est donc:
          \par
          $P_{0}=\begin{pmatrix}0,2 & 0,8\end{pmatrix}$
          \item
          En reprenant les données de l'énoncé :


\begin{center}
\imgsvg{copie-0012}{0.3}% alt="graphe probabiliste" style="width:35rem"
\end{center}
          \item
          \begin{enumerate}[label=\alph*.]
               \item
               Le graphe précédent correspond à la matrice de transition suivante :
               \par
               $M=\begin{pmatrix} 0,9 & 0,1  \\ 0,15 & 0,85 \end{pmatrix}$
               \item
               $P_{1}=P_{0}\times M=\begin{pmatrix}0,2 & 0,8\end{pmatrix}\times \begin{pmatrix} 0,9 & 0,1  \\ 0,15 & 0,85 \end{pmatrix}=\begin{pmatrix}0,3 & 0,7\end{pmatrix}$
          \end{enumerate}
          \item
          \begin{enumerate}[label=\alph*.]
               \item
               $P_{m}=P_{0}\times M^{n}$
               \item
               Pour n=3 :
               \par
               $P_{3}=P_{0}\times M^{3}$
               \par
               on calcule (à la calculatrice) :
               \par
               $M^{3}=\begin{pmatrix} 0,76875 & 0,23125  \\ 0,346875 & 0,653125 \end{pmatrix}$
               \par
               et
               \par
               $P_{3}=\begin{pmatrix}0,2 & 0,8\end{pmatrix}\times \begin{pmatrix} 0,76875 & 0,23125  \\ 0,346875 & 0,653125 \end{pmatrix}=\begin{pmatrix}0,43125 & 0,56875\end{pmatrix}$
               \par
               Cela signifie qu'après 3 semaines 43,125\% des personnes interrogées préféreront Aurore et 56,875\% préféreront Boréale.
          \end{enumerate}
          \item
          \begin{enumerate}[label=\alph*.]
               \item
               La matrice $P=\begin{pmatrix}a & b \end{pmatrix}$ de l'état stable  est caractérisé par :
               \par
               $P=PM$ avec $a+b=1$
               \par
               c'est à dire
               \par
               $\begin{pmatrix}a & b \end{pmatrix}=\begin{pmatrix}a & b \end{pmatrix}\times \begin{pmatrix} 0,9 & 0,1  \\ 0,15 & 0,85 \end{pmatrix}$
               \par
               et $a+b=1$
               \par
               On obtient le système :
               \par
               $\left\{ \begin{matrix} a+b=1 \\ 0,9a+0,15b=a \\ 0,1a+0,85b=b \end{matrix}\right.$
                    \par
                    Ce système est équivalent à :
                    \par
                    $\left\{ \begin{matrix} b=1-a \\ 0,9a+0,15\left(1-a\right)=a \\ 0,1a+0,85\left(1-a\right) =1-a\end{matrix}\right.$
                         \par
                         $\left\{ \begin{matrix} b=1-a \\ 0,25a=0,15 \end{matrix}\right.$
                              \par
                              $\left\{ \begin{matrix} a=0,6 \\ b=0,4 \end{matrix}\right.$
                                   \item
                                   D'après la question précédente dans l'état stable 60\% des personnes interrogées préfèrent Aurore (contre 40\% qui préfèrent Boréal).
                                   \par
                                   Au fil du temps on se rapproche de l'état stable donc \textbf{le parfum Aurore finira par être préféré} au parfum Boréal.
                              \end{enumerate}
                         \end{enumerate}
                    \end{corrige}

\end{document}