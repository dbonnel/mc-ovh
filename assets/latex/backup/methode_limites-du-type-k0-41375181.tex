\documentclass[a4paper]{article}

%================================================================================================================================
%
% Packages
%
%================================================================================================================================

\usepackage[T1]{fontenc} 	% pour caractères accentués
\usepackage[utf8]{inputenc}  % encodage utf8
\usepackage[french]{babel}	% langue : français
\usepackage{fourier}			% caractères plus lisibles
\usepackage[dvipsnames]{xcolor} % couleurs
\usepackage{fancyhdr}		% réglage header footer
\usepackage{needspace}		% empêcher sauts de page mal placés
\usepackage{graphicx}		% pour inclure des graphiques
\usepackage{enumitem,cprotect}		% personnalise les listes d'items (nécessaire pour ol, al ...)
\usepackage{hyperref}		% Liens hypertexte
\usepackage{pstricks,pst-all,pst-node,pstricks-add,pst-math,pst-plot,pst-tree,pst-eucl} % pstricks
\usepackage[a4paper,includeheadfoot,top=2cm,left=3cm, bottom=2cm,right=3cm]{geometry} % marges etc.
\usepackage{comment}			% commentaires multilignes
\usepackage{amsmath,environ} % maths (matrices, etc.)
\usepackage{amssymb,makeidx}
\usepackage{bm}				% bold maths
\usepackage{tabularx}		% tableaux
\usepackage{colortbl}		% tableaux en couleur
\usepackage{fontawesome}		% Fontawesome
\usepackage{environ}			% environment with command
\usepackage{fp}				% calculs pour ps-tricks
\usepackage{multido}			% pour ps tricks
\usepackage[np]{numprint}	% formattage nombre
\usepackage{tikz,tkz-tab} 			% package principal TikZ
\usepackage{pgfplots}   % axes
\usepackage{mathrsfs}    % cursives
\usepackage{calc}			% calcul taille boites
\usepackage[scaled=0.875]{helvet} % font sans serif
\usepackage{svg} % svg
\usepackage{scrextend} % local margin
\usepackage{scratch} %scratch
\usepackage{multicol} % colonnes
%\usepackage{infix-RPN,pst-func} % formule en notation polanaise inversée
\usepackage{listings}

%================================================================================================================================
%
% Réglages de base
%
%================================================================================================================================

\lstset{
language=Python,   % R code
literate=
{á}{{\'a}}1
{à}{{\`a}}1
{ã}{{\~a}}1
{é}{{\'e}}1
{è}{{\`e}}1
{ê}{{\^e}}1
{í}{{\'i}}1
{ó}{{\'o}}1
{õ}{{\~o}}1
{ú}{{\'u}}1
{ü}{{\"u}}1
{ç}{{\c{c}}}1
{~}{{ }}1
}


\definecolor{codegreen}{rgb}{0,0.6,0}
\definecolor{codegray}{rgb}{0.5,0.5,0.5}
\definecolor{codepurple}{rgb}{0.58,0,0.82}
\definecolor{backcolour}{rgb}{0.95,0.95,0.92}

\lstdefinestyle{mystyle}{
    backgroundcolor=\color{backcolour},   
    commentstyle=\color{codegreen},
    keywordstyle=\color{magenta},
    numberstyle=\tiny\color{codegray},
    stringstyle=\color{codepurple},
    basicstyle=\ttfamily\footnotesize,
    breakatwhitespace=false,         
    breaklines=true,                 
    captionpos=b,                    
    keepspaces=true,                 
    numbers=left,                    
xleftmargin=2em,
framexleftmargin=2em,            
    showspaces=false,                
    showstringspaces=false,
    showtabs=false,                  
    tabsize=2,
    upquote=true
}

\lstset{style=mystyle}


\lstset{style=mystyle}
\newcommand{\imgdir}{C:/laragon/www/newmc/assets/imgsvg/}
\newcommand{\imgsvgdir}{C:/laragon/www/newmc/assets/imgsvg/}

\definecolor{mcgris}{RGB}{220, 220, 220}% ancien~; pour compatibilité
\definecolor{mcbleu}{RGB}{52, 152, 219}
\definecolor{mcvert}{RGB}{125, 194, 70}
\definecolor{mcmauve}{RGB}{154, 0, 215}
\definecolor{mcorange}{RGB}{255, 96, 0}
\definecolor{mcturquoise}{RGB}{0, 153, 153}
\definecolor{mcrouge}{RGB}{255, 0, 0}
\definecolor{mclightvert}{RGB}{205, 234, 190}

\definecolor{gris}{RGB}{220, 220, 220}
\definecolor{bleu}{RGB}{52, 152, 219}
\definecolor{vert}{RGB}{125, 194, 70}
\definecolor{mauve}{RGB}{154, 0, 215}
\definecolor{orange}{RGB}{255, 96, 0}
\definecolor{turquoise}{RGB}{0, 153, 153}
\definecolor{rouge}{RGB}{255, 0, 0}
\definecolor{lightvert}{RGB}{205, 234, 190}
\setitemize[0]{label=\color{lightvert}  $\bullet$}

\pagestyle{fancy}
\renewcommand{\headrulewidth}{0.2pt}
\fancyhead[L]{maths-cours.fr}
\fancyhead[R]{\thepage}
\renewcommand{\footrulewidth}{0.2pt}
\fancyfoot[C]{}

\newcolumntype{C}{>{\centering\arraybackslash}X}
\newcolumntype{s}{>{\hsize=.35\hsize\arraybackslash}X}

\setlength{\parindent}{0pt}		 
\setlength{\parskip}{3mm}
\setlength{\headheight}{1cm}

\def\ebook{ebook}
\def\book{book}
\def\web{web}
\def\type{web}

\newcommand{\vect}[1]{\overrightarrow{\,\mathstrut#1\,}}

\def\Oij{$\left(\text{O}~;~\vect{\imath},~\vect{\jmath}\right)$}
\def\Oijk{$\left(\text{O}~;~\vect{\imath},~\vect{\jmath},~\vect{k}\right)$}
\def\Ouv{$\left(\text{O}~;~\vect{u},~\vect{v}\right)$}

\hypersetup{breaklinks=true, colorlinks = true, linkcolor = OliveGreen, urlcolor = OliveGreen, citecolor = OliveGreen, pdfauthor={Didier BONNEL - https://www.maths-cours.fr} } % supprime les bordures autour des liens

\renewcommand{\arg}[0]{\text{arg}}

\everymath{\displaystyle}

%================================================================================================================================
%
% Macros - Commandes
%
%================================================================================================================================

\newcommand\meta[2]{    			% Utilisé pour créer le post HTML.
	\def\titre{titre}
	\def\url{url}
	\def\arg{#1}
	\ifx\titre\arg
		\newcommand\maintitle{#2}
		\fancyhead[L]{#2}
		{\Large\sffamily \MakeUppercase{#2}}
		\vspace{1mm}\textcolor{mcvert}{\hrule}
	\fi 
	\ifx\url\arg
		\fancyfoot[L]{\href{https://www.maths-cours.fr#2}{\black \footnotesize{https://www.maths-cours.fr#2}}}
	\fi 
}


\newcommand\TitreC[1]{    		% Titre centré
     \needspace{3\baselineskip}
     \begin{center}\textbf{#1}\end{center}
}

\newcommand\newpar{    		% paragraphe
     \par
}

\newcommand\nosp {    		% commande vide (pas d'espace)
}
\newcommand{\id}[1]{} %ignore

\newcommand\boite[2]{				% Boite simple sans titre
	\vspace{5mm}
	\setlength{\fboxrule}{0.2mm}
	\setlength{\fboxsep}{5mm}	
	\fcolorbox{#1}{#1!3}{\makebox[\linewidth-2\fboxrule-2\fboxsep]{
  		\begin{minipage}[t]{\linewidth-2\fboxrule-4\fboxsep}\setlength{\parskip}{3mm}
  			 #2
  		\end{minipage}
	}}
	\vspace{5mm}
}

\newcommand\CBox[4]{				% Boites
	\vspace{5mm}
	\setlength{\fboxrule}{0.2mm}
	\setlength{\fboxsep}{5mm}
	
	\fcolorbox{#1}{#1!3}{\makebox[\linewidth-2\fboxrule-2\fboxsep]{
		\begin{minipage}[t]{1cm}\setlength{\parskip}{3mm}
	  		\textcolor{#1}{\LARGE{#2}}    
 	 	\end{minipage}  
  		\begin{minipage}[t]{\linewidth-2\fboxrule-4\fboxsep}\setlength{\parskip}{3mm}
			\raisebox{1.2mm}{\normalsize\sffamily{\textcolor{#1}{#3}}}						
  			 #4
  		\end{minipage}
	}}
	\vspace{5mm}
}

\newcommand\cadre[3]{				% Boites convertible html
	\par
	\vspace{2mm}
	\setlength{\fboxrule}{0.1mm}
	\setlength{\fboxsep}{5mm}
	\fcolorbox{#1}{white}{\makebox[\linewidth-2\fboxrule-2\fboxsep]{
  		\begin{minipage}[t]{\linewidth-2\fboxrule-4\fboxsep}\setlength{\parskip}{3mm}
			\raisebox{-2.5mm}{\sffamily \small{\textcolor{#1}{\MakeUppercase{#2}}}}		
			\par		
  			 #3
 	 		\end{minipage}
	}}
		\vspace{2mm}
	\par
}

\newcommand\bloc[3]{				% Boites convertible html sans bordure
     \needspace{2\baselineskip}
     {\sffamily \small{\textcolor{#1}{\MakeUppercase{#2}}}}    
		\par		
  			 #3
		\par
}

\newcommand\CHelp[1]{
     \CBox{Plum}{\faInfoCircle}{À RETENIR}{#1}
}

\newcommand\CUp[1]{
     \CBox{NavyBlue}{\faThumbsOUp}{EN PRATIQUE}{#1}
}

\newcommand\CInfo[1]{
     \CBox{Sepia}{\faArrowCircleRight}{REMARQUE}{#1}
}

\newcommand\CRedac[1]{
     \CBox{PineGreen}{\faEdit}{BIEN R\'EDIGER}{#1}
}

\newcommand\CError[1]{
     \CBox{Red}{\faExclamationTriangle}{ATTENTION}{#1}
}

\newcommand\TitreExo[2]{
\needspace{4\baselineskip}
 {\sffamily\large EXERCICE #1\ (\emph{#2 points})}
\vspace{5mm}
}

\newcommand\img[2]{
          \includegraphics[width=#2\paperwidth]{\imgdir#1}
}

\newcommand\imgsvg[2]{
       \begin{center}   \includegraphics[width=#2\paperwidth]{\imgsvgdir#1} \end{center}
}


\newcommand\Lien[2]{
     \href{#1}{#2 \tiny \faExternalLink}
}
\newcommand\mcLien[2]{
     \href{https~://www.maths-cours.fr/#1}{#2 \tiny \faExternalLink}
}

\newcommand{\euro}{\eurologo{}}

%================================================================================================================================
%
% Macros - Environement
%
%================================================================================================================================

\newenvironment{tex}{ %
}
{%
}

\newenvironment{indente}{ %
	\setlength\parindent{10mm}
}

{
	\setlength\parindent{0mm}
}

\newenvironment{corrige}{%
     \needspace{3\baselineskip}
     \medskip
     \textbf{\textsc{Corrigé}}
     \medskip
}
{
}

\newenvironment{extern}{%
     \begin{center}
     }
     {
     \end{center}
}

\NewEnviron{code}{%
	\par
     \boite{gray}{\texttt{%
     \BODY
     }}
     \par
}

\newenvironment{vbloc}{% boite sans cadre empeche saut de page
     \begin{minipage}[t]{\linewidth}
     }
     {
     \end{minipage}
}
\NewEnviron{h2}{%
    \needspace{3\baselineskip}
    \vspace{0.6cm}
	\noindent \MakeUppercase{\sffamily \large \BODY}
	\vspace{1mm}\textcolor{mcgris}{\hrule}\vspace{0.4cm}
	\par
}{}

\NewEnviron{h3}{%
    \needspace{3\baselineskip}
	\vspace{5mm}
	\textsc{\BODY}
	\par
}

\NewEnviron{margeneg}{ %
\begin{addmargin}[-1cm]{0cm}
\BODY
\end{addmargin}
}

\NewEnviron{html}{%
}

\begin{document}
\meta{url}{/methode/limites-du-type-k0/}
\meta{pid}{1302}
\meta{pi_}{1302}
\meta{titre}{Limites du type "k/0"}
\meta{type}{methode}
\cadre{vert}{Situation}{%id=s100
     On cherche à calculer la limite d'une fraction rationnelle lorsque$x$ tend vers une valeur $a$ qui annule le dénominateur; par exemple $\lim\limits_{x\rightarrow 1} \frac{x+2}{x^{2}-1}. $
}
\cadre{rouge}{Méthode}{%id=m100
     \begin{itemize}
          \item Si on a affaire à une limite du type \og $\frac{0}{0}$ \fg{} (forme indéterminée), on lève l'indétermination en factorisant le numérateur et le dénominateur puis en simplifiant la fraction
          \item Si on a affaire à une limite du type \og  $\frac{k}{0}$ \fg{} avec $k \neq 0$:\begin{itemize}
               \item on distingue les limites à gauche et à droite :
               \newpar
               $\lim\limits_{x\rightarrow  a^-} f\left(x\right)$ et $\lim\limits_{x\rightarrow  a^+} f\left(x\right)$
               \item les limites seront égales à $+\infty $ ou $-\infty $
               \item pour déterminer le signe de la limite on étudie le signe du quotient. On peut toutefois se limiter à l'étude de signe au voisinage de $a$ (voir exemple 3)
          \end{itemize}
     \end{itemize}
}
\par
\bloc{orange}{Exemple 1}{%id=e100
     Calculer $\lim\limits_{x\rightarrow 2} \frac{x^{2}-3x+2}{x^{2}-4}$
     \newpar
     En remplaçant $x$ par 2 dans la fraction rationnelle on obtient \og $\frac{0}{0}$ \fg{}.
     \newpar
     On lève l'indétermination en simplifiant la fraction.
     \newpar
     2 est racine de $x^{2}-3x+2$ comme on vient de le voir. Le produit des racines vaut $\frac{c}{a}=2$ donc l'autre racine est 1 (on peut, si l'on préfère, calculer le discriminant puis les racines, mais c'est plus long…).
     \newpar
     $x^{2}-3x+2$ peut donc se factoriser sous la forme $\left(x-1\right)\left(x-2\right)$.
     \newpar
     $x^{2}-4=\left(x-2\right)\left(x+2\right)$ (identité remarquable)
     \newpar
     Donc :
     \newpar
     $\lim\limits_{x\rightarrow 2} \frac{x^{2}-3x+2}{x^{2}-4} = \lim\limits_{x\rightarrow 2} \frac{\left(x-1\right)\left(x-2\right)}{\left(x-2\right)\left(x+2\right)}=\lim\limits_{x\rightarrow 2} \frac{x-1}{x+2} = \frac{1}{4}$
}
\bloc{orange}{Exemple 2}{%id=e200
     Calculer $\lim\limits_{x\rightarrow -1} \frac{2}{1+x}$
     \newpar
     En remplaçant $x$ par -1 dans la fraction rationnelle on obtient \og $\frac{2}{0}$  \fg{}.
     \newpar
     La limite est donc infinie.
     \newpar
     Pour \textbf{l'étude du signe} on distingue les limites à gauche et à droite.
     \newpar
     Le numérateur est toujours positif.
     \begin{itemize}
          \item si $x < -1$, $1+x$ est strictement négatif
          \item si $x > -1$, $1+x$ est strictement positif donc :
     \end{itemize}
     $\lim\limits_{x\rightarrow -1^-} \frac{2}{1+x}=-\infty $
     \newpar
     $\lim\limits_{x\rightarrow -1^+} \frac{2}{1+x}=+\infty $
}
\bloc{orange}{Exemple 3}{%id=e300
     Calculer $\lim\limits_{x\rightarrow 0} \frac{x^{3}+x-3}{x^{2}-x}$
     \newpar
     En «remplaçant $x$ par 0» dans la fraction rationnelle on obtient «$-\frac{3}{0}$».
     \newpar
     La limite sera donc infinie. On distingue les limites à gauche et à droite.
     \newpar
     Il n'est pas facile de factoriser le numérateur qui est du troisième degré. Heureusement, cela ne sera pas nécessaire ici !
     \newpar
     On ne va pas construire le tableau de signes sur $\mathbb{R}$ tout entier mais \textbf{seulement au voisinage de zéro}.
     \newpar
     Si $x$ est proche de zéro le numérateur sera proche de $-3$ donc négatif.
     \newpar
     Le dénominateur se factorise $x^{2}-x=x\left(x-1\right)$ et $x-1$ est proche de $-1$ (donc négatif) lorsque $x$ est proche de 0.
     \newpar
     On obtient alors le tableau de signe au voisinage de $0$ :
     \begin{center}
          \begin{extern}%width="500" alt="Exemple tableau de signes d'un quotient"
               \resizebox{11cm}{!}{
                    \begin{tikzpicture}[scale=0.875]
                         % Styles
                         \tikzstyle{cadre}=[thin]
                         \tikzstyle{fleche}=[->,>=latex,thin]
                         \tikzstyle{nondefini}=[lightgray]
                         % Dimensions Modifiables
                         \def\Lrg{1.5}
                         \def\HtX{1.2}
                         \def\HtY{0.5}
                         % Dimensions Calculées
                         \def\lignex{-0.5*\HtX}
                         \def\lignea{-1.5*\HtX}
                         \def\ligneb{-2.5*\HtX}
                         \def\lignec{-3.5*\HtX}
                         \def\ligned{-4.5*\HtX}
                         \def\separateur{-0.5*\Lrg}
                         % Largeur du tableau
                         \def\gauche{-3.1*\Lrg}
                         \def\droite{4.5*\Lrg}
                         % Hauteur du tableau
                         \def\haut{0.5*\HtX}
                         \def\bas{-2.5*\HtX-2*\HtY}
                         % Pointillés
                         \draw[gray] (2*\Lrg,\lignex) -- (2*\Lrg,\lignec); les tensions
                         \draw[double distance=2pt] (2*\Lrg,\lignec) -- (2*\Lrg,\ligned);
                         % Ligne de l'abscisse : x
                         \node at (-1.8*\Lrg,0) {$x$};
                         \node at (0*\Lrg,0) {$\cdots$};
                         \node at (2*\Lrg,0) {$0$};
                         \node at (4*\Lrg,0) {$\cdots$};
                         % Ligne a
                         \node at (-1.8*\Lrg,-1*\HtX) {$x^{3}+x-3$};
                         \node at (0*\Lrg,-1*\HtX) {$ $};
                         \node at (1*\Lrg,-1*\HtX) {$-$};
                         \node at (2*\Lrg,-1*\HtX) {$ $};
                         \node at (3*\Lrg,-1*\HtX) {$-$};
                         % Ligne b
                         \node at (-1.8*\Lrg,-2*\HtX) {$x$};
                         \node at (2*\Lrg,-2*\HtX) {$ $};
                         \node at (1*\Lrg,-2*\HtX) {$-$};
                         \node at (2*\Lrg,-2*\HtX) {$0$};
                         \node at (3*\Lrg,-2*\HtX) {$+$};
                         % Ligne c
                         \node at (-1.8*\Lrg,-3*\HtX) {$x-1$};
                         \node at (0*\Lrg,-3*\HtX) {$ $};
                         \node at (1*\Lrg,-3*\HtX) {$-$};
                         \node at (2*\Lrg,-3*\HtX) {$ $};
                         \node at (3*\Lrg,-3*\HtX) {$-$};
                         % Ligne d
                         \node at (-1.8*\Lrg,-4*\HtX) {$\frac{x^{3}+x-3}{x^{2}-x}$};
                         \node at (0*\Lrg,-4*\HtX) {$ $};
                         \node at (1*\Lrg,-4*\HtX) {$-$};
                         \node at (2*\Lrg,-4*\HtX) {$ $};
                         \node at (3*\Lrg,-4*\HtX) {$+$};
                         % Encadrement
                         \draw[cadre] (\separateur,\haut) -- (\separateur, \ligned);
                         \draw[cadre] (\gauche,\haut) rectangle  (\droite, \ligned);
                         \draw[cadre] (\gauche,\lignex) -- (\droite,\lignex);
                         \draw[cadre] (\gauche,\lignea) -- (\droite,\lignea);
                         \draw[cadre] (\gauche,\ligneb) -- (\droite,\ligneb);
                         \draw[cadre] (\gauche,\lignec) -- (\droite,\lignec);
                    \end{tikzpicture}
               }
          \end{extern}
     \end{center}
     Donc~:
     \newpar
     $\lim\limits_{x\rightarrow 0^-}\frac{x^{3}+x-3}{x^{2}-x}=-\infty $
     \newpar
     $\lim\limits_{x\rightarrow 0^+}\frac{x^{3}+x-3}{x^{2}-x}=+\infty $
}
\bloc{cyan}{Remarque}{%id=r400
     Une petite astuce pour vérifier votre résultat \textbf{à la calculatrice}.
     \newpar
     Pour avoir une idée de la valeur de $\lim\limits_{x\rightarrow  a}f\left(x\right)$, donnez à $x$ des valeurs proches de $a$ et calculer $f\left(x\right)$
     \newpar
     Par exemple, pour l'exemple 3, on saisit la fonction $x\mapsto \frac{x^{3}+x-3}{x^{2}-x}$ et on calcule :
     \newpar
     $f\left(-0,0000000001\right)\approx -3\times 10^{10}$
     \newpar
     $f\left(0,0000000001\right)\approx 3\times 10^{10}$
     \newpar
     ce qui confirme les valeurs ( et surtout les signes ! ) que nous avons trouvées ($-\infty $ et $+\infty $).
}

\end{document}