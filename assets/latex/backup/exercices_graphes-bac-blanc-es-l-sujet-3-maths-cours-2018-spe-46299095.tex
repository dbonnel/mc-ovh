\documentclass[a4paper]{article}

%================================================================================================================================
%
% Packages
%
%================================================================================================================================

\usepackage[T1]{fontenc} 	% pour caractères accentués
\usepackage[utf8]{inputenc}  % encodage utf8
\usepackage[french]{babel}	% langue : français
\usepackage{fourier}			% caractères plus lisibles
\usepackage[dvipsnames]{xcolor} % couleurs
\usepackage{fancyhdr}		% réglage header footer
\usepackage{needspace}		% empêcher sauts de page mal placés
\usepackage{graphicx}		% pour inclure des graphiques
\usepackage{enumitem,cprotect}		% personnalise les listes d'items (nécessaire pour ol, al ...)
\usepackage{hyperref}		% Liens hypertexte
\usepackage{pstricks,pst-all,pst-node,pstricks-add,pst-math,pst-plot,pst-tree,pst-eucl} % pstricks
\usepackage[a4paper,includeheadfoot,top=2cm,left=3cm, bottom=2cm,right=3cm]{geometry} % marges etc.
\usepackage{comment}			% commentaires multilignes
\usepackage{amsmath,environ} % maths (matrices, etc.)
\usepackage{amssymb,makeidx}
\usepackage{bm}				% bold maths
\usepackage{tabularx}		% tableaux
\usepackage{colortbl}		% tableaux en couleur
\usepackage{fontawesome}		% Fontawesome
\usepackage{environ}			% environment with command
\usepackage{fp}				% calculs pour ps-tricks
\usepackage{multido}			% pour ps tricks
\usepackage[np]{numprint}	% formattage nombre
\usepackage{tikz,tkz-tab} 			% package principal TikZ
\usepackage{pgfplots}   % axes
\usepackage{mathrsfs}    % cursives
\usepackage{calc}			% calcul taille boites
\usepackage[scaled=0.875]{helvet} % font sans serif
\usepackage{svg} % svg
\usepackage{scrextend} % local margin
\usepackage{scratch} %scratch
\usepackage{multicol} % colonnes
%\usepackage{infix-RPN,pst-func} % formule en notation polanaise inversée
\usepackage{listings}

%================================================================================================================================
%
% Réglages de base
%
%================================================================================================================================

\lstset{
language=Python,   % R code
literate=
{á}{{\'a}}1
{à}{{\`a}}1
{ã}{{\~a}}1
{é}{{\'e}}1
{è}{{\`e}}1
{ê}{{\^e}}1
{í}{{\'i}}1
{ó}{{\'o}}1
{õ}{{\~o}}1
{ú}{{\'u}}1
{ü}{{\"u}}1
{ç}{{\c{c}}}1
{~}{{ }}1
}


\definecolor{codegreen}{rgb}{0,0.6,0}
\definecolor{codegray}{rgb}{0.5,0.5,0.5}
\definecolor{codepurple}{rgb}{0.58,0,0.82}
\definecolor{backcolour}{rgb}{0.95,0.95,0.92}

\lstdefinestyle{mystyle}{
    backgroundcolor=\color{backcolour},   
    commentstyle=\color{codegreen},
    keywordstyle=\color{magenta},
    numberstyle=\tiny\color{codegray},
    stringstyle=\color{codepurple},
    basicstyle=\ttfamily\footnotesize,
    breakatwhitespace=false,         
    breaklines=true,                 
    captionpos=b,                    
    keepspaces=true,                 
    numbers=left,                    
xleftmargin=2em,
framexleftmargin=2em,            
    showspaces=false,                
    showstringspaces=false,
    showtabs=false,                  
    tabsize=2,
    upquote=true
}

\lstset{style=mystyle}


\lstset{style=mystyle}
\newcommand{\imgdir}{C:/laragon/www/newmc/assets/imgsvg/}
\newcommand{\imgsvgdir}{C:/laragon/www/newmc/assets/imgsvg/}

\definecolor{mcgris}{RGB}{220, 220, 220}% ancien~; pour compatibilité
\definecolor{mcbleu}{RGB}{52, 152, 219}
\definecolor{mcvert}{RGB}{125, 194, 70}
\definecolor{mcmauve}{RGB}{154, 0, 215}
\definecolor{mcorange}{RGB}{255, 96, 0}
\definecolor{mcturquoise}{RGB}{0, 153, 153}
\definecolor{mcrouge}{RGB}{255, 0, 0}
\definecolor{mclightvert}{RGB}{205, 234, 190}

\definecolor{gris}{RGB}{220, 220, 220}
\definecolor{bleu}{RGB}{52, 152, 219}
\definecolor{vert}{RGB}{125, 194, 70}
\definecolor{mauve}{RGB}{154, 0, 215}
\definecolor{orange}{RGB}{255, 96, 0}
\definecolor{turquoise}{RGB}{0, 153, 153}
\definecolor{rouge}{RGB}{255, 0, 0}
\definecolor{lightvert}{RGB}{205, 234, 190}
\setitemize[0]{label=\color{lightvert}  $\bullet$}

\pagestyle{fancy}
\renewcommand{\headrulewidth}{0.2pt}
\fancyhead[L]{maths-cours.fr}
\fancyhead[R]{\thepage}
\renewcommand{\footrulewidth}{0.2pt}
\fancyfoot[C]{}

\newcolumntype{C}{>{\centering\arraybackslash}X}
\newcolumntype{s}{>{\hsize=.35\hsize\arraybackslash}X}

\setlength{\parindent}{0pt}		 
\setlength{\parskip}{3mm}
\setlength{\headheight}{1cm}

\def\ebook{ebook}
\def\book{book}
\def\web{web}
\def\type{web}

\newcommand{\vect}[1]{\overrightarrow{\,\mathstrut#1\,}}

\def\Oij{$\left(\text{O}~;~\vect{\imath},~\vect{\jmath}\right)$}
\def\Oijk{$\left(\text{O}~;~\vect{\imath},~\vect{\jmath},~\vect{k}\right)$}
\def\Ouv{$\left(\text{O}~;~\vect{u},~\vect{v}\right)$}

\hypersetup{breaklinks=true, colorlinks = true, linkcolor = OliveGreen, urlcolor = OliveGreen, citecolor = OliveGreen, pdfauthor={Didier BONNEL - https://www.maths-cours.fr} } % supprime les bordures autour des liens

\renewcommand{\arg}[0]{\text{arg}}

\everymath{\displaystyle}

%================================================================================================================================
%
% Macros - Commandes
%
%================================================================================================================================

\newcommand\meta[2]{    			% Utilisé pour créer le post HTML.
	\def\titre{titre}
	\def\url{url}
	\def\arg{#1}
	\ifx\titre\arg
		\newcommand\maintitle{#2}
		\fancyhead[L]{#2}
		{\Large\sffamily \MakeUppercase{#2}}
		\vspace{1mm}\textcolor{mcvert}{\hrule}
	\fi 
	\ifx\url\arg
		\fancyfoot[L]{\href{https://www.maths-cours.fr#2}{\black \footnotesize{https://www.maths-cours.fr#2}}}
	\fi 
}


\newcommand\TitreC[1]{    		% Titre centré
     \needspace{3\baselineskip}
     \begin{center}\textbf{#1}\end{center}
}

\newcommand\newpar{    		% paragraphe
     \par
}

\newcommand\nosp {    		% commande vide (pas d'espace)
}
\newcommand{\id}[1]{} %ignore

\newcommand\boite[2]{				% Boite simple sans titre
	\vspace{5mm}
	\setlength{\fboxrule}{0.2mm}
	\setlength{\fboxsep}{5mm}	
	\fcolorbox{#1}{#1!3}{\makebox[\linewidth-2\fboxrule-2\fboxsep]{
  		\begin{minipage}[t]{\linewidth-2\fboxrule-4\fboxsep}\setlength{\parskip}{3mm}
  			 #2
  		\end{minipage}
	}}
	\vspace{5mm}
}

\newcommand\CBox[4]{				% Boites
	\vspace{5mm}
	\setlength{\fboxrule}{0.2mm}
	\setlength{\fboxsep}{5mm}
	
	\fcolorbox{#1}{#1!3}{\makebox[\linewidth-2\fboxrule-2\fboxsep]{
		\begin{minipage}[t]{1cm}\setlength{\parskip}{3mm}
	  		\textcolor{#1}{\LARGE{#2}}    
 	 	\end{minipage}  
  		\begin{minipage}[t]{\linewidth-2\fboxrule-4\fboxsep}\setlength{\parskip}{3mm}
			\raisebox{1.2mm}{\normalsize\sffamily{\textcolor{#1}{#3}}}						
  			 #4
  		\end{minipage}
	}}
	\vspace{5mm}
}

\newcommand\cadre[3]{				% Boites convertible html
	\par
	\vspace{2mm}
	\setlength{\fboxrule}{0.1mm}
	\setlength{\fboxsep}{5mm}
	\fcolorbox{#1}{white}{\makebox[\linewidth-2\fboxrule-2\fboxsep]{
  		\begin{minipage}[t]{\linewidth-2\fboxrule-4\fboxsep}\setlength{\parskip}{3mm}
			\raisebox{-2.5mm}{\sffamily \small{\textcolor{#1}{\MakeUppercase{#2}}}}		
			\par		
  			 #3
 	 		\end{minipage}
	}}
		\vspace{2mm}
	\par
}

\newcommand\bloc[3]{				% Boites convertible html sans bordure
     \needspace{2\baselineskip}
     {\sffamily \small{\textcolor{#1}{\MakeUppercase{#2}}}}    
		\par		
  			 #3
		\par
}

\newcommand\CHelp[1]{
     \CBox{Plum}{\faInfoCircle}{À RETENIR}{#1}
}

\newcommand\CUp[1]{
     \CBox{NavyBlue}{\faThumbsOUp}{EN PRATIQUE}{#1}
}

\newcommand\CInfo[1]{
     \CBox{Sepia}{\faArrowCircleRight}{REMARQUE}{#1}
}

\newcommand\CRedac[1]{
     \CBox{PineGreen}{\faEdit}{BIEN R\'EDIGER}{#1}
}

\newcommand\CError[1]{
     \CBox{Red}{\faExclamationTriangle}{ATTENTION}{#1}
}

\newcommand\TitreExo[2]{
\needspace{4\baselineskip}
 {\sffamily\large EXERCICE #1\ (\emph{#2 points})}
\vspace{5mm}
}

\newcommand\img[2]{
          \includegraphics[width=#2\paperwidth]{\imgdir#1}
}

\newcommand\imgsvg[2]{
       \begin{center}   \includegraphics[width=#2\paperwidth]{\imgsvgdir#1} \end{center}
}


\newcommand\Lien[2]{
     \href{#1}{#2 \tiny \faExternalLink}
}
\newcommand\mcLien[2]{
     \href{https~://www.maths-cours.fr/#1}{#2 \tiny \faExternalLink}
}

\newcommand{\euro}{\eurologo{}}

%================================================================================================================================
%
% Macros - Environement
%
%================================================================================================================================

\newenvironment{tex}{ %
}
{%
}

\newenvironment{indente}{ %
	\setlength\parindent{10mm}
}

{
	\setlength\parindent{0mm}
}

\newenvironment{corrige}{%
     \needspace{3\baselineskip}
     \medskip
     \textbf{\textsc{Corrigé}}
     \medskip
}
{
}

\newenvironment{extern}{%
     \begin{center}
     }
     {
     \end{center}
}

\NewEnviron{code}{%
	\par
     \boite{gray}{\texttt{%
     \BODY
     }}
     \par
}

\newenvironment{vbloc}{% boite sans cadre empeche saut de page
     \begin{minipage}[t]{\linewidth}
     }
     {
     \end{minipage}
}
\NewEnviron{h2}{%
    \needspace{3\baselineskip}
    \vspace{0.6cm}
	\noindent \MakeUppercase{\sffamily \large \BODY}
	\vspace{1mm}\textcolor{mcgris}{\hrule}\vspace{0.4cm}
	\par
}{}

\NewEnviron{h3}{%
    \needspace{3\baselineskip}
	\vspace{5mm}
	\textsc{\BODY}
	\par
}

\NewEnviron{margeneg}{ %
\begin{addmargin}[-1cm]{0cm}
\BODY
\end{addmargin}
}

\NewEnviron{html}{%
}

\begin{document}
\meta{url}{/exercices/graphes-bac-blanc-es-l-sujet-3-maths-cours-2018-spe/}
\meta{pid}{10479}
\meta{titre}{Graphes - Bac blanc ES/L Sujet 3 - Maths-cours 2018 (spé)}
\meta{type}{exercices}
%
\begin{h2}Exercice 3 (5 points)\end{h2}
\par
\textbf{Candidats ayant suivi l'enseignement de spécialité}
\par
\emph{Pour chacune des cinq affirmations suivantes, indiquer si elle est vraie ou fausse en justifiant la réponse.\\ Il est attribué un point par réponse exacte correctement justifiée.\\ \textbf{Une réponse non justifiée n'est pas prise en compte.}}\index{Vrai--Faux}
\par
On modélise le plan d'un village à l'aide du graphe (G) ci-dessous :
\par
\begin{center}
     \begin{extern}%width="350" alt="modélisation à l'aide d'un graphe"
          \psset{unit=0.7cm}
          \begin{pspicture}(12,8)
               \rput(0.75,7.5){\circlenode{A}{A}}
               \rput(7.2,7.5){\circlenode{B}{B}}
               \rput(11.3,5.5){\circlenode{C}{C}}
               \rput(10.3,2){\circlenode{D}{D}}
               \rput(0.3,2.7){\circlenode{E}{E}}
               \rput(6.7,4.3){\circlenode{F}{F}}
               \ncline{A}{B}
               \ncline{A}{E}
               \ncline{B}{E}
               \ncline{B}{C}
               \ncline{B}{D}
               \ncline{B}{F}
               \ncline{C}{D}
               \ncline{D}{E}
               \ncline{D}{F}
               \ncline{E}{F}
          \end{pspicture}
     \end{extern}
\end{center}
\par
Les sommets du graphe (G) représentent les carrefours et les arêtes du graphe schématisent les routes reliant ces carrefours.
\par
\begin{itemize}
     \item %1
     \textbf{Affirmation 1 :}\quad Le graphe (G) est connexe.
     \item %2
     \textbf{Affirmation 2 :}\quad Le graphe (G) contient un sous-graphe complet d'ordre 4.
     \item %3
     \textbf{Affirmation 3 :}\quad Une personne peut parcourir toutes les routes du village sans emprunter plusieurs fois la même route.
     \item %4
     \textbf{Affirmation 4 :}\quad Il y a exactement 5 trajets de trois routes reliant les carrefours C et E.\\
     \textit{On pourra utiliser la calculatrice pour justifier la réponse à l'aide d'un calcul matriciel.}
     \par
\end{itemize}
On pondère le graphe (G) par les longueurs, en centaines de mètres, de chacune des routes :
\par
\begin{center}
     \begin{extern}%width="350" alt=" graphe pondéré"
          \psset{unit=0.7cm}
          \begin{pspicture}(12,8)
               \rput(0.75,7.5){\circlenode{A}{A}}
               \rput(7.2,7.5){\circlenode{B}{B}}
               \rput(11.3,5.5){\circlenode{C}{C}}
               \rput(10.3,2){\circlenode{D}{D}}
               \rput(0.3,2.7){\circlenode{E}{E}}
               \rput(6.7,4.3){\circlenode{F}{F}}
               \ncline{A}{B}\ncput*[nrot=:U]{3}
               \ncline{A}{E}\ncput*[nrot=:U]{2}
               \ncline{E}{B}\ncput*[nrot=:U]{5}
               \ncline{B}{C}\ncput*[nrot=:U]{2}
               \ncline{B}{D}\ncput*[nrot=:U]{4}
               \ncline{B}{F}\ncput*[nrot=:U]{1}
               \ncline{C}{D}\ncput*[nrot=:U]{2}
               \ncline{E}{D}\ncput*[nrot=:U]{5}
               \ncline{F}{D}\ncput*[nrot=:U]{2}
               \ncline{E}{F}\ncput*[nrot=:U]{5}
          \end{pspicture}
     \end{extern}
\end{center}
\par
\begin{itemize}
     \item %5
     \textbf{Affirmation 5 :}\quad Le plus court chemin menant de A à D mesure 700 mètres .\\
     \textit{On justifiera la réponse à l'aide d'un algorithme.}
\end{itemize}
\begin{corrige}
     \begin{itemize}
          \item %1
          \textbf{Affirmation 1 :}\quad Le graphe (G) est connexe : \textbf{EXACT}.
          \par
          Un graphe est connexe si et seulement si on peut relier deux quelconques de ses sommets par une chaîne.
          \par
          C'est bien le cas ici donc le graphe (G) est connexe.
          \item %2
          \textbf{Affirmation 2 :}\quad Le graphe (G) contient un sous-graphe complet d'ordre 4 : \textbf{EXACT}.
          \par
          Considérons le sous-graphe d'ordre 4 composé des sommets B, D, E et F. Chacun de ces sommets est relié aux trois autres.\\
          Ce sous-graphe est donc complet.
          \item %3
          \textbf{Affirmation 3 :}\quad Une personne peut parcourir toutes les routes du village sans emprunter plusieurs fois la même route : \textbf{EXACT}.
          \par
          La question revient à déterminer s'il existe une chaîne qui contient une fois et une seule chacune des arêtes du graphe c'est à dire une \textbf{chaîne eulérienne}.
          \par
          Or, d'après le théorème d'Euler, un graphe connexe contient une chaîne eulérienne si et seulement s'il possède \textbf{0 ou 2 sommets de degré impair}.
          \par
          Le tableau ci-après recense le degré de chacun des sommets :
          \par
          \begin{center}
               \begin{tabular}{|l|c|c|c|c|c|c|c|c|}%class="compact"
                    \hline
                    Sommet & A & B & C & D & E & F  \\
                    \hline
                    Degré & 2 & 5 & 2 & 4 & 4 & 3 \\
                    \hline
               \end{tabular}
          \end{center}
          \par
          Le graphe (G) possède deux sommets de degré impair : B et F. Il existe donc au moins un trajet qui emprunte une fois et une seule chacune des routes du graphe.\\
          Ces trajets ont nécessairement comme extrémités B et F ; par exemple : B-C-D-E-A-B-E-F-D-B-F.
          \item %4
          \textbf{Affirmation 4 :}\quad Il y a exactement 5 trajets de trois routes reliant les carrefours C et E: \textbf{EXACT}.
          \par
          La matrice de transition du graphe (G) obtenue en classant les sommets par ordre alphabétique est :
          \[ M = \begin{pmatrix}
               0 &1 &0 &0 &1 &0 \\
               1 &0 &1 &1 &1 &1 \\
               0 &1 &0 &1 &0 &0 \\
               0 &1 &1 &0 &1 &1 \\
               1 &1 &0 &1 &0 &1\\
          0 &1 &0 &1 &1 &0  \end{pmatrix}
          \]
          \par
          Pour obtenir le nombre de chemins de trois routes reliant deux sommets on calcule $M^3$.
          \par
          \`A la calculatrice, on trouve :
          \par
          \[ M^3 = \begin{pmatrix}
               2 &8 &3 &5 &7 &4 \\
               8 &10 &8 &11 &11 &11 \\
               3 &8 &2 &7 &5 &4 \\
               5 &11 &7 &8 &11 &9 \\
               7 &11 &5 &11 &8 &9\\
          4 &11 &4 &9 &9 &6  \end{pmatrix}
          \]
          \par
          Le nombre de chemins de trois routes, reliant C à E, est le coefficient de $M^3$ situé sur la troisième ligne (correspondant au sommet de départ C) et la cinquième colonne (correspondant au sommet d'arrivé E).
          \par
          Il y a donc bien \textbf{5 trajets} de trois routes reliant les carrefours C et E.
          \par
          \cadre{vert}{En pratique}{
               Soit $M$ la matrice de transition d'un graphe G. Pour déterminer \textbf{le nombre de chemins de longueur $\bm{n}$} reliant deux sommets du graphe on calcule $\bm{M^n}$.
               \par
               Le coefficient de la matrice $M^n$ situé à la $i$-ème ligne et à la $j$-ème colonne indique le nombre de chemins de longueur $n$ menant du sommet numéro $i$ au sommet numéro $j$.
          }
          \item %5
          \textbf{Affirmation 5 :}\quad Le plus court chemin menant de A à D mesure 700 mètres : \textbf{FAUX}.
          \par
          On utilise l'algorithme de Dijkstra en partant de A :
          \par
          \begin{center}
               \begin{extern}%width="600" alt="algorithme de Dijkstra"
                    \begin{tabularx}{0.9\linewidth}{|c|C|C|C|C|C|C|}
                         \hline
                         Sommets			&   A 						& B 							& C							& D 							& E								& F   						\\ \hline
                         Départ			&  $\color{red}0_{\text{A}}$ 	& $\infty$					& $\infty$					& $\infty$					& $\infty$						& $\infty$	  				\\ \hline
                         A (0) 			&  \cellcolor{black!20}		& $3_{\text{A}}$	 			& $\infty$ 					& $\infty$ 					& $\color{red}2_{\text{A}}$		& $\infty$ 					\\ \hline
                         E (2)			&  \cellcolor{black!20}	 	& $\color{red}3_{\text{A}}$	& $\infty$ 					& $7_{\text{E}}$   			& \cellcolor{black!20} 			& $7_{\text{E}}$				\\ \hline
                         B (3)			&  \cellcolor{black!20}		& \cellcolor{black!20}	 	& $5_{\text{B}}$ 			& $7_{\text{E}}$ 			& \cellcolor{black!20}			& $\color{red}4_{\text{B}}$ 	\\ \hline
                         F (4)			&  \cellcolor{black!20}		& \cellcolor{black!20}		& $\color{red}5_{\text{B}}$ 	& $6_{\text{F}}$  			& \cellcolor{black!20}			& \cellcolor{black!20}  		\\ \hline
                         C (5)			&  \cellcolor{black!20} 		& \cellcolor{black!20} 		& \cellcolor{black!20} 		& $\color{red}6_{\text{F}}$ 	& \cellcolor{black!20} 			& \cellcolor{black!20} 		\\ \hline
                    \end{tabularx}
               \end{extern}
          \end{center}
          Le trajet le plus court menant de A à D mesure 6 centaines de mètres soit 600 mètres.
          \par
          Il s'agit du trajet A-B-F-D.
          \par
          \textit{Pour plus de détails sur la méthode employée dans cette question se reporter à la fiche  consacrée à \mcLien{/methode/algorithme-de-dijkstra-etape-par-etape/}{ l'algorithme de Dijkstra}.}
          \par
     \end{itemize}
\end{corrige}

\end{document}