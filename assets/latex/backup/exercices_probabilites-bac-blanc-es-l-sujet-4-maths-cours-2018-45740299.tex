\documentclass[a4paper]{article}

%================================================================================================================================
%
% Packages
%
%================================================================================================================================

\usepackage[T1]{fontenc} 	% pour caractères accentués
\usepackage[utf8]{inputenc}  % encodage utf8
\usepackage[french]{babel}	% langue : français
\usepackage{fourier}			% caractères plus lisibles
\usepackage[dvipsnames]{xcolor} % couleurs
\usepackage{fancyhdr}		% réglage header footer
\usepackage{needspace}		% empêcher sauts de page mal placés
\usepackage{graphicx}		% pour inclure des graphiques
\usepackage{enumitem,cprotect}		% personnalise les listes d'items (nécessaire pour ol, al ...)
\usepackage{hyperref}		% Liens hypertexte
\usepackage{pstricks,pst-all,pst-node,pstricks-add,pst-math,pst-plot,pst-tree,pst-eucl} % pstricks
\usepackage[a4paper,includeheadfoot,top=2cm,left=3cm, bottom=2cm,right=3cm]{geometry} % marges etc.
\usepackage{comment}			% commentaires multilignes
\usepackage{amsmath,environ} % maths (matrices, etc.)
\usepackage{amssymb,makeidx}
\usepackage{bm}				% bold maths
\usepackage{tabularx}		% tableaux
\usepackage{colortbl}		% tableaux en couleur
\usepackage{fontawesome}		% Fontawesome
\usepackage{environ}			% environment with command
\usepackage{fp}				% calculs pour ps-tricks
\usepackage{multido}			% pour ps tricks
\usepackage[np]{numprint}	% formattage nombre
\usepackage{tikz,tkz-tab} 			% package principal TikZ
\usepackage{pgfplots}   % axes
\usepackage{mathrsfs}    % cursives
\usepackage{calc}			% calcul taille boites
\usepackage[scaled=0.875]{helvet} % font sans serif
\usepackage{svg} % svg
\usepackage{scrextend} % local margin
\usepackage{scratch} %scratch
\usepackage{multicol} % colonnes
%\usepackage{infix-RPN,pst-func} % formule en notation polanaise inversée
\usepackage{listings}

%================================================================================================================================
%
% Réglages de base
%
%================================================================================================================================

\lstset{
language=Python,   % R code
literate=
{á}{{\'a}}1
{à}{{\`a}}1
{ã}{{\~a}}1
{é}{{\'e}}1
{è}{{\`e}}1
{ê}{{\^e}}1
{í}{{\'i}}1
{ó}{{\'o}}1
{õ}{{\~o}}1
{ú}{{\'u}}1
{ü}{{\"u}}1
{ç}{{\c{c}}}1
{~}{{ }}1
}


\definecolor{codegreen}{rgb}{0,0.6,0}
\definecolor{codegray}{rgb}{0.5,0.5,0.5}
\definecolor{codepurple}{rgb}{0.58,0,0.82}
\definecolor{backcolour}{rgb}{0.95,0.95,0.92}

\lstdefinestyle{mystyle}{
    backgroundcolor=\color{backcolour},   
    commentstyle=\color{codegreen},
    keywordstyle=\color{magenta},
    numberstyle=\tiny\color{codegray},
    stringstyle=\color{codepurple},
    basicstyle=\ttfamily\footnotesize,
    breakatwhitespace=false,         
    breaklines=true,                 
    captionpos=b,                    
    keepspaces=true,                 
    numbers=left,                    
xleftmargin=2em,
framexleftmargin=2em,            
    showspaces=false,                
    showstringspaces=false,
    showtabs=false,                  
    tabsize=2,
    upquote=true
}

\lstset{style=mystyle}


\lstset{style=mystyle}
\newcommand{\imgdir}{C:/laragon/www/newmc/assets/imgsvg/}
\newcommand{\imgsvgdir}{C:/laragon/www/newmc/assets/imgsvg/}

\definecolor{mcgris}{RGB}{220, 220, 220}% ancien~; pour compatibilité
\definecolor{mcbleu}{RGB}{52, 152, 219}
\definecolor{mcvert}{RGB}{125, 194, 70}
\definecolor{mcmauve}{RGB}{154, 0, 215}
\definecolor{mcorange}{RGB}{255, 96, 0}
\definecolor{mcturquoise}{RGB}{0, 153, 153}
\definecolor{mcrouge}{RGB}{255, 0, 0}
\definecolor{mclightvert}{RGB}{205, 234, 190}

\definecolor{gris}{RGB}{220, 220, 220}
\definecolor{bleu}{RGB}{52, 152, 219}
\definecolor{vert}{RGB}{125, 194, 70}
\definecolor{mauve}{RGB}{154, 0, 215}
\definecolor{orange}{RGB}{255, 96, 0}
\definecolor{turquoise}{RGB}{0, 153, 153}
\definecolor{rouge}{RGB}{255, 0, 0}
\definecolor{lightvert}{RGB}{205, 234, 190}
\setitemize[0]{label=\color{lightvert}  $\bullet$}

\pagestyle{fancy}
\renewcommand{\headrulewidth}{0.2pt}
\fancyhead[L]{maths-cours.fr}
\fancyhead[R]{\thepage}
\renewcommand{\footrulewidth}{0.2pt}
\fancyfoot[C]{}

\newcolumntype{C}{>{\centering\arraybackslash}X}
\newcolumntype{s}{>{\hsize=.35\hsize\arraybackslash}X}

\setlength{\parindent}{0pt}		 
\setlength{\parskip}{3mm}
\setlength{\headheight}{1cm}

\def\ebook{ebook}
\def\book{book}
\def\web{web}
\def\type{web}

\newcommand{\vect}[1]{\overrightarrow{\,\mathstrut#1\,}}

\def\Oij{$\left(\text{O}~;~\vect{\imath},~\vect{\jmath}\right)$}
\def\Oijk{$\left(\text{O}~;~\vect{\imath},~\vect{\jmath},~\vect{k}\right)$}
\def\Ouv{$\left(\text{O}~;~\vect{u},~\vect{v}\right)$}

\hypersetup{breaklinks=true, colorlinks = true, linkcolor = OliveGreen, urlcolor = OliveGreen, citecolor = OliveGreen, pdfauthor={Didier BONNEL - https://www.maths-cours.fr} } % supprime les bordures autour des liens

\renewcommand{\arg}[0]{\text{arg}}

\everymath{\displaystyle}

%================================================================================================================================
%
% Macros - Commandes
%
%================================================================================================================================

\newcommand\meta[2]{    			% Utilisé pour créer le post HTML.
	\def\titre{titre}
	\def\url{url}
	\def\arg{#1}
	\ifx\titre\arg
		\newcommand\maintitle{#2}
		\fancyhead[L]{#2}
		{\Large\sffamily \MakeUppercase{#2}}
		\vspace{1mm}\textcolor{mcvert}{\hrule}
	\fi 
	\ifx\url\arg
		\fancyfoot[L]{\href{https://www.maths-cours.fr#2}{\black \footnotesize{https://www.maths-cours.fr#2}}}
	\fi 
}


\newcommand\TitreC[1]{    		% Titre centré
     \needspace{3\baselineskip}
     \begin{center}\textbf{#1}\end{center}
}

\newcommand\newpar{    		% paragraphe
     \par
}

\newcommand\nosp {    		% commande vide (pas d'espace)
}
\newcommand{\id}[1]{} %ignore

\newcommand\boite[2]{				% Boite simple sans titre
	\vspace{5mm}
	\setlength{\fboxrule}{0.2mm}
	\setlength{\fboxsep}{5mm}	
	\fcolorbox{#1}{#1!3}{\makebox[\linewidth-2\fboxrule-2\fboxsep]{
  		\begin{minipage}[t]{\linewidth-2\fboxrule-4\fboxsep}\setlength{\parskip}{3mm}
  			 #2
  		\end{minipage}
	}}
	\vspace{5mm}
}

\newcommand\CBox[4]{				% Boites
	\vspace{5mm}
	\setlength{\fboxrule}{0.2mm}
	\setlength{\fboxsep}{5mm}
	
	\fcolorbox{#1}{#1!3}{\makebox[\linewidth-2\fboxrule-2\fboxsep]{
		\begin{minipage}[t]{1cm}\setlength{\parskip}{3mm}
	  		\textcolor{#1}{\LARGE{#2}}    
 	 	\end{minipage}  
  		\begin{minipage}[t]{\linewidth-2\fboxrule-4\fboxsep}\setlength{\parskip}{3mm}
			\raisebox{1.2mm}{\normalsize\sffamily{\textcolor{#1}{#3}}}						
  			 #4
  		\end{minipage}
	}}
	\vspace{5mm}
}

\newcommand\cadre[3]{				% Boites convertible html
	\par
	\vspace{2mm}
	\setlength{\fboxrule}{0.1mm}
	\setlength{\fboxsep}{5mm}
	\fcolorbox{#1}{white}{\makebox[\linewidth-2\fboxrule-2\fboxsep]{
  		\begin{minipage}[t]{\linewidth-2\fboxrule-4\fboxsep}\setlength{\parskip}{3mm}
			\raisebox{-2.5mm}{\sffamily \small{\textcolor{#1}{\MakeUppercase{#2}}}}		
			\par		
  			 #3
 	 		\end{minipage}
	}}
		\vspace{2mm}
	\par
}

\newcommand\bloc[3]{				% Boites convertible html sans bordure
     \needspace{2\baselineskip}
     {\sffamily \small{\textcolor{#1}{\MakeUppercase{#2}}}}    
		\par		
  			 #3
		\par
}

\newcommand\CHelp[1]{
     \CBox{Plum}{\faInfoCircle}{À RETENIR}{#1}
}

\newcommand\CUp[1]{
     \CBox{NavyBlue}{\faThumbsOUp}{EN PRATIQUE}{#1}
}

\newcommand\CInfo[1]{
     \CBox{Sepia}{\faArrowCircleRight}{REMARQUE}{#1}
}

\newcommand\CRedac[1]{
     \CBox{PineGreen}{\faEdit}{BIEN R\'EDIGER}{#1}
}

\newcommand\CError[1]{
     \CBox{Red}{\faExclamationTriangle}{ATTENTION}{#1}
}

\newcommand\TitreExo[2]{
\needspace{4\baselineskip}
 {\sffamily\large EXERCICE #1\ (\emph{#2 points})}
\vspace{5mm}
}

\newcommand\img[2]{
          \includegraphics[width=#2\paperwidth]{\imgdir#1}
}

\newcommand\imgsvg[2]{
       \begin{center}   \includegraphics[width=#2\paperwidth]{\imgsvgdir#1} \end{center}
}


\newcommand\Lien[2]{
     \href{#1}{#2 \tiny \faExternalLink}
}
\newcommand\mcLien[2]{
     \href{https~://www.maths-cours.fr/#1}{#2 \tiny \faExternalLink}
}

\newcommand{\euro}{\eurologo{}}

%================================================================================================================================
%
% Macros - Environement
%
%================================================================================================================================

\newenvironment{tex}{ %
}
{%
}

\newenvironment{indente}{ %
	\setlength\parindent{10mm}
}

{
	\setlength\parindent{0mm}
}

\newenvironment{corrige}{%
     \needspace{3\baselineskip}
     \medskip
     \textbf{\textsc{Corrigé}}
     \medskip
}
{
}

\newenvironment{extern}{%
     \begin{center}
     }
     {
     \end{center}
}

\NewEnviron{code}{%
	\par
     \boite{gray}{\texttt{%
     \BODY
     }}
     \par
}

\newenvironment{vbloc}{% boite sans cadre empeche saut de page
     \begin{minipage}[t]{\linewidth}
     }
     {
     \end{minipage}
}
\NewEnviron{h2}{%
    \needspace{3\baselineskip}
    \vspace{0.6cm}
	\noindent \MakeUppercase{\sffamily \large \BODY}
	\vspace{1mm}\textcolor{mcgris}{\hrule}\vspace{0.4cm}
	\par
}{}

\NewEnviron{h3}{%
    \needspace{3\baselineskip}
	\vspace{5mm}
	\textsc{\BODY}
	\par
}

\NewEnviron{margeneg}{ %
\begin{addmargin}[-1cm]{0cm}
\BODY
\end{addmargin}
}

\NewEnviron{html}{%
}

\begin{document}
\meta{url}{/exercices/probabilites-bac-blanc-es-l-sujet-4-maths-cours-2018/}
\meta{pid}{10501}
\meta{titre}{Probabilités - Bac blanc ES/L Sujet 4 - Maths-cours 2018}
\meta{type}{exercices}
%
\begin{h2}Exercice 2 (5 points)\end{h2}
\par
\textit{Les parties A et B sont indépendantes.}
\par
\textit{Les probabilités demandées seront arrondies au dix-millième.}
\par
%============================================================================================================================
%
\TitreC{Partie A}
%
%============================================================================================================================
\par
Dans un lycée parisien, on a dénombré 52\% de filles et 48\% de garçons.
\par
Une étude a révélé que, dans ce lycée, 59\% des filles et 68\% des garçons pratiquaient un sport en dehors de l'établissement.
\par
On choisit au hasard un élève dans ce lycée et on considère les événements suivants :
\par
\begin{itemize}
     \item $F$ : \og l'élève choisi est une fille \fg{} ;
     \item $G$ : \og l'élève choisi est un garçon \fg{} ;
     \item $S$ : \og l'élève choisi pratique un sport en dehors de l'établissement\fg{} ;
     \item $\overline{S}$ : l'événement contraire de $S$.
\end{itemize}
\par
\begin{enumerate}
     \item Recopier et compléter l'arbre de probabilité ci-après :
     %:-+-+-+- Engendré par : http://math.et.info.free.fr/TikZ/Arbre/
     \begin{center}
          % Racine à Gauche, développement vers la droite
          \begin{extern}%width="300" alt="Arbre de probabilité à compléter "
               \begin{tikzpicture}[xscale=1,yscale=1]
                    % Styles (MODIFIABLES)
                    \tikzstyle{fleche}=[-,>=latex,thick]
                    \tikzstyle{noeud}=[fill=white,circle,draw]
                    \tikzstyle{feuille}=[fill=white,circle,draw]
                    \tikzstyle{etiquette}=[midway,fill=white]
                    % Dimensions (MODIFIABLES)
                    \def\DistanceInterNiveaux{3}
                    \def\DistanceInterFeuilles{2}
                    % Dimensions calculées (NON MODIFIABLES)
                    \def\NiveauA{(0)*\DistanceInterNiveaux}
                    \def\NiveauB{(1.5)*\DistanceInterNiveaux}
                    \def\NiveauC{(2.5)*\DistanceInterNiveaux}
                    \def\InterFeuilles{(-1)*\DistanceInterFeuilles}
                    % Noeuds (MODIFIABLES : Styles et Coefficients d'InterFeuilles)
                    \node[noeud] (R) at ({\NiveauA},{(1.5)*\InterFeuilles}) {$\ $};
                    \node[noeud] (Ra) at ({\NiveauB},{(0.5)*\InterFeuilles}) {$F$};
                    \node[feuille] (Raa) at ({\NiveauC},{(0)*\InterFeuilles}) {$S$};
                    \node[feuille] (Rab) at ({\NiveauC},{(1)*\InterFeuilles}) {$\overline{S}$};
                    \node[noeud] (Rb) at ({\NiveauB},{(2.5)*\InterFeuilles}) {$G$};
                    \node[feuille] (Rba) at ({\NiveauC},{(2)*\InterFeuilles}) {$S$};
                    \node[feuille] (Rbb) at ({\NiveauC},{(3)*\InterFeuilles}) {$\overline{S}$};
                    % Arcs (MODIFIABLES : Styles)
                    \draw[fleche] (R)--(Ra) node[etiquette] {$\cdots$};
                    \draw[fleche] (Ra)--(Raa) node[etiquette] {$\cdots$};
                    \draw[fleche] (Ra)--(Rab) node[etiquette] {$\cdots$};
                    \draw[fleche] (R)--(Rb) node[etiquette] {$\cdots$};
                    \draw[fleche] (Rb)--(Rba) node[etiquette] {$\cdots$};
                    \draw[fleche] (Rb)--(Rbb) node[etiquette] {$\cdots$};
               \end{tikzpicture}
          \end{extern}
     \end{center}
     %:-+-+-+-+- Fin
     \item Quel est la probabilité que l'élève choisi soit un garçon pratiquant un sport en dehors du lycée ?
     \item Quel est la probabilité que l'élève choisi pratique un sport en dehors du lycée ?
     \item On sait que l'élève choisi pratique un sport en dehors de l'établissement. Quel est la probabilité que ce soit un garçon ?
\end{enumerate}
\par
%============================================================================================================================
%
\TitreC{Partie B}
%
%============================================================================================================================
\par
Luc doit se rendre, par les transports en commun,  à un cours de natation qui débute à 10h. En fonction de la circulation, il arrive entre 9h30 et 10h15.
\par
On suppose que son heure d'arrivée peut être modélisée par une variable aléatoire $T$ qui suit la loi uniforme sur l'intervalle ${[9,5~;~10,25]}$.
\par
\begin{enumerate}
     \item Quelle est la probabilité que Luc arrive à l'heure à son cours ?
     \item Quelle est la probabilité que Luc arrive avec plus d'un quart d'heure d'avance à son cours ?
     \item Quelle est l'espérance mathématique de la variable aléatoire $T$ ?
     Interpréter cette valeur dans le cadre de l'exercice.
     \par
\end{enumerate}
\begin{corrige}
     %============================================================================================================================
     %
     \TitreC{Partie A}
     %
     %============================================================================================================================
     \par
     \begin{enumerate}
          \item %1
          D'après les données de l'énoncé :
          \par
          \begin{itemize}
               \item $p(F)=0,52$ ;
               \item $p(G)=0,48$ ;
               \item $p_F(S)=0,59$ ;
               \item $p_G(S)=0,68$.
          \end{itemize}
          \par
          On obtient alors l'arbre ci-après :
          %:-+-+-+- Engendré par : http://math.et.info.free.fr/TikZ/Arbre/
          \begin{center}
               \begin{extern}%width="400" alt="Arbre de probabilité complété"
                    % Racine à Gauche, développement vers la droite
                    \begin{tikzpicture}[xscale=1,yscale=1]
                         % Styles (MODIFIABLES)
                         \tikzstyle{fleche}=[-,>=latex,thick]
                         \tikzstyle{noeud}=[fill=white,circle,draw]
                         \tikzstyle{feuille}=[fill=white,circle,draw]
                         \tikzstyle{etiquette}=[midway,fill=white]
                         % Dimensions (MODIFIABLES)
                         \def\DistanceInterNiveaux{3}
                         \def\DistanceInterFeuilles{2}
                         % Dimensions calculées (NON MODIFIABLES)
                         \def\NiveauA{(0)*\DistanceInterNiveaux}
                         \def\NiveauB{(1.5)*\DistanceInterNiveaux}
                         \def\NiveauC{(2.5)*\DistanceInterNiveaux}
                         \def\InterFeuilles{(-1)*\DistanceInterFeuilles}
                         % Noeuds (MODIFIABLES : Styles et Coefficients d'InterFeuilles)
                         \node[noeud] (R) at ({\NiveauA},{(1.5)*\InterFeuilles}) {$\ $};
                         \node[noeud] (Ra) at ({\NiveauB},{(0.5)*\InterFeuilles}) {$F$};
                         \node[feuille] (Raa) at ({\NiveauC},{(0)*\InterFeuilles}) {$S$};
                         \node[feuille] (Rab) at ({\NiveauC},{(1)*\InterFeuilles}) {$\overline{S}$};
                         \node[noeud] (Rb) at ({\NiveauB},{(2.5)*\InterFeuilles}) {$G$};
                         \node[feuille] (Rba) at ({\NiveauC},{(2)*\InterFeuilles}) {$S$};
                         \node[feuille] (Rbb) at ({\NiveauC},{(3)*\InterFeuilles}) {$\overline{S}$};
                         % Arcs (MODIFIABLES : Styles)
                         \draw[fleche] (R)--(Ra) node[etiquette] {$0,52$};
                         \draw[fleche] (Ra)--(Raa) node[etiquette] {$0,59$};
                         \draw[fleche] (Ra)--(Rab) node[etiquette] {$0,41$};
                         \draw[fleche] (R)--(Rb) node[etiquette] {$0,48$};
                         \draw[fleche] (Rb)--(Rba) node[etiquette] {$0,68$};
                         \draw[fleche] (Rb)--(Rbb) node[etiquette] {$0,32$};
                    \end{tikzpicture}
               \end{extern}
          \end{center}
          %:-+-+-+-+- Fin
          \item %2
          La probabilité demandée est $p(G \cap S)$ :
          \par
          $p(G \cap S)= p(G) \times p_S(G)=0,48 \times 0,68 = 0,3264$.
          \par
          \cadre{vert}{En pratique}{
               L'événement $G \cap S$ correspond à : \og les événements $G$ \textbf{et} $S$ sont \textbf{tous les deux} réalisés \fg{}.
               \par
               La probabilité de  $G \cap S$ peut se calculer à l'aide de la formule :
               \[ p(G \cap S)= p(G \times p_G(S). \]
               \par
               \`A partir de l'arbre pondéré, cela revient à multiplier les probabilités situées sur :
               \begin{itemize}
                    \item %
                    la branche qui aboutit à $G$,
                    \item %
                    La branche qui relie $G$ à $S$.
               \end{itemize}
          }
          \item %3
          La probabilité cherchée est $p(S)$.
          \par
          D'après la formule des probabilités totales :
          \par
          $p(S)=p(F\cap S) + p(G\cap S)$\\
          $\phantom{p(S)}=p(F) \times p_F(S) + p(G) \times p_{G}(S)$\\
          $\phantom{p(S)} = 0,52 \times 0,59 +0,48 \times 0,68=0,6332$.
          \item %4
          La probabilité demandée est $p_S(G)$.
          \par
          D'après la formule des probabilités conditionnelles :
          \par
          $p_S(G)=\dfrac{p(G\cap S)}{p(S)}=\dfrac{0,3264}{0,6332} \approx 0,5155\ $ (à $10^{-4}$ près).
          \par
     \end{enumerate}
     \par
     %============================================================================================================================
     %
     \TitreC{Partie B}
     %
     %============================================================================================================================
     \par
     \begin{enumerate}
          \item %1
          Luc est à l'heure à son cours s'il arrive entre 9h30 et 10h, c'est à dire si $9,5 \leqslant T \leqslant 10$.
          \par
          $T$ suivant la loi uniforme sur l'intervalle $[9,5~;~10,25]$ :
          \par
          $p(9,5 \leqslant T \leqslant 10)=\dfrac{10-9,5}{10,25-9,5}=\dfrac{0,5}{0,75}=\dfrac{2}{3} \approx 0,6667\ $ (à $10^{-4}$ près).
          \par
          \cadre{rouge}{À retenir}{
               Si $X$ suit la \textbf{loi uniforme} sur l'intervalle $[a~;~b]$, alors pour tous réels $c$ et $d$ de l'intervalle $[a~;~b]$ avec $c \leqslant d$ :
               \[ p(c \leqslant X \leqslant d) = \dfrac{d-c}{b-a}. \]
          }
          \item %2
          Luc arrive à son cours avec plus d'un quart d'heure d'avance s'il arrive entre 9h30 et 9h45, c'est à dire si ${9+\dfrac{1}{2} \leqslant T \leqslant 9+\dfrac{3}{4}}$ ou encore ${9,5 \leqslant T \leqslant 9,75}$.
          \par
          La probabilité de cet événement est :
          \par
          $p(9,5 \leqslant T \leqslant 9,75)=\dfrac{9,75-9,5}{10,25-9,5}=\dfrac{0,25}{0,75}=\dfrac{1}{3} \approx 0,3333\ $ (à $10^{-4}$ près).
          \item %3
          Comme $T$ suit la loi uniforme sur l'intervalle $[9,5~;~10,25]$ :
          \par
          $E(T)=\dfrac{9,5+10,25}{2}=\dfrac{19,75}{2}=9,875$.
          \par
          L'espérance mathématique de $T$ représente l'heure d'arrivée \textbf{moyenne} de Luc.
          \par
          $0,875$ heure correspond à $0,875 \times 60 = 52,5$ minutes.
          \par
          En moyenne, Luc arrivera à son cours à 9h 52min 30s.
          \par
          \cadre{rouge}{À retenir}{
               L'espérance mathématique de la \textbf{loi uniforme} sur l'intervalle $[a~;~b]$ est :
               \[ E(X) = \dfrac{a+b}{2}. \]
          }
          \par
     \end{enumerate}
\end{corrige}

\end{document}