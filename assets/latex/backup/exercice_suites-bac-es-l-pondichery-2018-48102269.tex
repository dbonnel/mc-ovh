\documentclass[a4paper]{article}

%================================================================================================================================
%
% Packages
%
%================================================================================================================================

\usepackage[T1]{fontenc} 	% pour caractères accentués
\usepackage[utf8]{inputenc}  % encodage utf8
\usepackage[french]{babel}	% langue : français
\usepackage{fourier}			% caractères plus lisibles
\usepackage[dvipsnames]{xcolor} % couleurs
\usepackage{fancyhdr}		% réglage header footer
\usepackage{needspace}		% empêcher sauts de page mal placés
\usepackage{graphicx}		% pour inclure des graphiques
\usepackage{enumitem,cprotect}		% personnalise les listes d'items (nécessaire pour ol, al ...)
\usepackage{hyperref}		% Liens hypertexte
\usepackage{pstricks,pst-all,pst-node,pstricks-add,pst-math,pst-plot,pst-tree,pst-eucl} % pstricks
\usepackage[a4paper,includeheadfoot,top=2cm,left=3cm, bottom=2cm,right=3cm]{geometry} % marges etc.
\usepackage{comment}			% commentaires multilignes
\usepackage{amsmath,environ} % maths (matrices, etc.)
\usepackage{amssymb,makeidx}
\usepackage{bm}				% bold maths
\usepackage{tabularx}		% tableaux
\usepackage{colortbl}		% tableaux en couleur
\usepackage{fontawesome}		% Fontawesome
\usepackage{environ}			% environment with command
\usepackage{fp}				% calculs pour ps-tricks
\usepackage{multido}			% pour ps tricks
\usepackage[np]{numprint}	% formattage nombre
\usepackage{tikz,tkz-tab} 			% package principal TikZ
\usepackage{pgfplots}   % axes
\usepackage{mathrsfs}    % cursives
\usepackage{calc}			% calcul taille boites
\usepackage[scaled=0.875]{helvet} % font sans serif
\usepackage{svg} % svg
\usepackage{scrextend} % local margin
\usepackage{scratch} %scratch
\usepackage{multicol} % colonnes
%\usepackage{infix-RPN,pst-func} % formule en notation polanaise inversée
\usepackage{listings}

%================================================================================================================================
%
% Réglages de base
%
%================================================================================================================================

\lstset{
language=Python,   % R code
literate=
{á}{{\'a}}1
{à}{{\`a}}1
{ã}{{\~a}}1
{é}{{\'e}}1
{è}{{\`e}}1
{ê}{{\^e}}1
{í}{{\'i}}1
{ó}{{\'o}}1
{õ}{{\~o}}1
{ú}{{\'u}}1
{ü}{{\"u}}1
{ç}{{\c{c}}}1
{~}{{ }}1
}


\definecolor{codegreen}{rgb}{0,0.6,0}
\definecolor{codegray}{rgb}{0.5,0.5,0.5}
\definecolor{codepurple}{rgb}{0.58,0,0.82}
\definecolor{backcolour}{rgb}{0.95,0.95,0.92}

\lstdefinestyle{mystyle}{
    backgroundcolor=\color{backcolour},   
    commentstyle=\color{codegreen},
    keywordstyle=\color{magenta},
    numberstyle=\tiny\color{codegray},
    stringstyle=\color{codepurple},
    basicstyle=\ttfamily\footnotesize,
    breakatwhitespace=false,         
    breaklines=true,                 
    captionpos=b,                    
    keepspaces=true,                 
    numbers=left,                    
xleftmargin=2em,
framexleftmargin=2em,            
    showspaces=false,                
    showstringspaces=false,
    showtabs=false,                  
    tabsize=2,
    upquote=true
}

\lstset{style=mystyle}


\lstset{style=mystyle}
\newcommand{\imgdir}{C:/laragon/www/newmc/assets/imgsvg/}
\newcommand{\imgsvgdir}{C:/laragon/www/newmc/assets/imgsvg/}

\definecolor{mcgris}{RGB}{220, 220, 220}% ancien~; pour compatibilité
\definecolor{mcbleu}{RGB}{52, 152, 219}
\definecolor{mcvert}{RGB}{125, 194, 70}
\definecolor{mcmauve}{RGB}{154, 0, 215}
\definecolor{mcorange}{RGB}{255, 96, 0}
\definecolor{mcturquoise}{RGB}{0, 153, 153}
\definecolor{mcrouge}{RGB}{255, 0, 0}
\definecolor{mclightvert}{RGB}{205, 234, 190}

\definecolor{gris}{RGB}{220, 220, 220}
\definecolor{bleu}{RGB}{52, 152, 219}
\definecolor{vert}{RGB}{125, 194, 70}
\definecolor{mauve}{RGB}{154, 0, 215}
\definecolor{orange}{RGB}{255, 96, 0}
\definecolor{turquoise}{RGB}{0, 153, 153}
\definecolor{rouge}{RGB}{255, 0, 0}
\definecolor{lightvert}{RGB}{205, 234, 190}
\setitemize[0]{label=\color{lightvert}  $\bullet$}

\pagestyle{fancy}
\renewcommand{\headrulewidth}{0.2pt}
\fancyhead[L]{maths-cours.fr}
\fancyhead[R]{\thepage}
\renewcommand{\footrulewidth}{0.2pt}
\fancyfoot[C]{}

\newcolumntype{C}{>{\centering\arraybackslash}X}
\newcolumntype{s}{>{\hsize=.35\hsize\arraybackslash}X}

\setlength{\parindent}{0pt}		 
\setlength{\parskip}{3mm}
\setlength{\headheight}{1cm}

\def\ebook{ebook}
\def\book{book}
\def\web{web}
\def\type{web}

\newcommand{\vect}[1]{\overrightarrow{\,\mathstrut#1\,}}

\def\Oij{$\left(\text{O}~;~\vect{\imath},~\vect{\jmath}\right)$}
\def\Oijk{$\left(\text{O}~;~\vect{\imath},~\vect{\jmath},~\vect{k}\right)$}
\def\Ouv{$\left(\text{O}~;~\vect{u},~\vect{v}\right)$}

\hypersetup{breaklinks=true, colorlinks = true, linkcolor = OliveGreen, urlcolor = OliveGreen, citecolor = OliveGreen, pdfauthor={Didier BONNEL - https://www.maths-cours.fr} } % supprime les bordures autour des liens

\renewcommand{\arg}[0]{\text{arg}}

\everymath{\displaystyle}

%================================================================================================================================
%
% Macros - Commandes
%
%================================================================================================================================

\newcommand\meta[2]{    			% Utilisé pour créer le post HTML.
	\def\titre{titre}
	\def\url{url}
	\def\arg{#1}
	\ifx\titre\arg
		\newcommand\maintitle{#2}
		\fancyhead[L]{#2}
		{\Large\sffamily \MakeUppercase{#2}}
		\vspace{1mm}\textcolor{mcvert}{\hrule}
	\fi 
	\ifx\url\arg
		\fancyfoot[L]{\href{https://www.maths-cours.fr#2}{\black \footnotesize{https://www.maths-cours.fr#2}}}
	\fi 
}


\newcommand\TitreC[1]{    		% Titre centré
     \needspace{3\baselineskip}
     \begin{center}\textbf{#1}\end{center}
}

\newcommand\newpar{    		% paragraphe
     \par
}

\newcommand\nosp {    		% commande vide (pas d'espace)
}
\newcommand{\id}[1]{} %ignore

\newcommand\boite[2]{				% Boite simple sans titre
	\vspace{5mm}
	\setlength{\fboxrule}{0.2mm}
	\setlength{\fboxsep}{5mm}	
	\fcolorbox{#1}{#1!3}{\makebox[\linewidth-2\fboxrule-2\fboxsep]{
  		\begin{minipage}[t]{\linewidth-2\fboxrule-4\fboxsep}\setlength{\parskip}{3mm}
  			 #2
  		\end{minipage}
	}}
	\vspace{5mm}
}

\newcommand\CBox[4]{				% Boites
	\vspace{5mm}
	\setlength{\fboxrule}{0.2mm}
	\setlength{\fboxsep}{5mm}
	
	\fcolorbox{#1}{#1!3}{\makebox[\linewidth-2\fboxrule-2\fboxsep]{
		\begin{minipage}[t]{1cm}\setlength{\parskip}{3mm}
	  		\textcolor{#1}{\LARGE{#2}}    
 	 	\end{minipage}  
  		\begin{minipage}[t]{\linewidth-2\fboxrule-4\fboxsep}\setlength{\parskip}{3mm}
			\raisebox{1.2mm}{\normalsize\sffamily{\textcolor{#1}{#3}}}						
  			 #4
  		\end{minipage}
	}}
	\vspace{5mm}
}

\newcommand\cadre[3]{				% Boites convertible html
	\par
	\vspace{2mm}
	\setlength{\fboxrule}{0.1mm}
	\setlength{\fboxsep}{5mm}
	\fcolorbox{#1}{white}{\makebox[\linewidth-2\fboxrule-2\fboxsep]{
  		\begin{minipage}[t]{\linewidth-2\fboxrule-4\fboxsep}\setlength{\parskip}{3mm}
			\raisebox{-2.5mm}{\sffamily \small{\textcolor{#1}{\MakeUppercase{#2}}}}		
			\par		
  			 #3
 	 		\end{minipage}
	}}
		\vspace{2mm}
	\par
}

\newcommand\bloc[3]{				% Boites convertible html sans bordure
     \needspace{2\baselineskip}
     {\sffamily \small{\textcolor{#1}{\MakeUppercase{#2}}}}    
		\par		
  			 #3
		\par
}

\newcommand\CHelp[1]{
     \CBox{Plum}{\faInfoCircle}{À RETENIR}{#1}
}

\newcommand\CUp[1]{
     \CBox{NavyBlue}{\faThumbsOUp}{EN PRATIQUE}{#1}
}

\newcommand\CInfo[1]{
     \CBox{Sepia}{\faArrowCircleRight}{REMARQUE}{#1}
}

\newcommand\CRedac[1]{
     \CBox{PineGreen}{\faEdit}{BIEN R\'EDIGER}{#1}
}

\newcommand\CError[1]{
     \CBox{Red}{\faExclamationTriangle}{ATTENTION}{#1}
}

\newcommand\TitreExo[2]{
\needspace{4\baselineskip}
 {\sffamily\large EXERCICE #1\ (\emph{#2 points})}
\vspace{5mm}
}

\newcommand\img[2]{
          \includegraphics[width=#2\paperwidth]{\imgdir#1}
}

\newcommand\imgsvg[2]{
       \begin{center}   \includegraphics[width=#2\paperwidth]{\imgsvgdir#1} \end{center}
}


\newcommand\Lien[2]{
     \href{#1}{#2 \tiny \faExternalLink}
}
\newcommand\mcLien[2]{
     \href{https~://www.maths-cours.fr/#1}{#2 \tiny \faExternalLink}
}

\newcommand{\euro}{\eurologo{}}

%================================================================================================================================
%
% Macros - Environement
%
%================================================================================================================================

\newenvironment{tex}{ %
}
{%
}

\newenvironment{indente}{ %
	\setlength\parindent{10mm}
}

{
	\setlength\parindent{0mm}
}

\newenvironment{corrige}{%
     \needspace{3\baselineskip}
     \medskip
     \textbf{\textsc{Corrigé}}
     \medskip
}
{
}

\newenvironment{extern}{%
     \begin{center}
     }
     {
     \end{center}
}

\NewEnviron{code}{%
	\par
     \boite{gray}{\texttt{%
     \BODY
     }}
     \par
}

\newenvironment{vbloc}{% boite sans cadre empeche saut de page
     \begin{minipage}[t]{\linewidth}
     }
     {
     \end{minipage}
}
\NewEnviron{h2}{%
    \needspace{3\baselineskip}
    \vspace{0.6cm}
	\noindent \MakeUppercase{\sffamily \large \BODY}
	\vspace{1mm}\textcolor{mcgris}{\hrule}\vspace{0.4cm}
	\par
}{}

\NewEnviron{h3}{%
    \needspace{3\baselineskip}
	\vspace{5mm}
	\textsc{\BODY}
	\par
}

\NewEnviron{margeneg}{ %
\begin{addmargin}[-1cm]{0cm}
\BODY
\end{addmargin}
}

\NewEnviron{html}{%
}

\begin{document}
\meta{url}{/exercices/suites-bac-es-l-pondichery-2018/}
\meta{pid}{7099}
\meta{titre}{Suites - Bac ES/L Pondichéry 2018}
\meta{type}{exercice}
\begin{h2}Exercice 3 (5 points)\end{h2}
\textbf{Candidats n'ayant pas suivi l'enseignement de spécialité}
\medskip
On considère la suite $\left(u_n\right)$ définie par $u_0 = 65$ et pour tout entier naturel $n$~:
\par
\[u_{n+1} = 0,8u_n + 18.\]
\medskip
\begin{enumerate}
     \item Calculer $u_1$ et $u_2$.
     \item Pour tout entier naturel $n$, on pose~: $v_n = u_n - 90$.
     \begin{enumerate}[label=\alph*.]
          \item Démontrer que la suite $\left(v_n\right)$ est géométrique de raison $0,8$.
          \par
          On précisera la valeur de $v_0$.
          \item Démontrer que, pour tout entier naturel $n$~:
          \[u_n = 90 - 25 \times  0,8^n.\]
     \end{enumerate}
     \item  On considère l'algorithme ci-dessous~:
     \begin{center}
          \begin{extern}%width="330px" alt="algorithme suite géométrique"
               \begin{tabularx}{0.5\linewidth}{|c|X|}\hline
                    ligne 1&$u \gets 65$\\
                    ligne 2&$n \gets 0$\\
                    ligne 3&Tant que .........\\
                    ligne 4&\hspace{1cm}$n \gets n+1$\\
                    ligne 5&\hspace{1cm}$u \gets 0,8 \times u + 18$\\
                    ligne 6&Fin Tant que\\ \hline
               \end{tabularx}
          \end{extern}
     \end{center}
     \medskip
     \begin{enumerate}[label=\alph*.]
          \item Recopier et compléter la ligne 3 de cet algorithme afin qu'il détermine le plus petit entier
          naturel $n$ tel que $u_n \geqslant 85$.
          \item Quelle est la valeur de la variable $n$ à la fin de l'exécution de l'algorithme~?
          \item Retrouver par le calcul le résultat de la question précédente en résolvant l'inéquation
          $u_n \geqslant 85$.
     \end{enumerate}
     \item  La société Biocagette propose la livraison hebdomadaire d'un panier bio qui contient des fruits
     et des légumes de saison issus de l'agriculture biologique. Les clients ont la possibilité de
     souscrire un abonnement de $52$~\euro par mois qui permet de recevoir chaque semaine ce panier
     bio.
     \par
     En juillet 2017, $65$ particuliers ont souscrit cet abonnement.
     \smallskip
     Les responsables de la société Biocagette font les hypothèses suivantes~:
     \begin{itemize}[label=---]
          \item d'un mois à l'autre, environ 20\,\% des abonnements sont résiliés~;
          \item chaque mois, $18$ particuliers supplémentaires souscrivent à l'abonnement.
     \end{itemize}
     \begin{enumerate}[label=\alph*.]
          \item Justifier que la suite $\left(u_n\right)$ permet de modéliser le nombre d'abonnés au panier bio le $n$-ième mois qui suit le mois de juillet 2017.
          \item Selon ce modèle, la recette mensuelle de la société Biocagette va-t-elle dépasser 4~420~\euro{} durant l'année 2018~? Justifier la réponse.
          \item Selon ce modèle, vers quelle valeur tend la recette mensuelle de la société Biocagette~?
          \par
          Argumenter la réponse.
     \end{enumerate}
\end{enumerate}
\begin{corrige}
     \begin{enumerate}
          \item
          Pour tout entier naturel $n$, $u_{n+1} = 0,8u_n + 18$, par conséquent~:
          $u_1 = 0,8 u_0 + 18 = 0,8\times 65 + 18 = 70$ \\
          $u_2 = 0,8 u_1 + 18 = 0,8\times 70 + 18 = 74$\\
          \item
          \begin{enumerate}[label=\alph*.]
               \item
               Pour tout entier naturel $n$~:
               \par
               $v_{n+1}=u_{n+1}-90 $\\
               $\phantom{v_{n+1}}=0,8 u_n + 18 -90$\\
               $\phantom{v_{n+1}}=0,8u_n-72$.
               \par
               Or $v_n = u_n - 90$~; donc $u_n=v_n+90$~; alors~:
               \par
               $v_{n+1}=0,8 \left ( v_n+90\right ) - 72$\\
               $\phantom{v_{n+1}}=0,8 v_n + 72-72$\\
               $\phantom{v_{n+1}}=0,8v_n$.
               \par
               De plus $v_0=u_0-90 = 65- 90 = -25$~; par conséquent, la suite $(v_n)$ est une suite géométrique de premier terme ${v_0=-25}$  et de raison ${q=0,8}$.
               \item
               On en déduit que~:
               \par
               $v_n=v_0q^n=-25 \times 0,8^n$.
               \par
               Et comme $u_n=v_n+90$, pour tout entier naturel $n$~:
               $u_n=90-25\times 0,8^{n}$.
          \end{enumerate}
          \item
          \begin{enumerate}[label=\alph*.]
               \item  On souhaite déterminer le plus petit entier naturel $n$ tel que $u_n \geqslant 85$.
               \par
               Pour cela, on doit rester dans la boucle \og Tant que \fg{} aussi longtemps que $u_n$ est strictement inférieur à 85.
               \par
               On doit donc compléter l'algorithme comme suit~:
               \begin{center}
                    \begin{extern}%width="330px" alt="algoritme bac correction"
                         \begin{tabularx}{0.5\linewidth}{|c|X|}\hline
                              ligne 1&$u \gets 65$\\
                              ligne 2&$n \gets 0$\\
                              ligne 3&Tant que  $\red{u<85}$\\
                              ligne 4&\hspace{1cm}$n \gets n+1$\\
                              ligne 5&\hspace{1cm}$u \gets 0,8 \times u + 18$\\
                              ligne 6&Fin Tant que\\ \hline
                         \end{tabularx}
                    \end{extern}
               \end{center}
               \medskip
               \item
               \par
               On peut programmer l'algorithme sur sa calculatrice ou plus simplement utiliser le menu \og Suite \fg{} de la calculatrice (ou même le menu \og Fonction \fg{} en entrant la fonction $Y_1= 90-25 \times 0,8\hat~X$ et un pas de 1) de façon à calculer les premiers termes de la suite.
               \par
               On trouve alors~:
               \par
               $u_ 7 ~=  84,8$\\
               \par
               et $u_8 ~= 85,8$\\
               \par
               \`A la fin de l'exécution de l'algorithme, la variable $n$ contiendra donc la valeur 8.
               \item
               $u_n \geqslant 85 \Leftrightarrow 90-25\times 0,8^{n} \geqslant 85$\\
               $\phantom{u_n \geqslant 85 } \Leftrightarrow -25\times 0,8^{n} \geqslant -5$\\
               $\phantom{u_n \geqslant 85 } \Leftrightarrow 25\times 0,8^{n} \leqslant 5$\\
               $\phantom{u_n \geqslant 85 }\Leftrightarrow   0,8^{n} \leqslant \dfrac{5}{25}$\\
               $\phantom{u_n \geqslant 85 }\Leftrightarrow   0,8^{n} \leqslant 0,2$\\
               \par
               Comme la fonction $\ln$ est strictement croissante sur l'intervalle $]0~;~+\infty[$, on peut appliquer, à chaque membre, la fonction $\ln$~:\\
               \par
               $u_n \geqslant 85 \Leftrightarrow   \ln \left ( 0,8^{n} \right )\leqslant \ln 0,2$\\
               $\phantom{u_n \geqslant 85}\Leftrightarrow n \times \ln 0,8 \leqslant \ln 0,2$\\
               \par
               On divise chaque membre par $ \ln 0,8 $ qui est strictement négatif~; il faut donc changer le sens de l'inégalité~:
               \par
               $u_n \geqslant 85 \Leftrightarrow n \geqslant \dfrac{\ln 0,2}{\ln 0,8} $
               \par
               Or $\dfrac{\ln 0,2}{\ln 0,8} \approx 7,21$~; le plus petit entier $n$ vérifiant l'inégalité est donc bien  $n=8$.
          \end{enumerate}
          \item
          \begin{enumerate}[label=\alph*.]
               \item
               \par
               Notons $a_n$ le nombre d'abonnés au panier bio le $n$-ième mois qui suit le mois de juillet 2017.
               \par
               En juillet 2017, 65 particuliers avaient souscrit l'abonnement, donc $a_0=65$.
               \par
               Une diminution de 20\% correspond à un coefficient multiplicateur de ${1-\dfrac{20}{100}=0,8}$~; on ajoute ensuite les 18 nouveaux abonnés.
               \par
               On a donc~:
               \[ a_{n+1}=0,8a_n+60. \]
               Les suites $(u_n)$ et $(a_n)$ sont définies par la même relation de récurrence et le même premier terme~; elles sont donc identiques.
               \par
               La suite $\left(u_n\right)$ permet donc bien de modéliser le nombre d'abonnés au panier bio le $n$-ième mois qui suit le mois de juillet 2017.
               \item
               \par
               Puisque le prix d'un abonnement est  $52$~\euro par mois, la recette mensuelle pour le mois $n$ est $52 u_n$ euros.
               \par
               On cherche donc à résoudre l'inéquation $52 u_n > 4~420$. Or~:
               \par
               $52 u_n > 4~420 \Leftrightarrow u_n > \dfrac{4~420}{52} \Leftrightarrow u_n > 85$.
               \par
               Et d'après la question \textbf{3} ceci produit pour $n=8$ soit 8 mois après le mois de juillet 2017 c'est à dire en mars 2018.
               \par
               La recette mensuelle dépassera donc 4~420~\euro{} durant l'année 2018.
               \item
               La suite $(v_n)$ est une géométrique de raison $q=0,8$. Comme $0<0,8<1$, la suite $(v_n)$ converge vers 0.
               \par
               Or, pour tout entier naturel $n$, $u_n=v_n+90$ donc  $\lim\limits_{n \rightarrow +\infty}u_n=90$.
               \par
               Au cours du temps le nombre d'abonnés se rapprochera de $90$.
               \par
               La recette mensuelle  de la société Biocagette tendra donc vers $52 \times 90= 4~680$~\euro.
          \end{enumerate}
     \end{enumerate}
\end{corrige}

\end{document}