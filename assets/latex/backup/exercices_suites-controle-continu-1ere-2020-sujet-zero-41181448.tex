\documentclass[a4paper]{article}

%================================================================================================================================
%
% Packages
%
%================================================================================================================================

\usepackage[T1]{fontenc} 	% pour caractères accentués
\usepackage[utf8]{inputenc}  % encodage utf8
\usepackage[french]{babel}	% langue : français
\usepackage{fourier}			% caractères plus lisibles
\usepackage[dvipsnames]{xcolor} % couleurs
\usepackage{fancyhdr}		% réglage header footer
\usepackage{needspace}		% empêcher sauts de page mal placés
\usepackage{graphicx}		% pour inclure des graphiques
\usepackage{enumitem,cprotect}		% personnalise les listes d'items (nécessaire pour ol, al ...)
\usepackage{hyperref}		% Liens hypertexte
\usepackage{pstricks,pst-all,pst-node,pstricks-add,pst-math,pst-plot,pst-tree,pst-eucl} % pstricks
\usepackage[a4paper,includeheadfoot,top=2cm,left=3cm, bottom=2cm,right=3cm]{geometry} % marges etc.
\usepackage{comment}			% commentaires multilignes
\usepackage{amsmath,environ} % maths (matrices, etc.)
\usepackage{amssymb,makeidx}
\usepackage{bm}				% bold maths
\usepackage{tabularx}		% tableaux
\usepackage{colortbl}		% tableaux en couleur
\usepackage{fontawesome}		% Fontawesome
\usepackage{environ}			% environment with command
\usepackage{fp}				% calculs pour ps-tricks
\usepackage{multido}			% pour ps tricks
\usepackage[np]{numprint}	% formattage nombre
\usepackage{tikz,tkz-tab} 			% package principal TikZ
\usepackage{pgfplots}   % axes
\usepackage{mathrsfs}    % cursives
\usepackage{calc}			% calcul taille boites
\usepackage[scaled=0.875]{helvet} % font sans serif
\usepackage{svg} % svg
\usepackage{scrextend} % local margin
\usepackage{scratch} %scratch
\usepackage{multicol} % colonnes
%\usepackage{infix-RPN,pst-func} % formule en notation polanaise inversée
\usepackage{listings}

%================================================================================================================================
%
% Réglages de base
%
%================================================================================================================================

\lstset{
language=Python,   % R code
literate=
{á}{{\'a}}1
{à}{{\`a}}1
{ã}{{\~a}}1
{é}{{\'e}}1
{è}{{\`e}}1
{ê}{{\^e}}1
{í}{{\'i}}1
{ó}{{\'o}}1
{õ}{{\~o}}1
{ú}{{\'u}}1
{ü}{{\"u}}1
{ç}{{\c{c}}}1
{~}{{ }}1
}


\definecolor{codegreen}{rgb}{0,0.6,0}
\definecolor{codegray}{rgb}{0.5,0.5,0.5}
\definecolor{codepurple}{rgb}{0.58,0,0.82}
\definecolor{backcolour}{rgb}{0.95,0.95,0.92}

\lstdefinestyle{mystyle}{
    backgroundcolor=\color{backcolour},   
    commentstyle=\color{codegreen},
    keywordstyle=\color{magenta},
    numberstyle=\tiny\color{codegray},
    stringstyle=\color{codepurple},
    basicstyle=\ttfamily\footnotesize,
    breakatwhitespace=false,         
    breaklines=true,                 
    captionpos=b,                    
    keepspaces=true,                 
    numbers=left,                    
xleftmargin=2em,
framexleftmargin=2em,            
    showspaces=false,                
    showstringspaces=false,
    showtabs=false,                  
    tabsize=2,
    upquote=true
}

\lstset{style=mystyle}


\lstset{style=mystyle}
\newcommand{\imgdir}{C:/laragon/www/newmc/assets/imgsvg/}
\newcommand{\imgsvgdir}{C:/laragon/www/newmc/assets/imgsvg/}

\definecolor{mcgris}{RGB}{220, 220, 220}% ancien~; pour compatibilité
\definecolor{mcbleu}{RGB}{52, 152, 219}
\definecolor{mcvert}{RGB}{125, 194, 70}
\definecolor{mcmauve}{RGB}{154, 0, 215}
\definecolor{mcorange}{RGB}{255, 96, 0}
\definecolor{mcturquoise}{RGB}{0, 153, 153}
\definecolor{mcrouge}{RGB}{255, 0, 0}
\definecolor{mclightvert}{RGB}{205, 234, 190}

\definecolor{gris}{RGB}{220, 220, 220}
\definecolor{bleu}{RGB}{52, 152, 219}
\definecolor{vert}{RGB}{125, 194, 70}
\definecolor{mauve}{RGB}{154, 0, 215}
\definecolor{orange}{RGB}{255, 96, 0}
\definecolor{turquoise}{RGB}{0, 153, 153}
\definecolor{rouge}{RGB}{255, 0, 0}
\definecolor{lightvert}{RGB}{205, 234, 190}
\setitemize[0]{label=\color{lightvert}  $\bullet$}

\pagestyle{fancy}
\renewcommand{\headrulewidth}{0.2pt}
\fancyhead[L]{maths-cours.fr}
\fancyhead[R]{\thepage}
\renewcommand{\footrulewidth}{0.2pt}
\fancyfoot[C]{}

\newcolumntype{C}{>{\centering\arraybackslash}X}
\newcolumntype{s}{>{\hsize=.35\hsize\arraybackslash}X}

\setlength{\parindent}{0pt}		 
\setlength{\parskip}{3mm}
\setlength{\headheight}{1cm}

\def\ebook{ebook}
\def\book{book}
\def\web{web}
\def\type{web}

\newcommand{\vect}[1]{\overrightarrow{\,\mathstrut#1\,}}

\def\Oij{$\left(\text{O}~;~\vect{\imath},~\vect{\jmath}\right)$}
\def\Oijk{$\left(\text{O}~;~\vect{\imath},~\vect{\jmath},~\vect{k}\right)$}
\def\Ouv{$\left(\text{O}~;~\vect{u},~\vect{v}\right)$}

\hypersetup{breaklinks=true, colorlinks = true, linkcolor = OliveGreen, urlcolor = OliveGreen, citecolor = OliveGreen, pdfauthor={Didier BONNEL - https://www.maths-cours.fr} } % supprime les bordures autour des liens

\renewcommand{\arg}[0]{\text{arg}}

\everymath{\displaystyle}

%================================================================================================================================
%
% Macros - Commandes
%
%================================================================================================================================

\newcommand\meta[2]{    			% Utilisé pour créer le post HTML.
	\def\titre{titre}
	\def\url{url}
	\def\arg{#1}
	\ifx\titre\arg
		\newcommand\maintitle{#2}
		\fancyhead[L]{#2}
		{\Large\sffamily \MakeUppercase{#2}}
		\vspace{1mm}\textcolor{mcvert}{\hrule}
	\fi 
	\ifx\url\arg
		\fancyfoot[L]{\href{https://www.maths-cours.fr#2}{\black \footnotesize{https://www.maths-cours.fr#2}}}
	\fi 
}


\newcommand\TitreC[1]{    		% Titre centré
     \needspace{3\baselineskip}
     \begin{center}\textbf{#1}\end{center}
}

\newcommand\newpar{    		% paragraphe
     \par
}

\newcommand\nosp {    		% commande vide (pas d'espace)
}
\newcommand{\id}[1]{} %ignore

\newcommand\boite[2]{				% Boite simple sans titre
	\vspace{5mm}
	\setlength{\fboxrule}{0.2mm}
	\setlength{\fboxsep}{5mm}	
	\fcolorbox{#1}{#1!3}{\makebox[\linewidth-2\fboxrule-2\fboxsep]{
  		\begin{minipage}[t]{\linewidth-2\fboxrule-4\fboxsep}\setlength{\parskip}{3mm}
  			 #2
  		\end{minipage}
	}}
	\vspace{5mm}
}

\newcommand\CBox[4]{				% Boites
	\vspace{5mm}
	\setlength{\fboxrule}{0.2mm}
	\setlength{\fboxsep}{5mm}
	
	\fcolorbox{#1}{#1!3}{\makebox[\linewidth-2\fboxrule-2\fboxsep]{
		\begin{minipage}[t]{1cm}\setlength{\parskip}{3mm}
	  		\textcolor{#1}{\LARGE{#2}}    
 	 	\end{minipage}  
  		\begin{minipage}[t]{\linewidth-2\fboxrule-4\fboxsep}\setlength{\parskip}{3mm}
			\raisebox{1.2mm}{\normalsize\sffamily{\textcolor{#1}{#3}}}						
  			 #4
  		\end{minipage}
	}}
	\vspace{5mm}
}

\newcommand\cadre[3]{				% Boites convertible html
	\par
	\vspace{2mm}
	\setlength{\fboxrule}{0.1mm}
	\setlength{\fboxsep}{5mm}
	\fcolorbox{#1}{white}{\makebox[\linewidth-2\fboxrule-2\fboxsep]{
  		\begin{minipage}[t]{\linewidth-2\fboxrule-4\fboxsep}\setlength{\parskip}{3mm}
			\raisebox{-2.5mm}{\sffamily \small{\textcolor{#1}{\MakeUppercase{#2}}}}		
			\par		
  			 #3
 	 		\end{minipage}
	}}
		\vspace{2mm}
	\par
}

\newcommand\bloc[3]{				% Boites convertible html sans bordure
     \needspace{2\baselineskip}
     {\sffamily \small{\textcolor{#1}{\MakeUppercase{#2}}}}    
		\par		
  			 #3
		\par
}

\newcommand\CHelp[1]{
     \CBox{Plum}{\faInfoCircle}{À RETENIR}{#1}
}

\newcommand\CUp[1]{
     \CBox{NavyBlue}{\faThumbsOUp}{EN PRATIQUE}{#1}
}

\newcommand\CInfo[1]{
     \CBox{Sepia}{\faArrowCircleRight}{REMARQUE}{#1}
}

\newcommand\CRedac[1]{
     \CBox{PineGreen}{\faEdit}{BIEN R\'EDIGER}{#1}
}

\newcommand\CError[1]{
     \CBox{Red}{\faExclamationTriangle}{ATTENTION}{#1}
}

\newcommand\TitreExo[2]{
\needspace{4\baselineskip}
 {\sffamily\large EXERCICE #1\ (\emph{#2 points})}
\vspace{5mm}
}

\newcommand\img[2]{
          \includegraphics[width=#2\paperwidth]{\imgdir#1}
}

\newcommand\imgsvg[2]{
       \begin{center}   \includegraphics[width=#2\paperwidth]{\imgsvgdir#1} \end{center}
}


\newcommand\Lien[2]{
     \href{#1}{#2 \tiny \faExternalLink}
}
\newcommand\mcLien[2]{
     \href{https~://www.maths-cours.fr/#1}{#2 \tiny \faExternalLink}
}

\newcommand{\euro}{\eurologo{}}

%================================================================================================================================
%
% Macros - Environement
%
%================================================================================================================================

\newenvironment{tex}{ %
}
{%
}

\newenvironment{indente}{ %
	\setlength\parindent{10mm}
}

{
	\setlength\parindent{0mm}
}

\newenvironment{corrige}{%
     \needspace{3\baselineskip}
     \medskip
     \textbf{\textsc{Corrigé}}
     \medskip
}
{
}

\newenvironment{extern}{%
     \begin{center}
     }
     {
     \end{center}
}

\NewEnviron{code}{%
	\par
     \boite{gray}{\texttt{%
     \BODY
     }}
     \par
}

\newenvironment{vbloc}{% boite sans cadre empeche saut de page
     \begin{minipage}[t]{\linewidth}
     }
     {
     \end{minipage}
}
\NewEnviron{h2}{%
    \needspace{3\baselineskip}
    \vspace{0.6cm}
	\noindent \MakeUppercase{\sffamily \large \BODY}
	\vspace{1mm}\textcolor{mcgris}{\hrule}\vspace{0.4cm}
	\par
}{}

\NewEnviron{h3}{%
    \needspace{3\baselineskip}
	\vspace{5mm}
	\textsc{\BODY}
	\par
}

\NewEnviron{margeneg}{ %
\begin{addmargin}[-1cm]{0cm}
\BODY
\end{addmargin}
}

\NewEnviron{html}{%
}

\begin{document}
\meta{url}{/exercices/suites-controle-continu-1ere-2020-sujet-zero/}
\meta{pid}{11213}
\meta{titre}{Suites - Contrôle continu 1ère - 2020 - Sujet zéro}
\meta{type}{exercices}
%
\begin{h2}Exercice 4 (5 points)\end{h2}
En traversant une plaque de verre teintée, un rayon lumineux perd 20  \% de son intensité lumineuse. L'intensité lumineuse est exprimée en candela ( cd ).
\newpar
On utilise une lampe torche qui émet un rayon d'intensité lumineuse réglée à 400 cd.
\newpar
On superpose $n$ plaques de verre identiques (  $n$ étant un entier naturel ) et on désire mesurer l'intensité lumineuse $I_{ n }$ du rayon à la sortie de la $n$-ième plaque.
\newpar
On note $I_{ 0 }=400$ l'intensité lumineuse du rayon émis par la lampe torche avant de traverser les plaques ( intensité lumineuse initiale ). Ainsi, cette situation est modélisée par la suite $( I_{ n } ).$
\begin{enumerate}
     \item
     Montrer par un calcul que $I_{ 1 }=320.$
     \item
     \begin{enumerate}[label=\alph*.]
          \item
          Pour tout entier naturel $n$ , exprimer $I_{ n+1 }$ en fonction de $I_{ n }.$
          \item
          En déduire la nature de la suite $( I_{ n } ).$ Préciser sa raison et son premier terme.
          \item
          Pour tout entier naturel $n$ , exprimer $I_{ n }$ en fonction de $n. $
     \end{enumerate}
     \item
     On souhaite déterminer le nombre minimal $n$ de plaques à superposer afin que le rayon initial ait perdu au moins 70 \% de son intensité lumineuse initiale après sa traversée des plaques.
     \begin{enumerate}[label=\alph*.]
          \item
          Afin de déterminer le nombre de plaques à superposer, on considère la fonction Python suivante~:
\begin{lstlisting}[language=Python]
def nombrePlaques(J):
	I=400
	n=0
	while I > J:
		I = 0.8*I
		n = n+1
	return n
     \end{lstlisting}
     Préciser, en justifiant, le nombre \texttt{J} de sorte que l'appel \texttt{nombrePlaques(J)} renvoie le nombre de plaques à superposer.
     \item
     Le tableau suivant donne des valeurs de $I_{ n }.$ Combien de plaques doit-on superposer ?
     \begin{center}
          \begin{tabular}{|c|c|c|c|c|c|c|c|c|}%class="compact" width="600"
               \hline
               $n$ &  0 & 1 & 2 & 3 & 4 & 5 & 6 & 7 \\
               \hline
               $I_{ n }$  & 400 & 320 & 256 & 204,8 & 163,84 & 131,07 & 104,85 & 83,886 \\
               \hline
          \end{tabular}
     \end{center}
\end{enumerate}
\end{enumerate}
\begin{corrige}
     \begin{enumerate}
          \item
          Le coefficient multiplicateur correspondant à une baisse de 20 \% est~:
          \newpar
          $CM=1 - \frac{ 20 }{ 100 }=0,8$
          \newpar
          L'intensité lumineuse $I_{ 1 }$ à la sortie de la première plaque est donc~:
          \newpar
          $I_{ 1 }=0,8 \times I_{ 0 }=0,8 \times 400=320$
          \medbreak
          \textbf{Remarque~: }On aurait également pu calculer la diminution de l'intensité lumineuse qui est égale à $\frac{ 20 }{ 100 } \times 400=80$, puis la nouvelle intensité $I_{ 1 }=400 - 80=320$ . Mais il est préférable de s'habituer à utiliser le coefficient multiplicateur qui facilite les calculs lors d'augmentation ou de diminution en pourcentage.
          \item
          \begin{enumerate}[label=\alph*.]
               \item
               De même, le raisonnement précédent indique que, pour tout entier naturel $n$ ~: \\
               $I_{ n+1 }=0,8 \times I_{ n }.$
               \item
               La formule précédente prouve que la suite $( I_{ n } )$ est une suite géométrique de raison  $q=0,8$~; son premier terme est $I_{ 0 }=400.$
               \item
               D'après le cours, le n-ième terme d'une suite géométrique de premier terme $u_{ 0 }$ et de raison $q$ est donné par la \mcLien{https://www.maths-cours.fr/cours/suites-geometriques/\#p110}{formule}~:
               \newpar
               \[
               u_{ n }=u_{ 0 } \times q{}^{ n }
               \]
               On obtient ici~:
               \newpar
               $I_{ n }=I_{ 0 } \times q{}^{ n }=400 \times 0,8{}^{ n }$
          \end{enumerate}
          \item
          \begin{enumerate}[label=\alph*.]
               \item
               L'appel à la fonction Python \texttt{nombrePlaques( )} avec l'argument  \texttt{J} renvoie le nombre minimal de plaques à superposer afin que l'intensité lumineuse du rayon à la sortie de la n-ième plaque soit inférieure ou égale à  \texttt{J}.
               \newpar
               Puisque l'on veut que le rayon initial perde au moins 70 \% de son intensité lumineuse, il faut que l'intensité lumineuse à la sortie de la n-ième plaque soit inférieure à 30 \% de  $I_{ 0 }$ c'est-à-dire inférieure à $\frac{ 30 }{ 100 } \times 400=120.$
               \newpar
               Il faut donc choisir le nombre  \texttt{J} = 120, pour obtenir, en sortie de la fonction \texttt{nombrePlaques( )}, le nombre de plaques à superposer.
               \item
               Le tableau montre qu'il faut choisir 6 plaques pour obtenir une intensité lumineuse inférieure ou égale à 120 cd.
          \end{enumerate}
     \end{enumerate}
\end{corrige}

\end{document}