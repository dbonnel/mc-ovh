\documentclass[a4paper]{article}

%================================================================================================================================
%
% Packages
%
%================================================================================================================================

\usepackage[T1]{fontenc} 	% pour caractères accentués
\usepackage[utf8]{inputenc}  % encodage utf8
\usepackage[french]{babel}	% langue : français
\usepackage{fourier}			% caractères plus lisibles
\usepackage[dvipsnames]{xcolor} % couleurs
\usepackage{fancyhdr}		% réglage header footer
\usepackage{needspace}		% empêcher sauts de page mal placés
\usepackage{graphicx}		% pour inclure des graphiques
\usepackage{enumitem,cprotect}		% personnalise les listes d'items (nécessaire pour ol, al ...)
\usepackage{hyperref}		% Liens hypertexte
\usepackage{pstricks,pst-all,pst-node,pstricks-add,pst-math,pst-plot,pst-tree,pst-eucl} % pstricks
\usepackage[a4paper,includeheadfoot,top=2cm,left=3cm, bottom=2cm,right=3cm]{geometry} % marges etc.
\usepackage{comment}			% commentaires multilignes
\usepackage{amsmath,environ} % maths (matrices, etc.)
\usepackage{amssymb,makeidx}
\usepackage{bm}				% bold maths
\usepackage{tabularx}		% tableaux
\usepackage{colortbl}		% tableaux en couleur
\usepackage{fontawesome}		% Fontawesome
\usepackage{environ}			% environment with command
\usepackage{fp}				% calculs pour ps-tricks
\usepackage{multido}			% pour ps tricks
\usepackage[np]{numprint}	% formattage nombre
\usepackage{tikz,tkz-tab} 			% package principal TikZ
\usepackage{pgfplots}   % axes
\usepackage{mathrsfs}    % cursives
\usepackage{calc}			% calcul taille boites
\usepackage[scaled=0.875]{helvet} % font sans serif
\usepackage{svg} % svg
\usepackage{scrextend} % local margin
\usepackage{scratch} %scratch
\usepackage{multicol} % colonnes
%\usepackage{infix-RPN,pst-func} % formule en notation polanaise inversée
\usepackage{listings}

%================================================================================================================================
%
% Réglages de base
%
%================================================================================================================================

\lstset{
language=Python,   % R code
literate=
{á}{{\'a}}1
{à}{{\`a}}1
{ã}{{\~a}}1
{é}{{\'e}}1
{è}{{\`e}}1
{ê}{{\^e}}1
{í}{{\'i}}1
{ó}{{\'o}}1
{õ}{{\~o}}1
{ú}{{\'u}}1
{ü}{{\"u}}1
{ç}{{\c{c}}}1
{~}{{ }}1
}


\definecolor{codegreen}{rgb}{0,0.6,0}
\definecolor{codegray}{rgb}{0.5,0.5,0.5}
\definecolor{codepurple}{rgb}{0.58,0,0.82}
\definecolor{backcolour}{rgb}{0.95,0.95,0.92}

\lstdefinestyle{mystyle}{
    backgroundcolor=\color{backcolour},   
    commentstyle=\color{codegreen},
    keywordstyle=\color{magenta},
    numberstyle=\tiny\color{codegray},
    stringstyle=\color{codepurple},
    basicstyle=\ttfamily\footnotesize,
    breakatwhitespace=false,         
    breaklines=true,                 
    captionpos=b,                    
    keepspaces=true,                 
    numbers=left,                    
xleftmargin=2em,
framexleftmargin=2em,            
    showspaces=false,                
    showstringspaces=false,
    showtabs=false,                  
    tabsize=2,
    upquote=true
}

\lstset{style=mystyle}


\lstset{style=mystyle}
\newcommand{\imgdir}{C:/laragon/www/newmc/assets/imgsvg/}
\newcommand{\imgsvgdir}{C:/laragon/www/newmc/assets/imgsvg/}

\definecolor{mcgris}{RGB}{220, 220, 220}% ancien~; pour compatibilité
\definecolor{mcbleu}{RGB}{52, 152, 219}
\definecolor{mcvert}{RGB}{125, 194, 70}
\definecolor{mcmauve}{RGB}{154, 0, 215}
\definecolor{mcorange}{RGB}{255, 96, 0}
\definecolor{mcturquoise}{RGB}{0, 153, 153}
\definecolor{mcrouge}{RGB}{255, 0, 0}
\definecolor{mclightvert}{RGB}{205, 234, 190}

\definecolor{gris}{RGB}{220, 220, 220}
\definecolor{bleu}{RGB}{52, 152, 219}
\definecolor{vert}{RGB}{125, 194, 70}
\definecolor{mauve}{RGB}{154, 0, 215}
\definecolor{orange}{RGB}{255, 96, 0}
\definecolor{turquoise}{RGB}{0, 153, 153}
\definecolor{rouge}{RGB}{255, 0, 0}
\definecolor{lightvert}{RGB}{205, 234, 190}
\setitemize[0]{label=\color{lightvert}  $\bullet$}

\pagestyle{fancy}
\renewcommand{\headrulewidth}{0.2pt}
\fancyhead[L]{maths-cours.fr}
\fancyhead[R]{\thepage}
\renewcommand{\footrulewidth}{0.2pt}
\fancyfoot[C]{}

\newcolumntype{C}{>{\centering\arraybackslash}X}
\newcolumntype{s}{>{\hsize=.35\hsize\arraybackslash}X}

\setlength{\parindent}{0pt}		 
\setlength{\parskip}{3mm}
\setlength{\headheight}{1cm}

\def\ebook{ebook}
\def\book{book}
\def\web{web}
\def\type{web}

\newcommand{\vect}[1]{\overrightarrow{\,\mathstrut#1\,}}

\def\Oij{$\left(\text{O}~;~\vect{\imath},~\vect{\jmath}\right)$}
\def\Oijk{$\left(\text{O}~;~\vect{\imath},~\vect{\jmath},~\vect{k}\right)$}
\def\Ouv{$\left(\text{O}~;~\vect{u},~\vect{v}\right)$}

\hypersetup{breaklinks=true, colorlinks = true, linkcolor = OliveGreen, urlcolor = OliveGreen, citecolor = OliveGreen, pdfauthor={Didier BONNEL - https://www.maths-cours.fr} } % supprime les bordures autour des liens

\renewcommand{\arg}[0]{\text{arg}}

\everymath{\displaystyle}

%================================================================================================================================
%
% Macros - Commandes
%
%================================================================================================================================

\newcommand\meta[2]{    			% Utilisé pour créer le post HTML.
	\def\titre{titre}
	\def\url{url}
	\def\arg{#1}
	\ifx\titre\arg
		\newcommand\maintitle{#2}
		\fancyhead[L]{#2}
		{\Large\sffamily \MakeUppercase{#2}}
		\vspace{1mm}\textcolor{mcvert}{\hrule}
	\fi 
	\ifx\url\arg
		\fancyfoot[L]{\href{https://www.maths-cours.fr#2}{\black \footnotesize{https://www.maths-cours.fr#2}}}
	\fi 
}


\newcommand\TitreC[1]{    		% Titre centré
     \needspace{3\baselineskip}
     \begin{center}\textbf{#1}\end{center}
}

\newcommand\newpar{    		% paragraphe
     \par
}

\newcommand\nosp {    		% commande vide (pas d'espace)
}
\newcommand{\id}[1]{} %ignore

\newcommand\boite[2]{				% Boite simple sans titre
	\vspace{5mm}
	\setlength{\fboxrule}{0.2mm}
	\setlength{\fboxsep}{5mm}	
	\fcolorbox{#1}{#1!3}{\makebox[\linewidth-2\fboxrule-2\fboxsep]{
  		\begin{minipage}[t]{\linewidth-2\fboxrule-4\fboxsep}\setlength{\parskip}{3mm}
  			 #2
  		\end{minipage}
	}}
	\vspace{5mm}
}

\newcommand\CBox[4]{				% Boites
	\vspace{5mm}
	\setlength{\fboxrule}{0.2mm}
	\setlength{\fboxsep}{5mm}
	
	\fcolorbox{#1}{#1!3}{\makebox[\linewidth-2\fboxrule-2\fboxsep]{
		\begin{minipage}[t]{1cm}\setlength{\parskip}{3mm}
	  		\textcolor{#1}{\LARGE{#2}}    
 	 	\end{minipage}  
  		\begin{minipage}[t]{\linewidth-2\fboxrule-4\fboxsep}\setlength{\parskip}{3mm}
			\raisebox{1.2mm}{\normalsize\sffamily{\textcolor{#1}{#3}}}						
  			 #4
  		\end{minipage}
	}}
	\vspace{5mm}
}

\newcommand\cadre[3]{				% Boites convertible html
	\par
	\vspace{2mm}
	\setlength{\fboxrule}{0.1mm}
	\setlength{\fboxsep}{5mm}
	\fcolorbox{#1}{white}{\makebox[\linewidth-2\fboxrule-2\fboxsep]{
  		\begin{minipage}[t]{\linewidth-2\fboxrule-4\fboxsep}\setlength{\parskip}{3mm}
			\raisebox{-2.5mm}{\sffamily \small{\textcolor{#1}{\MakeUppercase{#2}}}}		
			\par		
  			 #3
 	 		\end{minipage}
	}}
		\vspace{2mm}
	\par
}

\newcommand\bloc[3]{				% Boites convertible html sans bordure
     \needspace{2\baselineskip}
     {\sffamily \small{\textcolor{#1}{\MakeUppercase{#2}}}}    
		\par		
  			 #3
		\par
}

\newcommand\CHelp[1]{
     \CBox{Plum}{\faInfoCircle}{À RETENIR}{#1}
}

\newcommand\CUp[1]{
     \CBox{NavyBlue}{\faThumbsOUp}{EN PRATIQUE}{#1}
}

\newcommand\CInfo[1]{
     \CBox{Sepia}{\faArrowCircleRight}{REMARQUE}{#1}
}

\newcommand\CRedac[1]{
     \CBox{PineGreen}{\faEdit}{BIEN R\'EDIGER}{#1}
}

\newcommand\CError[1]{
     \CBox{Red}{\faExclamationTriangle}{ATTENTION}{#1}
}

\newcommand\TitreExo[2]{
\needspace{4\baselineskip}
 {\sffamily\large EXERCICE #1\ (\emph{#2 points})}
\vspace{5mm}
}

\newcommand\img[2]{
          \includegraphics[width=#2\paperwidth]{\imgdir#1}
}

\newcommand\imgsvg[2]{
       \begin{center}   \includegraphics[width=#2\paperwidth]{\imgsvgdir#1} \end{center}
}


\newcommand\Lien[2]{
     \href{#1}{#2 \tiny \faExternalLink}
}
\newcommand\mcLien[2]{
     \href{https~://www.maths-cours.fr/#1}{#2 \tiny \faExternalLink}
}

\newcommand{\euro}{\eurologo{}}

%================================================================================================================================
%
% Macros - Environement
%
%================================================================================================================================

\newenvironment{tex}{ %
}
{%
}

\newenvironment{indente}{ %
	\setlength\parindent{10mm}
}

{
	\setlength\parindent{0mm}
}

\newenvironment{corrige}{%
     \needspace{3\baselineskip}
     \medskip
     \textbf{\textsc{Corrigé}}
     \medskip
}
{
}

\newenvironment{extern}{%
     \begin{center}
     }
     {
     \end{center}
}

\NewEnviron{code}{%
	\par
     \boite{gray}{\texttt{%
     \BODY
     }}
     \par
}

\newenvironment{vbloc}{% boite sans cadre empeche saut de page
     \begin{minipage}[t]{\linewidth}
     }
     {
     \end{minipage}
}
\NewEnviron{h2}{%
    \needspace{3\baselineskip}
    \vspace{0.6cm}
	\noindent \MakeUppercase{\sffamily \large \BODY}
	\vspace{1mm}\textcolor{mcgris}{\hrule}\vspace{0.4cm}
	\par
}{}

\NewEnviron{h3}{%
    \needspace{3\baselineskip}
	\vspace{5mm}
	\textsc{\BODY}
	\par
}

\NewEnviron{margeneg}{ %
\begin{addmargin}[-1cm]{0cm}
\BODY
\end{addmargin}
}

\NewEnviron{html}{%
}

\begin{document}
\meta{url}{/exercices/geometrie-espace-bac-s-pondichery-2016/}
\meta{pid}{4070}
\meta{titre}{Géométrie dans l'espace – Bac S Pondichéry 2016}
\meta{type}{exercices}
%
\begin{h2}Exercice 3 - 5 points\end{h2}
\textbf{Candidats n'ayant pas suivi l'enseignement de spécialité}

\begin{center}
\imgsvg{geometrie-espace-bac-s-pondichery-2016-1}{0.3}% alt="Géométrie dans l'espace – Bac S Pondichéry 2016 - 1" style="width:30rem"
\end{center}

$ABCDEFGH$ désigne un cube de côté $1$.
\par
Le point $I$ est le milieu du segment $[BF]$.
\par
Le point $J$ est le milieu du segment $[BC]$.
\par
Le point $K$ est le milieu du segment $[CD]$.
\begin{h3}Partie A\end{h3}
\textbf{Dans cette partie, on ne demande aucune justification}
On admet que les droites $(IJ)$ et $(CG)$ sont sécantes en un point $L$.
\par
Construire, sur la figure fournie en annexe et en laissant apparents les traits de construction :
\begin{itemize}
     \item
     le point $L$;
     \item
     l'intersection$\mathscr{D}$ des plans $(IJK)$ et $(CDH)$;
     \item
     la section du cube par le plan $(IJK)$
\end{itemize}
\begin{h3}Partie B\end{h3}
L'espace est rapporté au repère $\left(A ~;~\overrightarrow{AB},~\overrightarrow{AD},~\overrightarrow{AE}\right)$.
\begin{enumerate}
     \item
     Donner les coordonnées de $A, G, I, J$ et $K$ dans ce repère.
     \item
     \begin{enumerate}[label=\alph*.]
          \item
          Montrer que le vecteur $\overrightarrow{AG}$ est normal au plan $(IJK)$.
          \item
          En déduire une équation cartésienne du plan $(IJK)$.
     \end{enumerate}
     \item
     On désigne par $M$ un point du segment $[AG]$ et $t$ le réel de l'intervalle $[0~;~1]$ tel que $\overrightarrow{AM} = t\overrightarrow{AG}$.
     \begin{enumerate}[label=\alph*.]
          \item
          Démontrer que $M\text{I}^2 = 3t^2-3t+\dfrac{5}{4}$.
          \item
     Démontrer que la distance $MI$ est minimale pour le point $M\left(\dfrac{1}{2}~;~\dfrac{1}{2}~;~\dfrac{1}{2}\right)$.\end{enumerate}
     \item
     Démontrer que pour ce point $M\left(\dfrac{1}{2}~;~\dfrac{1}{2}~;~\dfrac{1}{2}\right)$ :
     \begin{enumerate}[label=\alph*.]
          \item
          $M$ appartient au plan $(IJK)$.
          \item
          La droite ($IM$) est perpendiculaire aux droites $(AG)$ et $(BF)$.
     \end{enumerate}
\end{enumerate}
\begin{corrige}
     \begin{h3}Partie A\end{h3}
     \begin{itemize}
          \item
          Les points $I, J,C$ et $G$ sont coplanaires. Pour placer le point $L$, il suffit de prolonger les droites $(IJ)$ et $(GC)$.
          \item
          Les points $K$ et $L$ appartiennent tous deux aux plans $IJK$ et $CDH$. L'intersection$\mathscr{D}$ de ces plans est donc la droite $(LK)$. Cette droite coupe le côté $[DH]$ en un point $P$.
          \item
          La section du cube par le plan $(IJK)$ a pour côtés $[IJ], [JK]$ et $[KP]$. Les trois autres côtés s'obtiennent en traçant les parallèles à $[IJ], [JK]$ et $[KP]$. On obtient ainsi un hexagone régulier $IJKPQR$.
     \end{itemize}
\begin{center}
\imgsvg{geometrie-espace-bac-s-pondichery-2016-2}{0.3}% alt="Géométrie dans l'espace – Bac S Pondichéry 2016 - 2" style="width:30rem"
\end{center}
     \begin{h3}Partie B\end{h3}
     \begin{enumerate}
          \item
          Par lecture directe :
          \par
          $A(0;0;0)$
          \par
          $G(1;1;1)$
          \par
          $I\left(1;0;\frac{1}{2}\right)$
          \par
          $J\left(1;\frac{1}{2};0\right)$
          \par
          $K\left(\frac{1}{2};1;0\right)$
          \item
          \begin{enumerate}[label=\alph*.]
               \item
               Pour montrer que le vecteur $\overrightarrow{AG}$ est normal au plan $(IJK)$, il suffit de montrer que $\overrightarrow{AG}$ est orthogonal à deux vecteurs non colinéaires de ce plan, par exemple $\overrightarrow{IJ}$ et $\overrightarrow{JK}$.
               \begin{itemize}
                    \item
                    Les coordonnées de $\overrightarrow{IJ}$ sont $\begin{pmatrix} 0 \\ 1/2 \\ -1/2 \end{pmatrix}$ et les coordonnées de $\overrightarrow{AG}$ sont $\begin{pmatrix} 1 \\ 1 \\ 1 \end{pmatrix}$.
                    \par
                    $\overrightarrow{IJ}.\overrightarrow{AG}=0 \times 1+\frac{1}{2} \times 1-\frac{1}{2} \times 1 = 0$
                    \par
                    Donc les vecteurs $\overrightarrow{IJ}$ et $\overrightarrow{AG}$ sont orthogonaux.
                    \item
                    Les coordonnées de $\overrightarrow{JK}$ sont $\begin{pmatrix} -1/2 \\ 1/2 \\ 0 \end{pmatrix}$.
                    \par
                    $\overrightarrow{JK}.\overrightarrow{AG}=-\frac{1}{2} \times 1+\frac{1}{2} \times 1 +0 \times 1= 0$
                    \par
                    Donc les vecteurs $\overrightarrow{JK}$ et $\overrightarrow{AG}$ sont orthogonaux.
               \end{itemize}
               Le vecteur $\overrightarrow{AG}$ est donc normal au plan $(IJK)$.
               \item
               Le plan $(IJK)$ admet donc une équation cartésienne de la forme $x+y+z+d=0$.
               \par
               Ce plan passant par $I$, les coordonnées de $I$ vérifient l'équation.
               \par
               Par conséquent :
               \par
               $1+0+\frac{1}{2}+d=0$
               \par
               $d=-\frac{3}{2}$
               \par
               Une équation cartésienne du plan $(IJK)$ est donc $x+y+z-\frac{3}{2}=0$
          \end{enumerate}
          \item
          \begin{enumerate}
               \item
               Les coordonnées du point $G$ étant $(1;1;1)$ et $A$ étant l'origine du repère, la relation $\overrightarrow{AM} = t\overrightarrow{AG}$ entraîne que les coordonnées de $M$ sont $(t;t;t)$.
               \par
               Alors :
               \par
               $MI^2=(1-t)^2+(-t)^2+ \left(\frac{1}{2}-t \right)^2$
               \par
               $\phantom{MI^2}=1-2t+t^2+t^2+\frac{1}{4}-t +t^2$
               \par
               $\phantom{MI^2}= 3t^2-3t+\dfrac{5}{4}$
               \item
               La fonction carrée étant strictement croissante sur $\mathbb{R}^+$, $MI^2$ et $MI$ ont des sens de variations identiques.
               \par
               $MI^2$ est un polynôme du second degré en $t$ de coefficients $a=3,\ b=-3$ et $c=\frac{5}{4}$.
               \par
               $a>0$ donc $MI^2$ admet un minimum pour $t_0=-\frac{b}{2a}=\frac{1}{2}$. Les coordonnées de $M$ sont alors $\left(\dfrac{1}{2}~;~\dfrac{1}{2}~;~\dfrac{1}{2}\right)$.
               \par
               La distance $MI$ est donc minimale au point $M\left(\dfrac{1}{2}~;~\dfrac{1}{2}~;~\dfrac{1}{2}\right)$
          \end{enumerate}
          \item
          \begin{enumerate}
               \item
               Pour prouver que le point $M$ appartient au plan $(IJK)$, il suffit de montrer que les coordonnées de $M$ vérifient l'équation du plan $(IJK)$ (trouvée en \textbf{2.a.}).
               \par
               C'est immédiat :
               \par
               $\frac{1}{2}+\frac{1}{2}+\frac{1}{2}-\frac{3}{2}=0$
               \item
               Pour montrer que deux droites sont perpendiculaires ils faut montrer qu'elles sont orthogonales \textbf{et} sécantes.
               \begin{itemize}
                    \item
                    $(IM)$ et $(AG)$ sont sécantes en $M$ puisque, par hypothèse, $M$ est un point du segment $[AG]$. Par ailleurs, $(IM)$ est incluse dans le plan $(IJK)$ qui est perpendiculaire à $(AG)$ d'après \textbf{2.a.} donc $(IM)$ et $(AG)$ sont orthogonales.
                    \item
                    $(IM)$ et $(BF)$ sont sécantes en $I$.
                    \par
                    Les coordonnées des vecteurs $\overrightarrow{IM}$ et $\overrightarrow{BF}$ sont $\overrightarrow{IM}\begin{pmatrix} -1/2 \\ 1/2 \\ 0  \end{pmatrix}$et $\overrightarrow{BF}\begin{pmatrix}  0 \\ 0 \\ 1   \end{pmatrix}$
                    \par
                    $\overrightarrow{IM}.\overrightarrow{BF}=-\frac{1}{2} \times 0 + \frac{1}{2} \times 0 + 0 \times 1=0$.
                    \par
                    Donc $(IM)$ et $(BF)$ sont orthogonales.
               \end{itemize}
               La droite ($IM$) est donc perpendiculaire aux droites $(AG)$ et $(BF)$.
          \end{enumerate}
     \end{enumerate}
\end{corrige}

\end{document}