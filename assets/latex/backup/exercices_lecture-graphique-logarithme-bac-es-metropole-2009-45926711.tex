\documentclass[a4paper]{article}

%================================================================================================================================
%
% Packages
%
%================================================================================================================================

\usepackage[T1]{fontenc} 	% pour caractères accentués
\usepackage[utf8]{inputenc}  % encodage utf8
\usepackage[french]{babel}	% langue : français
\usepackage{fourier}			% caractères plus lisibles
\usepackage[dvipsnames]{xcolor} % couleurs
\usepackage{fancyhdr}		% réglage header footer
\usepackage{needspace}		% empêcher sauts de page mal placés
\usepackage{graphicx}		% pour inclure des graphiques
\usepackage{enumitem,cprotect}		% personnalise les listes d'items (nécessaire pour ol, al ...)
\usepackage{hyperref}		% Liens hypertexte
\usepackage{pstricks,pst-all,pst-node,pstricks-add,pst-math,pst-plot,pst-tree,pst-eucl} % pstricks
\usepackage[a4paper,includeheadfoot,top=2cm,left=3cm, bottom=2cm,right=3cm]{geometry} % marges etc.
\usepackage{comment}			% commentaires multilignes
\usepackage{amsmath,environ} % maths (matrices, etc.)
\usepackage{amssymb,makeidx}
\usepackage{bm}				% bold maths
\usepackage{tabularx}		% tableaux
\usepackage{colortbl}		% tableaux en couleur
\usepackage{fontawesome}		% Fontawesome
\usepackage{environ}			% environment with command
\usepackage{fp}				% calculs pour ps-tricks
\usepackage{multido}			% pour ps tricks
\usepackage[np]{numprint}	% formattage nombre
\usepackage{tikz,tkz-tab} 			% package principal TikZ
\usepackage{pgfplots}   % axes
\usepackage{mathrsfs}    % cursives
\usepackage{calc}			% calcul taille boites
\usepackage[scaled=0.875]{helvet} % font sans serif
\usepackage{svg} % svg
\usepackage{scrextend} % local margin
\usepackage{scratch} %scratch
\usepackage{multicol} % colonnes
%\usepackage{infix-RPN,pst-func} % formule en notation polanaise inversée
\usepackage{listings}

%================================================================================================================================
%
% Réglages de base
%
%================================================================================================================================

\lstset{
language=Python,   % R code
literate=
{á}{{\'a}}1
{à}{{\`a}}1
{ã}{{\~a}}1
{é}{{\'e}}1
{è}{{\`e}}1
{ê}{{\^e}}1
{í}{{\'i}}1
{ó}{{\'o}}1
{õ}{{\~o}}1
{ú}{{\'u}}1
{ü}{{\"u}}1
{ç}{{\c{c}}}1
{~}{{ }}1
}


\definecolor{codegreen}{rgb}{0,0.6,0}
\definecolor{codegray}{rgb}{0.5,0.5,0.5}
\definecolor{codepurple}{rgb}{0.58,0,0.82}
\definecolor{backcolour}{rgb}{0.95,0.95,0.92}

\lstdefinestyle{mystyle}{
    backgroundcolor=\color{backcolour},   
    commentstyle=\color{codegreen},
    keywordstyle=\color{magenta},
    numberstyle=\tiny\color{codegray},
    stringstyle=\color{codepurple},
    basicstyle=\ttfamily\footnotesize,
    breakatwhitespace=false,         
    breaklines=true,                 
    captionpos=b,                    
    keepspaces=true,                 
    numbers=left,                    
xleftmargin=2em,
framexleftmargin=2em,            
    showspaces=false,                
    showstringspaces=false,
    showtabs=false,                  
    tabsize=2,
    upquote=true
}

\lstset{style=mystyle}


\lstset{style=mystyle}
\newcommand{\imgdir}{C:/laragon/www/newmc/assets/imgsvg/}
\newcommand{\imgsvgdir}{C:/laragon/www/newmc/assets/imgsvg/}

\definecolor{mcgris}{RGB}{220, 220, 220}% ancien~; pour compatibilité
\definecolor{mcbleu}{RGB}{52, 152, 219}
\definecolor{mcvert}{RGB}{125, 194, 70}
\definecolor{mcmauve}{RGB}{154, 0, 215}
\definecolor{mcorange}{RGB}{255, 96, 0}
\definecolor{mcturquoise}{RGB}{0, 153, 153}
\definecolor{mcrouge}{RGB}{255, 0, 0}
\definecolor{mclightvert}{RGB}{205, 234, 190}

\definecolor{gris}{RGB}{220, 220, 220}
\definecolor{bleu}{RGB}{52, 152, 219}
\definecolor{vert}{RGB}{125, 194, 70}
\definecolor{mauve}{RGB}{154, 0, 215}
\definecolor{orange}{RGB}{255, 96, 0}
\definecolor{turquoise}{RGB}{0, 153, 153}
\definecolor{rouge}{RGB}{255, 0, 0}
\definecolor{lightvert}{RGB}{205, 234, 190}
\setitemize[0]{label=\color{lightvert}  $\bullet$}

\pagestyle{fancy}
\renewcommand{\headrulewidth}{0.2pt}
\fancyhead[L]{maths-cours.fr}
\fancyhead[R]{\thepage}
\renewcommand{\footrulewidth}{0.2pt}
\fancyfoot[C]{}

\newcolumntype{C}{>{\centering\arraybackslash}X}
\newcolumntype{s}{>{\hsize=.35\hsize\arraybackslash}X}

\setlength{\parindent}{0pt}		 
\setlength{\parskip}{3mm}
\setlength{\headheight}{1cm}

\def\ebook{ebook}
\def\book{book}
\def\web{web}
\def\type{web}

\newcommand{\vect}[1]{\overrightarrow{\,\mathstrut#1\,}}

\def\Oij{$\left(\text{O}~;~\vect{\imath},~\vect{\jmath}\right)$}
\def\Oijk{$\left(\text{O}~;~\vect{\imath},~\vect{\jmath},~\vect{k}\right)$}
\def\Ouv{$\left(\text{O}~;~\vect{u},~\vect{v}\right)$}

\hypersetup{breaklinks=true, colorlinks = true, linkcolor = OliveGreen, urlcolor = OliveGreen, citecolor = OliveGreen, pdfauthor={Didier BONNEL - https://www.maths-cours.fr} } % supprime les bordures autour des liens

\renewcommand{\arg}[0]{\text{arg}}

\everymath{\displaystyle}

%================================================================================================================================
%
% Macros - Commandes
%
%================================================================================================================================

\newcommand\meta[2]{    			% Utilisé pour créer le post HTML.
	\def\titre{titre}
	\def\url{url}
	\def\arg{#1}
	\ifx\titre\arg
		\newcommand\maintitle{#2}
		\fancyhead[L]{#2}
		{\Large\sffamily \MakeUppercase{#2}}
		\vspace{1mm}\textcolor{mcvert}{\hrule}
	\fi 
	\ifx\url\arg
		\fancyfoot[L]{\href{https://www.maths-cours.fr#2}{\black \footnotesize{https://www.maths-cours.fr#2}}}
	\fi 
}


\newcommand\TitreC[1]{    		% Titre centré
     \needspace{3\baselineskip}
     \begin{center}\textbf{#1}\end{center}
}

\newcommand\newpar{    		% paragraphe
     \par
}

\newcommand\nosp {    		% commande vide (pas d'espace)
}
\newcommand{\id}[1]{} %ignore

\newcommand\boite[2]{				% Boite simple sans titre
	\vspace{5mm}
	\setlength{\fboxrule}{0.2mm}
	\setlength{\fboxsep}{5mm}	
	\fcolorbox{#1}{#1!3}{\makebox[\linewidth-2\fboxrule-2\fboxsep]{
  		\begin{minipage}[t]{\linewidth-2\fboxrule-4\fboxsep}\setlength{\parskip}{3mm}
  			 #2
  		\end{minipage}
	}}
	\vspace{5mm}
}

\newcommand\CBox[4]{				% Boites
	\vspace{5mm}
	\setlength{\fboxrule}{0.2mm}
	\setlength{\fboxsep}{5mm}
	
	\fcolorbox{#1}{#1!3}{\makebox[\linewidth-2\fboxrule-2\fboxsep]{
		\begin{minipage}[t]{1cm}\setlength{\parskip}{3mm}
	  		\textcolor{#1}{\LARGE{#2}}    
 	 	\end{minipage}  
  		\begin{minipage}[t]{\linewidth-2\fboxrule-4\fboxsep}\setlength{\parskip}{3mm}
			\raisebox{1.2mm}{\normalsize\sffamily{\textcolor{#1}{#3}}}						
  			 #4
  		\end{minipage}
	}}
	\vspace{5mm}
}

\newcommand\cadre[3]{				% Boites convertible html
	\par
	\vspace{2mm}
	\setlength{\fboxrule}{0.1mm}
	\setlength{\fboxsep}{5mm}
	\fcolorbox{#1}{white}{\makebox[\linewidth-2\fboxrule-2\fboxsep]{
  		\begin{minipage}[t]{\linewidth-2\fboxrule-4\fboxsep}\setlength{\parskip}{3mm}
			\raisebox{-2.5mm}{\sffamily \small{\textcolor{#1}{\MakeUppercase{#2}}}}		
			\par		
  			 #3
 	 		\end{minipage}
	}}
		\vspace{2mm}
	\par
}

\newcommand\bloc[3]{				% Boites convertible html sans bordure
     \needspace{2\baselineskip}
     {\sffamily \small{\textcolor{#1}{\MakeUppercase{#2}}}}    
		\par		
  			 #3
		\par
}

\newcommand\CHelp[1]{
     \CBox{Plum}{\faInfoCircle}{À RETENIR}{#1}
}

\newcommand\CUp[1]{
     \CBox{NavyBlue}{\faThumbsOUp}{EN PRATIQUE}{#1}
}

\newcommand\CInfo[1]{
     \CBox{Sepia}{\faArrowCircleRight}{REMARQUE}{#1}
}

\newcommand\CRedac[1]{
     \CBox{PineGreen}{\faEdit}{BIEN R\'EDIGER}{#1}
}

\newcommand\CError[1]{
     \CBox{Red}{\faExclamationTriangle}{ATTENTION}{#1}
}

\newcommand\TitreExo[2]{
\needspace{4\baselineskip}
 {\sffamily\large EXERCICE #1\ (\emph{#2 points})}
\vspace{5mm}
}

\newcommand\img[2]{
          \includegraphics[width=#2\paperwidth]{\imgdir#1}
}

\newcommand\imgsvg[2]{
       \begin{center}   \includegraphics[width=#2\paperwidth]{\imgsvgdir#1} \end{center}
}


\newcommand\Lien[2]{
     \href{#1}{#2 \tiny \faExternalLink}
}
\newcommand\mcLien[2]{
     \href{https~://www.maths-cours.fr/#1}{#2 \tiny \faExternalLink}
}

\newcommand{\euro}{\eurologo{}}

%================================================================================================================================
%
% Macros - Environement
%
%================================================================================================================================

\newenvironment{tex}{ %
}
{%
}

\newenvironment{indente}{ %
	\setlength\parindent{10mm}
}

{
	\setlength\parindent{0mm}
}

\newenvironment{corrige}{%
     \needspace{3\baselineskip}
     \medskip
     \textbf{\textsc{Corrigé}}
     \medskip
}
{
}

\newenvironment{extern}{%
     \begin{center}
     }
     {
     \end{center}
}

\NewEnviron{code}{%
	\par
     \boite{gray}{\texttt{%
     \BODY
     }}
     \par
}

\newenvironment{vbloc}{% boite sans cadre empeche saut de page
     \begin{minipage}[t]{\linewidth}
     }
     {
     \end{minipage}
}
\NewEnviron{h2}{%
    \needspace{3\baselineskip}
    \vspace{0.6cm}
	\noindent \MakeUppercase{\sffamily \large \BODY}
	\vspace{1mm}\textcolor{mcgris}{\hrule}\vspace{0.4cm}
	\par
}{}

\NewEnviron{h3}{%
    \needspace{3\baselineskip}
	\vspace{5mm}
	\textsc{\BODY}
	\par
}

\NewEnviron{margeneg}{ %
\begin{addmargin}[-1cm]{0cm}
\BODY
\end{addmargin}
}

\NewEnviron{html}{%
}

\begin{document}
\meta{url}{/exercices/lecture-graphique-logarithme-bac-es-metropole-2009/}
\meta{pid}{1913}
\meta{titre}{Logarithmes - Bac ES Métropole 2009}
\meta{type}{exercices}
%
\begin{h2}Exercice 2\end{h2}
\textit{5 points - Pour les candidats n'ayant pas suivi l'enseignement de spécialité}
\par
Soit $f$ une fonction définie et dérivable sur l'intervalle [- 2; 5], décroissante sur chacun des intervalles [-2;0] et [2; 5] et croissante sur l'intervalle [0;2].
\par
On note $f^{\prime}$ sa fonction dérivée sur l'intervalle [- 2; 5].
\par
La courbe $\left(\Gamma \right)$ représentative de la fonction $f$ est tracée en annexe 1 dans le plan muni d'un repère orthogonal.  Elle passe par les points A(- 2; 9), B(0; 4), C(1;4,5), D(2;5) et E(4; 0).
\par
En chacun des points B et D. la tangente à la courbe $\left(\Gamma \right)$ est parallèle à l'axe des abscisses.
\par
On note F le point de coordonnées (3;6).
\par
La droite (CF) est la tangente à la courbe $\left(\Gamma \right)$ au point C.

\begin{center}
\imgsvg{Bac-ES-2009-courbe}{0.3}% alt="Logarithmes - Bac ES Métropole 2009" style="width:50rem"
\end{center}

\begin{enumerate}
     \item
     À l'aide des informations précédentes et de l'annexe 1, préciser sans justifier :
\begin{enumerate}[label=\alph*.]
          \item
          les valeurs de $f\left(0\right)$, $f^{\prime}\left(1\right)$ et $f^{\prime}\left(2\right)$.
          \item
          le signe de $f^{\prime}\left(x\right)$ suivant les valeurs du nombre réel $x$ de l'intervalle $\left[- 2; 5\right]$.
          \item
          le signe de $f\left(x\right)$ suivant les valeurs du nombre réel $x$ de l'intervalle $\left[- 2; 5\right]$.
     \end{enumerate}
     \item
     On considère la fonction $g$ définie par $g\left(x\right)=\ln \left(f\left(x\right)\right)$  où $\ln$ désigne la fonction logarithme népérien.
     \begin{enumerate}[label=\alph*.]
          \item
          Expliquer pourquoi la fonction $g$ est définie sur l'intervalle $\left[- 2; 4\right[$.
          \item
          Calculer $g\left(-2\right), g\left(0\right)$ et $g\left(2\right)$.
          \item
          Préciser, en le justifiant, le sens  de variations de la fonction $g$ sur l'intervalle $\left[- 2; 4\right[$.
          \item
          Déterminer la limite de la fonction $g$ lorsque $x$ tend vers 4.
          \par
          Interpréter ce résultat pour la représentation graphique de la fonction $g$.
          \item
          Dresser le tableau de variations de la fonction $g$.
     \end{enumerate}
\end{enumerate}
\begin{corrige}
     \begin{enumerate}
          \item
          \begin{enumerate}[label=\alph*.]
               \item
               $f\left(0\right)=4$
               \par
               $f^{\prime}\left(1\right)=\frac{3}{4}$
               \par
               $f^{\prime}\left(2\right)=0$
               \item
&nbsp;
\\
%##
% type=table; width=45; c1=30
%--
% x|   -2  ~    0   ~   2  ~  5
% f'(x)|  ~    -       :0     +  :0   -   ~
%--
\begin{center}
 \begin{extern}%style="width:45rem" alt="Exercice"
    \resizebox{11cm}{!}{
       \definecolor{dark}{gray}{0.1}
       \definecolor{light}{gray}{0.8}
       \tikzstyle{fleche}=[->,>=latex]
       \begin{tikzpicture}[scale=.8, line width=.5pt, dark]
       \def\width{.15}
       \def\height{.10}
       \draw (0, -10*\height) -- (92*\width, -10*\height);
       \draw (30*\width, 0*\height) -- (30*\width, -10*\height);
       \node (l0c0) at (15*\width,-5*\height) {$ x $};
       \node (l0c1) at (34*\width,-5*\height) {$ -2 $};
       \node (l0c2) at (43*\width,-5*\height) {$ ~ $};
       \node (l0c3) at (52*\width,-5*\height) {$ 0 $};
       \node (l0c4) at (61*\width,-5*\height) {$ ~ $};
       \node (l0c5) at (70*\width,-5*\height) {$ 2 $};
       \node (l0c6) at (79*\width,-5*\height) {$ ~ $};
       \node (l0c7) at (88*\width,-5*\height) {$ 5 $};
       \draw (0, -20*\height) -- (92*\width, -20*\height);
       \draw (30*\width, -10*\height) -- (30*\width, -20*\height);
       \node (l1c0) at (15*\width,-15*\height) {$ f'(x) $};
       \node (l1c1) at (34*\width,-15*\height) {$ ~ $};
       \node (l1c2) at (43*\width,-15*\height) {$ - $};
       \draw[light] (52*\width, -10*\height) -- (52*\width, -20*\height);
       \node (l1c3) at (52*\width,-15*\height) {$ 0 $};
       \node (l1c4) at (61*\width,-15*\height) {$ + $};
       \draw[light] (70*\width, -10*\height) -- (70*\width, -20*\height);
       \node (l1c5) at (70*\width,-15*\height) {$ 0 $};
       \node (l1c6) at (79*\width,-15*\height) {$ - $};
       \node (l1c7) at (88*\width,-15*\height) {$ ~ $};
       \draw (0, 0) rectangle (92*\width, -20*\height);

       \end{tikzpicture}
      }
   \end{extern}
\end{center}
%##
\item
&nbsp;
\\
%##
% type=table; width=35; c1=30
%--
% x|   -2  ~    4  ~  5
% f(x)|  ~     +  :0   -   ~
%--
\begin{center}
 \begin{extern}%style="width:35rem" alt="Exercice"
    \resizebox{11cm}{!}{
       \definecolor{dark}{gray}{0.1}
       \definecolor{light}{gray}{0.8}
       \tikzstyle{fleche}=[->,>=latex]
       \begin{tikzpicture}[scale=.8, line width=.5pt, dark]
       \def\width{.15}
       \def\height{.10}
       \draw (0, -10*\height) -- (74*\width, -10*\height);
       \draw (30*\width, 0*\height) -- (30*\width, -10*\height);
       \node (l0c0) at (15*\width,-5*\height) {$ x $};
       \node (l0c1) at (34*\width,-5*\height) {$ -2 $};
       \node (l0c2) at (43*\width,-5*\height) {$ ~ $};
       \node (l0c3) at (52*\width,-5*\height) {$ 4 $};
       \node (l0c4) at (61*\width,-5*\height) {$ ~ $};
       \node (l0c5) at (70*\width,-5*\height) {$ 5 $};
       \draw (0, -20*\height) -- (74*\width, -20*\height);
       \draw (30*\width, -10*\height) -- (30*\width, -20*\height);
       \node (l1c0) at (15*\width,-15*\height) {$ f(x) $};
       \node (l1c1) at (34*\width,-15*\height) {$ ~ $};
       \node (l1c2) at (43*\width,-15*\height) {$ + $};
       \draw[light] (52*\width, -10*\height) -- (52*\width, -20*\height);
       \node (l1c3) at (52*\width,-15*\height) {$ 0 $};
       \node (l1c4) at (61*\width,-15*\height) {$ - $};
       \node (l1c5) at (70*\width,-15*\height) {$ ~ $};
       \draw (0, 0) rectangle (74*\width, -20*\height);

       \end{tikzpicture}
      }
   \end{extern}
\end{center}
%##
<img src="/wp-content/uploads/t_a1a7f8e0dd8709e8ad5acfabf5088ff3.gif" alt="" class="aligncenter size-full  img-pc" />
          \end{enumerate}
          \item
          \begin{enumerate}[label=\alph*.]
               \item
               $g$ est définie si et seulement si $f\left(x\right) > 0$ donc si et seulement si $x\in \left[-2; 4\right[$
               \item
               $g\left(-2\right)=\ln\left(f\left(-2\right)\right)=\ln9=2\ln3$
               \par
               $g\left(0\right)=\ln\left(f\left(0\right)\right)=\ln4=2\ln2$
               \par
               $g\left(2\right)=\ln\left(f\left(2\right)\right)=\ln5$
               \item
               Sur [-2; 4[, par dérivation de fonctions composées:
               \par
               $g^{\prime}\left(x\right)=\frac{f^{\prime}\left(x\right)}{f\left(x\right)}$
               \par
               La fonction $f$ étant positive sur [-2; 4[ $g^{\prime}\left(x\right)$ est du signe de $f^{\prime}\left(x\right)$
               \par
               $g$ est donc strictement croissante sur [0; 2] et strictement décroissante sur [-2; 0] et sur [2;4[
               \item
               Lorsque x tend vers 4 (par valeurs inférieures) $f\left(x\right)$ tend vers 0 par valeurs supérieures.
               \par
               D'après le théorème de composition des limites pour les fonctions continues:
               \par
               $\lim\limits_{x\rightarrow 4}g\left(x\right)=\lim\limits_{x\rightarrow 4}\ln\left(f\left(x\right)\right)=-\infty $
               \item
&nbsp;
\\
%##
% type=table; width=35; l3=20
%--
% x|   -2   ~    0   ~   2  ~   ~  4 
% g'(x)|  ~       -               :0             +   :0   -     ~  ||
% g(x)|  2\ln 3    \          :2\ln 2             /   :\ln 5   \   -\infty  ||
%--
\begin{center}
 \begin{extern}%style="width:35rem" alt="Exercice"
    \resizebox{11cm}{!}{
       \definecolor{dark}{gray}{0.1}
       \definecolor{light}{gray}{0.8}
       \tikzstyle{fleche}=[->,>=latex]
       \begin{tikzpicture}[scale=.8, line width=.5pt, dark]
       \def\width{.15}
       \def\height{.10}
       \draw (0, -10*\height) -- (82*\width, -10*\height);
       \draw (10*\width, 0*\height) -- (10*\width, -10*\height);
       \node (l0c0) at (5*\width,-5*\height) {$ x $};
       \node (l0c1) at (14*\width,-5*\height) {$ -2 $};
       \node (l0c2) at (23*\width,-5*\height) {$ ~ $};
       \node (l0c3) at (32*\width,-5*\height) {$ 0 $};
       \node (l0c4) at (41*\width,-5*\height) {$ ~ $};
       \node (l0c5) at (50*\width,-5*\height) {$ 2 $};
       \node (l0c6) at (59*\width,-5*\height) {$ ~ $};
       \node (l0c7) at (68*\width,-5*\height) {$ ~ $};
       \node (l0c8) at (77*\width,-5*\height) {$ 4 $};
       \draw (0, -20*\height) -- (82*\width, -20*\height);
       \draw (10*\width, -10*\height) -- (10*\width, -20*\height);
       \node (l1c0) at (5*\width,-15*\height) {$ g'(x) $};
       \node (l1c1) at (14*\width,-15*\height) {$ ~ $};
       \node (l1c2) at (23*\width,-15*\height) {$ - $};
       \draw[light] (32*\width, -10*\height) -- (32*\width, -20*\height);
       \node (l1c3) at (32*\width,-15*\height) {$ 0 $};
       \node (l1c4) at (41*\width,-15*\height) {$ + $};
       \draw[light] (50*\width, -10*\height) -- (50*\width, -20*\height);
       \node (l1c5) at (50*\width,-15*\height) {$ 0 $};
       \node (l1c6) at (59*\width,-15*\height) {$ - $};
       \node (l1c7) at (68*\width,-15*\height) {$ ~ $};
       \draw[double distance=2pt] (77*\width, -10*\height) -- (77*\width, -20*\height);
       \node (l1c8) at (77*\width,-15*\height) {$ ~ $};
       \draw (0, -40*\height) -- (82*\width, -40*\height);
       \draw (10*\width, -20*\height) -- (10*\width, -40*\height);
       \node (l2c0) at (5*\width,-30*\height) {$ g(x) $};
       \node (l2c1) at (14*\width,-25*\height) {$ 2\ln 3 $};
       \node (l2c2) at (23*\width,-30*\height) {$ ~ $};
       \draw[light] (32*\width, -20*\height) -- (32*\width, -40*\height);
       \node (l2c3) at (32*\width,-35*\height) {$ 2\ln 2 $};
       \node (l2c4) at (41*\width,-30*\height) {$ ~ $};
       \draw[light] (50*\width, -20*\height) -- (50*\width, -40*\height);
       \node (l2c5) at (50*\width,-25*\height) {$ \ln 5 $};
       \node (l2c6) at (59*\width,-30*\height) {$ ~ $};
       \node (l2c7) at (68*\width,-35*\height) {$ -\infty $};
       \draw[double distance=2pt] (77*\width, -20*\height) -- (77*\width, -40*\height);
       \node (l2c8) at (77*\width,-30*\height) {$ ~ $};
       \draw (0, 0) rectangle (82*\width, -40*\height);
       \draw[fleche] (l2c1) -- (l2c3);
       \draw[fleche] (l2c3) -- (l2c5);
       \draw[fleche] (l2c5) -- (l2c7);
       \end{tikzpicture}
      }
   \end{extern}
\end{center}
%##
<img src="/wp-content/uploads/t_6e4ae5cbb7de177ef707dfa5ed3a042f.gif" alt="" class="aligncenter size-full  img-pc" />
          \end{enumerate}
     \end{enumerate}

\end{corrige}

\end{document}