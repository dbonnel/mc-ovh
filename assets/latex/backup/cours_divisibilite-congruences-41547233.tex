\documentclass[a4paper]{article}

%================================================================================================================================
%
% Packages
%
%================================================================================================================================

\usepackage[T1]{fontenc} 	% pour caractères accentués
\usepackage[utf8]{inputenc}  % encodage utf8
\usepackage[french]{babel}	% langue : français
\usepackage{fourier}			% caractères plus lisibles
\usepackage[dvipsnames]{xcolor} % couleurs
\usepackage{fancyhdr}		% réglage header footer
\usepackage{needspace}		% empêcher sauts de page mal placés
\usepackage{graphicx}		% pour inclure des graphiques
\usepackage{enumitem,cprotect}		% personnalise les listes d'items (nécessaire pour ol, al ...)
\usepackage{hyperref}		% Liens hypertexte
\usepackage{pstricks,pst-all,pst-node,pstricks-add,pst-math,pst-plot,pst-tree,pst-eucl} % pstricks
\usepackage[a4paper,includeheadfoot,top=2cm,left=3cm, bottom=2cm,right=3cm]{geometry} % marges etc.
\usepackage{comment}			% commentaires multilignes
\usepackage{amsmath,environ} % maths (matrices, etc.)
\usepackage{amssymb,makeidx}
\usepackage{bm}				% bold maths
\usepackage{tabularx}		% tableaux
\usepackage{colortbl}		% tableaux en couleur
\usepackage{fontawesome}		% Fontawesome
\usepackage{environ}			% environment with command
\usepackage{fp}				% calculs pour ps-tricks
\usepackage{multido}			% pour ps tricks
\usepackage[np]{numprint}	% formattage nombre
\usepackage{tikz,tkz-tab} 			% package principal TikZ
\usepackage{pgfplots}   % axes
\usepackage{mathrsfs}    % cursives
\usepackage{calc}			% calcul taille boites
\usepackage[scaled=0.875]{helvet} % font sans serif
\usepackage{svg} % svg
\usepackage{scrextend} % local margin
\usepackage{scratch} %scratch
\usepackage{multicol} % colonnes
%\usepackage{infix-RPN,pst-func} % formule en notation polanaise inversée
\usepackage{listings}

%================================================================================================================================
%
% Réglages de base
%
%================================================================================================================================

\lstset{
language=Python,   % R code
literate=
{á}{{\'a}}1
{à}{{\`a}}1
{ã}{{\~a}}1
{é}{{\'e}}1
{è}{{\`e}}1
{ê}{{\^e}}1
{í}{{\'i}}1
{ó}{{\'o}}1
{õ}{{\~o}}1
{ú}{{\'u}}1
{ü}{{\"u}}1
{ç}{{\c{c}}}1
{~}{{ }}1
}


\definecolor{codegreen}{rgb}{0,0.6,0}
\definecolor{codegray}{rgb}{0.5,0.5,0.5}
\definecolor{codepurple}{rgb}{0.58,0,0.82}
\definecolor{backcolour}{rgb}{0.95,0.95,0.92}

\lstdefinestyle{mystyle}{
    backgroundcolor=\color{backcolour},   
    commentstyle=\color{codegreen},
    keywordstyle=\color{magenta},
    numberstyle=\tiny\color{codegray},
    stringstyle=\color{codepurple},
    basicstyle=\ttfamily\footnotesize,
    breakatwhitespace=false,         
    breaklines=true,                 
    captionpos=b,                    
    keepspaces=true,                 
    numbers=left,                    
xleftmargin=2em,
framexleftmargin=2em,            
    showspaces=false,                
    showstringspaces=false,
    showtabs=false,                  
    tabsize=2,
    upquote=true
}

\lstset{style=mystyle}


\lstset{style=mystyle}
\newcommand{\imgdir}{C:/laragon/www/newmc/assets/imgsvg/}
\newcommand{\imgsvgdir}{C:/laragon/www/newmc/assets/imgsvg/}

\definecolor{mcgris}{RGB}{220, 220, 220}% ancien~; pour compatibilité
\definecolor{mcbleu}{RGB}{52, 152, 219}
\definecolor{mcvert}{RGB}{125, 194, 70}
\definecolor{mcmauve}{RGB}{154, 0, 215}
\definecolor{mcorange}{RGB}{255, 96, 0}
\definecolor{mcturquoise}{RGB}{0, 153, 153}
\definecolor{mcrouge}{RGB}{255, 0, 0}
\definecolor{mclightvert}{RGB}{205, 234, 190}

\definecolor{gris}{RGB}{220, 220, 220}
\definecolor{bleu}{RGB}{52, 152, 219}
\definecolor{vert}{RGB}{125, 194, 70}
\definecolor{mauve}{RGB}{154, 0, 215}
\definecolor{orange}{RGB}{255, 96, 0}
\definecolor{turquoise}{RGB}{0, 153, 153}
\definecolor{rouge}{RGB}{255, 0, 0}
\definecolor{lightvert}{RGB}{205, 234, 190}
\setitemize[0]{label=\color{lightvert}  $\bullet$}

\pagestyle{fancy}
\renewcommand{\headrulewidth}{0.2pt}
\fancyhead[L]{maths-cours.fr}
\fancyhead[R]{\thepage}
\renewcommand{\footrulewidth}{0.2pt}
\fancyfoot[C]{}

\newcolumntype{C}{>{\centering\arraybackslash}X}
\newcolumntype{s}{>{\hsize=.35\hsize\arraybackslash}X}

\setlength{\parindent}{0pt}		 
\setlength{\parskip}{3mm}
\setlength{\headheight}{1cm}

\def\ebook{ebook}
\def\book{book}
\def\web{web}
\def\type{web}

\newcommand{\vect}[1]{\overrightarrow{\,\mathstrut#1\,}}

\def\Oij{$\left(\text{O}~;~\vect{\imath},~\vect{\jmath}\right)$}
\def\Oijk{$\left(\text{O}~;~\vect{\imath},~\vect{\jmath},~\vect{k}\right)$}
\def\Ouv{$\left(\text{O}~;~\vect{u},~\vect{v}\right)$}

\hypersetup{breaklinks=true, colorlinks = true, linkcolor = OliveGreen, urlcolor = OliveGreen, citecolor = OliveGreen, pdfauthor={Didier BONNEL - https://www.maths-cours.fr} } % supprime les bordures autour des liens

\renewcommand{\arg}[0]{\text{arg}}

\everymath{\displaystyle}

%================================================================================================================================
%
% Macros - Commandes
%
%================================================================================================================================

\newcommand\meta[2]{    			% Utilisé pour créer le post HTML.
	\def\titre{titre}
	\def\url{url}
	\def\arg{#1}
	\ifx\titre\arg
		\newcommand\maintitle{#2}
		\fancyhead[L]{#2}
		{\Large\sffamily \MakeUppercase{#2}}
		\vspace{1mm}\textcolor{mcvert}{\hrule}
	\fi 
	\ifx\url\arg
		\fancyfoot[L]{\href{https://www.maths-cours.fr#2}{\black \footnotesize{https://www.maths-cours.fr#2}}}
	\fi 
}


\newcommand\TitreC[1]{    		% Titre centré
     \needspace{3\baselineskip}
     \begin{center}\textbf{#1}\end{center}
}

\newcommand\newpar{    		% paragraphe
     \par
}

\newcommand\nosp {    		% commande vide (pas d'espace)
}
\newcommand{\id}[1]{} %ignore

\newcommand\boite[2]{				% Boite simple sans titre
	\vspace{5mm}
	\setlength{\fboxrule}{0.2mm}
	\setlength{\fboxsep}{5mm}	
	\fcolorbox{#1}{#1!3}{\makebox[\linewidth-2\fboxrule-2\fboxsep]{
  		\begin{minipage}[t]{\linewidth-2\fboxrule-4\fboxsep}\setlength{\parskip}{3mm}
  			 #2
  		\end{minipage}
	}}
	\vspace{5mm}
}

\newcommand\CBox[4]{				% Boites
	\vspace{5mm}
	\setlength{\fboxrule}{0.2mm}
	\setlength{\fboxsep}{5mm}
	
	\fcolorbox{#1}{#1!3}{\makebox[\linewidth-2\fboxrule-2\fboxsep]{
		\begin{minipage}[t]{1cm}\setlength{\parskip}{3mm}
	  		\textcolor{#1}{\LARGE{#2}}    
 	 	\end{minipage}  
  		\begin{minipage}[t]{\linewidth-2\fboxrule-4\fboxsep}\setlength{\parskip}{3mm}
			\raisebox{1.2mm}{\normalsize\sffamily{\textcolor{#1}{#3}}}						
  			 #4
  		\end{minipage}
	}}
	\vspace{5mm}
}

\newcommand\cadre[3]{				% Boites convertible html
	\par
	\vspace{2mm}
	\setlength{\fboxrule}{0.1mm}
	\setlength{\fboxsep}{5mm}
	\fcolorbox{#1}{white}{\makebox[\linewidth-2\fboxrule-2\fboxsep]{
  		\begin{minipage}[t]{\linewidth-2\fboxrule-4\fboxsep}\setlength{\parskip}{3mm}
			\raisebox{-2.5mm}{\sffamily \small{\textcolor{#1}{\MakeUppercase{#2}}}}		
			\par		
  			 #3
 	 		\end{minipage}
	}}
		\vspace{2mm}
	\par
}

\newcommand\bloc[3]{				% Boites convertible html sans bordure
     \needspace{2\baselineskip}
     {\sffamily \small{\textcolor{#1}{\MakeUppercase{#2}}}}    
		\par		
  			 #3
		\par
}

\newcommand\CHelp[1]{
     \CBox{Plum}{\faInfoCircle}{À RETENIR}{#1}
}

\newcommand\CUp[1]{
     \CBox{NavyBlue}{\faThumbsOUp}{EN PRATIQUE}{#1}
}

\newcommand\CInfo[1]{
     \CBox{Sepia}{\faArrowCircleRight}{REMARQUE}{#1}
}

\newcommand\CRedac[1]{
     \CBox{PineGreen}{\faEdit}{BIEN R\'EDIGER}{#1}
}

\newcommand\CError[1]{
     \CBox{Red}{\faExclamationTriangle}{ATTENTION}{#1}
}

\newcommand\TitreExo[2]{
\needspace{4\baselineskip}
 {\sffamily\large EXERCICE #1\ (\emph{#2 points})}
\vspace{5mm}
}

\newcommand\img[2]{
          \includegraphics[width=#2\paperwidth]{\imgdir#1}
}

\newcommand\imgsvg[2]{
       \begin{center}   \includegraphics[width=#2\paperwidth]{\imgsvgdir#1} \end{center}
}


\newcommand\Lien[2]{
     \href{#1}{#2 \tiny \faExternalLink}
}
\newcommand\mcLien[2]{
     \href{https~://www.maths-cours.fr/#1}{#2 \tiny \faExternalLink}
}

\newcommand{\euro}{\eurologo{}}

%================================================================================================================================
%
% Macros - Environement
%
%================================================================================================================================

\newenvironment{tex}{ %
}
{%
}

\newenvironment{indente}{ %
	\setlength\parindent{10mm}
}

{
	\setlength\parindent{0mm}
}

\newenvironment{corrige}{%
     \needspace{3\baselineskip}
     \medskip
     \textbf{\textsc{Corrigé}}
     \medskip
}
{
}

\newenvironment{extern}{%
     \begin{center}
     }
     {
     \end{center}
}

\NewEnviron{code}{%
	\par
     \boite{gray}{\texttt{%
     \BODY
     }}
     \par
}

\newenvironment{vbloc}{% boite sans cadre empeche saut de page
     \begin{minipage}[t]{\linewidth}
     }
     {
     \end{minipage}
}
\NewEnviron{h2}{%
    \needspace{3\baselineskip}
    \vspace{0.6cm}
	\noindent \MakeUppercase{\sffamily \large \BODY}
	\vspace{1mm}\textcolor{mcgris}{\hrule}\vspace{0.4cm}
	\par
}{}

\NewEnviron{h3}{%
    \needspace{3\baselineskip}
	\vspace{5mm}
	\textsc{\BODY}
	\par
}

\NewEnviron{margeneg}{ %
\begin{addmargin}[-1cm]{0cm}
\BODY
\end{addmargin}
}

\NewEnviron{html}{%
}

\begin{document}
\meta{url}{/cours/divisibilite-congruences/}
\meta{pid}{561}
\meta{titre}{Divisibilité et congruences (Spécialité)}
\meta{type}{cours}
\begin{h2}1. Division euclidienne\end{h2}
\cadre{bleu}{Définition}{% id="d10"
     Soient $a$ et  $b$ deux entiers relatifs tels qu'il existe un entier relatif $k$ tel que $a=bk$.
     \par
     On dit alors que :
     \begin{itemize}
          \item $b$ \textbf{divise} $a$ ;
          \item $b$ est un \textbf{diviseur} de $a$ ;
          \item $a$ est un \textbf{multiple} de $b$.
     \end{itemize}
     Ceci se note $b|a$
}
\bloc{orange}{Exemple}{% id="e10"
     $15=3\times 5$ donc :
     \begin{itemize}
          \item 3 divise 15.
          \item 3 est un diviseur de 15.
          \item 15 est un multiple de 3.
     \end{itemize}
}
\bloc{vert}{Remarques}{% id="r10"
     \begin{itemize}
          \item 0 est un multiple de tout entier relatif.
          \item 1 et -1 sont des diviseurs de tout entier relatif.
          \item $a$ et $-a$ ont les mêmes diviseurs.
     \end{itemize}
}
\cadre{vert}{Propriétés}{% id="p20"
     \begin{itemize}
          \item Si $a$ divise $b$ et $b$ divise $a$, alors $a$ et $b$ sont égaux ou opposés.
          \item Si $a$ divise $b$ et $b$  divise $c$, alors $a$ divise $c$.
          \item Si $c$ divise $a$ et $c$ divise $b$, alors $c$ divise toute combinaison linéaire de $a$ et $b$ (c'est-à-dire tout nombre de la forme $au+bv ; u\in \mathbb{Z}, v\in \mathbb{Z}$).
     \end{itemize}
}
\cadre{rouge}{Théorème et définitions}{% id="d30"
     \textbf{Division euclidienne dans $\mathbb{Z}$}
     \par
     Soient $a$ et $b$ deux entiers relatifs avec $b\neq 0$.
     \par
     Il existe un et un seul couple d'entiers relatifs $\left(q,r\right)$ tels que :
     \par     \begin{center}
          $a=bq+r$ et $0 \leqslant  r < |b|$.
     \end{center}
     $q$ et $r$ s'appelle respectivement le \textbf{quotient} et le \textbf{reste} de la \textbf{division euclidienne} de $a$ par $b$.
}
\bloc{orange}{Exemple}{% id="e30"
     -14=3$\times $(-5)+1 et 0$\leqslant $1$ < $3
     \par
     La division euclidienne de -14 par 3 donne un quotient de -5 est un reste de 1.
}
\bloc{vert}{Remarques}{% id="r30"
     \begin{itemize}
          \item \textbf{Attention !} Ne pas oublier la condition $0 \leqslant  r <  |b|$. La seule égalité $a=bq+r$  ne suffit pas à prouver que $q$ et $r$ sont les quotient et reste dans la division euclidienne de $a$ par $b$.
          \item $a$ est divisible par $b$ si et seulement si le reste de la division de $a$ par $b$ est égal à zéro.
     \end{itemize}
}
\begin{h2}2. Congruences\end{h2}
\cadre{bleu}{Définition}{% id="d40"
     On dit que deux entiers relatifs $a$ et $b$ son congrus modulo $n$ ( $n\in \mathbb{N}^*$ ) et l'on écrit $a\equiv b  \left[n\right]$ si et seulement si $a$ et $b$ ont le même reste dans la division par $n$.
}
\bloc{orange}{Exemple}{% id="e40"
     $18\equiv 23  \left[5\right]$ car 18 et 23 ont tous les deux 3 comme reste dans la division par 5.
}
\cadre{vert}{Propriétés}{% id="p50"
     \begin{itemize}
          \item $a\equiv b  \left[n\right]$ si et seulement si $n$ divise $a-b$ en particulier $a\equiv 0  \left[n\right]$ si et seulement si $n$ divise $a$.
          \item Si $a\equiv b  \left[n\right]$ et $b\equiv c  \left[n\right]$, alors $a\equiv c  \left[n\right]$.
     \end{itemize}
}
\cadre{vert}{Propriétés (Congruences et opérations)}{% id="p60"
     Soient quatre entiers relatifs $a, b, c, d$ tels que $a\equiv b  \left[n\right]$ et $c\equiv d  \left[n\right]$. Alors :
     \begin{itemize}
          \item $a+c\equiv b+d  \left[n\right]$ et $a-c\equiv b-d  \left[n\right]$.
          \item $ac\equiv bd  \left[n\right]$.
          \item $ka\equiv kb  \left[n\right]$ pour tout entier relatif $k$.
          \item $a^{m}\equiv b^{m}  \left[n\right]$ pour tout entier naturel $m$.
     \end{itemize}
}
\cadre{vert}{Propriété}{% id="p70"
     $r$ est le reste de la division euclidienne de $a$ par $b$ si et seulement si :
     \par
     \begin{center}
          $\left\{ \begin{matrix} r\equiv a  \left[b\right] \\ r < |b| \end{matrix}\right.$
     \end{center}
}
\bloc{orange}{Exemple}{% id="e70"
     On cherche à déterminer le reste de la division euclidienne de $2009^{2009}$ par 5.
     \par
     $2009\equiv -1  \left[5\right]$ car 2009-(-1)=2010 est divisible par 5.
     \par
     Donc :
     \par
     $2009^{2009}\equiv \left(-1\right)^{2009}  \left[5\right]$ c'est-à-dire $2009^{2009}\equiv -1  \left[5\right]$
     \par
     Or $-1\equiv 4  \left[5\right]$ donc $2009^{2009}\equiv 4  \left[5\right]$
     \par
     Comme $0\leqslant 4 < 5$, le reste de la division euclidienne de $2009^{2009}$ par 5 est 4.
}

\end{document}