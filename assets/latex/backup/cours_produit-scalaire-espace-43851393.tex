\documentclass[a4paper]{article}

%================================================================================================================================
%
% Packages
%
%================================================================================================================================

\usepackage[T1]{fontenc} 	% pour caractères accentués
\usepackage[utf8]{inputenc}  % encodage utf8
\usepackage[french]{babel}	% langue : français
\usepackage{fourier}			% caractères plus lisibles
\usepackage[dvipsnames]{xcolor} % couleurs
\usepackage{fancyhdr}		% réglage header footer
\usepackage{needspace}		% empêcher sauts de page mal placés
\usepackage{graphicx}		% pour inclure des graphiques
\usepackage{enumitem,cprotect}		% personnalise les listes d'items (nécessaire pour ol, al ...)
\usepackage{hyperref}		% Liens hypertexte
\usepackage{pstricks,pst-all,pst-node,pstricks-add,pst-math,pst-plot,pst-tree,pst-eucl} % pstricks
\usepackage[a4paper,includeheadfoot,top=2cm,left=3cm, bottom=2cm,right=3cm]{geometry} % marges etc.
\usepackage{comment}			% commentaires multilignes
\usepackage{amsmath,environ} % maths (matrices, etc.)
\usepackage{amssymb,makeidx}
\usepackage{bm}				% bold maths
\usepackage{tabularx}		% tableaux
\usepackage{colortbl}		% tableaux en couleur
\usepackage{fontawesome}		% Fontawesome
\usepackage{environ}			% environment with command
\usepackage{fp}				% calculs pour ps-tricks
\usepackage{multido}			% pour ps tricks
\usepackage[np]{numprint}	% formattage nombre
\usepackage{tikz,tkz-tab} 			% package principal TikZ
\usepackage{pgfplots}   % axes
\usepackage{mathrsfs}    % cursives
\usepackage{calc}			% calcul taille boites
\usepackage[scaled=0.875]{helvet} % font sans serif
\usepackage{svg} % svg
\usepackage{scrextend} % local margin
\usepackage{scratch} %scratch
\usepackage{multicol} % colonnes
%\usepackage{infix-RPN,pst-func} % formule en notation polanaise inversée
\usepackage{listings}

%================================================================================================================================
%
% Réglages de base
%
%================================================================================================================================

\lstset{
language=Python,   % R code
literate=
{á}{{\'a}}1
{à}{{\`a}}1
{ã}{{\~a}}1
{é}{{\'e}}1
{è}{{\`e}}1
{ê}{{\^e}}1
{í}{{\'i}}1
{ó}{{\'o}}1
{õ}{{\~o}}1
{ú}{{\'u}}1
{ü}{{\"u}}1
{ç}{{\c{c}}}1
{~}{{ }}1
}


\definecolor{codegreen}{rgb}{0,0.6,0}
\definecolor{codegray}{rgb}{0.5,0.5,0.5}
\definecolor{codepurple}{rgb}{0.58,0,0.82}
\definecolor{backcolour}{rgb}{0.95,0.95,0.92}

\lstdefinestyle{mystyle}{
    backgroundcolor=\color{backcolour},   
    commentstyle=\color{codegreen},
    keywordstyle=\color{magenta},
    numberstyle=\tiny\color{codegray},
    stringstyle=\color{codepurple},
    basicstyle=\ttfamily\footnotesize,
    breakatwhitespace=false,         
    breaklines=true,                 
    captionpos=b,                    
    keepspaces=true,                 
    numbers=left,                    
xleftmargin=2em,
framexleftmargin=2em,            
    showspaces=false,                
    showstringspaces=false,
    showtabs=false,                  
    tabsize=2,
    upquote=true
}

\lstset{style=mystyle}


\lstset{style=mystyle}
\newcommand{\imgdir}{C:/laragon/www/newmc/assets/imgsvg/}
\newcommand{\imgsvgdir}{C:/laragon/www/newmc/assets/imgsvg/}

\definecolor{mcgris}{RGB}{220, 220, 220}% ancien~; pour compatibilité
\definecolor{mcbleu}{RGB}{52, 152, 219}
\definecolor{mcvert}{RGB}{125, 194, 70}
\definecolor{mcmauve}{RGB}{154, 0, 215}
\definecolor{mcorange}{RGB}{255, 96, 0}
\definecolor{mcturquoise}{RGB}{0, 153, 153}
\definecolor{mcrouge}{RGB}{255, 0, 0}
\definecolor{mclightvert}{RGB}{205, 234, 190}

\definecolor{gris}{RGB}{220, 220, 220}
\definecolor{bleu}{RGB}{52, 152, 219}
\definecolor{vert}{RGB}{125, 194, 70}
\definecolor{mauve}{RGB}{154, 0, 215}
\definecolor{orange}{RGB}{255, 96, 0}
\definecolor{turquoise}{RGB}{0, 153, 153}
\definecolor{rouge}{RGB}{255, 0, 0}
\definecolor{lightvert}{RGB}{205, 234, 190}
\setitemize[0]{label=\color{lightvert}  $\bullet$}

\pagestyle{fancy}
\renewcommand{\headrulewidth}{0.2pt}
\fancyhead[L]{maths-cours.fr}
\fancyhead[R]{\thepage}
\renewcommand{\footrulewidth}{0.2pt}
\fancyfoot[C]{}

\newcolumntype{C}{>{\centering\arraybackslash}X}
\newcolumntype{s}{>{\hsize=.35\hsize\arraybackslash}X}

\setlength{\parindent}{0pt}		 
\setlength{\parskip}{3mm}
\setlength{\headheight}{1cm}

\def\ebook{ebook}
\def\book{book}
\def\web{web}
\def\type{web}

\newcommand{\vect}[1]{\overrightarrow{\,\mathstrut#1\,}}

\def\Oij{$\left(\text{O}~;~\vect{\imath},~\vect{\jmath}\right)$}
\def\Oijk{$\left(\text{O}~;~\vect{\imath},~\vect{\jmath},~\vect{k}\right)$}
\def\Ouv{$\left(\text{O}~;~\vect{u},~\vect{v}\right)$}

\hypersetup{breaklinks=true, colorlinks = true, linkcolor = OliveGreen, urlcolor = OliveGreen, citecolor = OliveGreen, pdfauthor={Didier BONNEL - https://www.maths-cours.fr} } % supprime les bordures autour des liens

\renewcommand{\arg}[0]{\text{arg}}

\everymath{\displaystyle}

%================================================================================================================================
%
% Macros - Commandes
%
%================================================================================================================================

\newcommand\meta[2]{    			% Utilisé pour créer le post HTML.
	\def\titre{titre}
	\def\url{url}
	\def\arg{#1}
	\ifx\titre\arg
		\newcommand\maintitle{#2}
		\fancyhead[L]{#2}
		{\Large\sffamily \MakeUppercase{#2}}
		\vspace{1mm}\textcolor{mcvert}{\hrule}
	\fi 
	\ifx\url\arg
		\fancyfoot[L]{\href{https://www.maths-cours.fr#2}{\black \footnotesize{https://www.maths-cours.fr#2}}}
	\fi 
}


\newcommand\TitreC[1]{    		% Titre centré
     \needspace{3\baselineskip}
     \begin{center}\textbf{#1}\end{center}
}

\newcommand\newpar{    		% paragraphe
     \par
}

\newcommand\nosp {    		% commande vide (pas d'espace)
}
\newcommand{\id}[1]{} %ignore

\newcommand\boite[2]{				% Boite simple sans titre
	\vspace{5mm}
	\setlength{\fboxrule}{0.2mm}
	\setlength{\fboxsep}{5mm}	
	\fcolorbox{#1}{#1!3}{\makebox[\linewidth-2\fboxrule-2\fboxsep]{
  		\begin{minipage}[t]{\linewidth-2\fboxrule-4\fboxsep}\setlength{\parskip}{3mm}
  			 #2
  		\end{minipage}
	}}
	\vspace{5mm}
}

\newcommand\CBox[4]{				% Boites
	\vspace{5mm}
	\setlength{\fboxrule}{0.2mm}
	\setlength{\fboxsep}{5mm}
	
	\fcolorbox{#1}{#1!3}{\makebox[\linewidth-2\fboxrule-2\fboxsep]{
		\begin{minipage}[t]{1cm}\setlength{\parskip}{3mm}
	  		\textcolor{#1}{\LARGE{#2}}    
 	 	\end{minipage}  
  		\begin{minipage}[t]{\linewidth-2\fboxrule-4\fboxsep}\setlength{\parskip}{3mm}
			\raisebox{1.2mm}{\normalsize\sffamily{\textcolor{#1}{#3}}}						
  			 #4
  		\end{minipage}
	}}
	\vspace{5mm}
}

\newcommand\cadre[3]{				% Boites convertible html
	\par
	\vspace{2mm}
	\setlength{\fboxrule}{0.1mm}
	\setlength{\fboxsep}{5mm}
	\fcolorbox{#1}{white}{\makebox[\linewidth-2\fboxrule-2\fboxsep]{
  		\begin{minipage}[t]{\linewidth-2\fboxrule-4\fboxsep}\setlength{\parskip}{3mm}
			\raisebox{-2.5mm}{\sffamily \small{\textcolor{#1}{\MakeUppercase{#2}}}}		
			\par		
  			 #3
 	 		\end{minipage}
	}}
		\vspace{2mm}
	\par
}

\newcommand\bloc[3]{				% Boites convertible html sans bordure
     \needspace{2\baselineskip}
     {\sffamily \small{\textcolor{#1}{\MakeUppercase{#2}}}}    
		\par		
  			 #3
		\par
}

\newcommand\CHelp[1]{
     \CBox{Plum}{\faInfoCircle}{À RETENIR}{#1}
}

\newcommand\CUp[1]{
     \CBox{NavyBlue}{\faThumbsOUp}{EN PRATIQUE}{#1}
}

\newcommand\CInfo[1]{
     \CBox{Sepia}{\faArrowCircleRight}{REMARQUE}{#1}
}

\newcommand\CRedac[1]{
     \CBox{PineGreen}{\faEdit}{BIEN R\'EDIGER}{#1}
}

\newcommand\CError[1]{
     \CBox{Red}{\faExclamationTriangle}{ATTENTION}{#1}
}

\newcommand\TitreExo[2]{
\needspace{4\baselineskip}
 {\sffamily\large EXERCICE #1\ (\emph{#2 points})}
\vspace{5mm}
}

\newcommand\img[2]{
          \includegraphics[width=#2\paperwidth]{\imgdir#1}
}

\newcommand\imgsvg[2]{
       \begin{center}   \includegraphics[width=#2\paperwidth]{\imgsvgdir#1} \end{center}
}


\newcommand\Lien[2]{
     \href{#1}{#2 \tiny \faExternalLink}
}
\newcommand\mcLien[2]{
     \href{https~://www.maths-cours.fr/#1}{#2 \tiny \faExternalLink}
}

\newcommand{\euro}{\eurologo{}}

%================================================================================================================================
%
% Macros - Environement
%
%================================================================================================================================

\newenvironment{tex}{ %
}
{%
}

\newenvironment{indente}{ %
	\setlength\parindent{10mm}
}

{
	\setlength\parindent{0mm}
}

\newenvironment{corrige}{%
     \needspace{3\baselineskip}
     \medskip
     \textbf{\textsc{Corrigé}}
     \medskip
}
{
}

\newenvironment{extern}{%
     \begin{center}
     }
     {
     \end{center}
}

\NewEnviron{code}{%
	\par
     \boite{gray}{\texttt{%
     \BODY
     }}
     \par
}

\newenvironment{vbloc}{% boite sans cadre empeche saut de page
     \begin{minipage}[t]{\linewidth}
     }
     {
     \end{minipage}
}
\NewEnviron{h2}{%
    \needspace{3\baselineskip}
    \vspace{0.6cm}
	\noindent \MakeUppercase{\sffamily \large \BODY}
	\vspace{1mm}\textcolor{mcgris}{\hrule}\vspace{0.4cm}
	\par
}{}

\NewEnviron{h3}{%
    \needspace{3\baselineskip}
	\vspace{5mm}
	\textsc{\BODY}
	\par
}

\NewEnviron{margeneg}{ %
\begin{addmargin}[-1cm]{0cm}
\BODY
\end{addmargin}
}

\NewEnviron{html}{%
}

\begin{document}
\meta{url}{/cours/produit-scalaire-espace/}
\meta{pid}{555}
\meta{titre}{Orthogonalité et produit scalaire dans l'espace}
\meta{type}{cours}
\begin{h2}1. Produit scalaire\end{h2}
Deux vecteurs de l'espace sont toujours coplanaires (voir chapitre précédent). On peut alors définir le produit scalaire dans l'espace à l'aide de la définition donnée en Première pour deux vecteurs d'un plan.
\par
La plupart des propriétés vues en Première seront donc encore valables pour le produit scalaire dans l'espace, en particulier pour tous vecteurs $\vec{u}$ et $\vec{v}$ :
\begin{itemize}
     \item $\vec{u}.\vec{v}=||\vec{u}||\times ||\vec{v}||\times  \cos\left(\vec{u}, \vec{v}\right)$
     \item $\vec{u}.\vec{v}=\frac{1}{2} \left(||\vec{u}+\vec{v}||^{2}-||\vec{u}||^{2}-||\vec{v}||^{2}\right)$
     \item $\vec{u}^{2} = ||\vec{u}||^{2}$
\end{itemize}
La notion d'\textbf{orthogonalité de vecteurs} vue en Première est encore valable dans l'espace. Pour tous vecteurs $\vec{u}$ et $\vec{v}$ : $\vec{u}$ et $\vec{v}$ sont orthogonaux  $ \Leftrightarrow   \vec{u}.\vec{v}=0$.
\par
Les principales distinctions concernent les formules faisant intervenir les coordonnées puisque, dans l'espace, chaque vecteur possède trois coordonnées.
\cadre{vert}{Propriété}{% id="p10"
     L'espace est rapporté à un repère orthonormé $\left(O; \vec{i}, \vec{j}, \vec{k}\right)$
     \par
     Soient $\vec{u}$ et $\vec{v}$ deux vecteurs de coordonnées respectives $\left(x ; y ; z\right)$ et $\left(x^{\prime} ; y^{\prime} ; z^{\prime}\right)$ dans ce repère. Alors:
     \begin{center}$\vec{u}.\vec{v} =xx^{\prime}+yy^{\prime}+zz^{\prime}$\end{center}
}
\bloc{cyan}{Conséquences}{% id="r10"
     \begin{itemize}
          \item $||\vec{u}|| = \sqrt{x^{2}+y^{2}+z^{2}}$
          \item $AB=||\overrightarrow{AB}|| = \sqrt{\left(x_{B}-x_{A}\right)^{2}+\left(y_{B}-y_{A}\right)^{2}+\left(z_{B}-z_{A}\right)^{2}}$
     \end{itemize}
}
\begin{h2}2. Orthogonalité dans l'espace\end{h2}
\cadre{bleu}{Définition}{% id="d50"
     Deux droites $d_{1}$ et $d_{2}$ sont \textbf{orthogonales} si et seulement si il existe une droite qui est à la fois parallèle à $d_{1}$ et perpendiculaire à $d_{2}$
}
\begin{center}
     \begin{extern}%width="400" alt=""
          \newrgbcolor{afeeee}{0.7 0.9 0.9}
          \newrgbcolor{qqwuqq}{0. 0.4 0.}
          \newrgbcolor{wwzzff}{0.4 0.6 1.}
          \newrgbcolor{qqwwtt}{0. 0.4 0.2}
          \psset{xunit=1.0cm,yunit=1.0cm,algebraic=true,dimen=middle,dotstyle=o,dotsize=5pt 0,linewidth=1.6pt,arrowsize=3pt 2,arrowinset=0.25}
          \begin{pspicture*}(-5.6,-2.8)(5.6,3.5)
               \pspolygon[linewidth=0.8pt,linecolor=afeeee,fillcolor=afeeee,fillstyle=solid,opacity=0.1](-3.,0.)(-5.,-2.)(3.,-2.)(5.,0.)
               \psline[linewidth=0.8pt,linecolor=afeeee](-3.,0.)(-5.,-2.)
               \psline[linewidth=0.8pt,linecolor=afeeee](-5.,-2.)(3.,-2.)
               \psline[linewidth=0.8pt,linecolor=afeeee](3.,-2.)(5.,0.)
               \psline[linewidth=0.8pt,linecolor=afeeee](5.,0.)(-3.,0.)
               \rput[tl](-0.6097241218054483,2.8){$\qqwuqq{d_1}$}
               \rput[tl](2.672271188641368,-1.6){$\wwzzff{\mathscr{P}}$}
               \psline[linewidth=0.8pt,linecolor=qqwwtt](-4.69749899688947,0.6225150605167632)(0.39015188716309995,2.589170024100124)
               \psline[linewidth=0.8pt,linecolor=qqwuqq](-4.017190664344944,-2.)(0.8786019214951395,0.)
               \psline[linewidth=0.8pt,linecolor=red](-2.340323340435117,0.)(1.0884968352036755,-2)
               \psline[linewidth=0.4pt](-1.1428090352869673,-0.6557784132517245)(-0.9566973185809743,-0.5797491642822004)
               \psline[linewidth=0.4pt](-0.9566973185809743,-0.5797491642822004)(-0.7786761090029106,-0.6772364893524531)
               \rput[tl](0.812473846054839,-0.1){$\qqwuqq{d'_1}$}
               \rput[tl](1.2,-1.6){$\red{d_2}$}
          \end{pspicture*}
     \end{extern}
\end{center}
\begin{center}\textit{$d_{1}$ et $d_{2}$ sont orthogonales}\end{center}
\bloc{cyan}{Remarque}{% id="r50"
     Attention à ne pas confondre \textit{orthogonales} et \textit{perpendiculaires}. Le terme \textit{perpendiculaires} s'emploie uniquement pour des droites \textbf{sécantes} (donc \textbf{coplanaires}).
}
\cadre{vert}{Propriétés}{% id="t60"
     Soient deux droites $d_{1}$ et $d_{2}$, $\overrightarrow{u_{1}}$ un vecteur directeur de $d_{1}$ et $\overrightarrow{u_{2}}$ un vecteur directeur de $d_{2}$.
     \par
     $d_{1}$ et $d_{2}$ sont orthogonales si et seulement si les vecteurs $\overrightarrow{u_{1}}$ et $\overrightarrow{u_{2}}$ sont orthogonaux, c'est à dire si et seulement si $\overrightarrow{u_{1}}.\overrightarrow{u_{2}}=0$
}
\cadre{bleu}{Définition (Droite perpendiculaire à un plan)}{% id="d65"
     Une droite $d$ est \textbf{perpendiculaire} (ou \textbf{orthogonale}) à un plan $\mathscr P$ si et seulement si elle est orthogonale à toutes les droites incluses dans ce plan.
}
\begin{center}
     \begin{extern}%width="400" alt=""
          \newrgbcolor{afeeee}{0.7 0.9 0.9}
          \newrgbcolor{wwzzff}{0.4 0.6 1.}
          \newrgbcolor{qqwuqq}{0. 0.4 0.}
          \psset{xunit=1.0cm,yunit=1.0cm,algebraic=true,dimen=middle,dotstyle=o,dotsize=5pt 0,linewidth=1.6pt,arrowsize=3pt 2,arrowinset=0.25}
          \begin{pspicture*}(-5.5,-4.6)(6.4,2.6)
               \pspolygon[linewidth=0.8pt,linecolor=afeeee,fillcolor=afeeee,fillstyle=solid,opacity=0.1](-3.,0.)(-5.,-2.)(3.,-2.)(5.,0.)
               \psline[linewidth=0.8pt,linecolor=afeeee](-3.,0.)(-5.,-2.)
               \psline[linewidth=0.8pt,linecolor=afeeee](-5.,-2.)(3.,-2.)
               \psline[linewidth=0.8pt,linecolor=afeeee](3.,-2.)(5.,0.)
               \psline[linewidth=0.8pt,linecolor=afeeee](5.,0.)(-3.,0.)
               \rput[tl](0.20176035252591304,2.59630929745675){$\red{d}$}
               \rput[tl](2.6,-1.5){$\wwzzff{\mathscr{P}}$}
               \psline[linewidth=0.8pt,linecolor=red](0.,3.)(0.,-1.)
               \psline[linewidth=0.8pt,linecolor=red](0.,-2.)(0.,-7.)
               \begin{scriptsize}
                    \psdots[dotsize=1pt 0,dotstyle=*,linecolor=blue](0.,-1.)
               \end{scriptsize}
          \end{pspicture*}
     \end{extern}
\end{center}
\begin{center}\textbf{\textit{Droite perpendiculaire à un plan}}\end{center}
\bloc{cyan}{Remarque}{% id="r65"
     Une droite orthogonale à un plan coupe nécessairement ce plan en un point. Il n'y a donc plus lieu ici de distinguer orthogonalité et perpendicularité.
}
\cadre{vert}{Propriété}{% id="p70"
     La droite $d$ est perpendiculaire au plan $\mathscr P$ si et seulement si elle est orthogonale à \textbf{deux droites sécantes} incluses dans ce plan.
}
\begin{center}
     \begin{extern}%width="400" alt=""
          \newrgbcolor{afeeee}{0.7 0.9 0.9}
          \newrgbcolor{wwzzff}{0.4 0.6 1.}
          \newrgbcolor{qqwuqq}{0. 0.4 0.}
          \psset{xunit=1.0cm,yunit=1.0cm,algebraic=true,dimen=middle,dotstyle=o,dotsize=5pt 0,linewidth=1.6pt,arrowsize=3pt 2,arrowinset=0.25}
          \begin{pspicture*}(-5.5,-4.6)(6.4,2.6)
               \pspolygon[linewidth=0.8pt,linecolor=afeeee,fillcolor=afeeee,fillstyle=solid,opacity=0.1](-3.,0.)(-5.,-2.)(3.,-2.)(5.,0.)
               \psline[linewidth=0.8pt,linecolor=afeeee](-3.,0.)(-5.,-2.)
               \psline[linewidth=0.8pt,linecolor=afeeee](-5.,-2.)(3.,-2.)
               \psline[linewidth=0.8pt,linecolor=afeeee](3.,-2.)(5.,0.)
               \psline[linewidth=0.8pt,linecolor=afeeee](5.,0.)(-3.,0.)
               \rput[tl](0.20176035252591304,2.59630929745675){$\red{d}$}
               \rput[tl](2.6,-1.5){$\wwzzff{\mathscr{P}}$}
               \psline[linewidth=0.8pt,linecolor=red](0.,3.)(0.,-1.)
               \psline[linewidth=0.8pt,linecolor=red](0.,-2.)(0.,-7.)
               \psline[linewidth=0.8pt,linecolor=qqwuqq](-3.9857227292164437,-2.)(-0.5037395145067665,0.)
               \psline[linewidth=0.8pt,linecolor=qqwuqq](-3.332609727222885,-0.3326097272228852)(0.7024376775168468,-2.)
               \begin{scriptsize}
                    \psdots[dotsize=1pt 0,dotstyle=*,linecolor=blue](0.,-1.)
               \end{scriptsize}
          \end{pspicture*}
     \end{extern}
\end{center}
\cadre{bleu}{Définition (Plans perpendiculaires)}{% id="d70"
     Deux plans $\mathscr P_{1}$ et  $\mathscr P_{1}$ sont \textbf{perpendiculaires} (ou \textbf{orthogonaux}) si et seulement si $\mathscr P_{1}$ contient une droite $d$ perpendiculaire à $\mathscr P_{2}$.
}
\begin{center}
     \begin{extern}%width="350" alt=""
          \newrgbcolor{afeeee}{0.7 0.9 0.9}
          \newrgbcolor{zzffzz}{0.6 1. 0.6}
          \newrgbcolor{ccccff}{0.8 0.8 1.}
          \newrgbcolor{ffzzff}{1. 0.6 1.}
          \newrgbcolor{qqccqq}{0. 0.4 0.}
          \newrgbcolor{qqzzff}{0. 0.6 1.}
          \psset{xunit=1.0cm,yunit=1.0cm,algebraic=true,dimen=middle,dotstyle=o,dotsize=5pt 0,linewidth=1.6pt,arrowsize=3pt 2,arrowinset=0.25}
          \begin{pspicture*}(-2.3,-1.3)(6.9,7.4)
               \pspolygon[linewidth=0.8pt,linecolor=afeeee,fillcolor=afeeee,fillstyle=solid,opacity=0.1](0.061284443758347074,4.)(-1.938715556241653,2.)(4.061284443758347,2.)(6.061284443758347,4.)
               \pspolygon[linewidth=0.4pt,linecolor=qqccqq,fillcolor=qqccqq,fillstyle=solid,opacity=0.03](1.0080587194837574,-0.8588531512120481)(1.,5.)(3.,7.)(3.,1.)
               \psline[linewidth=0.8pt,linecolor=ccccff](1.0041264318847574,2.)(3.,4.)
               \psline[linewidth=0.8pt,linecolor=red](1.9956031708383426,2.0000048330325084)(1.9934714932952935,0.060720963628051905)
               \psline[linewidth=0.8pt,linecolor=red](1.9966964627665804,2.994622151160741)(2.,6.)
               \rput[tl](3.1583617599248424,7.0096617252930775){$\qqccqq{\mathscr{P}_1}$}
               \rput[tl](5.5,4.4){$\qqzzff{\mathscr{P}_2}$}
               \rput[tl](2.1,5.9){$\red{d}$}
          \end{pspicture*}
     \end{extern}
\end{center}
\bloc{cyan}{Remarque}{% id="r70"
     \textbf{Attention, } cela ne signifie \textbf{pas} que toutes les droites de $\mathscr P_{1}$ sont orthogonales à toutes les droites de  $\mathscr P_{2}$
}
\cadre{bleu}{Définition (Vecteur normal à un plan)}{% id="d80"
     On dit qu'un vecteur $\vec{n}$ non nul est un vecteur \textbf{normal} au plan $\mathscr P$ si et seulement si la droite dirigée par $\vec{n}$ est perpendiculaire au plan $\mathscr P$.
}
\begin{center}
     \begin{extern}%width="400" alt=""
          \newrgbcolor{afeeee}{0.7 0.9 0.9}
          \newrgbcolor{ttzzqq}{0.2 0.6 0.}
          \newrgbcolor{wwzzff}{0.4 0.6 1.}
          \psset{xunit=1.0cm,yunit=1.0cm,algebraic=true,dimen=middle,dotstyle=o,dotsize=5pt 0,linewidth=1.6pt,arrowsize=3pt 2,arrowinset=0.25}
          \begin{pspicture*}(-5.3,-3.8)(5.5,2.5)
               \pspolygon[linewidth=0.8pt,linecolor=afeeee,fillcolor=afeeee,fillstyle=solid,opacity=0.1](-3.,0.)(-5.,-2.)(3.,-2.)(5.,0.)
               \psline[linewidth=0.8pt,linecolor=afeeee](-3.,0.)(-5.,-2.)
               \psline[linewidth=0.8pt,linecolor=afeeee](-5.,-2.)(3.,-2.)
               \psline[linewidth=0.8pt,linecolor=afeeee](3.,-2.)(5.,0.)
               \psline[linewidth=0.8pt,linecolor=afeeee](5.,0.)(-3.,0.)
               \rput[tl](0.2,2.2){$\ttzzqq{d}$}
               \rput[tl](2.66,-1.5){$\wwzzff{\mathscr{P}}$}
               \psline[linewidth=0.8pt,linecolor=ttzzqq](0.,3.)(0.,-1.)
               \psline[linewidth=0.8pt,linecolor=ttzzqq](0.,-2.)(0.,-7.)
               \psline[linewidth=0.8pt,linecolor=red]{->}(0.,-1.)(0.,0.9122128406690555)
               \rput[tl](0.2,0.5){$\red{\vec{n}}$}
          \end{pspicture*}
     \end{extern}
\end{center}
\cadre{rouge}{Théorème}{% id="t90"
     Soit $\mathscr P$ un plan de vecteur normal $\vec{n}$ et soit $A$ un point de $\mathscr P$.
     \begin{center}$M \in  \mathscr P   \Leftrightarrow  \overrightarrow{AM}.\vec{n} = 0$.\end{center}
}
\cadre{rouge}{Théorème}{% id="t100"
     L'espace est rapporté à un repère orthonormé $\left(O; \vec{i}, \vec{j}, \vec{k}\right)$
     \par
     Le plan $\mathscr P$ de vecteur normal $\vec{n} \left(a ; b; c\right)$ admet une équation cartésienne de la forme :
     \begin{center}$ax+by+cz+d=0$\end{center}
     où $a$, $b$, $c$ sont les coordonnées de $\vec{n}$ et $d$ un nombre réel.
     \par
     \textbf{Réciproquement}, l'ensemble des points $M\left(x ; y ; z\right)$ tels que $ax+by+cz+d=0$ ($a, b, c, d$ étant des réels avec $a\neq 0$ ou $b\neq 0$ ou $c\neq 0$)  est un plan dont un vecteur normal est  $\vec{n}\left(a ; b ; c\right)$.
}
\bloc{cyan}{Démonstration}{% id="m100"
     Soit $A\left(x_{A} ; y_{A} ; z_{A}\right)$ un point de $\mathscr P$ :
     \par
     $M \in  \mathscr P   \Leftrightarrow   \overrightarrow{AM}.\vec{n} = 0   \Leftrightarrow   a\left(x-x_{A}\right)+b\left(y-y_{A}\right)+c\left(z-z_{A}\right)= 0$
     <span style="visibility:hidden">$M \in  \mathscr P   \Leftrightarrow   \overrightarrow{AM}.\vec{n} = 0 $</span>$ \Leftrightarrow   ax+by+cz-ax_{A}-by_{A}-cz_{A}= 0$
     \par
     et il suffit de poser $d=-ax_{A}-by_{A}-cz_{A}$.
     \par
     \textbf{Réciproquement}, supposons par exemple $a\neq 0$.
     \par
     Soit $A$ le point de coordonnées $\left(-\frac{d}{a} ; 0 ;0\right)$ Les coordonnées de $A$ vérifient :
     \par
     $ax_{A}+by_{A}+cz_{A}+d= 0$.
     \par
     On a alors $d = -ax_{A}-by_{A}-cz_{A}$ donc :
     \par
     $ax+by+cz+d=0  \Leftrightarrow   a\left(x-x_{A}\right)+b\left(y-y_{A}\right)+c\left(z-z_{A}\right)= 0  \Leftrightarrow   \overrightarrow{AM}.\vec{n} = 0$
     \par
     donc $M\left(x ; y ; z\right)$ appartient au plan passant par $A$ et dont un vecteur normal est $\vec{n}\left(a ; b ; c\right)$
}
\bloc{orange}{Exemple}{% id="e100"
     On cherche une équation cartésienne du plan passant par $A\left(1 ; 3 ; -2\right)$ et de vecteur normal $\vec{n}\left(1 ; 1 ; 1\right)$.
     \par
     Ce plan admet une équation cartésienne de la forme :
     \par
     $\left(E\right)      x+y+z+d=0$
     \par
     Le point $A\left(1 ; 3 ; -2\right)$ appartient à ce plan, donc les coordonnées de $A$ vérifient l'équation $\left(E\right)$ :
     \par
     $1+3-2+d=0$ soit $d=-2$
     \par
     Une équation cartésienne du plan est donc :
     \par
     $\left(E\right)      x+y+z-2=0$
}
\cadre{vert}{Propriétés}{% id="p110"
     \begin{itemize}
          \item Une droite $d$ est parallèle à un plan $\mathscr P$ si et seulement si un vecteur directeur de $d$ est orthogonal à un vecteur normal de $\mathscr P$.
          \item Une droite $d$ est perpendiculaire à un plan $\mathscr P$ si et seulement si un vecteur directeur de $d$ est colinéaire à un vecteur normal de $\mathscr P$.
          \item Deux plans sont parallèles si et seulement si leurs vecteurs normaux sont colinéaires.
          \item Deux plans sont perpendiculaires si et seulement si leurs vecteurs normaux sont  orthogonaux.
     \end{itemize}
}

\end{document}