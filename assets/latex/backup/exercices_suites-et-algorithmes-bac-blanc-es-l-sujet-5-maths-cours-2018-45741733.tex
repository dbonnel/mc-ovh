\documentclass[a4paper]{article}

%================================================================================================================================
%
% Packages
%
%================================================================================================================================

\usepackage[T1]{fontenc} 	% pour caractères accentués
\usepackage[utf8]{inputenc}  % encodage utf8
\usepackage[french]{babel}	% langue : français
\usepackage{fourier}			% caractères plus lisibles
\usepackage[dvipsnames]{xcolor} % couleurs
\usepackage{fancyhdr}		% réglage header footer
\usepackage{needspace}		% empêcher sauts de page mal placés
\usepackage{graphicx}		% pour inclure des graphiques
\usepackage{enumitem,cprotect}		% personnalise les listes d'items (nécessaire pour ol, al ...)
\usepackage{hyperref}		% Liens hypertexte
\usepackage{pstricks,pst-all,pst-node,pstricks-add,pst-math,pst-plot,pst-tree,pst-eucl} % pstricks
\usepackage[a4paper,includeheadfoot,top=2cm,left=3cm, bottom=2cm,right=3cm]{geometry} % marges etc.
\usepackage{comment}			% commentaires multilignes
\usepackage{amsmath,environ} % maths (matrices, etc.)
\usepackage{amssymb,makeidx}
\usepackage{bm}				% bold maths
\usepackage{tabularx}		% tableaux
\usepackage{colortbl}		% tableaux en couleur
\usepackage{fontawesome}		% Fontawesome
\usepackage{environ}			% environment with command
\usepackage{fp}				% calculs pour ps-tricks
\usepackage{multido}			% pour ps tricks
\usepackage[np]{numprint}	% formattage nombre
\usepackage{tikz,tkz-tab} 			% package principal TikZ
\usepackage{pgfplots}   % axes
\usepackage{mathrsfs}    % cursives
\usepackage{calc}			% calcul taille boites
\usepackage[scaled=0.875]{helvet} % font sans serif
\usepackage{svg} % svg
\usepackage{scrextend} % local margin
\usepackage{scratch} %scratch
\usepackage{multicol} % colonnes
%\usepackage{infix-RPN,pst-func} % formule en notation polanaise inversée
\usepackage{listings}

%================================================================================================================================
%
% Réglages de base
%
%================================================================================================================================

\lstset{
language=Python,   % R code
literate=
{á}{{\'a}}1
{à}{{\`a}}1
{ã}{{\~a}}1
{é}{{\'e}}1
{è}{{\`e}}1
{ê}{{\^e}}1
{í}{{\'i}}1
{ó}{{\'o}}1
{õ}{{\~o}}1
{ú}{{\'u}}1
{ü}{{\"u}}1
{ç}{{\c{c}}}1
{~}{{ }}1
}


\definecolor{codegreen}{rgb}{0,0.6,0}
\definecolor{codegray}{rgb}{0.5,0.5,0.5}
\definecolor{codepurple}{rgb}{0.58,0,0.82}
\definecolor{backcolour}{rgb}{0.95,0.95,0.92}

\lstdefinestyle{mystyle}{
    backgroundcolor=\color{backcolour},   
    commentstyle=\color{codegreen},
    keywordstyle=\color{magenta},
    numberstyle=\tiny\color{codegray},
    stringstyle=\color{codepurple},
    basicstyle=\ttfamily\footnotesize,
    breakatwhitespace=false,         
    breaklines=true,                 
    captionpos=b,                    
    keepspaces=true,                 
    numbers=left,                    
xleftmargin=2em,
framexleftmargin=2em,            
    showspaces=false,                
    showstringspaces=false,
    showtabs=false,                  
    tabsize=2,
    upquote=true
}

\lstset{style=mystyle}


\lstset{style=mystyle}
\newcommand{\imgdir}{C:/laragon/www/newmc/assets/imgsvg/}
\newcommand{\imgsvgdir}{C:/laragon/www/newmc/assets/imgsvg/}

\definecolor{mcgris}{RGB}{220, 220, 220}% ancien~; pour compatibilité
\definecolor{mcbleu}{RGB}{52, 152, 219}
\definecolor{mcvert}{RGB}{125, 194, 70}
\definecolor{mcmauve}{RGB}{154, 0, 215}
\definecolor{mcorange}{RGB}{255, 96, 0}
\definecolor{mcturquoise}{RGB}{0, 153, 153}
\definecolor{mcrouge}{RGB}{255, 0, 0}
\definecolor{mclightvert}{RGB}{205, 234, 190}

\definecolor{gris}{RGB}{220, 220, 220}
\definecolor{bleu}{RGB}{52, 152, 219}
\definecolor{vert}{RGB}{125, 194, 70}
\definecolor{mauve}{RGB}{154, 0, 215}
\definecolor{orange}{RGB}{255, 96, 0}
\definecolor{turquoise}{RGB}{0, 153, 153}
\definecolor{rouge}{RGB}{255, 0, 0}
\definecolor{lightvert}{RGB}{205, 234, 190}
\setitemize[0]{label=\color{lightvert}  $\bullet$}

\pagestyle{fancy}
\renewcommand{\headrulewidth}{0.2pt}
\fancyhead[L]{maths-cours.fr}
\fancyhead[R]{\thepage}
\renewcommand{\footrulewidth}{0.2pt}
\fancyfoot[C]{}

\newcolumntype{C}{>{\centering\arraybackslash}X}
\newcolumntype{s}{>{\hsize=.35\hsize\arraybackslash}X}

\setlength{\parindent}{0pt}		 
\setlength{\parskip}{3mm}
\setlength{\headheight}{1cm}

\def\ebook{ebook}
\def\book{book}
\def\web{web}
\def\type{web}

\newcommand{\vect}[1]{\overrightarrow{\,\mathstrut#1\,}}

\def\Oij{$\left(\text{O}~;~\vect{\imath},~\vect{\jmath}\right)$}
\def\Oijk{$\left(\text{O}~;~\vect{\imath},~\vect{\jmath},~\vect{k}\right)$}
\def\Ouv{$\left(\text{O}~;~\vect{u},~\vect{v}\right)$}

\hypersetup{breaklinks=true, colorlinks = true, linkcolor = OliveGreen, urlcolor = OliveGreen, citecolor = OliveGreen, pdfauthor={Didier BONNEL - https://www.maths-cours.fr} } % supprime les bordures autour des liens

\renewcommand{\arg}[0]{\text{arg}}

\everymath{\displaystyle}

%================================================================================================================================
%
% Macros - Commandes
%
%================================================================================================================================

\newcommand\meta[2]{    			% Utilisé pour créer le post HTML.
	\def\titre{titre}
	\def\url{url}
	\def\arg{#1}
	\ifx\titre\arg
		\newcommand\maintitle{#2}
		\fancyhead[L]{#2}
		{\Large\sffamily \MakeUppercase{#2}}
		\vspace{1mm}\textcolor{mcvert}{\hrule}
	\fi 
	\ifx\url\arg
		\fancyfoot[L]{\href{https://www.maths-cours.fr#2}{\black \footnotesize{https://www.maths-cours.fr#2}}}
	\fi 
}


\newcommand\TitreC[1]{    		% Titre centré
     \needspace{3\baselineskip}
     \begin{center}\textbf{#1}\end{center}
}

\newcommand\newpar{    		% paragraphe
     \par
}

\newcommand\nosp {    		% commande vide (pas d'espace)
}
\newcommand{\id}[1]{} %ignore

\newcommand\boite[2]{				% Boite simple sans titre
	\vspace{5mm}
	\setlength{\fboxrule}{0.2mm}
	\setlength{\fboxsep}{5mm}	
	\fcolorbox{#1}{#1!3}{\makebox[\linewidth-2\fboxrule-2\fboxsep]{
  		\begin{minipage}[t]{\linewidth-2\fboxrule-4\fboxsep}\setlength{\parskip}{3mm}
  			 #2
  		\end{minipage}
	}}
	\vspace{5mm}
}

\newcommand\CBox[4]{				% Boites
	\vspace{5mm}
	\setlength{\fboxrule}{0.2mm}
	\setlength{\fboxsep}{5mm}
	
	\fcolorbox{#1}{#1!3}{\makebox[\linewidth-2\fboxrule-2\fboxsep]{
		\begin{minipage}[t]{1cm}\setlength{\parskip}{3mm}
	  		\textcolor{#1}{\LARGE{#2}}    
 	 	\end{minipage}  
  		\begin{minipage}[t]{\linewidth-2\fboxrule-4\fboxsep}\setlength{\parskip}{3mm}
			\raisebox{1.2mm}{\normalsize\sffamily{\textcolor{#1}{#3}}}						
  			 #4
  		\end{minipage}
	}}
	\vspace{5mm}
}

\newcommand\cadre[3]{				% Boites convertible html
	\par
	\vspace{2mm}
	\setlength{\fboxrule}{0.1mm}
	\setlength{\fboxsep}{5mm}
	\fcolorbox{#1}{white}{\makebox[\linewidth-2\fboxrule-2\fboxsep]{
  		\begin{minipage}[t]{\linewidth-2\fboxrule-4\fboxsep}\setlength{\parskip}{3mm}
			\raisebox{-2.5mm}{\sffamily \small{\textcolor{#1}{\MakeUppercase{#2}}}}		
			\par		
  			 #3
 	 		\end{minipage}
	}}
		\vspace{2mm}
	\par
}

\newcommand\bloc[3]{				% Boites convertible html sans bordure
     \needspace{2\baselineskip}
     {\sffamily \small{\textcolor{#1}{\MakeUppercase{#2}}}}    
		\par		
  			 #3
		\par
}

\newcommand\CHelp[1]{
     \CBox{Plum}{\faInfoCircle}{À RETENIR}{#1}
}

\newcommand\CUp[1]{
     \CBox{NavyBlue}{\faThumbsOUp}{EN PRATIQUE}{#1}
}

\newcommand\CInfo[1]{
     \CBox{Sepia}{\faArrowCircleRight}{REMARQUE}{#1}
}

\newcommand\CRedac[1]{
     \CBox{PineGreen}{\faEdit}{BIEN R\'EDIGER}{#1}
}

\newcommand\CError[1]{
     \CBox{Red}{\faExclamationTriangle}{ATTENTION}{#1}
}

\newcommand\TitreExo[2]{
\needspace{4\baselineskip}
 {\sffamily\large EXERCICE #1\ (\emph{#2 points})}
\vspace{5mm}
}

\newcommand\img[2]{
          \includegraphics[width=#2\paperwidth]{\imgdir#1}
}

\newcommand\imgsvg[2]{
       \begin{center}   \includegraphics[width=#2\paperwidth]{\imgsvgdir#1} \end{center}
}


\newcommand\Lien[2]{
     \href{#1}{#2 \tiny \faExternalLink}
}
\newcommand\mcLien[2]{
     \href{https~://www.maths-cours.fr/#1}{#2 \tiny \faExternalLink}
}

\newcommand{\euro}{\eurologo{}}

%================================================================================================================================
%
% Macros - Environement
%
%================================================================================================================================

\newenvironment{tex}{ %
}
{%
}

\newenvironment{indente}{ %
	\setlength\parindent{10mm}
}

{
	\setlength\parindent{0mm}
}

\newenvironment{corrige}{%
     \needspace{3\baselineskip}
     \medskip
     \textbf{\textsc{Corrigé}}
     \medskip
}
{
}

\newenvironment{extern}{%
     \begin{center}
     }
     {
     \end{center}
}

\NewEnviron{code}{%
	\par
     \boite{gray}{\texttt{%
     \BODY
     }}
     \par
}

\newenvironment{vbloc}{% boite sans cadre empeche saut de page
     \begin{minipage}[t]{\linewidth}
     }
     {
     \end{minipage}
}
\NewEnviron{h2}{%
    \needspace{3\baselineskip}
    \vspace{0.6cm}
	\noindent \MakeUppercase{\sffamily \large \BODY}
	\vspace{1mm}\textcolor{mcgris}{\hrule}\vspace{0.4cm}
	\par
}{}

\NewEnviron{h3}{%
    \needspace{3\baselineskip}
	\vspace{5mm}
	\textsc{\BODY}
	\par
}

\NewEnviron{margeneg}{ %
\begin{addmargin}[-1cm]{0cm}
\BODY
\end{addmargin}
}

\NewEnviron{html}{%
}

\begin{document}
\meta{url}{/exercices/suites-et-algorithmes-bac-blanc-es-l-sujet-5-maths-cours-2018/}
\meta{pid}{10556}
\meta{titre}{Suites et algorithmes - Bac blanc ES/L Sujet 5 - Maths-cours 2018}
\meta{type}{exercices}
%
\begin{h2}Exercice 4 (4 points)\end{h2}
\par
On considère la suite géométrique $(u_n)$ de premier terme ${u_0= 1~000}$ et de raison ${q=0,9}$.
\par
\begin{enumerate}
     \item
     Exprimer $u_n$ en fonction de $n$.
     \item
     On pose $S_n=u_0+u_1+u_2+ \cdots +u_n$.
     \par
     Compléter l'algorithme ci-après afin qu'il calcule et affiche la valeur de $S_{10}$.
     \par
     \begin{center}
          \begin{extern}%width="340" alt="Algorithme de calcul de la somme S10"
               \begin{tabular}{|l l|}\hline
                    Variables :	&$I$ est un entier naturel\\
                    &$S$ est un nombre réel\\
                    & \\
                    Initialisation: &$S$ prend la valeur 0\\
                    & \\
                    Traitement: &Pour $I=0$ à $\cdots$ faire :\\
                    &\qquad$S$ prend la valeur $\cdots$\\
                    &Fin Pour\\
                    & \\
                    Sortie :	&Afficher $\cdots$ \\
                    \hline
               \end{tabular}
          \end{extern}
     \end{center}
     \item
     Montrer que, pour tout entier naturel $n$ :
     \[ S_n=10~000 \left( 1-0,9^{n+1} \right). \]
     \par
     En déduire la valeur affichée en sortie de l'algorithme précédent.
     \par
     On arrondira le résultat à l'unité.
     \item
     Quelle est la limite de la somme $S_n$ lorsque $n$ tend vers $+\infty$ ?
     \item
     Déterminer, par la méthode de votre choix, la plus petite valeur de l'entier $n$ telle que :
     \[ S_n > 9~000. \]
     \par
\end{enumerate}
\begin{corrige}
     \begin{enumerate}
          \item
          La suite $(u_n)$ étant une suite géométrique, pour tout entier naturel $n$ :
          \par
          $u_n=u_0 \times q^n=1~000 \times 0,9^n$.
          \item
          L'algorithme calculant et affichant la valeur de $S_{10}$ peut être complété comme suit :
          \par
          \begin{center}
               \begin{extern}%width="420" alt="Algorithme de calcul de S10 complété"
                    \begin{tabular}{|l l|}\hline
                         Variables :	&$I$ est un entier naturel\\
                         &$S$ est un nombre réel\\
                         & \\
                         Initialisation: &$S$ prend la valeur 0\\
                         & \\
                         Traitement: &Pour $I=0$ à \textcolor{red}{10} faire :\\
                         &\qquad$S$ prend la valeur $\textcolor{red}{S+1~000 \times 0,9^I}$\\
                         &Fin Pour\\
                         & \\
                         Sortie :	&Afficher \textcolor{red}{$S$} \\
                         \hline
                    \end{tabular}
               \end{extern}
          \end{center}
          \par
          \cadre{vert}{En pratique}{
               Pour calculer la somme $S=u_0+u_1+ \cdots + u_n$ à l'aide d'un algorithme :
               \par
               \begin{itemize}
                    \item %
                    On \textbf{initialise} $S$ à \textbf{0}.
                    \item %
                    On \textbf{cumule} les termes de la suite $(u_n)$ dans la variable $S$ grâce à une instruction du type :
                    \par
                    Pour $i=0$ à $n$ faire :\\
                    $\phantom{-}S$ prend la valeur $S+$\textit{<formule donnant $u_i$>}\\
                    Fin Pour\\
                    \par
               \end{itemize}
          }
          \item
          D'après la question \textbf{1.} :
          \par
          $u_0=1~000$\\
          $u_1=1~000 \times 0,9 $\\
          $u_2=1~000 \times 0,9^2 $\\
          $\qquad \cdots $\\
          $u_n=1~000 \times 0,9^n $.\\
          \par
          On a alors :
          \par
          $S_n=1~000 + 1~000 \times 0,9 + 1~000 \times 0,9^2 +$\nosp$ \cdots + 1~000 \times 0,9^n$.
          \par
          En mettant 1~000 en facteur, on obtient :
          \par
          $S_n=1~000~\left( 1 + 0,9 + 0,9^2 + \cdots + 0,9^n \right)$.
          \par
          Or :
          \par
          $1+0,9++0,9^{2}+\cdots+0,9^{n}=\dfrac{1-0,9^{n+1}}{1-0,9}$\nosp$=\dfrac{1-0,9^{n+1}}{0,1}=10 \left(1-0,9^{n+1} \right)$.
          \par
          Donc :
          \par
          $S_n=1~000 \times 10 \left(1-0,9^{n+1} \right)$\\
          $\phantom{S_n}=10~000 \left( 1-0,9^{n+1} \right)$.
          \par
          \cadre{rouge}{À retenir}{
               \par
               La formule suivante permet de calculer la somme des premiers termes d'une suite géométrique :
               \par
               \[ 1+q+q^2+\cdots+q^{n}=\dfrac{1-q^{n+1}}{1-q}. \]
               \par
          }
          \par
          L'algorithme précédent affiche la valeur $S_{10}$ c'est à dire :
          \par
          $S_{10}=10~000 \left( 1-0,9^{11} \right) \approx 6~862~$ (arrondi à l'unité).
          \item
          $0 \leqslant 0,9 < 1$ donc $\lim\limits_{n \rightarrow +\infty}0,9 ^n = 0$.
          \par
          Comme $0,9^{n+1} = 0,9 \times 0,9 ^n$, alors ${\lim\limits_{n \rightarrow +\infty}0,9 ^{n+1} = 0}$.
          \par
          Par conséquent :
          \par
          ${\lim\limits_{n \rightarrow +\infty}\left(1-0,9 ^{n+1}\right) = 1}$\quad et \quad${\lim\limits_{n \rightarrow +\infty}\left(10~000 \left( 1-0,9^{n+1} \right)\right) = 10~000}$.
          \par
          La somme $S_n$ tend vers 10~000 lorsque $n$ tend vers $+\infty$.
          \item
          \textbf{Méthode 1 : \`A la calculatrice}
          \par
          La suite $S_n$ est croissante. \`A l'aide d'un tableau de valeurs pour la fonction $x \longmapsto 10~000 \left( 1-0,9^{x+1} \right)$, on trouve :
          \[ S_{20} \approx 8~905 \quad \text{et} \quad S_{21} \approx 9~015. \]
          La plus petite valeur de l'entier $n$ telle que $S_n > 9~000$ est donc $21$.
          \vspace{0.5cm}
          \par
          \textbf{Méthode 2 : Par le calcul}
          \par
          $S_n > 9~000 ~ \Leftrightarrow ~10~000 \left( 1-0,9^{n+1}\right) > 9~000 $\\
          $	\phantom{S_n > 9~000 ~} \Leftrightarrow ~ 1-0,9^{n+1} > 0,9$ \\
          $	\phantom{S_n > 9~000 ~}  \Leftrightarrow ~ -0,9^{n+1} > - 0,1$ \\
          $	\phantom{S_n > 9~000 ~}  \Leftrightarrow ~ 0,9^{n+1} < 0,1 $
          \par
          La fonction $\ln$ étant strictement croissante sur $]0~;~+\infty[$ :
          \par
          $S_n > 9~000 ~ \Leftrightarrow ~ \ln \left(0,9^{n+1} \right) < \ln (0,1) $ \\
          $\phantom{S_n > 9~000 ~ } \Leftrightarrow ~ (n+1)\ln (0,9) < \ln (0,1)$
          \par
          $0,9 < 1$ donc $\ln (0,9) < 0$ ; par conséquent :
          \par
          $S_n > 9~000  ~\Leftrightarrow ~n+1 > \dfrac{\ln (0,1)}{\ln (0,9)}$ \\
          $\phantom{S_n > 9~000  ~}   \Leftrightarrow ~ n > \dfrac{\ln (0,1)}{\ln (0,9)} - 1$
          \par
          $\dfrac{\ln (0,1)}{\ln (0,9)} - 1 \approx 20,9$ (arrondi au dixième)
          \par
          La plus petite valeur de l'entier $n$ telle que $S_n > 9~000$ est donc $21$.
          \cadre{rouge}{Attention}{
               Lorsque $0 < \alpha < 1$, $\ln(\alpha)$ est strictement \textbf{négatif}.
               \par
               Il faut donc penser à \textbf{changer le sens de l'inégalité} lorsque l'on divise par $\ln(\alpha)$.
          }
          \par
     \end{enumerate}
\end{corrige}

\end{document}