\documentclass[a4paper]{article}

%================================================================================================================================
%
% Packages
%
%================================================================================================================================

\usepackage[T1]{fontenc} 	% pour caractères accentués
\usepackage[utf8]{inputenc}  % encodage utf8
\usepackage[french]{babel}	% langue : français
\usepackage{fourier}			% caractères plus lisibles
\usepackage[dvipsnames]{xcolor} % couleurs
\usepackage{fancyhdr}		% réglage header footer
\usepackage{needspace}		% empêcher sauts de page mal placés
\usepackage{graphicx}		% pour inclure des graphiques
\usepackage{enumitem,cprotect}		% personnalise les listes d'items (nécessaire pour ol, al ...)
\usepackage{hyperref}		% Liens hypertexte
\usepackage{pstricks,pst-all,pst-node,pstricks-add,pst-math,pst-plot,pst-tree,pst-eucl} % pstricks
\usepackage[a4paper,includeheadfoot,top=2cm,left=3cm, bottom=2cm,right=3cm]{geometry} % marges etc.
\usepackage{comment}			% commentaires multilignes
\usepackage{amsmath,environ} % maths (matrices, etc.)
\usepackage{amssymb,makeidx}
\usepackage{bm}				% bold maths
\usepackage{tabularx}		% tableaux
\usepackage{colortbl}		% tableaux en couleur
\usepackage{fontawesome}		% Fontawesome
\usepackage{environ}			% environment with command
\usepackage{fp}				% calculs pour ps-tricks
\usepackage{multido}			% pour ps tricks
\usepackage[np]{numprint}	% formattage nombre
\usepackage{tikz,tkz-tab} 			% package principal TikZ
\usepackage{pgfplots}   % axes
\usepackage{mathrsfs}    % cursives
\usepackage{calc}			% calcul taille boites
\usepackage[scaled=0.875]{helvet} % font sans serif
\usepackage{svg} % svg
\usepackage{scrextend} % local margin
\usepackage{scratch} %scratch
\usepackage{multicol} % colonnes
%\usepackage{infix-RPN,pst-func} % formule en notation polanaise inversée
\usepackage{listings}

%================================================================================================================================
%
% Réglages de base
%
%================================================================================================================================

\lstset{
language=Python,   % R code
literate=
{á}{{\'a}}1
{à}{{\`a}}1
{ã}{{\~a}}1
{é}{{\'e}}1
{è}{{\`e}}1
{ê}{{\^e}}1
{í}{{\'i}}1
{ó}{{\'o}}1
{õ}{{\~o}}1
{ú}{{\'u}}1
{ü}{{\"u}}1
{ç}{{\c{c}}}1
{~}{{ }}1
}


\definecolor{codegreen}{rgb}{0,0.6,0}
\definecolor{codegray}{rgb}{0.5,0.5,0.5}
\definecolor{codepurple}{rgb}{0.58,0,0.82}
\definecolor{backcolour}{rgb}{0.95,0.95,0.92}

\lstdefinestyle{mystyle}{
    backgroundcolor=\color{backcolour},   
    commentstyle=\color{codegreen},
    keywordstyle=\color{magenta},
    numberstyle=\tiny\color{codegray},
    stringstyle=\color{codepurple},
    basicstyle=\ttfamily\footnotesize,
    breakatwhitespace=false,         
    breaklines=true,                 
    captionpos=b,                    
    keepspaces=true,                 
    numbers=left,                    
xleftmargin=2em,
framexleftmargin=2em,            
    showspaces=false,                
    showstringspaces=false,
    showtabs=false,                  
    tabsize=2,
    upquote=true
}

\lstset{style=mystyle}


\lstset{style=mystyle}
\newcommand{\imgdir}{C:/laragon/www/newmc/assets/imgsvg/}
\newcommand{\imgsvgdir}{C:/laragon/www/newmc/assets/imgsvg/}

\definecolor{mcgris}{RGB}{220, 220, 220}% ancien~; pour compatibilité
\definecolor{mcbleu}{RGB}{52, 152, 219}
\definecolor{mcvert}{RGB}{125, 194, 70}
\definecolor{mcmauve}{RGB}{154, 0, 215}
\definecolor{mcorange}{RGB}{255, 96, 0}
\definecolor{mcturquoise}{RGB}{0, 153, 153}
\definecolor{mcrouge}{RGB}{255, 0, 0}
\definecolor{mclightvert}{RGB}{205, 234, 190}

\definecolor{gris}{RGB}{220, 220, 220}
\definecolor{bleu}{RGB}{52, 152, 219}
\definecolor{vert}{RGB}{125, 194, 70}
\definecolor{mauve}{RGB}{154, 0, 215}
\definecolor{orange}{RGB}{255, 96, 0}
\definecolor{turquoise}{RGB}{0, 153, 153}
\definecolor{rouge}{RGB}{255, 0, 0}
\definecolor{lightvert}{RGB}{205, 234, 190}
\setitemize[0]{label=\color{lightvert}  $\bullet$}

\pagestyle{fancy}
\renewcommand{\headrulewidth}{0.2pt}
\fancyhead[L]{maths-cours.fr}
\fancyhead[R]{\thepage}
\renewcommand{\footrulewidth}{0.2pt}
\fancyfoot[C]{}

\newcolumntype{C}{>{\centering\arraybackslash}X}
\newcolumntype{s}{>{\hsize=.35\hsize\arraybackslash}X}

\setlength{\parindent}{0pt}		 
\setlength{\parskip}{3mm}
\setlength{\headheight}{1cm}

\def\ebook{ebook}
\def\book{book}
\def\web{web}
\def\type{web}

\newcommand{\vect}[1]{\overrightarrow{\,\mathstrut#1\,}}

\def\Oij{$\left(\text{O}~;~\vect{\imath},~\vect{\jmath}\right)$}
\def\Oijk{$\left(\text{O}~;~\vect{\imath},~\vect{\jmath},~\vect{k}\right)$}
\def\Ouv{$\left(\text{O}~;~\vect{u},~\vect{v}\right)$}

\hypersetup{breaklinks=true, colorlinks = true, linkcolor = OliveGreen, urlcolor = OliveGreen, citecolor = OliveGreen, pdfauthor={Didier BONNEL - https://www.maths-cours.fr} } % supprime les bordures autour des liens

\renewcommand{\arg}[0]{\text{arg}}

\everymath{\displaystyle}

%================================================================================================================================
%
% Macros - Commandes
%
%================================================================================================================================

\newcommand\meta[2]{    			% Utilisé pour créer le post HTML.
	\def\titre{titre}
	\def\url{url}
	\def\arg{#1}
	\ifx\titre\arg
		\newcommand\maintitle{#2}
		\fancyhead[L]{#2}
		{\Large\sffamily \MakeUppercase{#2}}
		\vspace{1mm}\textcolor{mcvert}{\hrule}
	\fi 
	\ifx\url\arg
		\fancyfoot[L]{\href{https://www.maths-cours.fr#2}{\black \footnotesize{https://www.maths-cours.fr#2}}}
	\fi 
}


\newcommand\TitreC[1]{    		% Titre centré
     \needspace{3\baselineskip}
     \begin{center}\textbf{#1}\end{center}
}

\newcommand\newpar{    		% paragraphe
     \par
}

\newcommand\nosp {    		% commande vide (pas d'espace)
}
\newcommand{\id}[1]{} %ignore

\newcommand\boite[2]{				% Boite simple sans titre
	\vspace{5mm}
	\setlength{\fboxrule}{0.2mm}
	\setlength{\fboxsep}{5mm}	
	\fcolorbox{#1}{#1!3}{\makebox[\linewidth-2\fboxrule-2\fboxsep]{
  		\begin{minipage}[t]{\linewidth-2\fboxrule-4\fboxsep}\setlength{\parskip}{3mm}
  			 #2
  		\end{minipage}
	}}
	\vspace{5mm}
}

\newcommand\CBox[4]{				% Boites
	\vspace{5mm}
	\setlength{\fboxrule}{0.2mm}
	\setlength{\fboxsep}{5mm}
	
	\fcolorbox{#1}{#1!3}{\makebox[\linewidth-2\fboxrule-2\fboxsep]{
		\begin{minipage}[t]{1cm}\setlength{\parskip}{3mm}
	  		\textcolor{#1}{\LARGE{#2}}    
 	 	\end{minipage}  
  		\begin{minipage}[t]{\linewidth-2\fboxrule-4\fboxsep}\setlength{\parskip}{3mm}
			\raisebox{1.2mm}{\normalsize\sffamily{\textcolor{#1}{#3}}}						
  			 #4
  		\end{minipage}
	}}
	\vspace{5mm}
}

\newcommand\cadre[3]{				% Boites convertible html
	\par
	\vspace{2mm}
	\setlength{\fboxrule}{0.1mm}
	\setlength{\fboxsep}{5mm}
	\fcolorbox{#1}{white}{\makebox[\linewidth-2\fboxrule-2\fboxsep]{
  		\begin{minipage}[t]{\linewidth-2\fboxrule-4\fboxsep}\setlength{\parskip}{3mm}
			\raisebox{-2.5mm}{\sffamily \small{\textcolor{#1}{\MakeUppercase{#2}}}}		
			\par		
  			 #3
 	 		\end{minipage}
	}}
		\vspace{2mm}
	\par
}

\newcommand\bloc[3]{				% Boites convertible html sans bordure
     \needspace{2\baselineskip}
     {\sffamily \small{\textcolor{#1}{\MakeUppercase{#2}}}}    
		\par		
  			 #3
		\par
}

\newcommand\CHelp[1]{
     \CBox{Plum}{\faInfoCircle}{À RETENIR}{#1}
}

\newcommand\CUp[1]{
     \CBox{NavyBlue}{\faThumbsOUp}{EN PRATIQUE}{#1}
}

\newcommand\CInfo[1]{
     \CBox{Sepia}{\faArrowCircleRight}{REMARQUE}{#1}
}

\newcommand\CRedac[1]{
     \CBox{PineGreen}{\faEdit}{BIEN R\'EDIGER}{#1}
}

\newcommand\CError[1]{
     \CBox{Red}{\faExclamationTriangle}{ATTENTION}{#1}
}

\newcommand\TitreExo[2]{
\needspace{4\baselineskip}
 {\sffamily\large EXERCICE #1\ (\emph{#2 points})}
\vspace{5mm}
}

\newcommand\img[2]{
          \includegraphics[width=#2\paperwidth]{\imgdir#1}
}

\newcommand\imgsvg[2]{
       \begin{center}   \includegraphics[width=#2\paperwidth]{\imgsvgdir#1} \end{center}
}


\newcommand\Lien[2]{
     \href{#1}{#2 \tiny \faExternalLink}
}
\newcommand\mcLien[2]{
     \href{https~://www.maths-cours.fr/#1}{#2 \tiny \faExternalLink}
}

\newcommand{\euro}{\eurologo{}}

%================================================================================================================================
%
% Macros - Environement
%
%================================================================================================================================

\newenvironment{tex}{ %
}
{%
}

\newenvironment{indente}{ %
	\setlength\parindent{10mm}
}

{
	\setlength\parindent{0mm}
}

\newenvironment{corrige}{%
     \needspace{3\baselineskip}
     \medskip
     \textbf{\textsc{Corrigé}}
     \medskip
}
{
}

\newenvironment{extern}{%
     \begin{center}
     }
     {
     \end{center}
}

\NewEnviron{code}{%
	\par
     \boite{gray}{\texttt{%
     \BODY
     }}
     \par
}

\newenvironment{vbloc}{% boite sans cadre empeche saut de page
     \begin{minipage}[t]{\linewidth}
     }
     {
     \end{minipage}
}
\NewEnviron{h2}{%
    \needspace{3\baselineskip}
    \vspace{0.6cm}
	\noindent \MakeUppercase{\sffamily \large \BODY}
	\vspace{1mm}\textcolor{mcgris}{\hrule}\vspace{0.4cm}
	\par
}{}

\NewEnviron{h3}{%
    \needspace{3\baselineskip}
	\vspace{5mm}
	\textsc{\BODY}
	\par
}

\NewEnviron{margeneg}{ %
\begin{addmargin}[-1cm]{0cm}
\BODY
\end{addmargin}
}

\NewEnviron{html}{%
}

\begin{document}
\meta{url}{/exercices/probabilites-bac-blanc-es-l-sujet-3-maths-cours-2018/}
\meta{pid}{10477}
\meta{titre}{Probabilités - Bac blanc ES/L Sujet 3 - Maths-cours 2018}
\meta{type}{exercices}
%
\begin{h2}Exercice 4 (3 points)\end{h2}
\par
\textit{Dans cet exercice, toute trace de recherche, même incomplète, ou d'initiative, même infructueuse, sera prise en compte dans l'évaluation.}
\par


Dans le cadre d'essais cliniques, on souhaite tester l'efficacité d'un nouveau médicament destiné à lutter contre l'excès de cholestérol.
\par
L'expérimentation s'effectue sur un échantillon de patients présentant un excès de cholestérol dans le sang.
\par
Lors de cet essai clinique, 70\% des patients reçoivent le médicament tandis que les 30\% restant reçoivent un placebo (comprimé sans principe actif).
\par
\`A la fin de la période de test, le taux de cholestérol de chaque patient est mesuré et comparé au taux initial.
\par
On observe une baisse significative du taux de cholestérol chez 85\% des personnes ayant pris le médicament tandis que chez les personnes ayant pris le placebo, cette baisse n'est constatée que dans 20\% des cas.
\par
Le laboratoire pharmaceutique ayant réalisé cette étude affirme que \og plus de 90\% des patients chez qui une baisse significative a été constatée avaient pris le médicament \fg{}.
\par
Que pensez-vous de cette affirmation ? \\
Justifier votre réponse.
\begin{corrige}
     \par
     Choisissons un patient au hasard et notons :
     \par
     \begin{itemize}
          \item
          $M$ : l'événement \og le patient a pris le médicament \fg{} ;
          \item
          $\overline{M}$ : l'événement \og le patient a pris le placebo \fg{} ;
          \item
          $B$ : l'événement \og le taux de cholestérol du patient a baissé \fg{} ;
          \item
          $\overline{B}$ : l'événement \og le taux de cholestérol du patient n'a pas baissé \fg{}.
          \par
     \end{itemize}
     \par
     Les données de l'énoncé permettent de construire l'arbre suivant :
     \par
     %:-+-+-+- Engendré par : http://math.et.info.free.fr/TikZ/Arbre/
     \begin{center}
          % Racine à Gauche, développement vers la droite
          \begin{extern}%width="360" alt="Arbre bac blanc"
               \begin{tikzpicture}[xscale=1,yscale=1]
                    % Styles (MODIFIABLES)
                    \tikzstyle{fleche}=[-,>=latex,thick]
                    \tikzstyle{noeud}=[fill=white,circle,draw]
                    \tikzstyle{feuille}=[fill=white,circle,draw]
                    \tikzstyle{etiquette}=[midway,fill=white]
                    % Dimensions (MODIFIABLES)
                    \def\DistanceInterNiveaux{3}
                    \def\DistanceInterFeuilles{2}
                    % Dimensions calculées (NON MODIFIABLES)
                    \def\NiveauA{(0)*\DistanceInterNiveaux}
                    \def\NiveauB{(1.5)*\DistanceInterNiveaux}
                    \def\NiveauC{(2.5)*\DistanceInterNiveaux}
                    \def\InterFeuilles{(-1)*\DistanceInterFeuilles}
                    % Noeuds (MODIFIABLES : Styles et Coefficients d'InterFeuilles)
                    \node[noeud] (R) at ({\NiveauA},{(1.5)*\InterFeuilles}) {$\ $};
                    \node[noeud] (Ra) at ({\NiveauB},{(0.5)*\InterFeuilles}) {$M$};
                    \node[feuille] (Raa) at ({\NiveauC},{(0)*\InterFeuilles}) {$B$};
                    \node[feuille] (Rab) at ({\NiveauC},{(1)*\InterFeuilles}) {$\overline{B}$};
                    \node[noeud] (Rb) at ({\NiveauB},{(2.5)*\InterFeuilles}) {$\overline{M}$};
                    \node[feuille] (Rba) at ({\NiveauC},{(2)*\InterFeuilles}) {$B$};
                    \node[feuille] (Rbb) at ({\NiveauC},{(3)*\InterFeuilles}) {$\overline{B}$};
                    % Arcs (MODIFIABLES : Styles)
                    \draw[fleche] (R)--(Ra) node[etiquette] {$0,7$};
                    \draw[fleche] (Ra)--(Raa) node[etiquette] {$0,85$};
                    \draw[fleche] (Ra)--(Rab) node[etiquette] {$0,15$};
                    \draw[fleche] (R)--(Rb) node[etiquette] {$0,3$};
                    \draw[fleche] (Rb)--(Rba) node[etiquette] {$0,2$};
                    \draw[fleche] (Rb)--(Rbb) node[etiquette] {$0,8$};
               \end{tikzpicture}
          \end{extern}
     \end{center}
     Pour juger la validité de l'affirmation du laboratoire, il faut évaluer la probabilité qu'un patient ait pris le médicament, sachant que son taux de cholestérol a diminué.
     \par
     Il faut calculer $p_B(M)$.
     \par
     D'après la formule des probabilités conditionnelles :
     \par
     $p_B(M)=\dfrac{p(B \cap M)}{p(B)}$.
     \par
     Or :
     \par
     $p(B \cap M) = p(M) \times p_M(B)=0,7 \times 0,85 = 0,595$ ;
     \par
     et, d'après la formule des probabilités totales :
     \par
     $p(B)=p(M) \times p_M(B) + p(\overline{M})  p_{\overline{M}}(B) = 0,7 \times 0,85 +0,3 \times 0,2=0,655$.
     \par
     Par conséquent :
     \par
     $p_B(M)=\dfrac{0,595}{0,655} \approx 0,91 = 91\%$.
     \par
     Cette probabilité est supérieure à 90\% donc \textbf{l'affirmation du laboratoire pharmaceutique est exacte}.
\end{corrige}

\end{document}