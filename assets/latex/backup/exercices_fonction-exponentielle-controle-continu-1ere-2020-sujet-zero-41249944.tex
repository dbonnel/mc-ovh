\documentclass[a4paper]{article}

%================================================================================================================================
%
% Packages
%
%================================================================================================================================

\usepackage[T1]{fontenc} 	% pour caractères accentués
\usepackage[utf8]{inputenc}  % encodage utf8
\usepackage[french]{babel}	% langue : français
\usepackage{fourier}			% caractères plus lisibles
\usepackage[dvipsnames]{xcolor} % couleurs
\usepackage{fancyhdr}		% réglage header footer
\usepackage{needspace}		% empêcher sauts de page mal placés
\usepackage{graphicx}		% pour inclure des graphiques
\usepackage{enumitem,cprotect}		% personnalise les listes d'items (nécessaire pour ol, al ...)
\usepackage{hyperref}		% Liens hypertexte
\usepackage{pstricks,pst-all,pst-node,pstricks-add,pst-math,pst-plot,pst-tree,pst-eucl} % pstricks
\usepackage[a4paper,includeheadfoot,top=2cm,left=3cm, bottom=2cm,right=3cm]{geometry} % marges etc.
\usepackage{comment}			% commentaires multilignes
\usepackage{amsmath,environ} % maths (matrices, etc.)
\usepackage{amssymb,makeidx}
\usepackage{bm}				% bold maths
\usepackage{tabularx}		% tableaux
\usepackage{colortbl}		% tableaux en couleur
\usepackage{fontawesome}		% Fontawesome
\usepackage{environ}			% environment with command
\usepackage{fp}				% calculs pour ps-tricks
\usepackage{multido}			% pour ps tricks
\usepackage[np]{numprint}	% formattage nombre
\usepackage{tikz,tkz-tab} 			% package principal TikZ
\usepackage{pgfplots}   % axes
\usepackage{mathrsfs}    % cursives
\usepackage{calc}			% calcul taille boites
\usepackage[scaled=0.875]{helvet} % font sans serif
\usepackage{svg} % svg
\usepackage{scrextend} % local margin
\usepackage{scratch} %scratch
\usepackage{multicol} % colonnes
%\usepackage{infix-RPN,pst-func} % formule en notation polanaise inversée
\usepackage{listings}

%================================================================================================================================
%
% Réglages de base
%
%================================================================================================================================

\lstset{
language=Python,   % R code
literate=
{á}{{\'a}}1
{à}{{\`a}}1
{ã}{{\~a}}1
{é}{{\'e}}1
{è}{{\`e}}1
{ê}{{\^e}}1
{í}{{\'i}}1
{ó}{{\'o}}1
{õ}{{\~o}}1
{ú}{{\'u}}1
{ü}{{\"u}}1
{ç}{{\c{c}}}1
{~}{{ }}1
}


\definecolor{codegreen}{rgb}{0,0.6,0}
\definecolor{codegray}{rgb}{0.5,0.5,0.5}
\definecolor{codepurple}{rgb}{0.58,0,0.82}
\definecolor{backcolour}{rgb}{0.95,0.95,0.92}

\lstdefinestyle{mystyle}{
    backgroundcolor=\color{backcolour},   
    commentstyle=\color{codegreen},
    keywordstyle=\color{magenta},
    numberstyle=\tiny\color{codegray},
    stringstyle=\color{codepurple},
    basicstyle=\ttfamily\footnotesize,
    breakatwhitespace=false,         
    breaklines=true,                 
    captionpos=b,                    
    keepspaces=true,                 
    numbers=left,                    
xleftmargin=2em,
framexleftmargin=2em,            
    showspaces=false,                
    showstringspaces=false,
    showtabs=false,                  
    tabsize=2,
    upquote=true
}

\lstset{style=mystyle}


\lstset{style=mystyle}
\newcommand{\imgdir}{C:/laragon/www/newmc/assets/imgsvg/}
\newcommand{\imgsvgdir}{C:/laragon/www/newmc/assets/imgsvg/}

\definecolor{mcgris}{RGB}{220, 220, 220}% ancien~; pour compatibilité
\definecolor{mcbleu}{RGB}{52, 152, 219}
\definecolor{mcvert}{RGB}{125, 194, 70}
\definecolor{mcmauve}{RGB}{154, 0, 215}
\definecolor{mcorange}{RGB}{255, 96, 0}
\definecolor{mcturquoise}{RGB}{0, 153, 153}
\definecolor{mcrouge}{RGB}{255, 0, 0}
\definecolor{mclightvert}{RGB}{205, 234, 190}

\definecolor{gris}{RGB}{220, 220, 220}
\definecolor{bleu}{RGB}{52, 152, 219}
\definecolor{vert}{RGB}{125, 194, 70}
\definecolor{mauve}{RGB}{154, 0, 215}
\definecolor{orange}{RGB}{255, 96, 0}
\definecolor{turquoise}{RGB}{0, 153, 153}
\definecolor{rouge}{RGB}{255, 0, 0}
\definecolor{lightvert}{RGB}{205, 234, 190}
\setitemize[0]{label=\color{lightvert}  $\bullet$}

\pagestyle{fancy}
\renewcommand{\headrulewidth}{0.2pt}
\fancyhead[L]{maths-cours.fr}
\fancyhead[R]{\thepage}
\renewcommand{\footrulewidth}{0.2pt}
\fancyfoot[C]{}

\newcolumntype{C}{>{\centering\arraybackslash}X}
\newcolumntype{s}{>{\hsize=.35\hsize\arraybackslash}X}

\setlength{\parindent}{0pt}		 
\setlength{\parskip}{3mm}
\setlength{\headheight}{1cm}

\def\ebook{ebook}
\def\book{book}
\def\web{web}
\def\type{web}

\newcommand{\vect}[1]{\overrightarrow{\,\mathstrut#1\,}}

\def\Oij{$\left(\text{O}~;~\vect{\imath},~\vect{\jmath}\right)$}
\def\Oijk{$\left(\text{O}~;~\vect{\imath},~\vect{\jmath},~\vect{k}\right)$}
\def\Ouv{$\left(\text{O}~;~\vect{u},~\vect{v}\right)$}

\hypersetup{breaklinks=true, colorlinks = true, linkcolor = OliveGreen, urlcolor = OliveGreen, citecolor = OliveGreen, pdfauthor={Didier BONNEL - https://www.maths-cours.fr} } % supprime les bordures autour des liens

\renewcommand{\arg}[0]{\text{arg}}

\everymath{\displaystyle}

%================================================================================================================================
%
% Macros - Commandes
%
%================================================================================================================================

\newcommand\meta[2]{    			% Utilisé pour créer le post HTML.
	\def\titre{titre}
	\def\url{url}
	\def\arg{#1}
	\ifx\titre\arg
		\newcommand\maintitle{#2}
		\fancyhead[L]{#2}
		{\Large\sffamily \MakeUppercase{#2}}
		\vspace{1mm}\textcolor{mcvert}{\hrule}
	\fi 
	\ifx\url\arg
		\fancyfoot[L]{\href{https://www.maths-cours.fr#2}{\black \footnotesize{https://www.maths-cours.fr#2}}}
	\fi 
}


\newcommand\TitreC[1]{    		% Titre centré
     \needspace{3\baselineskip}
     \begin{center}\textbf{#1}\end{center}
}

\newcommand\newpar{    		% paragraphe
     \par
}

\newcommand\nosp {    		% commande vide (pas d'espace)
}
\newcommand{\id}[1]{} %ignore

\newcommand\boite[2]{				% Boite simple sans titre
	\vspace{5mm}
	\setlength{\fboxrule}{0.2mm}
	\setlength{\fboxsep}{5mm}	
	\fcolorbox{#1}{#1!3}{\makebox[\linewidth-2\fboxrule-2\fboxsep]{
  		\begin{minipage}[t]{\linewidth-2\fboxrule-4\fboxsep}\setlength{\parskip}{3mm}
  			 #2
  		\end{minipage}
	}}
	\vspace{5mm}
}

\newcommand\CBox[4]{				% Boites
	\vspace{5mm}
	\setlength{\fboxrule}{0.2mm}
	\setlength{\fboxsep}{5mm}
	
	\fcolorbox{#1}{#1!3}{\makebox[\linewidth-2\fboxrule-2\fboxsep]{
		\begin{minipage}[t]{1cm}\setlength{\parskip}{3mm}
	  		\textcolor{#1}{\LARGE{#2}}    
 	 	\end{minipage}  
  		\begin{minipage}[t]{\linewidth-2\fboxrule-4\fboxsep}\setlength{\parskip}{3mm}
			\raisebox{1.2mm}{\normalsize\sffamily{\textcolor{#1}{#3}}}						
  			 #4
  		\end{minipage}
	}}
	\vspace{5mm}
}

\newcommand\cadre[3]{				% Boites convertible html
	\par
	\vspace{2mm}
	\setlength{\fboxrule}{0.1mm}
	\setlength{\fboxsep}{5mm}
	\fcolorbox{#1}{white}{\makebox[\linewidth-2\fboxrule-2\fboxsep]{
  		\begin{minipage}[t]{\linewidth-2\fboxrule-4\fboxsep}\setlength{\parskip}{3mm}
			\raisebox{-2.5mm}{\sffamily \small{\textcolor{#1}{\MakeUppercase{#2}}}}		
			\par		
  			 #3
 	 		\end{minipage}
	}}
		\vspace{2mm}
	\par
}

\newcommand\bloc[3]{				% Boites convertible html sans bordure
     \needspace{2\baselineskip}
     {\sffamily \small{\textcolor{#1}{\MakeUppercase{#2}}}}    
		\par		
  			 #3
		\par
}

\newcommand\CHelp[1]{
     \CBox{Plum}{\faInfoCircle}{À RETENIR}{#1}
}

\newcommand\CUp[1]{
     \CBox{NavyBlue}{\faThumbsOUp}{EN PRATIQUE}{#1}
}

\newcommand\CInfo[1]{
     \CBox{Sepia}{\faArrowCircleRight}{REMARQUE}{#1}
}

\newcommand\CRedac[1]{
     \CBox{PineGreen}{\faEdit}{BIEN R\'EDIGER}{#1}
}

\newcommand\CError[1]{
     \CBox{Red}{\faExclamationTriangle}{ATTENTION}{#1}
}

\newcommand\TitreExo[2]{
\needspace{4\baselineskip}
 {\sffamily\large EXERCICE #1\ (\emph{#2 points})}
\vspace{5mm}
}

\newcommand\img[2]{
          \includegraphics[width=#2\paperwidth]{\imgdir#1}
}

\newcommand\imgsvg[2]{
       \begin{center}   \includegraphics[width=#2\paperwidth]{\imgsvgdir#1} \end{center}
}


\newcommand\Lien[2]{
     \href{#1}{#2 \tiny \faExternalLink}
}
\newcommand\mcLien[2]{
     \href{https~://www.maths-cours.fr/#1}{#2 \tiny \faExternalLink}
}

\newcommand{\euro}{\eurologo{}}

%================================================================================================================================
%
% Macros - Environement
%
%================================================================================================================================

\newenvironment{tex}{ %
}
{%
}

\newenvironment{indente}{ %
	\setlength\parindent{10mm}
}

{
	\setlength\parindent{0mm}
}

\newenvironment{corrige}{%
     \needspace{3\baselineskip}
     \medskip
     \textbf{\textsc{Corrigé}}
     \medskip
}
{
}

\newenvironment{extern}{%
     \begin{center}
     }
     {
     \end{center}
}

\NewEnviron{code}{%
	\par
     \boite{gray}{\texttt{%
     \BODY
     }}
     \par
}

\newenvironment{vbloc}{% boite sans cadre empeche saut de page
     \begin{minipage}[t]{\linewidth}
     }
     {
     \end{minipage}
}
\NewEnviron{h2}{%
    \needspace{3\baselineskip}
    \vspace{0.6cm}
	\noindent \MakeUppercase{\sffamily \large \BODY}
	\vspace{1mm}\textcolor{mcgris}{\hrule}\vspace{0.4cm}
	\par
}{}

\NewEnviron{h3}{%
    \needspace{3\baselineskip}
	\vspace{5mm}
	\textsc{\BODY}
	\par
}

\NewEnviron{margeneg}{ %
\begin{addmargin}[-1cm]{0cm}
\BODY
\end{addmargin}
}

\NewEnviron{html}{%
}

\begin{document}
\meta{url}{/exercices/fonction-exponentielle-controle-continu-1ere-2020-sujet-zero/}
\meta{pid}{11198}
\meta{titre}{Fonction exponentielle - Contrôle continu 1ère - 2020 - Sujet zéro}
\meta{type}{exercices}
%
\begin{h2}Exercice  2 (5 points)\end{h2}
Une entreprise de menuiserie réalise des découpes dans des plaques rectangulaires de bois.
\newpar
Dans un repère orthonormé d'unité 30 cm ci-dessous, on modélise la forme de la découpe dans la plaque rectangulaire par la courbe $ \mathscr{C}_{ f }$ représentatif de la fonction  $f$ définie sur l'intervalle $[  - 1~;~2 ]$ par~:
\[
f( x )=(  - x+2 )\text{e}^{ x }.   
\]
\begin{center}
\imgsvg{fonction-exponentielle-controle-continu-1ere-2020-sujet-zero}{0.3}% alt="fonction exponentielle controle continu-1ere-2020-sujet-zero" style="width:30rem" class="aligncenter"                     
\end{center}
Le bord supérieur de la plaque rectangulaire est tangent à la courbe $ \mathscr{C}_{ f }$. On nomme $L$ la longueur de la plaque rectangulaire et $ \mathscr{l}$ sa largeur.
\begin{enumerate}
     \item
     On note $f' $ la fonction dérivée de  $f$.
     \begin{enumerate}[label=\alph*.]
          \item
          Montrer que pour tout réel $x$ de l'intervalle $[  - 1~;~2 ]$ , $f' ( x )=(  - x+1 )\text{e}^{ x }. $
          \item
          En déduire le tableau de variations de la fonction $f$ sur $[  - 1~;~2 ].$
     \end{enumerate}
     \item
     La longueur $L$ de la plaque rectangulaire est de 90 cm. Trouver sa largeur $ \mathscr{l}$ exacte en centimètres.
\end{enumerate}
\begin{corrige}
     \begin{enumerate}
          \item
          \begin{enumerate}[label=\alph*.]
               \item
               Pour calculer la dérivée $f'$ de la fonction $f$ on utilise la \mcLien{https://www.maths-cours.fr/cours/fonction-derivee/\#p60}{formule}~:
               \[
               ( uv )' =u' v+uv'
               \]
               où  $u$ et $v$ sont les fonctions définies par~:\\
               \begin{itemize}
                    \item
                    $u( x )= - x+2$
                    \item
                    $v( x )=\text{e}^{ x }$
               \end{itemize}
               On a alors~:
               \begin{itemize}
                    \item
                    $u' ( x )= - 1$
                    \item
                    $v' ( x )=\text{e}^{ x }$
               \end{itemize}
               Par conséquent, pour tout réel $x$ de l'intervalle  $\left[  - 1~;~2\right]$~:
               \newpar
               $f' ( x )= - \text{e}^{ x }+(  - x+2 )\text{e}^{ x }$\\
               $\phantom{f' ( x )}=\text{e}^{ x }\left(  - 1 - x+2 \right)$\\
               $\phantom{f' ( x )}=\left(  - x+1 \right)\text{e}^{ x }.$
               \item
               Pour tout réel $x$,  $\text{e}^{ x }$ est strictement positif~; donc $f' $ est du signe de  $ - x+1$ c'est-à-dire~:
               \begin{itemize}
                    \item
                    $f' $ s'annule pour $x=1$
                    \item
                    $f' $ est strictement positive pour $x < 1$
                    \item
                    $f' $ est strictement négative pour $x > 1. $
               \end{itemize}
               \newpar
               On a par ailleurs :
               \begin{itemize}
                    \item
                    $f(  - 1 )=( 1+2 )\text{e}^{  - 1 }=3\text{e}^{  - 1 }=\frac{ 3 }{ \text{e} }$
                    \item
                    $f( 1 )=(  - 1+2 )\text{e}^{ 1 }=\text{e}$
                    \item
                    $f( 2)=(  - 2 +2)\text{e}^{ 2 }=0$
               \end{itemize}
               \newpar
               On obtient alors le tableau de variation ci-dessous~:\\
               \begin{center}
                    \begin{extern}%width="340" alt="Tableau de variation Contrôle continu"
                         \begin{tikzpicture}[scale=0.875]
                              % Styles
                              \tikzstyle{cadre}=[thin]
                              \tikzstyle{fleche}=[->,>=latex,thin]
                              \tikzstyle{nondefini}=[lightgray]
                              % Dimensions Modifiables
                              \def\Lrg{1.5}
                              \def\HtX{1}
                              \def\HtY{0.5}
                              % Dimensions Calculées
                              \def\lignex{-0.5*\HtX}
                              \def\lignef{-1.5*\HtX}
                              \def\separateur{-0.5*\Lrg}
                              % Largeur du tableau
                              \def\gauche{-1.5*\Lrg}
                              \def\droite{4.5*\Lrg}
                              % Hauteur du tableau
                              \def\haut{0.5*\HtX}
                              \def\bas{-2.5*\HtX-2*\HtY}
                              % Pointillés
                              % Ligne de l'abscisse : x
                              \node at (-1*\Lrg,0) {$x$};
                              \node at (0*\Lrg,0) {$-1$};
                              \node at (2*\Lrg,0) {$1$};
                              \node at (4*\Lrg,0) {$2$};
                              % Ligne de la dérivée : f'(x)
                              \node at (-1*\Lrg,-1*\HtX) {$f'(x)$};
                              \node at (0*\Lrg,-1*\HtX) {$$};
                              \node at (1*\Lrg,-1*\HtX) {$+$};
                              \node at (2*\Lrg,-1*\HtX) {$0$};
                              \draw[gray] (2*\Lrg,\lignex) -- (2*\Lrg,\lignef);
                              \node at (3*\Lrg,-1*\HtX) {$-$};
                              \node at (4*\Lrg,-1*\HtX) {$$};
                              % Ligne de la fonction : f(x)
                              \node  at (-1*\Lrg,{-2*\HtX+(-1)*\HtY}) {$f(x)$};
                              \node (f1) at (0*\Lrg,{-2*\HtX+(-2)*\HtY}) {$3\text{e}^{  - 1 }$};
                              \node (f2) at (2*\Lrg,{-2*\HtX+(-0)*\HtY}) { $\text{e}$ };
                              \draw[gray] (2*\Lrg,\lignef) -- (2*\Lrg,\bas);
                              \node (f3) at (4*\Lrg,{-2*\HtX+(-2)*\HtY}) { $0$ };
                              % Flèches
                              \draw[fleche] (f1) -- (f2);
                              \draw[fleche] (f2) -- (f3);
                              % Encadrement
                              \draw[cadre] (\separateur,\haut) -- (\separateur,\bas);
                              \draw[cadre] (\gauche,\haut) rectangle  (\droite,\bas);
                              \draw[cadre] (\gauche,\lignex) -- (\droite,\lignex);
                              \draw[cadre] (\gauche,\lignef) -- (\droite,\lignef);
                         \end{tikzpicture}
                    \end{extern}
               \end{center}
          \end{enumerate}
          \item
          Le maximum de la fonction  $f$ est $f( 1 )=\text{e}$~;  son minimum est  $f( 2 )=0$. La largeur de la plaque est donc $\text{e}$ unités. L'unité mesurant 30 cm, la largeur de la plaque est donc  $l=30\text{e}$ centimètres (soit environ 81,5 cm mais c'est la valeur exacte qui est demandée…).
     \end{enumerate}
\end{corrige}

\end{document}