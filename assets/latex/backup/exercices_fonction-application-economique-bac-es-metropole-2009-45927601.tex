\documentclass[a4paper]{article}

%================================================================================================================================
%
% Packages
%
%================================================================================================================================

\usepackage[T1]{fontenc} 	% pour caractères accentués
\usepackage[utf8]{inputenc}  % encodage utf8
\usepackage[french]{babel}	% langue : français
\usepackage{fourier}			% caractères plus lisibles
\usepackage[dvipsnames]{xcolor} % couleurs
\usepackage{fancyhdr}		% réglage header footer
\usepackage{needspace}		% empêcher sauts de page mal placés
\usepackage{graphicx}		% pour inclure des graphiques
\usepackage{enumitem,cprotect}		% personnalise les listes d'items (nécessaire pour ol, al ...)
\usepackage{hyperref}		% Liens hypertexte
\usepackage{pstricks,pst-all,pst-node,pstricks-add,pst-math,pst-plot,pst-tree,pst-eucl} % pstricks
\usepackage[a4paper,includeheadfoot,top=2cm,left=3cm, bottom=2cm,right=3cm]{geometry} % marges etc.
\usepackage{comment}			% commentaires multilignes
\usepackage{amsmath,environ} % maths (matrices, etc.)
\usepackage{amssymb,makeidx}
\usepackage{bm}				% bold maths
\usepackage{tabularx}		% tableaux
\usepackage{colortbl}		% tableaux en couleur
\usepackage{fontawesome}		% Fontawesome
\usepackage{environ}			% environment with command
\usepackage{fp}				% calculs pour ps-tricks
\usepackage{multido}			% pour ps tricks
\usepackage[np]{numprint}	% formattage nombre
\usepackage{tikz,tkz-tab} 			% package principal TikZ
\usepackage{pgfplots}   % axes
\usepackage{mathrsfs}    % cursives
\usepackage{calc}			% calcul taille boites
\usepackage[scaled=0.875]{helvet} % font sans serif
\usepackage{svg} % svg
\usepackage{scrextend} % local margin
\usepackage{scratch} %scratch
\usepackage{multicol} % colonnes
%\usepackage{infix-RPN,pst-func} % formule en notation polanaise inversée
\usepackage{listings}

%================================================================================================================================
%
% Réglages de base
%
%================================================================================================================================

\lstset{
language=Python,   % R code
literate=
{á}{{\'a}}1
{à}{{\`a}}1
{ã}{{\~a}}1
{é}{{\'e}}1
{è}{{\`e}}1
{ê}{{\^e}}1
{í}{{\'i}}1
{ó}{{\'o}}1
{õ}{{\~o}}1
{ú}{{\'u}}1
{ü}{{\"u}}1
{ç}{{\c{c}}}1
{~}{{ }}1
}


\definecolor{codegreen}{rgb}{0,0.6,0}
\definecolor{codegray}{rgb}{0.5,0.5,0.5}
\definecolor{codepurple}{rgb}{0.58,0,0.82}
\definecolor{backcolour}{rgb}{0.95,0.95,0.92}

\lstdefinestyle{mystyle}{
    backgroundcolor=\color{backcolour},   
    commentstyle=\color{codegreen},
    keywordstyle=\color{magenta},
    numberstyle=\tiny\color{codegray},
    stringstyle=\color{codepurple},
    basicstyle=\ttfamily\footnotesize,
    breakatwhitespace=false,         
    breaklines=true,                 
    captionpos=b,                    
    keepspaces=true,                 
    numbers=left,                    
xleftmargin=2em,
framexleftmargin=2em,            
    showspaces=false,                
    showstringspaces=false,
    showtabs=false,                  
    tabsize=2,
    upquote=true
}

\lstset{style=mystyle}


\lstset{style=mystyle}
\newcommand{\imgdir}{C:/laragon/www/newmc/assets/imgsvg/}
\newcommand{\imgsvgdir}{C:/laragon/www/newmc/assets/imgsvg/}

\definecolor{mcgris}{RGB}{220, 220, 220}% ancien~; pour compatibilité
\definecolor{mcbleu}{RGB}{52, 152, 219}
\definecolor{mcvert}{RGB}{125, 194, 70}
\definecolor{mcmauve}{RGB}{154, 0, 215}
\definecolor{mcorange}{RGB}{255, 96, 0}
\definecolor{mcturquoise}{RGB}{0, 153, 153}
\definecolor{mcrouge}{RGB}{255, 0, 0}
\definecolor{mclightvert}{RGB}{205, 234, 190}

\definecolor{gris}{RGB}{220, 220, 220}
\definecolor{bleu}{RGB}{52, 152, 219}
\definecolor{vert}{RGB}{125, 194, 70}
\definecolor{mauve}{RGB}{154, 0, 215}
\definecolor{orange}{RGB}{255, 96, 0}
\definecolor{turquoise}{RGB}{0, 153, 153}
\definecolor{rouge}{RGB}{255, 0, 0}
\definecolor{lightvert}{RGB}{205, 234, 190}
\setitemize[0]{label=\color{lightvert}  $\bullet$}

\pagestyle{fancy}
\renewcommand{\headrulewidth}{0.2pt}
\fancyhead[L]{maths-cours.fr}
\fancyhead[R]{\thepage}
\renewcommand{\footrulewidth}{0.2pt}
\fancyfoot[C]{}

\newcolumntype{C}{>{\centering\arraybackslash}X}
\newcolumntype{s}{>{\hsize=.35\hsize\arraybackslash}X}

\setlength{\parindent}{0pt}		 
\setlength{\parskip}{3mm}
\setlength{\headheight}{1cm}

\def\ebook{ebook}
\def\book{book}
\def\web{web}
\def\type{web}

\newcommand{\vect}[1]{\overrightarrow{\,\mathstrut#1\,}}

\def\Oij{$\left(\text{O}~;~\vect{\imath},~\vect{\jmath}\right)$}
\def\Oijk{$\left(\text{O}~;~\vect{\imath},~\vect{\jmath},~\vect{k}\right)$}
\def\Ouv{$\left(\text{O}~;~\vect{u},~\vect{v}\right)$}

\hypersetup{breaklinks=true, colorlinks = true, linkcolor = OliveGreen, urlcolor = OliveGreen, citecolor = OliveGreen, pdfauthor={Didier BONNEL - https://www.maths-cours.fr} } % supprime les bordures autour des liens

\renewcommand{\arg}[0]{\text{arg}}

\everymath{\displaystyle}

%================================================================================================================================
%
% Macros - Commandes
%
%================================================================================================================================

\newcommand\meta[2]{    			% Utilisé pour créer le post HTML.
	\def\titre{titre}
	\def\url{url}
	\def\arg{#1}
	\ifx\titre\arg
		\newcommand\maintitle{#2}
		\fancyhead[L]{#2}
		{\Large\sffamily \MakeUppercase{#2}}
		\vspace{1mm}\textcolor{mcvert}{\hrule}
	\fi 
	\ifx\url\arg
		\fancyfoot[L]{\href{https://www.maths-cours.fr#2}{\black \footnotesize{https://www.maths-cours.fr#2}}}
	\fi 
}


\newcommand\TitreC[1]{    		% Titre centré
     \needspace{3\baselineskip}
     \begin{center}\textbf{#1}\end{center}
}

\newcommand\newpar{    		% paragraphe
     \par
}

\newcommand\nosp {    		% commande vide (pas d'espace)
}
\newcommand{\id}[1]{} %ignore

\newcommand\boite[2]{				% Boite simple sans titre
	\vspace{5mm}
	\setlength{\fboxrule}{0.2mm}
	\setlength{\fboxsep}{5mm}	
	\fcolorbox{#1}{#1!3}{\makebox[\linewidth-2\fboxrule-2\fboxsep]{
  		\begin{minipage}[t]{\linewidth-2\fboxrule-4\fboxsep}\setlength{\parskip}{3mm}
  			 #2
  		\end{minipage}
	}}
	\vspace{5mm}
}

\newcommand\CBox[4]{				% Boites
	\vspace{5mm}
	\setlength{\fboxrule}{0.2mm}
	\setlength{\fboxsep}{5mm}
	
	\fcolorbox{#1}{#1!3}{\makebox[\linewidth-2\fboxrule-2\fboxsep]{
		\begin{minipage}[t]{1cm}\setlength{\parskip}{3mm}
	  		\textcolor{#1}{\LARGE{#2}}    
 	 	\end{minipage}  
  		\begin{minipage}[t]{\linewidth-2\fboxrule-4\fboxsep}\setlength{\parskip}{3mm}
			\raisebox{1.2mm}{\normalsize\sffamily{\textcolor{#1}{#3}}}						
  			 #4
  		\end{minipage}
	}}
	\vspace{5mm}
}

\newcommand\cadre[3]{				% Boites convertible html
	\par
	\vspace{2mm}
	\setlength{\fboxrule}{0.1mm}
	\setlength{\fboxsep}{5mm}
	\fcolorbox{#1}{white}{\makebox[\linewidth-2\fboxrule-2\fboxsep]{
  		\begin{minipage}[t]{\linewidth-2\fboxrule-4\fboxsep}\setlength{\parskip}{3mm}
			\raisebox{-2.5mm}{\sffamily \small{\textcolor{#1}{\MakeUppercase{#2}}}}		
			\par		
  			 #3
 	 		\end{minipage}
	}}
		\vspace{2mm}
	\par
}

\newcommand\bloc[3]{				% Boites convertible html sans bordure
     \needspace{2\baselineskip}
     {\sffamily \small{\textcolor{#1}{\MakeUppercase{#2}}}}    
		\par		
  			 #3
		\par
}

\newcommand\CHelp[1]{
     \CBox{Plum}{\faInfoCircle}{À RETENIR}{#1}
}

\newcommand\CUp[1]{
     \CBox{NavyBlue}{\faThumbsOUp}{EN PRATIQUE}{#1}
}

\newcommand\CInfo[1]{
     \CBox{Sepia}{\faArrowCircleRight}{REMARQUE}{#1}
}

\newcommand\CRedac[1]{
     \CBox{PineGreen}{\faEdit}{BIEN R\'EDIGER}{#1}
}

\newcommand\CError[1]{
     \CBox{Red}{\faExclamationTriangle}{ATTENTION}{#1}
}

\newcommand\TitreExo[2]{
\needspace{4\baselineskip}
 {\sffamily\large EXERCICE #1\ (\emph{#2 points})}
\vspace{5mm}
}

\newcommand\img[2]{
          \includegraphics[width=#2\paperwidth]{\imgdir#1}
}

\newcommand\imgsvg[2]{
       \begin{center}   \includegraphics[width=#2\paperwidth]{\imgsvgdir#1} \end{center}
}


\newcommand\Lien[2]{
     \href{#1}{#2 \tiny \faExternalLink}
}
\newcommand\mcLien[2]{
     \href{https~://www.maths-cours.fr/#1}{#2 \tiny \faExternalLink}
}

\newcommand{\euro}{\eurologo{}}

%================================================================================================================================
%
% Macros - Environement
%
%================================================================================================================================

\newenvironment{tex}{ %
}
{%
}

\newenvironment{indente}{ %
	\setlength\parindent{10mm}
}

{
	\setlength\parindent{0mm}
}

\newenvironment{corrige}{%
     \needspace{3\baselineskip}
     \medskip
     \textbf{\textsc{Corrigé}}
     \medskip
}
{
}

\newenvironment{extern}{%
     \begin{center}
     }
     {
     \end{center}
}

\NewEnviron{code}{%
	\par
     \boite{gray}{\texttt{%
     \BODY
     }}
     \par
}

\newenvironment{vbloc}{% boite sans cadre empeche saut de page
     \begin{minipage}[t]{\linewidth}
     }
     {
     \end{minipage}
}
\NewEnviron{h2}{%
    \needspace{3\baselineskip}
    \vspace{0.6cm}
	\noindent \MakeUppercase{\sffamily \large \BODY}
	\vspace{1mm}\textcolor{mcgris}{\hrule}\vspace{0.4cm}
	\par
}{}

\NewEnviron{h3}{%
    \needspace{3\baselineskip}
	\vspace{5mm}
	\textsc{\BODY}
	\par
}

\NewEnviron{margeneg}{ %
\begin{addmargin}[-1cm]{0cm}
\BODY
\end{addmargin}
}

\NewEnviron{html}{%
}

\begin{document}
\meta{url}{/exercices/fonction-application-economique-bac-es-metropole-2009/}
\meta{pid}{1925}
\meta{titre}{Étude d'une fonction - Application économique - Bac ES Métropole 2009}
\meta{type}{exercices}
%
\begin{h2}Exercice 4\end{h2}
\textit{6 points - Commun à tous les candidats}
\par
\begin{h3}Partie A : Étude d'une fonction\end{h3}
On considère la fonction $f$ définie sur l'intervalle [0,5; 8] par $f\left(x\right) = 20\left(x-1\right)\text{e}^{-0,5x}$.
\par
On note $f^{\prime}$ la fonction dérivée de la fonction $f$ sur l'intervalle [0,5; 8]
\begin{enumerate}
     \item
     \begin{enumerate}
          \item
          Démontrer que pour tout nombre réel $x$ de l'intervalle [0,5; 8]
          \par
          $f^{\prime}\left(x\right) = 10\left(- x + 3\right)\text{e}^{-0,5x}$
          \item
          Étudier le signe de la fonction $f^{\prime}$ sur l'intervalle [0,5; 8] et en déduire le tableau de variations de la fonction $f$.
     \end{enumerate}
     \item
     Construire la courbe représentative $\left(C\right)$ de la fonction $f$ dans le plan muni d'un repère orthogonal (O; veci, vecj).
     \par
     On prendra pour unités graphiques 2cm sur l'axe des abscisses et 1cm, sur l'axe des ordonnées.
     \item
     Justifier que la fonction $F$ définie sur l'intervalle  [0,5; 8] par $F\left(x\right) = \frac{-40\left(x + 1\right)}{\text{e}^{x}}$ est une primitive de la fonction $f$ sur l'intervalle [0,5; 8].
     \item
     Calculer la valeur exacte de l'intégrale  $I  = \int_{1,5}^{5} f\left(x\right) \text{d}x$.
\end{enumerate}
\begin{h3}Partie B : Application économique\end{h3}
Une entreprise produit sur commande des bicyclettes pour des municipalités.
\par
La production mensuelle peut varier de 50 à 800 bicyclettes.
\par
Le bénéfice mensuel réalisé par cette production peut être modélisé par la fonction $f$ de la partie A de la façon suivante :
\par
si, un mois donné, on produit $x$ centaines de bicyclettes, alors $f\left(x\right)$ modélise le bénéfice, exprimé en milliers d' euros, réalisé par l'entreprise ce même mois.
\par
Dans la suite de l'exercice, on utilise ce modèle.
\begin{enumerate}
     \item
     \begin{enumerate}
          \item
          Vérifier que si l'entreprise produit 220 bicyclettes un mois donné, alors elle réalise ce mois-là un bénéfice de 7 989 euros.
          \item
          Déterminer le bénéfice réalisé par une production de 408 bicyclettes un mois donné.
     \end{enumerate}
     \item
     \textit{Pour cette question, toute trace de recherche même non aboutie sera prise en compte}
     Répondre aux questions suivantes en utilisant les résultats de la \textbf{partie A} et le modèle précédent.
     \par
     Justifier chaque réponse.
     \begin{enumerate}
          \item
          Combien, pour un mois donné, l'entreprise doit-elle produire au minimum de bicyclettes pour ne pas travailler à perte ?
          \item
          Combien, pour un mois donné, l'entreprise doit-elle produire de bicyclettes pour réaliser un bénéfice maximum. Préciser alors ce bénéfice à l'euro près.
          \item
          Combien, pour un mois donné, l'entreprise doit-elle produire de bicyclettes pour réaliser un bénéfice supérieur à 8 000 euros ?
     \end{enumerate}
\end{enumerate}
\begin{corrige}
     \begin{h3}Partie A\end{h3}
     \begin{enumerate}
          \item
          \begin{enumerate}
               \item
               $f$ est le produit de deux fonctions dérivables sur [0,5; 8] :
               \par
               $u\left(x\right)=20\left(x-1\right)$
               \par
               $u^{\prime}\left(x\right)=20$
               \par
               $v\left(x\right)=e^{-0,5x}$
               \par
               $v^{\prime}\left(x\right)=-0,5e^{-0,5x}$
               \par
               On a donc :
               \par
               $f^{\prime}\left(x\right)=u^{\prime}\left(x\right)v\left(x\right)+u\left(x\right)v^{\prime}\left(x\right)$
               \par
               $f^{\prime}\left(x\right) = 20e^{-0,5x}+20\left(x-1\right)\times -0,5e^{-0,5x}$
               \par
               $f^{\prime}\left(x\right) = 20e^{-0,5x}\left(1,5-0,5x\right)$
               \par
               $f^{\prime}\left(x\right) = 10e^{-0,5x}\left(3-x\right)$
               \item
               $10e^{-0,5x} > 0$ sur $\left[0,5; 8\right]$ donc $f^{\prime}\left(x\right)$ est du signe de $3-x$
               \par
               On obtient le tableau de variations suivant :
               <img src="/wp-content/uploads/mc-0175.png" alt="" class="aligncenter size-full  img-pc" />
%##
% type=table; width=35; l3=20
%--
% x|   \frac{ 3 }{ 4 }   ~   4  ~   +\infty 
% f'(x)|           :0             +   :0   -     ~ 
% f(x)| -10\text{e}^{ 0,25 }  /  +\infty    \  ~ 
\begin{center}
 \begin{extern}%style="width:35rem" alt="Exercice"
    \resizebox{11cm}{!}{
       \definecolor{dark}{gray}{0.1}
       \definecolor{light}{gray}{0.8}
       \tikzstyle{fleche}=[->,>=latex]
       \begin{tikzpicture}[scale=.8, line width=.5pt, dark]
       \def\width{.15}
       \def\height{.10}
       \draw (0, -10*\height) -- (54*\width, -10*\height);
       \draw (10*\width, 0*\height) -- (10*\width, -10*\height);
       \node (l0c0) at (5*\width,-5*\height) {$ x $};
       \node (l0c1) at (14*\width,-5*\height) {$ \frac{ 3 }{ 4 } $};
       \node (l0c2) at (23*\width,-5*\height) {$ ~ $};
       \node (l0c3) at (32*\width,-5*\height) {$ 4 $};
       \node (l0c4) at (41*\width,-5*\height) {$ ~ $};
       \node (l0c5) at (50*\width,-5*\height) {$ +\infty $};
       \draw (0, -20*\height) -- (54*\width, -20*\height);
       \draw (10*\width, -10*\height) -- (10*\width, -20*\height);
       \node (l1c0) at (5*\width,-15*\height) {$ f'(x) $};
       \draw[light] (14*\width, -10*\height) -- (14*\width, -20*\height);
       \node (l1c1) at (14*\width,-15*\height) {$ 0 $};
       \node (l1c2) at (23*\width,-15*\height) {$ + $};
       \draw[light] (32*\width, -10*\height) -- (32*\width, -20*\height);
       \node (l1c3) at (32*\width,-15*\height) {$ 0 $};
       \node (l1c4) at (41*\width,-15*\height) {$ - $};
       \node (l1c5) at (50*\width,-15*\height) {$ ~ $};
       \draw (0, -40*\height) -- (54*\width, -40*\height);
       \draw (10*\width, -20*\height) -- (10*\width, -40*\height);
       \node (l2c0) at (5*\width,-30*\height) {$ f(x) $};
       \node (l2c1) at (14*\width,-25*\height) {$ +\infty $};
       \node (l2c2) at (23*\width,-30*\height) {$ ~ $};
       \draw[light] (32*\width, -20*\height) -- (32*\width, -40*\height);
       \node (l2c3) at (32*\width,-35*\height) {$ 0 $};
       \node (l2c4) at (41*\width,-30*\height) {$ ~ $};
       \draw[light] (50*\width, -20*\height) -- (50*\width, -40*\height);
       \node (l2c5) at (50*\width,-25*\height) {$ 5 $};
       \draw (0, 0) rectangle (54*\width, -40*\height);
       \draw[fleche] (l2c1) -- (l2c3);
       \draw[fleche] (l2c3) -- (l2c5);
       \end{tikzpicture}
      }
   \end{extern}
\end{center}
%##
\end{enumerate}
          \item
\\
\begin{center}
\imgsvg{bac-es-2009-fct}{0.3}% alt="Étude d'une fonction" style="width:50rem" 
\end{center}
          \item
          $F $est le quotient de deux fonctions u et v dérivables sur [0,5; 8] (et v est non nulle sur [0,5; 8])
          \par
          $u\left(x\right)=-40\left(x+1\right)$
          \par
          $u^{\prime}\left(x\right)=-40$
          \par
          $v\left(x\right)=e^{0,5x}$
          \par
          $v^{\prime}\left(x\right)=0,5e^{0,5x}$
          \par
          Donc :
          \par
          $F^{\prime}\left(x\right)=\frac{u^{\prime}\left(x\right)v\left(x\right)-u\left(x\right)v^{\prime}\left(x\right)}{v\left(x\right)^{2}}$
          \par
          $F^{\prime}\left(x\right) = \frac{-40e^{0,5x}+40\left(x+1\right)\times 0,5e^{0,5x}}{\left(e^{0,5x}\right)^{2}}$
          \par
          $F^{\prime}\left(x\right) = \frac{20\left(x-1\right)e^{0,5x}}{e^{x}}$
          \par
          $F^{\prime}\left(x\right) = 20\left(x-1\right)e^{-0,5x}=f\left(x\right)$
          \par
          Donc $F$ est une primitive de $f$ sur [0,5; 8].
          \item
          $I=\int_{1,5}^{5}f\left(x\right)dx=F\left(5\right)-F\left(1,5\right)=-240e^{-2,5x}+100e^{-0,75}$
     \end{enumerate}
     \begin{h3}Partie B\end{h3}
     \begin{enumerate}
          \item
          \begin{enumerate}
               \item
               $f\left(2,2\right)=20\times \left(2,2-1\right)e^{-0,5\times 2,2}=24e^{-1,1}\approx 7,989$
               \par
               Le bénéfice réalisé par la production de 220 bicyclettes est 7 989€
               \item
               $f\left(4,08\right)=61,6e^{-2.04}\approx 8,01$
               \par
               Le bénéfice réalisé par la production de 408 bicyclettes est 8 010€
          \end{enumerate}
          \item
          \begin{enumerate}
               \item
               L'entreprise fait des bénéfices si et seulement si $f\left(x\right) > 0$.
               \par
               Or $20e^{-0,5x} > 0$ sur [0,5; 8] donc $f\left(x\right)$ est du signe de $x-1$ et $f\left(x\right) > 0 \Leftrightarrow  x > 1$
               \par
               L'entreprise doit produire au moins 100 bicyclettes par mois pour réaliser des bénéfices.
               \item
               D'après la partie A, la fonction $f$ atteint son maximum pour $x = 3$
               \par
               L'entreprise doit donc produire 300 bicyclettes pour réaliser un bénéfice maximum de $1000\times f\left(3\right)\approx 8 925$€
               \item
               On sait d'après la question 1. que $f\left(2,2\right) < 8$ et $f\left(4,08\right) > 8$.
               \par
               On vérifie à la calculatrice que $f\left(2,21\right) > 8$ et $f\left(4,09\right) < 8$
               <img src="/wp-content/uploads/bac-es-2009-fct2.png" alt="" class="aligncenter size-full  img-pc" />
               Graphiquement ou d'après le tableau de variation de $f$ on en déduit que l'entreprise doit produire entre 221 et 408 bicyclettes pour réaliser un bénéfice supérieur à 8 000€
          \end{enumerate}
     \end{enumerate}
}\end{corrige}

\end{document}