\documentclass[a4paper]{article}

%================================================================================================================================
%
% Packages
%
%================================================================================================================================

\usepackage[T1]{fontenc} 	% pour caractères accentués
\usepackage[utf8]{inputenc}  % encodage utf8
\usepackage[french]{babel}	% langue : français
\usepackage{fourier}			% caractères plus lisibles
\usepackage[dvipsnames]{xcolor} % couleurs
\usepackage{fancyhdr}		% réglage header footer
\usepackage{needspace}		% empêcher sauts de page mal placés
\usepackage{graphicx}		% pour inclure des graphiques
\usepackage{enumitem,cprotect}		% personnalise les listes d'items (nécessaire pour ol, al ...)
\usepackage{hyperref}		% Liens hypertexte
\usepackage{pstricks,pst-all,pst-node,pstricks-add,pst-math,pst-plot,pst-tree,pst-eucl} % pstricks
\usepackage[a4paper,includeheadfoot,top=2cm,left=3cm, bottom=2cm,right=3cm]{geometry} % marges etc.
\usepackage{comment}			% commentaires multilignes
\usepackage{amsmath,environ} % maths (matrices, etc.)
\usepackage{amssymb,makeidx}
\usepackage{bm}				% bold maths
\usepackage{tabularx}		% tableaux
\usepackage{colortbl}		% tableaux en couleur
\usepackage{fontawesome}		% Fontawesome
\usepackage{environ}			% environment with command
\usepackage{fp}				% calculs pour ps-tricks
\usepackage{multido}			% pour ps tricks
\usepackage[np]{numprint}	% formattage nombre
\usepackage{tikz,tkz-tab} 			% package principal TikZ
\usepackage{pgfplots}   % axes
\usepackage{mathrsfs}    % cursives
\usepackage{calc}			% calcul taille boites
\usepackage[scaled=0.875]{helvet} % font sans serif
\usepackage{svg} % svg
\usepackage{scrextend} % local margin
\usepackage{scratch} %scratch
\usepackage{multicol} % colonnes
%\usepackage{infix-RPN,pst-func} % formule en notation polanaise inversée
\usepackage{listings}

%================================================================================================================================
%
% Réglages de base
%
%================================================================================================================================

\lstset{
language=Python,   % R code
literate=
{á}{{\'a}}1
{à}{{\`a}}1
{ã}{{\~a}}1
{é}{{\'e}}1
{è}{{\`e}}1
{ê}{{\^e}}1
{í}{{\'i}}1
{ó}{{\'o}}1
{õ}{{\~o}}1
{ú}{{\'u}}1
{ü}{{\"u}}1
{ç}{{\c{c}}}1
{~}{{ }}1
}


\definecolor{codegreen}{rgb}{0,0.6,0}
\definecolor{codegray}{rgb}{0.5,0.5,0.5}
\definecolor{codepurple}{rgb}{0.58,0,0.82}
\definecolor{backcolour}{rgb}{0.95,0.95,0.92}

\lstdefinestyle{mystyle}{
    backgroundcolor=\color{backcolour},   
    commentstyle=\color{codegreen},
    keywordstyle=\color{magenta},
    numberstyle=\tiny\color{codegray},
    stringstyle=\color{codepurple},
    basicstyle=\ttfamily\footnotesize,
    breakatwhitespace=false,         
    breaklines=true,                 
    captionpos=b,                    
    keepspaces=true,                 
    numbers=left,                    
xleftmargin=2em,
framexleftmargin=2em,            
    showspaces=false,                
    showstringspaces=false,
    showtabs=false,                  
    tabsize=2,
    upquote=true
}

\lstset{style=mystyle}


\lstset{style=mystyle}
\newcommand{\imgdir}{C:/laragon/www/newmc/assets/imgsvg/}
\newcommand{\imgsvgdir}{C:/laragon/www/newmc/assets/imgsvg/}

\definecolor{mcgris}{RGB}{220, 220, 220}% ancien~; pour compatibilité
\definecolor{mcbleu}{RGB}{52, 152, 219}
\definecolor{mcvert}{RGB}{125, 194, 70}
\definecolor{mcmauve}{RGB}{154, 0, 215}
\definecolor{mcorange}{RGB}{255, 96, 0}
\definecolor{mcturquoise}{RGB}{0, 153, 153}
\definecolor{mcrouge}{RGB}{255, 0, 0}
\definecolor{mclightvert}{RGB}{205, 234, 190}

\definecolor{gris}{RGB}{220, 220, 220}
\definecolor{bleu}{RGB}{52, 152, 219}
\definecolor{vert}{RGB}{125, 194, 70}
\definecolor{mauve}{RGB}{154, 0, 215}
\definecolor{orange}{RGB}{255, 96, 0}
\definecolor{turquoise}{RGB}{0, 153, 153}
\definecolor{rouge}{RGB}{255, 0, 0}
\definecolor{lightvert}{RGB}{205, 234, 190}
\setitemize[0]{label=\color{lightvert}  $\bullet$}

\pagestyle{fancy}
\renewcommand{\headrulewidth}{0.2pt}
\fancyhead[L]{maths-cours.fr}
\fancyhead[R]{\thepage}
\renewcommand{\footrulewidth}{0.2pt}
\fancyfoot[C]{}

\newcolumntype{C}{>{\centering\arraybackslash}X}
\newcolumntype{s}{>{\hsize=.35\hsize\arraybackslash}X}

\setlength{\parindent}{0pt}		 
\setlength{\parskip}{3mm}
\setlength{\headheight}{1cm}

\def\ebook{ebook}
\def\book{book}
\def\web{web}
\def\type{web}

\newcommand{\vect}[1]{\overrightarrow{\,\mathstrut#1\,}}

\def\Oij{$\left(\text{O}~;~\vect{\imath},~\vect{\jmath}\right)$}
\def\Oijk{$\left(\text{O}~;~\vect{\imath},~\vect{\jmath},~\vect{k}\right)$}
\def\Ouv{$\left(\text{O}~;~\vect{u},~\vect{v}\right)$}

\hypersetup{breaklinks=true, colorlinks = true, linkcolor = OliveGreen, urlcolor = OliveGreen, citecolor = OliveGreen, pdfauthor={Didier BONNEL - https://www.maths-cours.fr} } % supprime les bordures autour des liens

\renewcommand{\arg}[0]{\text{arg}}

\everymath{\displaystyle}

%================================================================================================================================
%
% Macros - Commandes
%
%================================================================================================================================

\newcommand\meta[2]{    			% Utilisé pour créer le post HTML.
	\def\titre{titre}
	\def\url{url}
	\def\arg{#1}
	\ifx\titre\arg
		\newcommand\maintitle{#2}
		\fancyhead[L]{#2}
		{\Large\sffamily \MakeUppercase{#2}}
		\vspace{1mm}\textcolor{mcvert}{\hrule}
	\fi 
	\ifx\url\arg
		\fancyfoot[L]{\href{https://www.maths-cours.fr#2}{\black \footnotesize{https://www.maths-cours.fr#2}}}
	\fi 
}


\newcommand\TitreC[1]{    		% Titre centré
     \needspace{3\baselineskip}
     \begin{center}\textbf{#1}\end{center}
}

\newcommand\newpar{    		% paragraphe
     \par
}

\newcommand\nosp {    		% commande vide (pas d'espace)
}
\newcommand{\id}[1]{} %ignore

\newcommand\boite[2]{				% Boite simple sans titre
	\vspace{5mm}
	\setlength{\fboxrule}{0.2mm}
	\setlength{\fboxsep}{5mm}	
	\fcolorbox{#1}{#1!3}{\makebox[\linewidth-2\fboxrule-2\fboxsep]{
  		\begin{minipage}[t]{\linewidth-2\fboxrule-4\fboxsep}\setlength{\parskip}{3mm}
  			 #2
  		\end{minipage}
	}}
	\vspace{5mm}
}

\newcommand\CBox[4]{				% Boites
	\vspace{5mm}
	\setlength{\fboxrule}{0.2mm}
	\setlength{\fboxsep}{5mm}
	
	\fcolorbox{#1}{#1!3}{\makebox[\linewidth-2\fboxrule-2\fboxsep]{
		\begin{minipage}[t]{1cm}\setlength{\parskip}{3mm}
	  		\textcolor{#1}{\LARGE{#2}}    
 	 	\end{minipage}  
  		\begin{minipage}[t]{\linewidth-2\fboxrule-4\fboxsep}\setlength{\parskip}{3mm}
			\raisebox{1.2mm}{\normalsize\sffamily{\textcolor{#1}{#3}}}						
  			 #4
  		\end{minipage}
	}}
	\vspace{5mm}
}

\newcommand\cadre[3]{				% Boites convertible html
	\par
	\vspace{2mm}
	\setlength{\fboxrule}{0.1mm}
	\setlength{\fboxsep}{5mm}
	\fcolorbox{#1}{white}{\makebox[\linewidth-2\fboxrule-2\fboxsep]{
  		\begin{minipage}[t]{\linewidth-2\fboxrule-4\fboxsep}\setlength{\parskip}{3mm}
			\raisebox{-2.5mm}{\sffamily \small{\textcolor{#1}{\MakeUppercase{#2}}}}		
			\par		
  			 #3
 	 		\end{minipage}
	}}
		\vspace{2mm}
	\par
}

\newcommand\bloc[3]{				% Boites convertible html sans bordure
     \needspace{2\baselineskip}
     {\sffamily \small{\textcolor{#1}{\MakeUppercase{#2}}}}    
		\par		
  			 #3
		\par
}

\newcommand\CHelp[1]{
     \CBox{Plum}{\faInfoCircle}{À RETENIR}{#1}
}

\newcommand\CUp[1]{
     \CBox{NavyBlue}{\faThumbsOUp}{EN PRATIQUE}{#1}
}

\newcommand\CInfo[1]{
     \CBox{Sepia}{\faArrowCircleRight}{REMARQUE}{#1}
}

\newcommand\CRedac[1]{
     \CBox{PineGreen}{\faEdit}{BIEN R\'EDIGER}{#1}
}

\newcommand\CError[1]{
     \CBox{Red}{\faExclamationTriangle}{ATTENTION}{#1}
}

\newcommand\TitreExo[2]{
\needspace{4\baselineskip}
 {\sffamily\large EXERCICE #1\ (\emph{#2 points})}
\vspace{5mm}
}

\newcommand\img[2]{
          \includegraphics[width=#2\paperwidth]{\imgdir#1}
}

\newcommand\imgsvg[2]{
       \begin{center}   \includegraphics[width=#2\paperwidth]{\imgsvgdir#1} \end{center}
}


\newcommand\Lien[2]{
     \href{#1}{#2 \tiny \faExternalLink}
}
\newcommand\mcLien[2]{
     \href{https~://www.maths-cours.fr/#1}{#2 \tiny \faExternalLink}
}

\newcommand{\euro}{\eurologo{}}

%================================================================================================================================
%
% Macros - Environement
%
%================================================================================================================================

\newenvironment{tex}{ %
}
{%
}

\newenvironment{indente}{ %
	\setlength\parindent{10mm}
}

{
	\setlength\parindent{0mm}
}

\newenvironment{corrige}{%
     \needspace{3\baselineskip}
     \medskip
     \textbf{\textsc{Corrigé}}
     \medskip
}
{
}

\newenvironment{extern}{%
     \begin{center}
     }
     {
     \end{center}
}

\NewEnviron{code}{%
	\par
     \boite{gray}{\texttt{%
     \BODY
     }}
     \par
}

\newenvironment{vbloc}{% boite sans cadre empeche saut de page
     \begin{minipage}[t]{\linewidth}
     }
     {
     \end{minipage}
}
\NewEnviron{h2}{%
    \needspace{3\baselineskip}
    \vspace{0.6cm}
	\noindent \MakeUppercase{\sffamily \large \BODY}
	\vspace{1mm}\textcolor{mcgris}{\hrule}\vspace{0.4cm}
	\par
}{}

\NewEnviron{h3}{%
    \needspace{3\baselineskip}
	\vspace{5mm}
	\textsc{\BODY}
	\par
}

\NewEnviron{margeneg}{ %
\begin{addmargin}[-1cm]{0cm}
\BODY
\end{addmargin}
}

\NewEnviron{html}{%
}

\begin{document}
\meta{url}{/exercices/fonctions-bac-es-metropole-2014/}
\meta{pid}{2206}
\meta{titre}{Fonctions - Bac ES/L Métropole 2014}
\meta{type}{exercices}
%
\begin{h2}Exercice 4   (5 points)\end{h2}
\textbf{Commun à tous les candidats}
\par
On injecte à un patient un médicament et on mesure régulièrement, pendant $15$ heures, la concentration, en grammes par litre, de ce médicament dans le sang.
\par
On obtient la courbe fournie ci-dessous:

\begin{center}
\imgsvg{mc-0307}{0.3}% alt="concentration d'un médicament" style="width:50rem"
\end{center}
\begin{h3}A. Étude graphique\end{h3}
Avec la précision permise par le graphique, indiquer :
\begin{enumerate}
     \item
     la concentration à l'instant initial;
     \item
     l'intervalle de temps pendant lequel la concentration est supérieure ou égale à $0,4$ gramme par litre.
     \textit{On fera apparaitre sur le graphique les traits de construction nécessaires.}
\end{enumerate}
\begin{h3}B. Étude théorique :\end{h3}
On admet que la concentration peut être modélisée par la fonction $f$ définie sur l'intervalle [0 ; 15] par :
\begin{center}$f\left(x\right)=\left(x+2\right)e^{- 0,5x},$\end{center}
où $x$ représente le nombre d'heures écoulées depuis l'instant initial et $f\left(x\right)$ la concentration, en grammes par litre, du médicament dans le sang.
\begin{enumerate}
     \item
     On note $f^{\prime}$ la fonction dérivée de la fonction $f$. Justifier que $f^{\prime}\left(x\right) =-0,5xe^{- 0,5x}$ et en déduire le tableau de variation de la fonction $f$ sur $\left[0 ; 15\right]$.
     \item
     Justifier que l'équation $f\left(x\right)=0,1$ admet une unique solution $a$ sur l'intervalle $\left[0; 15\right]$.
     \item
     Déterminer un encadrement de $a$ d'amplitude un dixième.
     \item
     Un logiciel de calcul formel donne le résultat ci-dessous.
     \begin{tabularx}{0.8\linewidth}{|*{3}{>{\centering \arraybackslash }X|}}%class="singleborder" width="600"
          \hline
          	& 1. derivez $\left(\left(x+2\right) \times  exp\left(-0.5 \times  x\right)\right)$
          \\ \hline
& $ \qquad \qquad \exp\left(-0.5x\right)-0.5\times \exp\left(-0.5x\right)\times \left(x+2\right)$
     \\ \hline
     	&2. derivez $\left(exp\left(-0.5\times x\right)-0.5\times exp\left(-0.5\times x\right) \times  \left(x+2\right)\right)$
     \\ \hline
& $ \qquad \qquad -exp\left(-0.5 \times x\right)+0.25 \times exp\left(-0.5\times x\right) \times \left(x+2\right)$
\\ \hline
 & 3.	factorisez $\left(-exp \left(-0.5\times x\right)+0.25\times exp\left(-0.5\times x\right)\times \left(x+2\right) \right)$
\\ \hline
& $ \qquad \qquad \left(0.25\times x-0.5\right)\times exp\left(-0.5\times x\right)$
\\ \hline
\end{tabularx}
\par
En vous appuyant sur ces résultats, étudier la convexité de la fonction $f$ sur l'intervalle $\left[0 ; 15\right]$ et préciser l'abscisse d'un éventuel point d'inflexion.
\end{enumerate}
\begin{h3}C. Interprétation des résultats :\end{h3}
En vous aidant des résultats obtenus, soit dans la partie B, soit par lecture graphique et sans justifier, répondre aux questions ci-dessous.
\begin{enumerate}
     \item
     On estime que le médicament n'est plus actif lorsque la concentration est strictement inférieure à $0,1$ gramme par litre. Pendant combien de temps le médicament est-il actif ?
     \item
     Au bout de combien d'heures la baisse de concentration ralentit-elle ?
\end{enumerate}
\begin{corrige}
     \begin{h3}A. Étude graphique\end{h3}
     \begin{enumerate}
          \item
          Par lecture graphique, la concentration à l'instant initial est 2 grammes par litre.
          \item
          Par lecture graphique, l'intervalle de temps pendant lequel la concentration est supérieure ou égale à $0,4$ gramme par litre est $\left[0;6\right]$.
     \end{enumerate}
     \begin{h3}B. Étude théorique :\end{h3}
     \begin{enumerate}
          \item
          On pose $u\left(x\right)=x+2$ et $v\left(x\right)=e^{- 0,5x},$
          \par
          On a $u^{\prime}\left(x\right)=1$ , $v^{\prime}\left(x\right)=-0,5e^{- 0,5x}$ et :
          \par
          $f^{\prime}\left(x\right)=u^{\prime}\left(x\right)v\left(x\right)+u\left(x\right)v^{\prime}\left(x\right)=e^{- 0,5x}+\left(x+2\right)\times \left(-0,5e^{- 0,5x}\right)$
          \par
          $f^{\prime}\left(x\right)=e^{- 0,5x}-0,5xe^{- 0,5x}-e^{- 0,5x}$
          \par
          $f^{\prime}\left(x\right)\equiv -0,5xe^{- 0,5x}$
          \par
          Sur l'intervalle $\left[0 ; 15\right]$, $-0,5x$ est négatif et  $e^{- 0,5x}$ est positif donc $f^{\prime}\left(x\right)$ est négative.
          \par
          On obtient le tableau de variation ci-dessous :
          <img src="/wp-content/uploads/mc-0311.png" alt="" class="aligncenter size-full  img-pc" />
          avec $f\left(15\right)\approx 0,01$
          \item
          La fonction $f$ est \textbf{continue} et \textbf{strictement décroissante} sur $\left[0; 15\right]$.
          \par
          $0,1$ est compris entre $f\left(15\right)\approx 0,01$ et $f\left(0\right)=2$.
          \par
          Donc, d'après le corollaire du théorème des valeurs intermédiaires, l'équation $f\left(x\right)=0,1$ admet une unique solution sur l'intervalle $\left[0; 15\right]$.
          \item
          A la calculatrice on trouve : $f\left(9,4\right)\approx 0,104$ et $f\left(9,5\right)\approx 0,099$ donc $9,4 < \alpha  < 9,5$
          \item
          Le logiciel de calcul formel montre que :
          \par
          $f^{\prime\prime}\left(x\right) = \left(0,25x-0,5\right) e^{-0,5x}$
          \par
          Comme $e^{- 0,5x}$ est strictement positif $f^{\prime\prime}\left(x\right)$ est du signe de $0,25x-0,5$.
          \par
          ♦ ~ Sur l'intervalle $\left[0; 2\right]$, $f^{\prime\prime}$ est négative ou nulle donc la fonction est \textbf{concave}.
          \par
          ♦ ~ Sur l'intervalle $\left[2; 15\right]$, $f^{\prime\prime}$ est positive ou nulle donc la fonction est \textbf{convexe}.
          \par
          $f^{\prime\prime}$ s'annule et change de signe pour $x=2$ donc le point de la courbe d'abscisse $2$ est un point d'inflexion.
     \end{enumerate}
     \begin{h3}C. Interprétation des résultats :\end{h3}
     \begin{enumerate}
          \item
          D'après la question \textbf{B.2.}, le médicament est actif pendant $\alpha $ heures, c'est à dire environ neuf heures et demi.
          \item
          D'après la question \textbf{B.4.}, la baisse de concentration ralentit au bout de deux heures
     \end{enumerate}
\end{corrige}

\end{document}