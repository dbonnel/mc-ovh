\documentclass[a4paper]{article}

%================================================================================================================================
%
% Packages
%
%================================================================================================================================

\usepackage[T1]{fontenc} 	% pour caractères accentués
\usepackage[utf8]{inputenc}  % encodage utf8
\usepackage[french]{babel}	% langue : français
\usepackage{fourier}			% caractères plus lisibles
\usepackage[dvipsnames]{xcolor} % couleurs
\usepackage{fancyhdr}		% réglage header footer
\usepackage{needspace}		% empêcher sauts de page mal placés
\usepackage{graphicx}		% pour inclure des graphiques
\usepackage{enumitem,cprotect}		% personnalise les listes d'items (nécessaire pour ol, al ...)
\usepackage{hyperref}		% Liens hypertexte
\usepackage{pstricks,pst-all,pst-node,pstricks-add,pst-math,pst-plot,pst-tree,pst-eucl} % pstricks
\usepackage[a4paper,includeheadfoot,top=2cm,left=3cm, bottom=2cm,right=3cm]{geometry} % marges etc.
\usepackage{comment}			% commentaires multilignes
\usepackage{amsmath,environ} % maths (matrices, etc.)
\usepackage{amssymb,makeidx}
\usepackage{bm}				% bold maths
\usepackage{tabularx}		% tableaux
\usepackage{colortbl}		% tableaux en couleur
\usepackage{fontawesome}		% Fontawesome
\usepackage{environ}			% environment with command
\usepackage{fp}				% calculs pour ps-tricks
\usepackage{multido}			% pour ps tricks
\usepackage[np]{numprint}	% formattage nombre
\usepackage{tikz,tkz-tab} 			% package principal TikZ
\usepackage{pgfplots}   % axes
\usepackage{mathrsfs}    % cursives
\usepackage{calc}			% calcul taille boites
\usepackage[scaled=0.875]{helvet} % font sans serif
\usepackage{svg} % svg
\usepackage{scrextend} % local margin
\usepackage{scratch} %scratch
\usepackage{multicol} % colonnes
%\usepackage{infix-RPN,pst-func} % formule en notation polanaise inversée
\usepackage{listings}

%================================================================================================================================
%
% Réglages de base
%
%================================================================================================================================

\lstset{
language=Python,   % R code
literate=
{á}{{\'a}}1
{à}{{\`a}}1
{ã}{{\~a}}1
{é}{{\'e}}1
{è}{{\`e}}1
{ê}{{\^e}}1
{í}{{\'i}}1
{ó}{{\'o}}1
{õ}{{\~o}}1
{ú}{{\'u}}1
{ü}{{\"u}}1
{ç}{{\c{c}}}1
{~}{{ }}1
}


\definecolor{codegreen}{rgb}{0,0.6,0}
\definecolor{codegray}{rgb}{0.5,0.5,0.5}
\definecolor{codepurple}{rgb}{0.58,0,0.82}
\definecolor{backcolour}{rgb}{0.95,0.95,0.92}

\lstdefinestyle{mystyle}{
    backgroundcolor=\color{backcolour},   
    commentstyle=\color{codegreen},
    keywordstyle=\color{magenta},
    numberstyle=\tiny\color{codegray},
    stringstyle=\color{codepurple},
    basicstyle=\ttfamily\footnotesize,
    breakatwhitespace=false,         
    breaklines=true,                 
    captionpos=b,                    
    keepspaces=true,                 
    numbers=left,                    
xleftmargin=2em,
framexleftmargin=2em,            
    showspaces=false,                
    showstringspaces=false,
    showtabs=false,                  
    tabsize=2,
    upquote=true
}

\lstset{style=mystyle}


\lstset{style=mystyle}
\newcommand{\imgdir}{C:/laragon/www/newmc/assets/imgsvg/}
\newcommand{\imgsvgdir}{C:/laragon/www/newmc/assets/imgsvg/}

\definecolor{mcgris}{RGB}{220, 220, 220}% ancien~; pour compatibilité
\definecolor{mcbleu}{RGB}{52, 152, 219}
\definecolor{mcvert}{RGB}{125, 194, 70}
\definecolor{mcmauve}{RGB}{154, 0, 215}
\definecolor{mcorange}{RGB}{255, 96, 0}
\definecolor{mcturquoise}{RGB}{0, 153, 153}
\definecolor{mcrouge}{RGB}{255, 0, 0}
\definecolor{mclightvert}{RGB}{205, 234, 190}

\definecolor{gris}{RGB}{220, 220, 220}
\definecolor{bleu}{RGB}{52, 152, 219}
\definecolor{vert}{RGB}{125, 194, 70}
\definecolor{mauve}{RGB}{154, 0, 215}
\definecolor{orange}{RGB}{255, 96, 0}
\definecolor{turquoise}{RGB}{0, 153, 153}
\definecolor{rouge}{RGB}{255, 0, 0}
\definecolor{lightvert}{RGB}{205, 234, 190}
\setitemize[0]{label=\color{lightvert}  $\bullet$}

\pagestyle{fancy}
\renewcommand{\headrulewidth}{0.2pt}
\fancyhead[L]{maths-cours.fr}
\fancyhead[R]{\thepage}
\renewcommand{\footrulewidth}{0.2pt}
\fancyfoot[C]{}

\newcolumntype{C}{>{\centering\arraybackslash}X}
\newcolumntype{s}{>{\hsize=.35\hsize\arraybackslash}X}

\setlength{\parindent}{0pt}		 
\setlength{\parskip}{3mm}
\setlength{\headheight}{1cm}

\def\ebook{ebook}
\def\book{book}
\def\web{web}
\def\type{web}

\newcommand{\vect}[1]{\overrightarrow{\,\mathstrut#1\,}}

\def\Oij{$\left(\text{O}~;~\vect{\imath},~\vect{\jmath}\right)$}
\def\Oijk{$\left(\text{O}~;~\vect{\imath},~\vect{\jmath},~\vect{k}\right)$}
\def\Ouv{$\left(\text{O}~;~\vect{u},~\vect{v}\right)$}

\hypersetup{breaklinks=true, colorlinks = true, linkcolor = OliveGreen, urlcolor = OliveGreen, citecolor = OliveGreen, pdfauthor={Didier BONNEL - https://www.maths-cours.fr} } % supprime les bordures autour des liens

\renewcommand{\arg}[0]{\text{arg}}

\everymath{\displaystyle}

%================================================================================================================================
%
% Macros - Commandes
%
%================================================================================================================================

\newcommand\meta[2]{    			% Utilisé pour créer le post HTML.
	\def\titre{titre}
	\def\url{url}
	\def\arg{#1}
	\ifx\titre\arg
		\newcommand\maintitle{#2}
		\fancyhead[L]{#2}
		{\Large\sffamily \MakeUppercase{#2}}
		\vspace{1mm}\textcolor{mcvert}{\hrule}
	\fi 
	\ifx\url\arg
		\fancyfoot[L]{\href{https://www.maths-cours.fr#2}{\black \footnotesize{https://www.maths-cours.fr#2}}}
	\fi 
}


\newcommand\TitreC[1]{    		% Titre centré
     \needspace{3\baselineskip}
     \begin{center}\textbf{#1}\end{center}
}

\newcommand\newpar{    		% paragraphe
     \par
}

\newcommand\nosp {    		% commande vide (pas d'espace)
}
\newcommand{\id}[1]{} %ignore

\newcommand\boite[2]{				% Boite simple sans titre
	\vspace{5mm}
	\setlength{\fboxrule}{0.2mm}
	\setlength{\fboxsep}{5mm}	
	\fcolorbox{#1}{#1!3}{\makebox[\linewidth-2\fboxrule-2\fboxsep]{
  		\begin{minipage}[t]{\linewidth-2\fboxrule-4\fboxsep}\setlength{\parskip}{3mm}
  			 #2
  		\end{minipage}
	}}
	\vspace{5mm}
}

\newcommand\CBox[4]{				% Boites
	\vspace{5mm}
	\setlength{\fboxrule}{0.2mm}
	\setlength{\fboxsep}{5mm}
	
	\fcolorbox{#1}{#1!3}{\makebox[\linewidth-2\fboxrule-2\fboxsep]{
		\begin{minipage}[t]{1cm}\setlength{\parskip}{3mm}
	  		\textcolor{#1}{\LARGE{#2}}    
 	 	\end{minipage}  
  		\begin{minipage}[t]{\linewidth-2\fboxrule-4\fboxsep}\setlength{\parskip}{3mm}
			\raisebox{1.2mm}{\normalsize\sffamily{\textcolor{#1}{#3}}}						
  			 #4
  		\end{minipage}
	}}
	\vspace{5mm}
}

\newcommand\cadre[3]{				% Boites convertible html
	\par
	\vspace{2mm}
	\setlength{\fboxrule}{0.1mm}
	\setlength{\fboxsep}{5mm}
	\fcolorbox{#1}{white}{\makebox[\linewidth-2\fboxrule-2\fboxsep]{
  		\begin{minipage}[t]{\linewidth-2\fboxrule-4\fboxsep}\setlength{\parskip}{3mm}
			\raisebox{-2.5mm}{\sffamily \small{\textcolor{#1}{\MakeUppercase{#2}}}}		
			\par		
  			 #3
 	 		\end{minipage}
	}}
		\vspace{2mm}
	\par
}

\newcommand\bloc[3]{				% Boites convertible html sans bordure
     \needspace{2\baselineskip}
     {\sffamily \small{\textcolor{#1}{\MakeUppercase{#2}}}}    
		\par		
  			 #3
		\par
}

\newcommand\CHelp[1]{
     \CBox{Plum}{\faInfoCircle}{À RETENIR}{#1}
}

\newcommand\CUp[1]{
     \CBox{NavyBlue}{\faThumbsOUp}{EN PRATIQUE}{#1}
}

\newcommand\CInfo[1]{
     \CBox{Sepia}{\faArrowCircleRight}{REMARQUE}{#1}
}

\newcommand\CRedac[1]{
     \CBox{PineGreen}{\faEdit}{BIEN R\'EDIGER}{#1}
}

\newcommand\CError[1]{
     \CBox{Red}{\faExclamationTriangle}{ATTENTION}{#1}
}

\newcommand\TitreExo[2]{
\needspace{4\baselineskip}
 {\sffamily\large EXERCICE #1\ (\emph{#2 points})}
\vspace{5mm}
}

\newcommand\img[2]{
          \includegraphics[width=#2\paperwidth]{\imgdir#1}
}

\newcommand\imgsvg[2]{
       \begin{center}   \includegraphics[width=#2\paperwidth]{\imgsvgdir#1} \end{center}
}


\newcommand\Lien[2]{
     \href{#1}{#2 \tiny \faExternalLink}
}
\newcommand\mcLien[2]{
     \href{https~://www.maths-cours.fr/#1}{#2 \tiny \faExternalLink}
}

\newcommand{\euro}{\eurologo{}}

%================================================================================================================================
%
% Macros - Environement
%
%================================================================================================================================

\newenvironment{tex}{ %
}
{%
}

\newenvironment{indente}{ %
	\setlength\parindent{10mm}
}

{
	\setlength\parindent{0mm}
}

\newenvironment{corrige}{%
     \needspace{3\baselineskip}
     \medskip
     \textbf{\textsc{Corrigé}}
     \medskip
}
{
}

\newenvironment{extern}{%
     \begin{center}
     }
     {
     \end{center}
}

\NewEnviron{code}{%
	\par
     \boite{gray}{\texttt{%
     \BODY
     }}
     \par
}

\newenvironment{vbloc}{% boite sans cadre empeche saut de page
     \begin{minipage}[t]{\linewidth}
     }
     {
     \end{minipage}
}
\NewEnviron{h2}{%
    \needspace{3\baselineskip}
    \vspace{0.6cm}
	\noindent \MakeUppercase{\sffamily \large \BODY}
	\vspace{1mm}\textcolor{mcgris}{\hrule}\vspace{0.4cm}
	\par
}{}

\NewEnviron{h3}{%
    \needspace{3\baselineskip}
	\vspace{5mm}
	\textsc{\BODY}
	\par
}

\NewEnviron{margeneg}{ %
\begin{addmargin}[-1cm]{0cm}
\BODY
\end{addmargin}
}

\NewEnviron{html}{%
}

\begin{document}
\meta{url}{/cours/loi-probabilite/}
\meta{pid}{298}
\meta{titre}{Variable aléatoire - Loi de probabilité}
\meta{type}{cours}
\begin{h2}I - Rappels de probabilités\end{h2}
\cadre{bleu}{Définitions}{%id="d10"
     Une expérience \textbf{aléatoire} est une expérience dont le résultat dépend du hasard.
     \par
     Chacun des résultats possibles s'appelle une \textbf{éventualité} (ou une \textbf{issue} ou un \textbf{évènement élémentaire})
     \par
     L'ensemble de tous les résultats possibles d'une expérience aléatoire s'appelle l'\textbf{univers} de l'expérience.
}
\bloc{orange}{Exemple}{%id="e10"
     Par exemple, le lancer d'un dé à six faces est une expérience aléatoire. \textit{"Obtenir un 6 avec le dé"} est une éventualité. L'univers possède 6 éventualités; on peut le représenter par l'ensemble:
     \par
     $\Omega =\left\{1;2;3;4;5;6\right\}$
}
\cadre{bleu}{Définition}{%id="d20"
     Soit une expérience aléatoire ayant comme univers:
     \par
     $\Omega =\left\{x_{1}; x_{2};. . .; x_{n}\right\}$
     \par
     On définit une \textbf{probabilité} sur $\Omega $ en associant, à chaque éventualité $x_{i}$, un réel $p_{i}$ compris entre 0 et 1 tel que la somme de tous les $p_{i}$ soit égale à 1.
}
\bloc{cyan}{Remarques}{%id="r20"
     \begin{itemize}
          \item En pratique, pour définir les probabilités $p_{i}$ on peut effectuer un très grand nombre de fois l'expérience aléatoire. La fréquence des résultats obtenus permet d'obtenir une estimation de la loi de probabilité. Par exemple, si en lançant 1 000 000 de fois un dé, on obtient 166 724 fois la face "6" on considérera que la probabilité d'obtenir un "6" est d'environ $\frac{166\ 724}{1\ 000\ 000} \approx \frac{1}{6}$
          \item A condition de faire certaines hypothèses (par exemple : "\textit{le dé n'est pas truqué}") les théorèmes qui suivent permettent de calculer les lois de probabilité de certaines expériences sans avoir recours aux statistiques. Les statistiques peuvent alors servir à valider les hypothèses que l'on a faites au départ.
     \end{itemize}
}
\cadre{bleu}{Définition et propriété}{%id="d30"
     On dit que l'on est en situation d'\textbf{équiprobabilité} si toutes les éventualités on la même probabilité.
     \par
     Cette probabilité est alors $p=\frac{1}{n}$ où $n$ est le nombre total d'éventualités.
}
\bloc{cyan}{Remarque}{%id="r30"
     Dans les exercices, on considérera qu'il y a équiprobabilité si l'énoncé indique que l'on jette une pièce "\textit{équilibrée}", qu'on lance un dé "\textit{non truqué}", qu'on tire une carte "\textit{au hasard}" , etc.
}
\bloc{orange}{Exemples}{%id="e30"
     \begin{itemize}
          \item Si l'on jette une pièce non truquée, la probabilité d'obtenir \textit{pile} est $p=\frac{1}{2}$
          \item Pour un dé à six faces non truqué, la probabilité d'obtenir une face donnée est $p=\frac{1}{6}$
     \end{itemize}
}
\begin{h2}II - Variables aléatoires\end{h2}
\cadre{bleu}{Définition}{%id="d40"
     On définit une \textbf{variable aléatoire} en associant un nombre réel à chaque éventualité d'une expérience aléatoire.
}
\bloc{orange}{Exemples}{%id="e40"
     \begin{itemize}
          \item On mise 1€ sur le numéro 1 à la roulette. On gagne 35€ (36€ - la mise) si le numéro sort. On perd sa mise (soit 1€) dans les autres cas. On peut définir une variable aléatoire représentant le gain algébrique du joueur. Cette variable aléatoire peut prendre la valeur 35 (en cas de gain) ou -1 (en cas de perte).
          \item On lance 4 fois une pièce de monnaie. On peut définir une variable aléatoire égale au nombre de "\textit{faces}" obtenues.
          \par
          Les valeurs possibles pour cette variable sont : 0; 1; 2; 3 ou 4.
     \end{itemize}
}
\bloc{cyan}{Notations}{%id="n40"
     \begin{itemize}
          \item On note généralement une variable aléatoire à l'aide d'une lettre majuscule (le plus souvent $X$)
          \item Si la variable aléatoire $X$ peut prendre les valeurs $a_{1}, a_{2}, . . . a_{n}$, on note $\left(X=a_{i}\right)$ l'évènement : "$X$ prend la valeur $a_{i}$"
     \end{itemize}
}
\cadre{bleu}{Définition}{%id="d50"
     La loi de probabilité d'une variable aléatoire $X$ associe à chaque valeur $a_{i}$ prise par $X$ la probabilité de l'événement $\left(X = a_{i}\right)$.
     \par
     On la représente généralement sous forme de tableau.
}
\bloc{orange}{Exemples}{%id="e50"
     \begin{itemize}
          \item Si l'on reprend l'exemple de la roulette (ci-dessus) et si on suppose que la probabilité de sortie de chacun des 37 numéros (0 à 36) est égale, la probabilité de gain est de $\frac{1}{37}$ et la probabilité de perte $\frac{36}{37}$.
          \par
          La loi de probabilité est donnée par le tableau suivant :
          \begin{tabularx}{0.8\linewidth}{|*{3}{>{\centering \arraybackslash }X|}}%class="compact" width="600"
               \hline
               $a_{i}$ & $-1$ & $35$\\ \hline
               $p\left(X=a_{i}\right)$ & $\frac{36}{37}$ & $\frac{1}{37}$\\ \hline
          \end{tabularx}
          \item Si on lance 4 fois une pièce de monnaie équilibrée, on montre à l'aide d'un arbre que la variable aléatoire $X$ donnant le nombre de "\textit{faces}" obtenues suit la loi de probabilité donnée par le tableau ci-dessous :
          \begin{center}
               \begin{tabularx}{0.8\linewidth}{|*{6}{>{\centering \arraybackslash }X|}}%class="compact" width="600"
                    \hline
                    $a_{i}$ & $0$ & $1$ & $2$ & $3$ & $4$ \\ \hline
                    $p\left(X=a_{i}\right)$ & $\frac{1}{16}$ & $\frac{1}{4}$ & $\frac{3}{8}$ & $\frac{1}{4}$ & $\frac{1}{16}$  \\ \hline
               \end{tabularx}
          \end{center}
     \end{itemize}
}
\cadre{bleu}{Définition (Espérance mathématique)}{%id="d60"
     Soit $X$ une variable aléatoire qui prend les valeurs $x_{i}$ avec les probabilités $p_{i}=p\left(X=x_{i}\right)$.
     \par
     On appelle \textbf{espérance mathématique} de $X$ le nombre :
     \begin{center}
          $E\left(X\right)= x_{1}\times p_{1}+x_{2}\times p_{2}+. . . +x_{n}\times p_{n} $\nosp$= \sum_{i=1}^{n}p_{i} x_{i}$
     \end{center}
}
\bloc{cyan}{Remarque}{%id="r60"
     Ce nombre peut s'interpréter comme une valeur moyenne de $X$ si l'on répète un grand nombre de fois l'expérience.
}
\bloc{orange}{Exemple}{%id="e60"
     Pour l'exemple de la roulette on a :
     \par
     $E\left(X\right)=-1\times \frac{36}{37}+35\times \frac{1}{37}=-\frac{1}{37}$
     \par
     L'espérance est négative, ce qui signifie qu'en moyenne, le jeu n'est pas favorable au joueur.
}
\cadre{bleu}{Définition (Variance - Ecart-type)}{%id="d70"
     Soit $X$ une variable aléatoire d'espérance mathématique $\overline X$.
     \par
     La \textbf{variance} de la variable aléatoire $X$ est le nombre réel positif :
     \begin{center}
          $V\left(X\right)=E\left(\left(X-\overline X\right)^{2}\right)$
     \end{center}
     \smallskip
     L'\textbf{écart-type} est égal à la racine carrée de la variance :
     \begin{center}
          $\sigma \left(X\right)=\sqrt{V\left(X\right)}$
     \end{center}
}
\bloc{cyan}{Remarque}{%id="r70"
     D'après la définition de la variance, si $X$ les valeurs $x_{i}$ avec les probabilités $p_{i}$ :
     \par
     $V\left(X\right)=\sum_{i=1}^{n}p_{i}\left(x_{i}-\overline X\right)^{2}$
     \par
     En développant les carrés, on montre que la variance peut également s'écrire :
     \par
     $V\left(X\right) = E\left(X^{2}\right)-\overline X^{2} =\left(\sum_{i=1}^{n}p_{i} x_{i}^{2}\right)- \overline X^{2}$
}
\cadre{vert}{Propriétés}{%id="p80"
     Soit $X$ une variable aléatoire qui prend les valeurs $x_{i}$ avec les probabilités $p_{i}$. On note $aX+b$ la variable aléatoire qui prend les valeurs $ax_{i}+b$ avec les mêmes probabilités $p_{i}$.
     \par
     On a alors :
     \begin{itemize}
          \item $E\left(aX+b\right)=aE\left(X\right)+b$
          \item $V\left(aX+b\right)=a^{2}\times V\left(X\right)$
          \item $\sigma \left(aX+b\right)=|a|\times \sigma \left(X\right)$
     \end{itemize}
}
\bloc{orange}{Exemple}{%id="e80"
     Soit $X$ un variable aléatoire qui représente le gain algébrique en euro à un jeu d'argent.
     \begin{itemize}
          \item Si on augmente les gains de 1 euro, l'espérance mathématique augmentera de 1, la variance et l'écart-type ne seront pas modifiés ($a=1 ; b=1$).
          \item Si on double les gains, l'espérance mathématique et l'écart-type seront doublés, la variance sera quadruplée ($a=2 ; b=0$).
     \end{itemize}
}

\end{document}