\documentclass[a4paper]{article}

%================================================================================================================================
%
% Packages
%
%================================================================================================================================

\usepackage[T1]{fontenc} 	% pour caractères accentués
\usepackage[utf8]{inputenc}  % encodage utf8
\usepackage[french]{babel}	% langue : français
\usepackage{fourier}			% caractères plus lisibles
\usepackage[dvipsnames]{xcolor} % couleurs
\usepackage{fancyhdr}		% réglage header footer
\usepackage{needspace}		% empêcher sauts de page mal placés
\usepackage{graphicx}		% pour inclure des graphiques
\usepackage{enumitem,cprotect}		% personnalise les listes d'items (nécessaire pour ol, al ...)
\usepackage{hyperref}		% Liens hypertexte
\usepackage{pstricks,pst-all,pst-node,pstricks-add,pst-math,pst-plot,pst-tree,pst-eucl} % pstricks
\usepackage[a4paper,includeheadfoot,top=2cm,left=3cm, bottom=2cm,right=3cm]{geometry} % marges etc.
\usepackage{comment}			% commentaires multilignes
\usepackage{amsmath,environ} % maths (matrices, etc.)
\usepackage{amssymb,makeidx}
\usepackage{bm}				% bold maths
\usepackage{tabularx}		% tableaux
\usepackage{colortbl}		% tableaux en couleur
\usepackage{fontawesome}		% Fontawesome
\usepackage{environ}			% environment with command
\usepackage{fp}				% calculs pour ps-tricks
\usepackage{multido}			% pour ps tricks
\usepackage[np]{numprint}	% formattage nombre
\usepackage{tikz,tkz-tab} 			% package principal TikZ
\usepackage{pgfplots}   % axes
\usepackage{mathrsfs}    % cursives
\usepackage{calc}			% calcul taille boites
\usepackage[scaled=0.875]{helvet} % font sans serif
\usepackage{svg} % svg
\usepackage{scrextend} % local margin
\usepackage{scratch} %scratch
\usepackage{multicol} % colonnes
%\usepackage{infix-RPN,pst-func} % formule en notation polanaise inversée
\usepackage{listings}

%================================================================================================================================
%
% Réglages de base
%
%================================================================================================================================

\lstset{
language=Python,   % R code
literate=
{á}{{\'a}}1
{à}{{\`a}}1
{ã}{{\~a}}1
{é}{{\'e}}1
{è}{{\`e}}1
{ê}{{\^e}}1
{í}{{\'i}}1
{ó}{{\'o}}1
{õ}{{\~o}}1
{ú}{{\'u}}1
{ü}{{\"u}}1
{ç}{{\c{c}}}1
{~}{{ }}1
}


\definecolor{codegreen}{rgb}{0,0.6,0}
\definecolor{codegray}{rgb}{0.5,0.5,0.5}
\definecolor{codepurple}{rgb}{0.58,0,0.82}
\definecolor{backcolour}{rgb}{0.95,0.95,0.92}

\lstdefinestyle{mystyle}{
    backgroundcolor=\color{backcolour},   
    commentstyle=\color{codegreen},
    keywordstyle=\color{magenta},
    numberstyle=\tiny\color{codegray},
    stringstyle=\color{codepurple},
    basicstyle=\ttfamily\footnotesize,
    breakatwhitespace=false,         
    breaklines=true,                 
    captionpos=b,                    
    keepspaces=true,                 
    numbers=left,                    
xleftmargin=2em,
framexleftmargin=2em,            
    showspaces=false,                
    showstringspaces=false,
    showtabs=false,                  
    tabsize=2,
    upquote=true
}

\lstset{style=mystyle}


\lstset{style=mystyle}
\newcommand{\imgdir}{C:/laragon/www/newmc/assets/imgsvg/}
\newcommand{\imgsvgdir}{C:/laragon/www/newmc/assets/imgsvg/}

\definecolor{mcgris}{RGB}{220, 220, 220}% ancien~; pour compatibilité
\definecolor{mcbleu}{RGB}{52, 152, 219}
\definecolor{mcvert}{RGB}{125, 194, 70}
\definecolor{mcmauve}{RGB}{154, 0, 215}
\definecolor{mcorange}{RGB}{255, 96, 0}
\definecolor{mcturquoise}{RGB}{0, 153, 153}
\definecolor{mcrouge}{RGB}{255, 0, 0}
\definecolor{mclightvert}{RGB}{205, 234, 190}

\definecolor{gris}{RGB}{220, 220, 220}
\definecolor{bleu}{RGB}{52, 152, 219}
\definecolor{vert}{RGB}{125, 194, 70}
\definecolor{mauve}{RGB}{154, 0, 215}
\definecolor{orange}{RGB}{255, 96, 0}
\definecolor{turquoise}{RGB}{0, 153, 153}
\definecolor{rouge}{RGB}{255, 0, 0}
\definecolor{lightvert}{RGB}{205, 234, 190}
\setitemize[0]{label=\color{lightvert}  $\bullet$}

\pagestyle{fancy}
\renewcommand{\headrulewidth}{0.2pt}
\fancyhead[L]{maths-cours.fr}
\fancyhead[R]{\thepage}
\renewcommand{\footrulewidth}{0.2pt}
\fancyfoot[C]{}

\newcolumntype{C}{>{\centering\arraybackslash}X}
\newcolumntype{s}{>{\hsize=.35\hsize\arraybackslash}X}

\setlength{\parindent}{0pt}		 
\setlength{\parskip}{3mm}
\setlength{\headheight}{1cm}

\def\ebook{ebook}
\def\book{book}
\def\web{web}
\def\type{web}

\newcommand{\vect}[1]{\overrightarrow{\,\mathstrut#1\,}}

\def\Oij{$\left(\text{O}~;~\vect{\imath},~\vect{\jmath}\right)$}
\def\Oijk{$\left(\text{O}~;~\vect{\imath},~\vect{\jmath},~\vect{k}\right)$}
\def\Ouv{$\left(\text{O}~;~\vect{u},~\vect{v}\right)$}

\hypersetup{breaklinks=true, colorlinks = true, linkcolor = OliveGreen, urlcolor = OliveGreen, citecolor = OliveGreen, pdfauthor={Didier BONNEL - https://www.maths-cours.fr} } % supprime les bordures autour des liens

\renewcommand{\arg}[0]{\text{arg}}

\everymath{\displaystyle}

%================================================================================================================================
%
% Macros - Commandes
%
%================================================================================================================================

\newcommand\meta[2]{    			% Utilisé pour créer le post HTML.
	\def\titre{titre}
	\def\url{url}
	\def\arg{#1}
	\ifx\titre\arg
		\newcommand\maintitle{#2}
		\fancyhead[L]{#2}
		{\Large\sffamily \MakeUppercase{#2}}
		\vspace{1mm}\textcolor{mcvert}{\hrule}
	\fi 
	\ifx\url\arg
		\fancyfoot[L]{\href{https://www.maths-cours.fr#2}{\black \footnotesize{https://www.maths-cours.fr#2}}}
	\fi 
}


\newcommand\TitreC[1]{    		% Titre centré
     \needspace{3\baselineskip}
     \begin{center}\textbf{#1}\end{center}
}

\newcommand\newpar{    		% paragraphe
     \par
}

\newcommand\nosp {    		% commande vide (pas d'espace)
}
\newcommand{\id}[1]{} %ignore

\newcommand\boite[2]{				% Boite simple sans titre
	\vspace{5mm}
	\setlength{\fboxrule}{0.2mm}
	\setlength{\fboxsep}{5mm}	
	\fcolorbox{#1}{#1!3}{\makebox[\linewidth-2\fboxrule-2\fboxsep]{
  		\begin{minipage}[t]{\linewidth-2\fboxrule-4\fboxsep}\setlength{\parskip}{3mm}
  			 #2
  		\end{minipage}
	}}
	\vspace{5mm}
}

\newcommand\CBox[4]{				% Boites
	\vspace{5mm}
	\setlength{\fboxrule}{0.2mm}
	\setlength{\fboxsep}{5mm}
	
	\fcolorbox{#1}{#1!3}{\makebox[\linewidth-2\fboxrule-2\fboxsep]{
		\begin{minipage}[t]{1cm}\setlength{\parskip}{3mm}
	  		\textcolor{#1}{\LARGE{#2}}    
 	 	\end{minipage}  
  		\begin{minipage}[t]{\linewidth-2\fboxrule-4\fboxsep}\setlength{\parskip}{3mm}
			\raisebox{1.2mm}{\normalsize\sffamily{\textcolor{#1}{#3}}}						
  			 #4
  		\end{minipage}
	}}
	\vspace{5mm}
}

\newcommand\cadre[3]{				% Boites convertible html
	\par
	\vspace{2mm}
	\setlength{\fboxrule}{0.1mm}
	\setlength{\fboxsep}{5mm}
	\fcolorbox{#1}{white}{\makebox[\linewidth-2\fboxrule-2\fboxsep]{
  		\begin{minipage}[t]{\linewidth-2\fboxrule-4\fboxsep}\setlength{\parskip}{3mm}
			\raisebox{-2.5mm}{\sffamily \small{\textcolor{#1}{\MakeUppercase{#2}}}}		
			\par		
  			 #3
 	 		\end{minipage}
	}}
		\vspace{2mm}
	\par
}

\newcommand\bloc[3]{				% Boites convertible html sans bordure
     \needspace{2\baselineskip}
     {\sffamily \small{\textcolor{#1}{\MakeUppercase{#2}}}}    
		\par		
  			 #3
		\par
}

\newcommand\CHelp[1]{
     \CBox{Plum}{\faInfoCircle}{À RETENIR}{#1}
}

\newcommand\CUp[1]{
     \CBox{NavyBlue}{\faThumbsOUp}{EN PRATIQUE}{#1}
}

\newcommand\CInfo[1]{
     \CBox{Sepia}{\faArrowCircleRight}{REMARQUE}{#1}
}

\newcommand\CRedac[1]{
     \CBox{PineGreen}{\faEdit}{BIEN R\'EDIGER}{#1}
}

\newcommand\CError[1]{
     \CBox{Red}{\faExclamationTriangle}{ATTENTION}{#1}
}

\newcommand\TitreExo[2]{
\needspace{4\baselineskip}
 {\sffamily\large EXERCICE #1\ (\emph{#2 points})}
\vspace{5mm}
}

\newcommand\img[2]{
          \includegraphics[width=#2\paperwidth]{\imgdir#1}
}

\newcommand\imgsvg[2]{
       \begin{center}   \includegraphics[width=#2\paperwidth]{\imgsvgdir#1} \end{center}
}


\newcommand\Lien[2]{
     \href{#1}{#2 \tiny \faExternalLink}
}
\newcommand\mcLien[2]{
     \href{https~://www.maths-cours.fr/#1}{#2 \tiny \faExternalLink}
}

\newcommand{\euro}{\eurologo{}}

%================================================================================================================================
%
% Macros - Environement
%
%================================================================================================================================

\newenvironment{tex}{ %
}
{%
}

\newenvironment{indente}{ %
	\setlength\parindent{10mm}
}

{
	\setlength\parindent{0mm}
}

\newenvironment{corrige}{%
     \needspace{3\baselineskip}
     \medskip
     \textbf{\textsc{Corrigé}}
     \medskip
}
{
}

\newenvironment{extern}{%
     \begin{center}
     }
     {
     \end{center}
}

\NewEnviron{code}{%
	\par
     \boite{gray}{\texttt{%
     \BODY
     }}
     \par
}

\newenvironment{vbloc}{% boite sans cadre empeche saut de page
     \begin{minipage}[t]{\linewidth}
     }
     {
     \end{minipage}
}
\NewEnviron{h2}{%
    \needspace{3\baselineskip}
    \vspace{0.6cm}
	\noindent \MakeUppercase{\sffamily \large \BODY}
	\vspace{1mm}\textcolor{mcgris}{\hrule}\vspace{0.4cm}
	\par
}{}

\NewEnviron{h3}{%
    \needspace{3\baselineskip}
	\vspace{5mm}
	\textsc{\BODY}
	\par
}

\NewEnviron{margeneg}{ %
\begin{addmargin}[-1cm]{0cm}
\BODY
\end{addmargin}
}

\NewEnviron{html}{%
}

\begin{document}
\meta{url}{/exercices/suites-integrales-bac-s-pondichery-2016/}
\meta{pid}{4085}
\meta{titre}{Suites - Intégrales – Bac S Pondichéry 2016}
\meta{type}{exercices}
%
\begin{h2}Exercice 5 - 3 points\end{h2} 
\par
\textbf{Commun à tous les candidats}
\par
On souhaite stériliser une boîte de conserve.
\par
Pour cela, on la prend à la température ambiante $T_0 = 25$°C et on la place dans un four à température constante $T_F = 100$°C.
\par
La stérilisation débute dès lors que la température de la boîte est supérieure à $85$°C.
\par
\textit{Les deux parties de cet exercice sont indépendantes}
\par
\begin{h3}Partie A : Modélisation discrète\end{h3}
Pour $n$ entier naturel, on note $T_n$ la température en degré Celsius de la boîte au bout de $n$ minutes. On a donc $T_0 = 25$.
\par
Pour $n$ non nul, la valeur $T_n$ est calculée puis affichée par l'algorithme suivant :
     \begin{tabularx}{0.8\linewidth}{|*{3}{>{\centering \arraybackslash }X|}}%class="singleborder" width="600"
          \hline

          \textbf{Initialisation}  &  $T$ prend la valeur $25$
          \hline
               \textbf{Traitement	}  & Demander la valeur de $n$
          \hline
       &  Pour $i$ allant de $1$ à $n$ faire
          \hline
& $  \quad T$ prend la valeur $0,85 \times T + 15$
          \hline
& Fin Pour
\textbf{Sortie }: &Afficher $T$
          \\ \hline
     \end{tabularx}

\begin{enumerate}
     \item
     Déterminer la température de la boîte de conserve au bout de 3 minutes.
     \par
     Arrondir à l'unité.
     \item
     Démontrer que, pour tout entier naturel $n$, on a $T_n = 100-75 \times 0,85^n$.
     \item
     Au bout de combien de minutes la stérilisation débute-elle ?
\end{enumerate}
\begin{h3}Partie B : Modélisation continue\end{h3}
Dans cette partie, $t$ désigne un réel positif.
\par
On suppose désormais qu'à l'instant $t$ (exprimé en minutes), la température de la boîte est donnée par $f(t)$ (exprimée en degré Celsius) avec :
\begin{center}$f(t) = 100-75{e}^{- \frac{\ln 5}{10}t}$.\end{center}
\begin{enumerate}
     \item
     \begin{enumerate}
          \item
          Étudier le sens de variations de $f$ sur $[0~;~+ \infty[$.
          \item
          Justifier que si $t \geqslant 10$ alors $f(t) \geqslant 85$.
     \end{enumerate}
     \item
     Soit $\theta$ un réel supérieur ou égal à $10$.
     \par
     On note $\mathcal{A}(\theta)$ le domaine délimité par les droites d'équation $t = 10$, $t = \theta$, $y = 85$ et la courbe représentative $\mathscr{C}_f$ de $f$.
     \par
     On considère que la stérilisation est finie au bout d'un temps $\theta$, si l'aire, exprimée en unité d'aire du domaine $\mathcal{A}(\theta)$ est supérieure à $80$.
\begin{center}
\imgsvg{suites-integrales-bac-s-pondichery-2016-1}{0.3}% alt="Suites - Intégrales – Bac S Pondichéry 2016 - 1" style="width:60rem"
\end{center}
     \begin{enumerate}
          \item
          Justifier, à l'aide du graphique donné, que l'on a $\mathcal{A}(25) > 80$.
          \item
          Justifier que, pour $\theta \geqslant 10$, on a $\mathcal{A}(\theta) = 15(\theta-10)-75 \int_{10}^{\theta} \text{e}^{- \frac{\ln 5}{10}t}\:\text{d}t$.
          \item
          La stérilisation est-elle finie au bout de $20$ minutes ?
     \end{enumerate}
\end{enumerate}
\begin{corrige}
     \begin{h3}Partie A : Modélisation discrète\end{h3}
     \begin{enumerate}
          \item
          À partir de l'algorithme proposé, on déduit que la suite $(T_n)$ est définie par :
          \par
          $\begin{cases} T_0=25 \\ T_{n+1}=0,85 \times T_n+15  \end{cases}$
          \par
          Par conséquent :
          \par
          $T_1=0,85 \times  25+15=36,25$
          \par
          $T_2=0,85 \times  36,25+15=45,8125$
          \par
          $T_3=0,85 \times  45,8125+15=45,8125 \approx 54$ à une unité près.
          \par
          Au bout de 3 minutes, la température de la boîte de conserve sera d'environ $54$°.
          \item
          Montrons la propriété $(P_n)\ :\ T_n = 100-75 \times 0,85^n$ par récurrence.
          \begin{itemize}
               \item
               \textbf{Initialisation :}
               $100-75 \times 0,85^0=100-75 \times 1=25$
               \par
               On a bien $T_0 = 100-75 \times 0,85^0$ donc la propriété $(P_0)$ est vérifiée.
               \item
               \textbf{Hérédité :}
               Supposons que la propriété $(P_n)$ soit vraie pour un certain rang $n \in \mathbb{N}$.
               \par
               Alors :
               \par
               $T_{n+1}=0,85 \times T_n+15$ (définition de la suite $(T_n)$)
               \par
               $\phantom{T_{n+1}}= 0,85 \times (100-75 \times 0,85^n)+15$ (d'après l'hypothèse de récurrence)
          <div class="note flr-40">$0,85 \times 0,85^n=0,85^{n+1}$\end{corrige}
          $\phantom{T_{n+1}}= 85 -75 \times 0,85 \times 0,85^n+15$
          \par
          $\phantom{T_{n+1}}= 100 -75 \times 0,85^{n+1}$
          \par
          La propriété $(P_{n+1})$ est donc vérifiée.
     \end{itemize}
     Finalement, la propriété $T_n = 100-75 \times 0,85^n$ est démontrée par récurrence.
     \item
     La stérilisation débute lorsque $T_n  > 85$ :
     \par
     $T_n  > 85 \Leftrightarrow 100-75 \times 0,85^n  > 85 $
     \par
     $\phantom{T_n  > 85 }\Leftrightarrow -75 \times 0,85^n  > 85-100 $
     \par
     $\phantom{T_n  > 85 }\Leftrightarrow 0,85^n  <\frac{15}{75}$
     \par
     $\phantom{T_n  > 85 }\Leftrightarrow \ln(0,85^n)  <\ln\frac{1}{5}$ (car la fonction $\ln$ est strictement croissante)
     \par
     $\phantom{T_n  > 85 }\Leftrightarrow n\ln(0,85)  <-\ln 5$
<div class="note flr-40">\textbf{Attention au sens :}$\ln(0,85)$ est négatif~!}
$\phantom{T_n  > 85 }\Leftrightarrow n >-\frac{\ln 5}{\ln(0,85)}$
\par
À la calculatrice $-\frac{\ln 5}{\ln(0,85)} \approx 9,9$.
\par
La stérilisation débutera au bout de $10$ minutes.
\end{enumerate}
\begin{h3}Partie B : Modélisation continue\end{h3}
\begin{enumerate}
     \item
     \begin{enumerate}
          \item
          $f$ est dérivable sur $[0~;~+ \infty[$ comme composée de fonctions dérivables.
          \par
          On calcule la dérivée de la fonction $f$ en utilisant la formule $(e^u)^{\prime}=u^{\prime}e^u$ :
          \par
          $f^{\prime}(t)=-75 \times \left(-\frac{\ln 5}{10} \right){e}^{-\frac{\ln 5}{10}t}$$=7,5\ln5\ {e}^{-\frac{\ln 5}{10}t} $
          \par
          La fonction exponentielle étant à valeurs strictement positives, $f^{\prime}(t)  > 0$ pour tout $t \in [0~;~+ \infty[$ donc la fonction $f$ est strictement croissante sur $[0~;~+ \infty[$.
          \item
          Comme $f$ est strictement croissante sur $[0~;~+ \infty[$ :
          \par
          $t \geqslant 10 \Rightarrow f(t) \geqslant f(10)$
          \par
          Or :
          \par
          $f(10)=100-75{e}^{- \frac{\ln 5}{10} \times 10}$
          \par
          $\phantom{f(10)}=100-75{e}^{-\ln 5}$
          \par
          $\phantom{f(10)}=100-\frac{75}{{e}^{\ln 5}}$
          \par
          $\phantom{f(10)}=100-\frac{75}{5}=85$
          \par
          Donc pour tout $t \in [0~;~+ \infty[$, $f(t) \geqslant 85$.
     \end{enumerate}
     \item
     \begin{enumerate}
          \item
\begin{center}
\imgsvg{suites-integrales-bac-s-pondichery-2016-2}{0.3}% alt="Suites - Intégrales – Bac S Pondichéry 2016 - 2" style="width:60rem"
\end{center}
          L'aire d'un carreau rectangulaire (coloré en bleu ci-dessus) est $5 \times 5 = 25$ u.a.
          \par
          $\mathcal{A}(25)$ est l'aire du domaine (coloré en vert ci-dessus) délimité par les droites d'équation $t = 10$, $t = 25$, $y = 85$ et la courbe représentative $\mathscr{C}_f$ de $f$.
          \par
          Il est visible que cette aire est (assez largement !) supérieure à trois fois et demi l'aire d'un carreau donc $\mathcal{A}(25)  > 3,5 \times 25  >  80$.
          \item
          Pour $t \geqslant 10$, $f(t) \geqslant 85$ donc :
          \par
          $\mathcal{A}(\theta) = \int_{10}^{\theta} f(t)-85\:\text{d}t$
          \par
          $\phantom{\mathcal{A}(\theta)} = \int_{10}^{\theta} 15-75{e}^{- \frac{\ln 5}{10}t}\:\text{d}t$
          \par
          $\phantom{\mathcal{A}(\theta)} = \int_{10}^{\theta} 15\:\text{d}t-75\int_{10}^{\theta}{e}^{-\frac{\ln 5}{10}t}\:\text{d}t$ (linéarité de l'intégrale)
          \par
          $\phantom{\mathcal{A}(\theta)} = \left[15t\right]_{10}^\theta-75\int_{10}^{\theta}{e}^{-\frac{\ln 5}{10}t}\:\text{d}t$
          \par
          $\phantom{\mathcal{A}(\theta)} = 15(\theta-10)-75\int_{10}^{\theta}{e}^{-\frac{\ln 5}{10}t}\:\text{d}t$
          \item
          Une primitive sur $\mathbb{R}$ de la fonction $g : t \longmapsto {e}^{-\frac{\ln 5}{10}t}$ est la fonction $G $ définie par :
          \par
          $G(t)=-\frac{10}{\ln 5}{e}^{-\frac{\ln 5}{10}t}$
          \par
          Par conséquent :
          \par
          $\int_{10}^{20}{e}^{-\frac{\ln 5}{10}t}\:\text{d}t = \left[-\frac{10}{\ln 5}{e}^{-\frac{\ln 5}{10}t}\right]_{10}^{20}$
          \par
          $\phantom{\int_{10}^{20}{e}^{-\frac{\ln 5}{10}t}\:\text{d}t} = -\frac{10}{\ln 5}\left({e}^{-2\ln 5}-{e}^{-\ln 5}\right) $
          \par
          $\phantom{\int_{10}^{20}{e}^{-\frac{\ln 5}{10}t}\:\text{d}t} = -\frac{10}{\ln 5}\left(\frac{1}{25}-\frac{1}{5}\right) $
          \par
          $\phantom{\int_{10}^{20}{e}^{-\frac{\ln 5}{10}t}\:\text{d}t} = \frac{8}{5\ln 5}$
          \par
          et :
          \par
          $\mathcal{A}(20) = 15(20-10)-75 \times \frac{8}{5\ln 5}$
          \par
          $\phantom{\mathcal{A}(20)} = 150-\frac{120}{\ln 5} \approx 75,44$u.a.
          \par
          Au bout de $20$ minutes, $\mathcal{A}(20)\approx 75,44$u.a.; le seuil de $80$u.a. n'a pas été atteint donc la stérilisation n'est pas encore finie.
          \par
          D'après la question \textbf{2.a.}, la stérilisation sera terminée au bout d'un temps compris entre $20$ et $25$ minutes.
     \end{enumerate}
\end{enumerate}
}

\end{document}