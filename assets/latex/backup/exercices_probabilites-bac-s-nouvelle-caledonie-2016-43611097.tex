\documentclass[a4paper]{article}

%================================================================================================================================
%
% Packages
%
%================================================================================================================================

\usepackage[T1]{fontenc} 	% pour caractères accentués
\usepackage[utf8]{inputenc}  % encodage utf8
\usepackage[french]{babel}	% langue : français
\usepackage{fourier}			% caractères plus lisibles
\usepackage[dvipsnames]{xcolor} % couleurs
\usepackage{fancyhdr}		% réglage header footer
\usepackage{needspace}		% empêcher sauts de page mal placés
\usepackage{graphicx}		% pour inclure des graphiques
\usepackage{enumitem,cprotect}		% personnalise les listes d'items (nécessaire pour ol, al ...)
\usepackage{hyperref}		% Liens hypertexte
\usepackage{pstricks,pst-all,pst-node,pstricks-add,pst-math,pst-plot,pst-tree,pst-eucl} % pstricks
\usepackage[a4paper,includeheadfoot,top=2cm,left=3cm, bottom=2cm,right=3cm]{geometry} % marges etc.
\usepackage{comment}			% commentaires multilignes
\usepackage{amsmath,environ} % maths (matrices, etc.)
\usepackage{amssymb,makeidx}
\usepackage{bm}				% bold maths
\usepackage{tabularx}		% tableaux
\usepackage{colortbl}		% tableaux en couleur
\usepackage{fontawesome}		% Fontawesome
\usepackage{environ}			% environment with command
\usepackage{fp}				% calculs pour ps-tricks
\usepackage{multido}			% pour ps tricks
\usepackage[np]{numprint}	% formattage nombre
\usepackage{tikz,tkz-tab} 			% package principal TikZ
\usepackage{pgfplots}   % axes
\usepackage{mathrsfs}    % cursives
\usepackage{calc}			% calcul taille boites
\usepackage[scaled=0.875]{helvet} % font sans serif
\usepackage{svg} % svg
\usepackage{scrextend} % local margin
\usepackage{scratch} %scratch
\usepackage{multicol} % colonnes
%\usepackage{infix-RPN,pst-func} % formule en notation polanaise inversée
\usepackage{listings}

%================================================================================================================================
%
% Réglages de base
%
%================================================================================================================================

\lstset{
language=Python,   % R code
literate=
{á}{{\'a}}1
{à}{{\`a}}1
{ã}{{\~a}}1
{é}{{\'e}}1
{è}{{\`e}}1
{ê}{{\^e}}1
{í}{{\'i}}1
{ó}{{\'o}}1
{õ}{{\~o}}1
{ú}{{\'u}}1
{ü}{{\"u}}1
{ç}{{\c{c}}}1
{~}{{ }}1
}


\definecolor{codegreen}{rgb}{0,0.6,0}
\definecolor{codegray}{rgb}{0.5,0.5,0.5}
\definecolor{codepurple}{rgb}{0.58,0,0.82}
\definecolor{backcolour}{rgb}{0.95,0.95,0.92}

\lstdefinestyle{mystyle}{
    backgroundcolor=\color{backcolour},   
    commentstyle=\color{codegreen},
    keywordstyle=\color{magenta},
    numberstyle=\tiny\color{codegray},
    stringstyle=\color{codepurple},
    basicstyle=\ttfamily\footnotesize,
    breakatwhitespace=false,         
    breaklines=true,                 
    captionpos=b,                    
    keepspaces=true,                 
    numbers=left,                    
xleftmargin=2em,
framexleftmargin=2em,            
    showspaces=false,                
    showstringspaces=false,
    showtabs=false,                  
    tabsize=2,
    upquote=true
}

\lstset{style=mystyle}


\lstset{style=mystyle}
\newcommand{\imgdir}{C:/laragon/www/newmc/assets/imgsvg/}
\newcommand{\imgsvgdir}{C:/laragon/www/newmc/assets/imgsvg/}

\definecolor{mcgris}{RGB}{220, 220, 220}% ancien~; pour compatibilité
\definecolor{mcbleu}{RGB}{52, 152, 219}
\definecolor{mcvert}{RGB}{125, 194, 70}
\definecolor{mcmauve}{RGB}{154, 0, 215}
\definecolor{mcorange}{RGB}{255, 96, 0}
\definecolor{mcturquoise}{RGB}{0, 153, 153}
\definecolor{mcrouge}{RGB}{255, 0, 0}
\definecolor{mclightvert}{RGB}{205, 234, 190}

\definecolor{gris}{RGB}{220, 220, 220}
\definecolor{bleu}{RGB}{52, 152, 219}
\definecolor{vert}{RGB}{125, 194, 70}
\definecolor{mauve}{RGB}{154, 0, 215}
\definecolor{orange}{RGB}{255, 96, 0}
\definecolor{turquoise}{RGB}{0, 153, 153}
\definecolor{rouge}{RGB}{255, 0, 0}
\definecolor{lightvert}{RGB}{205, 234, 190}
\setitemize[0]{label=\color{lightvert}  $\bullet$}

\pagestyle{fancy}
\renewcommand{\headrulewidth}{0.2pt}
\fancyhead[L]{maths-cours.fr}
\fancyhead[R]{\thepage}
\renewcommand{\footrulewidth}{0.2pt}
\fancyfoot[C]{}

\newcolumntype{C}{>{\centering\arraybackslash}X}
\newcolumntype{s}{>{\hsize=.35\hsize\arraybackslash}X}

\setlength{\parindent}{0pt}		 
\setlength{\parskip}{3mm}
\setlength{\headheight}{1cm}

\def\ebook{ebook}
\def\book{book}
\def\web{web}
\def\type{web}

\newcommand{\vect}[1]{\overrightarrow{\,\mathstrut#1\,}}

\def\Oij{$\left(\text{O}~;~\vect{\imath},~\vect{\jmath}\right)$}
\def\Oijk{$\left(\text{O}~;~\vect{\imath},~\vect{\jmath},~\vect{k}\right)$}
\def\Ouv{$\left(\text{O}~;~\vect{u},~\vect{v}\right)$}

\hypersetup{breaklinks=true, colorlinks = true, linkcolor = OliveGreen, urlcolor = OliveGreen, citecolor = OliveGreen, pdfauthor={Didier BONNEL - https://www.maths-cours.fr} } % supprime les bordures autour des liens

\renewcommand{\arg}[0]{\text{arg}}

\everymath{\displaystyle}

%================================================================================================================================
%
% Macros - Commandes
%
%================================================================================================================================

\newcommand\meta[2]{    			% Utilisé pour créer le post HTML.
	\def\titre{titre}
	\def\url{url}
	\def\arg{#1}
	\ifx\titre\arg
		\newcommand\maintitle{#2}
		\fancyhead[L]{#2}
		{\Large\sffamily \MakeUppercase{#2}}
		\vspace{1mm}\textcolor{mcvert}{\hrule}
	\fi 
	\ifx\url\arg
		\fancyfoot[L]{\href{https://www.maths-cours.fr#2}{\black \footnotesize{https://www.maths-cours.fr#2}}}
	\fi 
}


\newcommand\TitreC[1]{    		% Titre centré
     \needspace{3\baselineskip}
     \begin{center}\textbf{#1}\end{center}
}

\newcommand\newpar{    		% paragraphe
     \par
}

\newcommand\nosp {    		% commande vide (pas d'espace)
}
\newcommand{\id}[1]{} %ignore

\newcommand\boite[2]{				% Boite simple sans titre
	\vspace{5mm}
	\setlength{\fboxrule}{0.2mm}
	\setlength{\fboxsep}{5mm}	
	\fcolorbox{#1}{#1!3}{\makebox[\linewidth-2\fboxrule-2\fboxsep]{
  		\begin{minipage}[t]{\linewidth-2\fboxrule-4\fboxsep}\setlength{\parskip}{3mm}
  			 #2
  		\end{minipage}
	}}
	\vspace{5mm}
}

\newcommand\CBox[4]{				% Boites
	\vspace{5mm}
	\setlength{\fboxrule}{0.2mm}
	\setlength{\fboxsep}{5mm}
	
	\fcolorbox{#1}{#1!3}{\makebox[\linewidth-2\fboxrule-2\fboxsep]{
		\begin{minipage}[t]{1cm}\setlength{\parskip}{3mm}
	  		\textcolor{#1}{\LARGE{#2}}    
 	 	\end{minipage}  
  		\begin{minipage}[t]{\linewidth-2\fboxrule-4\fboxsep}\setlength{\parskip}{3mm}
			\raisebox{1.2mm}{\normalsize\sffamily{\textcolor{#1}{#3}}}						
  			 #4
  		\end{minipage}
	}}
	\vspace{5mm}
}

\newcommand\cadre[3]{				% Boites convertible html
	\par
	\vspace{2mm}
	\setlength{\fboxrule}{0.1mm}
	\setlength{\fboxsep}{5mm}
	\fcolorbox{#1}{white}{\makebox[\linewidth-2\fboxrule-2\fboxsep]{
  		\begin{minipage}[t]{\linewidth-2\fboxrule-4\fboxsep}\setlength{\parskip}{3mm}
			\raisebox{-2.5mm}{\sffamily \small{\textcolor{#1}{\MakeUppercase{#2}}}}		
			\par		
  			 #3
 	 		\end{minipage}
	}}
		\vspace{2mm}
	\par
}

\newcommand\bloc[3]{				% Boites convertible html sans bordure
     \needspace{2\baselineskip}
     {\sffamily \small{\textcolor{#1}{\MakeUppercase{#2}}}}    
		\par		
  			 #3
		\par
}

\newcommand\CHelp[1]{
     \CBox{Plum}{\faInfoCircle}{À RETENIR}{#1}
}

\newcommand\CUp[1]{
     \CBox{NavyBlue}{\faThumbsOUp}{EN PRATIQUE}{#1}
}

\newcommand\CInfo[1]{
     \CBox{Sepia}{\faArrowCircleRight}{REMARQUE}{#1}
}

\newcommand\CRedac[1]{
     \CBox{PineGreen}{\faEdit}{BIEN R\'EDIGER}{#1}
}

\newcommand\CError[1]{
     \CBox{Red}{\faExclamationTriangle}{ATTENTION}{#1}
}

\newcommand\TitreExo[2]{
\needspace{4\baselineskip}
 {\sffamily\large EXERCICE #1\ (\emph{#2 points})}
\vspace{5mm}
}

\newcommand\img[2]{
          \includegraphics[width=#2\paperwidth]{\imgdir#1}
}

\newcommand\imgsvg[2]{
       \begin{center}   \includegraphics[width=#2\paperwidth]{\imgsvgdir#1} \end{center}
}


\newcommand\Lien[2]{
     \href{#1}{#2 \tiny \faExternalLink}
}
\newcommand\mcLien[2]{
     \href{https~://www.maths-cours.fr/#1}{#2 \tiny \faExternalLink}
}

\newcommand{\euro}{\eurologo{}}

%================================================================================================================================
%
% Macros - Environement
%
%================================================================================================================================

\newenvironment{tex}{ %
}
{%
}

\newenvironment{indente}{ %
	\setlength\parindent{10mm}
}

{
	\setlength\parindent{0mm}
}

\newenvironment{corrige}{%
     \needspace{3\baselineskip}
     \medskip
     \textbf{\textsc{Corrigé}}
     \medskip
}
{
}

\newenvironment{extern}{%
     \begin{center}
     }
     {
     \end{center}
}

\NewEnviron{code}{%
	\par
     \boite{gray}{\texttt{%
     \BODY
     }}
     \par
}

\newenvironment{vbloc}{% boite sans cadre empeche saut de page
     \begin{minipage}[t]{\linewidth}
     }
     {
     \end{minipage}
}
\NewEnviron{h2}{%
    \needspace{3\baselineskip}
    \vspace{0.6cm}
	\noindent \MakeUppercase{\sffamily \large \BODY}
	\vspace{1mm}\textcolor{mcgris}{\hrule}\vspace{0.4cm}
	\par
}{}

\NewEnviron{h3}{%
    \needspace{3\baselineskip}
	\vspace{5mm}
	\textsc{\BODY}
	\par
}

\NewEnviron{margeneg}{ %
\begin{addmargin}[-1cm]{0cm}
\BODY
\end{addmargin}
}

\NewEnviron{html}{%
}

\begin{document}
\meta{url}{/exercices/probabilites-bac-s-nouvelle-caledonie-2016/}
\meta{pid}{3843}
\meta{titre}{Probabilités – Bac S Nouvelle Calédonie 2016}
\meta{type}{exercices}
%
\begin{h2}Exercice 1 - 6 points\end{h2}
\textbf{Commun à tous les candidats}
\\
\textit{Les parties A et B sont indépendantes}
\par
\begin{h3}Partie A\end{h3}
Une boite contient $200$ médailles souvenir dont $50$ sont argentées, les autres dorées.
\par
Parmi les argentées $60$\% représentent le château de Blois, $30$\% le château de Langeais, les autres le château de Saumur.
\par
Parmi les dorées $40$\% représentent le château de Blois, les autres le château de Langeais.
\par
On tire au hasard une médaille de la boite. Le tirage est considéré équiprobable et on note :
\begin{itemize}
     \item
     $A$ l'événement « la médaille tirée est argentée » ;
     \item
     $D$ l'événement « la médaille tirée est dorée » ;
     \item
     $B$ l'événement « la médaille tirée représente le château de Blois » ;
     \item
     $L$ l'événement « la médaille tirée représente le château de Langeais » ;
     \item
$S$ l'événement « la médaille tirée représente le château de Saumur ».\end{itemize}
\begin{enumerate}
     \item
     Dans cette question, on donnera les résultats sous la forme d'une fraction irréductible.
     \begin{enumerate}[label=\alph*.]
          \item
          Calculer la probabilité que la médaille tirée soit argentée et représente le château de Langeais.
          \item
          Montrer que la probabilité que la médaille tirée représente le château de Langeais est égale à $\frac{21}{40}$.
          \item
          Sachant que la médaille tirée représente le château de Langeais, quelle est la probabilité que celle-ci soit dorée ?
     \end{enumerate}
     \item
     Sachant que la médaille tirée représente le château de Saumur, donner la probabilité que celle-ci soit argentée.
\end{enumerate}
\begin{h3}Partie B\end{h3}
Une médaille est dite conforme lorsque sa masse est comprise entre $9,9$ et $10,1$ grammes.
\par
On dispose de deux machines $M_1$ et $M_2$ pour produire les médailles.
\begin{enumerate}
     \item
     Après plusieurs séries de tests, on estime qu'une machine $M_1$ produit des médailles dont la masse $X$ en grammes suit la loi normale d'espérance $10$ et d'écart-type $0,06$.
     \par
     On note $C$ l'événement « la médaille est conforme ».
     \par
     Calculer la probabilité qu'une médaille produite par la machine $M_1$ ne soit pas conforme. On donnera le résultat arrondi à $10^{-3}$ près.
     \item
     La proportion des médailles non conformes produites par la machine $M_1$ étant jugée trop importante, on utilise une machine $M_2$ qui produit des médailles dont la masse $Y$ en grammes suit la loi normale d'espérance $\mu = 10$ et d'écart-type $\sigma$.
     \begin{enumerate}[label=\alph*.]
          \item
          Soit $Z$ la variable aléatoire égale à $\frac{Y-10}{\sigma}$.
          \par
          Quelle est la loi suivie par la variable $Z$ ?
          \item
          Sachant que cette machine produit $6$\% de pièces non conformes, déterminer la valeur arrondie au millième de $\sigma$.
     \end{enumerate}
\end{enumerate}
\begin{corrige}
     \begin{h3}Partie A\end{h3}
     \begin{enumerate}
          \item
          Le tirage peut être modélisé par l'arbre pondéré ci-dessous :

\begin{center}
\imgsvg{probabilites-bac-s-nouvelle-caledonie-2016}{0.3}% alt="Arbre probabilités" style="width:30rem"
\end{center}

          \begin{enumerate}
               \item
               L'événement dont dont recherche la probabilité est $A \cap L$ :
               \par
               $p(A \cap L)=p(A) \times p_A(L)=\frac{1}{4} \times \frac{3}{10}=\frac{3}{40}$
               \item
               D'après la formule des probabilités totales, la probabilité de $L$ est égale à :
               \par
               $p(L)=p(A \cap L)+p(D \cap L)$
               \par
               $\phantom{p(L)}=p(A \cap L)+p(D) \times p_D(L)$
               \par
               $\phantom{p(L)}=\frac{3}{40}+\frac{3}{4}\times \frac{3}{5}=\frac{21}{40}$
               \item
               La probabilité cherchée est $p_L(D)$. D'après la formule des probabilités conditionnelles :
               \par
               $p_L(D)=\frac{p(D \cap L)}{p(L)}$
               \par
               $p_L(D)=\frac{\frac{9}{20}}{\frac{21}{40}}=\frac{9}{20} \times \frac{40}{21}=\frac{6}{7}$
          \end{enumerate}
          \item
          On cherche cette fois $p_S(A)$. D'après la formule des probabilités conditionnelles :
          \par
          $p_S(A)=\frac{p(A \cap S)}{p(S)}$
          \par
          D'après la formule des probabilités totales :
          \par
          $p(S)=p(A \cap S) + p(D \cap S)$
          \par
          Or $D \cap S$ est l'événement impossible (il n'y a pas de médaille dorée représentant le château de Saumur) donc $p(D \cap S)=0$.
          \par
          Finalement :
          \par
          $p_S(A)=\frac{p(A \cap S)}{p(A \cap S)}=1$
          \par
          Ce résultat était prévisible : comme il n'y a pas de médaille dorée représentant le château de Saumur, une médaille représentant le château de Saumur est nécessairement une médaille argentée. (D'ailleurs cet argument suffisait pour répondre à la question...)
     \end{enumerate}
     \begin{h3}Partie B\end{h3}
     \begin{enumerate}
          \item
          Une médaille est conforme si et seulement si sa masse est comprise entre $9,9$ et $10,1$ grammes. A la calculatrice ("NormalCdf(9.9 , 10.1 , 10 , 0.06)") on trouve :
          \par
          $p(C)=p(9,9 \leqslant X \leqslant 10,1) \approx 0,904$ à $10^{-3}$ près
          \item
          \begin{enumerate}
               \item
               $Y$ suit la loi normale d'espérance $\mu = 10$ et d'écart-type $\sigma$, donc la variable aléatoire $Z=\frac{Y-10}{\sigma}$ suit la loi normale centrée réduite (voir \mcLien{/cours/terminale-s/loi-normale#d70}{cours}) .
               \item
               La machine $M_2$ produit $6$\% de pièces non conformes donc elle produit $94$\% de pièces conformes.
               \par
               Cela se traduit par :
               \par
               $p(9,9 \leqslant Y \leqslant 1,1)=0,94$
               \par
               Or :
               \par
               $9,9 \leqslant Y \leqslant 1,1 \Leftrightarrow -0,1 \leqslant Y-10 \leqslant 0,1 $
               \par
               $\phantom{9,9 \leqslant Y \leqslant 1,1 }\Leftrightarrow -\frac{0,1}{\sigma} \leqslant \frac{Y-10}{\sigma} \leqslant \frac{0,1}{\sigma} $
               \par
               $\phantom{9,9 \leqslant Y \leqslant 1,1 }\Leftrightarrow -\frac{0,1}{\sigma} \leqslant Z \leqslant \frac{0,1}{\sigma} $
               \par
               Par conséquent :
               \par
               $p\left(-\frac{0,1}{\sigma} \leqslant Z \leqslant \frac{0,1}{\sigma}\right) =0,94$
               \par
               Donc par symétrie :
               \par
               $p\left(0 \leqslant Z \leqslant \frac{0,1}{\sigma}\right) =\frac{0,94}{2}=0,47$

\begin{center}
\imgsvg{probabilites-bac-s-nouvelle-caledonie-2016-1}{0.3}% alt="loi normale" style="width:35rem"
\end{center}
               Et par conséquent :
               \par
               $p\left(Z \leqslant \frac{0,1}{\sigma}\right) =p\left(Z \leqslant 0 \right) +p\left(0 \leqslant Z \leqslant \frac{0,1}{\sigma}\right)  $
               \par
               $\phantom{p\left(Z \leqslant \frac{0,1}{\sigma}\right)} =0,5 + 0,47=0,97 $
\begin{center}
\imgsvg{probabilites-bac-s-nouvelle-caledonie-2016-2}{0.3}% alt="loi normale" style="width:35rem"
\end{center}
               A la calculatrice ("InvNormale(0,97)"), on trouve que :
               \par
               $p\left(Z \leqslant 1,881\right)=0,97 $

\begin{center}
\imgsvg{probabilites-bac-s-nouvelle-caledonie-2016-3}{0.3}% alt="loi normale" style="width:35rem"
\end{center}
               donc
               \par
               $\frac{0,1}{\sigma}=1,881$
               \par
               $\sigma=\frac{0,1}{1,881} \approx 0,053$ au millième près.
          \end{enumerate}
     \end{enumerate}
\end{corrige}

\end{document}