\documentclass[a4paper]{article}

%================================================================================================================================
%
% Packages
%
%================================================================================================================================

\usepackage[T1]{fontenc} 	% pour caractères accentués
\usepackage[utf8]{inputenc}  % encodage utf8
\usepackage[french]{babel}	% langue : français
\usepackage{fourier}			% caractères plus lisibles
\usepackage[dvipsnames]{xcolor} % couleurs
\usepackage{fancyhdr}		% réglage header footer
\usepackage{needspace}		% empêcher sauts de page mal placés
\usepackage{graphicx}		% pour inclure des graphiques
\usepackage{enumitem,cprotect}		% personnalise les listes d'items (nécessaire pour ol, al ...)
\usepackage{hyperref}		% Liens hypertexte
\usepackage{pstricks,pst-all,pst-node,pstricks-add,pst-math,pst-plot,pst-tree,pst-eucl} % pstricks
\usepackage[a4paper,includeheadfoot,top=2cm,left=3cm, bottom=2cm,right=3cm]{geometry} % marges etc.
\usepackage{comment}			% commentaires multilignes
\usepackage{amsmath,environ} % maths (matrices, etc.)
\usepackage{amssymb,makeidx}
\usepackage{bm}				% bold maths
\usepackage{tabularx}		% tableaux
\usepackage{colortbl}		% tableaux en couleur
\usepackage{fontawesome}		% Fontawesome
\usepackage{environ}			% environment with command
\usepackage{fp}				% calculs pour ps-tricks
\usepackage{multido}			% pour ps tricks
\usepackage[np]{numprint}	% formattage nombre
\usepackage{tikz,tkz-tab} 			% package principal TikZ
\usepackage{pgfplots}   % axes
\usepackage{mathrsfs}    % cursives
\usepackage{calc}			% calcul taille boites
\usepackage[scaled=0.875]{helvet} % font sans serif
\usepackage{svg} % svg
\usepackage{scrextend} % local margin
\usepackage{scratch} %scratch
\usepackage{multicol} % colonnes
%\usepackage{infix-RPN,pst-func} % formule en notation polanaise inversée
\usepackage{listings}

%================================================================================================================================
%
% Réglages de base
%
%================================================================================================================================

\lstset{
language=Python,   % R code
literate=
{á}{{\'a}}1
{à}{{\`a}}1
{ã}{{\~a}}1
{é}{{\'e}}1
{è}{{\`e}}1
{ê}{{\^e}}1
{í}{{\'i}}1
{ó}{{\'o}}1
{õ}{{\~o}}1
{ú}{{\'u}}1
{ü}{{\"u}}1
{ç}{{\c{c}}}1
{~}{{ }}1
}


\definecolor{codegreen}{rgb}{0,0.6,0}
\definecolor{codegray}{rgb}{0.5,0.5,0.5}
\definecolor{codepurple}{rgb}{0.58,0,0.82}
\definecolor{backcolour}{rgb}{0.95,0.95,0.92}

\lstdefinestyle{mystyle}{
    backgroundcolor=\color{backcolour},   
    commentstyle=\color{codegreen},
    keywordstyle=\color{magenta},
    numberstyle=\tiny\color{codegray},
    stringstyle=\color{codepurple},
    basicstyle=\ttfamily\footnotesize,
    breakatwhitespace=false,         
    breaklines=true,                 
    captionpos=b,                    
    keepspaces=true,                 
    numbers=left,                    
xleftmargin=2em,
framexleftmargin=2em,            
    showspaces=false,                
    showstringspaces=false,
    showtabs=false,                  
    tabsize=2,
    upquote=true
}

\lstset{style=mystyle}


\lstset{style=mystyle}
\newcommand{\imgdir}{C:/laragon/www/newmc/assets/imgsvg/}
\newcommand{\imgsvgdir}{C:/laragon/www/newmc/assets/imgsvg/}

\definecolor{mcgris}{RGB}{220, 220, 220}% ancien~; pour compatibilité
\definecolor{mcbleu}{RGB}{52, 152, 219}
\definecolor{mcvert}{RGB}{125, 194, 70}
\definecolor{mcmauve}{RGB}{154, 0, 215}
\definecolor{mcorange}{RGB}{255, 96, 0}
\definecolor{mcturquoise}{RGB}{0, 153, 153}
\definecolor{mcrouge}{RGB}{255, 0, 0}
\definecolor{mclightvert}{RGB}{205, 234, 190}

\definecolor{gris}{RGB}{220, 220, 220}
\definecolor{bleu}{RGB}{52, 152, 219}
\definecolor{vert}{RGB}{125, 194, 70}
\definecolor{mauve}{RGB}{154, 0, 215}
\definecolor{orange}{RGB}{255, 96, 0}
\definecolor{turquoise}{RGB}{0, 153, 153}
\definecolor{rouge}{RGB}{255, 0, 0}
\definecolor{lightvert}{RGB}{205, 234, 190}
\setitemize[0]{label=\color{lightvert}  $\bullet$}

\pagestyle{fancy}
\renewcommand{\headrulewidth}{0.2pt}
\fancyhead[L]{maths-cours.fr}
\fancyhead[R]{\thepage}
\renewcommand{\footrulewidth}{0.2pt}
\fancyfoot[C]{}

\newcolumntype{C}{>{\centering\arraybackslash}X}
\newcolumntype{s}{>{\hsize=.35\hsize\arraybackslash}X}

\setlength{\parindent}{0pt}		 
\setlength{\parskip}{3mm}
\setlength{\headheight}{1cm}

\def\ebook{ebook}
\def\book{book}
\def\web{web}
\def\type{web}

\newcommand{\vect}[1]{\overrightarrow{\,\mathstrut#1\,}}

\def\Oij{$\left(\text{O}~;~\vect{\imath},~\vect{\jmath}\right)$}
\def\Oijk{$\left(\text{O}~;~\vect{\imath},~\vect{\jmath},~\vect{k}\right)$}
\def\Ouv{$\left(\text{O}~;~\vect{u},~\vect{v}\right)$}

\hypersetup{breaklinks=true, colorlinks = true, linkcolor = OliveGreen, urlcolor = OliveGreen, citecolor = OliveGreen, pdfauthor={Didier BONNEL - https://www.maths-cours.fr} } % supprime les bordures autour des liens

\renewcommand{\arg}[0]{\text{arg}}

\everymath{\displaystyle}

%================================================================================================================================
%
% Macros - Commandes
%
%================================================================================================================================

\newcommand\meta[2]{    			% Utilisé pour créer le post HTML.
	\def\titre{titre}
	\def\url{url}
	\def\arg{#1}
	\ifx\titre\arg
		\newcommand\maintitle{#2}
		\fancyhead[L]{#2}
		{\Large\sffamily \MakeUppercase{#2}}
		\vspace{1mm}\textcolor{mcvert}{\hrule}
	\fi 
	\ifx\url\arg
		\fancyfoot[L]{\href{https://www.maths-cours.fr#2}{\black \footnotesize{https://www.maths-cours.fr#2}}}
	\fi 
}


\newcommand\TitreC[1]{    		% Titre centré
     \needspace{3\baselineskip}
     \begin{center}\textbf{#1}\end{center}
}

\newcommand\newpar{    		% paragraphe
     \par
}

\newcommand\nosp {    		% commande vide (pas d'espace)
}
\newcommand{\id}[1]{} %ignore

\newcommand\boite[2]{				% Boite simple sans titre
	\vspace{5mm}
	\setlength{\fboxrule}{0.2mm}
	\setlength{\fboxsep}{5mm}	
	\fcolorbox{#1}{#1!3}{\makebox[\linewidth-2\fboxrule-2\fboxsep]{
  		\begin{minipage}[t]{\linewidth-2\fboxrule-4\fboxsep}\setlength{\parskip}{3mm}
  			 #2
  		\end{minipage}
	}}
	\vspace{5mm}
}

\newcommand\CBox[4]{				% Boites
	\vspace{5mm}
	\setlength{\fboxrule}{0.2mm}
	\setlength{\fboxsep}{5mm}
	
	\fcolorbox{#1}{#1!3}{\makebox[\linewidth-2\fboxrule-2\fboxsep]{
		\begin{minipage}[t]{1cm}\setlength{\parskip}{3mm}
	  		\textcolor{#1}{\LARGE{#2}}    
 	 	\end{minipage}  
  		\begin{minipage}[t]{\linewidth-2\fboxrule-4\fboxsep}\setlength{\parskip}{3mm}
			\raisebox{1.2mm}{\normalsize\sffamily{\textcolor{#1}{#3}}}						
  			 #4
  		\end{minipage}
	}}
	\vspace{5mm}
}

\newcommand\cadre[3]{				% Boites convertible html
	\par
	\vspace{2mm}
	\setlength{\fboxrule}{0.1mm}
	\setlength{\fboxsep}{5mm}
	\fcolorbox{#1}{white}{\makebox[\linewidth-2\fboxrule-2\fboxsep]{
  		\begin{minipage}[t]{\linewidth-2\fboxrule-4\fboxsep}\setlength{\parskip}{3mm}
			\raisebox{-2.5mm}{\sffamily \small{\textcolor{#1}{\MakeUppercase{#2}}}}		
			\par		
  			 #3
 	 		\end{minipage}
	}}
		\vspace{2mm}
	\par
}

\newcommand\bloc[3]{				% Boites convertible html sans bordure
     \needspace{2\baselineskip}
     {\sffamily \small{\textcolor{#1}{\MakeUppercase{#2}}}}    
		\par		
  			 #3
		\par
}

\newcommand\CHelp[1]{
     \CBox{Plum}{\faInfoCircle}{À RETENIR}{#1}
}

\newcommand\CUp[1]{
     \CBox{NavyBlue}{\faThumbsOUp}{EN PRATIQUE}{#1}
}

\newcommand\CInfo[1]{
     \CBox{Sepia}{\faArrowCircleRight}{REMARQUE}{#1}
}

\newcommand\CRedac[1]{
     \CBox{PineGreen}{\faEdit}{BIEN R\'EDIGER}{#1}
}

\newcommand\CError[1]{
     \CBox{Red}{\faExclamationTriangle}{ATTENTION}{#1}
}

\newcommand\TitreExo[2]{
\needspace{4\baselineskip}
 {\sffamily\large EXERCICE #1\ (\emph{#2 points})}
\vspace{5mm}
}

\newcommand\img[2]{
          \includegraphics[width=#2\paperwidth]{\imgdir#1}
}

\newcommand\imgsvg[2]{
       \begin{center}   \includegraphics[width=#2\paperwidth]{\imgsvgdir#1} \end{center}
}


\newcommand\Lien[2]{
     \href{#1}{#2 \tiny \faExternalLink}
}
\newcommand\mcLien[2]{
     \href{https~://www.maths-cours.fr/#1}{#2 \tiny \faExternalLink}
}

\newcommand{\euro}{\eurologo{}}

%================================================================================================================================
%
% Macros - Environement
%
%================================================================================================================================

\newenvironment{tex}{ %
}
{%
}

\newenvironment{indente}{ %
	\setlength\parindent{10mm}
}

{
	\setlength\parindent{0mm}
}

\newenvironment{corrige}{%
     \needspace{3\baselineskip}
     \medskip
     \textbf{\textsc{Corrigé}}
     \medskip
}
{
}

\newenvironment{extern}{%
     \begin{center}
     }
     {
     \end{center}
}

\NewEnviron{code}{%
	\par
     \boite{gray}{\texttt{%
     \BODY
     }}
     \par
}

\newenvironment{vbloc}{% boite sans cadre empeche saut de page
     \begin{minipage}[t]{\linewidth}
     }
     {
     \end{minipage}
}
\NewEnviron{h2}{%
    \needspace{3\baselineskip}
    \vspace{0.6cm}
	\noindent \MakeUppercase{\sffamily \large \BODY}
	\vspace{1mm}\textcolor{mcgris}{\hrule}\vspace{0.4cm}
	\par
}{}

\NewEnviron{h3}{%
    \needspace{3\baselineskip}
	\vspace{5mm}
	\textsc{\BODY}
	\par
}

\NewEnviron{margeneg}{ %
\begin{addmargin}[-1cm]{0cm}
\BODY
\end{addmargin}
}

\NewEnviron{html}{%
}

\begin{document}
\meta{url}{/exercices/etude-dune-fonction-a-laide-dune-fonction-annexe/}
\meta{pid}{11404}
\meta{titre}{Étude d'une fonction à l'aide d'une fonction annexe}
\meta{type}{exercices}
%
\begin{center}
     \begin{h3} Partie A \end{h3}
\end{center}
Soit la fonction $ g $ définie sur $  \mathbb{R}  $ par~:
\[
g (x) =2x^{ 3 } +2x^2  - 1.
\]
\begin{enumerate}
     \item
     Étudier les variations de la fonction $ g $ sur $  \mathbb{R} . $
     \item
     Calculer $ g (0)  $ et $ g (1).  $\\
     On admet que l'équation $ g (x) =0 $ admet une unique solution  $ x_{ 0 }  $ sur $  \mathbb{R}$.  \\
     Justifier que $ x_{ 0 }  \in  \left] 0 ; 1 \right[$.
     \item
     Déterminer le signe de $ g (x)  $ sur $  \mathbb{R}$.
     \item
     On considère le programme Python ci-dessous~:
\begin{lstlisting}[language=Python]
def g(x) :
    return 2*x**3 + 2*x**2 - 1
def solution() :
    x = 0
    y = g(x)
    while y < 0 :
        x = x + 0.01
        y = g(x)
    return x
\end{lstlisting}
L'appel de la fonction \texttt{solution() } définie ci-dessus retourne \texttt{0.57}.
\newpar
donner un encadrement d'amplitude 0,01 de  $ x_{ 0 }. $
\end{enumerate}
\begin{center}
     \begin{h3} Partie b \end{h3}
\end{center}
On considère la fonction $ f $ définie sur  $  \mathbb{R}  \backslash  \{ 0 \}  $ par~: \\
\[
f (x) = \frac{ x^{ 3 } +2x^2 +1 }{ x } .
\]
\begin{enumerate}
     \item
     Montrer que pour tout réel $ x $ non nul~:
     \[
     f'  (x)= \frac{ g (x)  }{ x^2  } .
     \]
     \item
     Dresser le tableau de variations de $ f $ sur $  \mathbb{R} . $  (On ne cherchera pas à déterminer la valeur de l’extremum de cette fonction.)
\end{enumerate}
%
\begin{corrige}
     \begin{center}
          \begin{h3} Partie A \end{h3}
     \end{center}
     \begin{enumerate}
          \item
          La fonction $ g $ est une fonction polynôme, donc, elle est dérivable sur  $  \mathbb{R}  $ et~:
          \newpar
          $ g'  (x) =2 \times 3x^2+2 \times 2x=6x^2 +4x$\nosp$=2x (3x+2).  $
          \newpar
          $ g'  $ possède donc 2 racines~:  $ x_{ 1 } =0 $ et $ x_{ 2 } =  - \frac{ 2 }{ 3 } . $
          \newpar
          Le coefficient du terme du second degré est positif donc $ g' $ est négative entre $   - \frac{ 2 }{ 3 }  $ et $ 0 $ et est positive à l'extérieur de ces racines.
          \newpar
          $ g (0) = - 1 $
          \newpar
          $ g  \left( -  \frac{ 2 }{ 3 }  \right)  =2 \times  \left(  -  \frac{ 2 }{ 3 }  \right) ^{3}$\nosp$ +2 \times  \left(  -  \frac{ 2 }{ 3 }  \right) ^2  - 1$\nosp$= -  \frac{ 19 }{ 27 } . $
          \newpar
          On peut alors dresser le tableau de variations  de  $ g $~:  \\
          \begin{center}
               \begin{extern}%width="450" alt="tableau de variations de la fonction"
                    \begin{tikzpicture}[scale=0.875]
                         % Styles
                         \tikzstyle{cadre}=[thin]
                         \tikzstyle{fleche}=[->,>=latex,thin]
                         \tikzstyle{nondefini}=[lightgray]
                         % Dimensions modifiables
                         \def\Lrg{1.5}
                         \def\HtX{1}
                         \def\HtY{0.5}
                         % Dimensions calculées
                         \def\lignex{-0.5*\HtX}
                         \def\lignef{-1.5*\HtX}
                         \def\separateur{-0.5*\Lrg}
                         % Largeur du tableau
                         \def\gauche{-1.5*\Lrg}
                         \def\droite{6.5*\Lrg}
                         % Hauteur du tableau
                         \def\haut{0.5*\HtX}
                         \def\bas{-2.5*\HtX-2*\HtY}
                         % Pointillés
                         \draw[dotted,black] (0*\Lrg,\lignex-0.1*\HtX) -- (0*\Lrg,\lignef+0.1*\HtX);
                         \draw[dotted,black] (0*\Lrg,\lignef-0.1*\HtX) -- (0*\Lrg,\bas+0.1*\HtX);
                         \draw[dotted,black] (2*\Lrg,\lignex-0.1*\HtX) -- (2*\Lrg,\lignef+0.1*\HtX);
                         \draw[dotted,black] (2*\Lrg,\lignef-0.1*\HtX) -- (2*\Lrg,\bas+0.1*\HtX);
                         \draw[dotted,black] (4*\Lrg,\lignex-0.1*\HtX) -- (4*\Lrg,\lignef+0.1*\HtX);
                         \draw[dotted,black] (4*\Lrg,\lignef-0.1*\HtX) -- (4*\Lrg,\bas+0.1*\HtX);
                         \draw[dotted,black] (6*\Lrg,\lignex-0.1*\HtX) -- (6*\Lrg,\lignef+0.1*\HtX);
                         \draw[dotted,black] (6*\Lrg,\lignef-0.1*\HtX) -- (6*\Lrg,\bas+0.1*\HtX);
                         % Ligne de l'abscisse~: x
                         \node at (-1*\Lrg,0) {$x$};
                         \node at (0*\Lrg,0) {$ -  \infty $};
                         \node at (2*\Lrg,0) {$ -  \frac{ 2 }{ 3 } $};
                         \node at (4*\Lrg,0) {$0$};
                         \node at (6*\Lrg,0) {$+ \infty $};
                         % Ligne de la dérivée~: f'(x)
                         \node at (-1*\Lrg,-1*\HtX) {$g'(x)$};
                         \node at (0*\Lrg,-1*\HtX) {$$};
                         \node at (1*\Lrg,-1*\HtX) {$+$};
                         \node at (2*\Lrg,-1*\HtX) {$0$};
                         \node at (3*\Lrg,-1*\HtX) {$-$};
                         \node at (4*\Lrg,-1*\HtX) {$0$};
                         \node at (5*\Lrg,-1*\HtX) {$+$};
                         \node at (6*\Lrg,-1*\HtX) {$$};
                         % Ligne de la fonction~: f(x)
                         \node  at (-1*\Lrg,{-2*\HtX+(-1)*\HtY}) {$g(x)$};
                         \node (f1) at (0*\Lrg,{-2*\HtX+(-2)*\HtY}) {$$};
                         \node (f2) at (2*\Lrg,{-2*\HtX+(0)*\HtY}) {$ -  \frac{ 19 }{ 27 } $};
                         \node (f3) at (4*\Lrg,{-2*\HtX+(-2)*\HtY}) {$-1$};
                         \node (f4) at (6*\Lrg,{-2*\HtX+(0)*\HtY}) {$$};
                         % Flèches
                         \draw[fleche] (f1) -- (f2);
                         \draw[fleche] (f2) -- (f3);
                         \draw[fleche] (f3) -- (f4);
                         % Encadrement
                         \draw[cadre] (\separateur,\haut) -- (\separateur,\bas);
                         \draw[cadre] (\gauche,\haut) rectangle  (\droite,\bas);
                         \draw[cadre] (\gauche,\lignex) -- (\droite,\lignex);
                         \draw[cadre] (\gauche,\lignef) -- (\droite,\lignef);
                    \end{tikzpicture}
               \end{extern}
          \end{center}
          \item
          On a déjà calculé $ g (0) = - 1 $ qui est strictement négatif. \\
          $ g (1) =3 $ est strictement positif.   \\
          $ g $ change de signe entre  $ 0 $  et  $ 1 $ donc s'annule pour un nombre $ x_{ 0 }  $ appartenant à l'intervalle  $  \left] 0 ; 1 \right[ . $
          \newpar
          \textbf{Remarque~: } Une démonstration plus rigoureuse nécessiterait l'emploi du théorème des valeurs intermédiaires qui n'est pas au programme de Première. Ici, seule une justification était demandée.
          \item
          D'après le tableau de variations, $ g $ est négative sur l'intervalle $  \left] -  \infty   ; 0 \right]  $. \\
          D'après la question précédente, $ g $ est également négative sur l'intervalle $  \left[ 0 ; x_{ 0 }  \right[  $ mais positive sur l'intervalle $  \left] x_{ 0 }  ; + \infty  \right[.$
          \newpar
          Le tableau de signes de $ g $ est donc le suivant~:  \\
          \begin{center}
               \begin{extern}%width="390" alt="Exemple tableau de signe 1"
                    \resizebox{8cm}{!}{
                         \begin{tikzpicture}[scale=0.875]
                              % Styles
                              \tikzstyle{cadre}=[thin]
                              \tikzstyle{fleche}=[->,>=latex,thin]
                              \tikzstyle{nondefini}=[lightgray]
                              % Dimensions modifiables
                              \def\Lrg{1.8}
                              \def\HtX{1.2}
                              \def\HtY{0.5}
                              % Dimensions calculées
                              \def\lignex{-0.5*\HtX}
                              \def\lignef{-1.5*\HtX}
                              \def\separateur{-0.5*\Lrg}
                              % Largeur du tableau
                              \def\gauche{-1.5*\Lrg}
                              \def\droite{4.5*\Lrg}
                              % Hauteur du tableau
                              \def\haut{0.5*\HtX}
                              \def\bas{-2.5*\HtX-2*\HtY}
                              % Pointillés
                              \draw[gray] (2*\Lrg,\lignex) -- (2*\Lrg,\lignef);
                              % Ligne de l'abscisse : x
                              \node at (-1*\Lrg,0) {$x$};
                              \node at (0*\Lrg,0) {$-\infty$};
                              \node at (2*\Lrg,0) {$x_{ 0 } $};
                              \node at (4*\Lrg,0) {$+\infty$};
                              % Ligne de la dérivée : f'(x)
                              \node at (-1*\Lrg,-1*\HtX) {$g (x) $};
                              \node at (0*\Lrg,-1*\HtX) {$ $};
                              \node at (1*\Lrg,-1*\HtX) {$-$};
                              \node at (2*\Lrg,-1*\HtX) {$0$};
                              \node at (3*\Lrg,-1*\HtX) {$+$};
                              \node at (4*\Lrg,-1*\HtX) {$ $};
                              % Ligne de la fonction : f(x)
                              % Encadrement
                              \draw[cadre] (\separateur,\haut) -- (\separateur, \lignef);
                              \draw[cadre] (\gauche,\haut) rectangle  (\droite, \lignef);
                              \draw[cadre] (\gauche,\lignex) -- (\droite,\lignex);
                         \end{tikzpicture}
                    }
               \end{extern}
          \end{center}
          \item
          La fonction  \og \texttt{solution () } \fg{}  calcule les valeurs de $ g (x)  $ pour $ x $ partant de  $ 0$ et augmentant par pas de $0,01.$
          \newpar
          La boucle \og  \texttt{while}  \fg{} s'arrête dès que $ g (x)   \geqslant 0$ et la fonction renvoie alors la valeur de la variable $ x. $  \\
          Ce nombre retourné est donc le plus petit nombre de deux décimales tel que  $ g (x)  \geqslant 0. $
          \newpar
          Ce nombre étant égal à $ 0,57 $ d'après l'énoncé, on a donc  $ g (0,57)  \geqslant 0 $ mais  $ g (0,56)  < 0. $
          \newpar
          Par conséquent~:  $ 0,56 < x_{ 0 }  \leqslant 0,57. $
     \end{enumerate}
     \begin{center}
          \begin{h3} Partie B \end{h3}
     \end{center}
     \begin{enumerate}
          \item
          Posons~:  $ u (x) =x^3 +2x^{ 2 } +1$ et $ v (x) =x $.
          \newpar
          Alors~:  $ u'  (x) =3x^2 +4x$ et $ v'  (x) =1 $
          \newpar
          Donc~:
          \newpar
          $ f'  (x) = \frac{ u'  (x) v (x)  - u (x) v'  (x)  }{ v (x) ^2  }  $
          \newpar
          $\phantom{ f'  (x) } =  \frac{ x (3x^2 +4x)  -  (x^{ 3 } +2x^2 +1)  }{  x^2  } $
          \newpar
          $\phantom{ f'  (x) } =  \frac{ 2x^{ 3 } +2x^2  - 1 }{  x^2  } $
          \newpar
          $\phantom{ f'  (x) } =  \frac{ g (x)  }{  x^2  } $
          \item
          $ x^2  $ est toujours strictement positif sur $  \mathbb{R} \backslash \{ 0 \}. $  \\
          $ f'  (x)  $ est donc du signe de  $ g (x)  $ qui est donné par le tableau de la question a.3. \\
          Toutefois, $ f $ n'est pas définie en $ 0. $
          \newpar
          On en déduit le tableau de variations de la fonction $ f $~:
          \begin{center}
               \begin{extern}%width="450" alt="tableau de variations de la fonction"
                    \begin{tikzpicture}[scale=0.875]
                         % Styles
                         \tikzstyle{cadre}=[thin]
                         \tikzstyle{fleche}=[->,>=latex,thin]
                         \tikzstyle{nondefini}=[lightgray]
                         % Dimensions modifiables
                         \def\Lrg{1.5}
                         \def\HtX{1}
                         \def\HtY{0.5}
                         % Dimensions calculées
                         \def\lignex{-0.5*\HtX}
                         \def\lignef{-1.5*\HtX}
                         \def\separateur{-0.5*\Lrg}
                         % Largeur du tableau
                         \def\gauche{-1.5*\Lrg}
                         \def\droite{6.5*\Lrg}
                         % Hauteur du tableau
                         \def\haut{0.5*\HtX}
                         \def\bas{-2.5*\HtX-2*\HtY}
                         % Pointillés
                         \draw[dotted,black] (4*\Lrg,\lignex-0.1*\HtX) -- (4*\Lrg,\lignef+0.1*\HtX);
                         \draw[dotted,black] (4*\Lrg,\lignef-0.1*\HtX) -- (4*\Lrg,\bas+0.1*\HtX);
                         % Ligne de l'abscisse~: x
                         \node at (-1*\Lrg,0) {$x$};
                         \node at (0*\Lrg,0) {$ -  \infty $};
                         \node at (2*\Lrg,0) {$0$};
                         \node at (4*\Lrg,0) {$x_0$};
                         \node at (6*\Lrg,0) {$+ \infty $};
                         % Ligne de la dérivée~: f'(x)
                         \node at (-1*\Lrg,-1*\HtX) {$f'(x)$};
                         \node at (0*\Lrg,-1*\HtX) {$$};
                         \node at (1*\Lrg,-1*\HtX) {$-$};
                         \node at (2*\Lrg,-1*\HtX) {$$};
                         \node at (3*\Lrg,-1*\HtX) {$-$};
                         \node at (4*\Lrg,-1*\HtX) {$0$};
                         \node at (5*\Lrg,-1*\HtX) {$+$};
                         \node at (6*\Lrg,-1*\HtX) {$$};
                         % Ligne de la fonction~: f(x)
                         \node  at (-1*\Lrg,{-2*\HtX+(-1)*\HtY}) {$f(x)$};
                         \node (f1) at (0*\Lrg,{-2*\HtX+(0)*\HtY}) {$$};
                         \node (f2) at (1.5*\Lrg,{-2*\HtX+(-2)*\HtY}) {$$};
                         \node (f5) at (2.5*\Lrg,{-2*\HtX+(0)*\HtY}) {$$};
                         \node (f3) at (4*\Lrg,{-2*\HtX+(-2)*\HtY}) {$f(x_0)$};
                         \node (f4) at (6*\Lrg,{-2*\HtX+(0)*\HtY}) {$$};
                         % Flèches
                         \draw[fleche] (f1) -- (f2);
                         \draw[fleche] (f5) -- (f3);
                         \draw[fleche] (f3) -- (f4);
                         % Doubles barres
                         \draw[double distance=2pt] (2*\Lrg,\lignex) -- (2*\Lrg,\bas);
                         % Encadrement
                         \draw[cadre] (\separateur,\haut) -- (\separateur,\bas);
                         \draw[cadre] (\gauche,\haut) rectangle  (\droite,\bas);
                         \draw[cadre] (\gauche,\lignex) -- (\droite,\lignex);
                         \draw[cadre] (\gauche,\lignef) -- (\droite,\lignef);
                    \end{tikzpicture}
               \end{extern}
          \end{center}
     \end{enumerate}
\end{corrige}

\end{document}