\documentclass[a4paper]{article}

%================================================================================================================================
%
% Packages
%
%================================================================================================================================

\usepackage[T1]{fontenc} 	% pour caractères accentués
\usepackage[utf8]{inputenc}  % encodage utf8
\usepackage[french]{babel}	% langue : français
\usepackage{fourier}			% caractères plus lisibles
\usepackage[dvipsnames]{xcolor} % couleurs
\usepackage{fancyhdr}		% réglage header footer
\usepackage{needspace}		% empêcher sauts de page mal placés
\usepackage{graphicx}		% pour inclure des graphiques
\usepackage{enumitem,cprotect}		% personnalise les listes d'items (nécessaire pour ol, al ...)
\usepackage{hyperref}		% Liens hypertexte
\usepackage{pstricks,pst-all,pst-node,pstricks-add,pst-math,pst-plot,pst-tree,pst-eucl} % pstricks
\usepackage[a4paper,includeheadfoot,top=2cm,left=3cm, bottom=2cm,right=3cm]{geometry} % marges etc.
\usepackage{comment}			% commentaires multilignes
\usepackage{amsmath,environ} % maths (matrices, etc.)
\usepackage{amssymb,makeidx}
\usepackage{bm}				% bold maths
\usepackage{tabularx}		% tableaux
\usepackage{colortbl}		% tableaux en couleur
\usepackage{fontawesome}		% Fontawesome
\usepackage{environ}			% environment with command
\usepackage{fp}				% calculs pour ps-tricks
\usepackage{multido}			% pour ps tricks
\usepackage[np]{numprint}	% formattage nombre
\usepackage{tikz,tkz-tab} 			% package principal TikZ
\usepackage{pgfplots}   % axes
\usepackage{mathrsfs}    % cursives
\usepackage{calc}			% calcul taille boites
\usepackage[scaled=0.875]{helvet} % font sans serif
\usepackage{svg} % svg
\usepackage{scrextend} % local margin
\usepackage{scratch} %scratch
\usepackage{multicol} % colonnes
%\usepackage{infix-RPN,pst-func} % formule en notation polanaise inversée
\usepackage{listings}

%================================================================================================================================
%
% Réglages de base
%
%================================================================================================================================

\lstset{
language=Python,   % R code
literate=
{á}{{\'a}}1
{à}{{\`a}}1
{ã}{{\~a}}1
{é}{{\'e}}1
{è}{{\`e}}1
{ê}{{\^e}}1
{í}{{\'i}}1
{ó}{{\'o}}1
{õ}{{\~o}}1
{ú}{{\'u}}1
{ü}{{\"u}}1
{ç}{{\c{c}}}1
{~}{{ }}1
}


\definecolor{codegreen}{rgb}{0,0.6,0}
\definecolor{codegray}{rgb}{0.5,0.5,0.5}
\definecolor{codepurple}{rgb}{0.58,0,0.82}
\definecolor{backcolour}{rgb}{0.95,0.95,0.92}

\lstdefinestyle{mystyle}{
    backgroundcolor=\color{backcolour},   
    commentstyle=\color{codegreen},
    keywordstyle=\color{magenta},
    numberstyle=\tiny\color{codegray},
    stringstyle=\color{codepurple},
    basicstyle=\ttfamily\footnotesize,
    breakatwhitespace=false,         
    breaklines=true,                 
    captionpos=b,                    
    keepspaces=true,                 
    numbers=left,                    
xleftmargin=2em,
framexleftmargin=2em,            
    showspaces=false,                
    showstringspaces=false,
    showtabs=false,                  
    tabsize=2,
    upquote=true
}

\lstset{style=mystyle}


\lstset{style=mystyle}
\newcommand{\imgdir}{C:/laragon/www/newmc/assets/imgsvg/}
\newcommand{\imgsvgdir}{C:/laragon/www/newmc/assets/imgsvg/}

\definecolor{mcgris}{RGB}{220, 220, 220}% ancien~; pour compatibilité
\definecolor{mcbleu}{RGB}{52, 152, 219}
\definecolor{mcvert}{RGB}{125, 194, 70}
\definecolor{mcmauve}{RGB}{154, 0, 215}
\definecolor{mcorange}{RGB}{255, 96, 0}
\definecolor{mcturquoise}{RGB}{0, 153, 153}
\definecolor{mcrouge}{RGB}{255, 0, 0}
\definecolor{mclightvert}{RGB}{205, 234, 190}

\definecolor{gris}{RGB}{220, 220, 220}
\definecolor{bleu}{RGB}{52, 152, 219}
\definecolor{vert}{RGB}{125, 194, 70}
\definecolor{mauve}{RGB}{154, 0, 215}
\definecolor{orange}{RGB}{255, 96, 0}
\definecolor{turquoise}{RGB}{0, 153, 153}
\definecolor{rouge}{RGB}{255, 0, 0}
\definecolor{lightvert}{RGB}{205, 234, 190}
\setitemize[0]{label=\color{lightvert}  $\bullet$}

\pagestyle{fancy}
\renewcommand{\headrulewidth}{0.2pt}
\fancyhead[L]{maths-cours.fr}
\fancyhead[R]{\thepage}
\renewcommand{\footrulewidth}{0.2pt}
\fancyfoot[C]{}

\newcolumntype{C}{>{\centering\arraybackslash}X}
\newcolumntype{s}{>{\hsize=.35\hsize\arraybackslash}X}

\setlength{\parindent}{0pt}		 
\setlength{\parskip}{3mm}
\setlength{\headheight}{1cm}

\def\ebook{ebook}
\def\book{book}
\def\web{web}
\def\type{web}

\newcommand{\vect}[1]{\overrightarrow{\,\mathstrut#1\,}}

\def\Oij{$\left(\text{O}~;~\vect{\imath},~\vect{\jmath}\right)$}
\def\Oijk{$\left(\text{O}~;~\vect{\imath},~\vect{\jmath},~\vect{k}\right)$}
\def\Ouv{$\left(\text{O}~;~\vect{u},~\vect{v}\right)$}

\hypersetup{breaklinks=true, colorlinks = true, linkcolor = OliveGreen, urlcolor = OliveGreen, citecolor = OliveGreen, pdfauthor={Didier BONNEL - https://www.maths-cours.fr} } % supprime les bordures autour des liens

\renewcommand{\arg}[0]{\text{arg}}

\everymath{\displaystyle}

%================================================================================================================================
%
% Macros - Commandes
%
%================================================================================================================================

\newcommand\meta[2]{    			% Utilisé pour créer le post HTML.
	\def\titre{titre}
	\def\url{url}
	\def\arg{#1}
	\ifx\titre\arg
		\newcommand\maintitle{#2}
		\fancyhead[L]{#2}
		{\Large\sffamily \MakeUppercase{#2}}
		\vspace{1mm}\textcolor{mcvert}{\hrule}
	\fi 
	\ifx\url\arg
		\fancyfoot[L]{\href{https://www.maths-cours.fr#2}{\black \footnotesize{https://www.maths-cours.fr#2}}}
	\fi 
}


\newcommand\TitreC[1]{    		% Titre centré
     \needspace{3\baselineskip}
     \begin{center}\textbf{#1}\end{center}
}

\newcommand\newpar{    		% paragraphe
     \par
}

\newcommand\nosp {    		% commande vide (pas d'espace)
}
\newcommand{\id}[1]{} %ignore

\newcommand\boite[2]{				% Boite simple sans titre
	\vspace{5mm}
	\setlength{\fboxrule}{0.2mm}
	\setlength{\fboxsep}{5mm}	
	\fcolorbox{#1}{#1!3}{\makebox[\linewidth-2\fboxrule-2\fboxsep]{
  		\begin{minipage}[t]{\linewidth-2\fboxrule-4\fboxsep}\setlength{\parskip}{3mm}
  			 #2
  		\end{minipage}
	}}
	\vspace{5mm}
}

\newcommand\CBox[4]{				% Boites
	\vspace{5mm}
	\setlength{\fboxrule}{0.2mm}
	\setlength{\fboxsep}{5mm}
	
	\fcolorbox{#1}{#1!3}{\makebox[\linewidth-2\fboxrule-2\fboxsep]{
		\begin{minipage}[t]{1cm}\setlength{\parskip}{3mm}
	  		\textcolor{#1}{\LARGE{#2}}    
 	 	\end{minipage}  
  		\begin{minipage}[t]{\linewidth-2\fboxrule-4\fboxsep}\setlength{\parskip}{3mm}
			\raisebox{1.2mm}{\normalsize\sffamily{\textcolor{#1}{#3}}}						
  			 #4
  		\end{minipage}
	}}
	\vspace{5mm}
}

\newcommand\cadre[3]{				% Boites convertible html
	\par
	\vspace{2mm}
	\setlength{\fboxrule}{0.1mm}
	\setlength{\fboxsep}{5mm}
	\fcolorbox{#1}{white}{\makebox[\linewidth-2\fboxrule-2\fboxsep]{
  		\begin{minipage}[t]{\linewidth-2\fboxrule-4\fboxsep}\setlength{\parskip}{3mm}
			\raisebox{-2.5mm}{\sffamily \small{\textcolor{#1}{\MakeUppercase{#2}}}}		
			\par		
  			 #3
 	 		\end{minipage}
	}}
		\vspace{2mm}
	\par
}

\newcommand\bloc[3]{				% Boites convertible html sans bordure
     \needspace{2\baselineskip}
     {\sffamily \small{\textcolor{#1}{\MakeUppercase{#2}}}}    
		\par		
  			 #3
		\par
}

\newcommand\CHelp[1]{
     \CBox{Plum}{\faInfoCircle}{À RETENIR}{#1}
}

\newcommand\CUp[1]{
     \CBox{NavyBlue}{\faThumbsOUp}{EN PRATIQUE}{#1}
}

\newcommand\CInfo[1]{
     \CBox{Sepia}{\faArrowCircleRight}{REMARQUE}{#1}
}

\newcommand\CRedac[1]{
     \CBox{PineGreen}{\faEdit}{BIEN R\'EDIGER}{#1}
}

\newcommand\CError[1]{
     \CBox{Red}{\faExclamationTriangle}{ATTENTION}{#1}
}

\newcommand\TitreExo[2]{
\needspace{4\baselineskip}
 {\sffamily\large EXERCICE #1\ (\emph{#2 points})}
\vspace{5mm}
}

\newcommand\img[2]{
          \includegraphics[width=#2\paperwidth]{\imgdir#1}
}

\newcommand\imgsvg[2]{
       \begin{center}   \includegraphics[width=#2\paperwidth]{\imgsvgdir#1} \end{center}
}


\newcommand\Lien[2]{
     \href{#1}{#2 \tiny \faExternalLink}
}
\newcommand\mcLien[2]{
     \href{https~://www.maths-cours.fr/#1}{#2 \tiny \faExternalLink}
}

\newcommand{\euro}{\eurologo{}}

%================================================================================================================================
%
% Macros - Environement
%
%================================================================================================================================

\newenvironment{tex}{ %
}
{%
}

\newenvironment{indente}{ %
	\setlength\parindent{10mm}
}

{
	\setlength\parindent{0mm}
}

\newenvironment{corrige}{%
     \needspace{3\baselineskip}
     \medskip
     \textbf{\textsc{Corrigé}}
     \medskip
}
{
}

\newenvironment{extern}{%
     \begin{center}
     }
     {
     \end{center}
}

\NewEnviron{code}{%
	\par
     \boite{gray}{\texttt{%
     \BODY
     }}
     \par
}

\newenvironment{vbloc}{% boite sans cadre empeche saut de page
     \begin{minipage}[t]{\linewidth}
     }
     {
     \end{minipage}
}
\NewEnviron{h2}{%
    \needspace{3\baselineskip}
    \vspace{0.6cm}
	\noindent \MakeUppercase{\sffamily \large \BODY}
	\vspace{1mm}\textcolor{mcgris}{\hrule}\vspace{0.4cm}
	\par
}{}

\NewEnviron{h3}{%
    \needspace{3\baselineskip}
	\vspace{5mm}
	\textsc{\BODY}
	\par
}

\NewEnviron{margeneg}{ %
\begin{addmargin}[-1cm]{0cm}
\BODY
\end{addmargin}
}

\NewEnviron{html}{%
}

\begin{document}
\meta{url}{/methode/montrer-quune-suite-croissante-decroissante/}
\meta{pid}{6242}
\meta{titre}{Montrer qu'une suite est croissante (ou décroissante)}
\meta{type}{methode}
%
\cadre{bleu}{Remarque}{%
     Pour simplifier les explications, on supposera que les suites $(u_n)$ étudiées ici sont définies pour tout entier naturel $n$, c'est à dire à partir de $u_0$.
     \par
     Les méthodes ci-dessous se généralisent facilement aux suites commençant à $u_1$, $u_2$, etc.
}
\cadre{vert}{Rappel}{%
     On considère une suite $(u_n)$ définie pour tout entier naturel $n$.
     \begin{itemize}
          \item
          la suite $\left(u_{n}\right)$ est \textbf{croissante} si pour tout entier naturel $n$ : $u_{n+1} \geqslant u_{n}$
          \item
          la suite $\left(u_{n}\right)$ est \textbf{décroissante} si pour tout entier naturel $n$ : $u_{n+1} \leqslant u_{n}$
          \item
          la suite $\left(u_{n}\right)$ est \textbf{constante} si pour tout entier naturel $n$ : $u_{n+1} = u_{n}$
          \item
          la suite $\left(u_{n}\right)$ est \textbf{strictement croissante} si pour tout entier naturel $n$ : $u_{n+1} > u_{n}$
          \item
          la suite $\left(u_{n}\right)$ est \textbf{strictement décroissante} si pour tout entier naturel $n$ : $u_{n+1} < u_{n}$
     \end{itemize}
}
\begin{h2}Première méthode\end{h2}
\begin{h3}Étude du signe de $u_{n+1}-u_{n} $ \end{h3}
\cadre{rouge}{Méthode}{% id="m010"
     On calcule $u_{n+1}-u_{n}$ puis on étudie le signe du résultat.
     \begin{itemize}
          \item
          si pour tout entier naturel $n$ : $u_{n+1}- u_{n} \geqslant 0 $, la suite $\left(u_{n}\right)$ est \textbf{croissante}
          \item
          si pour tout entier naturel $n$ : $u_{n+1}- u_{n} \leqslant 0 $, la suite $\left(u_{n}\right)$ est \textbf{décroissante}
          \item
          si pour tout entier naturel $n$ : $u_{n+1}- u_{n} = 0 $, la suite $\left(u_{n}\right)$ est \textbf{constante}
          \item
          si pour tout entier naturel $n$ : $u_{n+1}- u_{n} > 0 $, la suite $\left(u_{n}\right)$ est \textbf{strictement croissante}
          \item
          si pour tout entier naturel $n$ : $u_{n+1}- u_{n} < 0 $, la suite $\left(u_{n}\right)$ est \textbf{strictement décroissante}
     \end{itemize}
     \textit{\textbf{Remarque 1} : Pour l'étude du signe on n'oubliera pas que $n$ étant un entier naturel, il est positif ou nul.}
     \textit{\textbf{Remarque 2 }: Une suite peut très bien n'être ni croissante, ni décroissante, ni constante (cas des suites non monotones comme la suite $(u_n)$ définie par  $u_n=(-1)^n$})
}
\bloc{orange}{Exemple 1}{% id="e020"
     Etudier le sens de variation de la suite $(u_n)$ définie pour tout $n \in \mathbb{N}$ par $u_n= \frac{n}{n+1} $.
\par        \textbf{Solution :}
     On calcule $u_{n+1}$ en remplaçant $n$ par $n+1$ dans la formule donnant $u_n$ :
     \par
     $u_{n+1}= \frac{n+1}{(n+1)+1}= \frac{n+1}{n+2} $.
     \par
     Par conséquent :
     \par
     $u_{n+1}-u_n= \frac{n+1}{n+2}- \frac{n}{n+1}  $
     \par
     On réduit au même dénominateur :
     \par
     $u_{n+1}-u_n= \frac{(n+1)^2}{(n+2)(n+1)}- \frac{n(n+2)}{(n+1)(n+2)}  $
     \par
     $u_{n+1}-u_n= \frac{n^2+2n+1}{(n+2)(n+1)}- \frac{n^2+2n}{(n+1)(n+2)}  $
     \\   
     $u_{n+1}-u_n= \frac{1}{(n+1)(n+2)}  $
     \par
     Le numérateur et le dénominateur étant positifs (car $n$ est un entier naturel) $u_{n+1}-u_n >0  $ donc la suite $(u_n)$ est strictement croissante.
}
\bloc{orange}{Exemple 2}{% id="e030"
     Montrer que la suite $(u_n)$ définie par $u_0=0$ et pour tout $n \in \mathbb{N}$ : $u_{n+1}= u_n+n-1 $ est croissante pour $n \geqslant 1$.
\par        \textbf{Solution :}
     $u_{n+1}-u_n= (u_n+n-1)-u_n=n-1  $
     \par
     $u_{n+1}-u_n \geqslant 0 $ pour $n  \geqslant 1$ donc la suite $(u_n)$ est croissante à partir du rang 1.
}
\bloc{orange}{Cas particulier 1 : Suites arithmétiques}{% id="e030"
     Une suite arithmétique de raison $r$ est définie par une relation du type $u_{n+1}=u_n + r$.
     \par
     On a donc $u_{n+1}-u_n=r$
\par        \textbf{Résultat :
          \par
     Une suite arithmétique est croissante (resp. décroissante) si et seulement si sa raison est positive (resp. négative).}
}
\bloc{orange}{Cas particulier 2 : Suites géométriques}{% id="e030"
     On considère une suite géométrique \textbf{de premier terme et de raison tous deux positifs}.
     \par
     Pour une suite géométrique de raison $q$ : $u_{n}=u_0 q^n$.
     \par
     Par conséquent :
     \par
     $u_{n+1}-u_n=u_0 q^{n+1}-u_0 q^n = u_0 q^n(q-1)$
     \par
     $u_{n+1}-u_n$ est donc du signe de $q-1$ (puisqu'on a supposé $u_0$ et $q$ positifs).
\par        \textbf{Résultat :
          \par
     Une suite géométrique de raison $q>0$ et de premier terme $u_0>0$ est croissante (resp. décroissante) si et seulement si $q \geqslant 1$(resp. $q \leqslant 1$).}
}
\begin{h2}Deuxième méthode\end{h2}
\begin{h3}Étude de fonction \end{h3}
\cadre{rouge}{Méthode}{% id="m010"
     Si la suite $(u_n)$ est définie par une \textbf{formule explicite} du type $u_n=f(n)$, on peut étudier les variations de la fonction $x \longmapsto f(x)$ sur $[0; +\infty[ $
     \begin{itemize}
          \item
          si $f$ est croissante (resp. strictement croissante), la suite $\left(u_{n}\right)$ est \textbf{croissante}  (resp. \textbf{strictement croissante})
          \item
          si $f$ est décroissante (resp. strictement décroissante), la suite $\left(u_{n}\right)$ est \textbf{décroissante}  (resp. \textbf{strictement décroissante})
          \item
          si $f$ est constante, la suite $\left(u_{n}\right)$ est \textbf{constante}
     \end{itemize}
}
\bloc{orange}{Exemple 3}{% id="e040"
     On reprend la suite $(u_n)$ de \textbf{l'exemple 1} définie pour tout $n \in \mathbb{N}$ par $u_n= \frac{n}{n+1} $.
\par        \textbf{Solution :}
     On définit $f$ sur $[0 ; + \infty [ $ par $f(x)= \frac{x}{x+1} $.
     \\   $f^\prime (x)= \frac{1\times(x+1)-1\times x}{(x+1)^2} = \frac{1}{(x+1)^2} > 0$
     \\   $f^\prime $ est strictement positive sur $[0 ; + \infty [$ donc la fonction $f$ est strictement croissante sur $[0 ; + \infty [$ et la suite $(u_n)$ est \textbf{strictement croissante}.
}
\begin{h2}Troisième méthode\end{h2}
\begin{h3}Démonstration par récurrence (en terminale S) \end{h3}
\cadre{rouge}{Méthode}{% id="m010"
     Si la suite $(u_n)$ est définie par une \textbf{formule par récurrence} (par exemple par une formule du type $u_{n+1}=f(u_n)$), on peut démontrer par récurrence que $u_{n+1} \geqslant u_n$ (resp. $u_{n+1} \leqslant u_n$) pour montrer que la suite est croissante (resp. décroissante)
}
\bloc{orange}{Exemple 4}{% id="e050"
     Soit la suite $(u_n)$ définie sur $\mathbb{N}$ par $u_0=1$ et pour tout $n \in \mathbb{N}   $ : $u_{n+1}=2u_n-3$.
     \par
     Montrer que la suite $(u_n)$ est strictement décroissante.
\par        \textbf{Solution :}
     Montrons par récurrence que pour tout entier naturel $n$ : $u_{n+1} < u_n$.
     \\   \textit{Initialisation}
     $u_0=1$ et $u_1=2 \times 1-3=-1$$u_1 < u_0$ donc la propriété est vraie au rang 0.
     \\   \textit{Hérédité}
     Supposons que la propriété $u_{n+1} < u_n$ est vraie pour un certain entier $n$ et montrons que $u_{n+2} < u_{n+1}$.
     \par
     $ u_{n+1} < u_n  \Rightarrow 2u_{n+1} < 2u_n $
     \par
     $ \phantom{u_{n+1} < u_n}  \Rightarrow 2u_{n+1}-3< 2u_n-3 $
     \par
     $ \phantom{u_{n+1} < u_n}  \Rightarrow u_{n+2}< u_{n+1} $
     \par
     ce qui prouve l'hérédité.
     \\   \textit{Conclusion}
     Pour tout entier naturel $n$ : $u_{n+1} < u_n$ donc la suite $(u_n) $ est strictement décroissante.
}
\bloc{orange}{Exemple 5}{% id="e050"
     Soit la suite $(u_n)$ définie par $u_0=0$ et pour tout entier naturel $n$ : $u_{n+1}=u_n^3+u_n-1$.
     \par
     Etudier le sens de variation de la suite $(u_n)$.
\par        \textbf{Solution :}
     Le calcul des premiers termes ($u_0=0$, $u_1=-1$, $u_2=-3$) laisse présager que la suite $(u_n)$ est strictement décroissante.
     \par
     Montrons par récurrence que pour tout entier naturel $n$ : $u_{n+1} < u_n$.
     \\   \textit{Initialisation}
     $u_0=0$ et $u_1= -1$.
     \par
     $u_1 < u_0$ donc la propriété est vraie au rang 0.
     \\   \textit{Hérédité}
     Supposons que la propriété $u_{n+1} < u_n$ est vraie pour un certain entier $n$ et montrons que $u_{n+2} < u_{n+1}$.
     \par
     Posons $f(x)=x^3+x-1$ pour tout $x \in \mathbb{R}$.
     \par
     Alors :$f^\prime (x) = 3x^2+1$ est strictement positif pour tout réel $x$ donc la fonction $f$ est strictement croissante sur $\mathbb{R}$.
     \par
     Par conséquent :
     \par
     $ u_{n+1} < u_n  \Rightarrow f(u_{n+1}) < f(u_n) $ puisque $f$ est strictement croissante !
     \par
     $ \phantom{u_{n+1} < u_n}  \Rightarrow u_{n+2}< u_{n+1} $
     \par
     ce qui prouve l'hérédité.
     \\   \textit{Conclusion}
     Pour tout entier naturel $n$ : $u_{n+1} < u_n$ donc la suite $(u_n) $ est strictement décroissante.
}

\end{document}