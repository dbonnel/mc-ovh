\documentclass[a4paper]{article}

%================================================================================================================================
%
% Packages
%
%================================================================================================================================

\usepackage[T1]{fontenc} 	% pour caractères accentués
\usepackage[utf8]{inputenc}  % encodage utf8
\usepackage[french]{babel}	% langue : français
\usepackage{fourier}			% caractères plus lisibles
\usepackage[dvipsnames]{xcolor} % couleurs
\usepackage{fancyhdr}		% réglage header footer
\usepackage{needspace}		% empêcher sauts de page mal placés
\usepackage{graphicx}		% pour inclure des graphiques
\usepackage{enumitem,cprotect}		% personnalise les listes d'items (nécessaire pour ol, al ...)
\usepackage{hyperref}		% Liens hypertexte
\usepackage{pstricks,pst-all,pst-node,pstricks-add,pst-math,pst-plot,pst-tree,pst-eucl} % pstricks
\usepackage[a4paper,includeheadfoot,top=2cm,left=3cm, bottom=2cm,right=3cm]{geometry} % marges etc.
\usepackage{comment}			% commentaires multilignes
\usepackage{amsmath,environ} % maths (matrices, etc.)
\usepackage{amssymb,makeidx}
\usepackage{bm}				% bold maths
\usepackage{tabularx}		% tableaux
\usepackage{colortbl}		% tableaux en couleur
\usepackage{fontawesome}		% Fontawesome
\usepackage{environ}			% environment with command
\usepackage{fp}				% calculs pour ps-tricks
\usepackage{multido}			% pour ps tricks
\usepackage[np]{numprint}	% formattage nombre
\usepackage{tikz,tkz-tab} 			% package principal TikZ
\usepackage{pgfplots}   % axes
\usepackage{mathrsfs}    % cursives
\usepackage{calc}			% calcul taille boites
\usepackage[scaled=0.875]{helvet} % font sans serif
\usepackage{svg} % svg
\usepackage{scrextend} % local margin
\usepackage{scratch} %scratch
\usepackage{multicol} % colonnes
%\usepackage{infix-RPN,pst-func} % formule en notation polanaise inversée
\usepackage{listings}

%================================================================================================================================
%
% Réglages de base
%
%================================================================================================================================

\lstset{
language=Python,   % R code
literate=
{á}{{\'a}}1
{à}{{\`a}}1
{ã}{{\~a}}1
{é}{{\'e}}1
{è}{{\`e}}1
{ê}{{\^e}}1
{í}{{\'i}}1
{ó}{{\'o}}1
{õ}{{\~o}}1
{ú}{{\'u}}1
{ü}{{\"u}}1
{ç}{{\c{c}}}1
{~}{{ }}1
}


\definecolor{codegreen}{rgb}{0,0.6,0}
\definecolor{codegray}{rgb}{0.5,0.5,0.5}
\definecolor{codepurple}{rgb}{0.58,0,0.82}
\definecolor{backcolour}{rgb}{0.95,0.95,0.92}

\lstdefinestyle{mystyle}{
    backgroundcolor=\color{backcolour},   
    commentstyle=\color{codegreen},
    keywordstyle=\color{magenta},
    numberstyle=\tiny\color{codegray},
    stringstyle=\color{codepurple},
    basicstyle=\ttfamily\footnotesize,
    breakatwhitespace=false,         
    breaklines=true,                 
    captionpos=b,                    
    keepspaces=true,                 
    numbers=left,                    
xleftmargin=2em,
framexleftmargin=2em,            
    showspaces=false,                
    showstringspaces=false,
    showtabs=false,                  
    tabsize=2,
    upquote=true
}

\lstset{style=mystyle}


\lstset{style=mystyle}
\newcommand{\imgdir}{C:/laragon/www/newmc/assets/imgsvg/}
\newcommand{\imgsvgdir}{C:/laragon/www/newmc/assets/imgsvg/}

\definecolor{mcgris}{RGB}{220, 220, 220}% ancien~; pour compatibilité
\definecolor{mcbleu}{RGB}{52, 152, 219}
\definecolor{mcvert}{RGB}{125, 194, 70}
\definecolor{mcmauve}{RGB}{154, 0, 215}
\definecolor{mcorange}{RGB}{255, 96, 0}
\definecolor{mcturquoise}{RGB}{0, 153, 153}
\definecolor{mcrouge}{RGB}{255, 0, 0}
\definecolor{mclightvert}{RGB}{205, 234, 190}

\definecolor{gris}{RGB}{220, 220, 220}
\definecolor{bleu}{RGB}{52, 152, 219}
\definecolor{vert}{RGB}{125, 194, 70}
\definecolor{mauve}{RGB}{154, 0, 215}
\definecolor{orange}{RGB}{255, 96, 0}
\definecolor{turquoise}{RGB}{0, 153, 153}
\definecolor{rouge}{RGB}{255, 0, 0}
\definecolor{lightvert}{RGB}{205, 234, 190}
\setitemize[0]{label=\color{lightvert}  $\bullet$}

\pagestyle{fancy}
\renewcommand{\headrulewidth}{0.2pt}
\fancyhead[L]{maths-cours.fr}
\fancyhead[R]{\thepage}
\renewcommand{\footrulewidth}{0.2pt}
\fancyfoot[C]{}

\newcolumntype{C}{>{\centering\arraybackslash}X}
\newcolumntype{s}{>{\hsize=.35\hsize\arraybackslash}X}

\setlength{\parindent}{0pt}		 
\setlength{\parskip}{3mm}
\setlength{\headheight}{1cm}

\def\ebook{ebook}
\def\book{book}
\def\web{web}
\def\type{web}

\newcommand{\vect}[1]{\overrightarrow{\,\mathstrut#1\,}}

\def\Oij{$\left(\text{O}~;~\vect{\imath},~\vect{\jmath}\right)$}
\def\Oijk{$\left(\text{O}~;~\vect{\imath},~\vect{\jmath},~\vect{k}\right)$}
\def\Ouv{$\left(\text{O}~;~\vect{u},~\vect{v}\right)$}

\hypersetup{breaklinks=true, colorlinks = true, linkcolor = OliveGreen, urlcolor = OliveGreen, citecolor = OliveGreen, pdfauthor={Didier BONNEL - https://www.maths-cours.fr} } % supprime les bordures autour des liens

\renewcommand{\arg}[0]{\text{arg}}

\everymath{\displaystyle}

%================================================================================================================================
%
% Macros - Commandes
%
%================================================================================================================================

\newcommand\meta[2]{    			% Utilisé pour créer le post HTML.
	\def\titre{titre}
	\def\url{url}
	\def\arg{#1}
	\ifx\titre\arg
		\newcommand\maintitle{#2}
		\fancyhead[L]{#2}
		{\Large\sffamily \MakeUppercase{#2}}
		\vspace{1mm}\textcolor{mcvert}{\hrule}
	\fi 
	\ifx\url\arg
		\fancyfoot[L]{\href{https://www.maths-cours.fr#2}{\black \footnotesize{https://www.maths-cours.fr#2}}}
	\fi 
}


\newcommand\TitreC[1]{    		% Titre centré
     \needspace{3\baselineskip}
     \begin{center}\textbf{#1}\end{center}
}

\newcommand\newpar{    		% paragraphe
     \par
}

\newcommand\nosp {    		% commande vide (pas d'espace)
}
\newcommand{\id}[1]{} %ignore

\newcommand\boite[2]{				% Boite simple sans titre
	\vspace{5mm}
	\setlength{\fboxrule}{0.2mm}
	\setlength{\fboxsep}{5mm}	
	\fcolorbox{#1}{#1!3}{\makebox[\linewidth-2\fboxrule-2\fboxsep]{
  		\begin{minipage}[t]{\linewidth-2\fboxrule-4\fboxsep}\setlength{\parskip}{3mm}
  			 #2
  		\end{minipage}
	}}
	\vspace{5mm}
}

\newcommand\CBox[4]{				% Boites
	\vspace{5mm}
	\setlength{\fboxrule}{0.2mm}
	\setlength{\fboxsep}{5mm}
	
	\fcolorbox{#1}{#1!3}{\makebox[\linewidth-2\fboxrule-2\fboxsep]{
		\begin{minipage}[t]{1cm}\setlength{\parskip}{3mm}
	  		\textcolor{#1}{\LARGE{#2}}    
 	 	\end{minipage}  
  		\begin{minipage}[t]{\linewidth-2\fboxrule-4\fboxsep}\setlength{\parskip}{3mm}
			\raisebox{1.2mm}{\normalsize\sffamily{\textcolor{#1}{#3}}}						
  			 #4
  		\end{minipage}
	}}
	\vspace{5mm}
}

\newcommand\cadre[3]{				% Boites convertible html
	\par
	\vspace{2mm}
	\setlength{\fboxrule}{0.1mm}
	\setlength{\fboxsep}{5mm}
	\fcolorbox{#1}{white}{\makebox[\linewidth-2\fboxrule-2\fboxsep]{
  		\begin{minipage}[t]{\linewidth-2\fboxrule-4\fboxsep}\setlength{\parskip}{3mm}
			\raisebox{-2.5mm}{\sffamily \small{\textcolor{#1}{\MakeUppercase{#2}}}}		
			\par		
  			 #3
 	 		\end{minipage}
	}}
		\vspace{2mm}
	\par
}

\newcommand\bloc[3]{				% Boites convertible html sans bordure
     \needspace{2\baselineskip}
     {\sffamily \small{\textcolor{#1}{\MakeUppercase{#2}}}}    
		\par		
  			 #3
		\par
}

\newcommand\CHelp[1]{
     \CBox{Plum}{\faInfoCircle}{À RETENIR}{#1}
}

\newcommand\CUp[1]{
     \CBox{NavyBlue}{\faThumbsOUp}{EN PRATIQUE}{#1}
}

\newcommand\CInfo[1]{
     \CBox{Sepia}{\faArrowCircleRight}{REMARQUE}{#1}
}

\newcommand\CRedac[1]{
     \CBox{PineGreen}{\faEdit}{BIEN R\'EDIGER}{#1}
}

\newcommand\CError[1]{
     \CBox{Red}{\faExclamationTriangle}{ATTENTION}{#1}
}

\newcommand\TitreExo[2]{
\needspace{4\baselineskip}
 {\sffamily\large EXERCICE #1\ (\emph{#2 points})}
\vspace{5mm}
}

\newcommand\img[2]{
          \includegraphics[width=#2\paperwidth]{\imgdir#1}
}

\newcommand\imgsvg[2]{
       \begin{center}   \includegraphics[width=#2\paperwidth]{\imgsvgdir#1} \end{center}
}


\newcommand\Lien[2]{
     \href{#1}{#2 \tiny \faExternalLink}
}
\newcommand\mcLien[2]{
     \href{https~://www.maths-cours.fr/#1}{#2 \tiny \faExternalLink}
}

\newcommand{\euro}{\eurologo{}}

%================================================================================================================================
%
% Macros - Environement
%
%================================================================================================================================

\newenvironment{tex}{ %
}
{%
}

\newenvironment{indente}{ %
	\setlength\parindent{10mm}
}

{
	\setlength\parindent{0mm}
}

\newenvironment{corrige}{%
     \needspace{3\baselineskip}
     \medskip
     \textbf{\textsc{Corrigé}}
     \medskip
}
{
}

\newenvironment{extern}{%
     \begin{center}
     }
     {
     \end{center}
}

\NewEnviron{code}{%
	\par
     \boite{gray}{\texttt{%
     \BODY
     }}
     \par
}

\newenvironment{vbloc}{% boite sans cadre empeche saut de page
     \begin{minipage}[t]{\linewidth}
     }
     {
     \end{minipage}
}
\NewEnviron{h2}{%
    \needspace{3\baselineskip}
    \vspace{0.6cm}
	\noindent \MakeUppercase{\sffamily \large \BODY}
	\vspace{1mm}\textcolor{mcgris}{\hrule}\vspace{0.4cm}
	\par
}{}

\NewEnviron{h3}{%
    \needspace{3\baselineskip}
	\vspace{5mm}
	\textsc{\BODY}
	\par
}

\NewEnviron{margeneg}{ %
\begin{addmargin}[-1cm]{0cm}
\BODY
\end{addmargin}
}

\NewEnviron{html}{%
}

\begin{document}
\meta{url}{/cours/droites/}
\meta{pid}{166}
\meta{titre}{Équations de droites}
\meta{type}{cours}
\begin{h2}1. Équation réduite d'une droite\end{h2}
\cadre{vert}{Propriété}{%
     Une droite du plan peut être caractérisée une équation de la forme :
     \begin{itemize}
          \item $x=c$ si cette droite est parallèle à l'axe des ordonnées (\textit{« verticale »})
          \item $y=mx+p$ si cette droite n'est pas parallèle à l'axe des ordonnées.
     \end{itemize}
     Dans le second cas, $m$ est appelé coefficient directeur et $p$ ordonnée à l'origine.
}
\bloc{orange}{Exemples}{%
     \begin{center}
          \begin{extern}%width="550" alt="équations de droites"
               \begin{tabular}{c c c}
                    \resizebox{5.5cm}{!}{%
                         % -+-+-+ variables modifiables
                         \def\fonction{1+0.2*x*x }
                         \def\xmin{-2.5}
                         \def\xmax{3.5}
                         \def\ymin{-2.5}
                         \def\ymax{4.5}
                         \def\xunit{1}  % unités en cm
                         \def\yunit{1}
                         \psset{xunit=\xunit,yunit=\yunit,algebraic=true}
                         \fontsize{12pt}{12pt}\selectfont
                         \begin{pspicture*}[linewidth=1pt](\xmin,\ymin)(\xmax,\ymax)
                              %      \psgrid[gridcolor=mcgris, subgriddiv=5, gridlabels=0pt](\xmin,\ymin)(\xmax,\ymax)
                              \psaxes[linewidth=0.75pt,Dx=1,Dy=1]{->}(0,0)(\xmin,\ymin)(\xmax,\ymax)
                              \psline[linewidth=0.75pt,linecolor=red](1,\ymin)(1,\ymax)
                              \rput[tr](-0.3,-0.3){$O$}
                              \rput[l](1.2,4){$\color{red} x=1$}
                         \end{pspicture*}
                    }
                    & ~~~~ &%
                    \resizebox{5.5cm}{!}{%
                         % -+-+-+ variables modifiables
                         \def\fonction{2*x-1 }
                         \def\xmin{-2.5}
                         \def\xmax{3.5}
                         \def\ymin{-2.5}
                         \def\ymax{4.5}
                         \def\xunit{1}  % unités en cm
                         \def\yunit{1}
                         \psset{xunit=\xunit,yunit=\yunit,algebraic=true}
                         \fontsize{12pt}{12pt}\selectfont
                         \begin{pspicture*}[linewidth=1pt](\xmin,\ymin)(\xmax,\ymax)
                              %      \psgrid[gridcolor=mcgris, subgriddiv=5, gridlabels=0pt](\xmin,\ymin)(\xmax,\ymax)
                              \psaxes[linewidth=0.75pt,Dx=1,Dy=1]{->}(0,0)(\xmin,\ymin)(\xmax,\ymax)
                              \psplot[plotpoints=2000,linecolor=red]{\xmin}{\xmax}{\fonction}
                              \rput[tr](-0.3,-0.3){$O$}
                              \rput[tl](2.7,4){$\color{red} y=2x+1$}
                         \end{pspicture*}
                    }
                    \\
                    Droite d'équation $x=1$ & ~~~~ & Droite d'équation $y=2x-1$ %
                    \\
               \end{tabular}
          \end{extern}
     \end{center}
}
\bloc{cyan}{Remarques}{%
     \begin{itemize}
          \item L'équation d'une droite peut s'écrire sous plusieurs formes. Par exemple $y=2x-1$ est équivalente à $y-2x+1=0$ ou $2y-4x+2=0$, etc.
          \par
          Les formes $x=c$ et $y=mx+p$ sont appelées \textbf{équation réduite} de la droite.
          \item Cette propriété indique que toute droite qui n'est pas parallèle à l'axe des ordonnées est la représentation graphique d'une fonction affine.(Voir chapitre \mcLien{/cours/fonctions-lineaires-affines}{Fonctions linéaires et affines})
          \item Une droite parallèle à l'axe des abscisses a un coefficient direct $m$ égal à zéro. Son équation est donc de la forme $y=p$. C'est la représentation graphique d'une fonction constante.
     \end{itemize}
}
\cadre{vert}{Propriété}{%
     Soient $A$ et $B$ deux points du plan tels que $x_A\neq x_B$.
     \par
     Le coefficient directeur de la droite $\left(AB\right)$ est :
     \begin{center}$m = \frac{y_B-y_A}{x_B-x_A}$\end{center}
}
\bloc{cyan}{Remarque}{%
     Une fois que le coefficient directeur de la droite $\left(AB\right)$ est connu, on peut trouver l'ordonnée à l'origine en sachant que la droite $\left(AB\right)$ passe par le point $A$ donc que les coordonnées de $A$ vérifient l'équation de la droite.
}
\bloc{orange}{Exemple}{%
     On recherche l'équation de la droite passant par les points $A\left(1 ; 3\right)$ et $B\left(3 ; 5\right)$.
     \par
     Les points $A$ et $B$ n'ayant pas la même abscisse, cette équation est du type $y=mx+p$ avec :
     \par
     $m = \frac{y_B-y_A}{x_B-x_A}=\frac{5-3}{3-1}=\frac{2}{2}=1$
     \par
     Donc l'équation de $\left(AB\right)$ est de la forme $y=x+p$. Comme cette droite passe par $A$, l'équation est vérifiée si on remplace $x$ et $y$ par les coordonnées de $A$ donc :
     \par
     $3=1+p$ soit $p=2$.
     \par
     L'équation de $\left(AB\right)$ est donc $y=x+2$.
}
\begin{h2}2. Droites parallèles - Droites sécantes\end{h2}
\cadre{vert}{Propriété}{%
     Deux droites d'équations respectives $y=mx+p$ et $y=m^{\prime}x+p^{\prime}$ sont \textbf{parallèles} si et seulement si elles ont le même coefficient directeur : $m=m^{\prime}$.
}
\bloc{orange}{Exemple}{%
     \begin{center}
          \begin{extern}%width="320" alt="Droites parallèles"
               \resizebox{6cm}{!}{%
                    % -+-+-+ variables modifiables
                    \def\fonction{2*x-1 }
                    \def\g{2*x+3 }
                    \def\xmin{-4.5}
                    \def\xmax{3.5}
                    \def\ymin{-2.5}
                    \def\ymax{4.5}
                    \def\xunit{1}  % unités en cm
                    \def\yunit{1}
                    \psset{xunit=\xunit,yunit=\yunit,algebraic=true}
                    \fontsize{12pt}{12pt}\selectfont
                    \begin{pspicture*}[linewidth=1pt](\xmin,\ymin)(\xmax,\ymax)
                         %      \psgrid[gridcolor=mcgris, subgriddiv=5, gridlabels=0pt](\xmin,\ymin)(\xmax,\ymax)
                         \psaxes[linewidth=0.75pt,Dx=1,Dy=1]{->}(0,0)(\xmin,\ymin)(\xmax,\ymax)
                         \psplot[plotpoints=2000,linecolor=red]{\xmin}{\xmax}{\fonction}
                         \psplot[plotpoints=2000,linecolor=blue]{\xmin}{\xmax}{\g}
                         \rput[tr](-0.3,-0.3){$O$}
                         \rput[tl](2.6,4){$\color{red} y=2x-1$}
                         \rput[tr](-2.8,-2){$\color{blue} y=2x+3$}
                    \end{pspicture*}
               }
          \end{extern}
     \end{center}
     \begin{center}
          \'Equations de droites parallèles
     \end{center}
}
\cadre{vert}{Méthode}{%
     Soient $\mathscr D$ et $\mathscr D^{\prime}$ deux droites sécantes d'équations respectives $y=mx+p$ et $y=m^{\prime}x+p^{\prime}$.
     \par
     Les coordonnées $\left(x ; y\right)$ du point d'intersection des droites $\mathscr D$ et $\mathscr D^{\prime}$ s'obtiennent en résolvant le système :
     \par
     $\left\{ \begin{matrix} y=mx+p \\ y=m^{\prime}x+p^{\prime} \end{matrix}\right.$
}
\bloc{cyan}{Remarque}{%
     Ce système se résout simplement par substitution. Il est équivalent à :
     \par
     $\left\{ \begin{matrix} mx+p=m^{\prime}x+p^{\prime} \\ y=mx+p \end{matrix}\right.$
}
\bloc{orange}{Exemple}{%
     On cherche les coordonnées du point d'intersection des droites $\mathscr D$ et $\mathscr D^{\prime}$ d'équations respectives $y=2x+1$ et $y=3x-1$.
     \par
     Ces droites n'ont pas le même coefficient directeur donc elles sont sécantes.
     \par
     Les coordonnées du point d'intersection vérifient le système :
     \par
     $\left\{ \begin{matrix} y=2x+1 \\ y=3x-1 \end{matrix}\right.$
     \par
     qui équivaut à :
     \par
     $\left\{ \begin{matrix} y=2x+1 \\ y=3x-1 \end{matrix}\right. \Leftrightarrow \left\{ \begin{matrix} 2x+1=3x-1 \\ y=3x-1 \end{matrix}\right.$
     \par
     $\phantom{\left\{ \begin{matrix} y=2x+1 \\ y=3x-1 \end{matrix}\right.} \Leftrightarrow \left\{ \begin{matrix} x=2 \\ y=3x-1 \end{matrix}\right.$
     \par
     $\phantom{\left\{ \begin{matrix} y=2x+1 \\ y=3x-1 \end{matrix}\right.} \Leftrightarrow \left\{ \begin{matrix} x=2 \\ y=5 \end{matrix}\right.$
     \par
     Le point d'intersection a pour coordonnées $\left(2 ; 5\right)$.
}

\end{document}