\documentclass[a4paper]{article}

%================================================================================================================================
%
% Packages
%
%================================================================================================================================

\usepackage[T1]{fontenc} 	% pour caractères accentués
\usepackage[utf8]{inputenc}  % encodage utf8
\usepackage[french]{babel}	% langue : français
\usepackage{fourier}			% caractères plus lisibles
\usepackage[dvipsnames]{xcolor} % couleurs
\usepackage{fancyhdr}		% réglage header footer
\usepackage{needspace}		% empêcher sauts de page mal placés
\usepackage{graphicx}		% pour inclure des graphiques
\usepackage{enumitem,cprotect}		% personnalise les listes d'items (nécessaire pour ol, al ...)
\usepackage{hyperref}		% Liens hypertexte
\usepackage{pstricks,pst-all,pst-node,pstricks-add,pst-math,pst-plot,pst-tree,pst-eucl} % pstricks
\usepackage[a4paper,includeheadfoot,top=2cm,left=3cm, bottom=2cm,right=3cm]{geometry} % marges etc.
\usepackage{comment}			% commentaires multilignes
\usepackage{amsmath,environ} % maths (matrices, etc.)
\usepackage{amssymb,makeidx}
\usepackage{bm}				% bold maths
\usepackage{tabularx}		% tableaux
\usepackage{colortbl}		% tableaux en couleur
\usepackage{fontawesome}		% Fontawesome
\usepackage{environ}			% environment with command
\usepackage{fp}				% calculs pour ps-tricks
\usepackage{multido}			% pour ps tricks
\usepackage[np]{numprint}	% formattage nombre
\usepackage{tikz,tkz-tab} 			% package principal TikZ
\usepackage{pgfplots}   % axes
\usepackage{mathrsfs}    % cursives
\usepackage{calc}			% calcul taille boites
\usepackage[scaled=0.875]{helvet} % font sans serif
\usepackage{svg} % svg
\usepackage{scrextend} % local margin
\usepackage{scratch} %scratch
\usepackage{multicol} % colonnes
%\usepackage{infix-RPN,pst-func} % formule en notation polanaise inversée
\usepackage{listings}

%================================================================================================================================
%
% Réglages de base
%
%================================================================================================================================

\lstset{
language=Python,   % R code
literate=
{á}{{\'a}}1
{à}{{\`a}}1
{ã}{{\~a}}1
{é}{{\'e}}1
{è}{{\`e}}1
{ê}{{\^e}}1
{í}{{\'i}}1
{ó}{{\'o}}1
{õ}{{\~o}}1
{ú}{{\'u}}1
{ü}{{\"u}}1
{ç}{{\c{c}}}1
{~}{{ }}1
}


\definecolor{codegreen}{rgb}{0,0.6,0}
\definecolor{codegray}{rgb}{0.5,0.5,0.5}
\definecolor{codepurple}{rgb}{0.58,0,0.82}
\definecolor{backcolour}{rgb}{0.95,0.95,0.92}

\lstdefinestyle{mystyle}{
    backgroundcolor=\color{backcolour},   
    commentstyle=\color{codegreen},
    keywordstyle=\color{magenta},
    numberstyle=\tiny\color{codegray},
    stringstyle=\color{codepurple},
    basicstyle=\ttfamily\footnotesize,
    breakatwhitespace=false,         
    breaklines=true,                 
    captionpos=b,                    
    keepspaces=true,                 
    numbers=left,                    
xleftmargin=2em,
framexleftmargin=2em,            
    showspaces=false,                
    showstringspaces=false,
    showtabs=false,                  
    tabsize=2,
    upquote=true
}

\lstset{style=mystyle}


\lstset{style=mystyle}
\newcommand{\imgdir}{C:/laragon/www/newmc/assets/imgsvg/}
\newcommand{\imgsvgdir}{C:/laragon/www/newmc/assets/imgsvg/}

\definecolor{mcgris}{RGB}{220, 220, 220}% ancien~; pour compatibilité
\definecolor{mcbleu}{RGB}{52, 152, 219}
\definecolor{mcvert}{RGB}{125, 194, 70}
\definecolor{mcmauve}{RGB}{154, 0, 215}
\definecolor{mcorange}{RGB}{255, 96, 0}
\definecolor{mcturquoise}{RGB}{0, 153, 153}
\definecolor{mcrouge}{RGB}{255, 0, 0}
\definecolor{mclightvert}{RGB}{205, 234, 190}

\definecolor{gris}{RGB}{220, 220, 220}
\definecolor{bleu}{RGB}{52, 152, 219}
\definecolor{vert}{RGB}{125, 194, 70}
\definecolor{mauve}{RGB}{154, 0, 215}
\definecolor{orange}{RGB}{255, 96, 0}
\definecolor{turquoise}{RGB}{0, 153, 153}
\definecolor{rouge}{RGB}{255, 0, 0}
\definecolor{lightvert}{RGB}{205, 234, 190}
\setitemize[0]{label=\color{lightvert}  $\bullet$}

\pagestyle{fancy}
\renewcommand{\headrulewidth}{0.2pt}
\fancyhead[L]{maths-cours.fr}
\fancyhead[R]{\thepage}
\renewcommand{\footrulewidth}{0.2pt}
\fancyfoot[C]{}

\newcolumntype{C}{>{\centering\arraybackslash}X}
\newcolumntype{s}{>{\hsize=.35\hsize\arraybackslash}X}

\setlength{\parindent}{0pt}		 
\setlength{\parskip}{3mm}
\setlength{\headheight}{1cm}

\def\ebook{ebook}
\def\book{book}
\def\web{web}
\def\type{web}

\newcommand{\vect}[1]{\overrightarrow{\,\mathstrut#1\,}}

\def\Oij{$\left(\text{O}~;~\vect{\imath},~\vect{\jmath}\right)$}
\def\Oijk{$\left(\text{O}~;~\vect{\imath},~\vect{\jmath},~\vect{k}\right)$}
\def\Ouv{$\left(\text{O}~;~\vect{u},~\vect{v}\right)$}

\hypersetup{breaklinks=true, colorlinks = true, linkcolor = OliveGreen, urlcolor = OliveGreen, citecolor = OliveGreen, pdfauthor={Didier BONNEL - https://www.maths-cours.fr} } % supprime les bordures autour des liens

\renewcommand{\arg}[0]{\text{arg}}

\everymath{\displaystyle}

%================================================================================================================================
%
% Macros - Commandes
%
%================================================================================================================================

\newcommand\meta[2]{    			% Utilisé pour créer le post HTML.
	\def\titre{titre}
	\def\url{url}
	\def\arg{#1}
	\ifx\titre\arg
		\newcommand\maintitle{#2}
		\fancyhead[L]{#2}
		{\Large\sffamily \MakeUppercase{#2}}
		\vspace{1mm}\textcolor{mcvert}{\hrule}
	\fi 
	\ifx\url\arg
		\fancyfoot[L]{\href{https://www.maths-cours.fr#2}{\black \footnotesize{https://www.maths-cours.fr#2}}}
	\fi 
}


\newcommand\TitreC[1]{    		% Titre centré
     \needspace{3\baselineskip}
     \begin{center}\textbf{#1}\end{center}
}

\newcommand\newpar{    		% paragraphe
     \par
}

\newcommand\nosp {    		% commande vide (pas d'espace)
}
\newcommand{\id}[1]{} %ignore

\newcommand\boite[2]{				% Boite simple sans titre
	\vspace{5mm}
	\setlength{\fboxrule}{0.2mm}
	\setlength{\fboxsep}{5mm}	
	\fcolorbox{#1}{#1!3}{\makebox[\linewidth-2\fboxrule-2\fboxsep]{
  		\begin{minipage}[t]{\linewidth-2\fboxrule-4\fboxsep}\setlength{\parskip}{3mm}
  			 #2
  		\end{minipage}
	}}
	\vspace{5mm}
}

\newcommand\CBox[4]{				% Boites
	\vspace{5mm}
	\setlength{\fboxrule}{0.2mm}
	\setlength{\fboxsep}{5mm}
	
	\fcolorbox{#1}{#1!3}{\makebox[\linewidth-2\fboxrule-2\fboxsep]{
		\begin{minipage}[t]{1cm}\setlength{\parskip}{3mm}
	  		\textcolor{#1}{\LARGE{#2}}    
 	 	\end{minipage}  
  		\begin{minipage}[t]{\linewidth-2\fboxrule-4\fboxsep}\setlength{\parskip}{3mm}
			\raisebox{1.2mm}{\normalsize\sffamily{\textcolor{#1}{#3}}}						
  			 #4
  		\end{minipage}
	}}
	\vspace{5mm}
}

\newcommand\cadre[3]{				% Boites convertible html
	\par
	\vspace{2mm}
	\setlength{\fboxrule}{0.1mm}
	\setlength{\fboxsep}{5mm}
	\fcolorbox{#1}{white}{\makebox[\linewidth-2\fboxrule-2\fboxsep]{
  		\begin{minipage}[t]{\linewidth-2\fboxrule-4\fboxsep}\setlength{\parskip}{3mm}
			\raisebox{-2.5mm}{\sffamily \small{\textcolor{#1}{\MakeUppercase{#2}}}}		
			\par		
  			 #3
 	 		\end{minipage}
	}}
		\vspace{2mm}
	\par
}

\newcommand\bloc[3]{				% Boites convertible html sans bordure
     \needspace{2\baselineskip}
     {\sffamily \small{\textcolor{#1}{\MakeUppercase{#2}}}}    
		\par		
  			 #3
		\par
}

\newcommand\CHelp[1]{
     \CBox{Plum}{\faInfoCircle}{À RETENIR}{#1}
}

\newcommand\CUp[1]{
     \CBox{NavyBlue}{\faThumbsOUp}{EN PRATIQUE}{#1}
}

\newcommand\CInfo[1]{
     \CBox{Sepia}{\faArrowCircleRight}{REMARQUE}{#1}
}

\newcommand\CRedac[1]{
     \CBox{PineGreen}{\faEdit}{BIEN R\'EDIGER}{#1}
}

\newcommand\CError[1]{
     \CBox{Red}{\faExclamationTriangle}{ATTENTION}{#1}
}

\newcommand\TitreExo[2]{
\needspace{4\baselineskip}
 {\sffamily\large EXERCICE #1\ (\emph{#2 points})}
\vspace{5mm}
}

\newcommand\img[2]{
          \includegraphics[width=#2\paperwidth]{\imgdir#1}
}

\newcommand\imgsvg[2]{
       \begin{center}   \includegraphics[width=#2\paperwidth]{\imgsvgdir#1} \end{center}
}


\newcommand\Lien[2]{
     \href{#1}{#2 \tiny \faExternalLink}
}
\newcommand\mcLien[2]{
     \href{https~://www.maths-cours.fr/#1}{#2 \tiny \faExternalLink}
}

\newcommand{\euro}{\eurologo{}}

%================================================================================================================================
%
% Macros - Environement
%
%================================================================================================================================

\newenvironment{tex}{ %
}
{%
}

\newenvironment{indente}{ %
	\setlength\parindent{10mm}
}

{
	\setlength\parindent{0mm}
}

\newenvironment{corrige}{%
     \needspace{3\baselineskip}
     \medskip
     \textbf{\textsc{Corrigé}}
     \medskip
}
{
}

\newenvironment{extern}{%
     \begin{center}
     }
     {
     \end{center}
}

\NewEnviron{code}{%
	\par
     \boite{gray}{\texttt{%
     \BODY
     }}
     \par
}

\newenvironment{vbloc}{% boite sans cadre empeche saut de page
     \begin{minipage}[t]{\linewidth}
     }
     {
     \end{minipage}
}
\NewEnviron{h2}{%
    \needspace{3\baselineskip}
    \vspace{0.6cm}
	\noindent \MakeUppercase{\sffamily \large \BODY}
	\vspace{1mm}\textcolor{mcgris}{\hrule}\vspace{0.4cm}
	\par
}{}

\NewEnviron{h3}{%
    \needspace{3\baselineskip}
	\vspace{5mm}
	\textsc{\BODY}
	\par
}

\NewEnviron{margeneg}{ %
\begin{addmargin}[-1cm]{0cm}
\BODY
\end{addmargin}
}

\NewEnviron{html}{%
}

\begin{document}
\meta{url}{/exercices/suites-bac-s-asie-2013/}
\meta{pid}{1439}
\meta{titre}{Suites - Bac S  Asie 2013}
\meta{type}{exercices}
%
\begin{h2}Exercice 4   5 points\end{h2}
\textit{Candidats n'ayant pas choisi l'enseignement de spécialité}
\begin{h3}Partie A\end{h3}
On considère la suite $\left(u_{n}\right)$ définie par $u_{0}=2$ et, pour tout entier naturel $n$ :
\begin{center}$u_{n+1}=\frac{1+3u_{n}}{3+u_{n}}.$\end{center}
On admet que tous les termes de cette suite sont définis et strictement positifs.
\begin{enumerate}
     \item
     Démontrer par récurrence que, pour tout entier naturel $n$, on a :
     \begin{center}$u_{n} > 1$\end{center}
     \item
     \begin{enumerate}[label=\alph*.]
          \item
          Établir que, pour tout entier naturel $n$, on a :
          \begin{center}$u_{n+1}- u_{n}=\frac{\left(1-u_{n} \right)\left(1+u_{n}\right)}{3+ u_{n}}$.\end{center}
          \item
          Déterminer le sens de variation de la suite $\left(u_{n}\right)$.
          \par
          En déduire que la suite $\left(u_{n}\right)$ converge.
     \end{enumerate}
\end{enumerate}
\begin{h3}Partie B\end{h3}
On considère la suite $\left(u_{n}\right)$ définie par : $u_{0}=2$ et, pour tout entier naturel $n$ :
\begin{center}$ u_{n+1}=\frac{1+0,5u_{n}}{0,5+u_{n}}.$\end{center}
On admet que tous les termes de cette suite sont définis et strictement positifs.
\begin{enumerate}
     \item
     On considère l'algorithme suivant :
\begin{tabularx}{0.8\linewidth}{|*{3}{>{\centering \arraybackslash }X|}}%class="compact" width="600"
          \hline
          Entrée &  Soit un entier naturel non nul $n$
          \\ \hline
          Initialisation  & Affecter à $u$ la valeur 2
          \\ \hline
          Traitement et sortie & POUR $i$ allant de 1 à $n$
          \\ \hline
          & <span style="color:transparent">...</span>Affecter à $u$ la valeur $\frac{1+0,5u}{0,5+u}$
          \\ \hline
          & <span style="color:transparent">...</span>Afficher $u$
          \\ \hline
          & FIN POUR
          \\ \hline
     \end{tabularx}

Reproduire et compléter le tableau suivant, en faisant fonctionner cet algorithme pour $n=3$. Les valeurs de $u$ seront arrondies au millième.
\begin{tabularx}{0.8\linewidth}{|*{3}{>{\centering \arraybackslash }X|}}%class="compact" width="600"
     \hline
       $i$   &   1   &   2   &    3  
     \\ \hline
       $u$   &     &  &
     \\ \hline
\end{tabularx}
\item
Pour $n=12$, on a prolongé le tableau précédent et on a obtenu :
\begin{tabularx}{0.8\linewidth}{|*{3}{>{\centering \arraybackslash }X|}}%class="compact" width="600"
     \hline
     $i$ & 4 & 5 & 6 & 7 & 8 & 9 & 10 & 11 & 12
     \\ \hline
     $u$ & 1,0083 & 0,9973 & 1,0009 & 0,9997 & 1,0001 & 0,99997 & 1,00001 & 0,999996 & 1,000001
     \\ \hline
\end{tabularx}
Conjecturer le comportement de la suite $\left(u_{n}\right)$ à l'infini.
\item
On considère la suite $\left(v_{n}\right)$ définie, pour tout entier naturel $n$, par : $v_{n}=\frac{u_{n}-1}{u_{n}+1}$.
\begin{enumerate}[label=\alph*.]
     \item
     Démontrer que la suite $\left(v_{n}\right)$ est géométrique de raison $-\frac{1}{3}$.
     \item
     Calculer $v_{0}$ puis écrire $v_{n}$ en fonction de $n$.
     \end{enumerate}
          \item 
\begin{enumerate}[label=\alph*.]
  \item 
     Montrer que, pour tout entier naturel $n$, on a : $v_{n} \neq  1$.
     \item
     Montrer que, pour tout entier naturel $n$, on a : $u_{n}=\frac{1+v_{n}}{1-v_{n}}$.
     \item
     Déterminer la limite de la suite $\left(u_{n}\right)$.
\end{enumerate}
\end{enumerate}
\begin{corrige}
     \begin{h3}Partie A\end{h3}
     \begin{enumerate}
          \item
          Soit $P_{n}$ la propriété «$u_{n} > 1$»
          \textbf{Initialisation :}
          $u_{0}=2 > 1$ donc $P_{0}$ est vraie.
          \textbf{Hérédité}
          Supposons que $P_{n}$ soit vraie pour un entier $n$ fixé. Alors :
          \par
          $u_{n+1}-1=\frac{1+3u_{n}}{3+u_{n}}-1=\frac{1+3u_{n}}{3+u_{n}}-\frac{3+u_{n}}{3+u_{n}}=\frac{-2+2u_{n}}{3+u_{n}}=\frac{2\left(u_{n}-1\right)}{3+u_{n}}$
          \par
          Comme $u_{n} > 1$ par hypothèse de récurrence, le numérateur et le dénominateur sont strictement positifs donc  $u_{n+1}-1 > 0$ donc $u_{n+1} > 1$ ce qui prouve l'hérédité.
          \par
          Par conséquent, $u_{n} > 1$  pour tout entier naturel $n$.
          \item
          \begin{enumerate}[label=\alph*.]
               \item
               $u_{n+1}- u_{n}=\frac{1+3u_{n}}{3+u_{n}}-u_{n}=\frac{1+3u_{n}}{3+u_{n}}-\frac{u_{n}\left(3+u_{n}\right)}{3+u_{n}}=\frac{1-u_{n}^{2}}{3+u_{n}}$
               \par
               $u_{n+1}- u_{n}=\frac{\left(1-u_{n} \right)\left(1+u_{n}\right)}{3+ u_{n}}$
               \item
               D'après la question \textbf{1.} $u_{n} > 1$  pour tout entier naturel $n$. Par conséquent :
               \par
               ♦  $1-u_{n} < 0$
               \par
               ♦  $1+u_{n} > 0$
               \par
               ♦  $3+u_{n} > 0$
               \par
               $u_{n+1}- u_{n}$ est donc strictement négatif pour tout entier $n$. Par conséquent, la suite $\left(u_{n}\right)$ est strictement décroissante.
               \par
               La suite $\left(u_{n}\right)$ est décroissante et minorée par $1$ donc convergente (voir \mcLien{/terminale-s/variations-convergence-suite#t80}{cours})
          \end{enumerate}
     \end{enumerate}
     \begin{h3}Partie B\end{h3}
     \begin{enumerate}
          \item
          A la calculatrice (en utilisant le menu \textit{Suites}) on trouve :
          \begin{tabularx}{0.8\linewidth}{|*{3}{>{\centering \arraybackslash }X|}}%class="compact" width="600"
               \hline
               $i$   &   1   &   2   &    3
               \\ \hline
               $u$   & 0,8    & 1,077 & 0,976
               \\ \hline
          \end{tabularx}
          \item
          La suite $\left(u_{n}\right)$ semble converger vers $1$.
          \item
          \begin{enumerate}
               \item
               $v_{n+1} = \frac{u_{n+1}-1}{u_{n+1}+1} = \frac{\frac{1+0,5u_{n}}{0,5+u_{n}}-1}{\frac{1+0,5u_{n}}{0,5+u_{n}}+1}=\frac{\frac{-0,5u_{n}-0,5}{0,5+u_{n}}}{\frac{1,5u_{n}+1,5}{0,5+u_{n}}}=\frac{-0,5u_{n}-0,5}{0,5+u_{n}}\times \frac{0,5+u_{n}}{1,5u_{n}+1,5}$
               \par
               $v_{n+1} =\frac{-0,5u_{n}-0,5}{1,5u_{n}+1,5}=-\frac{0,5}{1,5}\times \frac{u_{n}-1}{u_{n}+1}=-\frac{1}{3}v_{n}$
               \par
               Donc, la suite $\left(v_{n}\right)$ est une suite géométrique de raison $-\frac{1}{3}$
               \item
               $v_{0}=\frac{u_{0}-1}{u_{0}+1}=\frac{1}{3}$
               \par
               Par conséquent :
               \par
               $v_{n}=v_{0}\times \left(-\frac{1}{3}\right)^{n}=\frac{1}{3}\times \left(-\frac{1}{3}\right)^{n}$
               \par
               (Remarque : le résultat peut aussi s'écrire $-\left(-\frac{1}{3}\right)^{n+1}$)
          \end{enumerate}
          \item
          \begin{enumerate}
               \item
               Pour tout entier $n$, $\left(-\frac{1}{3}\right)^{n} \leqslant  1$ donc $v_{n} \leqslant  \frac{1}{3}$ et par conséquent $v_{n}\neq 1$
               \item
               $v_{n}=\frac{u_{n}-1}{u_{n}+1}$ équivaut à :
               \par
               $v_{n}\left(u_{n}+1\right)=u_{n}-1$
               \par
               $v_{n}u_{n}+v_{n}=u_{n}-1$
               \par
               $v_{n}u_{n}-u_{n}=-v_{n}-1$
               \par
               $-v_{n}u_{n}+u_{n}=v_{n}+1$
               \par
               $u_{n}\left(1-v_{n}\right)=v_{n}+1$
               \par
               $u_{n}=\frac{1+v_{n}}{1-v_{n}}$ car pour tout $n \in  \mathbb{N}$, $v_{n}\neq 1$
               \item
               $v_{n}$ est une suite géométrique dont la raison est strictement inférieure à $1$ en valeur absolue.
               \par
               La suite $\left(v_{n}\right)$ converge donc vers $0$ (voir \mcLien{/terminale-s/variations-convergence-suite#t90}{limite d'une suite géométrique}).
               \par
               D'après la formule $u_{n}=\frac{1+v_{n}}{1-v_{n}}$ et les règles de calcul sur les limites, la suite $\left(u_{n}\right)$ converge donc vers $1$.
          \end{enumerate}
     \end{enumerate}
\end{corrige}

\end{document}