\alpha\documentclass[a4paper]{article}

%================================================================================================================================
%
% Packages
%
%================================================================================================================================

\usepackage[T1]{fontenc} 	% pour caractères accentués
\usepackage[utf8]{inputenc}  % encodage utf8
\usepackage[french]{babel}	% langue : français
\usepackage{fourier}			% caractères plus lisibles
\usepackage[dvipsnames]{xcolor} % couleurs
\usepackage{fancyhdr}		% réglage header footer
\usepackage{needspace}		% empêcher sauts de page mal placés
\usepackage{graphicx}		% pour inclure des graphiques
\usepackage{enumitem,cprotect}		% personnalise les listes d'items (nécessaire pour ol, al ...)
\usepackage{hyperref}		% Liens hypertexte
\usepackage{pstricks,pst-all,pst-node,pstricks-add,pst-math,pst-plot,pst-tree,pst-eucl} % pstricks
\usepackage[a4paper,includeheadfoot,top=2cm,left=3cm, bottom=2cm,right=3cm]{geometry} % marges etc.
\usepackage{comment}			% commentaires multilignes
\usepackage{amsmath,environ} % maths (matrices, etc.)
\usepackage{amssymb,makeidx}
\usepackage{bm}				% bold maths
\usepackage{tabularx}		% tableaux
\usepackage{colortbl}		% tableaux en couleur
\usepackage{fontawesome}		% Fontawesome
\usepackage{environ}			% environment with command
\usepackage{fp}				% calculs pour ps-tricks
\usepackage{multido}			% pour ps tricks
\usepackage[np]{numprint}	% formattage nombre
\usepackage{tikz,tkz-tab} 			% package principal TikZ
\usepackage{pgfplots}   % axes
\usepackage{mathrsfs}    % cursives
\usepackage{calc}			% calcul taille boites
\usepackage[scaled=0.875]{helvet} % font sans serif
\usepackage{svg} % svg
\usepackage{scrextend} % local margin
\usepackage{scratch} %scratch
\usepackage{multicol} % colonnes
%\usepackage{infix-RPN,pst-func} % formule en notation polanaise inversée
\usepackage{listings}

%================================================================================================================================
%
% Réglages de base
%
%================================================================================================================================

\lstset{
language=Python,   % R code
literate=
{á}{{\'a}}1
{à}{{\`a}}1
{ã}{{\~a}}1
{é}{{\'e}}1
{è}{{\`e}}1
{ê}{{\^e}}1
{í}{{\'i}}1
{ó}{{\'o}}1
{õ}{{\~o}}1
{ú}{{\'u}}1
{ü}{{\"u}}1
{ç}{{\c{c}}}1
{~}{{ }}1
}


\definecolor{codegreen}{rgb}{0,0.6,0}
\definecolor{codegray}{rgb}{0.5,0.5,0.5}
\definecolor{codepurple}{rgb}{0.58,0,0.82}
\definecolor{backcolour}{rgb}{0.95,0.95,0.92}

\lstdefinestyle{mystyle}{
    backgroundcolor=\color{backcolour},   
    commentstyle=\color{codegreen},
    keywordstyle=\color{magenta},
    numberstyle=\tiny\color{codegray},
    stringstyle=\color{codepurple},
    basicstyle=\ttfamily\footnotesize,
    breakatwhitespace=false,         
    breaklines=true,                 
    captionpos=b,                    
    keepspaces=true,                 
    numbers=left,                    
xleftmargin=2em,
framexleftmargin=2em,            
    showspaces=false,                
    showstringspaces=false,
    showtabs=false,                  
    tabsize=2,
    upquote=true
}

\lstset{style=mystyle}


\lstset{style=mystyle}
\newcommand{\imgdir}{C:/laragon/www/newmc/assets/imgsvg/}
\newcommand{\imgsvgdir}{C:/laragon/www/newmc/assets/imgsvg/}

\definecolor{mcgris}{RGB}{220, 220, 220}% ancien~; pour compatibilité
\definecolor{mcbleu}{RGB}{52, 152, 219}
\definecolor{mcvert}{RGB}{125, 194, 70}
\definecolor{mcmauve}{RGB}{154, 0, 215}
\definecolor{mcorange}{RGB}{255, 96, 0}
\definecolor{mcturquoise}{RGB}{0, 153, 153}
\definecolor{mcrouge}{RGB}{255, 0, 0}
\definecolor{mclightvert}{RGB}{205, 234, 190}

\definecolor{gris}{RGB}{220, 220, 220}
\definecolor{bleu}{RGB}{52, 152, 219}
\definecolor{vert}{RGB}{125, 194, 70}
\definecolor{mauve}{RGB}{154, 0, 215}
\definecolor{orange}{RGB}{255, 96, 0}
\definecolor{turquoise}{RGB}{0, 153, 153}
\definecolor{rouge}{RGB}{255, 0, 0}
\definecolor{lightvert}{RGB}{205, 234, 190}
\setitemize[0]{label=\color{lightvert}  $\bullet$}

\pagestyle{fancy}
\renewcommand{\headrulewidth}{0.2pt}
\fancyhead[L]{maths-cours.fr}
\fancyhead[R]{\thepage}
\renewcommand{\footrulewidth}{0.2pt}
\fancyfoot[C]{}

\newcolumntype{C}{>{\centering\arraybackslash}X}
\newcolumntype{s}{>{\hsize=.35\hsize\arraybackslash}X}

\setlength{\parindent}{0pt}		 
\setlength{\parskip}{3mm}
\setlength{\headheight}{1cm}

\def\ebook{ebook}
\def\book{book}
\def\web{web}
\def\type{web}

\newcommand{\vect}[1]{\overrightarrow{\,\mathstrut#1\,}}

\def\Oij{$\left(\text{O}~;~\vect{\imath},~\vect{\jmath}\right)$}
\def\Oijk{$\left(\text{O}~;~\vect{\imath},~\vect{\jmath},~\vect{k}\right)$}
\def\Ouv{$\left(\text{O}~;~\vect{u},~\vect{v}\right)$}

\hypersetup{breaklinks=true, colorlinks = true, linkcolor = OliveGreen, urlcolor = OliveGreen, citecolor = OliveGreen, pdfauthor={Didier BONNEL - https://www.maths-cours.fr} } % supprime les bordures autour des liens

\renewcommand{\arg}[0]{\text{arg}}

\everymath{\displaystyle}

%================================================================================================================================
%
% Macros - Commandes
%
%================================================================================================================================

\newcommand\meta[2]{    			% Utilisé pour créer le post HTML.
	\def\titre{titre}
	\def\url{url}
	\def\arg{#1}
	\ifx\titre\arg
		\newcommand\maintitle{#2}
		\fancyhead[L]{#2}
		{\Large\sffamily \MakeUppercase{#2}}
		\vspace{1mm}\textcolor{mcvert}{\hrule}
	\fi 
	\ifx\url\arg
		\fancyfoot[L]{\href{https://www.maths-cours.fr#2}{\black \footnotesize{https://www.maths-cours.fr#2}}}
	\fi 
}


\newcommand\TitreC[1]{    		% Titre centré
     \needspace{3\baselineskip}
     \begin{center}\textbf{#1}\end{center}
}

\newcommand\newpar{    		% paragraphe
     \par
}

\newcommand\nosp {    		% commande vide (pas d'espace)
}
\newcommand{\id}[1]{} %ignore

\newcommand\boite[2]{				% Boite simple sans titre
	\vspace{5mm}
	\setlength{\fboxrule}{0.2mm}
	\setlength{\fboxsep}{5mm}	
	\fcolorbox{#1}{#1!3}{\makebox[\linewidth-2\fboxrule-2\fboxsep]{
  		\begin{minipage}[t]{\linewidth-2\fboxrule-4\fboxsep}\setlength{\parskip}{3mm}
  			 #2
  		\end{minipage}
	}}
	\vspace{5mm}
}

\newcommand\CBox[4]{				% Boites
	\vspace{5mm}
	\setlength{\fboxrule}{0.2mm}
	\setlength{\fboxsep}{5mm}
	
	\fcolorbox{#1}{#1!3}{\makebox[\linewidth-2\fboxrule-2\fboxsep]{
		\begin{minipage}[t]{1cm}\setlength{\parskip}{3mm}
	  		\textcolor{#1}{\LARGE{#2}}    
 	 	\end{minipage}  
  		\begin{minipage}[t]{\linewidth-2\fboxrule-4\fboxsep}\setlength{\parskip}{3mm}
			\raisebox{1.2mm}{\normalsize\sffamily{\textcolor{#1}{#3}}}						
  			 #4
  		\end{minipage}
	}}
	\vspace{5mm}
}

\newcommand\cadre[3]{				% Boites convertible html
	\par
	\vspace{2mm}
	\setlength{\fboxrule}{0.1mm}
	\setlength{\fboxsep}{5mm}
	\fcolorbox{#1}{white}{\makebox[\linewidth-2\fboxrule-2\fboxsep]{
  		\begin{minipage}[t]{\linewidth-2\fboxrule-4\fboxsep}\setlength{\parskip}{3mm}
			\raisebox{-2.5mm}{\sffamily \small{\textcolor{#1}{\MakeUppercase{#2}}}}		
			\par		
  			 #3
 	 		\end{minipage}
	}}
		\vspace{2mm}
	\par
}

\newcommand\bloc[3]{				% Boites convertible html sans bordure
     \needspace{2\baselineskip}
     {\sffamily \small{\textcolor{#1}{\MakeUppercase{#2}}}}    
		\par		
  			 #3
		\par
}

\newcommand\CHelp[1]{
     \CBox{Plum}{\faInfoCircle}{À RETENIR}{#1}
}

\newcommand\CUp[1]{
     \CBox{NavyBlue}{\faThumbsOUp}{EN PRATIQUE}{#1}
}

\newcommand\CInfo[1]{
     \CBox{Sepia}{\faArrowCircleRight}{REMARQUE}{#1}
}

\newcommand\CRedac[1]{
     \CBox{PineGreen}{\faEdit}{BIEN R\'EDIGER}{#1}
}

\newcommand\CError[1]{
     \CBox{Red}{\faExclamationTriangle}{ATTENTION}{#1}
}

\newcommand\TitreExo[2]{
\needspace{4\baselineskip}
 {\sffamily\large EXERCICE #1\ (\emph{#2 points})}
\vspace{5mm}
}

\newcommand\img[2]{
          \includegraphics[width=#2\paperwidth]{\imgdir#1}
}

\newcommand\imgsvg[2]{
       \begin{center}   \includegraphics[width=#2\paperwidth]{\imgsvgdir#1} \end{center}
}


\newcommand\Lien[2]{
     \href{#1}{#2 \tiny \faExternalLink}
}
\newcommand\mcLien[2]{
     \href{https~://www.maths-cours.fr/#1}{#2 \tiny \faExternalLink}
}

\newcommand{\euro}{\eurologo{}}

%================================================================================================================================
%
% Macros - Environement
%
%================================================================================================================================

\newenvironment{tex}{ %
}
{%
}

\newenvironment{indente}{ %
	\setlength\parindent{10mm}
}

{
	\setlength\parindent{0mm}
}

\newenvironment{corrige}{%
     \needspace{3\baselineskip}
     \medskip
     \textbf{\textsc{Corrigé}}
     \medskip
}
{
}

\newenvironment{extern}{%
     \begin{center}
     }
     {
     \end{center}
}

\NewEnviron{code}{%
	\par
     \boite{gray}{\texttt{%
     \BODY
     }}
     \par
}

\newenvironment{vbloc}{% boite sans cadre empeche saut de page
     \begin{minipage}[t]{\linewidth}
     }
     {
     \end{minipage}
}
\NewEnviron{h2}{%
    \needspace{3\baselineskip}
    \vspace{0.6cm}
	\noindent \MakeUppercase{\sffamily \large \BODY}
	\vspace{1mm}\textcolor{mcgris}{\hrule}\vspace{0.4cm}
	\par
}{}

\NewEnviron{h3}{%
    \needspace{3\baselineskip}
	\vspace{5mm}
	\textsc{\BODY}
	\par
}

\NewEnviron{margeneg}{ %
\begin{addmargin}[-1cm]{0cm}
\BODY
\end{addmargin}
}

\NewEnviron{html}{%
}

\begin{document}
\meta{url}{/exercices/fonction-trigonometrique-bac/}
\meta{pid}{1357}
\meta{titre}{[Bac] Etude d'une fonction trigonométrique}
\meta{type}{exercices}
%
\textit{(d'après Bac S Nouvelle Calédonie 2005 - Sujet modifié pour être conforme au programme actuel)}
\par
Un lapin désire traverser une route de $4$ mètres de largeur. Un camion, occupant toute la route, arrive à sa rencontre à la vitesse de $60$ km/h. Le lapin décide au dernier moment de traverser, alors que le camion n'est plus qu'à $7$ mètres de lui. Son démarrage est foudroyant et on suppose qu'il effectue la traversée en ligne droite au maximum de ses possibilités, c'est à dire à ...  $30$ km/h !
\par
L'avant du camion est représenté par le segment $\left[CC^{\prime}\right]$ sur le schéma ci-dessous.

\begin{center}
\imgsvg{mc-0341}{0.3}% alt="Bac S Nouvelle Calédonie 2005" style="width:40rem"
\end{center}
\\
Le lapin part du point $A$ en direction de $D$.
\par
Cette direction est repérée par l'angle $\theta  =\widehat{BAD}$ avec $0 \leqslant  \theta  < \frac{\pi }{2}$(en radians).
\begin{enumerate}
     \item
     Déterminer les distances $AD$ et $CD$ en fonction de $\theta $ et   les temps $t_{1}$ et $t_{2}$ mis par le lapin et le camion pour  parcourir respectivement les distances $AD$ et $CD$.
     \item
     On pose $f\left(\theta \right)=\frac{7}{2}+\frac{2 \sin \theta -4}{\cos \theta }$.
     \par
     Montrer que le lapin aura traversé la route avant le passage du camion si et seulement si $f\left(\theta \right) > 0$.
     \item
     Etudier la fonction $f$ sur l'intervalle $\left[0 ; \frac{\pi }{2}\right[$.
     \par
     Conclure.
\end{enumerate}
\begin{corrige}
     \begin{enumerate}
          \item
          Le triangle $ABC$ étant rectangle en $B$ :
          \par
          $\cos \theta =\frac{AB}{AD}$ donc $AD=\frac{AB}{\cos \theta }=\frac{4}{\cos \theta }$
          \par
          $\sin \theta =\frac{BD}{AD}$ donc $BD=AD \sin \theta = \frac{4 \sin \theta }{\cos \theta }$
          \par
          $CD=CB+BD=7+\frac{4 \sin \theta }{\cos \theta }$
          \par
          $60$km/h=$1 000$m/minute et $30$km/h=$500$m/minute
          \par
          Le temps, en minutes, mis par le lapin pour parcourir $AD$ est donc :
          \par
          $t_{1}=\frac{1}{500}\left(\frac{4}{\cos \theta }\right)$
          \par
          Le temps, en minutes, mis par le camion pour parcourir $CD$ :
          \par
          $t_{2}= \frac{1}{1000}\left(7+\frac{4 \sin \theta }{\cos \theta }\right)$
          \item
          Le lapin aura traversé la route avant le passage du camion si et seulement si $t_{2} > t_{1}$, c'est à dire $t_{2}-t_{1} > 0$. Or :
          \par
          $t_{2}-t_{1}=\frac{1}{1000}\left(7+\frac{4 \sin \theta }{\cos \theta }\right)-\frac{1}{500}\left(\frac{4}{\cos \theta }\right)$
          \par
          $t_{2}-t_{1}=\frac{1}{500}\left(\frac{1}{2}\left(7+\frac{4 \sin \theta }{\cos \theta }\right)-\frac{4}{\cos \theta }\right)$
          \par
          $t_{2}-t_{1}=\frac{1}{500}\left(\frac{7}{2}+\frac{2 \sin \theta }{\cos \theta }-\frac{4}{\cos \theta }\right)$
          \par
          $t_{2}-t_{1}=\frac{1}{500}f\left(\theta \right)$
          \par
          Donc le lapin aura traversé la route avant le passage du camion si et seulement si $f\left(\theta \right) > 0$.
          \item
          On pose $u\left(\theta \right)=2\sin \theta -4$ et $v\left(\theta \right)=\cos\left(\theta \right)$
          \par
          $f^{\prime}\left(\theta \right)=\frac{u^{\prime}\left(\theta \right)v\left(\theta \right)-u\left(\theta \right)v^{\prime}\left(\theta \right)}{v\left(\theta \right)^{2}}=\frac{2\cos^{2}\theta -\left(2\sin \theta -4\right)\times -\sin \theta }{\cos^{2}\theta }$
          \par
          $f^{\prime}\left(\theta \right)=\frac{2\left(\cos^{2}\theta +\sin^{2}\theta \right)-4\sin \theta }{\cos^{2}\theta }=\frac{2-4\sin \theta }{\cos^{2}\theta }=\frac{2\left(1-2\sin \theta \right)}{\cos^{2}\theta }$
          \par
          $f^{\prime}\left(\theta \right) > 0  \Leftrightarrow   1-2\sin \theta  > 0  \Leftrightarrow  -2\sin \theta  > -1 \Leftrightarrow  \sin \theta  < \frac{1}{2}  \Leftrightarrow   \sin \theta  < \sin \frac{\pi }{6}$
          \par
          Or la fonction sinus étant strictement croissante sur l'intervalle $\left[0 ; \frac{\pi }{2}\right[$, $\sin \theta  < \sin \frac{\pi }{6}  \Leftrightarrow   \theta  < \frac{\pi }{6}$
          \par
          $f\left(0\right)=\frac{7}{2}-4=-\frac{1}{2}$
          \par
          Lorsque $\theta $ tend vers $\frac{\pi }{2}$ en restant inférieur à $\frac{\pi }{2}$, $\cos \theta $ tend vers zéro en restant positif et $2 \sin \theta -4$ est négatif donc, par quotient :
          \par
          $\lim\limits_{\theta \rightarrow \frac{\pi }{2}^-}\frac{2 \sin \theta -4}{\cos \theta }=-\infty $
          \par
          et par somme :
          \par
          $\lim\limits_{\theta \rightarrow \frac{\pi }{2}^-}f\left(x\right)=-\infty $
          \par
          On en déduit le tableau de variation de la fonction $f$ :
%##
% type=table; width=      25; l3=20
%--
% x|  0  ~    \frac{ \pi }{ 6 }   ~   \frac{ \pi }{ 2 }  
% f'(x)|  ~       +               :0        -     ~
% f(x)|   -\frac{1}{2}  /   :f \left(\frac{\pi}{6}\right)    \   -\infty
%--
\begin{center}
 \begin{extern}%style="width:25rem" alt="Exercice"
    \resizebox{11cm}{!}{
       \definecolor{dark}{gray}{0.1}
       \definecolor{light}{gray}{0.6}
       \tikzstyle{fleche}=[->,>=latex]
       \begin{tikzpicture}[scale=.8, line width=.5pt, dark]
       \def\width{.15}
       \def\height{.10}
       \draw (0, -10*\height) -- (54*\width, -10*\height);
       \draw (10*\width, 0*\height) -- (10*\width, -10*\height);
       \node (l0c0) at (5*\width,-5*\height) {$ x $};
       \node (l0c1) at (14*\width,-5*\height) {$ 0 $};
       \node (l0c2) at (23*\width,-5*\height) {$ ~ $};
       \node (l0c3) at (32*\width,-5*\height) {$ \frac{ \pi }{ 6 } $};
       \node (l0c4) at (41*\width,-5*\height) {$ ~ $};
       \node (l0c5) at (50*\width,-5*\height) {$ \frac{ \pi }{ 2 } $};
       \draw (0, -20*\height) -- (54*\width, -20*\height);
       \draw (10*\width, -10*\height) -- (10*\width, -20*\height);
       \node (l1c0) at (5*\width,-15*\height) {$ f'(x) $};
       \node (l1c1) at (14*\width,-15*\height) {$ ~ $};
       \node (l1c2) at (23*\width,-15*\height) {$ + $};
       \draw[light] (32*\width, -10*\height) -- (32*\width, -20*\height);
       \node (l1c3) at (32*\width,-15*\height) {$ 0 $};
       \node (l1c4) at (41*\width,-15*\height) {$ - $};
       \node (l1c5) at (50*\width,-15*\height) {$ ~ $};
       \draw (0, -40*\height) -- (54*\width, -40*\height);
       \draw (10*\width, -20*\height) -- (10*\width, -40*\height);
       \node (l2c0) at (5*\width,-30*\height) {$ f(x) $};
       \node (l2c1) at (14*\width,-35*\height) {$ -\frac{1}{2} $};
       \node (l2c2) at (23*\width,-30*\height) {$ ~ $};
       \draw[light] (32*\width, -20*\height) -- (32*\width, -40*\height);
       \node (l2c3) at (32*\width,-25*\height) {$ f \left(\frac{\pi}{6}\right) $};
       \node (l2c4) at (41*\width,-30*\height) {$ ~ $};
       \node (l2c5) at (50*\width,-35*\height) {$ -\infty $};
       \draw (0, 0) rectangle (54*\width, -40*\height);
       \draw[fleche] (l2c1) -- (l2c3);
       \draw[fleche] (l2c3) -- (l2c5);
       \end{tikzpicture}
      }
   \end{extern}
\end{center}
%##
et sa courbe représentative  :

\begin{center}
 \begin{extern}%style="width:30rem" alt="Fonction"
    \resizebox{11cm}{!}{
       \definecolor{dark}{gray}{0.1}
       \definecolor{light}{gray}{0.8}
       \def\xmin{-0.3}
       \def\xmax{1.2}
       \def\ymin{-0.9}
       \def\ymax{0.6}
       \def\xunit{5}
       \def\yunit{5}
       \psset{xunit=\xunit, yunit=\yunit, algebraic=true}
       \fontsize{15pt}{15pt}\selectfont
       \begin{pspicture*}[linewidth=1pt](\xmin,\ymin)(\xmax,\ymax)
       \psaxes[linewidth=0.75pt,showorigin=false]{->}(0,0)(\xmin,\ymin)(\xmax,\ymax)
       \rput(-0.1,-0.1){$O$}
       \psplot[linewidth=1pt,plotpoints=2000,linecolor=blue]{0.01}{\xmax}{7/2+(2*sin(x)-4)/cos(x)}
                    \psdots[dotsize=.15,linecolor=red](.53,.04)
       \rput[t](.51,-.02){$\red \frac{  \pi  }{ 6 } $}
       \end{pspicture*}
      }
   \end{extern}
\end{center}

A la calculatrice on trouve : $f\left(\frac{\pi }{6}\right)\approx 0.04 > 0$
          \par
          Le lapin peut donc être sauvé si l'angle $\theta $ est proche de $\frac{\pi }{6}$
     \end{enumerate}
\end{corrige}

\end{document}