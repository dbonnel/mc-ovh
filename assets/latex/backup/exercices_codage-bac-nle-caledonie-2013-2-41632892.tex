\documentclass[a4paper]{article}

%================================================================================================================================
%
% Packages
%
%================================================================================================================================

\usepackage[T1]{fontenc} 	% pour caractères accentués
\usepackage[utf8]{inputenc}  % encodage utf8
\usepackage[french]{babel}	% langue : français
\usepackage{fourier}			% caractères plus lisibles
\usepackage[dvipsnames]{xcolor} % couleurs
\usepackage{fancyhdr}		% réglage header footer
\usepackage{needspace}		% empêcher sauts de page mal placés
\usepackage{graphicx}		% pour inclure des graphiques
\usepackage{enumitem,cprotect}		% personnalise les listes d'items (nécessaire pour ol, al ...)
\usepackage{hyperref}		% Liens hypertexte
\usepackage{pstricks,pst-all,pst-node,pstricks-add,pst-math,pst-plot,pst-tree,pst-eucl} % pstricks
\usepackage[a4paper,includeheadfoot,top=2cm,left=3cm, bottom=2cm,right=3cm]{geometry} % marges etc.
\usepackage{comment}			% commentaires multilignes
\usepackage{amsmath,environ} % maths (matrices, etc.)
\usepackage{amssymb,makeidx}
\usepackage{bm}				% bold maths
\usepackage{tabularx}		% tableaux
\usepackage{colortbl}		% tableaux en couleur
\usepackage{fontawesome}		% Fontawesome
\usepackage{environ}			% environment with command
\usepackage{fp}				% calculs pour ps-tricks
\usepackage{multido}			% pour ps tricks
\usepackage[np]{numprint}	% formattage nombre
\usepackage{tikz,tkz-tab} 			% package principal TikZ
\usepackage{pgfplots}   % axes
\usepackage{mathrsfs}    % cursives
\usepackage{calc}			% calcul taille boites
\usepackage[scaled=0.875]{helvet} % font sans serif
\usepackage{svg} % svg
\usepackage{scrextend} % local margin
\usepackage{scratch} %scratch
\usepackage{multicol} % colonnes
%\usepackage{infix-RPN,pst-func} % formule en notation polanaise inversée
\usepackage{listings}

%================================================================================================================================
%
% Réglages de base
%
%================================================================================================================================

\lstset{
language=Python,   % R code
literate=
{á}{{\'a}}1
{à}{{\`a}}1
{ã}{{\~a}}1
{é}{{\'e}}1
{è}{{\`e}}1
{ê}{{\^e}}1
{í}{{\'i}}1
{ó}{{\'o}}1
{õ}{{\~o}}1
{ú}{{\'u}}1
{ü}{{\"u}}1
{ç}{{\c{c}}}1
{~}{{ }}1
}


\definecolor{codegreen}{rgb}{0,0.6,0}
\definecolor{codegray}{rgb}{0.5,0.5,0.5}
\definecolor{codepurple}{rgb}{0.58,0,0.82}
\definecolor{backcolour}{rgb}{0.95,0.95,0.92}

\lstdefinestyle{mystyle}{
    backgroundcolor=\color{backcolour},   
    commentstyle=\color{codegreen},
    keywordstyle=\color{magenta},
    numberstyle=\tiny\color{codegray},
    stringstyle=\color{codepurple},
    basicstyle=\ttfamily\footnotesize,
    breakatwhitespace=false,         
    breaklines=true,                 
    captionpos=b,                    
    keepspaces=true,                 
    numbers=left,                    
xleftmargin=2em,
framexleftmargin=2em,            
    showspaces=false,                
    showstringspaces=false,
    showtabs=false,                  
    tabsize=2,
    upquote=true
}

\lstset{style=mystyle}


\lstset{style=mystyle}
\newcommand{\imgdir}{C:/laragon/www/newmc/assets/imgsvg/}
\newcommand{\imgsvgdir}{C:/laragon/www/newmc/assets/imgsvg/}

\definecolor{mcgris}{RGB}{220, 220, 220}% ancien~; pour compatibilité
\definecolor{mcbleu}{RGB}{52, 152, 219}
\definecolor{mcvert}{RGB}{125, 194, 70}
\definecolor{mcmauve}{RGB}{154, 0, 215}
\definecolor{mcorange}{RGB}{255, 96, 0}
\definecolor{mcturquoise}{RGB}{0, 153, 153}
\definecolor{mcrouge}{RGB}{255, 0, 0}
\definecolor{mclightvert}{RGB}{205, 234, 190}

\definecolor{gris}{RGB}{220, 220, 220}
\definecolor{bleu}{RGB}{52, 152, 219}
\definecolor{vert}{RGB}{125, 194, 70}
\definecolor{mauve}{RGB}{154, 0, 215}
\definecolor{orange}{RGB}{255, 96, 0}
\definecolor{turquoise}{RGB}{0, 153, 153}
\definecolor{rouge}{RGB}{255, 0, 0}
\definecolor{lightvert}{RGB}{205, 234, 190}
\setitemize[0]{label=\color{lightvert}  $\bullet$}

\pagestyle{fancy}
\renewcommand{\headrulewidth}{0.2pt}
\fancyhead[L]{maths-cours.fr}
\fancyhead[R]{\thepage}
\renewcommand{\footrulewidth}{0.2pt}
\fancyfoot[C]{}

\newcolumntype{C}{>{\centering\arraybackslash}X}
\newcolumntype{s}{>{\hsize=.35\hsize\arraybackslash}X}

\setlength{\parindent}{0pt}		 
\setlength{\parskip}{3mm}
\setlength{\headheight}{1cm}

\def\ebook{ebook}
\def\book{book}
\def\web{web}
\def\type{web}

\newcommand{\vect}[1]{\overrightarrow{\,\mathstrut#1\,}}

\def\Oij{$\left(\text{O}~;~\vect{\imath},~\vect{\jmath}\right)$}
\def\Oijk{$\left(\text{O}~;~\vect{\imath},~\vect{\jmath},~\vect{k}\right)$}
\def\Ouv{$\left(\text{O}~;~\vect{u},~\vect{v}\right)$}

\hypersetup{breaklinks=true, colorlinks = true, linkcolor = OliveGreen, urlcolor = OliveGreen, citecolor = OliveGreen, pdfauthor={Didier BONNEL - https://www.maths-cours.fr} } % supprime les bordures autour des liens

\renewcommand{\arg}[0]{\text{arg}}

\everymath{\displaystyle}

%================================================================================================================================
%
% Macros - Commandes
%
%================================================================================================================================

\newcommand\meta[2]{    			% Utilisé pour créer le post HTML.
	\def\titre{titre}
	\def\url{url}
	\def\arg{#1}
	\ifx\titre\arg
		\newcommand\maintitle{#2}
		\fancyhead[L]{#2}
		{\Large\sffamily \MakeUppercase{#2}}
		\vspace{1mm}\textcolor{mcvert}{\hrule}
	\fi 
	\ifx\url\arg
		\fancyfoot[L]{\href{https://www.maths-cours.fr#2}{\black \footnotesize{https://www.maths-cours.fr#2}}}
	\fi 
}


\newcommand\TitreC[1]{    		% Titre centré
     \needspace{3\baselineskip}
     \begin{center}\textbf{#1}\end{center}
}

\newcommand\newpar{    		% paragraphe
     \par
}

\newcommand\nosp {    		% commande vide (pas d'espace)
}
\newcommand{\id}[1]{} %ignore

\newcommand\boite[2]{				% Boite simple sans titre
	\vspace{5mm}
	\setlength{\fboxrule}{0.2mm}
	\setlength{\fboxsep}{5mm}	
	\fcolorbox{#1}{#1!3}{\makebox[\linewidth-2\fboxrule-2\fboxsep]{
  		\begin{minipage}[t]{\linewidth-2\fboxrule-4\fboxsep}\setlength{\parskip}{3mm}
  			 #2
  		\end{minipage}
	}}
	\vspace{5mm}
}

\newcommand\CBox[4]{				% Boites
	\vspace{5mm}
	\setlength{\fboxrule}{0.2mm}
	\setlength{\fboxsep}{5mm}
	
	\fcolorbox{#1}{#1!3}{\makebox[\linewidth-2\fboxrule-2\fboxsep]{
		\begin{minipage}[t]{1cm}\setlength{\parskip}{3mm}
	  		\textcolor{#1}{\LARGE{#2}}    
 	 	\end{minipage}  
  		\begin{minipage}[t]{\linewidth-2\fboxrule-4\fboxsep}\setlength{\parskip}{3mm}
			\raisebox{1.2mm}{\normalsize\sffamily{\textcolor{#1}{#3}}}						
  			 #4
  		\end{minipage}
	}}
	\vspace{5mm}
}

\newcommand\cadre[3]{				% Boites convertible html
	\par
	\vspace{2mm}
	\setlength{\fboxrule}{0.1mm}
	\setlength{\fboxsep}{5mm}
	\fcolorbox{#1}{white}{\makebox[\linewidth-2\fboxrule-2\fboxsep]{
  		\begin{minipage}[t]{\linewidth-2\fboxrule-4\fboxsep}\setlength{\parskip}{3mm}
			\raisebox{-2.5mm}{\sffamily \small{\textcolor{#1}{\MakeUppercase{#2}}}}		
			\par		
  			 #3
 	 		\end{minipage}
	}}
		\vspace{2mm}
	\par
}

\newcommand\bloc[3]{				% Boites convertible html sans bordure
     \needspace{2\baselineskip}
     {\sffamily \small{\textcolor{#1}{\MakeUppercase{#2}}}}    
		\par		
  			 #3
		\par
}

\newcommand\CHelp[1]{
     \CBox{Plum}{\faInfoCircle}{À RETENIR}{#1}
}

\newcommand\CUp[1]{
     \CBox{NavyBlue}{\faThumbsOUp}{EN PRATIQUE}{#1}
}

\newcommand\CInfo[1]{
     \CBox{Sepia}{\faArrowCircleRight}{REMARQUE}{#1}
}

\newcommand\CRedac[1]{
     \CBox{PineGreen}{\faEdit}{BIEN R\'EDIGER}{#1}
}

\newcommand\CError[1]{
     \CBox{Red}{\faExclamationTriangle}{ATTENTION}{#1}
}

\newcommand\TitreExo[2]{
\needspace{4\baselineskip}
 {\sffamily\large EXERCICE #1\ (\emph{#2 points})}
\vspace{5mm}
}

\newcommand\img[2]{
          \includegraphics[width=#2\paperwidth]{\imgdir#1}
}

\newcommand\imgsvg[2]{
       \begin{center}   \includegraphics[width=#2\paperwidth]{\imgsvgdir#1} \end{center}
}


\newcommand\Lien[2]{
     \href{#1}{#2 \tiny \faExternalLink}
}
\newcommand\mcLien[2]{
     \href{https~://www.maths-cours.fr/#1}{#2 \tiny \faExternalLink}
}

\newcommand{\euro}{\eurologo{}}

%================================================================================================================================
%
% Macros - Environement
%
%================================================================================================================================

\newenvironment{tex}{ %
}
{%
}

\newenvironment{indente}{ %
	\setlength\parindent{10mm}
}

{
	\setlength\parindent{0mm}
}

\newenvironment{corrige}{%
     \needspace{3\baselineskip}
     \medskip
     \textbf{\textsc{Corrigé}}
     \medskip
}
{
}

\newenvironment{extern}{%
     \begin{center}
     }
     {
     \end{center}
}

\NewEnviron{code}{%
	\par
     \boite{gray}{\texttt{%
     \BODY
     }}
     \par
}

\newenvironment{vbloc}{% boite sans cadre empeche saut de page
     \begin{minipage}[t]{\linewidth}
     }
     {
     \end{minipage}
}
\NewEnviron{h2}{%
    \needspace{3\baselineskip}
    \vspace{0.6cm}
	\noindent \MakeUppercase{\sffamily \large \BODY}
	\vspace{1mm}\textcolor{mcgris}{\hrule}\vspace{0.4cm}
	\par
}{}

\NewEnviron{h3}{%
    \needspace{3\baselineskip}
	\vspace{5mm}
	\textsc{\BODY}
	\par
}

\NewEnviron{margeneg}{ %
\begin{addmargin}[-1cm]{0cm}
\BODY
\end{addmargin}
}

\NewEnviron{html}{%
}

\begin{document}
\meta{url}{/exercices/codage-bac-nle-caledonie-2013-2/}
\meta{pid}{1522}
\meta{titre}{Codage - Bac Nle Calédonie 2013}
\meta{type}{exercices}
\meta{canonical}{/exercices/codage-bac-nle-caledonie-2013-2/}
%
\textit{Bac S Nouvelle Calédonie 2013}
\par
On note E l'ensemble des vingt-sept nombres entiers compris entre 0 et 26.
\par
On note A l'ensemble dont les éléments sont les vingt-six lettres de l'alphabet et un séparateur entre deux mots, noté «*» considéré comme un caractère.
\par
Pour coder les éléments de A, on procède de la façon suivante :
\par
\textbf{Premièrement} : On associe à chacune des lettres de l'alphabet, rangées par ordre alphabétique, un nombre entier naturel compris entre 0 et 25, rangés par ordre croissant. On a donc \textit{a} → 0, \textit{b} → 1, ... \textit{z} → 25.
\par
On associe au séparateur «*» le nombre 26.
\begin{tabularx}{0.8\linewidth}{|*{3}{ {\centering \arraybackslash }X|}}%class="compact" width="600"     \hline
a       &  b       &  c       &  d       &  e       &  f       &  g       &  h       &  i       &  j       &  k       &  l       &  m       &  n  & o       \\ \hline
0       &  1       &   2       &  3       &   4       &  5       &   6       &  7       &   8       &  9       &   10       &  11       &   12       &  13 & 14   \\ \hline
\end{tabularx}

\begin{tabularx}{0.8\linewidth}{|*{3}{ {\centering \arraybackslash }X|}}%class="compact" width="600"     \hline
  p       &  q       &  r       &  s       &  t       &  u       &  v       &  w       &  x       &  y       &  z &  *\\ \hline     
      15       &   13       &  17       &   18       &  19       &   20       &  21       &   22       &  23       &   24       &  25       &   26
	   \\ \hline
\end{tabularx}
On dit que \textit{a} a pour rang 0, \textit{b} a pour rang 1, ... , \textit{z} a pour rang 25 et le séparateur «*» a pour rang 26.
\par
\textbf{Deuxièmement} : à chaque élément \textit{x} de E, l'application \textit{g} associe le reste de la division euclidienne de 4\textit{x}+3 par 27.
\par
On remarquera que pour tout \textit{x} de E, \textit{g}(\textit{x}) appartient à E.
\par
\textbf{Troisièmement} : Le caractère initial est alors remplacé par le caractère de rang \textit{g}(\textit{x}).
\textit{Exemple :} \textit{s} → 18,   \textit{g}(18)=21 et 21 → \textit{v}. Donc la lettre \textit{s} est remplacée lors du codage par la lettre \textit{v}.
\begin{enumerate}
     \item
     Trouver tous les entiers \textit{x} de E tels que \textit{g}(\textit{x})=\textit{x} c'est-à-dire invariants par \textit{g}.
     \par
     En déduire les caractères invariants dans ce codage
     \item
     Démontrer que, pour tout entier naturel \textit{x} appartenant à E et tout entier naturel \textit{y} appartenant à E, si \textit{y} ≡ 4\textit{x}+3 modulo 27 alors \textit{x} ≡ 7\textit{y}+6 modulo 27.
     \par
     En déduire que deux caractères distincts sont codés par deux caractères distincts.
     \item
     Proposer une méthode de décodage.
     \item
Décoder le mot « vfv ».\end{enumerate}
\begin{corrige}
     \begin{enumerate}
          \item
          \textit{g}(\textit{x})=\textit{x} si et seulement si 0 ≤ \textit{x} ≤ 26 et :
          \par
          4\textit{x}+3 ≡ \textit{x}   (mod. 27)
          \par
          Cette congruence est vérifiée si et seulement si il existe un entier relatif \textit{k} tel que :
          \par
          4\textit{x}+3 = \textit{x}+27\textit{k}
          3\textit{x} = 27\textit{k}−3
          \par
          3\textit{x} = 27\textit{k}−3
          \textit{x} = 9\textit{k}−1Pour \textit{k}≤0, les valeurs de \textit{x} obtenues sont strictement négatives et pour \textit{k} > 3 elles sont strictement supérieures à 26.
          \par
          On obtient donc trois solutions comprises entre 0 et 26 :
          \begin{itemize}
               \item
               \textit{x}=8 (pour \textit{k}=1)
               \item
               \textit{x}=17 (pour \textit{k}=2)
               \item
               \textit{x}=26 (pour \textit{k}31)
          \end{itemize}
          Par conséquent, les caractères invariants dans ce codage sont : \textit{i}, \textit{r}, *.
          \item
          Si \textit{y} ≡ 4\textit{x}+3   (mod. 27) alors :
          \par
          7\textit{y} ≡ 7(4\textit{x}+3)   (mod. 27)
          \par
          7\textit{y} ≡ 28\textit{x}+21   (mod. 27)Comme 28 ≡ 1  (mod. 27) et 21≡−6  (mod. 27) on a alors :
          \par
          7\textit{y} ≡ \textit{x}−6   (mod. 27)
          \textit{x} ≡ 7\textit{y}+6   (mod. 27)
          \par
          Soient deux entiers naturels \textit{x} et \textit{x}′, compris entre 0 et 26, ayant la même image \textit{y} par \textit{g}. Alors \textit{g}(\textit{x})=\textit{y} et \textit{g}(\textit{x}′)=\textit{y}.
          \par
          Par conséquent, \textit{x} ≡ 7\textit{y}+6   (mod. 27) et \textit{x}′ ≡ 7\textit{y}+6   (mod. 27).
          \par
          Donc, comme \textit{x} est compris entre 0 et 26, \textit{x} est le reste de la division euclidienne de 7\textit{y}+6 par 27 ainsi que \textit{x}′. L'unicité du reste entraîne que \textit{x}=\textit{x}′.
          \par
          Par conséquent, si deux caractères sont codés de façon identique, c'est qu'ils sont identiques. Autrement dit, deux caractères distincts sont codés par deux caractères distincts
          \item
          La formule \textit{x} ≡ 7\textit{y}+6 permet de décoder un caractère. Il suffit de procéder de la façon suivante :
          \begin{itemize}
               \item
               \textbf{1ère étape:} A chaque lettre on associe son rang \textit{y}
               \item
               \textbf{2ème étape} : à chaque valeur de \textit{y} , l'application \textit{h} associe le reste de la division euclidienne de 7\textit{y}+6 par 27.
               \item
               \textbf{3ème étape} : Le caractère initial est alors remplacé par le caractère de rang \textit{h}(\textit{y}) trouvé à la seconde étape.
          \end{itemize}
          \item
          On utilise la méthode décrite précédemment :
          \begin{itemize}
               \item
               \textit{v} → \textit{y}=21;
               \textit{h}(21) est le reste de la division de 7×21+6=153 par 27 donc \textit{h}(21)=18;
               \par
               18 → \textit{s}
               \item
               \textit{f} → \textit{y}=5;
               \textit{h}(5) est le reste de la division de 7×5+6=41 par 27 donc \textit{h}(21)=14;
               \par
               14 → \textit{o}
          \end{itemize}
          Le mot « vfv » se décode : « sos ».
     \end{enumerate}
\end{corrige}

\end{document}