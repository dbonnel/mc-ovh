\documentclass[a4paper]{article}

%================================================================================================================================
%
% Packages
%
%================================================================================================================================

\usepackage[T1]{fontenc} 	% pour caractères accentués
\usepackage[utf8]{inputenc}  % encodage utf8
\usepackage[french]{babel}	% langue : français
\usepackage{fourier}			% caractères plus lisibles
\usepackage[dvipsnames]{xcolor} % couleurs
\usepackage{fancyhdr}		% réglage header footer
\usepackage{needspace}		% empêcher sauts de page mal placés
\usepackage{graphicx}		% pour inclure des graphiques
\usepackage{enumitem,cprotect}		% personnalise les listes d'items (nécessaire pour ol, al ...)
\usepackage{hyperref}		% Liens hypertexte
\usepackage{pstricks,pst-all,pst-node,pstricks-add,pst-math,pst-plot,pst-tree,pst-eucl} % pstricks
\usepackage[a4paper,includeheadfoot,top=2cm,left=3cm, bottom=2cm,right=3cm]{geometry} % marges etc.
\usepackage{comment}			% commentaires multilignes
\usepackage{amsmath,environ} % maths (matrices, etc.)
\usepackage{amssymb,makeidx}
\usepackage{bm}				% bold maths
\usepackage{tabularx}		% tableaux
\usepackage{colortbl}		% tableaux en couleur
\usepackage{fontawesome}		% Fontawesome
\usepackage{environ}			% environment with command
\usepackage{fp}				% calculs pour ps-tricks
\usepackage{multido}			% pour ps tricks
\usepackage[np]{numprint}	% formattage nombre
\usepackage{tikz,tkz-tab} 			% package principal TikZ
\usepackage{pgfplots}   % axes
\usepackage{mathrsfs}    % cursives
\usepackage{calc}			% calcul taille boites
\usepackage[scaled=0.875]{helvet} % font sans serif
\usepackage{svg} % svg
\usepackage{scrextend} % local margin
\usepackage{scratch} %scratch
\usepackage{multicol} % colonnes
%\usepackage{infix-RPN,pst-func} % formule en notation polanaise inversée
\usepackage{listings}

%================================================================================================================================
%
% Réglages de base
%
%================================================================================================================================

\lstset{
language=Python,   % R code
literate=
{á}{{\'a}}1
{à}{{\`a}}1
{ã}{{\~a}}1
{é}{{\'e}}1
{è}{{\`e}}1
{ê}{{\^e}}1
{í}{{\'i}}1
{ó}{{\'o}}1
{õ}{{\~o}}1
{ú}{{\'u}}1
{ü}{{\"u}}1
{ç}{{\c{c}}}1
{~}{{ }}1
}


\definecolor{codegreen}{rgb}{0,0.6,0}
\definecolor{codegray}{rgb}{0.5,0.5,0.5}
\definecolor{codepurple}{rgb}{0.58,0,0.82}
\definecolor{backcolour}{rgb}{0.95,0.95,0.92}

\lstdefinestyle{mystyle}{
    backgroundcolor=\color{backcolour},   
    commentstyle=\color{codegreen},
    keywordstyle=\color{magenta},
    numberstyle=\tiny\color{codegray},
    stringstyle=\color{codepurple},
    basicstyle=\ttfamily\footnotesize,
    breakatwhitespace=false,         
    breaklines=true,                 
    captionpos=b,                    
    keepspaces=true,                 
    numbers=left,                    
xleftmargin=2em,
framexleftmargin=2em,            
    showspaces=false,                
    showstringspaces=false,
    showtabs=false,                  
    tabsize=2,
    upquote=true
}

\lstset{style=mystyle}


\lstset{style=mystyle}
\newcommand{\imgdir}{C:/laragon/www/newmc/assets/imgsvg/}
\newcommand{\imgsvgdir}{C:/laragon/www/newmc/assets/imgsvg/}

\definecolor{mcgris}{RGB}{220, 220, 220}% ancien~; pour compatibilité
\definecolor{mcbleu}{RGB}{52, 152, 219}
\definecolor{mcvert}{RGB}{125, 194, 70}
\definecolor{mcmauve}{RGB}{154, 0, 215}
\definecolor{mcorange}{RGB}{255, 96, 0}
\definecolor{mcturquoise}{RGB}{0, 153, 153}
\definecolor{mcrouge}{RGB}{255, 0, 0}
\definecolor{mclightvert}{RGB}{205, 234, 190}

\definecolor{gris}{RGB}{220, 220, 220}
\definecolor{bleu}{RGB}{52, 152, 219}
\definecolor{vert}{RGB}{125, 194, 70}
\definecolor{mauve}{RGB}{154, 0, 215}
\definecolor{orange}{RGB}{255, 96, 0}
\definecolor{turquoise}{RGB}{0, 153, 153}
\definecolor{rouge}{RGB}{255, 0, 0}
\definecolor{lightvert}{RGB}{205, 234, 190}
\setitemize[0]{label=\color{lightvert}  $\bullet$}

\pagestyle{fancy}
\renewcommand{\headrulewidth}{0.2pt}
\fancyhead[L]{maths-cours.fr}
\fancyhead[R]{\thepage}
\renewcommand{\footrulewidth}{0.2pt}
\fancyfoot[C]{}

\newcolumntype{C}{>{\centering\arraybackslash}X}
\newcolumntype{s}{>{\hsize=.35\hsize\arraybackslash}X}

\setlength{\parindent}{0pt}		 
\setlength{\parskip}{3mm}
\setlength{\headheight}{1cm}

\def\ebook{ebook}
\def\book{book}
\def\web{web}
\def\type{web}

\newcommand{\vect}[1]{\overrightarrow{\,\mathstrut#1\,}}

\def\Oij{$\left(\text{O}~;~\vect{\imath},~\vect{\jmath}\right)$}
\def\Oijk{$\left(\text{O}~;~\vect{\imath},~\vect{\jmath},~\vect{k}\right)$}
\def\Ouv{$\left(\text{O}~;~\vect{u},~\vect{v}\right)$}

\hypersetup{breaklinks=true, colorlinks = true, linkcolor = OliveGreen, urlcolor = OliveGreen, citecolor = OliveGreen, pdfauthor={Didier BONNEL - https://www.maths-cours.fr} } % supprime les bordures autour des liens

\renewcommand{\arg}[0]{\text{arg}}

\everymath{\displaystyle}

%================================================================================================================================
%
% Macros - Commandes
%
%================================================================================================================================

\newcommand\meta[2]{    			% Utilisé pour créer le post HTML.
	\def\titre{titre}
	\def\url{url}
	\def\arg{#1}
	\ifx\titre\arg
		\newcommand\maintitle{#2}
		\fancyhead[L]{#2}
		{\Large\sffamily \MakeUppercase{#2}}
		\vspace{1mm}\textcolor{mcvert}{\hrule}
	\fi 
	\ifx\url\arg
		\fancyfoot[L]{\href{https://www.maths-cours.fr#2}{\black \footnotesize{https://www.maths-cours.fr#2}}}
	\fi 
}


\newcommand\TitreC[1]{    		% Titre centré
     \needspace{3\baselineskip}
     \begin{center}\textbf{#1}\end{center}
}

\newcommand\newpar{    		% paragraphe
     \par
}

\newcommand\nosp {    		% commande vide (pas d'espace)
}
\newcommand{\id}[1]{} %ignore

\newcommand\boite[2]{				% Boite simple sans titre
	\vspace{5mm}
	\setlength{\fboxrule}{0.2mm}
	\setlength{\fboxsep}{5mm}	
	\fcolorbox{#1}{#1!3}{\makebox[\linewidth-2\fboxrule-2\fboxsep]{
  		\begin{minipage}[t]{\linewidth-2\fboxrule-4\fboxsep}\setlength{\parskip}{3mm}
  			 #2
  		\end{minipage}
	}}
	\vspace{5mm}
}

\newcommand\CBox[4]{				% Boites
	\vspace{5mm}
	\setlength{\fboxrule}{0.2mm}
	\setlength{\fboxsep}{5mm}
	
	\fcolorbox{#1}{#1!3}{\makebox[\linewidth-2\fboxrule-2\fboxsep]{
		\begin{minipage}[t]{1cm}\setlength{\parskip}{3mm}
	  		\textcolor{#1}{\LARGE{#2}}    
 	 	\end{minipage}  
  		\begin{minipage}[t]{\linewidth-2\fboxrule-4\fboxsep}\setlength{\parskip}{3mm}
			\raisebox{1.2mm}{\normalsize\sffamily{\textcolor{#1}{#3}}}						
  			 #4
  		\end{minipage}
	}}
	\vspace{5mm}
}

\newcommand\cadre[3]{				% Boites convertible html
	\par
	\vspace{2mm}
	\setlength{\fboxrule}{0.1mm}
	\setlength{\fboxsep}{5mm}
	\fcolorbox{#1}{white}{\makebox[\linewidth-2\fboxrule-2\fboxsep]{
  		\begin{minipage}[t]{\linewidth-2\fboxrule-4\fboxsep}\setlength{\parskip}{3mm}
			\raisebox{-2.5mm}{\sffamily \small{\textcolor{#1}{\MakeUppercase{#2}}}}		
			\par		
  			 #3
 	 		\end{minipage}
	}}
		\vspace{2mm}
	\par
}

\newcommand\bloc[3]{				% Boites convertible html sans bordure
     \needspace{2\baselineskip}
     {\sffamily \small{\textcolor{#1}{\MakeUppercase{#2}}}}    
		\par		
  			 #3
		\par
}

\newcommand\CHelp[1]{
     \CBox{Plum}{\faInfoCircle}{À RETENIR}{#1}
}

\newcommand\CUp[1]{
     \CBox{NavyBlue}{\faThumbsOUp}{EN PRATIQUE}{#1}
}

\newcommand\CInfo[1]{
     \CBox{Sepia}{\faArrowCircleRight}{REMARQUE}{#1}
}

\newcommand\CRedac[1]{
     \CBox{PineGreen}{\faEdit}{BIEN R\'EDIGER}{#1}
}

\newcommand\CError[1]{
     \CBox{Red}{\faExclamationTriangle}{ATTENTION}{#1}
}

\newcommand\TitreExo[2]{
\needspace{4\baselineskip}
 {\sffamily\large EXERCICE #1\ (\emph{#2 points})}
\vspace{5mm}
}

\newcommand\img[2]{
          \includegraphics[width=#2\paperwidth]{\imgdir#1}
}

\newcommand\imgsvg[2]{
       \begin{center}   \includegraphics[width=#2\paperwidth]{\imgsvgdir#1} \end{center}
}


\newcommand\Lien[2]{
     \href{#1}{#2 \tiny \faExternalLink}
}
\newcommand\mcLien[2]{
     \href{https~://www.maths-cours.fr/#1}{#2 \tiny \faExternalLink}
}

\newcommand{\euro}{\eurologo{}}

%================================================================================================================================
%
% Macros - Environement
%
%================================================================================================================================

\newenvironment{tex}{ %
}
{%
}

\newenvironment{indente}{ %
	\setlength\parindent{10mm}
}

{
	\setlength\parindent{0mm}
}

\newenvironment{corrige}{%
     \needspace{3\baselineskip}
     \medskip
     \textbf{\textsc{Corrigé}}
     \medskip
}
{
}

\newenvironment{extern}{%
     \begin{center}
     }
     {
     \end{center}
}

\NewEnviron{code}{%
	\par
     \boite{gray}{\texttt{%
     \BODY
     }}
     \par
}

\newenvironment{vbloc}{% boite sans cadre empeche saut de page
     \begin{minipage}[t]{\linewidth}
     }
     {
     \end{minipage}
}
\NewEnviron{h2}{%
    \needspace{3\baselineskip}
    \vspace{0.6cm}
	\noindent \MakeUppercase{\sffamily \large \BODY}
	\vspace{1mm}\textcolor{mcgris}{\hrule}\vspace{0.4cm}
	\par
}{}

\NewEnviron{h3}{%
    \needspace{3\baselineskip}
	\vspace{5mm}
	\textsc{\BODY}
	\par
}

\NewEnviron{margeneg}{ %
\begin{addmargin}[-1cm]{0cm}
\BODY
\end{addmargin}
}

\NewEnviron{html}{%
}

\begin{document}
\meta{url}{/supplement/fiche-de-revision-bac-les-nombres-complexes/}
\meta{pid}{10842}
\meta{titre}{Fiche de révision BAC : les nombres complexes}
\meta{type}{supplement}
%
\begin{enumerate}
     \item %
     Quelle est la forme algébrique d'un nombre complexe~? Quelle est la partie réelle~? La partie imaginaire~?
     \item %
     Qu'est-ce que le conjugué d'un nombre complexe~?
     \item %
     Comment représente-t-on graphiquement un nombre complexe~?
     \item %
     Qu'est-ce que le module et un argument d'un nombre complexe~? Comment s'interprètent-ils graphiquement ~?
     \item %
     Quelles sont les propriétés des conjugués, des modules et des arguments (produit, etc…)~?
     \item %
     Comment obtient-on  la forme trigonométrique d'un nombre complexe~? La forme exponentielle~?
     \item %
     Comment s'obtient la distance $AB$ à partir des affixes des points $A$ et $B$~?
     \item %
     Quels sont les arguments possibles pour un nombre réel~? un nombre imaginaire pur~?
     \item %
     Quelles sont, dans $\mathbb{C}$,  les solutions de l'équation $az^2+bz+c=0$~?
     \bigskip
     \textbf{Rappels de collège utiles pour certains exercices portant sur les nombres complexes.}
     \par
     $A$ et $B$ désignent des points du plan.
     \item %
     Quel est l'ensemble des points $M$ tels que $AM=BM$~?
     \item %
     Quel est l'ensemble des points $M$ tels que $AM=k $ (où $k$ est un réel donné)~?
     \item %
     Quel est l'ensemble des points $M$ tels que $(\overrightarrow{MA}~;~\overrightarrow{MB})=\pm \dfrac{\pi}{2}~(\text{mod.}~2\pi)$~?
\end{enumerate}
\begin{reponses}
     \begin{enumerate}
          \item %
          \textit{Quelle est la forme algébrique d'un nombre complexe~? Quelle est la partie réelle~? La partie imaginaire~?}
          \par
          La forme algébrique d'un nombre complexe $z$ est $z=x+iy$ (ou $z=a+ib$...) où $x$ et $y$ sont deux réels.
          $x$ est la partie réelle de $z$ et $y$ sa partie imaginaire.
          \item %
          \textit{ Qu'est-ce que le conjugué d'un nombre complexe~?}
          \par
          Le conjugué de $z=x+iy$ est le nombre complexe $\overline{z}=x-iy$.
          \item %
          \textit{Comment représente-t-on graphiquement un nombre complexe~?}
          \par
          Dans un repère orthonormé, on représente ee nombre complexe $z=x+iy$ par le point $M(x~;~y)$.\\
          On dit que $M$ est l'image de $z$ et que $z$ est l'affixe de $M$.
          \item %
          \textit{Qu'est-ce que le module et un argument d'un nombre complexe~? Comment s'interprètent-ils graphiquement ~?}
          \par
          Si le plan est rapporté au repère $(O~;~\vec{u},~\vec{v})$, le module de $z$ d'image $M$ est la distance $OM$~:\\
          $|z|=OM=\sqrt{x^2+y^2}$
          \par
          Un argument $\theta$ de $z$ (pour $z$ non nul) est une mesure, en radians, de l'angle $( \vec{u}~;~\vec{OM})$.\\
          On a $\cos \theta = \dfrac{x}{|z|}$ et  $\sin \theta = \dfrac{y}{|z|}$
          \item %
          \textit{Quelles sont les propriétés des conjugués, des modules et des arguments (produit, etc…)~?}
          \par
          $z$, $z_1$, $z_2$ désignent des nombres complexes quelconques et $n$ un entier relatif.
          \textbf{Conjugués~:}
          \begin{itemize}
               \item $\overline{z_1+z_2} = \overline{z_1}+\overline{z_2}$
               \item $\overline{z_1z_2} = \overline{z_1}\times \overline{z_2}$
               \item $\overline{\left(\frac{z_1}{z_2}\right)} = \frac{\overline{z_1}}{\overline{z_2}}  $ (si $z_2\neq 0$)
               \item $\overline{\left(z^{n}\right)} = \left(\overline{z}\right)^{n}$.
          \end{itemize}
          \textbf{Modules~:}
          \begin{itemize}
               \item $|z_1z_2| = |z_1|\times |z_2|$
               \item $|\frac{z_1}{z_2}| = \frac{|z_1|}{|z_2|}  $ (si $z_2\neq 0$)
               \item $|z_1+z_2| \leqslant |z_1| + |z_2|$ (inégalité triangulaire)
          \end{itemize}
          \textbf{Arguments~:}
          \begin{itemize}
               \item $\text{arg}\left(\overline{z}\right)=-\text{arg}\left(z\right)$
               \item $\text{arg}\left(z_1z_2\right)=\text{arg}\left(z_1\right)+\text{arg}\left(z_2\right)$
               \item $\text{arg}\left(z^{n}\right)=n\times \text{arg}\left(z\right)$
               \item $\text{arg}\left(\frac{z_1}{z_2}\right)=\text{arg}\left(z_1\right)-\text{arg}\left(z_2\right)$
          \end{itemize}
          \par
          \item %
          \textit{Comment obtient-on  la forme trigonométrique d'un nombre complexe~? La forme exponentielle~?}
          \par
          La forme trigonométrique d'un nombre complexe $z$ de module $r$ et dont un argument est $\theta$ est~:\\
          $z=r(\cos \theta + i \sin \theta)$.
          \par
          La forme exponentielle est~:
          $z=r\text{e}^{i\theta}$
          \item %
          \textit{Comment s'obtient la distance $AB$ à partir des affixes des points $A$ et $B$~?}
          \par
          Si $A$ et $B$ ont pour affixes respectives $z_A$ et $z_B$~:
          $AB=\left|z_B-z_A\right|$
          \item %
          \textit{Quels sont les arguments possibles pour un nombre réel~? un nombre imaginaire pur~?}
          \par
          Un nombre réel non nul a pour argument $0~(\text{mod.}~2\pi)$ (s'il est positif) ou $\pi~(\text{mod.}~2\pi)$ (s'il est négatif).
          Un nombre imaginaire pur non nul a pour argument $\dfrac{\pi}{2}~(\text{mod.}~2\pi)$ (si sa partie imaginaire est positive) ou $-\dfrac{\pi}{2}~(\text{mod.}~2\pi)$ (si sa partie imaginaire est négative)
          \item %
          \textit{Quelles sont, dans $\mathbb{C}$,  les solutions de l'équation $az^2+bz+c=0$~?}
          \par
          Si $\Delta$ est positif ou nul, on retrouve les solutions réelles.\\
          Si $\Delta$ est strictement négatif, l'équation possède deux solutions conjuguées~:\\
          $z_{1}=\frac{-b-i\sqrt{-\Delta }}{2a}  $ \\
          $z_{2}=\frac{-b+i\sqrt{-\Delta }}{2a}$.
          \par
          \bigskip
          \textbf{Rappels de collège utiles pour certains exercices portant sur les nombres complexes.}
          \par
          $A$ et $B$ désignent des points du plan.
          \item %
          \textit{Quel est l'ensemble des points $M$ tels que $AM=BM$~?}
          \par
          L'ensemble des points $M$ tels que $AM=BM$ est la médiatrice du segment $[AB]$.
          \item %
          \textit{Quel est l'ensemble des points $M$ tels que $AM=k $ (où $k$ est un réel donné)~?}
          \par
          L'ensemble des points $M$ tels que $AM=k$ est :
          \begin{itemize}
               \item %
               le cercle de centre $A$ et de rayon $k$ si $k > 0$
               \item %
               le point $A$  si $k = 0$
               \item %
               l'ensemble vide  si $k < 0$
          \end{itemize}
          \item %
          \textit{Quel est l'ensemble des points $M$ tels que $(\overrightarrow{MA}~;~\overrightarrow{MB})=\pm \dfrac{\pi}{2}~(\text{mod.}~2\pi)$~?}
          \par
          l'ensemble des points $M$ tels que $(\overrightarrow{MA}~;~\overrightarrow{MB})=\pm \dfrac{\pi}{2}~(\text{mod.}~2\pi)$ est le cercle de diamètre $[AB]$ privé des points $A$ et $B$ (pour lesquels l'angle $(\overrightarrow{MA}~;~\overrightarrow{MB})$ n'est pas défini).
     \end{enumerate}
\end{reponses}

\end{document}