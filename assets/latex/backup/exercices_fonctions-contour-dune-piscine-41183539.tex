\documentclass[a4paper]{article}

%================================================================================================================================
%
% Packages
%
%================================================================================================================================

\usepackage[T1]{fontenc} 	% pour caractères accentués
\usepackage[utf8]{inputenc}  % encodage utf8
\usepackage[french]{babel}	% langue : français
\usepackage{fourier}			% caractères plus lisibles
\usepackage[dvipsnames]{xcolor} % couleurs
\usepackage{fancyhdr}		% réglage header footer
\usepackage{needspace}		% empêcher sauts de page mal placés
\usepackage{graphicx}		% pour inclure des graphiques
\usepackage{enumitem,cprotect}		% personnalise les listes d'items (nécessaire pour ol, al ...)
\usepackage{hyperref}		% Liens hypertexte
\usepackage{pstricks,pst-all,pst-node,pstricks-add,pst-math,pst-plot,pst-tree,pst-eucl} % pstricks
\usepackage[a4paper,includeheadfoot,top=2cm,left=3cm, bottom=2cm,right=3cm]{geometry} % marges etc.
\usepackage{comment}			% commentaires multilignes
\usepackage{amsmath,environ} % maths (matrices, etc.)
\usepackage{amssymb,makeidx}
\usepackage{bm}				% bold maths
\usepackage{tabularx}		% tableaux
\usepackage{colortbl}		% tableaux en couleur
\usepackage{fontawesome}		% Fontawesome
\usepackage{environ}			% environment with command
\usepackage{fp}				% calculs pour ps-tricks
\usepackage{multido}			% pour ps tricks
\usepackage[np]{numprint}	% formattage nombre
\usepackage{tikz,tkz-tab} 			% package principal TikZ
\usepackage{pgfplots}   % axes
\usepackage{mathrsfs}    % cursives
\usepackage{calc}			% calcul taille boites
\usepackage[scaled=0.875]{helvet} % font sans serif
\usepackage{svg} % svg
\usepackage{scrextend} % local margin
\usepackage{scratch} %scratch
\usepackage{multicol} % colonnes
%\usepackage{infix-RPN,pst-func} % formule en notation polanaise inversée
\usepackage{listings}

%================================================================================================================================
%
% Réglages de base
%
%================================================================================================================================

\lstset{
language=Python,   % R code
literate=
{á}{{\'a}}1
{à}{{\`a}}1
{ã}{{\~a}}1
{é}{{\'e}}1
{è}{{\`e}}1
{ê}{{\^e}}1
{í}{{\'i}}1
{ó}{{\'o}}1
{õ}{{\~o}}1
{ú}{{\'u}}1
{ü}{{\"u}}1
{ç}{{\c{c}}}1
{~}{{ }}1
}


\definecolor{codegreen}{rgb}{0,0.6,0}
\definecolor{codegray}{rgb}{0.5,0.5,0.5}
\definecolor{codepurple}{rgb}{0.58,0,0.82}
\definecolor{backcolour}{rgb}{0.95,0.95,0.92}

\lstdefinestyle{mystyle}{
    backgroundcolor=\color{backcolour},   
    commentstyle=\color{codegreen},
    keywordstyle=\color{magenta},
    numberstyle=\tiny\color{codegray},
    stringstyle=\color{codepurple},
    basicstyle=\ttfamily\footnotesize,
    breakatwhitespace=false,         
    breaklines=true,                 
    captionpos=b,                    
    keepspaces=true,                 
    numbers=left,                    
xleftmargin=2em,
framexleftmargin=2em,            
    showspaces=false,                
    showstringspaces=false,
    showtabs=false,                  
    tabsize=2,
    upquote=true
}

\lstset{style=mystyle}


\lstset{style=mystyle}
\newcommand{\imgdir}{C:/laragon/www/newmc/assets/imgsvg/}
\newcommand{\imgsvgdir}{C:/laragon/www/newmc/assets/imgsvg/}

\definecolor{mcgris}{RGB}{220, 220, 220}% ancien~; pour compatibilité
\definecolor{mcbleu}{RGB}{52, 152, 219}
\definecolor{mcvert}{RGB}{125, 194, 70}
\definecolor{mcmauve}{RGB}{154, 0, 215}
\definecolor{mcorange}{RGB}{255, 96, 0}
\definecolor{mcturquoise}{RGB}{0, 153, 153}
\definecolor{mcrouge}{RGB}{255, 0, 0}
\definecolor{mclightvert}{RGB}{205, 234, 190}

\definecolor{gris}{RGB}{220, 220, 220}
\definecolor{bleu}{RGB}{52, 152, 219}
\definecolor{vert}{RGB}{125, 194, 70}
\definecolor{mauve}{RGB}{154, 0, 215}
\definecolor{orange}{RGB}{255, 96, 0}
\definecolor{turquoise}{RGB}{0, 153, 153}
\definecolor{rouge}{RGB}{255, 0, 0}
\definecolor{lightvert}{RGB}{205, 234, 190}
\setitemize[0]{label=\color{lightvert}  $\bullet$}

\pagestyle{fancy}
\renewcommand{\headrulewidth}{0.2pt}
\fancyhead[L]{maths-cours.fr}
\fancyhead[R]{\thepage}
\renewcommand{\footrulewidth}{0.2pt}
\fancyfoot[C]{}

\newcolumntype{C}{>{\centering\arraybackslash}X}
\newcolumntype{s}{>{\hsize=.35\hsize\arraybackslash}X}

\setlength{\parindent}{0pt}		 
\setlength{\parskip}{3mm}
\setlength{\headheight}{1cm}

\def\ebook{ebook}
\def\book{book}
\def\web{web}
\def\type{web}

\newcommand{\vect}[1]{\overrightarrow{\,\mathstrut#1\,}}

\def\Oij{$\left(\text{O}~;~\vect{\imath},~\vect{\jmath}\right)$}
\def\Oijk{$\left(\text{O}~;~\vect{\imath},~\vect{\jmath},~\vect{k}\right)$}
\def\Ouv{$\left(\text{O}~;~\vect{u},~\vect{v}\right)$}

\hypersetup{breaklinks=true, colorlinks = true, linkcolor = OliveGreen, urlcolor = OliveGreen, citecolor = OliveGreen, pdfauthor={Didier BONNEL - https://www.maths-cours.fr} } % supprime les bordures autour des liens

\renewcommand{\arg}[0]{\text{arg}}

\everymath{\displaystyle}

%================================================================================================================================
%
% Macros - Commandes
%
%================================================================================================================================

\newcommand\meta[2]{    			% Utilisé pour créer le post HTML.
	\def\titre{titre}
	\def\url{url}
	\def\arg{#1}
	\ifx\titre\arg
		\newcommand\maintitle{#2}
		\fancyhead[L]{#2}
		{\Large\sffamily \MakeUppercase{#2}}
		\vspace{1mm}\textcolor{mcvert}{\hrule}
	\fi 
	\ifx\url\arg
		\fancyfoot[L]{\href{https://www.maths-cours.fr#2}{\black \footnotesize{https://www.maths-cours.fr#2}}}
	\fi 
}


\newcommand\TitreC[1]{    		% Titre centré
     \needspace{3\baselineskip}
     \begin{center}\textbf{#1}\end{center}
}

\newcommand\newpar{    		% paragraphe
     \par
}

\newcommand\nosp {    		% commande vide (pas d'espace)
}
\newcommand{\id}[1]{} %ignore

\newcommand\boite[2]{				% Boite simple sans titre
	\vspace{5mm}
	\setlength{\fboxrule}{0.2mm}
	\setlength{\fboxsep}{5mm}	
	\fcolorbox{#1}{#1!3}{\makebox[\linewidth-2\fboxrule-2\fboxsep]{
  		\begin{minipage}[t]{\linewidth-2\fboxrule-4\fboxsep}\setlength{\parskip}{3mm}
  			 #2
  		\end{minipage}
	}}
	\vspace{5mm}
}

\newcommand\CBox[4]{				% Boites
	\vspace{5mm}
	\setlength{\fboxrule}{0.2mm}
	\setlength{\fboxsep}{5mm}
	
	\fcolorbox{#1}{#1!3}{\makebox[\linewidth-2\fboxrule-2\fboxsep]{
		\begin{minipage}[t]{1cm}\setlength{\parskip}{3mm}
	  		\textcolor{#1}{\LARGE{#2}}    
 	 	\end{minipage}  
  		\begin{minipage}[t]{\linewidth-2\fboxrule-4\fboxsep}\setlength{\parskip}{3mm}
			\raisebox{1.2mm}{\normalsize\sffamily{\textcolor{#1}{#3}}}						
  			 #4
  		\end{minipage}
	}}
	\vspace{5mm}
}

\newcommand\cadre[3]{				% Boites convertible html
	\par
	\vspace{2mm}
	\setlength{\fboxrule}{0.1mm}
	\setlength{\fboxsep}{5mm}
	\fcolorbox{#1}{white}{\makebox[\linewidth-2\fboxrule-2\fboxsep]{
  		\begin{minipage}[t]{\linewidth-2\fboxrule-4\fboxsep}\setlength{\parskip}{3mm}
			\raisebox{-2.5mm}{\sffamily \small{\textcolor{#1}{\MakeUppercase{#2}}}}		
			\par		
  			 #3
 	 		\end{minipage}
	}}
		\vspace{2mm}
	\par
}

\newcommand\bloc[3]{				% Boites convertible html sans bordure
     \needspace{2\baselineskip}
     {\sffamily \small{\textcolor{#1}{\MakeUppercase{#2}}}}    
		\par		
  			 #3
		\par
}

\newcommand\CHelp[1]{
     \CBox{Plum}{\faInfoCircle}{À RETENIR}{#1}
}

\newcommand\CUp[1]{
     \CBox{NavyBlue}{\faThumbsOUp}{EN PRATIQUE}{#1}
}

\newcommand\CInfo[1]{
     \CBox{Sepia}{\faArrowCircleRight}{REMARQUE}{#1}
}

\newcommand\CRedac[1]{
     \CBox{PineGreen}{\faEdit}{BIEN R\'EDIGER}{#1}
}

\newcommand\CError[1]{
     \CBox{Red}{\faExclamationTriangle}{ATTENTION}{#1}
}

\newcommand\TitreExo[2]{
\needspace{4\baselineskip}
 {\sffamily\large EXERCICE #1\ (\emph{#2 points})}
\vspace{5mm}
}

\newcommand\img[2]{
          \includegraphics[width=#2\paperwidth]{\imgdir#1}
}

\newcommand\imgsvg[2]{
       \begin{center}   \includegraphics[width=#2\paperwidth]{\imgsvgdir#1} \end{center}
}


\newcommand\Lien[2]{
     \href{#1}{#2 \tiny \faExternalLink}
}
\newcommand\mcLien[2]{
     \href{https~://www.maths-cours.fr/#1}{#2 \tiny \faExternalLink}
}

\newcommand{\euro}{\eurologo{}}

%================================================================================================================================
%
% Macros - Environement
%
%================================================================================================================================

\newenvironment{tex}{ %
}
{%
}

\newenvironment{indente}{ %
	\setlength\parindent{10mm}
}

{
	\setlength\parindent{0mm}
}

\newenvironment{corrige}{%
     \needspace{3\baselineskip}
     \medskip
     \textbf{\textsc{Corrigé}}
     \medskip
}
{
}

\newenvironment{extern}{%
     \begin{center}
     }
     {
     \end{center}
}

\NewEnviron{code}{%
	\par
     \boite{gray}{\texttt{%
     \BODY
     }}
     \par
}

\newenvironment{vbloc}{% boite sans cadre empeche saut de page
     \begin{minipage}[t]{\linewidth}
     }
     {
     \end{minipage}
}
\NewEnviron{h2}{%
    \needspace{3\baselineskip}
    \vspace{0.6cm}
	\noindent \MakeUppercase{\sffamily \large \BODY}
	\vspace{1mm}\textcolor{mcgris}{\hrule}\vspace{0.4cm}
	\par
}{}

\NewEnviron{h3}{%
    \needspace{3\baselineskip}
	\vspace{5mm}
	\textsc{\BODY}
	\par
}

\NewEnviron{margeneg}{ %
\begin{addmargin}[-1cm]{0cm}
\BODY
\end{addmargin}
}

\NewEnviron{html}{%
}

\begin{document}
\meta{url}{/exercices/fonctions-contour-dune-piscine/}
\meta{pid}{3483}
\meta{titre}{Fonctions - Contour d'une piscine}
\meta{type}{exercices}
%
Pour les besoins d'un centre de loisirs, un architecte élabore les plans d'une future piscine carrelée.
\par
Le graphique ci-dessous présente le contour de cette piscine dans un repère orthonormé $(O,I,J)$ d'unité 1 mètre.

\begin{center}
\imgsvg{fonctions-contour-dune-piscine}{0.3}% alt="fonctions-contour-dune-piscine" style="width:50rem" 
\end{center}
\\
$C_1$ est un demi-cercle de centre $O$ et de rayon $8$ ;
\par
$C_2$ est un demi-cercle de centre $P(12 ; 0)$ et de rayon $6$. Les courbes $F_1$ et $F_2$ relient ces deux demi-cercles.
\par
Le contour de la piscine est symétrique par rapport à l'axe des abscisses. On suppose, par ailleurs, que les tangentes à la courbe $F_1$ aux points $M(0;8)$ et $N(12;6)$ sont parallèles à l'axe des abscisses.
\begin{h3}Partie 1\end{h3}
\begin{enumerate}
     \item
     La courbe $F_1$ est la représentation graphique d'une fonction $f$ définie sur $[0;12]$.
     \par
     Quelles sont les valeurs de $f(0)$,  $f(12)$ ,  $f'(0)$,  $f'(12)$ ?
     \item
     $f$ est définie sur $[0;12]$ par $f(x)=ax^3+bx^2+cx+d$.
     \par
     Déduire de la question précédente un système de quatre équations à quatre inconnues vérifié par $(a;b;c;d)$.
     \item
     En déduire les valeurs de $a, b, c$ et $d$.
\end{enumerate}
\begin{h3}Partie 2\end{h3}
Dans la suite du problème on suppose que $f$ est définie sur $[0;12]$ par $f(x)=\frac{1}{432}x^3-\frac{1}{24}x^2+8$.
\begin{enumerate}
     \item
     Montrer que le milieu $I$ de $[MN]$ appartient à la courbe $F_1$
     \item
     Donner une équation de la tangente $(T)$  à la courbe $F_1$ au point $I$.
\end{enumerate}
\begin{h3}Partie 3\end{h3}
\begin{enumerate}
     \item
     Dans le but de carreler le fond de la piscine, l'architecte cherche à estimer l'aire $\mathscr A$  de la surface située à l'intérieur de ce contour.
     \par
     On admet que l'aire de la surface délimitée par  la courbe $F_1$ et les segments $[OM], [OP], [PN]$ est égale à l'aire du trapèze $OMNP$.
     \par
     Calculer l'aire $\mathscr A$ en $\text{m}^2$ (on arrondira au $\text{m}^2$ près).
     \item
     La profondeur de la piscine sera constante et égale à $1,5\text{m}$.
     \par
     Quel sera, en $\text{m}^3$, le volume d'eau de la piscine ?
\end{enumerate}
\begin{corrige}
     \begin{h3}Partie 1\end{h3}
     \begin{enumerate}
          \item
          La courbe $F_1$ passe par le point $M(0;8)$ donc $f(0)=8$.
          \par
          La courbe $F_1$ passe par le point $N(12;6)$ donc $f(12)=6$.
     <div class="note flr-40">La tangente à $F_1$ au point d'abscisse $a$ est parallèle à l'axe des abscisses si et seulement si  $f ^{\prime}(a)=0$
     La tangente à $F_1$ au point $M$ est parallèle à l'axe des abscisses donc $f ^{\prime}(0)=0$.
     \par
     La tangente à $F_1$ au point $N$ est parallèle à l'axe des abscisses donc $f ^{\prime}(12)=0$.
     \item
     $f(0)=a\times 0^3+b \times 0^2+c \times 0+d=d$
     \par
     Donc d'après la question précédente $d=8$.
     \par
     De même :
     \par
     $f(12)=a\times 12^3+b \times 12^2+c \times 12+d$$=1728a+144b+12c+d$
     \par
     Donc  $1728a+144b+12c+d=6$.
     \par
     $f ^{\prime}(x)=3ax^2+2bx+c$
     \par
     $f ^{\prime}(0)=3a \times 0^2+2b \times 0+c=c$
     \par
     Donc $c=0$.
     \par
     $f ^{\prime}(12)=3a \times 12^2+2b \times 12+c=432a+24b+c$
     \par
     Donc $432a+24b+c=0$.
     \par
     Le quadruplet $(a;b;c;d)$ est donc solution du système :
     \par
     $\begin{cases}  d=8 \\ 1728a+144b+12c+d=6 \\ c=0 \\   432a+24b+c=0 \end{cases}$
     \item
     Les première et troisième équations donnent $c=0$ et $d=8$.
     \par
     En remplaçant $c$ par $0$ et $d$ par $8$ dans les deux autres équations on obtient :
     \par
     $(S) \ \begin{cases} 1728a+144b+8=6 \\ 432a+24b=0  \end{cases}$
     \par
     Ce système équivaut à :
     \par
     $(S) \Leftrightarrow \ \begin{cases} 1728a+144b+8=6 \\ b=-18a  \end{cases}$~
     \par
     ~
     \par
     $(S) \Leftrightarrow \ \begin{cases} 1728a+144 \times (-18a)=-2 \\ b=-18a  \end{cases}$~
     \par
     ~
     \par
     $(S) \Leftrightarrow \ \begin{cases} -864a=-2 \\ b=-18a  \end{cases}$ ~
     \par
     ~
     \par
     $(S) \Leftrightarrow \ \begin{cases} a=\frac{2}{864} \\ \\ b=-18 \times \frac{2}{864}   \end{cases}$ ~
     \par
     ~
     \par
     $(S) \Leftrightarrow \ \begin{cases} a=\frac{1}{432} \\ \\ b=-\frac{1}{24}   \end{cases}$ ~
     \par
     ~
     \par
     Finalement $ a=\frac{1}{432},  b=-\frac{1}{24}, c=0  $ et $d=8$.
     \par
     Donc $f(x)=\frac{1}{432}x^3-\frac{1}{24}x^2+8$.
\end{enumerate}
\begin{h3}Partie 2\end{h3}
\begin{enumerate}
     \item
     Les coordonnées du point $I$ sont :
     \par
     $x_I=\frac{x_M+x_N}{2}=\frac{0+12}{2}=6$
     \par
     $y_I=\frac{y_M+y_N}{2}=\frac{8+6}{2}=7$
     <div class="note flr-40">
     Le point $I$ appartient à la courbe $F_1$ si et seulement si $f(x_I)=y_I$
}
$f(6)=\frac{1}{432} \times 216-\frac{1}{24} \times 36+8$$=0,5-1,5+8=7$
\par
$f(x_I)=y_I$ donc le milieu $I$ de $[MN]$ appartient à la courbe $F_1$.
\item
L'équation réduite de la tangente à la courbe $F_1$ en $I$ est :
\par
$y=f ^{\prime}(6)(x-6)+f(6)$
\par
$f ^{\prime}(x)=\frac{3}{432}x^2-\frac{2}{24}x$$=\frac{x^2}{144}-\frac{x}{12}$
\par
~
\par
$f'(6)=\frac{36}{144}-\frac{6}{12}=-\frac{1}{4}$
\par
L'équation réduite de $(T)$ est donc :
\par
$y=-\frac{1}{4}(x-6)-7$
\par
~
\par
$y=-\frac{1}{4}x+\frac{17}{2}$
\begin{center}
\imgsvg{fonctions-contour-dune-piscine}{0.3}% alt="fonctions-contour-dune-piscine-2" style="width:50rem" 
\end{center}   
\end{enumerate}
\begin{h3}Partie 3\end{h3}
\begin{enumerate}
     \item
     On "découpe" l'aire $\mathscr A$ en quatre aires :
     \begin{itemize}
          \item
          l'aire $\mathscr A_1$ du demi-disque de rayon $[OM]$
          \item
          l'aire $\mathscr A_2$ du demi-disque de rayon $[PN]$
          \item
          l'aire $\mathscr A_3$ du trapèze $OMNP$ et du trapèze symétrique par rapport à l'axe des abscisses.
     \end{itemize}
     $\mathscr A_1=\frac{1}{2}\pi OM^2=32\pi$
     \par
     $\mathscr A_2=\frac{1}{2}\pi PN^2=18\pi$
     <div class="note flr-40">L'aire d'un trapèze de bases $b$ et $B$ et de hauteur $h$ est $\mathscr A=\frac{b+B}{2} \times  h$
}
$\mathscr A_3=\frac{8+6}{2} \times 12$
\par
$\phantom{\mathscr A_3}=7 \times 12$
\par
$\phantom{\mathscr A_3}=84$
\par
L'aire totale est donc :
\par
$\mathscr A=\mathscr A_1+\mathscr A_2+2 \times \mathscr A_3$
\par
$\phantom{\mathscr A}=50\pi+168 \approx 325 \text{m}^2$
\item
Le volume d'eau de la piscine est :
\par
$\mathscr V = 1,5 \times \mathscr A \approx 488 \text{m}^3$
\end{enumerate}
}

\end{document}