\documentclass[a4paper]{article}

%================================================================================================================================
%
% Packages
%
%================================================================================================================================

\usepackage[T1]{fontenc} 	% pour caractères accentués
\usepackage[utf8]{inputenc}  % encodage utf8
\usepackage[french]{babel}	% langue : français
\usepackage{fourier}			% caractères plus lisibles
\usepackage[dvipsnames]{xcolor} % couleurs
\usepackage{fancyhdr}		% réglage header footer
\usepackage{needspace}		% empêcher sauts de page mal placés
\usepackage{graphicx}		% pour inclure des graphiques
\usepackage{enumitem,cprotect}		% personnalise les listes d'items (nécessaire pour ol, al ...)
\usepackage{hyperref}		% Liens hypertexte
\usepackage{pstricks,pst-all,pst-node,pstricks-add,pst-math,pst-plot,pst-tree,pst-eucl} % pstricks
\usepackage[a4paper,includeheadfoot,top=2cm,left=3cm, bottom=2cm,right=3cm]{geometry} % marges etc.
\usepackage{comment}			% commentaires multilignes
\usepackage{amsmath,environ} % maths (matrices, etc.)
\usepackage{amssymb,makeidx}
\usepackage{bm}				% bold maths
\usepackage{tabularx}		% tableaux
\usepackage{colortbl}		% tableaux en couleur
\usepackage{fontawesome}		% Fontawesome
\usepackage{environ}			% environment with command
\usepackage{fp}				% calculs pour ps-tricks
\usepackage{multido}			% pour ps tricks
\usepackage[np]{numprint}	% formattage nombre
\usepackage{tikz,tkz-tab} 			% package principal TikZ
\usepackage{pgfplots}   % axes
\usepackage{mathrsfs}    % cursives
\usepackage{calc}			% calcul taille boites
\usepackage[scaled=0.875]{helvet} % font sans serif
\usepackage{svg} % svg
\usepackage{scrextend} % local margin
\usepackage{scratch} %scratch
\usepackage{multicol} % colonnes
%\usepackage{infix-RPN,pst-func} % formule en notation polanaise inversée
\usepackage{listings}

%================================================================================================================================
%
% Réglages de base
%
%================================================================================================================================

\lstset{
language=Python,   % R code
literate=
{á}{{\'a}}1
{à}{{\`a}}1
{ã}{{\~a}}1
{é}{{\'e}}1
{è}{{\`e}}1
{ê}{{\^e}}1
{í}{{\'i}}1
{ó}{{\'o}}1
{õ}{{\~o}}1
{ú}{{\'u}}1
{ü}{{\"u}}1
{ç}{{\c{c}}}1
{~}{{ }}1
}


\definecolor{codegreen}{rgb}{0,0.6,0}
\definecolor{codegray}{rgb}{0.5,0.5,0.5}
\definecolor{codepurple}{rgb}{0.58,0,0.82}
\definecolor{backcolour}{rgb}{0.95,0.95,0.92}

\lstdefinestyle{mystyle}{
    backgroundcolor=\color{backcolour},   
    commentstyle=\color{codegreen},
    keywordstyle=\color{magenta},
    numberstyle=\tiny\color{codegray},
    stringstyle=\color{codepurple},
    basicstyle=\ttfamily\footnotesize,
    breakatwhitespace=false,         
    breaklines=true,                 
    captionpos=b,                    
    keepspaces=true,                 
    numbers=left,                    
xleftmargin=2em,
framexleftmargin=2em,            
    showspaces=false,                
    showstringspaces=false,
    showtabs=false,                  
    tabsize=2,
    upquote=true
}

\lstset{style=mystyle}


\lstset{style=mystyle}
\newcommand{\imgdir}{C:/laragon/www/newmc/assets/imgsvg/}
\newcommand{\imgsvgdir}{C:/laragon/www/newmc/assets/imgsvg/}

\definecolor{mcgris}{RGB}{220, 220, 220}% ancien~; pour compatibilité
\definecolor{mcbleu}{RGB}{52, 152, 219}
\definecolor{mcvert}{RGB}{125, 194, 70}
\definecolor{mcmauve}{RGB}{154, 0, 215}
\definecolor{mcorange}{RGB}{255, 96, 0}
\definecolor{mcturquoise}{RGB}{0, 153, 153}
\definecolor{mcrouge}{RGB}{255, 0, 0}
\definecolor{mclightvert}{RGB}{205, 234, 190}

\definecolor{gris}{RGB}{220, 220, 220}
\definecolor{bleu}{RGB}{52, 152, 219}
\definecolor{vert}{RGB}{125, 194, 70}
\definecolor{mauve}{RGB}{154, 0, 215}
\definecolor{orange}{RGB}{255, 96, 0}
\definecolor{turquoise}{RGB}{0, 153, 153}
\definecolor{rouge}{RGB}{255, 0, 0}
\definecolor{lightvert}{RGB}{205, 234, 190}
\setitemize[0]{label=\color{lightvert}  $\bullet$}

\pagestyle{fancy}
\renewcommand{\headrulewidth}{0.2pt}
\fancyhead[L]{maths-cours.fr}
\fancyhead[R]{\thepage}
\renewcommand{\footrulewidth}{0.2pt}
\fancyfoot[C]{}

\newcolumntype{C}{>{\centering\arraybackslash}X}
\newcolumntype{s}{>{\hsize=.35\hsize\arraybackslash}X}

\setlength{\parindent}{0pt}		 
\setlength{\parskip}{3mm}
\setlength{\headheight}{1cm}

\def\ebook{ebook}
\def\book{book}
\def\web{web}
\def\type{web}

\newcommand{\vect}[1]{\overrightarrow{\,\mathstrut#1\,}}

\def\Oij{$\left(\text{O}~;~\vect{\imath},~\vect{\jmath}\right)$}
\def\Oijk{$\left(\text{O}~;~\vect{\imath},~\vect{\jmath},~\vect{k}\right)$}
\def\Ouv{$\left(\text{O}~;~\vect{u},~\vect{v}\right)$}

\hypersetup{breaklinks=true, colorlinks = true, linkcolor = OliveGreen, urlcolor = OliveGreen, citecolor = OliveGreen, pdfauthor={Didier BONNEL - https://www.maths-cours.fr} } % supprime les bordures autour des liens

\renewcommand{\arg}[0]{\text{arg}}

\everymath{\displaystyle}

%================================================================================================================================
%
% Macros - Commandes
%
%================================================================================================================================

\newcommand\meta[2]{    			% Utilisé pour créer le post HTML.
	\def\titre{titre}
	\def\url{url}
	\def\arg{#1}
	\ifx\titre\arg
		\newcommand\maintitle{#2}
		\fancyhead[L]{#2}
		{\Large\sffamily \MakeUppercase{#2}}
		\vspace{1mm}\textcolor{mcvert}{\hrule}
	\fi 
	\ifx\url\arg
		\fancyfoot[L]{\href{https://www.maths-cours.fr#2}{\black \footnotesize{https://www.maths-cours.fr#2}}}
	\fi 
}


\newcommand\TitreC[1]{    		% Titre centré
     \needspace{3\baselineskip}
     \begin{center}\textbf{#1}\end{center}
}

\newcommand\newpar{    		% paragraphe
     \par
}

\newcommand\nosp {    		% commande vide (pas d'espace)
}
\newcommand{\id}[1]{} %ignore

\newcommand\boite[2]{				% Boite simple sans titre
	\vspace{5mm}
	\setlength{\fboxrule}{0.2mm}
	\setlength{\fboxsep}{5mm}	
	\fcolorbox{#1}{#1!3}{\makebox[\linewidth-2\fboxrule-2\fboxsep]{
  		\begin{minipage}[t]{\linewidth-2\fboxrule-4\fboxsep}\setlength{\parskip}{3mm}
  			 #2
  		\end{minipage}
	}}
	\vspace{5mm}
}

\newcommand\CBox[4]{				% Boites
	\vspace{5mm}
	\setlength{\fboxrule}{0.2mm}
	\setlength{\fboxsep}{5mm}
	
	\fcolorbox{#1}{#1!3}{\makebox[\linewidth-2\fboxrule-2\fboxsep]{
		\begin{minipage}[t]{1cm}\setlength{\parskip}{3mm}
	  		\textcolor{#1}{\LARGE{#2}}    
 	 	\end{minipage}  
  		\begin{minipage}[t]{\linewidth-2\fboxrule-4\fboxsep}\setlength{\parskip}{3mm}
			\raisebox{1.2mm}{\normalsize\sffamily{\textcolor{#1}{#3}}}						
  			 #4
  		\end{minipage}
	}}
	\vspace{5mm}
}

\newcommand\cadre[3]{				% Boites convertible html
	\par
	\vspace{2mm}
	\setlength{\fboxrule}{0.1mm}
	\setlength{\fboxsep}{5mm}
	\fcolorbox{#1}{white}{\makebox[\linewidth-2\fboxrule-2\fboxsep]{
  		\begin{minipage}[t]{\linewidth-2\fboxrule-4\fboxsep}\setlength{\parskip}{3mm}
			\raisebox{-2.5mm}{\sffamily \small{\textcolor{#1}{\MakeUppercase{#2}}}}		
			\par		
  			 #3
 	 		\end{minipage}
	}}
		\vspace{2mm}
	\par
}

\newcommand\bloc[3]{				% Boites convertible html sans bordure
     \needspace{2\baselineskip}
     {\sffamily \small{\textcolor{#1}{\MakeUppercase{#2}}}}    
		\par		
  			 #3
		\par
}

\newcommand\CHelp[1]{
     \CBox{Plum}{\faInfoCircle}{À RETENIR}{#1}
}

\newcommand\CUp[1]{
     \CBox{NavyBlue}{\faThumbsOUp}{EN PRATIQUE}{#1}
}

\newcommand\CInfo[1]{
     \CBox{Sepia}{\faArrowCircleRight}{REMARQUE}{#1}
}

\newcommand\CRedac[1]{
     \CBox{PineGreen}{\faEdit}{BIEN R\'EDIGER}{#1}
}

\newcommand\CError[1]{
     \CBox{Red}{\faExclamationTriangle}{ATTENTION}{#1}
}

\newcommand\TitreExo[2]{
\needspace{4\baselineskip}
 {\sffamily\large EXERCICE #1\ (\emph{#2 points})}
\vspace{5mm}
}

\newcommand\img[2]{
          \includegraphics[width=#2\paperwidth]{\imgdir#1}
}

\newcommand\imgsvg[2]{
       \begin{center}   \includegraphics[width=#2\paperwidth]{\imgsvgdir#1} \end{center}
}


\newcommand\Lien[2]{
     \href{#1}{#2 \tiny \faExternalLink}
}
\newcommand\mcLien[2]{
     \href{https~://www.maths-cours.fr/#1}{#2 \tiny \faExternalLink}
}

\newcommand{\euro}{\eurologo{}}

%================================================================================================================================
%
% Macros - Environement
%
%================================================================================================================================

\newenvironment{tex}{ %
}
{%
}

\newenvironment{indente}{ %
	\setlength\parindent{10mm}
}

{
	\setlength\parindent{0mm}
}

\newenvironment{corrige}{%
     \needspace{3\baselineskip}
     \medskip
     \textbf{\textsc{Corrigé}}
     \medskip
}
{
}

\newenvironment{extern}{%
     \begin{center}
     }
     {
     \end{center}
}

\NewEnviron{code}{%
	\par
     \boite{gray}{\texttt{%
     \BODY
     }}
     \par
}

\newenvironment{vbloc}{% boite sans cadre empeche saut de page
     \begin{minipage}[t]{\linewidth}
     }
     {
     \end{minipage}
}
\NewEnviron{h2}{%
    \needspace{3\baselineskip}
    \vspace{0.6cm}
	\noindent \MakeUppercase{\sffamily \large \BODY}
	\vspace{1mm}\textcolor{mcgris}{\hrule}\vspace{0.4cm}
	\par
}{}

\NewEnviron{h3}{%
    \needspace{3\baselineskip}
	\vspace{5mm}
	\textsc{\BODY}
	\par
}

\NewEnviron{margeneg}{ %
\begin{addmargin}[-1cm]{0cm}
\BODY
\end{addmargin}
}

\NewEnviron{html}{%
}

\begin{document}
\meta{url}{/cours/sinus-cosinus/}
\meta{pid}{341}
\meta{titre}{Trigonométrie}
\meta{type}{cours}
\begin{h2}1. Mesures en radians d'un angle orienté  \end{h2}
Dans tout le chapitre, le plan $\mathscr P$ est muni d'un repère orthonormé $\left(O~; \vec{i} , \vec{j}\right)$.
\cadre{bleu}{Définition}{% id="d10"
     Soit $I$ le point de coordonnées $\left(1~; 0\right)$ et $d$ la droite parallèle à l'axe des ordonnées passant par $I$.
     \par
     A tout réel $x$ on associe le point $N$ de la droite $d$ d'ordonnée $x$ puis le point $M$ obtenu en «~enroulant~» la droite $d$ sur le cercle trigonométrique (voir figure ci-dessous).
     \par
     On dit que $x$ est une \textbf{mesure en radians} de l'angle orienté $\left(\overrightarrow{OI}, \overrightarrow{OM}\right)$
}
\begin{center}
     \begin{extern}%width="400" alt="Mesure en radians d'un angle orienté"
          \resizebox{8cm}{!}{
               \newrgbcolor{dblue}{0. 0. 0.7}
               \newrgbcolor{dvert}{0. 0.4 0.}
               \newrgbcolor{dmauve}{0.5 0. 0.5}
               \psset{xunit=5.0cm,yunit=5.0cm,algebraic=true,dimen=middle,dotstyle=o,dotsize=5pt 0,linewidth=0.8pt,arrowsize=3pt 2,arrowinset=0.25}
               \begin{pspicture*}(-1.2,-1.2)(1.2,2.2)
                    \psaxes[linewidth=0.75pt,labelFontSize=\scriptstyle,xAxis=true,yAxis=true,Dx=10.,Dy=10.,ticksize=-2pt 0,subticks=1]{->}(0,0)(-1.2,-1.2)(1.2,2.2)
                    \pscircle[linewidth=0.8pt](0.,0.){5.} %cercle trigo
                    %             \pscircle[linewidth=0.8pt](1.373,0.){9.779} %cercle trigo
                    \parametricplot[linewidth=1.2pt,linecolor=red]{0.0}{1.92}{cos(t)|sin(t)}%arc angle
                    \parametricplot[linewidth=0.8pt,arrows=->]{0.8}{1.3}{1.15*cos(t)|1.15*sin(t)}% sens trigo
                    \rput[tl](0.58,1.07){+}
                    \pscustom[linewidth=0.8pt,linecolor=dmauve,fillcolor=dmauve,fillstyle=solid,opacity=0.1]{ % color angle
                         \parametricplot{0.0}{1.92}{0.15*cos(t)|0.15*sin(t)}
                    \lineto(0.,0.)\closepath}
                    \psellipticarc[linewidth=0.8pt,linecolor=dmauve,arrows=->](0.,0.)(0.15,0.15){0.}{110} % fleche angle
                    \psellipticarc[linewidth=0.8pt,linecolor=dblue,arrows=->](1.373,0.)(1.956,1.956) {101.3}{151} % fleche mn
                    \psline[linewidth=0.8pt,linecolor=dmauve]{->}(0.,0.)(-0.342,0.94)%rayon
                    \psline[linewidth=0.8pt]{->}(0.,0.)(1.,0.) %vecteurs unités
                    \psline[linewidth=0.8pt]{->}(0.,0.)(0,1)
                    \psline[linewidth=0.8pt]{->}(1,-1.5)(1,2.5)
                    %\rput[tl](0.4,0.1){$\vec{i}$}
                    %\rput[tl](-0.06,0.5){$\vec{j}$}
                    \psdots[dotsize=2pt 0,dotstyle=*](0.,0.)
                    \rput[bl](-0.09,-0.09){$O$}
                    \psdots[dotsize=2pt 0,dotstyle=*,linecolor=dblue](1.,0.)
                    \rput[bl](1.02,0.02){\dblue{$I$}}
                    \psdots[dotsize=2pt 0,dotstyle=*,linecolor=dblue](-0.342,0.94)
                    \rput[br](-0.342,0.96){\dblue{$M$}}
                    \psdots[dotsize=2pt 0,dotstyle=*,linecolor=dblue](1.0,1.92)
                    \rput[bl](1.03,1.92){\dblue{$N$}}\rput[l](1.03,0.86){\red{$x$}} \rput[bl](1.03,2.12){$d$}
                    \rput[t](0.5,-0.04){$\overrightarrow{i}$}\rput[l](-0.1,0.5){$\overrightarrow{j}$}
                    \psline[linewidth=1pt,linecolor=red](1.,0.)(1.,1.92)
                    \psdots[dotsize=2pt 0,dotstyle=*,linecolor=dblue](0,1)
                    \rput[bl](0.02,1.03){\dblue{$J$}}
               \end{pspicture*}
          }
     \end{extern}
\end{center}
\begin{center}
     \begin{center}
          \textit{Mesures d'un angle orienté}
     \end{center}
\end{center}
\bloc{cyan}{Remarque}{% id="r10"
     \begin{itemize}
          \item Une infinités de points de la droite $d$ se superposent à $M$ par enroulement (en faisant plusieurs tours). Chaque angle possède une infinité de mesures qui diffèrent entre elles d'un multiple de $2\pi $. Si $x$ est une mesure d'un angle, les autres mesures sont $x+2\pi  , x+4\pi  ,$  etc. et $x-2\pi  , x-4\pi $ ,  etc.
          \par
          Ces différentes mesures s'écrivent $x+2k\pi $ avec $k \in  \mathbb{Z}$
          \item On note de la même façon $\left(\vec{u}, \vec{v}\right)$ l'angle orienté de $\vec{u}$ vers $\vec{v}$et la mesure en radians de cet angle.
     \end{itemize}
}
\cadre{bleu}{Propriété et définition (Mesure principale)}{% id="d20"
     Tout angle orienté $\left(\vec{u}, \vec{v}\right)$ possède une unique mesure dans l'intervalle $\left]-\pi ~; \pi \right]$.
     \par
     Cette mesure s'appelle \textbf{la mesure principale }de l'angle $\left(\vec{u}, \vec{v}\right)$.
}
\bloc{orange}{Exemple}{% id="e20"
     Soit un angle dont une mesure est $-\frac{5\pi }{2}$. Comme $-\frac{5\pi }{2} \notin  \left]-\pi ~; \pi \right]$, ce n'est pas la mesure principale. Comme~: $-\frac{5\pi }{2} = -\frac{\pi }{2}-\frac{4\pi }{2} = -\frac{\pi }{2}-2\pi $ et $-\frac{\pi }{2}\in  \left]-\pi ~; \pi \right]$, $-\frac{\pi }{2}$ est la mesure principale de cet angle.
}
\bloc{orange}{Mesures d'angles à connaitre}{% id="e30"
     \begin{center}
          \begin{extern}%width="400" alt="Mesures d'angles remarquables"
               \newrgbcolor{dblue}{0. 0. 0.7}
               \newrgbcolor{dvert}{0. 0.4 0.}
               \newrgbcolor{dmauve}{0.5 0. 0.5}
               \psset{xunit=5.0cm,yunit=5.0cm,algebraic=true,dimen=middle,dotstyle=o,dotsize=5pt 0,linewidth=0.8pt,arrowsize=3pt 2,arrowinset=0.25}
               \resizebox{8cm}{!}{
                    \begin{pspicture*}(-1.2,-1.2)(1.2,1.2)
                         \psaxes[linewidth=0.75pt,labelFontSize=\scriptstyle,xAxis=true,yAxis=true,Dx=10.,Dy=10.,ticksize=-2pt 0,subticks=1]{->}(0,0)(-1.2,-1.2)(1.2,1.2)
                         \pscircle[linewidth=0.8pt](0.,0.){5.} %cercle trigo
                         \psline[linewidth=0.8pt]{->}(0.,0.)(1.,0.) %vecteurs unités
                         \psline[linewidth=0.8pt]{->}(0.,0.)(0,1)
                         %\rput[tl](0.4,0.1){$\vec{i}$}
                         %\rput[tl](-0.06,0.5){$\vec{j}$}
                         \psdots[dotsize=2pt 0,dotstyle=*](0.,0.)
                         %\rput[bl](-0.09,-0.09){$O$}
                         \psline[linewidth=0.8pt,linecolor=dvert](-0.707,-0.707)(0.707,0.707)
                         \psline[linewidth=0.8pt,linecolor=dvert](-0.707,0.707)(0.707,-0.707)
                         \psline[linewidth=0.8pt,linecolor=red](-0.866,-0.5)(0.866,0.5)
                         \psline[linewidth=0.8pt,linecolor=red](0.866,-0.5)(-0.866,0.5)
                         \rput(0.943,0.55){$\red{\dfrac{\pi}{6}}$}
                         \rput(-0.943,0.55){$\red{\dfrac{5\pi}{6}}$}
                         \rput(-0.973,-0.55){$\red{-\dfrac{5\pi}{6}}$}
                         \rput(0.943,-0.55){$\red{-\dfrac{\pi}{6}}$}
                         %
                         \rput(0.777,0.777){$\dvert{\dfrac{\pi}{4}}$}
                         \rput(-0.777,0.777){$\dvert{\dfrac{3\pi}{4}}$}
                         \rput(-0.807,-0.777){$\dvert{-\dfrac{3\pi}{4}}$}
                         \rput(0.777,-0.777){$\dvert{-\dfrac{\pi}{4}}$}
                         %
                         \psline[linewidth=0.8pt,linecolor=dblue](-0.5,-0.866)(0.5,0.866)
                         \psline[linewidth=0.8pt,linecolor=dblue](-0.5,0.866)(0.5,-0.866)
                         \rput(0.55,0.943){$\dblue{\dfrac{\pi}{3}}$}
                         \rput(-0.55,0.943){$\dblue{\dfrac{2\pi}{3}}$}
                         \rput(-0.58,-0.943){$\dblue{-\dfrac{2\pi}{3}}$}
                         \rput(0.55,-0.943){$\dblue{-\dfrac{\pi}{3}}$}
                         %
                         \rput(1.06,0.06){$0$}
                         \rput(0.06,1.1){$\dfrac{\pi}{2}$}
                         \rput(0.06,-1.1){$-\dfrac{\pi}{2}$}
                         \rput(-1.06,0.06){$\pi$}
                    \end{pspicture*}
               }
          \end{extern}
\end{center}}
\begin{center}
     \textit{ Mesures d'angles remarquables}
\end{center}
\begin{h2}2. Sinus et cosinus - Équations trigonométriques\end{h2}
\cadre{bleu}{Définition}{% id="d40"
     Soit $M$ un point du cercle trigonométrique et $x$ une mesure de l'angle $\widehat{IOM}$.
     \par
     On appelle \textbf{cosinus} de $x$, noté\textbf{ $\cos x$} l'abscisse du point $M$.
     \par
     On appelle \textbf{sinus} de $x$, noté\textbf{ $\sin x$} l'ordonnée du point $M$
}
\begin{center}
     \begin{extern}%width="400" alt="sinus et cosinus d'un angle orienté"
          \resizebox{7cm}{!}{
               \newrgbcolor{dblue}{0. 0. 0.7}
               \newrgbcolor{dvert}{0. 0.4 0.}
               \newrgbcolor{dmauve}{0.5 0. 0.5}
               \psset{xunit=5.0cm,yunit=5.0cm,algebraic=true,dimen=middle,dotstyle=o,dotsize=5pt 0,linewidth=0.8pt,arrowsize=3pt 2,arrowinset=0.25}
               \begin{pspicture*}(-1.2,-1.2)(1.2,1.2)
                    \psaxes[linewidth=0.75pt,labelFontSize=\scriptstyle,xAxis=true,yAxis=true,Dx=10.,Dy=10.,ticksize=-2pt 0,subticks=1]{->}(0,0)(-1.2,-1.2)(1.2,1.2)
                    \pscircle[linewidth=0.8pt](0.,0.){5.} %cercle trigo
                    \parametricplot[linewidth=1.2pt,linecolor=red]{0.0}{0.698}{cos(t)|sin(t)}%arc angle
                    \pscustom[linewidth=0.8pt,linecolor=dmauve,fillcolor=dmauve,fillstyle=solid,opacity=0.1]{ % color angle
                         \parametricplot{0.0}{0.698}{0.15*cos(t)|0.15*sin(t)}
                    \lineto(0.,0.)\closepath}
                    \psellipticarc[linewidth=0.8pt,linecolor=dmauve,arrows=->](0.,0.)(0.15,0.15){0.}{40} % fleche angle
                    \psline[linewidth=0.8pt,linecolor=dmauve](0.,0.)(0.766,0.643)%rayon
                    \psline[linewidth=0.8pt]{->}(0.,0.)(1.,0.) %vecteurs unités
                    \psline[linewidth=0.8pt]{->}(0.,0.)(0,1)
                    %\rput[tl](0.4,0.1){$\vec{i}$}
                    %\rput[tl](-0.06,0.5){$\vec{j}$}
                    \psdots[dotsize=2pt 0,dotstyle=*](0.,0.)
                    \rput[bl](-0.09,-0.09){$O$}
                    \psdots[dotsize=2pt 0,dotstyle=*,linecolor=dblue](1.,0.)
                    \rput[bl](1.02,0.02){\dblue{$I$}}
                    \psdots[dotsize=2pt 0,dotstyle=*,linecolor=dblue](0.766,0.643)
                    \rput[bl](0.78,0.66){\dblue{$M$}}
                    \psdots[dotsize=2pt 0,dotstyle=*,linecolor=dblue](0,1)
                    \rput[bl](0.02,1.03){\dblue{$J$}}
                    \rput[bl](0.19,0.05){\dmauve{$x$}}
                    \psline[linewidth=1pt,linecolor=dvert](0.,0.)(0.766,0)
                    \psline[linewidth=1pt,linecolor=dvert](0.,0.)(0,0.643)
                    \psline[linewidth=0.4pt,linecolor=dvert](0.,0.643)(0.766,0.643)
                    \psline[linewidth=0.4pt,linecolor=dvert](0.766,0.)(0.766,0.643)
                    \rput(0.766,-0.05){\dvert{$\cos x$}}
                    \rput(-0.10,0.643){\dvert{$\sin x$}}
               \end{pspicture*}
          }
     \end{extern}
\end{center}
\begin{center}
     \textit{Sinus et cosinus}
\end{center}
\bloc{cyan}{Remarques}{% id="r40"
     Pour tout réel $x$~:
     \begin{itemize}
          \item $-1 \leqslant  \cos x \leqslant  1$
          \item $-1 \leqslant  \sin x \leqslant  1$
          \item Comme $M$ appartient au cercle trigonométrique, $OM=1$ donc $OM^{2}=1=1$ donc~:
          \par
          $\sin^{2}x+\cos^{2}x=1$ ($\sin^{2}x$ étant une écriture abrégée pour $\left(\sin x\right)^{2}$)
     \end{itemize}
}
\bloc{orange}{Valeurs de sinus et de cosinus à retenir}{% id="r45"
     \begin{center}
          \begin{extern}%width="450" alt="Valeurs de sinus et de cosinus"
               \newrgbcolor{dblue}{0. 0. 0.7}
               \newrgbcolor{dvert}{0. 0.4 0.}
               \newrgbcolor{dmauve}{0.5 0. 0.5}
               \psset{xunit=5.0cm,yunit=5.0cm,algebraic=true,dimen=middle,dotstyle=o,dotsize=5pt 0,linewidth=0.8pt,arrowsize=3pt 2,arrowinset=0.25}
               \begin{pspicture*}(-1.2,-1.2)(1.2,1.2)
                    \psaxes[linewidth=0.75pt,labelFontSize=\scriptstyle,xAxis=true,yAxis=true,Dx=10.,Dy=10.,ticksize=-2pt 0,subticks=1]{->}(0,0)(-1.2,-1.2)(1.2,1.2)
                    \pscircle[linewidth=0.8pt](0.,0.){5.} %cercle trigo
                    \psline[linewidth=0.8pt]{->}(0.,0.)(1.,0.) %vecteurs unités
                    \psline[linewidth=0.8pt]{->}(0.,0.)(0,1)
                    %\rput[tl](0.4,0.1){$\vec{i}$}
                    %\rput[tl](-0.06,0.5){$\vec{j}$}
                    \psdots[dotsize=2pt 0,dotstyle=*](0.,0.)
                    %\rput[bl](-0.09,-0.09){$O$}
                    \psframe[linewidth=0.4pt,linecolor=dvert](-0.707,-0.707)(0.707,0.707)
                    \psline[linewidth=0.8pt,linecolor=dvert](-0.707,-0.707)(0.707,0.707)
                    \psline[linewidth=0.8pt,linecolor=dvert](-0.707,0.707)(0.707,-0.707)
                    \psframe[linewidth=0.4pt,linecolor=red](-0.866,-0.5)(0.866,0.5)
                    \psline[linewidth=0.8pt,linecolor=red](-0.866,-0.5)(0.866,0.5)
                    \psline[linewidth=0.8pt,linecolor=red](0.866,-0.5)(-0.866,0.5)
                    \rput(0.943,0.55){$\red{\dfrac{\pi}{6}}$}
                    \rput(-0.943,0.55){$\red{\dfrac{5\pi}{6}}$}
                    \rput(-0.973,-0.55){$\red{-\dfrac{5\pi}{6}}$}
                    \rput(0.943,-0.55){$\red{-\dfrac{\pi}{6}}$}
                    \rput(0.05,-0.554){\fontsize{7 pt}{7 pt}\selectfont $\red{ -\dfrac{1}{2}}$}
                    \rput(0.05,0.554){\fontsize{7 pt}{7 pt}\selectfont $\red{ \dfrac{1}{2}}$}
                    \rput(0.93,0.07){\fontsize{7 pt}{7 pt}\selectfont $\red{ \dfrac{\sqrt{3}}{2}}$}
                    \rput(-0.93,0.07){\fontsize{7 pt}{7 pt}\selectfont $\red{-\dfrac{\sqrt{3}}{2}}$}
                    %
                    \rput(0.777,0.777){$\dvert{\dfrac{\pi}{4}}$}
                    \rput(-0.777,0.777){$\dvert{\dfrac{3\pi}{4}}$}
                    \rput(-0.807,-0.777){$\dvert{-\dfrac{3\pi}{4}}$}
                    \rput(0.777,-0.777){$\dvert{-\dfrac{\pi}{4}}$}
                    \rput(0.06,-0.77){\fontsize{7 pt}{7 pt}\selectfont $\dvert{ -\dfrac{\sqrt{2}}{2}}$}
                    \rput(0.06,0.77){\fontsize{7 pt}{7 pt}\selectfont $\dvert{ \dfrac{\sqrt{2}}{2}}$}
                    \rput(0.764,0.07){\fontsize{7 pt}{7 pt}\selectfont $\dvert{ \dfrac{\sqrt{2}}{2}}$}
                    \rput(-0.764,0.07){\fontsize{7 pt}{7 pt}\selectfont $\dvert{-\dfrac{\sqrt{2}}{2}}$}
                    %
                    \psframe[linewidth=0.4pt,linecolor=dblue](-0.5,-0.866)(0.5,0.866)
                    \psline[linewidth=0.8pt,linecolor=dblue](-0.5,-0.866)(0.5,0.866)
                    \psline[linewidth=0.8pt,linecolor=dblue](-0.5,0.866)(0.5,-0.866)
                    \rput(0.55,0.943){$\dblue{\dfrac{\pi}{3}}$}
                    \rput(-0.55,0.943){$\dblue{\dfrac{2\pi}{3}}$}
                    \rput(-0.58,-0.943){$\dblue{-\dfrac{2\pi}{3}}$}
                    \rput(0.55,-0.943){$\dblue{-\dfrac{\pi}{3}}$}
                    \rput(0.06,-0.933){\fontsize{7 pt}{7 pt}\selectfont $\dblue{ -\dfrac{\sqrt{3}}{2}}$}
                    \rput(0.06,0.933){\fontsize{7 pt}{7 pt}\selectfont $\dblue{ \dfrac{\sqrt{3}}{2}}$}
                    \rput(0.538,0.07){\fontsize{7 pt}{7 pt}\selectfont $\dblue{ \dfrac{1}{2}}$}
                    \rput(-0.538,0.07){\fontsize{7 pt}{7 pt}\selectfont $\dblue{-\dfrac{1}{2}}$}
                    %
                    \rput(1.06,0.06){$0$}
                    \rput(0.06,1.1){$\dfrac{\pi}{2}$}
                    \rput(0.06,-1.1){$-\dfrac{\pi}{2}$}
                    \rput(-1.06,0.06){$\pi$}
               \end{pspicture*}
          \end{extern}
     \end{center}
     \begin{tabularx}{0.8\linewidth}{|*{10}{>{\centering \arraybackslash }X|}}%class="compact" width="600"
          \hline
          \textbf{$x$}  & $0$ & $\frac{\pi }{6}$ & $\frac{\pi }{4}$ & $\frac{\pi }{3}$ & $\frac{\pi }{2}$ & $\frac{2\pi }{3}$ & $\frac{3\pi }{4}$ & $\frac{5\pi }{6}$ & $\pi $
          \\ \hline
          \textbf{$\cos x$} & $1$ & $\frac{\sqrt{3}}{2}$ & $\frac{\sqrt{2}}{2}$ & $\frac{1}{2}$ & $0$ &  $-\frac{1}{2}$ & $-\frac{\sqrt{2}}{2}$ & $-\frac{\sqrt{3}}{2}$ & $-1$
          \\ \hline
          \textbf{$\sin x$} & $0$ & $\frac{1}{2}$ & $\frac{\sqrt{2}}{2}$ & $\frac{\sqrt{3}}{2}$ & $1$ & $\frac{\sqrt{3}}{2}$ & $\frac{\sqrt{2}}{2}$ & $\frac{1}{2}$ & $0$
          \\  \hline
     \end{tabularx}
     \begin{tabularx}{0.8\linewidth}{|*{8}{>{\centering \arraybackslash }X|}}%class="compact" width="600"
          \hline
          \textbf{$x$} & $-\frac{\pi }{6}$ & $-\frac{\pi }{4}$ & $-\frac{\pi }{3}$ & $-\frac{\pi }{2}$ & $-\frac{2\pi }{3}$ & $-\frac{3\pi }{4}$ & $-\frac{5\pi }{6}$
          \\ \hline
          \textbf{$\cos x$} & $\frac{\sqrt{3}}{2}$ & $\frac{\sqrt{2}}{2}$ & $\frac{1}{2}$ & $0$ &  $-\frac{1}{2}$ & $-\frac{\sqrt{2}}{2}$ & $-\frac{\sqrt{3}}{2}$
          \\ \hline
          \textbf{$\sin x$} & $-\frac{1}{2}$ & $-\frac{\sqrt{2}}{2}$ & $-\frac{\sqrt{3}}{2}$ & $-1$ & $-\frac{\sqrt{3}}{2}$ & $-\frac{\sqrt{2}}{2}$ & $-\frac{1}{2}$
          \\    \hline
     \end{tabularx}
}
\cadre{vert}{Propriétés}{% id="p60"
     Pour tout réel $x$~:
     \begin{itemize}
          \item %
          $\sin\left(-x\right)=-\sin\left(x\right)$
          \item %
          $\cos\left(-x\right)=\cos\left(x\right)$
          \item %
          $\sin\left(\pi +x\right)=-\sin\left(x\right)$
          \item %
          $\cos\left(\pi +x\right)=-\cos\left(x\right)$
     \end{itemize}
}
\begin{center}
     \begin{extern}%width="400" alt=" Angles x, -x et pi+x"
          \resizebox{7cm}{!}{
               \newrgbcolor{dblue}{0. 0. 0.7}
               \newrgbcolor{dvert}{0. 0.4 0.}
               \newrgbcolor{dmauve}{0.5 0. 0.5}
               \psset{xunit=5.0cm,yunit=5.0cm,algebraic=true,dimen=middle,dotstyle=o,dotsize=5pt 0,linewidth=0.8pt,arrowsize=3pt 2,arrowinset=0.25}
               \begin{pspicture*}(-1.2,-1.2)(1.2,1.2)
                    \psaxes[linewidth=0.75pt,labelFontSize=\scriptstyle,xAxis=true,yAxis=true,Dx=10.,Dy=10.,ticksize=-2pt 0,subticks=1]{->}(0,0)(-1.2,-1.2)(1.2,1.2)
                    \pscircle[linewidth=0.8pt](0.,0.){5.} %cercle trigo
                    \psellipticarc[linewidth=0.8pt,linecolor=dmauve,arrows=->](0.,0.)(0.15,0.15){0.}{40} % fleche angle
                    \psellipticarc[linewidth=0.8pt,linecolor=dvert,arrows=<-](0.,0.)(0.15,0.15){-40}{0} % fleche angle -x
                    \psellipticarc[linewidth=0.8pt,linecolor=red,arrows=->](0.,0.)(0.1,0.1){0.}{220} % fleche angle pi+x
                    \psline[linewidth=0.8pt,linecolor=dmauve](0.,0.)(0.766,0.643)%rayon
                    \psline[linewidth=0.8pt,linecolor=dvert](0.,0.)(0.766,-0.643)%rayon -x
                    \psline[linewidth=0.8pt,linecolor=red](0.,0.)(-0.766,-0.643)%rayon -x
                    \psline[linewidth=0.8pt]{->}(0.,0.)(1.,0.) %vecteurs unités
                    \psline[linewidth=0.8pt]{->}(0.,0.)(0,1)
                    %\rput[tl](0.4,0.1){$\vec{i}$}
                    %\rput[tl](-0.06,0.5){$\vec{j}$}
                    \psdots[dotsize=2pt 0,dotstyle=*](0.,0.)
                    \rput[bl](-0.09,-0.09){$O$}
                    \psdots[dotsize=2pt 0,dotstyle=*,linecolor=dblue](1.,0.)
                    \rput[bl](1.02,0.02){\dblue{$I$}}
                    \psdots[dotsize=2pt 0,dotstyle=*,linecolor=dblue](0,1)
                    \rput[bl](0.02,1.03){\dblue{$J$}}
                    \rput[bl](0.19,0.05){\dmauve{$x$}}
                    \rput[tl](0.19,-0.05){\dvert{$-x$}}
                    \rput[br](-0.13,0.07){\red{$\pi+x$}}
                    \psframe[linewidth=0.4pt,linecolor=lightgray](-0.766,-0.643)(0.766,0.643)
                    \rput(0.766,-0.05){\dvert{$\cos x$}}
                    \rput(-0.766,-0.05){\dvert{$-\cos x$}}
                    \rput(-0.11,0.69){\dvert{$\sin x$}}
                    \rput(-0.14,-0.69){\dvert{$-\sin x$}}
               \end{pspicture*}
          }
     \end{extern}
\end{center}
\begin{center}
     \textit{   Angles $x$, $-x$ et $\pi+x$}
\end{center}
\cadre{vert}{Formules d'addition}{% id="p60"
     Pour tous réels $a$ et $b$~:
     \begin{itemize}
          \item $\cos\left(a+b\right)=\cos\left(a\right) \cos\left(b\right)-\sin\left(a\right) \sin\left(b\right)$
          \item $\sin\left(a+b\right)=\sin\left(a\right) \cos\left(b\right)+\cos\left(a\right) \sin\left(b\right)$
     \end{itemize}
}
\cadre{rouge}{Théorème}{% id="t70"
     Soit $a$ un réel fixé.
     \par
     Les solutions de l'équation $\cos\left(x\right)=\cos\left(a\right)$ sont les réels de la forme~:
     \begin{center}$a+2k\pi   $ ou $  -a+2k\pi       $ où $k$ décrit $\mathbb{Z}$\end{center}
}
\bloc{orange}{Exemple}{% id="e70"
     On cherche à résoudre l'équation $\cos\left(x\right)=0$
     \par
     On sait que $\cos\left(\frac{\pi }{2}\right)=0$ ce qui fournit une solution de l'équation mais permet aussi d'écrire l'équation sous la forme $\cos\left(x\right)=\cos\left(\frac{\pi }{2}\right)$
     \par
     D'après le théorème ci-dessus les solutions sont de la forme~:
     \par
     $x=\frac{\pi }{2}+2k\pi $ ou $x=-\frac{\pi }{2}+2k\pi $ avec $k \in  \mathbb{Z}$
}
\cadre{rouge}{Théorème}{% id="t80"
     Soit $a$ un réel fixé.
     \par
     Les solutions de l'équation $\sin\left(x\right)=\sin\left(a\right)$ sont les réels de la forme~:
     \begin{center}$a+2k\pi   $ ou $  \pi -a+2k\pi       $ où $k$ décrit $\mathbb{Z}$\end{center}
}

\end{document}