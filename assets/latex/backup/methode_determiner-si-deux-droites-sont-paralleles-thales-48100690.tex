\documentclass[a4paper]{article}

%================================================================================================================================
%
% Packages
%
%================================================================================================================================

\usepackage[T1]{fontenc} 	% pour caractères accentués
\usepackage[utf8]{inputenc}  % encodage utf8
\usepackage[french]{babel}	% langue : français
\usepackage{fourier}			% caractères plus lisibles
\usepackage[dvipsnames]{xcolor} % couleurs
\usepackage{fancyhdr}		% réglage header footer
\usepackage{needspace}		% empêcher sauts de page mal placés
\usepackage{graphicx}		% pour inclure des graphiques
\usepackage{enumitem,cprotect}		% personnalise les listes d'items (nécessaire pour ol, al ...)
\usepackage{hyperref}		% Liens hypertexte
\usepackage{pstricks,pst-all,pst-node,pstricks-add,pst-math,pst-plot,pst-tree,pst-eucl} % pstricks
\usepackage[a4paper,includeheadfoot,top=2cm,left=3cm, bottom=2cm,right=3cm]{geometry} % marges etc.
\usepackage{comment}			% commentaires multilignes
\usepackage{amsmath,environ} % maths (matrices, etc.)
\usepackage{amssymb,makeidx}
\usepackage{bm}				% bold maths
\usepackage{tabularx}		% tableaux
\usepackage{colortbl}		% tableaux en couleur
\usepackage{fontawesome}		% Fontawesome
\usepackage{environ}			% environment with command
\usepackage{fp}				% calculs pour ps-tricks
\usepackage{multido}			% pour ps tricks
\usepackage[np]{numprint}	% formattage nombre
\usepackage{tikz,tkz-tab} 			% package principal TikZ
\usepackage{pgfplots}   % axes
\usepackage{mathrsfs}    % cursives
\usepackage{calc}			% calcul taille boites
\usepackage[scaled=0.875]{helvet} % font sans serif
\usepackage{svg} % svg
\usepackage{scrextend} % local margin
\usepackage{scratch} %scratch
\usepackage{multicol} % colonnes
%\usepackage{infix-RPN,pst-func} % formule en notation polanaise inversée
\usepackage{listings}

%================================================================================================================================
%
% Réglages de base
%
%================================================================================================================================

\lstset{
language=Python,   % R code
literate=
{á}{{\'a}}1
{à}{{\`a}}1
{ã}{{\~a}}1
{é}{{\'e}}1
{è}{{\`e}}1
{ê}{{\^e}}1
{í}{{\'i}}1
{ó}{{\'o}}1
{õ}{{\~o}}1
{ú}{{\'u}}1
{ü}{{\"u}}1
{ç}{{\c{c}}}1
{~}{{ }}1
}


\definecolor{codegreen}{rgb}{0,0.6,0}
\definecolor{codegray}{rgb}{0.5,0.5,0.5}
\definecolor{codepurple}{rgb}{0.58,0,0.82}
\definecolor{backcolour}{rgb}{0.95,0.95,0.92}

\lstdefinestyle{mystyle}{
    backgroundcolor=\color{backcolour},   
    commentstyle=\color{codegreen},
    keywordstyle=\color{magenta},
    numberstyle=\tiny\color{codegray},
    stringstyle=\color{codepurple},
    basicstyle=\ttfamily\footnotesize,
    breakatwhitespace=false,         
    breaklines=true,                 
    captionpos=b,                    
    keepspaces=true,                 
    numbers=left,                    
xleftmargin=2em,
framexleftmargin=2em,            
    showspaces=false,                
    showstringspaces=false,
    showtabs=false,                  
    tabsize=2,
    upquote=true
}

\lstset{style=mystyle}


\lstset{style=mystyle}
\newcommand{\imgdir}{C:/laragon/www/newmc/assets/imgsvg/}
\newcommand{\imgsvgdir}{C:/laragon/www/newmc/assets/imgsvg/}

\definecolor{mcgris}{RGB}{220, 220, 220}% ancien~; pour compatibilité
\definecolor{mcbleu}{RGB}{52, 152, 219}
\definecolor{mcvert}{RGB}{125, 194, 70}
\definecolor{mcmauve}{RGB}{154, 0, 215}
\definecolor{mcorange}{RGB}{255, 96, 0}
\definecolor{mcturquoise}{RGB}{0, 153, 153}
\definecolor{mcrouge}{RGB}{255, 0, 0}
\definecolor{mclightvert}{RGB}{205, 234, 190}

\definecolor{gris}{RGB}{220, 220, 220}
\definecolor{bleu}{RGB}{52, 152, 219}
\definecolor{vert}{RGB}{125, 194, 70}
\definecolor{mauve}{RGB}{154, 0, 215}
\definecolor{orange}{RGB}{255, 96, 0}
\definecolor{turquoise}{RGB}{0, 153, 153}
\definecolor{rouge}{RGB}{255, 0, 0}
\definecolor{lightvert}{RGB}{205, 234, 190}
\setitemize[0]{label=\color{lightvert}  $\bullet$}

\pagestyle{fancy}
\renewcommand{\headrulewidth}{0.2pt}
\fancyhead[L]{maths-cours.fr}
\fancyhead[R]{\thepage}
\renewcommand{\footrulewidth}{0.2pt}
\fancyfoot[C]{}

\newcolumntype{C}{>{\centering\arraybackslash}X}
\newcolumntype{s}{>{\hsize=.35\hsize\arraybackslash}X}

\setlength{\parindent}{0pt}		 
\setlength{\parskip}{3mm}
\setlength{\headheight}{1cm}

\def\ebook{ebook}
\def\book{book}
\def\web{web}
\def\type{web}

\newcommand{\vect}[1]{\overrightarrow{\,\mathstrut#1\,}}

\def\Oij{$\left(\text{O}~;~\vect{\imath},~\vect{\jmath}\right)$}
\def\Oijk{$\left(\text{O}~;~\vect{\imath},~\vect{\jmath},~\vect{k}\right)$}
\def\Ouv{$\left(\text{O}~;~\vect{u},~\vect{v}\right)$}

\hypersetup{breaklinks=true, colorlinks = true, linkcolor = OliveGreen, urlcolor = OliveGreen, citecolor = OliveGreen, pdfauthor={Didier BONNEL - https://www.maths-cours.fr} } % supprime les bordures autour des liens

\renewcommand{\arg}[0]{\text{arg}}

\everymath{\displaystyle}

%================================================================================================================================
%
% Macros - Commandes
%
%================================================================================================================================

\newcommand\meta[2]{    			% Utilisé pour créer le post HTML.
	\def\titre{titre}
	\def\url{url}
	\def\arg{#1}
	\ifx\titre\arg
		\newcommand\maintitle{#2}
		\fancyhead[L]{#2}
		{\Large\sffamily \MakeUppercase{#2}}
		\vspace{1mm}\textcolor{mcvert}{\hrule}
	\fi 
	\ifx\url\arg
		\fancyfoot[L]{\href{https://www.maths-cours.fr#2}{\black \footnotesize{https://www.maths-cours.fr#2}}}
	\fi 
}


\newcommand\TitreC[1]{    		% Titre centré
     \needspace{3\baselineskip}
     \begin{center}\textbf{#1}\end{center}
}

\newcommand\newpar{    		% paragraphe
     \par
}

\newcommand\nosp {    		% commande vide (pas d'espace)
}
\newcommand{\id}[1]{} %ignore

\newcommand\boite[2]{				% Boite simple sans titre
	\vspace{5mm}
	\setlength{\fboxrule}{0.2mm}
	\setlength{\fboxsep}{5mm}	
	\fcolorbox{#1}{#1!3}{\makebox[\linewidth-2\fboxrule-2\fboxsep]{
  		\begin{minipage}[t]{\linewidth-2\fboxrule-4\fboxsep}\setlength{\parskip}{3mm}
  			 #2
  		\end{minipage}
	}}
	\vspace{5mm}
}

\newcommand\CBox[4]{				% Boites
	\vspace{5mm}
	\setlength{\fboxrule}{0.2mm}
	\setlength{\fboxsep}{5mm}
	
	\fcolorbox{#1}{#1!3}{\makebox[\linewidth-2\fboxrule-2\fboxsep]{
		\begin{minipage}[t]{1cm}\setlength{\parskip}{3mm}
	  		\textcolor{#1}{\LARGE{#2}}    
 	 	\end{minipage}  
  		\begin{minipage}[t]{\linewidth-2\fboxrule-4\fboxsep}\setlength{\parskip}{3mm}
			\raisebox{1.2mm}{\normalsize\sffamily{\textcolor{#1}{#3}}}						
  			 #4
  		\end{minipage}
	}}
	\vspace{5mm}
}

\newcommand\cadre[3]{				% Boites convertible html
	\par
	\vspace{2mm}
	\setlength{\fboxrule}{0.1mm}
	\setlength{\fboxsep}{5mm}
	\fcolorbox{#1}{white}{\makebox[\linewidth-2\fboxrule-2\fboxsep]{
  		\begin{minipage}[t]{\linewidth-2\fboxrule-4\fboxsep}\setlength{\parskip}{3mm}
			\raisebox{-2.5mm}{\sffamily \small{\textcolor{#1}{\MakeUppercase{#2}}}}		
			\par		
  			 #3
 	 		\end{minipage}
	}}
		\vspace{2mm}
	\par
}

\newcommand\bloc[3]{				% Boites convertible html sans bordure
     \needspace{2\baselineskip}
     {\sffamily \small{\textcolor{#1}{\MakeUppercase{#2}}}}    
		\par		
  			 #3
		\par
}

\newcommand\CHelp[1]{
     \CBox{Plum}{\faInfoCircle}{À RETENIR}{#1}
}

\newcommand\CUp[1]{
     \CBox{NavyBlue}{\faThumbsOUp}{EN PRATIQUE}{#1}
}

\newcommand\CInfo[1]{
     \CBox{Sepia}{\faArrowCircleRight}{REMARQUE}{#1}
}

\newcommand\CRedac[1]{
     \CBox{PineGreen}{\faEdit}{BIEN R\'EDIGER}{#1}
}

\newcommand\CError[1]{
     \CBox{Red}{\faExclamationTriangle}{ATTENTION}{#1}
}

\newcommand\TitreExo[2]{
\needspace{4\baselineskip}
 {\sffamily\large EXERCICE #1\ (\emph{#2 points})}
\vspace{5mm}
}

\newcommand\img[2]{
          \includegraphics[width=#2\paperwidth]{\imgdir#1}
}

\newcommand\imgsvg[2]{
       \begin{center}   \includegraphics[width=#2\paperwidth]{\imgsvgdir#1} \end{center}
}


\newcommand\Lien[2]{
     \href{#1}{#2 \tiny \faExternalLink}
}
\newcommand\mcLien[2]{
     \href{https~://www.maths-cours.fr/#1}{#2 \tiny \faExternalLink}
}

\newcommand{\euro}{\eurologo{}}

%================================================================================================================================
%
% Macros - Environement
%
%================================================================================================================================

\newenvironment{tex}{ %
}
{%
}

\newenvironment{indente}{ %
	\setlength\parindent{10mm}
}

{
	\setlength\parindent{0mm}
}

\newenvironment{corrige}{%
     \needspace{3\baselineskip}
     \medskip
     \textbf{\textsc{Corrigé}}
     \medskip
}
{
}

\newenvironment{extern}{%
     \begin{center}
     }
     {
     \end{center}
}

\NewEnviron{code}{%
	\par
     \boite{gray}{\texttt{%
     \BODY
     }}
     \par
}

\newenvironment{vbloc}{% boite sans cadre empeche saut de page
     \begin{minipage}[t]{\linewidth}
     }
     {
     \end{minipage}
}
\NewEnviron{h2}{%
    \needspace{3\baselineskip}
    \vspace{0.6cm}
	\noindent \MakeUppercase{\sffamily \large \BODY}
	\vspace{1mm}\textcolor{mcgris}{\hrule}\vspace{0.4cm}
	\par
}{}

\NewEnviron{h3}{%
    \needspace{3\baselineskip}
	\vspace{5mm}
	\textsc{\BODY}
	\par
}

\NewEnviron{margeneg}{ %
\begin{addmargin}[-1cm]{0cm}
\BODY
\end{addmargin}
}

\NewEnviron{html}{%
}

\begin{document}
\meta{url}{/methode/determiner-si-deux-droites-sont-paralleles-thales/}
\meta{pid}{11546}
\meta{titre}{Déterminer si deux droites sont parallèles (Thalès)}
\meta{type}{methode}
%
\begin{h2} Problème~: \end{h2}
$ A $ et $ B $ sont deux points d'une droite $\mathscr{D} $ et $ C $ et $ D $ deux points d'une droite $ \mathscr{D'} $.
\\
Les droites $ \left( AC \right) $ et $ \left( BD \right) $ sont sécantes en un point $ M $.
\\
Deux cas de figure sont possibles~:

     \begin{center}
          \begin{extern}%style="width:30rem" alt="Thales 1"
               \newrgbcolor{tttttt}{0.2 0.2 0.2}
               \psset{xunit=1.0cm,yunit=1.0cm,algebraic=true,dimen=middle,dotstyle=o,dotsize=5pt 0,linewidth=1.6pt,arrowsize=3pt 2,arrowinset=0.25}
               \begin{pspicture*}(1.,1.)(7.,8.)
                    \rput[tl](5.3728597107438,7.1){$\tttttt{\mathscr{D}}$}
                    \rput[tl](6.487572210743799,4.1){$\tttttt{\mathscr{D'}}$}
                    \psline[linewidth=0.4pt,linecolor=tttttt](5.,7.)(2.,2.)
                    \psline[linewidth=0.4pt,linecolor=tttttt](6.,4.)(3.,6.)
                    \psplot[linewidth=0.4pt,linecolor=tttttt]{1.}{7.}{(--9.--1.*x)/2.}
                    \psplot[linewidth=0.4pt,linecolor=tttttt]{1.}{7.}{(--4.--2.*x)/4.}
                    \begin{scriptsize}
                         \psdots[dotsize=2pt 0,dotstyle=*,linecolor=tttttt](3.,6.)
                         \rput[bl](2.7690909607438003,6.065112062122526){\tttttt{$A$}}
                         \psdots[dotsize=2pt 0,dotstyle=*,linecolor=tttttt](5.,7.)
                         \rput[bl](5.0151534607438,7.115625918895607){\tttttt{$B$}}
                         \psdots[dotsize=2pt 0,dotstyle=*,linecolor=tttttt](6.,4.)
                         \rput[bl](6.079953460743799,3.780661294219158){\tttttt{$D$}}
                         \psdots[dotsize=2pt 0,dotstyle=*,linecolor=tttttt](2.,2.)
                         \rput[bl](1.8873034607438006,1.7630076962899062){\tttttt{$C$}}
                         \psdots[dotsize=2pt 0,dotstyle=*,linecolor=tttttt](4.,5.333333333333333)
                         \rput[bl](3.6,5.2){\tttttt{$M$}}
                    \end{scriptsize}
               \end{pspicture*}
          \end{extern}
     \end{center}

     \begin{center}
          \begin{extern}%width="500" alt=""
               \newrgbcolor{tttttt}{0.2 0.2 0.2}
               \psset{xunit=1.0cm,yunit=1.0cm,algebraic=true,dimen=middle,dotstyle=o,dotsize=5pt 0,linewidth=1.6pt,arrowsize=3pt 2,arrowinset=0.25}
               \begin{pspicture*}(1.,1.)(7.,8.)
                    \psplot[linewidth=0.4pt,linecolor=tttttt]{1.}{7.}{(--5.841086716091681--1.0004893874029346*x)/2.0131375}
                    \psplot[linewidth=0.4pt,linecolor=tttttt]{1.}{7.}{(--4.--2.*x)/4.}
                    \rput[tl](5.3728597107438,5.5){$\tttttt{\mathscr{D}}$}
                    \rput[tl](6.487572210743799,4.1){$\tttttt{\mathscr{D'}}$}
                    \psline[linewidth=0.4pt,linecolor=tttttt](3.997317828962791,6.802444984387303)(2.,2.)
                    \psline[linewidth=0.4pt,linecolor=tttttt](3.997317828962791,6.802444984387303)(6.,4.)
                    \begin{scriptsize}
                         \psdots[dotsize=2pt 0,dotstyle=*,linecolor=tttttt](2.9936972107438002,4.38929233822261)
                         \rput[bl](2.7607722107438004,4.455991630716139){\tttttt{$A$}}
                         \psdots[dotsize=2pt 0,dotstyle=*,linecolor=tttttt](5.0068347107438,5.389781725625545)
                         \rput[bl](5.0234722107438,5.50650548748922){\tttttt{$B$}}
                         \psdots[dotsize=2pt 0,dotstyle=*,linecolor=tttttt](6.,4.)
                         \rput[bl](6.079953460743799,3.780661294219158){\tttttt{$D$}}
                         \psdots[dotsize=2pt 0,dotstyle=*,linecolor=tttttt](2.,2.)
                         \rput[bl](1.8873034607438006,1.7630076962899062){\tttttt{$C$}}
                         \psdots[dotsize=2pt 0,dotstyle=*,linecolor=darkgray](3.997317828962791,6.802444984387303)
                         \rput[bl](4.0335409607438,6.832153925798109){\darkgray{$M$}}
                    \end{scriptsize}
               \end{pspicture*}
          \end{extern}
     \end{center}

On connait les longueurs $ AM, BM, CM, DM. $
\\
On se pose la question suivante~:
\begin{center}
     \textbf{ Les droites $\mathscr{D} $ et $ \mathscr{D'} $ sont-elles parallèles~?}
\end{center}
\begin{h2} Méthode \end{h2}
On calcule séparément chacun des deux rapports $ \frac{ MA }{ MD } $ et $ \frac{ MB }{ MC } . $
\begin{itemize}
     \item
     Si ces deux rapports sont égaux, les droites $\mathscr{D} $ et $\mathscr{D'} $ sont parallèles d'après la \mcLien{https://www.maths-cours.fr/cours/theoreme-thales/\#t50}{réciproque du théorème de Thalès}
     \item
     Sinon, les droites $\mathscr{D} $ et $\mathscr{D'} $ ne sont pas parallèles (en effet, si elles étaient parallèles, on aurait $ \frac{ MA }{ MD } = \frac{ MB }{ MC } $ d'après le théorème de Thalès.)
\end{itemize}
\bloc{cyan}{Attention à la rédaction~:}{ % id=r010
     \begin{itemize}
          \item
          N'écrivez pas $ \frac{ MA }{ MD } = \frac{ MB }{ MC } $ tant que vous n'avez pas prouvé que ces rapports étaient égaux~!
          \item
          La \textbf{ réciproque } du théorème de Thalès sert à prouver que les droites $\mathscr{D} $ et $\mathscr{D'} $ \textbf{sont} parallèles. Ne citez pas la réciproque du théorème de Thalès si $\mathscr{D} $ et $\mathscr{D'} $ \textbf{ne sont pas} parallèles~!
          \item
          Il faut faire attention à l'alignement et à l' \textbf{ordre} des points.
          //
          Dans le cas de la première figure ci-dessus, on signalera que $ A, M, D $ et $ B, M, C $ sont dans le même ordre~; dans le cas de la seconde figure, que $ M, A, D $ et $ M, B, C $ sont dans le même ordre.
     \end{itemize}
} % fin Attention à la rédaction
\bloc{cyan}{Remarque:}{ % id=r030
     N'oubliez pas qu'il existe d'autres méthodes pour démontrer que deux droites sont parallèles.
     \par
     En particulier, un théorème couramment utilisé au collège pour montrer que deux droites sont parallèles est le suivant~:
     \cadre{rouge}{Théorème}{ % id=t030
          Deux droites prependiculaires à une même troisième sont parallèles entre elles.
     } % fin théorème
     \begin{center}
          \begin{extern}%width="400" alt="perpendiculaires et parallèles "
               \newrgbcolor{grey}{0.2 0.2 0.2}
               \psset{xunit=1.0cm,yunit=1.0cm,algebraic=true,dimen=middle,dotstyle=o,dotsize=5pt 0,linewidth=1.6pt,arrowsize=3pt 2,arrowinset=0.25}
               \fontsize{9pt}{9pt}\selectfont
               \begin{pspicture*}(0.,0.)(5.,5.)
                    \pspolygon[linewidth=0.4pt,linecolor=grey](2.2980044729391027,3.6490022364695514)(2.2490022364695514,3.747006709408654)(2.150997763530449,3.6980044729391026)(2.2,3.6)
                    \pspolygon[linewidth=0.4pt,linecolor=grey](2.6980044729391026,2.849002236469551)(2.6490022364695514,2.9470067094086536)(2.550997763530449,2.8980044729391023)(2.6,2.8)
                    \psplot[linewidth=0.4pt,linecolor=red]{0.}{5.}{(--5.--1.*x)/2.}
                    \psplot[linewidth=0.4pt,linecolor=red]{0.}{5.}{(--3.--1.*x)/2.}
                    \psplot[linewidth=0.4pt,linecolor=blue]{0.}{3.5}{(--8.-2.*x)/1.}
                    \rput[tl](2.884090909090908,4.434685073339088){$\red{\mathscr{D}}$}
                    \rput[tl](3.1940082644628086,3.596031061259709){$\red{\mathscr{D'}}$}
                    \rput[tl](1.9369834710743792,4.6106988783434025){$\blue{(d)}$}
               \end{pspicture*}
          \end{extern}
     \end{center}
} % fin remarque
\begin{h2}Exemple 1 \end{h2}
Dans la figure ci-dessous, tracée à main levée, on a~: $ IM=5 $cm, $ IJ = 3 $cm, $ IL=4 $cm et $ IK=2,5 $cm.
\begin{center}
     \imgsvg{reciproque-thales}{0.1}%width="400" alt="thales et reciproque"
\end{center}
Les droites $(JK)$ et $ (LM) $ sont-elles parallèles~?
\bloc{orange}{Solution}{ % id=s040
     Les droites $ \left( JM \right) $ et $ \left( KL \right) $ sont sécantes en $I$.
     \\
     Les points $ J, I, M $ et $ K, I, L $ sont dans le même ordre.
     \par
     On calcule $ \frac{ IJ }{ IM } $ et $ \frac{ IK }{ IL } $~:
     \par
     $ \frac{ IJ }{ IM } = \frac{ 3 }{ 5 } $
     \par
     $ \frac{ IK }{ IL } = \frac{ 2,5 }{ 4 } = \frac{ 25 }{ 40 } = \frac{ 5 }{ 8 } $
     \par
     Les rapports $ \frac{ IJ }{ IM } $ et $ \frac{ IK }{ IL } $ ne sont pas égaux~; par conséquent, les droites $ \left( JK \right) $ et $ \left( LM \right) $ ne sont pas parallèles.
} % fin Solution
\begin{h2}Exemple 2 \end{h2}
\begin{center}
     \begin{extern}%width="500" alt="réciproque du théorème de Thalès"
          \newrgbcolor{tttttt}{0.2 0.2 0.2}
          \psset{xunit=1.0cm,yunit=1.0cm,algebraic=true,dimen=middle,dotstyle=o,dotsize=5pt 0,linewidth=1.6pt,arrowsize=3pt 2,arrowinset=0.25}
          \begin{pspicture*}(0.,0.)(8.,6.)
               \psplot[linewidth=0.4pt,linecolor=tttttt]{0.}{8.}{(--5.389132322680942--0.606992187729279*x)/5.996124510410222}
               \psline[linewidth=0.4pt,linecolor=tttttt](2.,5.)(1.,1.)
               \psplot[linewidth=0.4pt,linecolor=tttttt]{0.}{8.}{(--19.372549727742015--0.606992187729279*x)/5.996124510410222}
               \psline[linewidth=0.4pt,linecolor=tttttt](2.,5.)(6.996124510410222,1.606992187729279)
               \begin{scriptsize}
                    \psdots[dotsize=2pt 0,dotstyle=*,linecolor=tttttt](2.,5.)
                    \rput[bl](1.9064419707276834,5.085640095555049){\tttttt{$A$}}
                    \psdots[dotsize=2pt 0,dotstyle=*,linecolor=tttttt](1.,1.)
                    \rput[bl](0.9049680478251896,0.6757659957857199){\tttttt{$B$}}
                    \psdots[dotsize=2pt 0,dotstyle=*,linecolor=tttttt](6.996124510410222,1.606992187729279)
                    \rput[bl](7.050999793856933,1.3144978370986039){\tttttt{$C$}}
                    \psdots[dotsize=2pt 0,dotstyle=*](1.5981569416534365,3.3926277666137454)
                    \rput[bl](1.35,3.449438115194311){$D$}
                    \psdots[dotsize=2pt 0,dotstyle=*,linecolor=darkgray](4.007657953143472,3.63654336372335)
                    \rput[bl](4.07,3.710056053482655){\darkgray{$E$}}
               \end{scriptsize}
          \end{pspicture*}
     \end{extern}
\end{center}
Sur la figure ci-dessus, on sait que~:
\begin{itemize}
     \item
     $ AD = 2 $ cm
     \item
     $ AB = 5 $ cm
     \item
     $ AE = 3 $ cm
     \item
     $ EC = 4,5 $ cm
\end{itemize}
Les droites $ \left( BD \right) $ et $ \left( EC \right) $ sont-elles parallèles~?
\bloc{orange}{Solution}{ % id=e060
     Les droites $ \left(BD \right) $ et $ \left( EC \right) $ sont sécantes au point $A$.
     \\
     Les points $ A, D, B $ et les points $ A, E, C $ sont situés dans le même ordre.
     \par
     On va comparer les rapports $ \frac{AD}{ AB } $ et $ \frac{ AE }{ AC } $.
     \par
     Pour cela, on calcule d'abord $ AC $~:
     \\
     $ AC = AE + EC = 3 + 4,5 = 7,5 .$
     \par
     Alors~:
     \par
     $ \frac{ AD }{ AB } = \frac{ 2 }{ 5 } $
     \par
     $ \frac{ AE }{ AC } = \frac{ 3 }{ 7,5 } = \frac{ 2 }{ 5 } $
     \par
     Les rapports $ \frac{ AD }{ AB } $ et $ \frac{ AE }{ AC }$ sont égaux, donc, d'après \textbf{la réciproque du théorème de Thalès} les droites $ \left( DE \right) $ et $ \left( BC \right) $ sont parallèles.
} % fin Solution

\end{document}