\documentclass[a4paper]{article}

%================================================================================================================================
%
% Packages
%
%================================================================================================================================

\usepackage[T1]{fontenc} 	% pour caractères accentués
\usepackage[utf8]{inputenc}  % encodage utf8
\usepackage[french]{babel}	% langue : français
\usepackage{fourier}			% caractères plus lisibles
\usepackage[dvipsnames]{xcolor} % couleurs
\usepackage{fancyhdr}		% réglage header footer
\usepackage{needspace}		% empêcher sauts de page mal placés
\usepackage{graphicx}		% pour inclure des graphiques
\usepackage{enumitem,cprotect}		% personnalise les listes d'items (nécessaire pour ol, al ...)
\usepackage{hyperref}		% Liens hypertexte
\usepackage{pstricks,pst-all,pst-node,pstricks-add,pst-math,pst-plot,pst-tree,pst-eucl} % pstricks
\usepackage[a4paper,includeheadfoot,top=2cm,left=3cm, bottom=2cm,right=3cm]{geometry} % marges etc.
\usepackage{comment}			% commentaires multilignes
\usepackage{amsmath,environ} % maths (matrices, etc.)
\usepackage{amssymb,makeidx}
\usepackage{bm}				% bold maths
\usepackage{tabularx}		% tableaux
\usepackage{colortbl}		% tableaux en couleur
\usepackage{fontawesome}		% Fontawesome
\usepackage{environ}			% environment with command
\usepackage{fp}				% calculs pour ps-tricks
\usepackage{multido}			% pour ps tricks
\usepackage[np]{numprint}	% formattage nombre
\usepackage{tikz,tkz-tab} 			% package principal TikZ
\usepackage{pgfplots}   % axes
\usepackage{mathrsfs}    % cursives
\usepackage{calc}			% calcul taille boites
\usepackage[scaled=0.875]{helvet} % font sans serif
\usepackage{svg} % svg
\usepackage{scrextend} % local margin
\usepackage{scratch} %scratch
\usepackage{multicol} % colonnes
%\usepackage{infix-RPN,pst-func} % formule en notation polanaise inversée
\usepackage{listings}

%================================================================================================================================
%
% Réglages de base
%
%================================================================================================================================

\lstset{
language=Python,   % R code
literate=
{á}{{\'a}}1
{à}{{\`a}}1
{ã}{{\~a}}1
{é}{{\'e}}1
{è}{{\`e}}1
{ê}{{\^e}}1
{í}{{\'i}}1
{ó}{{\'o}}1
{õ}{{\~o}}1
{ú}{{\'u}}1
{ü}{{\"u}}1
{ç}{{\c{c}}}1
{~}{{ }}1
}


\definecolor{codegreen}{rgb}{0,0.6,0}
\definecolor{codegray}{rgb}{0.5,0.5,0.5}
\definecolor{codepurple}{rgb}{0.58,0,0.82}
\definecolor{backcolour}{rgb}{0.95,0.95,0.92}

\lstdefinestyle{mystyle}{
    backgroundcolor=\color{backcolour},   
    commentstyle=\color{codegreen},
    keywordstyle=\color{magenta},
    numberstyle=\tiny\color{codegray},
    stringstyle=\color{codepurple},
    basicstyle=\ttfamily\footnotesize,
    breakatwhitespace=false,         
    breaklines=true,                 
    captionpos=b,                    
    keepspaces=true,                 
    numbers=left,                    
xleftmargin=2em,
framexleftmargin=2em,            
    showspaces=false,                
    showstringspaces=false,
    showtabs=false,                  
    tabsize=2,
    upquote=true
}

\lstset{style=mystyle}


\lstset{style=mystyle}
\newcommand{\imgdir}{C:/laragon/www/newmc/assets/imgsvg/}
\newcommand{\imgsvgdir}{C:/laragon/www/newmc/assets/imgsvg/}

\definecolor{mcgris}{RGB}{220, 220, 220}% ancien~; pour compatibilité
\definecolor{mcbleu}{RGB}{52, 152, 219}
\definecolor{mcvert}{RGB}{125, 194, 70}
\definecolor{mcmauve}{RGB}{154, 0, 215}
\definecolor{mcorange}{RGB}{255, 96, 0}
\definecolor{mcturquoise}{RGB}{0, 153, 153}
\definecolor{mcrouge}{RGB}{255, 0, 0}
\definecolor{mclightvert}{RGB}{205, 234, 190}

\definecolor{gris}{RGB}{220, 220, 220}
\definecolor{bleu}{RGB}{52, 152, 219}
\definecolor{vert}{RGB}{125, 194, 70}
\definecolor{mauve}{RGB}{154, 0, 215}
\definecolor{orange}{RGB}{255, 96, 0}
\definecolor{turquoise}{RGB}{0, 153, 153}
\definecolor{rouge}{RGB}{255, 0, 0}
\definecolor{lightvert}{RGB}{205, 234, 190}
\setitemize[0]{label=\color{lightvert}  $\bullet$}

\pagestyle{fancy}
\renewcommand{\headrulewidth}{0.2pt}
\fancyhead[L]{maths-cours.fr}
\fancyhead[R]{\thepage}
\renewcommand{\footrulewidth}{0.2pt}
\fancyfoot[C]{}

\newcolumntype{C}{>{\centering\arraybackslash}X}
\newcolumntype{s}{>{\hsize=.35\hsize\arraybackslash}X}

\setlength{\parindent}{0pt}		 
\setlength{\parskip}{3mm}
\setlength{\headheight}{1cm}

\def\ebook{ebook}
\def\book{book}
\def\web{web}
\def\type{web}

\newcommand{\vect}[1]{\overrightarrow{\,\mathstrut#1\,}}

\def\Oij{$\left(\text{O}~;~\vect{\imath},~\vect{\jmath}\right)$}
\def\Oijk{$\left(\text{O}~;~\vect{\imath},~\vect{\jmath},~\vect{k}\right)$}
\def\Ouv{$\left(\text{O}~;~\vect{u},~\vect{v}\right)$}

\hypersetup{breaklinks=true, colorlinks = true, linkcolor = OliveGreen, urlcolor = OliveGreen, citecolor = OliveGreen, pdfauthor={Didier BONNEL - https://www.maths-cours.fr} } % supprime les bordures autour des liens

\renewcommand{\arg}[0]{\text{arg}}

\everymath{\displaystyle}

%================================================================================================================================
%
% Macros - Commandes
%
%================================================================================================================================

\newcommand\meta[2]{    			% Utilisé pour créer le post HTML.
	\def\titre{titre}
	\def\url{url}
	\def\arg{#1}
	\ifx\titre\arg
		\newcommand\maintitle{#2}
		\fancyhead[L]{#2}
		{\Large\sffamily \MakeUppercase{#2}}
		\vspace{1mm}\textcolor{mcvert}{\hrule}
	\fi 
	\ifx\url\arg
		\fancyfoot[L]{\href{https://www.maths-cours.fr#2}{\black \footnotesize{https://www.maths-cours.fr#2}}}
	\fi 
}


\newcommand\TitreC[1]{    		% Titre centré
     \needspace{3\baselineskip}
     \begin{center}\textbf{#1}\end{center}
}

\newcommand\newpar{    		% paragraphe
     \par
}

\newcommand\nosp {    		% commande vide (pas d'espace)
}
\newcommand{\id}[1]{} %ignore

\newcommand\boite[2]{				% Boite simple sans titre
	\vspace{5mm}
	\setlength{\fboxrule}{0.2mm}
	\setlength{\fboxsep}{5mm}	
	\fcolorbox{#1}{#1!3}{\makebox[\linewidth-2\fboxrule-2\fboxsep]{
  		\begin{minipage}[t]{\linewidth-2\fboxrule-4\fboxsep}\setlength{\parskip}{3mm}
  			 #2
  		\end{minipage}
	}}
	\vspace{5mm}
}

\newcommand\CBox[4]{				% Boites
	\vspace{5mm}
	\setlength{\fboxrule}{0.2mm}
	\setlength{\fboxsep}{5mm}
	
	\fcolorbox{#1}{#1!3}{\makebox[\linewidth-2\fboxrule-2\fboxsep]{
		\begin{minipage}[t]{1cm}\setlength{\parskip}{3mm}
	  		\textcolor{#1}{\LARGE{#2}}    
 	 	\end{minipage}  
  		\begin{minipage}[t]{\linewidth-2\fboxrule-4\fboxsep}\setlength{\parskip}{3mm}
			\raisebox{1.2mm}{\normalsize\sffamily{\textcolor{#1}{#3}}}						
  			 #4
  		\end{minipage}
	}}
	\vspace{5mm}
}

\newcommand\cadre[3]{				% Boites convertible html
	\par
	\vspace{2mm}
	\setlength{\fboxrule}{0.1mm}
	\setlength{\fboxsep}{5mm}
	\fcolorbox{#1}{white}{\makebox[\linewidth-2\fboxrule-2\fboxsep]{
  		\begin{minipage}[t]{\linewidth-2\fboxrule-4\fboxsep}\setlength{\parskip}{3mm}
			\raisebox{-2.5mm}{\sffamily \small{\textcolor{#1}{\MakeUppercase{#2}}}}		
			\par		
  			 #3
 	 		\end{minipage}
	}}
		\vspace{2mm}
	\par
}

\newcommand\bloc[3]{				% Boites convertible html sans bordure
     \needspace{2\baselineskip}
     {\sffamily \small{\textcolor{#1}{\MakeUppercase{#2}}}}    
		\par		
  			 #3
		\par
}

\newcommand\CHelp[1]{
     \CBox{Plum}{\faInfoCircle}{À RETENIR}{#1}
}

\newcommand\CUp[1]{
     \CBox{NavyBlue}{\faThumbsOUp}{EN PRATIQUE}{#1}
}

\newcommand\CInfo[1]{
     \CBox{Sepia}{\faArrowCircleRight}{REMARQUE}{#1}
}

\newcommand\CRedac[1]{
     \CBox{PineGreen}{\faEdit}{BIEN R\'EDIGER}{#1}
}

\newcommand\CError[1]{
     \CBox{Red}{\faExclamationTriangle}{ATTENTION}{#1}
}

\newcommand\TitreExo[2]{
\needspace{4\baselineskip}
 {\sffamily\large EXERCICE #1\ (\emph{#2 points})}
\vspace{5mm}
}

\newcommand\img[2]{
          \includegraphics[width=#2\paperwidth]{\imgdir#1}
}

\newcommand\imgsvg[2]{
       \begin{center}   \includegraphics[width=#2\paperwidth]{\imgsvgdir#1} \end{center}
}


\newcommand\Lien[2]{
     \href{#1}{#2 \tiny \faExternalLink}
}
\newcommand\mcLien[2]{
     \href{https~://www.maths-cours.fr/#1}{#2 \tiny \faExternalLink}
}

\newcommand{\euro}{\eurologo{}}

%================================================================================================================================
%
% Macros - Environement
%
%================================================================================================================================

\newenvironment{tex}{ %
}
{%
}

\newenvironment{indente}{ %
	\setlength\parindent{10mm}
}

{
	\setlength\parindent{0mm}
}

\newenvironment{corrige}{%
     \needspace{3\baselineskip}
     \medskip
     \textbf{\textsc{Corrigé}}
     \medskip
}
{
}

\newenvironment{extern}{%
     \begin{center}
     }
     {
     \end{center}
}

\NewEnviron{code}{%
	\par
     \boite{gray}{\texttt{%
     \BODY
     }}
     \par
}

\newenvironment{vbloc}{% boite sans cadre empeche saut de page
     \begin{minipage}[t]{\linewidth}
     }
     {
     \end{minipage}
}
\NewEnviron{h2}{%
    \needspace{3\baselineskip}
    \vspace{0.6cm}
	\noindent \MakeUppercase{\sffamily \large \BODY}
	\vspace{1mm}\textcolor{mcgris}{\hrule}\vspace{0.4cm}
	\par
}{}

\NewEnviron{h3}{%
    \needspace{3\baselineskip}
	\vspace{5mm}
	\textsc{\BODY}
	\par
}

\NewEnviron{margeneg}{ %
\begin{addmargin}[-1cm]{0cm}
\BODY
\end{addmargin}
}

\NewEnviron{html}{%
}

\begin{document}
\meta{url}{/exercices/lancers-successifs-bac-s-pondichery-2009/}
\meta{pid}{2356}
\meta{titre}{Probabilités Lancers successifs - Bac S Pondichéry 2009}
\meta{type}{exercices}
%
\begin{h2}Exercice 4\end{h2}
\textit{4 points - Commun à tous les candidats}

On dispose de deux dés cubiques dont les faces sont numérotées de 1 à 6. Ces dés sont en apparence identiques mais l'un est bien équilibré et l'autre truqué. Avec le dé truqué la probabilité d'obtenir 6 lors d'un lancer est égale à $\frac{1}{3}$.
\textit{Les résultats seront donnés sous forme de fractions irréductibles.}
\begin{enumerate}
     \item
     On lance le dé bien équilibré trois fois de suite et on désigne par X la variable aléatoire donnant le nombre de 6 obtenus.
     \begin{enumerate}
          \item
          Quelle loi de probabilité suit la variable aléatoire X ?
          \item
          Quelle est son espérance ?
          \item
          Calculer $P\left(X=2\right)$.
     \end{enumerate}
     \item
     On choisit au hasard l'un des deux dés, les choix étant équiprobables. Et on lance le dé choisi trois fois de suite.
     \par
     On considère les événements D et A suivants:
     \par
     •ᅠᅠ D : « le dé choisi est le dé bien équilibré » ;
     \par
     •ᅠᅠ A : « obtenir exactement deux 6 ».
     \begin{enumerate}
          \item
          Calculer la probabilité des événements suivants :
          \par
          •ᅠᅠ « choisir le dé bien équilibré et obtenir exactement deux 6 » ;
          \par
          •ᅠᅠ « choisir le dé truqué et obtenir exactement deux 6 ».
          \par
          (On pourra construire un arbre de probabilité).
          \item
          En déduire que: $p\left(A\right)=\frac{7}{48}$.
          \item
          Ayant choisi au hasard l'un des deux dés et l'ayant lancé trois fois de suite, on a obtenu exactement deux 6. Quelle est la probabilité d'avoir choisi le dé truqué ?
     \end{enumerate}
     \item
     On choisit au hasard l'un des deux dés, les choix étant équiprobables, et on lance le dé $n$ fois de suite ($n$ désigne un entier naturel supérieur ou égal à 2).
     \par
     On note $B_{n}$ l'événement « obtenir au moins un 6 parmi ces $n$ lancers successifs ».
     \begin{enumerate}
          \item
          Déterminer, en fonction de $n$, la probabilité $p_{n}$ de l'événement $B_{n}$.
          \item
          Calculer la limite de la suite $\left(p_{n}\right)$. Commenter ce résultat.
     \end{enumerate}
\end{enumerate}
\begin{corrige}
     \begin{enumerate}
          \item
          \begin{enumerate}
               \item
               La variable aléatoire $X$ suit une loi binômiale de paramètres $n=3$ et $p=\frac{1}{6}$
               \item
               $E\left(X\right)=np=3\times \frac{1}{6}=\frac{1}{2}$
               \item
               $P\left(X=2\right)=\begin{pmatrix} 3 \\ 2 \end{pmatrix}\times \left(\frac{1}{6}\right)^{2}\times \frac{5}{6}=3\times \frac{5}{216}=\frac{5}{72}$.
          \end{enumerate}
          \item
          \begin{enumerate}
               \item
               L'évènement « choisir le dé bien équilibré et obtenir exactement deux 6 » est $D \cap A$ :
               \par
               $p\left(D \cap A\right)=p\left(D\right)\times p_{D}\left(A\right)$
               \par
               La probabilité $ p_{D}\left(A\right)$ est la probabilité d'obtenir exactement deux 6 sachant le dé choisi est le dé bien équilibré; c'est à dire  $p_{D}(A)=p(X=2)=\frac{5}{72}$ d'après la première question donc :
               \par
               $p\left(D \cap A\right)=\frac{1}{2}\times \frac{5}{72}=\frac{5}{144}$
               \par
               L'évènement « choisir le dé truqué et obtenir exactement deux 6 » est $\overline{D} \cap A$
               \par
               $p\left(\overline{D} \cap A\right)=p\left(\overline{D}\right)\times p_{\overline{D}}\left(A\right)$
               \par
               La probabilité $p_{\overline{D}}\left(A\right)$ correspond à « la probabilité d'obtenir exactement deux 6 sachant le dé choisi est le dé truqué » :
               \par
               $p_{\overline{D}}\left(A\right)=\begin{pmatrix} 3 \\ 2 \end{pmatrix}\times \left(\frac{1}{3}\right)^{2}\times \frac{2}{3}=\frac{2}{9}$
               \par
               Donc :
               \par
               $p\left(\overline{D} \cap A\right)=\frac{1}{2}\times \frac{2}{9}=\frac{1}{9}$
               <img class="aligncenter size-full img-pc" src="/wp-content/uploads/Bac_S_Pondichery_2009_cor2.png" alt="" />
%##
% type=arbre; width=25; wcell=3.5; hcell=1.5
%--
% >D:1 / 2
% >>A:5 / 72 
% >>\overline{A}:67 / 72
% >\overline{D}:1 / 2
% >>A:2 / 9 
% >>\overline{A}:7 / 9
%--
\begin{center}
 \begin{extern}%style="width:25rem" alt="Arbre pondéré"
    \resizebox{11cm}{!}{
       \definecolor{dark}{gray}{0.1}
       \begin{tikzpicture}[scale=.8, line width=.5pt, dark]
       \def\width{3.5}
       \def\height{1.5}
       \tikzstyle{noeud}=[fill=white,circle,draw]
       \tikzstyle{poids}=[fill=white,font=\footnotesize,midway]
    \node[noeud] (r) at ({1*\width},{-1.5*\height}) {$$};
    \node[noeud] (ra) at ({2*\width},{-0.5*\height}) {$D$};
     \draw (r) -- (ra) node [poids] {$1/2$};
    \node[noeud] (raa) at ({3*\width},{0*\height}) {$A$};
     \draw (ra) -- (raa) node [poids] {$5/72$};
    \node[noeud] (rab) at ({3*\width},{-1*\height}) {$\overline{A}$};
     \draw (ra) -- (rab) node [poids] {$67/72$};
    \node[noeud] (rb) at ({2*\width},{-2.5*\height}) {$\overline{D}$};
     \draw (r) -- (rb) node [poids] {$1/2$};
    \node[noeud] (rba) at ({3*\width},{-2*\height}) {$A$};
     \draw (rb) -- (rba) node [poids] {$2/9$};
    \node[noeud] (rbb) at ({3*\width},{-3*\height}) {$\overline{A}$};
     \draw (rb) -- (rbb) node [poids] {$7/9$};
       \end{tikzpicture}
      }
   \end{extern}
\end{center}
%##
\item
               D'après le théorème des probabilités totales :
               \par
               $p\left(A\right)=p\left(\overline{D} \cap A\right)+p\left(D \cap A\right)=\frac{1}{9}+\frac{5}{144}=\frac{16}{144}+\frac{5}{144}=\frac{21}{144}=\frac{7}{48}$
               \item
               Ayant choisi au hasard l'un des deux dés et l'ayant lancé trois fois de suite, on a obtenu exactement deux 6. Quelle est la probabilité d'avoir choisi le dé truqué est :
               \par
               $p_{A}\left(\overline{D}\right)=\frac{p\left(\overline{D} \cap A\right)}{p\left(A\right)}=\frac{\frac{1}{9}}{\frac{7}{48}}=\frac{1}{9}\times \frac{48}{7}=\frac{16}{21}$
          \end{enumerate}
          \item
          \begin{enumerate}
               \item
               L'évènement $\overline{B_{n}}$ contraire de $B_{n}$ est l'événement « n'obtenir aucun 6 parmi ces $n$ lancers successifs ».
               \par
               $p\left(\overline{B_{n}}\right)=p\left(\overline{B_{n}} \cap D\right)+p\left(\overline{B_{n}} \cap \overline{D}\right)=p_{D}\left(\overline{B_{n}}\right)\times p\left(D\right)+p_{\overline{D}}\left(\overline{B_{n}}\right)\times p\left(\overline{D}\right)$
               \par
               $p\left(\overline{B_{n}}\right)=\frac{1}{2}\times \left(\frac{5}{6}\right)^{n}+\frac{1}{2}\times \left(\frac{2}{3}\right)^{n}$
               \par
               Donc
               \par
               $p_{n}=1-p\left(\overline{B_{n}}\right)=1-\frac{1}{2}\times \left(\frac{5}{6}\right)^{n}-\frac{1}{2}\times \left(\frac{2}{3}\right)^{n}$
               \item
               Comme $\frac{5}{6} < 1$ et $\frac{2}{3} < 1$:
               \par
               $\lim\limits_{n\rightarrow \infty } p_{n}=\lim\limits_{n\rightarrow \infty }1-\frac{1}{2}\times \left(\frac{5}{6}\right)^{n}-\frac{1}{2}\times \left(\frac{2}{3}\right)^{n}=1$.
               \par
               Si on lance le dé "un très grand nombre de fois", on est "pratiquement assuré" d'obtenir au moins un 6 quel que soit le dé choisi.
          \end{enumerate}
     \end{enumerate}
}
\end{corrige}

\end{document}