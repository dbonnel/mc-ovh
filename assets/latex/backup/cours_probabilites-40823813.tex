\documentclass[a4paper]{article}

%================================================================================================================================
%
% Packages
%
%================================================================================================================================

\usepackage[T1]{fontenc} 	% pour caractères accentués
\usepackage[utf8]{inputenc}  % encodage utf8
\usepackage[french]{babel}	% langue : français
\usepackage{fourier}			% caractères plus lisibles
\usepackage[dvipsnames]{xcolor} % couleurs
\usepackage{fancyhdr}		% réglage header footer
\usepackage{needspace}		% empêcher sauts de page mal placés
\usepackage{graphicx}		% pour inclure des graphiques
\usepackage{enumitem,cprotect}		% personnalise les listes d'items (nécessaire pour ol, al ...)
\usepackage{hyperref}		% Liens hypertexte
\usepackage{pstricks,pst-all,pst-node,pstricks-add,pst-math,pst-plot,pst-tree,pst-eucl} % pstricks
\usepackage[a4paper,includeheadfoot,top=2cm,left=3cm, bottom=2cm,right=3cm]{geometry} % marges etc.
\usepackage{comment}			% commentaires multilignes
\usepackage{amsmath,environ} % maths (matrices, etc.)
\usepackage{amssymb,makeidx}
\usepackage{bm}				% bold maths
\usepackage{tabularx}		% tableaux
\usepackage{colortbl}		% tableaux en couleur
\usepackage{fontawesome}		% Fontawesome
\usepackage{environ}			% environment with command
\usepackage{fp}				% calculs pour ps-tricks
\usepackage{multido}			% pour ps tricks
\usepackage[np]{numprint}	% formattage nombre
\usepackage{tikz,tkz-tab} 			% package principal TikZ
\usepackage{pgfplots}   % axes
\usepackage{mathrsfs}    % cursives
\usepackage{calc}			% calcul taille boites
\usepackage[scaled=0.875]{helvet} % font sans serif
\usepackage{svg} % svg
\usepackage{scrextend} % local margin
\usepackage{scratch} %scratch
\usepackage{multicol} % colonnes
%\usepackage{infix-RPN,pst-func} % formule en notation polanaise inversée
\usepackage{listings}

%================================================================================================================================
%
% Réglages de base
%
%================================================================================================================================

\lstset{
language=Python,   % R code
literate=
{á}{{\'a}}1
{à}{{\`a}}1
{ã}{{\~a}}1
{é}{{\'e}}1
{è}{{\`e}}1
{ê}{{\^e}}1
{í}{{\'i}}1
{ó}{{\'o}}1
{õ}{{\~o}}1
{ú}{{\'u}}1
{ü}{{\"u}}1
{ç}{{\c{c}}}1
{~}{{ }}1
}


\definecolor{codegreen}{rgb}{0,0.6,0}
\definecolor{codegray}{rgb}{0.5,0.5,0.5}
\definecolor{codepurple}{rgb}{0.58,0,0.82}
\definecolor{backcolour}{rgb}{0.95,0.95,0.92}

\lstdefinestyle{mystyle}{
    backgroundcolor=\color{backcolour},   
    commentstyle=\color{codegreen},
    keywordstyle=\color{magenta},
    numberstyle=\tiny\color{codegray},
    stringstyle=\color{codepurple},
    basicstyle=\ttfamily\footnotesize,
    breakatwhitespace=false,         
    breaklines=true,                 
    captionpos=b,                    
    keepspaces=true,                 
    numbers=left,                    
xleftmargin=2em,
framexleftmargin=2em,            
    showspaces=false,                
    showstringspaces=false,
    showtabs=false,                  
    tabsize=2,
    upquote=true
}

\lstset{style=mystyle}


\lstset{style=mystyle}
\newcommand{\imgdir}{C:/laragon/www/newmc/assets/imgsvg/}
\newcommand{\imgsvgdir}{C:/laragon/www/newmc/assets/imgsvg/}

\definecolor{mcgris}{RGB}{220, 220, 220}% ancien~; pour compatibilité
\definecolor{mcbleu}{RGB}{52, 152, 219}
\definecolor{mcvert}{RGB}{125, 194, 70}
\definecolor{mcmauve}{RGB}{154, 0, 215}
\definecolor{mcorange}{RGB}{255, 96, 0}
\definecolor{mcturquoise}{RGB}{0, 153, 153}
\definecolor{mcrouge}{RGB}{255, 0, 0}
\definecolor{mclightvert}{RGB}{205, 234, 190}

\definecolor{gris}{RGB}{220, 220, 220}
\definecolor{bleu}{RGB}{52, 152, 219}
\definecolor{vert}{RGB}{125, 194, 70}
\definecolor{mauve}{RGB}{154, 0, 215}
\definecolor{orange}{RGB}{255, 96, 0}
\definecolor{turquoise}{RGB}{0, 153, 153}
\definecolor{rouge}{RGB}{255, 0, 0}
\definecolor{lightvert}{RGB}{205, 234, 190}
\setitemize[0]{label=\color{lightvert}  $\bullet$}

\pagestyle{fancy}
\renewcommand{\headrulewidth}{0.2pt}
\fancyhead[L]{maths-cours.fr}
\fancyhead[R]{\thepage}
\renewcommand{\footrulewidth}{0.2pt}
\fancyfoot[C]{}

\newcolumntype{C}{>{\centering\arraybackslash}X}
\newcolumntype{s}{>{\hsize=.35\hsize\arraybackslash}X}

\setlength{\parindent}{0pt}		 
\setlength{\parskip}{3mm}
\setlength{\headheight}{1cm}

\def\ebook{ebook}
\def\book{book}
\def\web{web}
\def\type{web}

\newcommand{\vect}[1]{\overrightarrow{\,\mathstrut#1\,}}

\def\Oij{$\left(\text{O}~;~\vect{\imath},~\vect{\jmath}\right)$}
\def\Oijk{$\left(\text{O}~;~\vect{\imath},~\vect{\jmath},~\vect{k}\right)$}
\def\Ouv{$\left(\text{O}~;~\vect{u},~\vect{v}\right)$}

\hypersetup{breaklinks=true, colorlinks = true, linkcolor = OliveGreen, urlcolor = OliveGreen, citecolor = OliveGreen, pdfauthor={Didier BONNEL - https://www.maths-cours.fr} } % supprime les bordures autour des liens

\renewcommand{\arg}[0]{\text{arg}}

\everymath{\displaystyle}

%================================================================================================================================
%
% Macros - Commandes
%
%================================================================================================================================

\newcommand\meta[2]{    			% Utilisé pour créer le post HTML.
	\def\titre{titre}
	\def\url{url}
	\def\arg{#1}
	\ifx\titre\arg
		\newcommand\maintitle{#2}
		\fancyhead[L]{#2}
		{\Large\sffamily \MakeUppercase{#2}}
		\vspace{1mm}\textcolor{mcvert}{\hrule}
	\fi 
	\ifx\url\arg
		\fancyfoot[L]{\href{https://www.maths-cours.fr#2}{\black \footnotesize{https://www.maths-cours.fr#2}}}
	\fi 
}


\newcommand\TitreC[1]{    		% Titre centré
     \needspace{3\baselineskip}
     \begin{center}\textbf{#1}\end{center}
}

\newcommand\newpar{    		% paragraphe
     \par
}

\newcommand\nosp {    		% commande vide (pas d'espace)
}
\newcommand{\id}[1]{} %ignore

\newcommand\boite[2]{				% Boite simple sans titre
	\vspace{5mm}
	\setlength{\fboxrule}{0.2mm}
	\setlength{\fboxsep}{5mm}	
	\fcolorbox{#1}{#1!3}{\makebox[\linewidth-2\fboxrule-2\fboxsep]{
  		\begin{minipage}[t]{\linewidth-2\fboxrule-4\fboxsep}\setlength{\parskip}{3mm}
  			 #2
  		\end{minipage}
	}}
	\vspace{5mm}
}

\newcommand\CBox[4]{				% Boites
	\vspace{5mm}
	\setlength{\fboxrule}{0.2mm}
	\setlength{\fboxsep}{5mm}
	
	\fcolorbox{#1}{#1!3}{\makebox[\linewidth-2\fboxrule-2\fboxsep]{
		\begin{minipage}[t]{1cm}\setlength{\parskip}{3mm}
	  		\textcolor{#1}{\LARGE{#2}}    
 	 	\end{minipage}  
  		\begin{minipage}[t]{\linewidth-2\fboxrule-4\fboxsep}\setlength{\parskip}{3mm}
			\raisebox{1.2mm}{\normalsize\sffamily{\textcolor{#1}{#3}}}						
  			 #4
  		\end{minipage}
	}}
	\vspace{5mm}
}

\newcommand\cadre[3]{				% Boites convertible html
	\par
	\vspace{2mm}
	\setlength{\fboxrule}{0.1mm}
	\setlength{\fboxsep}{5mm}
	\fcolorbox{#1}{white}{\makebox[\linewidth-2\fboxrule-2\fboxsep]{
  		\begin{minipage}[t]{\linewidth-2\fboxrule-4\fboxsep}\setlength{\parskip}{3mm}
			\raisebox{-2.5mm}{\sffamily \small{\textcolor{#1}{\MakeUppercase{#2}}}}		
			\par		
  			 #3
 	 		\end{minipage}
	}}
		\vspace{2mm}
	\par
}

\newcommand\bloc[3]{				% Boites convertible html sans bordure
     \needspace{2\baselineskip}
     {\sffamily \small{\textcolor{#1}{\MakeUppercase{#2}}}}    
		\par		
  			 #3
		\par
}

\newcommand\CHelp[1]{
     \CBox{Plum}{\faInfoCircle}{À RETENIR}{#1}
}

\newcommand\CUp[1]{
     \CBox{NavyBlue}{\faThumbsOUp}{EN PRATIQUE}{#1}
}

\newcommand\CInfo[1]{
     \CBox{Sepia}{\faArrowCircleRight}{REMARQUE}{#1}
}

\newcommand\CRedac[1]{
     \CBox{PineGreen}{\faEdit}{BIEN R\'EDIGER}{#1}
}

\newcommand\CError[1]{
     \CBox{Red}{\faExclamationTriangle}{ATTENTION}{#1}
}

\newcommand\TitreExo[2]{
\needspace{4\baselineskip}
 {\sffamily\large EXERCICE #1\ (\emph{#2 points})}
\vspace{5mm}
}

\newcommand\img[2]{
          \includegraphics[width=#2\paperwidth]{\imgdir#1}
}

\newcommand\imgsvg[2]{
       \begin{center}   \includegraphics[width=#2\paperwidth]{\imgsvgdir#1} \end{center}
}


\newcommand\Lien[2]{
     \href{#1}{#2 \tiny \faExternalLink}
}
\newcommand\mcLien[2]{
     \href{https~://www.maths-cours.fr/#1}{#2 \tiny \faExternalLink}
}

\newcommand{\euro}{\eurologo{}}

%================================================================================================================================
%
% Macros - Environement
%
%================================================================================================================================

\newenvironment{tex}{ %
}
{%
}

\newenvironment{indente}{ %
	\setlength\parindent{10mm}
}

{
	\setlength\parindent{0mm}
}

\newenvironment{corrige}{%
     \needspace{3\baselineskip}
     \medskip
     \textbf{\textsc{Corrigé}}
     \medskip
}
{
}

\newenvironment{extern}{%
     \begin{center}
     }
     {
     \end{center}
}

\NewEnviron{code}{%
	\par
     \boite{gray}{\texttt{%
     \BODY
     }}
     \par
}

\newenvironment{vbloc}{% boite sans cadre empeche saut de page
     \begin{minipage}[t]{\linewidth}
     }
     {
     \end{minipage}
}
\NewEnviron{h2}{%
    \needspace{3\baselineskip}
    \vspace{0.6cm}
	\noindent \MakeUppercase{\sffamily \large \BODY}
	\vspace{1mm}\textcolor{mcgris}{\hrule}\vspace{0.4cm}
	\par
}{}

\NewEnviron{h3}{%
    \needspace{3\baselineskip}
	\vspace{5mm}
	\textsc{\BODY}
	\par
}

\NewEnviron{margeneg}{ %
\begin{addmargin}[-1cm]{0cm}
\BODY
\end{addmargin}
}

\NewEnviron{html}{%
}

\begin{document}
\meta{url}{/cours/probabilites/}
\meta{pid}{155}
\meta{titre}{Probabilités en Seconde}
\meta{type}{cours}
\begin{h2}1. Expérience aléatoire\end{h2}
\cadre{bleu}{Définitions}{%
     Une expérience \textbf{aléatoire} est une expérience dont le résultat dépend du hasard.
     \par
     L'ensemble de tous les résultats possibles d'une expérience aléatoire s'appelle l'\textbf{univers} de l'expérience.
     \par
     On le note en général \textbf{$\Omega $}.
}
\cadre{bleu}{Définition}{%
     Soit une expérience aléatoire d'univers $\Omega $.
     \par
     Chacun des résultats possibles s'appelle une \textbf{éventualité} (ou un \textbf{événement élémentaire }ou une \textbf{issue}).
     \par
     On appelle \textbf{événement} tout sous ensemble de $\Omega $.
     \par
     Un événement est donc constitué de zéro, une ou plusieurs éventualités.
}
\bloc{orange}{Exemples}{%
     Le lancer d'un dé à six faces est une expérience aléatoire d'univers :
     \par
     $\Omega =\left\{1;2;3;4;5;6\right\}$
     \begin{itemize}
          \item L'ensemble $E_1=\left\{2;4;6\right\}$ est un événement. En français, cet événement peut se traduire par la phrase : « \textit{le résultat du dé est un nombre pair} »
          \item L'ensemble $E_2=\left\{1;2;3\right\}$ est un autre événement. Ce second événement peut se traduire par la phrase : « \textit{le résultat du dé est strictement inférieur à 4} »
     \end{itemize}
     Ces événements peuvent être représentés par un diagramme de Venn :
     \begin{center}
          \imgsvg{Venn}{0.33}%width="300" alt="Diagramme de Venn"
     \end{center}
}
\cadre{bleu}{Définition}{%
     \begin{itemize}
          \item l'\textbf{événement impossible} est la partie vide, noté $\varnothing $, lorsque aucune issue ne le réalise.
          \item l'\textbf{événement certain} est $\Omega $, lorsque toutes les issues le réalisent.
          \item l'\textbf{événement contraire} de $A$ noté $\overline A$ est l'ensemble des éventualités de $\Omega $ qui n'appartiennent pas à $A$.
          \item l'événement $A \cup  B$ (lire « $A$ union $B$ » ou « $A$ \textbf{ou} $B$ ») est constitué des éventualités qui appartiennent soit à $A$, soit à $B$, soit aux deux ensembles.
          \item l'événement $A \cap  B$ (lire « $A$ inter $B$ » ou « $A$ \textbf{et }$B$ ») est constitué des éventualités qui appartiennent à la fois à $A$ et à $B$.
     \end{itemize}
}
\bloc{orange}{Exemple}{%
     On reprend l'exemple précédent avec :
     \par
     $\Omega =\left\{1;2;3;4;5;6\right\}$
     \par
     $E_1=\left\{2;4;6\right\}$
     \par
     $E_2=\left\{1;2;3\right\}$
     \begin{itemize}
          \item  L'événement « obtenir un nombre supérieur à 7 » est l' événement impossible.
          \item  L'événement « obtenir un nombre entier » est l' événement certain.
          \item $\overline{E}_{1}=\left\{1; 3; 5\right\}$ : cet événement peut se traduire par \og le résultat est un nombre impair \fg{}~:
          \begin{center}
               \imgsvg{Venn-complementaire}{0.3}%width="300" alt="Diagramme de Venn - Complémentaire"
          \end{center}
          \item $E_{1} \cup  E_{2}=\left\{1; 2; 3; 4; 6\right\}$ : cet événement peut se traduire par \og le résultat est pair \textbf{ou} strictement inférieur à 4 \fg{}~:
          \begin{center}
               \imgsvg{Venn-union}{0.3}%width="300" alt="Diagramme de Venn - Union"
          \end{center}
          \item $E_{1} \cap  E_{2}=\left\{2\right\}$ : cet événement peut se traduire par \og le résultat est pair \textbf{et} strictement inférieur à 4 \fg{}~:
          \begin{center}
               \imgsvg{Venn-inter}{0.3}%width="300" alt="Diagramme de Venn - Intersection "
          \end{center}
     \end{itemize}
}
\cadre{bleu}{Définition}{%
     On dit que A et B sont \textbf{incompatibles} si et seulement si $A \cap  B=\varnothing$
     \par
     Deux événements sont incompatibles lorsqu'aucun événement ne les réalise simultanément.
}
\bloc{vert}{Remarque}{%
     Deux événements contraires sont incompatibles mais deux événements peuvent être incompatibles sans être contraires.
}
\bloc{orange}{Exemple}{%
     « Obtenir un chiffre inférieur à 2 » et « obtenir un chiffre supérieur à 4 » sont deux événements incompatibles.
}
\begin{h2}2. Probabilités\end{h2}
\cadre{bleu}{Définition}{%
     La probabilité d'un événement élémentaire est un nombre réel tel que:
     \begin{itemize}
          \item Ce nombre est compris entre 0 et 1
          \item La somme des probabilités de tous les événements élémentaires de l'univers vaut 1
     \end{itemize}
}
\cadre{vert}{Propriétés}{%
     \begin{itemize}
          \item $p\left(\varnothing\right)=0$
          \item $p\left(\Omega \right)=1$
          \item $p\left(\overline A\right)=1-p\left(A\right)$
     \end{itemize}
}
\bloc{orange}{Exemple}{%
     On lance un dé à six faces. On note $S$ l'événement : « obtenir un $6$. On suppose que le dé est bien équilibré et que la probabilité de $S$ est de $\frac{1}{6}$. La probabilité d'obtenir un résultat différent de $6$ est alors :
     \par
     $p\left(\overline S\right)=1-p\left(S\right)=1-\frac{1}{6}=\frac{5}{6}$
}
\cadre{rouge}{Théorème}{%
     Quels que soient les événements $A$ et $B$ de $\Omega $ :
     \par
     $p\left(A \cup  B\right)=p\left(A\right)+p\left(B\right)-p\left(A \cap  B\right)$
     \par
     En particulier, si $A$ et $B$ sont \textbf{incompatibles} :
     \par
     $p\left(A \cup  B\right)=p\left(A\right)+p\left(B\right)$
}
\cadre{bleu}{Définition}{%
     Deux événements qui ont la même probabilité sont dits \textbf{ équiprobables}.
     \par
     Lorsque tous les événements élémentaires sont équiprobables, on dit qu'il y a \textbf{équiprobabilité}.
}
\bloc{orange}{Exemple}{%
     Un lancer d'un dé non truqué est une situation d'équiprobabilité.
}
\cadre{vert}{Propriétés}{%
     On suppose que l'univers est composé de $n$ événements élémentaires
     \begin{itemize}
          \item Dans le cas d'équiprobabilité, chaque événement élémentaire a pour probabilité : $\frac{1}{n}$
          \item Si un événement $A$ de $ \Omega $ est composé de $m$ événements élémentaires, alors $P\left(A\right)=\frac{m}{n}$.
     \end{itemize}
}
\bloc{orange}{Exemple}{%
     On reprend l'exemple du lancer d'un dé avec $E_1$ : « le résultat du dé est un nombre pair »
     \par
     $P\left(E_1\right)=\frac{3}{6}=\frac{1}{2}$
}

\end{document}