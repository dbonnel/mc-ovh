\documentclass[a4paper]{article}

%================================================================================================================================
%
% Packages
%
%================================================================================================================================

\usepackage[T1]{fontenc} 	% pour caractères accentués
\usepackage[utf8]{inputenc}  % encodage utf8
\usepackage[french]{babel}	% langue : français
\usepackage{fourier}			% caractères plus lisibles
\usepackage[dvipsnames]{xcolor} % couleurs
\usepackage{fancyhdr}		% réglage header footer
\usepackage{needspace}		% empêcher sauts de page mal placés
\usepackage{graphicx}		% pour inclure des graphiques
\usepackage{enumitem,cprotect}		% personnalise les listes d'items (nécessaire pour ol, al ...)
\usepackage{hyperref}		% Liens hypertexte
\usepackage{pstricks,pst-all,pst-node,pstricks-add,pst-math,pst-plot,pst-tree,pst-eucl} % pstricks
\usepackage[a4paper,includeheadfoot,top=2cm,left=3cm, bottom=2cm,right=3cm]{geometry} % marges etc.
\usepackage{comment}			% commentaires multilignes
\usepackage{amsmath,environ} % maths (matrices, etc.)
\usepackage{amssymb,makeidx}
\usepackage{bm}				% bold maths
\usepackage{tabularx}		% tableaux
\usepackage{colortbl}		% tableaux en couleur
\usepackage{fontawesome}		% Fontawesome
\usepackage{environ}			% environment with command
\usepackage{fp}				% calculs pour ps-tricks
\usepackage{multido}			% pour ps tricks
\usepackage[np]{numprint}	% formattage nombre
\usepackage{tikz,tkz-tab} 			% package principal TikZ
\usepackage{pgfplots}   % axes
\usepackage{mathrsfs}    % cursives
\usepackage{calc}			% calcul taille boites
\usepackage[scaled=0.875]{helvet} % font sans serif
\usepackage{svg} % svg
\usepackage{scrextend} % local margin
\usepackage{scratch} %scratch
\usepackage{multicol} % colonnes
%\usepackage{infix-RPN,pst-func} % formule en notation polanaise inversée
\usepackage{listings}

%================================================================================================================================
%
% Réglages de base
%
%================================================================================================================================

\lstset{
language=Python,   % R code
literate=
{á}{{\'a}}1
{à}{{\`a}}1
{ã}{{\~a}}1
{é}{{\'e}}1
{è}{{\`e}}1
{ê}{{\^e}}1
{í}{{\'i}}1
{ó}{{\'o}}1
{õ}{{\~o}}1
{ú}{{\'u}}1
{ü}{{\"u}}1
{ç}{{\c{c}}}1
{~}{{ }}1
}


\definecolor{codegreen}{rgb}{0,0.6,0}
\definecolor{codegray}{rgb}{0.5,0.5,0.5}
\definecolor{codepurple}{rgb}{0.58,0,0.82}
\definecolor{backcolour}{rgb}{0.95,0.95,0.92}

\lstdefinestyle{mystyle}{
    backgroundcolor=\color{backcolour},   
    commentstyle=\color{codegreen},
    keywordstyle=\color{magenta},
    numberstyle=\tiny\color{codegray},
    stringstyle=\color{codepurple},
    basicstyle=\ttfamily\footnotesize,
    breakatwhitespace=false,         
    breaklines=true,                 
    captionpos=b,                    
    keepspaces=true,                 
    numbers=left,                    
xleftmargin=2em,
framexleftmargin=2em,            
    showspaces=false,                
    showstringspaces=false,
    showtabs=false,                  
    tabsize=2,
    upquote=true
}

\lstset{style=mystyle}


\lstset{style=mystyle}
\newcommand{\imgdir}{C:/laragon/www/newmc/assets/imgsvg/}
\newcommand{\imgsvgdir}{C:/laragon/www/newmc/assets/imgsvg/}

\definecolor{mcgris}{RGB}{220, 220, 220}% ancien~; pour compatibilité
\definecolor{mcbleu}{RGB}{52, 152, 219}
\definecolor{mcvert}{RGB}{125, 194, 70}
\definecolor{mcmauve}{RGB}{154, 0, 215}
\definecolor{mcorange}{RGB}{255, 96, 0}
\definecolor{mcturquoise}{RGB}{0, 153, 153}
\definecolor{mcrouge}{RGB}{255, 0, 0}
\definecolor{mclightvert}{RGB}{205, 234, 190}

\definecolor{gris}{RGB}{220, 220, 220}
\definecolor{bleu}{RGB}{52, 152, 219}
\definecolor{vert}{RGB}{125, 194, 70}
\definecolor{mauve}{RGB}{154, 0, 215}
\definecolor{orange}{RGB}{255, 96, 0}
\definecolor{turquoise}{RGB}{0, 153, 153}
\definecolor{rouge}{RGB}{255, 0, 0}
\definecolor{lightvert}{RGB}{205, 234, 190}
\setitemize[0]{label=\color{lightvert}  $\bullet$}

\pagestyle{fancy}
\renewcommand{\headrulewidth}{0.2pt}
\fancyhead[L]{maths-cours.fr}
\fancyhead[R]{\thepage}
\renewcommand{\footrulewidth}{0.2pt}
\fancyfoot[C]{}

\newcolumntype{C}{>{\centering\arraybackslash}X}
\newcolumntype{s}{>{\hsize=.35\hsize\arraybackslash}X}

\setlength{\parindent}{0pt}		 
\setlength{\parskip}{3mm}
\setlength{\headheight}{1cm}

\def\ebook{ebook}
\def\book{book}
\def\web{web}
\def\type{web}

\newcommand{\vect}[1]{\overrightarrow{\,\mathstrut#1\,}}

\def\Oij{$\left(\text{O}~;~\vect{\imath},~\vect{\jmath}\right)$}
\def\Oijk{$\left(\text{O}~;~\vect{\imath},~\vect{\jmath},~\vect{k}\right)$}
\def\Ouv{$\left(\text{O}~;~\vect{u},~\vect{v}\right)$}

\hypersetup{breaklinks=true, colorlinks = true, linkcolor = OliveGreen, urlcolor = OliveGreen, citecolor = OliveGreen, pdfauthor={Didier BONNEL - https://www.maths-cours.fr} } % supprime les bordures autour des liens

\renewcommand{\arg}[0]{\text{arg}}

\everymath{\displaystyle}

%================================================================================================================================
%
% Macros - Commandes
%
%================================================================================================================================

\newcommand\meta[2]{    			% Utilisé pour créer le post HTML.
	\def\titre{titre}
	\def\url{url}
	\def\arg{#1}
	\ifx\titre\arg
		\newcommand\maintitle{#2}
		\fancyhead[L]{#2}
		{\Large\sffamily \MakeUppercase{#2}}
		\vspace{1mm}\textcolor{mcvert}{\hrule}
	\fi 
	\ifx\url\arg
		\fancyfoot[L]{\href{https://www.maths-cours.fr#2}{\black \footnotesize{https://www.maths-cours.fr#2}}}
	\fi 
}


\newcommand\TitreC[1]{    		% Titre centré
     \needspace{3\baselineskip}
     \begin{center}\textbf{#1}\end{center}
}

\newcommand\newpar{    		% paragraphe
     \par
}

\newcommand\nosp {    		% commande vide (pas d'espace)
}
\newcommand{\id}[1]{} %ignore

\newcommand\boite[2]{				% Boite simple sans titre
	\vspace{5mm}
	\setlength{\fboxrule}{0.2mm}
	\setlength{\fboxsep}{5mm}	
	\fcolorbox{#1}{#1!3}{\makebox[\linewidth-2\fboxrule-2\fboxsep]{
  		\begin{minipage}[t]{\linewidth-2\fboxrule-4\fboxsep}\setlength{\parskip}{3mm}
  			 #2
  		\end{minipage}
	}}
	\vspace{5mm}
}

\newcommand\CBox[4]{				% Boites
	\vspace{5mm}
	\setlength{\fboxrule}{0.2mm}
	\setlength{\fboxsep}{5mm}
	
	\fcolorbox{#1}{#1!3}{\makebox[\linewidth-2\fboxrule-2\fboxsep]{
		\begin{minipage}[t]{1cm}\setlength{\parskip}{3mm}
	  		\textcolor{#1}{\LARGE{#2}}    
 	 	\end{minipage}  
  		\begin{minipage}[t]{\linewidth-2\fboxrule-4\fboxsep}\setlength{\parskip}{3mm}
			\raisebox{1.2mm}{\normalsize\sffamily{\textcolor{#1}{#3}}}						
  			 #4
  		\end{minipage}
	}}
	\vspace{5mm}
}

\newcommand\cadre[3]{				% Boites convertible html
	\par
	\vspace{2mm}
	\setlength{\fboxrule}{0.1mm}
	\setlength{\fboxsep}{5mm}
	\fcolorbox{#1}{white}{\makebox[\linewidth-2\fboxrule-2\fboxsep]{
  		\begin{minipage}[t]{\linewidth-2\fboxrule-4\fboxsep}\setlength{\parskip}{3mm}
			\raisebox{-2.5mm}{\sffamily \small{\textcolor{#1}{\MakeUppercase{#2}}}}		
			\par		
  			 #3
 	 		\end{minipage}
	}}
		\vspace{2mm}
	\par
}

\newcommand\bloc[3]{				% Boites convertible html sans bordure
     \needspace{2\baselineskip}
     {\sffamily \small{\textcolor{#1}{\MakeUppercase{#2}}}}    
		\par		
  			 #3
		\par
}

\newcommand\CHelp[1]{
     \CBox{Plum}{\faInfoCircle}{À RETENIR}{#1}
}

\newcommand\CUp[1]{
     \CBox{NavyBlue}{\faThumbsOUp}{EN PRATIQUE}{#1}
}

\newcommand\CInfo[1]{
     \CBox{Sepia}{\faArrowCircleRight}{REMARQUE}{#1}
}

\newcommand\CRedac[1]{
     \CBox{PineGreen}{\faEdit}{BIEN R\'EDIGER}{#1}
}

\newcommand\CError[1]{
     \CBox{Red}{\faExclamationTriangle}{ATTENTION}{#1}
}

\newcommand\TitreExo[2]{
\needspace{4\baselineskip}
 {\sffamily\large EXERCICE #1\ (\emph{#2 points})}
\vspace{5mm}
}

\newcommand\img[2]{
          \includegraphics[width=#2\paperwidth]{\imgdir#1}
}

\newcommand\imgsvg[2]{
       \begin{center}   \includegraphics[width=#2\paperwidth]{\imgsvgdir#1} \end{center}
}


\newcommand\Lien[2]{
     \href{#1}{#2 \tiny \faExternalLink}
}
\newcommand\mcLien[2]{
     \href{https~://www.maths-cours.fr/#1}{#2 \tiny \faExternalLink}
}

\newcommand{\euro}{\eurologo{}}

%================================================================================================================================
%
% Macros - Environement
%
%================================================================================================================================

\newenvironment{tex}{ %
}
{%
}

\newenvironment{indente}{ %
	\setlength\parindent{10mm}
}

{
	\setlength\parindent{0mm}
}

\newenvironment{corrige}{%
     \needspace{3\baselineskip}
     \medskip
     \textbf{\textsc{Corrigé}}
     \medskip
}
{
}

\newenvironment{extern}{%
     \begin{center}
     }
     {
     \end{center}
}

\NewEnviron{code}{%
	\par
     \boite{gray}{\texttt{%
     \BODY
     }}
     \par
}

\newenvironment{vbloc}{% boite sans cadre empeche saut de page
     \begin{minipage}[t]{\linewidth}
     }
     {
     \end{minipage}
}
\NewEnviron{h2}{%
    \needspace{3\baselineskip}
    \vspace{0.6cm}
	\noindent \MakeUppercase{\sffamily \large \BODY}
	\vspace{1mm}\textcolor{mcgris}{\hrule}\vspace{0.4cm}
	\par
}{}

\NewEnviron{h3}{%
    \needspace{3\baselineskip}
	\vspace{5mm}
	\textsc{\BODY}
	\par
}

\NewEnviron{margeneg}{ %
\begin{addmargin}[-1cm]{0cm}
\BODY
\end{addmargin}
}

\NewEnviron{html}{%
}

\begin{document}
\meta{url}{/exercices/suites-bac-es-amerique-nord-2013/}
\meta{pid}{2071}
\meta{titre}{Suites arithmético-géométrique - Bac ES/L Amérique du Nord 2013}
\meta{type}{exercices}
%
\begin{h2}Exercice 3  (5 points)\end{h2}
\par
\textbf{Candidats de la série ES n'ayant pas suivi l'enseignement de spécialité et candidats de L.}
\par
La bibliothèque municipale étant devenue trop petite, une commune a décidé d'ouvrir une médiathèque qui pourra contenir 100 000 ouvrages au total.
\par
Pour l'ouverture prévue le 1er janvier 2013, la médiathèque dispose du stock de 35 000 ouvrages de l'ancienne bibliothèque augmenté de 7 000 ouvrages supplémentaires neufs offerts par la commune.
\begin{h3}Partie A\end{h3}
Chaque année, la bibliothécaire est chargée de supprimer 5\% des ouvrages, trop vieux ou abîmés, et d'acheter 6 000 ouvrages neufs.
\par
On appelle $u_{n}$ le nombre, en milliers, d'ouvrages disponibles le 1er  janvier de l'année $\left(2013+n\right)$.
\par
On donne $u_{0}=42$.
\begin{enumerate}
     \item
     Justifier que, pour tout entier naturel $n$ , on a $u_{n+1}=u_{n}\times 0,95+6$.
     \item
     On propose, ci-dessous, un algorithme, en langage naturel.
     \par
     Expliquer ce que permet de calculer cet algorithme.

\begin{tabularx}{0.8\linewidth}{|*{3}{>{\centering \arraybackslash }X|}}%class="singleborder" width="600"
          \hline
          \textbf{Variables :} &
          \\ \hline
          $ \quad $$U, N$ &
          \\ \hline
  \textbf{Initialisation :} &
          \\ \hline
          $ \quad $Mettre $42$ dans $U$ &
          \\ \hline
          $ \quad $Mettre $0$ dans $N$ &
          \\ \hline
          \textbf{Traitement :} &
          \\ \hline
           $ \quad $  Tant que $U < 100$ &
          \\ \hline
          $ \quad $$ \quad $$U$ prend la valeur $U\times 0,95+6$ &
          \\ \hline
          $ \quad $$ \quad $$N$ prend la valeur $N+1$ &
          \\ \hline
          $ \quad $Fin du Tant que &
          \\ \hline
          \textbf{Sortie :} &
          \\ \hline
          $ \quad $Afficher $N$ &
          \\ \hline
     \end{tabularx}
\item
À l'aide de votre calculatrice, déterminer le résultat obtenu grâce à cet algorithme.
\end{enumerate}
\begin{h3}Partie B\end{h3}
La commune doit finalement revoir ses dépenses à la baisse, elle ne pourra financer que 4 000 nouveaux ouvrages par an au lieu des 6 000 prévus.
\par
On appelle $v_{n}$ le nombre, en milliers, d'ouvrages disponibles le 1er janvier de l'année $\left(2013 +n\right)$.
\begin{enumerate}
     \item
     Identifier et écrire la ligne qu'il faut modifier dans l'algorithme pour prendre en compte ce changement.
     \item
     On admet que $v_{n+1}=v_{n}\times 0,95+4$ avec $v_{0}=42$.
     \par
     On considère la suite $\left(w_{n}\right)$ définie, pour tout entier $n$, par $w_{n}=v_{n}-80$.
     \par
     Montrer que $\left(w_{n}\right)$ est une suite géométrique de raison $q=0,95$ et préciser son premier terme $w_{0}$.
     \item
     On admet que, pour tout entier naturel $n : w_{n}=-38\times \left(0,95\right)^{n}$.
     \begin{enumerate}[label=\alph*.]
          \item
          Déterminer la limite de $\left(w_{n}\right)$.
          \item
          En déduire la limite de $\left(v_{n}\right)$.
          \item
          Interpréter ce résultat.
     \end{enumerate}
\end{enumerate}
\begin{corrige}
     \begin{h3}Partie A\end{h3}
     \begin{enumerate}
          \item
          Chaque année 5\% des ouvrages sont supprimés. Le nombre d'ouvrages restant est alors :
          \par
          $\left(1-\frac{5}{100}\right) \times  u_{n} = 0,95\times  u_{n} $.
          \par
          Ensuite, 6 000 ouvrages neufs (soit 6 milliers) sont achetés.
          \par
          Le nombre d'ouvrages disponibles le 1er  janvier de l'année $\left(2013+n+1\right)$ est donc :
          \begin{center}$u_{n+1}=0,95\times  u_{n} +6$\end{center}
          \item
          Cet algorithme affiche le plus petit rang $N$ à partir duquel $u_{N}  > 100$ (c'est à dire à partir duquel le nombre d'ouvrages dépassera 10 000).
          \item
          A la calculatrice on trouve :
          \par
          $u_{26} \approx  99,45$ et $u_{27} \approx  100,47$
          \par
          donc le résultat affiché par l'algorithme est  27.
     \end{enumerate}
     \begin{h3}Partie B\end{h3}
     \begin{enumerate}
          \item
          Il faut remplacer 6 par 4 à la ligne :
\par
          $U$ prend la valeur $U\times 0,95+6$
\par
     L'algorithme devient :

\begin{tabularx}{0.8\linewidth}{|*{3}{>{\centering \arraybackslash }X|}}%class="singleborder" width="600"
          \hline
          \textbf{Variables :}&
          \\ \hline
          $ \quad $$U, N$&
          \\ \hline
          \textbf{Initialisation :}&
          \\ \hline
           $ \quad $Mettre $42$ dans $U$ &
          \\ \hline
          $ \quad $Mettre $0$ dans $N$ &
          \\ \hline
          \textbf{Traitement :} &
          \\ \hline
          $ \quad $Tant que $U < 100$ &
          \\ \hline
          $ \quad $$ \quad $$U$ prend la valeur $U\times 0,95+4$ &
          \\ \hline
          $ \quad $$ \quad $$N$ prend la valeur $N+1$ &
          \\ \hline
          $ \quad $Fin du Tant que &
          \\ \hline
          \textbf{Sortie :} &
          \\ \hline
          $ \quad $Afficher $N$ &
          \\ \hline
     \end{tabularx}

\item
$w_{n+1} = v_{n+1}-80 = 0,95 v_{n}+4 -80 = 0,95 v_{n}-76$
\par
Or $w_{n}=v_{n}-80$ donc $v_{n}=w_{n}+80$
\par
Par conséquent :
\par
$w_{n+1}= 0,95\left(w_{n}+80\right)-76 = 0,95w_{n}+76-76 = 0,95 w_{n} $
\par
On a  $w_{n+1}= 0,95\times v_{n}$ a ce qui montre que \textbf{la suite $w_{n}$ est une suite géométrique de raison $0,95$}.
\par
Par ailleurs,  $ w_{0} = v_{0}-80 = 42-80 = -38$, donc le premier terme de la suite $\left(w_{n}\right)$ est $-38$.
\item
\begin{enumerate}[label=\alph*.]
     \item
     La raison de la suite $\left(w_{n}\right)$ est strictement comprise entre $0$ et $1$ donc $\lim\limits_{n \rightarrow +\infty } w_{n}=0$.
     \item
     Pour tout entier $n$, $v_{n} = w_{n} + 80$ donc $\lim\limits_{n \rightarrow +\infty } v_{n}=80$.
     \item
     Au fil du temps, le nombre d'ouvrages se rapprochera de $80 000$ sans jamais dépasser ce nombre.
     \par
     (En particulier, il n'atteindra jamais les $100 000$ et l'algorithme ci-dessus bouclera sans fin).
\end{enumerate}
\end{enumerate}
\end{corrige}

\end{document}