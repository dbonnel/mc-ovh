\documentclass[a4paper]{article}

%================================================================================================================================
%
% Packages
%
%================================================================================================================================

\usepackage[T1]{fontenc} 	% pour caractères accentués
\usepackage[utf8]{inputenc}  % encodage utf8
\usepackage[french]{babel}	% langue : français
\usepackage{fourier}			% caractères plus lisibles
\usepackage[dvipsnames]{xcolor} % couleurs
\usepackage{fancyhdr}		% réglage header footer
\usepackage{needspace}		% empêcher sauts de page mal placés
\usepackage{graphicx}		% pour inclure des graphiques
\usepackage{enumitem,cprotect}		% personnalise les listes d'items (nécessaire pour ol, al ...)
\usepackage{hyperref}		% Liens hypertexte
\usepackage{pstricks,pst-all,pst-node,pstricks-add,pst-math,pst-plot,pst-tree,pst-eucl} % pstricks
\usepackage[a4paper,includeheadfoot,top=2cm,left=3cm, bottom=2cm,right=3cm]{geometry} % marges etc.
\usepackage{comment}			% commentaires multilignes
\usepackage{amsmath,environ} % maths (matrices, etc.)
\usepackage{amssymb,makeidx}
\usepackage{bm}				% bold maths
\usepackage{tabularx}		% tableaux
\usepackage{colortbl}		% tableaux en couleur
\usepackage{fontawesome}		% Fontawesome
\usepackage{environ}			% environment with command
\usepackage{fp}				% calculs pour ps-tricks
\usepackage{multido}			% pour ps tricks
\usepackage[np]{numprint}	% formattage nombre
\usepackage{tikz,tkz-tab} 			% package principal TikZ
\usepackage{pgfplots}   % axes
\usepackage{mathrsfs}    % cursives
\usepackage{calc}			% calcul taille boites
\usepackage[scaled=0.875]{helvet} % font sans serif
\usepackage{svg} % svg
\usepackage{scrextend} % local margin
\usepackage{scratch} %scratch
\usepackage{multicol} % colonnes
%\usepackage{infix-RPN,pst-func} % formule en notation polanaise inversée
\usepackage{listings}

%================================================================================================================================
%
% Réglages de base
%
%================================================================================================================================

\lstset{
language=Python,   % R code
literate=
{á}{{\'a}}1
{à}{{\`a}}1
{ã}{{\~a}}1
{é}{{\'e}}1
{è}{{\`e}}1
{ê}{{\^e}}1
{í}{{\'i}}1
{ó}{{\'o}}1
{õ}{{\~o}}1
{ú}{{\'u}}1
{ü}{{\"u}}1
{ç}{{\c{c}}}1
{~}{{ }}1
}


\definecolor{codegreen}{rgb}{0,0.6,0}
\definecolor{codegray}{rgb}{0.5,0.5,0.5}
\definecolor{codepurple}{rgb}{0.58,0,0.82}
\definecolor{backcolour}{rgb}{0.95,0.95,0.92}

\lstdefinestyle{mystyle}{
    backgroundcolor=\color{backcolour},   
    commentstyle=\color{codegreen},
    keywordstyle=\color{magenta},
    numberstyle=\tiny\color{codegray},
    stringstyle=\color{codepurple},
    basicstyle=\ttfamily\footnotesize,
    breakatwhitespace=false,         
    breaklines=true,                 
    captionpos=b,                    
    keepspaces=true,                 
    numbers=left,                    
xleftmargin=2em,
framexleftmargin=2em,            
    showspaces=false,                
    showstringspaces=false,
    showtabs=false,                  
    tabsize=2,
    upquote=true
}

\lstset{style=mystyle}


\lstset{style=mystyle}
\newcommand{\imgdir}{C:/laragon/www/newmc/assets/imgsvg/}
\newcommand{\imgsvgdir}{C:/laragon/www/newmc/assets/imgsvg/}

\definecolor{mcgris}{RGB}{220, 220, 220}% ancien~; pour compatibilité
\definecolor{mcbleu}{RGB}{52, 152, 219}
\definecolor{mcvert}{RGB}{125, 194, 70}
\definecolor{mcmauve}{RGB}{154, 0, 215}
\definecolor{mcorange}{RGB}{255, 96, 0}
\definecolor{mcturquoise}{RGB}{0, 153, 153}
\definecolor{mcrouge}{RGB}{255, 0, 0}
\definecolor{mclightvert}{RGB}{205, 234, 190}

\definecolor{gris}{RGB}{220, 220, 220}
\definecolor{bleu}{RGB}{52, 152, 219}
\definecolor{vert}{RGB}{125, 194, 70}
\definecolor{mauve}{RGB}{154, 0, 215}
\definecolor{orange}{RGB}{255, 96, 0}
\definecolor{turquoise}{RGB}{0, 153, 153}
\definecolor{rouge}{RGB}{255, 0, 0}
\definecolor{lightvert}{RGB}{205, 234, 190}
\setitemize[0]{label=\color{lightvert}  $\bullet$}

\pagestyle{fancy}
\renewcommand{\headrulewidth}{0.2pt}
\fancyhead[L]{maths-cours.fr}
\fancyhead[R]{\thepage}
\renewcommand{\footrulewidth}{0.2pt}
\fancyfoot[C]{}

\newcolumntype{C}{>{\centering\arraybackslash}X}
\newcolumntype{s}{>{\hsize=.35\hsize\arraybackslash}X}

\setlength{\parindent}{0pt}		 
\setlength{\parskip}{3mm}
\setlength{\headheight}{1cm}

\def\ebook{ebook}
\def\book{book}
\def\web{web}
\def\type{web}

\newcommand{\vect}[1]{\overrightarrow{\,\mathstrut#1\,}}

\def\Oij{$\left(\text{O}~;~\vect{\imath},~\vect{\jmath}\right)$}
\def\Oijk{$\left(\text{O}~;~\vect{\imath},~\vect{\jmath},~\vect{k}\right)$}
\def\Ouv{$\left(\text{O}~;~\vect{u},~\vect{v}\right)$}

\hypersetup{breaklinks=true, colorlinks = true, linkcolor = OliveGreen, urlcolor = OliveGreen, citecolor = OliveGreen, pdfauthor={Didier BONNEL - https://www.maths-cours.fr} } % supprime les bordures autour des liens

\renewcommand{\arg}[0]{\text{arg}}

\everymath{\displaystyle}

%================================================================================================================================
%
% Macros - Commandes
%
%================================================================================================================================

\newcommand\meta[2]{    			% Utilisé pour créer le post HTML.
	\def\titre{titre}
	\def\url{url}
	\def\arg{#1}
	\ifx\titre\arg
		\newcommand\maintitle{#2}
		\fancyhead[L]{#2}
		{\Large\sffamily \MakeUppercase{#2}}
		\vspace{1mm}\textcolor{mcvert}{\hrule}
	\fi 
	\ifx\url\arg
		\fancyfoot[L]{\href{https://www.maths-cours.fr#2}{\black \footnotesize{https://www.maths-cours.fr#2}}}
	\fi 
}


\newcommand\TitreC[1]{    		% Titre centré
     \needspace{3\baselineskip}
     \begin{center}\textbf{#1}\end{center}
}

\newcommand\newpar{    		% paragraphe
     \par
}

\newcommand\nosp {    		% commande vide (pas d'espace)
}
\newcommand{\id}[1]{} %ignore

\newcommand\boite[2]{				% Boite simple sans titre
	\vspace{5mm}
	\setlength{\fboxrule}{0.2mm}
	\setlength{\fboxsep}{5mm}	
	\fcolorbox{#1}{#1!3}{\makebox[\linewidth-2\fboxrule-2\fboxsep]{
  		\begin{minipage}[t]{\linewidth-2\fboxrule-4\fboxsep}\setlength{\parskip}{3mm}
  			 #2
  		\end{minipage}
	}}
	\vspace{5mm}
}

\newcommand\CBox[4]{				% Boites
	\vspace{5mm}
	\setlength{\fboxrule}{0.2mm}
	\setlength{\fboxsep}{5mm}
	
	\fcolorbox{#1}{#1!3}{\makebox[\linewidth-2\fboxrule-2\fboxsep]{
		\begin{minipage}[t]{1cm}\setlength{\parskip}{3mm}
	  		\textcolor{#1}{\LARGE{#2}}    
 	 	\end{minipage}  
  		\begin{minipage}[t]{\linewidth-2\fboxrule-4\fboxsep}\setlength{\parskip}{3mm}
			\raisebox{1.2mm}{\normalsize\sffamily{\textcolor{#1}{#3}}}						
  			 #4
  		\end{minipage}
	}}
	\vspace{5mm}
}

\newcommand\cadre[3]{				% Boites convertible html
	\par
	\vspace{2mm}
	\setlength{\fboxrule}{0.1mm}
	\setlength{\fboxsep}{5mm}
	\fcolorbox{#1}{white}{\makebox[\linewidth-2\fboxrule-2\fboxsep]{
  		\begin{minipage}[t]{\linewidth-2\fboxrule-4\fboxsep}\setlength{\parskip}{3mm}
			\raisebox{-2.5mm}{\sffamily \small{\textcolor{#1}{\MakeUppercase{#2}}}}		
			\par		
  			 #3
 	 		\end{minipage}
	}}
		\vspace{2mm}
	\par
}

\newcommand\bloc[3]{				% Boites convertible html sans bordure
     \needspace{2\baselineskip}
     {\sffamily \small{\textcolor{#1}{\MakeUppercase{#2}}}}    
		\par		
  			 #3
		\par
}

\newcommand\CHelp[1]{
     \CBox{Plum}{\faInfoCircle}{À RETENIR}{#1}
}

\newcommand\CUp[1]{
     \CBox{NavyBlue}{\faThumbsOUp}{EN PRATIQUE}{#1}
}

\newcommand\CInfo[1]{
     \CBox{Sepia}{\faArrowCircleRight}{REMARQUE}{#1}
}

\newcommand\CRedac[1]{
     \CBox{PineGreen}{\faEdit}{BIEN R\'EDIGER}{#1}
}

\newcommand\CError[1]{
     \CBox{Red}{\faExclamationTriangle}{ATTENTION}{#1}
}

\newcommand\TitreExo[2]{
\needspace{4\baselineskip}
 {\sffamily\large EXERCICE #1\ (\emph{#2 points})}
\vspace{5mm}
}

\newcommand\img[2]{
          \includegraphics[width=#2\paperwidth]{\imgdir#1}
}

\newcommand\imgsvg[2]{
       \begin{center}   \includegraphics[width=#2\paperwidth]{\imgsvgdir#1} \end{center}
}


\newcommand\Lien[2]{
     \href{#1}{#2 \tiny \faExternalLink}
}
\newcommand\mcLien[2]{
     \href{https~://www.maths-cours.fr/#1}{#2 \tiny \faExternalLink}
}

\newcommand{\euro}{\eurologo{}}

%================================================================================================================================
%
% Macros - Environement
%
%================================================================================================================================

\newenvironment{tex}{ %
}
{%
}

\newenvironment{indente}{ %
	\setlength\parindent{10mm}
}

{
	\setlength\parindent{0mm}
}

\newenvironment{corrige}{%
     \needspace{3\baselineskip}
     \medskip
     \textbf{\textsc{Corrigé}}
     \medskip
}
{
}

\newenvironment{extern}{%
     \begin{center}
     }
     {
     \end{center}
}

\NewEnviron{code}{%
	\par
     \boite{gray}{\texttt{%
     \BODY
     }}
     \par
}

\newenvironment{vbloc}{% boite sans cadre empeche saut de page
     \begin{minipage}[t]{\linewidth}
     }
     {
     \end{minipage}
}
\NewEnviron{h2}{%
    \needspace{3\baselineskip}
    \vspace{0.6cm}
	\noindent \MakeUppercase{\sffamily \large \BODY}
	\vspace{1mm}\textcolor{mcgris}{\hrule}\vspace{0.4cm}
	\par
}{}

\NewEnviron{h3}{%
    \needspace{3\baselineskip}
	\vspace{5mm}
	\textsc{\BODY}
	\par
}

\NewEnviron{margeneg}{ %
\begin{addmargin}[-1cm]{0cm}
\BODY
\end{addmargin}
}

\NewEnviron{html}{%
}

\begin{document}
\meta{url}{/exercices/nombres-complexes-bac-s-pondichery-2018/}
\meta{pid}{7157}
\meta{titre}{Nombres complexes – Bac S Pondichéry 2018}
\meta{type}{exercice}
\begin{h2}Exercice 2 (4 points)\end{h2}
\textbf{Commun  à tous les candidats}
\medskip
Le plan est muni d'un repère orthonormé $(O~;~\overrightarrow{u},~\overrightarrow{v})$.
\smallskip
Les points A, B et C ont pour affixes respectives $a = -4,\: b = 2$ et $c = 4$.
\medskip
\begin{enumerate}
     \item On considère les trois points A$'$, B$'$ et C$'$ d'affixes respectives $a'= ja$, $b'= jb$ et $c'= jc$ où $j$ est le nombre complexe $-\dfrac{1}{2} + \text{i}\dfrac{\sqrt{3}}{2}$.
     \begin{enumerate}[label=\alph*.]
          \item Donner la forme trigonométrique et la forme exponentielle de $j$.
          \par
          En déduire les formes algébriques et exponentielles de $a'$, $b'$ et $c'$.
          \item Les points A, B et C ainsi que les cercles de centre O et de rayon 2, 3 et 4 sont
          représentés sur le graphique fourni en \textbf{Annexe}.
          \par
          Placer les points A$'$, B$'$ et C$'$ sur ce graphique.
     \end{enumerate}
     \item  Montrer que les points A$'$, B$'$ et C$'$ sont alignés.
     \item  On note M le milieu du segment [A$'$C], N le milieu du segment [C$'$C] et P le milieu du
     segment $[\text{C}'\text{A}]$.
     \par
     Démontrer que le triangle MNP est isocèle.
\end{enumerate}
\begin{center}
     \bigskip
     \textbf{ANNEXE}
     \par
     \textit{À compléter et à remettre avec la copie}
     \bigskip
     \begin{extern}%width="400px"
          % src~:nombres-complexes-bac-s-pondichery-2018-1.ggb
          \psset{xunit=1.0cm,yunit=1.0cm,algebraic=true,dimen=middle,dotstyle=*,dotsize=3pt 0,linewidth=0.5pt,arrowsize=3pt 2,arrowinset=0.25}
          \begin{pspicture*}(-4.74,-4.846666666666671)(4.86,4.713333333333333)
               \multips(0,-4)(0,1.0){10}{\psline[linestyle=dashed,linecap=1,dash=1.5pt 1.5pt,linewidth=0.4pt,linecolor=lightgray]{c-c}(-4.74,0)(4.86,0)}
               \multips(-4,0)(1.0,0){10}{\psline[linestyle=dashed,linecap=1,dash=1.5pt 1.5pt,linewidth=0.4pt,linecolor=lightgray]{c-c}(0,-4.846666666666671)(0,4.713333333333333)}
               \psaxes[labelFontSize=\scriptstyle,xAxis=true,yAxis=true,labels=none,Dx=1.,Dy=1.,ticksize=-0.5pt 0,subticks=2]{->}(0,0)(-4.74,-4.84666671)(4.86,4.71333333)
               \pscircle[linewidth=0.5pt](0.,0.){2.}
               \pscircle[linewidth=0.5pt](0.,0.){4.}
               \psline[linewidth=0.5pt]{->}(0.,0.)(1.,0.)
               \psline[linewidth=0.5pt]{->}(0.,0.)(0.,1.)
               \pscircle[linewidth=0.5pt](0.,0.){3.}
               \begin{scriptsize}
                    \psdots[dotsize=3pt 0,dotstyle=*,linecolor=darkgray](2.,0.)
                    \rput[bl](2.1,0.193333333){\normalsize \darkgray{$B$}}
                    \psdots[dotsize=3pt 0,dotstyle=*,linecolor=darkgray](4.,0.)
                    \rput[bl](4.14,0.2133333333){\normalsize \darkgray{$C$}}
                    \psdots[dotsize=3pt 0,dotstyle=*,linecolor=darkgray](0.,0.)
                    \rput[bl](-0.64,-0.5266666666){\normalsize \darkgray{$O$}}
                    \rput[bl](0.24,-0.6266666666){\normalsize $\overrightarrow{u}$}
                    \rput[bl](-0.6,0.25333333){\normalsize $\overrightarrow{v}$}
                    \psdots[dotsize=3pt 0,dotstyle=*,linecolor=darkgray](-4.,0.)
                    \rput[bl](-4.48,0.213333333){\normalsize \darkgray{$A$}}
               \end{scriptsize}
          \end{pspicture*}
     \end{extern}
\end{center}
\begin{corrige}
     \begin{enumerate}
          \item
          \begin{enumerate}[label=\alph*.]
               \item
               $j=-\dfrac{1}{2} + \text{i}\dfrac{\sqrt{3}}{2}$\\
               $\left| j \right| = \sqrt{\left(-\dfrac{1}{2}\right)^2+\left(\dfrac{\sqrt{3}}{2}\right)^2} = \sqrt{\dfrac{1}{4}+\dfrac{3}{4}}=1$\\
               $\theta$ est un argument de $j$ si et seulement si $\cos \theta = -\dfrac{1}{2}$ et $\sin \theta = \dfrac{\sqrt{3}}{2}$. Donc $\dfrac{2\pi}{3}$ est un argument de $j$.
               \par
               La forme trigonométrique de $j$ est~:
               \par
               $j=\cos\left(\dfrac{2\pi}{3}\right) + \text{i}\sin\left(\dfrac{2\pi}{3}\right)$
               \par
               et sa forme exponentielle~:
               \par
               $j= \text{e}^{\frac{2\text{i}\pi}{3}}$.
               \par
               La forme algébrique de $a'$ est~:
               \par
               $a'=aj=-4j=2-2\text{i}\sqrt{3}$.
               \par
               Par ailleurs~:
               \par
               $a'=-4j= -4\text{e}^{\frac{2\text{i}\pi}{3}}$
               \par
               Toutefois $-4$ étant négatif, l'écriture ci-dessus n'est pas la forme exponentielle de $a'$.
               \par
               Pour obtenir la forme exponentielle de $a'$ on utilise le fait que $-1=\text{e}^{\text{i}\pi}$~; par conséquent~:
               \par
               $a'=-4\left( \text{e}^{\frac{2\text{i}\pi}{3}}\right)$\\
               $\phantom{a'}=4 \text{e}^{\text{i}\pi}\text{e}^{\frac{2\text{i}\pi}{3}}$\\
               $\phantom{a'}=4\text{e}^{\text{i}\left( \pi+\frac{2\pi}{3}\right)  }$.
               \par
               La forme exponentielle de $a'$ est donc~:
               \par
               $a'=4\text{e}^{\frac{5\text{i}\pi}{3}}$.
               \smallskip
               La forme algébrique de $b'$ est~:
               \par
               $b'= bj=2j=-1+\text{i}\sqrt{3}$
               \par
               et sa forme exponentielle~:
               \par
               $b'=2j=2\text{e}^{\frac{2\text{i}\pi}{3}}$.
               \par
               Enfin, la forme algébrique de $c'$ est~:
               \par
               $c'= cj=4j=-2+2\text{i}\sqrt{3}$
               \par
               et sa forme exponentielle~:
               \par
               $c'=4j=4\text{e}^{\frac{2\text{i}\pi}{3}}$.
               \item
               Voir figure ci-après.
          \end{enumerate}
          \item
          L'affixe du vecteur $\overrightarrow{A'B'}$ est~:
          \par
          $b'-a'=2j-(-4j)=6j$.
          \par
          L'affixe du vecteur $\overrightarrow{B'C'}$ est~:
          \par
          $c'-b'=4j-2j=2j$.
          \par
          Par conséquent $\overrightarrow{A'B'}$ =3$\overrightarrow{B'C'}$.
          \par
          Les vecteurs $\overrightarrow{A'B'}$ et $\overrightarrow{B'C'}$ sont colinéaires donc les points $A'$, $B'$ et $C'$ sont alignés.
          \item
          ~
          \begin{center}
               \begin{extern}%width="400px"
                    % src~:nombres-complexes-bac-s-pondichery-2018-2.ggb
                    \newrgbcolor{qqwuqq}{0. 0.39215686274509803 0.}
                    \psset{xunit=1.0cm,yunit=1.0cm,algebraic=true,dimen=middle,dotstyle=o,dotsize=3pt 0,linewidth=0.5pt,arrowsize=3pt 2,arrowinset=0.25}
                    \begin{pspicture*}(-4.74,-4.846666666666671)(4.86,4.713333333333333)
                         \multips(0,-4)(0,1.0){10}{\psline[linestyle=dashed,linecap=1,dash=1.5pt 1.5pt,linewidth=0.4pt,linecolor=lightgray]{c-c}(-4.74,0)(4.86,0)}
                         \multips(-4,0)(1.0,0){10}{\psline[linestyle=dashed,linecap=1,dash=1.5pt 1.5pt,linewidth=0.4pt,linecolor=lightgray]{c-c}(0,-4.846666666666671)(0,4.713333333333333)}
                         \psaxes[labelFontSize=\scriptstyle,xAxis=true,yAxis=true,labels=none,Dx=1.,Dy=1.,ticksize=-2pt 0,subticks=2]{->}(0,0)(-4.74,-4.846666667)(4.86,4.71333333)
                         \pscircle[linewidth=0.5pt](0.,0.){2.}
                         \pscircle[linewidth=0.5pt](0.,0.){4.}
                         \psline[linewidth=0.5pt]{->}(0.,0.)(1.,0.)
                         \psline[linewidth=0.5pt]{->}(0.,0.)(0.,1.)
                         \pscircle[linewidth=0.5pt](0.,0.){3.}
                         \psline[linecolor=qqwuqq](-2.995923,1.7343978)(1.004,1.7343978)
                         \psline[linecolor=qqwuqq](1.0040,1.7343978)(2.981,-1.74288)
                         \psline[linecolor=qqwuqq](2.981,-1.74288)(-2.995923759375434,1.734397839456537)
                         \begin{scriptsize}
                              \normalsize
                              \psdots[dotsize=3pt 0,dotstyle=*,linecolor=darkgray](2.,0.)
                              \rput[bl](2.1,0.19333333){\darkgray{$B$}}
                              \psdots[dotsize=3pt 0,dotstyle=*,linecolor=darkgray](4.,0.)
                              \rput[bl](4.14,0.2133333){\darkgray{$C$}}
                              \psdots[dotsize=3pt 0,dotstyle=*,linecolor=darkgray](0.,0.)
                              \rput[bl](-0.74,-0.626666666){\darkgray{$O$}}
                              \rput[bl](0.24,-0.726666){$\overrightarrow{u}$}
                              \rput[bl](-0.6,0.2533333){$\overrightarrow{v}$}
                              \psdots[dotsize=3pt 0,dotstyle=*,linecolor=darkgray](-4.,0.)
                              \rput[bl](-4.48,0.21333333){\darkgray{$A$}}
                              \psdots[dotsize=3pt 0,dotstyle=*,linecolor=blue](1.962,-3.48576)
                              \rput[bl](1.78,-3.246667){\blue{$A'$}}
                              \psdots[dotsize=3pt 0,dotstyle=*,linecolor=blue](-0.9508987,1.759486)
                              \rput[bl](-1.02,2.07333333){\blue{$B'$}}
                              \psdots[dotsize=3pt 0,dotstyle=*,linecolor=blue](-1.99184751875,3.4687956)
                              \rput[bl](-2.08,3.7533333){\blue{$C'$}}
                              \psdots[dotsize=3pt 0,dotstyle=*,linecolor=qqwuqq](2.981,-1.74288)
                              \rput[bl](3.06,-1.66666){\qqwuqq{$M$}}
                              \psdots[dotsize=3pt 0,dotstyle=*,linecolor=qqwuqq](1.00407624,1.7343978)
                              \rput[bl](1.08,1.813333){\qqwuqq{$N$}}
                              \psdots[dotsize=3pt 0,dotstyle=*,linecolor=qqwuqq](-2.99592,1.734)
                              \rput[bl](-3.02,1.9133333){\qqwuqq{$P$}}
                         \end{scriptsize}
                    \end{pspicture*}
               \end{extern}
          \end{center}
          L'affixe de M est~:
          \par
          $m=\dfrac{a'+c}{2}=3-\text{i}\sqrt{3}$
          \par
          L'affixe de N est~:
          \par
          $n=\dfrac{c'+c}{2}=1+\text{i}\sqrt{3}$
          \par
          L'affixe de P est~:
          \par
          $p=\dfrac{c'+a}{2}=-3+\text{i}\sqrt{3}$
          \par
          Montrons que $MN=PN$\\
          $MN=\left|m-n \right| = \left|2-2\text{i}\sqrt{3} \right| $\\
          $\phantom{MN}=\sqrt{2^2+\left(2 \sqrt{3}\right)^2}=\sqrt{4+12}=4$\\
          $PN=\left|n-p \right| =\left|4 \right| = 4$
          \par
          Le triangle $MNP$ est donc isocèle en $N$.
     \end{enumerate}
\end{corrige}
\par

\end{document}