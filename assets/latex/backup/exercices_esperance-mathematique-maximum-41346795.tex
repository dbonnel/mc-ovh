\documentclass[a4paper]{article}

%================================================================================================================================
%
% Packages
%
%================================================================================================================================

\usepackage[T1]{fontenc} 	% pour caractères accentués
\usepackage[utf8]{inputenc}  % encodage utf8
\usepackage[french]{babel}	% langue : français
\usepackage{fourier}			% caractères plus lisibles
\usepackage[dvipsnames]{xcolor} % couleurs
\usepackage{fancyhdr}		% réglage header footer
\usepackage{needspace}		% empêcher sauts de page mal placés
\usepackage{graphicx}		% pour inclure des graphiques
\usepackage{enumitem,cprotect}		% personnalise les listes d'items (nécessaire pour ol, al ...)
\usepackage{hyperref}		% Liens hypertexte
\usepackage{pstricks,pst-all,pst-node,pstricks-add,pst-math,pst-plot,pst-tree,pst-eucl} % pstricks
\usepackage[a4paper,includeheadfoot,top=2cm,left=3cm, bottom=2cm,right=3cm]{geometry} % marges etc.
\usepackage{comment}			% commentaires multilignes
\usepackage{amsmath,environ} % maths (matrices, etc.)
\usepackage{amssymb,makeidx}
\usepackage{bm}				% bold maths
\usepackage{tabularx}		% tableaux
\usepackage{colortbl}		% tableaux en couleur
\usepackage{fontawesome}		% Fontawesome
\usepackage{environ}			% environment with command
\usepackage{fp}				% calculs pour ps-tricks
\usepackage{multido}			% pour ps tricks
\usepackage[np]{numprint}	% formattage nombre
\usepackage{tikz,tkz-tab} 			% package principal TikZ
\usepackage{pgfplots}   % axes
\usepackage{mathrsfs}    % cursives
\usepackage{calc}			% calcul taille boites
\usepackage[scaled=0.875]{helvet} % font sans serif
\usepackage{svg} % svg
\usepackage{scrextend} % local margin
\usepackage{scratch} %scratch
\usepackage{multicol} % colonnes
%\usepackage{infix-RPN,pst-func} % formule en notation polanaise inversée
\usepackage{listings}

%================================================================================================================================
%
% Réglages de base
%
%================================================================================================================================

\lstset{
language=Python,   % R code
literate=
{á}{{\'a}}1
{à}{{\`a}}1
{ã}{{\~a}}1
{é}{{\'e}}1
{è}{{\`e}}1
{ê}{{\^e}}1
{í}{{\'i}}1
{ó}{{\'o}}1
{õ}{{\~o}}1
{ú}{{\'u}}1
{ü}{{\"u}}1
{ç}{{\c{c}}}1
{~}{{ }}1
}


\definecolor{codegreen}{rgb}{0,0.6,0}
\definecolor{codegray}{rgb}{0.5,0.5,0.5}
\definecolor{codepurple}{rgb}{0.58,0,0.82}
\definecolor{backcolour}{rgb}{0.95,0.95,0.92}

\lstdefinestyle{mystyle}{
    backgroundcolor=\color{backcolour},   
    commentstyle=\color{codegreen},
    keywordstyle=\color{magenta},
    numberstyle=\tiny\color{codegray},
    stringstyle=\color{codepurple},
    basicstyle=\ttfamily\footnotesize,
    breakatwhitespace=false,         
    breaklines=true,                 
    captionpos=b,                    
    keepspaces=true,                 
    numbers=left,                    
xleftmargin=2em,
framexleftmargin=2em,            
    showspaces=false,                
    showstringspaces=false,
    showtabs=false,                  
    tabsize=2,
    upquote=true
}

\lstset{style=mystyle}


\lstset{style=mystyle}
\newcommand{\imgdir}{C:/laragon/www/newmc/assets/imgsvg/}
\newcommand{\imgsvgdir}{C:/laragon/www/newmc/assets/imgsvg/}

\definecolor{mcgris}{RGB}{220, 220, 220}% ancien~; pour compatibilité
\definecolor{mcbleu}{RGB}{52, 152, 219}
\definecolor{mcvert}{RGB}{125, 194, 70}
\definecolor{mcmauve}{RGB}{154, 0, 215}
\definecolor{mcorange}{RGB}{255, 96, 0}
\definecolor{mcturquoise}{RGB}{0, 153, 153}
\definecolor{mcrouge}{RGB}{255, 0, 0}
\definecolor{mclightvert}{RGB}{205, 234, 190}

\definecolor{gris}{RGB}{220, 220, 220}
\definecolor{bleu}{RGB}{52, 152, 219}
\definecolor{vert}{RGB}{125, 194, 70}
\definecolor{mauve}{RGB}{154, 0, 215}
\definecolor{orange}{RGB}{255, 96, 0}
\definecolor{turquoise}{RGB}{0, 153, 153}
\definecolor{rouge}{RGB}{255, 0, 0}
\definecolor{lightvert}{RGB}{205, 234, 190}
\setitemize[0]{label=\color{lightvert}  $\bullet$}

\pagestyle{fancy}
\renewcommand{\headrulewidth}{0.2pt}
\fancyhead[L]{maths-cours.fr}
\fancyhead[R]{\thepage}
\renewcommand{\footrulewidth}{0.2pt}
\fancyfoot[C]{}

\newcolumntype{C}{>{\centering\arraybackslash}X}
\newcolumntype{s}{>{\hsize=.35\hsize\arraybackslash}X}

\setlength{\parindent}{0pt}		 
\setlength{\parskip}{3mm}
\setlength{\headheight}{1cm}

\def\ebook{ebook}
\def\book{book}
\def\web{web}
\def\type{web}

\newcommand{\vect}[1]{\overrightarrow{\,\mathstrut#1\,}}

\def\Oij{$\left(\text{O}~;~\vect{\imath},~\vect{\jmath}\right)$}
\def\Oijk{$\left(\text{O}~;~\vect{\imath},~\vect{\jmath},~\vect{k}\right)$}
\def\Ouv{$\left(\text{O}~;~\vect{u},~\vect{v}\right)$}

\hypersetup{breaklinks=true, colorlinks = true, linkcolor = OliveGreen, urlcolor = OliveGreen, citecolor = OliveGreen, pdfauthor={Didier BONNEL - https://www.maths-cours.fr} } % supprime les bordures autour des liens

\renewcommand{\arg}[0]{\text{arg}}

\everymath{\displaystyle}

%================================================================================================================================
%
% Macros - Commandes
%
%================================================================================================================================

\newcommand\meta[2]{    			% Utilisé pour créer le post HTML.
	\def\titre{titre}
	\def\url{url}
	\def\arg{#1}
	\ifx\titre\arg
		\newcommand\maintitle{#2}
		\fancyhead[L]{#2}
		{\Large\sffamily \MakeUppercase{#2}}
		\vspace{1mm}\textcolor{mcvert}{\hrule}
	\fi 
	\ifx\url\arg
		\fancyfoot[L]{\href{https://www.maths-cours.fr#2}{\black \footnotesize{https://www.maths-cours.fr#2}}}
	\fi 
}


\newcommand\TitreC[1]{    		% Titre centré
     \needspace{3\baselineskip}
     \begin{center}\textbf{#1}\end{center}
}

\newcommand\newpar{    		% paragraphe
     \par
}

\newcommand\nosp {    		% commande vide (pas d'espace)
}
\newcommand{\id}[1]{} %ignore

\newcommand\boite[2]{				% Boite simple sans titre
	\vspace{5mm}
	\setlength{\fboxrule}{0.2mm}
	\setlength{\fboxsep}{5mm}	
	\fcolorbox{#1}{#1!3}{\makebox[\linewidth-2\fboxrule-2\fboxsep]{
  		\begin{minipage}[t]{\linewidth-2\fboxrule-4\fboxsep}\setlength{\parskip}{3mm}
  			 #2
  		\end{minipage}
	}}
	\vspace{5mm}
}

\newcommand\CBox[4]{				% Boites
	\vspace{5mm}
	\setlength{\fboxrule}{0.2mm}
	\setlength{\fboxsep}{5mm}
	
	\fcolorbox{#1}{#1!3}{\makebox[\linewidth-2\fboxrule-2\fboxsep]{
		\begin{minipage}[t]{1cm}\setlength{\parskip}{3mm}
	  		\textcolor{#1}{\LARGE{#2}}    
 	 	\end{minipage}  
  		\begin{minipage}[t]{\linewidth-2\fboxrule-4\fboxsep}\setlength{\parskip}{3mm}
			\raisebox{1.2mm}{\normalsize\sffamily{\textcolor{#1}{#3}}}						
  			 #4
  		\end{minipage}
	}}
	\vspace{5mm}
}

\newcommand\cadre[3]{				% Boites convertible html
	\par
	\vspace{2mm}
	\setlength{\fboxrule}{0.1mm}
	\setlength{\fboxsep}{5mm}
	\fcolorbox{#1}{white}{\makebox[\linewidth-2\fboxrule-2\fboxsep]{
  		\begin{minipage}[t]{\linewidth-2\fboxrule-4\fboxsep}\setlength{\parskip}{3mm}
			\raisebox{-2.5mm}{\sffamily \small{\textcolor{#1}{\MakeUppercase{#2}}}}		
			\par		
  			 #3
 	 		\end{minipage}
	}}
		\vspace{2mm}
	\par
}

\newcommand\bloc[3]{				% Boites convertible html sans bordure
     \needspace{2\baselineskip}
     {\sffamily \small{\textcolor{#1}{\MakeUppercase{#2}}}}    
		\par		
  			 #3
		\par
}

\newcommand\CHelp[1]{
     \CBox{Plum}{\faInfoCircle}{À RETENIR}{#1}
}

\newcommand\CUp[1]{
     \CBox{NavyBlue}{\faThumbsOUp}{EN PRATIQUE}{#1}
}

\newcommand\CInfo[1]{
     \CBox{Sepia}{\faArrowCircleRight}{REMARQUE}{#1}
}

\newcommand\CRedac[1]{
     \CBox{PineGreen}{\faEdit}{BIEN R\'EDIGER}{#1}
}

\newcommand\CError[1]{
     \CBox{Red}{\faExclamationTriangle}{ATTENTION}{#1}
}

\newcommand\TitreExo[2]{
\needspace{4\baselineskip}
 {\sffamily\large EXERCICE #1\ (\emph{#2 points})}
\vspace{5mm}
}

\newcommand\img[2]{
          \includegraphics[width=#2\paperwidth]{\imgdir#1}
}

\newcommand\imgsvg[2]{
       \begin{center}   \includegraphics[width=#2\paperwidth]{\imgsvgdir#1} \end{center}
}


\newcommand\Lien[2]{
     \href{#1}{#2 \tiny \faExternalLink}
}
\newcommand\mcLien[2]{
     \href{https~://www.maths-cours.fr/#1}{#2 \tiny \faExternalLink}
}

\newcommand{\euro}{\eurologo{}}

%================================================================================================================================
%
% Macros - Environement
%
%================================================================================================================================

\newenvironment{tex}{ %
}
{%
}

\newenvironment{indente}{ %
	\setlength\parindent{10mm}
}

{
	\setlength\parindent{0mm}
}

\newenvironment{corrige}{%
     \needspace{3\baselineskip}
     \medskip
     \textbf{\textsc{Corrigé}}
     \medskip
}
{
}

\newenvironment{extern}{%
     \begin{center}
     }
     {
     \end{center}
}

\NewEnviron{code}{%
	\par
     \boite{gray}{\texttt{%
     \BODY
     }}
     \par
}

\newenvironment{vbloc}{% boite sans cadre empeche saut de page
     \begin{minipage}[t]{\linewidth}
     }
     {
     \end{minipage}
}
\NewEnviron{h2}{%
    \needspace{3\baselineskip}
    \vspace{0.6cm}
	\noindent \MakeUppercase{\sffamily \large \BODY}
	\vspace{1mm}\textcolor{mcgris}{\hrule}\vspace{0.4cm}
	\par
}{}

\NewEnviron{h3}{%
    \needspace{3\baselineskip}
	\vspace{5mm}
	\textsc{\BODY}
	\par
}

\NewEnviron{margeneg}{ %
\begin{addmargin}[-1cm]{0cm}
\BODY
\end{addmargin}
}

\NewEnviron{html}{%
}

\begin{document}
\meta{url}{/exercices/esperance-mathematique-maximum/}
\meta{pid}{3319}
\meta{titre}{Espérance mathématique maximale}
\meta{type}{exercices}
%
Une urne contient $n$ boules blanches et $2n$ boules rouges (avec $n \geqslant 1$).
\par
On tire au hasard et \textbf{sans remise}, trois boules de l'urne.
\begin{enumerate}
     \item
     Représenter cette situation à l'aide d'un arbre pondéré.
     \item
     On considère le jeu suivant :
     \begin{itemize}
          \item
          Si toutes les boules tirées sont blanches, le joueur perd 16 euros.
          \item
          Si toutes les boules tirées sont rouges, le joueur gagne 2 euros.
          \item
          Dans les autres cas, le joueur ne gagne rien et ne perd rien.
     \end{itemize}
     On note $X$ la variable aléatoire représentant le gain algébrique du joueur.
     \par
     Déterminer la loi de probabilité de $X$.
     \par
     Calculer l'espérance mathématique $E(X)$.
     \item
     Soit la fonction $f$ définie sur $\left[1~;~+\infty\right[$ par $f(x)=\frac{x-1}{(3x-1)(3x-2)}$.
     \begin{enumerate}
          \item
          Etudier les variations de la fonction $f$.
          \item
          Combien de boules doit contenir l'urne pour que l'espérance mathématique $E(X)$ soit maximale ?
          \par
          Quelle est alors la valeur de cette espérance mathématique ?
     \end{enumerate}
\end{enumerate}
\begin{corrige}
     \begin{enumerate}
          \item
          \textit{Pour ne pas surcharger la figure, seules les probabilités utilisées lors des questions suivantes ont été indiquées.}
%##
% type=arbre; width=35; wcell=3.5; hcell=1.5
%--
% >B: \dfrac{ 1 }{ 3 }  
% >>B: \dfrac{ n-1 }{ 3n-1 }  
% >>>B: \dfrac{ n-2 }{ 3n-2 } 
% >>>R
% >>R
% >>>B
% >>>R
% >R: \dfrac{ 2 }{ 3 }  
% >>B
% >>>B
% >>>R
% >>R:\dfrac{ 2n-1 }{ 3n-1 }  
% >>>B
% >>>R: \dfrac{ 2n-2 }{ 3n-2 }
%--
\begin{center}
 \begin{extern}%style="width:35rem" alt="Arbre pondéré"
    \resizebox{11cm}{!}{
       \definecolor{dark}{gray}{0.1}
       \begin{tikzpicture}[scale=.8, line width=.5pt, dark]
       \def\width{3.5}
       \def\height{1.5}
       \tikzstyle{noeud}=[fill=white,circle,draw]
       \tikzstyle{poids}=[fill=white,font=\footnotesize,midway]
    \node[noeud] (r) at ({1*\width},{-3.5*\height}) {$$};
    \node[noeud] (ra) at ({2*\width},{-1.5*\height}) {$B$};
     \draw (r) -- (ra) node [poids] {$\dfrac{1}{3}$};
    \node[noeud] (raa) at ({3*\width},{-0.5*\height}) {$B$};
     \draw (ra) -- (raa) node [poids] {$\dfrac{n-1}{3n-1}$};
    \node[noeud] (raaa) at ({4*\width},{0*\height}) {$B$};
     \draw (raa) -- (raaa) node [poids] {$\dfrac{n-2}{3n-2}$};
    \node[noeud] (raab) at ({4*\width},{-1*\height}) {$R$};
     \draw (raa) -- (raab);
    \node[noeud] (rab) at ({3*\width},{-2.5*\height}) {$R$};
     \draw (ra) -- (rab);
    \node[noeud] (raba) at ({4*\width},{-2*\height}) {$B$};
     \draw (rab) -- (raba);
    \node[noeud] (rabb) at ({4*\width},{-3*\height}) {$R$};
     \draw (rab) -- (rabb);
    \node[noeud] (rb) at ({2*\width},{-5.5*\height}) {$R$};
     \draw (r) -- (rb) node [poids] {$\dfrac{2}{3}$};
    \node[noeud] (rba) at ({3*\width},{-4.5*\height}) {$B$};
     \draw (rb) -- (rba);
    \node[noeud] (rbaa) at ({4*\width},{-4*\height}) {$B$};
     \draw (rba) -- (rbaa);
    \node[noeud] (rbab) at ({4*\width},{-5*\height}) {$R$};
     \draw (rba) -- (rbab);
    \node[noeud] (rbb) at ({3*\width},{-6.5*\height}) {$R$};
     \draw (rb) -- (rbb) node [poids] {$\dfrac{2n-1}{3n-1}$};
    \node[noeud] (rbba) at ({4*\width},{-6*\height}) {$B$};
     \draw (rbb) -- (rbba);
    \node[noeud] (rbbb) at ({4*\width},{-7*\height}) {$R$};
     \draw (rbb) -- (rbbb) node [poids] {$\dfrac{2n-2}{3n-2}$};
       \end{tikzpicture}
      }
   \end{extern}
\end{center}
%##
\begin{itemize}
               \item
               \textbf{Au premier niveau} (tirage de la première boule), l'urne contient $n$ boules blanches sur un total de $3n$ boules. La probabilité de tirer une boule blanche est donc $\frac{n}{3n} = \frac{1}{3}$.
               \par
               L'urne contient $2n$ boules rouges sur un total de $3n$ boules. La probabilité de tirer une boule rouge est donc $\frac{2n}{3n} = \frac{2}{3}$.
               \item
               \textbf{Au second niveau} (tirage de la seconde boule) :
               \begin{itemize}
                    \item
                    Si l'on a tiré une boule blanche en premier ,  il reste alors $n-1$ boules blanches sur un total de $3n-1$ boules. La probabilité de tirer une boule blanche est donc alors $\frac{n-1}{3n-1}$.
                    \item
                    ...
                    \item
                    Si l'on a tiré une boule rouge  en premier ,  il reste alors $2n-1$ boules rouges sur un total de $3n-1$ boules. La probabilité de tirer une boule rouge est donc alors $\frac{2n-1}{3n-1}$.
               \end{itemize}
               \item
               \textbf{Au troisième niveau} (tirage de la troisième boule) :
               \begin{itemize}
                    \item
                    Si l'on a tiré deux boules blanches lors des deux premiers tirages,  il reste alors $n-2$ boules blanches sur un total de $3n-2$ boules. La probabilité de tirer une boule blanche est donc alors $\frac{n-2}{3n-2}$.
                    \item
                    ...
                    \item
                    Si l'on a tiré deux boules rouges lors des deux premiers tirages, il reste alors $2n-2$ boules rouges sur un total de $3n-2$ boules. La probabilité de tirer une boule rouge est donc alors $\frac{2n-2}{3n-2}$.
               \end{itemize}
          \end{itemize}
          \item
          $X$ prend la valeur -16 si les trois boules sont blanches, c'est à dire avec une probabilité :
          \par
          $p(X=-16)=\frac{1}{3} \times \frac{n-1}{3n-1} \times \frac{n-2}{3n-2} $
          \par
          $p(X=-16)=\frac{(n-1)(n-2)}{3(3n-1)(3n-2)}$
          \par
          $X$ prend la valeur -2 si les trois boules sont rouges, c'est à dire avec une probabilité :
          \par
          $p(X=2)=\frac{2}{3} \times \frac{2n-1}{3n-1} \times \frac{2n-2}{3n-2} $
          \par
          $p(X=2)=\frac{2(2n-1)(2n-2)}{3(3n-1)(3n-2)}$
          \par
          Dans les autres cas $X$ prend la valeur $0$.
          \par
          Le total des probabilités étant égal à $1$ on obtient :
          \par
          $p(X=0)=1-p(X=-16)-p(X=2) $
          \par
          $p(X=0)=1-\frac{(n-1)(n-2)}{3(3n-1)(3n-2)}$$-\frac{2(2n-1)(2n-2)}{3(3n-1)(3n-2)}$
          \par
          $p(X=0)=\frac{6n^2-4n}{(3n-1)(3n-2)}$
          \par
          La loi de probabilité de $X$ est donc :
          \par
          [table class=compact]$x_i$ | $-16$ | $0$ | $2$
          \par
          $p(X=x_i)$ |$\frac{(n-1)(n-2)}{3(3n-1)(3n-2)}$ | $\frac{6n^2-4n}{(3n-1)(3n-2)}$| $\frac{2(2n-1)(2n-2)}{3(3n-1)(3n-2)}$[/table]
          \par
          L'espérance mathématique de $X$ est :
          \par
          $E(X)=-16\times p(X=-16)+0 \times p(X=0)+2 \times p(X=2)$
          \par
          $\phantom{E(X)}=-16\times \frac{(n-1)(n-2)}{3(3n-1)(3n-2)}$$+ 2 \times \frac{2(2n-1)(2n-2)}{3(3n-1)(3n-2)}$
          \par
          $\phantom{E(X)}=-16\times \frac{n^2-3n+2}{3(3n-1)(3n-2)}$$+ 2 \times \frac{8n^2-12n+4}{3(3n-1)(3n-2)}$
          \par
          $\phantom{E(X)}=\frac{24n-24}{3(3n-1)(3n-2)}$
          \par
          $\phantom{E(X)}=\frac{8(n-1)}{(3n-1)(3n-2)}$
          \item
          \begin{enumerate}
               \item
               La fonction $f$ est définie et dérivable sur $\left[1;+\infty \right[$.
               \par
               $f$ est de la forme $\frac{u}{v}$ avec
               \par
               $u(x)=x-1$ donc $u ^{\prime}(x)=1$
               \par
               $v(x)=(3x-1)(3x-2)=9x^2-9x+2$ donc $v^{\prime}(x) = 18x-9$
               \par
               Par conséquent :
               \par
               $f^{\prime}(x)=\frac{9x^2-9x+2-(x-1)(18x-9)}{(3x-1)(3x-2))^2}$
               \par
               $f^{\prime}(x)=\frac{-9x^2+18x-7}{((3x-1)(3x-2))^2}$
               \par
               Le dénominateur est positif et le numérateur est un polynôme du second degré.
               \par
               $\Delta = 18^2-4 \times 7 \times 9=72  >  0$
               \par
               Le numérateur admet deux racines :
               \par
               $x_1 = \frac{-18+6\sqrt{2}}{-18}= \frac{3-\sqrt{2}}{3}$
               \par
               $x_2 = \frac{-18-6\sqrt{2}}{-18}= \frac{3+\sqrt{2}}{3}$
               \par
               $x_1$ est inférieur à $1$ et $x_2 \approx 1,47 \in \left[1;+\infty \right[$.
               \par
               On obtient le tableau de variations suivant sur  $\left[1;+\infty \right[$ :

\begin{center}
\imgsvg{esperance-mathematique-maximum-2}{0.3}% alt="Espérance mathématique maximale" style="width:40rem"
\end{center}

               \item
               D'après la question \textbf{2.}, $E(X)=8f(n)$
               \par
               L'espérance mathématique est donc maximale pour la valeur de $n$ qui maximise $f(n)$.
               \par
               D'après la question précédente $f$ est décroissante sur $\left[x_2;+\infty\right[$ donc sur $\left[2;+\infty\right[$ puisque $x_2  < 2$.
               \par
               Les seules valeurs \textbf{entières} susceptibles de maximiser $f$ sont donc $1$ ou $2$.
               \par
               Or $f(1)=0$ et $f(2)=\frac{1}{20}$.
               \par
               Donc, l'espérance mathématique est maximale pour $n=2$ c'est à dire si l'urne contient $2$ boules blanches et $4$ boules rouges.
               \par
               Cette espérance vaut alors $E(X)=8f(2)=\frac{8}{20}=0,4$
          \end{enumerate}
     \end{enumerate}
\end{corrige} 

\end{document}