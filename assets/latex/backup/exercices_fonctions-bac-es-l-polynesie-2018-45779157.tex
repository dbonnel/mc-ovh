\documentclass[a4paper]{article}

%================================================================================================================================
%
% Packages
%
%================================================================================================================================

\usepackage[T1]{fontenc} 	% pour caractères accentués
\usepackage[utf8]{inputenc}  % encodage utf8
\usepackage[french]{babel}	% langue : français
\usepackage{fourier}			% caractères plus lisibles
\usepackage[dvipsnames]{xcolor} % couleurs
\usepackage{fancyhdr}		% réglage header footer
\usepackage{needspace}		% empêcher sauts de page mal placés
\usepackage{graphicx}		% pour inclure des graphiques
\usepackage{enumitem,cprotect}		% personnalise les listes d'items (nécessaire pour ol, al ...)
\usepackage{hyperref}		% Liens hypertexte
\usepackage{pstricks,pst-all,pst-node,pstricks-add,pst-math,pst-plot,pst-tree,pst-eucl} % pstricks
\usepackage[a4paper,includeheadfoot,top=2cm,left=3cm, bottom=2cm,right=3cm]{geometry} % marges etc.
\usepackage{comment}			% commentaires multilignes
\usepackage{amsmath,environ} % maths (matrices, etc.)
\usepackage{amssymb,makeidx}
\usepackage{bm}				% bold maths
\usepackage{tabularx}		% tableaux
\usepackage{colortbl}		% tableaux en couleur
\usepackage{fontawesome}		% Fontawesome
\usepackage{environ}			% environment with command
\usepackage{fp}				% calculs pour ps-tricks
\usepackage{multido}			% pour ps tricks
\usepackage[np]{numprint}	% formattage nombre
\usepackage{tikz,tkz-tab} 			% package principal TikZ
\usepackage{pgfplots}   % axes
\usepackage{mathrsfs}    % cursives
\usepackage{calc}			% calcul taille boites
\usepackage[scaled=0.875]{helvet} % font sans serif
\usepackage{svg} % svg
\usepackage{scrextend} % local margin
\usepackage{scratch} %scratch
\usepackage{multicol} % colonnes
%\usepackage{infix-RPN,pst-func} % formule en notation polanaise inversée
\usepackage{listings}

%================================================================================================================================
%
% Réglages de base
%
%================================================================================================================================

\lstset{
language=Python,   % R code
literate=
{á}{{\'a}}1
{à}{{\`a}}1
{ã}{{\~a}}1
{é}{{\'e}}1
{è}{{\`e}}1
{ê}{{\^e}}1
{í}{{\'i}}1
{ó}{{\'o}}1
{õ}{{\~o}}1
{ú}{{\'u}}1
{ü}{{\"u}}1
{ç}{{\c{c}}}1
{~}{{ }}1
}


\definecolor{codegreen}{rgb}{0,0.6,0}
\definecolor{codegray}{rgb}{0.5,0.5,0.5}
\definecolor{codepurple}{rgb}{0.58,0,0.82}
\definecolor{backcolour}{rgb}{0.95,0.95,0.92}

\lstdefinestyle{mystyle}{
    backgroundcolor=\color{backcolour},   
    commentstyle=\color{codegreen},
    keywordstyle=\color{magenta},
    numberstyle=\tiny\color{codegray},
    stringstyle=\color{codepurple},
    basicstyle=\ttfamily\footnotesize,
    breakatwhitespace=false,         
    breaklines=true,                 
    captionpos=b,                    
    keepspaces=true,                 
    numbers=left,                    
xleftmargin=2em,
framexleftmargin=2em,            
    showspaces=false,                
    showstringspaces=false,
    showtabs=false,                  
    tabsize=2,
    upquote=true
}

\lstset{style=mystyle}


\lstset{style=mystyle}
\newcommand{\imgdir}{C:/laragon/www/newmc/assets/imgsvg/}
\newcommand{\imgsvgdir}{C:/laragon/www/newmc/assets/imgsvg/}

\definecolor{mcgris}{RGB}{220, 220, 220}% ancien~; pour compatibilité
\definecolor{mcbleu}{RGB}{52, 152, 219}
\definecolor{mcvert}{RGB}{125, 194, 70}
\definecolor{mcmauve}{RGB}{154, 0, 215}
\definecolor{mcorange}{RGB}{255, 96, 0}
\definecolor{mcturquoise}{RGB}{0, 153, 153}
\definecolor{mcrouge}{RGB}{255, 0, 0}
\definecolor{mclightvert}{RGB}{205, 234, 190}

\definecolor{gris}{RGB}{220, 220, 220}
\definecolor{bleu}{RGB}{52, 152, 219}
\definecolor{vert}{RGB}{125, 194, 70}
\definecolor{mauve}{RGB}{154, 0, 215}
\definecolor{orange}{RGB}{255, 96, 0}
\definecolor{turquoise}{RGB}{0, 153, 153}
\definecolor{rouge}{RGB}{255, 0, 0}
\definecolor{lightvert}{RGB}{205, 234, 190}
\setitemize[0]{label=\color{lightvert}  $\bullet$}

\pagestyle{fancy}
\renewcommand{\headrulewidth}{0.2pt}
\fancyhead[L]{maths-cours.fr}
\fancyhead[R]{\thepage}
\renewcommand{\footrulewidth}{0.2pt}
\fancyfoot[C]{}

\newcolumntype{C}{>{\centering\arraybackslash}X}
\newcolumntype{s}{>{\hsize=.35\hsize\arraybackslash}X}

\setlength{\parindent}{0pt}		 
\setlength{\parskip}{3mm}
\setlength{\headheight}{1cm}

\def\ebook{ebook}
\def\book{book}
\def\web{web}
\def\type{web}

\newcommand{\vect}[1]{\overrightarrow{\,\mathstrut#1\,}}

\def\Oij{$\left(\text{O}~;~\vect{\imath},~\vect{\jmath}\right)$}
\def\Oijk{$\left(\text{O}~;~\vect{\imath},~\vect{\jmath},~\vect{k}\right)$}
\def\Ouv{$\left(\text{O}~;~\vect{u},~\vect{v}\right)$}

\hypersetup{breaklinks=true, colorlinks = true, linkcolor = OliveGreen, urlcolor = OliveGreen, citecolor = OliveGreen, pdfauthor={Didier BONNEL - https://www.maths-cours.fr} } % supprime les bordures autour des liens

\renewcommand{\arg}[0]{\text{arg}}

\everymath{\displaystyle}

%================================================================================================================================
%
% Macros - Commandes
%
%================================================================================================================================

\newcommand\meta[2]{    			% Utilisé pour créer le post HTML.
	\def\titre{titre}
	\def\url{url}
	\def\arg{#1}
	\ifx\titre\arg
		\newcommand\maintitle{#2}
		\fancyhead[L]{#2}
		{\Large\sffamily \MakeUppercase{#2}}
		\vspace{1mm}\textcolor{mcvert}{\hrule}
	\fi 
	\ifx\url\arg
		\fancyfoot[L]{\href{https://www.maths-cours.fr#2}{\black \footnotesize{https://www.maths-cours.fr#2}}}
	\fi 
}


\newcommand\TitreC[1]{    		% Titre centré
     \needspace{3\baselineskip}
     \begin{center}\textbf{#1}\end{center}
}

\newcommand\newpar{    		% paragraphe
     \par
}

\newcommand\nosp {    		% commande vide (pas d'espace)
}
\newcommand{\id}[1]{} %ignore

\newcommand\boite[2]{				% Boite simple sans titre
	\vspace{5mm}
	\setlength{\fboxrule}{0.2mm}
	\setlength{\fboxsep}{5mm}	
	\fcolorbox{#1}{#1!3}{\makebox[\linewidth-2\fboxrule-2\fboxsep]{
  		\begin{minipage}[t]{\linewidth-2\fboxrule-4\fboxsep}\setlength{\parskip}{3mm}
  			 #2
  		\end{minipage}
	}}
	\vspace{5mm}
}

\newcommand\CBox[4]{				% Boites
	\vspace{5mm}
	\setlength{\fboxrule}{0.2mm}
	\setlength{\fboxsep}{5mm}
	
	\fcolorbox{#1}{#1!3}{\makebox[\linewidth-2\fboxrule-2\fboxsep]{
		\begin{minipage}[t]{1cm}\setlength{\parskip}{3mm}
	  		\textcolor{#1}{\LARGE{#2}}    
 	 	\end{minipage}  
  		\begin{minipage}[t]{\linewidth-2\fboxrule-4\fboxsep}\setlength{\parskip}{3mm}
			\raisebox{1.2mm}{\normalsize\sffamily{\textcolor{#1}{#3}}}						
  			 #4
  		\end{minipage}
	}}
	\vspace{5mm}
}

\newcommand\cadre[3]{				% Boites convertible html
	\par
	\vspace{2mm}
	\setlength{\fboxrule}{0.1mm}
	\setlength{\fboxsep}{5mm}
	\fcolorbox{#1}{white}{\makebox[\linewidth-2\fboxrule-2\fboxsep]{
  		\begin{minipage}[t]{\linewidth-2\fboxrule-4\fboxsep}\setlength{\parskip}{3mm}
			\raisebox{-2.5mm}{\sffamily \small{\textcolor{#1}{\MakeUppercase{#2}}}}		
			\par		
  			 #3
 	 		\end{minipage}
	}}
		\vspace{2mm}
	\par
}

\newcommand\bloc[3]{				% Boites convertible html sans bordure
     \needspace{2\baselineskip}
     {\sffamily \small{\textcolor{#1}{\MakeUppercase{#2}}}}    
		\par		
  			 #3
		\par
}

\newcommand\CHelp[1]{
     \CBox{Plum}{\faInfoCircle}{À RETENIR}{#1}
}

\newcommand\CUp[1]{
     \CBox{NavyBlue}{\faThumbsOUp}{EN PRATIQUE}{#1}
}

\newcommand\CInfo[1]{
     \CBox{Sepia}{\faArrowCircleRight}{REMARQUE}{#1}
}

\newcommand\CRedac[1]{
     \CBox{PineGreen}{\faEdit}{BIEN R\'EDIGER}{#1}
}

\newcommand\CError[1]{
     \CBox{Red}{\faExclamationTriangle}{ATTENTION}{#1}
}

\newcommand\TitreExo[2]{
\needspace{4\baselineskip}
 {\sffamily\large EXERCICE #1\ (\emph{#2 points})}
\vspace{5mm}
}

\newcommand\img[2]{
          \includegraphics[width=#2\paperwidth]{\imgdir#1}
}

\newcommand\imgsvg[2]{
       \begin{center}   \includegraphics[width=#2\paperwidth]{\imgsvgdir#1} \end{center}
}


\newcommand\Lien[2]{
     \href{#1}{#2 \tiny \faExternalLink}
}
\newcommand\mcLien[2]{
     \href{https~://www.maths-cours.fr/#1}{#2 \tiny \faExternalLink}
}

\newcommand{\euro}{\eurologo{}}

%================================================================================================================================
%
% Macros - Environement
%
%================================================================================================================================

\newenvironment{tex}{ %
}
{%
}

\newenvironment{indente}{ %
	\setlength\parindent{10mm}
}

{
	\setlength\parindent{0mm}
}

\newenvironment{corrige}{%
     \needspace{3\baselineskip}
     \medskip
     \textbf{\textsc{Corrigé}}
     \medskip
}
{
}

\newenvironment{extern}{%
     \begin{center}
     }
     {
     \end{center}
}

\NewEnviron{code}{%
	\par
     \boite{gray}{\texttt{%
     \BODY
     }}
     \par
}

\newenvironment{vbloc}{% boite sans cadre empeche saut de page
     \begin{minipage}[t]{\linewidth}
     }
     {
     \end{minipage}
}
\NewEnviron{h2}{%
    \needspace{3\baselineskip}
    \vspace{0.6cm}
	\noindent \MakeUppercase{\sffamily \large \BODY}
	\vspace{1mm}\textcolor{mcgris}{\hrule}\vspace{0.4cm}
	\par
}{}

\NewEnviron{h3}{%
    \needspace{3\baselineskip}
	\vspace{5mm}
	\textsc{\BODY}
	\par
}

\NewEnviron{margeneg}{ %
\begin{addmargin}[-1cm]{0cm}
\BODY
\end{addmargin}
}

\NewEnviron{html}{%
}

\begin{document}
\meta{url}{/exercices/fonctions-bac-es-l-polynesie-2018/}
\meta{pid}{9326}
\meta{titre}{Fonctions – Bac ES/L Polynésie 2018}
\meta{type}{exercices}
%
\begin{h2}Exercice 4 (5 points)\end{h2}
\textbf{Commun à  tous les candidats}
\par
\begin{center}
     \textit{ Les parties de cet exercice peuvent être traitées indépendamment.}
\end{center}
\par
Une usine qui fabrique un produit A, décide de fabriquer un nouveau produit B afin d'augmenter son chiffre d'affaires. La quantité, exprimée en tonnes, fabriquée par jour par l'usine est modélisée par~:
\begin{itemize}
     \item la fonction $f$ définie sur [0~;~14] par
     \par
     \[f(x) = 2~000\text{e}^{-0,2x}\]
     \par
     pour le produit A~;
     \item  la fonction $g$ définie sur [0~;~14] par
     \par
     \[g (x)= 15x^2 + 50 x\]
     \par
     pour le produit B, où $x$ est la durée écoulée depuis le lancement du nouveau produit B exprimée en mois.
\end{itemize}
Leurs courbes représentatives respectives $\mathscr{C}_f$ et $\mathscr{C}_g$ sont données ci-dessous.
\begin{center}
     \begin{extern}%width="600" alt="Fonctions quantités Bac ES/L Polynésie 2018"
          \resizebox{10cm}{!}{
               \psset{xunit=0.8cm,yunit=0.0025cm}
               \begin{pspicture}(-1,-200)(17.5,3700)
                    \multido{\n=0+1}{18}{\psline[linewidth=0.2pt,linecolor=gray](\n,0)(\n,3500)}
                    \multido{\n=0+100}{36}{\psline[linewidth=0.1pt,linecolor=lightgray](0,\n)(17,\n)}
                    \multido{\n=0+500}{7}{\psline[linewidth=0.2pt,linecolor=gray](0,\n)(17,\n)}
                    \psaxes[linewidth=1.25pt,Dy=500]{->}(0,0)(0,0)(17,3500)
                    \uput[r](-1.9,3600){$y$ en tonnes}
                    \uput[u](16,-450){$x$ en mois}
                    \psplot[plotpoints=3000,linewidth=1.25pt,linecolor=blue]{0}{14}{2000 2.71828 0.2 x mul exp div}\uput[u](1,1750){\blue $\mathcal{C}_f$}
                    \psplot[plotpoints=3000,linewidth=1.25pt,linecolor=red]{0}{14}{x dup mul 15 mul 50 x mul add}\uput[u](1,100){\red $\mathcal{C}_g$}
               \end{pspicture}
          }
     \end{extern}
\end{center}
\TitreC{Partie A}
\medbreak
Par lecture graphique, sans justification et avec la précision permise par le graphique~:
\medbreak
\begin{enumerate}
     \item Déterminer la durée nécessaire pour que la quantité de produit B dépasse celle du produit A.
     \item L'usine ne peut pas fabriquer une quantité journalière de produit B supérieure à 3~000~tonnes.
     \par
     Au bout de combien de mois cette quantité journalière sera atteinte~?
\end{enumerate}
\bigbreak
\TitreC{Partie B}
\medbreak
Pour tout nombre réel $x$ de l'intervalle [0~;~14] on pose $h(x) = f(x) + g(x)$.
\par
On admet que la fonction $h$ ainsi définie est dérivable sur [0~;~14].
\medbreak
\begin{enumerate}
     \item
     \begin{enumerate}[label=\alph*.]
          \item Que modélise cette fonction dans le contexte de l'exercice~?
          \item Montrer que, pour tout nombre réel $x$ de l'intervalle [0~;~14]
          $h'(x) = - 400\text{e}^{-0,2x} + 30x + 50$.
     \end{enumerate}
     \item On admet que le tableau de variation de la fonction $h'$ sur l'intervalle [0~;~14] est~:
     %:-+-+-+-+- Engendré par : http://math.et.info.free.fr/TikZ/TableauxVariations/
     \begin{center}
          \begin{extern}%width="350" alt=""
               \begin{tikzpicture}[scale=0.875]
                    % Styles
                    \tikzstyle{cadre}=[thin]
                    \tikzstyle{fleche}=[->,>=latex,thin]
                    \tikzstyle{nondefini}=[lightgray]
                    % Dimensions Modifiables
                    \def\Lrg{1.5}
                    \def\HtX{0.8}
                    \def\HtY{0.5}
                    % Dimensions Calculées
                    \def\lignex{-0.5*\HtX}
                    \def\lignef{-1.5*\HtX}
                    \def\separateur{-0.5*\Lrg}
                    % Largeur du tableau
                    \def\gauche{-3*\Lrg}
                    \def\droite{3*\Lrg}
                    % Hauteur du tableau
                    \def\haut{0.5*\HtX}
                    \def\bas{-1.5*\HtX-2*\HtY}
                    % Ligne de l'abscisse : x
                    \node at (-1.8*\Lrg,0) {$x$};
                    \node at (0*\Lrg,0) {$0$};
                    \node at (2*\Lrg,0) {$14$};
                    % Ligne de la fonction : f(x)
                    \node  at (-1.69*\Lrg,{-1*\HtX+(-1)*\HtY}) {variation de $h'$};
                    \node (f1) at (0*\Lrg,{-1*\HtX+(-2)*\HtY}) {$-350$};
                    \node (f2) at (2*\Lrg,{-1*\HtX+(0)*\HtY}) {$h'(14) \approx 446$};
                    % Flèches
                    \draw[fleche] (f1) -- (f2);
                    % Encadrement
                    \draw[cadre] (\separateur,\haut) -- (\separateur,\bas);
                    \draw[cadre] (\gauche,\haut) rectangle  (\droite,\bas);
                    \draw[cadre] (\gauche,\lignex) -- (\droite,\lignex);
               \end{tikzpicture}
          \end{extern}
     \end{center}
     %:-+-+-+-+- Fin
     \begin{enumerate}[label=\alph*.]
          \item Justifier que l'équation $h'(x)= 0$ admet une unique solution $\alpha$ sur l'intervalle [0~;~14] et donner un encadrement d'amplitude $0,1$ de $\alpha$.
          \item  En déduire les variations de la fonction $h$ sur l'intervalle [0~;~14].
     \end{enumerate}
     \item Voici un algorithme~:
     \begin{center}
          \begin{extern}%width="320" alt="Algorithme Bac ES/L Polynésie 2018"
               \begin{tabularx}{0.5\linewidth}{|X|}\hline
                    $Y \gets -400 \text{exp}(- 0,2X) + 30X + 50$\\
                    Tant que $Y \leqslant 0$\\
                    \hspace{0.7cm}$X \gets X + 0,1$\\
                    \hspace{0.7cm}$Y \gets  -400 \text{exp}(- 0,2X)+ 30X +50$\\
                    Fin Tant que\\ \hline
               \end{tabularx}
          \end{extern}
     \end{center}
     \begin{enumerate}[label=\alph*.]
          \item Si la variable $X$ contient la valeur 3 avant l'exécution de cet algorithme, que contient la variable $X$ après l'exécution de cet algorithme~?
          \item En supposant toujours que la variable $X$ contient la valeur $3$ avant l'exécution de cet algorithme, modifier l'algorithme de façon à ce que X contienne une valeur approchée à $0,001$ près de a après l'exécution de l'algorithme.
     \end{enumerate}
     \item
     \begin{enumerate}[label=\alph*.]
          \item Vérifier qu'une primitive $H$ de la fonction $h$ sur [0~;~14] est~:
          \par
          \[H(x) = - 10~000 \text{e}^{- 0,2x} + 5x^3 + 25x^2.\]
          \item  Calculer une valeur approchée à l'unité près de
          $\dfrac{1}{12} \displaystyle\int_0^{12}  h(x)\:\text{d}x$.
          \item  Donner une interprétation dans le contexte de l'exercice.
     \end{enumerate}
\end{enumerate}

\end{document}