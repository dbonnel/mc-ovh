\documentclass[a4paper]{article}

%================================================================================================================================
%
% Packages
%
%================================================================================================================================

\usepackage[T1]{fontenc} 	% pour caractères accentués
\usepackage[utf8]{inputenc}  % encodage utf8
\usepackage[french]{babel}	% langue : français
\usepackage{fourier}			% caractères plus lisibles
\usepackage[dvipsnames]{xcolor} % couleurs
\usepackage{fancyhdr}		% réglage header footer
\usepackage{needspace}		% empêcher sauts de page mal placés
\usepackage{graphicx}		% pour inclure des graphiques
\usepackage{enumitem,cprotect}		% personnalise les listes d'items (nécessaire pour ol, al ...)
\usepackage{hyperref}		% Liens hypertexte
\usepackage{pstricks,pst-all,pst-node,pstricks-add,pst-math,pst-plot,pst-tree,pst-eucl} % pstricks
\usepackage[a4paper,includeheadfoot,top=2cm,left=3cm, bottom=2cm,right=3cm]{geometry} % marges etc.
\usepackage{comment}			% commentaires multilignes
\usepackage{amsmath,environ} % maths (matrices, etc.)
\usepackage{amssymb,makeidx}
\usepackage{bm}				% bold maths
\usepackage{tabularx}		% tableaux
\usepackage{colortbl}		% tableaux en couleur
\usepackage{fontawesome}		% Fontawesome
\usepackage{environ}			% environment with command
\usepackage{fp}				% calculs pour ps-tricks
\usepackage{multido}			% pour ps tricks
\usepackage[np]{numprint}	% formattage nombre
\usepackage{tikz,tkz-tab} 			% package principal TikZ
\usepackage{pgfplots}   % axes
\usepackage{mathrsfs}    % cursives
\usepackage{calc}			% calcul taille boites
\usepackage[scaled=0.875]{helvet} % font sans serif
\usepackage{svg} % svg
\usepackage{scrextend} % local margin
\usepackage{scratch} %scratch
\usepackage{multicol} % colonnes
%\usepackage{infix-RPN,pst-func} % formule en notation polanaise inversée
\usepackage{listings}

%================================================================================================================================
%
% Réglages de base
%
%================================================================================================================================

\lstset{
language=Python,   % R code
literate=
{á}{{\'a}}1
{à}{{\`a}}1
{ã}{{\~a}}1
{é}{{\'e}}1
{è}{{\`e}}1
{ê}{{\^e}}1
{í}{{\'i}}1
{ó}{{\'o}}1
{õ}{{\~o}}1
{ú}{{\'u}}1
{ü}{{\"u}}1
{ç}{{\c{c}}}1
{~}{{ }}1
}


\definecolor{codegreen}{rgb}{0,0.6,0}
\definecolor{codegray}{rgb}{0.5,0.5,0.5}
\definecolor{codepurple}{rgb}{0.58,0,0.82}
\definecolor{backcolour}{rgb}{0.95,0.95,0.92}

\lstdefinestyle{mystyle}{
    backgroundcolor=\color{backcolour},   
    commentstyle=\color{codegreen},
    keywordstyle=\color{magenta},
    numberstyle=\tiny\color{codegray},
    stringstyle=\color{codepurple},
    basicstyle=\ttfamily\footnotesize,
    breakatwhitespace=false,         
    breaklines=true,                 
    captionpos=b,                    
    keepspaces=true,                 
    numbers=left,                    
xleftmargin=2em,
framexleftmargin=2em,            
    showspaces=false,                
    showstringspaces=false,
    showtabs=false,                  
    tabsize=2,
    upquote=true
}

\lstset{style=mystyle}


\lstset{style=mystyle}
\newcommand{\imgdir}{C:/laragon/www/newmc/assets/imgsvg/}
\newcommand{\imgsvgdir}{C:/laragon/www/newmc/assets/imgsvg/}

\definecolor{mcgris}{RGB}{220, 220, 220}% ancien~; pour compatibilité
\definecolor{mcbleu}{RGB}{52, 152, 219}
\definecolor{mcvert}{RGB}{125, 194, 70}
\definecolor{mcmauve}{RGB}{154, 0, 215}
\definecolor{mcorange}{RGB}{255, 96, 0}
\definecolor{mcturquoise}{RGB}{0, 153, 153}
\definecolor{mcrouge}{RGB}{255, 0, 0}
\definecolor{mclightvert}{RGB}{205, 234, 190}

\definecolor{gris}{RGB}{220, 220, 220}
\definecolor{bleu}{RGB}{52, 152, 219}
\definecolor{vert}{RGB}{125, 194, 70}
\definecolor{mauve}{RGB}{154, 0, 215}
\definecolor{orange}{RGB}{255, 96, 0}
\definecolor{turquoise}{RGB}{0, 153, 153}
\definecolor{rouge}{RGB}{255, 0, 0}
\definecolor{lightvert}{RGB}{205, 234, 190}
\setitemize[0]{label=\color{lightvert}  $\bullet$}

\pagestyle{fancy}
\renewcommand{\headrulewidth}{0.2pt}
\fancyhead[L]{maths-cours.fr}
\fancyhead[R]{\thepage}
\renewcommand{\footrulewidth}{0.2pt}
\fancyfoot[C]{}

\newcolumntype{C}{>{\centering\arraybackslash}X}
\newcolumntype{s}{>{\hsize=.35\hsize\arraybackslash}X}

\setlength{\parindent}{0pt}		 
\setlength{\parskip}{3mm}
\setlength{\headheight}{1cm}

\def\ebook{ebook}
\def\book{book}
\def\web{web}
\def\type{web}

\newcommand{\vect}[1]{\overrightarrow{\,\mathstrut#1\,}}

\def\Oij{$\left(\text{O}~;~\vect{\imath},~\vect{\jmath}\right)$}
\def\Oijk{$\left(\text{O}~;~\vect{\imath},~\vect{\jmath},~\vect{k}\right)$}
\def\Ouv{$\left(\text{O}~;~\vect{u},~\vect{v}\right)$}

\hypersetup{breaklinks=true, colorlinks = true, linkcolor = OliveGreen, urlcolor = OliveGreen, citecolor = OliveGreen, pdfauthor={Didier BONNEL - https://www.maths-cours.fr} } % supprime les bordures autour des liens

\renewcommand{\arg}[0]{\text{arg}}

\everymath{\displaystyle}

%================================================================================================================================
%
% Macros - Commandes
%
%================================================================================================================================

\newcommand\meta[2]{    			% Utilisé pour créer le post HTML.
	\def\titre{titre}
	\def\url{url}
	\def\arg{#1}
	\ifx\titre\arg
		\newcommand\maintitle{#2}
		\fancyhead[L]{#2}
		{\Large\sffamily \MakeUppercase{#2}}
		\vspace{1mm}\textcolor{mcvert}{\hrule}
	\fi 
	\ifx\url\arg
		\fancyfoot[L]{\href{https://www.maths-cours.fr#2}{\black \footnotesize{https://www.maths-cours.fr#2}}}
	\fi 
}


\newcommand\TitreC[1]{    		% Titre centré
     \needspace{3\baselineskip}
     \begin{center}\textbf{#1}\end{center}
}

\newcommand\newpar{    		% paragraphe
     \par
}

\newcommand\nosp {    		% commande vide (pas d'espace)
}
\newcommand{\id}[1]{} %ignore

\newcommand\boite[2]{				% Boite simple sans titre
	\vspace{5mm}
	\setlength{\fboxrule}{0.2mm}
	\setlength{\fboxsep}{5mm}	
	\fcolorbox{#1}{#1!3}{\makebox[\linewidth-2\fboxrule-2\fboxsep]{
  		\begin{minipage}[t]{\linewidth-2\fboxrule-4\fboxsep}\setlength{\parskip}{3mm}
  			 #2
  		\end{minipage}
	}}
	\vspace{5mm}
}

\newcommand\CBox[4]{				% Boites
	\vspace{5mm}
	\setlength{\fboxrule}{0.2mm}
	\setlength{\fboxsep}{5mm}
	
	\fcolorbox{#1}{#1!3}{\makebox[\linewidth-2\fboxrule-2\fboxsep]{
		\begin{minipage}[t]{1cm}\setlength{\parskip}{3mm}
	  		\textcolor{#1}{\LARGE{#2}}    
 	 	\end{minipage}  
  		\begin{minipage}[t]{\linewidth-2\fboxrule-4\fboxsep}\setlength{\parskip}{3mm}
			\raisebox{1.2mm}{\normalsize\sffamily{\textcolor{#1}{#3}}}						
  			 #4
  		\end{minipage}
	}}
	\vspace{5mm}
}

\newcommand\cadre[3]{				% Boites convertible html
	\par
	\vspace{2mm}
	\setlength{\fboxrule}{0.1mm}
	\setlength{\fboxsep}{5mm}
	\fcolorbox{#1}{white}{\makebox[\linewidth-2\fboxrule-2\fboxsep]{
  		\begin{minipage}[t]{\linewidth-2\fboxrule-4\fboxsep}\setlength{\parskip}{3mm}
			\raisebox{-2.5mm}{\sffamily \small{\textcolor{#1}{\MakeUppercase{#2}}}}		
			\par		
  			 #3
 	 		\end{minipage}
	}}
		\vspace{2mm}
	\par
}

\newcommand\bloc[3]{				% Boites convertible html sans bordure
     \needspace{2\baselineskip}
     {\sffamily \small{\textcolor{#1}{\MakeUppercase{#2}}}}    
		\par		
  			 #3
		\par
}

\newcommand\CHelp[1]{
     \CBox{Plum}{\faInfoCircle}{À RETENIR}{#1}
}

\newcommand\CUp[1]{
     \CBox{NavyBlue}{\faThumbsOUp}{EN PRATIQUE}{#1}
}

\newcommand\CInfo[1]{
     \CBox{Sepia}{\faArrowCircleRight}{REMARQUE}{#1}
}

\newcommand\CRedac[1]{
     \CBox{PineGreen}{\faEdit}{BIEN R\'EDIGER}{#1}
}

\newcommand\CError[1]{
     \CBox{Red}{\faExclamationTriangle}{ATTENTION}{#1}
}

\newcommand\TitreExo[2]{
\needspace{4\baselineskip}
 {\sffamily\large EXERCICE #1\ (\emph{#2 points})}
\vspace{5mm}
}

\newcommand\img[2]{
          \includegraphics[width=#2\paperwidth]{\imgdir#1}
}

\newcommand\imgsvg[2]{
       \begin{center}   \includegraphics[width=#2\paperwidth]{\imgsvgdir#1} \end{center}
}


\newcommand\Lien[2]{
     \href{#1}{#2 \tiny \faExternalLink}
}
\newcommand\mcLien[2]{
     \href{https~://www.maths-cours.fr/#1}{#2 \tiny \faExternalLink}
}

\newcommand{\euro}{\eurologo{}}

%================================================================================================================================
%
% Macros - Environement
%
%================================================================================================================================

\newenvironment{tex}{ %
}
{%
}

\newenvironment{indente}{ %
	\setlength\parindent{10mm}
}

{
	\setlength\parindent{0mm}
}

\newenvironment{corrige}{%
     \needspace{3\baselineskip}
     \medskip
     \textbf{\textsc{Corrigé}}
     \medskip
}
{
}

\newenvironment{extern}{%
     \begin{center}
     }
     {
     \end{center}
}

\NewEnviron{code}{%
	\par
     \boite{gray}{\texttt{%
     \BODY
     }}
     \par
}

\newenvironment{vbloc}{% boite sans cadre empeche saut de page
     \begin{minipage}[t]{\linewidth}
     }
     {
     \end{minipage}
}
\NewEnviron{h2}{%
    \needspace{3\baselineskip}
    \vspace{0.6cm}
	\noindent \MakeUppercase{\sffamily \large \BODY}
	\vspace{1mm}\textcolor{mcgris}{\hrule}\vspace{0.4cm}
	\par
}{}

\NewEnviron{h3}{%
    \needspace{3\baselineskip}
	\vspace{5mm}
	\textsc{\BODY}
	\par
}

\NewEnviron{margeneg}{ %
\begin{addmargin}[-1cm]{0cm}
\BODY
\end{addmargin}
}

\NewEnviron{html}{%
}

\begin{document}
\meta{url}{/exercices/suites-bac-blanc-es-l-sujet-2-maths-cours-2018/}
\meta{pid}{10448}
\meta{titre}{Suites - Bac blanc ES/L Sujet 2 - Maths-cours 2018}
\meta{type}{exercices}
%
\begin{h2}Exercice 3 (4 points)\end{h2}
\par
Un cinéma de trois salles propose le choix entre les films \textbf{A}, \textbf{B} ou \textbf{C}. Suivant leur âge, les spectateurs payent leur place plein tarif ou bénéficient d'un tarif réduit.
\par
Le directeur de la salle a constaté que :
\par
\begin{itemize}
     \item 30\% des spectateurs bénéficient du tarif réduit (les 70\% restant payant plein tarif) ;
     \item 45\% des spectateurs payant plein tarif et 40\% des spectateurs bénéficiant du tarif réduit ont été voir le film \textbf{A} ;
     \item 30\% des spectateurs payant plein tarif et 37\% des spectateurs bénéficiant du tarif réduit ont été voir le film \textbf{B} ;
     \item 25\% des spectateurs payant plein tarif et 23\% des spectateurs bénéficiant du tarif réduit ont été voir le film \textbf{C}.
\end{itemize}
\par
On choisit au hasard un spectateur à la sortie du cinéma. On note :
\par
\begin{itemize}
     \item $R$ : l'événement \og le spectateur bénéficie du tarif réduit \fg{} ;
     \item $A$ : l'événement \og le spectateur a été voir le film \textbf{A} \fg{} ;
     \item $B$ : l'événement \og le spectateur a été voir le film \textbf{B} \fg{} ;
     \item $C$ : l'événement \og le spectateur a été voir le film \textbf{C} \fg{}.
\end{itemize}
\par
\begin{enumerate}
     \item %1
     Représenter la situation à l'aide d'un arbre pondéré.
     \item %2
     Montrer que la probabilité que le spectateur choisi vienne d'aller voir le film \textbf{A} est égale à $0,435$.
     \item %3
     On sait que le spectateur vient de voir le film \textbf{A}. Quelle est la probabilité qu'il bénéficie du tarif réduit ?
     \item %4
     On choisit maintenant au hasard et de façon indépendante, trois spectateurs. On suppose que ces choix peuvent être assimilés à des tirages successifs avec remise.
     \par
     On note $X$ la variable aléatoire correspondant au nombre de ces spectateurs qui viennent de voir le film \textbf{A}.
     \par
     \begin{enumerate}
          \item %4a
          Quelle est la loi de probabilité suivie par $X$ ? Préciser ses paramètres.
          \item %4b
          Calculer la probabilité $p(X \geqslant 1)$. Interpréter cette probabilité dans le cadre de l'énoncé.
          \par
     \end{enumerate}
     \par
\end{enumerate}
\begin{corrige}
     \begin{enumerate}
          \item %1
          La situation peut être modélisée par l'arbre pondéré ci-après :
          \par
          %:-+-+-+- Engendré par : http://math.et.info.free.fr/TikZ/Arbre/
          \begin{center}
               \begin{extern}%width="320" alt="Arbre pondéré de probabilité"
                    % Racine à Gauche, développement vers la droite
                    \begin{tikzpicture}[xscale=1,yscale=1]
                         % Styles (MODIFIABLES)
                         \tikzstyle{fleche}=[-,>=latex,thick]
                         \tikzstyle{noeud}=[fill=white,circle,draw]
                         \tikzstyle{feuille}=[fill=white,circle,draw]
                         \tikzstyle{etiquette}=[midway,fill=white]
                         % Dimensions (MODIFIABLES)
                         \def\DistanceInterNiveaux{2.5}
                         \def\DistanceInterFeuilles{1.2}
                         % Dimensions calculées (NON MODIFIABLES)
                         \def\NiveauA{(0)*\DistanceInterNiveaux}
                         \def\NiveauB{(1.5)*\DistanceInterNiveaux}
                         \def\NiveauC{(2.5)*\DistanceInterNiveaux}
                         \def\InterFeuilles{(-1)*\DistanceInterFeuilles}
                         % Noeuds (MODIFIABLES : Styles et Coefficients d'InterFeuilles)
                         \node[noeud] (R) at ({\NiveauA},{(2.5)*\InterFeuilles}) {$\ $};
                         \node[noeud] (Ra) at ({\NiveauB},{(1)*\InterFeuilles}) {$R$};
                         \node[feuille] (Raa) at ({\NiveauC},{(0)*\InterFeuilles}) {$A$};
                         \node[feuille] (Rab) at ({\NiveauC},{(1)*\InterFeuilles}) {$B$};
                         \node[feuille] (Rac) at ({\NiveauC},{(2)*\InterFeuilles}) {$C$};
                         \node[noeud] (Rb) at ({\NiveauB},{(4)*\InterFeuilles}) {$\overline{R}$};
                         \node[feuille] (Rba) at ({\NiveauC},{(3)*\InterFeuilles}) {$A$};
                         \node[feuille] (Rbb) at ({\NiveauC},{(4)*\InterFeuilles}) {$B$};
                         \node[feuille] (Rbc) at ({\NiveauC},{(5)*\InterFeuilles}) {$C$};
                         % Arcs (MODIFIABLES : Styles)
                         \draw[fleche] (R)--(Ra) node[etiquette] {$0,3$};
                         \draw[fleche] (Ra)--(Raa) node[etiquette] {$0,4$};
                         \draw[fleche] (Ra)--(Rab) node[etiquette] {$0,37$};
                         \draw[fleche] (Ra)--(Rac) node[etiquette] {$0,23$};
                         \draw[fleche] (R)--(Rb) node[etiquette] {$0,7$};
                         \draw[fleche] (Rb)--(Rba) node[etiquette] {$0,45$};
                         \draw[fleche] (Rb)--(Rbb) node[etiquette] {$0,3$};
                         \draw[fleche] (Rb)--(Rbc) node[etiquette] {$0,25$};
                    \end{tikzpicture}
               \end{extern}
          \end{center}
          \cadre{rouge}{À retenir}{
               Le total des probabilités figurant sur l'ensemble des branches partant d'un même nœud est toujours égal à 1.
          }
          \item %2
          La probabilité que le spectateur ait été voir le film \textbf{A} est $p(A)$.
          \par
          D'après la formule des probabilités totales :
          \par
          $p(A)=p(A\cap R)+p(A\cap \overline{R})$\\
          $\phantom{p(A)}=p(R) \times p_R(A)+ p({\overline{R}}) \times p_{\overline{R}}(A)$\\
          $\phantom{p(A)}=0,3 \times 0,4 + 0,7 \times 0,45 = 0,435.$
          \par
          \cadre{rouge}{À retenir}{
               \textbf{Formule des probabilités totales :}
               \par
               Si les événements $B_1, B_2, \cdots , B_n$ forment une partition de l'univers (c'est à dire regroupent toutes les éventualités) alors, pour tout événement $A$ :
               \begin{center}
                    $p(A)= p(A\cap B_1)+p(A\cap B_2)$\nosp$+\cdots+p(A\cap B_n).$
               \end{center}
               \par
               Un cas particulier très fréquent, dû au fait que $B$ et $\overline{B}$ forment une partition de l'univers, donne :
               \[ p(A)= p(A\cap B)+p(A\cap \overline{B}). \]
          }
          \item %3
          La probabilité demandée est $p_A(R)$.
          \cadre{vert}{En pratique}{
               Très souvent, en probabilités, la première étape consiste à traduire la probabilité cherchée en utilisant les notations de l'énoncé.
               \par
               Dans le cas présent, on sait que l'événement $A$ est vérifié et on souhaite déterminer la probabilité de l'événement $R$. On recherche donc $p_A(R)$.
          }
          \cadre{rouge}{Attention}{
               Ne pas confondre  :
               \begin{itemize}
                    \item
                    $p(A\cap R)$ : probabilité que $A$ \textbf{et} $R$ se réalisent (alors que l'on n'a, \textit{a priori}, aucune information concernant la réalisation de $A$ ou de $R$) ;
                    \item
                    $p_A(R)$ : probabilité que $R$ se réalise alors que l'\textbf{on sait que $A$ est réalisé}.
               \end{itemize}
          }
          \par
          D'après la formule des probabilités conditionnelles :
          \par
          $p_A(R)=\dfrac{p(A\cap R)}{p(A)}=\dfrac{0,3 \times 0,4}{0,435}$\nosp$=\dfrac{0,12}{0,435} \approx 0,276\ $ (à $10^{-3}$ près).
          \item %4
          \begin{enumerate}
               \item %4a
               La variable aléatoire $X$ suit une loi binomiale de paramètres ${n=3}$ et ${p=0,435}$.
               \par
               En effet :
               \par
               \begin{itemize}
                    \item on assimile l'expérience aux tirages successifs et avec remise de 3 spectateurs ;
                    \item pour chaque spectateur, deux issues sont possibles :
                    \begin{itemize}
                         \item \textit{succès} : le spectateur vient d'aller voir le film \textbf{A} (probabilité $p=0,435$) ;
                         \item \textit{échec} : le spectateur ne vient pas d'aller voir le film \textbf{A}.
                    \end{itemize}
                    \item la variable aléatoire $X$ comptabilise le nombre de succès.
               \end{itemize}
               \item
               L'événement contraire de $(X \geqslant 1)$ est $(X<1)$ c'est à dire $(X=0)$.
               \cadre{rouge}{Attention}{
                    L'événement contraire de ($X \geqslant a$) est ($X < a$) et non ($X \leqslant a$).
               }
               Comme $X$ suit une loi binomiale :
               \par
               $p(X=0)=\begin{pmatrix} 3 \\ 0 \end{pmatrix} \times 0,435^0 \times 0,565^{3}$\nosp$ = 0,565^{3}$.
               \par
               Par conséquent :
               \par
               $p(X \geqslant 1)=1-p(X=0)$\nosp$=1-0,565^{3} \approx 0,820\ $ (à $10^{-3}$ près).
               \par
          \end{enumerate}
          \par
     \end{enumerate}
\end{corrige}

\end{document}