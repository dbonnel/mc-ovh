\documentclass[a4paper]{article}

%================================================================================================================================
%
% Packages
%
%================================================================================================================================

\usepackage[T1]{fontenc} 	% pour caractères accentués
\usepackage[utf8]{inputenc}  % encodage utf8
\usepackage[french]{babel}	% langue : français
\usepackage{fourier}			% caractères plus lisibles
\usepackage[dvipsnames]{xcolor} % couleurs
\usepackage{fancyhdr}		% réglage header footer
\usepackage{needspace}		% empêcher sauts de page mal placés
\usepackage{graphicx}		% pour inclure des graphiques
\usepackage{enumitem,cprotect}		% personnalise les listes d'items (nécessaire pour ol, al ...)
\usepackage{hyperref}		% Liens hypertexte
\usepackage{pstricks,pst-all,pst-node,pstricks-add,pst-math,pst-plot,pst-tree,pst-eucl} % pstricks
\usepackage[a4paper,includeheadfoot,top=2cm,left=3cm, bottom=2cm,right=3cm]{geometry} % marges etc.
\usepackage{comment}			% commentaires multilignes
\usepackage{amsmath,environ} % maths (matrices, etc.)
\usepackage{amssymb,makeidx}
\usepackage{bm}				% bold maths
\usepackage{tabularx}		% tableaux
\usepackage{colortbl}		% tableaux en couleur
\usepackage{fontawesome}		% Fontawesome
\usepackage{environ}			% environment with command
\usepackage{fp}				% calculs pour ps-tricks
\usepackage{multido}			% pour ps tricks
\usepackage[np]{numprint}	% formattage nombre
\usepackage{tikz,tkz-tab} 			% package principal TikZ
\usepackage{pgfplots}   % axes
\usepackage{mathrsfs}    % cursives
\usepackage{calc}			% calcul taille boites
\usepackage[scaled=0.875]{helvet} % font sans serif
\usepackage{svg} % svg
\usepackage{scrextend} % local margin
\usepackage{scratch} %scratch
\usepackage{multicol} % colonnes
%\usepackage{infix-RPN,pst-func} % formule en notation polanaise inversée
\usepackage{listings}

%================================================================================================================================
%
% Réglages de base
%
%================================================================================================================================

\lstset{
language=Python,   % R code
literate=
{á}{{\'a}}1
{à}{{\`a}}1
{ã}{{\~a}}1
{é}{{\'e}}1
{è}{{\`e}}1
{ê}{{\^e}}1
{í}{{\'i}}1
{ó}{{\'o}}1
{õ}{{\~o}}1
{ú}{{\'u}}1
{ü}{{\"u}}1
{ç}{{\c{c}}}1
{~}{{ }}1
}


\definecolor{codegreen}{rgb}{0,0.6,0}
\definecolor{codegray}{rgb}{0.5,0.5,0.5}
\definecolor{codepurple}{rgb}{0.58,0,0.82}
\definecolor{backcolour}{rgb}{0.95,0.95,0.92}

\lstdefinestyle{mystyle}{
    backgroundcolor=\color{backcolour},   
    commentstyle=\color{codegreen},
    keywordstyle=\color{magenta},
    numberstyle=\tiny\color{codegray},
    stringstyle=\color{codepurple},
    basicstyle=\ttfamily\footnotesize,
    breakatwhitespace=false,         
    breaklines=true,                 
    captionpos=b,                    
    keepspaces=true,                 
    numbers=left,                    
xleftmargin=2em,
framexleftmargin=2em,            
    showspaces=false,                
    showstringspaces=false,
    showtabs=false,                  
    tabsize=2,
    upquote=true
}

\lstset{style=mystyle}


\lstset{style=mystyle}
\newcommand{\imgdir}{C:/laragon/www/newmc/assets/imgsvg/}
\newcommand{\imgsvgdir}{C:/laragon/www/newmc/assets/imgsvg/}

\definecolor{mcgris}{RGB}{220, 220, 220}% ancien~; pour compatibilité
\definecolor{mcbleu}{RGB}{52, 152, 219}
\definecolor{mcvert}{RGB}{125, 194, 70}
\definecolor{mcmauve}{RGB}{154, 0, 215}
\definecolor{mcorange}{RGB}{255, 96, 0}
\definecolor{mcturquoise}{RGB}{0, 153, 153}
\definecolor{mcrouge}{RGB}{255, 0, 0}
\definecolor{mclightvert}{RGB}{205, 234, 190}

\definecolor{gris}{RGB}{220, 220, 220}
\definecolor{bleu}{RGB}{52, 152, 219}
\definecolor{vert}{RGB}{125, 194, 70}
\definecolor{mauve}{RGB}{154, 0, 215}
\definecolor{orange}{RGB}{255, 96, 0}
\definecolor{turquoise}{RGB}{0, 153, 153}
\definecolor{rouge}{RGB}{255, 0, 0}
\definecolor{lightvert}{RGB}{205, 234, 190}
\setitemize[0]{label=\color{lightvert}  $\bullet$}

\pagestyle{fancy}
\renewcommand{\headrulewidth}{0.2pt}
\fancyhead[L]{maths-cours.fr}
\fancyhead[R]{\thepage}
\renewcommand{\footrulewidth}{0.2pt}
\fancyfoot[C]{}

\newcolumntype{C}{>{\centering\arraybackslash}X}
\newcolumntype{s}{>{\hsize=.35\hsize\arraybackslash}X}

\setlength{\parindent}{0pt}		 
\setlength{\parskip}{3mm}
\setlength{\headheight}{1cm}

\def\ebook{ebook}
\def\book{book}
\def\web{web}
\def\type{web}

\newcommand{\vect}[1]{\overrightarrow{\,\mathstrut#1\,}}

\def\Oij{$\left(\text{O}~;~\vect{\imath},~\vect{\jmath}\right)$}
\def\Oijk{$\left(\text{O}~;~\vect{\imath},~\vect{\jmath},~\vect{k}\right)$}
\def\Ouv{$\left(\text{O}~;~\vect{u},~\vect{v}\right)$}

\hypersetup{breaklinks=true, colorlinks = true, linkcolor = OliveGreen, urlcolor = OliveGreen, citecolor = OliveGreen, pdfauthor={Didier BONNEL - https://www.maths-cours.fr} } % supprime les bordures autour des liens

\renewcommand{\arg}[0]{\text{arg}}

\everymath{\displaystyle}

%================================================================================================================================
%
% Macros - Commandes
%
%================================================================================================================================

\newcommand\meta[2]{    			% Utilisé pour créer le post HTML.
	\def\titre{titre}
	\def\url{url}
	\def\arg{#1}
	\ifx\titre\arg
		\newcommand\maintitle{#2}
		\fancyhead[L]{#2}
		{\Large\sffamily \MakeUppercase{#2}}
		\vspace{1mm}\textcolor{mcvert}{\hrule}
	\fi 
	\ifx\url\arg
		\fancyfoot[L]{\href{https://www.maths-cours.fr#2}{\black \footnotesize{https://www.maths-cours.fr#2}}}
	\fi 
}


\newcommand\TitreC[1]{    		% Titre centré
     \needspace{3\baselineskip}
     \begin{center}\textbf{#1}\end{center}
}

\newcommand\newpar{    		% paragraphe
     \par
}

\newcommand\nosp {    		% commande vide (pas d'espace)
}
\newcommand{\id}[1]{} %ignore

\newcommand\boite[2]{				% Boite simple sans titre
	\vspace{5mm}
	\setlength{\fboxrule}{0.2mm}
	\setlength{\fboxsep}{5mm}	
	\fcolorbox{#1}{#1!3}{\makebox[\linewidth-2\fboxrule-2\fboxsep]{
  		\begin{minipage}[t]{\linewidth-2\fboxrule-4\fboxsep}\setlength{\parskip}{3mm}
  			 #2
  		\end{minipage}
	}}
	\vspace{5mm}
}

\newcommand\CBox[4]{				% Boites
	\vspace{5mm}
	\setlength{\fboxrule}{0.2mm}
	\setlength{\fboxsep}{5mm}
	
	\fcolorbox{#1}{#1!3}{\makebox[\linewidth-2\fboxrule-2\fboxsep]{
		\begin{minipage}[t]{1cm}\setlength{\parskip}{3mm}
	  		\textcolor{#1}{\LARGE{#2}}    
 	 	\end{minipage}  
  		\begin{minipage}[t]{\linewidth-2\fboxrule-4\fboxsep}\setlength{\parskip}{3mm}
			\raisebox{1.2mm}{\normalsize\sffamily{\textcolor{#1}{#3}}}						
  			 #4
  		\end{minipage}
	}}
	\vspace{5mm}
}

\newcommand\cadre[3]{				% Boites convertible html
	\par
	\vspace{2mm}
	\setlength{\fboxrule}{0.1mm}
	\setlength{\fboxsep}{5mm}
	\fcolorbox{#1}{white}{\makebox[\linewidth-2\fboxrule-2\fboxsep]{
  		\begin{minipage}[t]{\linewidth-2\fboxrule-4\fboxsep}\setlength{\parskip}{3mm}
			\raisebox{-2.5mm}{\sffamily \small{\textcolor{#1}{\MakeUppercase{#2}}}}		
			\par		
  			 #3
 	 		\end{minipage}
	}}
		\vspace{2mm}
	\par
}

\newcommand\bloc[3]{				% Boites convertible html sans bordure
     \needspace{2\baselineskip}
     {\sffamily \small{\textcolor{#1}{\MakeUppercase{#2}}}}    
		\par		
  			 #3
		\par
}

\newcommand\CHelp[1]{
     \CBox{Plum}{\faInfoCircle}{À RETENIR}{#1}
}

\newcommand\CUp[1]{
     \CBox{NavyBlue}{\faThumbsOUp}{EN PRATIQUE}{#1}
}

\newcommand\CInfo[1]{
     \CBox{Sepia}{\faArrowCircleRight}{REMARQUE}{#1}
}

\newcommand\CRedac[1]{
     \CBox{PineGreen}{\faEdit}{BIEN R\'EDIGER}{#1}
}

\newcommand\CError[1]{
     \CBox{Red}{\faExclamationTriangle}{ATTENTION}{#1}
}

\newcommand\TitreExo[2]{
\needspace{4\baselineskip}
 {\sffamily\large EXERCICE #1\ (\emph{#2 points})}
\vspace{5mm}
}

\newcommand\img[2]{
          \includegraphics[width=#2\paperwidth]{\imgdir#1}
}

\newcommand\imgsvg[2]{
       \begin{center}   \includegraphics[width=#2\paperwidth]{\imgsvgdir#1} \end{center}
}


\newcommand\Lien[2]{
     \href{#1}{#2 \tiny \faExternalLink}
}
\newcommand\mcLien[2]{
     \href{https~://www.maths-cours.fr/#1}{#2 \tiny \faExternalLink}
}

\newcommand{\euro}{\eurologo{}}

%================================================================================================================================
%
% Macros - Environement
%
%================================================================================================================================

\newenvironment{tex}{ %
}
{%
}

\newenvironment{indente}{ %
	\setlength\parindent{10mm}
}

{
	\setlength\parindent{0mm}
}

\newenvironment{corrige}{%
     \needspace{3\baselineskip}
     \medskip
     \textbf{\textsc{Corrigé}}
     \medskip
}
{
}

\newenvironment{extern}{%
     \begin{center}
     }
     {
     \end{center}
}

\NewEnviron{code}{%
	\par
     \boite{gray}{\texttt{%
     \BODY
     }}
     \par
}

\newenvironment{vbloc}{% boite sans cadre empeche saut de page
     \begin{minipage}[t]{\linewidth}
     }
     {
     \end{minipage}
}
\NewEnviron{h2}{%
    \needspace{3\baselineskip}
    \vspace{0.6cm}
	\noindent \MakeUppercase{\sffamily \large \BODY}
	\vspace{1mm}\textcolor{mcgris}{\hrule}\vspace{0.4cm}
	\par
}{}

\NewEnviron{h3}{%
    \needspace{3\baselineskip}
	\vspace{5mm}
	\textsc{\BODY}
	\par
}

\NewEnviron{margeneg}{ %
\begin{addmargin}[-1cm]{0cm}
\BODY
\end{addmargin}
}

\NewEnviron{html}{%
}

\begin{document}
\meta{url}{/exercices/fonctions-bac-s-metropole-2014/}
\meta{pid}{2705}
\meta{titre}{Fonctions Intégrales - Bac S  Métropole 2014}
\meta{type}{exercices}
%
\begin{h2}Exercice 1 (5 points)\end{h2}
\textit{Commun à tous les candidats}
\begin{h3}Partie A\end{h3}
Dans le plan muni d'un repère orthonormé, on désigne par  $\mathscr C_{1}$ la courbe représentative de la fonction $f_{1}$  définie sur $\mathbb{R}$ par :
\begin{center}$f_{1}\left(x\right)=x+ e^{-x}.$\end{center}
\begin{enumerate}
     \item
     Justifier que $\mathscr C_{1}$ passe par le point $A$ de coordonnées $\left(0 ; 1\right)$.
     \item
     Déterminer le tableau de variation de la fonction $f_{1}$. On précisera les limites de $f_{1}$ en $+ \infty $ et en $-\infty $.
\end{enumerate}
\begin{h3}Partie B\end{h3}
L'objet de cette partie est d'étudier la suite $\left(I_{n}\right)$ définie sur $\mathbb{N}$ par :
\begin{center}$I_{n}=\int_{0}^{1}\left(x+e^{-nx}\right) dx.$\end{center}
\begin{enumerate}
     \item
     Dans le plan muni d'un repère orthonormé $\left(O; \vec{i}, \vec{j}\right)$ , pour tout entier naturel $n$, on note
     \par
     $\mathscr C_{n}$ la courbe représentative de la fonction $f_{n} $ définie sur $\mathbb{R}$ par
     \begin{center}$f_{n}\left(x\right)=x+e^{-nx}. $\end{center}
     Sur le graphique ci-dessous on a tracé la courbe  $\mathscr C_{n}$ pour plusieurs valeurs de l'entier $n$ et la droite $\mathscr D$ d'équation $x=1$.

\begin{center}
\imgsvg{mc-0297}{0.3}% alt="Fonctions Intégrales - Bac S  Métropole 2014" style="width:50rem"
\end{center}

     \begin{enumerate}[label=\alph*.]
          \item
          Interpréter géométriquement l'intégrale $I_{n}$.
          \item
     En utilisant cette interprétation, formuler une conjecture sur le sens de   variation de la suite $\left(I_{n}\right)$ et sa limite éventuelle. On précisera les éléments sur lesquels on s'appuie pour conjecturer.\end{enumerate}
     \item
     Démontrer que pour tout entier naturel $n$ supérieur ou égal à 1,
     \begin{center}$I_{n+1}-I_{n}=\int_{0}^{1}e^{-\left(n+1\right)x} \left(1-e^{x}\right)dx.$\end{center}
     En déduire le signe de $I_{n+1}-I_{n}$ puis démontrer que la suite $\left(I_{n}\right)$ est convergente.
     \item
     Déterminer l'expression de $I_{n}$ en fonction de $n$ et déterminer la limite de la suite $\left(I_{n}\right)$.
\end{enumerate}
\begin{corrige}
     \begin{h3}Partie A\end{h3}
     \begin{enumerate}
          \item
          $f_{1}\left(0\right)=0+e^{0}=1$
          \par
          Donc la courbe $\mathscr C_{1}$ passe par le point $A$ de coordonnées $\left(0 ; 1\right)$.
          \item
          $f_{1}$ est définie et dérivable sur $\mathbb{R}$ et :
          \par
          $f_{1}^{\prime}\left(x\right)=1-e^{-x}$
          \par
          $f_{1}^{\prime}\left(x\right) \geqslant  0  \Leftrightarrow   1-e^{-x} \geqslant 0  \Leftrightarrow  e^{-x}\leqslant 1$
          \par
          Du fait de la stricte croissance de la fonction exponentielle :
          \par
          $f_{1}^{\prime}\left(x\right) \geqslant  0  \Leftrightarrow   -x \leqslant  0   \Leftrightarrow  x\geqslant 0$
          \par
          $\lim\limits_{x\rightarrow +\infty }x= + \infty $ et $\lim\limits_{x\rightarrow +\infty }e^{-x}= 0$
          \par
          Donc, par somme : $\lim\limits_{x\rightarrow +\infty }f_{1}\left(x\right)= +\infty $
          \par
          Pour la limite en $-\infty $ on a une forme indéterminée dy type « $+\infty -\infty  $». On peut lever cette indétermination en mettant $x$ en facteur :
          \par
          $f_{1}\left(x\right)=x\left(1+\frac{e^{-x}}{x}\right)$
          \par
          En posant $X=-x$ on voit que :
          \par
          $\lim\limits_{x\rightarrow -\infty }\frac{e^{-x}}{x}= \lim\limits_{X\rightarrow +\infty }-\frac{e^{X}}{X} =-\infty $
          \par
          Donc $\lim\limits_{x\rightarrow -\infty }\left(1+\frac{e^{-x}}{x}\right)= -\infty $ et par produit :
          \par
          $\lim\limits_{x\rightarrow -\infty }x\left(1+\frac{e^{-x}}{x}\right)= +\infty $
          \par
          On obtient le tableau de variations suivant :
%##
% type=table; width=35; l3=20
%--
% x|   -\infty   ~    \frac{ 3 }{ 4 }   ~   +\infty 
% f'(x)|  ~       -               :0             +   ~
% f(x)|  +\infty    \          :0              /   +\infty
%--
\begin{center}
 \begin{extern}%style="width:35rem" alt="Exercice"
    \resizebox{11cm}{!}{
       \definecolor{dark}{gray}{0.1}
       \definecolor{light}{gray}{0.8}
       \tikzstyle{fleche}=[->,>=latex]
       \begin{tikzpicture}[scale=.8, line width=.5pt, dark]
       \def\width{.15}
       \def\height{.10}
       \draw (0, -10*\height) -- (54*\width, -10*\height);
       \draw (10*\width, 0*\height) -- (10*\width, -10*\height);
       \node (l0c0) at (5*\width,-5*\height) {$ x $};
       \node (l0c1) at (14*\width,-5*\height) {$ -\infty $};
       \node (l0c2) at (23*\width,-5*\height) {$ ~ $};
       \node (l0c3) at (32*\width,-5*\height) {$ \frac{ 3 }{ 4 } $};
       \node (l0c4) at (41*\width,-5*\height) {$ ~ $};
       \node (l0c5) at (50*\width,-5*\height) {$ +\infty $};
       \draw (0, -20*\height) -- (54*\width, -20*\height);
       \draw (10*\width, -10*\height) -- (10*\width, -20*\height);
       \node (l1c0) at (5*\width,-15*\height) {$ f'(x) $};
       \node (l1c1) at (14*\width,-15*\height) {$ ~ $};
       \node (l1c2) at (23*\width,-15*\height) {$ - $};
       \draw[light] (32*\width, -10*\height) -- (32*\width, -20*\height);
       \node (l1c3) at (32*\width,-15*\height) {$ 0 $};
       \node (l1c4) at (41*\width,-15*\height) {$ + $};
       \node (l1c5) at (50*\width,-15*\height) {$ ~ $};
       \draw (0, -40*\height) -- (54*\width, -40*\height);
       \draw (10*\width, -20*\height) -- (10*\width, -40*\height);
       \node (l2c0) at (5*\width,-30*\height) {$ f(x) $};
       \node (l2c1) at (14*\width,-25*\height) {$ +\infty $};
       \node (l2c2) at (23*\width,-30*\height) {$ ~ $};
       \draw[light] (32*\width, -20*\height) -- (32*\width, -40*\height);
       \node (l2c3) at (32*\width,-35*\height) {$ 0 $};
       \node (l2c4) at (41*\width,-30*\height) {$ ~ $};
       \node (l2c5) at (50*\width,-25*\height) {$ +\infty $};
       \draw (0, 0) rectangle (54*\width, -40*\height);
       \draw[fleche] (l2c1) -- (l2c3);
       \draw[fleche] (l2c3) -- (l2c5);
       \end{tikzpicture}
      }
   \end{extern}
\end{center}
%##
<img src="/wp-content/uploads/mc-0298.png" alt="" class="aligncenter size-full  img-pc" />
     \end{enumerate}
     \begin{h3}Partie B\end{h3}
     \begin{enumerate}
          \item
          \begin{enumerate}[label=\alph*.]
               \item
               Sur $\left[0;1\right]$ les fonctions $f_{n}$ sont strictement positives puisque $x \geqslant 0$ et $e^{-nx} > 0$
               \par
               L'intégrale $I_{n}$ représente donc l'aire du plan délimité par la courbe $\mathscr C_{n}$, l'axe des abscisses et les droites d'équations $x=0$ et $x=1$.
               \item
               D'après la figure, il semble que la suite $I_{n}$ soit décroissante et tende vers $\frac{1}{2}$.
               \par
               En effet, sur $\left[0;1\right]$ les courbes $\mathscr C_{n}$ semble se rapprocher de la droite d'équation $y=x$;
               \par
               l'aire comprise entre cette droite, l'axe des abscisses et les droites d'équations $x=0$ et $x=1$ vaut $\frac{1}{2}$ (triangle rectangle isocèle dont les côtés mesurent 1 unité).
          \end{enumerate}
          \item
          $I_{n+1}-I_{n}=\int_{0}^{1}x+e^{-\left(n+1\right)x}dx-\int_{0}^{1}x+e^{-nx}dx$.
          \par
          D'après la propriété de linéarité de l'intégrale :
          \par
          $I_{n+1}-I_{n}=\int_{0}^{1}x+e^{-\left(n+1\right)x}-\left(x+e^{-nx}\right)dx=\int_{0}^{1}e^{-\left(n+1\right)x}-e^{-nx}dx $
          \par
          $I_{n+1}-I_{n}= \int_{0}^{1}e^{-\left(n+1\right)x} \left(1-e^{x}\right)dx  $
          \par
          Or, sur $\left[0;1\right]$,  $e^{x} \geqslant  1$ donc $1-e^{x}\leqslant 0$ et $e^{-\left(n+1\right)x} > 0$ donc d'après la propriété de positivité de l'intégrale : $I_{n+1}-I_{n} \leqslant  0$.
          \par
          La suite $\left(I_{n}\right)$ est donc décroissante. Comme elle est minorée par zéro elle est convergente.
          \item
          Une primitive de $x\mapsto x$ sur $\mathbb{R}$ est $x\mapsto \frac{x^{2}}{2}$
          \par
          Une primitive de $x\mapsto e^{-nx}$ sur $\mathbb{R}$ est $x\mapsto -\frac{e^{-nx}}{n}$
          \par
          Par conséquent :
          \par
          $I_{n}=\int_{0}^{1}\left(x+e^{-nx}\right) dx = \left[\frac{x^{2}}{2}-\frac{e^{-nx}}{n}\right]_{0}^{1} = \frac{1}{2}-\frac{e^{-n}}{n}+\frac{1}{n}$
          \par
          Comme $\lim\limits_{n\rightarrow +\infty }\frac{e^{-n}}{n}=0$ (par quotient) et  $\lim\limits_{n\rightarrow +\infty }\frac{1}{n}=0$ on a donc :
          \begin{center}$\lim\limits_{n\rightarrow +\infty }I_{n}=\frac{1}{2}$\end{center}
     \end{enumerate}
\end{corrige}

\end{document}