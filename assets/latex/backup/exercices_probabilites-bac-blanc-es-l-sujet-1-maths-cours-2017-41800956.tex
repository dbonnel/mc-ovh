\documentclass[a4paper]{article}

%================================================================================================================================
%
% Packages
%
%================================================================================================================================

\usepackage[T1]{fontenc} 	% pour caractères accentués
\usepackage[utf8]{inputenc}  % encodage utf8
\usepackage[french]{babel}	% langue : français
\usepackage{fourier}			% caractères plus lisibles
\usepackage[dvipsnames]{xcolor} % couleurs
\usepackage{fancyhdr}		% réglage header footer
\usepackage{needspace}		% empêcher sauts de page mal placés
\usepackage{graphicx}		% pour inclure des graphiques
\usepackage{enumitem,cprotect}		% personnalise les listes d'items (nécessaire pour ol, al ...)
\usepackage{hyperref}		% Liens hypertexte
\usepackage{pstricks,pst-all,pst-node,pstricks-add,pst-math,pst-plot,pst-tree,pst-eucl} % pstricks
\usepackage[a4paper,includeheadfoot,top=2cm,left=3cm, bottom=2cm,right=3cm]{geometry} % marges etc.
\usepackage{comment}			% commentaires multilignes
\usepackage{amsmath,environ} % maths (matrices, etc.)
\usepackage{amssymb,makeidx}
\usepackage{bm}				% bold maths
\usepackage{tabularx}		% tableaux
\usepackage{colortbl}		% tableaux en couleur
\usepackage{fontawesome}		% Fontawesome
\usepackage{environ}			% environment with command
\usepackage{fp}				% calculs pour ps-tricks
\usepackage{multido}			% pour ps tricks
\usepackage[np]{numprint}	% formattage nombre
\usepackage{tikz,tkz-tab} 			% package principal TikZ
\usepackage{pgfplots}   % axes
\usepackage{mathrsfs}    % cursives
\usepackage{calc}			% calcul taille boites
\usepackage[scaled=0.875]{helvet} % font sans serif
\usepackage{svg} % svg
\usepackage{scrextend} % local margin
\usepackage{scratch} %scratch
\usepackage{multicol} % colonnes
%\usepackage{infix-RPN,pst-func} % formule en notation polanaise inversée
\usepackage{listings}

%================================================================================================================================
%
% Réglages de base
%
%================================================================================================================================

\lstset{
language=Python,   % R code
literate=
{á}{{\'a}}1
{à}{{\`a}}1
{ã}{{\~a}}1
{é}{{\'e}}1
{è}{{\`e}}1
{ê}{{\^e}}1
{í}{{\'i}}1
{ó}{{\'o}}1
{õ}{{\~o}}1
{ú}{{\'u}}1
{ü}{{\"u}}1
{ç}{{\c{c}}}1
{~}{{ }}1
}


\definecolor{codegreen}{rgb}{0,0.6,0}
\definecolor{codegray}{rgb}{0.5,0.5,0.5}
\definecolor{codepurple}{rgb}{0.58,0,0.82}
\definecolor{backcolour}{rgb}{0.95,0.95,0.92}

\lstdefinestyle{mystyle}{
    backgroundcolor=\color{backcolour},   
    commentstyle=\color{codegreen},
    keywordstyle=\color{magenta},
    numberstyle=\tiny\color{codegray},
    stringstyle=\color{codepurple},
    basicstyle=\ttfamily\footnotesize,
    breakatwhitespace=false,         
    breaklines=true,                 
    captionpos=b,                    
    keepspaces=true,                 
    numbers=left,                    
xleftmargin=2em,
framexleftmargin=2em,            
    showspaces=false,                
    showstringspaces=false,
    showtabs=false,                  
    tabsize=2,
    upquote=true
}

\lstset{style=mystyle}


\lstset{style=mystyle}
\newcommand{\imgdir}{C:/laragon/www/newmc/assets/imgsvg/}
\newcommand{\imgsvgdir}{C:/laragon/www/newmc/assets/imgsvg/}

\definecolor{mcgris}{RGB}{220, 220, 220}% ancien~; pour compatibilité
\definecolor{mcbleu}{RGB}{52, 152, 219}
\definecolor{mcvert}{RGB}{125, 194, 70}
\definecolor{mcmauve}{RGB}{154, 0, 215}
\definecolor{mcorange}{RGB}{255, 96, 0}
\definecolor{mcturquoise}{RGB}{0, 153, 153}
\definecolor{mcrouge}{RGB}{255, 0, 0}
\definecolor{mclightvert}{RGB}{205, 234, 190}

\definecolor{gris}{RGB}{220, 220, 220}
\definecolor{bleu}{RGB}{52, 152, 219}
\definecolor{vert}{RGB}{125, 194, 70}
\definecolor{mauve}{RGB}{154, 0, 215}
\definecolor{orange}{RGB}{255, 96, 0}
\definecolor{turquoise}{RGB}{0, 153, 153}
\definecolor{rouge}{RGB}{255, 0, 0}
\definecolor{lightvert}{RGB}{205, 234, 190}
\setitemize[0]{label=\color{lightvert}  $\bullet$}

\pagestyle{fancy}
\renewcommand{\headrulewidth}{0.2pt}
\fancyhead[L]{maths-cours.fr}
\fancyhead[R]{\thepage}
\renewcommand{\footrulewidth}{0.2pt}
\fancyfoot[C]{}

\newcolumntype{C}{>{\centering\arraybackslash}X}
\newcolumntype{s}{>{\hsize=.35\hsize\arraybackslash}X}

\setlength{\parindent}{0pt}		 
\setlength{\parskip}{3mm}
\setlength{\headheight}{1cm}

\def\ebook{ebook}
\def\book{book}
\def\web{web}
\def\type{web}

\newcommand{\vect}[1]{\overrightarrow{\,\mathstrut#1\,}}

\def\Oij{$\left(\text{O}~;~\vect{\imath},~\vect{\jmath}\right)$}
\def\Oijk{$\left(\text{O}~;~\vect{\imath},~\vect{\jmath},~\vect{k}\right)$}
\def\Ouv{$\left(\text{O}~;~\vect{u},~\vect{v}\right)$}

\hypersetup{breaklinks=true, colorlinks = true, linkcolor = OliveGreen, urlcolor = OliveGreen, citecolor = OliveGreen, pdfauthor={Didier BONNEL - https://www.maths-cours.fr} } % supprime les bordures autour des liens

\renewcommand{\arg}[0]{\text{arg}}

\everymath{\displaystyle}

%================================================================================================================================
%
% Macros - Commandes
%
%================================================================================================================================

\newcommand\meta[2]{    			% Utilisé pour créer le post HTML.
	\def\titre{titre}
	\def\url{url}
	\def\arg{#1}
	\ifx\titre\arg
		\newcommand\maintitle{#2}
		\fancyhead[L]{#2}
		{\Large\sffamily \MakeUppercase{#2}}
		\vspace{1mm}\textcolor{mcvert}{\hrule}
	\fi 
	\ifx\url\arg
		\fancyfoot[L]{\href{https://www.maths-cours.fr#2}{\black \footnotesize{https://www.maths-cours.fr#2}}}
	\fi 
}


\newcommand\TitreC[1]{    		% Titre centré
     \needspace{3\baselineskip}
     \begin{center}\textbf{#1}\end{center}
}

\newcommand\newpar{    		% paragraphe
     \par
}

\newcommand\nosp {    		% commande vide (pas d'espace)
}
\newcommand{\id}[1]{} %ignore

\newcommand\boite[2]{				% Boite simple sans titre
	\vspace{5mm}
	\setlength{\fboxrule}{0.2mm}
	\setlength{\fboxsep}{5mm}	
	\fcolorbox{#1}{#1!3}{\makebox[\linewidth-2\fboxrule-2\fboxsep]{
  		\begin{minipage}[t]{\linewidth-2\fboxrule-4\fboxsep}\setlength{\parskip}{3mm}
  			 #2
  		\end{minipage}
	}}
	\vspace{5mm}
}

\newcommand\CBox[4]{				% Boites
	\vspace{5mm}
	\setlength{\fboxrule}{0.2mm}
	\setlength{\fboxsep}{5mm}
	
	\fcolorbox{#1}{#1!3}{\makebox[\linewidth-2\fboxrule-2\fboxsep]{
		\begin{minipage}[t]{1cm}\setlength{\parskip}{3mm}
	  		\textcolor{#1}{\LARGE{#2}}    
 	 	\end{minipage}  
  		\begin{minipage}[t]{\linewidth-2\fboxrule-4\fboxsep}\setlength{\parskip}{3mm}
			\raisebox{1.2mm}{\normalsize\sffamily{\textcolor{#1}{#3}}}						
  			 #4
  		\end{minipage}
	}}
	\vspace{5mm}
}

\newcommand\cadre[3]{				% Boites convertible html
	\par
	\vspace{2mm}
	\setlength{\fboxrule}{0.1mm}
	\setlength{\fboxsep}{5mm}
	\fcolorbox{#1}{white}{\makebox[\linewidth-2\fboxrule-2\fboxsep]{
  		\begin{minipage}[t]{\linewidth-2\fboxrule-4\fboxsep}\setlength{\parskip}{3mm}
			\raisebox{-2.5mm}{\sffamily \small{\textcolor{#1}{\MakeUppercase{#2}}}}		
			\par		
  			 #3
 	 		\end{minipage}
	}}
		\vspace{2mm}
	\par
}

\newcommand\bloc[3]{				% Boites convertible html sans bordure
     \needspace{2\baselineskip}
     {\sffamily \small{\textcolor{#1}{\MakeUppercase{#2}}}}    
		\par		
  			 #3
		\par
}

\newcommand\CHelp[1]{
     \CBox{Plum}{\faInfoCircle}{À RETENIR}{#1}
}

\newcommand\CUp[1]{
     \CBox{NavyBlue}{\faThumbsOUp}{EN PRATIQUE}{#1}
}

\newcommand\CInfo[1]{
     \CBox{Sepia}{\faArrowCircleRight}{REMARQUE}{#1}
}

\newcommand\CRedac[1]{
     \CBox{PineGreen}{\faEdit}{BIEN R\'EDIGER}{#1}
}

\newcommand\CError[1]{
     \CBox{Red}{\faExclamationTriangle}{ATTENTION}{#1}
}

\newcommand\TitreExo[2]{
\needspace{4\baselineskip}
 {\sffamily\large EXERCICE #1\ (\emph{#2 points})}
\vspace{5mm}
}

\newcommand\img[2]{
          \includegraphics[width=#2\paperwidth]{\imgdir#1}
}

\newcommand\imgsvg[2]{
       \begin{center}   \includegraphics[width=#2\paperwidth]{\imgsvgdir#1} \end{center}
}


\newcommand\Lien[2]{
     \href{#1}{#2 \tiny \faExternalLink}
}
\newcommand\mcLien[2]{
     \href{https~://www.maths-cours.fr/#1}{#2 \tiny \faExternalLink}
}

\newcommand{\euro}{\eurologo{}}

%================================================================================================================================
%
% Macros - Environement
%
%================================================================================================================================

\newenvironment{tex}{ %
}
{%
}

\newenvironment{indente}{ %
	\setlength\parindent{10mm}
}

{
	\setlength\parindent{0mm}
}

\newenvironment{corrige}{%
     \needspace{3\baselineskip}
     \medskip
     \textbf{\textsc{Corrigé}}
     \medskip
}
{
}

\newenvironment{extern}{%
     \begin{center}
     }
     {
     \end{center}
}

\NewEnviron{code}{%
	\par
     \boite{gray}{\texttt{%
     \BODY
     }}
     \par
}

\newenvironment{vbloc}{% boite sans cadre empeche saut de page
     \begin{minipage}[t]{\linewidth}
     }
     {
     \end{minipage}
}
\NewEnviron{h2}{%
    \needspace{3\baselineskip}
    \vspace{0.6cm}
	\noindent \MakeUppercase{\sffamily \large \BODY}
	\vspace{1mm}\textcolor{mcgris}{\hrule}\vspace{0.4cm}
	\par
}{}

\NewEnviron{h3}{%
    \needspace{3\baselineskip}
	\vspace{5mm}
	\textsc{\BODY}
	\par
}

\NewEnviron{margeneg}{ %
\begin{addmargin}[-1cm]{0cm}
\BODY
\end{addmargin}
}

\NewEnviron{html}{%
}

\begin{document}
\meta{url}{/exercices/probabilites-bac-blanc-es-l-sujet-1-maths-cours-2017/}
\meta{pid}{10412}
\meta{titre}{Probabilités - Bac blanc ES/L Sujet 1 - Maths-cours 2017}
\meta{type}{exercices}
%
\begin{h2}Exercice 4 (5 points)\end{h2}
\par
Un constructeur fabrique des tablettes informatiques. Le coût de production est 250~euros par unité.
\par
Les tablettes sont garanties contre un défaut de fonctionnement de l'écran ou du disque dur.
\par
Cette garantie permet à l'acheteur, en cas de panne, d'effectuer les réparations suivantes aux frais du constructeur~:
\begin{itemize}
     \item réparation de l'écran (coût pour le constructeur ~: 50~euros) ;
     \item réparation du disque dur (coût pour le constructeur ~: 30~euros).
\end{itemize}
\par
Une étude statistique a montré que ~:
\begin{itemize}
     \item 3\% des tablettes présentent un défaut de disque dur ;
     \item 4\% des tablettes présentent un défaut d'écran ;
     \item 95\% des tablettes ne présentent aucun des deux défauts.
\end{itemize}
\par
%
%
\TitreC{Partie A}
%
%
\par
\begin{enumerate}
     \par
     \item %1
     Recopier et compléter le tableau ci-après à l'aide des données de l'énoncé.
     \begin{center}
          \begin{tabular}{|c|p{2cm}|p{2cm}|c|}%class="compact"
               \hline
               $\ $ & Disque dur OK & Disque dur défectueux & Total \\
               \hline
               \'Ecran OK &  $\cdots$ & $\cdots$ & $\cdots$ \\
               \hline
               \'Ecran défectueux &  $\cdots$ & $\cdots$ & $\cdots$ \\
               \hline
               Total & $\cdots$ & 3\% & 100 \% \\
               \hline
          \end{tabular}
     \end{center}
     \item %2
     Le prix de revient d'une tablette est égal à son coût de production augmenté du coût de réparation éventuel.
     On note $X$ la variable aléatoire correspondant au prix de revient d'une tablette.\\
     \'Etablir la loi de probabilité de $X$.
     \par
     \item %3
     Calculer l'espérance mathématique de $X$. Interpréter ce résultat dans le contexte de l'énoncé.
     \par
     \item %4
     L'entreprise vend chaque tablette 400~euros. Quel sera son bénéfice mensuel moyen si elle vend 750 tablettes par mois ?
     \par
\end{enumerate}
\par
%
%
\TitreC{Partie B}
%
%
\par
Un établissement scolaire achète 50 tablettes à ce constructeur.
\par
On suppose que l'on peut assimiler cet achat à un tirage aléatoire de 50 tablettes avec remise, les tirages étant supposés indépendants.
\par
On rappelle que 95\% des tablettes ne présentent aucun défaut couvert par la garantie constructeur.
\par
On note $Y$ la variable aléatoire égale au nombre de tablettes achetées par l'établissement présentant un défaut couvert par la garantie constructeur.
\par
\begin{enumerate}
     \par
     \item %1
     Justifier que $Y$ suit une loi binomiale dont on précisera les paramètres.
     \par
     \item %2
     Quelle est la probabilité qu'aucune des tablettes achetées par l'établissement ne présente de défaut couvert par la garantie constructeur ?
     \par
     \item %3
     Quelle est l'espérance mathématique de $Y$ ? Interpréter ce résultat.
     \par
\end{enumerate}
\begin{corrige}
     %
     %
     \TitreC{Partie A}
     %
    %
     \par
     \begin{enumerate}
          \par
          \item %1
          On place dans le tableau les données fournies par l'énoncé ~:
          \par
          \begin{itemize}
               \item %
               3\% des tablettes présentent un défaut de disque dur ;
               \item %
               4\% des tablettes présentent un défaut d'écran ;
               \item %
               95\% des tablettes ne présentent aucun des deux défauts.
          \end{itemize}
          \begin{center}
               \begin{tabular}{|c|p{2cm}|p{2cm}|c|}%class="compact"
                    \hline
                    $\ $ & Disque dur OK & Disque dur défectueux & Total \\
                    \hline
                    \'Ecran OK &  95\% & $\cdots$ & $\cdots$ \\
                    \hline
                    \'Ecran défectueux & $\cdots$ & $\cdots$ & 4\% \\
                    \hline
                    Total & $\cdots$ & 3\% & 100 \% \\
                    \hline
               \end{tabular}
          \end{center}
          On complète ensuite les totaux partiels afin que le total global soit égal à 100\% ~:
          \begin{center}
               \begin{tabular}{|c|p{2cm}|p{2cm}|c|}%class="compact"
                    \hline
                    $\ $ & Disque dur OK & Disque dur défectueux & Total \\
                    \hline
                    \'Ecran OK &  95\% & $\cdots$ & 96\% \\
                    \hline
                    \'Ecran défectueux & $\cdots$ & $\cdots$ & 4\% \\
                    \hline
                    Total & 97\% & 3\% & 100 \% \\
                    \hline
               \end{tabular}
          \end{center}
          Les données restantes peuvent être calculées simplement à partir des totaux ~:
          \begin{center}
               \begin{tabular}{|c|p{2cm}|p{2cm}|c|}%class="compact"
                    \hline
                    $\ $ & Disque dur OK & Disque dur défectueux & Total \\
                    \hline
                    \'Ecran OK &  95\% & 1\% & 96\% \\
                    \hline
                    \'Ecran défectueux & 2\% & 2\%  & 4\% \\
                    \hline
                    Total & 97\% & 3\% & 100 \% \\
                    \hline
               \end{tabular}
          \end{center}
          \item %2
          La variable aléatoire $X$ peut prendre quatre valeurs distinctes ; le tableau de la question précédente fournit la probabilité de chacune d'elle ~:
          \par
          \begin{itemize}
               \item si la tablette ne présente aucun défaut ~: ${X=250}$ \textit{(probabilité ~: 0,95)}~;
               \par
               \item si la tablette présente \textbf{uniquement} un défaut de disque dur ~: ${X=250+30=280}$ \textit{(probabilité ~: 0,01)} ;
               \par
               \item si la tablette présente \textbf{uniquement} un défaut d'écran ~:${X=250+50=300}$ \textit{(probabilité ~: 0,02)} ;
               \par
               \item si la tablette présente \textbf{à la fois un défaut de disque dur et un défaut d'écran} ~: ${X=250+50+30=330}$ \textit{(probabilité ~: 0,02)}.
               \par
          \end{itemize}
          \par
          On peut regrouper ces résultats dans un tableau ~:
          \par
          \begin{center}
               \begin{tabular}{|c|c|c|c|c|}%class="compact"
                    \hline
                    $x_i$ & 250 & 280 & 300 & 330 \\
                    \hline
                    $p(X=x_i)$ & 0,95 & 0,01 & 0,02 & 0,02 \\
                    \hline
               \end{tabular}
          \end{center}
          \par
          \cadre{rouge}{À retenir}{
               La \textbf{loi de probabilité} d'une variable aléatoire X est un tableau qui recense les différentes valeurs $x_1, x_2, \cdots, x_n$ prises par X et les probabilités des événements ${(X=x_1), (X=x_2), \cdots, (X=x_n)}$
          }
          \par
          \item %3
          L'espérance mathématique de $X$ est ~:
          \par
          $E(X)=250 \times 0,95 + 280 \times 0,01 + 300 \times 0,02 + 330 \times 0,02 = 252,9$.
          \par
          \cadre{rouge}{À retenir}{
               Si X est une variable aléatoire qui prend valeurs $x_1, x_2, \cdots, x_n$ avec les probabilités respectives $p_1, p_2, \cdots, p_n$, l'\textbf{espérance mathématique} de X est ~:
               \[ E(X)=p_1x_1+p_2x_2+ \cdots +p_nx_n \]
               \par
          }
          \par
          \item %4
          D'après la question précédente, le prix de revient moyen d'une tablette est de 252,9~euros.
          \par
          Si chaque tablette est vendu 400~euros, le bénéfice moyen par tablette vendue sera de $400 - 252,9 = 147,1$~euros.
          \par
          Pour une vente mensuelle de 750 tablettes, l'entreprise fera un bénéfice mensuel moyen de $750 \times 147,1 =\bm{110\ 325}$ euros.
          \par
     \end{enumerate}
     \par
     %============================================================================================================================
     %
     \TitreC{Partie B}
     %
     %============================================================================================================================
     \par
     \begin{enumerate}
          \par
          \item %1
          \par
          La variable aléatoire $Y$ suit une loi binomiale de paramètres $n=50$ et $p=0,05$ puisque ~:
          \par
          \begin{itemize}
               \par
               \item on assimile l'expérience à la répétition de 50 tirages aléatoires identiques et indépendants ;
               \par
               \item chaque tirage possède deux issues ~:
               \par
               \begin{itemize}
                    \par
                    \item \textit{succès}, correspondant au tirage d'une tablette défectueuse (probabilité $p=0,05$) ;
                    \item \textit{échec}, correspondant au tirage d'une tablette fonctionnant correctement ;
                    \par
               \end{itemize}
               \par
               \item la variable aléatoire $Y$ comptabilise le nombre de succès.
               \par
          \end{itemize}
          \par
          \cadre{rouge}{Bien rédiger}{
               Pour montrer qu'une variable aléatoire suit une \textbf{loi binomiale} $\mathscr{B}(n~;~p)$ de paramètres $n$ et $p$, on précise que ~:
               \par
               \begin{itemize}
                    \par
                    \item l'expérience aléatoire est la \textbf{répétition} de $n$ épreuves de Bernoulli \textbf{identiques et indépendantes} ;
                    \par
                    \item chaque épreuve de Bernoulli possède \textbf{deux issues} ~:
                    \begin{itemize}[label=---]
                         \item %
                         \textit{succès}, de probabilité $p$;
                         \item %
                         \textit{échec}, de probabilité $1-p$ ;
                    \end{itemize}
                    \item la variable aléatoire $X$ \textbf{comptabilise le nombre de succès}.
                    \par
               \end{itemize}
          }
          \item %2
          La probabilité qu'aucune des tablettes achetées par l'établissement ne présente de défaut est ~:
          \par
          $P(Y=0)=\begin{pmatrix} 50 \\ 0 \end{pmatrix} \times 0,05^0 \times 0,95^{50} = 0,95^{50}$
          \par
          $P(Y=0) \approx 0,077$ (arrondi au millième).
          \par
          \cadre{rouge}{À retenir}{
               Si la variable aléatoire $X$ suit une \textbf{loi binomiale} $\mathscr B \left(n ; p\right)$, pour tout entier naturel $k$ compris entre $0$ et $n$, la probabilité que $X$ prenne la valeur $k$ est ~:
               \[ P\left(X=k\right)=\begin{pmatrix} n \\ k \end{pmatrix}p^{k} \left(1-p\right)^{n-k} \]
          }
          \par
          \item %3
          L'espérance mathématique de $Y$ est ~:
          \par
          $E(Y)=np=50 \times 0,05=2,5$.
          \par
          En moyenne, parmi les 50 tablettes achetées par l'école, 2,5~tablettes présenteront un défaut.
          \par
          \cadre{rouge}{À retenir}{
               Pour une variable aléatoire $X$ qui une \textbf{loi binomiale} $\mathscr B \left(n ; p\right)$, l'\textbf{espérance mathématique} vaut ~:
               \[ E(X)=np \]
          }
          \par
     \end{enumerate}
\end{corrige}

\end{document}