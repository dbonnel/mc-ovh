\documentclass[a4paper]{article}

%================================================================================================================================
%
% Packages
%
%================================================================================================================================

\usepackage[T1]{fontenc} 	% pour caractères accentués
\usepackage[utf8]{inputenc}  % encodage utf8
\usepackage[french]{babel}	% langue : français
\usepackage{fourier}			% caractères plus lisibles
\usepackage[dvipsnames]{xcolor} % couleurs
\usepackage{fancyhdr}		% réglage header footer
\usepackage{needspace}		% empêcher sauts de page mal placés
\usepackage{graphicx}		% pour inclure des graphiques
\usepackage{enumitem,cprotect}		% personnalise les listes d'items (nécessaire pour ol, al ...)
\usepackage{hyperref}		% Liens hypertexte
\usepackage{pstricks,pst-all,pst-node,pstricks-add,pst-math,pst-plot,pst-tree,pst-eucl} % pstricks
\usepackage[a4paper,includeheadfoot,top=2cm,left=3cm, bottom=2cm,right=3cm]{geometry} % marges etc.
\usepackage{comment}			% commentaires multilignes
\usepackage{amsmath,environ} % maths (matrices, etc.)
\usepackage{amssymb,makeidx}
\usepackage{bm}				% bold maths
\usepackage{tabularx}		% tableaux
\usepackage{colortbl}		% tableaux en couleur
\usepackage{fontawesome}		% Fontawesome
\usepackage{environ}			% environment with command
\usepackage{fp}				% calculs pour ps-tricks
\usepackage{multido}			% pour ps tricks
\usepackage[np]{numprint}	% formattage nombre
\usepackage{tikz,tkz-tab} 			% package principal TikZ
\usepackage{pgfplots}   % axes
\usepackage{mathrsfs}    % cursives
\usepackage{calc}			% calcul taille boites
\usepackage[scaled=0.875]{helvet} % font sans serif
\usepackage{svg} % svg
\usepackage{scrextend} % local margin
\usepackage{scratch} %scratch
\usepackage{multicol} % colonnes
%\usepackage{infix-RPN,pst-func} % formule en notation polanaise inversée
\usepackage{listings}

%================================================================================================================================
%
% Réglages de base
%
%================================================================================================================================

\lstset{
language=Python,   % R code
literate=
{á}{{\'a}}1
{à}{{\`a}}1
{ã}{{\~a}}1
{é}{{\'e}}1
{è}{{\`e}}1
{ê}{{\^e}}1
{í}{{\'i}}1
{ó}{{\'o}}1
{õ}{{\~o}}1
{ú}{{\'u}}1
{ü}{{\"u}}1
{ç}{{\c{c}}}1
{~}{{ }}1
}


\definecolor{codegreen}{rgb}{0,0.6,0}
\definecolor{codegray}{rgb}{0.5,0.5,0.5}
\definecolor{codepurple}{rgb}{0.58,0,0.82}
\definecolor{backcolour}{rgb}{0.95,0.95,0.92}

\lstdefinestyle{mystyle}{
    backgroundcolor=\color{backcolour},   
    commentstyle=\color{codegreen},
    keywordstyle=\color{magenta},
    numberstyle=\tiny\color{codegray},
    stringstyle=\color{codepurple},
    basicstyle=\ttfamily\footnotesize,
    breakatwhitespace=false,         
    breaklines=true,                 
    captionpos=b,                    
    keepspaces=true,                 
    numbers=left,                    
xleftmargin=2em,
framexleftmargin=2em,            
    showspaces=false,                
    showstringspaces=false,
    showtabs=false,                  
    tabsize=2,
    upquote=true
}

\lstset{style=mystyle}


\lstset{style=mystyle}
\newcommand{\imgdir}{C:/laragon/www/newmc/assets/imgsvg/}
\newcommand{\imgsvgdir}{C:/laragon/www/newmc/assets/imgsvg/}

\definecolor{mcgris}{RGB}{220, 220, 220}% ancien~; pour compatibilité
\definecolor{mcbleu}{RGB}{52, 152, 219}
\definecolor{mcvert}{RGB}{125, 194, 70}
\definecolor{mcmauve}{RGB}{154, 0, 215}
\definecolor{mcorange}{RGB}{255, 96, 0}
\definecolor{mcturquoise}{RGB}{0, 153, 153}
\definecolor{mcrouge}{RGB}{255, 0, 0}
\definecolor{mclightvert}{RGB}{205, 234, 190}

\definecolor{gris}{RGB}{220, 220, 220}
\definecolor{bleu}{RGB}{52, 152, 219}
\definecolor{vert}{RGB}{125, 194, 70}
\definecolor{mauve}{RGB}{154, 0, 215}
\definecolor{orange}{RGB}{255, 96, 0}
\definecolor{turquoise}{RGB}{0, 153, 153}
\definecolor{rouge}{RGB}{255, 0, 0}
\definecolor{lightvert}{RGB}{205, 234, 190}
\setitemize[0]{label=\color{lightvert}  $\bullet$}

\pagestyle{fancy}
\renewcommand{\headrulewidth}{0.2pt}
\fancyhead[L]{maths-cours.fr}
\fancyhead[R]{\thepage}
\renewcommand{\footrulewidth}{0.2pt}
\fancyfoot[C]{}

\newcolumntype{C}{>{\centering\arraybackslash}X}
\newcolumntype{s}{>{\hsize=.35\hsize\arraybackslash}X}

\setlength{\parindent}{0pt}		 
\setlength{\parskip}{3mm}
\setlength{\headheight}{1cm}

\def\ebook{ebook}
\def\book{book}
\def\web{web}
\def\type{web}

\newcommand{\vect}[1]{\overrightarrow{\,\mathstrut#1\,}}

\def\Oij{$\left(\text{O}~;~\vect{\imath},~\vect{\jmath}\right)$}
\def\Oijk{$\left(\text{O}~;~\vect{\imath},~\vect{\jmath},~\vect{k}\right)$}
\def\Ouv{$\left(\text{O}~;~\vect{u},~\vect{v}\right)$}

\hypersetup{breaklinks=true, colorlinks = true, linkcolor = OliveGreen, urlcolor = OliveGreen, citecolor = OliveGreen, pdfauthor={Didier BONNEL - https://www.maths-cours.fr} } % supprime les bordures autour des liens

\renewcommand{\arg}[0]{\text{arg}}

\everymath{\displaystyle}

%================================================================================================================================
%
% Macros - Commandes
%
%================================================================================================================================

\newcommand\meta[2]{    			% Utilisé pour créer le post HTML.
	\def\titre{titre}
	\def\url{url}
	\def\arg{#1}
	\ifx\titre\arg
		\newcommand\maintitle{#2}
		\fancyhead[L]{#2}
		{\Large\sffamily \MakeUppercase{#2}}
		\vspace{1mm}\textcolor{mcvert}{\hrule}
	\fi 
	\ifx\url\arg
		\fancyfoot[L]{\href{https://www.maths-cours.fr#2}{\black \footnotesize{https://www.maths-cours.fr#2}}}
	\fi 
}


\newcommand\TitreC[1]{    		% Titre centré
     \needspace{3\baselineskip}
     \begin{center}\textbf{#1}\end{center}
}

\newcommand\newpar{    		% paragraphe
     \par
}

\newcommand\nosp {    		% commande vide (pas d'espace)
}
\newcommand{\id}[1]{} %ignore

\newcommand\boite[2]{				% Boite simple sans titre
	\vspace{5mm}
	\setlength{\fboxrule}{0.2mm}
	\setlength{\fboxsep}{5mm}	
	\fcolorbox{#1}{#1!3}{\makebox[\linewidth-2\fboxrule-2\fboxsep]{
  		\begin{minipage}[t]{\linewidth-2\fboxrule-4\fboxsep}\setlength{\parskip}{3mm}
  			 #2
  		\end{minipage}
	}}
	\vspace{5mm}
}

\newcommand\CBox[4]{				% Boites
	\vspace{5mm}
	\setlength{\fboxrule}{0.2mm}
	\setlength{\fboxsep}{5mm}
	
	\fcolorbox{#1}{#1!3}{\makebox[\linewidth-2\fboxrule-2\fboxsep]{
		\begin{minipage}[t]{1cm}\setlength{\parskip}{3mm}
	  		\textcolor{#1}{\LARGE{#2}}    
 	 	\end{minipage}  
  		\begin{minipage}[t]{\linewidth-2\fboxrule-4\fboxsep}\setlength{\parskip}{3mm}
			\raisebox{1.2mm}{\normalsize\sffamily{\textcolor{#1}{#3}}}						
  			 #4
  		\end{minipage}
	}}
	\vspace{5mm}
}

\newcommand\cadre[3]{				% Boites convertible html
	\par
	\vspace{2mm}
	\setlength{\fboxrule}{0.1mm}
	\setlength{\fboxsep}{5mm}
	\fcolorbox{#1}{white}{\makebox[\linewidth-2\fboxrule-2\fboxsep]{
  		\begin{minipage}[t]{\linewidth-2\fboxrule-4\fboxsep}\setlength{\parskip}{3mm}
			\raisebox{-2.5mm}{\sffamily \small{\textcolor{#1}{\MakeUppercase{#2}}}}		
			\par		
  			 #3
 	 		\end{minipage}
	}}
		\vspace{2mm}
	\par
}

\newcommand\bloc[3]{				% Boites convertible html sans bordure
     \needspace{2\baselineskip}
     {\sffamily \small{\textcolor{#1}{\MakeUppercase{#2}}}}    
		\par		
  			 #3
		\par
}

\newcommand\CHelp[1]{
     \CBox{Plum}{\faInfoCircle}{À RETENIR}{#1}
}

\newcommand\CUp[1]{
     \CBox{NavyBlue}{\faThumbsOUp}{EN PRATIQUE}{#1}
}

\newcommand\CInfo[1]{
     \CBox{Sepia}{\faArrowCircleRight}{REMARQUE}{#1}
}

\newcommand\CRedac[1]{
     \CBox{PineGreen}{\faEdit}{BIEN R\'EDIGER}{#1}
}

\newcommand\CError[1]{
     \CBox{Red}{\faExclamationTriangle}{ATTENTION}{#1}
}

\newcommand\TitreExo[2]{
\needspace{4\baselineskip}
 {\sffamily\large EXERCICE #1\ (\emph{#2 points})}
\vspace{5mm}
}

\newcommand\img[2]{
          \includegraphics[width=#2\paperwidth]{\imgdir#1}
}

\newcommand\imgsvg[2]{
       \begin{center}   \includegraphics[width=#2\paperwidth]{\imgsvgdir#1} \end{center}
}


\newcommand\Lien[2]{
     \href{#1}{#2 \tiny \faExternalLink}
}
\newcommand\mcLien[2]{
     \href{https~://www.maths-cours.fr/#1}{#2 \tiny \faExternalLink}
}

\newcommand{\euro}{\eurologo{}}

%================================================================================================================================
%
% Macros - Environement
%
%================================================================================================================================

\newenvironment{tex}{ %
}
{%
}

\newenvironment{indente}{ %
	\setlength\parindent{10mm}
}

{
	\setlength\parindent{0mm}
}

\newenvironment{corrige}{%
     \needspace{3\baselineskip}
     \medskip
     \textbf{\textsc{Corrigé}}
     \medskip
}
{
}

\newenvironment{extern}{%
     \begin{center}
     }
     {
     \end{center}
}

\NewEnviron{code}{%
	\par
     \boite{gray}{\texttt{%
     \BODY
     }}
     \par
}

\newenvironment{vbloc}{% boite sans cadre empeche saut de page
     \begin{minipage}[t]{\linewidth}
     }
     {
     \end{minipage}
}
\NewEnviron{h2}{%
    \needspace{3\baselineskip}
    \vspace{0.6cm}
	\noindent \MakeUppercase{\sffamily \large \BODY}
	\vspace{1mm}\textcolor{mcgris}{\hrule}\vspace{0.4cm}
	\par
}{}

\NewEnviron{h3}{%
    \needspace{3\baselineskip}
	\vspace{5mm}
	\textsc{\BODY}
	\par
}

\NewEnviron{margeneg}{ %
\begin{addmargin}[-1cm]{0cm}
\BODY
\end{addmargin}
}

\NewEnviron{html}{%
}

\begin{document}
\meta{url}{/supplement/fiche-de-revision-bac-les-suites/}
\meta{pid}{10791}
\meta{titre}{Fiche de révision BAC : les suites}
\meta{type}{supplement}
%
\begin{enumerate}
     \item %
     Comment peut-on montrer qu'une suite est croissante ? décroissante ? constante ?
     \item %
     Qu'est-ce qu'une suite majorée ? minorée ? bornée ?
     \item %
     Quelles méthodes peut-on utiliser pour montrer qu'une suite est convergente ?
     \item %
     Comment montre-t-on qu'une suite est arithmétique ?
     \item %
     Pour une suite arithmétique de raison $r$, quelle formule permet de calculer $u_n$ en fonction de $u_0 $ ? en fonction de $u_p$  $(p \in \mathbb{N})$ ?
     \item %
     Que vaut la somme : $1+2+3+\cdots+n$ ?
     \item %
     Comment montre-t-on qu'une suite est géométrique ?
     \item %
     Pour une suite géométrique de raison $q$, quelle formule permet de calculer $u_n$ en fonction de $u_0 $? en fonction de $u_p$  $(p \in \mathbb{N})$ ?
     \item %
     Que vaut la somme : $1+q+q^2+\cdots+q^n $?
     \item %
     Quelle est (en fonction de $q$) la limite de $q^n$ ?
     \item %
     Écrire un algorithme affichant les $n$ premiers termes d'une suite.
     \item %
     Quelles sont les étapes d'une démonstration par récurrence ?
\end{enumerate}
\begin{reponses}
     \begin{enumerate}
          \item %
          \textit{Comment peut-on montrer qu'une suite est croissante ? décroissante ? constante ?}
          \par
          Voici 3 des principales méthodes :
          \begin{enumerate}[label=\alph*.]
               \item %
               \textbf{Calcul de $u_{n+1}-u_n$.}
               \par
               Si cette différence est positive pour tout entier naturel $n$ la suite $(u_n)$ est croissante~;
               \par
               si cette différence est négative pour tout entier naturel $n$ la suite $(u_n)$ est décroissante~;
               \par
               enfin, si cette différence est nulle pour tout entier naturel $n$ la suite $(u_n)$ est constante.
               \item %
               \textbf{Par récurrence.}
               \par
               Dans ce cas, c'est la comparaison des deux premiers termes (e.g. $u_0$ et $u_1$) qui dira si la suite est croissante ou décroissante.
               \item %
               Si la suite $(u_n)$ est définie de façon explicite par une formule du type $u_n=f(n)$, on peut étudier les variations de $f$ sur $[0~;~+\infty[$ (calcul de la dérivée $f'$...).
          \end{enumerate}
          \item %
          \textit{Qu'est-ce qu'une suite majorée ? minorée ? bornée ?}
          \par
          Une suite $(u_n)$ est \textbf{majorée} s'il existe un réel $M$ tel que pour tout entier naturel $n$~: $u_n \leqslant M$.
          \par
          Une suite $(u_n)$ est \textbf{minorée} s'il existe un réel $m$ tel que pour tout entier naturel $n$~: $u_n \geqslant m$.
          \par
          Une suite est bornée si elle est à la fois majorée et minorée.
          \item %
          \textit{Quelles méthodes peut-on utiliser pour montrer qu'une suite est convergente ?}
          \par
          Voici 3 méthodes. La plus utilisée dans les sujets du bac est la première.
          \begin{enumerate}[label=\alph*.]
               \item %
               \textbf{Suite croissante majorée ou décroissante minorée.}
               Si une suite est croissante et majorée alors elle est convergente. De même, une suite décroissante et minorée est convergente.
               \item %
               \textbf{Théorème des gendarmes} (Voir \mcLien{/cours/variations-convergence-suite\#t60}{cours}).
               \item %
               Si la suite $(u_n)$ est définie de façon explicite on peut calculer la limite en utilisant les règles de calculs des limites (similaires à celles utilisées pour les fonctions).
               \par
               Dans ce cas, gardez aussi à l'esprit la formule donnant la limite de $q^n$ (voir ci-dessous)
          \end{enumerate}
          \item %
          \textit{Comment montre-t-on qu'une suite est arithmétique ?}
          Pour montrer que la suite $(u_n)$ est arithmétique on calcule $u_{n+1}-u_n$ et on montre que le résultat est constant (indépendant de $n$). Ce résultat est la raison de la suite arithmétique.
          \item %
          \textit{Pour une suite arithmétique de raison $r$, quelle formule permet de calculer $u_n$ en fonction de $u_0 $~? en fonction de $u_p$  $(p \in \mathbb{N})$~?}
          \par
          En fonction de $u_0~:~u_n=u_0+nr$
          \par
          En fonction de $u_p~:~u_n=u_p+(n-p)r$
          \item %
          \textit{Que vaut la somme : $1+2+3+\cdots+n$ ?}
          \par
          $1+2+3+\cdots+n=\dfrac{n(n+1)}{2}$
          \item %
          \textit{Comment montre-t-on qu'une suite $(u_n)$ est géométrique ?}
          On montre qu'il existe un réel $q$, indépendant de $n$, tel que pour tout entier naturel $n$~: $u_{n+1}=qu_n$.
          (on peut également montrer que le rapport $\dfrac{u_{n+1}}{u_n}$ est constant si on sait que la suite $(u_n)$ ne s'annule pas.)
          \item %
          \textit{Pour une suite géométrique de raison $q$, quelle formule permet de calculer $u_n$ en fonction de $u_0 $? en fonction de $u_p$  $(p \in \mathbb{N})$ ?}
          En fonction de $u_0~:~u_n=u_0q^n$
          En fonction de $u_p~:~u_n=u_pq^{n-p}$
          \item %
          \textit{ Que vaut la somme : $1+q+q^2+\cdots+q^n $?}
          \par
          Pour tout réel $q \neq 1$~:
          \par
          $1+q+q^2+\cdots+q^n =\dfrac{1-q^{n+1}}{1-q}$
          \item %
          \textit{Quelle est (en fonction de $q$) la limite de $q^n$ ?}
          \begin{itemize}
               \item %
               si $q>1~:~\lim\limits_{n \rightarrow +\infty }q^n=+\infty$~;~la suite est divergente ;
               \item %
               si $-1<q<1~:~\lim\limits_{n \rightarrow +\infty }q^n=0$~;~la suite converge vers 0 ;
               \item %
               si $q \leqslant -1~:$~la suite est divergente (pas de limite) ;
               \item %
               pour $q=1$, la suite est constante.
          \end{itemize}
          \item %
          \textit{Écrire un algorithme affichant les $n$ premiers termes d'une suite.}
          \par
          Voir la fiche \mcLien{/methode/algorithme-premiers-termes}{Algorithme de calcul des premiers termes d'une suite}.
          \item %
          \textit{ Quelles sont les étapes d'une démonstration par récurrence ?}
          \begin{itemize}
               \item %
               \textbf{Initialisation}~:~On montre que la propriété est vraie au premier rang (e.g. au rang 0).
               \item %
               \textbf{Hérédité}~:~On montre que si la propriété est vraie à un certain rang , alors elle est vraie au rang suivant.
               \item %
               \textbf{Conclusion}~:~On en déduit que la propriété est vraie pour tout entier naturel $n$ (ou pour tout entier $n \geqslant n_0$ si l'initialisation a été faite au rang $n_0$).
          \end{itemize}
     \end{enumerate}
\end{reponses}

\end{document}