\documentclass[a4paper]{article}

%================================================================================================================================
%
% Packages
%
%================================================================================================================================

\usepackage[T1]{fontenc} 	% pour caractères accentués
\usepackage[utf8]{inputenc}  % encodage utf8
\usepackage[french]{babel}	% langue : français
\usepackage{fourier}			% caractères plus lisibles
\usepackage[dvipsnames]{xcolor} % couleurs
\usepackage{fancyhdr}		% réglage header footer
\usepackage{needspace}		% empêcher sauts de page mal placés
\usepackage{graphicx}		% pour inclure des graphiques
\usepackage{enumitem,cprotect}		% personnalise les listes d'items (nécessaire pour ol, al ...)
\usepackage{hyperref}		% Liens hypertexte
\usepackage{pstricks,pst-all,pst-node,pstricks-add,pst-math,pst-plot,pst-tree,pst-eucl} % pstricks
\usepackage[a4paper,includeheadfoot,top=2cm,left=3cm, bottom=2cm,right=3cm]{geometry} % marges etc.
\usepackage{comment}			% commentaires multilignes
\usepackage{amsmath,environ} % maths (matrices, etc.)
\usepackage{amssymb,makeidx}
\usepackage{bm}				% bold maths
\usepackage{tabularx}		% tableaux
\usepackage{colortbl}		% tableaux en couleur
\usepackage{fontawesome}		% Fontawesome
\usepackage{environ}			% environment with command
\usepackage{fp}				% calculs pour ps-tricks
\usepackage{multido}			% pour ps tricks
\usepackage[np]{numprint}	% formattage nombre
\usepackage{tikz,tkz-tab} 			% package principal TikZ
\usepackage{pgfplots}   % axes
\usepackage{mathrsfs}    % cursives
\usepackage{calc}			% calcul taille boites
\usepackage[scaled=0.875]{helvet} % font sans serif
\usepackage{svg} % svg
\usepackage{scrextend} % local margin
\usepackage{scratch} %scratch
\usepackage{multicol} % colonnes
%\usepackage{infix-RPN,pst-func} % formule en notation polanaise inversée
\usepackage{listings}

%================================================================================================================================
%
% Réglages de base
%
%================================================================================================================================

\lstset{
language=Python,   % R code
literate=
{á}{{\'a}}1
{à}{{\`a}}1
{ã}{{\~a}}1
{é}{{\'e}}1
{è}{{\`e}}1
{ê}{{\^e}}1
{í}{{\'i}}1
{ó}{{\'o}}1
{õ}{{\~o}}1
{ú}{{\'u}}1
{ü}{{\"u}}1
{ç}{{\c{c}}}1
{~}{{ }}1
}


\definecolor{codegreen}{rgb}{0,0.6,0}
\definecolor{codegray}{rgb}{0.5,0.5,0.5}
\definecolor{codepurple}{rgb}{0.58,0,0.82}
\definecolor{backcolour}{rgb}{0.95,0.95,0.92}

\lstdefinestyle{mystyle}{
    backgroundcolor=\color{backcolour},   
    commentstyle=\color{codegreen},
    keywordstyle=\color{magenta},
    numberstyle=\tiny\color{codegray},
    stringstyle=\color{codepurple},
    basicstyle=\ttfamily\footnotesize,
    breakatwhitespace=false,         
    breaklines=true,                 
    captionpos=b,                    
    keepspaces=true,                 
    numbers=left,                    
xleftmargin=2em,
framexleftmargin=2em,            
    showspaces=false,                
    showstringspaces=false,
    showtabs=false,                  
    tabsize=2,
    upquote=true
}

\lstset{style=mystyle}


\lstset{style=mystyle}
\newcommand{\imgdir}{C:/laragon/www/newmc/assets/imgsvg/}
\newcommand{\imgsvgdir}{C:/laragon/www/newmc/assets/imgsvg/}

\definecolor{mcgris}{RGB}{220, 220, 220}% ancien~; pour compatibilité
\definecolor{mcbleu}{RGB}{52, 152, 219}
\definecolor{mcvert}{RGB}{125, 194, 70}
\definecolor{mcmauve}{RGB}{154, 0, 215}
\definecolor{mcorange}{RGB}{255, 96, 0}
\definecolor{mcturquoise}{RGB}{0, 153, 153}
\definecolor{mcrouge}{RGB}{255, 0, 0}
\definecolor{mclightvert}{RGB}{205, 234, 190}

\definecolor{gris}{RGB}{220, 220, 220}
\definecolor{bleu}{RGB}{52, 152, 219}
\definecolor{vert}{RGB}{125, 194, 70}
\definecolor{mauve}{RGB}{154, 0, 215}
\definecolor{orange}{RGB}{255, 96, 0}
\definecolor{turquoise}{RGB}{0, 153, 153}
\definecolor{rouge}{RGB}{255, 0, 0}
\definecolor{lightvert}{RGB}{205, 234, 190}
\setitemize[0]{label=\color{lightvert}  $\bullet$}

\pagestyle{fancy}
\renewcommand{\headrulewidth}{0.2pt}
\fancyhead[L]{maths-cours.fr}
\fancyhead[R]{\thepage}
\renewcommand{\footrulewidth}{0.2pt}
\fancyfoot[C]{}

\newcolumntype{C}{>{\centering\arraybackslash}X}
\newcolumntype{s}{>{\hsize=.35\hsize\arraybackslash}X}

\setlength{\parindent}{0pt}		 
\setlength{\parskip}{3mm}
\setlength{\headheight}{1cm}

\def\ebook{ebook}
\def\book{book}
\def\web{web}
\def\type{web}

\newcommand{\vect}[1]{\overrightarrow{\,\mathstrut#1\,}}

\def\Oij{$\left(\text{O}~;~\vect{\imath},~\vect{\jmath}\right)$}
\def\Oijk{$\left(\text{O}~;~\vect{\imath},~\vect{\jmath},~\vect{k}\right)$}
\def\Ouv{$\left(\text{O}~;~\vect{u},~\vect{v}\right)$}

\hypersetup{breaklinks=true, colorlinks = true, linkcolor = OliveGreen, urlcolor = OliveGreen, citecolor = OliveGreen, pdfauthor={Didier BONNEL - https://www.maths-cours.fr} } % supprime les bordures autour des liens

\renewcommand{\arg}[0]{\text{arg}}

\everymath{\displaystyle}

%================================================================================================================================
%
% Macros - Commandes
%
%================================================================================================================================

\newcommand\meta[2]{    			% Utilisé pour créer le post HTML.
	\def\titre{titre}
	\def\url{url}
	\def\arg{#1}
	\ifx\titre\arg
		\newcommand\maintitle{#2}
		\fancyhead[L]{#2}
		{\Large\sffamily \MakeUppercase{#2}}
		\vspace{1mm}\textcolor{mcvert}{\hrule}
	\fi 
	\ifx\url\arg
		\fancyfoot[L]{\href{https://www.maths-cours.fr#2}{\black \footnotesize{https://www.maths-cours.fr#2}}}
	\fi 
}


\newcommand\TitreC[1]{    		% Titre centré
     \needspace{3\baselineskip}
     \begin{center}\textbf{#1}\end{center}
}

\newcommand\newpar{    		% paragraphe
     \par
}

\newcommand\nosp {    		% commande vide (pas d'espace)
}
\newcommand{\id}[1]{} %ignore

\newcommand\boite[2]{				% Boite simple sans titre
	\vspace{5mm}
	\setlength{\fboxrule}{0.2mm}
	\setlength{\fboxsep}{5mm}	
	\fcolorbox{#1}{#1!3}{\makebox[\linewidth-2\fboxrule-2\fboxsep]{
  		\begin{minipage}[t]{\linewidth-2\fboxrule-4\fboxsep}\setlength{\parskip}{3mm}
  			 #2
  		\end{minipage}
	}}
	\vspace{5mm}
}

\newcommand\CBox[4]{				% Boites
	\vspace{5mm}
	\setlength{\fboxrule}{0.2mm}
	\setlength{\fboxsep}{5mm}
	
	\fcolorbox{#1}{#1!3}{\makebox[\linewidth-2\fboxrule-2\fboxsep]{
		\begin{minipage}[t]{1cm}\setlength{\parskip}{3mm}
	  		\textcolor{#1}{\LARGE{#2}}    
 	 	\end{minipage}  
  		\begin{minipage}[t]{\linewidth-2\fboxrule-4\fboxsep}\setlength{\parskip}{3mm}
			\raisebox{1.2mm}{\normalsize\sffamily{\textcolor{#1}{#3}}}						
  			 #4
  		\end{minipage}
	}}
	\vspace{5mm}
}

\newcommand\cadre[3]{				% Boites convertible html
	\par
	\vspace{2mm}
	\setlength{\fboxrule}{0.1mm}
	\setlength{\fboxsep}{5mm}
	\fcolorbox{#1}{white}{\makebox[\linewidth-2\fboxrule-2\fboxsep]{
  		\begin{minipage}[t]{\linewidth-2\fboxrule-4\fboxsep}\setlength{\parskip}{3mm}
			\raisebox{-2.5mm}{\sffamily \small{\textcolor{#1}{\MakeUppercase{#2}}}}		
			\par		
  			 #3
 	 		\end{minipage}
	}}
		\vspace{2mm}
	\par
}

\newcommand\bloc[3]{				% Boites convertible html sans bordure
     \needspace{2\baselineskip}
     {\sffamily \small{\textcolor{#1}{\MakeUppercase{#2}}}}    
		\par		
  			 #3
		\par
}

\newcommand\CHelp[1]{
     \CBox{Plum}{\faInfoCircle}{À RETENIR}{#1}
}

\newcommand\CUp[1]{
     \CBox{NavyBlue}{\faThumbsOUp}{EN PRATIQUE}{#1}
}

\newcommand\CInfo[1]{
     \CBox{Sepia}{\faArrowCircleRight}{REMARQUE}{#1}
}

\newcommand\CRedac[1]{
     \CBox{PineGreen}{\faEdit}{BIEN R\'EDIGER}{#1}
}

\newcommand\CError[1]{
     \CBox{Red}{\faExclamationTriangle}{ATTENTION}{#1}
}

\newcommand\TitreExo[2]{
\needspace{4\baselineskip}
 {\sffamily\large EXERCICE #1\ (\emph{#2 points})}
\vspace{5mm}
}

\newcommand\img[2]{
          \includegraphics[width=#2\paperwidth]{\imgdir#1}
}

\newcommand\imgsvg[2]{
       \begin{center}   \includegraphics[width=#2\paperwidth]{\imgsvgdir#1} \end{center}
}


\newcommand\Lien[2]{
     \href{#1}{#2 \tiny \faExternalLink}
}
\newcommand\mcLien[2]{
     \href{https~://www.maths-cours.fr/#1}{#2 \tiny \faExternalLink}
}

\newcommand{\euro}{\eurologo{}}

%================================================================================================================================
%
% Macros - Environement
%
%================================================================================================================================

\newenvironment{tex}{ %
}
{%
}

\newenvironment{indente}{ %
	\setlength\parindent{10mm}
}

{
	\setlength\parindent{0mm}
}

\newenvironment{corrige}{%
     \needspace{3\baselineskip}
     \medskip
     \textbf{\textsc{Corrigé}}
     \medskip
}
{
}

\newenvironment{extern}{%
     \begin{center}
     }
     {
     \end{center}
}

\NewEnviron{code}{%
	\par
     \boite{gray}{\texttt{%
     \BODY
     }}
     \par
}

\newenvironment{vbloc}{% boite sans cadre empeche saut de page
     \begin{minipage}[t]{\linewidth}
     }
     {
     \end{minipage}
}
\NewEnviron{h2}{%
    \needspace{3\baselineskip}
    \vspace{0.6cm}
	\noindent \MakeUppercase{\sffamily \large \BODY}
	\vspace{1mm}\textcolor{mcgris}{\hrule}\vspace{0.4cm}
	\par
}{}

\NewEnviron{h3}{%
    \needspace{3\baselineskip}
	\vspace{5mm}
	\textsc{\BODY}
	\par
}

\NewEnviron{margeneg}{ %
\begin{addmargin}[-1cm]{0cm}
\BODY
\end{addmargin}
}

\NewEnviron{html}{%
}

\begin{document}
\meta{url}{/exercices/probabilites-et-suites-bac-blanc-es-l-sujet-6-maths-cours-2018/}
\meta{pid}{10592}
\meta{titre}{Probabilités et suites - Bac blanc ES/L Sujet 6 - Maths-cours 2018}
\meta{type}{exercices}
%
\begin{h2}Exercice 2 (5 points)\end{h2}
\par
Lors d'un tournoi de jeux vidéo, Loïc dispute plusieurs parties d'affilée.
\par
La probabilité qu'il gagne la première partie est 0,5.
\par
Lorsqu'il gagne une partie, la probabilité qu'il gagne la suivante est 0,7.
\par
Si, par contre, il perd une partie, la probabilité qu'il gagne la suivante est seulement 0,4.
\par
Pour tout entier naturel $n$ supérieur ou égal à 1, on note $G_n$ l'événement \og Loïc a gagné la $n$-ième partie \fg{}, $\overline{G_n}$ l'événement contraire et $p_n$ la probabilité de l'événement $G_n$. On a donc $p_1=0,5$.
\par
\begin{enumerate}
     \item
     Recopier et compléter l'arbre ci-après :
     \par
     %:-+-+-+- Engendré par : http://math.et.info.free.fr/TikZ/Arbre/
     \begin{center}
          \begin{extern}%width="340" alt="Arbre de probabilités à compléter"
               % Racine à Gauche, développement vers la droite
               \begin{tikzpicture}[xscale=1,yscale=1]
                    % Styles (MODIFIABLES)
                    \tikzstyle{fleche}=[-,>=latex,thick]
                    \tikzstyle{noeud}=[fill=white,circle,draw]
                    \tikzstyle{feuille}=[fill=white,circle,draw]
                    \tikzstyle{etiquette}=[midway,fill=white]
                    % Dimensions (MODIFIABLES)
                    \def\DistanceInterNiveaux{3}
                    \def\DistanceInterFeuilles{2}
                    % Dimensions calculées (NON MODIFIABLES)
                    \def\NiveauA{(0)*\DistanceInterNiveaux}
                    \def\NiveauB{(1.5)*\DistanceInterNiveaux}
                    \def\NiveauC{(2.5)*\DistanceInterNiveaux}
                    \def\InterFeuilles{(-1)*\DistanceInterFeuilles}
                    % Noeuds (MODIFIABLES : Styles et Coefficients d'InterFeuilles)
                    \node[noeud] (R) at ({\NiveauA},{(1.5)*\InterFeuilles}) {$\ $};
                    \node[noeud] (Ra) at ({\NiveauB},{(0.5)*\InterFeuilles}) {$G_n$};
                    \node[feuille] (Raa) at ({\NiveauC},{(0)*\InterFeuilles}) {$G_{n+1}$};
                    \node[feuille] (Rab) at ({\NiveauC},{(1)*\InterFeuilles}) {$\overline{G_{n+1}}$};
                    \node[noeud] (Rb) at ({\NiveauB},{(2.5)*\InterFeuilles}) {$\overline{G_n}$};
                    \node[feuille] (Rba) at ({\NiveauC},{(2)*\InterFeuilles}) {$G_{n+1}$};
                    \node[feuille] (Rbb) at ({\NiveauC},{(3)*\InterFeuilles}) {$\overline{G_{n+1}}$};
                    % Arcs (MODIFIABLES : Styles)
                    \draw[fleche] (R)--(Ra) node[etiquette] {$p_n$};
                    \draw[fleche] (Ra)--(Raa) node[etiquette] {$\cdots$};
                    \draw[fleche] (Ra)--(Rab) node[etiquette] {$\cdots$};
                    \draw[fleche] (R)--(Rb) node[etiquette] {$1-p_n$};
                    \draw[fleche] (Rb)--(Rba) node[etiquette] {$\cdots$};
                    \draw[fleche] (Rb)--(Rbb) node[etiquette] {$\cdots$};
               \end{tikzpicture}
          \end{extern}
     \end{center}
     %:-+-+-+-+- Fin
     \item
     Montrer que, pour tout entier naturel $n \geqslant 1$ :
     \[ p_{n+1}=0,3p_n+0,4. \]
     \item
     On considère la suite $(u_n)$ définie, pour tout entier naturel $n$ supérieur ou égal à $1$, par :
     \[ u_n=p_n-\dfrac{4}{7}. \]
     \par
     \begin{enumerate}[label=\alph*.]
          \item
          Montrer que la suite $(u_n)$ est une suite géométrique dont on précisera la raison et le premier terme $u_1$.
          \item
          En déduire que, pour tout entier naturel $n$ supérieur ou égal à $1$ :
          \[ p_n=\dfrac{4}{7} - \dfrac{1}{14} \times (0,3)^{n-1}. \]
          \par
     \end{enumerate}
     \item
     Déterminer la limite de la suite $(p_n)$.
     \par
     Interpréter ce résultat dans le contexte de l'exercice.
     \par
\end{enumerate}
\begin{corrige}
     \begin{enumerate}
          \item
          D'après l'énoncé :
          \par
          \begin{itemize}
               \item
               \og Lorsqu'il gagne une partie, la probabilité qu'il gagne la suivante est 0,7 \fg{} : donc $\ {p_{G_n}(G_{n+1})=0,7}$.
               \item
               \og Si, par contre, il perd une partie, la probabilité qu'il gagne la suivante est seulement 0,4 \fg{} ; donc $\ {p_{\overline{G_n}}(G_{n+1})=0,4}$.
               \par
          \end{itemize}
          \par
          La somme des probabilités inscrites sur les branches partant d'un même nœud est égale à 1 ; on peut donc compléter l'arbre comme suit :
          \par
          %:-+-+-+- Engendré par : http://math.et.info.free.fr/TikZ/Arbre/
          \begin{center}
               \begin{extern}%width="340" alt="Arbre de probabilités complété"
                    % Racine à Gauche, développement vers la droite
                    \begin{tikzpicture}[xscale=1,yscale=1]
                         % Styles (MODIFIABLES)
                         \tikzstyle{fleche}=[-,>=latex,thick]
                         \tikzstyle{noeud}=[fill=white,circle,draw]
                         \tikzstyle{feuille}=[fill=white,circle,draw]
                         \tikzstyle{etiquette}=[midway,fill=white]
                         % Dimensions (MODIFIABLES)
                         \def\DistanceInterNiveaux{3}
                         \def\DistanceInterFeuilles{2}
                         % Dimensions calculées (NON MODIFIABLES)
                         \def\NiveauA{(0)*\DistanceInterNiveaux}
                         \def\NiveauB{(1.5)*\DistanceInterNiveaux}
                         \def\NiveauC{(2.5)*\DistanceInterNiveaux}
                         \def\InterFeuilles{(-1)*\DistanceInterFeuilles}
                         % Noeuds (MODIFIABLES : Styles et Coefficients d'InterFeuilles)
                         \node[noeud] (R) at ({\NiveauA},{(1.5)*\InterFeuilles}) {$\ $};
                         \node[noeud] (Ra) at ({\NiveauB},{(0.5)*\InterFeuilles}) {$G_n$};
                         \node[feuille] (Raa) at ({\NiveauC},{(0)*\InterFeuilles}) {$G_{n+1}$};
                         \node[feuille] (Rab) at ({\NiveauC},{(1)*\InterFeuilles}) {$\overline{G_{n+1}}$};
                         \node[noeud] (Rb) at ({\NiveauB},{(2.5)*\InterFeuilles}) {$\overline{G_n}$};
                         \node[feuille] (Rba) at ({\NiveauC},{(2)*\InterFeuilles}) {$G_{n+1}$};
                         \node[feuille] (Rbb) at ({\NiveauC},{(3)*\InterFeuilles}) {$\overline{G_{n+1}}$};
                         % Arcs (MODIFIABLES : Styles)
                         \draw[fleche] (R)--(Ra) node[etiquette] {$p_n$};
                         \draw[fleche] (Ra)--(Raa) node[etiquette] {\textcolor{red}{$0,7$}};
                         \draw[fleche] (Ra)--(Rab) node[etiquette] {\textcolor{red}{$0,3$}};
                         \draw[fleche] (R)--(Rb) node[etiquette] {$1-p_n$};
                         \draw[fleche] (Rb)--(Rba) node[etiquette] {\textcolor{red}{$0,4$}};
                         \draw[fleche] (Rb)--(Rbb) node[etiquette] {\textcolor{red}{$0,6$}};
                    \end{tikzpicture}
               \end{extern}
          \end{center}
          %:-+-+-+-+- Fin
          \item
          $p_{n+1}$ représente $p(G_{n+1})$. D'après la formule des probabilités totales :
          \par
          $p_{n+1} = p(G_n) \times p_{G_n}(G_{n+1}) + p(\overline{G_n}) \times  p_{\overline{G_n}}(G_{n+1})$ \\
          $\phantom{p_{n+1}} =  p_n \times 0,7 + (1-p_n) \times  0,4 $\\
          $\phantom{p_{n+1}} = 0,7p_n + 0,4-0,4p_n $\\
          $\phantom{p_{n+1}} = 0,3p_n+0,4$
          \par
          \cadre{bleu}{Remarque}{
               La suite $(p_n)$ vérifie une relation de récurrence de la forme : $p_{n+1}=ap_n+b$ ; c'est donc une suite arithmético-géométrique.
          }
          \item %3
          \textit{Pour plus de détails sur la méthode employée dans cette question se reporter à la \hyperlink{suite-ag-pap}{page \pageref*{suite-ag-pap}} : \og \'Etude d'une suite arithmético-géométrique étape par étape \fg{}.}
          \begin{enumerate}[label=\alph*.]
               \item %a
               Pour tout entier naturel $n$ strictement positif : $u_n=p_n-\dfrac{4}{7}$ ; par conséquent :
               \par
               $u_{n+1} = p_{n+1}-\dfrac{4}{7}$ \\
               $\phantom{u_{n+1}} = (0,3p_n+0,4)-\dfrac{4}{7}$ \\
               $\phantom{u_{n+1}}= 0,3p_n+\dfrac{2,8}{7}-\dfrac{4}{7}$ \\
               $\phantom{u_{n+1}}= 0,3p_n-\dfrac{1,2}{7}$.
               \par
               Or $u_n=p_n-\dfrac{4}{7}$ donc $p_n=u_n+\dfrac{4}{7}$ ; on obtient donc :
               \par
               $u_{n+1}  = 0,3\left(u_n+\dfrac{4}{7}\right)-\dfrac{1,2}{7}$\\
               $\phantom{u_{n+1}}= 0,3u_n+\dfrac{1,2}{7}-\dfrac{1,2}{7}$\\
               $\phantom{u_{n+1}} = 0,3u_n.$
               \par
               Par ailleurs, ${u_1=p_1-\dfrac{4}{7}=\dfrac{1}{2}-\dfrac{4}{7}=-\dfrac{1}{14}}$.
               \par
               La suite $(u_n)$ est donc une suite géométrique de premier terme ${u_1=-\dfrac{1}{14}}$ et de raison ${q=0,3}$.
               \item %b
               On en déduit que pour tout entier naturel $n$ strictement positif:
               \par
               $u_n=u_1q^{n-1}=-\dfrac{1}{14} \times 0,3^{n-1}$.
               \par
               \cadre{rouge}{À retenir}{
                    Pour une suite \textbf{géométrique} $(u_n)$ de premier terme $u_0$ et de raison $q$, le $n$-ième terme vaut :
                    \[u_{n}=u_0 \times q^n.\]
                    \par
                    Pour une suite \textbf{géométrique} $(u_n)$ de premier terme $u_1$ et de raison $q$, le $n$-ième terme vaut :
                    \[u_{n}=u_1 \times q^{n-1}.\]
                    \par
                    Plus généralement, pour une suite \textbf{géométrique} $(u_n)$ de raison $q$ dont on connait le terme $u_p$, le $n$-ième terme vaut :
                    \[u_{n}=u_p \times q^{n-p}.\]
               }
               \par
               En utilisant la relation $p_n=u_n+\dfrac{4}{7}$, on obtient :
               \[ p_n=\dfrac{4}{7} - \dfrac{1}{14} \times 0,3^{n-1}. \]
\medskip
          \end{enumerate}
          \item
          Comme $0 \leqslant 0,3 < 1$, alors $\lim\limits_{n \rightarrow +\infty}0,3^n=0$.
          \par
          $0,3^{n-1} = \dfrac{0,3^n}{0,3}$ donc on a également  $\lim\limits_{n \rightarrow +\infty}0,3^{n-1}=0$.
          \par
          Par conséquent $\lim\limits_{n \rightarrow +\infty} \left(\dfrac{1}{14} \times 0,3^{n-1}\right)=0$ et :
          \[ \lim\limits_{n \rightarrow +\infty} \left(\dfrac{4}{7} - \dfrac{1}{14} \times 0,3^{n-1}\right)=\dfrac{4}{7}. \]
          \par
          La suite $(p_n)$ converge donc vers $\dfrac{4}{7}$.
          \par
          Lorsque Loïc a joué beaucoup de parties, sa probabilité de gagner une partie est proche de $\dfrac{4}{7}$.
          \par
          \cadre{rouge}{À retenir}{
               Soit $q$ un nombre réel positif ou nul.
               \par
               \begin{itemize}
                    \item %
                    Si $\bm{0 \leqslant q < 1}$, alors $\lim\limits_{n \rightarrow +\infty}q^n=\bm{0}$.
                    \item %
                    Si $\bm{q > 1}$, alors $\lim\limits_{n \rightarrow +\infty}q^n=\bm{+\infty}$.
               \end{itemize}
          }
     \end{enumerate}
\end{corrige}

\end{document}