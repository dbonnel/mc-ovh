\documentclass[a4paper]{article}

%================================================================================================================================
%
% Packages
%
%================================================================================================================================

\usepackage[T1]{fontenc} 	% pour caractères accentués
\usepackage[utf8]{inputenc}  % encodage utf8
\usepackage[french]{babel}	% langue : français
\usepackage{fourier}			% caractères plus lisibles
\usepackage[dvipsnames]{xcolor} % couleurs
\usepackage{fancyhdr}		% réglage header footer
\usepackage{needspace}		% empêcher sauts de page mal placés
\usepackage{graphicx}		% pour inclure des graphiques
\usepackage{enumitem,cprotect}		% personnalise les listes d'items (nécessaire pour ol, al ...)
\usepackage{hyperref}		% Liens hypertexte
\usepackage{pstricks,pst-all,pst-node,pstricks-add,pst-math,pst-plot,pst-tree,pst-eucl} % pstricks
\usepackage[a4paper,includeheadfoot,top=2cm,left=3cm, bottom=2cm,right=3cm]{geometry} % marges etc.
\usepackage{comment}			% commentaires multilignes
\usepackage{amsmath,environ} % maths (matrices, etc.)
\usepackage{amssymb,makeidx}
\usepackage{bm}				% bold maths
\usepackage{tabularx}		% tableaux
\usepackage{colortbl}		% tableaux en couleur
\usepackage{fontawesome}		% Fontawesome
\usepackage{environ}			% environment with command
\usepackage{fp}				% calculs pour ps-tricks
\usepackage{multido}			% pour ps tricks
\usepackage[np]{numprint}	% formattage nombre
\usepackage{tikz,tkz-tab} 			% package principal TikZ
\usepackage{pgfplots}   % axes
\usepackage{mathrsfs}    % cursives
\usepackage{calc}			% calcul taille boites
\usepackage[scaled=0.875]{helvet} % font sans serif
\usepackage{svg} % svg
\usepackage{scrextend} % local margin
\usepackage{scratch} %scratch
\usepackage{multicol} % colonnes
%\usepackage{infix-RPN,pst-func} % formule en notation polanaise inversée
\usepackage{listings}

%================================================================================================================================
%
% Réglages de base
%
%================================================================================================================================

\lstset{
language=Python,   % R code
literate=
{á}{{\'a}}1
{à}{{\`a}}1
{ã}{{\~a}}1
{é}{{\'e}}1
{è}{{\`e}}1
{ê}{{\^e}}1
{í}{{\'i}}1
{ó}{{\'o}}1
{õ}{{\~o}}1
{ú}{{\'u}}1
{ü}{{\"u}}1
{ç}{{\c{c}}}1
{~}{{ }}1
}


\definecolor{codegreen}{rgb}{0,0.6,0}
\definecolor{codegray}{rgb}{0.5,0.5,0.5}
\definecolor{codepurple}{rgb}{0.58,0,0.82}
\definecolor{backcolour}{rgb}{0.95,0.95,0.92}

\lstdefinestyle{mystyle}{
    backgroundcolor=\color{backcolour},   
    commentstyle=\color{codegreen},
    keywordstyle=\color{magenta},
    numberstyle=\tiny\color{codegray},
    stringstyle=\color{codepurple},
    basicstyle=\ttfamily\footnotesize,
    breakatwhitespace=false,         
    breaklines=true,                 
    captionpos=b,                    
    keepspaces=true,                 
    numbers=left,                    
xleftmargin=2em,
framexleftmargin=2em,            
    showspaces=false,                
    showstringspaces=false,
    showtabs=false,                  
    tabsize=2,
    upquote=true
}

\lstset{style=mystyle}


\lstset{style=mystyle}
\newcommand{\imgdir}{C:/laragon/www/newmc/assets/imgsvg/}
\newcommand{\imgsvgdir}{C:/laragon/www/newmc/assets/imgsvg/}

\definecolor{mcgris}{RGB}{220, 220, 220}% ancien~; pour compatibilité
\definecolor{mcbleu}{RGB}{52, 152, 219}
\definecolor{mcvert}{RGB}{125, 194, 70}
\definecolor{mcmauve}{RGB}{154, 0, 215}
\definecolor{mcorange}{RGB}{255, 96, 0}
\definecolor{mcturquoise}{RGB}{0, 153, 153}
\definecolor{mcrouge}{RGB}{255, 0, 0}
\definecolor{mclightvert}{RGB}{205, 234, 190}

\definecolor{gris}{RGB}{220, 220, 220}
\definecolor{bleu}{RGB}{52, 152, 219}
\definecolor{vert}{RGB}{125, 194, 70}
\definecolor{mauve}{RGB}{154, 0, 215}
\definecolor{orange}{RGB}{255, 96, 0}
\definecolor{turquoise}{RGB}{0, 153, 153}
\definecolor{rouge}{RGB}{255, 0, 0}
\definecolor{lightvert}{RGB}{205, 234, 190}
\setitemize[0]{label=\color{lightvert}  $\bullet$}

\pagestyle{fancy}
\renewcommand{\headrulewidth}{0.2pt}
\fancyhead[L]{maths-cours.fr}
\fancyhead[R]{\thepage}
\renewcommand{\footrulewidth}{0.2pt}
\fancyfoot[C]{}

\newcolumntype{C}{>{\centering\arraybackslash}X}
\newcolumntype{s}{>{\hsize=.35\hsize\arraybackslash}X}

\setlength{\parindent}{0pt}		 
\setlength{\parskip}{3mm}
\setlength{\headheight}{1cm}

\def\ebook{ebook}
\def\book{book}
\def\web{web}
\def\type{web}

\newcommand{\vect}[1]{\overrightarrow{\,\mathstrut#1\,}}

\def\Oij{$\left(\text{O}~;~\vect{\imath},~\vect{\jmath}\right)$}
\def\Oijk{$\left(\text{O}~;~\vect{\imath},~\vect{\jmath},~\vect{k}\right)$}
\def\Ouv{$\left(\text{O}~;~\vect{u},~\vect{v}\right)$}

\hypersetup{breaklinks=true, colorlinks = true, linkcolor = OliveGreen, urlcolor = OliveGreen, citecolor = OliveGreen, pdfauthor={Didier BONNEL - https://www.maths-cours.fr} } % supprime les bordures autour des liens

\renewcommand{\arg}[0]{\text{arg}}

\everymath{\displaystyle}

%================================================================================================================================
%
% Macros - Commandes
%
%================================================================================================================================

\newcommand\meta[2]{    			% Utilisé pour créer le post HTML.
	\def\titre{titre}
	\def\url{url}
	\def\arg{#1}
	\ifx\titre\arg
		\newcommand\maintitle{#2}
		\fancyhead[L]{#2}
		{\Large\sffamily \MakeUppercase{#2}}
		\vspace{1mm}\textcolor{mcvert}{\hrule}
	\fi 
	\ifx\url\arg
		\fancyfoot[L]{\href{https://www.maths-cours.fr#2}{\black \footnotesize{https://www.maths-cours.fr#2}}}
	\fi 
}


\newcommand\TitreC[1]{    		% Titre centré
     \needspace{3\baselineskip}
     \begin{center}\textbf{#1}\end{center}
}

\newcommand\newpar{    		% paragraphe
     \par
}

\newcommand\nosp {    		% commande vide (pas d'espace)
}
\newcommand{\id}[1]{} %ignore

\newcommand\boite[2]{				% Boite simple sans titre
	\vspace{5mm}
	\setlength{\fboxrule}{0.2mm}
	\setlength{\fboxsep}{5mm}	
	\fcolorbox{#1}{#1!3}{\makebox[\linewidth-2\fboxrule-2\fboxsep]{
  		\begin{minipage}[t]{\linewidth-2\fboxrule-4\fboxsep}\setlength{\parskip}{3mm}
  			 #2
  		\end{minipage}
	}}
	\vspace{5mm}
}

\newcommand\CBox[4]{				% Boites
	\vspace{5mm}
	\setlength{\fboxrule}{0.2mm}
	\setlength{\fboxsep}{5mm}
	
	\fcolorbox{#1}{#1!3}{\makebox[\linewidth-2\fboxrule-2\fboxsep]{
		\begin{minipage}[t]{1cm}\setlength{\parskip}{3mm}
	  		\textcolor{#1}{\LARGE{#2}}    
 	 	\end{minipage}  
  		\begin{minipage}[t]{\linewidth-2\fboxrule-4\fboxsep}\setlength{\parskip}{3mm}
			\raisebox{1.2mm}{\normalsize\sffamily{\textcolor{#1}{#3}}}						
  			 #4
  		\end{minipage}
	}}
	\vspace{5mm}
}

\newcommand\cadre[3]{				% Boites convertible html
	\par
	\vspace{2mm}
	\setlength{\fboxrule}{0.1mm}
	\setlength{\fboxsep}{5mm}
	\fcolorbox{#1}{white}{\makebox[\linewidth-2\fboxrule-2\fboxsep]{
  		\begin{minipage}[t]{\linewidth-2\fboxrule-4\fboxsep}\setlength{\parskip}{3mm}
			\raisebox{-2.5mm}{\sffamily \small{\textcolor{#1}{\MakeUppercase{#2}}}}		
			\par		
  			 #3
 	 		\end{minipage}
	}}
		\vspace{2mm}
	\par
}

\newcommand\bloc[3]{				% Boites convertible html sans bordure
     \needspace{2\baselineskip}
     {\sffamily \small{\textcolor{#1}{\MakeUppercase{#2}}}}    
		\par		
  			 #3
		\par
}

\newcommand\CHelp[1]{
     \CBox{Plum}{\faInfoCircle}{À RETENIR}{#1}
}

\newcommand\CUp[1]{
     \CBox{NavyBlue}{\faThumbsOUp}{EN PRATIQUE}{#1}
}

\newcommand\CInfo[1]{
     \CBox{Sepia}{\faArrowCircleRight}{REMARQUE}{#1}
}

\newcommand\CRedac[1]{
     \CBox{PineGreen}{\faEdit}{BIEN R\'EDIGER}{#1}
}

\newcommand\CError[1]{
     \CBox{Red}{\faExclamationTriangle}{ATTENTION}{#1}
}

\newcommand\TitreExo[2]{
\needspace{4\baselineskip}
 {\sffamily\large EXERCICE #1\ (\emph{#2 points})}
\vspace{5mm}
}

\newcommand\img[2]{
          \includegraphics[width=#2\paperwidth]{\imgdir#1}
}

\newcommand\imgsvg[2]{
       \begin{center}   \includegraphics[width=#2\paperwidth]{\imgsvgdir#1} \end{center}
}


\newcommand\Lien[2]{
     \href{#1}{#2 \tiny \faExternalLink}
}
\newcommand\mcLien[2]{
     \href{https~://www.maths-cours.fr/#1}{#2 \tiny \faExternalLink}
}

\newcommand{\euro}{\eurologo{}}

%================================================================================================================================
%
% Macros - Environement
%
%================================================================================================================================

\newenvironment{tex}{ %
}
{%
}

\newenvironment{indente}{ %
	\setlength\parindent{10mm}
}

{
	\setlength\parindent{0mm}
}

\newenvironment{corrige}{%
     \needspace{3\baselineskip}
     \medskip
     \textbf{\textsc{Corrigé}}
     \medskip
}
{
}

\newenvironment{extern}{%
     \begin{center}
     }
     {
     \end{center}
}

\NewEnviron{code}{%
	\par
     \boite{gray}{\texttt{%
     \BODY
     }}
     \par
}

\newenvironment{vbloc}{% boite sans cadre empeche saut de page
     \begin{minipage}[t]{\linewidth}
     }
     {
     \end{minipage}
}
\NewEnviron{h2}{%
    \needspace{3\baselineskip}
    \vspace{0.6cm}
	\noindent \MakeUppercase{\sffamily \large \BODY}
	\vspace{1mm}\textcolor{mcgris}{\hrule}\vspace{0.4cm}
	\par
}{}

\NewEnviron{h3}{%
    \needspace{3\baselineskip}
	\vspace{5mm}
	\textsc{\BODY}
	\par
}

\NewEnviron{margeneg}{ %
\begin{addmargin}[-1cm]{0cm}
\BODY
\end{addmargin}
}

\NewEnviron{html}{%
}

\begin{document}
\meta{url}{/exercices/calcul-de-cos15/}
\meta{pid}{4543}
\meta{titre}{Calcul de cos(15°)}
\meta{type}{exercices}
%
Sur la figure ci-dessous, $A$ et $I$ sont les points de coordonnées respectives $(-1;0)$ et $(1;0)$, $B$ est le point du cercle de centre $ O $ et $H$ le pied de la hauteur issue de $B$ du triangle $OBA$.
<img src="/wp-content/uploads/cos15.svg" alt="" class="aligncenter" style="width:450px;"/>

\begin{center}
\imgsvg{cos15}{0.3}% alt="Calcul de cos(15°)" style="width:35rem"
\end{center}
\begin{enumerate}
     \item
     Donner, en degré, la mesure de l'angle $ \widehat{BOI} $ puis de l'angle $\widehat{AOB}$.
     \item
     Que peut-on dire du triangle $AOB$ ? En déduire la mesure, en degré, de l'angle $\widehat{BAH}$.
     \item
     Calculer les valeurs exactes de $OH$, $BH$ puis $AB$.
     \item
     En déduire que $\cos(\widehat{BAH}) = \frac{\sqrt{2+\sqrt{3}}}{2}$.
     \item
     Calculer $( \sqrt{2} +  \sqrt{6})^2  $.
     \item
     Déduire des questions précédentes que $\cos(15^{\circ})=  \frac{ \sqrt{2} +  \sqrt{6}}{4}  $
\end{enumerate}
\begin{corrige}
     \begin{enumerate}
          \item
          Dire que $B$ est le point du cercle trigonométrique \textit{image de $\frac{\pi}{6}$}, signifie que l'angle $\widehat{BOI} $ mesure $ \frac{ \pi }{6} $ radians soit $30$ degrés.
          \par
          Les points $A, O$ et $I$ étant alignés (sur l'axe des abscisses), les angles $\widehat{AOB}$ et $\widehat{BOI}$ sont supplémentaires.
          \par
          Par conséquent :
          \par
          $\widehat{AOB}=180^\circ-30^\circ=150^\circ$
          \item
          Les côtés $[OA]$ et $[OB]$ sont des rayons du cercle trigonométrique donc $OA=OB=1$. Le triangle $AOB$ est donc isocèle.
          \par
          La somme des mesures des angles d'un triangle vaut $180^\circ$, par conséquent :
          \par
          $\widehat{OAB}+\widehat{ABO}+\widehat{BOA}=180^\circ$
          \par
          $\widehat{OAB}+\widehat{ABO} = 180^\circ-150^\circ = 30^\circ$
          \par
          et, comme le triangle $AOB$ est isocèle, les angles $\widehat{OAB}$ et $\widehat{ABO}$ ont la même mesure :
          \par
          $\widehat{OAB} = \widehat{ABO} = 15^\circ$
          \item
          Dans le triangle $OBH$ rectangle en $H$ :
          \par
          $\cos(\widehat{BOH})= \frac{OH}{OB} $
          \par
          $\cos(30^\circ)= \frac{OH}{1} $
          \par
          $OH=\cos(30^\circ)=\frac{\sqrt{3}}{2}$
          \par
          $\sin(\widehat{BOH})= \frac{BH}{OB}$
          \par
          $\sin(30^\circ)= \frac{BH}{1} $
          \par
          $BH=\sin(30^\circ)=\frac{1}{2}$
          \par
          $AH=AO+OH=1+\frac{\sqrt{3}}{2}$
          \par
          D'après le théorème de Pythagore :
          \par
          $AB^2=AH^2+HB^2$
          \par
          $AB^2=\left(1+ \frac{ \sqrt{3} }{2} \right)^2+\left( \frac{1}{2} \right)^2$
          \begin{note}Pour tous réels $a$ et $b$ :
               \par
               $(a+b)^2$$=a^2+2ab+b^2$
          \end{note}
          $AB^2=1 + 2 \times \frac{ \sqrt{3} }{2} +  \frac{3}{4} + \frac{1}{4} $
          \par
          $AB^2=2+  \sqrt{3}$
          \par
          $AB=\sqrt{2+  \sqrt{3}}$
          \item
          Dans le triangle $ABH$ rectangle en $H$ :
          \par
          $\cos(\widehat{BAH})= \frac{AH}{AB} $
          \par
          $\cos(\widehat{BAH})= \frac{1+\frac{\sqrt{3}}{2}}{\sqrt{2+  \sqrt{3}} } $
          \begin{note}Pour $b \neq 0$ et $c \neq 0$:
               \par
               $\frac{\frac{a}{b}}{c}=\frac{a}{b}\times \frac{1}{c}=\frac{a}{bc}$
          \end{note}
          $\cos(\widehat{BAH})= \frac{\frac{2+\sqrt{3}}{2}}{\sqrt{2+  \sqrt{3}} } $
          \par
          $\cos(\widehat{BAH})= \frac{2+\sqrt{3}} {2\sqrt{2+  \sqrt{3}} } $
          \begin{note}Pour tout réel $a$ \textbf{positif} ou nul $\sqrt{a^2}=a$ et  $(\sqrt{a})^2=a$
          \end{note}
          ~
          \par
          Or $2+  \sqrt{3} =\left(\sqrt{2+  \sqrt{3}} \right)^2$ donc :
          \par
          $\cos(\widehat{BAH})= \frac{\left(\sqrt{2+  \sqrt{3}} \right)^2} {2\sqrt{2+  \sqrt{3}} } $
          \par
          $\cos(\widehat{BAH})= \frac{\sqrt{2+  \sqrt{3} }} {2} $
          \par
          après simplification par $\sqrt{2+  \sqrt{3}}$.
          \item
          $( \sqrt{2} +  \sqrt{6})^2 =2+2\times \sqrt{2} \times \sqrt{6} + 6$
          \par
          $( \sqrt{2} +  \sqrt{6})^2 =8+2 \sqrt{12} $
          \par
          $( \sqrt{2} +  \sqrt{6})^2 =8+4 \sqrt{3} $
          \par
          On peut mettre $4$ en facteur (utile pour la suite...) :
          \par
          $( \sqrt{2} +  \sqrt{6})^2 =4(2+ \sqrt{3}) $
          \item
          On déduit de la question précédente que :
          \par
          $2+ \sqrt{3}=\frac{( \sqrt{2} +  \sqrt{6})^2} {4} $
          \par
          Chaque membre de l'égalité étant positif, on en déduit, en prenant la racine carrée de chaque membre :
          \par
          $\sqrt{2+ \sqrt{3}}=\sqrt{\frac{( \sqrt{2} +  \sqrt{6})^2} {4}}$
          \par
          $\sqrt{2+ \sqrt{3}}=\frac{ \sqrt{2} +  \sqrt{6}} {2}$
          \par
          D'après la question \textbf{2.}, $\widehat{BAH}=15^\circ$ et d'après la question \textbf{4.}, $\cos(\widehat{BAH})= \frac{\sqrt{2+  \sqrt{3} }} {2} $, par conséquent :
          \par
          $\cos(15^\circ)= \frac{\sqrt{2+  \sqrt{3} }} {2} = \frac{\frac{ \sqrt{2} +  \sqrt{6}} {2} } {2} $
          \par
          $\cos(15^{\circ})=  \frac{ \sqrt{2} +  \sqrt{6}}{4} $
     \end{enumerate}
} 						\end{corrige}

\end{document}