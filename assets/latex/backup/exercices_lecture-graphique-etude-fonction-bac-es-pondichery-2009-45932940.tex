\documentclass[a4paper]{article}

%================================================================================================================================
%
% Packages
%
%================================================================================================================================

\usepackage[T1]{fontenc} 	% pour caractères accentués
\usepackage[utf8]{inputenc}  % encodage utf8
\usepackage[french]{babel}	% langue : français
\usepackage{fourier}			% caractères plus lisibles
\usepackage[dvipsnames]{xcolor} % couleurs
\usepackage{fancyhdr}		% réglage header footer
\usepackage{needspace}		% empêcher sauts de page mal placés
\usepackage{graphicx}		% pour inclure des graphiques
\usepackage{enumitem,cprotect}		% personnalise les listes d'items (nécessaire pour ol, al ...)
\usepackage{hyperref}		% Liens hypertexte
\usepackage{pstricks,pst-all,pst-node,pstricks-add,pst-math,pst-plot,pst-tree,pst-eucl} % pstricks
\usepackage[a4paper,includeheadfoot,top=2cm,left=3cm, bottom=2cm,right=3cm]{geometry} % marges etc.
\usepackage{comment}			% commentaires multilignes
\usepackage{amsmath,environ} % maths (matrices, etc.)
\usepackage{amssymb,makeidx}
\usepackage{bm}				% bold maths
\usepackage{tabularx}		% tableaux
\usepackage{colortbl}		% tableaux en couleur
\usepackage{fontawesome}		% Fontawesome
\usepackage{environ}			% environment with command
\usepackage{fp}				% calculs pour ps-tricks
\usepackage{multido}			% pour ps tricks
\usepackage[np]{numprint}	% formattage nombre
\usepackage{tikz,tkz-tab} 			% package principal TikZ
\usepackage{pgfplots}   % axes
\usepackage{mathrsfs}    % cursives
\usepackage{calc}			% calcul taille boites
\usepackage[scaled=0.875]{helvet} % font sans serif
\usepackage{svg} % svg
\usepackage{scrextend} % local margin
\usepackage{scratch} %scratch
\usepackage{multicol} % colonnes
%\usepackage{infix-RPN,pst-func} % formule en notation polanaise inversée
\usepackage{listings}

%================================================================================================================================
%
% Réglages de base
%
%================================================================================================================================

\lstset{
language=Python,   % R code
literate=
{á}{{\'a}}1
{à}{{\`a}}1
{ã}{{\~a}}1
{é}{{\'e}}1
{è}{{\`e}}1
{ê}{{\^e}}1
{í}{{\'i}}1
{ó}{{\'o}}1
{õ}{{\~o}}1
{ú}{{\'u}}1
{ü}{{\"u}}1
{ç}{{\c{c}}}1
{~}{{ }}1
}


\definecolor{codegreen}{rgb}{0,0.6,0}
\definecolor{codegray}{rgb}{0.5,0.5,0.5}
\definecolor{codepurple}{rgb}{0.58,0,0.82}
\definecolor{backcolour}{rgb}{0.95,0.95,0.92}

\lstdefinestyle{mystyle}{
    backgroundcolor=\color{backcolour},   
    commentstyle=\color{codegreen},
    keywordstyle=\color{magenta},
    numberstyle=\tiny\color{codegray},
    stringstyle=\color{codepurple},
    basicstyle=\ttfamily\footnotesize,
    breakatwhitespace=false,         
    breaklines=true,                 
    captionpos=b,                    
    keepspaces=true,                 
    numbers=left,                    
xleftmargin=2em,
framexleftmargin=2em,            
    showspaces=false,                
    showstringspaces=false,
    showtabs=false,                  
    tabsize=2,
    upquote=true
}

\lstset{style=mystyle}


\lstset{style=mystyle}
\newcommand{\imgdir}{C:/laragon/www/newmc/assets/imgsvg/}
\newcommand{\imgsvgdir}{C:/laragon/www/newmc/assets/imgsvg/}

\definecolor{mcgris}{RGB}{220, 220, 220}% ancien~; pour compatibilité
\definecolor{mcbleu}{RGB}{52, 152, 219}
\definecolor{mcvert}{RGB}{125, 194, 70}
\definecolor{mcmauve}{RGB}{154, 0, 215}
\definecolor{mcorange}{RGB}{255, 96, 0}
\definecolor{mcturquoise}{RGB}{0, 153, 153}
\definecolor{mcrouge}{RGB}{255, 0, 0}
\definecolor{mclightvert}{RGB}{205, 234, 190}

\definecolor{gris}{RGB}{220, 220, 220}
\definecolor{bleu}{RGB}{52, 152, 219}
\definecolor{vert}{RGB}{125, 194, 70}
\definecolor{mauve}{RGB}{154, 0, 215}
\definecolor{orange}{RGB}{255, 96, 0}
\definecolor{turquoise}{RGB}{0, 153, 153}
\definecolor{rouge}{RGB}{255, 0, 0}
\definecolor{lightvert}{RGB}{205, 234, 190}
\setitemize[0]{label=\color{lightvert}  $\bullet$}

\pagestyle{fancy}
\renewcommand{\headrulewidth}{0.2pt}
\fancyhead[L]{maths-cours.fr}
\fancyhead[R]{\thepage}
\renewcommand{\footrulewidth}{0.2pt}
\fancyfoot[C]{}

\newcolumntype{C}{>{\centering\arraybackslash}X}
\newcolumntype{s}{>{\hsize=.35\hsize\arraybackslash}X}

\setlength{\parindent}{0pt}		 
\setlength{\parskip}{3mm}
\setlength{\headheight}{1cm}

\def\ebook{ebook}
\def\book{book}
\def\web{web}
\def\type{web}

\newcommand{\vect}[1]{\overrightarrow{\,\mathstrut#1\,}}

\def\Oij{$\left(\text{O}~;~\vect{\imath},~\vect{\jmath}\right)$}
\def\Oijk{$\left(\text{O}~;~\vect{\imath},~\vect{\jmath},~\vect{k}\right)$}
\def\Ouv{$\left(\text{O}~;~\vect{u},~\vect{v}\right)$}

\hypersetup{breaklinks=true, colorlinks = true, linkcolor = OliveGreen, urlcolor = OliveGreen, citecolor = OliveGreen, pdfauthor={Didier BONNEL - https://www.maths-cours.fr} } % supprime les bordures autour des liens

\renewcommand{\arg}[0]{\text{arg}}

\everymath{\displaystyle}

%================================================================================================================================
%
% Macros - Commandes
%
%================================================================================================================================

\newcommand\meta[2]{    			% Utilisé pour créer le post HTML.
	\def\titre{titre}
	\def\url{url}
	\def\arg{#1}
	\ifx\titre\arg
		\newcommand\maintitle{#2}
		\fancyhead[L]{#2}
		{\Large\sffamily \MakeUppercase{#2}}
		\vspace{1mm}\textcolor{mcvert}{\hrule}
	\fi 
	\ifx\url\arg
		\fancyfoot[L]{\href{https://www.maths-cours.fr#2}{\black \footnotesize{https://www.maths-cours.fr#2}}}
	\fi 
}


\newcommand\TitreC[1]{    		% Titre centré
     \needspace{3\baselineskip}
     \begin{center}\textbf{#1}\end{center}
}

\newcommand\newpar{    		% paragraphe
     \par
}

\newcommand\nosp {    		% commande vide (pas d'espace)
}
\newcommand{\id}[1]{} %ignore

\newcommand\boite[2]{				% Boite simple sans titre
	\vspace{5mm}
	\setlength{\fboxrule}{0.2mm}
	\setlength{\fboxsep}{5mm}	
	\fcolorbox{#1}{#1!3}{\makebox[\linewidth-2\fboxrule-2\fboxsep]{
  		\begin{minipage}[t]{\linewidth-2\fboxrule-4\fboxsep}\setlength{\parskip}{3mm}
  			 #2
  		\end{minipage}
	}}
	\vspace{5mm}
}

\newcommand\CBox[4]{				% Boites
	\vspace{5mm}
	\setlength{\fboxrule}{0.2mm}
	\setlength{\fboxsep}{5mm}
	
	\fcolorbox{#1}{#1!3}{\makebox[\linewidth-2\fboxrule-2\fboxsep]{
		\begin{minipage}[t]{1cm}\setlength{\parskip}{3mm}
	  		\textcolor{#1}{\LARGE{#2}}    
 	 	\end{minipage}  
  		\begin{minipage}[t]{\linewidth-2\fboxrule-4\fboxsep}\setlength{\parskip}{3mm}
			\raisebox{1.2mm}{\normalsize\sffamily{\textcolor{#1}{#3}}}						
  			 #4
  		\end{minipage}
	}}
	\vspace{5mm}
}

\newcommand\cadre[3]{				% Boites convertible html
	\par
	\vspace{2mm}
	\setlength{\fboxrule}{0.1mm}
	\setlength{\fboxsep}{5mm}
	\fcolorbox{#1}{white}{\makebox[\linewidth-2\fboxrule-2\fboxsep]{
  		\begin{minipage}[t]{\linewidth-2\fboxrule-4\fboxsep}\setlength{\parskip}{3mm}
			\raisebox{-2.5mm}{\sffamily \small{\textcolor{#1}{\MakeUppercase{#2}}}}		
			\par		
  			 #3
 	 		\end{minipage}
	}}
		\vspace{2mm}
	\par
}

\newcommand\bloc[3]{				% Boites convertible html sans bordure
     \needspace{2\baselineskip}
     {\sffamily \small{\textcolor{#1}{\MakeUppercase{#2}}}}    
		\par		
  			 #3
		\par
}

\newcommand\CHelp[1]{
     \CBox{Plum}{\faInfoCircle}{À RETENIR}{#1}
}

\newcommand\CUp[1]{
     \CBox{NavyBlue}{\faThumbsOUp}{EN PRATIQUE}{#1}
}

\newcommand\CInfo[1]{
     \CBox{Sepia}{\faArrowCircleRight}{REMARQUE}{#1}
}

\newcommand\CRedac[1]{
     \CBox{PineGreen}{\faEdit}{BIEN R\'EDIGER}{#1}
}

\newcommand\CError[1]{
     \CBox{Red}{\faExclamationTriangle}{ATTENTION}{#1}
}

\newcommand\TitreExo[2]{
\needspace{4\baselineskip}
 {\sffamily\large EXERCICE #1\ (\emph{#2 points})}
\vspace{5mm}
}

\newcommand\img[2]{
          \includegraphics[width=#2\paperwidth]{\imgdir#1}
}

\newcommand\imgsvg[2]{
       \begin{center}   \includegraphics[width=#2\paperwidth]{\imgsvgdir#1} \end{center}
}


\newcommand\Lien[2]{
     \href{#1}{#2 \tiny \faExternalLink}
}
\newcommand\mcLien[2]{
     \href{https~://www.maths-cours.fr/#1}{#2 \tiny \faExternalLink}
}

\newcommand{\euro}{\eurologo{}}

%================================================================================================================================
%
% Macros - Environement
%
%================================================================================================================================

\newenvironment{tex}{ %
}
{%
}

\newenvironment{indente}{ %
	\setlength\parindent{10mm}
}

{
	\setlength\parindent{0mm}
}

\newenvironment{corrige}{%
     \needspace{3\baselineskip}
     \medskip
     \textbf{\textsc{Corrigé}}
     \medskip
}
{
}

\newenvironment{extern}{%
     \begin{center}
     }
     {
     \end{center}
}

\NewEnviron{code}{%
	\par
     \boite{gray}{\texttt{%
     \BODY
     }}
     \par
}

\newenvironment{vbloc}{% boite sans cadre empeche saut de page
     \begin{minipage}[t]{\linewidth}
     }
     {
     \end{minipage}
}
\NewEnviron{h2}{%
    \needspace{3\baselineskip}
    \vspace{0.6cm}
	\noindent \MakeUppercase{\sffamily \large \BODY}
	\vspace{1mm}\textcolor{mcgris}{\hrule}\vspace{0.4cm}
	\par
}{}

\NewEnviron{h3}{%
    \needspace{3\baselineskip}
	\vspace{5mm}
	\textsc{\BODY}
	\par
}

\NewEnviron{margeneg}{ %
\begin{addmargin}[-1cm]{0cm}
\BODY
\end{addmargin}
}

\NewEnviron{html}{%
}

\begin{document}
\meta{url}{/exercices/lecture-graphique-etude-fonction-bac-es-pondichery-2009/}
\meta{pid}{2758}
\meta{titre}{Lecture graphique Etude de fonction - Bac ES Pondichéry 2009}
\meta{type}{exercices}
%
\begin{h2}Exercice 3 \end{h2}
\textit{10 points - Commun à tous les candidats }
\textit{Les parties A et B de cet exercice sont indépendantes. }
\begin{h3}Partie A. ~ Lectures graphiques\end{h3}
La courbe $C$ ci-dessous représente, dans un repère orthonormé, une fonction $f$ définie et dérivable sur $\left]0 ; +\infty \right[$.
\par
On note $f^{\prime}$ la fonction dérivée de $f$
\par
La courbe $C$ passe par les points $A\left(e ; 0\right)$ et $B\left(1 ; -1\right)$.
\par
La courbe admet une tangente parallèle à l'axe des abscisses au point d'abscisse 1 et la tangente au point d'abscisse e passe par le point $D\left(0 ; -e\right)$.

\begin{center}
\imgsvg{Bac_ES_Pondichery_2009}{0.3}% alt="Bac_ES_Pondichery_2009" style="width:50rem"
\end{center}
\begin{enumerate}
     \item
     Déterminer une équation de la droite (AD).

\textit{Aucune justification n'est exigée pour les réponses à la question 2. }

\item
Par lectures graphiques:
\begin{enumerate}[label=\alph*.]
     \item
     Déterminer $f\left(1\right)$ et $f^{\prime}\left(1\right)$.
     \item
     Dresser le tableau de signes de $f$ sur $\left]0 ; 5\right]$.
     \item
     Dresser le tableau de signes de $f^{\prime}$ sur $\left]0 ; 5\right]$.
     \item
     Soit $F$ une primitive de $f$ sur $\left]0 ; +\infty \right[$. Déterminer les variations de $F$ sur $\left]0 ; 5\right]$.
     \item
     Encadrer par deux entiers consécutifs l'aire (en unités d'aire) du domaine délimité par l'axe des abscisses, la courbe $C$ et les droites d'équation $x=4$ et $x=5$.
\end{enumerate}
\end{enumerate}
\begin{h3}Partie B. ~ Étude de la fonction\end{h3}
La courbe $C$ de la partie A est la représentation graphique de la fonction $f$ définie sur $\left]0 ; +\infty \right[$ par $f\left(x\right)=x \left(\ln x -1\right)$.
\begin{enumerate}
     \item
     \begin{enumerate}[label=\alph*.]
          \item
          Déterminer la limite de $f$ en $+\infty $.
          \item
          Soit $h$ la fonction définie sur $\left]0 ; +\infty \right[$ par $h\left(x\right)=x\ln x$. On rappelle que $\lim\limits_{x\rightarrow 0}h\left(x\right)=0$.
          \par
          Déterminer la limite de $f$ en 0.
     \end{enumerate}
     \item
     \begin{enumerate}[label=\alph*.]
          \item
          Montrer que, pour tout $x$ de $\left]0 ; +\infty \right[$, on a : $f^{\prime}\left(x\right)=\ln x$.
          \item
          Étudier le signe de $f^{\prime}\left(x\right)$ sur $\left]0 ; +\infty \right[$ et en déduire le tableau de variation de $f$ sur $\left]0 ; +\infty \right[$.
     \end{enumerate}
     \item
     \begin{enumerate}[label=\alph*.]
          \item
          Démontrer que la fonction $H$ définie sur $\left]0 ; +\infty \right[$ par $H\left(x\right)=\frac{1}{2}x^{2} \ln x-\frac{1}{4} x^{2}$ est une primitive sur $\left]0 ; +\infty \right[$ de la fonction $h$ définie à la question 1.b).
          \item
          En déduire une primitive $F$ de $f$ et calculer $\int_{1}^{e}f\left(x\right)dx$.
          \item
          En déduire l'aire, en unités d'aire, de la partie du plan délimitée par $C$, l'axe des abscisses et les droites d'équation $x=1$ et $x=e$. On arrondira le résultat au dixième.
     \end{enumerate}
\end{enumerate}
\begin{corrige}
     \begin{h3}Partie A.\end{h3}
     \begin{enumerate}
          \item
          Le coefficient directeur de la droite $\left(AD\right)$ est :
          \par
          $a=\frac{y_{D}-y_{A}}{x_{D}-x_{A}}=\frac{-e-0}{0-e}=1$
          \par
          L'équation réduite de la droite $\left(AD\right)$ est donc de la forme $y=x+b$
          \par
          Les coordonnées de $D$ vérifient cette équation donc :
          \par
          $-e=0+b$
          \par
          L'équation réduite de la droite $\left(AD\right)$ est donc :
          \par
          $y=x-e$
          \item
          Par lectures graphiques:
          \begin{enumerate}
               \item
               $f\left(1\right)=-1$
               \par
               $f^{\prime}\left(1\right)=0$ (tangente parallèle à l'axe des abscisses)
               \item
               La courbe $C$ est au-dessous de l'axe des abscisses sur $\left]0; e\right[$ et au dessus sur $\left]e; 5\right]$ donc :
%##
% type=table; width=30; c1=20
%--
% x|   0   ~    \text{e}   ~   5
% f(x)|   ||       -                :0   +     ~
%--
\begin{center}
 \begin{extern}%style="width:30rem" alt="Exercice"
    \resizebox{11cm}{!}{
       \definecolor{dark}{gray}{0.1}
       \definecolor{light}{gray}{0.8}
       \tikzstyle{fleche}=[->,>=latex]
       \begin{tikzpicture}[scale=.8, line width=.5pt, dark]
       \def\width{.15}
       \def\height{.10}
       \draw (0, -10*\height) -- (64*\width, -10*\height);
       \draw (20*\width, 0*\height) -- (20*\width, -10*\height);
       \node (l0c0) at (10*\width,-5*\height) {$ x $};
       \node (l0c1) at (24*\width,-5*\height) {$ 0 $};
       \node (l0c2) at (33*\width,-5*\height) {$ ~ $};
       \node (l0c3) at (42*\width,-5*\height) {$ \text{e} $};
       \node (l0c4) at (51*\width,-5*\height) {$ ~ $};
       \node (l0c5) at (60*\width,-5*\height) {$ 5 $};
       \draw (0, -20*\height) -- (64*\width, -20*\height);
       \draw (20*\width, -10*\height) -- (20*\width, -20*\height);
       \node (l1c0) at (10*\width,-15*\height) {$ f(x) $};
       \draw[double distance=2pt] (24*\width, -10*\height) -- (24*\width, -20*\height);
       \node (l1c1) at (24*\width,-15*\height) {$ ~ $};
       \node (l1c2) at (33*\width,-15*\height) {$ - $};
       \draw[light] (42*\width, -10*\height) -- (42*\width, -20*\height);
       \node (l1c3) at (42*\width,-15*\height) {$ 0 $};
       \node (l1c4) at (51*\width,-15*\height) {$ + $};
       \node (l1c5) at (60*\width,-15*\height) {$ ~ $};
       \draw (0, 0) rectangle (64*\width, -20*\height);

       \end{tikzpicture}
      }
   \end{extern}
\end{center}
%##
<img src="/wp-content/uploads/t_97e31ee1256b5da9a72ce9d83fecbe7d.gif" alt="" class="aligncenter size-full  img-pc" />
               \item
               $f$ est décroissante sur $\left]0; 1\right]$ et croissante sur $\left[1;5\right]$ donc :
               <img src="/wp-content/uploads/t_b97b6c4b355b692825c0db7c2df598da.gif" alt="" class="aligncenter size-full  img-pc" />
               \item
               $F$ est une primitive de $f$ donc $F^{\prime}=f$. D'après la question \textbf{1.b.}, on obtient le tableau de variations suivant: :
               <img src="/wp-content/uploads/t_bc1a3dd7ddc7e73d356517baf4590a94.gif" alt="" class="aligncenter size-full  img-pc" />
               \item
               L'aire du domaine délimité par l'axe des abscisses, la courbe $C$ et les droites d'équation $x=4$ et $x=5$ est comprise entre 2 et 3 (il suffit de compter les carreaux!).
          \end{enumerate}
     \end{enumerate}
     \begin{h3}Partie B.\end{h3}
     \begin{enumerate}
          \item
          \begin{enumerate}[label=\alph*.]
               \item
               $\lim\limits_{x\rightarrow +\infty }x=+\infty $
               \par
               $\lim\limits_{x\rightarrow +\infty }\ln x-1=+\infty $
               \par
               donc en effectuant le produit des limites :
               \par
               $\lim\limits_{x\rightarrow +\infty }f\left(x\right)=+\infty $
               \item
               $f\left(x\right)=x\left(\ln x-1\right)=x\ln x-x=h\left(x\right)-x$
               \par
               $\lim\limits_{x\rightarrow 0}h\left(x\right)=0$
               \par
               $\lim\limits_{x\rightarrow 0}x=0$
               \par
               donc en effectuant la différence des limites :
               \par
               $\lim\limits_{x\rightarrow 0}f\left(x\right)=0$
               .
          \end{enumerate}
          \item
          \begin{enumerate}[label=\alph*.]
               \item
               Sur $\left]0; +\infty \right[$, $f$ est de la forme $uv$ avec $u\left(x\right)=x$ et $v\left(x\right)=\ln x-1$ donc :
               \par
               $f^{\prime}\left(x\right)=u^{\prime}\left(x\right)v\left(x\right)+u\left(x\right)v^{\prime}\left(x\right)=1\times \left(\ln x-1\right)+x\times \frac{1}{x}=\ln x$
               \item
               La fonction logarithme népérien est strictement négative sur $\left]0; 1\right[$ et  strictement positive sur $\left]1; +\infty \right[$.
               \par
               $f\left(1\right)=1\times \left(\ln1-1\right)=-1$
               \par
               Le tableau de variations de $f$ sur $\left]0 ; +\infty \right[$ est donc :
               <img src="/wp-content/uploads/t_6b461ea5b9ecf7ef9c477f36fb335d4e.gif" alt="" class="aligncenter size-full  img-pc" />
          \end{enumerate}
          \item
          \begin{enumerate}[label=\alph*.]
               \item
               En employant la formule $\left(uv\right)^{\prime}=u^{\prime}v+uv^{\prime}$ pour dériver le premier terme on obtient :
               \par
               $H^{\prime}\left(x\right)=\frac{1}{2}\times \left(2x \ln x +x^{2}\times \frac{1}{x}\right)-\frac{1}{4}\times 2x=\frac{1}{2}\times \left(2x\ln x+x\right)-\frac{1}{2}x$
               \par
               $H^{\prime}\left(x\right)=x\ln x+\frac{1}{2}x-\frac{1}{2}x=x\ln x=h\left(x\right)$
               \par
               Donc $H$ est une primitive de $h$ sur $\left]0; +\infty \right[$
               \item
               Comme $f\left(x\right)=h\left(x\right)-x$, une primitive $F$ de $f$ est définie par :
               \par
               $F\left(x\right)=H\left(x\right)-\frac{1}{2}x^{2}=\frac{1}{2}x^{2} \ln x-\frac{1}{4} x^{2}-\frac{1}{2}x^{2}=\frac{1}{2}x^{2} \ln x-\frac{3}{4} x^{2}$
               \par
               Donc :
               \par
               $\int_{1}^{e}f\left(x\right)dx=\left[F\left(x\right)\right]_{1}^{e}=F\left(e\right)-F\left(1\right)=\frac{1}{2}e^{2}\left(\ln e-\frac{3}{2}\right)-\frac{1}{2}\left(\ln 1-\frac{3}{2}\right)$
               \par
               $\int_{1}^{e}f\left(x\right)dx=\frac{1}{2}e^{2}\left(-\frac{1}{2}\right)-\frac{1}{2}\left(-\frac{3}{2}\right)=\frac{-e^{2}+3}{4}$
               \item
               Sur $\left[1; e\right]$ la fonction $f$ est strictement négative donc l'aire, en unités d'aire, de la partie du plan délimitée par $C$, l'axe des abscisses et les droites d'équation $x=1$ et $x=e$ est :
               \par
               $A=-\int_{1}^{e}f\left(x\right)dx=\frac{e^{2}-3}{4}$
               \par
               La valeur arrondie de $A$ au dixième près est 1,1 unités d'aire.
          \end{enumerate}
     \end{enumerate}
\end{corrige}

\end{document}