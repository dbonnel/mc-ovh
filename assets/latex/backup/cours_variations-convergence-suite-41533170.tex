\documentclass[a4paper]{article}

%================================================================================================================================
%
% Packages
%
%================================================================================================================================

\usepackage[T1]{fontenc} 	% pour caractères accentués
\usepackage[utf8]{inputenc}  % encodage utf8
\usepackage[french]{babel}	% langue : français
\usepackage{fourier}			% caractères plus lisibles
\usepackage[dvipsnames]{xcolor} % couleurs
\usepackage{fancyhdr}		% réglage header footer
\usepackage{needspace}		% empêcher sauts de page mal placés
\usepackage{graphicx}		% pour inclure des graphiques
\usepackage{enumitem,cprotect}		% personnalise les listes d'items (nécessaire pour ol, al ...)
\usepackage{hyperref}		% Liens hypertexte
\usepackage{pstricks,pst-all,pst-node,pstricks-add,pst-math,pst-plot,pst-tree,pst-eucl} % pstricks
\usepackage[a4paper,includeheadfoot,top=2cm,left=3cm, bottom=2cm,right=3cm]{geometry} % marges etc.
\usepackage{comment}			% commentaires multilignes
\usepackage{amsmath,environ} % maths (matrices, etc.)
\usepackage{amssymb,makeidx}
\usepackage{bm}				% bold maths
\usepackage{tabularx}		% tableaux
\usepackage{colortbl}		% tableaux en couleur
\usepackage{fontawesome}		% Fontawesome
\usepackage{environ}			% environment with command
\usepackage{fp}				% calculs pour ps-tricks
\usepackage{multido}			% pour ps tricks
\usepackage[np]{numprint}	% formattage nombre
\usepackage{tikz,tkz-tab} 			% package principal TikZ
\usepackage{pgfplots}   % axes
\usepackage{mathrsfs}    % cursives
\usepackage{calc}			% calcul taille boites
\usepackage[scaled=0.875]{helvet} % font sans serif
\usepackage{svg} % svg
\usepackage{scrextend} % local margin
\usepackage{scratch} %scratch
\usepackage{multicol} % colonnes
%\usepackage{infix-RPN,pst-func} % formule en notation polanaise inversée
\usepackage{listings}

%================================================================================================================================
%
% Réglages de base
%
%================================================================================================================================

\lstset{
language=Python,   % R code
literate=
{á}{{\'a}}1
{à}{{\`a}}1
{ã}{{\~a}}1
{é}{{\'e}}1
{è}{{\`e}}1
{ê}{{\^e}}1
{í}{{\'i}}1
{ó}{{\'o}}1
{õ}{{\~o}}1
{ú}{{\'u}}1
{ü}{{\"u}}1
{ç}{{\c{c}}}1
{~}{{ }}1
}


\definecolor{codegreen}{rgb}{0,0.6,0}
\definecolor{codegray}{rgb}{0.5,0.5,0.5}
\definecolor{codepurple}{rgb}{0.58,0,0.82}
\definecolor{backcolour}{rgb}{0.95,0.95,0.92}

\lstdefinestyle{mystyle}{
    backgroundcolor=\color{backcolour},   
    commentstyle=\color{codegreen},
    keywordstyle=\color{magenta},
    numberstyle=\tiny\color{codegray},
    stringstyle=\color{codepurple},
    basicstyle=\ttfamily\footnotesize,
    breakatwhitespace=false,         
    breaklines=true,                 
    captionpos=b,                    
    keepspaces=true,                 
    numbers=left,                    
xleftmargin=2em,
framexleftmargin=2em,            
    showspaces=false,                
    showstringspaces=false,
    showtabs=false,                  
    tabsize=2,
    upquote=true
}

\lstset{style=mystyle}


\lstset{style=mystyle}
\newcommand{\imgdir}{C:/laragon/www/newmc/assets/imgsvg/}
\newcommand{\imgsvgdir}{C:/laragon/www/newmc/assets/imgsvg/}

\definecolor{mcgris}{RGB}{220, 220, 220}% ancien~; pour compatibilité
\definecolor{mcbleu}{RGB}{52, 152, 219}
\definecolor{mcvert}{RGB}{125, 194, 70}
\definecolor{mcmauve}{RGB}{154, 0, 215}
\definecolor{mcorange}{RGB}{255, 96, 0}
\definecolor{mcturquoise}{RGB}{0, 153, 153}
\definecolor{mcrouge}{RGB}{255, 0, 0}
\definecolor{mclightvert}{RGB}{205, 234, 190}

\definecolor{gris}{RGB}{220, 220, 220}
\definecolor{bleu}{RGB}{52, 152, 219}
\definecolor{vert}{RGB}{125, 194, 70}
\definecolor{mauve}{RGB}{154, 0, 215}
\definecolor{orange}{RGB}{255, 96, 0}
\definecolor{turquoise}{RGB}{0, 153, 153}
\definecolor{rouge}{RGB}{255, 0, 0}
\definecolor{lightvert}{RGB}{205, 234, 190}
\setitemize[0]{label=\color{lightvert}  $\bullet$}

\pagestyle{fancy}
\renewcommand{\headrulewidth}{0.2pt}
\fancyhead[L]{maths-cours.fr}
\fancyhead[R]{\thepage}
\renewcommand{\footrulewidth}{0.2pt}
\fancyfoot[C]{}

\newcolumntype{C}{>{\centering\arraybackslash}X}
\newcolumntype{s}{>{\hsize=.35\hsize\arraybackslash}X}

\setlength{\parindent}{0pt}		 
\setlength{\parskip}{3mm}
\setlength{\headheight}{1cm}

\def\ebook{ebook}
\def\book{book}
\def\web{web}
\def\type{web}

\newcommand{\vect}[1]{\overrightarrow{\,\mathstrut#1\,}}

\def\Oij{$\left(\text{O}~;~\vect{\imath},~\vect{\jmath}\right)$}
\def\Oijk{$\left(\text{O}~;~\vect{\imath},~\vect{\jmath},~\vect{k}\right)$}
\def\Ouv{$\left(\text{O}~;~\vect{u},~\vect{v}\right)$}

\hypersetup{breaklinks=true, colorlinks = true, linkcolor = OliveGreen, urlcolor = OliveGreen, citecolor = OliveGreen, pdfauthor={Didier BONNEL - https://www.maths-cours.fr} } % supprime les bordures autour des liens

\renewcommand{\arg}[0]{\text{arg}}

\everymath{\displaystyle}

%================================================================================================================================
%
% Macros - Commandes
%
%================================================================================================================================

\newcommand\meta[2]{    			% Utilisé pour créer le post HTML.
	\def\titre{titre}
	\def\url{url}
	\def\arg{#1}
	\ifx\titre\arg
		\newcommand\maintitle{#2}
		\fancyhead[L]{#2}
		{\Large\sffamily \MakeUppercase{#2}}
		\vspace{1mm}\textcolor{mcvert}{\hrule}
	\fi 
	\ifx\url\arg
		\fancyfoot[L]{\href{https://www.maths-cours.fr#2}{\black \footnotesize{https://www.maths-cours.fr#2}}}
	\fi 
}


\newcommand\TitreC[1]{    		% Titre centré
     \needspace{3\baselineskip}
     \begin{center}\textbf{#1}\end{center}
}

\newcommand\newpar{    		% paragraphe
     \par
}

\newcommand\nosp {    		% commande vide (pas d'espace)
}
\newcommand{\id}[1]{} %ignore

\newcommand\boite[2]{				% Boite simple sans titre
	\vspace{5mm}
	\setlength{\fboxrule}{0.2mm}
	\setlength{\fboxsep}{5mm}	
	\fcolorbox{#1}{#1!3}{\makebox[\linewidth-2\fboxrule-2\fboxsep]{
  		\begin{minipage}[t]{\linewidth-2\fboxrule-4\fboxsep}\setlength{\parskip}{3mm}
  			 #2
  		\end{minipage}
	}}
	\vspace{5mm}
}

\newcommand\CBox[4]{				% Boites
	\vspace{5mm}
	\setlength{\fboxrule}{0.2mm}
	\setlength{\fboxsep}{5mm}
	
	\fcolorbox{#1}{#1!3}{\makebox[\linewidth-2\fboxrule-2\fboxsep]{
		\begin{minipage}[t]{1cm}\setlength{\parskip}{3mm}
	  		\textcolor{#1}{\LARGE{#2}}    
 	 	\end{minipage}  
  		\begin{minipage}[t]{\linewidth-2\fboxrule-4\fboxsep}\setlength{\parskip}{3mm}
			\raisebox{1.2mm}{\normalsize\sffamily{\textcolor{#1}{#3}}}						
  			 #4
  		\end{minipage}
	}}
	\vspace{5mm}
}

\newcommand\cadre[3]{				% Boites convertible html
	\par
	\vspace{2mm}
	\setlength{\fboxrule}{0.1mm}
	\setlength{\fboxsep}{5mm}
	\fcolorbox{#1}{white}{\makebox[\linewidth-2\fboxrule-2\fboxsep]{
  		\begin{minipage}[t]{\linewidth-2\fboxrule-4\fboxsep}\setlength{\parskip}{3mm}
			\raisebox{-2.5mm}{\sffamily \small{\textcolor{#1}{\MakeUppercase{#2}}}}		
			\par		
  			 #3
 	 		\end{minipage}
	}}
		\vspace{2mm}
	\par
}

\newcommand\bloc[3]{				% Boites convertible html sans bordure
     \needspace{2\baselineskip}
     {\sffamily \small{\textcolor{#1}{\MakeUppercase{#2}}}}    
		\par		
  			 #3
		\par
}

\newcommand\CHelp[1]{
     \CBox{Plum}{\faInfoCircle}{À RETENIR}{#1}
}

\newcommand\CUp[1]{
     \CBox{NavyBlue}{\faThumbsOUp}{EN PRATIQUE}{#1}
}

\newcommand\CInfo[1]{
     \CBox{Sepia}{\faArrowCircleRight}{REMARQUE}{#1}
}

\newcommand\CRedac[1]{
     \CBox{PineGreen}{\faEdit}{BIEN R\'EDIGER}{#1}
}

\newcommand\CError[1]{
     \CBox{Red}{\faExclamationTriangle}{ATTENTION}{#1}
}

\newcommand\TitreExo[2]{
\needspace{4\baselineskip}
 {\sffamily\large EXERCICE #1\ (\emph{#2 points})}
\vspace{5mm}
}

\newcommand\img[2]{
          \includegraphics[width=#2\paperwidth]{\imgdir#1}
}

\newcommand\imgsvg[2]{
       \begin{center}   \includegraphics[width=#2\paperwidth]{\imgsvgdir#1} \end{center}
}


\newcommand\Lien[2]{
     \href{#1}{#2 \tiny \faExternalLink}
}
\newcommand\mcLien[2]{
     \href{https~://www.maths-cours.fr/#1}{#2 \tiny \faExternalLink}
}

\newcommand{\euro}{\eurologo{}}

%================================================================================================================================
%
% Macros - Environement
%
%================================================================================================================================

\newenvironment{tex}{ %
}
{%
}

\newenvironment{indente}{ %
	\setlength\parindent{10mm}
}

{
	\setlength\parindent{0mm}
}

\newenvironment{corrige}{%
     \needspace{3\baselineskip}
     \medskip
     \textbf{\textsc{Corrigé}}
     \medskip
}
{
}

\newenvironment{extern}{%
     \begin{center}
     }
     {
     \end{center}
}

\NewEnviron{code}{%
	\par
     \boite{gray}{\texttt{%
     \BODY
     }}
     \par
}

\newenvironment{vbloc}{% boite sans cadre empeche saut de page
     \begin{minipage}[t]{\linewidth}
     }
     {
     \end{minipage}
}
\NewEnviron{h2}{%
    \needspace{3\baselineskip}
    \vspace{0.6cm}
	\noindent \MakeUppercase{\sffamily \large \BODY}
	\vspace{1mm}\textcolor{mcgris}{\hrule}\vspace{0.4cm}
	\par
}{}

\NewEnviron{h3}{%
    \needspace{3\baselineskip}
	\vspace{5mm}
	\textsc{\BODY}
	\par
}

\NewEnviron{margeneg}{ %
\begin{addmargin}[-1cm]{0cm}
\BODY
\end{addmargin}
}

\NewEnviron{html}{%
}

\begin{document}
\meta{url}{/cours/variations-convergence-suite/}
\meta{pid}{543}
\meta{titre}{Suites et récurrence}
\meta{type}{cours}
\begin{h2}I - Démonstration par récurrence\end{h2}
\cadre{rouge}{Théorème}{%id="t10"
     Soit $P\left(n\right)$ une proposition qui dépend d'un entier naturel $n$.
     \begin{itemize}
          \item Si $P\left(n_{0}\right)$ est vraie \textbf{(initialisation)}
          \item Et si $P\left(n\right)$ vraie entraîne $P\left(n+1\right)$ vraie \textbf{(hérédité)}
     \end{itemize}
     alors la propriété $P\left(n\right)$ est vraie pour tout entier $n\geqslant n_{0}$
}
\bloc{vert}{Remarques}{%id="r10"
     \begin{itemize}
          \item La démonstration par récurrence s'apparente au "principe des dominos" :
          \begin{center}
               \imgsvg{dominos}{0.4}%width="450" alt="dominos et récurrence"
          \end{center}
          \item L'étape d'initialisation est souvent facile à démontrer~; toutefois, faites attention à \textbf{ne pas l'oublier}~!
          \item Pour prouver l'hérédité, on suppose que la propriété est vraie \textbf{pour un certain entier $n$} (cette supposition est appelée \textbf{hypothèse de récurrence}) et on démontre qu'elle est alors vraie pour l'entier $n+1$. Pour cela, il est conseillé d'écrire ce que signifie $P\left(n+1\right)$ (que l'on souhaite démontrer), en remplaçant $n$ par $n+$1 dans la propriété $P\left(n\right)$
     \end{itemize}
}
\bloc{orange}{Exemple}{%id="e10"
     Montrons que pour tout entier n strictement positif $1+2+. . .+n=\frac{n\left(n+1\right)}{2}$.
     \bigbreak
     \textbf{Initialisation}
     \medbreak
     On commence à $n_{0}=1$ car l'énoncé précise "strictement positif".
     \par
     La proposition devient :
     \par
     $1=\frac{1\times 2}{2}$
     \par
     ce qui est vrai.
     \bigbreak
     \textbf{Hérédité}
     \medbreak
     On suppose que pour un certain entier $n$:
     \par
     $1+2+. . .+n=\frac{n\left(n+1\right)}{2}$ (\textbf{Hypothèse de récurrence})
     \par
     et on va montrer qu'alors :
     \par
     $1+2+. . .+n+1=\frac{\left(n+1\right)\left(n+2\right)}{2}$ (on a remplacé $n$ par $n+1$ dans la formule que l'on souhaite prouver).
     \par
     Isolons le dernier terme de notre somme
     \par
     $1+2+. . .+n+1=\left(1+2+. . . +n\right) + n+1$
     \par
     On applique maintenant notre hypothèse de récurrence à $1+2+. . . +n$:
     \par
     $1+2+. . .+n+1=\frac{n\left(n+1\right)}{2}+n+1=\frac{n\left(n+1\right)}{2}+\frac{2\left(n+1\right)}{2}=\frac{n\left(n+1\right)+2\left(n+1\right)}{2}$
     \par
     $1+2+. . .+n+1=\frac{\left(n+1\right)\left(n+2\right)}{2}$
     \par
     ce qui correspond bien à ce que nous voulions montrer.
     \medbreak
     En conclusion nous avons bien prouvé que pour pour tout entier n strictement positif :
     \par
     $1+2+. . .+n=\frac{n\left(n+1\right)}{2}$.
}
\begin{h2}II - Sens de variation - Suites majorées, minorées\end{h2}
\cadre{bleu}{Définitions (rappel)}{%id="d20"
     \begin{itemize}
          \item On dit que la suite $\left(u_{n}\right)$ est \textbf{croissante} si pour tout entier naturel $n$ : $u_{n+1} \geqslant u_{n}$
          \item On dit que la suite $\left(u_{n}\right)$ est \textbf{strictement croissante} si pour tout entier naturel $n$ : $u_{n+1} > u_{n}$
          \item On dit que la suite $\left(u_{n}\right)$ est \textbf{décroissante} si pour tout entier naturel $n$ : $u_{n+1} \leqslant u_{n}$
          \item On dit que la suite $\left(u_{n}\right)$ est \textbf{strictement décroissante} si pour tout entier naturel $n$ : $u_{n+1} < u_{n}$
          \item On dit que la suite $\left(u_{n}\right)$ est \textbf{constante} si pour tout entier naturel $n$ : $u_{n+1} = u_{n}$
     \end{itemize}
}
\cadre{bleu}{Définitions}{%id="d30"
     \begin{itemize}
          \item On dit que la suite $\left(u_{n}\right)$ est \textbf{majorée} par le réel $M$ si tout entier naturel $n$ : $u_{n} \leqslant M$.
          \par
          $M$ s'appelle alors un \textbf{majorant} de la suite $\left(u_{n}\right)$
          \item On dit que la suite $\left(u_{n}\right)$ est \textbf{minorée} par le réel $m$ si pour tout entier naturel $n$ : $u_{n} \geqslant m$.
          \par
          $m$ s'appelle un \textbf{minorant} de la suite $\left(u_{n}\right)$
     \end{itemize}
}
\bloc{cyan}{Remarque}{%id="r30"
     Si la suite $\left(u_{n}\right)$ est majorée (ou minorée), les majorants (ou minorants) \textbf{ne sont pas uniques}. Bien au contraire, si $M$ est un majorant de la suite $\left(u_{n}\right)$, tout réel supérieur à $M$ est aussi un majorant de la suite $\left(u_{n}\right)$
}
\bloc{orange}{Exemple}{%id="e30"
     Soit la suite $\left(u_{n}\right)$ définie par :
     \par
     $\left\{ \begin{matrix} u_{0}=1 \\ u_{n+1} =u_{n}^{2}+1 \end{matrix}\right.\text{pour tout} n \in \mathbb{N}$
     \par
     On vérifie aisément que pour tout $n \in \mathbb{N}$, $u_{n}$ est supérieur ou égal à $1$ donc la suite $\left(u_{n}\right)$ est minorée par $1$. Par contre cette suite n'est pas majorée (on peut, par exemple, démonter par récurrence que pour tout $n \in \mathbb{N}$ $u_{n} > n$.
}
\begin{h2}III - Convergence - Limite\end{h2}
\cadre{bleu}{Définition}{%id="d40"
     On dit que la suite $(u_{n})$ \textbf{converge} vers le nombre réel $l$ (ou \textbf{admet pour limite} le nombre réel $l$) si tout intervalle ouvert contenant $l$ contient tous les termes de la suite à partir d'un certain rang.
     \par
     On note alors $\lim\limits_{n\rightarrow +\infty }u_{n}=l$
}
\bloc{orange}{Exemple}{%id="e40"
     \begin{center}
          \begin{extern}%width="600" alt="suite convergente"
               % -+-+-+ variables modifiables
               \resizebox{8cm}{!}{%
                    \def\xmin{-1}
                    \def\xmax{13.5}
                    \def\ymin{-2}
                    \def\ymax{3.8}
                    \def\xunit{1}  % unités en cm
                    \def\yunit{1}
                    \psset{xunit=\xunit,yunit=\yunit,algebraic=true}
                    \fontsize{15pt}{15pt}\selectfont
                    \begin{pspicture*}[linewidth=1pt](\xmin,\ymin)(\xmax,\ymax)
                         \psaxes[Dx=100,Dy=100,linewidth=0.75pt]{->}(0,0)(\xmin,\ymin)(\xmax,\ymax)
                         \rput[tr](-0.1,-0.1){$O$}
                         \multido{\n=0.0+1}{15}{
                              \FPeval{\suite}{3*cos(\n*3.14159)/(\n+1)+2}
                              \psdots[linecolor=blue](\n,\suite)
                         }
                         \psline[linecolor=red,linewidth=0.5pt](-3,2)(16,2)
                         \rput[br](-0.1,2.1){$\red l$}
                         \psdots[linecolor=red](0,2)
                         \psframe[fillstyle=solid,fillcolor=vert,linecolor=vert,linewidth=0.75pt,opacity=0.2](-3,2.6)(16,1.4)
                    \end{pspicture*}
               }
          \end{extern}
     \end{center}
     \begin{center}
          \textit{Suite convergeant vers $l$}
     \end{center}
}
\bloc{cyan}{Remarques}{%id="r40"
     \begin{itemize}
          \item Une suite qui n'est pas convergente (c'est à dire qui n'a pas de limite ou qui a une limite infinie - voir ci-dessous) est dite \textbf{divergente}.
          \item La limite, si elle existe, est \textbf{unique}.
     \end{itemize}
}
\bloc{orange}{Exemple}{%id="e40"
     Les suites définies pour $n > 0$ par $u_{n}=\frac{1}{n^{k}}$ où $k$ est un entier strictement positif, \textbf{convergent vers zéro}
}
\cadre{bleu}{Définition}{%id="d50"
     On dit que la suite $u_{n}$ admet pour limite $+\infty $ si tout intervalle de la forme $\left]A;+\infty \right[$ contient tous les termes de la suite à partir d'un certain rang.
}
\bloc{orange}{Exemple}{%id="e50"
     Les suites définies pour $n > 0$ par $u_{n}=n^{k}$ où $k$ est un entier strictement positif, divergent vers $+\infty $
}
\cadre{rouge}{Théorème (des gendarmes)}{%id="t60"
     Si les suites $\left(v_{n}\right)$ et $\left(w_{n}\right)$ convergent vers \textbf{la même limite} $l$ et si $v_{n}\leqslant u_{n}\leqslant w_{n}$ pour tout entier $n$ à partir d'un certain rang, alors la suite $\left(u_{n}\right)$ converge vers $l$.
}
\bloc{orange}{Exemple}{%id="e60"
     Soit la suite définie pour $n > 0$ par $u_{n}=\frac{\sin\left(n\right)}{n}$.
     \par
     On sait que pour tout $n$, $-1\leqslant \sin\left(n\right)\leqslant 1$ donc $-\frac{1}{n}\leqslant \frac{\sin\left(n\right)}{n}\leqslant \frac{1}{n}$.
     \par
     Or les suites $\left(v_{n}\right)$ et $\left(w_{n}\right)$ définie sur $\mathbb{N}^*$ par $v_{n}=-\frac{1}{n}$ et $w_{n}=\frac{1}{n}$ convergent vers zéro donc, d'après le théorème des gendarmes \textbf{$\left(u_{n}\right)$ converge vers zéro}.
}
\cadre{rouge}{Théorème}{%id="t70"
     Soient deux suites $\left(u_{n}\right)$ et $\left(v_{n}\right)$ telles que pour tout $n \in \mathbb{N}$, $u_{n}\geqslant v_{n}$.
     \par
     Si $\lim\limits_{n\rightarrow +\infty }v_{n}=+\infty $, alors $\lim\limits_{n\rightarrow +\infty }u_{n}=+\infty $
}
\cadre{rouge}{Théorème}{%id="t80"
     Une suite \textbf{croissante et majorée} est convergente.
     \par
     Une suite \textbf{décroissante et minorée} est convergente.
}
\bloc{cyan}{Remarques}{%id="r80"
     \begin{itemize}
          \item Ce théorème est fréquemment utilisé dans les exercices
          \item Ce théorème permet de montrer qu'une suite est convergente mais, à lui seul, il ne permet pas de trouver la valeur de la limite $l$
     \end{itemize}
}
\bloc{orange}{Exemple}{%id="e80"
     Un cas particulier assez fréquent est celui d'une suite \textbf{décroissante et positive}. Puisqu'elle est positive, elle est minorée par zéro, donc d'après le théorème précédent, elle est convergente.
}
\cadre{rouge}{Théorème (limite d'une suite géométrique)}{%id="t90"
     Soit $\left(u_{n}\right)$ une suite géométrique de raison $q$.
     \begin{itemize}
          \item Si $-1 < q < 1$ la suite $\left(u_{n}\right)$ \textbf{converge vers 0}
          \item Si $q > 1$ la suite $\left(u_{n}\right)$ \textbf{tend vers $+\infty $}
          \item Si $q\leqslant -1$ la suite $\left(u_{n}\right)$ \textbf{n'a pas de limite.}
     \end{itemize}
}
\bloc{cyan}{Remarque}{%id="r90"
     Si $q=1$ la suite $\left(u_{n}\right)$ est constante (donc convergente)
}
\bloc{orange}{Exemple}{%id="e90"
     $\lim\limits_{n\rightarrow +\infty }\left(\frac{2}{3}\right)^{n}=0$ (suite géométrique de raison $q=\frac{2}{3} < 1$)
     \par
     $\lim\limits_{n\rightarrow +\infty }\left(\frac{4}{3}\right)^{n}=+\infty $ (suite géométrique de raison $q=\frac{4}{3} > 1$)
}

\end{document}