\documentclass[a4paper]{article}

%================================================================================================================================
%
% Packages
%
%================================================================================================================================

\usepackage[T1]{fontenc} 	% pour caractères accentués
\usepackage[utf8]{inputenc}  % encodage utf8
\usepackage[french]{babel}	% langue : français
\usepackage{fourier}			% caractères plus lisibles
\usepackage[dvipsnames]{xcolor} % couleurs
\usepackage{fancyhdr}		% réglage header footer
\usepackage{needspace}		% empêcher sauts de page mal placés
\usepackage{graphicx}		% pour inclure des graphiques
\usepackage{enumitem,cprotect}		% personnalise les listes d'items (nécessaire pour ol, al ...)
\usepackage{hyperref}		% Liens hypertexte
\usepackage{pstricks,pst-all,pst-node,pstricks-add,pst-math,pst-plot,pst-tree,pst-eucl} % pstricks
\usepackage[a4paper,includeheadfoot,top=2cm,left=3cm, bottom=2cm,right=3cm]{geometry} % marges etc.
\usepackage{comment}			% commentaires multilignes
\usepackage{amsmath,environ} % maths (matrices, etc.)
\usepackage{amssymb,makeidx}
\usepackage{bm}				% bold maths
\usepackage{tabularx}		% tableaux
\usepackage{colortbl}		% tableaux en couleur
\usepackage{fontawesome}		% Fontawesome
\usepackage{environ}			% environment with command
\usepackage{fp}				% calculs pour ps-tricks
\usepackage{multido}			% pour ps tricks
\usepackage[np]{numprint}	% formattage nombre
\usepackage{tikz,tkz-tab} 			% package principal TikZ
\usepackage{pgfplots}   % axes
\usepackage{mathrsfs}    % cursives
\usepackage{calc}			% calcul taille boites
\usepackage[scaled=0.875]{helvet} % font sans serif
\usepackage{svg} % svg
\usepackage{scrextend} % local margin
\usepackage{scratch} %scratch
\usepackage{multicol} % colonnes
%\usepackage{infix-RPN,pst-func} % formule en notation polanaise inversée
\usepackage{listings}

%================================================================================================================================
%
% Réglages de base
%
%================================================================================================================================

\lstset{
language=Python,   % R code
literate=
{á}{{\'a}}1
{à}{{\`a}}1
{ã}{{\~a}}1
{é}{{\'e}}1
{è}{{\`e}}1
{ê}{{\^e}}1
{í}{{\'i}}1
{ó}{{\'o}}1
{õ}{{\~o}}1
{ú}{{\'u}}1
{ü}{{\"u}}1
{ç}{{\c{c}}}1
{~}{{ }}1
}


\definecolor{codegreen}{rgb}{0,0.6,0}
\definecolor{codegray}{rgb}{0.5,0.5,0.5}
\definecolor{codepurple}{rgb}{0.58,0,0.82}
\definecolor{backcolour}{rgb}{0.95,0.95,0.92}

\lstdefinestyle{mystyle}{
    backgroundcolor=\color{backcolour},   
    commentstyle=\color{codegreen},
    keywordstyle=\color{magenta},
    numberstyle=\tiny\color{codegray},
    stringstyle=\color{codepurple},
    basicstyle=\ttfamily\footnotesize,
    breakatwhitespace=false,         
    breaklines=true,                 
    captionpos=b,                    
    keepspaces=true,                 
    numbers=left,                    
xleftmargin=2em,
framexleftmargin=2em,            
    showspaces=false,                
    showstringspaces=false,
    showtabs=false,                  
    tabsize=2,
    upquote=true
}

\lstset{style=mystyle}


\lstset{style=mystyle}
\newcommand{\imgdir}{C:/laragon/www/newmc/assets/imgsvg/}
\newcommand{\imgsvgdir}{C:/laragon/www/newmc/assets/imgsvg/}

\definecolor{mcgris}{RGB}{220, 220, 220}% ancien~; pour compatibilité
\definecolor{mcbleu}{RGB}{52, 152, 219}
\definecolor{mcvert}{RGB}{125, 194, 70}
\definecolor{mcmauve}{RGB}{154, 0, 215}
\definecolor{mcorange}{RGB}{255, 96, 0}
\definecolor{mcturquoise}{RGB}{0, 153, 153}
\definecolor{mcrouge}{RGB}{255, 0, 0}
\definecolor{mclightvert}{RGB}{205, 234, 190}

\definecolor{gris}{RGB}{220, 220, 220}
\definecolor{bleu}{RGB}{52, 152, 219}
\definecolor{vert}{RGB}{125, 194, 70}
\definecolor{mauve}{RGB}{154, 0, 215}
\definecolor{orange}{RGB}{255, 96, 0}
\definecolor{turquoise}{RGB}{0, 153, 153}
\definecolor{rouge}{RGB}{255, 0, 0}
\definecolor{lightvert}{RGB}{205, 234, 190}
\setitemize[0]{label=\color{lightvert}  $\bullet$}

\pagestyle{fancy}
\renewcommand{\headrulewidth}{0.2pt}
\fancyhead[L]{maths-cours.fr}
\fancyhead[R]{\thepage}
\renewcommand{\footrulewidth}{0.2pt}
\fancyfoot[C]{}

\newcolumntype{C}{>{\centering\arraybackslash}X}
\newcolumntype{s}{>{\hsize=.35\hsize\arraybackslash}X}

\setlength{\parindent}{0pt}		 
\setlength{\parskip}{3mm}
\setlength{\headheight}{1cm}

\def\ebook{ebook}
\def\book{book}
\def\web{web}
\def\type{web}

\newcommand{\vect}[1]{\overrightarrow{\,\mathstrut#1\,}}

\def\Oij{$\left(\text{O}~;~\vect{\imath},~\vect{\jmath}\right)$}
\def\Oijk{$\left(\text{O}~;~\vect{\imath},~\vect{\jmath},~\vect{k}\right)$}
\def\Ouv{$\left(\text{O}~;~\vect{u},~\vect{v}\right)$}

\hypersetup{breaklinks=true, colorlinks = true, linkcolor = OliveGreen, urlcolor = OliveGreen, citecolor = OliveGreen, pdfauthor={Didier BONNEL - https://www.maths-cours.fr} } % supprime les bordures autour des liens

\renewcommand{\arg}[0]{\text{arg}}

\everymath{\displaystyle}

%================================================================================================================================
%
% Macros - Commandes
%
%================================================================================================================================

\newcommand\meta[2]{    			% Utilisé pour créer le post HTML.
	\def\titre{titre}
	\def\url{url}
	\def\arg{#1}
	\ifx\titre\arg
		\newcommand\maintitle{#2}
		\fancyhead[L]{#2}
		{\Large\sffamily \MakeUppercase{#2}}
		\vspace{1mm}\textcolor{mcvert}{\hrule}
	\fi 
	\ifx\url\arg
		\fancyfoot[L]{\href{https://www.maths-cours.fr#2}{\black \footnotesize{https://www.maths-cours.fr#2}}}
	\fi 
}


\newcommand\TitreC[1]{    		% Titre centré
     \needspace{3\baselineskip}
     \begin{center}\textbf{#1}\end{center}
}

\newcommand\newpar{    		% paragraphe
     \par
}

\newcommand\nosp {    		% commande vide (pas d'espace)
}
\newcommand{\id}[1]{} %ignore

\newcommand\boite[2]{				% Boite simple sans titre
	\vspace{5mm}
	\setlength{\fboxrule}{0.2mm}
	\setlength{\fboxsep}{5mm}	
	\fcolorbox{#1}{#1!3}{\makebox[\linewidth-2\fboxrule-2\fboxsep]{
  		\begin{minipage}[t]{\linewidth-2\fboxrule-4\fboxsep}\setlength{\parskip}{3mm}
  			 #2
  		\end{minipage}
	}}
	\vspace{5mm}
}

\newcommand\CBox[4]{				% Boites
	\vspace{5mm}
	\setlength{\fboxrule}{0.2mm}
	\setlength{\fboxsep}{5mm}
	
	\fcolorbox{#1}{#1!3}{\makebox[\linewidth-2\fboxrule-2\fboxsep]{
		\begin{minipage}[t]{1cm}\setlength{\parskip}{3mm}
	  		\textcolor{#1}{\LARGE{#2}}    
 	 	\end{minipage}  
  		\begin{minipage}[t]{\linewidth-2\fboxrule-4\fboxsep}\setlength{\parskip}{3mm}
			\raisebox{1.2mm}{\normalsize\sffamily{\textcolor{#1}{#3}}}						
  			 #4
  		\end{minipage}
	}}
	\vspace{5mm}
}

\newcommand\cadre[3]{				% Boites convertible html
	\par
	\vspace{2mm}
	\setlength{\fboxrule}{0.1mm}
	\setlength{\fboxsep}{5mm}
	\fcolorbox{#1}{white}{\makebox[\linewidth-2\fboxrule-2\fboxsep]{
  		\begin{minipage}[t]{\linewidth-2\fboxrule-4\fboxsep}\setlength{\parskip}{3mm}
			\raisebox{-2.5mm}{\sffamily \small{\textcolor{#1}{\MakeUppercase{#2}}}}		
			\par		
  			 #3
 	 		\end{minipage}
	}}
		\vspace{2mm}
	\par
}

\newcommand\bloc[3]{				% Boites convertible html sans bordure
     \needspace{2\baselineskip}
     {\sffamily \small{\textcolor{#1}{\MakeUppercase{#2}}}}    
		\par		
  			 #3
		\par
}

\newcommand\CHelp[1]{
     \CBox{Plum}{\faInfoCircle}{À RETENIR}{#1}
}

\newcommand\CUp[1]{
     \CBox{NavyBlue}{\faThumbsOUp}{EN PRATIQUE}{#1}
}

\newcommand\CInfo[1]{
     \CBox{Sepia}{\faArrowCircleRight}{REMARQUE}{#1}
}

\newcommand\CRedac[1]{
     \CBox{PineGreen}{\faEdit}{BIEN R\'EDIGER}{#1}
}

\newcommand\CError[1]{
     \CBox{Red}{\faExclamationTriangle}{ATTENTION}{#1}
}

\newcommand\TitreExo[2]{
\needspace{4\baselineskip}
 {\sffamily\large EXERCICE #1\ (\emph{#2 points})}
\vspace{5mm}
}

\newcommand\img[2]{
          \includegraphics[width=#2\paperwidth]{\imgdir#1}
}

\newcommand\imgsvg[2]{
       \begin{center}   \includegraphics[width=#2\paperwidth]{\imgsvgdir#1} \end{center}
}


\newcommand\Lien[2]{
     \href{#1}{#2 \tiny \faExternalLink}
}
\newcommand\mcLien[2]{
     \href{https~://www.maths-cours.fr/#1}{#2 \tiny \faExternalLink}
}

\newcommand{\euro}{\eurologo{}}

%================================================================================================================================
%
% Macros - Environement
%
%================================================================================================================================

\newenvironment{tex}{ %
}
{%
}

\newenvironment{indente}{ %
	\setlength\parindent{10mm}
}

{
	\setlength\parindent{0mm}
}

\newenvironment{corrige}{%
     \needspace{3\baselineskip}
     \medskip
     \textbf{\textsc{Corrigé}}
     \medskip
}
{
}

\newenvironment{extern}{%
     \begin{center}
     }
     {
     \end{center}
}

\NewEnviron{code}{%
	\par
     \boite{gray}{\texttt{%
     \BODY
     }}
     \par
}

\newenvironment{vbloc}{% boite sans cadre empeche saut de page
     \begin{minipage}[t]{\linewidth}
     }
     {
     \end{minipage}
}
\NewEnviron{h2}{%
    \needspace{3\baselineskip}
    \vspace{0.6cm}
	\noindent \MakeUppercase{\sffamily \large \BODY}
	\vspace{1mm}\textcolor{mcgris}{\hrule}\vspace{0.4cm}
	\par
}{}

\NewEnviron{h3}{%
    \needspace{3\baselineskip}
	\vspace{5mm}
	\textsc{\BODY}
	\par
}

\NewEnviron{margeneg}{ %
\begin{addmargin}[-1cm]{0cm}
\BODY
\end{addmargin}
}

\NewEnviron{html}{%
}

\begin{document}
\meta{url}{/methode/5-methodes-pour-calculer-un-produit-scalaire/}
\meta{pid}{11181}
\meta{titre}{5 méthodes pour calculer un produit scalaire}
\meta{type}{methode}
%
\cadre{vert}{Propriété}{ % id=100
     Il existe de nombreuses méthodes permettant de calculer un produit scalaire. C'est, en partie, ce qui fait la puissance de cet outil en mathématiques.
     \newpar
     Nous allons voir, dans ce chapitre, 5 des principales méthodes utilisées en classe de Première pour calculer un produit scalaire :
     \begin{enumerate}
          \item
          Utiliser une projection orthogonale,
          \item
          Appliquer une formule utilisant le cosinus d'un angle,
          \item
          Appliquer une formule utilisant les normes de 3 vecteurs,
          \item
          Se placer dans un repère orthonormé,
          \item
          Utiliser la relation de Chasles.
     \end{enumerate}
}% fin propriété
\begin{h2}1. Utiliser une projection orthogonale\end{h2}
Pour calculer le produit scalaire $ \overrightarrow{AB}  \cdot \overrightarrow{AC} $ , on projette orthogonalement le point $C$ sur la droite $(AB)$ .\\
Notons $H$ ce projeté orthogonal~:\\
     \begin{center}
\imgsvg{5-methodes-pour-calculer-un-produit-scalaire-1}{0.3}% alt="Calculer du produit scalaire 1" style="width:25rem" 
 \end{center}
On utilise alors le théorème suivant (voir \mcLien{https://www.maths-cours.fr/cours/produit-scalaire\#t50}{cours})~:\\
\cadre{rouge}{Théorème}{% id="t50"
     Soient $A, B, C$ trois points du plan et si $H$ est la projection orthogonale de $C$ sur la droite $\left(AB\right).$
     \par
     Alors :
     \begin{itemize}
          \item $\overrightarrow{AB} \cdot \overrightarrow{AC}=AB\times AH   $ si l'angle $\widehat{BAC}$ est aigu
          \item $\overrightarrow{AB} \cdot \overrightarrow{AC}=-AB\times AH   $ si l'angle $\widehat{BAC}$ est obtus
     \end{itemize}
}
\bloc{cyan}{Remarque}{% id="r60"
     \begin{itemize}
          \item
          Dire que l'angle $\widehat{BAC}$ est aigu revient à dire que les vecteurs  $\overrightarrow{AB}$ et  $\overrightarrow{AH}$ ont le même sens.
          \item
          Dire que l'angle $\widehat{BAC}$ est obtus revient à dire que les vecteurs  $\overrightarrow{AB}$ et  $\overrightarrow{AH}$ ont des sens opposés.
     \end{itemize}
}% fin remarque
\bloc{orange}{Exemple}{ % id="e70"
     Sur la figure ci-dessous,  $ABCD$  est un carré de côté 4 unités et  $I$  et le milieu du segment  $[AB]$.\\
     On cherche à calculer la valeur du produit scalaire $\overrightarrow{IB}  \cdot \overrightarrow{ID} $.
     \begin{center}
\imgsvg{5-methodes-pour-calculer-un-produit-scalaire-2}{0.3}% alt="Calculer du produit scalaire 2" style="width:25rem" 
 \end{center}     La méthode utilisant la projection orthogonale est particulièrement bien adaptée ici puisque l'on connaît la projection orthogonale $A$ du point $D$ sur la droite $(IB).$
     \newpar
     L'angle $ \widehat{DIB}$ est ici un angle obtus. \\
     Les segments $IB$ et $AI$ mesure chacun 2 unités.\\
     On a donc~:
     \newpar
     $\overrightarrow{IB} \cdot \overrightarrow{ID}= - IB \times IA $\\
     $\overrightarrow{IB} \cdot \overrightarrow{ID}= - 2 \times 2= - 4$
}% fin exemple
\begin{h2}2. Appliquer une formule utilisant le cosinus d'un angle\end{h2}
     \begin{center}
\imgsvg{5-methodes-pour-calculer-un-produit-scalaire-3}{0.3}% alt="Calculer du produit scalaire 7" style="width:25rem" 
 \end{center}Si l'on connaît l'angle $ \widehat{BAC}$, on peut calculer le produit scalaire $ \overrightarrow{AB}  \cdot \overrightarrow{AC} $ en utilisant les longueurs $AB$ et  $AC$ ainsi que le cosinus de l'angle $ \widehat{BAC}$(Voir \mcLien{https://www.maths-cours.fr/cours/produit-scalaire\#d10}{Définition du produit scalaire.})
\cadre{bleu}{Définition}{% id="d100"
     Le \textbf{produit scalaire} de $ \overrightarrow{AB} $ et $ \overrightarrow{AC} $ est le \textbf{nombre réel} noté $ \overrightarrow{AB}  \cdot  \overrightarrow{AC} $ défini par :
     \begin{center}$\overrightarrow{AB}  \cdot  \overrightarrow{AC} =AB \times AC \times  \cos \left(\overrightarrow{AB} ; \overrightarrow{AC} \right) $\end{center}
}
\bloc{cyan}{Remarque}{% id="r110"
     Le sens de l'angle n'a pas d'importance dans cette formule puisque pour tout angle $\theta \ :$ $\cos \theta =\cos( -  \theta ).$\\
     On peut donc aussi bien utiliser des angles orientés ( comme $ \left(\overrightarrow{AB} ; \overrightarrow{AC} \right) $ )  que des angles géométriques ( comme $ \widehat{BAC}$ ).
} % fin remarque
\bloc{orange}{Exemple}{ % id="e120"
     Pour la figure ci-dessous, on souhaite déterminer une valeur approchée à  $10{}^{ - 2}$  près du produit scalaire $\overrightarrow{AB} \cdot \overrightarrow{AC}$ .
     \begin{center}
          \begin{extern} %width="400" alt="Calcul du produit scalaire à partir du cosinus"
               \resizebox{8cm}{!}{
                    %
                    \begin{tikzpicture}[line cap=round,line join=round,>=triangle 45,x=1.0cm,y=1.0cm]
                         \clip(0.,1.) rectangle (7.,4.6);
                         \draw [shift={(1.,2.)},line width=0.4pt,color=ffqqqq,fill=ffqqqq,fill opacity=0.10000000149011612] (0,0) -- (0.:0.36157024793388415) arc (0.:50.1826180001947:0.36157024793388415) -- cycle;
                         \draw [line width=0.4pt,color=tttttt] (2.6472246317724393,3.97584124245039)-- (6.,2.);
                         \draw [line width=0.4pt,color=tttttt] (2.6472246317724393,3.97584124245039)-- (1.,2.);
                         \draw [color=ffqqqq](1.6,2.2) node{\scriptsize 50°};
                         \draw [color=tttttt](3.4400826446280974,1.770491803278689) node {\scriptsize 12};
                         \draw [color=tttttt](1.6735537190082637,3.2510785159620372) node {\scriptsize 6};
                         \draw [line width=0.4pt,color=tttttt] (1.,2.)-- (6.,2.);
                         \begin{scriptsize}
                              \draw [fill=tttttt] (1.,2.) circle (0.5pt);
                              \draw[color=tttttt] (0.9245867768595036,1.8248490077653154) node {$A$};
                              \draw [fill=tttttt] (6.,2.) circle (0.5pt);
                              \draw[color=tttttt] (6.058884297520659,1.855910267471959) node {$B$};
                              \draw [fill=black] (2.6472246317724393,3.97584124245039) circle (0.5pt);
                              \draw[color=black] (2.665289256198346,4.164797238999138) node {$C$};
                         \end{scriptsize}
                    \end{tikzpicture}
               }
          \end{extern}
     \end{center}
     Bien sûr, on utilise la définition du produit scalaire à l'aide des angles puisqu'ici on connaît l'angle  $ \widehat{BAC}$ .
     \newpar
     $\overrightarrow{AB} \cdot \overrightarrow{AC}=AB  \times AC \times \cos \widehat{BAC}$\\
     $\overrightarrow{AB} \cdot \overrightarrow{AC}=12 \times 6 \times \cos(50 \degree)$\\
     $\overrightarrow{AB} \cdot \overrightarrow{AC} \approx 12 \times 6 \times 0,643 \approx 46,28.$
}% fin exemple
\begin{h2}3. Appliquer une formule utilisant les normes de 3 vecteurs\end{h2}
Lorsque l'on connaît trois distances, par exemple, les longueurs des trois côtés d'un triangle, On peut calculer un produit scalaire en utilisant l'une des égalités ci-dessous (Voir \mcLien{https://www.maths-cours.fr/cours/produit-scalaire\#t40}{propriété})~:
\cadre{rouge}{Théorème}{% id="t40"
     Pour tous vecteurs $\vec{u}$ et $\vec{v}$ :
     \begin{center}$\vec{u} \cdot \vec{v}=\frac{1}{2} \left(||\vec{u}+\vec{v}||^{2}-||\vec{u}||^{2}-||\vec{v}||^{2}\right)$\end{center}
     \begin{center} $\vec{u} \cdot \vec{v}=\frac{1}{2}\left(\left\Vert \vec{u}\right\Vert{}^2 +\left\Vert \vec{v}\right\Vert{}^2  - \left\Vert \vec{u} - \vec{v}\right\Vert{}^2 \right)$
     \end{center}
}
Cette formule est particulièrement utile lorsque l'on connaît les trois côtés d'un triangle ou lorsque l'on connaît 2 côtés et la médiane issus du même point~; on utilise alors souvent une des relations ci-dessous~:
\begin{itemize}
     \item
     $\overrightarrow{BC}=\overrightarrow{BA}+\overrightarrow{AC}=\overrightarrow{AC} - \overrightarrow{AB}$ (Relation de Chasles) \\
     \item
     Si $M$ et le milieu du segment $[BC]\ :$ \\
     $\overrightarrow{AB}+\overrightarrow{AC}=2\overrightarrow{AM}$ (Propriété de la médiane)
\end{itemize}
\bloc{orange}{Exemple}{ % id=150
     Pour la figure ci-dessous, on cherche, là encore, à calculer le produit scalaire $\overrightarrow{AB} \cdot \overrightarrow{AC}$ .
     \begin{center}
          \begin{extern} %width="360" alt="Produit scalaire connaissant trois longueurs"
               \resizebox{8cm}{!}{
                    %
                    \begin{tikzpicture}[line cap=round,line join=round,>=triangle 45,x=1.0cm,y=1.0cm]
                         \clip(0.,0.) rectangle (6.,4.6);
                         \draw [line width=0.4pt,color=tttttt] (1.7793181669564129,3.483291080200642)-- (5.,2.);
                         \draw [line width=0.4pt,color=tttttt] (1.7793181669564129,3.483291080200642)-- (1.,1.);
                         \draw [color=tttttt](3.034842223891809,1.2320966350302005) node {\scriptsize 9};
                         \draw [color=tttttt](1.1443463561232148,2.5470232959447823) node {\scriptsize 6};
                         \draw [line width=0.4pt,color=tttttt] (1.,1.)-- (5.,2.);
                         \draw [color=tttttt](3.396412471825693,3.0854184641932725) node {\scriptsize 8};
                         \begin{scriptsize}
                              \draw [fill=tttttt] (1.,1.) circle (0.5pt);
                              \draw[color=tttttt] (0.922238918106686,0.8257118205349459) node {$A$};
                              \draw [fill=tttttt] (5.,2.) circle (0.5pt);
                              \draw[color=tttttt] (5.188767843726519,1.8766177739430565) node {$B$};
                              \draw [fill=black] (1.7793181669564129,3.483291080200642) circle (0.5pt);
                              \draw[color=black] (1.6815364387678426,3.641932700603972) node {$C$};
                         \end{scriptsize}
                    \end{tikzpicture}
               }
          \end{extern}
     \end{center}
     Dans le triangle ci-dessus, d'après la relation de Chasles~:
     \newpar
     $\overrightarrow{BC}=\overrightarrow{BA}+\overrightarrow{AC}=\overrightarrow{AC} - \overrightarrow{AB}$
     \newpar
     On en déduit, d'après la seconde égalité du théorème précédent~:
     \newpar
     $\overrightarrow{AB} \cdot \overrightarrow{AC} =\frac{1}{2} \left( ||\overrightarrow{AB}||{}^2 +||\overrightarrow{AC}||{}^2  - ||\overrightarrow{BC}{}||^2  \right) $\\
     $\overrightarrow{AB} \cdot \overrightarrow{AC} =\frac{1}{2} \left( 9{}^2 +6{}^2  - 8{}^2  \right)$\\
     $\overrightarrow{AB} \cdot \overrightarrow{AC} =\frac{1}{2}  \times 53=26,5$
}% fin exemple
\begin{h2}4. Se placer dans un repère orthonormé\end{h2}
Dans un \textbf{repère orthonormé}, il est facile de calculer le produit scalaire des vecteurs $\overrightarrow{u}\begin{pmatrix} x \\ y \end{pmatrix} $ et $\overrightarrow{v}\begin{pmatrix} x'  \\ y'  \end{pmatrix} $ grâce à la \mcLien{https://www.maths-cours.fr/cours/produit-scalaire\#t60}{formule} suivante~:
\cadre{rouge}{Théorème}{% id="t200"
     Le plan étant rapporté à un repère orthonormé $\left(O; \vec{i}, \vec{j}\right)$, soient $\overrightarrow{u}\begin{pmatrix} x \\ y \end{pmatrix} $ et $\overrightarrow{v}\begin{pmatrix} x'  \\ y'  \end{pmatrix}$ deux vecteurs du plan; alors :
     \begin{center}$\vec{u} \cdot \vec{v}=xx^{\prime}+yy^{\prime}$\end{center}
}
\bloc{cyan}{Remarque}{ % id=r210
     Lorsque la figure ne comporte pas de repère orthonormé, il est toujours possible d'en choisir un soi-même. Attention toutefois, pour que la formule précédente soit valable, il est important que le repère soit \textbf{orthonormé}.
}% fin remarque
\bloc{orange}{Exemple}{ % id=e220
     Reprenons l'exemple étudié lors de la première méthode en nous plaçant, cette fois, dans le repère  $(A~;~\vec{i},~\vec{j})$ représenté ci-dessous~:
     \begin{center}
          \begin{extern} %width="500" alt="Produit scalaire dans un repère orthonormé"
               \resizebox{8cm}{!}{
                    %
                    \begin{tikzpicture}[line cap=round,line join=round,>=triangle 45,x=1.0cm,y=1.0cm]
                         \draw [color=cqcqcq,, xstep=1.0cm,ystep=1.0cm] (0.,0.) grid (8.,6.);
                         \clip(0.,0.) rectangle (8.,6.);
                         \draw [line width=0.4pt,color=tttttt] (2.,5.)-- (2.,1.);
                         \draw [line width=0.4pt,color=tttttt] (6.,1.)-- (6.,5.);
                         \draw [line width=0.4pt,color=tttttt] (6.,5.)-- (2.,5.);
                         \draw [line width=0.4pt,color=tttttt] (2.,1.)-- (4.,1.);
                         \draw [line width=0.4pt,color=tttttt] (4.,1.)-- (6.,1.);
                         \draw [->,line width=0.4pt,color=qqqqff] (4.,1.) -- (6.,1.);
                         \draw [->,line width=0.4pt,color=qqqqff] (4.,1.) -- (2.,5.);
                         \draw [->,line width=0.8pt,color=ffqqqq] (2.,1.) -- (3.,1.);
                         \draw [->,line width=0.8pt,color=ffqqqq] (2.,1.) -- (2.,2.);
                         \draw [color=ffqqqq](2.4,0.7) node {\scriptsize $\vec{i}$};
                         \draw [color=ffqqqq](1.7,1.5) node {\scriptsize $\vec{j}$};
                         \begin{scriptsize}
                              \draw [fill=tttttt] (2.,1.) circle (0.5pt);
                              \draw[color=ffqqqq] (1.9111570247933876,0.8050043140638496) node {$A$};
                              \draw [fill=tttttt] (6.,1.) circle (0.5pt);
                              \draw[color=tttttt] (6.038223140495865,0.8153580672993975) node {$B$};
                              \draw [fill=tttttt] (6.,5.) circle (0.5pt);
                              \draw[color=tttttt] (6.048553719008262,5.210526315789477) node {$C$};
                              \draw [fill=tttttt] (2.,5.) circle (0.5pt);
                              \draw[color=tttttt] (1.8801652892561975,5.179465056082833) node {$D$};
                              \draw [fill=tttttt] (4.,1.) circle (0.5pt);
                              \draw[color=tttttt] (4.003099173553717,0.8153580672993975) node {$I$};
                         \end{scriptsize}
                    \end{tikzpicture}
               }
          \end{extern}
     \end{center}
     Les coordonnées des points $A, B, C, D, I$ dans le repère orthonormé $(A~;~\vec{i},~\vec{j})$ sont~:\\
     $A(0~;~0)~; B(4~;~0)~;~C(4~;~4)~; D(0~;~4)~;~I(2~;~0) $
     \newpar
     On on déduit les coordonnées des vecteurs $\overrightarrow{IB}$ et $\overrightarrow{ID}~:$\\
     $\overrightarrow{IB}\begin{pmatrix} x_{B}  - x_{I} \\ y_{B} - y_{I} \end{pmatrix} $ donc $\overrightarrow{IB}\begin{pmatrix} 2 \\ 0 \end{pmatrix} $
     \newpar
     $\overrightarrow{ID}\begin{pmatrix} x_{D}  - x_{I} \\ y_{D} - y_{I} \end{pmatrix} $ donc $\overrightarrow{ID}\begin{pmatrix}  - 2 \\ 4 \end{pmatrix} $
     \medbreak
     Par conséquent~:
     \newpar
     $\overrightarrow{IB} \cdot \overrightarrow{ID}=2 \times ( - 2) +4 \times 0= - 4$
}% fin exemple
\begin{h2}5. Utiliser la relation de Chasles\end{h2}
Une autre façon de calculer le produit scalaire de 2 vecteurs consiste à décomposer ces vecteurs en utilisant la relation de Chasles puis à utiliser la \mcLien{https://www.maths-cours.fr/cours/produit-scalaire\#p30}{distributivité} du produit scalaire par rapport à l'addition ou à la soustraction de vecteurs.
\cadre{vert}{Propriété}{% id="p20"
     Pour tous vecteurs $\vec{u}, \vec{v}, \vec{w}~:$
     \begin{center}
          $\vec{u} \cdot \left(\vec{v}+\vec{w}\right)=\vec{u} \cdot \vec{v}+\vec{u} \cdot \vec{w}$
     \end{center}
}
\bloc{cyan}{Remarque}{ % id=r250
     Cette méthode est très générale et elle peut souvent remplacer les méthodes 1 ou 4~;~cependant, elle peut être parfois plus difficile à manier.
}% fin remarque
Sur la figure ci-dessous,   $ABCD$ est un losange dont les diagonales mesurent~: $AC=12$ et $BD=6. $\\
On souhaite calculer le produit scalaire  $\overrightarrow{AB} \cdot \overrightarrow{BC}. $
\bloc{orange}{Exemple}{ % id=e260
     \begin{center}
\imgsvg{5-methodes-pour-calculer-un-produit-scalaire-7}{0.3}% alt="Calculer du produit scalaire 7" style="width:25rem" 
 \end{center}
     Pour trouver le résultat demandé, on peut se placer dans un repère de centre $I$ et employer la méthode précédente. Toutefois, Il est également possible ici de décomposer les vecteurs $\overrightarrow{AB}$ et $\overrightarrow{BC}$ en utilisant la relation de Chasles et en faisant intervenir le point $I$~:
     $\overrightarrow{AB}=\overrightarrow{AI}+\overrightarrow{IB}$ \\
     $\overrightarrow{BC}=\overrightarrow{BI}+\overrightarrow{IC}$
     \newpar
     On peut alors calculer le produit scalaire  $\overrightarrow{AB} \cdot \overrightarrow{BC}$ de la façon suivante~:
     \newpar
     $\overrightarrow{AB} \cdot \overrightarrow{BC}=\left( \overrightarrow{AI}+\overrightarrow{IB} \right)  \cdot \left( \overrightarrow{BI}+\overrightarrow{IC}  \right) $\\
     $\overrightarrow{AB} \cdot \overrightarrow{BC}=\overrightarrow{AI} \cdot \overrightarrow{BI}+\overrightarrow{AI} \cdot \overrightarrow{IC}+\overrightarrow{IB} \cdot \overrightarrow{BI}+\overrightarrow{IB} \cdot \overrightarrow{IC}$
     \newpar
     Comme les vecteurs   $\overrightarrow{AI}$ et $\overrightarrow{BI}$ sont orthogonaux le produit scalaire  $\overrightarrow{AI} \cdot \overrightarrow{BI}$ est nul~;~pour la même raison le produit scalaire $ \overrightarrow{IB}  \cdot \overrightarrow{IC}$ est lui aussi nul.\\
     De plus, $\overrightarrow{IC}=  \overrightarrow{AI}$, $IB=\frac{1}{2} DB=3$ et  $IC=AI=\frac{1}{2} AC=6.$
     \newpar
     Par conséquent~:\\
     $\overrightarrow{AB} \cdot \overrightarrow{BC}=\overrightarrow{AI}{}^2  - \overrightarrow{IB}{}^2  =AI{}^2  - IB{}^2 $\\
     $\overrightarrow{AB} \cdot \overrightarrow{BC}=6{}^2  - 3{}^2 =36 - 9=27.$
     \newpar
     (\textbf{Remarque}~: On peut montrer que ce résultat est encore correct si $ABCD$ est un parallélogramme quelconque et non nécessairement un losange)
}% fin exemple

\end{document}