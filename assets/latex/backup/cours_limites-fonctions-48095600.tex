\documentclass[a4paper]{article}

%================================================================================================================================
%
% Packages
%
%================================================================================================================================

\usepackage[T1]{fontenc} 	% pour caractères accentués
\usepackage[utf8]{inputenc}  % encodage utf8
\usepackage[french]{babel}	% langue : français
\usepackage{fourier}			% caractères plus lisibles
\usepackage[dvipsnames]{xcolor} % couleurs
\usepackage{fancyhdr}		% réglage header footer
\usepackage{needspace}		% empêcher sauts de page mal placés
\usepackage{graphicx}		% pour inclure des graphiques
\usepackage{enumitem,cprotect}		% personnalise les listes d'items (nécessaire pour ol, al ...)
\usepackage{hyperref}		% Liens hypertexte
\usepackage{pstricks,pst-all,pst-node,pstricks-add,pst-math,pst-plot,pst-tree,pst-eucl} % pstricks
\usepackage[a4paper,includeheadfoot,top=2cm,left=3cm, bottom=2cm,right=3cm]{geometry} % marges etc.
\usepackage{comment}			% commentaires multilignes
\usepackage{amsmath,environ} % maths (matrices, etc.)
\usepackage{amssymb,makeidx}
\usepackage{bm}				% bold maths
\usepackage{tabularx}		% tableaux
\usepackage{colortbl}		% tableaux en couleur
\usepackage{fontawesome}		% Fontawesome
\usepackage{environ}			% environment with command
\usepackage{fp}				% calculs pour ps-tricks
\usepackage{multido}			% pour ps tricks
\usepackage[np]{numprint}	% formattage nombre
\usepackage{tikz,tkz-tab} 			% package principal TikZ
\usepackage{pgfplots}   % axes
\usepackage{mathrsfs}    % cursives
\usepackage{calc}			% calcul taille boites
\usepackage[scaled=0.875]{helvet} % font sans serif
\usepackage{svg} % svg
\usepackage{scrextend} % local margin
\usepackage{scratch} %scratch
\usepackage{multicol} % colonnes
%\usepackage{infix-RPN,pst-func} % formule en notation polanaise inversée
\usepackage{listings}

%================================================================================================================================
%
% Réglages de base
%
%================================================================================================================================

\lstset{
language=Python,   % R code
literate=
{á}{{\'a}}1
{à}{{\`a}}1
{ã}{{\~a}}1
{é}{{\'e}}1
{è}{{\`e}}1
{ê}{{\^e}}1
{í}{{\'i}}1
{ó}{{\'o}}1
{õ}{{\~o}}1
{ú}{{\'u}}1
{ü}{{\"u}}1
{ç}{{\c{c}}}1
{~}{{ }}1
}


\definecolor{codegreen}{rgb}{0,0.6,0}
\definecolor{codegray}{rgb}{0.5,0.5,0.5}
\definecolor{codepurple}{rgb}{0.58,0,0.82}
\definecolor{backcolour}{rgb}{0.95,0.95,0.92}

\lstdefinestyle{mystyle}{
    backgroundcolor=\color{backcolour},   
    commentstyle=\color{codegreen},
    keywordstyle=\color{magenta},
    numberstyle=\tiny\color{codegray},
    stringstyle=\color{codepurple},
    basicstyle=\ttfamily\footnotesize,
    breakatwhitespace=false,         
    breaklines=true,                 
    captionpos=b,                    
    keepspaces=true,                 
    numbers=left,                    
xleftmargin=2em,
framexleftmargin=2em,            
    showspaces=false,                
    showstringspaces=false,
    showtabs=false,                  
    tabsize=2,
    upquote=true
}

\lstset{style=mystyle}


\lstset{style=mystyle}
\newcommand{\imgdir}{C:/laragon/www/newmc/assets/imgsvg/}
\newcommand{\imgsvgdir}{C:/laragon/www/newmc/assets/imgsvg/}

\definecolor{mcgris}{RGB}{220, 220, 220}% ancien~; pour compatibilité
\definecolor{mcbleu}{RGB}{52, 152, 219}
\definecolor{mcvert}{RGB}{125, 194, 70}
\definecolor{mcmauve}{RGB}{154, 0, 215}
\definecolor{mcorange}{RGB}{255, 96, 0}
\definecolor{mcturquoise}{RGB}{0, 153, 153}
\definecolor{mcrouge}{RGB}{255, 0, 0}
\definecolor{mclightvert}{RGB}{205, 234, 190}

\definecolor{gris}{RGB}{220, 220, 220}
\definecolor{bleu}{RGB}{52, 152, 219}
\definecolor{vert}{RGB}{125, 194, 70}
\definecolor{mauve}{RGB}{154, 0, 215}
\definecolor{orange}{RGB}{255, 96, 0}
\definecolor{turquoise}{RGB}{0, 153, 153}
\definecolor{rouge}{RGB}{255, 0, 0}
\definecolor{lightvert}{RGB}{205, 234, 190}
\setitemize[0]{label=\color{lightvert}  $\bullet$}

\pagestyle{fancy}
\renewcommand{\headrulewidth}{0.2pt}
\fancyhead[L]{maths-cours.fr}
\fancyhead[R]{\thepage}
\renewcommand{\footrulewidth}{0.2pt}
\fancyfoot[C]{}

\newcolumntype{C}{>{\centering\arraybackslash}X}
\newcolumntype{s}{>{\hsize=.35\hsize\arraybackslash}X}

\setlength{\parindent}{0pt}		 
\setlength{\parskip}{3mm}
\setlength{\headheight}{1cm}

\def\ebook{ebook}
\def\book{book}
\def\web{web}
\def\type{web}

\newcommand{\vect}[1]{\overrightarrow{\,\mathstrut#1\,}}

\def\Oij{$\left(\text{O}~;~\vect{\imath},~\vect{\jmath}\right)$}
\def\Oijk{$\left(\text{O}~;~\vect{\imath},~\vect{\jmath},~\vect{k}\right)$}
\def\Ouv{$\left(\text{O}~;~\vect{u},~\vect{v}\right)$}

\hypersetup{breaklinks=true, colorlinks = true, linkcolor = OliveGreen, urlcolor = OliveGreen, citecolor = OliveGreen, pdfauthor={Didier BONNEL - https://www.maths-cours.fr} } % supprime les bordures autour des liens

\renewcommand{\arg}[0]{\text{arg}}

\everymath{\displaystyle}

%================================================================================================================================
%
% Macros - Commandes
%
%================================================================================================================================

\newcommand\meta[2]{    			% Utilisé pour créer le post HTML.
	\def\titre{titre}
	\def\url{url}
	\def\arg{#1}
	\ifx\titre\arg
		\newcommand\maintitle{#2}
		\fancyhead[L]{#2}
		{\Large\sffamily \MakeUppercase{#2}}
		\vspace{1mm}\textcolor{mcvert}{\hrule}
	\fi 
	\ifx\url\arg
		\fancyfoot[L]{\href{https://www.maths-cours.fr#2}{\black \footnotesize{https://www.maths-cours.fr#2}}}
	\fi 
}


\newcommand\TitreC[1]{    		% Titre centré
     \needspace{3\baselineskip}
     \begin{center}\textbf{#1}\end{center}
}

\newcommand\newpar{    		% paragraphe
     \par
}

\newcommand\nosp {    		% commande vide (pas d'espace)
}
\newcommand{\id}[1]{} %ignore

\newcommand\boite[2]{				% Boite simple sans titre
	\vspace{5mm}
	\setlength{\fboxrule}{0.2mm}
	\setlength{\fboxsep}{5mm}	
	\fcolorbox{#1}{#1!3}{\makebox[\linewidth-2\fboxrule-2\fboxsep]{
  		\begin{minipage}[t]{\linewidth-2\fboxrule-4\fboxsep}\setlength{\parskip}{3mm}
  			 #2
  		\end{minipage}
	}}
	\vspace{5mm}
}

\newcommand\CBox[4]{				% Boites
	\vspace{5mm}
	\setlength{\fboxrule}{0.2mm}
	\setlength{\fboxsep}{5mm}
	
	\fcolorbox{#1}{#1!3}{\makebox[\linewidth-2\fboxrule-2\fboxsep]{
		\begin{minipage}[t]{1cm}\setlength{\parskip}{3mm}
	  		\textcolor{#1}{\LARGE{#2}}    
 	 	\end{minipage}  
  		\begin{minipage}[t]{\linewidth-2\fboxrule-4\fboxsep}\setlength{\parskip}{3mm}
			\raisebox{1.2mm}{\normalsize\sffamily{\textcolor{#1}{#3}}}						
  			 #4
  		\end{minipage}
	}}
	\vspace{5mm}
}

\newcommand\cadre[3]{				% Boites convertible html
	\par
	\vspace{2mm}
	\setlength{\fboxrule}{0.1mm}
	\setlength{\fboxsep}{5mm}
	\fcolorbox{#1}{white}{\makebox[\linewidth-2\fboxrule-2\fboxsep]{
  		\begin{minipage}[t]{\linewidth-2\fboxrule-4\fboxsep}\setlength{\parskip}{3mm}
			\raisebox{-2.5mm}{\sffamily \small{\textcolor{#1}{\MakeUppercase{#2}}}}		
			\par		
  			 #3
 	 		\end{minipage}
	}}
		\vspace{2mm}
	\par
}

\newcommand\bloc[3]{				% Boites convertible html sans bordure
     \needspace{2\baselineskip}
     {\sffamily \small{\textcolor{#1}{\MakeUppercase{#2}}}}    
		\par		
  			 #3
		\par
}

\newcommand\CHelp[1]{
     \CBox{Plum}{\faInfoCircle}{À RETENIR}{#1}
}

\newcommand\CUp[1]{
     \CBox{NavyBlue}{\faThumbsOUp}{EN PRATIQUE}{#1}
}

\newcommand\CInfo[1]{
     \CBox{Sepia}{\faArrowCircleRight}{REMARQUE}{#1}
}

\newcommand\CRedac[1]{
     \CBox{PineGreen}{\faEdit}{BIEN R\'EDIGER}{#1}
}

\newcommand\CError[1]{
     \CBox{Red}{\faExclamationTriangle}{ATTENTION}{#1}
}

\newcommand\TitreExo[2]{
\needspace{4\baselineskip}
 {\sffamily\large EXERCICE #1\ (\emph{#2 points})}
\vspace{5mm}
}

\newcommand\img[2]{
          \includegraphics[width=#2\paperwidth]{\imgdir#1}
}

\newcommand\imgsvg[2]{
       \begin{center}   \includegraphics[width=#2\paperwidth]{\imgsvgdir#1} \end{center}
}


\newcommand\Lien[2]{
     \href{#1}{#2 \tiny \faExternalLink}
}
\newcommand\mcLien[2]{
     \href{https~://www.maths-cours.fr/#1}{#2 \tiny \faExternalLink}
}

\newcommand{\euro}{\eurologo{}}

%================================================================================================================================
%
% Macros - Environement
%
%================================================================================================================================

\newenvironment{tex}{ %
}
{%
}

\newenvironment{indente}{ %
	\setlength\parindent{10mm}
}

{
	\setlength\parindent{0mm}
}

\newenvironment{corrige}{%
     \needspace{3\baselineskip}
     \medskip
     \textbf{\textsc{Corrigé}}
     \medskip
}
{
}

\newenvironment{extern}{%
     \begin{center}
     }
     {
     \end{center}
}

\NewEnviron{code}{%
	\par
     \boite{gray}{\texttt{%
     \BODY
     }}
     \par
}

\newenvironment{vbloc}{% boite sans cadre empeche saut de page
     \begin{minipage}[t]{\linewidth}
     }
     {
     \end{minipage}
}
\NewEnviron{h2}{%
    \needspace{3\baselineskip}
    \vspace{0.6cm}
	\noindent \MakeUppercase{\sffamily \large \BODY}
	\vspace{1mm}\textcolor{mcgris}{\hrule}\vspace{0.4cm}
	\par
}{}

\NewEnviron{h3}{%
    \needspace{3\baselineskip}
	\vspace{5mm}
	\textsc{\BODY}
	\par
}

\NewEnviron{margeneg}{ %
\begin{addmargin}[-1cm]{0cm}
\BODY
\end{addmargin}
}

\NewEnviron{html}{%
}

\begin{document}
\meta{url}{/cours/limites-fonctions/}
\meta{pid}{509}
\meta{titre}{Limites d'une fonction}
\meta{type}{cours}
\begin{h2}1. Définitions\end{h2}
\cadre{bleu}{Définition}{%id="d10"
     \textbf{Limite infinie quand $x$ tend vers l'infini.}
     \par
     Soit $f$ une fonction définie sur un intervalle $\left[a; +\infty \right[$.
     \par
     On dit que que $f\left(x\right)$ tend vers $+\infty $ quand $x$ tend vers $+\infty $ lorsque pour $x$ suffisamment grand, $f\left(x\right)$ est aussi grand que l'on veut. On écrit alors que $\lim\limits_{x\rightarrow +\infty } f\left(x\right)=+\infty $.
}
%<img src="/wp-content/uploads/lim_inf.svg" alt="" class="aligncenter" style="width:350px;"/>
\begin{center}
     \begin{extern} %width="400" alt="limite infinie"
          % -+-+-+ variables modifiables
          \resizebox{8cm}{!}{%
               \def\fonction{ 0.05*x*x + 0.5*x - 1 }
               \def\xmin{-1}
               \def\xmax{13}
               \def\ymin{-2}
               \def\ymax{9}
               \def\xunit{1}  % unités en cm
               \def\yunit{1}
               \psset{xunit=\xunit,yunit=\yunit,algebraic=true}
               \fontsize{15pt}{15pt}\selectfont
               \begin{pspicture*}[linewidth=1pt](\xmin,\ymin)(\xmax,\ymax)
                    %      \psgrid[gridcolor=mcgris, subgriddiv=5, gridlabels=0pt](-5,-0.3)(5,1)
                    \psaxes[Dx=100,Dy=100,linewidth=0.75pt]{->}(0,0)(\xmin,\ymin)(\xmax,\ymax)
                    \psplot[plotpoints=2000,linecolor=blue]{\xmin}{\xmax}{\fonction}
                    \rput[tr](-0.1,-0.1){$O$}
                    \rput[tl](9.5,8){$\color{blue} \mathcal{C}_f$}
               \end{pspicture*}
          }
     \end{extern}
\end{center}
\begin{center}$\lim\limits_{x\rightarrow +\infty } f\left(x\right)=+\infty $\end{center}
\bloc{cyan}{Remarque}{%id="r10"
     On définit de façon similaire les limites :
     \par
     $\lim\limits_{x\rightarrow +\infty } f\left(x\right)=-\infty $ ; $\lim\limits_{x\rightarrow -\infty } f\left(x\right)=+\infty $ ; $\lim\limits_{x\rightarrow -\infty } f\left(x\right)=-\infty $.
}
\cadre{bleu}{Définition}{%id="d20"
     \textbf{Limite finie quand $x$ tend vers l'infini.}
     \par
     Soit $f$ une fonction définie sur un intervalle $\left[a ; +\infty \right[$.
     \par
     On dit que que $f\left(x\right)$ tend vers $l$ quand $x$ tend vers $+\infty $ lorsque pour $x$ suffisamment grand, $f\left(x\right)$ est aussi proche de $l$ que l'on veut. On écrit alors que $\lim\limits_{x\rightarrow +\infty } f\left(x\right)=l$.
}
\begin{center}
     \begin{extern} %width="400" alt="limite nulle"
          \resizebox{8cm}{!}{%
               % -+-+-+ variables modifiables
               \def\fonction{15/(x*x+3*x+4) }
               \def\xmin{-1}
               \def\xmax{13}
               \def\ymin{-2}
               \def\ymax{9}
               \def\xunit{1}  % unités en cm
               \def\yunit{1}
               \psset{xunit=\xunit,yunit=\yunit,algebraic=true}
               \fontsize{15pt}{15pt}\selectfont
               \begin{pspicture*}[linewidth=1pt](\xmin,\ymin)(\xmax,\ymax)
                    %      \psgrid[gridcolor=mcgris, subgriddiv=5, gridlabels=0pt](-5,-0.3)(5,1)
                    \psaxes[Dx=100,Dy=100,linewidth=0.75pt]{->}(0,0)(\xmin,\ymin)(\xmax,\ymax)
                    \psplot[plotpoints=2000,linecolor=blue]{\xmin}{\xmax}{\fonction}
                    \rput[tr](-0.1,-0.1){$O$}
                    \rput[tl](-1,4.5){$\color{blue} \mathcal{C}_f$}
               \end{pspicture*}
          }
     \end{extern}
\end{center}
\begin{center}$\lim\limits_{x\rightarrow +\infty } f\left(x\right)=0$\end{center}
\bloc{cyan}{Remarque}{%id="r20"
     On définit de façon similaire la limite $\lim\limits_{x\rightarrow -\infty } f\left(x\right)=l$.
}
\cadre{bleu}{Définition}{%id="d30"
     Si $\lim\limits_{x\rightarrow -\infty }f\left(x\right)=l$ ou $\lim\limits_{x\rightarrow +\infty }f\left(x\right)=l$, on dit que la droite d'équation $y=l$ est \textbf{asymptote horizontale} à la courbe représentative de la fonction $f$.
}
\bloc{orange}{Exemple}{%id="e30"
     Sur la courbe ci-dessus, la droite d'équation $y=0$ est \textbf{asymptote horizontale} à la courbe représentative de $f$.
}
\cadre{bleu}{Définition}{%id="d40"
     \textbf{Limite infinie quand $x$ tend vers un réel.}
     \par
     Soit $f$ une fonction définie sur un intervalle $\left]a; b\right[$ (avec $a < b$).
     \par
     On dit que que $f\left(x\right)$ tend vers $+\infty $ quand $x$ tend vers $a$ par valeurs supérieures lorsque $f\left(x\right)$ est aussi grand que l'on veut quand $x$ se rapproche de $a$ en restant supérieur à $a$. On écrit alors $\lim\limits_{x\rightarrow a^+} f\left(x\right)=+\infty $ ou $\lim\limits_{\scriptstyle x\rightarrow a \atop\scriptstyle x > a} f\left(x\right)=+\infty $.
     \par
     De même, on dit que que $f\left(x\right)$ tend vers $+\infty $ quand $x$ tend vers $b$ par valeurs inférieures lorsque $f\left(x\right)$ est aussi grand que l'on veut quand $x$ se rapproche de $b$ en restant inférieur à $b$. On écrit alors $\lim\limits_{x\rightarrow b^-} f\left(x\right)=+\infty $ ou $\lim\limits_{\scriptstyle x\rightarrow b \atop\scriptstyle  x < b} f\left(x\right)=+\infty $.
     \par
     Enfin, si $c\in \left]a;b\right[$ , on dit que que $f\left(x\right)$ tend vers $+\infty $ quand $x$ tend vers $c$ si $f\left(x\right)$ tend vers $+\infty $ quand $x$ tend vers $c$ par valeurs supérieures et par valeurs inférieures. On écrit alors $\lim\limits_{x\rightarrow c} f\left(x\right)=+\infty $.
}
\bloc{cyan}{Remarque}{%id="r40"
     On définit de façon symétrique $\lim\limits_{x\rightarrow a^-} f\left(x\right)=-\infty $, $\lim\limits_{x\rightarrow a^+} f\left(x\right)=-\infty $ et $\lim\limits_{x\rightarrow a} f\left(x\right)=-\infty $ en remplaçant «\textit{ $f\left(x\right)$ est aussi grand que l'on veut} » par « \textit{$f\left(x\right)$ est aussi petit que l'on veut} » dans la définition.
}
\cadre{bleu}{Définition}{%id="d50"
     Si $\lim\limits_{x\rightarrow c^-}f\left(x\right)=\pm \infty $ ou $\lim\limits_{x\rightarrow c^+}f\left(x\right)=\pm \infty $ ou $\lim\limits_{x\rightarrow c}f\left(x\right)=\pm \infty $, on dit que la droite d'équation $x=c$ est \textbf{asymptote verticale} à la courbe représentative de la fonction $f$.
}
\bloc{orange}{Exemple}{%id="e50"
     Sur les trois courbes de la figure ci-dessous, la droite d'équation $x=0$ est \textbf{asymptote verticale} à la courbe représentative de $f$.
}
\begin{center}
     \begin{extern} %width="600" alt="limite à droite et à gauche"
          % -+-+-+ variables modifiables
          \def\f{4/abs(x)-abs(x) }
          \def\g{4/(x+9)-(x+9) }
          \def\h{-4/(x+14)+(x+14) }
          \def\xmin{-18}
          \def\xmax{3}
          \def\ymin{-4.8}
          \def\ymax{9}
          \def\xunit{0.5}  % unités en cm
          \def\yunit{0.5}
          \psset{xunit=\xunit,yunit=\yunit,algebraic=true}
          \fontsize{9pt}{9pt}\selectfont
          \begin{pspicture*}[linewidth=1pt](\xmin,\ymin)(\xmax,\ymax)
               %      \psgrid[gridcolor=mcgris, subgriddiv=5, gridlabels=0pt](-5,-0.3)(5,1)
               %   \psaxes[Dx=100,Dy=100,linewidth=0.75pt]{->}(0,0)(\xmin,-2)(\xmax,\ymax)
               \psplot[plotpoints=2000,linecolor=blue,linewidth=0.75pt]{-3}{3}{\f}
               \psplot[plotpoints=2000,linecolor=rouge,linewidth=0.75pt]{-8.9}{-6}{\g}
               \psplot[plotpoints=2000,linecolor=vert,linewidth=0.75pt]{-17}{-14.01}{\h}
               \rput[tr](-0.1,-0.1){$O$}
               \rput[tr](-9.1,-0.1){$O$}
               \rput[tr](-14.1,-0.1){$O$}
               \rput[tl](0.7,7.5){$\color{blue} \mathcal{C}_f$}
               \rput[tl](-8.3,7.5){$\color{rouge} \mathcal{C}_f$}
               \rput[tl](-16,7.5){$\color{vert} \mathcal{C}_f$}
               \psline[linewidth=0.75pt]{->}(0,-2)(0,9)
               \psline[linewidth=0.75pt]{->}(-9,-2)(-9,9)
               \psline[linewidth=0.75pt]{->}(-14,-2)(-14,9)
               \psline[linewidth=0.75pt]{->}(-3,0)(3,0)
               \psline[linewidth=0.75pt]{->}(-10,0)(-5,0)
               \psline[linewidth=0.75pt]{->}(-17,0)(-12,0)
               \rput[t](-14.5,-3){$\lim\limits_{x\rightarrow 0^{-}} f\left(x\right)=+\infty$}
               \rput[t](-7.5,-3){$\lim\limits_{x\rightarrow 0^{+}} f\left(x\right)=+\infty$}
               \rput[t](0,-3){$\lim\limits_{x\rightarrow 0} f\left(x\right)=+\infty$}
          \end{pspicture*}
     \end{extern}
\end{center}
\cadre{bleu}{Définition}{%id="d60"
     \textbf{Limite finie quand x tend vers un réel}.
     \par
     Soit $f$ une fonction définie sur un intervalle $\left]a;b\right[$ (avec $a < b$).
     \par
     On dit que que $f\left(x\right)$ tend vers $l$ quand $x$ tend vers $a$ par valeurs supérieures lorsque $f\left(x\right)$ se rapproche de $l$ quand x se rapproche de $a$ en restant supérieur à $a$.
     \par
     On écrit alors $\lim\limits_{x\rightarrow a^+} f\left(x\right)=l$ ou $\lim\limits_{\begin{matrix}x\rightarrow a \\ x > a\end{matrix}} f\left(x\right)=l$.
     \par
     De même, on dit que que $f\left(x\right)$ tend vers $l$ quand $x$ tend vers $b$ par valeurs inférieures lorsque $f\left(x\right)$ se rapproche de $l$ quand x se rapproche de $b$ en restant inférieur à $b$.
     \par
     On écrit alors $\lim\limits_{x\rightarrow b^-} f\left(x\right)=l$ ou $\lim\limits_{\begin{matrix}x\rightarrow b \\ x < b\end{matrix}} f\left(x\right)=l $.
     \par
     Enfin, si $c\in \left]a; b\right[$ , on dit que que $f\left(x\right)$ tend vers $l$ quand $x$ tend vers $c$ si $f\left(x\right)$ tend vers $l$ quand $x$ tend vers $c$ par valeurs supérieures et par valeurs inférieures.
     \par
     On écrit alors $\lim\limits_{x\rightarrow c} f\left(x\right)=l$.
}
\begin{h2}2. Limites usuelles\end{h2}
\cadre{vert}{Propriétés}{%id="p70"
     Pour tout entier $n > 1$ :
     \begin{itemize}
          \item $\lim\limits_{x\rightarrow -\infty }x^{n}=\left\{ \begin{matrix} -\infty \text{ si n est impair} \\ +\infty \text{ si n est pair} \end{matrix}\right. $
          \item $\lim\limits_{x\rightarrow +\infty }x^{n}=+\infty $
          \item $\lim\limits_{x\rightarrow -\infty }\frac{1}{x^{n}}=0$
          \item $\lim\limits_{x\rightarrow +\infty }\frac{1}{x^{n}}=0$
          \item $\lim\limits_{x\rightarrow 0^-}\frac{1}{x}=-\infty $
          \item $\lim\limits_{x\rightarrow 0^+}\frac{1}{x}=+\infty $
          \item $\lim\limits_{x\rightarrow +\infty }\sqrt{x}=+\infty $.
     \end{itemize}
}
\begin{h2}3. Opérations sur les limites\end{h2}
\cadre{vert}{Propriétés}{%id="p70"
     \textbf{Limite d'une somme.}
     \par
     $a$ désigne un réel ou $+\infty $ ou $-\infty $.
     \begin{center}
           \begin{tabularx}{0.8\linewidth}{|*{3}{>{\centering \arraybackslash }X|}}%class="compact" width="600"
               \hline
               $\lim\limits_{x\rightarrow a}f\left(x\right)$ & $\lim\limits_{x\rightarrow a}g\left(x\right)$ & $\lim\limits_{x\rightarrow a}f\left(x\right)+g\left(x\right)$          \\ \hline
               $l$ & $l^{\prime}$ & $l+l^{\prime}$          \\ \hline
               $l$ & $+\infty $ & $+\infty $          \\ \hline
               $l$ & $-\infty $ & $-\infty $          \\ \hline
               $+\infty $ & $+\infty $ & $+\infty $          \\ \hline
               $-\infty $ & $-\infty $ & $-\infty $          \\ \hline
               $+\infty $ & $-\infty $ & $F.I.$           \\ \hline
          \end{tabularx}
     \end{center}     $F.I.$ signifie forme indéterminée.
}
\bloc{cyan}{Remarque}{%id="r70"
     \textit{« Forme indéterminée »} ne signifie \textbf{pas} que la limite n'existe pas mais que les formules d'opérations sur les limites ne permettent pas de trouver directement limite. Pour la calculer, il faut alors \textit{« lever l'indétermination »} par exemple en simplifiant une fraction (cf. \textit{fiches méthodes}).
}
\cadre{vert}{Propriétés}{%id="p80"
     \textbf{Limite d'un produit.}
     \par
     $a$ désigne un réel ou $+\infty $ ou $-\infty $.
     \begin{center}

          \begin{tabularx}{0.8\linewidth}{|*{3}{>{\centering \arraybackslash }X|}}%class="compact" width="600"
               \hline
               $\lim\limits_{x\rightarrow a}f\left(x\right)$ & $\lim\limits_{x\rightarrow a}g\left(x\right)$ & $\lim\limits_{x\rightarrow a}f\left(x\right)\times g\left(x\right)$          \\ \hline
               $l$ & $l^{\prime}$ & $l\times l^{\prime}$          \\ \hline
               $l\neq 0$ & $\pm \infty $ & $\left(signe\right)\infty $          \\ \hline
               $\pm \infty $ & $\pm \infty $ & $\left(signe\right)\infty $          \\ \hline
               $0$ & $\pm \infty $ & $F.I.$     \\ \hline
          \end{tabularx}
     \end{center}
     \begin{itemize}
          \item $F.I.$ signifie forme indéterminée.
          \item $\pm \infty $ signifie que la formule s'applique pour $+\infty $ et pour $-\infty $.
          \item $\left(signe\right)\infty $ signifie que l'on utilise la règle des signes usuelle :
          \par
          $+\times +=+$
          \par
          $+\times -=-$
          \par
          $-\times -=+$
          \par
          pour déterminer si la limite vaut $+\infty $ ou $-\infty $.
     \end{itemize}
}
\cadre{vert}{Propriétés}{%id="p90"
     \textbf{Limite d'un quotient.}
     \par
     $a$ désigne un réel ou $+\infty $ ou $-\infty $.
     \begin{center}
          \def\arraystretch{3}%
          \begin{tabularx}{0.8\linewidth}{|*{3}{>{\centering \arraybackslash }X|}}%class="compact" width="600"
               \hline
               $\lim\limits_{x\rightarrow a}f\left(x\right)$ & $\lim\limits_{x\rightarrow a}g\left(x\right)$ & $\lim\limits_{x\rightarrow a}\frac{f\left(x\right)}{g\left(x\right)}$          \\ \hline
               $l$ & $l^{\prime}\neq 0$ & $\frac{l}{l^{\prime}}$          \\ \hline
               $l\neq 0$ & $0$ & $\left(signe\right)\infty $          \\ \hline
               $0$ & $0$ & $F.I.$          \\ \hline
               $l$ & $\pm \infty $ & $0$          \\ \hline
               $\pm \infty $ & $l$ & $\left(signe\right)\infty $          \\ \hline
               $\pm \infty $ & $\pm \infty $ & $F.I.$         \\ \hline
          \end{tabularx}
     \end{center}
}
\cadre{vert}{Propriété}{%id="p100"
     \textbf{Limite d'une fonction composée.}
     \par
     $a$, $b$ et $c$ désignent des réels ou $+\infty $ ou $-\infty $.
     \par
     Si $\lim\limits_{x\rightarrow a}f\left(x\right)=\color{red}{b}$ et $\lim\limits_{x\rightarrow \color{red}{b}}g\left(x\right)=c$ alors :
     \par
     $\lim\limits_{x\rightarrow a}g\left(f\left(x\right)\right)=c$.
}
\bloc{cyan}{Remarque}{%id="r100"
     On pose souvent $X=f\left(x\right)$ («changement de variable») et on écrit alors :
     \par
     $\lim\limits_{x\rightarrow a}X=\lim\limits_{x\rightarrow a}f\left(x\right)=b$
     \par
     $\lim\limits_{x\rightarrow a}g\left(f\left(x\right)\right)=\lim\limits_{X\rightarrow b}g\left(X\right)=c$.
}
\bloc{orange}{Exemple}{%id="e100"
     On cherche à calculer :
     \par
     $\lim\limits_{x\rightarrow -\infty }\sqrt{1+x^{2}}$.
     \par
     On pose $X=1+x^{2}$. Alors :
     \par
     $\lim\limits_{x\rightarrow -\infty }X=\lim\limits_{x\rightarrow -\infty }1+x^{2}=+\infty $
     \par
     et
     \par
     $\lim\limits_{x\rightarrow -\infty }\sqrt{1+x^{2}}=\lim\limits_{X\rightarrow +\infty }\sqrt{X}=+\infty $.
}
\begin{h2}4. Théorèmes de comparaison\end{h2}
\cadre{rouge}{Théorèmes}{%id="t110"
     \begin{itemize}
          \item Si $f\left(x\right)\geqslant g\left(x\right)$ sur un intervalle de la forme $\left[a;+\infty \right[$ et si $\lim\limits_{x\rightarrow +\infty }g\left(x\right)=+\infty $ alors :
          \par
          $\lim\limits_{x\rightarrow +\infty }f\left(x\right)=+\infty $.
          \item Si $f\left(x\right)\leqslant g\left(x\right)$ sur un intervalle de la forme $\left[a;+\infty \right[$ et si $\lim\limits_{x\rightarrow +\infty }g\left(x\right)=-\infty $ alors :
          \par
          $\lim\limits_{x\rightarrow +\infty }f\left(x\right)=-\infty $.
     \end{itemize}
}
\cadre{rouge}{Théorème}{%id="t120"
     \textbf{Théorème des "gendarmes".}
     \par
     Si $g\left(x\right)\leqslant f\left(x\right)\leqslant h\left(x\right)$ sur un intervalle de la forme $\left[a;+\infty \right[$ et si $\lim\limits_{x\rightarrow +\infty }g\left(x\right)=\lim\limits_{x\rightarrow +\infty }h\left(x\right)=l$ alors :
     \par
     $\lim\limits_{x\rightarrow +\infty }f\left(x\right)=l.$
}
\begin{center}
     \begin{extern} %width="400" alt="Théorème des gendarmes"
          \resizebox{8cm}{!}{%
               % -+-+-+ variables modifiables
               \def\fonction{10* sin(2*x)/(x+3)^2+3 }
               \def\g{25/(x+3)^2 +3}
               \def\h{-25/(x+3)^2 +3}
               \def\xmin{-1}
               \def\xmax{13}
               \def\ymin{-2}
               \def\ymax{9}
               \def\xunit{1}  % unités en cm
               \def\yunit{1}
               \psset{xunit=\xunit,yunit=\yunit,algebraic=true}
               \fontsize{15pt}{15pt}\selectfont
               \begin{pspicture*}[linewidth=1pt](\xmin,\ymin)(\xmax,\ymax)
                    %      \psgrid[gridcolor=mcgris, subgriddiv=5, gridlabels=0pt](-5,-0.3)(5,1)
                    \psaxes[Dx=100,Dy=100,linewidth=0.75pt]{->}(0,0)(\xmin,\ymin)(\xmax,\ymax)
                    \psline[linecolor=gray](-3,3)(14,3)
                    \psplot[plotpoints=2000,linecolor=rouge]{\xmin}{\xmax}{\fonction}
                    \psplot[plotpoints=2000,linecolor=blue]{\xmin}{\xmax}{\g}
                    \psplot[plotpoints=2000,linecolor=vert]{\xmin}{\xmax}{\h}
                    \rput[tr](-0.1,-0.1){$O$}
                    \rput[tl](-1,6.5){$\color{blue} \mathcal{C}_h$}
                    \rput[tl](-1,3.5){$\color{rouge} \mathcal{C}_f$}
                    \rput[tl](-1,0.6){$\color{vert} \mathcal{C}_g$}
               \end{pspicture*}
          }
     \end{extern}
\end{center}
\begin{center}Théorème des gendarmes \end{center}
\bloc{cyan}{Remarque}{%id="r120"
     On a des théorèmes similaires lorsque $x \rightarrow -\infty $.
}

\end{document}