\documentclass[a4paper]{article}

%================================================================================================================================
%
% Packages
%
%================================================================================================================================

\usepackage[T1]{fontenc} 	% pour caractères accentués
\usepackage[utf8]{inputenc}  % encodage utf8
\usepackage[french]{babel}	% langue : français
\usepackage{fourier}			% caractères plus lisibles
\usepackage[dvipsnames]{xcolor} % couleurs
\usepackage{fancyhdr}		% réglage header footer
\usepackage{needspace}		% empêcher sauts de page mal placés
\usepackage{graphicx}		% pour inclure des graphiques
\usepackage{enumitem,cprotect}		% personnalise les listes d'items (nécessaire pour ol, al ...)
\usepackage{hyperref}		% Liens hypertexte
\usepackage{pstricks,pst-all,pst-node,pstricks-add,pst-math,pst-plot,pst-tree,pst-eucl} % pstricks
\usepackage[a4paper,includeheadfoot,top=2cm,left=3cm, bottom=2cm,right=3cm]{geometry} % marges etc.
\usepackage{comment}			% commentaires multilignes
\usepackage{amsmath,environ} % maths (matrices, etc.)
\usepackage{amssymb,makeidx}
\usepackage{bm}				% bold maths
\usepackage{tabularx}		% tableaux
\usepackage{colortbl}		% tableaux en couleur
\usepackage{fontawesome}		% Fontawesome
\usepackage{environ}			% environment with command
\usepackage{fp}				% calculs pour ps-tricks
\usepackage{multido}			% pour ps tricks
\usepackage[np]{numprint}	% formattage nombre
\usepackage{tikz,tkz-tab} 			% package principal TikZ
\usepackage{pgfplots}   % axes
\usepackage{mathrsfs}    % cursives
\usepackage{calc}			% calcul taille boites
\usepackage[scaled=0.875]{helvet} % font sans serif
\usepackage{svg} % svg
\usepackage{scrextend} % local margin
\usepackage{scratch} %scratch
\usepackage{multicol} % colonnes
%\usepackage{infix-RPN,pst-func} % formule en notation polanaise inversée
\usepackage{listings}

%================================================================================================================================
%
% Réglages de base
%
%================================================================================================================================

\lstset{
language=Python,   % R code
literate=
{á}{{\'a}}1
{à}{{\`a}}1
{ã}{{\~a}}1
{é}{{\'e}}1
{è}{{\`e}}1
{ê}{{\^e}}1
{í}{{\'i}}1
{ó}{{\'o}}1
{õ}{{\~o}}1
{ú}{{\'u}}1
{ü}{{\"u}}1
{ç}{{\c{c}}}1
{~}{{ }}1
}


\definecolor{codegreen}{rgb}{0,0.6,0}
\definecolor{codegray}{rgb}{0.5,0.5,0.5}
\definecolor{codepurple}{rgb}{0.58,0,0.82}
\definecolor{backcolour}{rgb}{0.95,0.95,0.92}

\lstdefinestyle{mystyle}{
    backgroundcolor=\color{backcolour},   
    commentstyle=\color{codegreen},
    keywordstyle=\color{magenta},
    numberstyle=\tiny\color{codegray},
    stringstyle=\color{codepurple},
    basicstyle=\ttfamily\footnotesize,
    breakatwhitespace=false,         
    breaklines=true,                 
    captionpos=b,                    
    keepspaces=true,                 
    numbers=left,                    
xleftmargin=2em,
framexleftmargin=2em,            
    showspaces=false,                
    showstringspaces=false,
    showtabs=false,                  
    tabsize=2,
    upquote=true
}

\lstset{style=mystyle}


\lstset{style=mystyle}
\newcommand{\imgdir}{C:/laragon/www/newmc/assets/imgsvg/}
\newcommand{\imgsvgdir}{C:/laragon/www/newmc/assets/imgsvg/}

\definecolor{mcgris}{RGB}{220, 220, 220}% ancien~; pour compatibilité
\definecolor{mcbleu}{RGB}{52, 152, 219}
\definecolor{mcvert}{RGB}{125, 194, 70}
\definecolor{mcmauve}{RGB}{154, 0, 215}
\definecolor{mcorange}{RGB}{255, 96, 0}
\definecolor{mcturquoise}{RGB}{0, 153, 153}
\definecolor{mcrouge}{RGB}{255, 0, 0}
\definecolor{mclightvert}{RGB}{205, 234, 190}

\definecolor{gris}{RGB}{220, 220, 220}
\definecolor{bleu}{RGB}{52, 152, 219}
\definecolor{vert}{RGB}{125, 194, 70}
\definecolor{mauve}{RGB}{154, 0, 215}
\definecolor{orange}{RGB}{255, 96, 0}
\definecolor{turquoise}{RGB}{0, 153, 153}
\definecolor{rouge}{RGB}{255, 0, 0}
\definecolor{lightvert}{RGB}{205, 234, 190}
\setitemize[0]{label=\color{lightvert}  $\bullet$}

\pagestyle{fancy}
\renewcommand{\headrulewidth}{0.2pt}
\fancyhead[L]{maths-cours.fr}
\fancyhead[R]{\thepage}
\renewcommand{\footrulewidth}{0.2pt}
\fancyfoot[C]{}

\newcolumntype{C}{>{\centering\arraybackslash}X}
\newcolumntype{s}{>{\hsize=.35\hsize\arraybackslash}X}

\setlength{\parindent}{0pt}		 
\setlength{\parskip}{3mm}
\setlength{\headheight}{1cm}

\def\ebook{ebook}
\def\book{book}
\def\web{web}
\def\type{web}

\newcommand{\vect}[1]{\overrightarrow{\,\mathstrut#1\,}}

\def\Oij{$\left(\text{O}~;~\vect{\imath},~\vect{\jmath}\right)$}
\def\Oijk{$\left(\text{O}~;~\vect{\imath},~\vect{\jmath},~\vect{k}\right)$}
\def\Ouv{$\left(\text{O}~;~\vect{u},~\vect{v}\right)$}

\hypersetup{breaklinks=true, colorlinks = true, linkcolor = OliveGreen, urlcolor = OliveGreen, citecolor = OliveGreen, pdfauthor={Didier BONNEL - https://www.maths-cours.fr} } % supprime les bordures autour des liens

\renewcommand{\arg}[0]{\text{arg}}

\everymath{\displaystyle}

%================================================================================================================================
%
% Macros - Commandes
%
%================================================================================================================================

\newcommand\meta[2]{    			% Utilisé pour créer le post HTML.
	\def\titre{titre}
	\def\url{url}
	\def\arg{#1}
	\ifx\titre\arg
		\newcommand\maintitle{#2}
		\fancyhead[L]{#2}
		{\Large\sffamily \MakeUppercase{#2}}
		\vspace{1mm}\textcolor{mcvert}{\hrule}
	\fi 
	\ifx\url\arg
		\fancyfoot[L]{\href{https://www.maths-cours.fr#2}{\black \footnotesize{https://www.maths-cours.fr#2}}}
	\fi 
}


\newcommand\TitreC[1]{    		% Titre centré
     \needspace{3\baselineskip}
     \begin{center}\textbf{#1}\end{center}
}

\newcommand\newpar{    		% paragraphe
     \par
}

\newcommand\nosp {    		% commande vide (pas d'espace)
}
\newcommand{\id}[1]{} %ignore

\newcommand\boite[2]{				% Boite simple sans titre
	\vspace{5mm}
	\setlength{\fboxrule}{0.2mm}
	\setlength{\fboxsep}{5mm}	
	\fcolorbox{#1}{#1!3}{\makebox[\linewidth-2\fboxrule-2\fboxsep]{
  		\begin{minipage}[t]{\linewidth-2\fboxrule-4\fboxsep}\setlength{\parskip}{3mm}
  			 #2
  		\end{minipage}
	}}
	\vspace{5mm}
}

\newcommand\CBox[4]{				% Boites
	\vspace{5mm}
	\setlength{\fboxrule}{0.2mm}
	\setlength{\fboxsep}{5mm}
	
	\fcolorbox{#1}{#1!3}{\makebox[\linewidth-2\fboxrule-2\fboxsep]{
		\begin{minipage}[t]{1cm}\setlength{\parskip}{3mm}
	  		\textcolor{#1}{\LARGE{#2}}    
 	 	\end{minipage}  
  		\begin{minipage}[t]{\linewidth-2\fboxrule-4\fboxsep}\setlength{\parskip}{3mm}
			\raisebox{1.2mm}{\normalsize\sffamily{\textcolor{#1}{#3}}}						
  			 #4
  		\end{minipage}
	}}
	\vspace{5mm}
}

\newcommand\cadre[3]{				% Boites convertible html
	\par
	\vspace{2mm}
	\setlength{\fboxrule}{0.1mm}
	\setlength{\fboxsep}{5mm}
	\fcolorbox{#1}{white}{\makebox[\linewidth-2\fboxrule-2\fboxsep]{
  		\begin{minipage}[t]{\linewidth-2\fboxrule-4\fboxsep}\setlength{\parskip}{3mm}
			\raisebox{-2.5mm}{\sffamily \small{\textcolor{#1}{\MakeUppercase{#2}}}}		
			\par		
  			 #3
 	 		\end{minipage}
	}}
		\vspace{2mm}
	\par
}

\newcommand\bloc[3]{				% Boites convertible html sans bordure
     \needspace{2\baselineskip}
     {\sffamily \small{\textcolor{#1}{\MakeUppercase{#2}}}}    
		\par		
  			 #3
		\par
}

\newcommand\CHelp[1]{
     \CBox{Plum}{\faInfoCircle}{À RETENIR}{#1}
}

\newcommand\CUp[1]{
     \CBox{NavyBlue}{\faThumbsOUp}{EN PRATIQUE}{#1}
}

\newcommand\CInfo[1]{
     \CBox{Sepia}{\faArrowCircleRight}{REMARQUE}{#1}
}

\newcommand\CRedac[1]{
     \CBox{PineGreen}{\faEdit}{BIEN R\'EDIGER}{#1}
}

\newcommand\CError[1]{
     \CBox{Red}{\faExclamationTriangle}{ATTENTION}{#1}
}

\newcommand\TitreExo[2]{
\needspace{4\baselineskip}
 {\sffamily\large EXERCICE #1\ (\emph{#2 points})}
\vspace{5mm}
}

\newcommand\img[2]{
          \includegraphics[width=#2\paperwidth]{\imgdir#1}
}

\newcommand\imgsvg[2]{
       \begin{center}   \includegraphics[width=#2\paperwidth]{\imgsvgdir#1} \end{center}
}


\newcommand\Lien[2]{
     \href{#1}{#2 \tiny \faExternalLink}
}
\newcommand\mcLien[2]{
     \href{https~://www.maths-cours.fr/#1}{#2 \tiny \faExternalLink}
}

\newcommand{\euro}{\eurologo{}}

%================================================================================================================================
%
% Macros - Environement
%
%================================================================================================================================

\newenvironment{tex}{ %
}
{%
}

\newenvironment{indente}{ %
	\setlength\parindent{10mm}
}

{
	\setlength\parindent{0mm}
}

\newenvironment{corrige}{%
     \needspace{3\baselineskip}
     \medskip
     \textbf{\textsc{Corrigé}}
     \medskip
}
{
}

\newenvironment{extern}{%
     \begin{center}
     }
     {
     \end{center}
}

\NewEnviron{code}{%
	\par
     \boite{gray}{\texttt{%
     \BODY
     }}
     \par
}

\newenvironment{vbloc}{% boite sans cadre empeche saut de page
     \begin{minipage}[t]{\linewidth}
     }
     {
     \end{minipage}
}
\NewEnviron{h2}{%
    \needspace{3\baselineskip}
    \vspace{0.6cm}
	\noindent \MakeUppercase{\sffamily \large \BODY}
	\vspace{1mm}\textcolor{mcgris}{\hrule}\vspace{0.4cm}
	\par
}{}

\NewEnviron{h3}{%
    \needspace{3\baselineskip}
	\vspace{5mm}
	\textsc{\BODY}
	\par
}

\NewEnviron{margeneg}{ %
\begin{addmargin}[-1cm]{0cm}
\BODY
\end{addmargin}
}

\NewEnviron{html}{%
}

\begin{document}
\meta{url}{/exercices/qcm-bac-es-l-asie-2018/}
\meta{pid}{9371}
\meta{titre}{QCM – Bac ES/L Asie 2018}
\meta{type}{exercices}
%
\begin{h2}Exercice 1 (5 points)\end{h2}
\par
\textbf{Commun  à tous les candidats}
\bigbreak
\par
\emph{Cet exercice est un QCM (questionnaire à choix multiples). Pour chacune des questions
     posées, une seule des quatre réponses proposées est exacte. Indiquer sur la copie le
numéro de la question et recopier la lettre de la réponse choisie.}
\medbreak
\textbf{Aucune justification n'est demandée.}
\medbreak
\emph{Une réponse exacte rapporte 1 point~; une réponse fausse, une réponse multiple ou l'absence de réponse ne rapporte ni n'enlève de point.}
\medbreak
\begin{enumerate}
     \item Pour la recherche d'un emploi, une personne envoie sa candidature à $25$ entreprises.
     \par
     La probabilité qu'une entreprise lui réponde est de $0,2$ et on suppose que ces réponses
     sont indépendantes.
     \par
     Quelle est la probabilité, arrondie au centième, que la personne reçoive au moins $5$
     réponses~?
     \medbreak
     \begin{tabularx}{\linewidth}{*{2}{X}}%class="noborder cel50"
          \textbf{a.~~} 0,20 &\textbf{b.~~} 0,62\\
          \textbf{c.~~}0,38  &\textbf{d.~~} 0,58
     \end{tabularx}
     \medbreak
     \item Pour tout événement $E$ on note $P(E)$ sa probabilité. $X$ est une variable aléatoire suivant la loi normale d'espérance $30$ et d'écart type $\sigma$. Alors~:
     \medbreak
     \textbf{a.~~} $P(X = 30) = 0,5$ \\
     \textbf{b.~~} $P(X < 40 ) < 0,5$\\
     \textbf{c.~~} $P(X < 20) = P(X > 40)$\\
     \textbf{d.~~} $P(X) < 20) > P(X < 30)$\\
     \medbreak
     \item En France, les ventes de tablettes numériques sont passées de 6,2 millions d'unités en 2014 à 4,3 millions d'unités en 2016. Les ventes ont diminué, entre 2014 et 2016,
     d'environ~:
     \medbreak
     \begin{tabularx}{\linewidth}{*{2}{X}}%class="noborder cel50"
          \textbf{a.~~} 65\,\%&\textbf{b.~~}31\,\% \\
          \textbf{c.~~} 20\,\%&\textbf{d.~~}  17\,\%
     \end{tabularx}
     \medbreak
     Pour les questions 4 et 5, on donne ci-dessous
     la représentation graphique d'une
     fonction $f$ définie sur $\mathbb{R}$.
     \medbreak
     \begin{center}
          \begin{extern}%width="600" alt="représentation graphique de la  fonction f"
               \resizebox{8cm}{!}{
                    \begin{pspicture*}(-5.3,-3.5)(8.3,8.5)
                         \psgrid[gridlabels=0pt,subgriddiv=1,gridwidth=0.2pt](-5.6,-3.5)(8.6,8.5)
                         \psaxes[linewidth=0.8pt]{->}(0,0)(-5.3,-3.5)(8.3,8.5)
                         \psaxes[linewidth=1pt]{->}(0,0)(1,1)
                         \psplot[plotpoints=3000,linewidth=1pt,linecolor=blue]{-4.5}{7.2}{x 4 exp 40 div x 3 exp 6 div sub x dup mul 5 div sub 2 x mul add}
                    \end{pspicture*}
               }
          \end{extern}
     \end{center}
     \item Soit $f'$ la dérivée de $f$ et $F$ une
     primitive de $f$ sur $\mathbb{R}$.
     \begin{enumerate}[label=\alph*.]
          \item $f'$ est positive sur [2~;~4].
          \item $f'$ est négative sur [-3~;~-1]-
          \item $F$ est décroissante sur [2~;~4].
          \item $F$ est décroissante sur [-3~;~-1].
     \end{enumerate}
     \item Une des courbes ci-dessous
     représente la fonction $f''$. Laquelle~?
     \medbreak
     \textbf{a.}
     \begin{center}
          \begin{extern}%width="600" alt="représentation graphique de la dérivée seconde - 1"
               \resizebox{8cm}{!}{
                    \begin{pspicture*}(-5.3,-3.5)(8.3,8.5)
                         \psgrid[gridlabels=0pt,subgriddiv=1,gridwidth=0.2pt](-5.6,-3.5)(8.6,8.5)
                         \psaxes[linewidth=0.8pt]{->}(0,0)(-5.3,-3.5)(8.3,8.5)
                         \psaxes[linewidth=1pt]{->}(0,0)(1,1)
                         \psplot[plotpoints=3000,linewidth=1pt,linecolor=blue]{-5}{7.2}{x 5 exp 5 div x 4 exp 1.7 mul sub x 3 exp 2.41667 mul sub x dup mul 39.6 mul add 35 div}
                    \end{pspicture*}
               }
          \end{extern}
     \end{center}
     \textbf{b.}
     \begin{center}
          \begin{extern}%width="500" alt="représentation graphique de la dérivée seconde - 2"
               \resizebox{8cm}{!}{
                    \begin{pspicture*}(-5.3,-3.5)(8.3,8.5)
                         \psgrid[gridlabels=0pt,subgriddiv=1,gridwidth=0.2pt](-5.6,-3.5)(8.6,8.5)
                         \psaxes[linewidth=0.8pt]{->}(0,0)(-5.3,-3.5)(8.3,8.5)
                         \psaxes[linewidth=1pt]{->}(0,0)(1,1)
                         \psplot[plotpoints=3000,linewidth=1pt,linecolor=blue]{-4.5}{7.2}{x 3 exp x dup mul 5 mul sub 4 x mul sub  20 add 10 div}
                    \end{pspicture*}
               }
          \end{extern}
     \end{center}
     \textbf{c.}
     \begin{center}
          \begin{extern}%width="500" alt="représentation graphique de la dérivée seconde - 3"
               \resizebox{8cm}{!}{
                    \begin{pspicture*}(-5.3,-3.5)(8.3,8.5)
                         \psgrid[gridlabels=0pt,subgriddiv=1,gridwidth=0.2pt](-5.6,-3.5)(8.6,8.5)
                         \psaxes[linewidth=0.8pt]{->}(0,0)(-5.3,-3.5)(8.3,8.5)
                         \psaxes[linewidth=1pt]{->}(0,0)(1,1)
                         \psplot[plotpoints=3000,linewidth=1pt,linecolor=blue]{-4.5}{7.2}{x x 3.4 add mul x 4.5 sub mul x 5.5 sub mul  68 neg div}
                    \end{pspicture*}
               }
          \end{extern}
     \end{center}
     \textbf{d.}
     \begin{center}
          \begin{extern}%width="500" alt="représentation graphique de la dérivée seconde - 4"
               \resizebox{8cm}{!}{
                    \begin{pspicture*}(-5.3,-3.5)(8.3,8.5)
                         \psgrid[gridlabels=0pt,subgriddiv=1,gridwidth=0.2pt](-5.6,-3.5)(8.6,8.5)
                         \psaxes[linewidth=0.8pt]{->}(0,0)(-5.3,-3.5)(8.3,8.5)
                         \psaxes[linewidth=1pt]{->}(0,0)(1,1)
                         \psplot[plotpoints=3000,linewidth=1pt,linecolor=blue]{-4.5}{7.2}{x 1.8 sub 2 exp 4 sub 0.35 mul}
                    \end{pspicture*}
               }
          \end{extern}
     \end{center}
\end{enumerate}

\end{document}