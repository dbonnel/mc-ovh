\documentclass[a4paper]{article}

%================================================================================================================================
%
% Packages
%
%================================================================================================================================

\usepackage[T1]{fontenc} 	% pour caractères accentués
\usepackage[utf8]{inputenc}  % encodage utf8
\usepackage[french]{babel}	% langue : français
\usepackage{fourier}			% caractères plus lisibles
\usepackage[dvipsnames]{xcolor} % couleurs
\usepackage{fancyhdr}		% réglage header footer
\usepackage{needspace}		% empêcher sauts de page mal placés
\usepackage{graphicx}		% pour inclure des graphiques
\usepackage{enumitem,cprotect}		% personnalise les listes d'items (nécessaire pour ol, al ...)
\usepackage{hyperref}		% Liens hypertexte
\usepackage{pstricks,pst-all,pst-node,pstricks-add,pst-math,pst-plot,pst-tree,pst-eucl} % pstricks
\usepackage[a4paper,includeheadfoot,top=2cm,left=3cm, bottom=2cm,right=3cm]{geometry} % marges etc.
\usepackage{comment}			% commentaires multilignes
\usepackage{amsmath,environ} % maths (matrices, etc.)
\usepackage{amssymb,makeidx}
\usepackage{bm}				% bold maths
\usepackage{tabularx}		% tableaux
\usepackage{colortbl}		% tableaux en couleur
\usepackage{fontawesome}		% Fontawesome
\usepackage{environ}			% environment with command
\usepackage{fp}				% calculs pour ps-tricks
\usepackage{multido}			% pour ps tricks
\usepackage[np]{numprint}	% formattage nombre
\usepackage{tikz,tkz-tab} 			% package principal TikZ
\usepackage{pgfplots}   % axes
\usepackage{mathrsfs}    % cursives
\usepackage{calc}			% calcul taille boites
\usepackage[scaled=0.875]{helvet} % font sans serif
\usepackage{svg} % svg
\usepackage{scrextend} % local margin
\usepackage{scratch} %scratch
\usepackage{multicol} % colonnes
%\usepackage{infix-RPN,pst-func} % formule en notation polanaise inversée
\usepackage{listings}

%================================================================================================================================
%
% Réglages de base
%
%================================================================================================================================

\lstset{
language=Python,   % R code
literate=
{á}{{\'a}}1
{à}{{\`a}}1
{ã}{{\~a}}1
{é}{{\'e}}1
{è}{{\`e}}1
{ê}{{\^e}}1
{í}{{\'i}}1
{ó}{{\'o}}1
{õ}{{\~o}}1
{ú}{{\'u}}1
{ü}{{\"u}}1
{ç}{{\c{c}}}1
{~}{{ }}1
}


\definecolor{codegreen}{rgb}{0,0.6,0}
\definecolor{codegray}{rgb}{0.5,0.5,0.5}
\definecolor{codepurple}{rgb}{0.58,0,0.82}
\definecolor{backcolour}{rgb}{0.95,0.95,0.92}

\lstdefinestyle{mystyle}{
    backgroundcolor=\color{backcolour},   
    commentstyle=\color{codegreen},
    keywordstyle=\color{magenta},
    numberstyle=\tiny\color{codegray},
    stringstyle=\color{codepurple},
    basicstyle=\ttfamily\footnotesize,
    breakatwhitespace=false,         
    breaklines=true,                 
    captionpos=b,                    
    keepspaces=true,                 
    numbers=left,                    
xleftmargin=2em,
framexleftmargin=2em,            
    showspaces=false,                
    showstringspaces=false,
    showtabs=false,                  
    tabsize=2,
    upquote=true
}

\lstset{style=mystyle}


\lstset{style=mystyle}
\newcommand{\imgdir}{C:/laragon/www/newmc/assets/imgsvg/}
\newcommand{\imgsvgdir}{C:/laragon/www/newmc/assets/imgsvg/}

\definecolor{mcgris}{RGB}{220, 220, 220}% ancien~; pour compatibilité
\definecolor{mcbleu}{RGB}{52, 152, 219}
\definecolor{mcvert}{RGB}{125, 194, 70}
\definecolor{mcmauve}{RGB}{154, 0, 215}
\definecolor{mcorange}{RGB}{255, 96, 0}
\definecolor{mcturquoise}{RGB}{0, 153, 153}
\definecolor{mcrouge}{RGB}{255, 0, 0}
\definecolor{mclightvert}{RGB}{205, 234, 190}

\definecolor{gris}{RGB}{220, 220, 220}
\definecolor{bleu}{RGB}{52, 152, 219}
\definecolor{vert}{RGB}{125, 194, 70}
\definecolor{mauve}{RGB}{154, 0, 215}
\definecolor{orange}{RGB}{255, 96, 0}
\definecolor{turquoise}{RGB}{0, 153, 153}
\definecolor{rouge}{RGB}{255, 0, 0}
\definecolor{lightvert}{RGB}{205, 234, 190}
\setitemize[0]{label=\color{lightvert}  $\bullet$}

\pagestyle{fancy}
\renewcommand{\headrulewidth}{0.2pt}
\fancyhead[L]{maths-cours.fr}
\fancyhead[R]{\thepage}
\renewcommand{\footrulewidth}{0.2pt}
\fancyfoot[C]{}

\newcolumntype{C}{>{\centering\arraybackslash}X}
\newcolumntype{s}{>{\hsize=.35\hsize\arraybackslash}X}

\setlength{\parindent}{0pt}		 
\setlength{\parskip}{3mm}
\setlength{\headheight}{1cm}

\def\ebook{ebook}
\def\book{book}
\def\web{web}
\def\type{web}

\newcommand{\vect}[1]{\overrightarrow{\,\mathstrut#1\,}}

\def\Oij{$\left(\text{O}~;~\vect{\imath},~\vect{\jmath}\right)$}
\def\Oijk{$\left(\text{O}~;~\vect{\imath},~\vect{\jmath},~\vect{k}\right)$}
\def\Ouv{$\left(\text{O}~;~\vect{u},~\vect{v}\right)$}

\hypersetup{breaklinks=true, colorlinks = true, linkcolor = OliveGreen, urlcolor = OliveGreen, citecolor = OliveGreen, pdfauthor={Didier BONNEL - https://www.maths-cours.fr} } % supprime les bordures autour des liens

\renewcommand{\arg}[0]{\text{arg}}

\everymath{\displaystyle}

%================================================================================================================================
%
% Macros - Commandes
%
%================================================================================================================================

\newcommand\meta[2]{    			% Utilisé pour créer le post HTML.
	\def\titre{titre}
	\def\url{url}
	\def\arg{#1}
	\ifx\titre\arg
		\newcommand\maintitle{#2}
		\fancyhead[L]{#2}
		{\Large\sffamily \MakeUppercase{#2}}
		\vspace{1mm}\textcolor{mcvert}{\hrule}
	\fi 
	\ifx\url\arg
		\fancyfoot[L]{\href{https://www.maths-cours.fr#2}{\black \footnotesize{https://www.maths-cours.fr#2}}}
	\fi 
}


\newcommand\TitreC[1]{    		% Titre centré
     \needspace{3\baselineskip}
     \begin{center}\textbf{#1}\end{center}
}

\newcommand\newpar{    		% paragraphe
     \par
}

\newcommand\nosp {    		% commande vide (pas d'espace)
}
\newcommand{\id}[1]{} %ignore

\newcommand\boite[2]{				% Boite simple sans titre
	\vspace{5mm}
	\setlength{\fboxrule}{0.2mm}
	\setlength{\fboxsep}{5mm}	
	\fcolorbox{#1}{#1!3}{\makebox[\linewidth-2\fboxrule-2\fboxsep]{
  		\begin{minipage}[t]{\linewidth-2\fboxrule-4\fboxsep}\setlength{\parskip}{3mm}
  			 #2
  		\end{minipage}
	}}
	\vspace{5mm}
}

\newcommand\CBox[4]{				% Boites
	\vspace{5mm}
	\setlength{\fboxrule}{0.2mm}
	\setlength{\fboxsep}{5mm}
	
	\fcolorbox{#1}{#1!3}{\makebox[\linewidth-2\fboxrule-2\fboxsep]{
		\begin{minipage}[t]{1cm}\setlength{\parskip}{3mm}
	  		\textcolor{#1}{\LARGE{#2}}    
 	 	\end{minipage}  
  		\begin{minipage}[t]{\linewidth-2\fboxrule-4\fboxsep}\setlength{\parskip}{3mm}
			\raisebox{1.2mm}{\normalsize\sffamily{\textcolor{#1}{#3}}}						
  			 #4
  		\end{minipage}
	}}
	\vspace{5mm}
}

\newcommand\cadre[3]{				% Boites convertible html
	\par
	\vspace{2mm}
	\setlength{\fboxrule}{0.1mm}
	\setlength{\fboxsep}{5mm}
	\fcolorbox{#1}{white}{\makebox[\linewidth-2\fboxrule-2\fboxsep]{
  		\begin{minipage}[t]{\linewidth-2\fboxrule-4\fboxsep}\setlength{\parskip}{3mm}
			\raisebox{-2.5mm}{\sffamily \small{\textcolor{#1}{\MakeUppercase{#2}}}}		
			\par		
  			 #3
 	 		\end{minipage}
	}}
		\vspace{2mm}
	\par
}

\newcommand\bloc[3]{				% Boites convertible html sans bordure
     \needspace{2\baselineskip}
     {\sffamily \small{\textcolor{#1}{\MakeUppercase{#2}}}}    
		\par		
  			 #3
		\par
}

\newcommand\CHelp[1]{
     \CBox{Plum}{\faInfoCircle}{À RETENIR}{#1}
}

\newcommand\CUp[1]{
     \CBox{NavyBlue}{\faThumbsOUp}{EN PRATIQUE}{#1}
}

\newcommand\CInfo[1]{
     \CBox{Sepia}{\faArrowCircleRight}{REMARQUE}{#1}
}

\newcommand\CRedac[1]{
     \CBox{PineGreen}{\faEdit}{BIEN R\'EDIGER}{#1}
}

\newcommand\CError[1]{
     \CBox{Red}{\faExclamationTriangle}{ATTENTION}{#1}
}

\newcommand\TitreExo[2]{
\needspace{4\baselineskip}
 {\sffamily\large EXERCICE #1\ (\emph{#2 points})}
\vspace{5mm}
}

\newcommand\img[2]{
          \includegraphics[width=#2\paperwidth]{\imgdir#1}
}

\newcommand\imgsvg[2]{
       \begin{center}   \includegraphics[width=#2\paperwidth]{\imgsvgdir#1} \end{center}
}


\newcommand\Lien[2]{
     \href{#1}{#2 \tiny \faExternalLink}
}
\newcommand\mcLien[2]{
     \href{https~://www.maths-cours.fr/#1}{#2 \tiny \faExternalLink}
}

\newcommand{\euro}{\eurologo{}}

%================================================================================================================================
%
% Macros - Environement
%
%================================================================================================================================

\newenvironment{tex}{ %
}
{%
}

\newenvironment{indente}{ %
	\setlength\parindent{10mm}
}

{
	\setlength\parindent{0mm}
}

\newenvironment{corrige}{%
     \needspace{3\baselineskip}
     \medskip
     \textbf{\textsc{Corrigé}}
     \medskip
}
{
}

\newenvironment{extern}{%
     \begin{center}
     }
     {
     \end{center}
}

\NewEnviron{code}{%
	\par
     \boite{gray}{\texttt{%
     \BODY
     }}
     \par
}

\newenvironment{vbloc}{% boite sans cadre empeche saut de page
     \begin{minipage}[t]{\linewidth}
     }
     {
     \end{minipage}
}
\NewEnviron{h2}{%
    \needspace{3\baselineskip}
    \vspace{0.6cm}
	\noindent \MakeUppercase{\sffamily \large \BODY}
	\vspace{1mm}\textcolor{mcgris}{\hrule}\vspace{0.4cm}
	\par
}{}

\NewEnviron{h3}{%
    \needspace{3\baselineskip}
	\vspace{5mm}
	\textsc{\BODY}
	\par
}

\NewEnviron{margeneg}{ %
\begin{addmargin}[-1cm]{0cm}
\BODY
\end{addmargin}
}

\NewEnviron{html}{%
}

\begin{document}
\meta{url}{/exercices/modelisation-par-une-fonction-exponentielle/}
\meta{pid}{15434}
\meta{titre}{Modélisation par une fonction exponentielle}
\meta{type}{exercices}
%
\\
Le maire d'une ville française a effectué un recensement de la population de sa municipalité pendant 7 ans.
\\
Les données recueillies sont présentées dans le tableau ci-dessous~:
\begin{center}
     \begin{tabularx}{0.8linewidth}{|*{8}{>{centering arraybackslash }X|}}%class="compact" style="width:50rem"
          \hline
          Année & 2013 & 2014 & 2015 & 2016 & 2017 & 2018 & 2019\\ \hline
          Rang & 0 & 1 & 2 & 3 & 4 & 5 & 6 \\ \hline
          Habitants & 2~502 & 2~475 & 2~452 & 2~430 & 2~398 & 2~378 & 2~351 \\ \hline
     \end{tabularx}
\end{center}

Dans la première partie de l'exercice, on modélisera le nombre d'habitants à l'aide d'une suite géométrique et dans la seconde partie, on utilisera une fonction exponentielle.

\begin{h2} Partie 1~: Modélisation à l'aide d'une suite \end{h2}
\begin{enumerate}
     \item
     Calculer le pourcentage d'évolution de la population de la ville entre 2013 et 2014, entre 2014 et 2015, entre 2015 et 2016 et entre 2018 et 2019.
     \item
     Par la suite on estimera que la population diminue de 1\% par an.
     \par
     On note $ p_n $ le nombre d'habitants l'année 2013+$n$.
     \par
     Montrer que la suite $(p_n)$ est une suite géométrique dont on donnera le premier terme et la raison.
     \item
     À l'aide de la suite $ (p_n) $ estimer la population de la ville en 2030 en supposant que la diminution de la population s'effectue au même rythme pendant les années à venir.
\end{enumerate}
\begin{h2} Partie 2~: Modélisation à l'aide d'une fonction exponentielle\end{h2}
\begin{enumerate}
     \item
     On cherche à modéliser le nombre d'habitants à l'aide de la fonction $f$ définie sur $ \left[ 0~;~ +\infty \right[ $ par~:
     \begin{center}
          $f~: \ t \longmapsto 2500\ \text{e}^{ -0,01t } $
     \end{center}
     où $t$ désigne la durée écoulée, en année, depuis 2013.
     \par
     Montrer que la fonction $f$ est strictement décroissante sur l'intervalle $ \left[ 0~;~ +\infty \right[ $.
     \item
     Compléter la fonction Python ci-dessous afin qu'elle retourne les images de la variable $t$ par la fonction $f$~:
\begin{lstlisting}[language=Python]
def f(t) :
    return ...
\end{lstlisting}
À l'aide d'une boucle, écrire un script Python qui retourne les images par $f$ des entiers compris entre 0 et 6.
\\ Comparer aux données de l'énoncé.
\\ Cette modélisation vous semble-t-elle valable~?
\item
Le maire souhaite prévoir en quelle année le nombre d'habitants de sa ville passera sous la barre des 2~200 d'après ce modèle.
\par
En utilisant la fonction précédente, écrire un programme Python qui répond à cette question.
\end{enumerate}
\begin{corrige}
     \TitreC{ Partie 1 }
     \begin{enumerate}
          \item
          Le pourcentage d'évolution de la population entre 2013 et 2014 est (voir \mcLien{https://www.maths-cours.fr/cours/pourcentages/\#d40}{formule de calcul d'une évolution})~:
          \par
          $ t_1 = \frac{ p_1 - p_0 }{ p_0 } = \frac{ 2~475 - 2~502 }{ 2~502 } $\nosp$ \approx -0,0108 \approx \frac{ -1,08 }{ 100 } = -1,08\%$
          \medskip
          De même, le pourcentage d'évolution entre 2014 et 2015 est~:
          \par
          $ t_2 = \frac{ p_2 - p_1}{ p_1 } = \frac{ 2~452 - 2~475 }{ 2~475 } $\nosp$ \approx -0,0093 \approx \frac{ -0,93 }{ 100 } = -0,93\%$
          \medskip
          entre 2015 et 2016~:
          \par
          $ t_3 = \frac{ p_3 - p_2}{ p_2 } = \frac{ 2~430 - 2~452 }{ 2~452 } $\nosp$ \approx -0,0090 \approx \frac{ -0,90 }{ 100 } = -0,90\%$
          \medskip
          enfin, entre 2018 et 2019~:
          \par
          $ t_6 = \frac{ p_6 - p_5}{ p_5 } = \frac{ 2~351 - 2~378 }{ 2~378 } $\nosp$ \approx -0,0114 \approx \frac{ -1,14 }{ 100 } = -1,14\%$
          \medskip
          On remarque que, dans tous les cas, la diminution est proche de 1\%.
          \item
          Le coefficient multiplicateur qui fait passer de $p_{n+1}$ à $p_n$ correspondant à une baisse de 1\% est (voir \mcLien{https://www.maths-cours.fr/cours/pourcentages/#d30}{coefficient multiplicateur})~:
          \par
          $ CM=1- \frac{ 1 }{ 100 } =0,99 $
          \par
          On a donc, pour tout entier naturel $n$~:
          \par
          $p_{n+1} = 0,99p_n $
          \par
          La suite $ \left( p_n \right) $ est donc une suite géométrique de raison $ q = 0,99. $ Son premier terme est $ p_0=2502. $
          \item
          La population de la ville à l'année de rang $n$ est~:
          \par
          $ p_n=p_0\ q^n = 2502 \times 0,99^n$
          \medskip
          L'année 2030 correspond au rang 17. La population en 2030 peut donc, d'après ce modèle, être estimée à~:
          \par
          $ p_{ 17 } = 2502 \times 0,99^{ 17 } \approx 2109.$
     \end{enumerate}
     \TitreC{ Partie 2 }
     \begin{enumerate}
          \item
          $f$ est dérivable sur $ \left[ 0~;~ +\infty \right[ $. Pour déterminer le sens de variation de $f$, on calcule sa dérivée $ f' $.
          \\Sachant que la dérivée de la fonction $ t \longmapsto \text{e}^{ at } $ est la fonction $ t \longmapsto a\ \text{e}^{ at }$ on obtient~:
          \par
          $f'(t)=2500 \times -0,01 \text{e}^{ -0,01t } =-25 \ \text{e}^{ -0,01t } $
          \medskip
          $ -25 $ est strictement négatif tandis que $\text{e}^{ -0,01t } $ est strictement positif (car la fonction exponentielle ne prend que des valeurs strictement positives) donc $f'(t) < 0$ sur $ \left[ 0~;~ +\infty \right[ $.
          \par
          Par conséquent, la fonction $f$ est strictement décroissante sur l'intervalle $ \left[ 0~;~ +\infty \right[ $.
          \item
          La fonction Python se définit simplement comme suit~:
\begin{lstlisting}[language=Python]
def f(t) :
    return 2500 * exp(-0.01 * t)
     \end{lstlisting}
     \par
     On doit toutefois importer le module math qui contient la fonction exp~; par exemple~:
\begin{lstlisting}[language=Python]
from math import exp
def f(t) :
    return 2500 * exp(0.01 * t)
\end{lstlisting}
\par
Comme on connait le nombre d'itérations, on peut employer une boucle \texttt{for} pour afficher les images des 7 premières valeurs entières de $t$~:
\begin{lstlisting}[language=Python]
from math import exp
def f(t) :
    return 2500 * exp(-0.01 * t)

for t in range(7) : 
    print(f(t))
\end{lstlisting}
\par
\\
On obtient le résultat suivant~:
\begin{lstlisting}[language=Python]
2500.0
2475.1245843729203
2450.4966832668883
2426.1138338712703
2401.973597880808
2378.073561251785
2354.411333960622
\end{lstlisting}
\par
Ces valeurs sont suffisamment proches de celles du tableau donné dans l'énoncé pour considérer que cette modélisation est satisfaisante.
\item
On utilise une boucle \texttt{while} pour répondre à la question.
\\On reste dans la boucle tant que le nombre d'habitants est supérieur ou égal à 2~200 et on sort de la boucle dès que ce nombre devient strictement inférieur à 2~200.
\par
Il faut penser à initialiser la variable \texttt{t} avant la boucle et à l’incrémenter à l'intérieur de la boucle (voir~: \mcLien{https://www.maths-cours.fr/cours/python-au-lycee-3-les-boucles/#h10}{boucles while}). On peut ensuite afficher la valeur de \texttt{t} à la sortie de la boucle~:
\begin{lstlisting}[language=Python]
from math import exp
def f(t) :
    return 2500 * exp(-0.01 * t)

t=0
while f(t) >= 2200: 
    t=t+1
print(t)
\end{lstlisting}
\par
Ce programme affiche la valeur 13.
\par
D'après ce modèle, la population passera sous la barre des 2~200 l'année de rang 13 c'est à dire en 2013+13 = 2026.
\end{enumerate}
\end{corrige}

\end{document}