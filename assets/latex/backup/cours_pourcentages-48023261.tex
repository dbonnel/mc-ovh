\documentclass[a4paper]{article}

%================================================================================================================================
%
% Packages
%
%================================================================================================================================

\usepackage[T1]{fontenc} 	% pour caractères accentués
\usepackage[utf8]{inputenc}  % encodage utf8
\usepackage[french]{babel}	% langue : français
\usepackage{fourier}			% caractères plus lisibles
\usepackage[dvipsnames]{xcolor} % couleurs
\usepackage{fancyhdr}		% réglage header footer
\usepackage{needspace}		% empêcher sauts de page mal placés
\usepackage{graphicx}		% pour inclure des graphiques
\usepackage{enumitem,cprotect}		% personnalise les listes d'items (nécessaire pour ol, al ...)
\usepackage{hyperref}		% Liens hypertexte
\usepackage{pstricks,pst-all,pst-node,pstricks-add,pst-math,pst-plot,pst-tree,pst-eucl} % pstricks
\usepackage[a4paper,includeheadfoot,top=2cm,left=3cm, bottom=2cm,right=3cm]{geometry} % marges etc.
\usepackage{comment}			% commentaires multilignes
\usepackage{amsmath,environ} % maths (matrices, etc.)
\usepackage{amssymb,makeidx}
\usepackage{bm}				% bold maths
\usepackage{tabularx}		% tableaux
\usepackage{colortbl}		% tableaux en couleur
\usepackage{fontawesome}		% Fontawesome
\usepackage{environ}			% environment with command
\usepackage{fp}				% calculs pour ps-tricks
\usepackage{multido}			% pour ps tricks
\usepackage[np]{numprint}	% formattage nombre
\usepackage{tikz,tkz-tab} 			% package principal TikZ
\usepackage{pgfplots}   % axes
\usepackage{mathrsfs}    % cursives
\usepackage{calc}			% calcul taille boites
\usepackage[scaled=0.875]{helvet} % font sans serif
\usepackage{svg} % svg
\usepackage{scrextend} % local margin
\usepackage{scratch} %scratch
\usepackage{multicol} % colonnes
%\usepackage{infix-RPN,pst-func} % formule en notation polanaise inversée
\usepackage{listings}

%================================================================================================================================
%
% Réglages de base
%
%================================================================================================================================

\lstset{
language=Python,   % R code
literate=
{á}{{\'a}}1
{à}{{\`a}}1
{ã}{{\~a}}1
{é}{{\'e}}1
{è}{{\`e}}1
{ê}{{\^e}}1
{í}{{\'i}}1
{ó}{{\'o}}1
{õ}{{\~o}}1
{ú}{{\'u}}1
{ü}{{\"u}}1
{ç}{{\c{c}}}1
{~}{{ }}1
}


\definecolor{codegreen}{rgb}{0,0.6,0}
\definecolor{codegray}{rgb}{0.5,0.5,0.5}
\definecolor{codepurple}{rgb}{0.58,0,0.82}
\definecolor{backcolour}{rgb}{0.95,0.95,0.92}

\lstdefinestyle{mystyle}{
    backgroundcolor=\color{backcolour},   
    commentstyle=\color{codegreen},
    keywordstyle=\color{magenta},
    numberstyle=\tiny\color{codegray},
    stringstyle=\color{codepurple},
    basicstyle=\ttfamily\footnotesize,
    breakatwhitespace=false,         
    breaklines=true,                 
    captionpos=b,                    
    keepspaces=true,                 
    numbers=left,                    
xleftmargin=2em,
framexleftmargin=2em,            
    showspaces=false,                
    showstringspaces=false,
    showtabs=false,                  
    tabsize=2,
    upquote=true
}

\lstset{style=mystyle}


\lstset{style=mystyle}
\newcommand{\imgdir}{C:/laragon/www/newmc/assets/imgsvg/}
\newcommand{\imgsvgdir}{C:/laragon/www/newmc/assets/imgsvg/}

\definecolor{mcgris}{RGB}{220, 220, 220}% ancien~; pour compatibilité
\definecolor{mcbleu}{RGB}{52, 152, 219}
\definecolor{mcvert}{RGB}{125, 194, 70}
\definecolor{mcmauve}{RGB}{154, 0, 215}
\definecolor{mcorange}{RGB}{255, 96, 0}
\definecolor{mcturquoise}{RGB}{0, 153, 153}
\definecolor{mcrouge}{RGB}{255, 0, 0}
\definecolor{mclightvert}{RGB}{205, 234, 190}

\definecolor{gris}{RGB}{220, 220, 220}
\definecolor{bleu}{RGB}{52, 152, 219}
\definecolor{vert}{RGB}{125, 194, 70}
\definecolor{mauve}{RGB}{154, 0, 215}
\definecolor{orange}{RGB}{255, 96, 0}
\definecolor{turquoise}{RGB}{0, 153, 153}
\definecolor{rouge}{RGB}{255, 0, 0}
\definecolor{lightvert}{RGB}{205, 234, 190}
\setitemize[0]{label=\color{lightvert}  $\bullet$}

\pagestyle{fancy}
\renewcommand{\headrulewidth}{0.2pt}
\fancyhead[L]{maths-cours.fr}
\fancyhead[R]{\thepage}
\renewcommand{\footrulewidth}{0.2pt}
\fancyfoot[C]{}

\newcolumntype{C}{>{\centering\arraybackslash}X}
\newcolumntype{s}{>{\hsize=.35\hsize\arraybackslash}X}

\setlength{\parindent}{0pt}		 
\setlength{\parskip}{3mm}
\setlength{\headheight}{1cm}

\def\ebook{ebook}
\def\book{book}
\def\web{web}
\def\type{web}

\newcommand{\vect}[1]{\overrightarrow{\,\mathstrut#1\,}}

\def\Oij{$\left(\text{O}~;~\vect{\imath},~\vect{\jmath}\right)$}
\def\Oijk{$\left(\text{O}~;~\vect{\imath},~\vect{\jmath},~\vect{k}\right)$}
\def\Ouv{$\left(\text{O}~;~\vect{u},~\vect{v}\right)$}

\hypersetup{breaklinks=true, colorlinks = true, linkcolor = OliveGreen, urlcolor = OliveGreen, citecolor = OliveGreen, pdfauthor={Didier BONNEL - https://www.maths-cours.fr} } % supprime les bordures autour des liens

\renewcommand{\arg}[0]{\text{arg}}

\everymath{\displaystyle}

%================================================================================================================================
%
% Macros - Commandes
%
%================================================================================================================================

\newcommand\meta[2]{    			% Utilisé pour créer le post HTML.
	\def\titre{titre}
	\def\url{url}
	\def\arg{#1}
	\ifx\titre\arg
		\newcommand\maintitle{#2}
		\fancyhead[L]{#2}
		{\Large\sffamily \MakeUppercase{#2}}
		\vspace{1mm}\textcolor{mcvert}{\hrule}
	\fi 
	\ifx\url\arg
		\fancyfoot[L]{\href{https://www.maths-cours.fr#2}{\black \footnotesize{https://www.maths-cours.fr#2}}}
	\fi 
}


\newcommand\TitreC[1]{    		% Titre centré
     \needspace{3\baselineskip}
     \begin{center}\textbf{#1}\end{center}
}

\newcommand\newpar{    		% paragraphe
     \par
}

\newcommand\nosp {    		% commande vide (pas d'espace)
}
\newcommand{\id}[1]{} %ignore

\newcommand\boite[2]{				% Boite simple sans titre
	\vspace{5mm}
	\setlength{\fboxrule}{0.2mm}
	\setlength{\fboxsep}{5mm}	
	\fcolorbox{#1}{#1!3}{\makebox[\linewidth-2\fboxrule-2\fboxsep]{
  		\begin{minipage}[t]{\linewidth-2\fboxrule-4\fboxsep}\setlength{\parskip}{3mm}
  			 #2
  		\end{minipage}
	}}
	\vspace{5mm}
}

\newcommand\CBox[4]{				% Boites
	\vspace{5mm}
	\setlength{\fboxrule}{0.2mm}
	\setlength{\fboxsep}{5mm}
	
	\fcolorbox{#1}{#1!3}{\makebox[\linewidth-2\fboxrule-2\fboxsep]{
		\begin{minipage}[t]{1cm}\setlength{\parskip}{3mm}
	  		\textcolor{#1}{\LARGE{#2}}    
 	 	\end{minipage}  
  		\begin{minipage}[t]{\linewidth-2\fboxrule-4\fboxsep}\setlength{\parskip}{3mm}
			\raisebox{1.2mm}{\normalsize\sffamily{\textcolor{#1}{#3}}}						
  			 #4
  		\end{minipage}
	}}
	\vspace{5mm}
}

\newcommand\cadre[3]{				% Boites convertible html
	\par
	\vspace{2mm}
	\setlength{\fboxrule}{0.1mm}
	\setlength{\fboxsep}{5mm}
	\fcolorbox{#1}{white}{\makebox[\linewidth-2\fboxrule-2\fboxsep]{
  		\begin{minipage}[t]{\linewidth-2\fboxrule-4\fboxsep}\setlength{\parskip}{3mm}
			\raisebox{-2.5mm}{\sffamily \small{\textcolor{#1}{\MakeUppercase{#2}}}}		
			\par		
  			 #3
 	 		\end{minipage}
	}}
		\vspace{2mm}
	\par
}

\newcommand\bloc[3]{				% Boites convertible html sans bordure
     \needspace{2\baselineskip}
     {\sffamily \small{\textcolor{#1}{\MakeUppercase{#2}}}}    
		\par		
  			 #3
		\par
}

\newcommand\CHelp[1]{
     \CBox{Plum}{\faInfoCircle}{À RETENIR}{#1}
}

\newcommand\CUp[1]{
     \CBox{NavyBlue}{\faThumbsOUp}{EN PRATIQUE}{#1}
}

\newcommand\CInfo[1]{
     \CBox{Sepia}{\faArrowCircleRight}{REMARQUE}{#1}
}

\newcommand\CRedac[1]{
     \CBox{PineGreen}{\faEdit}{BIEN R\'EDIGER}{#1}
}

\newcommand\CError[1]{
     \CBox{Red}{\faExclamationTriangle}{ATTENTION}{#1}
}

\newcommand\TitreExo[2]{
\needspace{4\baselineskip}
 {\sffamily\large EXERCICE #1\ (\emph{#2 points})}
\vspace{5mm}
}

\newcommand\img[2]{
          \includegraphics[width=#2\paperwidth]{\imgdir#1}
}

\newcommand\imgsvg[2]{
       \begin{center}   \includegraphics[width=#2\paperwidth]{\imgsvgdir#1} \end{center}
}


\newcommand\Lien[2]{
     \href{#1}{#2 \tiny \faExternalLink}
}
\newcommand\mcLien[2]{
     \href{https~://www.maths-cours.fr/#1}{#2 \tiny \faExternalLink}
}

\newcommand{\euro}{\eurologo{}}

%================================================================================================================================
%
% Macros - Environement
%
%================================================================================================================================

\newenvironment{tex}{ %
}
{%
}

\newenvironment{indente}{ %
	\setlength\parindent{10mm}
}

{
	\setlength\parindent{0mm}
}

\newenvironment{corrige}{%
     \needspace{3\baselineskip}
     \medskip
     \textbf{\textsc{Corrigé}}
     \medskip
}
{
}

\newenvironment{extern}{%
     \begin{center}
     }
     {
     \end{center}
}

\NewEnviron{code}{%
	\par
     \boite{gray}{\texttt{%
     \BODY
     }}
     \par
}

\newenvironment{vbloc}{% boite sans cadre empeche saut de page
     \begin{minipage}[t]{\linewidth}
     }
     {
     \end{minipage}
}
\NewEnviron{h2}{%
    \needspace{3\baselineskip}
    \vspace{0.6cm}
	\noindent \MakeUppercase{\sffamily \large \BODY}
	\vspace{1mm}\textcolor{mcgris}{\hrule}\vspace{0.4cm}
	\par
}{}

\NewEnviron{h3}{%
    \needspace{3\baselineskip}
	\vspace{5mm}
	\textsc{\BODY}
	\par
}

\NewEnviron{margeneg}{ %
\begin{addmargin}[-1cm]{0cm}
\BODY
\end{addmargin}
}

\NewEnviron{html}{%
}

\begin{document}
\meta{url}{/cours/pourcentages/}
\meta{pid}{365}
\meta{titre}{Pourcentages}
\meta{type}{cours}
\begin{h2}1. Part en pourcentage\end{h2}
\cadre{bleu}{Définition}{% id="d10"
     Soit $E$ un ensemble fini (que l'on appellera \textbf{ensemble de référence}) et $F$ une partie de l'ensemble $E$. La \textbf{part en pourcentage} de $F$ par rapport à  $E$ est le nombre :
     \begin{center}$t \% =\frac{t}{100}= \frac{card \left(F\right)}{card \left(E\right)}$\end{center}
     où $card \left(E\right)$ (cardinal de $E$) désigne le nombre d'éléments de $E$ et $card \left(F\right)$ le nombre d'éléments de $F$.
     \par
     On dit également que $F$ représente $t\%$ de $E$.
}
\bloc{cyan}{Remarques}{% id="r10"
     \begin{itemize}
          \item $5\%$, $\frac{5}{100}$ et $0,05$ sont trois écritures différentes du \textbf{même nombre} (pourcentage, fraction, écriture décimale).
          \item On est en présence d'une situation de proportionnalité que l'on peut représenter par le tableau suivant :
          \begin{tabularx}{0.8\linewidth}{|*{3}{>{\centering \arraybackslash }X|}}%class="compact" width="600"
               \hline
               $t$ & nombre d'éléments de $F$
               \\ \hline
               $100$ & nombre d'éléments de $E$
               \\ \hline
          \end{tabularx}
          \item Ceci peut également s'écrire : nombre d'éléments de $F =\frac{t}{100} \times $ nombre d'élements de $E$.
          \par
          Cette dernière égalité permet de calculer le nombre d'éléments de $F$ connaissant sa part en pourcentage par rapport à $E$
     \end{itemize}
}
\bloc{orange}{Exemples}{% id="e10"
     \begin{itemize}
          \item Dans une classe de $25$ élèves qui compte $15$ garçons le pourcentage de garçons est :
          \par
          $\frac{15}{25}=0,6=\frac{60}{100}=60\%$
          \item $16\%$ de $75$€ font : $\frac{16}{100}\times 75=12$€
     \end{itemize}
}
\cadre{vert}{Propriété}{% id="p20"
     \textbf{Pourcentages de pourcentages}
     Soit 3 ensembles $E, F, G$ tels que $G \subset F \subset E$.
     \par
     Si $G$ représente $t_{1}$\% de $F$ et si $F$ représente $t_{2}$\% de $E$, la part en pourcentage de $G$ par rapport à $E$ est :
     \begin{center}$\frac{t}{100}=\frac{t_{1}}{100}\times \frac{t_{2}}{100}$\end{center}
}
\bloc{orange}{Exemple}{% id="e20"
     Dans un lycée de $800$ élèves :
     \begin{itemize}
          \item $25$ \% des élèves sont en Seconde;
          \item $45$ \% des élèves de Seconde sont des filles.
     \end{itemize}
     La part des filles de Seconde dans le lycée est :
     \par
     $\frac{t}{100}=\frac{25}{100}\times \frac{45}{100}=\frac{1125}{10000}=\frac{11,25}{100}=11,25\%$
     \par
     Le nombre de filles en Seconde est $\frac{11,25}{100}\times 800=90$
}
\begin{h2}2. Pourcentages d'évolution\end{h2}
\cadre{bleu}{Définition}{% id="d40"
     On considère une quantité passant d'une valeur $V_{0}$ à une valeur $V_{1}$.
     \par
     Le pourcentage d'évolution de cette quantité est le nombre
     \begin{center}$\frac{t}{100}=\frac{V_{1}-V_{0}}{V_{0}}$\end{center}
}
\bloc{cyan}{Remarques}{% id="r40"
     Le pourcentage d'évolution est \textbf{positif} dans le cas d'une \textbf{augmentation} et \textbf{négatif} dans le cas d'une \textbf{diminution}.
}
\bloc{orange}{Exemple}{% id="e40"
     Le prix d'un article passe de 80€ à 76€. Le pourcentage d'évolution est :
     \begin{center}$\frac{t}{100}=\frac{76-80}{80}=-\frac{4}{80}=-0,05=-5\%$\end{center}
     Le prix de l'article a diminué de 5\%
}    
\cadre{bleu}{Définition}{% id="d30"
     On considère une quantité passant d'une valeur $V_{0}$ à une valeur $V_{1}$.
     \par
     Le \textbf{coefficient multiplicateur} est le nombre par lequel il faut multiplier $V_{0}$ pour obtenir $V_{1}$ :
     \par
     $V_{1}=CM \times  V_{0}$
}
\bloc{cyan}{Remarques}{% id="r30"
     \begin{itemize}
          \item On a donc  $CM=\frac{V_{1}}{V_{0}}$
          \item Le coefficient multiplicateur est \textbf{supérieur à 1} dans le cas d'une \textbf{augmentation} et \textbf{inférieur à 1} dans le cas d'une \textbf{diminution}.
          \item La fonction qui à l'ancienne valeur associe la nouvelle valeur est : $x\mapsto CM\times x$
          \par
          C'est une \textbf{fonction linéaire} de coefficient directeur $CM$
     \end{itemize}
}
\cadre{vert}{Propriété}{% id="p50"
     Le coefficient multiplicateur s'exprime en fonction du pourcentage d'évolution par:
     \par
     $CM=1+\frac{t}{100}$
     \par
     (où $t$ est positif en cas d'augmentation, négatif en cas de diminution)
}
\bloc{cyan}{Remarques}{% id="r50"
     \begin{itemize}
          \item On a donc : $V_{1}=\left(1+\frac{t}{100}\right)V_{0}$.
          \item Dans le cas d'une diminution de $5$\%, par exemple, on pourra au choix considérer que :
          \par
          $CM=1+\frac{t}{100}$ avec $t=-5$
          \par
          ou
          \par
          $CM=1-\frac{t}{100}$ avec $t=5$
          \par
          Dans les deux raisonnements, on obtient évidemment le même coefficient multiplicateur $0,95$.
          \item Connaissant le coefficient multiplicateur, on a facilement le pourcentage d'évolution grâce à la relation : $\frac{t}{100}=CM-1$
          \item Le tableau ci-dessous résume les différents cas :
          \begin{tabularx}{0.8\linewidth}{|*{3}{>{\centering \arraybackslash }X|}}%class="compact" width="600"
               \hline
               &  Prendre $t\%$ de $x$                &   Augmenter $x$ de $t\%$                 & Diminuer $x$ de $t\%$
               \\ \hline
               Calculs à effectuer  &  Multiplier $x$ par \textbf{$\frac{t}{100}$}  &   Multiplier $x$ par \textbf{$1+\frac{t}{100}$} & Multiplier $x$ par \textbf{$1-\frac{t}{100}$}
               \\ \hline
               Fonction linéaire    & $x\mapsto \frac{t}{100}\times x$                    &  $x\mapsto \left(1+\frac{t}{100}\right)\times x$                    & $x\mapsto \left(1-\frac{t}{100}\right)\times x$
               \\ \hline
          \end{tabularx}
     \end{itemize}
}
\bloc{orange}{Exemple}{% id="e50"
     \begin{tabularx}{0.8\linewidth}{|*{3}{>{\centering \arraybackslash }X|}}%class="compact" width="600"
          \hline
          &  Prendre $25\%$ de $x$         &   Augmenter $x$ de $25\%$        & Diminuer $x$ de $25\%$
          \\ \hline
          Calculs à effectuer  &  Multiplier $x$ par $\frac{25}{100}$  &   Multiplier $x$ par $1,25$     & Multiplier $x$ par $0,75$
          \\ \hline
          Fonction linéaire    & $x\mapsto 0,25\times x$                 &   $x\mapsto 1,25\times x$                 & $x\mapsto 0,75\times x$
          \\ \hline
          Exemples             & Prendre $25\%$ de 200          &   Augmenter 50 de $25\%$         & Diminuer 50 de $25\%$
          \\ \hline
          Résultat             & $0,25\times 200=50$          &   $1,25\times 50=62,5$          & $0,75\times 50=37,5$
          \\ \hline
     \end{tabularx}
}
\cadre{vert}{Propriété (Évolutions successives)}{% id="p60"
     Lors d'évolutions successives, le coefficient multiplicateur global est égal au \textbf{produit} des coefficients multiplicateurs de chaque évolution
}
\bloc{orange}{Exemple}{% id="e60"
     Le prix d'un objet augmente de $10\%$ puis diminue de $10\%$.
     \par
     Le coefficient multiplicateur global est :
     \begin{center}$CM=\left(1+\frac{10}{100}\right)\left(1-\frac{10}{100}\right)=0,99$ \end{center}
     Si $t$ désigne le pourcentage d'évolution global en \%, on a donc :
     \par
     $1+ \frac{t}{100}=0,99 $
     \par
     $\frac{t}{100}=0,99-1=-0,01=-\frac{1}{100}$
     \par
     Le prix de l'objet a globalement \textbf{diminué} de $1\%$.
}
\bloc{cyan}{Remarques}{% id="r60"
     \begin{itemize}
          \item \textbf{Une hausse de $t\%$ ne "compense" pas une baisse de $t\%$}. C'est dû au fait que les deux pourcentages ne portent pas sur le même montant.
          \par
          En effet, si un objet coûtant 100 euros subit une augmentation de $10\%$ son prix passera à $110$€ (les $10\%$ ont été calculé par rapport à $100$€).
          \par
          Si son prix subit ensuite une diminution de $10\%$, le montant de la baisse sera calculé par rapport au prix de $110$€ et non plus de $100$€. La baisse sera donc de $11$€ et non $10$€.
          \item En cas d'évolution successives,\textbf{ les pourcentages d'évolutions ne s'ajoutent (ni ne soustraient) jamais}.
     \end{itemize}
}
\cadre{vert}{Définition et propriété (Taux d'évolution réciproque)}{% id="p70"
     Si le taux d'évolution $t \%$ fait passer de $V_{0}$ à $V_{1}$, on appelle taux d'évolution réciproque $t^{\prime} \%$, le taux d'évolution qui fait passer de $V_{1}$ à $V_{0}$.
     \par
     On a alors la relation suivante :
     \begin{center}$\left(1+\frac{t}{100}\right)\left(1+\frac{t^{\prime}}{100}\right)=1$\end{center}
}
\bloc{orange}{Exemple}{% id="e70"
     Le prix d'un article augmente de 60\%. Pour qu'il revienne à son prix de départ, il faut qu'ensuite il varie de $t^{\prime} \%$ tel que :
     \par
     $\left(1+\frac{60}{100}\right)\left(1+\frac{t^{\prime}}{100}\right)=1$
     \par
     $1,6\times \left(1+\frac{t^{\prime}}{100}\right)=1$
     \par
     $1+\frac{t^{\prime}}{100}=\frac{1}{1,6}$
     \par
     $1+\frac{t^{\prime}}{100}=0,625$
     \par
     $\frac{t^{\prime}}{100}=-0,375$
     \par
     $t^{\prime}=-37,5$
     \par
     Il faut donc que le prix diminue de 37,5\% pour compenser la hausse de 60\%.
}

\end{document}