\documentclass[a4paper]{article}

%================================================================================================================================
%
% Packages
%
%================================================================================================================================

\usepackage[T1]{fontenc} 	% pour caractères accentués
\usepackage[utf8]{inputenc}  % encodage utf8
\usepackage[french]{babel}	% langue : français
\usepackage{fourier}			% caractères plus lisibles
\usepackage[dvipsnames]{xcolor} % couleurs
\usepackage{fancyhdr}		% réglage header footer
\usepackage{needspace}		% empêcher sauts de page mal placés
\usepackage{graphicx}		% pour inclure des graphiques
\usepackage{enumitem,cprotect}		% personnalise les listes d'items (nécessaire pour ol, al ...)
\usepackage{hyperref}		% Liens hypertexte
\usepackage{pstricks,pst-all,pst-node,pstricks-add,pst-math,pst-plot,pst-tree,pst-eucl} % pstricks
\usepackage[a4paper,includeheadfoot,top=2cm,left=3cm, bottom=2cm,right=3cm]{geometry} % marges etc.
\usepackage{comment}			% commentaires multilignes
\usepackage{amsmath,environ} % maths (matrices, etc.)
\usepackage{amssymb,makeidx}
\usepackage{bm}				% bold maths
\usepackage{tabularx}		% tableaux
\usepackage{colortbl}		% tableaux en couleur
\usepackage{fontawesome}		% Fontawesome
\usepackage{environ}			% environment with command
\usepackage{fp}				% calculs pour ps-tricks
\usepackage{multido}			% pour ps tricks
\usepackage[np]{numprint}	% formattage nombre
\usepackage{tikz,tkz-tab} 			% package principal TikZ
\usepackage{pgfplots}   % axes
\usepackage{mathrsfs}    % cursives
\usepackage{calc}			% calcul taille boites
\usepackage[scaled=0.875]{helvet} % font sans serif
\usepackage{svg} % svg
\usepackage{scrextend} % local margin
\usepackage{scratch} %scratch
\usepackage{multicol} % colonnes
%\usepackage{infix-RPN,pst-func} % formule en notation polanaise inversée
\usepackage{listings}

%================================================================================================================================
%
% Réglages de base
%
%================================================================================================================================

\lstset{
language=Python,   % R code
literate=
{á}{{\'a}}1
{à}{{\`a}}1
{ã}{{\~a}}1
{é}{{\'e}}1
{è}{{\`e}}1
{ê}{{\^e}}1
{í}{{\'i}}1
{ó}{{\'o}}1
{õ}{{\~o}}1
{ú}{{\'u}}1
{ü}{{\"u}}1
{ç}{{\c{c}}}1
{~}{{ }}1
}


\definecolor{codegreen}{rgb}{0,0.6,0}
\definecolor{codegray}{rgb}{0.5,0.5,0.5}
\definecolor{codepurple}{rgb}{0.58,0,0.82}
\definecolor{backcolour}{rgb}{0.95,0.95,0.92}

\lstdefinestyle{mystyle}{
    backgroundcolor=\color{backcolour},   
    commentstyle=\color{codegreen},
    keywordstyle=\color{magenta},
    numberstyle=\tiny\color{codegray},
    stringstyle=\color{codepurple},
    basicstyle=\ttfamily\footnotesize,
    breakatwhitespace=false,         
    breaklines=true,                 
    captionpos=b,                    
    keepspaces=true,                 
    numbers=left,                    
xleftmargin=2em,
framexleftmargin=2em,            
    showspaces=false,                
    showstringspaces=false,
    showtabs=false,                  
    tabsize=2,
    upquote=true
}

\lstset{style=mystyle}


\lstset{style=mystyle}
\newcommand{\imgdir}{C:/laragon/www/newmc/assets/imgsvg/}
\newcommand{\imgsvgdir}{C:/laragon/www/newmc/assets/imgsvg/}

\definecolor{mcgris}{RGB}{220, 220, 220}% ancien~; pour compatibilité
\definecolor{mcbleu}{RGB}{52, 152, 219}
\definecolor{mcvert}{RGB}{125, 194, 70}
\definecolor{mcmauve}{RGB}{154, 0, 215}
\definecolor{mcorange}{RGB}{255, 96, 0}
\definecolor{mcturquoise}{RGB}{0, 153, 153}
\definecolor{mcrouge}{RGB}{255, 0, 0}
\definecolor{mclightvert}{RGB}{205, 234, 190}

\definecolor{gris}{RGB}{220, 220, 220}
\definecolor{bleu}{RGB}{52, 152, 219}
\definecolor{vert}{RGB}{125, 194, 70}
\definecolor{mauve}{RGB}{154, 0, 215}
\definecolor{orange}{RGB}{255, 96, 0}
\definecolor{turquoise}{RGB}{0, 153, 153}
\definecolor{rouge}{RGB}{255, 0, 0}
\definecolor{lightvert}{RGB}{205, 234, 190}
\setitemize[0]{label=\color{lightvert}  $\bullet$}

\pagestyle{fancy}
\renewcommand{\headrulewidth}{0.2pt}
\fancyhead[L]{maths-cours.fr}
\fancyhead[R]{\thepage}
\renewcommand{\footrulewidth}{0.2pt}
\fancyfoot[C]{}

\newcolumntype{C}{>{\centering\arraybackslash}X}
\newcolumntype{s}{>{\hsize=.35\hsize\arraybackslash}X}

\setlength{\parindent}{0pt}		 
\setlength{\parskip}{3mm}
\setlength{\headheight}{1cm}

\def\ebook{ebook}
\def\book{book}
\def\web{web}
\def\type{web}

\newcommand{\vect}[1]{\overrightarrow{\,\mathstrut#1\,}}

\def\Oij{$\left(\text{O}~;~\vect{\imath},~\vect{\jmath}\right)$}
\def\Oijk{$\left(\text{O}~;~\vect{\imath},~\vect{\jmath},~\vect{k}\right)$}
\def\Ouv{$\left(\text{O}~;~\vect{u},~\vect{v}\right)$}

\hypersetup{breaklinks=true, colorlinks = true, linkcolor = OliveGreen, urlcolor = OliveGreen, citecolor = OliveGreen, pdfauthor={Didier BONNEL - https://www.maths-cours.fr} } % supprime les bordures autour des liens

\renewcommand{\arg}[0]{\text{arg}}

\everymath{\displaystyle}

%================================================================================================================================
%
% Macros - Commandes
%
%================================================================================================================================

\newcommand\meta[2]{    			% Utilisé pour créer le post HTML.
	\def\titre{titre}
	\def\url{url}
	\def\arg{#1}
	\ifx\titre\arg
		\newcommand\maintitle{#2}
		\fancyhead[L]{#2}
		{\Large\sffamily \MakeUppercase{#2}}
		\vspace{1mm}\textcolor{mcvert}{\hrule}
	\fi 
	\ifx\url\arg
		\fancyfoot[L]{\href{https://www.maths-cours.fr#2}{\black \footnotesize{https://www.maths-cours.fr#2}}}
	\fi 
}


\newcommand\TitreC[1]{    		% Titre centré
     \needspace{3\baselineskip}
     \begin{center}\textbf{#1}\end{center}
}

\newcommand\newpar{    		% paragraphe
     \par
}

\newcommand\nosp {    		% commande vide (pas d'espace)
}
\newcommand{\id}[1]{} %ignore

\newcommand\boite[2]{				% Boite simple sans titre
	\vspace{5mm}
	\setlength{\fboxrule}{0.2mm}
	\setlength{\fboxsep}{5mm}	
	\fcolorbox{#1}{#1!3}{\makebox[\linewidth-2\fboxrule-2\fboxsep]{
  		\begin{minipage}[t]{\linewidth-2\fboxrule-4\fboxsep}\setlength{\parskip}{3mm}
  			 #2
  		\end{minipage}
	}}
	\vspace{5mm}
}

\newcommand\CBox[4]{				% Boites
	\vspace{5mm}
	\setlength{\fboxrule}{0.2mm}
	\setlength{\fboxsep}{5mm}
	
	\fcolorbox{#1}{#1!3}{\makebox[\linewidth-2\fboxrule-2\fboxsep]{
		\begin{minipage}[t]{1cm}\setlength{\parskip}{3mm}
	  		\textcolor{#1}{\LARGE{#2}}    
 	 	\end{minipage}  
  		\begin{minipage}[t]{\linewidth-2\fboxrule-4\fboxsep}\setlength{\parskip}{3mm}
			\raisebox{1.2mm}{\normalsize\sffamily{\textcolor{#1}{#3}}}						
  			 #4
  		\end{minipage}
	}}
	\vspace{5mm}
}

\newcommand\cadre[3]{				% Boites convertible html
	\par
	\vspace{2mm}
	\setlength{\fboxrule}{0.1mm}
	\setlength{\fboxsep}{5mm}
	\fcolorbox{#1}{white}{\makebox[\linewidth-2\fboxrule-2\fboxsep]{
  		\begin{minipage}[t]{\linewidth-2\fboxrule-4\fboxsep}\setlength{\parskip}{3mm}
			\raisebox{-2.5mm}{\sffamily \small{\textcolor{#1}{\MakeUppercase{#2}}}}		
			\par		
  			 #3
 	 		\end{minipage}
	}}
		\vspace{2mm}
	\par
}

\newcommand\bloc[3]{				% Boites convertible html sans bordure
     \needspace{2\baselineskip}
     {\sffamily \small{\textcolor{#1}{\MakeUppercase{#2}}}}    
		\par		
  			 #3
		\par
}

\newcommand\CHelp[1]{
     \CBox{Plum}{\faInfoCircle}{À RETENIR}{#1}
}

\newcommand\CUp[1]{
     \CBox{NavyBlue}{\faThumbsOUp}{EN PRATIQUE}{#1}
}

\newcommand\CInfo[1]{
     \CBox{Sepia}{\faArrowCircleRight}{REMARQUE}{#1}
}

\newcommand\CRedac[1]{
     \CBox{PineGreen}{\faEdit}{BIEN R\'EDIGER}{#1}
}

\newcommand\CError[1]{
     \CBox{Red}{\faExclamationTriangle}{ATTENTION}{#1}
}

\newcommand\TitreExo[2]{
\needspace{4\baselineskip}
 {\sffamily\large EXERCICE #1\ (\emph{#2 points})}
\vspace{5mm}
}

\newcommand\img[2]{
          \includegraphics[width=#2\paperwidth]{\imgdir#1}
}

\newcommand\imgsvg[2]{
       \begin{center}   \includegraphics[width=#2\paperwidth]{\imgsvgdir#1} \end{center}
}


\newcommand\Lien[2]{
     \href{#1}{#2 \tiny \faExternalLink}
}
\newcommand\mcLien[2]{
     \href{https~://www.maths-cours.fr/#1}{#2 \tiny \faExternalLink}
}

\newcommand{\euro}{\eurologo{}}

%================================================================================================================================
%
% Macros - Environement
%
%================================================================================================================================

\newenvironment{tex}{ %
}
{%
}

\newenvironment{indente}{ %
	\setlength\parindent{10mm}
}

{
	\setlength\parindent{0mm}
}

\newenvironment{corrige}{%
     \needspace{3\baselineskip}
     \medskip
     \textbf{\textsc{Corrigé}}
     \medskip
}
{
}

\newenvironment{extern}{%
     \begin{center}
     }
     {
     \end{center}
}

\NewEnviron{code}{%
	\par
     \boite{gray}{\texttt{%
     \BODY
     }}
     \par
}

\newenvironment{vbloc}{% boite sans cadre empeche saut de page
     \begin{minipage}[t]{\linewidth}
     }
     {
     \end{minipage}
}
\NewEnviron{h2}{%
    \needspace{3\baselineskip}
    \vspace{0.6cm}
	\noindent \MakeUppercase{\sffamily \large \BODY}
	\vspace{1mm}\textcolor{mcgris}{\hrule}\vspace{0.4cm}
	\par
}{}

\NewEnviron{h3}{%
    \needspace{3\baselineskip}
	\vspace{5mm}
	\textsc{\BODY}
	\par
}

\NewEnviron{margeneg}{ %
\begin{addmargin}[-1cm]{0cm}
\BODY
\end{addmargin}
}

\NewEnviron{html}{%
}

\begin{document}
\meta{url}{/exercices/cryptographie-bac-s-pondichery-2016-spe/}
\meta{pid}{4103}
\meta{titre}{Cryptographie - Bac S Pondichéry 2016 (spé)}
\meta{type}{exercices}
%
\begin{h2}Exercice 3 - 5 points\end{h2}
\textbf{Candidats ayant suivi l'enseignement de spécialité}
\begin{h3}Partie A\end{h3}
On considère les matrices $M$ de la forme $M = \begin{pmatrix}a&b \\ 5&3\end{pmatrix}$ où $a$ et $b$ sont des nombres entiers.
\par
Le nombre $3a-5b$ est appelé le déterminant de $M$. On le note det$(M)$.
\par
Ainsi det$(M) = 3a-5b$.
\begin{enumerate}
     \item
     Dans cette question on suppose que det$(M) \ne 0$ et on pose $N = \dfrac{1}{\text{det}(M)}\begin{pmatrix}3&-b \\ -5&a\end{pmatrix}$.
     \par
     Justifier que $N$ est l'inverse de $M$.
     \item
     On considère l'équation $(E) :\quad \text{det}(M) = 3$.
     \par
     On souhaite déterminer tous les couples d'entiers $(a~;~b)$ solutions de l'équation $(E)$.
     \begin{enumerate}
          \item
          Vérifier que le couple $(6~;~3)$ est une solution de $(E)$.
          \item
          Montrer que le couple d'entiers $(a~;~b)$ est solution de $(E)$ si et seulement si$3(a-6) = 5(b-3)$.
          \par
          En déduire l'ensemble des solutions de l'équation $(E)$.
     \end{enumerate}
\end{enumerate}
\begin{h3}Partie B\end{h3}
\begin{enumerate}
     \item
     On pose $Q = \begin{pmatrix}6 & 3 \\ 5 & 3\end{pmatrix}$.
     \par
     En utilisant la partie A, déterminer la matrice inverse de $Q$.
     \item
     \textit{Codage avec la matrice } $Q$
     \par
     Pour coder un mot de deux lettres à l'aide de la matrice $Q = \begin{pmatrix}6 &3 \\ 5& 3\end{pmatrix}$ on utilise la procédure ci-après :
     \textbf{Étape 1 :} On associe au mot la matrice $X = \begin{pmatrix}x_1 \\ x_2\end{pmatrix}$ où $x_1$ est l'entier correspondant à la première lettre du mot et $x_2$ l'entier correspondant à la deuxième lettre du mot selon le tableau de correspondance ci-dessous :
     \par
\begin{tabularx}{0.8\linewidth}{|*{3}{>{\centering \arraybackslash }X|}}%class="compact" width="600"
     \hline
 &   &  &   &   &  
     \\ \hline
        0  &  1  &  2  &  3  &  4  &  5  &  6  &  7  &  8  &  9  &  10  &  11  &  12
     \\ \hline
     &   &   &   &  & 
     \\ \hline
\end{tabularx}
     [table class=compact]A|B|C|D|E|F|G|H|I|J|K|L|M
     \par
     0|1|2|3|4|5|6|7|8|9|10|11|12[/table][table class=compact]N|O|P|Q|R|S|T|U|V|W|X|Y|Z
     \par
     13|14|15|16|17|18|19|20|21|22|23|24|25[/table]
     \textbf{Étape 2 :} La matrice $X$ est transformée en la matrice $Y = \begin{pmatrix}y_1 \\ y_2\end{pmatrix}$ telle que $Y = QX$.
     \textbf{Étape 3 :} La matrice $Y$ est transformée en la matrice $R = \begin{pmatrix}r_1 \\ r_2\end{pmatrix}$ telle que $r_1$ est le reste de la division euclidienne de $y_1$ par 26 et $r_2$ est le reste de la division euclidienne de $y_2$ par 26.
     \textbf{Étape 4 :} À la matrice $R = \begin{pmatrix}r_1 \\ r_2\end{pmatrix}$ on associe un mot de deux lettres selon le tableau de correspondance de l'étape 1.
     \begin{center}Exemple : JE$ \to X = \begin{pmatrix}9 \\ 4\end{pmatrix} \to Y = \begin{pmatrix}66 \\ 57\end{pmatrix} \to R \begin{pmatrix}14 \\ 5\end{pmatrix} \to $OF.\end{center}
     Le mot JE est codé en le mot OF.
     \par
     Coder le mot DO.
     \item
     \textit{Procédure de décodage}
     On conserve les mêmes notations que pour le codage.
     \par
     Lors du codage, la matrice $X$ a été transformée en la matrice $Y$ telle que $Y = QX$.
     \begin{enumerate}
          \item
          Démontrer que $3X = 3Q^{-1}Y$ puis que $\begin{cases} 3x_1 \equiv 3r_1-3r_2 \quad [26]\\ 3x_2 \equiv -5r_1+6r_2 \quad [26] \end{cases}$
          \item
          En remarquant que $9 \times 3 \equiv 1 \quad [26]$, montrer que $\begin{cases} x_1 \equiv r_1-r_2 \quad [26] \\ x_2 \equiv 7r_1+2r_2 \quad [26] \end{cases}$
          \item
          Décoder le mot SG.
     \end{enumerate}
\end{enumerate}
\begin{corrige}
\imgsvg{pdf-icon}{0.1}% alt="pdf" style="width:2rem"
\mcLien{/assets/pdf/cryptographie-bac-s-pondichery-2016-spe/cryptographie-bac-s-pondichery-2016-spe .pdf}{Solution rédigée par Paki} 
\end{corrige}

\end{document}