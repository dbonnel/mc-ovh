\documentclass[a4paper]{article}

%================================================================================================================================
%
% Packages
%
%================================================================================================================================

\usepackage[T1]{fontenc} 	% pour caractères accentués
\usepackage[utf8]{inputenc}  % encodage utf8
\usepackage[french]{babel}	% langue : français
\usepackage{fourier}			% caractères plus lisibles
\usepackage[dvipsnames]{xcolor} % couleurs
\usepackage{fancyhdr}		% réglage header footer
\usepackage{needspace}		% empêcher sauts de page mal placés
\usepackage{graphicx}		% pour inclure des graphiques
\usepackage{enumitem,cprotect}		% personnalise les listes d'items (nécessaire pour ol, al ...)
\usepackage{hyperref}		% Liens hypertexte
\usepackage{pstricks,pst-all,pst-node,pstricks-add,pst-math,pst-plot,pst-tree,pst-eucl} % pstricks
\usepackage[a4paper,includeheadfoot,top=2cm,left=3cm, bottom=2cm,right=3cm]{geometry} % marges etc.
\usepackage{comment}			% commentaires multilignes
\usepackage{amsmath,environ} % maths (matrices, etc.)
\usepackage{amssymb,makeidx}
\usepackage{bm}				% bold maths
\usepackage{tabularx}		% tableaux
\usepackage{colortbl}		% tableaux en couleur
\usepackage{fontawesome}		% Fontawesome
\usepackage{environ}			% environment with command
\usepackage{fp}				% calculs pour ps-tricks
\usepackage{multido}			% pour ps tricks
\usepackage[np]{numprint}	% formattage nombre
\usepackage{tikz,tkz-tab} 			% package principal TikZ
\usepackage{pgfplots}   % axes
\usepackage{mathrsfs}    % cursives
\usepackage{calc}			% calcul taille boites
\usepackage[scaled=0.875]{helvet} % font sans serif
\usepackage{svg} % svg
\usepackage{scrextend} % local margin
\usepackage{scratch} %scratch
\usepackage{multicol} % colonnes
%\usepackage{infix-RPN,pst-func} % formule en notation polanaise inversée
\usepackage{listings}

%================================================================================================================================
%
% Réglages de base
%
%================================================================================================================================

\lstset{
language=Python,   % R code
literate=
{á}{{\'a}}1
{à}{{\`a}}1
{ã}{{\~a}}1
{é}{{\'e}}1
{è}{{\`e}}1
{ê}{{\^e}}1
{í}{{\'i}}1
{ó}{{\'o}}1
{õ}{{\~o}}1
{ú}{{\'u}}1
{ü}{{\"u}}1
{ç}{{\c{c}}}1
{~}{{ }}1
}


\definecolor{codegreen}{rgb}{0,0.6,0}
\definecolor{codegray}{rgb}{0.5,0.5,0.5}
\definecolor{codepurple}{rgb}{0.58,0,0.82}
\definecolor{backcolour}{rgb}{0.95,0.95,0.92}

\lstdefinestyle{mystyle}{
    backgroundcolor=\color{backcolour},   
    commentstyle=\color{codegreen},
    keywordstyle=\color{magenta},
    numberstyle=\tiny\color{codegray},
    stringstyle=\color{codepurple},
    basicstyle=\ttfamily\footnotesize,
    breakatwhitespace=false,         
    breaklines=true,                 
    captionpos=b,                    
    keepspaces=true,                 
    numbers=left,                    
xleftmargin=2em,
framexleftmargin=2em,            
    showspaces=false,                
    showstringspaces=false,
    showtabs=false,                  
    tabsize=2,
    upquote=true
}

\lstset{style=mystyle}


\lstset{style=mystyle}
\newcommand{\imgdir}{C:/laragon/www/newmc/assets/imgsvg/}
\newcommand{\imgsvgdir}{C:/laragon/www/newmc/assets/imgsvg/}

\definecolor{mcgris}{RGB}{220, 220, 220}% ancien~; pour compatibilité
\definecolor{mcbleu}{RGB}{52, 152, 219}
\definecolor{mcvert}{RGB}{125, 194, 70}
\definecolor{mcmauve}{RGB}{154, 0, 215}
\definecolor{mcorange}{RGB}{255, 96, 0}
\definecolor{mcturquoise}{RGB}{0, 153, 153}
\definecolor{mcrouge}{RGB}{255, 0, 0}
\definecolor{mclightvert}{RGB}{205, 234, 190}

\definecolor{gris}{RGB}{220, 220, 220}
\definecolor{bleu}{RGB}{52, 152, 219}
\definecolor{vert}{RGB}{125, 194, 70}
\definecolor{mauve}{RGB}{154, 0, 215}
\definecolor{orange}{RGB}{255, 96, 0}
\definecolor{turquoise}{RGB}{0, 153, 153}
\definecolor{rouge}{RGB}{255, 0, 0}
\definecolor{lightvert}{RGB}{205, 234, 190}
\setitemize[0]{label=\color{lightvert}  $\bullet$}

\pagestyle{fancy}
\renewcommand{\headrulewidth}{0.2pt}
\fancyhead[L]{maths-cours.fr}
\fancyhead[R]{\thepage}
\renewcommand{\footrulewidth}{0.2pt}
\fancyfoot[C]{}

\newcolumntype{C}{>{\centering\arraybackslash}X}
\newcolumntype{s}{>{\hsize=.35\hsize\arraybackslash}X}

\setlength{\parindent}{0pt}		 
\setlength{\parskip}{3mm}
\setlength{\headheight}{1cm}

\def\ebook{ebook}
\def\book{book}
\def\web{web}
\def\type{web}

\newcommand{\vect}[1]{\overrightarrow{\,\mathstrut#1\,}}

\def\Oij{$\left(\text{O}~;~\vect{\imath},~\vect{\jmath}\right)$}
\def\Oijk{$\left(\text{O}~;~\vect{\imath},~\vect{\jmath},~\vect{k}\right)$}
\def\Ouv{$\left(\text{O}~;~\vect{u},~\vect{v}\right)$}

\hypersetup{breaklinks=true, colorlinks = true, linkcolor = OliveGreen, urlcolor = OliveGreen, citecolor = OliveGreen, pdfauthor={Didier BONNEL - https://www.maths-cours.fr} } % supprime les bordures autour des liens

\renewcommand{\arg}[0]{\text{arg}}

\everymath{\displaystyle}

%================================================================================================================================
%
% Macros - Commandes
%
%================================================================================================================================

\newcommand\meta[2]{    			% Utilisé pour créer le post HTML.
	\def\titre{titre}
	\def\url{url}
	\def\arg{#1}
	\ifx\titre\arg
		\newcommand\maintitle{#2}
		\fancyhead[L]{#2}
		{\Large\sffamily \MakeUppercase{#2}}
		\vspace{1mm}\textcolor{mcvert}{\hrule}
	\fi 
	\ifx\url\arg
		\fancyfoot[L]{\href{https://www.maths-cours.fr#2}{\black \footnotesize{https://www.maths-cours.fr#2}}}
	\fi 
}


\newcommand\TitreC[1]{    		% Titre centré
     \needspace{3\baselineskip}
     \begin{center}\textbf{#1}\end{center}
}

\newcommand\newpar{    		% paragraphe
     \par
}

\newcommand\nosp {    		% commande vide (pas d'espace)
}
\newcommand{\id}[1]{} %ignore

\newcommand\boite[2]{				% Boite simple sans titre
	\vspace{5mm}
	\setlength{\fboxrule}{0.2mm}
	\setlength{\fboxsep}{5mm}	
	\fcolorbox{#1}{#1!3}{\makebox[\linewidth-2\fboxrule-2\fboxsep]{
  		\begin{minipage}[t]{\linewidth-2\fboxrule-4\fboxsep}\setlength{\parskip}{3mm}
  			 #2
  		\end{minipage}
	}}
	\vspace{5mm}
}

\newcommand\CBox[4]{				% Boites
	\vspace{5mm}
	\setlength{\fboxrule}{0.2mm}
	\setlength{\fboxsep}{5mm}
	
	\fcolorbox{#1}{#1!3}{\makebox[\linewidth-2\fboxrule-2\fboxsep]{
		\begin{minipage}[t]{1cm}\setlength{\parskip}{3mm}
	  		\textcolor{#1}{\LARGE{#2}}    
 	 	\end{minipage}  
  		\begin{minipage}[t]{\linewidth-2\fboxrule-4\fboxsep}\setlength{\parskip}{3mm}
			\raisebox{1.2mm}{\normalsize\sffamily{\textcolor{#1}{#3}}}						
  			 #4
  		\end{minipage}
	}}
	\vspace{5mm}
}

\newcommand\cadre[3]{				% Boites convertible html
	\par
	\vspace{2mm}
	\setlength{\fboxrule}{0.1mm}
	\setlength{\fboxsep}{5mm}
	\fcolorbox{#1}{white}{\makebox[\linewidth-2\fboxrule-2\fboxsep]{
  		\begin{minipage}[t]{\linewidth-2\fboxrule-4\fboxsep}\setlength{\parskip}{3mm}
			\raisebox{-2.5mm}{\sffamily \small{\textcolor{#1}{\MakeUppercase{#2}}}}		
			\par		
  			 #3
 	 		\end{minipage}
	}}
		\vspace{2mm}
	\par
}

\newcommand\bloc[3]{				% Boites convertible html sans bordure
     \needspace{2\baselineskip}
     {\sffamily \small{\textcolor{#1}{\MakeUppercase{#2}}}}    
		\par		
  			 #3
		\par
}

\newcommand\CHelp[1]{
     \CBox{Plum}{\faInfoCircle}{À RETENIR}{#1}
}

\newcommand\CUp[1]{
     \CBox{NavyBlue}{\faThumbsOUp}{EN PRATIQUE}{#1}
}

\newcommand\CInfo[1]{
     \CBox{Sepia}{\faArrowCircleRight}{REMARQUE}{#1}
}

\newcommand\CRedac[1]{
     \CBox{PineGreen}{\faEdit}{BIEN R\'EDIGER}{#1}
}

\newcommand\CError[1]{
     \CBox{Red}{\faExclamationTriangle}{ATTENTION}{#1}
}

\newcommand\TitreExo[2]{
\needspace{4\baselineskip}
 {\sffamily\large EXERCICE #1\ (\emph{#2 points})}
\vspace{5mm}
}

\newcommand\img[2]{
          \includegraphics[width=#2\paperwidth]{\imgdir#1}
}

\newcommand\imgsvg[2]{
       \begin{center}   \includegraphics[width=#2\paperwidth]{\imgsvgdir#1} \end{center}
}


\newcommand\Lien[2]{
     \href{#1}{#2 \tiny \faExternalLink}
}
\newcommand\mcLien[2]{
     \href{https~://www.maths-cours.fr/#1}{#2 \tiny \faExternalLink}
}

\newcommand{\euro}{\eurologo{}}

%================================================================================================================================
%
% Macros - Environement
%
%================================================================================================================================

\newenvironment{tex}{ %
}
{%
}

\newenvironment{indente}{ %
	\setlength\parindent{10mm}
}

{
	\setlength\parindent{0mm}
}

\newenvironment{corrige}{%
     \needspace{3\baselineskip}
     \medskip
     \textbf{\textsc{Corrigé}}
     \medskip
}
{
}

\newenvironment{extern}{%
     \begin{center}
     }
     {
     \end{center}
}

\NewEnviron{code}{%
	\par
     \boite{gray}{\texttt{%
     \BODY
     }}
     \par
}

\newenvironment{vbloc}{% boite sans cadre empeche saut de page
     \begin{minipage}[t]{\linewidth}
     }
     {
     \end{minipage}
}
\NewEnviron{h2}{%
    \needspace{3\baselineskip}
    \vspace{0.6cm}
	\noindent \MakeUppercase{\sffamily \large \BODY}
	\vspace{1mm}\textcolor{mcgris}{\hrule}\vspace{0.4cm}
	\par
}{}

\NewEnviron{h3}{%
    \needspace{3\baselineskip}
	\vspace{5mm}
	\textsc{\BODY}
	\par
}

\NewEnviron{margeneg}{ %
\begin{addmargin}[-1cm]{0cm}
\BODY
\end{addmargin}
}

\NewEnviron{html}{%
}

\begin{document}
\meta{url}{/methode/algorithme-premiers-termes/}
\meta{pid}{871}
\meta{titre}{Algorithme de calcul des premiers termes d'une suite}
\meta{type}{methode}
%
\cadre{vert}{Situation}{%
     On considère une suite $\left(u_{n}\right)$ définie par son premier terme $u_{0}$ et par une relation de récurrence du type $u_{n+1}=f\left(u_{n}\right)$
     \par
     On souhaite écrire un algorithme permettant de calculer et d'afficher les termes $u_{0}$ à $u_{k}$ où $k$ est un nombre entré par l'utilisateur.
}
\begin{h3}1. Algorithme\end{h3}
Voici un algorithme répondant à la question pour la suite $\left(u_{n}\right)$ définie par :
\par
$\left\{ \begin{matrix} u_{0}=3 \\ u_{n+1} = 0,5u_{n}+2\end{matrix}\right.$
\par
     \textbf{Remarque : }Cet algorithme n'est pas le seul possible.
\\
    \begin{tabularx}{0.8\linewidth}{|*{3}{>{\centering \arraybackslash }X|}}%class="singleborder" width="600"
          \hline
          \textbf{ 1.} & \textbf{Variables}  &  	$i$ et $k$ sont des entiers naturels
          \\ \hline
          \textbf{ 2.} &  & $u$ est un réel
          \\ \hline
          \textbf{ 3.} & \textbf{Entrée}  & Saisir la valeur de $k$
          \\ \hline
          \textbf{ 4.} & \textbf{Début traitement} : &  	$u$ prend la valeur 3
          \\ \hline
          \textbf{ 5.} &  & Afficher $u$
               \\ \hline
               \textbf{ 6.} &   & Pour $i$ allant de $1$ à $k$
               \\ \hline
               \textbf{ 7.} &  & $\quad$$\quad$$u$ prend la valeur $0,5\times u+2$
               \\ \hline
               \textbf{ 8.} &  & $\quad$$\quad$Afficher $u$
               \\ \hline
               \textbf{ 9.} &  &  Fin Pour
               \\ \hline
               \textbf{10.} & \textbf{Fin traitement}  &
               \\ \hline
          \end{tabularx}
     \begin{h3}2. Commentaires\end{h3}
\\      \textbf{Lignes 1 et 2 : } On définit 3 variables :
     \begin{itemize}
          \item
          $k$ contiendra la valeur saisie par l'utilisateur qui déterminera l'arrêt de la boucle. $k$ ne sera pas modifié lors du traitement mais gardera une valeur constante
          \item
          $i$ contiendra le rang (indice) du terme que l'on calcule à partir du rang 1. $i$ variera de $1$ à $k$
          \item
          $u$ contiendra les valeurs de $u_{i}$. Notez que l'on définit \textbf{une seule variable pour l'ensemble des termes de la suites}. Au départ cette variable sera initialisée à $u_{0}$. Puis on calculera $u_{1}$ qui viendra «écraser» $u_{0}$. Puis $u_{2}$ viendra écraser $u_{1}$ et ainsi de suite...
     \end{itemize}
\\      \textbf{Ligne 3 :} La valeur saisie par l'utilisateur qui déterminera l'arrêt de l'algorithme est stockée dans la variable $k$
\\      \textbf{Ligne 4 :} On initialise $u$ en lui donnant la valeur de $u_{0}$ (ici $u_{0}=3$).
\\      \textbf{Ligne 5 :} On affiche la valeur de $u$ (qui contient actuellement $3$). Cette ligne est nécessaire pour afficher la valeur de $u_{0}$ car la boucle qui suit n'affichera que les valeurs de $u_{1}$ à $u_{n}$.
\\      \textbf{Ligne 6 :} On crée une boucle qui fera varier l'indice $i$ de $1$ à $k$. Puisqu'ici on connait le nombre d'itérations $k$, une boucle \textit{Pour} a été préférée à une boucle \textit{Tant que}.
\\      \textbf{Ligne 7 :} On modifie la valeur de $u$ : La nouvelle valeur de $u$ sera égale à l'ancienne valeur de $u$ fois $0,5$ plus $2$. Cela traduit bien la relation de récurrence  $u_{n+1}= 0,5u_{n}+2$.
\\      \textbf{Ligne 8 :} On affiche le terme que l'on vient de calculer (à savoir $u_{i}$).
\\      \textbf{Ligne 9 :} On « ferme » la boucle; on retourne à la ligne 6; si $i$ valait $k$, la boucle se terminera alors et  on passera à la ligne 10.
\\      \textbf{Ligne 10 :} L'algorithme est terminé !
     \textbf{Remarque : }Il faut toujours être très attentif au nombre de passages dans la boucle et au nombre d'affichages. Pour vérifier son algorithme, on peut :
     \begin{itemize}
          \item
          faire « tourner » l'algorithme (c'est à dire créer un tableau contenant les valeurs des variables étape par étape) - voir \textbf{3.} ci-dessous.
          \item
          compter le nombre d'affichages :
          \par
          Ici on souhaite afficher les valeurs de $u_{0}$ à $u_{k}$, c'est à dire $k+1$ valeurs.
          \par
          La ligne \textbf{5.} effectue un premier affichage (de $u_{0}$).
          \par
          La boucle affichera, quant à elle, $k$ valeurs puisque $i$ varie de $1$ à $k$
          \par
          En tout on a donc bien effectué $k+1$ affichages.
     \end{itemize}
     \begin{h3}3. Résultats\end{h3}
     Le tableau ci dessous récapitule les valeurs prises par les variables pour $k=4$
     \begin{tabularx}{0.8\linewidth}{|*{3}{>{\centering \arraybackslash }X|}}%class="compact" width="600"
          \hline
          $   k   $ & $  i  $ & fin de boucle ? & $   u   $
          \\ \hline
          4 & - & - & 3
          \\ \hline
          4 & 1 & non & 3,5
          \\ \hline
          4 & 2 & non & 3,75
          \\ \hline
          4 & 3 & non & 3,875
          \\ \hline
          4 & 4 & non & 3,9375
          \\ \hline
          4 & 5 & oui &
          \\ \hline
     \end{tabularx}
     \begin{h3}4. Variante\end{h3}
     Cette fois, on ne souhaite pas afficher toutes les valeurs de $u_{0}$ à $u_{k}$ mais uniquement la valeur $u_{k}$.
     \par
     Les modifications à apporter à l'algorithme sont les suivantes :
     \begin{itemize}
          \item
          On supprime la ligne 5 puisque l'on ne souhaite plus afficher $u_{0}$
          \item
          On supprime la ligne 8 puisque l'on ne souhaite plus afficher tous les termes de $u_{1}$ à $u_{n}$
          \item
          On ajoute une ligne « Afficher $u$ » \textbf{après la boucle} pour afficher la dernière valeur calculée dans la boucle (et qui correspond à $u_{k}$)
     \end{itemize}
     On obtient l'algorithme ci-dessous :
     \begin{tabularx}{0.8\linewidth}{|*{3}{>{\centering \arraybackslash }X|}}%class="singleborder" width="600"
          \hline
          \textbf{ 1.} & \textbf{Variables}  &  	$i$ et $k$ sont des entiers naturels
          \\ \hline
          \textbf{ 2.} &  & $u$ est un réel
          \\ \hline
          \textbf{ 3.} & \textbf{Entrée}  & Saisir la valeur de $k$
          \\ \hline
          \textbf{ 4.} & \textbf{Début traitement} : &  	$u$ prend la valeur 3
          \\ \hline
          \textbf{ 5.} &   & Pour $i$ allant de $1$ à $k$
          \\ \hline
          \textbf{ 6.} &  & $\quad$$\quad$$\quad$$u$ prend la valeur $0,5\times u+2$
          \\ \hline
          \textbf{ 7.} &  &  Fin Pour
          \\ \hline
          \textbf{ 8.} &  & \textbf{Afficher $u$}
          \\ \hline
          \textbf{9.} & \textbf{Fin traitement}  &
          \\ \hline
     \end{tabularx}

\end{document}