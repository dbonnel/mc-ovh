\documentclass[a4paper]{article}

%================================================================================================================================
%
% Packages
%
%================================================================================================================================

\usepackage[T1]{fontenc} 	% pour caractères accentués
\usepackage[utf8]{inputenc}  % encodage utf8
\usepackage[french]{babel}	% langue : français
\usepackage{fourier}			% caractères plus lisibles
\usepackage[dvipsnames]{xcolor} % couleurs
\usepackage{fancyhdr}		% réglage header footer
\usepackage{needspace}		% empêcher sauts de page mal placés
\usepackage{graphicx}		% pour inclure des graphiques
\usepackage{enumitem,cprotect}		% personnalise les listes d'items (nécessaire pour ol, al ...)
\usepackage{hyperref}		% Liens hypertexte
\usepackage{pstricks,pst-all,pst-node,pstricks-add,pst-math,pst-plot,pst-tree,pst-eucl} % pstricks
\usepackage[a4paper,includeheadfoot,top=2cm,left=3cm, bottom=2cm,right=3cm]{geometry} % marges etc.
\usepackage{comment}			% commentaires multilignes
\usepackage{amsmath,environ} % maths (matrices, etc.)
\usepackage{amssymb,makeidx}
\usepackage{bm}				% bold maths
\usepackage{tabularx}		% tableaux
\usepackage{colortbl}		% tableaux en couleur
\usepackage{fontawesome}		% Fontawesome
\usepackage{environ}			% environment with command
\usepackage{fp}				% calculs pour ps-tricks
\usepackage{multido}			% pour ps tricks
\usepackage[np]{numprint}	% formattage nombre
\usepackage{tikz,tkz-tab} 			% package principal TikZ
\usepackage{pgfplots}   % axes
\usepackage{mathrsfs}    % cursives
\usepackage{calc}			% calcul taille boites
\usepackage[scaled=0.875]{helvet} % font sans serif
\usepackage{svg} % svg
\usepackage{scrextend} % local margin
\usepackage{scratch} %scratch
\usepackage{multicol} % colonnes
%\usepackage{infix-RPN,pst-func} % formule en notation polanaise inversée
\usepackage{listings}

%================================================================================================================================
%
% Réglages de base
%
%================================================================================================================================

\lstset{
language=Python,   % R code
literate=
{á}{{\'a}}1
{à}{{\`a}}1
{ã}{{\~a}}1
{é}{{\'e}}1
{è}{{\`e}}1
{ê}{{\^e}}1
{í}{{\'i}}1
{ó}{{\'o}}1
{õ}{{\~o}}1
{ú}{{\'u}}1
{ü}{{\"u}}1
{ç}{{\c{c}}}1
{~}{{ }}1
}


\definecolor{codegreen}{rgb}{0,0.6,0}
\definecolor{codegray}{rgb}{0.5,0.5,0.5}
\definecolor{codepurple}{rgb}{0.58,0,0.82}
\definecolor{backcolour}{rgb}{0.95,0.95,0.92}

\lstdefinestyle{mystyle}{
    backgroundcolor=\color{backcolour},   
    commentstyle=\color{codegreen},
    keywordstyle=\color{magenta},
    numberstyle=\tiny\color{codegray},
    stringstyle=\color{codepurple},
    basicstyle=\ttfamily\footnotesize,
    breakatwhitespace=false,         
    breaklines=true,                 
    captionpos=b,                    
    keepspaces=true,                 
    numbers=left,                    
xleftmargin=2em,
framexleftmargin=2em,            
    showspaces=false,                
    showstringspaces=false,
    showtabs=false,                  
    tabsize=2,
    upquote=true
}

\lstset{style=mystyle}


\lstset{style=mystyle}
\newcommand{\imgdir}{C:/laragon/www/newmc/assets/imgsvg/}
\newcommand{\imgsvgdir}{C:/laragon/www/newmc/assets/imgsvg/}

\definecolor{mcgris}{RGB}{220, 220, 220}% ancien~; pour compatibilité
\definecolor{mcbleu}{RGB}{52, 152, 219}
\definecolor{mcvert}{RGB}{125, 194, 70}
\definecolor{mcmauve}{RGB}{154, 0, 215}
\definecolor{mcorange}{RGB}{255, 96, 0}
\definecolor{mcturquoise}{RGB}{0, 153, 153}
\definecolor{mcrouge}{RGB}{255, 0, 0}
\definecolor{mclightvert}{RGB}{205, 234, 190}

\definecolor{gris}{RGB}{220, 220, 220}
\definecolor{bleu}{RGB}{52, 152, 219}
\definecolor{vert}{RGB}{125, 194, 70}
\definecolor{mauve}{RGB}{154, 0, 215}
\definecolor{orange}{RGB}{255, 96, 0}
\definecolor{turquoise}{RGB}{0, 153, 153}
\definecolor{rouge}{RGB}{255, 0, 0}
\definecolor{lightvert}{RGB}{205, 234, 190}
\setitemize[0]{label=\color{lightvert}  $\bullet$}

\pagestyle{fancy}
\renewcommand{\headrulewidth}{0.2pt}
\fancyhead[L]{maths-cours.fr}
\fancyhead[R]{\thepage}
\renewcommand{\footrulewidth}{0.2pt}
\fancyfoot[C]{}

\newcolumntype{C}{>{\centering\arraybackslash}X}
\newcolumntype{s}{>{\hsize=.35\hsize\arraybackslash}X}

\setlength{\parindent}{0pt}		 
\setlength{\parskip}{3mm}
\setlength{\headheight}{1cm}

\def\ebook{ebook}
\def\book{book}
\def\web{web}
\def\type{web}

\newcommand{\vect}[1]{\overrightarrow{\,\mathstrut#1\,}}

\def\Oij{$\left(\text{O}~;~\vect{\imath},~\vect{\jmath}\right)$}
\def\Oijk{$\left(\text{O}~;~\vect{\imath},~\vect{\jmath},~\vect{k}\right)$}
\def\Ouv{$\left(\text{O}~;~\vect{u},~\vect{v}\right)$}

\hypersetup{breaklinks=true, colorlinks = true, linkcolor = OliveGreen, urlcolor = OliveGreen, citecolor = OliveGreen, pdfauthor={Didier BONNEL - https://www.maths-cours.fr} } % supprime les bordures autour des liens

\renewcommand{\arg}[0]{\text{arg}}

\everymath{\displaystyle}

%================================================================================================================================
%
% Macros - Commandes
%
%================================================================================================================================

\newcommand\meta[2]{    			% Utilisé pour créer le post HTML.
	\def\titre{titre}
	\def\url{url}
	\def\arg{#1}
	\ifx\titre\arg
		\newcommand\maintitle{#2}
		\fancyhead[L]{#2}
		{\Large\sffamily \MakeUppercase{#2}}
		\vspace{1mm}\textcolor{mcvert}{\hrule}
	\fi 
	\ifx\url\arg
		\fancyfoot[L]{\href{https://www.maths-cours.fr#2}{\black \footnotesize{https://www.maths-cours.fr#2}}}
	\fi 
}


\newcommand\TitreC[1]{    		% Titre centré
     \needspace{3\baselineskip}
     \begin{center}\textbf{#1}\end{center}
}

\newcommand\newpar{    		% paragraphe
     \par
}

\newcommand\nosp {    		% commande vide (pas d'espace)
}
\newcommand{\id}[1]{} %ignore

\newcommand\boite[2]{				% Boite simple sans titre
	\vspace{5mm}
	\setlength{\fboxrule}{0.2mm}
	\setlength{\fboxsep}{5mm}	
	\fcolorbox{#1}{#1!3}{\makebox[\linewidth-2\fboxrule-2\fboxsep]{
  		\begin{minipage}[t]{\linewidth-2\fboxrule-4\fboxsep}\setlength{\parskip}{3mm}
  			 #2
  		\end{minipage}
	}}
	\vspace{5mm}
}

\newcommand\CBox[4]{				% Boites
	\vspace{5mm}
	\setlength{\fboxrule}{0.2mm}
	\setlength{\fboxsep}{5mm}
	
	\fcolorbox{#1}{#1!3}{\makebox[\linewidth-2\fboxrule-2\fboxsep]{
		\begin{minipage}[t]{1cm}\setlength{\parskip}{3mm}
	  		\textcolor{#1}{\LARGE{#2}}    
 	 	\end{minipage}  
  		\begin{minipage}[t]{\linewidth-2\fboxrule-4\fboxsep}\setlength{\parskip}{3mm}
			\raisebox{1.2mm}{\normalsize\sffamily{\textcolor{#1}{#3}}}						
  			 #4
  		\end{minipage}
	}}
	\vspace{5mm}
}

\newcommand\cadre[3]{				% Boites convertible html
	\par
	\vspace{2mm}
	\setlength{\fboxrule}{0.1mm}
	\setlength{\fboxsep}{5mm}
	\fcolorbox{#1}{white}{\makebox[\linewidth-2\fboxrule-2\fboxsep]{
  		\begin{minipage}[t]{\linewidth-2\fboxrule-4\fboxsep}\setlength{\parskip}{3mm}
			\raisebox{-2.5mm}{\sffamily \small{\textcolor{#1}{\MakeUppercase{#2}}}}		
			\par		
  			 #3
 	 		\end{minipage}
	}}
		\vspace{2mm}
	\par
}

\newcommand\bloc[3]{				% Boites convertible html sans bordure
     \needspace{2\baselineskip}
     {\sffamily \small{\textcolor{#1}{\MakeUppercase{#2}}}}    
		\par		
  			 #3
		\par
}

\newcommand\CHelp[1]{
     \CBox{Plum}{\faInfoCircle}{À RETENIR}{#1}
}

\newcommand\CUp[1]{
     \CBox{NavyBlue}{\faThumbsOUp}{EN PRATIQUE}{#1}
}

\newcommand\CInfo[1]{
     \CBox{Sepia}{\faArrowCircleRight}{REMARQUE}{#1}
}

\newcommand\CRedac[1]{
     \CBox{PineGreen}{\faEdit}{BIEN R\'EDIGER}{#1}
}

\newcommand\CError[1]{
     \CBox{Red}{\faExclamationTriangle}{ATTENTION}{#1}
}

\newcommand\TitreExo[2]{
\needspace{4\baselineskip}
 {\sffamily\large EXERCICE #1\ (\emph{#2 points})}
\vspace{5mm}
}

\newcommand\img[2]{
          \includegraphics[width=#2\paperwidth]{\imgdir#1}
}

\newcommand\imgsvg[2]{
       \begin{center}   \includegraphics[width=#2\paperwidth]{\imgsvgdir#1} \end{center}
}


\newcommand\Lien[2]{
     \href{#1}{#2 \tiny \faExternalLink}
}
\newcommand\mcLien[2]{
     \href{https~://www.maths-cours.fr/#1}{#2 \tiny \faExternalLink}
}

\newcommand{\euro}{\eurologo{}}

%================================================================================================================================
%
% Macros - Environement
%
%================================================================================================================================

\newenvironment{tex}{ %
}
{%
}

\newenvironment{indente}{ %
	\setlength\parindent{10mm}
}

{
	\setlength\parindent{0mm}
}

\newenvironment{corrige}{%
     \needspace{3\baselineskip}
     \medskip
     \textbf{\textsc{Corrigé}}
     \medskip
}
{
}

\newenvironment{extern}{%
     \begin{center}
     }
     {
     \end{center}
}

\NewEnviron{code}{%
	\par
     \boite{gray}{\texttt{%
     \BODY
     }}
     \par
}

\newenvironment{vbloc}{% boite sans cadre empeche saut de page
     \begin{minipage}[t]{\linewidth}
     }
     {
     \end{minipage}
}
\NewEnviron{h2}{%
    \needspace{3\baselineskip}
    \vspace{0.6cm}
	\noindent \MakeUppercase{\sffamily \large \BODY}
	\vspace{1mm}\textcolor{mcgris}{\hrule}\vspace{0.4cm}
	\par
}{}

\NewEnviron{h3}{%
    \needspace{3\baselineskip}
	\vspace{5mm}
	\textsc{\BODY}
	\par
}

\NewEnviron{margeneg}{ %
\begin{addmargin}[-1cm]{0cm}
\BODY
\end{addmargin}
}

\NewEnviron{html}{%
}

\begin{document}
\meta{url}{/cours/division-euclidienne-pgcd/}
\meta{pid}{1552}
\meta{titre}{Division euclidienne - Nombres premiers - PGCD}
\meta{type}{cours}
\begin{h2}1 - Division euclidienne\end{h2}
\cadre{bleu}{Définition}{% id="d10"
     Soient $a$ et $b$, deux nombres entiers naturels (c'est à dire positifs) avec $b\neq 0$.
     \par
     Effectuer la \textbf{division euclidienne} de $a$ par $b$, c'est trouver deux entiers naturels $q$ et $r$ tels que~:
     \begin{center}$a = b\times q+r $ et $ r < b$\end{center}
     $q$ s'appelle le \textbf{quotient} et $r$ le \textbf{reste}.
}
\bloc{orange}{Exemple}{% id="e10"
     \begin{center}
          \imgsvg{division}{0.2}%width="180" alt="division euclidienne"
     \end{center}
     Écriture en ligne~:
     \par
     $6894 = 23\times 299 + 17$
     \par
     $299$ est le \textbf{quotient} et $17$ le \textbf{reste}.
}
\bloc{cyan}{Remarque}{% id="r10"
     Sur la plupart des calculatrices de collège la touche qui permet d'effectuer la division euclidienne est notée~: \img{touche-divise}{0.008}%width="8" alt="touche division euclidienne"
     .
     \par
     Par exemple, la suite de touches à entrer pour obtenir la division euclidienne de $6894$ par $23$ sur une TI-Collège est~:
     \begin{center}
          \imgsvg{touches}{0.25}%width="300" alt="suite de touches division euclidienne"
     \end{center}
     et voici le résultat obtenu à l'écran~:
     \begin{center}
          \imgsvg{result}{0.15}%width="160" alt="resultat division euclidienne"
     \end{center}
}
\cadre{bleu}{Définition}{% id="d20"
     On dit que $a$ est \textbf{divisible} par $b$ si le reste de la division euclidienne de $a$ par $b$ est nul.
     \par
     Cela revient à dire qu'il existe un entier naturel $q$ tel que $a = b\times q$.
     \par
     Les expressions suivantes sont synonymes~:
     \begin{itemize}
          \item $a$ est divisible par $b$
          \item $a$ est un multiple de $b$
          \item $b$ est un diviseur de $a$
          \item $b$ divise $a$ (que l'on écrit parfois $b | a$)
     \end{itemize}
}
\bloc{orange}{Exemple}{% id="e20"
     La division euclidienne de $630$ par $15$ donne un quotient de $42$ et un reste nul.
     \par
     On a donc $630 = 15\times 42$.
     \par
     On peut dire que~:
     \begin{itemize}
          \item $630$ est divisible par $15$
          \item $630$ est un multiple de $15$
          \item $15$ est un diviseur de $630$
          \item $15$ divise $630$
     \end{itemize}
     (On peut aussi dire que $630$ est divisible par $42$, etc.)
}
\cadre{rouge}{Critères de divisibilité}{% id="t30"
     \begin{itemize}
          \item Un entier naturel est divisible par 2 si son \textbf{chiffre des unités} est 0, 2, 4, 6 ou 8.
          \item Un entier naturel est divisible par 3 si la \textbf{somme de ses chiffres} est divisible par 3.
          \item Un entier naturel est divisible par 4 si le nombre formé par ses \textbf{deux derniers chiffres} est divisible par 4.
          \item Un entier naturel est divisible par 5 si son \textbf{chiffre des unités} est 0 ou 5.
          \item Un entier naturel est divisible par 9 si la \textbf{somme de ses chiffres} est divisible par 9.
          \item Un entier naturel est divisible par 10 si son\textbf{ chiffre des unités} est 0.
     \end{itemize}
}
\bloc{cyan}{Remarques}{% id="r30"
     \begin{itemize}
          \item \textbf{Attention~: }Pour les critères de divisibilité par 3 et par 9, il faut effectuer \textbf{la somme des chiffres} (et non regarder le chiffre des unités)
          \item Il n'existe pas de critère de divisibilité par 7 qui soit très simple. Le plus rapide est en général d'effectuer la division~!
     \end{itemize}
}
\bloc{orange}{Exemple}{% id="e30"
     \begin{itemize}
          \item $1314$ est divisible par $2$ (chiffre des unités~: 4)
          \item $1314$ est divisible par $3$ (somme des chiffres~: 9)
          \item $1314$ n'est pas divisible par $4$ (deux derniers chiffres~: 14)
          \item $1314$ n'est pas divisible par $5$ (chiffre des unités~: 4)
          \item $1314$ est divisible par $9$ (somme des chiffres~: 9)
          \item $1314$ n'est pas divisible par $10$ (chiffre des unités~: 4)
     \end{itemize}
}
\begin{h2}2 - Nombres premiers\end{h2}
\cadre{bleu}{Définition}{ % id="d45"
     On dit qu'un nombre entier naturel est \textbf{premier} s'il possède exactement deux diviseurs~: 1 et lui-même.
} % fin définition
\bloc{orange}{Exemples}{ % id=e47
     \begin{itemize}
          \item
          2; 3; 5 sont des nombres premiers~;
          \item
          0 \textbf{n'est pas} un nombre premier car il est divisible par tous les entiers supérieurs ou égal à 1.
          \item
          1 \textbf{n'est pas} un nombre premier car il n'admet qu' \textbf{un seul} diviseur (lui-même).
          \item
          À l'exception du nombre 2, tous les entiers pairs \textbf{ne sont pas} des nombres premiers (car ils sont divisibles par 2). Cela signifie qu'à l'exception du nombre 2, tous les nombres premiers sont impairs. Par contre, la réciproque est fausse~: tous les nombres impairs ne sont pas premiers~; par exemple 1 (voir ci-dessus) et 15 (divisible par 1; 3; 5 et 15) ne sont pas premiers.
     \end{itemize}
} % fin exemple
\bloc{cyan}{Remarque}{ % id=r47
     Il est utile de connaître par cœur la liste des nombres premiers inférieurs à 20 (ou plus ...):
     \begin{center}
          \textbf{2~; 3~; 5~; 7~; 11~; 13~; 17~; 19 }
     \end{center}
} % fin remarque
\cadre{rouge}{Théorème}{ % id=t48
     \textbf{Décomposition en produit de facteurs premiers}
     \par
     Tout nombre entier supérieur ou égal à 2 peut s'écrire sous la forme d'un produit de nombres premiers. Cette décomposition est \textbf{unique} (à l'ordre des facteurs près).
} % fin théorème
\bloc{cyan}{Remarque}{ % id=r48
     Ce résultat très important est également appelé \textbf{ \og Théorème fondamental de l'arithmétique \fg{} }
} % fin remarque
\bloc{orange}{Exemple}{ % id=e48
     \begin{itemize}
          \item
          $10 = 2 \times 5$
          \item
          $84 = 2 \times 2 \times 3 \times 7 = 2^2 \times 3 \times 7$
          \item
          $23 = 23$ (un seul facteur car 23 est premier~!)
     \end{itemize}
} % fin exemple
\cadre{vert}{Méthode}{ % id=p48
     Pour décomposer un nombre $ N $ en produit de facteurs premiers, on peut essayer de le diviser successivement par chaque nombre premier inférieur ou égal à $ \sqrt{ n } $ . Le méthode détaillée est décrite sur la fiche~: Décomposition en produit de facteurs premiers.
} % fin propriété@
\begin{h2}3 - PGCD\end{h2}
\cadre{bleu}{Définition}{% id="d50"
     Le \textbf{PGCD} de deux entiers naturels non nuls $a$ et $b$ est le plus grand diviseur commun à $a$ et à $b$, c'est à dire le plus grand entier naturel qui divise à la fois $a$ et $b$.
}
\bloc{orange}{Exemple}{% id="e50"
     Soit à déterminer le PGCD de $600$ et $315$.
     \par
     Les diviseurs de $600$ sont~:
     \par
     $1; 2; 3; 4; 5; 6; 8; 10; 12; 15; 20; 24; 25; 30; 40; 50; 60; 75; 100; 120; 150; 200; 300; 600$
     \par
     Les diviseurs de $315$ sont~:
     \par
     $1; 3; 5; 7; 9; 15; 21; 35; 45; 63; 105; 315$
     \par
     Le plus grand diviseur commun est donc $15$ (le plus grand nombre figurant à la fois dans les deux listes).
     \par
     $PGCD\left(600~; 315\right)=15$.
     \par
     Il existe plusieurs méthodes permettant de trouver le PGCD de deux nombres de façon plus rapide, sans avoir besoin de faire la liste de tous les diviseurs.
     \medskip
     En classe de Troisième, il faut connaître la méthode utilisant la décomposition en facteurs premiers (voir ci-dessous). D'autres méthodes sont proposées en compléments~: \mcLien{https://www.maths-cours.fr/supplement/deux-methodes-de-calcul-du-pgcd/}{Calcul du PGCD par soustractions successives et algorithme d'Euclide}.
     \par
     Par ailleurs, de nombreuses calculatrices (de niveau collège ou lycée) possède une touche permettant de calculer le PGCD de deux entiers naturels.
}
\bloc{orange}{Exemples}{ % id=e59
     \textbf{Calcul du PGCD à l'aide de décomposition en produit de facteurs premiers}
     \par
     \begin{itemize}
          \item
          Exemple 1~: Calcul du PGCD de 45 et de 150~:
          \par
          Les décompositions en facteurs premiers de 45 et de 150 sont~:
          \par
          $45 = \color{red}{3 }\color{black} \times 3 \times \color{red}{5} \color{black}= 3^2 \times 5$
          \par
          $ 150 = 2 \times \color{red}{3}\color{black} \times \color{red}{5}\color{black} \times 5 = 2 \times 3 \times 5^2 $
          \par
          $3$ et $5$ sont les facteurs premiers figurant dans les deux décompositions donc le PGCD de $45$ et de $150$ est $ 3 \times 5 = 15. $
          \item
          Exemple 2~: Calcul du PGCD de 108 et de 144~:
          \par
          Les décompositions en produit de facteurs premiers de 108 et de 144 sont~:
          \par
          $108 = \color{red}{2 \times 2}\color{black} \times \color{red}{ 3 \times 3}\color{black} \times 3 = 2^2 \times 3^3$
          \par
          $ 144 = \color{red}{2 \times 2}\color{black} \times 2 \times 2 \times \color{red}{3 \times 3}\color{black} = 2^4 \times 3^2 $
          \par
          Le facteur $2$ est présent (au moins) deux fois dans chacune des décompositions ainsi que le facteur $ 3 $~; donc le PGCD de $108$ et de $ 144 $ est $ 2 \times 2 \times 3 \times 3 = 36. $
     \end{itemize}
} % fin exemple
\cadre{bleu}{Définition}{% id="d90"
     Une fraction est \textbf{irréductible} si son numérateur et son dénominateur n'ont aucun diviseur commun mis à part $1$, c'est à dire si le PGCD du numérateur et du dénominateur est égal à 1.
}
\bloc{orange}{Exemples}{% id="e90"
     \begin{itemize}
          \item $\frac{5}{6}$ est une fraction irréductible car $PGCD\left(5~; 6\right)=1$.
          \item $\frac{121}{99}$ n'est pas une fraction irréductible car $PGCD\left(121~; 99\right)=11$.\\ La fraction se simplifie donc par $11$~:
          \par
          $\frac{121}{99}=\frac{11\times 11}{9\times 11}=\frac{11}{9}$
     \end{itemize}
}

\end{document}