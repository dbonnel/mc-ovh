\documentclass[a4paper]{article}

%================================================================================================================================
%
% Packages
%
%================================================================================================================================

\usepackage[T1]{fontenc} 	% pour caractères accentués
\usepackage[utf8]{inputenc}  % encodage utf8
\usepackage[french]{babel}	% langue : français
\usepackage{fourier}			% caractères plus lisibles
\usepackage[dvipsnames]{xcolor} % couleurs
\usepackage{fancyhdr}		% réglage header footer
\usepackage{needspace}		% empêcher sauts de page mal placés
\usepackage{graphicx}		% pour inclure des graphiques
\usepackage{enumitem,cprotect}		% personnalise les listes d'items (nécessaire pour ol, al ...)
\usepackage{hyperref}		% Liens hypertexte
\usepackage{pstricks,pst-all,pst-node,pstricks-add,pst-math,pst-plot,pst-tree,pst-eucl} % pstricks
\usepackage[a4paper,includeheadfoot,top=2cm,left=3cm, bottom=2cm,right=3cm]{geometry} % marges etc.
\usepackage{comment}			% commentaires multilignes
\usepackage{amsmath,environ} % maths (matrices, etc.)
\usepackage{amssymb,makeidx}
\usepackage{bm}				% bold maths
\usepackage{tabularx}		% tableaux
\usepackage{colortbl}		% tableaux en couleur
\usepackage{fontawesome}		% Fontawesome
\usepackage{environ}			% environment with command
\usepackage{fp}				% calculs pour ps-tricks
\usepackage{multido}			% pour ps tricks
\usepackage[np]{numprint}	% formattage nombre
\usepackage{tikz,tkz-tab} 			% package principal TikZ
\usepackage{pgfplots}   % axes
\usepackage{mathrsfs}    % cursives
\usepackage{calc}			% calcul taille boites
\usepackage[scaled=0.875]{helvet} % font sans serif
\usepackage{svg} % svg
\usepackage{scrextend} % local margin
\usepackage{scratch} %scratch
\usepackage{multicol} % colonnes
%\usepackage{infix-RPN,pst-func} % formule en notation polanaise inversée
\usepackage{listings}

%================================================================================================================================
%
% Réglages de base
%
%================================================================================================================================

\lstset{
language=Python,   % R code
literate=
{á}{{\'a}}1
{à}{{\`a}}1
{ã}{{\~a}}1
{é}{{\'e}}1
{è}{{\`e}}1
{ê}{{\^e}}1
{í}{{\'i}}1
{ó}{{\'o}}1
{õ}{{\~o}}1
{ú}{{\'u}}1
{ü}{{\"u}}1
{ç}{{\c{c}}}1
{~}{{ }}1
}


\definecolor{codegreen}{rgb}{0,0.6,0}
\definecolor{codegray}{rgb}{0.5,0.5,0.5}
\definecolor{codepurple}{rgb}{0.58,0,0.82}
\definecolor{backcolour}{rgb}{0.95,0.95,0.92}

\lstdefinestyle{mystyle}{
    backgroundcolor=\color{backcolour},   
    commentstyle=\color{codegreen},
    keywordstyle=\color{magenta},
    numberstyle=\tiny\color{codegray},
    stringstyle=\color{codepurple},
    basicstyle=\ttfamily\footnotesize,
    breakatwhitespace=false,         
    breaklines=true,                 
    captionpos=b,                    
    keepspaces=true,                 
    numbers=left,                    
xleftmargin=2em,
framexleftmargin=2em,            
    showspaces=false,                
    showstringspaces=false,
    showtabs=false,                  
    tabsize=2,
    upquote=true
}

\lstset{style=mystyle}


\lstset{style=mystyle}
\newcommand{\imgdir}{C:/laragon/www/newmc/assets/imgsvg/}
\newcommand{\imgsvgdir}{C:/laragon/www/newmc/assets/imgsvg/}

\definecolor{mcgris}{RGB}{220, 220, 220}% ancien~; pour compatibilité
\definecolor{mcbleu}{RGB}{52, 152, 219}
\definecolor{mcvert}{RGB}{125, 194, 70}
\definecolor{mcmauve}{RGB}{154, 0, 215}
\definecolor{mcorange}{RGB}{255, 96, 0}
\definecolor{mcturquoise}{RGB}{0, 153, 153}
\definecolor{mcrouge}{RGB}{255, 0, 0}
\definecolor{mclightvert}{RGB}{205, 234, 190}

\definecolor{gris}{RGB}{220, 220, 220}
\definecolor{bleu}{RGB}{52, 152, 219}
\definecolor{vert}{RGB}{125, 194, 70}
\definecolor{mauve}{RGB}{154, 0, 215}
\definecolor{orange}{RGB}{255, 96, 0}
\definecolor{turquoise}{RGB}{0, 153, 153}
\definecolor{rouge}{RGB}{255, 0, 0}
\definecolor{lightvert}{RGB}{205, 234, 190}
\setitemize[0]{label=\color{lightvert}  $\bullet$}

\pagestyle{fancy}
\renewcommand{\headrulewidth}{0.2pt}
\fancyhead[L]{maths-cours.fr}
\fancyhead[R]{\thepage}
\renewcommand{\footrulewidth}{0.2pt}
\fancyfoot[C]{}

\newcolumntype{C}{>{\centering\arraybackslash}X}
\newcolumntype{s}{>{\hsize=.35\hsize\arraybackslash}X}

\setlength{\parindent}{0pt}		 
\setlength{\parskip}{3mm}
\setlength{\headheight}{1cm}

\def\ebook{ebook}
\def\book{book}
\def\web{web}
\def\type{web}

\newcommand{\vect}[1]{\overrightarrow{\,\mathstrut#1\,}}

\def\Oij{$\left(\text{O}~;~\vect{\imath},~\vect{\jmath}\right)$}
\def\Oijk{$\left(\text{O}~;~\vect{\imath},~\vect{\jmath},~\vect{k}\right)$}
\def\Ouv{$\left(\text{O}~;~\vect{u},~\vect{v}\right)$}

\hypersetup{breaklinks=true, colorlinks = true, linkcolor = OliveGreen, urlcolor = OliveGreen, citecolor = OliveGreen, pdfauthor={Didier BONNEL - https://www.maths-cours.fr} } % supprime les bordures autour des liens

\renewcommand{\arg}[0]{\text{arg}}

\everymath{\displaystyle}

%================================================================================================================================
%
% Macros - Commandes
%
%================================================================================================================================

\newcommand\meta[2]{    			% Utilisé pour créer le post HTML.
	\def\titre{titre}
	\def\url{url}
	\def\arg{#1}
	\ifx\titre\arg
		\newcommand\maintitle{#2}
		\fancyhead[L]{#2}
		{\Large\sffamily \MakeUppercase{#2}}
		\vspace{1mm}\textcolor{mcvert}{\hrule}
	\fi 
	\ifx\url\arg
		\fancyfoot[L]{\href{https://www.maths-cours.fr#2}{\black \footnotesize{https://www.maths-cours.fr#2}}}
	\fi 
}


\newcommand\TitreC[1]{    		% Titre centré
     \needspace{3\baselineskip}
     \begin{center}\textbf{#1}\end{center}
}

\newcommand\newpar{    		% paragraphe
     \par
}

\newcommand\nosp {    		% commande vide (pas d'espace)
}
\newcommand{\id}[1]{} %ignore

\newcommand\boite[2]{				% Boite simple sans titre
	\vspace{5mm}
	\setlength{\fboxrule}{0.2mm}
	\setlength{\fboxsep}{5mm}	
	\fcolorbox{#1}{#1!3}{\makebox[\linewidth-2\fboxrule-2\fboxsep]{
  		\begin{minipage}[t]{\linewidth-2\fboxrule-4\fboxsep}\setlength{\parskip}{3mm}
  			 #2
  		\end{minipage}
	}}
	\vspace{5mm}
}

\newcommand\CBox[4]{				% Boites
	\vspace{5mm}
	\setlength{\fboxrule}{0.2mm}
	\setlength{\fboxsep}{5mm}
	
	\fcolorbox{#1}{#1!3}{\makebox[\linewidth-2\fboxrule-2\fboxsep]{
		\begin{minipage}[t]{1cm}\setlength{\parskip}{3mm}
	  		\textcolor{#1}{\LARGE{#2}}    
 	 	\end{minipage}  
  		\begin{minipage}[t]{\linewidth-2\fboxrule-4\fboxsep}\setlength{\parskip}{3mm}
			\raisebox{1.2mm}{\normalsize\sffamily{\textcolor{#1}{#3}}}						
  			 #4
  		\end{minipage}
	}}
	\vspace{5mm}
}

\newcommand\cadre[3]{				% Boites convertible html
	\par
	\vspace{2mm}
	\setlength{\fboxrule}{0.1mm}
	\setlength{\fboxsep}{5mm}
	\fcolorbox{#1}{white}{\makebox[\linewidth-2\fboxrule-2\fboxsep]{
  		\begin{minipage}[t]{\linewidth-2\fboxrule-4\fboxsep}\setlength{\parskip}{3mm}
			\raisebox{-2.5mm}{\sffamily \small{\textcolor{#1}{\MakeUppercase{#2}}}}		
			\par		
  			 #3
 	 		\end{minipage}
	}}
		\vspace{2mm}
	\par
}

\newcommand\bloc[3]{				% Boites convertible html sans bordure
     \needspace{2\baselineskip}
     {\sffamily \small{\textcolor{#1}{\MakeUppercase{#2}}}}    
		\par		
  			 #3
		\par
}

\newcommand\CHelp[1]{
     \CBox{Plum}{\faInfoCircle}{À RETENIR}{#1}
}

\newcommand\CUp[1]{
     \CBox{NavyBlue}{\faThumbsOUp}{EN PRATIQUE}{#1}
}

\newcommand\CInfo[1]{
     \CBox{Sepia}{\faArrowCircleRight}{REMARQUE}{#1}
}

\newcommand\CRedac[1]{
     \CBox{PineGreen}{\faEdit}{BIEN R\'EDIGER}{#1}
}

\newcommand\CError[1]{
     \CBox{Red}{\faExclamationTriangle}{ATTENTION}{#1}
}

\newcommand\TitreExo[2]{
\needspace{4\baselineskip}
 {\sffamily\large EXERCICE #1\ (\emph{#2 points})}
\vspace{5mm}
}

\newcommand\img[2]{
          \includegraphics[width=#2\paperwidth]{\imgdir#1}
}

\newcommand\imgsvg[2]{
       \begin{center}   \includegraphics[width=#2\paperwidth]{\imgsvgdir#1} \end{center}
}


\newcommand\Lien[2]{
     \href{#1}{#2 \tiny \faExternalLink}
}
\newcommand\mcLien[2]{
     \href{https~://www.maths-cours.fr/#1}{#2 \tiny \faExternalLink}
}

\newcommand{\euro}{\eurologo{}}

%================================================================================================================================
%
% Macros - Environement
%
%================================================================================================================================

\newenvironment{tex}{ %
}
{%
}

\newenvironment{indente}{ %
	\setlength\parindent{10mm}
}

{
	\setlength\parindent{0mm}
}

\newenvironment{corrige}{%
     \needspace{3\baselineskip}
     \medskip
     \textbf{\textsc{Corrigé}}
     \medskip
}
{
}

\newenvironment{extern}{%
     \begin{center}
     }
     {
     \end{center}
}

\NewEnviron{code}{%
	\par
     \boite{gray}{\texttt{%
     \BODY
     }}
     \par
}

\newenvironment{vbloc}{% boite sans cadre empeche saut de page
     \begin{minipage}[t]{\linewidth}
     }
     {
     \end{minipage}
}
\NewEnviron{h2}{%
    \needspace{3\baselineskip}
    \vspace{0.6cm}
	\noindent \MakeUppercase{\sffamily \large \BODY}
	\vspace{1mm}\textcolor{mcgris}{\hrule}\vspace{0.4cm}
	\par
}{}

\NewEnviron{h3}{%
    \needspace{3\baselineskip}
	\vspace{5mm}
	\textsc{\BODY}
	\par
}

\NewEnviron{margeneg}{ %
\begin{addmargin}[-1cm]{0cm}
\BODY
\end{addmargin}
}

\NewEnviron{html}{%
}

\begin{document}
\meta{url}{/cours/trigonometrie/}
\meta{pid}{204}
\meta{titre}{Trigonométrie}
\meta{type}{cours}
\begin{h2}1. Angle dans le cercle trigonométrique\end{h2}
Dans tout le chapitre, le plan $P$ est muni d'un repère orthonormé $\left(O ; I , J\right)$
\cadre{bleu}{Définition}{%id="d10"
     On appelle \textbf{cercle trigonométrique} le cercle de centre $O$ et de rayon $1$ orienté dans le sens inverse des aiguilles d'une montre (aussi appelé \textit{« sens direct »} ou \textit{« sens trigonométrique»}).
}
\begin{center}
     \begin{extern}%width="250" alt="cercle trigonométrique"
          \resizebox{6cm}{!}{
               \psset{xunit=3.0cm,yunit=3.0cm,algebraic=true,dimen=middle,dotstyle=o,dotsize=5pt 0,linewidth=0.8pt,arrowsize=3pt 2,arrowinset=0.25}
               \begin{pspicture*}(-1.2,-1.2)(1.2,1.2)
                    \psaxes[linewidth=0.75pt,xAxis=true,yAxis=true,Dx=1.,Dy=1.,ticksize=-2pt 0,subticks=1]{->}(0,0)(-1.2,-1.2)(1.2,1.2)
                    \pscircle[linewidth=0.8pt,linecolor=blue](0.,0.){3.} %cercle trigo
                    \parametricplot[linewidth=0.8pt,arrows=->]{0.8}{1.3}{1.15*cos(t)|1.15*sin(t)}% sens trigo
                    \rput[tl](0.6,1.1){+}
                    \psline[linewidth=0.8pt]{->}(0.,0.)(1.,0.) %vecteurs unités
                    \psline[linewidth=0.8pt]{->}(0.,0.)(0,1)
                    \psdots[dotsize=2pt 0,dotstyle=*](0.,0.)
                    \rput[tr](-0.05,-0.05){$O$}
                    \psdots[dotsize=2pt 0,dotstyle=*](0,1)
                    \rput[bl](0.05,1.05){$J$}
                    \psdots[dotsize=2pt 0,dotstyle=*](1.,0.)
                    \rput[bl](1.05,0.05){$I$}
               \end{pspicture*}
          }
     \end{extern}
\end{center}
\bloc{cyan}{Mesure d'un angle en radians}{%
     Dans le plan $P$ muni d'un repère orthonormé $\left(O ; I , J\right)$, on trace le cercle trigonométrique et la droite d'équation $x=1$ qui est tangente à ce cercle.
     \begin{center}
          \begin{extern}%width="250" alt="cercle trigonométrique"
               \resizebox{6cm}{!}{
                    \psset{xunit=3.0cm,yunit=3.0cm,algebraic=true,dimen=middle,dotstyle=o,dotsize=5pt 0,linewidth=0.8pt,arrowsize=3pt 2,arrowinset=0.25}
                    \begin{pspicture*}(-1.2,-1.2)(1.2,1.2)
                         \psaxes[linewidth=0.75pt,,xAxis=true,yAxis=true,Dx=1.,Dy=1.,ticksize=-2pt 0,subticks=1]{->}(0,0)(-1.2,-1.2)(1.2,1.2)
                         \pscircle[linewidth=0.8pt,linecolor=blue](0.,0.){3.} %cercle trigo
                         \parametricplot[linewidth=0.8pt,arrows=->]{0.8}{1.3}{1.15*cos(t)|1.15*sin(t)}% sens trigo
                         \rput[tl](0.6,1.1){+}
                         \psline[linewidth=0.8pt]{->}(0.,0.)(1.,0.) %vecteurs unités
                         \psline[linewidth=0.8pt]{->}(0.,0.)(0,1)
                         \psdots[dotsize=2pt 0,dotstyle=*](0.,0.)
                         \rput[tr](-0.05,-0.05){$O$}
                         \psdots[dotsize=2pt 0,dotstyle=*](0,1)
                         \rput[bl](0.05,1.05){$J$}
                         \psdots[dotsize=2pt 0,dotstyle=*](1.,0.)
                         \rput[bl](1.05,0.05){$I$}
                         \psline[linewidth=0.8pt]{->}(1,-1.5)(1,1.5)
                    \end{pspicture*}
               }
          \end{extern}
     \end{center}
     Soit $N$ un point du cercle. Pour mesurer en radians l'angle $\widehat{ION}$ on mesure la longueur de l'arc $\left(IN\right)$.
     \begin{center}
          \begin{extern}%width="250" alt="longueur d'un arc en radians"
               \resizebox{6cm}{!}{
                    \newrgbcolor{dblue}{0. 0. 0.7}
                    \newrgbcolor{dvert}{0. 0.4 0.}
                    \newrgbcolor{dmauve}{0.5 0. 0.5}
                    \psset{xunit=3.0cm,yunit=3.0cm,algebraic=true,dimen=middle,dotstyle=o,dotsize=5pt 0,linewidth=0.8pt,arrowsize=3pt 2,arrowinset=0.25}
                    \begin{pspicture*}(-1.2,-1.2)(1.2,1.2)
                         \psaxes[linewidth=0.75pt,xAxis=true,yAxis=true,Dx=1.,Dy=1.,ticksize=-2pt 0,subticks=1]{->}(0,0)(-1.2,-1.2)(1.2,1.2)
                         \pscircle[linewidth=0.8pt,linecolor=blue](0.,0.){3.} %cercle trigo
                         \parametricplot[linewidth=0.8pt,arrows=->]{0.8}{1.3}{1.15*cos(t)|1.15*sin(t)}% sens trigo
                         \rput[tl](0.6,1.1){+}
                         \psline[linewidth=0.8pt]{->}(0.,0.)(1.,0.) %vecteurs unités
                         \psline[linewidth=0.8pt]{->}(0.,0.)(0,1)
                         \psdots[dotsize=2pt 0,dotstyle=*](0.,0.)
                         \rput[tr](-0.05,-0.05){$O$}
                         \psdots[dotsize=2pt 0,dotstyle=*](0,1)
                         \rput[bl](0.05,1.05){$J$}
                         \psdots[dotsize=2pt 0,dotstyle=*](1.,0.)
                         \rput[bl](1.05,0.05){$I$}
                         \psline[linewidth=0.8pt](1,-1.5)(1,1.5)
                         \pscustom[linewidth=0.8pt,linecolor=dmauve,fillcolor=dmauve,fillstyle=solid,opacity=0.1]					{ % color angle
                              \parametricplot{0.0}{1.92}{0.15*cos(t)|0.15*sin(t)}
                         \lineto(0.,0.)\closepath}
                         \psline[linewidth=0.8pt,linecolor=dmauve](0.,0.)(-0.342,0.94)%rayon
                         \parametricplot[linewidth=1.2pt,linecolor=red]{0.0}{1.92}{cos(t)|sin(t)}%arc angle
                         \psdots[dotsize=2pt 0,dotstyle=*,linecolor=dblue](-0.342,0.94)
                         \rput[br](-0.342,1){\dblue{$N$}}
                    \end{pspicture*}
               }
          \end{extern}
     \end{center}
     Pour cela on \textit{« enroule »} la tangente sur le cercle trigonométrique et on fait correspondre au point $N$ un point $M$ situé sur cette tangente.
     \begin{center}
          \begin{extern}%width="250" alt="mesure d'un angle"
               \resizebox{6cm}{!}{
                    \newrgbcolor{dblue}{0. 0. 0.7}
                    \newrgbcolor{dvert}{0. 0.4 0.}
                    \newrgbcolor{dmauve}{0.5 0. 0.5}
                    \psset{xunit=3.0cm,yunit=3.0cm,algebraic=true,dimen=middle,dotstyle=o,dotsize=5pt 0,linewidth=0.8pt,arrowsize=3pt 2,arrowinset=0.25}
                    \begin{pspicture*}(-1.2,-1.2)(1.2,2.2)
                         \psaxes[linewidth=0.75pt,xAxis=true,yAxis=true,Dx=1.,Dy=1.,ticksize=-2pt 0,subticks=1]{->}(0,0)(-1.2,-1.2)(1.2,2.2)
                         \pscircle[linewidth=0.8pt,linecolor=blue](0.,0.){3.} %cercle trigo
                         \parametricplot[linewidth=0.8pt,arrows=->]{0.8}{1.3}{1.15*cos(t)|1.15*sin(t)}% sens trigo
                         \rput[tl](0.6,1.1){+}
                         \psline[linewidth=0.8pt]{->}(0.,0.)(1.,0.) %vecteurs unités
                         \psline[linewidth=0.8pt]{->}(0.,0.)(0,1)
                         \psdots[dotsize=2pt 0,dotstyle=*](0.,0.)
                         \rput[tr](-0.05,-0.05){$O$}
                         \psdots[dotsize=2pt 0,dotstyle=*](0,1)
                         \rput[bl](0.05,1.05){$J$}
                         \psdots[dotsize=2pt 0,dotstyle=*](1.,0.)
                         \rput[bl](1.05,0.05){$I$}
                         \psline[linewidth=0.8pt](1,-1.5)(1,2.5)
                         \pscustom[linewidth=0.8pt,linecolor=dmauve,fillcolor=dmauve,fillstyle=solid,opacity=0.1]					{ % color angle
                              \parametricplot{0.0}{1.92}{0.15*cos(t)|0.15*sin(t)}
                         \lineto(0.,0.)\closepath}
                         \psline[linewidth=0.8pt,linecolor=dmauve](0.,0.)(-0.342,0.94)%rayon
                         \parametricplot[linewidth=1.2pt,linecolor=red]{0.0}{1.92}{cos(t)|sin(t)}%arc angle
                         \psdots[dotsize=2pt 0,dotstyle=*,linecolor=dblue](-0.342,0.94)
                         \rput[br](-0.342,1){\dblue{$N$}}
                         \psdots[dotsize=2pt 0,dotstyle=*,linecolor=dblue](1.0,1.92)
                         \rput[bl](1.03,1.92){\dblue{$M$}}
                         \psline[linewidth=1pt,linecolor=red](1.,0.)(1.,1.92) % segment IM
                         \psellipticarc[linewidth=0.8pt,linecolor=dvert,arrows=<->](1.373,0.)(1.956,1.956) {101.3}{151} % fleche MN
                    \end{pspicture*}
               }
          \end{extern}
     \end{center}
     L'ordonnée de $M$ est une \textit{mesure en radians} de l'angle $\widehat{ION}$ (sur la figure ci-dessus cette mesure vaut environ 1,9 radians).
     \par
     Cette mesure n'est pas unique.En effet, si l'on poursuit « l'enroulement » de la droite sur le cercle trigonométrique, on voit que plusieurs points de cette droite vont venir se positionner sur le point $N$.
     \par
     Il en est de même si l'on « enroule » la droite dans l'autre sens ; dans ce cas on obtiendra des mesures négatives de l'angle.
     \begin{center}
          \begin{extern}%width="320" alt="mesures négatives d'un angle"
               \resizebox{6cm}{!}{
                    \newrgbcolor{dblue}{0. 0. 0.7}
                    \newrgbcolor{dvert}{0. 0.4 0.}
                    \newrgbcolor{dmauve}{0.5 0. 0.5}
                    \psset{xunit=3.0cm,yunit=3.0cm,algebraic=true,dimen=middle,dotstyle=o,dotsize=5pt 0,linewidth=0.8pt,arrowsize=3pt 2,arrowinset=0.25}
                    \begin{pspicture*}(-2.6,-4.6)(1.2,1.2)
                         \psaxes[linewidth=0.75pt,xAxis=true,yAxis=true,Dx=1.,Dy=1.,ticksize=-2pt 0,subticks=1]{->}(0,0)(-2.6,-4.6)(1.2,1.2)
                         \pscircle[linewidth=0.8pt,linecolor=blue](0.,0.){3.} %cercle trigo
                         \parametricplot[linewidth=0.8pt,arrows=->]{0.8}{1.3}{1.15*cos(t)|1.15*sin(t)}% sens trigo
                         \rput[tl](0.6,1.1){+}
                         \psline[linewidth=0.8pt]{->}(0.,0.)(1.,0.) %vecteurs unités
                         \psline[linewidth=0.8pt]{->}(0.,0.)(0,1)
                         \psdots[dotsize=2pt 0,dotstyle=*](0.,0.)
                         \rput[tr](-0.05,-0.05){$O$}
                         \psdots[dotsize=2pt 0,dotstyle=*](0,1)
                         \rput[bl](0.05,1.05){$J$}
                         \psdots[dotsize=2pt 0,dotstyle=*](1.,0.)
                         \rput[bl](1.05,0.05){$I$}
                         \psline[linewidth=0.8pt](1,-4.9)(1,1.5)
                         \pscustom[linewidth=0.8pt,linecolor=dmauve,fillcolor=dmauve,fillstyle=solid,opacity=0.1]
                         { % color angle
                              \parametricplot{0.0}{1.92}{0.15*cos(t)|0.15*sin(t)}
                         \lineto(0.,0.)\closepath}
                         \psline[linewidth=0.8pt,linecolor=dmauve](0.,0.)(-0.342,0.94)%rayon
                         \parametricplot[linewidth=1.2pt,linecolor=red]{1.92}{6.283}{cos(t)|sin(t)}%arc angle
                         \psdots[dotsize=2pt 0,dotstyle=*,linecolor=dblue](-0.342,0.94)
                         \rput[br](-0.342,1){\dblue{$N$}}
                         \psdots[dotsize=2pt 0,dotstyle=*,linecolor=dblue](1.0,-4.36)
                         \rput[bl](1.03,-4.36){\dblue{$M$}}
                         \psline[linewidth=1pt,linecolor=red](1.,0.)(1.,-4.36) % segment IM
                         \psellipticarc[linewidth=0.8pt,linecolor=dvert,arrows=<->](0.329,-1.71)(2.73,2.73) {104}{284} % fleche MN
                    \end{pspicture*}
               }
          \end{extern}
\end{center}   }
\cadre{vert}{Propriété}{%
     Chaque angle possède une infinité de mesures (en radians) qui diffèrent d'un multiple de $2\pi $.
}
\bloc{cyan}{Remarques}{%
     \begin{itemize}
          \item Cela signifie que si $x$ est une mesure d'un angle, les autres mesures sont $x+2\pi , x+4\pi , $ etc. et $x-2\pi , x-4\pi , $ etc.
          \item Ces différentes mesures s'écrivent donc $x+2k\pi $ avec $k \in \mathbb{Z}$
     \end{itemize}
}
\bloc{orange}{Mesures d'angles à retenir}{%
     \begin{center}
          \begin{extern}%width="400" alt="Mesures d'angles remarquables"
               \newrgbcolor{dblue}{0. 0. 0.7}
               \newrgbcolor{dvert}{0. 0.4 0.}
               \newrgbcolor{dmauve}{0.5 0. 0.5}
               \psset{xunit=5.0cm,yunit=5.0cm,algebraic=true,dimen=middle,dotstyle=o,dotsize=5pt 0,linewidth=0.8pt,arrowsize=3pt 2,arrowinset=0.25}
               \resizebox{8cm}{!}{
                    \begin{pspicture*}(-1.2,-1.2)(1.2,1.2)
                         \psaxes[linewidth=0.75pt,labelFontSize=\scriptstyle,xAxis=true,yAxis=true,Dx=10.,Dy=10.,ticksize=-2pt 0,subticks=1]{->}(0,0)(-1.2,-1.2)(1.2,1.2)
                         \pscircle[linewidth=0.8pt](0.,0.){5.} %cercle trigo
                         \psline[linewidth=0.8pt]{->}(0.,0.)(1.,0.) %vecteurs unités
                         \psline[linewidth=0.8pt]{->}(0.,0.)(0,1)
                         %\rput[tl](0.4,0.1){$\vec{i}$}
                         %\rput[tl](-0.06,0.5){$\vec{j}$}
                         \psdots[dotsize=2pt 0,dotstyle=*](0.,0.)
                         %\rput[bl](-0.09,-0.09){$O$}
                         \psline[linewidth=0.8pt,linecolor=dvert](-0.707,-0.707)(0.707,0.707)
                         \psline[linewidth=0.8pt,linecolor=dvert](-0.707,0.707)(0.707,-0.707)
                         \psline[linewidth=0.8pt,linecolor=red](-0.866,-0.5)(0.866,0.5)
                         \psline[linewidth=0.8pt,linecolor=red](0.866,-0.5)(-0.866,0.5)
                         \rput(0.943,0.55){$\red{\dfrac{\pi}{6}}$}
                         \rput(-0.943,0.55){$\red{\dfrac{5\pi}{6}}$}
                         \rput(-0.973,-0.55){$\red{-\dfrac{5\pi}{6}}$}
                         \rput(0.943,-0.55){$\red{-\dfrac{\pi}{6}}$}
                         %
                         \rput(0.777,0.777){$\dvert{\dfrac{\pi}{4}}$}
                         \rput(-0.777,0.777){$\dvert{\dfrac{3\pi}{4}}$}
                         \rput(-0.807,-0.777){$\dvert{-\dfrac{3\pi}{4}}$}
                         \rput(0.777,-0.777){$\dvert{-\dfrac{\pi}{4}}$}
                         %
                         \psline[linewidth=0.8pt,linecolor=dblue](-0.5,-0.866)(0.5,0.866)
                         \psline[linewidth=0.8pt,linecolor=dblue](-0.5,0.866)(0.5,-0.866)
                         \rput(0.55,0.943){$\dblue{\dfrac{\pi}{3}}$}
                         \rput(-0.55,0.943){$\dblue{\dfrac{2\pi}{3}}$}
                         \rput(-0.58,-0.943){$\dblue{-\dfrac{2\pi}{3}}$}
                         \rput(0.55,-0.943){$\dblue{-\dfrac{\pi}{3}}$}
                         %
                         \rput(1.06,0.06){$0$}
                         \rput(0.06,1.1){$\dfrac{\pi}{2}$}
                         \rput(0.06,-1.1){$-\dfrac{\pi}{2}$}
                         \rput(-1.06,0.06){$\pi$}
                    \end{pspicture*}
               }
          \end{extern}
     \end{center}
     \begin{center}
          \textit{ Mesures d'angles remarquables}
     \end{center}
}
\begin{h2}2. Sinus et cosinus\end{h2}
\cadre{bleu}{Définition}{%
     Soit $N$ un point du cercle trigonométrique. On note $x$ une mesure de l'angle $\widehat{ION}$.
     \par
     On appelle \textbf{cosinus} de $x$, noté\textbf{ $\cos x$} l'abscisse du point $N$.
     \par
     On appelle \textbf{sinus} de $x$, noté\textbf{ $\sin x$} l'ordonnée du point $N$
}
\begin{center}
     \begin{extern}%width="330" alt="sinus et cosinus d'un angle"
          \resizebox{6cm}{!}{
               \newrgbcolor{dblue}{0. 0. 0.7}
               \newrgbcolor{dvert}{0. 0.4 0.}
               \newrgbcolor{dmauve}{0.5 0. 0.5}
               \psset{xunit=5.0cm,yunit=5.0cm,algebraic=true,dimen=middle,dotstyle=o,dotsize=5pt 0,linewidth=0.8pt,arrowsize=3pt 2,arrowinset=0.25}
               \begin{pspicture*}(-1.2,-1.2)(1.2,1.2)
                    \psaxes[linewidth=0.75pt,labelFontSize=\scriptstyle,xAxis=true,yAxis=true,Dx=10.,Dy=10.,ticksize=-2pt 0,subticks=1]{->}(0,0)(-1.2,-1.2)(1.2,1.2)
                    \pscircle[linewidth=0.8pt](0.,0.){5.} %cercle trigo
                    \parametricplot[linewidth=1.2pt,linecolor=red]{0.0}{0.698}{cos(t)|sin(t)}%arc angle
                    \pscustom[linewidth=0.8pt,linecolor=dmauve,fillcolor=dmauve,fillstyle=solid,opacity=0.1]{ % color angle
                         \parametricplot{0.0}{0.698}{0.15*cos(t)|0.15*sin(t)}
                    \lineto(0.,0.)\closepath}
                    \psellipticarc[linewidth=0.8pt,linecolor=dmauve](0.,0.)(0.15,0.15){0.}{40} % fleche angle
                    \psline[linewidth=0.8pt,linecolor=dmauve](0.,0.)(0.766,0.643)%rayon
                    \psline[linewidth=0.8pt]{->}(0.,0.)(1.,0.) %vecteurs unités
                    \psline[linewidth=0.8pt]{->}(0.,0.)(0,1)
                    %\rput[tl](0.4,0.1){$\vec{i}$}
                    %\rput[tl](-0.06,0.5){$\vec{j}$}
                    \psdots[dotsize=2pt 0,dotstyle=*](0.,0.)
                    \rput[bl](-0.09,-0.09){$O$}
                    \psdots[dotsize=2pt 0,dotstyle=*,linecolor=dblue](1.,0.)
                    \rput[bl](1.02,0.02){\dblue{$I$}}
                    \psdots[dotsize=2pt 0,dotstyle=*,linecolor=dblue](0.766,0.643)
                    \rput[bl](0.78,0.66){\dvert{$N$}}
                    \psdots[dotsize=2pt 0,dotstyle=*,linecolor=dblue](0,1)
                    \rput[bl](0.02,1.03){\dblue{$J$}}
                    \rput[bl](0.19,0.05){\dmauve{$x$}}
                    \psline[linewidth=0.8pt,linecolor=dvert](0.,0.643)(0.766,0.643)
                    \psline[linewidth=0.8pt,linecolor=dvert](0.766,0.)(0.766,0.643)
                    \rput(0.766,-0.05){\dvert{$\cos x$}}
                    \rput(-0.10,0.643){\dvert{$\sin x$}}
               \end{pspicture*}
          }
     \end{extern}
\end{center}
\bloc{cyan}{Remarque}{%
     Ces notions généralisent celles vues au collège.
     \par
     En effet si l'angle $\widehat{ION}$ est aigu :
     \begin{center}
          \begin{extern}%width="330" alt="sinus et cosinus d'un angle aigu"
               \resizebox{6cm}{!}{
                    \newrgbcolor{dblue}{0. 0. 0.7}
                    \newrgbcolor{dvert}{0. 0.4 0.}
                    \newrgbcolor{dmauve}{0.5 0. 0.5}
                    \psset{xunit=5.0cm,yunit=5.0cm,algebraic=true,dimen=middle,dotstyle=o,dotsize=5pt 0,linewidth=0.8pt,arrowsize=3pt 2,arrowinset=0.25}
                    \begin{pspicture*}(-1.2,-1.2)(1.2,1.2)
                         \psaxes[linewidth=0.75pt,labelFontSize=\scriptstyle,xAxis=true,yAxis=true,Dx=10.,Dy=10.,ticksize=-2pt 0,subticks=1]{->}(0,0)(-1.2,-1.2)(1.2,1.2)
                         \pscircle[linewidth=0.8pt](0.,0.){5.} %cercle trigo
                         \pscustom[linewidth=0.8pt,linecolor=dmauve,fillcolor=dmauve,fillstyle=solid,opacity=0.1]{ % color angle
                              \parametricplot{0.0}{0.698}{0.15*cos(t)|0.15*sin(t)}
                         \lineto(0.,0.)\closepath}
                         \psellipticarc[linewidth=0.8pt,linecolor=dmauve](0.,0.)(0.15,0.15){0.}{40} % fleche angle
                         \psline[linewidth=0.8pt,linecolor=dmauve](0.,0.)(0.766,0.643)%rayon
                         \psline[linewidth=0.8pt]{->}(0.,0.)(1.,0.) %vecteurs unités
                         \psline[linewidth=0.8pt]{->}(0.,0.)(0,1)
                         %\rput[tl](0.4,0.1){$\vec{i}$}
                         %\rput[tl](-0.06,0.5){$\vec{j}$}
                         \psdots[dotsize=2pt 0,dotstyle=*](0.,0.)
                         \rput[bl](-0.09,-0.09){$O$}
                         \psdots[dotsize=2pt 0,dotstyle=*,linecolor=dblue](1.,0.)
                         \rput[bl](1.02,0.02){\dblue{$I$}}
                         \psdots[dotsize=2pt 0,dotstyle=*,linecolor=dblue](0.766,0.643)
                         \rput[bl](0.78,0.66){\dvert{$N$}}
                         \psdots[dotsize=2pt 0,dotstyle=*,linecolor=dblue](0,1)
                         \rput[bl](0.02,1.03){\dblue{$J$}}
                         \psline[linewidth=0.8pt,linecolor=dvert](0.766,0.)(0.766,0.643)
                         \psframe[linewidth=0.4pt,linecolor=dvert](0.766,0.)(0.71,0.056)
                         \rput(0.766,-0.05){\dvert{$A$}}
                    \end{pspicture*}
               }
          \end{extern}
     \end{center}
     Le triangle $OAN$ est rectangle en $A$ et $ON=1$ car $\left[ON\right]$ est un rayon du cercle; par conséquent :
     \par
     $\cos\left(\widehat{ION}\right)=\cos\left(\widehat{AON}\right)=\frac{OA}{ON}=\frac{OA}{1}=OA$
     \par
     $\sin\left(\widehat{ION}\right)=\sin\left(\widehat{AON}\right)=\frac{AN}{ON}=\frac{AN}{1}=AN$
}
\bloc{orange}{Valeurs de sinus et de cosinus à retenir}{%
     \begin{center}
          \begin{extern}%width="450" alt="Valeurs de sinus et de cosinus"
               \newrgbcolor{dblue}{0. 0. 0.7}
               \newrgbcolor{dvert}{0. 0.4 0.}
               \newrgbcolor{dmauve}{0.5 0. 0.5}
               \psset{xunit=5.0cm,yunit=5.0cm,algebraic=true,dimen=middle,dotstyle=o,dotsize=5pt 0,linewidth=0.8pt,arrowsize=3pt 2,arrowinset=0.25}
               \begin{pspicture*}(-1.2,-1.2)(1.2,1.2)
                    \psaxes[linewidth=0.75pt,labelFontSize=\scriptstyle,xAxis=true,yAxis=true,Dx=10.,Dy=10.,ticksize=-2pt 0,subticks=1]{->}(0,0)(-1.2,-1.2)(1.2,1.2)
                    \pscircle[linewidth=0.8pt](0.,0.){5.} %cercle trigo
                    \psline[linewidth=0.8pt]{->}(0.,0.)(1.,0.) %vecteurs unités
                    \psline[linewidth=0.8pt]{->}(0.,0.)(0,1)
                    %\rput[tl](0.4,0.1){$\vec{i}$}
                    %\rput[tl](-0.06,0.5){$\vec{j}$}
                    \psdots[dotsize=2pt 0,dotstyle=*](0.,0.)
                    %\rput[bl](-0.09,-0.09){$O$}
                    \psframe[linewidth=0.4pt,linecolor=dvert](-0.707,-0.707)(0.707,0.707)
                    \psline[linewidth=0.8pt,linecolor=dvert](-0.707,-0.707)(0.707,0.707)
                    \psline[linewidth=0.8pt,linecolor=dvert](-0.707,0.707)(0.707,-0.707)
                    \psframe[linewidth=0.4pt,linecolor=red](-0.866,-0.5)(0.866,0.5)
                    \psline[linewidth=0.8pt,linecolor=red](-0.866,-0.5)(0.866,0.5)
                    \psline[linewidth=0.8pt,linecolor=red](0.866,-0.5)(-0.866,0.5)
                    \rput(0.943,0.55){$\red{\dfrac{\pi}{6}}$}
                    \rput(-0.943,0.55){$\red{\dfrac{5\pi}{6}}$}
                    \rput(-0.973,-0.55){$\red{-\dfrac{5\pi}{6}}$}
                    \rput(0.943,-0.55){$\red{-\dfrac{\pi}{6}}$}
                    \rput(0.05,-0.554){\fontsize{7 pt}{7 pt}\selectfont $\red{ -\dfrac{1}{2}}$}
                    \rput(0.05,0.554){\fontsize{7 pt}{7 pt}\selectfont $\red{ \dfrac{1}{2}}$}
                    \rput(0.93,0.07){\fontsize{7 pt}{7 pt}\selectfont $\red{ \dfrac{\sqrt{3}}{2}}$}
                    \rput(-0.93,0.07){\fontsize{7 pt}{7 pt}\selectfont $\red{-\dfrac{\sqrt{3}}{2}}$}
                    %
                    \rput(0.777,0.777){$\dvert{\dfrac{\pi}{4}}$}
                    \rput(-0.777,0.777){$\dvert{\dfrac{3\pi}{4}}$}
                    \rput(-0.807,-0.777){$\dvert{-\dfrac{3\pi}{4}}$}
                    \rput(0.777,-0.777){$\dvert{-\dfrac{\pi}{4}}$}
                    \rput(0.06,-0.77){\fontsize{7 pt}{7 pt}\selectfont $\dvert{ -\dfrac{\sqrt{2}}{2}}$}
                    \rput(0.06,0.77){\fontsize{7 pt}{7 pt}\selectfont $\dvert{ \dfrac{\sqrt{2}}{2}}$}
                    \rput(0.764,0.07){\fontsize{7 pt}{7 pt}\selectfont $\dvert{ \dfrac{\sqrt{2}}{2}}$}
                    \rput(-0.764,0.07){\fontsize{7 pt}{7 pt}\selectfont $\dvert{-\dfrac{\sqrt{2}}{2}}$}
                    %
                    \psframe[linewidth=0.4pt,linecolor=dblue](-0.5,-0.866)(0.5,0.866)
                    \psline[linewidth=0.8pt,linecolor=dblue](-0.5,-0.866)(0.5,0.866)
                    \psline[linewidth=0.8pt,linecolor=dblue](-0.5,0.866)(0.5,-0.866)
                    \rput(0.55,0.943){$\dblue{\dfrac{\pi}{3}}$}
                    \rput(-0.55,0.943){$\dblue{\dfrac{2\pi}{3}}$}
                    \rput(-0.58,-0.943){$\dblue{-\dfrac{2\pi}{3}}$}
                    \rput(0.55,-0.943){$\dblue{-\dfrac{\pi}{3}}$}
                    \rput(0.06,-0.933){\fontsize{7 pt}{7 pt}\selectfont $\dblue{ -\dfrac{\sqrt{3}}{2}}$}
                    \rput(0.06,0.933){\fontsize{7 pt}{7 pt}\selectfont $\dblue{ \dfrac{\sqrt{3}}{2}}$}
                    \rput(0.538,0.07){\fontsize{7 pt}{7 pt}\selectfont $\dblue{ \dfrac{1}{2}}$}
                    \rput(-0.538,0.07){\fontsize{7 pt}{7 pt}\selectfont $\dblue{-\dfrac{1}{2}}$}
                    %
                    \rput(1.06,0.06){$0$}
                    \rput(0.06,1.1){$\dfrac{\pi}{2}$}
                    \rput(0.06,-1.1){$-\dfrac{\pi}{2}$}
                    \rput(-1.06,0.06){$\pi$}
               \end{pspicture*}
          \end{extern}
     \end{center}
     \begin{center}
          \begin{tabularx}{0.8\linewidth}{|*{10}{>{\centering \arraybackslash }X|}}%class="compact" width="600"
               \hline
               \textbf{$x$} & $0$ & $\frac{\pi }{6}$ & $\frac{\pi }{4}$ & $\frac{\pi }{3}$ & $\frac{\pi }{2}$ & $\frac{2\pi }{3}$ & $\frac{3\pi }{4}$ & $\frac{5\pi }{6}$ & $\pi $\\ \hline
               \textbf{$\cos x$} & $1$ & $\frac{\sqrt{3}}{2}$ & $\frac{\sqrt{2}}{2}$ & $\frac{1}{2}$ & $0$ & $-\frac{1}{2}$ & $-\frac{\sqrt{2}}{2}$ & $-\frac{\sqrt{3}}{2}$ & $-1$\\ \hline
               \textbf{$\sin x$} & $0$ & $\frac{1}{2}$ & $\frac{\sqrt{2}}{2}$ & $\frac{\sqrt{3}}{2}$ & $1$ & $\frac{\sqrt{3}}{2}$ & $\frac{\sqrt{2}}{2}$ & $\frac{1}{2}$ & $0$\\ \hline
          \end{tabularx}
     \end{center}
     \begin{center}
          \begin{tabularx}{0.8\linewidth}{|*{8}{>{\centering \arraybackslash }X|}}%class="compact" width="600"
               \hline
               \textbf{$x$} & $-\frac{\pi }{6}$ & $-\frac{\pi }{4}$ & $-\frac{\pi }{3}$ & $-\frac{\pi }{2}$ & $-\frac{2\pi }{3}$ & $-\frac{3\pi }{4}$ & $-\frac{5\pi }{6}$\\ \hline
               \textbf{$\cos x$} & $\frac{\sqrt{3}}{2}$ & $\frac{\sqrt{2}}{2}$ & $\frac{1}{2}$ & $0$ & $-\frac{1}{2}$ & $-\frac{\sqrt{2}}{2}$ & $-\frac{\sqrt{3}}{2}$\\ \hline
               \textbf{$\sin x$} & $-\frac{1}{2}$ & $-\frac{\sqrt{2}}{2}$ & $-\frac{\sqrt{3}}{2}$ & $-1$ & $-\frac{\sqrt{3}}{2}$ & $-\frac{\sqrt{2}}{2}$ & $-\frac{1}{2}$\\ \hline
          \end{tabularx}
     \end{center}
}
\cadre{vert}{Propriétés}{%
     Pour tout réel $x$ :
     \begin{itemize}
          \item $-1 \leqslant \cos x \leqslant 1$
          \item $-1 \leqslant \sin x \leqslant 1$
          \item $\left(\cos x\right)^2 + \left(\sin x\right)^2 = 1$
     \end{itemize}
}
\bloc{cyan}{Remarque}{%
     On écrit souvent $\cos^2 x$ et $\sin^2 x$ à la place de $\left(\cos x\right)^2$ et $\left(\sin x\right)^2$ afin de simplifier les notations.
     \par
     La dernière propriété s'écrit alors :
     \par
     $\cos^2 x + \sin^2 x = 1$
}

\end{document}