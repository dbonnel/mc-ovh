\documentclass[a4paper]{article}

%================================================================================================================================
%
% Packages
%
%================================================================================================================================

\usepackage[T1]{fontenc} 	% pour caractères accentués
\usepackage[utf8]{inputenc}  % encodage utf8
\usepackage[french]{babel}	% langue : français
\usepackage{fourier}			% caractères plus lisibles
\usepackage[dvipsnames]{xcolor} % couleurs
\usepackage{fancyhdr}		% réglage header footer
\usepackage{needspace}		% empêcher sauts de page mal placés
\usepackage{graphicx}		% pour inclure des graphiques
\usepackage{enumitem,cprotect}		% personnalise les listes d'items (nécessaire pour ol, al ...)
\usepackage{hyperref}		% Liens hypertexte
\usepackage{pstricks,pst-all,pst-node,pstricks-add,pst-math,pst-plot,pst-tree,pst-eucl} % pstricks
\usepackage[a4paper,includeheadfoot,top=2cm,left=3cm, bottom=2cm,right=3cm]{geometry} % marges etc.
\usepackage{comment}			% commentaires multilignes
\usepackage{amsmath,environ} % maths (matrices, etc.)
\usepackage{amssymb,makeidx}
\usepackage{bm}				% bold maths
\usepackage{tabularx}		% tableaux
\usepackage{colortbl}		% tableaux en couleur
\usepackage{fontawesome}		% Fontawesome
\usepackage{environ}			% environment with command
\usepackage{fp}				% calculs pour ps-tricks
\usepackage{multido}			% pour ps tricks
\usepackage[np]{numprint}	% formattage nombre
\usepackage{tikz,tkz-tab} 			% package principal TikZ
\usepackage{pgfplots}   % axes
\usepackage{mathrsfs}    % cursives
\usepackage{calc}			% calcul taille boites
\usepackage[scaled=0.875]{helvet} % font sans serif
\usepackage{svg} % svg
\usepackage{scrextend} % local margin
\usepackage{scratch} %scratch
\usepackage{multicol} % colonnes
%\usepackage{infix-RPN,pst-func} % formule en notation polanaise inversée
\usepackage{listings}

%================================================================================================================================
%
% Réglages de base
%
%================================================================================================================================

\lstset{
language=Python,   % R code
literate=
{á}{{\'a}}1
{à}{{\`a}}1
{ã}{{\~a}}1
{é}{{\'e}}1
{è}{{\`e}}1
{ê}{{\^e}}1
{í}{{\'i}}1
{ó}{{\'o}}1
{õ}{{\~o}}1
{ú}{{\'u}}1
{ü}{{\"u}}1
{ç}{{\c{c}}}1
{~}{{ }}1
}


\definecolor{codegreen}{rgb}{0,0.6,0}
\definecolor{codegray}{rgb}{0.5,0.5,0.5}
\definecolor{codepurple}{rgb}{0.58,0,0.82}
\definecolor{backcolour}{rgb}{0.95,0.95,0.92}

\lstdefinestyle{mystyle}{
    backgroundcolor=\color{backcolour},   
    commentstyle=\color{codegreen},
    keywordstyle=\color{magenta},
    numberstyle=\tiny\color{codegray},
    stringstyle=\color{codepurple},
    basicstyle=\ttfamily\footnotesize,
    breakatwhitespace=false,         
    breaklines=true,                 
    captionpos=b,                    
    keepspaces=true,                 
    numbers=left,                    
xleftmargin=2em,
framexleftmargin=2em,            
    showspaces=false,                
    showstringspaces=false,
    showtabs=false,                  
    tabsize=2,
    upquote=true
}

\lstset{style=mystyle}


\lstset{style=mystyle}
\newcommand{\imgdir}{C:/laragon/www/newmc/assets/imgsvg/}
\newcommand{\imgsvgdir}{C:/laragon/www/newmc/assets/imgsvg/}

\definecolor{mcgris}{RGB}{220, 220, 220}% ancien~; pour compatibilité
\definecolor{mcbleu}{RGB}{52, 152, 219}
\definecolor{mcvert}{RGB}{125, 194, 70}
\definecolor{mcmauve}{RGB}{154, 0, 215}
\definecolor{mcorange}{RGB}{255, 96, 0}
\definecolor{mcturquoise}{RGB}{0, 153, 153}
\definecolor{mcrouge}{RGB}{255, 0, 0}
\definecolor{mclightvert}{RGB}{205, 234, 190}

\definecolor{gris}{RGB}{220, 220, 220}
\definecolor{bleu}{RGB}{52, 152, 219}
\definecolor{vert}{RGB}{125, 194, 70}
\definecolor{mauve}{RGB}{154, 0, 215}
\definecolor{orange}{RGB}{255, 96, 0}
\definecolor{turquoise}{RGB}{0, 153, 153}
\definecolor{rouge}{RGB}{255, 0, 0}
\definecolor{lightvert}{RGB}{205, 234, 190}
\setitemize[0]{label=\color{lightvert}  $\bullet$}

\pagestyle{fancy}
\renewcommand{\headrulewidth}{0.2pt}
\fancyhead[L]{maths-cours.fr}
\fancyhead[R]{\thepage}
\renewcommand{\footrulewidth}{0.2pt}
\fancyfoot[C]{}

\newcolumntype{C}{>{\centering\arraybackslash}X}
\newcolumntype{s}{>{\hsize=.35\hsize\arraybackslash}X}

\setlength{\parindent}{0pt}		 
\setlength{\parskip}{3mm}
\setlength{\headheight}{1cm}

\def\ebook{ebook}
\def\book{book}
\def\web{web}
\def\type{web}

\newcommand{\vect}[1]{\overrightarrow{\,\mathstrut#1\,}}

\def\Oij{$\left(\text{O}~;~\vect{\imath},~\vect{\jmath}\right)$}
\def\Oijk{$\left(\text{O}~;~\vect{\imath},~\vect{\jmath},~\vect{k}\right)$}
\def\Ouv{$\left(\text{O}~;~\vect{u},~\vect{v}\right)$}

\hypersetup{breaklinks=true, colorlinks = true, linkcolor = OliveGreen, urlcolor = OliveGreen, citecolor = OliveGreen, pdfauthor={Didier BONNEL - https://www.maths-cours.fr} } % supprime les bordures autour des liens

\renewcommand{\arg}[0]{\text{arg}}

\everymath{\displaystyle}

%================================================================================================================================
%
% Macros - Commandes
%
%================================================================================================================================

\newcommand\meta[2]{    			% Utilisé pour créer le post HTML.
	\def\titre{titre}
	\def\url{url}
	\def\arg{#1}
	\ifx\titre\arg
		\newcommand\maintitle{#2}
		\fancyhead[L]{#2}
		{\Large\sffamily \MakeUppercase{#2}}
		\vspace{1mm}\textcolor{mcvert}{\hrule}
	\fi 
	\ifx\url\arg
		\fancyfoot[L]{\href{https://www.maths-cours.fr#2}{\black \footnotesize{https://www.maths-cours.fr#2}}}
	\fi 
}


\newcommand\TitreC[1]{    		% Titre centré
     \needspace{3\baselineskip}
     \begin{center}\textbf{#1}\end{center}
}

\newcommand\newpar{    		% paragraphe
     \par
}

\newcommand\nosp {    		% commande vide (pas d'espace)
}
\newcommand{\id}[1]{} %ignore

\newcommand\boite[2]{				% Boite simple sans titre
	\vspace{5mm}
	\setlength{\fboxrule}{0.2mm}
	\setlength{\fboxsep}{5mm}	
	\fcolorbox{#1}{#1!3}{\makebox[\linewidth-2\fboxrule-2\fboxsep]{
  		\begin{minipage}[t]{\linewidth-2\fboxrule-4\fboxsep}\setlength{\parskip}{3mm}
  			 #2
  		\end{minipage}
	}}
	\vspace{5mm}
}

\newcommand\CBox[4]{				% Boites
	\vspace{5mm}
	\setlength{\fboxrule}{0.2mm}
	\setlength{\fboxsep}{5mm}
	
	\fcolorbox{#1}{#1!3}{\makebox[\linewidth-2\fboxrule-2\fboxsep]{
		\begin{minipage}[t]{1cm}\setlength{\parskip}{3mm}
	  		\textcolor{#1}{\LARGE{#2}}    
 	 	\end{minipage}  
  		\begin{minipage}[t]{\linewidth-2\fboxrule-4\fboxsep}\setlength{\parskip}{3mm}
			\raisebox{1.2mm}{\normalsize\sffamily{\textcolor{#1}{#3}}}						
  			 #4
  		\end{minipage}
	}}
	\vspace{5mm}
}

\newcommand\cadre[3]{				% Boites convertible html
	\par
	\vspace{2mm}
	\setlength{\fboxrule}{0.1mm}
	\setlength{\fboxsep}{5mm}
	\fcolorbox{#1}{white}{\makebox[\linewidth-2\fboxrule-2\fboxsep]{
  		\begin{minipage}[t]{\linewidth-2\fboxrule-4\fboxsep}\setlength{\parskip}{3mm}
			\raisebox{-2.5mm}{\sffamily \small{\textcolor{#1}{\MakeUppercase{#2}}}}		
			\par		
  			 #3
 	 		\end{minipage}
	}}
		\vspace{2mm}
	\par
}

\newcommand\bloc[3]{				% Boites convertible html sans bordure
     \needspace{2\baselineskip}
     {\sffamily \small{\textcolor{#1}{\MakeUppercase{#2}}}}    
		\par		
  			 #3
		\par
}

\newcommand\CHelp[1]{
     \CBox{Plum}{\faInfoCircle}{À RETENIR}{#1}
}

\newcommand\CUp[1]{
     \CBox{NavyBlue}{\faThumbsOUp}{EN PRATIQUE}{#1}
}

\newcommand\CInfo[1]{
     \CBox{Sepia}{\faArrowCircleRight}{REMARQUE}{#1}
}

\newcommand\CRedac[1]{
     \CBox{PineGreen}{\faEdit}{BIEN R\'EDIGER}{#1}
}

\newcommand\CError[1]{
     \CBox{Red}{\faExclamationTriangle}{ATTENTION}{#1}
}

\newcommand\TitreExo[2]{
\needspace{4\baselineskip}
 {\sffamily\large EXERCICE #1\ (\emph{#2 points})}
\vspace{5mm}
}

\newcommand\img[2]{
          \includegraphics[width=#2\paperwidth]{\imgdir#1}
}

\newcommand\imgsvg[2]{
       \begin{center}   \includegraphics[width=#2\paperwidth]{\imgsvgdir#1} \end{center}
}


\newcommand\Lien[2]{
     \href{#1}{#2 \tiny \faExternalLink}
}
\newcommand\mcLien[2]{
     \href{https~://www.maths-cours.fr/#1}{#2 \tiny \faExternalLink}
}

\newcommand{\euro}{\eurologo{}}

%================================================================================================================================
%
% Macros - Environement
%
%================================================================================================================================

\newenvironment{tex}{ %
}
{%
}

\newenvironment{indente}{ %
	\setlength\parindent{10mm}
}

{
	\setlength\parindent{0mm}
}

\newenvironment{corrige}{%
     \needspace{3\baselineskip}
     \medskip
     \textbf{\textsc{Corrigé}}
     \medskip
}
{
}

\newenvironment{extern}{%
     \begin{center}
     }
     {
     \end{center}
}

\NewEnviron{code}{%
	\par
     \boite{gray}{\texttt{%
     \BODY
     }}
     \par
}

\newenvironment{vbloc}{% boite sans cadre empeche saut de page
     \begin{minipage}[t]{\linewidth}
     }
     {
     \end{minipage}
}
\NewEnviron{h2}{%
    \needspace{3\baselineskip}
    \vspace{0.6cm}
	\noindent \MakeUppercase{\sffamily \large \BODY}
	\vspace{1mm}\textcolor{mcgris}{\hrule}\vspace{0.4cm}
	\par
}{}

\NewEnviron{h3}{%
    \needspace{3\baselineskip}
	\vspace{5mm}
	\textsc{\BODY}
	\par
}

\NewEnviron{margeneg}{ %
\begin{addmargin}[-1cm]{0cm}
\BODY
\end{addmargin}
}

\NewEnviron{html}{%
}

\begin{document}
\meta{url}{/cours/polynomes-du-second-degre/}
\meta{pid}{356}
\meta{titre}{Polynômes et équations du second degré}
\meta{type}{cours}

\begin{h2}1. Fonctions polynômes\end{h2}
\cadre{bleu}{Définition}{% id="d10"
     Une fonction $P$ est une \textbf{fonction polynôme} si elle est définie sur $\mathbb{R}$ et si on peut l'écrire sous la forme~:
     \begin{center}$P\left(x\right)=a_{n}x^{n}+a_{n-1}x^{n-1}+ . . . +a_{1}x+a_{0}$\end{center}
}
\bloc{vert}{Remarques}{% id="r10"
     \begin{itemize}
          \item par abus de langage, on dit souvent polynôme au lieu de fonction polynôme.
          \item les nombres $a_{i}$ s'appellent les \textbf{coefficients} du polynôme.
     \end{itemize}
}
\cadre{bleu}{Définition (Degré d'un polynôme)}{% id="d20"
     Si $P\left(x\right)=a_{n}x^{n}+a_{n-1}x^{n-1}+ . . . +a_{1}x+a_{0}$ (où le coefficient $ a_n $ est non nul), on dit que $P$ est une fonction polynôme de \textbf{degré} $n$.
}
\bloc{orange}{Cas particuliers}{% id="r20"
     \begin{itemize}
          \item la fonction nulle n'a pas de degré.
          \item une fonction constante non nulle définie par $f\left(x\right)=a$ avec $a\neq 0$ est une fonction polynôme de degré~0.
          \item une fonction affine $f\left(x\right)=ax+b$ avec $a\neq 0$ est une fonction polynôme de degré~1.
     \end{itemize}
}
\cadre{vert}{Propriété}{% id="p30"
     Le produit d'un polynôme de degré $n$ par un polynôme de degré $m$ est un polynôme de degré $m+n$.
}
\bloc{vert}{Remarque}{% id="r30"
     Il n'existe pas de formule donnant le degré d'une somme de polynôme. On peut tout au plus dire que le degré de $ P+Q $ est inférieur ou égal à la fois au degré de $P$ et au degré de $Q$.
}
\cadre{vert}{Propriété}{% id="p40"
     Deux polynômes sont égaux si et seulement si les coefficients des termes de même degré sont égaux.
}
\bloc{orange}{Cas particulier}{% id="r40"
     $P$ est le polynôme nul si et seulement si tous ses coefficients sont nuls.
}
\cadre{bleu}{Définition}{% id="d50"
     On dit que $a \in \mathbb{R}$ est une racine du polynôme $P$ si et seulement si $P\left(a\right)=0$.
}
\bloc{orange}{Exemple}{% id="e50"
     $1$ est racine du polynôme $P\left(x\right)=x^{3}-2x+1$ car $P\left(1\right)=0$
}
\cadre{rouge}{Théorème}{% id="t60"
     Si $P$ est un polynôme de degré $n\geqslant 1$ et si $a$ est une racine de $P$ alors $P\left(x\right)$ peut s'écrire sous la forme~:
     \begin{center}$P\left(x\right)=\left(x-a\right)Q\left(x\right)$\end{center}
     où $Q$ est un polynôme de degré $n-1$.
}
\begin{h2}2. Fonctions polynômes du second degré\end{h2}
\cadre{bleu}{Définition}{% id="d70"
     On appelle \textbf{polynôme (ou trinôme) du second degré} toute expression pouvant se mettre sous la forme~:
     \begin{center}$P\left(x\right)=ax^{2}+bx+c$\end{center}
     où $a$, $b$ et $c$ sont des réels avec $a \neq 0$.
}
\bloc{orange}{Exemples}{% id="e60"
     \begin{itemize}
          \item $P\left(x\right)=2x^{2}+3x-5$ est un polynôme du second degré.
          \item $P\left(x\right)=x^{2}-1$ est un polynôme du second degré avec $b=0$ mais $Q\left(x\right)=x-1$ n'en est pas un car $a$ n'est pas différent de zéro (c'est un polynôme du premier degré - ou une fonction affine).
          \item $P\left(x\right)=5\left(x-1\right)\left(3-2x\right)$ est un polynôme du second degré car en développant on obtient une expression du type souhaité.
     \end{itemize}
}
\cadre{rouge}{Théorème et définition}{% id="d70"
     Tout polynôme du second degré $P\left(x\right)=ax^{2}+bx+c$ peut s'écrire sous la forme~:
     \begin{center}$P\left(x\right)=a\left(x-\alpha \right)^{2}+ \beta $\end{center}
     avec $\alpha =-\frac{b}{2a}$ et $\beta =P\left(\alpha \right)$.
     \par
     Cette expression s'appelle \textbf{forme canonique} du polynôme $P$.
}
\cadre{bleu}{Définition}{% id="e70"
     Le nombre $\Delta =b^{2}-4ac$ s'appelle le \textbf{discriminant} du trinôme $ax^{2}+bx+c$.
}
\cadre{vert}{Propriété (Racines d'un polynôme du second degré)}{% id="p80"
     L'équation $ax^{2}+bx+c=0$~:
     \begin{itemize}
          \item n'a aucune solution réelle si $\Delta < 0$~;
          \item a une solution unique $x_{0}=\alpha =-\frac{b}{2a}$ si $\Delta =0$~;
          \item a deux solutions $x_{1}=\frac{-b+\sqrt{\Delta }}{2a}$ et $x_{2}=\frac{-b-\sqrt{\Delta }}{2a}$ si $\Delta > 0$.
     \end{itemize}
}
\bloc{orange}{Exemples}{% id="e80"
     \begin{itemize}
          \item \textbf{$P_{1}\left(x\right)=-x^{2}+3x-2$}~:
          \par
          $\Delta =9-4\times \left(-1\right)\times \left(-2\right)=1$.
          \par
          $P_{1}$ possède 2 racines~:
          \par
          $x_{1}=\frac{-3-1}{-2}=2$ et $x_{2}=\frac{-3+1}{-2}=1$
          \item \textbf{$P_{2}\left(x\right)=x^{2}-4x+4$}~:
          \par
          $\Delta =16-4\times 1\times 4=0$.
          \par
          $P_{2}$ possède une seule racine~:
          \par
          $x_{0}=-\frac{-4}{2}=2$.
          \item \textbf{$P_{3}\left(x\right)=x^{2}+x+1$}~:
          \par
          $\Delta =1-4\times 1\times 1=-3$.
          \par
          $P_{3}$ ne possède aucune racine.
     \end{itemize}
}
\cadre{vert}{Propriété (Somme et produit des racines)}{% id="p90"
     Soit un polynôme $P\left(x\right)=ax^{2}+bx+c$ dont le discriminant est strictement positif.
     \begin{itemize}
          \item La somme des racines vaut $x_{1}+x_{2}=-\frac{b}{a}$.
          \item Le produit des racines vaut $x_{1}x_{2}=\frac{c}{a}$.
     \end{itemize}
}
\bloc{cyan}{Remarque}{% id="r90"
     Ces propriétés sont souvent utilisées pour résoudre rapidement une équation qui possède une racine "évidente".
     \par
     Par exemple l'équation $x^{2}-4x+3=0$ admet $x_{1}=1$ comme racine puisque $1^{2}-4\times 1+3=0$~; comme $x_{1}\times x_{2}=\frac{c}{a}=3$ l'autre racine est $x_{2}=3$ .
}
\cadre{vert}{Propriété (Signe d'un polynôme du second degré)}{% id="p100"
     Le polynôme $P\left(x\right)=ax^{2}+bx+c$~:
     \begin{itemize}
          \item est toujours du signe de $a$ si $\Delta < 0$~;
          \item est toujours du signe de $a$ mais s'annule en $x_{0}=\alpha =-\frac{b}{2a}$ si $\Delta =0$~;
          \item est du signe de $a$ \og à l'extérieur des racines \fg{} (c'est à dire sur $\left]-\infty~; x_{1}\right[ \cup \left]x_{2}; +\infty \right[$) et du signe opposé \og entre les racines \fg{} ( sur $\left]x_{1}; x_{2}\right[$).
     \end{itemize}
}
\bloc{cyan}{Remarque}{% id="r100"
     Suivant chacun des cas on peut représenter le tableau de signe de $P$ de la façon suivante~:
     \begin{itemize}
          \item \textbf{Si $\Delta > 0$~:} $P\left(x\right)$ est du signe de $a$ à l'extérieur des racines (c'est à dire si $x < x_{1}$ ou $x > x_{2}$ ) et du signe opposé entre les racines (si $x_{1} < x < x_{2}$).
          %:-+-+-+-+- Engendré par~: http://math.et.info.free.fr/TikZ/TableauxVariations/
          \begin{center}
               \begin{extern}%width="500" alt="Tableau de signe plynôme du second degré delta positif"
                    \begin{tikzpicture}[scale=0.875]
                         % Styles
                         \tikzstyle{cadre}=[thin]
                         \tikzstyle{fleche}=[->,>=latex,thin]
                         \tikzstyle{nondefini}=[lightgray]
                         % Dimensions Modifiables
                         \def\Lrg{1.8}
                         \def\HtX{1.2}
                         \def\HtY{0.5}
                         % Dimensions Calculées
                         \def\lignex{-0.5*\HtX}
                         \def\lignef{-1.5*\HtX}
                         \def\separateur{-0.5*\Lrg}
                         % Largeur du tableau
                         \def\gauche{-1.5*\Lrg}
                         \def\droite{6.5*\Lrg}
                         % Hauteur du tableau
                         \def\haut{0.5*\HtX}
                         \def\bas{-2.5*\HtX-2*\HtY}
                         % Pointillés
                         \draw[gray] (2*\Lrg,\lignex) -- (2*\Lrg,\lignef);
                         \draw[gray] (4*\Lrg,\lignex) -- (4*\Lrg,\lignef);
                         % Ligne de l'abscisse~: x
                         \node at (-1*\Lrg,0) {$x$};
                         \node at (0*\Lrg,0) {$-\infty$};
                         \node at (2*\Lrg,0) {$x_1$};
                         \node at (4*\Lrg,0) {$x_2$};
                         \node at (6*\Lrg,0) {$+\infty$};
                         % Ligne de la dérivée~: f'(x)
                         \node at (-1*\Lrg,-1*\HtX) {$P(x)$};
                         \node at (0*\Lrg,-1*\HtX) {$ $};
                         \node at (1*\Lrg,-1*\HtX) {signe de $a$};
                         \node at (2*\Lrg,-1*\HtX) {$0$};
                         \node at (3*\Lrg,-1*\HtX) {signe de $-a$};
                         \node at (4*\Lrg,-1*\HtX) {$0$};
                         \node at (5*\Lrg,-1*\HtX) {signe de $a$};
                         \node at (6*\Lrg,-1*\HtX) {$ $};
                         % Ligne de la fonction~: f(x)
                         % Encadrement
                         \draw[cadre] (\separateur,\haut) -- (\separateur, \lignef);
                         \draw[cadre] (\gauche,\haut) rectangle (\droite, \lignef);
                         \draw[cadre] (\gauche,\lignex) -- (\droite,\lignex);
                    \end{tikzpicture}
               \end{extern}
          \end{center}
          %:-+-+-+-+- Fin
          \item \textbf{Si $\Delta =0$~:} $P\left(x\right)$ est toujours du signe de $a$ sauf en $x_{0}$ (où il s'annule).
          %:-+-+-+-+- Engendré par~: http://math.et.info.free.fr/TikZ/TableauxVariations/
          \begin{center}
               \begin{extern}%width="390" alt="Tableau de signe plynôme du second degré delta nul"
                    \begin{tikzpicture}[scale=0.875]
                         % Styles
                         \tikzstyle{cadre}=[thin]
                         \tikzstyle{fleche}=[->,>=latex,thin]
                         \tikzstyle{nondefini}=[lightgray]
                         % Dimensions Modifiables
                         \def\Lrg{1.8}
                         \def\HtX{1.2}
                         \def\HtY{0.5}
                         % Dimensions Calculées
                         \def\lignex{-0.5*\HtX}
                         \def\lignef{-1.5*\HtX}
                         \def\separateur{-0.5*\Lrg}
                         % Largeur du tableau
                         \def\gauche{-1.5*\Lrg}
                         \def\droite{4.5*\Lrg}
                         % Hauteur du tableau
                         \def\haut{0.5*\HtX}
                         \def\bas{-2.5*\HtX-2*\HtY}
                         % Pointillés
                         \draw[gray] (2*\Lrg,\lignex) -- (2*\Lrg,\lignef);
                         % Ligne de l'abscisse~: x
                         \node at (-1*\Lrg,0) {$x$};
                         \node at (0*\Lrg,0) {$-\infty$};
                         \node at (2*\Lrg,0) {$x_0$};
                         \node at (4*\Lrg,0) {$+\infty$};
                         % Ligne de la dérivée~: f'(x)
                         \node at (-1*\Lrg,-1*\HtX) {$P(x)$};
                         \node at (0*\Lrg,-1*\HtX) {$ $};
                         \node at (1*\Lrg,-1*\HtX) {signe de $a$};
                         \node at (2*\Lrg,-1*\HtX) {$0$};
                         \node at (3*\Lrg,-1*\HtX) {signe de $a$};
                         \node at (4*\Lrg,-1*\HtX) {$ $};
                         % Ligne de la fonction~: f(x)
                         % Encadrement
                         \draw[cadre] (\separateur,\haut) -- (\separateur, \lignef);
                         \draw[cadre] (\gauche,\haut) rectangle (\droite, \lignef);
                         \draw[cadre] (\gauche,\lignex) -- (\droite,\lignex);
                    \end{tikzpicture}
               \end{extern}
          \end{center}
          %:-+-+-+-+- Fin
          \item \textbf{Si $\Delta < 0$~:} $P\left(x\right)$ est toujours du signe de $a$.
          \begin{center}
               \begin{extern}%width="230" alt="Tableau de signe plynôme du second degré delta négatif"
                    \begin{tikzpicture}[scale=0.875]
                         % Styles
                         \tikzstyle{cadre}=[thin]
                         \tikzstyle{fleche}=[->,>=latex,thin]
                         \tikzstyle{nondefini}=[lightgray]
                         % Dimensions Modifiables
                         \def\Lrg{1.5}
                         \def\HtX{1.2}
                         \def\HtY{0.5}
                         % Dimensions Calculées
                         \def\lignex{-0.5*\HtX}
                         \def\lignef{-1.5*\HtX}
                         \def\separateur{-0.5*\Lrg}
                         % Largeur du tableau
                         \def\gauche{-1.5*\Lrg}
                         \def\droite{2.5*\Lrg}
                         % Hauteur du tableau
                         \def\haut{0.5*\HtX}
                         \def\bas{-2.5*\HtX-2*\HtY}
                         % Ligne de l'abscisse~: x
                         \node at (-1*\Lrg,0) {$x$};
                         \node at (0*\Lrg,0) {$-\infty$};
                         \node at (2*\Lrg,0) {$+\infty$};
                         % Ligne de la dérivée~: f'(x)
                         \node at (-1*\Lrg,-1*\HtX) {$P(x)$};
                         \node at (0*\Lrg,-1*\HtX) {$ $};
                         \node at (1*\Lrg,-1*\HtX) {signe de $a$};
                         \node at (2*\Lrg,-1*\HtX) {$ $};
                         % Ligne de la fonction~: f(x)
                         % Encadrement
                         \draw[cadre] (\separateur,\haut) -- (\separateur, \lignef);
                         \draw[cadre] (\gauche,\haut) rectangle (\droite, \lignef);
                         \draw[cadre] (\gauche,\lignex) -- (\droite,\lignex);
                    \end{tikzpicture}
               \end{extern}
          \end{center}
     \end{itemize}
}
\bloc{orange}{Exemples}{% id="e100"
     Si l'on reprend les exemples précédents~:
     \begin{itemize}
          \item $P_{1}\left(x\right)=-x^{2}+3x-2$~:
          \par
          $\Delta > 0$ et $a < 0$.
          %:-+-+-+-+- Engendré par~: http://math.et.info.free.fr/TikZ/TableauxVariations/
          \begin{center}
               \begin{extern}%width="430" alt="Exemple tableau de signe plynôme du second degré delta positif"
                    \begin{tikzpicture}[scale=0.875]
                         % Styles
                         \tikzstyle{cadre}=[thin]
                         \tikzstyle{fleche}=[->,>=latex,thin]
                         \tikzstyle{nondefini}=[lightgray]
                         % Dimensions Modifiables
                         \def\Lrg{1.5}
                         \def\HtX{1.2}
                         \def\HtY{0.5}
                         % Dimensions Calculées
                         \def\lignex{-0.5*\HtX}
                         \def\lignef{-1.5*\HtX}
                         \def\separateur{-0.5*\Lrg}
                         % Largeur du tableau
                         \def\gauche{-1.5*\Lrg}
                         \def\droite{6.5*\Lrg}
                         % Hauteur du tableau
                         \def\haut{0.5*\HtX}
                         \def\bas{-2.5*\HtX-2*\HtY}
                         % Pointillés
                         \draw[gray] (2*\Lrg,\lignex) -- (2*\Lrg,\lignef);
                         \draw[gray] (4*\Lrg,\lignex) -- (4*\Lrg,\lignef);
                         % Ligne de l'abscisse~: x
                         \node at (-1*\Lrg,0) {$x$};
                         \node at (0*\Lrg,0) {$-\infty$};
                         \node at (2*\Lrg,0) {$1$};
                         \node at (4*\Lrg,0) {$2$};
                         \node at (6*\Lrg,0) {$+\infty$};
                         % Ligne de la dérivée~: f'(x)
                         \node at (-1*\Lrg,-1*\HtX) {$P(x)$};
                         \node at (0*\Lrg,-1*\HtX) {$ $};
                         \node at (1*\Lrg,-1*\HtX) {$-$};
                         \node at (2*\Lrg,-1*\HtX) {$0$};
                         \node at (3*\Lrg,-1*\HtX) {$+$};
                         \node at (4*\Lrg,-1*\HtX) {$0$};
                         \node at (5*\Lrg,-1*\HtX) {$-$};
                         \node at (6*\Lrg,-1*\HtX) {$ $};
                         % Ligne de la fonction~: f(x)
                         % Encadrement
                         \draw[cadre] (\separateur,\haut) -- (\separateur, \lignef);
                         \draw[cadre] (\gauche,\haut) rectangle (\droite, \lignef);
                         \draw[cadre] (\gauche,\lignex) -- (\droite,\lignex);
                    \end{tikzpicture}
               \end{extern}
          \end{center}
          %:-+-+-+-+- Fin
          \item $P_{2}\left(x\right)=x^{2}-4x+4$~:
          \par
          $\Delta =0$ et $a > 0$.
          %:-+-+-+-+- Engendré par~: http://math.et.info.free.fr/TikZ/TableauxVariations/
          \begin{center}
               \begin{extern}%width="330" alt="Exemple tableau de signe plynôme du second degré delta nul"
                    \begin{tikzpicture}[scale=0.875]
                         % Styles
                         \tikzstyle{cadre}=[thin]
                         \tikzstyle{fleche}=[->,>=latex,thin]
                         \tikzstyle{nondefini}=[lightgray]
                         % Dimensions Modifiables
                         \def\Lrg{1.5}
                         \def\HtX{1.2}
                         \def\HtY{0.5}
                         % Dimensions Calculées
                         \def\lignex{-0.5*\HtX}
                         \def\lignef{-1.5*\HtX}
                         \def\separateur{-0.5*\Lrg}
                         % Largeur du tableau
                         \def\gauche{-1.5*\Lrg}
                         \def\droite{4.5*\Lrg}
                         % Hauteur du tableau
                         \def\haut{0.5*\HtX}
                         \def\bas{-2.5*\HtX-2*\HtY}
                         % Pointillés
                         \draw[gray] (2*\Lrg,\lignex) -- (2*\Lrg,\lignef);
                         % Ligne de l'abscisse~: x
                         \node at (-1*\Lrg,0) {$x$};
                         \node at (0*\Lrg,0) {$-\infty$};
                         \node at (2*\Lrg,0) {$2$};
                         \node at (4*\Lrg,0) {$+\infty$};
                         % Ligne de la dérivée~: f'(x)
                         \node at (-1*\Lrg,-1*\HtX) {$P(x)$};
                         \node at (0*\Lrg,-1*\HtX) {$ $};
                         \node at (1*\Lrg,-1*\HtX) {+};
                         \node at (2*\Lrg,-1*\HtX) {$0$};
                         \node at (3*\Lrg,-1*\HtX) {+};
                         \node at (4*\Lrg,-1*\HtX) {$ $};
                         % Ligne de la fonction~: f(x)
                         % Encadrement
                         \draw[cadre] (\separateur,\haut) -- (\separateur, \lignef);
                         \draw[cadre] (\gauche,\haut) rectangle (\droite, \lignef);
                         \draw[cadre] (\gauche,\lignex) -- (\droite,\lignex);
                    \end{tikzpicture}
               \end{extern}
          \end{center}
          %:-+-+-+-+- Fin
          \item $P_{3}\left(x\right)=x^{2}+x+1$~:
          \par
          $\Delta < 0$ et $a > 0$.
          \begin{center}
               \begin{extern}%width="230" alt="Exemple tableau de signe plynôme du second degré delta négatif"
                    \begin{tikzpicture}[scale=0.875]
                         % Styles
                         \tikzstyle{cadre}=[thin]
                         \tikzstyle{fleche}=[->,>=latex,thin]
                         \tikzstyle{nondefini}=[lightgray]
                         % Dimensions Modifiables
                         \def\Lrg{1.5}
                         \def\HtX{1.2}
                         \def\HtY{0.5}
                         % Dimensions Calculées
                         \def\lignex{-0.5*\HtX}
                         \def\lignef{-1.5*\HtX}
                         \def\separateur{-0.5*\Lrg}
                         % Largeur du tableau
                         \def\gauche{-1.5*\Lrg}
                         \def\droite{2.5*\Lrg}
                         % Hauteur du tableau
                         \def\haut{0.5*\HtX}
                         \def\bas{-2.5*\HtX-2*\HtY}
                         % Ligne de l'abscisse~: x
                         \node at (-1*\Lrg,0) {$x$};
                         \node at (0*\Lrg,0) {$-\infty$};
                         \node at (2*\Lrg,0) {$+\infty$};
                         % Ligne de la dérivée~: f'(x)
                         \node at (-1*\Lrg,-1*\HtX) {$P(x)$};
                         \node at (0*\Lrg,-1*\HtX) {$ $};
                         \node at (1*\Lrg,-1*\HtX) {$+$};
                         \node at (2*\Lrg,-1*\HtX) {$ $};
                         % Ligne de la fonction~: f(x)
                         % Encadrement
                         \draw[cadre] (\separateur,\haut) -- (\separateur, \lignef);
                         \draw[cadre] (\gauche,\haut) rectangle (\droite, \lignef);
                         \draw[cadre] (\gauche,\lignex) -- (\droite,\lignex);
                    \end{tikzpicture}
               \end{extern}
          \end{center}
     \end{itemize}
}
On rappelle que les solutions de l'équation $f\left(x\right)=0$ sont les abscisses des \textbf{points d'intersection de la courbe} $C_{f}$ et de l'\textbf{axe des abscisses}.
\par
En regroupant les propriétés de ce chapitre et celles vues en Seconde on peut résumer ces résultats dans le tableau~:

\begin{center}
\imgsvg{cours_trinomes-du-second-degre}{0.3}% alt="cours_trinomes-du-second-degre" style="width:70rem"
\end{center}


\end{document}