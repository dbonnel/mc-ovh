\documentclass[a4paper]{article}

%================================================================================================================================
%
% Packages
%
%================================================================================================================================

\usepackage[T1]{fontenc} 	% pour caractères accentués
\usepackage[utf8]{inputenc}  % encodage utf8
\usepackage[french]{babel}	% langue : français
\usepackage{fourier}			% caractères plus lisibles
\usepackage[dvipsnames]{xcolor} % couleurs
\usepackage{fancyhdr}		% réglage header footer
\usepackage{needspace}		% empêcher sauts de page mal placés
\usepackage{graphicx}		% pour inclure des graphiques
\usepackage{enumitem,cprotect}		% personnalise les listes d'items (nécessaire pour ol, al ...)
\usepackage{hyperref}		% Liens hypertexte
\usepackage{pstricks,pst-all,pst-node,pstricks-add,pst-math,pst-plot,pst-tree,pst-eucl} % pstricks
\usepackage[a4paper,includeheadfoot,top=2cm,left=3cm, bottom=2cm,right=3cm]{geometry} % marges etc.
\usepackage{comment}			% commentaires multilignes
\usepackage{amsmath,environ} % maths (matrices, etc.)
\usepackage{amssymb,makeidx}
\usepackage{bm}				% bold maths
\usepackage{tabularx}		% tableaux
\usepackage{colortbl}		% tableaux en couleur
\usepackage{fontawesome}		% Fontawesome
\usepackage{environ}			% environment with command
\usepackage{fp}				% calculs pour ps-tricks
\usepackage{multido}			% pour ps tricks
\usepackage[np]{numprint}	% formattage nombre
\usepackage{tikz,tkz-tab} 			% package principal TikZ
\usepackage{pgfplots}   % axes
\usepackage{mathrsfs}    % cursives
\usepackage{calc}			% calcul taille boites
\usepackage[scaled=0.875]{helvet} % font sans serif
\usepackage{svg} % svg
\usepackage{scrextend} % local margin
\usepackage{scratch} %scratch
\usepackage{multicol} % colonnes
%\usepackage{infix-RPN,pst-func} % formule en notation polanaise inversée
\usepackage{listings}

%================================================================================================================================
%
% Réglages de base
%
%================================================================================================================================

\lstset{
language=Python,   % R code
literate=
{á}{{\'a}}1
{à}{{\`a}}1
{ã}{{\~a}}1
{é}{{\'e}}1
{è}{{\`e}}1
{ê}{{\^e}}1
{í}{{\'i}}1
{ó}{{\'o}}1
{õ}{{\~o}}1
{ú}{{\'u}}1
{ü}{{\"u}}1
{ç}{{\c{c}}}1
{~}{{ }}1
}


\definecolor{codegreen}{rgb}{0,0.6,0}
\definecolor{codegray}{rgb}{0.5,0.5,0.5}
\definecolor{codepurple}{rgb}{0.58,0,0.82}
\definecolor{backcolour}{rgb}{0.95,0.95,0.92}

\lstdefinestyle{mystyle}{
    backgroundcolor=\color{backcolour},   
    commentstyle=\color{codegreen},
    keywordstyle=\color{magenta},
    numberstyle=\tiny\color{codegray},
    stringstyle=\color{codepurple},
    basicstyle=\ttfamily\footnotesize,
    breakatwhitespace=false,         
    breaklines=true,                 
    captionpos=b,                    
    keepspaces=true,                 
    numbers=left,                    
xleftmargin=2em,
framexleftmargin=2em,            
    showspaces=false,                
    showstringspaces=false,
    showtabs=false,                  
    tabsize=2,
    upquote=true
}

\lstset{style=mystyle}


\lstset{style=mystyle}
\newcommand{\imgdir}{C:/laragon/www/newmc/assets/imgsvg/}
\newcommand{\imgsvgdir}{C:/laragon/www/newmc/assets/imgsvg/}

\definecolor{mcgris}{RGB}{220, 220, 220}% ancien~; pour compatibilité
\definecolor{mcbleu}{RGB}{52, 152, 219}
\definecolor{mcvert}{RGB}{125, 194, 70}
\definecolor{mcmauve}{RGB}{154, 0, 215}
\definecolor{mcorange}{RGB}{255, 96, 0}
\definecolor{mcturquoise}{RGB}{0, 153, 153}
\definecolor{mcrouge}{RGB}{255, 0, 0}
\definecolor{mclightvert}{RGB}{205, 234, 190}

\definecolor{gris}{RGB}{220, 220, 220}
\definecolor{bleu}{RGB}{52, 152, 219}
\definecolor{vert}{RGB}{125, 194, 70}
\definecolor{mauve}{RGB}{154, 0, 215}
\definecolor{orange}{RGB}{255, 96, 0}
\definecolor{turquoise}{RGB}{0, 153, 153}
\definecolor{rouge}{RGB}{255, 0, 0}
\definecolor{lightvert}{RGB}{205, 234, 190}
\setitemize[0]{label=\color{lightvert}  $\bullet$}

\pagestyle{fancy}
\renewcommand{\headrulewidth}{0.2pt}
\fancyhead[L]{maths-cours.fr}
\fancyhead[R]{\thepage}
\renewcommand{\footrulewidth}{0.2pt}
\fancyfoot[C]{}

\newcolumntype{C}{>{\centering\arraybackslash}X}
\newcolumntype{s}{>{\hsize=.35\hsize\arraybackslash}X}

\setlength{\parindent}{0pt}		 
\setlength{\parskip}{3mm}
\setlength{\headheight}{1cm}

\def\ebook{ebook}
\def\book{book}
\def\web{web}
\def\type{web}

\newcommand{\vect}[1]{\overrightarrow{\,\mathstrut#1\,}}

\def\Oij{$\left(\text{O}~;~\vect{\imath},~\vect{\jmath}\right)$}
\def\Oijk{$\left(\text{O}~;~\vect{\imath},~\vect{\jmath},~\vect{k}\right)$}
\def\Ouv{$\left(\text{O}~;~\vect{u},~\vect{v}\right)$}

\hypersetup{breaklinks=true, colorlinks = true, linkcolor = OliveGreen, urlcolor = OliveGreen, citecolor = OliveGreen, pdfauthor={Didier BONNEL - https://www.maths-cours.fr} } % supprime les bordures autour des liens

\renewcommand{\arg}[0]{\text{arg}}

\everymath{\displaystyle}

%================================================================================================================================
%
% Macros - Commandes
%
%================================================================================================================================

\newcommand\meta[2]{    			% Utilisé pour créer le post HTML.
	\def\titre{titre}
	\def\url{url}
	\def\arg{#1}
	\ifx\titre\arg
		\newcommand\maintitle{#2}
		\fancyhead[L]{#2}
		{\Large\sffamily \MakeUppercase{#2}}
		\vspace{1mm}\textcolor{mcvert}{\hrule}
	\fi 
	\ifx\url\arg
		\fancyfoot[L]{\href{https://www.maths-cours.fr#2}{\black \footnotesize{https://www.maths-cours.fr#2}}}
	\fi 
}


\newcommand\TitreC[1]{    		% Titre centré
     \needspace{3\baselineskip}
     \begin{center}\textbf{#1}\end{center}
}

\newcommand\newpar{    		% paragraphe
     \par
}

\newcommand\nosp {    		% commande vide (pas d'espace)
}
\newcommand{\id}[1]{} %ignore

\newcommand\boite[2]{				% Boite simple sans titre
	\vspace{5mm}
	\setlength{\fboxrule}{0.2mm}
	\setlength{\fboxsep}{5mm}	
	\fcolorbox{#1}{#1!3}{\makebox[\linewidth-2\fboxrule-2\fboxsep]{
  		\begin{minipage}[t]{\linewidth-2\fboxrule-4\fboxsep}\setlength{\parskip}{3mm}
  			 #2
  		\end{minipage}
	}}
	\vspace{5mm}
}

\newcommand\CBox[4]{				% Boites
	\vspace{5mm}
	\setlength{\fboxrule}{0.2mm}
	\setlength{\fboxsep}{5mm}
	
	\fcolorbox{#1}{#1!3}{\makebox[\linewidth-2\fboxrule-2\fboxsep]{
		\begin{minipage}[t]{1cm}\setlength{\parskip}{3mm}
	  		\textcolor{#1}{\LARGE{#2}}    
 	 	\end{minipage}  
  		\begin{minipage}[t]{\linewidth-2\fboxrule-4\fboxsep}\setlength{\parskip}{3mm}
			\raisebox{1.2mm}{\normalsize\sffamily{\textcolor{#1}{#3}}}						
  			 #4
  		\end{minipage}
	}}
	\vspace{5mm}
}

\newcommand\cadre[3]{				% Boites convertible html
	\par
	\vspace{2mm}
	\setlength{\fboxrule}{0.1mm}
	\setlength{\fboxsep}{5mm}
	\fcolorbox{#1}{white}{\makebox[\linewidth-2\fboxrule-2\fboxsep]{
  		\begin{minipage}[t]{\linewidth-2\fboxrule-4\fboxsep}\setlength{\parskip}{3mm}
			\raisebox{-2.5mm}{\sffamily \small{\textcolor{#1}{\MakeUppercase{#2}}}}		
			\par		
  			 #3
 	 		\end{minipage}
	}}
		\vspace{2mm}
	\par
}

\newcommand\bloc[3]{				% Boites convertible html sans bordure
     \needspace{2\baselineskip}
     {\sffamily \small{\textcolor{#1}{\MakeUppercase{#2}}}}    
		\par		
  			 #3
		\par
}

\newcommand\CHelp[1]{
     \CBox{Plum}{\faInfoCircle}{À RETENIR}{#1}
}

\newcommand\CUp[1]{
     \CBox{NavyBlue}{\faThumbsOUp}{EN PRATIQUE}{#1}
}

\newcommand\CInfo[1]{
     \CBox{Sepia}{\faArrowCircleRight}{REMARQUE}{#1}
}

\newcommand\CRedac[1]{
     \CBox{PineGreen}{\faEdit}{BIEN R\'EDIGER}{#1}
}

\newcommand\CError[1]{
     \CBox{Red}{\faExclamationTriangle}{ATTENTION}{#1}
}

\newcommand\TitreExo[2]{
\needspace{4\baselineskip}
 {\sffamily\large EXERCICE #1\ (\emph{#2 points})}
\vspace{5mm}
}

\newcommand\img[2]{
          \includegraphics[width=#2\paperwidth]{\imgdir#1}
}

\newcommand\imgsvg[2]{
       \begin{center}   \includegraphics[width=#2\paperwidth]{\imgsvgdir#1} \end{center}
}


\newcommand\Lien[2]{
     \href{#1}{#2 \tiny \faExternalLink}
}
\newcommand\mcLien[2]{
     \href{https~://www.maths-cours.fr/#1}{#2 \tiny \faExternalLink}
}

\newcommand{\euro}{\eurologo{}}

%================================================================================================================================
%
% Macros - Environement
%
%================================================================================================================================

\newenvironment{tex}{ %
}
{%
}

\newenvironment{indente}{ %
	\setlength\parindent{10mm}
}

{
	\setlength\parindent{0mm}
}

\newenvironment{corrige}{%
     \needspace{3\baselineskip}
     \medskip
     \textbf{\textsc{Corrigé}}
     \medskip
}
{
}

\newenvironment{extern}{%
     \begin{center}
     }
     {
     \end{center}
}

\NewEnviron{code}{%
	\par
     \boite{gray}{\texttt{%
     \BODY
     }}
     \par
}

\newenvironment{vbloc}{% boite sans cadre empeche saut de page
     \begin{minipage}[t]{\linewidth}
     }
     {
     \end{minipage}
}
\NewEnviron{h2}{%
    \needspace{3\baselineskip}
    \vspace{0.6cm}
	\noindent \MakeUppercase{\sffamily \large \BODY}
	\vspace{1mm}\textcolor{mcgris}{\hrule}\vspace{0.4cm}
	\par
}{}

\NewEnviron{h3}{%
    \needspace{3\baselineskip}
	\vspace{5mm}
	\textsc{\BODY}
	\par
}

\NewEnviron{margeneg}{ %
\begin{addmargin}[-1cm]{0cm}
\BODY
\end{addmargin}
}

\NewEnviron{html}{%
}

\begin{document}
\meta{url}{/exercices/fonctions-et-volumes-bac-s-polynesie-2018/}
\meta{pid}{9276}
\meta{titre}{Fonctions et Volumes – Bac S Polynésie 2018}
\meta{type}{exercices}
%
\textbf{\textit{Exercice 2} \quad 6 points}
\par
\textbf{Commun  à tous les candidats}
\medbreak
Dans cet exercice, on s'intéresse au volume d'une ampoule basse consommation.
\bigbreak
\TitreC{Partie A - Modélisation de la forme de l'ampoule}
\medbreak
Le plan est muni d'un repère orthonormé $(O~;~\overrightarrow{u},~\overrightarrow{v})$.
\par
On considère les points A$(-1~;~1)$, B$(0~;~1)$, C$(4~;~3)$, D$(7~;~0)$, E$(4~;~-3)$, F$(O~;~-1)$ et G$(- 1~;~- 1)$.
\par
On modélise la section de l'ampoule par un plan passant par son axe de révolution à l'aide de la figure ci-dessous~:
\begin{center}
     \begin{extern}%width="400" alt="Coupe ampoule Bac S Polynésie 2018"
          \psset{unit=1.2cm,algebraic=true}
          \begin{pspicture*}(-1.8,-3.5)(7.8,3.5)
               \psgrid[gridlabels=0pt,subgriddiv=1,gridwidth=0.3pt](-2,-4)(8,4)
               \psaxes[linewidth=1pt,Dx=10,Dy=10]{->}(0,0)(-1.8,-3.5)(7.8,3.5)
               \psaxes[linewidth=1.2pt,Dx=10,Dy=10]{->}(0,0)(1,1)
               \uput[d](0.5,0){$\overrightarrow{u}$}
               \uput[l](0,0.5){$\overrightarrow{v}$}
               \psline[linecolor=blue](-1,1)(0,1)\psline[linecolor=blue](-1,-1)(0,-1)
               \psplot[linewidth=1pt,plotpoints=1000,linecolor=blue]{0}{4}{2-cos(x*3.141659/4)}
               \psplot[linewidth=1pt,plotpoints=1000,linecolor=blue]{0}{4}{cos(x*3.141659/4)-2}
               \psarc[linewidth=1pt,linecolor=blue](4,0){3}{-90}{90}
               \psdots(-1,1)(0,1)(4,3)(7,0)(4,-3)(0,-1)(-1,-1)
               \uput[dl](0,0){\small O}\uput[ul](-1,1){\small A}\uput[ur](0,1){\small B}\uput[u](4,3){\small C}
               \uput[ur](7,0){\small D}\uput[d](4,-3){\small E}\uput[dr](0,-1){\small F}\uput[d](-1,-1){\small G}
          \end{pspicture*}
     \end{extern}
\end{center}
\medbreak
La partie de la courbe située au-dessus de l'axe des abscisses se décompose de la manière suivante~:
\begin{indent}
     \begin{itemize}
          \item la portion située entre les points A et B est la représentation graphique de la fonction constante
          $h$ définie sur l'intervalle $[-1~;~0]$ par $h(x) = 1$~;
          \item la portion située entre les points B et C est la représentation graphique d'une fonction $f$ définie sur l'intervalle [0~;~4] par $f(x) = a + b \sin \left(c + \frac{\pi}{4} x\right)$, où $a$, $b$ et $c$ sont des réels non nuls
          fixés et où le réel $c$ appartient à l'intervalle $\left[0~;~\frac{\pi}{2}\right]$~;
          \item la portion située entre les points C et D est un quart de cercle de diamètre [CE].
     \end{itemize}
\end{indent}
La partie de la courbe située en-dessous de l'axe des abscisses est obtenue par symétrie par rapport à l'axe des abscisses.
\medbreak
\begin{enumerate}
     \item
     \begin{enumerate}[label=\alph*.]
          \item On appelle $f'$ la fonction dérivée de la fonction $f$. Pour tout réel $x$ de l'intervalle [0~;~4], déterminer $f'(x)$.
          \item On impose que les tangentes aux points B et C à la représentation graphique de la fonction $f$ soient parallèles à l'axe des abscisses. Déterminer la valeur du réel $c$.
     \end{enumerate}
     \item  Déterminer les réels $a$ et $b$.
\end{enumerate}
\bigbreak
\TitreC{Partie B - Approximation du volume de l'ampoule}
\medbreak
Par rotation de la figure précédente autour de l'axe des abscisses, on obtient un modèle de l'ampoule.
\par
Afin d'en calculer le volume, on la décompose en trois parties comme illustré ci-dessous~:
\begin{center}
     \begin{extern}%width="400" alt="Volume ampoule Bac S Polynésie 2018"
          \psset{unit=1.2cm,algebraic=true}
          \begin{pspicture*}(-2,-4)(8,3.5)
               %\psgrid[gridlabels=0pt,subgriddiv=1,gridwidth=0.3pt]
               \par
               \uput[d](0.5,0){$\overrightarrow{u}$}
               \uput[l](0,0.5){$\overrightarrow{v}$}
               \psline(-1,1)(0,1)\psline(-1,-1)(0,-1)
               \psplot[linewidth=1pt,plotpoints=1000]{0}{4}{2-cos(x*3.141659/4)}
               \psplot[linewidth=1pt,plotpoints=1000]{0}{4}{cos(x*3.141659/4)-2}
               \psarc[linewidth=1pt](4,0){3}{-90}{90}
               \psdots(-1,1)(0,1)(4,3)(7,0)(4,-3)(0,-1)(-1,-1)
               \uput[dl](0,0){\small O}\uput[ul](-1,1){\small A}\uput[ur](0,1){\small B}\uput[u](4,3){\small C}
               \uput[ur](7,0){\small D}\uput[d](4,-3){\small E}\uput[dr](0,-1){\small F}\uput[d](-1,-1){\small G}
               \pscustom[fillstyle=solid,fillcolor=mcmauve]{
                    \psplot[linewidth=1pt,plotpoints=1000]{0}{4}{2-cos(x*3.141659/4)}
               \psplot[linewidth=1pt,plotpoints=1000]{4}{0}{cos(x*3.141659/4)-2}}
               \pscustom[fillstyle=solid,fillcolor=mcvert]{
               \psline(4,3)(4,-3)\psarc[linewidth=1pt](4,0){3}{-90}{90}}
               \psframe[fillstyle=solid,fillcolor=red](-1,-1)(0,1)
               \psaxes[linewidth=1pt,Dx=10,Dy=10](0,0)(-2,-3.5)(8,3.5)
               \psaxes[linewidth=1.5pt,Dx=10,Dy=10]{->}(0,0)(1,1)
          \end{pspicture*}
     \end{extern}
\end{center}
\begin{center}
     Vue dans le plan (BCE)
\end{center}
\medbreak
On rappelle que~:
\begin{indent}
     \begin{itemize}
          \item le volume d'un cylindre est donné par la formule $\pi r^2 h$ où $r$ est le rayon du disque de base et $h$ est la hauteur~;
          \item le volume d'une boule de rayon $r$ est donné par la formule $\dfrac{4}{3}\pi r^3$.
     \end{itemize}
\end{indent}
On admet également que, pour tout réel $x$ de l'intervalle [0~;~4], $f(x) = 2 - \cos \left(\frac{\pi}{4}x\right)$.
\medbreak
\begin{enumerate}
     \item Calculer le volume du cylindre de section le rectangle ABFG.
     \item Calculer le volume de la demi-sphère de section le demi -disque de diamètre [CE].
     \item Pour approcher le volume du solide de section la zone colorée en mauve BCEF, on partage le segment [OO$'$] en $n$ segments de même longueur $\dfrac{4}{n}$ puis on construit $n$ cylindres de même hauteur $\dfrac{4}{n}$.
     \begin{enumerate}[label=\alph*.]
          \item \textbf{Cas particulier~:} dans cette question uniquement on choisit $n = 5$.
          \par
          Calculer le volume du troisième cylindre, grisé dans les figures ci-dessous, puis en donner
          la valeur arrondie à $10^{-2}$.
          \begin{center}
               \begin{extern}%width="400" alt="Cylindres vue plan Bac S Polynésie 2018"
                    \psset{unit=1.2cm,algebraic=true}
                    \begin{pspicture*}(-2,-4)(8,3.5)
                         %\psgrid[gridlabels=0pt,subgriddiv=1,gridwidth=0.3pt]
                         \psaxes[linewidth=1pt,Dx=10,Dy=10](0,0)(-2,-3.5)(8,3.5)
                         \psaxes[linewidth=1.5pt,Dx=10,Dy=10]{->}(0,0)(1,1)
                         %\uput[d](0.5,0){$\overrightarrow{i}$}
                         %\uput[l](0,0.5){$\overrightarrow{j}$}
                         \psline(-1,1)(0,1)\psline(-1,-1)(0,-1)
                         \psplot[linewidth=1pt,plotpoints=1000]{0}{4}{2-cos(x*3.141659/4)}
                         \psplot[linewidth=1pt,plotpoints=1000]{0}{4}{cos(x*3.141659/4)-2}
                         \psarc[linewidth=1pt](4,0){3}{-90}{90}
                         \psdots(-1,1)(0,1)(4,3)(7,0)(4,-3)(0,-1)(-1,-1)
                         \uput[dl](0,0){\small O}
                         %\uput[ul](-1,1){\small A}
                         \uput[ur](0,1){\small B}\uput[u](4,3){\small C}
                         \uput[ur](7,0){\small D}\uput[d](4,-3){\small E}\uput[dr](0,-1){\small F}
                         %\uput[d](-1,-1){\small G}
                         \psframe[fillstyle=solid,fillcolor=mcmauve](0,-1)(0.8,1)
                         \psframe[fillstyle=solid,fillcolor=mcmauve](0.8,-1.2)(1.6,1.2)
                         \psframe[fillstyle=solid,fillcolor=blue](1.6,-1.7)(2.4,1.7)
                         \psframe[fillstyle=solid,fillcolor=mcmauve](2.4,-2.3)(3.2,2.3)
                         \psframe[fillstyle=solid,fillcolor=mcmauve](3.2,-2.8)(4,2.8)
                         \psline(-1,-1)(-1,1)\psline(4,3)(4,-3)
                    \end{pspicture*}
               \end{extern}
          \end{center}
          \begin{center}
               Vue dans le plan (BCE)
          \end{center}
          \begin{center}
               \begin{extern}%width="360" alt="Cylindres vue espace Bac S Polynésie 2018"
                    \psset{unit=1.2cm,algebraic=true}
                    \begin{pspicture*}(-2,-4)(8,3.5)
                         %\psgrid
                         \psline(-1,1)(0,1)\psline(-1,-1)(0,-1)
                         \psplot[linewidth=1pt,plotpoints=1000]{0}{4}{2-cos(x*3.141659/4)}
                         \psplot[linewidth=1pt,plotpoints=1000]{0}{4}{cos(x*3.141659/4)-2}
                         \psarc[linewidth=1pt](4,0){3}{-90}{90}
                         \psline(-1,-1)(-1,1)
                         \psframe[fillstyle=solid,fillcolor=mcmauve,linewidth=0pt](3.2,-2.8)(4,2.8)
                         \psellipse[fillstyle=solid,fillcolor=mcmauve,linewidth=0pt](3.2,0)(0.3,2.8)
                         \psellipse[fillstyle=solid,fillcolor=mcmauve,linewidth=0pt](4,0)(0.3,2.8)
                         \psframe[fillstyle=solid,fillcolor=mcmauve,linewidth=0pt](2.4,-2.3)(3.2,2.3)
                         \psellipse[fillstyle=solid,fillcolor=mcmauve,linewidth=0pt](2.4,0)(0.3,2.3)
                         \psellipse[fillstyle=solid,fillcolor=mcmauve,linewidth=0pt](3.2,0)(0.3,2.3)
                         \psframe[fillstyle=solid,fillcolor=blue,linewidth=0pt](1.6,-1.7)(2.4,1.7)
                         \psellipse[fillstyle=solid,fillcolor=blue,linewidth=0pt](1.6,0)(0.3,1.7)
                         \psellipse[fillstyle=solid,fillcolor=blue,linewidth=0pt](2.4,0)(0.3,1.7)
                         \psframe[fillstyle=solid,fillcolor=mcmauve,linewidth=0pt](0.8,-1.2)(1.6,1.2)
                         \psellipse[fillstyle=solid,fillcolor=mcmauve,linewidth=0pt](0.8,0)(0.3,1.2)
                         \psellipse[fillstyle=solid,fillcolor=mcmauve,linewidth=0pt](1.6,0)(0.3,1.2)
                         \psframe[fillstyle=solid,fillcolor=mcmauve,linewidth=0pt](0,-1)(0.8,1)
                         \psellipse[fillstyle=solid,fillcolor=mcmauve,linewidth=0pt](0,0)(0.3,1)
                         \psellipse[fillstyle=solid,fillcolor=mcmauve,linewidth=0pt](0.8,0)(0.3,1)
                    \end{pspicture*}
               \end{extern}
          \end{center}
          \begin{center}
               Vue dans l'espace
          \end{center}
          \item \textbf{Cas général~:} dans cette question, $n$ désigne un entier naturel quelconque non nul.
          \par
          On approche le volume du solide de section BCEF par la somme des volumes des $n$ cylindres
          ainsi créés en choisissant une valeur de $n$ suffisamment grande.
          \par
          Recopier et compléter l'algorithme suivant de sorte qu'à la fin de son exécution, la variable $V$ contienne la somme des volumes des $n$ cylindres créés lorsque l'on saisit $n$.
          \begin{center}
               \begin{extern}%width="300" alt="Algorithme Bac S Polynésie 2018"
                    \begin{tabularx}{0.45\linewidth}{|l X|}\hline
                         1&$V \gets 0$\\
                         2& Pour $k$ allant de \ldots à \ldots~:\\
                         3& \hspace{0.5cm} $V \gets \ldots$\\
                         4& Fin Pour\\ \hline
                    \end{tabularx}
               \end{extern}
          \end{center}
     \end{enumerate}
\end{enumerate}

\end{document}