\documentclass[a4paper]{article}

%================================================================================================================================
%
% Packages
%
%================================================================================================================================

\usepackage[T1]{fontenc} 	% pour caractères accentués
\usepackage[utf8]{inputenc}  % encodage utf8
\usepackage[french]{babel}	% langue : français
\usepackage{fourier}			% caractères plus lisibles
\usepackage[dvipsnames]{xcolor} % couleurs
\usepackage{fancyhdr}		% réglage header footer
\usepackage{needspace}		% empêcher sauts de page mal placés
\usepackage{graphicx}		% pour inclure des graphiques
\usepackage{enumitem,cprotect}		% personnalise les listes d'items (nécessaire pour ol, al ...)
\usepackage{hyperref}		% Liens hypertexte
\usepackage{pstricks,pst-all,pst-node,pstricks-add,pst-math,pst-plot,pst-tree,pst-eucl} % pstricks
\usepackage[a4paper,includeheadfoot,top=2cm,left=3cm, bottom=2cm,right=3cm]{geometry} % marges etc.
\usepackage{comment}			% commentaires multilignes
\usepackage{amsmath,environ} % maths (matrices, etc.)
\usepackage{amssymb,makeidx}
\usepackage{bm}				% bold maths
\usepackage{tabularx}		% tableaux
\usepackage{colortbl}		% tableaux en couleur
\usepackage{fontawesome}		% Fontawesome
\usepackage{environ}			% environment with command
\usepackage{fp}				% calculs pour ps-tricks
\usepackage{multido}			% pour ps tricks
\usepackage[np]{numprint}	% formattage nombre
\usepackage{tikz,tkz-tab} 			% package principal TikZ
\usepackage{pgfplots}   % axes
\usepackage{mathrsfs}    % cursives
\usepackage{calc}			% calcul taille boites
\usepackage[scaled=0.875]{helvet} % font sans serif
\usepackage{svg} % svg
\usepackage{scrextend} % local margin
\usepackage{scratch} %scratch
\usepackage{multicol} % colonnes
%\usepackage{infix-RPN,pst-func} % formule en notation polanaise inversée
\usepackage{listings}

%================================================================================================================================
%
% Réglages de base
%
%================================================================================================================================

\lstset{
language=Python,   % R code
literate=
{á}{{\'a}}1
{à}{{\`a}}1
{ã}{{\~a}}1
{é}{{\'e}}1
{è}{{\`e}}1
{ê}{{\^e}}1
{í}{{\'i}}1
{ó}{{\'o}}1
{õ}{{\~o}}1
{ú}{{\'u}}1
{ü}{{\"u}}1
{ç}{{\c{c}}}1
{~}{{ }}1
}


\definecolor{codegreen}{rgb}{0,0.6,0}
\definecolor{codegray}{rgb}{0.5,0.5,0.5}
\definecolor{codepurple}{rgb}{0.58,0,0.82}
\definecolor{backcolour}{rgb}{0.95,0.95,0.92}

\lstdefinestyle{mystyle}{
    backgroundcolor=\color{backcolour},   
    commentstyle=\color{codegreen},
    keywordstyle=\color{magenta},
    numberstyle=\tiny\color{codegray},
    stringstyle=\color{codepurple},
    basicstyle=\ttfamily\footnotesize,
    breakatwhitespace=false,         
    breaklines=true,                 
    captionpos=b,                    
    keepspaces=true,                 
    numbers=left,                    
xleftmargin=2em,
framexleftmargin=2em,            
    showspaces=false,                
    showstringspaces=false,
    showtabs=false,                  
    tabsize=2,
    upquote=true
}

\lstset{style=mystyle}


\lstset{style=mystyle}
\newcommand{\imgdir}{C:/laragon/www/newmc/assets/imgsvg/}
\newcommand{\imgsvgdir}{C:/laragon/www/newmc/assets/imgsvg/}

\definecolor{mcgris}{RGB}{220, 220, 220}% ancien~; pour compatibilité
\definecolor{mcbleu}{RGB}{52, 152, 219}
\definecolor{mcvert}{RGB}{125, 194, 70}
\definecolor{mcmauve}{RGB}{154, 0, 215}
\definecolor{mcorange}{RGB}{255, 96, 0}
\definecolor{mcturquoise}{RGB}{0, 153, 153}
\definecolor{mcrouge}{RGB}{255, 0, 0}
\definecolor{mclightvert}{RGB}{205, 234, 190}

\definecolor{gris}{RGB}{220, 220, 220}
\definecolor{bleu}{RGB}{52, 152, 219}
\definecolor{vert}{RGB}{125, 194, 70}
\definecolor{mauve}{RGB}{154, 0, 215}
\definecolor{orange}{RGB}{255, 96, 0}
\definecolor{turquoise}{RGB}{0, 153, 153}
\definecolor{rouge}{RGB}{255, 0, 0}
\definecolor{lightvert}{RGB}{205, 234, 190}
\setitemize[0]{label=\color{lightvert}  $\bullet$}

\pagestyle{fancy}
\renewcommand{\headrulewidth}{0.2pt}
\fancyhead[L]{maths-cours.fr}
\fancyhead[R]{\thepage}
\renewcommand{\footrulewidth}{0.2pt}
\fancyfoot[C]{}

\newcolumntype{C}{>{\centering\arraybackslash}X}
\newcolumntype{s}{>{\hsize=.35\hsize\arraybackslash}X}

\setlength{\parindent}{0pt}		 
\setlength{\parskip}{3mm}
\setlength{\headheight}{1cm}

\def\ebook{ebook}
\def\book{book}
\def\web{web}
\def\type{web}

\newcommand{\vect}[1]{\overrightarrow{\,\mathstrut#1\,}}

\def\Oij{$\left(\text{O}~;~\vect{\imath},~\vect{\jmath}\right)$}
\def\Oijk{$\left(\text{O}~;~\vect{\imath},~\vect{\jmath},~\vect{k}\right)$}
\def\Ouv{$\left(\text{O}~;~\vect{u},~\vect{v}\right)$}

\hypersetup{breaklinks=true, colorlinks = true, linkcolor = OliveGreen, urlcolor = OliveGreen, citecolor = OliveGreen, pdfauthor={Didier BONNEL - https://www.maths-cours.fr} } % supprime les bordures autour des liens

\renewcommand{\arg}[0]{\text{arg}}

\everymath{\displaystyle}

%================================================================================================================================
%
% Macros - Commandes
%
%================================================================================================================================

\newcommand\meta[2]{    			% Utilisé pour créer le post HTML.
	\def\titre{titre}
	\def\url{url}
	\def\arg{#1}
	\ifx\titre\arg
		\newcommand\maintitle{#2}
		\fancyhead[L]{#2}
		{\Large\sffamily \MakeUppercase{#2}}
		\vspace{1mm}\textcolor{mcvert}{\hrule}
	\fi 
	\ifx\url\arg
		\fancyfoot[L]{\href{https://www.maths-cours.fr#2}{\black \footnotesize{https://www.maths-cours.fr#2}}}
	\fi 
}


\newcommand\TitreC[1]{    		% Titre centré
     \needspace{3\baselineskip}
     \begin{center}\textbf{#1}\end{center}
}

\newcommand\newpar{    		% paragraphe
     \par
}

\newcommand\nosp {    		% commande vide (pas d'espace)
}
\newcommand{\id}[1]{} %ignore

\newcommand\boite[2]{				% Boite simple sans titre
	\vspace{5mm}
	\setlength{\fboxrule}{0.2mm}
	\setlength{\fboxsep}{5mm}	
	\fcolorbox{#1}{#1!3}{\makebox[\linewidth-2\fboxrule-2\fboxsep]{
  		\begin{minipage}[t]{\linewidth-2\fboxrule-4\fboxsep}\setlength{\parskip}{3mm}
  			 #2
  		\end{minipage}
	}}
	\vspace{5mm}
}

\newcommand\CBox[4]{				% Boites
	\vspace{5mm}
	\setlength{\fboxrule}{0.2mm}
	\setlength{\fboxsep}{5mm}
	
	\fcolorbox{#1}{#1!3}{\makebox[\linewidth-2\fboxrule-2\fboxsep]{
		\begin{minipage}[t]{1cm}\setlength{\parskip}{3mm}
	  		\textcolor{#1}{\LARGE{#2}}    
 	 	\end{minipage}  
  		\begin{minipage}[t]{\linewidth-2\fboxrule-4\fboxsep}\setlength{\parskip}{3mm}
			\raisebox{1.2mm}{\normalsize\sffamily{\textcolor{#1}{#3}}}						
  			 #4
  		\end{minipage}
	}}
	\vspace{5mm}
}

\newcommand\cadre[3]{				% Boites convertible html
	\par
	\vspace{2mm}
	\setlength{\fboxrule}{0.1mm}
	\setlength{\fboxsep}{5mm}
	\fcolorbox{#1}{white}{\makebox[\linewidth-2\fboxrule-2\fboxsep]{
  		\begin{minipage}[t]{\linewidth-2\fboxrule-4\fboxsep}\setlength{\parskip}{3mm}
			\raisebox{-2.5mm}{\sffamily \small{\textcolor{#1}{\MakeUppercase{#2}}}}		
			\par		
  			 #3
 	 		\end{minipage}
	}}
		\vspace{2mm}
	\par
}

\newcommand\bloc[3]{				% Boites convertible html sans bordure
     \needspace{2\baselineskip}
     {\sffamily \small{\textcolor{#1}{\MakeUppercase{#2}}}}    
		\par		
  			 #3
		\par
}

\newcommand\CHelp[1]{
     \CBox{Plum}{\faInfoCircle}{À RETENIR}{#1}
}

\newcommand\CUp[1]{
     \CBox{NavyBlue}{\faThumbsOUp}{EN PRATIQUE}{#1}
}

\newcommand\CInfo[1]{
     \CBox{Sepia}{\faArrowCircleRight}{REMARQUE}{#1}
}

\newcommand\CRedac[1]{
     \CBox{PineGreen}{\faEdit}{BIEN R\'EDIGER}{#1}
}

\newcommand\CError[1]{
     \CBox{Red}{\faExclamationTriangle}{ATTENTION}{#1}
}

\newcommand\TitreExo[2]{
\needspace{4\baselineskip}
 {\sffamily\large EXERCICE #1\ (\emph{#2 points})}
\vspace{5mm}
}

\newcommand\img[2]{
          \includegraphics[width=#2\paperwidth]{\imgdir#1}
}

\newcommand\imgsvg[2]{
       \begin{center}   \includegraphics[width=#2\paperwidth]{\imgsvgdir#1} \end{center}
}


\newcommand\Lien[2]{
     \href{#1}{#2 \tiny \faExternalLink}
}
\newcommand\mcLien[2]{
     \href{https~://www.maths-cours.fr/#1}{#2 \tiny \faExternalLink}
}

\newcommand{\euro}{\eurologo{}}

%================================================================================================================================
%
% Macros - Environement
%
%================================================================================================================================

\newenvironment{tex}{ %
}
{%
}

\newenvironment{indente}{ %
	\setlength\parindent{10mm}
}

{
	\setlength\parindent{0mm}
}

\newenvironment{corrige}{%
     \needspace{3\baselineskip}
     \medskip
     \textbf{\textsc{Corrigé}}
     \medskip
}
{
}

\newenvironment{extern}{%
     \begin{center}
     }
     {
     \end{center}
}

\NewEnviron{code}{%
	\par
     \boite{gray}{\texttt{%
     \BODY
     }}
     \par
}

\newenvironment{vbloc}{% boite sans cadre empeche saut de page
     \begin{minipage}[t]{\linewidth}
     }
     {
     \end{minipage}
}
\NewEnviron{h2}{%
    \needspace{3\baselineskip}
    \vspace{0.6cm}
	\noindent \MakeUppercase{\sffamily \large \BODY}
	\vspace{1mm}\textcolor{mcgris}{\hrule}\vspace{0.4cm}
	\par
}{}

\NewEnviron{h3}{%
    \needspace{3\baselineskip}
	\vspace{5mm}
	\textsc{\BODY}
	\par
}

\NewEnviron{margeneg}{ %
\begin{addmargin}[-1cm]{0cm}
\BODY
\end{addmargin}
}

\NewEnviron{html}{%
}

\begin{document}
\meta{url}{/cours/geometrie-espace/}
\meta{pid}{219}
\meta{titre}{Géométrie dans l'espace}
\meta{type}{cours}

\cadre{bleu}{Définition}{%
     Un \textbf{prisme droit} est un solide ayant deux bases polygonales identiques et dont les faces latérales  sont des rectangles.
}
\begin{center}
     \begin{extern}%width="250" alt="prisme droit"
          \resizebox{7cm}{!}{
               \psset{xunit=1.0cm,yunit=1.0cm,algebraic=true,dimen=middle,dotstyle=o,dotsize=5pt 0,linewidth=1.pt,arrowsize=3pt 2,arrowinset=0.25}
               \begin{pspicture*}(-1.,0.)(9.,10.)
                    \pspolygon[linecolor=blue,fillcolor=blue,fillstyle=solid,opacity=0.1](1.,8.)(3.,6.5)(7.,7.)(8.,8.)(4.,9.)
                    \pspolygon[linecolor=blue,fillcolor=blue,fillstyle=solid,opacity=0.1](1.,2.)(3.,0.5)(7.,1.)(8.,2.)(4.,3.)
                    \psline(1.,8.)(1.,2.)
                    \psline(3.,6.5)(3.,0.5)
                    \psline(7.,7.)(7.,1.)
                    \psline(8.,8.)(8.,2.)
                    \psline[linestyle=dashed,dash=2pt 2pt](4.,9.)(4.,3.)
                    \rput(4.,2.){$\Large \blue{\mathscr{B}}$}
                    \psline{<->}(0.,8.)(0.,2.)
                    \rput(-0.5,5.){$\Large h$}
               \end{pspicture*}
          }
     \end{extern}
\end{center}
\begin{center}
     \textit{Prisme droit de hauteur $h$}
\end{center}
\cadre{vert}{Propriété}{%
     Si on désigne par $h$ la hauteur du prisme et $\mathscr B$ l'aire de la base, le volume du prisme est égal à :
     \begin{center}$V=\mathscr B\times h$\end{center}
}
\bloc{cyan}{Cas particuliers}{%
     \begin{itemize}
          \item Si le volume est un \textbf{pavé droit} (\textit{parallélépipède rectangle}) de dimensions $l, L, h$,  la base est un rectangle de largeur $l$ et de longueur $L$. Le volume vaut alors $V=L\times l\times h$
          \item Si le volume est un \textbf{cube} dont le côté mesure $c$, la base est un carré de côté $c$. Le volume vaut alors $V=c^3$
     \end{itemize}
}
\cadre{vert}{Propriété}{%
     Le volume d'un cylindre de révolution est égal à :
     \begin{center}$V=\mathscr B\times h$\end{center}
     où
     \begin{itemize}
          \item $h$ est la hauteur du cylindre de révolution
          \item $\mathscr B = \pi R^2$ est l'aire de la base de rayon $R$.
     \end{itemize}
}
\begin{center}
     \begin{extern}%width="250" alt="cylindre de révolution"
          \resizebox{7cm}{!}{
               \psset{xunit=1.0cm,yunit=1.0cm,algebraic=true,dimen=middle,dotstyle=o,dotsize=5pt 0,linewidth=1.pt,arrowsize=3pt 2,arrowinset=0.25}
               \begin{pspicture*}(-1.,1.)(9.,10.)
                    \psline(1.,8.)(1.,2.)
                    \psline(7.,8.)(7.,2.)
                    \rput{-0.172}(4,1.94){\psellipse[linecolor=blue,fillcolor=blue,fillstyle=solid,opacity=0.1](0,0)(3.,0.59)}
                    \rput{-0.172}(4,7.94){\psellipse[linecolor=blue,fillcolor=blue,fillstyle=solid,opacity=0.1](0,0)(3.,0.59)}
                    \rput(4.,2.){$\Large \blue{\mathscr{B}}$}
                    \psline{<->}(0.,8.)(0.,2.)
                    \rput(-0.5,5.){$\Large h$}
               \end{pspicture*}
          }
     \end{extern}
\end{center}
\begin{center}
     \textit{Cylindre de révolution de hauteur $h$}
\end{center}
\cadre{bleu}{Définition}{%
     Une \textbf{pyramide} est un solide ayant une base polygonale, un sommet et dont les faces latérales sont des triangles.
}
\begin{center}
     \begin{extern}%width="250" alt="pyramide"
          \resizebox{7cm}{!}{
               \psset{xunit=1.0cm,yunit=1.0cm,algebraic=true,dimen=middle,dotstyle=o,dotsize=5pt 0,linewidth=1.pt,arrowsize=3pt 2,arrowinset=0.25}
               \begin{pspicture*}(-1.,0.5)(9.,10.)
                    \pspolygon[linecolor=blue,fillcolor=blue,fillstyle=solid,opacity=0.1](1.,2.)(3.,0.5)(7.,1.)(8.,2.)(4.,3.)
                    \psline(4.,8.)(1.,2.)
                    \psline(4.,8.)(3.,0.5)
                    \psline(4.,8.)(7.,1.)
                    \psline(4.,8.)(8.,2.)
                    \psline[linestyle=dashed,dash=2pt 2pt](4.,8.)(4.,3.)
                    \rput(4.,2.){$\Large \blue{\mathscr{B}}$}
                    \psline{<->}(0.,8.)(0.,2.)
                    \rput(-0.5,5.){$\Large h$}
               \end{pspicture*}
          }
     \end{extern}
\end{center}
\begin{center}
     \textit{Pyramide de hauteur $h$}
\end{center}
\cadre{vert}{Propriété}{%
     Si on désigne par $h$ la hauteur de la pyramide et $\mathscr B$ l'aire de la base, le volume de la pyramide est égal à :
     \begin{center}$V=\frac{1}{3}\times \mathscr B\times h$\end{center}
}
\cadre{vert}{Propriété}{%
     Le volume d'un cône de révolution est égal à :
     \begin{center}$V=\frac{1}{3}\times \mathscr B\times h$\end{center}
     où
     \begin{itemize}
          \item $h$ est la hauteur du cône
          \item $\mathscr B = \pi R^2$ est l'aire de la base de rayon $R$.
     \end{itemize}
}
\begin{center}
     \begin{extern}%width="250" alt="cône de révolution"
          \resizebox{7cm}{!}{
               \psset{xunit=1.0cm,yunit=1.0cm,algebraic=true,dimen=middle,dotstyle=o,dotsize=5pt 0,linewidth=1.pt,arrowsize=3pt 2,arrowinset=0.25}
               \begin{pspicture*}(-1.,1)(9.,9.)
                    \psline(4.,8.)(1.,2.)
                    \psline(4.,8.)(7.,2.)
                    \rput{-0.172}(4,1.94){\psellipse[linecolor=blue,fillcolor=blue,fillstyle=solid,opacity=0.1](0,0)(3.,0.59)}
                    \rput(4.,2.){$\Large \blue{\mathscr{B}}$}
                    \psline{<->}(0.,8.)(0.,2.)
                    \rput(-0.5,5.){$\Large h$}
               \end{pspicture*}
          }
     \end{extern}
\end{center}
\begin{center}
     \textit{Cône de révolution de hauteur $h$}
\end{center}
\cadre{vert}{Propriété}{%
     Le volume d'une sphère de rayon $r$ est égal à :
     \begin{center}$V = \frac{4}{3}\times \pi \times r^3$ \end{center}
}
\begin{center}
     \begin{extern}%width="220" alt="sphère"
          \resizebox{7cm}{!}{
               \psset{xunit=1.0cm,yunit=1.0cm,algebraic=true,dimen=middle,dotstyle=o,dotsize=5pt 0,linewidth=1.pt,arrowsize=3pt 2,arrowinset=0.25}
               \begin{pspicture*}(0.,-1.2)(8.5,5.)
                    \newrgbcolor{lblue}{0.9 0.9 1}
                    \pscircle[linecolor=blue,GradientPos={(3,3.5)},fillstyle=gradient,gradmidpoint=0.01,GradientCircle=true, gradbegin=lblue, gradend=white](4.,2.){3}
                    \psline[linecolor=red](4.,2.)(4.28,1.36)
                    \rput[l](4.3,1.7){\red{$r$}}
                    \rput{-0.172}(4,1.94){\psellipse[linecolor=blue,fillcolor=blue,fillstyle=solid,opacity=0.05](0,0)(3.,0.59)}
                    \psdots[dotsize=3pt,dotstyle=*,linecolor=blue](4,2)
                    \rput(3.8,2.2){$\Large \blue{O}$}
               \end{pspicture*}
          }
     \end{extern}
\end{center}
\begin{center}
     \textit{Sphère de rayon $r$}
\end{center}

\end{document}