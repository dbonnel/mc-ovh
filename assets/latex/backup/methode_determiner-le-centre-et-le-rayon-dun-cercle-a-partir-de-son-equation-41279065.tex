\documentclass[a4paper]{article}

%================================================================================================================================
%
% Packages
%
%================================================================================================================================

\usepackage[T1]{fontenc} 	% pour caractères accentués
\usepackage[utf8]{inputenc}  % encodage utf8
\usepackage[french]{babel}	% langue : français
\usepackage{fourier}			% caractères plus lisibles
\usepackage[dvipsnames]{xcolor} % couleurs
\usepackage{fancyhdr}		% réglage header footer
\usepackage{needspace}		% empêcher sauts de page mal placés
\usepackage{graphicx}		% pour inclure des graphiques
\usepackage{enumitem,cprotect}		% personnalise les listes d'items (nécessaire pour ol, al ...)
\usepackage{hyperref}		% Liens hypertexte
\usepackage{pstricks,pst-all,pst-node,pstricks-add,pst-math,pst-plot,pst-tree,pst-eucl} % pstricks
\usepackage[a4paper,includeheadfoot,top=2cm,left=3cm, bottom=2cm,right=3cm]{geometry} % marges etc.
\usepackage{comment}			% commentaires multilignes
\usepackage{amsmath,environ} % maths (matrices, etc.)
\usepackage{amssymb,makeidx}
\usepackage{bm}				% bold maths
\usepackage{tabularx}		% tableaux
\usepackage{colortbl}		% tableaux en couleur
\usepackage{fontawesome}		% Fontawesome
\usepackage{environ}			% environment with command
\usepackage{fp}				% calculs pour ps-tricks
\usepackage{multido}			% pour ps tricks
\usepackage[np]{numprint}	% formattage nombre
\usepackage{tikz,tkz-tab} 			% package principal TikZ
\usepackage{pgfplots}   % axes
\usepackage{mathrsfs}    % cursives
\usepackage{calc}			% calcul taille boites
\usepackage[scaled=0.875]{helvet} % font sans serif
\usepackage{svg} % svg
\usepackage{scrextend} % local margin
\usepackage{scratch} %scratch
\usepackage{multicol} % colonnes
%\usepackage{infix-RPN,pst-func} % formule en notation polanaise inversée
\usepackage{listings}

%================================================================================================================================
%
% Réglages de base
%
%================================================================================================================================

\lstset{
language=Python,   % R code
literate=
{á}{{\'a}}1
{à}{{\`a}}1
{ã}{{\~a}}1
{é}{{\'e}}1
{è}{{\`e}}1
{ê}{{\^e}}1
{í}{{\'i}}1
{ó}{{\'o}}1
{õ}{{\~o}}1
{ú}{{\'u}}1
{ü}{{\"u}}1
{ç}{{\c{c}}}1
{~}{{ }}1
}


\definecolor{codegreen}{rgb}{0,0.6,0}
\definecolor{codegray}{rgb}{0.5,0.5,0.5}
\definecolor{codepurple}{rgb}{0.58,0,0.82}
\definecolor{backcolour}{rgb}{0.95,0.95,0.92}

\lstdefinestyle{mystyle}{
    backgroundcolor=\color{backcolour},   
    commentstyle=\color{codegreen},
    keywordstyle=\color{magenta},
    numberstyle=\tiny\color{codegray},
    stringstyle=\color{codepurple},
    basicstyle=\ttfamily\footnotesize,
    breakatwhitespace=false,         
    breaklines=true,                 
    captionpos=b,                    
    keepspaces=true,                 
    numbers=left,                    
xleftmargin=2em,
framexleftmargin=2em,            
    showspaces=false,                
    showstringspaces=false,
    showtabs=false,                  
    tabsize=2,
    upquote=true
}

\lstset{style=mystyle}


\lstset{style=mystyle}
\newcommand{\imgdir}{C:/laragon/www/newmc/assets/imgsvg/}
\newcommand{\imgsvgdir}{C:/laragon/www/newmc/assets/imgsvg/}

\definecolor{mcgris}{RGB}{220, 220, 220}% ancien~; pour compatibilité
\definecolor{mcbleu}{RGB}{52, 152, 219}
\definecolor{mcvert}{RGB}{125, 194, 70}
\definecolor{mcmauve}{RGB}{154, 0, 215}
\definecolor{mcorange}{RGB}{255, 96, 0}
\definecolor{mcturquoise}{RGB}{0, 153, 153}
\definecolor{mcrouge}{RGB}{255, 0, 0}
\definecolor{mclightvert}{RGB}{205, 234, 190}

\definecolor{gris}{RGB}{220, 220, 220}
\definecolor{bleu}{RGB}{52, 152, 219}
\definecolor{vert}{RGB}{125, 194, 70}
\definecolor{mauve}{RGB}{154, 0, 215}
\definecolor{orange}{RGB}{255, 96, 0}
\definecolor{turquoise}{RGB}{0, 153, 153}
\definecolor{rouge}{RGB}{255, 0, 0}
\definecolor{lightvert}{RGB}{205, 234, 190}
\setitemize[0]{label=\color{lightvert}  $\bullet$}

\pagestyle{fancy}
\renewcommand{\headrulewidth}{0.2pt}
\fancyhead[L]{maths-cours.fr}
\fancyhead[R]{\thepage}
\renewcommand{\footrulewidth}{0.2pt}
\fancyfoot[C]{}

\newcolumntype{C}{>{\centering\arraybackslash}X}
\newcolumntype{s}{>{\hsize=.35\hsize\arraybackslash}X}

\setlength{\parindent}{0pt}		 
\setlength{\parskip}{3mm}
\setlength{\headheight}{1cm}

\def\ebook{ebook}
\def\book{book}
\def\web{web}
\def\type{web}

\newcommand{\vect}[1]{\overrightarrow{\,\mathstrut#1\,}}

\def\Oij{$\left(\text{O}~;~\vect{\imath},~\vect{\jmath}\right)$}
\def\Oijk{$\left(\text{O}~;~\vect{\imath},~\vect{\jmath},~\vect{k}\right)$}
\def\Ouv{$\left(\text{O}~;~\vect{u},~\vect{v}\right)$}

\hypersetup{breaklinks=true, colorlinks = true, linkcolor = OliveGreen, urlcolor = OliveGreen, citecolor = OliveGreen, pdfauthor={Didier BONNEL - https://www.maths-cours.fr} } % supprime les bordures autour des liens

\renewcommand{\arg}[0]{\text{arg}}

\everymath{\displaystyle}

%================================================================================================================================
%
% Macros - Commandes
%
%================================================================================================================================

\newcommand\meta[2]{    			% Utilisé pour créer le post HTML.
	\def\titre{titre}
	\def\url{url}
	\def\arg{#1}
	\ifx\titre\arg
		\newcommand\maintitle{#2}
		\fancyhead[L]{#2}
		{\Large\sffamily \MakeUppercase{#2}}
		\vspace{1mm}\textcolor{mcvert}{\hrule}
	\fi 
	\ifx\url\arg
		\fancyfoot[L]{\href{https://www.maths-cours.fr#2}{\black \footnotesize{https://www.maths-cours.fr#2}}}
	\fi 
}


\newcommand\TitreC[1]{    		% Titre centré
     \needspace{3\baselineskip}
     \begin{center}\textbf{#1}\end{center}
}

\newcommand\newpar{    		% paragraphe
     \par
}

\newcommand\nosp {    		% commande vide (pas d'espace)
}
\newcommand{\id}[1]{} %ignore

\newcommand\boite[2]{				% Boite simple sans titre
	\vspace{5mm}
	\setlength{\fboxrule}{0.2mm}
	\setlength{\fboxsep}{5mm}	
	\fcolorbox{#1}{#1!3}{\makebox[\linewidth-2\fboxrule-2\fboxsep]{
  		\begin{minipage}[t]{\linewidth-2\fboxrule-4\fboxsep}\setlength{\parskip}{3mm}
  			 #2
  		\end{minipage}
	}}
	\vspace{5mm}
}

\newcommand\CBox[4]{				% Boites
	\vspace{5mm}
	\setlength{\fboxrule}{0.2mm}
	\setlength{\fboxsep}{5mm}
	
	\fcolorbox{#1}{#1!3}{\makebox[\linewidth-2\fboxrule-2\fboxsep]{
		\begin{minipage}[t]{1cm}\setlength{\parskip}{3mm}
	  		\textcolor{#1}{\LARGE{#2}}    
 	 	\end{minipage}  
  		\begin{minipage}[t]{\linewidth-2\fboxrule-4\fboxsep}\setlength{\parskip}{3mm}
			\raisebox{1.2mm}{\normalsize\sffamily{\textcolor{#1}{#3}}}						
  			 #4
  		\end{minipage}
	}}
	\vspace{5mm}
}

\newcommand\cadre[3]{				% Boites convertible html
	\par
	\vspace{2mm}
	\setlength{\fboxrule}{0.1mm}
	\setlength{\fboxsep}{5mm}
	\fcolorbox{#1}{white}{\makebox[\linewidth-2\fboxrule-2\fboxsep]{
  		\begin{minipage}[t]{\linewidth-2\fboxrule-4\fboxsep}\setlength{\parskip}{3mm}
			\raisebox{-2.5mm}{\sffamily \small{\textcolor{#1}{\MakeUppercase{#2}}}}		
			\par		
  			 #3
 	 		\end{minipage}
	}}
		\vspace{2mm}
	\par
}

\newcommand\bloc[3]{				% Boites convertible html sans bordure
     \needspace{2\baselineskip}
     {\sffamily \small{\textcolor{#1}{\MakeUppercase{#2}}}}    
		\par		
  			 #3
		\par
}

\newcommand\CHelp[1]{
     \CBox{Plum}{\faInfoCircle}{À RETENIR}{#1}
}

\newcommand\CUp[1]{
     \CBox{NavyBlue}{\faThumbsOUp}{EN PRATIQUE}{#1}
}

\newcommand\CInfo[1]{
     \CBox{Sepia}{\faArrowCircleRight}{REMARQUE}{#1}
}

\newcommand\CRedac[1]{
     \CBox{PineGreen}{\faEdit}{BIEN R\'EDIGER}{#1}
}

\newcommand\CError[1]{
     \CBox{Red}{\faExclamationTriangle}{ATTENTION}{#1}
}

\newcommand\TitreExo[2]{
\needspace{4\baselineskip}
 {\sffamily\large EXERCICE #1\ (\emph{#2 points})}
\vspace{5mm}
}

\newcommand\img[2]{
          \includegraphics[width=#2\paperwidth]{\imgdir#1}
}

\newcommand\imgsvg[2]{
       \begin{center}   \includegraphics[width=#2\paperwidth]{\imgsvgdir#1} \end{center}
}


\newcommand\Lien[2]{
     \href{#1}{#2 \tiny \faExternalLink}
}
\newcommand\mcLien[2]{
     \href{https~://www.maths-cours.fr/#1}{#2 \tiny \faExternalLink}
}

\newcommand{\euro}{\eurologo{}}

%================================================================================================================================
%
% Macros - Environement
%
%================================================================================================================================

\newenvironment{tex}{ %
}
{%
}

\newenvironment{indente}{ %
	\setlength\parindent{10mm}
}

{
	\setlength\parindent{0mm}
}

\newenvironment{corrige}{%
     \needspace{3\baselineskip}
     \medskip
     \textbf{\textsc{Corrigé}}
     \medskip
}
{
}

\newenvironment{extern}{%
     \begin{center}
     }
     {
     \end{center}
}

\NewEnviron{code}{%
	\par
     \boite{gray}{\texttt{%
     \BODY
     }}
     \par
}

\newenvironment{vbloc}{% boite sans cadre empeche saut de page
     \begin{minipage}[t]{\linewidth}
     }
     {
     \end{minipage}
}
\NewEnviron{h2}{%
    \needspace{3\baselineskip}
    \vspace{0.6cm}
	\noindent \MakeUppercase{\sffamily \large \BODY}
	\vspace{1mm}\textcolor{mcgris}{\hrule}\vspace{0.4cm}
	\par
}{}

\NewEnviron{h3}{%
    \needspace{3\baselineskip}
	\vspace{5mm}
	\textsc{\BODY}
	\par
}

\NewEnviron{margeneg}{ %
\begin{addmargin}[-1cm]{0cm}
\BODY
\end{addmargin}
}

\NewEnviron{html}{%
}

\begin{document}
\meta{url}{/methode/determiner-le-centre-et-le-rayon-dun-cercle-a-partir-de-son-equation/}
\meta{pid}{11187}
\meta{titre}{Déterminer le centre et le rayon d'un cercle à partir de son équation}
\meta{type}{methode}
%
\cadre{rouge}{Méthode}{ % id=m010
     Dans cette fiche, on cherchera à déterminer si une équation du type~:
     \[
     x^2 +y^2 +ax+by+c=0
     \]
     correspond à l'équation d'un cercle et, si c'est le cas, à déterminer les coordonnées du centre et du rayon de ce cercle.
}% fin méthode
On utilisera, pour cela, le \mcLien{https://www.maths-cours.fr/cours/produit-scalaire\#t110}{résultat suivant}~:
\cadre{bleu}{Rappel}{ % id=r015
     Le plan est rapporté à un repère orthonormé $\left(O, \vec{i}, \vec{j}\right)$.
     \par
     Soit $ \Omega ( \alpha  ;  \beta) $ un point quelconque du plan et $r$ un réel positif.
     \par
     Une équation du cercle de centre $ \Omega $ et de rayon $r$ est :
     \[
     \left(x- \alpha \right)^{2}+\left(y- \beta \right)^{2}=r^{2}
     \]
}%
\textbf{L'objectif} sera donc de chercher si l'équation $x^2 +y^2 +ax+by+c=0 $ peut s'écrire sous la forme $\left(x- \alpha \right)^{2}+\left(y- \beta \right)^{2}=r^{2}$.
\medbreak
\textbf{En pratique}, pour déterminer si l'ensemble cherché est un cercle et déterminer les éléments caractéristiques de ce cercle, on procède de la manière suivante~:
\begin{enumerate}
     \item
     on regroupe les termes en  $x$ et les termes en $y$
     \item
     on fait apparaître des carrés de la forme $(x -  \alpha )^2 $ et $(y -  \beta )^2 $ en utilisant des identités remarquables
     \item
     On écrit l'équation sous la forme $(x- \alpha )^2 +(y -  \beta )^2 = \gamma $ ;\\
     À partir de cette équation~:
     \begin{itemize}
          \item
          si $ \gamma >0$ , l'ensemble cherché est un cercle de centre  $ \Omega ( \alpha ; \beta )$ et de rayon $r=\sqrt{ \gamma }$ .
          \item
          si $ \gamma =0$ , l'ensemble cherché est le point $ \Omega ( \alpha ; \beta )$
          \item
          si $ \gamma <0$ , l'ensemble cherché est l'ensemble vide.
     \end{itemize}
\end{enumerate}
\bloc{orange}{Exemple 1}{ % id=e020
     \cadre{vert}{Énoncé}{ % id=r020
          Déterminer l'ensemble des points  $M$ tels que~:
          \[
          x^2 +y^2 +2x - 4y=0
          \]
     }% fin énoncé
     \begin{enumerate}
          \item
          En regroupant les termes en $x$ et les termes en $y$ on obtient~:\\
          $x^2 +2x+ y^2  - 4y=0$
          \item
          $x^2 +2x$ est le début de l'identité remarquable~:  $x^2 +2x+1=(x+1)^2.$ \\
          Donc $x^2 +2x=(x+1)^2 - 1$
          \newpar
          De la même manière, $y^2  - 4y$ est le début de l'identité remarquable~: $y^2  - 4y+4=(y - 2)^2 $ \\
          Donc $y^2  - 4y=(y - 2)^2  - 4$
          \item
          En utilisant les résultats précédents, l'équation de départ peut s'écrire sous la forme~:\\
          $(x+1)^2 - 1+(y - 2)^2  - 4=0$
          \newpar
          C'est-à-dire~:\\
          $(x+1)^2+(y - 2)^2 =5$ \\
          $(x - ( - 1))^2+(y - 2)^2 =\left(\sqrt{5}\right)^2 $
          \newpar
          L'ensemble des points  $M$ cherché est donc le cercle de centre  $ \Omega ( - 1~;~2)$ et de rayon  $\sqrt{5}.$
          \begin{center}
\imgsvg{determiner-le-centre-et-le-rayon-dun-cercle}{0.3}% alt="Déterminer le centre et le rayon d'un cercle" style="width:37rem" class="aligncenter"          \end{center}
          \textbf{Remarque} : Ce cercle passe par l'origine du repère puisque l'équation est vérifiée pour $x=0$ et $y=0.$
     \end{enumerate}
}% fin exemple
\bloc{orange}{Exemple 2}{ % id=e030
     \cadre{vert}{Énoncé}{ % id=r030
          Quel est l'ensemble des points  $M( x ; y )$ dont les coordonnées vérifient~:
          \[
          x^2 +y^2 +x - 3y+7=0 \quad ?
          \]
     }% fin énoncé
     \begin{enumerate}
          \item
          On commence par regrouper les termes en $x$ et les termes en $y$~:\\
          $x^2 +x+ y^2  - 3y + 7=0$
          \item
          $x^2 +x$ est le début de l'identité remarquable~:  $x^2 +x+\frac{ 1 }{ 4 }=\left(x+\frac{ 1 }{ 2 }\right)^2$ \\
          Donc $x^2 +x=\left(x+\frac{ 1 }{ 2 }\right)^2 - \frac{ 1 }{ 4 } $
          \newpar
          $y^2  - 3y$ est le début de l'identité remarquable~: $y^2  - 3y+\frac{ 9 }{ 4 }=\left(y - \frac{ 3 }{ 2 }\right)^2 $ \\
          Ce qui donne : $y^2  - 3y=\left(y - \frac{ 3 }{ 2 }\right)^2  - \frac{ 9 }{ 4 } $
          \item
          Finalement, notre équation peut s'écrire~:\\
          $\left(x+\frac{ 1 }{ 2 }\right)^2 - \frac{ 1 }{ 4 } +\left(y - \frac{ 3 }{ 2 }\right)^2  - \frac{ 9 }{ 4 } + 7 =0$
          \newpar
          Ou encore~:\\
          $\left(x+\frac{ 1 }{ 2 }\right)^2  + \left(y - \frac{ 3 }{ 2 }\right)^2 =  - \frac{ 9 }{ 2 }$
          \newpar
          La somme de deux carrés ne peut jamais être strictement négative~; par conséquent, l'équation n'admet aucun couple solution.
          \newpar
          L'ensemble cherché est donc l'ensemble vide.
     \end{enumerate}
}% fin exemple
\bloc{orange}{Exemple 3}{ % id=e040
     \cadre{vert}{Énoncé}{ % id=r040
          L'équation~:
          \[
          x^2 +y^2 -6x +2y+10=0
          \]
          est-elle l'équation d'un cercle ?
     }% fin énoncé
     \begin{enumerate}
          \item
          Là encore, la première étape consiste à regrouper les termes en $x$ et les termes en $y$~:\\
          $x^2 -6x+ y^2  +2y + 10=0$
          \item
          Ensuite, on recherche le début d'identités remarquables~:
          $x^2 -6x$ est le début de~:  $x^2 -6x+9=\left(x - 3\right)^2$ \\
          Donc $x^2 -6x=\left(x - 3\right)^2-9$
          \newpar
          De même, pour les termes en $y$~:
          $y^2  +2y$ est le début de l'identité remarquable~: $y^2  +2y+1=\left(y+1\right)^2 $ \\
          Ce qui donne : $y^2  +y=\left(y +1\right)^2  - 1 $
          \item
          L'équation de départ peut alors s'écrire~:\\
          $\left(x - 3\right)^2-9+\left(y +1\right)^2  - 1  + 10 =0$
          \newpar
          Soit~:\\
          $\left(x - 3\right)^2+\left(y +1\right)^2 =0$
          \newpar
          La somme de deux carrés et nulle si et seulement si chacun de ces carrés est nul, c'est-à-dire si et seulement si~:\\
          $x - 3=0$ et $y+1=0.$
          \newpar
          L'équation proposée admet donc un unique couple solution qui est $\left( 3~;~ - 1 \right).$
          \newpar
          L'ensemble d'équation  $x^2 +y^2 -6x +2y+10=0$ est donc réduit au point de coordonnées $\left( 3~;~ - 1 \right).$
     \end{enumerate}
}% fin exemple

\end{document}