\documentclass[a4paper]{article}

%================================================================================================================================
%
% Packages
%
%================================================================================================================================

\usepackage[T1]{fontenc} 	% pour caractères accentués
\usepackage[utf8]{inputenc}  % encodage utf8
\usepackage[french]{babel}	% langue : français
\usepackage{fourier}			% caractères plus lisibles
\usepackage[dvipsnames]{xcolor} % couleurs
\usepackage{fancyhdr}		% réglage header footer
\usepackage{needspace}		% empêcher sauts de page mal placés
\usepackage{graphicx}		% pour inclure des graphiques
\usepackage{enumitem,cprotect}		% personnalise les listes d'items (nécessaire pour ol, al ...)
\usepackage{hyperref}		% Liens hypertexte
\usepackage{pstricks,pst-all,pst-node,pstricks-add,pst-math,pst-plot,pst-tree,pst-eucl} % pstricks
\usepackage[a4paper,includeheadfoot,top=2cm,left=3cm, bottom=2cm,right=3cm]{geometry} % marges etc.
\usepackage{comment}			% commentaires multilignes
\usepackage{amsmath,environ} % maths (matrices, etc.)
\usepackage{amssymb,makeidx}
\usepackage{bm}				% bold maths
\usepackage{tabularx}		% tableaux
\usepackage{colortbl}		% tableaux en couleur
\usepackage{fontawesome}		% Fontawesome
\usepackage{environ}			% environment with command
\usepackage{fp}				% calculs pour ps-tricks
\usepackage{multido}			% pour ps tricks
\usepackage[np]{numprint}	% formattage nombre
\usepackage{tikz,tkz-tab} 			% package principal TikZ
\usepackage{pgfplots}   % axes
\usepackage{mathrsfs}    % cursives
\usepackage{calc}			% calcul taille boites
\usepackage[scaled=0.875]{helvet} % font sans serif
\usepackage{svg} % svg
\usepackage{scrextend} % local margin
\usepackage{scratch} %scratch
\usepackage{multicol} % colonnes
%\usepackage{infix-RPN,pst-func} % formule en notation polanaise inversée
\usepackage{listings}

%================================================================================================================================
%
% Réglages de base
%
%================================================================================================================================

\lstset{
language=Python,   % R code
literate=
{á}{{\'a}}1
{à}{{\`a}}1
{ã}{{\~a}}1
{é}{{\'e}}1
{è}{{\`e}}1
{ê}{{\^e}}1
{í}{{\'i}}1
{ó}{{\'o}}1
{õ}{{\~o}}1
{ú}{{\'u}}1
{ü}{{\"u}}1
{ç}{{\c{c}}}1
{~}{{ }}1
}


\definecolor{codegreen}{rgb}{0,0.6,0}
\definecolor{codegray}{rgb}{0.5,0.5,0.5}
\definecolor{codepurple}{rgb}{0.58,0,0.82}
\definecolor{backcolour}{rgb}{0.95,0.95,0.92}

\lstdefinestyle{mystyle}{
    backgroundcolor=\color{backcolour},   
    commentstyle=\color{codegreen},
    keywordstyle=\color{magenta},
    numberstyle=\tiny\color{codegray},
    stringstyle=\color{codepurple},
    basicstyle=\ttfamily\footnotesize,
    breakatwhitespace=false,         
    breaklines=true,                 
    captionpos=b,                    
    keepspaces=true,                 
    numbers=left,                    
xleftmargin=2em,
framexleftmargin=2em,            
    showspaces=false,                
    showstringspaces=false,
    showtabs=false,                  
    tabsize=2,
    upquote=true
}

\lstset{style=mystyle}


\lstset{style=mystyle}
\newcommand{\imgdir}{C:/laragon/www/newmc/assets/imgsvg/}
\newcommand{\imgsvgdir}{C:/laragon/www/newmc/assets/imgsvg/}

\definecolor{mcgris}{RGB}{220, 220, 220}% ancien~; pour compatibilité
\definecolor{mcbleu}{RGB}{52, 152, 219}
\definecolor{mcvert}{RGB}{125, 194, 70}
\definecolor{mcmauve}{RGB}{154, 0, 215}
\definecolor{mcorange}{RGB}{255, 96, 0}
\definecolor{mcturquoise}{RGB}{0, 153, 153}
\definecolor{mcrouge}{RGB}{255, 0, 0}
\definecolor{mclightvert}{RGB}{205, 234, 190}

\definecolor{gris}{RGB}{220, 220, 220}
\definecolor{bleu}{RGB}{52, 152, 219}
\definecolor{vert}{RGB}{125, 194, 70}
\definecolor{mauve}{RGB}{154, 0, 215}
\definecolor{orange}{RGB}{255, 96, 0}
\definecolor{turquoise}{RGB}{0, 153, 153}
\definecolor{rouge}{RGB}{255, 0, 0}
\definecolor{lightvert}{RGB}{205, 234, 190}
\setitemize[0]{label=\color{lightvert}  $\bullet$}

\pagestyle{fancy}
\renewcommand{\headrulewidth}{0.2pt}
\fancyhead[L]{maths-cours.fr}
\fancyhead[R]{\thepage}
\renewcommand{\footrulewidth}{0.2pt}
\fancyfoot[C]{}

\newcolumntype{C}{>{\centering\arraybackslash}X}
\newcolumntype{s}{>{\hsize=.35\hsize\arraybackslash}X}

\setlength{\parindent}{0pt}		 
\setlength{\parskip}{3mm}
\setlength{\headheight}{1cm}

\def\ebook{ebook}
\def\book{book}
\def\web{web}
\def\type{web}

\newcommand{\vect}[1]{\overrightarrow{\,\mathstrut#1\,}}

\def\Oij{$\left(\text{O}~;~\vect{\imath},~\vect{\jmath}\right)$}
\def\Oijk{$\left(\text{O}~;~\vect{\imath},~\vect{\jmath},~\vect{k}\right)$}
\def\Ouv{$\left(\text{O}~;~\vect{u},~\vect{v}\right)$}

\hypersetup{breaklinks=true, colorlinks = true, linkcolor = OliveGreen, urlcolor = OliveGreen, citecolor = OliveGreen, pdfauthor={Didier BONNEL - https://www.maths-cours.fr} } % supprime les bordures autour des liens

\renewcommand{\arg}[0]{\text{arg}}

\everymath{\displaystyle}

%================================================================================================================================
%
% Macros - Commandes
%
%================================================================================================================================

\newcommand\meta[2]{    			% Utilisé pour créer le post HTML.
	\def\titre{titre}
	\def\url{url}
	\def\arg{#1}
	\ifx\titre\arg
		\newcommand\maintitle{#2}
		\fancyhead[L]{#2}
		{\Large\sffamily \MakeUppercase{#2}}
		\vspace{1mm}\textcolor{mcvert}{\hrule}
	\fi 
	\ifx\url\arg
		\fancyfoot[L]{\href{https://www.maths-cours.fr#2}{\black \footnotesize{https://www.maths-cours.fr#2}}}
	\fi 
}


\newcommand\TitreC[1]{    		% Titre centré
     \needspace{3\baselineskip}
     \begin{center}\textbf{#1}\end{center}
}

\newcommand\newpar{    		% paragraphe
     \par
}

\newcommand\nosp {    		% commande vide (pas d'espace)
}
\newcommand{\id}[1]{} %ignore

\newcommand\boite[2]{				% Boite simple sans titre
	\vspace{5mm}
	\setlength{\fboxrule}{0.2mm}
	\setlength{\fboxsep}{5mm}	
	\fcolorbox{#1}{#1!3}{\makebox[\linewidth-2\fboxrule-2\fboxsep]{
  		\begin{minipage}[t]{\linewidth-2\fboxrule-4\fboxsep}\setlength{\parskip}{3mm}
  			 #2
  		\end{minipage}
	}}
	\vspace{5mm}
}

\newcommand\CBox[4]{				% Boites
	\vspace{5mm}
	\setlength{\fboxrule}{0.2mm}
	\setlength{\fboxsep}{5mm}
	
	\fcolorbox{#1}{#1!3}{\makebox[\linewidth-2\fboxrule-2\fboxsep]{
		\begin{minipage}[t]{1cm}\setlength{\parskip}{3mm}
	  		\textcolor{#1}{\LARGE{#2}}    
 	 	\end{minipage}  
  		\begin{minipage}[t]{\linewidth-2\fboxrule-4\fboxsep}\setlength{\parskip}{3mm}
			\raisebox{1.2mm}{\normalsize\sffamily{\textcolor{#1}{#3}}}						
  			 #4
  		\end{minipage}
	}}
	\vspace{5mm}
}

\newcommand\cadre[3]{				% Boites convertible html
	\par
	\vspace{2mm}
	\setlength{\fboxrule}{0.1mm}
	\setlength{\fboxsep}{5mm}
	\fcolorbox{#1}{white}{\makebox[\linewidth-2\fboxrule-2\fboxsep]{
  		\begin{minipage}[t]{\linewidth-2\fboxrule-4\fboxsep}\setlength{\parskip}{3mm}
			\raisebox{-2.5mm}{\sffamily \small{\textcolor{#1}{\MakeUppercase{#2}}}}		
			\par		
  			 #3
 	 		\end{minipage}
	}}
		\vspace{2mm}
	\par
}

\newcommand\bloc[3]{				% Boites convertible html sans bordure
     \needspace{2\baselineskip}
     {\sffamily \small{\textcolor{#1}{\MakeUppercase{#2}}}}    
		\par		
  			 #3
		\par
}

\newcommand\CHelp[1]{
     \CBox{Plum}{\faInfoCircle}{À RETENIR}{#1}
}

\newcommand\CUp[1]{
     \CBox{NavyBlue}{\faThumbsOUp}{EN PRATIQUE}{#1}
}

\newcommand\CInfo[1]{
     \CBox{Sepia}{\faArrowCircleRight}{REMARQUE}{#1}
}

\newcommand\CRedac[1]{
     \CBox{PineGreen}{\faEdit}{BIEN R\'EDIGER}{#1}
}

\newcommand\CError[1]{
     \CBox{Red}{\faExclamationTriangle}{ATTENTION}{#1}
}

\newcommand\TitreExo[2]{
\needspace{4\baselineskip}
 {\sffamily\large EXERCICE #1\ (\emph{#2 points})}
\vspace{5mm}
}

\newcommand\img[2]{
          \includegraphics[width=#2\paperwidth]{\imgdir#1}
}

\newcommand\imgsvg[2]{
       \begin{center}   \includegraphics[width=#2\paperwidth]{\imgsvgdir#1} \end{center}
}


\newcommand\Lien[2]{
     \href{#1}{#2 \tiny \faExternalLink}
}
\newcommand\mcLien[2]{
     \href{https~://www.maths-cours.fr/#1}{#2 \tiny \faExternalLink}
}

\newcommand{\euro}{\eurologo{}}

%================================================================================================================================
%
% Macros - Environement
%
%================================================================================================================================

\newenvironment{tex}{ %
}
{%
}

\newenvironment{indente}{ %
	\setlength\parindent{10mm}
}

{
	\setlength\parindent{0mm}
}

\newenvironment{corrige}{%
     \needspace{3\baselineskip}
     \medskip
     \textbf{\textsc{Corrigé}}
     \medskip
}
{
}

\newenvironment{extern}{%
     \begin{center}
     }
     {
     \end{center}
}

\NewEnviron{code}{%
	\par
     \boite{gray}{\texttt{%
     \BODY
     }}
     \par
}

\newenvironment{vbloc}{% boite sans cadre empeche saut de page
     \begin{minipage}[t]{\linewidth}
     }
     {
     \end{minipage}
}
\NewEnviron{h2}{%
    \needspace{3\baselineskip}
    \vspace{0.6cm}
	\noindent \MakeUppercase{\sffamily \large \BODY}
	\vspace{1mm}\textcolor{mcgris}{\hrule}\vspace{0.4cm}
	\par
}{}

\NewEnviron{h3}{%
    \needspace{3\baselineskip}
	\vspace{5mm}
	\textsc{\BODY}
	\par
}

\NewEnviron{margeneg}{ %
\begin{addmargin}[-1cm]{0cm}
\BODY
\end{addmargin}
}

\NewEnviron{html}{%
}

\begin{document}
\meta{url}{/exercices/graphes-probabilistes-bac-blanc-es-l-sujet-5-maths-cours-2018-spe/}
\meta{pid}{10559}
\meta{titre}{Graphes probabilistes - Bac blanc ES/L Sujet 5 - Maths-cours 2018 (spé)}
\meta{type}{exercices}
%
\begin{h2}Exercice 2 (5 points)\end{h2}
\par
\textbf{Candidats ayant suivi l'enseignement de spécialité}
\par
Depuis le début de l'année 2015, une agence bancaire propose à ses clients, titulaires d'une carte de crédit, une assurance \og Tranquillité \fg{} qui leur permet d'être mieux indemnisé en cas de perte ou de vol de leur carte.
\par
De 2015 à 2017, on a constaté que :
\par
\begin{itemize}
     \item
     20\% des titulaires d'une carte de crédit qui ne bénéficient pas de l'assurance \og Tranquillité \fg{} souscrivent à cette assurance l'année suivante ;
     \item
     5\% des titulaires d'une carte de crédit qui ont souscrit à l'assurance \og Tranquillité \fg{} résilient cette assurance l'année suivante  ;
\end{itemize}
\par
On suppose que cette évolution se poursuivra de manière identique durant les années à venir.
\par
On sélectionne au hasard un client titulaire d'une carte de crédit et, pour tout entier naturel $n$, on note :
\par
\begin{itemize}
     \item
     $a_{n}$, la probabilité que le client ait souscrit à l'assurance \og Tranquillité \fg{} au début de l'année $2015 + n$ ;
     \item
     $b_{n}$, la probabilité que le client choisi n'ait pas souscrit à l'assurance \og Tranquillité \fg{} au début de l'année $2015 + n$ ;
     \item
     $P_{n}$, la matrice ligne $\left(a_{n} \quad b_{n}\right)$ donnant l'état probabiliste au début de l'année $2015 + n$.
\end{itemize}
\par
Au début de l'année 2015, aucun client n'a encore souscrit à l'assurance \og Tranquillité \fg{}. \\
On a donc $P_0=\left(0 \quad 1\right)$.
\par
%============================================================================================================================
%
\TitreC{Partie A}
%
%============================================================================================================================
\par
\begin{enumerate}
     \item %1
     Représenter la situation par un graphe probabiliste de sommets $T$ et $\overline{T}$ où $T$ correspond à l'état " le client a souscrit à l'assurance \og Tranquillité \fg{} " et $\overline{T}$ correspond à l'état contraire.
     \item %2
     Déterminer la matrice de transition $M$ associée à ce graphe, les sommets $T$ et $\overline{T}$ étant classés dans cet ordre.
     \item %3
     Déterminer l'état stable de ce graphe probabiliste. \\
     Que peut-on en conclure concernant le pourcentage de clients qui souscriront à l'assurance \og Tranquillité \fg{} dans les années futures ?
     \par
\end{enumerate}
\par
%============================================================================================================================
%
\TitreC{Partie B}
%
%============================================================================================================================
\par
Le directeur de l'agence cherche à déterminer au début de quelle année plus de 70\% des titulaires d'une carte de crédit auront souscrit à l'assurance \og Tranquillité \fg{}.
\par
\begin{enumerate}
     \item
     Recopier et compléter l'algorithme ci-après afin qu'il réponde à l'interrogation du directeur.
     \par
     \begin{center}
          \begin{extern}%width="400" alt="Algorithme"
               \begin{tabular}{|l l|}\hline
                    \textbf{Variables :}	& 	$n$ un nombre entier naturel \\
                    &$a$ et $b$ sont des nombres réels\\
                    \textbf{Initialisation :}	& Affecter à $n$ la valeur ...\\
                    & Affecter à $a$ la valeur ...\\
                    & Affecter à $b$ la valeur ...\\
                    \textbf{Traitement :} & Tant que ...\\
                    &\qquad Affecter à $a$ la valeur \\
                    &\qquad \phantom{Affecter }$0,95 \times a + 0,2 \times b$\\
                    &\qquad Affecter à $b$ la valeur $1-a$ \\
                    &\qquad Affecter à $n$ la valeur ...\\
                    &Fin Tant que\\
                    \textbf{Sortie :}		&Afficher $2015+n$ \\ \hline
               \end{tabular}
          \end{extern}
     \end{center}
     \item
     On admet que pour tout entier naturel $n$ :
     \[ a_{n}=0,8(1-0,75^n). \]
     Déterminer la valeur affichée par l'algorithme de la question précédente.
     \par
\end{enumerate}
\begin{corrige}
     %============================================================================================================================
     %
     \TitreC{Partie A}
     %
     %============================================================================================================================
     \par
     \begin{enumerate}
          \item On traduit les données de l'énoncé par un graphe probabiliste de sommets $T$ et $\overline{T}$:
          \begin{center}
               %\hspace*{1cm}
               \begin{extern}%width="400" alt="Graphe probabiliste"
                    \begin{pspicture}(-2,-0.5)(4,1)
                         \circlenode{T}{$T$} \hskip 4cm \circlenode{B}{$\overline{T}$}% définition des sommets
                         \psset{arcangle=15,arrowsize=2pt 3}%  différents paramètres
                         \ncarc{->}{T}{B} \Aput{0,05}%              arc pondéré partant de T
                         \ncarc{->}{B}{T} \Aput{0,2}%              arc pondéré arrivant à B
                         \nccircle[angleA=90]{->}{T}{4mm}   \Bput{0,95}%    boucle autour de T
                         \nccircle[angleA=-90]{->}{B}{.4cm} \Bput{0,8}%    boucle autour de B
                    \end{pspicture}
               \end{extern}
          \end{center}
          \item  La matrice de transition de ce graphe en considérant les sommets dans l'ordre $T$, $\overline{T}$ est:
          \[ M=
          \begin{pmatrix}
               0,95 & 0,05\\
               0,2 & 0,8
          \end{pmatrix}. \]
          \item
          Les états stables sont les matrices-ligne $P = (a\quad b)$ telles que ${a + b = 1}$ et ${PM = P}$.
          \par
          $PM=P \Leftrightarrow \begin{pmatrix} a&b\end{pmatrix}
          \times \begin{pmatrix} 0,95 & 0,05 \\ 0,2 & 0,8 \end{pmatrix}
          $\nosp$=\begin{pmatrix} a&b\end{pmatrix}$
          \par
          $\phantom{PM=P} \Leftrightarrow \begin{pmatrix} 0,95a+0,2b & 0,05a+0,8b\end{pmatrix}
          $\nosp$=\begin{pmatrix} a&b\end{pmatrix}$
          \par
          $\phantom{PM=P} \Leftrightarrow
          \left\lbrace
          \begin{array}{r c l}
               0,95a+0,2b &=& a\\
               0,05a+0,8b &=& b
          \end{array}
     \right.$
     \par
     $\phantom{PM=P}
     \Leftrightarrow
     \left\lbrace
     \begin{array}{r c l}
          -0,05a+0,2b &=& 0\\
          0,05a-0,2b &=& 0
     \end{array}
\right.$
\par
$\phantom{PM=P}
\Leftrightarrow
0,05a-0,2b = 0$.
\par
Or $a+b=1$ ; donc  $b=1-a$ et:
\par
$0,05a-0,2(1-a) = 0$
\par
$0,25a-0,2 = 0$
\par
$a = \dfrac{0,2}{0.25}=0,8$.
\par
Et $b=1-a=1-0,8=0,2$.
\par
L'état stable du graphe est donc
$P=\begin{pmatrix} 0,8 & 0,2 \end{pmatrix}$.
\par
Au fil du temps, le pourcentage de clients qui choisiront l'assurance \og Tranquillité \fg{} se rapprochera de 80\% ($=0,8$).
\end{enumerate}
\par
%============================================================================================================================
%
\TitreC{Partie B}
%
%============================================================================================================================
\par
\begin{enumerate}
     \item
     On complète l'algorithme comme suit :
     \begin{center}
          \begin{extern}%width="400" alt="Algorithme"
               \begin{tabular}{|l l|}\hline
                    \textbf{Variables :}	& 	$n$ un nombre entier naturel \\
                    &$a$ et $b$ sont des nombres réels\\
                    \textbf{Initialisation :}	& Affecter à $n$ la valeur $\color{red}{0}$\\
                    & Affecter à $a$ la valeur $\color{red}{0}$\\
                    & Affecter à $b$ la valeur $\color{red}{1}$\\
                    \textbf{Traitement :} & Tant que $\color{red}{a \leqslant 0,7}$ \\
                    &\qquad Affecter à $a$ la valeur \\
                    &\qquad \phantom{Affecter }$0,95 \times a + 0,2 \times b$\\
                    &\qquad Affecter à $b$ la valeur $1-a$ \\
                    &\qquad Affecter à $n$ la valeur $\color{red}{n+1}$ \\
                    &Fin Tant que\\
                    \textbf{Sortie :}		&Afficher $2015+n$ \\ \hline
               \end{tabular}
          \end{extern}
     \end{center}
     \item
     L'algorithme affiche l'année à partir de laquelle plus de 70\% des titulaires d'une carte de crédit auront souscrit à l'assurance \og Tranquillité \fg{}.
     \par
     Pour trouver cette année, il nous faut donc résoudre l'inéquation $a_n > 0,7$.
     \par
     D'après l'énoncé $a_{n}=0,8(1-0,75^n)$ donc :
     \par
     $a_n > 0,7  \Leftrightarrow 0,8(1-0,75^n) > 0,7$\\
     $\phantom{ a_n > 0,7 } \Leftrightarrow \  1-0,75^n > \dfrac{0,7}{0,8}$ \\
     $\phantom{ a_n > 0,7 } \Leftrightarrow \  1-0,75^n > \dfrac{7}{8}$ \\
     $\phantom{ a_n > 0,7 } \Leftrightarrow \  -0,75^n  > \dfrac{7}{8}-1 $\\
     $\phantom{ a_n > 0,7 } \Leftrightarrow \  -0,75^n  > -\dfrac{1}{8}$\\
     $\phantom{ a_n > 0,7 } \Leftrightarrow \  0,75^n < \dfrac{1}{8}$
     \par
     La fonction $\ln$ étant strictement croissante sur $]0~;~+\infty[$ :
     \par
     $a_n > 0,7  \Leftrightarrow \  \ln \left(0,75^n \right) < \ln \left(\dfrac{1}{8}\right) $ \\
     $\phantom{ a_n > 0,7 } \Leftrightarrow \ n\ln (0,75) < -\ln (8)$
     \par
     Or $0,9 < 1$ donc $\ln (0,75)$ est strictement négatif ; par conséquent :
     \par
     $a_n > 0,7 \Leftrightarrow \ n > \dfrac{-\ln (8)}{\ln (0,75)}$
     \par
     \`A la calculatrice : $\dfrac{-\ln (8)}{\ln (0,75)} - 1 \approx 7,2$ (arrondi au dixième).
     \par
     La plus petite valeur de l'entier $n$ telle que $a_n > 0,7$, est donc $8$.
     \par
     Par conséquent, l'algorithme affichera l'année 2015+8=\textbf{2023}.
     \par
\end{enumerate}
\end{corrige}

\end{document}