\documentclass[a4paper]{article}

%================================================================================================================================
%
% Packages
%
%================================================================================================================================

\usepackage[T1]{fontenc} 	% pour caractères accentués
\usepackage[utf8]{inputenc}  % encodage utf8
\usepackage[french]{babel}	% langue : français
\usepackage{fourier}			% caractères plus lisibles
\usepackage[dvipsnames]{xcolor} % couleurs
\usepackage{fancyhdr}		% réglage header footer
\usepackage{needspace}		% empêcher sauts de page mal placés
\usepackage{graphicx}		% pour inclure des graphiques
\usepackage{enumitem,cprotect}		% personnalise les listes d'items (nécessaire pour ol, al ...)
\usepackage{hyperref}		% Liens hypertexte
\usepackage{pstricks,pst-all,pst-node,pstricks-add,pst-math,pst-plot,pst-tree,pst-eucl} % pstricks
\usepackage[a4paper,includeheadfoot,top=2cm,left=3cm, bottom=2cm,right=3cm]{geometry} % marges etc.
\usepackage{comment}			% commentaires multilignes
\usepackage{amsmath,environ} % maths (matrices, etc.)
\usepackage{amssymb,makeidx}
\usepackage{bm}				% bold maths
\usepackage{tabularx}		% tableaux
\usepackage{colortbl}		% tableaux en couleur
\usepackage{fontawesome}		% Fontawesome
\usepackage{environ}			% environment with command
\usepackage{fp}				% calculs pour ps-tricks
\usepackage{multido}			% pour ps tricks
\usepackage[np]{numprint}	% formattage nombre
\usepackage{tikz,tkz-tab} 			% package principal TikZ
\usepackage{pgfplots}   % axes
\usepackage{mathrsfs}    % cursives
\usepackage{calc}			% calcul taille boites
\usepackage[scaled=0.875]{helvet} % font sans serif
\usepackage{svg} % svg
\usepackage{scrextend} % local margin
\usepackage{scratch} %scratch
\usepackage{multicol} % colonnes
%\usepackage{infix-RPN,pst-func} % formule en notation polanaise inversée
\usepackage{listings}

%================================================================================================================================
%
% Réglages de base
%
%================================================================================================================================

\lstset{
language=Python,   % R code
literate=
{á}{{\'a}}1
{à}{{\`a}}1
{ã}{{\~a}}1
{é}{{\'e}}1
{è}{{\`e}}1
{ê}{{\^e}}1
{í}{{\'i}}1
{ó}{{\'o}}1
{õ}{{\~o}}1
{ú}{{\'u}}1
{ü}{{\"u}}1
{ç}{{\c{c}}}1
{~}{{ }}1
}


\definecolor{codegreen}{rgb}{0,0.6,0}
\definecolor{codegray}{rgb}{0.5,0.5,0.5}
\definecolor{codepurple}{rgb}{0.58,0,0.82}
\definecolor{backcolour}{rgb}{0.95,0.95,0.92}

\lstdefinestyle{mystyle}{
    backgroundcolor=\color{backcolour},   
    commentstyle=\color{codegreen},
    keywordstyle=\color{magenta},
    numberstyle=\tiny\color{codegray},
    stringstyle=\color{codepurple},
    basicstyle=\ttfamily\footnotesize,
    breakatwhitespace=false,         
    breaklines=true,                 
    captionpos=b,                    
    keepspaces=true,                 
    numbers=left,                    
xleftmargin=2em,
framexleftmargin=2em,            
    showspaces=false,                
    showstringspaces=false,
    showtabs=false,                  
    tabsize=2,
    upquote=true
}

\lstset{style=mystyle}


\lstset{style=mystyle}
\newcommand{\imgdir}{C:/laragon/www/newmc/assets/imgsvg/}
\newcommand{\imgsvgdir}{C:/laragon/www/newmc/assets/imgsvg/}

\definecolor{mcgris}{RGB}{220, 220, 220}% ancien~; pour compatibilité
\definecolor{mcbleu}{RGB}{52, 152, 219}
\definecolor{mcvert}{RGB}{125, 194, 70}
\definecolor{mcmauve}{RGB}{154, 0, 215}
\definecolor{mcorange}{RGB}{255, 96, 0}
\definecolor{mcturquoise}{RGB}{0, 153, 153}
\definecolor{mcrouge}{RGB}{255, 0, 0}
\definecolor{mclightvert}{RGB}{205, 234, 190}

\definecolor{gris}{RGB}{220, 220, 220}
\definecolor{bleu}{RGB}{52, 152, 219}
\definecolor{vert}{RGB}{125, 194, 70}
\definecolor{mauve}{RGB}{154, 0, 215}
\definecolor{orange}{RGB}{255, 96, 0}
\definecolor{turquoise}{RGB}{0, 153, 153}
\definecolor{rouge}{RGB}{255, 0, 0}
\definecolor{lightvert}{RGB}{205, 234, 190}
\setitemize[0]{label=\color{lightvert}  $\bullet$}

\pagestyle{fancy}
\renewcommand{\headrulewidth}{0.2pt}
\fancyhead[L]{maths-cours.fr}
\fancyhead[R]{\thepage}
\renewcommand{\footrulewidth}{0.2pt}
\fancyfoot[C]{}

\newcolumntype{C}{>{\centering\arraybackslash}X}
\newcolumntype{s}{>{\hsize=.35\hsize\arraybackslash}X}

\setlength{\parindent}{0pt}		 
\setlength{\parskip}{3mm}
\setlength{\headheight}{1cm}

\def\ebook{ebook}
\def\book{book}
\def\web{web}
\def\type{web}

\newcommand{\vect}[1]{\overrightarrow{\,\mathstrut#1\,}}

\def\Oij{$\left(\text{O}~;~\vect{\imath},~\vect{\jmath}\right)$}
\def\Oijk{$\left(\text{O}~;~\vect{\imath},~\vect{\jmath},~\vect{k}\right)$}
\def\Ouv{$\left(\text{O}~;~\vect{u},~\vect{v}\right)$}

\hypersetup{breaklinks=true, colorlinks = true, linkcolor = OliveGreen, urlcolor = OliveGreen, citecolor = OliveGreen, pdfauthor={Didier BONNEL - https://www.maths-cours.fr} } % supprime les bordures autour des liens

\renewcommand{\arg}[0]{\text{arg}}

\everymath{\displaystyle}

%================================================================================================================================
%
% Macros - Commandes
%
%================================================================================================================================

\newcommand\meta[2]{    			% Utilisé pour créer le post HTML.
	\def\titre{titre}
	\def\url{url}
	\def\arg{#1}
	\ifx\titre\arg
		\newcommand\maintitle{#2}
		\fancyhead[L]{#2}
		{\Large\sffamily \MakeUppercase{#2}}
		\vspace{1mm}\textcolor{mcvert}{\hrule}
	\fi 
	\ifx\url\arg
		\fancyfoot[L]{\href{https://www.maths-cours.fr#2}{\black \footnotesize{https://www.maths-cours.fr#2}}}
	\fi 
}


\newcommand\TitreC[1]{    		% Titre centré
     \needspace{3\baselineskip}
     \begin{center}\textbf{#1}\end{center}
}

\newcommand\newpar{    		% paragraphe
     \par
}

\newcommand\nosp {    		% commande vide (pas d'espace)
}
\newcommand{\id}[1]{} %ignore

\newcommand\boite[2]{				% Boite simple sans titre
	\vspace{5mm}
	\setlength{\fboxrule}{0.2mm}
	\setlength{\fboxsep}{5mm}	
	\fcolorbox{#1}{#1!3}{\makebox[\linewidth-2\fboxrule-2\fboxsep]{
  		\begin{minipage}[t]{\linewidth-2\fboxrule-4\fboxsep}\setlength{\parskip}{3mm}
  			 #2
  		\end{minipage}
	}}
	\vspace{5mm}
}

\newcommand\CBox[4]{				% Boites
	\vspace{5mm}
	\setlength{\fboxrule}{0.2mm}
	\setlength{\fboxsep}{5mm}
	
	\fcolorbox{#1}{#1!3}{\makebox[\linewidth-2\fboxrule-2\fboxsep]{
		\begin{minipage}[t]{1cm}\setlength{\parskip}{3mm}
	  		\textcolor{#1}{\LARGE{#2}}    
 	 	\end{minipage}  
  		\begin{minipage}[t]{\linewidth-2\fboxrule-4\fboxsep}\setlength{\parskip}{3mm}
			\raisebox{1.2mm}{\normalsize\sffamily{\textcolor{#1}{#3}}}						
  			 #4
  		\end{minipage}
	}}
	\vspace{5mm}
}

\newcommand\cadre[3]{				% Boites convertible html
	\par
	\vspace{2mm}
	\setlength{\fboxrule}{0.1mm}
	\setlength{\fboxsep}{5mm}
	\fcolorbox{#1}{white}{\makebox[\linewidth-2\fboxrule-2\fboxsep]{
  		\begin{minipage}[t]{\linewidth-2\fboxrule-4\fboxsep}\setlength{\parskip}{3mm}
			\raisebox{-2.5mm}{\sffamily \small{\textcolor{#1}{\MakeUppercase{#2}}}}		
			\par		
  			 #3
 	 		\end{minipage}
	}}
		\vspace{2mm}
	\par
}

\newcommand\bloc[3]{				% Boites convertible html sans bordure
     \needspace{2\baselineskip}
     {\sffamily \small{\textcolor{#1}{\MakeUppercase{#2}}}}    
		\par		
  			 #3
		\par
}

\newcommand\CHelp[1]{
     \CBox{Plum}{\faInfoCircle}{À RETENIR}{#1}
}

\newcommand\CUp[1]{
     \CBox{NavyBlue}{\faThumbsOUp}{EN PRATIQUE}{#1}
}

\newcommand\CInfo[1]{
     \CBox{Sepia}{\faArrowCircleRight}{REMARQUE}{#1}
}

\newcommand\CRedac[1]{
     \CBox{PineGreen}{\faEdit}{BIEN R\'EDIGER}{#1}
}

\newcommand\CError[1]{
     \CBox{Red}{\faExclamationTriangle}{ATTENTION}{#1}
}

\newcommand\TitreExo[2]{
\needspace{4\baselineskip}
 {\sffamily\large EXERCICE #1\ (\emph{#2 points})}
\vspace{5mm}
}

\newcommand\img[2]{
          \includegraphics[width=#2\paperwidth]{\imgdir#1}
}

\newcommand\imgsvg[2]{
       \begin{center}   \includegraphics[width=#2\paperwidth]{\imgsvgdir#1} \end{center}
}


\newcommand\Lien[2]{
     \href{#1}{#2 \tiny \faExternalLink}
}
\newcommand\mcLien[2]{
     \href{https~://www.maths-cours.fr/#1}{#2 \tiny \faExternalLink}
}

\newcommand{\euro}{\eurologo{}}

%================================================================================================================================
%
% Macros - Environement
%
%================================================================================================================================

\newenvironment{tex}{ %
}
{%
}

\newenvironment{indente}{ %
	\setlength\parindent{10mm}
}

{
	\setlength\parindent{0mm}
}

\newenvironment{corrige}{%
     \needspace{3\baselineskip}
     \medskip
     \textbf{\textsc{Corrigé}}
     \medskip
}
{
}

\newenvironment{extern}{%
     \begin{center}
     }
     {
     \end{center}
}

\NewEnviron{code}{%
	\par
     \boite{gray}{\texttt{%
     \BODY
     }}
     \par
}

\newenvironment{vbloc}{% boite sans cadre empeche saut de page
     \begin{minipage}[t]{\linewidth}
     }
     {
     \end{minipage}
}
\NewEnviron{h2}{%
    \needspace{3\baselineskip}
    \vspace{0.6cm}
	\noindent \MakeUppercase{\sffamily \large \BODY}
	\vspace{1mm}\textcolor{mcgris}{\hrule}\vspace{0.4cm}
	\par
}{}

\NewEnviron{h3}{%
    \needspace{3\baselineskip}
	\vspace{5mm}
	\textsc{\BODY}
	\par
}

\NewEnviron{margeneg}{ %
\begin{addmargin}[-1cm]{0cm}
\BODY
\end{addmargin}
}

\NewEnviron{html}{%
}

\begin{document}
\meta{url}{/exercices/suites-bac-blanc-es-l-sujet-4-maths-cours-2018/}
\meta{pid}{10505}
\meta{titre}{Suites - Bac blanc ES/L Sujet 4 - Maths-cours 2018}
\meta{type}{exercices}
%
\begin{h2}Exercice 4 (6 points)\end{h2}
\par
Un fournisseur d'accès internet propose deux formules, nommées \og Start \fg{} et \og Plus \fg{}, à ses abonnés.
\par
On suppose que le nombre global d'abonnés à ce fournisseur d'accès est stable d'une année sur l'autre et égal à 2 millions.
\par
En 2010, 1,5 million de personnes étaient abonnés à la formule \og Start \fg{} et 500 000 personnes étaient abonnés à la formule \og Plus \fg{}.
\par
Chaque année :
\par
\begin{itemize}
     \item 30\% des abonnés à la formule \og Start \fg{} choisissent de passer à la formule \og Plus  \fg{} l'année suivante (les 70\% restant conservant la formule \og Start \fg{}) ;
     \item 10\% des abonnés à la formule \og Plus \fg{} décident de migrer vers la formule \og Start  \fg{} l'année suivante (les 90\% restant conservant la formule \og Plus \fg{}).
\end{itemize}
\par
Pour tout entier naturel $n$, on note $a_n$ (respectivement $b_n$) le nombre d'abonnés, en milliers, à la formule \og Start \fg{} (respectivement à la formule \og Plus \fg{} ) l'année $2010+n$.
\par
On a donc ${a_0=1~500}$ et ${b_0=500}$.
\par
\begin{enumerate}
     \item %1
     Expliquer pourquoi, pour tout entier naturel $n$, ${a_n+b_n=2~000}$.
     \item %2
     Montrer que, pour tout entier naturel $n$ :
     \[ a_{n+1} =0,7a_n+0,1b_n. \]
     \item %3
     En déduire que, pour tout entier naturel $n$ :
     \[ a_{n+1} =0,6a_n+200. \]
     \item %4
     Soit la suite $(u_n)$ définie, pour tout entier naturel $n$, par :
     \[ u_n=a_n-500. \]
     \par
     \begin{enumerate}[label=\alph*.]
          \item %a
          Montrer que la suite $(u_n)$ est une suite géométrique dont on précisera le premier terme et la raison.
          \item %b
          Exprimer $u_n$ en fonction de $n$.
          \item %c
          En déduire que, pour tout entier naturel $n$ :
          \[ a_n=500+1000 \times 0,6^n. \]
     \end{enumerate}
     \item %5
     Montrer que la suite $(a_n)$ est décroissante et converge vers une limite que l'on déterminera.
     Que peut-on en déduire concernant le nombre d'abonnés à la formule \og Start \fg{} ?
     \item %6
     On souhaite utiliser un tableur pour calculer les termes $a_n$ et $b_n$.
     \par
     \begin{center}
          \img{BBESL-s4-4-1}{0.4}%width="450" alt="utilisation d'un tableur""
     \end{center}
     \par
     \begin{enumerate}
          \item %a
          Proposer une formule à saisir dans la cellule \textbf{C2} pour calculer $a_1$.
          \par
          \textit{Cette formule devra permettre de calculer les valeurs successives de la suite $(a_n)$ en la \og tirant vers la droite \fg{}. }
          \item %b
          Proposer une formule à saisir dans la cellule \textbf{B3} pour calculer $b_0$.
          \par
          \textit{Cette formule devra permettre de calculer les valeurs successives de la suite $(b_n)$ en la \og tirant vers la droite \fg{}. }
          \par
     \end{enumerate}
     \par
\end{enumerate}
\begin{corrige}
     \begin{enumerate}
          \item %1
          Pour tout entier naturel $n$, la somme ${a_n+b_n}$ représente le nombre total d'abonnés (en milliers) chez ce fournisseur d'accès internet.
          \par
          D'après l'énoncé ce nombre est stable et correspond à 2 millions, soit 2~000 milliers d'abonnés.
          \par
          Par conséquent, pour tout entier naturel $n$ :
          \[ a_n+b_n = 2~000. \]
          \item %2
          $a_{n+1}$ représente le nombre d'abonnés, en milliers, à la formule \og Start \fg{} l'année $2010+n+1$.
          \par
          Ce nombre comporte :
          \par
          \begin{itemize}
               \item
               les abonnés à la formule \og Start \fg{} de l'année précédente qui se réinscrivent à cette même formule, c'est à dire 70\% de $a_n$ soit $0,7a_n$ ;
               \item
               les abonnés à la formule \og Plus \fg{} de l'année précédente qui décident de migrer vers la formule \og Start  \fg{}, c'est à dire 10\% de $b_n$ soit $0,1b_n$ ;
               \par
          \end{itemize}
          \par
          Au total, on obtient :
          \par
          \[ a_{n+1} =0,7a_n+0,1b_n. \]
          \item %3
          D'après la question \textbf{1.}, $a_n+b_n = 2~000$ donc $b_n = 2~000 - a_n$.
          \par
          Comme $a_{n+1} =0,7a_n+0,1b_n$, alors :
          \par
          $a_{n+1} =0,7a_n+0,1(2~000 - a_n)$\\
          $\phantom{a_{n+1}} =0,7a_n+200 - 0,1a_n$\\
          $\phantom{a_{n+1}} =0,6a_n+200.$
          \item %4
          \begin{enumerate}[label=\alph*.]
               \item %a
               Pour tout entier naturel $n$ :
               \par
               $u_{n+1}=a_{n+1}-500$
               \par
               $\phantom{u_{n+1}}=0,6a_n+200-500$
               \par
               $\phantom{u_{n+1}}=0,6a_n-300$.
               \par
               Or $u_n=a_n-500$ donc $a_n=u_n+500$ ; alors :
               \par
               $u_{n+1}=0,6(u_n+500)-300$
               \par
               $\phantom{u_{n+1}}=0,6u_n+500-500$
               \par
               $\phantom{u_{n+1}}=0,6u_n$.
               \par
               De plus, comme ${u_0=a_0-500=1~500-500=1~000}$, la suite $(u_n)$ est une suite géométrique de premier terme ${u_0=1~000}$ et de raison ${q=0,6}$.
               \item %b
               Par conséquent :
               \par
               $u_n=u_0q^n=1~000 \times 0,6^n$.
               \item %c
               En utilisant la question précédente et la relation ${a_n=u_n+500}$, on en déduit que pour tout entier naturel $n$ :
               \par
               $a_n=u_n+500=500+1~000 \times 0,6^n$.
               \par
          \end{enumerate}
          \item %5
          D'après la question précédente, pour tout entier naturel $n$ :
          \par
          $a_{n+1}-a_n=500+1000 \times 0,6^{n+1}-\left[500+1000 \times 0,6^n\right]$
          \par
          $\phantom{a_{n+1}-a_n}=500+1000 \times 0,6^{n+1}-500-1000 \times 0,6^n$
          \par
          $\phantom{a_{n+1}-a_n}=1000 \times 0,6^{n+1}-1000 \times 0,6^n$.
          \par
          \vspace{2mm}
          Or $0,6^{n+1}=0,6^n \times 0,6$, donc :
          \par
          $a_{n+1}-a_n=1000 \times 0,6^n \times 0,6-1000 \times 0,6^n$
          \par
          $\phantom{a_{n+1}-a_n}=1000 \times 0,6^{n}[0,6-1)$
          \par
          $\phantom{a_{n+1}-a_n}=-0,4 \times 1000 \times 0,6^{n}$.
          \par
          $a_{n+1}-a_n$ est strictement négatif pour tout entier naturel $n$, donc la suite $(a_n)$ est strictement \textbf{décroissante}.
          \par
          \cadre{rouge}{À retenir}{
               Pour démontrer qu'une suite $(u_n)$ est \textbf{croissante}, on montre que pour tout entier naturel $n$ :
               \[ u_{n+1}-u_n \geqslant 0. \]
               \par
               Pour démontrer qu'une suite $(u_n)$ est \textbf{décroissante}, on montre que pour tout entier naturel $n$ :
               \[ u_{n+1}-u_n \leqslant 0. \]
          }
          \par
          \cadre{vert}{En pratique}{
               La formule :
               \[ a^{n+1} = a^n \times a. \]
               qui est un cas particulier de la formule ${a^{n+m} = a^n \times a^m}$ est très souvent utilisée dans les calculs concernant les suites géométriques.
          }
          Comme $0 < 0,6 < 1$, ${\lim\limits_{n \rightarrow +\infty}0,6^n=0}$ et ${\lim\limits_{n \rightarrow +\infty}1000 \times 0,6^n=0}$. Par conséquent :
          \par
          $\lim\limits_{n \rightarrow +\infty}a_n = \lim\limits_{n \rightarrow +\infty}500+1000 \times 0,6^n =500.$
          \par
          On en déduit que le nombre d'abonnés à la formule \og Start \fg{} va \textbf{décroître et se rapprocher de} $\bm{500~000}$.
          \item %6
          \begin{enumerate}[label=\alph*.]
               \item %a
               \textbf{Solution n°1}
               \par
               On sait que pour tout entier naturel $n$ : $a_{n+1}=0,6a_n+200$.
               \par
               En particulier $a_1=0,6a_0+200$.
               \par
               $a_0$ est situé dans la cellule \textbf{B2}. On peut donc saisir dans la cellule \textbf{C2} :
               \par
               \begin{center}
                    \mintxtbox{=0,6*B2+200}
               \end{center}
               \par
               \textbf{Solution n°2}
               \par
               On sait que pour tout entier naturel $n$ : $a_{n}=500+1000 \times 0,6^n$.
               \par
               En particulier $a_1=500+1000 \times 0,6^1$.
               \par
               Les indices sont situés sur la ligne n°1 ; l'indice 1 est situé dans la cellule \textbf{C1}. On peut donc saisir dans la cellule \textbf{C2} :
               \par
               \begin{center}
                    \mintxtbox{=500+1000*PUISSANCE(0,6~;~C1)}
               \end{center}
               \item %b
               Pour tout entier naturel $n$, $b_n=2~000-a_n$.\\
               En particulier : ${b_0=2~000-a_0}$.
               \par
               $a_0$ est situé dans la cellule \textbf{B2}. On peut donc saisir dans la cellule \textbf{B3} :
               \par
               \begin{center}
                    \mintxtbox{=2000-B2}
               \end{center}
               \par
               \textit{Remarque : D'autres solutions sont également possibles.}
               \par
               \cadre{rouge}{À retenir}{
                    Dans un tableur, une formule mathématique doit débuter par le signe \textbf{=} pour être exécutée.
               }
               \par
          \end{enumerate}
          \par
     \end{enumerate}
\end{corrige}

\end{document}