\documentclass[a4paper]{article}

%================================================================================================================================
%
% Packages
%
%================================================================================================================================

\usepackage[T1]{fontenc} 	% pour caractères accentués
\usepackage[utf8]{inputenc}  % encodage utf8
\usepackage[french]{babel}	% langue : français
\usepackage{fourier}			% caractères plus lisibles
\usepackage[dvipsnames]{xcolor} % couleurs
\usepackage{fancyhdr}		% réglage header footer
\usepackage{needspace}		% empêcher sauts de page mal placés
\usepackage{graphicx}		% pour inclure des graphiques
\usepackage{enumitem,cprotect}		% personnalise les listes d'items (nécessaire pour ol, al ...)
\usepackage{hyperref}		% Liens hypertexte
\usepackage{pstricks,pst-all,pst-node,pstricks-add,pst-math,pst-plot,pst-tree,pst-eucl} % pstricks
\usepackage[a4paper,includeheadfoot,top=2cm,left=3cm, bottom=2cm,right=3cm]{geometry} % marges etc.
\usepackage{comment}			% commentaires multilignes
\usepackage{amsmath,environ} % maths (matrices, etc.)
\usepackage{amssymb,makeidx}
\usepackage{bm}				% bold maths
\usepackage{tabularx}		% tableaux
\usepackage{colortbl}		% tableaux en couleur
\usepackage{fontawesome}		% Fontawesome
\usepackage{environ}			% environment with command
\usepackage{fp}				% calculs pour ps-tricks
\usepackage{multido}			% pour ps tricks
\usepackage[np]{numprint}	% formattage nombre
\usepackage{tikz,tkz-tab} 			% package principal TikZ
\usepackage{pgfplots}   % axes
\usepackage{mathrsfs}    % cursives
\usepackage{calc}			% calcul taille boites
\usepackage[scaled=0.875]{helvet} % font sans serif
\usepackage{svg} % svg
\usepackage{scrextend} % local margin
\usepackage{scratch} %scratch
\usepackage{multicol} % colonnes
%\usepackage{infix-RPN,pst-func} % formule en notation polanaise inversée
\usepackage{listings}

%================================================================================================================================
%
% Réglages de base
%
%================================================================================================================================

\lstset{
language=Python,   % R code
literate=
{á}{{\'a}}1
{à}{{\`a}}1
{ã}{{\~a}}1
{é}{{\'e}}1
{è}{{\`e}}1
{ê}{{\^e}}1
{í}{{\'i}}1
{ó}{{\'o}}1
{õ}{{\~o}}1
{ú}{{\'u}}1
{ü}{{\"u}}1
{ç}{{\c{c}}}1
{~}{{ }}1
}


\definecolor{codegreen}{rgb}{0,0.6,0}
\definecolor{codegray}{rgb}{0.5,0.5,0.5}
\definecolor{codepurple}{rgb}{0.58,0,0.82}
\definecolor{backcolour}{rgb}{0.95,0.95,0.92}

\lstdefinestyle{mystyle}{
    backgroundcolor=\color{backcolour},   
    commentstyle=\color{codegreen},
    keywordstyle=\color{magenta},
    numberstyle=\tiny\color{codegray},
    stringstyle=\color{codepurple},
    basicstyle=\ttfamily\footnotesize,
    breakatwhitespace=false,         
    breaklines=true,                 
    captionpos=b,                    
    keepspaces=true,                 
    numbers=left,                    
xleftmargin=2em,
framexleftmargin=2em,            
    showspaces=false,                
    showstringspaces=false,
    showtabs=false,                  
    tabsize=2,
    upquote=true
}

\lstset{style=mystyle}


\lstset{style=mystyle}
\newcommand{\imgdir}{C:/laragon/www/newmc/assets/imgsvg/}
\newcommand{\imgsvgdir}{C:/laragon/www/newmc/assets/imgsvg/}

\definecolor{mcgris}{RGB}{220, 220, 220}% ancien~; pour compatibilité
\definecolor{mcbleu}{RGB}{52, 152, 219}
\definecolor{mcvert}{RGB}{125, 194, 70}
\definecolor{mcmauve}{RGB}{154, 0, 215}
\definecolor{mcorange}{RGB}{255, 96, 0}
\definecolor{mcturquoise}{RGB}{0, 153, 153}
\definecolor{mcrouge}{RGB}{255, 0, 0}
\definecolor{mclightvert}{RGB}{205, 234, 190}

\definecolor{gris}{RGB}{220, 220, 220}
\definecolor{bleu}{RGB}{52, 152, 219}
\definecolor{vert}{RGB}{125, 194, 70}
\definecolor{mauve}{RGB}{154, 0, 215}
\definecolor{orange}{RGB}{255, 96, 0}
\definecolor{turquoise}{RGB}{0, 153, 153}
\definecolor{rouge}{RGB}{255, 0, 0}
\definecolor{lightvert}{RGB}{205, 234, 190}
\setitemize[0]{label=\color{lightvert}  $\bullet$}

\pagestyle{fancy}
\renewcommand{\headrulewidth}{0.2pt}
\fancyhead[L]{maths-cours.fr}
\fancyhead[R]{\thepage}
\renewcommand{\footrulewidth}{0.2pt}
\fancyfoot[C]{}

\newcolumntype{C}{>{\centering\arraybackslash}X}
\newcolumntype{s}{>{\hsize=.35\hsize\arraybackslash}X}

\setlength{\parindent}{0pt}		 
\setlength{\parskip}{3mm}
\setlength{\headheight}{1cm}

\def\ebook{ebook}
\def\book{book}
\def\web{web}
\def\type{web}

\newcommand{\vect}[1]{\overrightarrow{\,\mathstrut#1\,}}

\def\Oij{$\left(\text{O}~;~\vect{\imath},~\vect{\jmath}\right)$}
\def\Oijk{$\left(\text{O}~;~\vect{\imath},~\vect{\jmath},~\vect{k}\right)$}
\def\Ouv{$\left(\text{O}~;~\vect{u},~\vect{v}\right)$}

\hypersetup{breaklinks=true, colorlinks = true, linkcolor = OliveGreen, urlcolor = OliveGreen, citecolor = OliveGreen, pdfauthor={Didier BONNEL - https://www.maths-cours.fr} } % supprime les bordures autour des liens

\renewcommand{\arg}[0]{\text{arg}}

\everymath{\displaystyle}

%================================================================================================================================
%
% Macros - Commandes
%
%================================================================================================================================

\newcommand\meta[2]{    			% Utilisé pour créer le post HTML.
	\def\titre{titre}
	\def\url{url}
	\def\arg{#1}
	\ifx\titre\arg
		\newcommand\maintitle{#2}
		\fancyhead[L]{#2}
		{\Large\sffamily \MakeUppercase{#2}}
		\vspace{1mm}\textcolor{mcvert}{\hrule}
	\fi 
	\ifx\url\arg
		\fancyfoot[L]{\href{https://www.maths-cours.fr#2}{\black \footnotesize{https://www.maths-cours.fr#2}}}
	\fi 
}


\newcommand\TitreC[1]{    		% Titre centré
     \needspace{3\baselineskip}
     \begin{center}\textbf{#1}\end{center}
}

\newcommand\newpar{    		% paragraphe
     \par
}

\newcommand\nosp {    		% commande vide (pas d'espace)
}
\newcommand{\id}[1]{} %ignore

\newcommand\boite[2]{				% Boite simple sans titre
	\vspace{5mm}
	\setlength{\fboxrule}{0.2mm}
	\setlength{\fboxsep}{5mm}	
	\fcolorbox{#1}{#1!3}{\makebox[\linewidth-2\fboxrule-2\fboxsep]{
  		\begin{minipage}[t]{\linewidth-2\fboxrule-4\fboxsep}\setlength{\parskip}{3mm}
  			 #2
  		\end{minipage}
	}}
	\vspace{5mm}
}

\newcommand\CBox[4]{				% Boites
	\vspace{5mm}
	\setlength{\fboxrule}{0.2mm}
	\setlength{\fboxsep}{5mm}
	
	\fcolorbox{#1}{#1!3}{\makebox[\linewidth-2\fboxrule-2\fboxsep]{
		\begin{minipage}[t]{1cm}\setlength{\parskip}{3mm}
	  		\textcolor{#1}{\LARGE{#2}}    
 	 	\end{minipage}  
  		\begin{minipage}[t]{\linewidth-2\fboxrule-4\fboxsep}\setlength{\parskip}{3mm}
			\raisebox{1.2mm}{\normalsize\sffamily{\textcolor{#1}{#3}}}						
  			 #4
  		\end{minipage}
	}}
	\vspace{5mm}
}

\newcommand\cadre[3]{				% Boites convertible html
	\par
	\vspace{2mm}
	\setlength{\fboxrule}{0.1mm}
	\setlength{\fboxsep}{5mm}
	\fcolorbox{#1}{white}{\makebox[\linewidth-2\fboxrule-2\fboxsep]{
  		\begin{minipage}[t]{\linewidth-2\fboxrule-4\fboxsep}\setlength{\parskip}{3mm}
			\raisebox{-2.5mm}{\sffamily \small{\textcolor{#1}{\MakeUppercase{#2}}}}		
			\par		
  			 #3
 	 		\end{minipage}
	}}
		\vspace{2mm}
	\par
}

\newcommand\bloc[3]{				% Boites convertible html sans bordure
     \needspace{2\baselineskip}
     {\sffamily \small{\textcolor{#1}{\MakeUppercase{#2}}}}    
		\par		
  			 #3
		\par
}

\newcommand\CHelp[1]{
     \CBox{Plum}{\faInfoCircle}{À RETENIR}{#1}
}

\newcommand\CUp[1]{
     \CBox{NavyBlue}{\faThumbsOUp}{EN PRATIQUE}{#1}
}

\newcommand\CInfo[1]{
     \CBox{Sepia}{\faArrowCircleRight}{REMARQUE}{#1}
}

\newcommand\CRedac[1]{
     \CBox{PineGreen}{\faEdit}{BIEN R\'EDIGER}{#1}
}

\newcommand\CError[1]{
     \CBox{Red}{\faExclamationTriangle}{ATTENTION}{#1}
}

\newcommand\TitreExo[2]{
\needspace{4\baselineskip}
 {\sffamily\large EXERCICE #1\ (\emph{#2 points})}
\vspace{5mm}
}

\newcommand\img[2]{
          \includegraphics[width=#2\paperwidth]{\imgdir#1}
}

\newcommand\imgsvg[2]{
       \begin{center}   \includegraphics[width=#2\paperwidth]{\imgsvgdir#1} \end{center}
}


\newcommand\Lien[2]{
     \href{#1}{#2 \tiny \faExternalLink}
}
\newcommand\mcLien[2]{
     \href{https~://www.maths-cours.fr/#1}{#2 \tiny \faExternalLink}
}

\newcommand{\euro}{\eurologo{}}

%================================================================================================================================
%
% Macros - Environement
%
%================================================================================================================================

\newenvironment{tex}{ %
}
{%
}

\newenvironment{indente}{ %
	\setlength\parindent{10mm}
}

{
	\setlength\parindent{0mm}
}

\newenvironment{corrige}{%
     \needspace{3\baselineskip}
     \medskip
     \textbf{\textsc{Corrigé}}
     \medskip
}
{
}

\newenvironment{extern}{%
     \begin{center}
     }
     {
     \end{center}
}

\NewEnviron{code}{%
	\par
     \boite{gray}{\texttt{%
     \BODY
     }}
     \par
}

\newenvironment{vbloc}{% boite sans cadre empeche saut de page
     \begin{minipage}[t]{\linewidth}
     }
     {
     \end{minipage}
}
\NewEnviron{h2}{%
    \needspace{3\baselineskip}
    \vspace{0.6cm}
	\noindent \MakeUppercase{\sffamily \large \BODY}
	\vspace{1mm}\textcolor{mcgris}{\hrule}\vspace{0.4cm}
	\par
}{}

\NewEnviron{h3}{%
    \needspace{3\baselineskip}
	\vspace{5mm}
	\textsc{\BODY}
	\par
}

\NewEnviron{margeneg}{ %
\begin{addmargin}[-1cm]{0cm}
\BODY
\end{addmargin}
}

\NewEnviron{html}{%
}

\begin{document}
\meta{url}{/exercices/fonction-ln-primitives-integrales-aire/}
\meta{pid}{1379}
\meta{titre}{[Bac] Fonction ln - Primitives - Intégrales - Aires}
\meta{type}{exercices}
%
Extrait d'un exercice du Bac S Pondichéry 2011.
\par
Le sujet complet est disponible ici : \mcLien{/exercices/revisions/fonction-calcul-aire-bac-s-pondichery-2011}{Bac S Pondichéry 2011}
\begin{h3}Partie I \end{h3}
On considère la fonction $f$ définie sur l'intervalle $\left]0; +\infty \right[$ par

\begin{center}
$f\left(x\right)=\ln \left(x\right)+1-\frac{1}{x}.$
\end{center}

\begin{enumerate}
     \item
     Déterminer les limites de la fonction $f$ aux bornes de son ensemble de définition.
     \item
     Étudier les variations de la fonction $f$ sur l'intervalle $\left]0; +\infty \right[$.
     \item
     En déduire le signe de $f\left(x\right)$ lorsque $x$ décrit l'intervalle $\left]0; +\infty \right[$.
     \item
     Montrer que la fonction $F$ définie sur l'intervalle $\left]0; +\infty \right[$ par $F\left(x\right)=x \ln x-\ln x$ est une primitive de la fonction $f$ sur cet intervalle.
     \item
     Démontrer que la fonction $F$ est strictement croissante sur l'intervalle $\left]1; +\infty \right[$.
     \item
     Montrer que l'équation $F\left(x\right)=1-\frac{1}{\text{e}}$ admet une unique solution dans l'intervalle $\left]1;+\infty \right[$ qu'on note $\alpha $.
     \item
     Donner un encadrement de $\alpha $ d'amplitude $10^{-1}$.
\end{enumerate}
\begin{h3} Partie II \end{h3}
Soit $g$ et $h$ les fonctions définies sur l'intervalle $\left]0; +\infty \right[$ par :
\par
$g\left(x\right)=\frac{1}{x}$ ~ et ~ $ h\left(x\right)=\ln \left(x\right)+1.$
\par
Sur le graphique ci-dessous, on a représenté dans un repère orthonormal, les courbes $\left(\mathscr C_{g}\right)$ et $\left(\mathscr C_{h}\right)$ représentatives des fonctions $g$ et $h$.

\begin{center}
\imgsvg{Bac_S_Pondichery_2011-fonction2}{0.3}% alt="Fonction ln - Primitives - Intégrales" style="width:40rem" 
\end{center}

\begin{enumerate}
     \item
     $A$ est le point d'intersection de la courbe $\left(\mathscr C_{h}\right)$ et de l'axe des abscisses. Déterminer les coordonnées du point $A$.
     \item
     $P$ est le point d'intersection des courbes $\left(\mathscr C_{g}\right)$ et $\left(\mathscr C_{h}\right)$. Justifier que les coordonnées du point $P$ sont $\left(1 ; 1\right)$.
     \item
     On note $\mathscr A$ l'aire du domaine délimité par les courbes $\left(\mathscr C_{g}\right)$, $\left(\mathscr C_{h}\right)$ et les droites d'équations respectives $x=\frac{1}{\text{e}}$ et $x=1$ (domaine grisé sur le graphique).


     \begin{enumerate}[label=\alph*.]
          \item
          Exprimer l'aire $\mathscr A$ à l'aide de la fonction $f$ définie dans la \textbf{partie II}.
          \item
          Montrer que $\mathscr A=1-\frac{1}{\text{e}}$.
     \end{enumerate}
     \item
     Soit $t$ un nombre réel de l'intervalle $\left]1; +\infty \right[$. On note $\mathscr B_{t}$ l'aire du domaine délimité par les droites d'équations respectives $x=1, x=t$ et les courbes $\left(\mathscr C_{g}\right)$ et $\left(\mathscr C_{h}\right)$ (domaine hachuré sur le graphique).
     \par
     On souhaite déterminer une valeur de $t$ telle que $\mathscr A=\mathscr B_{t}$.
     \begin{enumerate}[label=\alph*.]
          \item
          Montrer que $\mathscr B_{t}=t \ln \left(t\right)-\ln \left(t\right)$.
          \item
          Conclure.
     \end{enumerate}
\end{enumerate}
\begin{corrige}
     \begin{h3} Partie I \end{h3}
     \begin{enumerate}
          \item
          $\lim_{x\rightarrow 0^+} \ln\left(x\right)=-\infty   $  et  $ \lim_{x\rightarrow 0^+} \frac{1}{x}=+\infty $
          \par
          Par conséquent, par somme $\lim_{x\rightarrow 0^+} f\left(x\right)=-\infty $
          \par
          $\lim\limits_{x\rightarrow +\infty } \ln\left(x\right)=+\infty   $ et $ \lim\limits_{x\rightarrow +\infty } \frac{1}{x}=0$
          \par
          Et par somme $\lim\limits_{x\rightarrow +\infty }f\left(x\right)=+\infty $
          \item
          $f^{\prime}\left(x\right)= \frac{1}{x}+\frac{1}{x^{2}}$
          \par
          Sur l'intervalle $\left]0;+\infty \right[$, $\frac{1}{x} > 0$ donc $f^{\prime}\left(x\right) > 0$ et par conséquent $f$ est \textbf{strictement croissante}.
          \item
          Comme $f\left(1\right)=\ln\left(1\right)+1-\frac{1}{1}=0$ et comme $f$ est strictement croissante, $f$ est strictement négative sur $\left]0;1\right[$ et strictement positive sur $\left]1 ;+\infty \right[$.
          \par
          Le tableau de signe de $f$ est :
          <img src="/assets/imgsvg/mc-0346.png" alt="" class="aligncenter size-full  img-pc" />
%##
% type=table; width=45
%--
% x|   0   ~    1   ~   3  ~   +\infty 
% x-3|  ~       -               :        -   :0   +     ~
% 4-3x|  ~       +               :0      -   :   -     ~
% (x-3)(4-3x)|  ~    -       :0     +  :0   -   ~
%--
\begin{center}
 \begin{extern}%style="width:45rem" alt="Exercice"
    \resizebox{11cm}{!}{
       \definecolor{dark}{gray}{0.1}
       \definecolor{light}{gray}{0.8}
       \tikzstyle{fleche}=[->,>=latex]
       \begin{tikzpicture}[scale=.8, line width=.5pt, dark]
       \def\width{.15}
       \def\height{.10}
       \draw (0, -10*\height) -- (72*\width, -10*\height);
       \draw (10*\width, 0*\height) -- (10*\width, -10*\height);
       \node (l0c0) at (5*\width,-5*\height) {$ x $};
       \node (l0c1) at (14*\width,-5*\height) {$ 0 $};
       \node (l0c2) at (23*\width,-5*\height) {$ ~ $};
       \node (l0c3) at (32*\width,-5*\height) {$ 1 $};
       \node (l0c4) at (41*\width,-5*\height) {$ ~ $};
       \node (l0c5) at (50*\width,-5*\height) {$ 3 $};
       \node (l0c6) at (59*\width,-5*\height) {$ ~ $};
       \node (l0c7) at (68*\width,-5*\height) {$ +\infty $};
       \draw (0, -20*\height) -- (72*\width, -20*\height);
       \draw (10*\width, -10*\height) -- (10*\width, -20*\height);
       \node (l1c0) at (5*\width,-15*\height) {$ x-3 $};
       \node (l1c1) at (14*\width,-15*\height) {$ ~ $};
       \node (l1c2) at (23*\width,-15*\height) {$ - $};
       \draw[light] (32*\width, -10*\height) -- (32*\width, -20*\height);
       \node (l1c3) at (32*\width,-15*\height) {$  $};
       \node (l1c4) at (41*\width,-15*\height) {$ - $};
       \draw[light] (50*\width, -10*\height) -- (50*\width, -20*\height);
       \node (l1c5) at (50*\width,-15*\height) {$ 0 $};
       \node (l1c6) at (59*\width,-15*\height) {$ + $};
       \node (l1c7) at (68*\width,-15*\height) {$ ~ $};
       \draw (0, -30*\height) -- (72*\width, -30*\height);
       \draw (10*\width, -20*\height) -- (10*\width, -30*\height);
       \node (l2c0) at (5*\width,-25*\height) {$ 4-3x $};
       \node (l2c1) at (14*\width,-25*\height) {$ ~ $};
       \node (l2c2) at (23*\width,-25*\height) {$ + $};
       \draw[light] (32*\width, -20*\height) -- (32*\width, -30*\height);
       \node (l2c3) at (32*\width,-25*\height) {$ 0 $};
       \node (l2c4) at (41*\width,-25*\height) {$ - $};
       \draw[light] (50*\width, -20*\height) -- (50*\width, -30*\height);
       \node (l2c5) at (50*\width,-25*\height) {$  $};
       \node (l2c6) at (59*\width,-25*\height) {$ - $};
       \node (l2c7) at (68*\width,-25*\height) {$ ~ $};
       \draw (0, -40*\height) -- (72*\width, -40*\height);
       \draw (10*\width, -30*\height) -- (10*\width, -40*\height);
       \node (l3c0) at (5*\width,-35*\height) {$ (x-3)(4-3x) $};
       \node (l3c1) at (14*\width,-35*\height) {$ ~ $};
       \node (l3c2) at (23*\width,-35*\height) {$ - $};
       \draw[light] (32*\width, -30*\height) -- (32*\width, -40*\height);
       \node (l3c3) at (32*\width,-35*\height) {$ 0 $};
       \node (l3c4) at (41*\width,-35*\height) {$ + $};
       \draw[light] (50*\width, -30*\height) -- (50*\width, -40*\height);
       \node (l3c5) at (50*\width,-35*\height) {$ 0 $};
       \node (l3c6) at (59*\width,-35*\height) {$ - $};
       \node (l3c7) at (68*\width,-35*\height) {$ ~ $};
       \draw (0, 0) rectangle (72*\width, -40*\height);

       \end{tikzpicture}
      }
   \end{extern}
\end{center}
%##
\item
          On calcule la dérivée $F^{\prime}\left(x\right)$ :
          \par
          $F^{\prime}\left(x\right)=\ln\left(x\right)+x\times \frac{1}{x}-\frac{1}{x}=\ln\left(x\right)+1-\frac{1}{x}=f\left(x\right)$
          \par
          Donc $F$ est une primitive de $f$ sur $\left]0;+\infty \right[$.
          \item
          La dérivée de $F$ est $f$ et est strictement positive sur $\left]1;+\infty \right[$ d'après \textbf{3.}. Donc $F$ est strictement croissante sur cet intervalle.
          \item
          $F\left(1\right)=0$
          \par
          $\lim\limits_{x\rightarrow +\infty }F\left(x\right)=\lim\limits_{x\rightarrow +\infty }x\ln\left(x\right)-\ln\left(x\right)=\lim\limits_{x\rightarrow +\infty }x\left(\ln\left(x\right)-\frac{\ln\left(x\right)}{x}\right)$
          \par
          Or $\lim\limits_{x\rightarrow +\infty }\frac{\ln\left(x\right)}{x}=0$ (\mcLien{/cours/fonction-logarithme-neperien#t60}{Croissance comparée})
          \par
          donc (par différence et produit) $\lim\limits_{x\rightarrow +\infty }F\left(x\right)=+\infty $
          \par
          Sur l'intervalle $\left]1;+\infty \right[$, $F$ est \textbf{continue} car dérivable, \textbf{strictement croissante} et  $1-\frac{1}{e}$ est compris entre $F\left(1\right)=0$ et  $\lim\limits_{x\rightarrow +\infty }F\left(x\right)=+\infty $.
          \par
          D'après le \mcLien{/cours/fonctions-continues#t70}{corolaire du théorème des valeurs intermédiaires},  l'équation $F\left(x\right)=1-\frac{1}{e}$ admet une unique solution sur l'intervalle $\left]1;+\infty \right[$.
          \item
          A la calculatrice, on trouve : $F\left(1,9\right)\approx -0,05$ et $F\left(2\right)\approx 0,06$ donc $1,9 < \alpha  < 2$.
     \end{enumerate}
     \begin{h3} Partie II \end{h3}
     \begin{enumerate}
          \item
          L'abscisse du point $A$ est solution de l'équation : $h\left(x\right)=0$. Donc :
          \par
          $\ln\left(x_{A}\right)+1=0$
          \par
          $\ln\left(x_{A}\right)=-1$
          \par
          $x_{A}=e^{-1}=\frac{1}{e}$
          \par
          Donc $A\left(\frac{1}{e};0\right)$.
          \item
          L'abscisse du point $P$ vérifie l'équation :
          \par
          $\frac{1}{x}=\ln\left(x\right)+1$
          \par
          $\ln\left(x\right)+1-\frac{1}{x}=0$
          \par
          $f\left(x\right)=0$
          \par
          Donc d'après la\textbf{ partie II},  $x_{P}=1$ et $y_{P}=g\left(x_{P}\right)=g\left(1\right)=1$
          \par
          Donc $P\left(1;1\right)$
          \item
          \begin{enumerate}[label=\alph*.]
               \item
               Sur l'intervalle $\left[\frac{1}{e} ; 1\right]$, $g\geqslant h$. L'aire $\mathscr A$ est donc:
               \par
               $\mathscr A=\int_{1/e}^{1}g\left(x\right)-h\left(x\right)dx = \int_{1/e}^{1}-f\left(x\right)dx = -\int_{1/e}^{1}f\left(x\right)dx$
               \item
               $\mathscr A=-\left[F\left(x\right)\right]_{1/e}^{1} = -F\left(1\right)+F\left(\frac{1}{e}\right) = 1\ln 1-\ln 1+\frac{1}{e} \ln \frac{1}{e}-\ln \frac{1}{e} = 1-\frac{1}{e}$
               \par
               car $\ln 1 = 0 $ et $ \ln \frac{1}{e} = -\ln e = -1$
          \end{enumerate}
          \item
          \begin{enumerate}[label=\alph*.]
               \item
               Sur l'intervalle $\left[1 ; +\infty \right[$, $g\leqslant h$. Par conséquent :
               \par
               $\mathscr B_{t}=\int_{1}^{t}h\left(x\right)-g\left(x\right)dx = \int_{1}^{t}f\left(x\right)dx = F\left(t\right)-F\left(1\right) = F\left(t\right) = t \ln t-\ln t$
               \item
               $\mathscr A=\mathscr B_{t}   \Leftrightarrow    F\left(t\right) = 1-\frac{1}{e}$
               \par
               D'après la question \textbf{6} de la partie précédente, cette équation admet $t=\alpha $ comme unique solution.
          \end{enumerate}
     \end{enumerate}
\end{corrige}

\end{document}