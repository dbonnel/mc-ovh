\documentclass[a4paper]{article}

%================================================================================================================================
%
% Packages
%
%================================================================================================================================

\usepackage[T1]{fontenc} 	% pour caractères accentués
\usepackage[utf8]{inputenc}  % encodage utf8
\usepackage[french]{babel}	% langue : français
\usepackage{fourier}			% caractères plus lisibles
\usepackage[dvipsnames]{xcolor} % couleurs
\usepackage{fancyhdr}		% réglage header footer
\usepackage{needspace}		% empêcher sauts de page mal placés
\usepackage{graphicx}		% pour inclure des graphiques
\usepackage{enumitem,cprotect}		% personnalise les listes d'items (nécessaire pour ol, al ...)
\usepackage{hyperref}		% Liens hypertexte
\usepackage{pstricks,pst-all,pst-node,pstricks-add,pst-math,pst-plot,pst-tree,pst-eucl} % pstricks
\usepackage[a4paper,includeheadfoot,top=2cm,left=3cm, bottom=2cm,right=3cm]{geometry} % marges etc.
\usepackage{comment}			% commentaires multilignes
\usepackage{amsmath,environ} % maths (matrices, etc.)
\usepackage{amssymb,makeidx}
\usepackage{bm}				% bold maths
\usepackage{tabularx}		% tableaux
\usepackage{colortbl}		% tableaux en couleur
\usepackage{fontawesome}		% Fontawesome
\usepackage{environ}			% environment with command
\usepackage{fp}				% calculs pour ps-tricks
\usepackage{multido}			% pour ps tricks
\usepackage[np]{numprint}	% formattage nombre
\usepackage{tikz,tkz-tab} 			% package principal TikZ
\usepackage{pgfplots}   % axes
\usepackage{mathrsfs}    % cursives
\usepackage{calc}			% calcul taille boites
\usepackage[scaled=0.875]{helvet} % font sans serif
\usepackage{svg} % svg
\usepackage{scrextend} % local margin
\usepackage{scratch} %scratch
\usepackage{multicol} % colonnes
%\usepackage{infix-RPN,pst-func} % formule en notation polanaise inversée
\usepackage{listings}

%================================================================================================================================
%
% Réglages de base
%
%================================================================================================================================

\lstset{
language=Python,   % R code
literate=
{á}{{\'a}}1
{à}{{\`a}}1
{ã}{{\~a}}1
{é}{{\'e}}1
{è}{{\`e}}1
{ê}{{\^e}}1
{í}{{\'i}}1
{ó}{{\'o}}1
{õ}{{\~o}}1
{ú}{{\'u}}1
{ü}{{\"u}}1
{ç}{{\c{c}}}1
{~}{{ }}1
}


\definecolor{codegreen}{rgb}{0,0.6,0}
\definecolor{codegray}{rgb}{0.5,0.5,0.5}
\definecolor{codepurple}{rgb}{0.58,0,0.82}
\definecolor{backcolour}{rgb}{0.95,0.95,0.92}

\lstdefinestyle{mystyle}{
    backgroundcolor=\color{backcolour},   
    commentstyle=\color{codegreen},
    keywordstyle=\color{magenta},
    numberstyle=\tiny\color{codegray},
    stringstyle=\color{codepurple},
    basicstyle=\ttfamily\footnotesize,
    breakatwhitespace=false,         
    breaklines=true,                 
    captionpos=b,                    
    keepspaces=true,                 
    numbers=left,                    
xleftmargin=2em,
framexleftmargin=2em,            
    showspaces=false,                
    showstringspaces=false,
    showtabs=false,                  
    tabsize=2,
    upquote=true
}

\lstset{style=mystyle}


\lstset{style=mystyle}
\newcommand{\imgdir}{C:/laragon/www/newmc/assets/imgsvg/}
\newcommand{\imgsvgdir}{C:/laragon/www/newmc/assets/imgsvg/}

\definecolor{mcgris}{RGB}{220, 220, 220}% ancien~; pour compatibilité
\definecolor{mcbleu}{RGB}{52, 152, 219}
\definecolor{mcvert}{RGB}{125, 194, 70}
\definecolor{mcmauve}{RGB}{154, 0, 215}
\definecolor{mcorange}{RGB}{255, 96, 0}
\definecolor{mcturquoise}{RGB}{0, 153, 153}
\definecolor{mcrouge}{RGB}{255, 0, 0}
\definecolor{mclightvert}{RGB}{205, 234, 190}

\definecolor{gris}{RGB}{220, 220, 220}
\definecolor{bleu}{RGB}{52, 152, 219}
\definecolor{vert}{RGB}{125, 194, 70}
\definecolor{mauve}{RGB}{154, 0, 215}
\definecolor{orange}{RGB}{255, 96, 0}
\definecolor{turquoise}{RGB}{0, 153, 153}
\definecolor{rouge}{RGB}{255, 0, 0}
\definecolor{lightvert}{RGB}{205, 234, 190}
\setitemize[0]{label=\color{lightvert}  $\bullet$}

\pagestyle{fancy}
\renewcommand{\headrulewidth}{0.2pt}
\fancyhead[L]{maths-cours.fr}
\fancyhead[R]{\thepage}
\renewcommand{\footrulewidth}{0.2pt}
\fancyfoot[C]{}

\newcolumntype{C}{>{\centering\arraybackslash}X}
\newcolumntype{s}{>{\hsize=.35\hsize\arraybackslash}X}

\setlength{\parindent}{0pt}		 
\setlength{\parskip}{3mm}
\setlength{\headheight}{1cm}

\def\ebook{ebook}
\def\book{book}
\def\web{web}
\def\type{web}

\newcommand{\vect}[1]{\overrightarrow{\,\mathstrut#1\,}}

\def\Oij{$\left(\text{O}~;~\vect{\imath},~\vect{\jmath}\right)$}
\def\Oijk{$\left(\text{O}~;~\vect{\imath},~\vect{\jmath},~\vect{k}\right)$}
\def\Ouv{$\left(\text{O}~;~\vect{u},~\vect{v}\right)$}

\hypersetup{breaklinks=true, colorlinks = true, linkcolor = OliveGreen, urlcolor = OliveGreen, citecolor = OliveGreen, pdfauthor={Didier BONNEL - https://www.maths-cours.fr} } % supprime les bordures autour des liens

\renewcommand{\arg}[0]{\text{arg}}

\everymath{\displaystyle}

%================================================================================================================================
%
% Macros - Commandes
%
%================================================================================================================================

\newcommand\meta[2]{    			% Utilisé pour créer le post HTML.
	\def\titre{titre}
	\def\url{url}
	\def\arg{#1}
	\ifx\titre\arg
		\newcommand\maintitle{#2}
		\fancyhead[L]{#2}
		{\Large\sffamily \MakeUppercase{#2}}
		\vspace{1mm}\textcolor{mcvert}{\hrule}
	\fi 
	\ifx\url\arg
		\fancyfoot[L]{\href{https://www.maths-cours.fr#2}{\black \footnotesize{https://www.maths-cours.fr#2}}}
	\fi 
}


\newcommand\TitreC[1]{    		% Titre centré
     \needspace{3\baselineskip}
     \begin{center}\textbf{#1}\end{center}
}

\newcommand\newpar{    		% paragraphe
     \par
}

\newcommand\nosp {    		% commande vide (pas d'espace)
}
\newcommand{\id}[1]{} %ignore

\newcommand\boite[2]{				% Boite simple sans titre
	\vspace{5mm}
	\setlength{\fboxrule}{0.2mm}
	\setlength{\fboxsep}{5mm}	
	\fcolorbox{#1}{#1!3}{\makebox[\linewidth-2\fboxrule-2\fboxsep]{
  		\begin{minipage}[t]{\linewidth-2\fboxrule-4\fboxsep}\setlength{\parskip}{3mm}
  			 #2
  		\end{minipage}
	}}
	\vspace{5mm}
}

\newcommand\CBox[4]{				% Boites
	\vspace{5mm}
	\setlength{\fboxrule}{0.2mm}
	\setlength{\fboxsep}{5mm}
	
	\fcolorbox{#1}{#1!3}{\makebox[\linewidth-2\fboxrule-2\fboxsep]{
		\begin{minipage}[t]{1cm}\setlength{\parskip}{3mm}
	  		\textcolor{#1}{\LARGE{#2}}    
 	 	\end{minipage}  
  		\begin{minipage}[t]{\linewidth-2\fboxrule-4\fboxsep}\setlength{\parskip}{3mm}
			\raisebox{1.2mm}{\normalsize\sffamily{\textcolor{#1}{#3}}}						
  			 #4
  		\end{minipage}
	}}
	\vspace{5mm}
}

\newcommand\cadre[3]{				% Boites convertible html
	\par
	\vspace{2mm}
	\setlength{\fboxrule}{0.1mm}
	\setlength{\fboxsep}{5mm}
	\fcolorbox{#1}{white}{\makebox[\linewidth-2\fboxrule-2\fboxsep]{
  		\begin{minipage}[t]{\linewidth-2\fboxrule-4\fboxsep}\setlength{\parskip}{3mm}
			\raisebox{-2.5mm}{\sffamily \small{\textcolor{#1}{\MakeUppercase{#2}}}}		
			\par		
  			 #3
 	 		\end{minipage}
	}}
		\vspace{2mm}
	\par
}

\newcommand\bloc[3]{				% Boites convertible html sans bordure
     \needspace{2\baselineskip}
     {\sffamily \small{\textcolor{#1}{\MakeUppercase{#2}}}}    
		\par		
  			 #3
		\par
}

\newcommand\CHelp[1]{
     \CBox{Plum}{\faInfoCircle}{À RETENIR}{#1}
}

\newcommand\CUp[1]{
     \CBox{NavyBlue}{\faThumbsOUp}{EN PRATIQUE}{#1}
}

\newcommand\CInfo[1]{
     \CBox{Sepia}{\faArrowCircleRight}{REMARQUE}{#1}
}

\newcommand\CRedac[1]{
     \CBox{PineGreen}{\faEdit}{BIEN R\'EDIGER}{#1}
}

\newcommand\CError[1]{
     \CBox{Red}{\faExclamationTriangle}{ATTENTION}{#1}
}

\newcommand\TitreExo[2]{
\needspace{4\baselineskip}
 {\sffamily\large EXERCICE #1\ (\emph{#2 points})}
\vspace{5mm}
}

\newcommand\img[2]{
          \includegraphics[width=#2\paperwidth]{\imgdir#1}
}

\newcommand\imgsvg[2]{
       \begin{center}   \includegraphics[width=#2\paperwidth]{\imgsvgdir#1} \end{center}
}


\newcommand\Lien[2]{
     \href{#1}{#2 \tiny \faExternalLink}
}
\newcommand\mcLien[2]{
     \href{https~://www.maths-cours.fr/#1}{#2 \tiny \faExternalLink}
}

\newcommand{\euro}{\eurologo{}}

%================================================================================================================================
%
% Macros - Environement
%
%================================================================================================================================

\newenvironment{tex}{ %
}
{%
}

\newenvironment{indente}{ %
	\setlength\parindent{10mm}
}

{
	\setlength\parindent{0mm}
}

\newenvironment{corrige}{%
     \needspace{3\baselineskip}
     \medskip
     \textbf{\textsc{Corrigé}}
     \medskip
}
{
}

\newenvironment{extern}{%
     \begin{center}
     }
     {
     \end{center}
}

\NewEnviron{code}{%
	\par
     \boite{gray}{\texttt{%
     \BODY
     }}
     \par
}

\newenvironment{vbloc}{% boite sans cadre empeche saut de page
     \begin{minipage}[t]{\linewidth}
     }
     {
     \end{minipage}
}
\NewEnviron{h2}{%
    \needspace{3\baselineskip}
    \vspace{0.6cm}
	\noindent \MakeUppercase{\sffamily \large \BODY}
	\vspace{1mm}\textcolor{mcgris}{\hrule}\vspace{0.4cm}
	\par
}{}

\NewEnviron{h3}{%
    \needspace{3\baselineskip}
	\vspace{5mm}
	\textsc{\BODY}
	\par
}

\NewEnviron{margeneg}{ %
\begin{addmargin}[-1cm]{0cm}
\BODY
\end{addmargin}
}

\NewEnviron{html}{%
}

\begin{document}
\meta{url}{/cours/fonctions-lineaires-et-fonctions-affines/}
\meta{pid}{1568}
\meta{titre}{Fonctions linéaires et fonctions affines}
\meta{type}{cours}
\begin{h2}1. Fonctions linéaires\end{h2}
\cadre{bleu}{Définition}{% id="d10"
     Une fonction \textbf{linéaire} est une fonction définie par une formule du type : $x\mapsto ax$ .
     \par
     $a$ s'appelle le \textbf{coefficient directeur}.
}
\bloc{orange}{Exemple}{% id="e10"
     La fonction qui à tout nombre réel associe son double est une fonction linéaire de coefficient directeur $2$.
     \par
     On la note : $f :  x\mapsto 2x$.
}
\cadre{vert}{Propriété}{% id="p20"
     Pour une fonction linéaire $f$, les valeurs de $f\left(x\right)$ sont \textbf{proportionnelles} aux valeurs de $x$
}
\bloc{orange}{Exemple}{% id="e20"
     Voici un tableau de valeur de la fonction $f :  x\mapsto 2x$ :
     \begin{center}
          \begin{tabularx}{0.6\linewidth}{|*{8}{>{\centering \arraybackslash }X|}}%class="compact" width="600"
               \hline
               $x$ & -3 & -2 & -1 & 0 & 1 & 2 & 3
               \\ \hline
               $f\left(x\right)$ & -6 & -4 & -2 & 0 & 2 & 4 & 6
               \\ \hline
          \end{tabularx}
     \end{center}
     Ce tableau est un tableau de proportionnalité.
}
\cadre{vert}{Propriété}{% id="p30"
     La représentation graphique d'une fonction linéaire est une droite qui passe par le point $O$ origine du repère.
}
\begin{center}
     \begin{extern}%width="380" alt="fonction linéaire"
          \resizebox{7cm}{!}{%
               % -+-+-+ variables modifiables
               \def\fonction{2*x}
               \def\xmin{-3.6}
               \def\xmax{4.6}
               \def\ymin{-4.6}
               \def\ymax{7.6}
               \def\xunit{1.5}  % unités en cm
               \def\yunit{1.5}
               \psset{xunit=\xunit,yunit=\yunit,algebraic=true}
               \fontsize{15pt}{15pt}\selectfont
               \begin{pspicture*}[linewidth=1pt,arrowsize=6pt](\xmin,\ymin)(\xmax,\ymax)
                    \psgrid[gridcolor=mcgris, subgriddiv=1, gridlabels=0pt](\xmin,\ymin)(\xmax,\ymax)
                    \psaxes[linewidth=0.75pt,Dx=1,Dy=1]{->}(0,0)(\xmin,\ymin)(\xmax,\ymax)
                    \psplot[plotpoints=2000,linecolor=blue]{\xmin}{\xmax}{\fonction}
                    \rput[tr](-0.2,-0.2){$O$}
               \end{pspicture*}
          }
     \end{extern}
\end{center}
\begin{center}\textit{Représentation graphique de la fonction linéaire $x\mapsto 2x$}\end{center}
\begin{h2}2. Fonctions affines\end{h2}
\cadre{bleu}{Définition}{% id="d50"
     Une fonction \textbf{affine} est une fonction définie par une formule du type : $x\mapsto ax+b$.
     \par
     $a$ s'appelle le \textbf{coefficient directeur }et $b$ s'appelle \textbf{l'ordonnée à l'origine}.
}
\bloc{cyan}{Remarques}{% id="r50"
     Si $b=0$, la fonction est linéaire. Les fonctions linéaires sont donc des cas particuliers des fonctions affines.
}
\bloc{orange}{Exemple}{% id="e50"
     La fonction $f : x\mapsto -2x+1$ est une fonction affine avec $a=-2$ et $b=1$
}
\cadre{rouge}{Théorème}{% id="t60"
     Soit une fonction affine $f : x\mapsto ax+b$.
     \par
     Pour tous nombres réels distincts $x_{1}$ et $x_{2}$, le coefficient directeur $a$ est égal à :
     \begin{center}
          $a=\frac{f\left(x_{2}\right)-f\left(x_{1}\right)}{x_{2}-x_{1}}$
     \end{center}
}
\bloc{orange}{Exercice corrigé}{% id="e60"
     \textbf{Déterminer la fonction affine $f$ telle $f\left(2\right)=1$ et $f\left(4\right)=5$.}
     \par
     $f$ étant une fonction affine, la formule donnant $f\left(x\right)$ est de la forme $f\left(x\right)=ax+b$.
     \par
     D'après le théorème précédent, le coefficient directeur $a$ est égal à :
     \par
     $a=\frac{f\left(4\right)-f\left(2\right)}{4-2}=\frac{5-1}{4-2}=\frac{4}{2}=2$
     \par
     On a donc $f\left(x\right)=2x+b$
     \par
     Pour trouver la valeur de $b$, on utilise le fait que $f\left(2\right)=1$ donc $2\times 2+b=1$.
     \par
     $4+b=1$
     \par
     $b=1-4$
     \par
     $b=-3$
     \par
     Par conséquent $f$ est définie par $f\left(x\right)=2x-3$
}
\cadre{vert}{Propriété}{% id="p70"
     La représentation graphique d'une fonction affine est une droite.
}
\bloc{cyan}{Remarque}{% id="r70"
     Pour tracer une droite, il suffit de connaître deux points de cette droite. Il suffit donc de calculer les images de deux nombres pour tracer la représentation graphique d'une fonction affine.
}
\bloc{orange}{Exemple}{% id="e70"
     On veut tracer la représentation graphique de la fonction $f : x\mapsto -2x+1$.
     \par
     Cette représentation graphique est une droite.
     \begin{itemize}
          \item comme $f\left(0\right)=-2\times 0+1=1$, cette droite passe par le point $A\left(0;1\right)$
          \item comme $f\left(1\right)=-2\times 1+1=-1$, cette droite passe par le point $B\left(1;-1\right)$
     \end{itemize}
     On en déduit la représentation ci-dessous :
     \begin{center}
          \begin{extern}%width="380" alt="fonction affine"
               \resizebox{7cm}{!}{%
                    % -+-+-+ variables modifiables
                    \def\fonction{1-2*x}
                    \def\xmin{-3.6}
                    \def\xmax{4.6}
                    \def\ymin{-4.6}
                    \def\ymax{7.6}
                    \def\xunit{1.5}  % unités en cm
                    \def\yunit{1.5}
                    \psset{xunit=\xunit,yunit=\yunit,algebraic=true}
                    \fontsize{15pt}{15pt}\selectfont
                    \begin{pspicture*}[linewidth=1pt,arrowsize=6pt](\xmin,\ymin)(\xmax,\ymax)
                         \psgrid[gridcolor=mcgris, subgriddiv=1, gridlabels=0pt](\xmin,\ymin)(\xmax,\ymax)
                         \psaxes[linewidth=0.75pt,Dx=1,Dy=1]{->}(0,0)(\xmin,\ymin)(\xmax,\ymax)
                         \psplot[plotpoints=2000,linecolor=blue]{\xmin}{\xmax}{\fonction}
                         \psdots[linecolor=red,dotsize=4pt,dotstyle=*](0,1)(1,-1)
                         \rput[tr](-0.2,-0.2){$O$}
                         \rput[bl](0.1,1.1){$\red A$}
                         \rput[bl](1.1,-0.9){$\red B$}
                    \end{pspicture*}
               }
          \end{extern}
     \end{center}
     \begin{center}\textit{Représentation graphique de la fonction $x\mapsto -2x+1$}\end{center}
}

\end{document}