\documentclass[a4paper]{article}

%================================================================================================================================
%
% Packages
%
%================================================================================================================================

\usepackage[T1]{fontenc} 	% pour caractères accentués
\usepackage[utf8]{inputenc}  % encodage utf8
\usepackage[french]{babel}	% langue : français
\usepackage{fourier}			% caractères plus lisibles
\usepackage[dvipsnames]{xcolor} % couleurs
\usepackage{fancyhdr}		% réglage header footer
\usepackage{needspace}		% empêcher sauts de page mal placés
\usepackage{graphicx}		% pour inclure des graphiques
\usepackage{enumitem,cprotect}		% personnalise les listes d'items (nécessaire pour ol, al ...)
\usepackage{hyperref}		% Liens hypertexte
\usepackage{pstricks,pst-all,pst-node,pstricks-add,pst-math,pst-plot,pst-tree,pst-eucl} % pstricks
\usepackage[a4paper,includeheadfoot,top=2cm,left=3cm, bottom=2cm,right=3cm]{geometry} % marges etc.
\usepackage{comment}			% commentaires multilignes
\usepackage{amsmath,environ} % maths (matrices, etc.)
\usepackage{amssymb,makeidx}
\usepackage{bm}				% bold maths
\usepackage{tabularx}		% tableaux
\usepackage{colortbl}		% tableaux en couleur
\usepackage{fontawesome}		% Fontawesome
\usepackage{environ}			% environment with command
\usepackage{fp}				% calculs pour ps-tricks
\usepackage{multido}			% pour ps tricks
\usepackage[np]{numprint}	% formattage nombre
\usepackage{tikz,tkz-tab} 			% package principal TikZ
\usepackage{pgfplots}   % axes
\usepackage{mathrsfs}    % cursives
\usepackage{calc}			% calcul taille boites
\usepackage[scaled=0.875]{helvet} % font sans serif
\usepackage{svg} % svg
\usepackage{scrextend} % local margin
\usepackage{scratch} %scratch
\usepackage{multicol} % colonnes
%\usepackage{infix-RPN,pst-func} % formule en notation polanaise inversée
\usepackage{listings}

%================================================================================================================================
%
% Réglages de base
%
%================================================================================================================================

\lstset{
language=Python,   % R code
literate=
{á}{{\'a}}1
{à}{{\`a}}1
{ã}{{\~a}}1
{é}{{\'e}}1
{è}{{\`e}}1
{ê}{{\^e}}1
{í}{{\'i}}1
{ó}{{\'o}}1
{õ}{{\~o}}1
{ú}{{\'u}}1
{ü}{{\"u}}1
{ç}{{\c{c}}}1
{~}{{ }}1
}


\definecolor{codegreen}{rgb}{0,0.6,0}
\definecolor{codegray}{rgb}{0.5,0.5,0.5}
\definecolor{codepurple}{rgb}{0.58,0,0.82}
\definecolor{backcolour}{rgb}{0.95,0.95,0.92}

\lstdefinestyle{mystyle}{
    backgroundcolor=\color{backcolour},   
    commentstyle=\color{codegreen},
    keywordstyle=\color{magenta},
    numberstyle=\tiny\color{codegray},
    stringstyle=\color{codepurple},
    basicstyle=\ttfamily\footnotesize,
    breakatwhitespace=false,         
    breaklines=true,                 
    captionpos=b,                    
    keepspaces=true,                 
    numbers=left,                    
xleftmargin=2em,
framexleftmargin=2em,            
    showspaces=false,                
    showstringspaces=false,
    showtabs=false,                  
    tabsize=2,
    upquote=true
}

\lstset{style=mystyle}


\lstset{style=mystyle}
\newcommand{\imgdir}{C:/laragon/www/newmc/assets/imgsvg/}
\newcommand{\imgsvgdir}{C:/laragon/www/newmc/assets/imgsvg/}

\definecolor{mcgris}{RGB}{220, 220, 220}% ancien~; pour compatibilité
\definecolor{mcbleu}{RGB}{52, 152, 219}
\definecolor{mcvert}{RGB}{125, 194, 70}
\definecolor{mcmauve}{RGB}{154, 0, 215}
\definecolor{mcorange}{RGB}{255, 96, 0}
\definecolor{mcturquoise}{RGB}{0, 153, 153}
\definecolor{mcrouge}{RGB}{255, 0, 0}
\definecolor{mclightvert}{RGB}{205, 234, 190}

\definecolor{gris}{RGB}{220, 220, 220}
\definecolor{bleu}{RGB}{52, 152, 219}
\definecolor{vert}{RGB}{125, 194, 70}
\definecolor{mauve}{RGB}{154, 0, 215}
\definecolor{orange}{RGB}{255, 96, 0}
\definecolor{turquoise}{RGB}{0, 153, 153}
\definecolor{rouge}{RGB}{255, 0, 0}
\definecolor{lightvert}{RGB}{205, 234, 190}
\setitemize[0]{label=\color{lightvert}  $\bullet$}

\pagestyle{fancy}
\renewcommand{\headrulewidth}{0.2pt}
\fancyhead[L]{maths-cours.fr}
\fancyhead[R]{\thepage}
\renewcommand{\footrulewidth}{0.2pt}
\fancyfoot[C]{}

\newcolumntype{C}{>{\centering\arraybackslash}X}
\newcolumntype{s}{>{\hsize=.35\hsize\arraybackslash}X}

\setlength{\parindent}{0pt}		 
\setlength{\parskip}{3mm}
\setlength{\headheight}{1cm}

\def\ebook{ebook}
\def\book{book}
\def\web{web}
\def\type{web}

\newcommand{\vect}[1]{\overrightarrow{\,\mathstrut#1\,}}

\def\Oij{$\left(\text{O}~;~\vect{\imath},~\vect{\jmath}\right)$}
\def\Oijk{$\left(\text{O}~;~\vect{\imath},~\vect{\jmath},~\vect{k}\right)$}
\def\Ouv{$\left(\text{O}~;~\vect{u},~\vect{v}\right)$}

\hypersetup{breaklinks=true, colorlinks = true, linkcolor = OliveGreen, urlcolor = OliveGreen, citecolor = OliveGreen, pdfauthor={Didier BONNEL - https://www.maths-cours.fr} } % supprime les bordures autour des liens

\renewcommand{\arg}[0]{\text{arg}}

\everymath{\displaystyle}

%================================================================================================================================
%
% Macros - Commandes
%
%================================================================================================================================

\newcommand\meta[2]{    			% Utilisé pour créer le post HTML.
	\def\titre{titre}
	\def\url{url}
	\def\arg{#1}
	\ifx\titre\arg
		\newcommand\maintitle{#2}
		\fancyhead[L]{#2}
		{\Large\sffamily \MakeUppercase{#2}}
		\vspace{1mm}\textcolor{mcvert}{\hrule}
	\fi 
	\ifx\url\arg
		\fancyfoot[L]{\href{https://www.maths-cours.fr#2}{\black \footnotesize{https://www.maths-cours.fr#2}}}
	\fi 
}


\newcommand\TitreC[1]{    		% Titre centré
     \needspace{3\baselineskip}
     \begin{center}\textbf{#1}\end{center}
}

\newcommand\newpar{    		% paragraphe
     \par
}

\newcommand\nosp {    		% commande vide (pas d'espace)
}
\newcommand{\id}[1]{} %ignore

\newcommand\boite[2]{				% Boite simple sans titre
	\vspace{5mm}
	\setlength{\fboxrule}{0.2mm}
	\setlength{\fboxsep}{5mm}	
	\fcolorbox{#1}{#1!3}{\makebox[\linewidth-2\fboxrule-2\fboxsep]{
  		\begin{minipage}[t]{\linewidth-2\fboxrule-4\fboxsep}\setlength{\parskip}{3mm}
  			 #2
  		\end{minipage}
	}}
	\vspace{5mm}
}

\newcommand\CBox[4]{				% Boites
	\vspace{5mm}
	\setlength{\fboxrule}{0.2mm}
	\setlength{\fboxsep}{5mm}
	
	\fcolorbox{#1}{#1!3}{\makebox[\linewidth-2\fboxrule-2\fboxsep]{
		\begin{minipage}[t]{1cm}\setlength{\parskip}{3mm}
	  		\textcolor{#1}{\LARGE{#2}}    
 	 	\end{minipage}  
  		\begin{minipage}[t]{\linewidth-2\fboxrule-4\fboxsep}\setlength{\parskip}{3mm}
			\raisebox{1.2mm}{\normalsize\sffamily{\textcolor{#1}{#3}}}						
  			 #4
  		\end{minipage}
	}}
	\vspace{5mm}
}

\newcommand\cadre[3]{				% Boites convertible html
	\par
	\vspace{2mm}
	\setlength{\fboxrule}{0.1mm}
	\setlength{\fboxsep}{5mm}
	\fcolorbox{#1}{white}{\makebox[\linewidth-2\fboxrule-2\fboxsep]{
  		\begin{minipage}[t]{\linewidth-2\fboxrule-4\fboxsep}\setlength{\parskip}{3mm}
			\raisebox{-2.5mm}{\sffamily \small{\textcolor{#1}{\MakeUppercase{#2}}}}		
			\par		
  			 #3
 	 		\end{minipage}
	}}
		\vspace{2mm}
	\par
}

\newcommand\bloc[3]{				% Boites convertible html sans bordure
     \needspace{2\baselineskip}
     {\sffamily \small{\textcolor{#1}{\MakeUppercase{#2}}}}    
		\par		
  			 #3
		\par
}

\newcommand\CHelp[1]{
     \CBox{Plum}{\faInfoCircle}{À RETENIR}{#1}
}

\newcommand\CUp[1]{
     \CBox{NavyBlue}{\faThumbsOUp}{EN PRATIQUE}{#1}
}

\newcommand\CInfo[1]{
     \CBox{Sepia}{\faArrowCircleRight}{REMARQUE}{#1}
}

\newcommand\CRedac[1]{
     \CBox{PineGreen}{\faEdit}{BIEN R\'EDIGER}{#1}
}

\newcommand\CError[1]{
     \CBox{Red}{\faExclamationTriangle}{ATTENTION}{#1}
}

\newcommand\TitreExo[2]{
\needspace{4\baselineskip}
 {\sffamily\large EXERCICE #1\ (\emph{#2 points})}
\vspace{5mm}
}

\newcommand\img[2]{
          \includegraphics[width=#2\paperwidth]{\imgdir#1}
}

\newcommand\imgsvg[2]{
       \begin{center}   \includegraphics[width=#2\paperwidth]{\imgsvgdir#1} \end{center}
}


\newcommand\Lien[2]{
     \href{#1}{#2 \tiny \faExternalLink}
}
\newcommand\mcLien[2]{
     \href{https~://www.maths-cours.fr/#1}{#2 \tiny \faExternalLink}
}

\newcommand{\euro}{\eurologo{}}

%================================================================================================================================
%
% Macros - Environement
%
%================================================================================================================================

\newenvironment{tex}{ %
}
{%
}

\newenvironment{indente}{ %
	\setlength\parindent{10mm}
}

{
	\setlength\parindent{0mm}
}

\newenvironment{corrige}{%
     \needspace{3\baselineskip}
     \medskip
     \textbf{\textsc{Corrigé}}
     \medskip
}
{
}

\newenvironment{extern}{%
     \begin{center}
     }
     {
     \end{center}
}

\NewEnviron{code}{%
	\par
     \boite{gray}{\texttt{%
     \BODY
     }}
     \par
}

\newenvironment{vbloc}{% boite sans cadre empeche saut de page
     \begin{minipage}[t]{\linewidth}
     }
     {
     \end{minipage}
}
\NewEnviron{h2}{%
    \needspace{3\baselineskip}
    \vspace{0.6cm}
	\noindent \MakeUppercase{\sffamily \large \BODY}
	\vspace{1mm}\textcolor{mcgris}{\hrule}\vspace{0.4cm}
	\par
}{}

\NewEnviron{h3}{%
    \needspace{3\baselineskip}
	\vspace{5mm}
	\textsc{\BODY}
	\par
}

\NewEnviron{margeneg}{ %
\begin{addmargin}[-1cm]{0cm}
\BODY
\end{addmargin}
}

\NewEnviron{html}{%
}

\begin{document}
\meta{url}{/exercices/coordonnees-representation-parametrique/}
\meta{pid}{1496}
\meta{titre}{Coordonnées et représentations paramétriques}
\meta{type}{exercices}
%
$ABCDEFGH$ est un cube. $I$ et $J$ sont les deux points définis par :
\begin{itemize}
     \item
     $\overrightarrow{HI}=\frac{1}{3}\overrightarrow{HE}$
     \item
     $\overrightarrow{HJ}=\frac{1}{3}\overrightarrow{HG}$
\end{itemize}
<img src="/wp-content/uploads/coordonnees-representation-parametrique.svg" alt="" class="aligncenter" style="width:350px;"/>
\imgsvg{}{0.3}% alt="@" style="width:30rem" class="aligncenter"
\begin{enumerate}
     \item
     On se place dans le repère $\left(D; \overrightarrow{DA}; \overrightarrow{DC}; \overrightarrow{DH}\right)$.
     \par
     Préciser les coordonnées des points $A, B, C, D, E, F, G, H, I, J$ dans ce repère.
     \textit{Aucune justification n'est demandée pour les coordonnées des sommets du cube.}
     \item
     Donner une représentation paramétrique de la droite $\left(AI\right)$ et de la droite $\left(CJ\right)$
     \item
     Montrer que les droites $\left(AI\right)$ et $\left(CJ\right)$ sont sécantes et déterminer les coordonnées de leur point d'intersection.
\end{enumerate}
\begin{corrige}
     \begin{enumerate}
          \item
          $A\left(1 ; 0 ; 0\right)$
          \par
          $B\left(1 ; 1 ; 0\right)$
          \par
          $C\left(0 ; 1 ; 0\right)$
          \par
          $D\left(0 ; 0 ; 0\right)$
          \par
          $E\left(1 ; 0 ; 1\right)$
          \par
          $F\left(1 ; 1 ; 1\right)$
          \par
          $G\left(0 ; 1 ; 1\right)$
          \par
          $H\left(0 ; 0 ; 1\right)$
          \par
          $\overrightarrow{DI}=\overrightarrow{DH}+\overrightarrow{HI} = \overrightarrow{DH}+\frac{1}{3}\overrightarrow{HE}$
          \par
          Les coordonnées de $\overrightarrow{DH}$ et $\overrightarrow{HE}$ sont $\overrightarrow{DH} \left(0 ; 0 ; 1\right)$ et $\overrightarrow{HE} \left(1 ; 0 ; 0\right)$
          \par
          Les coordonnées de $\overrightarrow{DI}$ et du point $I$ sont donc $I \left(\frac{1}{3} ; 0 ; 1\right)$
          \par
          Avec un calcul similaire on trouve $J \left(0 ; \frac{1}{3} ; 1\right)$
          \item
          La droite $\left(AI\right)$ passe par le point $A\left(1 ; 0 ; 0\right)$ et est dirigée par le vecteur $\overrightarrow{AI}\left(-\frac{2}{3} ; 0 ; 1\right)$. Pour éviter l'emploi de fraction, on peut dire que le vecteur $3\overrightarrow{AI}\left(-2 ; 0 ; 3\right)$ est également un vecteur directeur de $\left(AI\right)$.
          \par
          Une représentation paramétrique de la droite $\left(AI\right)$ est donc :
          \begin{center}$\left\{ \begin{matrix} x=1-2t \\ y=0 \\ z=3t \end{matrix}\right.      $  avec $t \in  \mathbb{R}$\end{center}
               \textbf{Remarque : }Cette représentation n'est pas unique. Le système :
               \begin{center}$\left\{ \begin{matrix} x=1-\frac{2}{3}t \\ y=0 \\ z=t \end{matrix}\right.      $  avec $t \in  \mathbb{R}$\end{center}
                    par exemple, est lui aussi correct.
                    \par
                    Avec un raisonnement identique on montre que le système :
                    \begin{center}$\left\{ \begin{matrix} x=0 \\ y=1-2t \\ z=3t \end{matrix}\right.      $  avec $t \in  \mathbb{R}$\end{center}
                         est une représentation paramétrique de la droite $\left(CJ\right)$
                         \item
                         Les droites $\left(AI\right)$ et $\left(CJ\right)$ sont sécantes en un point $M\left(x ; y ; z\right)$ s'il existe deux réels $t$ et $t^{\prime}$ tels que simultanément :
                         \begin{center}$\left\{ \begin{matrix} x=1-2t \\ y=0 \\ z=3t \end{matrix}\right.  \qquad    $ et $    \qquad      \left\{ \begin{matrix} x=0 \\ y=1-2t^{\prime} \\ z=3t^{\prime} \end{matrix}\right.      $  \end{center}
                                   \textbf{Remarque :} Attention à remplacer $t$ par $t^{\prime}$ dans le second système car les paramètres ne sont pas nécessairement égaux dans les deux représentations paramétriques.
                                   \par
                                   Cela entraine :
                                   \par
                                   $\left\{ \begin{matrix} 1-2t=0 \\ 0=1-2t^{\prime} \\ 3t=3t^{\prime} \end{matrix}\right.$
                                        \par
                                        Ce système admet une solution : $t=\frac{1}{2}$, $t^{\prime}=\frac{1}{2}$ et on obtient alors :
                                        \par
                                        $\left\{ \begin{matrix} x=0 \\ y=0 \\ z=\frac{3}{2} \end{matrix}\right.$
                                             \par
                                             Les droites $\left(AI\right)$ et $\left(CJ\right)$ sont donc sécantes en un point $M\left(0 ; 0 ; \frac{3}{2}\right)$
                                        \end{enumerate}
                                   \end{corrige}

\end{document}