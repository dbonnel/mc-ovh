\documentclass[a4paper]{article}

%================================================================================================================================
%
% Packages
%
%================================================================================================================================

\usepackage[T1]{fontenc} 	% pour caractères accentués
\usepackage[utf8]{inputenc}  % encodage utf8
\usepackage[french]{babel}	% langue : français
\usepackage{fourier}			% caractères plus lisibles
\usepackage[dvipsnames]{xcolor} % couleurs
\usepackage{fancyhdr}		% réglage header footer
\usepackage{needspace}		% empêcher sauts de page mal placés
\usepackage{graphicx}		% pour inclure des graphiques
\usepackage{enumitem,cprotect}		% personnalise les listes d'items (nécessaire pour ol, al ...)
\usepackage{hyperref}		% Liens hypertexte
\usepackage{pstricks,pst-all,pst-node,pstricks-add,pst-math,pst-plot,pst-tree,pst-eucl} % pstricks
\usepackage[a4paper,includeheadfoot,top=2cm,left=3cm, bottom=2cm,right=3cm]{geometry} % marges etc.
\usepackage{comment}			% commentaires multilignes
\usepackage{amsmath,environ} % maths (matrices, etc.)
\usepackage{amssymb,makeidx}
\usepackage{bm}				% bold maths
\usepackage{tabularx}		% tableaux
\usepackage{colortbl}		% tableaux en couleur
\usepackage{fontawesome}		% Fontawesome
\usepackage{environ}			% environment with command
\usepackage{fp}				% calculs pour ps-tricks
\usepackage{multido}			% pour ps tricks
\usepackage[np]{numprint}	% formattage nombre
\usepackage{tikz,tkz-tab} 			% package principal TikZ
\usepackage{pgfplots}   % axes
\usepackage{mathrsfs}    % cursives
\usepackage{calc}			% calcul taille boites
\usepackage[scaled=0.875]{helvet} % font sans serif
\usepackage{svg} % svg
\usepackage{scrextend} % local margin
\usepackage{scratch} %scratch
\usepackage{multicol} % colonnes
%\usepackage{infix-RPN,pst-func} % formule en notation polanaise inversée
\usepackage{listings}

%================================================================================================================================
%
% Réglages de base
%
%================================================================================================================================

\lstset{
language=Python,   % R code
literate=
{á}{{\'a}}1
{à}{{\`a}}1
{ã}{{\~a}}1
{é}{{\'e}}1
{è}{{\`e}}1
{ê}{{\^e}}1
{í}{{\'i}}1
{ó}{{\'o}}1
{õ}{{\~o}}1
{ú}{{\'u}}1
{ü}{{\"u}}1
{ç}{{\c{c}}}1
{~}{{ }}1
}


\definecolor{codegreen}{rgb}{0,0.6,0}
\definecolor{codegray}{rgb}{0.5,0.5,0.5}
\definecolor{codepurple}{rgb}{0.58,0,0.82}
\definecolor{backcolour}{rgb}{0.95,0.95,0.92}

\lstdefinestyle{mystyle}{
    backgroundcolor=\color{backcolour},   
    commentstyle=\color{codegreen},
    keywordstyle=\color{magenta},
    numberstyle=\tiny\color{codegray},
    stringstyle=\color{codepurple},
    basicstyle=\ttfamily\footnotesize,
    breakatwhitespace=false,         
    breaklines=true,                 
    captionpos=b,                    
    keepspaces=true,                 
    numbers=left,                    
xleftmargin=2em,
framexleftmargin=2em,            
    showspaces=false,                
    showstringspaces=false,
    showtabs=false,                  
    tabsize=2,
    upquote=true
}

\lstset{style=mystyle}


\lstset{style=mystyle}
\newcommand{\imgdir}{C:/laragon/www/newmc/assets/imgsvg/}
\newcommand{\imgsvgdir}{C:/laragon/www/newmc/assets/imgsvg/}

\definecolor{mcgris}{RGB}{220, 220, 220}% ancien~; pour compatibilité
\definecolor{mcbleu}{RGB}{52, 152, 219}
\definecolor{mcvert}{RGB}{125, 194, 70}
\definecolor{mcmauve}{RGB}{154, 0, 215}
\definecolor{mcorange}{RGB}{255, 96, 0}
\definecolor{mcturquoise}{RGB}{0, 153, 153}
\definecolor{mcrouge}{RGB}{255, 0, 0}
\definecolor{mclightvert}{RGB}{205, 234, 190}

\definecolor{gris}{RGB}{220, 220, 220}
\definecolor{bleu}{RGB}{52, 152, 219}
\definecolor{vert}{RGB}{125, 194, 70}
\definecolor{mauve}{RGB}{154, 0, 215}
\definecolor{orange}{RGB}{255, 96, 0}
\definecolor{turquoise}{RGB}{0, 153, 153}
\definecolor{rouge}{RGB}{255, 0, 0}
\definecolor{lightvert}{RGB}{205, 234, 190}
\setitemize[0]{label=\color{lightvert}  $\bullet$}

\pagestyle{fancy}
\renewcommand{\headrulewidth}{0.2pt}
\fancyhead[L]{maths-cours.fr}
\fancyhead[R]{\thepage}
\renewcommand{\footrulewidth}{0.2pt}
\fancyfoot[C]{}

\newcolumntype{C}{>{\centering\arraybackslash}X}
\newcolumntype{s}{>{\hsize=.35\hsize\arraybackslash}X}

\setlength{\parindent}{0pt}		 
\setlength{\parskip}{3mm}
\setlength{\headheight}{1cm}

\def\ebook{ebook}
\def\book{book}
\def\web{web}
\def\type{web}

\newcommand{\vect}[1]{\overrightarrow{\,\mathstrut#1\,}}

\def\Oij{$\left(\text{O}~;~\vect{\imath},~\vect{\jmath}\right)$}
\def\Oijk{$\left(\text{O}~;~\vect{\imath},~\vect{\jmath},~\vect{k}\right)$}
\def\Ouv{$\left(\text{O}~;~\vect{u},~\vect{v}\right)$}

\hypersetup{breaklinks=true, colorlinks = true, linkcolor = OliveGreen, urlcolor = OliveGreen, citecolor = OliveGreen, pdfauthor={Didier BONNEL - https://www.maths-cours.fr} } % supprime les bordures autour des liens

\renewcommand{\arg}[0]{\text{arg}}

\everymath{\displaystyle}

%================================================================================================================================
%
% Macros - Commandes
%
%================================================================================================================================

\newcommand\meta[2]{    			% Utilisé pour créer le post HTML.
	\def\titre{titre}
	\def\url{url}
	\def\arg{#1}
	\ifx\titre\arg
		\newcommand\maintitle{#2}
		\fancyhead[L]{#2}
		{\Large\sffamily \MakeUppercase{#2}}
		\vspace{1mm}\textcolor{mcvert}{\hrule}
	\fi 
	\ifx\url\arg
		\fancyfoot[L]{\href{https://www.maths-cours.fr#2}{\black \footnotesize{https://www.maths-cours.fr#2}}}
	\fi 
}


\newcommand\TitreC[1]{    		% Titre centré
     \needspace{3\baselineskip}
     \begin{center}\textbf{#1}\end{center}
}

\newcommand\newpar{    		% paragraphe
     \par
}

\newcommand\nosp {    		% commande vide (pas d'espace)
}
\newcommand{\id}[1]{} %ignore

\newcommand\boite[2]{				% Boite simple sans titre
	\vspace{5mm}
	\setlength{\fboxrule}{0.2mm}
	\setlength{\fboxsep}{5mm}	
	\fcolorbox{#1}{#1!3}{\makebox[\linewidth-2\fboxrule-2\fboxsep]{
  		\begin{minipage}[t]{\linewidth-2\fboxrule-4\fboxsep}\setlength{\parskip}{3mm}
  			 #2
  		\end{minipage}
	}}
	\vspace{5mm}
}

\newcommand\CBox[4]{				% Boites
	\vspace{5mm}
	\setlength{\fboxrule}{0.2mm}
	\setlength{\fboxsep}{5mm}
	
	\fcolorbox{#1}{#1!3}{\makebox[\linewidth-2\fboxrule-2\fboxsep]{
		\begin{minipage}[t]{1cm}\setlength{\parskip}{3mm}
	  		\textcolor{#1}{\LARGE{#2}}    
 	 	\end{minipage}  
  		\begin{minipage}[t]{\linewidth-2\fboxrule-4\fboxsep}\setlength{\parskip}{3mm}
			\raisebox{1.2mm}{\normalsize\sffamily{\textcolor{#1}{#3}}}						
  			 #4
  		\end{minipage}
	}}
	\vspace{5mm}
}

\newcommand\cadre[3]{				% Boites convertible html
	\par
	\vspace{2mm}
	\setlength{\fboxrule}{0.1mm}
	\setlength{\fboxsep}{5mm}
	\fcolorbox{#1}{white}{\makebox[\linewidth-2\fboxrule-2\fboxsep]{
  		\begin{minipage}[t]{\linewidth-2\fboxrule-4\fboxsep}\setlength{\parskip}{3mm}
			\raisebox{-2.5mm}{\sffamily \small{\textcolor{#1}{\MakeUppercase{#2}}}}		
			\par		
  			 #3
 	 		\end{minipage}
	}}
		\vspace{2mm}
	\par
}

\newcommand\bloc[3]{				% Boites convertible html sans bordure
     \needspace{2\baselineskip}
     {\sffamily \small{\textcolor{#1}{\MakeUppercase{#2}}}}    
		\par		
  			 #3
		\par
}

\newcommand\CHelp[1]{
     \CBox{Plum}{\faInfoCircle}{À RETENIR}{#1}
}

\newcommand\CUp[1]{
     \CBox{NavyBlue}{\faThumbsOUp}{EN PRATIQUE}{#1}
}

\newcommand\CInfo[1]{
     \CBox{Sepia}{\faArrowCircleRight}{REMARQUE}{#1}
}

\newcommand\CRedac[1]{
     \CBox{PineGreen}{\faEdit}{BIEN R\'EDIGER}{#1}
}

\newcommand\CError[1]{
     \CBox{Red}{\faExclamationTriangle}{ATTENTION}{#1}
}

\newcommand\TitreExo[2]{
\needspace{4\baselineskip}
 {\sffamily\large EXERCICE #1\ (\emph{#2 points})}
\vspace{5mm}
}

\newcommand\img[2]{
          \includegraphics[width=#2\paperwidth]{\imgdir#1}
}

\newcommand\imgsvg[2]{
       \begin{center}   \includegraphics[width=#2\paperwidth]{\imgsvgdir#1} \end{center}
}


\newcommand\Lien[2]{
     \href{#1}{#2 \tiny \faExternalLink}
}
\newcommand\mcLien[2]{
     \href{https~://www.maths-cours.fr/#1}{#2 \tiny \faExternalLink}
}

\newcommand{\euro}{\eurologo{}}

%================================================================================================================================
%
% Macros - Environement
%
%================================================================================================================================

\newenvironment{tex}{ %
}
{%
}

\newenvironment{indente}{ %
	\setlength\parindent{10mm}
}

{
	\setlength\parindent{0mm}
}

\newenvironment{corrige}{%
     \needspace{3\baselineskip}
     \medskip
     \textbf{\textsc{Corrigé}}
     \medskip
}
{
}

\newenvironment{extern}{%
     \begin{center}
     }
     {
     \end{center}
}

\NewEnviron{code}{%
	\par
     \boite{gray}{\texttt{%
     \BODY
     }}
     \par
}

\newenvironment{vbloc}{% boite sans cadre empeche saut de page
     \begin{minipage}[t]{\linewidth}
     }
     {
     \end{minipage}
}
\NewEnviron{h2}{%
    \needspace{3\baselineskip}
    \vspace{0.6cm}
	\noindent \MakeUppercase{\sffamily \large \BODY}
	\vspace{1mm}\textcolor{mcgris}{\hrule}\vspace{0.4cm}
	\par
}{}

\NewEnviron{h3}{%
    \needspace{3\baselineskip}
	\vspace{5mm}
	\textsc{\BODY}
	\par
}

\NewEnviron{margeneg}{ %
\begin{addmargin}[-1cm]{0cm}
\BODY
\end{addmargin}
}

\NewEnviron{html}{%
}

\begin{document}
\meta{url}{/exercices/geometrie-espace-bac-s-metropole-2014/}
\meta{pid}{1506}
\meta{titre}{Géométrie dans l'espace - Bac S  Métropole 2014}
\meta{type}{exercices}
%
\begin{h2}Exercice 4 (5 points)\end{h2}
\textit{Candidats n'ayant pas suivi l'enseignement de spécialité}
\par
Dans l'espace, on considère un tétraèdre $ABCD$ dont les faces $ABC, ACD$ et $ABD$ sont des triangles rectangles et isocèles en $A$. On désigne par $E, F$ et $G$ les milieux respectifs des côtés $\left[AB\right], \left[BC\right]$ et $\left[CA\right]$.
\par
On choisit $AB$ pour unité de longueur et on se place dans le repère orthonormé $\left(A ; \overrightarrow{AB}, \overrightarrow{AC}, \overrightarrow{AD}\right)$ de l'espace.
\begin{enumerate}
     \item
     On désigne par $\mathscr P$ le plan qui passe par $A$ et qui est orthogonal à la droite $\left(DF\right)$.
     \par
     On note $H$ le point d'intersection du plan $\mathscr P$ et de la droite $\left(DF\right)$.
     \begin{enumerate}[label=\alph*.]
          \item
          Donner les coordonnées des points $D$ et $F$.
          \item
          Donner une représentation paramétrique de la droite $\left(DF\right)$.
          \item
          Déterminer une équation cartésienne du plan $\mathscr P$.
          \item
          Calculer les coordonnées du point $H$.
          \item
     Démontrer que l'angle $\widehat{EHG}$ est un angle droit.\end{enumerate}
     \item
     On désigne par $M$ un point de la droite $\left(DF\right)$ et par $t$ le réel tel que $\overrightarrow{DM}=t \overrightarrow{DF}$. On note $\alpha $ la mesure en radians de l'angle géométrique $\widehat{EMG}$.
     \par
     Le but de cette question est de déterminer la position du point $M$ pour que $\alpha $ soit maximale.
     \begin{enumerate}[label=\alph*.]
          \item
          Démontrer que $ME^{2}=\frac{3}{2}t^{2}-\frac{5}{2}t+\frac{5}{4}$.
          \item
          Démontrer que le triangle $MEG$ est isocèle en $M$.
          \par
          En déduire que $ME \sin \left(\frac{\alpha }{2}\right)=\frac{1}{2\sqrt{2}}$.
          \item
          Justifier que $\alpha $ est maximale si et seulement si $\sin\left(\frac{\alpha }{2}\right)$ est maximal.
          \par
          En déduire que $\alpha $ est maximale si et seulement si $ME^{2}$ est minimal.
          \item
     Conclure.\end{enumerate}
\end{enumerate}
\begin{corrige}
     \begin{enumerate}
          \item
          \begin{enumerate}[label=\alph*.]
               \item
               Vu le choix du repère, $D$ a pour coordonnées $\left(0;0;1\right)$
               \par
               $F$ est le milieu de $\left[BC\right]$ avec $B\left(1;0;0\right)$ et $C\left(0;1;0\right)$ donc $F$ a pour coordonnées
               \par
               $\left(\frac{1}{2};\frac{1}{2};0\right)$
               \item
               La droite $\left(DF\right)$ a pour vecteur directeur $\overrightarrow{DF}\left(\frac{1}{2};\frac{1}{2};-1\right)$ et passe par le point  $D\left(0;0;1\right)$
               \par
               Une représentation paramétrique de $\left(DF\right)$ est donc :
               \par
               $\left\{ \begin{matrix} x=1/2 t  \\ y=1/2 t  \\ z=1-t \end{matrix}\right.$    $       t\in \mathbb{R}$
\par
                    \textbf{Remarque : }
                    Cette représentation n'est pas unique; ce n'est donc pas la seule réponse possible ! On peut, en particulier, obtenir une représentation plus simple en choisissant comme vecteur directeur $2\overrightarrow{DF}\left(1;1;-2\right)$ ce qui donne :
                    \par
                    $\left\{ \begin{matrix} x=t  \\ y=t  \\ z=1-2t \end{matrix}\right.$    $       t\in \mathbb{R}$
                         \item
                         Le vecteur $2\overrightarrow{DF}\left(1;1;-2\right)$ est normal au plan $\mathscr P$. L'équation de $\mathscr P$ est donc du type :
                         \par
                         $x+y-2z+d=0$
                         \par
                         Ce plan passe par $A$ donc les coordonnées de $A\left(0;0;0\right)$ vérifient l'équation du plan :
                         \par
                         $0+0-2\times 0+d=0$ soit $d=0$.
                         \par
                         Une équation cartésienne de $\mathscr P$ est donc :
                         \par
                         $x+y-2z=0$
\par
                         \textbf{Remarque : }
                         Là encore, ce n'est pas la seule réponse possible !
                         \item
                         $H$ appartient a l'intersection du plan $\mathscr P$ et de la droite $\left(DF\right)$. Ses coordonnées sont donc de la forme :
                         \par
                         $\left\{ \begin{matrix} x_{H}=t  \\ y_{H}=t  \\ z_{H}=1-2t \end{matrix}\right.$
                              \par
                              avec $t$ tel que
                              \par
                              $t+t-2\left(1-2t\right)=0$ (équation obtenue en remplaçant $x$, $y$, et $z$ par les coordonnées ci-dessus dans l'équation cartésienne de $\mathscr P$).
                              \par
                              Cette équation donne $6t=2$ donc $t=\frac{1}{3}$ et $H \left(\frac{1}{3};\frac{1}{3};\frac{1}{3}\right)$.
                              \item
                              $E\left(\frac{1}{2};0;0\right)$ et $G\left(0;\frac{1}{2};0\right)$ donc :
                              \par
                              $\overrightarrow{EH}\left(-\frac{1}{6};\frac{1}{3};\frac{1}{3}\right)$ et $\overrightarrow{GH}\left(\frac{1}{3};-\frac{1}{6};\frac{1}{3}\right)$
                              \par
                              $\overrightarrow{EH}.\overrightarrow{GH} =\frac{1}{3}\times \left(-\frac{1}{6}\right)-\frac{1}{6}\times \frac{1}{3}+\frac{1}{3}\times \frac{1}{3}=0$
                              \par
                              Les vecteurs $\overrightarrow{EH}$ et $\overrightarrow{GH}$ sont donc orthogonaux et l'angle $\widehat{EHG}$ est un angle droit.
                         \end{enumerate}
                         \item
                         \begin{enumerate}[label=\alph*.]
                              \item
                              Soit $M\left(x;y;z\right)$.
                              \par
                              $\overrightarrow{DM}=t\overrightarrow{DF}$ avec $\overrightarrow{DF}\left(\frac{1}{2};\frac{1}{2};-1\right)$ et $\overrightarrow{DM}\left(x;y;z-1\right)$
                              \par
                              Donc $x=\frac{1}{2}t$, $y=\frac{1}{2}t$ et $z=1-t$.
                              \par
                              $ME^{2}=\left(x-\frac{1}{2}\right)^{2}+y^{2}+z^{2}=\left(\frac{1}{2}t-\frac{1}{2}\right)^{2}+\left(\frac{1}{2}t\right)^{2}+\left(1-t\right)^{2}$
                              \par
                              $ME^{2}=\frac{1}{4}t^{2}-\frac{1}{2}t+\frac{1}{4}+\frac{1}{4}t^{2}+1-2t+t^{2}=\frac{3}{2}t^{2}-\frac{5}{2}t+\frac{5}{4}$
                              \par
                              Un calcul similaire pour $MG^{2}$ (il suffit de permuter $x$ et $y$ dans les calculs!) conduit également à :
                              \par
                              $MG^{2}=\frac{3}{2}t^{2}-\frac{5}{2}t+\frac{5}{4}$
                              \par
                              Donc $MG=ME$ et le triangle $MEG$ est isocèle en $M$.
                              \item
                              Soit $I$ le milieu de $\left[EG\right]$. $EI$ est une médiane donc une hauteur du triangle isocèle $MEG$

\begin{center}
\imgsvg{mc-0302}{0.3}% alt="Géométrie - Bac S  Métropole 2014" style="width:20rem"
\end{center}

                              $EG=2 EI=2 ME \sin\left(\frac{ \alpha}{2}\right)$
                              \par
                              Or, à partir des coordonnées de $E$ et de $G$ : $EG^{2}=\frac{1}{4}+\frac{1}{4}+0=\frac{1}{2}$
                              \par
                              Donc
                              \par
                              $2ME \sin \left(\frac{\alpha }{2}\right)=\frac{1}{\sqrt{2}}$
                              \par
                              c'est à dire :
                              \par
                              $ME \sin \left(\frac{\alpha }{2}\right)=\frac{1}{2\sqrt{2}}$
                              \item
                              La fonction $\alpha  \mapsto  \sin\left(\frac{\alpha}{2}\right)$ est strictement croissante sur l'intervalle $\left[0; \pi \right]$
                              \par
                              $\sin\left(\frac{\alpha}{2}\right)$ est donc maximal lorsque la mesure $\alpha $ est maximale;
                              \par
                              Le produit $ME \sin \left(\frac{\alpha }{2}\right)$ étant constant  $\sin \left(\frac{\alpha }{2}\right)$ est maximal lorsque $ME$ est minimal, c'est à dire lorsque $ME^{2}$ est minimal (la fonction \textit{carrée} étant strictement croissante sur $\left[0;+\infty \right[$)
                              \item
                              La fonction $t \mapsto  \frac{3}{2}t^{2}-\frac{5}{2}t+\frac{5}{4}$ est une fonction polynôme qu second degré qui atteint son minimum pour $t=-\frac{b}{2a}=\frac{5}{6}$.
                              \par
                              Le point $M$ pour lequel la mesure $\alpha $ est maximale est donc situé au $\frac{5}{6}$ du segment $\left[DF\right]$ en partant de $D$.
                              \par
                              Les coordonnées de $M$ sont alors $\left(\frac{5}{12};\frac{5}{12};\frac{1}{6}\right)$ (cf. \textbf{2.a.})
                         \end{enumerate}
                    \end{enumerate}
          \end{corrige}

\end{document}