\documentclass[a4paper]{article}

%================================================================================================================================
%
% Packages
%
%================================================================================================================================

\usepackage[T1]{fontenc} 	% pour caractères accentués
\usepackage[utf8]{inputenc}  % encodage utf8
\usepackage[french]{babel}	% langue : français
\usepackage{fourier}			% caractères plus lisibles
\usepackage[dvipsnames]{xcolor} % couleurs
\usepackage{fancyhdr}		% réglage header footer
\usepackage{needspace}		% empêcher sauts de page mal placés
\usepackage{graphicx}		% pour inclure des graphiques
\usepackage{enumitem,cprotect}		% personnalise les listes d'items (nécessaire pour ol, al ...)
\usepackage{hyperref}		% Liens hypertexte
\usepackage{pstricks,pst-all,pst-node,pstricks-add,pst-math,pst-plot,pst-tree,pst-eucl} % pstricks
\usepackage[a4paper,includeheadfoot,top=2cm,left=3cm, bottom=2cm,right=3cm]{geometry} % marges etc.
\usepackage{comment}			% commentaires multilignes
\usepackage{amsmath,environ} % maths (matrices, etc.)
\usepackage{amssymb,makeidx}
\usepackage{bm}				% bold maths
\usepackage{tabularx}		% tableaux
\usepackage{colortbl}		% tableaux en couleur
\usepackage{fontawesome}		% Fontawesome
\usepackage{environ}			% environment with command
\usepackage{fp}				% calculs pour ps-tricks
\usepackage{multido}			% pour ps tricks
\usepackage[np]{numprint}	% formattage nombre
\usepackage{tikz,tkz-tab} 			% package principal TikZ
\usepackage{pgfplots}   % axes
\usepackage{mathrsfs}    % cursives
\usepackage{calc}			% calcul taille boites
\usepackage[scaled=0.875]{helvet} % font sans serif
\usepackage{svg} % svg
\usepackage{scrextend} % local margin
\usepackage{scratch} %scratch
\usepackage{multicol} % colonnes
%\usepackage{infix-RPN,pst-func} % formule en notation polanaise inversée
\usepackage{listings}

%================================================================================================================================
%
% Réglages de base
%
%================================================================================================================================

\lstset{
language=Python,   % R code
literate=
{á}{{\'a}}1
{à}{{\`a}}1
{ã}{{\~a}}1
{é}{{\'e}}1
{è}{{\`e}}1
{ê}{{\^e}}1
{í}{{\'i}}1
{ó}{{\'o}}1
{õ}{{\~o}}1
{ú}{{\'u}}1
{ü}{{\"u}}1
{ç}{{\c{c}}}1
{~}{{ }}1
}


\definecolor{codegreen}{rgb}{0,0.6,0}
\definecolor{codegray}{rgb}{0.5,0.5,0.5}
\definecolor{codepurple}{rgb}{0.58,0,0.82}
\definecolor{backcolour}{rgb}{0.95,0.95,0.92}

\lstdefinestyle{mystyle}{
    backgroundcolor=\color{backcolour},   
    commentstyle=\color{codegreen},
    keywordstyle=\color{magenta},
    numberstyle=\tiny\color{codegray},
    stringstyle=\color{codepurple},
    basicstyle=\ttfamily\footnotesize,
    breakatwhitespace=false,         
    breaklines=true,                 
    captionpos=b,                    
    keepspaces=true,                 
    numbers=left,                    
xleftmargin=2em,
framexleftmargin=2em,            
    showspaces=false,                
    showstringspaces=false,
    showtabs=false,                  
    tabsize=2,
    upquote=true
}

\lstset{style=mystyle}


\lstset{style=mystyle}
\newcommand{\imgdir}{C:/laragon/www/newmc/assets/imgsvg/}
\newcommand{\imgsvgdir}{C:/laragon/www/newmc/assets/imgsvg/}

\definecolor{mcgris}{RGB}{220, 220, 220}% ancien~; pour compatibilité
\definecolor{mcbleu}{RGB}{52, 152, 219}
\definecolor{mcvert}{RGB}{125, 194, 70}
\definecolor{mcmauve}{RGB}{154, 0, 215}
\definecolor{mcorange}{RGB}{255, 96, 0}
\definecolor{mcturquoise}{RGB}{0, 153, 153}
\definecolor{mcrouge}{RGB}{255, 0, 0}
\definecolor{mclightvert}{RGB}{205, 234, 190}

\definecolor{gris}{RGB}{220, 220, 220}
\definecolor{bleu}{RGB}{52, 152, 219}
\definecolor{vert}{RGB}{125, 194, 70}
\definecolor{mauve}{RGB}{154, 0, 215}
\definecolor{orange}{RGB}{255, 96, 0}
\definecolor{turquoise}{RGB}{0, 153, 153}
\definecolor{rouge}{RGB}{255, 0, 0}
\definecolor{lightvert}{RGB}{205, 234, 190}
\setitemize[0]{label=\color{lightvert}  $\bullet$}

\pagestyle{fancy}
\renewcommand{\headrulewidth}{0.2pt}
\fancyhead[L]{maths-cours.fr}
\fancyhead[R]{\thepage}
\renewcommand{\footrulewidth}{0.2pt}
\fancyfoot[C]{}

\newcolumntype{C}{>{\centering\arraybackslash}X}
\newcolumntype{s}{>{\hsize=.35\hsize\arraybackslash}X}

\setlength{\parindent}{0pt}		 
\setlength{\parskip}{3mm}
\setlength{\headheight}{1cm}

\def\ebook{ebook}
\def\book{book}
\def\web{web}
\def\type{web}

\newcommand{\vect}[1]{\overrightarrow{\,\mathstrut#1\,}}

\def\Oij{$\left(\text{O}~;~\vect{\imath},~\vect{\jmath}\right)$}
\def\Oijk{$\left(\text{O}~;~\vect{\imath},~\vect{\jmath},~\vect{k}\right)$}
\def\Ouv{$\left(\text{O}~;~\vect{u},~\vect{v}\right)$}

\hypersetup{breaklinks=true, colorlinks = true, linkcolor = OliveGreen, urlcolor = OliveGreen, citecolor = OliveGreen, pdfauthor={Didier BONNEL - https://www.maths-cours.fr} } % supprime les bordures autour des liens

\renewcommand{\arg}[0]{\text{arg}}

\everymath{\displaystyle}

%================================================================================================================================
%
% Macros - Commandes
%
%================================================================================================================================

\newcommand\meta[2]{    			% Utilisé pour créer le post HTML.
	\def\titre{titre}
	\def\url{url}
	\def\arg{#1}
	\ifx\titre\arg
		\newcommand\maintitle{#2}
		\fancyhead[L]{#2}
		{\Large\sffamily \MakeUppercase{#2}}
		\vspace{1mm}\textcolor{mcvert}{\hrule}
	\fi 
	\ifx\url\arg
		\fancyfoot[L]{\href{https://www.maths-cours.fr#2}{\black \footnotesize{https://www.maths-cours.fr#2}}}
	\fi 
}


\newcommand\TitreC[1]{    		% Titre centré
     \needspace{3\baselineskip}
     \begin{center}\textbf{#1}\end{center}
}

\newcommand\newpar{    		% paragraphe
     \par
}

\newcommand\nosp {    		% commande vide (pas d'espace)
}
\newcommand{\id}[1]{} %ignore

\newcommand\boite[2]{				% Boite simple sans titre
	\vspace{5mm}
	\setlength{\fboxrule}{0.2mm}
	\setlength{\fboxsep}{5mm}	
	\fcolorbox{#1}{#1!3}{\makebox[\linewidth-2\fboxrule-2\fboxsep]{
  		\begin{minipage}[t]{\linewidth-2\fboxrule-4\fboxsep}\setlength{\parskip}{3mm}
  			 #2
  		\end{minipage}
	}}
	\vspace{5mm}
}

\newcommand\CBox[4]{				% Boites
	\vspace{5mm}
	\setlength{\fboxrule}{0.2mm}
	\setlength{\fboxsep}{5mm}
	
	\fcolorbox{#1}{#1!3}{\makebox[\linewidth-2\fboxrule-2\fboxsep]{
		\begin{minipage}[t]{1cm}\setlength{\parskip}{3mm}
	  		\textcolor{#1}{\LARGE{#2}}    
 	 	\end{minipage}  
  		\begin{minipage}[t]{\linewidth-2\fboxrule-4\fboxsep}\setlength{\parskip}{3mm}
			\raisebox{1.2mm}{\normalsize\sffamily{\textcolor{#1}{#3}}}						
  			 #4
  		\end{minipage}
	}}
	\vspace{5mm}
}

\newcommand\cadre[3]{				% Boites convertible html
	\par
	\vspace{2mm}
	\setlength{\fboxrule}{0.1mm}
	\setlength{\fboxsep}{5mm}
	\fcolorbox{#1}{white}{\makebox[\linewidth-2\fboxrule-2\fboxsep]{
  		\begin{minipage}[t]{\linewidth-2\fboxrule-4\fboxsep}\setlength{\parskip}{3mm}
			\raisebox{-2.5mm}{\sffamily \small{\textcolor{#1}{\MakeUppercase{#2}}}}		
			\par		
  			 #3
 	 		\end{minipage}
	}}
		\vspace{2mm}
	\par
}

\newcommand\bloc[3]{				% Boites convertible html sans bordure
     \needspace{2\baselineskip}
     {\sffamily \small{\textcolor{#1}{\MakeUppercase{#2}}}}    
		\par		
  			 #3
		\par
}

\newcommand\CHelp[1]{
     \CBox{Plum}{\faInfoCircle}{À RETENIR}{#1}
}

\newcommand\CUp[1]{
     \CBox{NavyBlue}{\faThumbsOUp}{EN PRATIQUE}{#1}
}

\newcommand\CInfo[1]{
     \CBox{Sepia}{\faArrowCircleRight}{REMARQUE}{#1}
}

\newcommand\CRedac[1]{
     \CBox{PineGreen}{\faEdit}{BIEN R\'EDIGER}{#1}
}

\newcommand\CError[1]{
     \CBox{Red}{\faExclamationTriangle}{ATTENTION}{#1}
}

\newcommand\TitreExo[2]{
\needspace{4\baselineskip}
 {\sffamily\large EXERCICE #1\ (\emph{#2 points})}
\vspace{5mm}
}

\newcommand\img[2]{
          \includegraphics[width=#2\paperwidth]{\imgdir#1}
}

\newcommand\imgsvg[2]{
       \begin{center}   \includegraphics[width=#2\paperwidth]{\imgsvgdir#1} \end{center}
}


\newcommand\Lien[2]{
     \href{#1}{#2 \tiny \faExternalLink}
}
\newcommand\mcLien[2]{
     \href{https~://www.maths-cours.fr/#1}{#2 \tiny \faExternalLink}
}

\newcommand{\euro}{\eurologo{}}

%================================================================================================================================
%
% Macros - Environement
%
%================================================================================================================================

\newenvironment{tex}{ %
}
{%
}

\newenvironment{indente}{ %
	\setlength\parindent{10mm}
}

{
	\setlength\parindent{0mm}
}

\newenvironment{corrige}{%
     \needspace{3\baselineskip}
     \medskip
     \textbf{\textsc{Corrigé}}
     \medskip
}
{
}

\newenvironment{extern}{%
     \begin{center}
     }
     {
     \end{center}
}

\NewEnviron{code}{%
	\par
     \boite{gray}{\texttt{%
     \BODY
     }}
     \par
}

\newenvironment{vbloc}{% boite sans cadre empeche saut de page
     \begin{minipage}[t]{\linewidth}
     }
     {
     \end{minipage}
}
\NewEnviron{h2}{%
    \needspace{3\baselineskip}
    \vspace{0.6cm}
	\noindent \MakeUppercase{\sffamily \large \BODY}
	\vspace{1mm}\textcolor{mcgris}{\hrule}\vspace{0.4cm}
	\par
}{}

\NewEnviron{h3}{%
    \needspace{3\baselineskip}
	\vspace{5mm}
	\textsc{\BODY}
	\par
}

\NewEnviron{margeneg}{ %
\begin{addmargin}[-1cm]{0cm}
\BODY
\end{addmargin}
}

\NewEnviron{html}{%
}

\begin{document}
\meta{url}{/exercices/suites-bac-blanc-es-l-sujet-2-maths-cours-2018/}
\meta{pid}{10455}
\meta{titre}{Suites - Bac blanc ES/L Sujet 2 - Maths-cours 2018}
\meta{type}{exercices}
%
\begin{h2}Exercice 4 (5 points)\end{h2}
\par
On considère la suite $(u_n)$ définie par $u_0=250$ et, pour tout entier naturel $n$ :
\[u_{n+1}=0,8u_n+60.\]
\begin{enumerate}
     \item Calculer $u_1$ et $u_2$.
     \item
     Compléter l'algorithme ci-après afin qu'il affiche le plus petit entier naturel $n$ tel que $u_n \geqslant 290$.
     \begin{center}
          \begin{extern}%width="340" alt="Algorithme de calcul de la somme S10"
               \begin{tabular}{|l l|}\hline
                    Variables :	& $N$ est un entier naturel\\
                    & $U$ est un nombre réel\\
                    & \\
                    Initialisation: &$U$ prend la valeur $250$ \\
                    &$N$ prend la valeur $0$ \\
                    & \\
                    Traitement: &Tant que ... faire\\
                    &\qquad$U$ prend la valeur ...\\
                    &\qquad$N$ prend la valeur ...\\
                    &Fin Tant que\\
                    & \\
                    Sortie :	&Afficher ... \\
                    \hline
               \end{tabular}
          \end{extern}
     \end{center}
     \item %3
     Soit la suite $(v_n)$ définie, pour tout entier naturel $n$, par :
     \[ v_n=u_n-300.\]
     \begin{enumerate}[label=\alph*.]
          \item %3a
          Montrer que la suite $(v_n)$ est une suite géométrique dont on précisera le premier terme et la raison.
          \item %3b
          Exprimer $v_n$ en fonction de $n$.
          \item %3c
          En déduire que pour tout entier naturel $n$ :
          \[ u_n=300-50 \times 0,8^n. \]
     \end{enumerate}
     \item %4
     \`A l'aide de la calculatrice, déterminer la valeur affichée par l'algorithme de la question \textbf{2.}
     \item %5
     Une ville organise chaque année un tournoi d'\'Echecs.
     En 2016, $200$ joueurs ont participé à ce tournoi.
     Les organisateurs font l'hypothèse que, d'une année sur l'autre :
     \begin{itemize}
          \item 20\% des joueurs ne reviennent pas l'année suivante,
          \item 60 nouveaux joueurs s'inscrivent au tournoi.
     \end{itemize}
     La taille de la salle dans laquelle se déroule le tournoi limite le nombre de joueurs à 320.
     Les organisateurs vont-ils devoir refuser des inscriptions par manque de places dans les années à venir ?
     Justifier la réponse.
\end{enumerate}
\begin{corrige}
     \begin{enumerate}
          \item %1
          Pour tout entier naturel $n$, ${u_{n+1}=0,8u_n+60}$ ; par conséquent :
          \par
          $u_{1}=0,8u_0+60=0,8 \times 250+60=260$.
          \par
          $u_{2}=0,8u_1+60=0,8 \times 260+60=268$.
          \item %2
          L'algorithme peut être complété de la façon suivante :
          \begin{center}
               \begin{extern}%width="380" alt="Algorithme de calcul de la somme S10"
                    \begin{tabular}{|l l|}\hline
                         Variables :	& $N$ est un entier naturel\\
                         & $U$ est un nombre réel\\
                         & \\
                         Initialisation: &$U$ prend la valeur $250$ \\
                         &$N$ prend la valeur $0$ \\
                         & \\
                         Traitement: &Tant que \textcolor{red}{\ $U<290$\ } faire\\
                         &\quad$U$ prend la valeur \textcolor{red}{\ $0,8U+60$}\\
                         &\quad$N$ prend la valeur \textcolor{red}{\ $N+1$\ }\\
                         &Fin Tant que\\
                         & \\
                         Sortie :	&Afficher \textcolor{red}{\ $N$\ } \\
                         \hline
                    \end{tabular}
               \end{extern}
          \end{center}
          (\textbf{Attention au sens de la condition} \og Tant que ${U<290}$ \fg{}. On veut que la boucle \og Tant que \fg{} \textbf{se termine} lorsque  $\bm{U \geqslant 290}$ ; on souhaite donc qu'elle \textbf{continue} à s'effectuer dans le cas contraire, c'est à dire tant que $\bm{U<290}$.)
          \item %3
          \begin{enumerate}[label=\alph*.]
               \item %3a
               Pour tout entier naturel $n$, $v_{n}= u_{n}-300$ ; par conséquent :
               \par
               $v_{n+1}= u_{n+1}-300$.
               \par
               Comme $u_{n+1}=0,8u_n + 60$ :
               \par
               $v_{n+1} = 0,8u_n+60-300$\\
               $\phantom{v_{n+1}} = 0,8u_n-240.$
               \par
               Puisque $v_{n}= u_{n}-300$, alors $u_{n}= v_{n}+300$. On en déduit :
               \par
               $v_{n+1} = 0,8(v_n+300)-240$\\
               $\phantom{v_{n+1}} = 0,8v_n+240-240$\\
               $\phantom{v_{n+1}} = 0,8v_n.$
               \par
               Par ailleurs :
               \par
               $v_{0}= u_{0}-300=250-300=-50$.
               \par
               La suite $(v_n)$ est une suite géométrique de premier terme ${v_0=-50}$ et de raison $0,8$.
               \item %3b
               La suite $(v_n)$ étant une suite géométrique :
               \par
               $v_n=v_0q^n=-50 \times 0,8^n$.
               \item %3c
               D'après les questions précédentes :
               \par
               $u_{n}= v_{n}+300 = 300 -50 \times 0,8^n$.
          \end{enumerate}
          \item %4
          \`A la calculatrice, on affiche un tableau de valeurs de la fonction $x \longmapsto 300 -50 \times 0,8^x$.
          \par
          On trouve alors :
          \par
          $u_7 \approx 289,51 \quad $ et $\quad u_8 \approx 291,61 $
          \par
          L'algorithme affiche le plus petit entier naturel $n$ tel que $u_n \geqslant 290$. L'algorithme affichera donc la valeur 8.
          \item %5
          Notons $a_n$ le nombre de joueurs inscrits au tournoi l'année $2016+n$.
          \par
          En 2016, 250 joueurs ont participé au tournoi donc $a_0=250$.
          \par
          Une diminution de 20\% correspond à un coefficient multiplicateur de ${1-\dfrac{20}{100}=0,8}$ ; on ajoute ensuite les 60 nouveaux inscrits.
          \par
          On a donc :
          \par
          \[ a_{n+1}=0,8a_n+60. \]
          \par
          Les suites $(u_n)$ et $(a_n)$ sont définies par la même relation de récurrence et le même premier terme ; elles sont donc identiques.
          \par
          Par conséquent, d'après la question \textbf{3.c.} :
          \par
          $a_{n}= 300 -50 \times 0,8^n$.
          \par
          Comme $50 \times 0,8^n$ est strictement positif pour tout entier $n$, le nombre $300 -50 \times 0,8^n$ est strictement inférieur à 300.
          \par
          Quelle que soit l'année, \textbf{le nombre d'inscrits sera inférieur à 300}. Les organisateurs \textbf{n'auront donc pas à refuser des inscriptions} par manque de places dans les années à venir.
     \end{enumerate}
\end{corrige}

\end{document}