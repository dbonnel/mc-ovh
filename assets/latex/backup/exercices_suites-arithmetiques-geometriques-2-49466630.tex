\documentclass[a4paper]{article}

%================================================================================================================================
%
% Packages
%
%================================================================================================================================

\usepackage[T1]{fontenc} 	% pour caractères accentués
\usepackage[utf8]{inputenc}  % encodage utf8
\usepackage[french]{babel}	% langue : français
\usepackage{fourier}			% caractères plus lisibles
\usepackage[dvipsnames]{xcolor} % couleurs
\usepackage{fancyhdr}		% réglage header footer
\usepackage{needspace}		% empêcher sauts de page mal placés
\usepackage{graphicx}		% pour inclure des graphiques
\usepackage{enumitem,cprotect}		% personnalise les listes d'items (nécessaire pour ol, al ...)
\usepackage{hyperref}		% Liens hypertexte
\usepackage{pstricks,pst-all,pst-node,pstricks-add,pst-math,pst-plot,pst-tree,pst-eucl} % pstricks
\usepackage[a4paper,includeheadfoot,top=2cm,left=3cm, bottom=2cm,right=3cm]{geometry} % marges etc.
\usepackage{comment}			% commentaires multilignes
\usepackage{amsmath,environ} % maths (matrices, etc.)
\usepackage{amssymb,makeidx}
\usepackage{bm}				% bold maths
\usepackage{tabularx}		% tableaux
\usepackage{colortbl}		% tableaux en couleur
\usepackage{fontawesome}		% Fontawesome
\usepackage{environ}			% environment with command
\usepackage{fp}				% calculs pour ps-tricks
\usepackage{multido}			% pour ps tricks
\usepackage[np]{numprint}	% formattage nombre
\usepackage{tikz,tkz-tab} 			% package principal TikZ
\usepackage{pgfplots}   % axes
\usepackage{mathrsfs}    % cursives
\usepackage{calc}			% calcul taille boites
\usepackage[scaled=0.875]{helvet} % font sans serif
\usepackage{svg} % svg
\usepackage{scrextend} % local margin
\usepackage{scratch} %scratch
\usepackage{multicol} % colonnes
%\usepackage{infix-RPN,pst-func} % formule en notation polanaise inversée
\usepackage{listings}

%================================================================================================================================
%
% Réglages de base
%
%================================================================================================================================

\lstset{
language=Python,   % R code
literate=
{á}{{\'a}}1
{à}{{\`a}}1
{ã}{{\~a}}1
{é}{{\'e}}1
{è}{{\`e}}1
{ê}{{\^e}}1
{í}{{\'i}}1
{ó}{{\'o}}1
{õ}{{\~o}}1
{ú}{{\'u}}1
{ü}{{\"u}}1
{ç}{{\c{c}}}1
{~}{{ }}1
}


\definecolor{codegreen}{rgb}{0,0.6,0}
\definecolor{codegray}{rgb}{0.5,0.5,0.5}
\definecolor{codepurple}{rgb}{0.58,0,0.82}
\definecolor{backcolour}{rgb}{0.95,0.95,0.92}

\lstdefinestyle{mystyle}{
    backgroundcolor=\color{backcolour},   
    commentstyle=\color{codegreen},
    keywordstyle=\color{magenta},
    numberstyle=\tiny\color{codegray},
    stringstyle=\color{codepurple},
    basicstyle=\ttfamily\footnotesize,
    breakatwhitespace=false,         
    breaklines=true,                 
    captionpos=b,                    
    keepspaces=true,                 
    numbers=left,                    
xleftmargin=2em,
framexleftmargin=2em,            
    showspaces=false,                
    showstringspaces=false,
    showtabs=false,                  
    tabsize=2,
    upquote=true
}

\lstset{style=mystyle}


\lstset{style=mystyle}
\newcommand{\imgdir}{C:/laragon/www/newmc/assets/imgsvg/}
\newcommand{\imgsvgdir}{C:/laragon/www/newmc/assets/imgsvg/}

\definecolor{mcgris}{RGB}{220, 220, 220}% ancien~; pour compatibilité
\definecolor{mcbleu}{RGB}{52, 152, 219}
\definecolor{mcvert}{RGB}{125, 194, 70}
\definecolor{mcmauve}{RGB}{154, 0, 215}
\definecolor{mcorange}{RGB}{255, 96, 0}
\definecolor{mcturquoise}{RGB}{0, 153, 153}
\definecolor{mcrouge}{RGB}{255, 0, 0}
\definecolor{mclightvert}{RGB}{205, 234, 190}

\definecolor{gris}{RGB}{220, 220, 220}
\definecolor{bleu}{RGB}{52, 152, 219}
\definecolor{vert}{RGB}{125, 194, 70}
\definecolor{mauve}{RGB}{154, 0, 215}
\definecolor{orange}{RGB}{255, 96, 0}
\definecolor{turquoise}{RGB}{0, 153, 153}
\definecolor{rouge}{RGB}{255, 0, 0}
\definecolor{lightvert}{RGB}{205, 234, 190}
\setitemize[0]{label=\color{lightvert}  $\bullet$}

\pagestyle{fancy}
\renewcommand{\headrulewidth}{0.2pt}
\fancyhead[L]{maths-cours.fr}
\fancyhead[R]{\thepage}
\renewcommand{\footrulewidth}{0.2pt}
\fancyfoot[C]{}

\newcolumntype{C}{>{\centering\arraybackslash}X}
\newcolumntype{s}{>{\hsize=.35\hsize\arraybackslash}X}

\setlength{\parindent}{0pt}		 
\setlength{\parskip}{3mm}
\setlength{\headheight}{1cm}

\def\ebook{ebook}
\def\book{book}
\def\web{web}
\def\type{web}

\newcommand{\vect}[1]{\overrightarrow{\,\mathstrut#1\,}}

\def\Oij{$\left(\text{O}~;~\vect{\imath},~\vect{\jmath}\right)$}
\def\Oijk{$\left(\text{O}~;~\vect{\imath},~\vect{\jmath},~\vect{k}\right)$}
\def\Ouv{$\left(\text{O}~;~\vect{u},~\vect{v}\right)$}

\hypersetup{breaklinks=true, colorlinks = true, linkcolor = OliveGreen, urlcolor = OliveGreen, citecolor = OliveGreen, pdfauthor={Didier BONNEL - https://www.maths-cours.fr} } % supprime les bordures autour des liens

\renewcommand{\arg}[0]{\text{arg}}

\everymath{\displaystyle}

%================================================================================================================================
%
% Macros - Commandes
%
%================================================================================================================================

\newcommand\meta[2]{    			% Utilisé pour créer le post HTML.
	\def\titre{titre}
	\def\url{url}
	\def\arg{#1}
	\ifx\titre\arg
		\newcommand\maintitle{#2}
		\fancyhead[L]{#2}
		{\Large\sffamily \MakeUppercase{#2}}
		\vspace{1mm}\textcolor{mcvert}{\hrule}
	\fi 
	\ifx\url\arg
		\fancyfoot[L]{\href{https://www.maths-cours.fr#2}{\black \footnotesize{https://www.maths-cours.fr#2}}}
	\fi 
}


\newcommand\TitreC[1]{    		% Titre centré
     \needspace{3\baselineskip}
     \begin{center}\textbf{#1}\end{center}
}

\newcommand\newpar{    		% paragraphe
     \par
}

\newcommand\nosp {    		% commande vide (pas d'espace)
}
\newcommand{\id}[1]{} %ignore

\newcommand\boite[2]{				% Boite simple sans titre
	\vspace{5mm}
	\setlength{\fboxrule}{0.2mm}
	\setlength{\fboxsep}{5mm}	
	\fcolorbox{#1}{#1!3}{\makebox[\linewidth-2\fboxrule-2\fboxsep]{
  		\begin{minipage}[t]{\linewidth-2\fboxrule-4\fboxsep}\setlength{\parskip}{3mm}
  			 #2
  		\end{minipage}
	}}
	\vspace{5mm}
}

\newcommand\CBox[4]{				% Boites
	\vspace{5mm}
	\setlength{\fboxrule}{0.2mm}
	\setlength{\fboxsep}{5mm}
	
	\fcolorbox{#1}{#1!3}{\makebox[\linewidth-2\fboxrule-2\fboxsep]{
		\begin{minipage}[t]{1cm}\setlength{\parskip}{3mm}
	  		\textcolor{#1}{\LARGE{#2}}    
 	 	\end{minipage}  
  		\begin{minipage}[t]{\linewidth-2\fboxrule-4\fboxsep}\setlength{\parskip}{3mm}
			\raisebox{1.2mm}{\normalsize\sffamily{\textcolor{#1}{#3}}}						
  			 #4
  		\end{minipage}
	}}
	\vspace{5mm}
}

\newcommand\cadre[3]{				% Boites convertible html
	\par
	\vspace{2mm}
	\setlength{\fboxrule}{0.1mm}
	\setlength{\fboxsep}{5mm}
	\fcolorbox{#1}{white}{\makebox[\linewidth-2\fboxrule-2\fboxsep]{
  		\begin{minipage}[t]{\linewidth-2\fboxrule-4\fboxsep}\setlength{\parskip}{3mm}
			\raisebox{-2.5mm}{\sffamily \small{\textcolor{#1}{\MakeUppercase{#2}}}}		
			\par		
  			 #3
 	 		\end{minipage}
	}}
		\vspace{2mm}
	\par
}

\newcommand\bloc[3]{				% Boites convertible html sans bordure
     \needspace{2\baselineskip}
     {\sffamily \small{\textcolor{#1}{\MakeUppercase{#2}}}}    
		\par		
  			 #3
		\par
}

\newcommand\CHelp[1]{
     \CBox{Plum}{\faInfoCircle}{À RETENIR}{#1}
}

\newcommand\CUp[1]{
     \CBox{NavyBlue}{\faThumbsOUp}{EN PRATIQUE}{#1}
}

\newcommand\CInfo[1]{
     \CBox{Sepia}{\faArrowCircleRight}{REMARQUE}{#1}
}

\newcommand\CRedac[1]{
     \CBox{PineGreen}{\faEdit}{BIEN R\'EDIGER}{#1}
}

\newcommand\CError[1]{
     \CBox{Red}{\faExclamationTriangle}{ATTENTION}{#1}
}

\newcommand\TitreExo[2]{
\needspace{4\baselineskip}
 {\sffamily\large EXERCICE #1\ (\emph{#2 points})}
\vspace{5mm}
}

\newcommand\img[2]{
          \includegraphics[width=#2\paperwidth]{\imgdir#1}
}

\newcommand\imgsvg[2]{
       \begin{center}   \includegraphics[width=#2\paperwidth]{\imgsvgdir#1} \end{center}
}


\newcommand\Lien[2]{
     \href{#1}{#2 \tiny \faExternalLink}
}
\newcommand\mcLien[2]{
     \href{https~://www.maths-cours.fr/#1}{#2 \tiny \faExternalLink}
}

\newcommand{\euro}{\eurologo{}}

%================================================================================================================================
%
% Macros - Environement
%
%================================================================================================================================

\newenvironment{tex}{ %
}
{%
}

\newenvironment{indente}{ %
	\setlength\parindent{10mm}
}

{
	\setlength\parindent{0mm}
}

\newenvironment{corrige}{%
     \needspace{3\baselineskip}
     \medskip
     \textbf{\textsc{Corrigé}}
     \medskip
}
{
}

\newenvironment{extern}{%
     \begin{center}
     }
     {
     \end{center}
}

\NewEnviron{code}{%
	\par
     \boite{gray}{\texttt{%
     \BODY
     }}
     \par
}

\newenvironment{vbloc}{% boite sans cadre empeche saut de page
     \begin{minipage}[t]{\linewidth}
     }
     {
     \end{minipage}
}
\NewEnviron{h2}{%
    \needspace{3\baselineskip}
    \vspace{0.6cm}
	\noindent \MakeUppercase{\sffamily \large \BODY}
	\vspace{1mm}\textcolor{mcgris}{\hrule}\vspace{0.4cm}
	\par
}{}

\NewEnviron{h3}{%
    \needspace{3\baselineskip}
	\vspace{5mm}
	\textsc{\BODY}
	\par
}

\NewEnviron{margeneg}{ %
\begin{addmargin}[-1cm]{0cm}
\BODY
\end{addmargin}
}

\NewEnviron{html}{%
}

\begin{document}
\meta{url}{/exercices/suites-arithmetiques-geometriques-2/}
\meta{pid}{4294}
\meta{titre}{Suites arithmétiques et géométriques}
\meta{type}{exercices}
%
Pour son appartement, Alexandre paye, tous les mois, un loyer brut et des charges locatives. On appelle loyer net, la somme du loyer brut et des charges locatives.
\par
En 2016, le loyer brut était de 450 euros (mensuel) et les charges de 60 euros (mensuel).
\par
Au premier janvier de chaque année, le loyer brut mensuel augmente de 1,5 \% et les charges locatives mensuelles augmentent de 1€.
\par
On note :
\begin{itemize}
     \item
     $b_n$ : le total des loyers bruts (en euros) pour l'année 2016 + $n$
     \item
     $c_n$ : le total des charges (en euros) pour l'année 2016 + $n$
     \item
     $l_n$ : le total des loyers nets (en euros) pour l'année 2016 + $n$.
\end{itemize}
\begin{enumerate}
     \item
     Calculer $b_0$ et $c_0$.
     \par
     En déduire que $l_0=6120$.
     \item
     Calculer $b_1, c_1$ et $l_1$ puis $b_2, c_2$ et $l_2$.
     \item
     Exprimer $b_{n+1}$ en fonction de $b_n$, puis $c_{n+1}$ en fonction de $c_n$.
     \item
     Pour chacune des suites $(b_n), (c_n)$ et $(l_n)$ indiquer s'il s'agit d'une suite arithmétique, d'une suite géométrique ou d'une suite qui n'est ni arithmétique ni géométrique.
     \item
     Exprimer $b_n, c_n$ puis $l_n$ en fonction de $n$.
     \item
     Quel sera le total des loyers nets payés par Alexandre au cours des dix premières années (de 2016 à 2025) ?
\end{enumerate}
\begin{corrige}
     \begin{enumerate}
          \item
          En 2016, Alexandre paiera 450 euros de loyer brut tous les mois donc le total en euros sera :
          \par
          $b_0=12 \times 450=5400$
          \par
          De même, le total en euros des charges locatives pour 2016 sera :
          \par
          $c_0=12 \times 60=720$
          \par
          Le total des loyers nets s'obtiendra en faisant la somme des loyers bruts et des charges locatives :
          \par
          $l_0=b_0+c_0=5400+720=6120$
          \item
          Augmenter un montant de $1,5$\% revient à multiplier ce montant par $1,015$.
          \par
          Le montant des loyers bruts mensuels en 2017 sera donc de $450 \times 1,015 = 456,75$ euros et le total annuel des loyers bruts :
          \par
          $b_1=450 \times 1,015 \times 12 = 5481$
          \par
          On remarque que pour obtenir $b_1$ il suffit de multiplier $b_0$ par $1,015$.
          \par
          En 2017, Alexandre paiera $1$ euro de charges supplémentaires tous les mois. Sur l'année, il paiera donc $12$ euros de charges de plus qu'en 2016.
          \par
          Le total des charges locatives en euros pour l'année 2017 sera donc :
          \par
          $c_1=c_0+12=720+12=732$
          \par
          Le total des loyers nets pour 2012 sera :
          \par
          $l_1=b_1+c_1=5481+732=6213$
          \par
          Un raisonnement analogue permet de calculer les montants des loyers et des charges en 2018 :
          \par
          $b_2=b_1 \times 1,015=5563,215$ (ou $5563,22$ arrondi au centime)
          \par
          $c_2=c_1+12=732+12=744$
          \par
          $l_2=b_2+c_2=6307,215$ (ou $6307,22$ arrondi au centime)
          \item
          Les loyers bruts de l'année de rang $n+1$ s'obtiennent en multipliant les loyers bruts de l'année de rang $n$ par $1,015$. On a donc :
          \par
          $b_{n+1}=1,015 \times b_n$
          \par
          Les charges de l'année de rang $n+1$ s'obtiennent en ajoutant $12$ aux charges de l'année de rang $n$. Par conséquent :
          \par
          $c_{n+1}=c_n+12$
          \item
          D'après les questions précédentes:
          \par
          $(b_n)$ est une suite géométrique de premier terme $b_0=5400$ et de raison $1,015$.
          \par
          $(c_n)$ est une suite arithmétique de premier terme $c_0=720$ et de raison $12$.
          \par
          Montrons que la suite $(l_n)$ n'est ni arithmétique ni géométrique :
          \par
          $l_1-l_0=6213-6120=93$
          \par
          $l_2-l_1=6307,215-6213=94,215$
          \par
          La différence entre deux termes consécutifs n'est pas constante donc la suite $(l_n)$ n'est pas arithmétique.
          \par
          $ \frac{l_1}{l_0} = \frac{6213}{6120}  \approx 1,01520$ (à $10^{^-5}$ près)
          \par
          $\frac{l_2}{l_1} = \frac{6307,215}{6213}  \approx 1,01516$ (à $10^{^-5}$ près)
          \par
          Le quotient de deux termes consécutifs n'est pas constant donc la suite $(l_n)$ n'est pas géométrique.
          \item
          La suite $(b_n)$ est une suite géométrique de premier terme $b_0=5400$ et de raison $q=1,015$, par conséquent :
          \par
          $b_n=b_0 \times q^n=5400 \times 1,015^n$
          \par
          La suite $(c_n)$ est une suite arithmétique de premier terme $c_0=720$ et de raison $r=12$, donc :
          \par
          $c_n=c_0 + n r=720 + 12n$
          \par
          $l_n$ est la somme de $b_n$ et $c_n$ :
          \par
          $l_n=5400 \times 1,015^n+720+12n
          $
          \item
          Le total des loyers bruts lors des 10 premières années est :
          \par
          $B=b_0+b_1+ \cdots +b_9$
          \par
          $\phantom{B}=5400+5400 \times 1,015 + \cdots +5400 \times 1,015^9$
          \par
          $\phantom{B}=5400(1+1,015 + \cdots +1,015^9)$
          \par
          donc d'après la formule $1+q+q^2+\cdots+q^n= \frac{1-q^{n+1}}{1-q} $ :
          \par
          $B=5400 \times  \frac{1-1.015^{10}}{1-1.015} $
          \par
          $\phantom{B} \approx 57794,70$ (au centime près)
          \par
          Le total des charges locatives lors des 10 premières années est :
          \par
          $C=c_0+c_1+ \cdots +c_9$
          \par
          $C=720+ 720+12 \times 1+ 720+12 \times 2  +$$ \cdots  +720+12 \times 9$
          \par
          On regroupe les termes égaux à $720$; il y en a 10, donc :
          \par
          $C=720\times 10+12 \times 1+12 \times 2  + \cdots +12 \times 9$
          \par
          $\phantom{C}=7200+12 (1+2+\cdots +9)$
          \par
          On applique la formule $1+2+\cdots +n= \frac{n(n+1)}{2} $ :
          \par
          $C=7200+12\times  \frac{9\times 10}{2} = 7740$
          \par
          Le total des loyers nets que paiera Alexandre au cours des 10 premières années est donc :
          \par
          $L=B+C=57794,70+7740=65534,70$ euros
     \end{enumerate}
\end{corrige}

\end{document}