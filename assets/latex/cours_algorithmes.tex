\documentclass[a4paper]{article}

%================================================================================================================================
%
% Packages
%
%================================================================================================================================

\usepackage[T1]{fontenc} 	% pour caractères accentués
\usepackage[utf8]{inputenc}  % encodage utf8
\usepackage[french]{babel}	% langue : français
\usepackage{fourier}			% caractères plus lisibles
\usepackage[dvipsnames]{xcolor} % couleurs
\usepackage{fancyhdr}		% réglage header footer
\usepackage{needspace}		% empêcher sauts de page mal placés
\usepackage{graphicx}		% pour inclure des graphiques
\usepackage{enumitem,cprotect}		% personnalise les listes d'items (nécessaire pour ol, al ...)
\usepackage{hyperref}		% Liens hypertexte
\usepackage{pstricks,pst-all,pst-node,pstricks-add,pst-math,pst-plot,pst-tree,pst-eucl} % pstricks
\usepackage[a4paper,includeheadfoot,top=2cm,left=3cm, bottom=2cm,right=3cm]{geometry} % marges etc.
\usepackage{comment}			% commentaires multilignes
\usepackage{amsmath,environ} % maths (matrices, etc.)
\usepackage{amssymb,makeidx}
\usepackage{bm}				% bold maths
\usepackage{tabularx}		% tableaux
\usepackage{colortbl}		% tableaux en couleur
\usepackage{fontawesome}		% Fontawesome
\usepackage{environ}			% environment with command
\usepackage{fp}				% calculs pour ps-tricks
\usepackage{multido}			% pour ps tricks
\usepackage[np]{numprint}	% formattage nombre
\usepackage{tikz,tkz-tab} 			% package principal TikZ
\usepackage{pgfplots}   % axes
\usepackage{mathrsfs}    % cursives
\usepackage{calc}			% calcul taille boites
\usepackage[scaled=0.875]{helvet} % font sans serif
\usepackage{svg} % svg
\usepackage{scrextend} % local margin
\usepackage{scratch} %scratch
\usepackage{multicol} % colonnes
%\usepackage{infix-RPN,pst-func} % formule en notation polanaise inversée
\usepackage{listings}

%================================================================================================================================
%
% Réglages de base
%
%================================================================================================================================

\lstset{
language=Python,   % R code
literate=
{á}{{\'a}}1
{à}{{\`a}}1
{ã}{{\~a}}1
{é}{{\'e}}1
{è}{{\`e}}1
{ê}{{\^e}}1
{í}{{\'i}}1
{ó}{{\'o}}1
{õ}{{\~o}}1
{ú}{{\'u}}1
{ü}{{\"u}}1
{ç}{{\c{c}}}1
{~}{{ }}1
}


\definecolor{codegreen}{rgb}{0,0.6,0}
\definecolor{codegray}{rgb}{0.5,0.5,0.5}
\definecolor{codepurple}{rgb}{0.58,0,0.82}
\definecolor{backcolour}{rgb}{0.95,0.95,0.92}

\lstdefinestyle{mystyle}{
    backgroundcolor=\color{backcolour},   
    commentstyle=\color{codegreen},
    keywordstyle=\color{magenta},
    numberstyle=\tiny\color{codegray},
    stringstyle=\color{codepurple},
    basicstyle=\ttfamily\footnotesize,
    breakatwhitespace=false,         
    breaklines=true,                 
    captionpos=b,                    
    keepspaces=true,                 
    numbers=left,                    
xleftmargin=2em,
framexleftmargin=2em,            
    showspaces=false,                
    showstringspaces=false,
    showtabs=false,                  
    tabsize=2,
    upquote=true
}

\lstset{style=mystyle}


\lstset{style=mystyle}
\newcommand{\imgdir}{C:/laragon/www/newmc/assets/imgsvg/}
\newcommand{\imgsvgdir}{C:/laragon/www/newmc/assets/imgsvg/}

\definecolor{mcgris}{RGB}{220, 220, 220}% ancien~; pour compatibilité
\definecolor{mcbleu}{RGB}{52, 152, 219}
\definecolor{mcvert}{RGB}{125, 194, 70}
\definecolor{mcmauve}{RGB}{154, 0, 215}
\definecolor{mcorange}{RGB}{255, 96, 0}
\definecolor{mcturquoise}{RGB}{0, 153, 153}
\definecolor{mcrouge}{RGB}{255, 0, 0}
\definecolor{mclightvert}{RGB}{205, 234, 190}

\definecolor{gris}{RGB}{220, 220, 220}
\definecolor{bleu}{RGB}{52, 152, 219}
\definecolor{vert}{RGB}{125, 194, 70}
\definecolor{mauve}{RGB}{154, 0, 215}
\definecolor{orange}{RGB}{255, 96, 0}
\definecolor{turquoise}{RGB}{0, 153, 153}
\definecolor{rouge}{RGB}{255, 0, 0}
\definecolor{lightvert}{RGB}{205, 234, 190}
\setitemize[0]{label=\color{lightvert}  $\bullet$}

\pagestyle{fancy}
\renewcommand{\headrulewidth}{0.2pt}
\fancyhead[L]{maths-cours.fr}
\fancyhead[R]{\thepage}
\renewcommand{\footrulewidth}{0.2pt}
\fancyfoot[C]{}

\newcolumntype{C}{>{\centering\arraybackslash}X}
\newcolumntype{s}{>{\hsize=.35\hsize\arraybackslash}X}

\setlength{\parindent}{0pt}		 
\setlength{\parskip}{3mm}
\setlength{\headheight}{1cm}

\def\ebook{ebook}
\def\book{book}
\def\web{web}
\def\type{web}

\newcommand{\vect}[1]{\overrightarrow{\,\mathstrut#1\,}}

\def\Oij{$\left(\text{O}~;~\vect{\imath},~\vect{\jmath}\right)$}
\def\Oijk{$\left(\text{O}~;~\vect{\imath},~\vect{\jmath},~\vect{k}\right)$}
\def\Ouv{$\left(\text{O}~;~\vect{u},~\vect{v}\right)$}

\hypersetup{breaklinks=true, colorlinks = true, linkcolor = OliveGreen, urlcolor = OliveGreen, citecolor = OliveGreen, pdfauthor={Didier BONNEL - https://www.maths-cours.fr} } % supprime les bordures autour des liens

\renewcommand{\arg}[0]{\text{arg}}

\everymath{\displaystyle}

%================================================================================================================================
%
% Macros - Commandes
%
%================================================================================================================================

\newcommand\meta[2]{    			% Utilisé pour créer le post HTML.
	\def\titre{titre}
	\def\url{url}
	\def\arg{#1}
	\ifx\titre\arg
		\newcommand\maintitle{#2}
		\fancyhead[L]{#2}
		{\Large\sffamily \MakeUppercase{#2}}
		\vspace{1mm}\textcolor{mcvert}{\hrule}
	\fi 
	\ifx\url\arg
		\fancyfoot[L]{\href{https://www.maths-cours.fr#2}{\black \footnotesize{https://www.maths-cours.fr#2}}}
	\fi 
}


\newcommand\TitreC[1]{    		% Titre centré
     \needspace{3\baselineskip}
     \begin{center}\textbf{#1}\end{center}
}

\newcommand\newpar{    		% paragraphe
     \par
}

\newcommand\nosp {    		% commande vide (pas d'espace)
}
\newcommand{\id}[1]{} %ignore

\newcommand\boite[2]{				% Boite simple sans titre
	\vspace{5mm}
	\setlength{\fboxrule}{0.2mm}
	\setlength{\fboxsep}{5mm}	
	\fcolorbox{#1}{#1!3}{\makebox[\linewidth-2\fboxrule-2\fboxsep]{
  		\begin{minipage}[t]{\linewidth-2\fboxrule-4\fboxsep}\setlength{\parskip}{3mm}
  			 #2
  		\end{minipage}
	}}
	\vspace{5mm}
}

\newcommand\CBox[4]{				% Boites
	\vspace{5mm}
	\setlength{\fboxrule}{0.2mm}
	\setlength{\fboxsep}{5mm}
	
	\fcolorbox{#1}{#1!3}{\makebox[\linewidth-2\fboxrule-2\fboxsep]{
		\begin{minipage}[t]{1cm}\setlength{\parskip}{3mm}
	  		\textcolor{#1}{\LARGE{#2}}    
 	 	\end{minipage}  
  		\begin{minipage}[t]{\linewidth-2\fboxrule-4\fboxsep}\setlength{\parskip}{3mm}
			\raisebox{1.2mm}{\normalsize\sffamily{\textcolor{#1}{#3}}}						
  			 #4
  		\end{minipage}
	}}
	\vspace{5mm}
}

\newcommand\cadre[3]{				% Boites convertible html
	\par
	\vspace{2mm}
	\setlength{\fboxrule}{0.1mm}
	\setlength{\fboxsep}{5mm}
	\fcolorbox{#1}{white}{\makebox[\linewidth-2\fboxrule-2\fboxsep]{
  		\begin{minipage}[t]{\linewidth-2\fboxrule-4\fboxsep}\setlength{\parskip}{3mm}
			\raisebox{-2.5mm}{\sffamily \small{\textcolor{#1}{\MakeUppercase{#2}}}}		
			\par		
  			 #3
 	 		\end{minipage}
	}}
		\vspace{2mm}
	\par
}

\newcommand\bloc[3]{				% Boites convertible html sans bordure
     \needspace{2\baselineskip}
     {\sffamily \small{\textcolor{#1}{\MakeUppercase{#2}}}}    
		\par		
  			 #3
		\par
}

\newcommand\CHelp[1]{
     \CBox{Plum}{\faInfoCircle}{À RETENIR}{#1}
}

\newcommand\CUp[1]{
     \CBox{NavyBlue}{\faThumbsOUp}{EN PRATIQUE}{#1}
}

\newcommand\CInfo[1]{
     \CBox{Sepia}{\faArrowCircleRight}{REMARQUE}{#1}
}

\newcommand\CRedac[1]{
     \CBox{PineGreen}{\faEdit}{BIEN R\'EDIGER}{#1}
}

\newcommand\CError[1]{
     \CBox{Red}{\faExclamationTriangle}{ATTENTION}{#1}
}

\newcommand\TitreExo[2]{
\needspace{4\baselineskip}
 {\sffamily\large EXERCICE #1\ (\emph{#2 points})}
\vspace{5mm}
}

\newcommand\img[2]{
          \includegraphics[width=#2\paperwidth]{\imgdir#1}
}

\newcommand\imgsvg[2]{
       \begin{center}   \includegraphics[width=#2\paperwidth]{\imgsvgdir#1} \end{center}
}


\newcommand\Lien[2]{
     \href{#1}{#2 \tiny \faExternalLink}
}
\newcommand\mcLien[2]{
     \href{https~://www.maths-cours.fr/#1}{#2 \tiny \faExternalLink}
}

\newcommand{\euro}{\eurologo{}}

%================================================================================================================================
%
% Macros - Environement
%
%================================================================================================================================

\newenvironment{tex}{ %
}
{%
}

\newenvironment{indente}{ %
	\setlength\parindent{10mm}
}

{
	\setlength\parindent{0mm}
}

\newenvironment{corrige}{%
     \needspace{3\baselineskip}
     \medskip
     \textbf{\textsc{Corrigé}}
     \medskip
}
{
}

\newenvironment{extern}{%
     \begin{center}
     }
     {
     \end{center}
}

\NewEnviron{code}{%
	\par
     \boite{gray}{\texttt{%
     \BODY
     }}
     \par
}

\newenvironment{vbloc}{% boite sans cadre empeche saut de page
     \begin{minipage}[t]{\linewidth}
     }
     {
     \end{minipage}
}
\NewEnviron{h2}{%
    \needspace{3\baselineskip}
    \vspace{0.6cm}
	\noindent \MakeUppercase{\sffamily \large \BODY}
	\vspace{1mm}\textcolor{mcgris}{\hrule}\vspace{0.4cm}
	\par
}{}

\NewEnviron{h3}{%
    \needspace{3\baselineskip}
	\vspace{5mm}
	\textsc{\BODY}
	\par
}

\NewEnviron{margeneg}{ %
\begin{addmargin}[-1cm]{0cm}
\BODY
\end{addmargin}
}

\NewEnviron{html}{%
}

\begin{document}
\meta{url}{/cours/algorithmes/}
\meta{pid}{230}
\meta{titre}{Algorithmes : Présentation}
\meta{type}{cours}
\begin{h2}1. Notion d'algorithme\end{h2}
\cadre{bleu}{Définition}{%
     Un \textbf{algorithme} est une suite d'instructions détaillées qui, si elles sont correctement exécutées, conduit à un résultat donné.
}
\bloc{orange}{Exemples}{%
     \begin{itemize}
          \item une recette de cuisine, une notice de montage peuvent être considérées comme des algorithmes.
          \item la suite d'instructions suivantes :
          \begin{code}
    1. choisir un nombre entier
    2. le multiplier par lui-même
    3. énoncer le résultat obtenu
\end{code}
          est un algorithme permettant d'obtenir le carré d'un nombre entier.
     \end{itemize}
}
\bloc{cyan}{Remarque}{%
     Dans la définition précédente, "détaillées" signifie que les instructions sont suffisamment précises pour pouvoir être mises en œuvre correctement par l'exécutant (homme ou machine)
}
\begin{h2}2. Pseudo-code\end{h2}
Les instructions doivent être formulées dans un langage compréhensible par l'exécutant. Dans le cas d'un humain, il s'agira du langage courant (langue maternelle), ; dans le cas d'une machine, il faudra recourir à un langage de \textbf{programmation} (assembleur, basic, C, java, php ...).
\par
En algorithmique, nous utiliserons un langage situé à mi-chemin entre le langage courant et un langage de programmation appelé \textit{pseudo-code}. Il n'y a pas de norme concernant ce pseudo-code qui peut varier légèrement d'un enseignant à l'autre. Le but est surtout de mettre l'accent sur la logique de l'algorithme. L'avantage du pseudo-code est qu'il permet de rester proche d'un langage informatique sans qu'il soit nécessaire de connaître toutes les règles et spécificités d'un langage particulier.
\begin{h2}3. Les variables\end{h2}
Un algorithme (ou un programme informatique), agit sur des nombres, des textes, ... Ces différents éléments sont stockés dans des \textbf{variables}. On peut se représenter une variable comme une boîte portant une étiquette ("le nom de la variable") à l'intérieur de laquelle on peut placer un contenu.
\begin{center}
     \imgsvg{var-5}{0.15}%width="119" alt="contenu d'une variable"
\end{center}
En informatique, les variables sont des emplacements réservés dans la mémoire de l'ordinateur auxquels on attribue une étiquette.
\cadre{bleu}{Définition}{%
     \textbf{Déclarer une variable}  c'est indiquer le nom et le type (nombre texte, tableau,...) d'une variable que l'on utilisera dans l'algorithme.
     \par
     La déclaration des variables se fait au début de l'algorithme avant la première instruction.
}
\bloc{cyan}{Remarques}{%
     \begin{itemize}
          \item Pour reprendre l'image précédente, déclarer une variable consiste à "créer la boîte"
          \item Les principaux types de variables que nous utiliserons seront : entier, nombre (=réel), texte (=chaîne de caractères), tableau de nombres ou de textes, logique (=booléen -cf chapitre suivant)
          \item Lorsqu'on déclare une variable dans un programme informatique, l'ordinateur affecte une étiquette à une zone de mémoire et éventuellement réserve de l'espace pour le contenu de cette variable en fonction de son type.
     \end{itemize}
}
\bloc{orange}{Exemple}{%
     Dans notre pseudo-code, nous déclarerons les variables de la façon suivante :
     \begin{code}
    variables
        x : nombre
        y : texte
        a, b, c : entiers
     \end{code}
     (Dans l'exemple précédent on définit 5 variables : x du type nombre (réel), y du type texte, et a, b et c de type entier.)
     \par
     Nous distinguerons  la déclaration des variables en plaçant le reste de l'algorithme entre les instructions "début algorithme" et "fin algorithme".
}
\cadre{bleu}{Définition}{%
     \textbf{Affecter une variable}, c'est attribuer une valeur à cette variable. Si la variable contenait déjà une valeur, cette ancienne valeur est effacée.
}
\bloc{cyan}{Remarques}{%
     \begin{itemize}
          \item Affecter une variable revient à "remplir la boîte"
          <img src="/wp-content/uploads/affect-var5.png" alt="" class="aligncenter size-full  img-pc" />
          \item Dans notre pseudo-code, nous utiliserons l'expression \textit{«prend la valeur»} pour l'affectation. Voici la déclaration et l'affectation de la variable x :
          \begin{code}
   variables
     x : entier 
   début algorithme 
     x prend la valeur 5
   fin algorithme
          \end{code}
          \item on ne peut affecter à une variable qu'une valeur du type qui a été défini lors de la déclaration. Le code suivant est incorrect (le // indique le début d'un commentaire):
          \begin{code}
   variables
     x : entier
   début algorithme 
     x prend la valeur "bonjour" // Erreur! x est de type entier !
   fin algorithme
          \end{code}
          \item Les textes (ou chaînes de caractères) doivent être entourés d'apostrophes afin de ne pas être confondus avec des noms de variables.
     \end{itemize}
}
Il est possible d'affecter à une variable le contenu d'une autre variable ou le résultat d'un calcul. Le contenu de l'autre variable n'est alors pas modifié. Par exemple :
\begin{code}
     variables 
        x, y, z : entiers
     début algorithme   
        x prend la valeur 5
        y prend la valeur x
        z prend la valeur x+y+1
     fin algorithme
\end{code}
A la fin de cet algorithme, x et y contiennent la valeur 5 et z la valeur 11 ( = 5+5+1).
\begin{h2}4. Les instructions d'entrée-sortie\end{h2}
Faire effectuer un calcul à une machine c'est bien ... Mais il faut au moins être capable d'entrer des valeurs et il faut aussi que la machine puisse afficher un résultat !
\par
Les instructions qui permettent de "dialoguer" avec une machine s'appellent les instructions "\textbf{d'entrée/sortie}" ou de "\textbf{lecture/écriture}"
\bloc{orange}{Lecture}{%
     Dans notre pseudo-code nous utiliserons l'instruction \textbf{\textit{lire}} (ou \textit{entrer}, ou \textit{saisir}, etc.) \textbf{suivie du nom d'une variable } pour pouvoir saisir une valeur (en anglais cette instruction se nomme généralement \textit{input}).
     \par
     Lorsqu'elle rencontre une telle instruction, la machine s'\textbf{arrête} et \textbf{attend que l'utilisateur entre une valeur}. Une fois la valeur saisie, la machine\textbf{ affecte la valeur saisie} à la variable dont le nom figure après \textit{lire}. Ensuite, elle passe à l'\textbf{instruction suivante}.
     \begin{code}
   variables
     x : entier
   début algorithme 
     lire x
     y prend la valeur 2*x
   fin algorithme
     \end{code}
     Cet algorithme demande d'\textbf{entrer} un nombre entier, \textbf{stocke} la valeur de ce nombre dans la variable x, puis \textbf{calcule le double} du nombre entré et \textbf{affecte ce double à la variable y}.
     \par
     Le résultat n'est pas affiché (d'où le paragraphe suivant...)
     \par
     Remarque : Dans un véritable programme, il faudrait vérifier que la valeur entrée est bien du type désiré (ici un entier). Dans un algorithme, on n'écrit pas cette vérification.
}

\bloc{orange}{Ecriture}{%
     Dans notre pseudo-code nous utiliserons l'instruction \textbf{\textit{afficher}} suivie du nom d'une variable ou d'une constante (nombre, texte ...) pour afficher une valeur (on peut également utiliser \textit{\textbf{"écrire"}} ou \textit{\textbf{"print"}} en anglais).
     \par
     Pour afficher un texte on utilise des \textbf{guillemets} (simples ou doubles) :
     \begin{code}
          ...
          afficher 'Ce texte sera affiché'
          ...
     \end{code}
     Il est fréquent d'afficher un texte pour donner des \textbf{consignes} ou des \textbf{informations} à l'utilisateur.
     \par
     Pour afficher le contenu d'une variable on fait suivre \textit{"afficher"} du nom de la variable \textbf{sans guillemet} :
     \begin{code}
	  ...
          afficher x
          ...
     \end{code}
     L'algorithme suivant calcule puis affiche l'âge qu'aura une personne en 2100 :
     \begin{code}
    variables 
       annee, age, age\_en\_2100 : entiers  
    début algorithme  
       afficher "Entrer l'année actuelle" 
       lire année
       afficher "Entrer votre âge"
       lire âge
       âge\_en\_2100 prend la valeur âge + 2100 - année
       afficher "En 2100, vous aurez ", âge\_en\_2100, " ans."
    fin algorithme
     \end{code}
     Si l'utilisateur entre \textit{"2014"} comme année et \textit{"16"} comme âge, l'algorithme affichera~:\\
     \textit{En 2100, vous aurez 102 ans.}
}

\end{document}