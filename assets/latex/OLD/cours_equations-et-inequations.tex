\documentclass[a4paper]{article}

%================================================================================================================================
%
% Packages
%
%================================================================================================================================

\usepackage[T1]{fontenc} 	% pour caractères accentués
\usepackage[utf8]{inputenc}  % encodage utf8
\usepackage[french]{babel}	% langue : français
\usepackage{fourier}			% caractères plus lisibles
\usepackage[dvipsnames]{xcolor} % couleurs
\usepackage{fancyhdr}		% réglage header footer
\usepackage{needspace}		% empêcher sauts de page mal placés
\usepackage{graphicx}		% pour inclure des graphiques
\usepackage{enumitem,cprotect}		% personnalise les listes d'items (nécessaire pour ol, al ...)
\usepackage{hyperref}		% Liens hypertexte
\usepackage{pstricks,pst-all,pst-node,pstricks-add,pst-math,pst-plot,pst-tree,pst-eucl} % pstricks
\usepackage[a4paper,includeheadfoot,top=2cm,left=3cm, bottom=2cm,right=3cm]{geometry} % marges etc.
\usepackage{comment}			% commentaires multilignes
\usepackage{amsmath,environ} % maths (matrices, etc.)
\usepackage{amssymb,makeidx}
\usepackage{bm}				% bold maths
\usepackage{tabularx}		% tableaux
\usepackage{colortbl}		% tableaux en couleur
\usepackage{fontawesome}		% Fontawesome
\usepackage{environ}			% environment with command
\usepackage{fp}				% calculs pour ps-tricks
\usepackage{multido}			% pour ps tricks
\usepackage[np]{numprint}	% formattage nombre
\usepackage{tikz,tkz-tab} 			% package principal TikZ
\usepackage{pgfplots}   % axes
\usepackage{mathrsfs}    % cursives
\usepackage{calc}			% calcul taille boites
\usepackage[scaled=0.875]{helvet} % font sans serif
\usepackage{svg} % svg
\usepackage{scrextend} % local margin
\usepackage{scratch} %scratch
\usepackage{multicol} % colonnes
%\usepackage{infix-RPN,pst-func} % formule en notation polanaise inversée
\usepackage{listings}

%================================================================================================================================
%
% Réglages de base
%
%================================================================================================================================

\lstset{
language=Python,   % R code
literate=
{á}{{\'a}}1
{à}{{\`a}}1
{ã}{{\~a}}1
{é}{{\'e}}1
{è}{{\`e}}1
{ê}{{\^e}}1
{í}{{\'i}}1
{ó}{{\'o}}1
{õ}{{\~o}}1
{ú}{{\'u}}1
{ü}{{\"u}}1
{ç}{{\c{c}}}1
{~}{{ }}1
}


\definecolor{codegreen}{rgb}{0,0.6,0}
\definecolor{codegray}{rgb}{0.5,0.5,0.5}
\definecolor{codepurple}{rgb}{0.58,0,0.82}
\definecolor{backcolour}{rgb}{0.95,0.95,0.92}

\lstdefinestyle{mystyle}{
    backgroundcolor=\color{backcolour},   
    commentstyle=\color{codegreen},
    keywordstyle=\color{magenta},
    numberstyle=\tiny\color{codegray},
    stringstyle=\color{codepurple},
    basicstyle=\ttfamily\footnotesize,
    breakatwhitespace=false,         
    breaklines=true,                 
    captionpos=b,                    
    keepspaces=true,                 
    numbers=left,                    
xleftmargin=2em,
framexleftmargin=2em,            
    showspaces=false,                
    showstringspaces=false,
    showtabs=false,                  
    tabsize=2,
    upquote=true
}

\lstset{style=mystyle}


\lstset{style=mystyle}
\newcommand{\imgdir}{C:/laragon/www/newmc/assets/imgsvg/}
\newcommand{\imgsvgdir}{C:/laragon/www/newmc/assets/imgsvg/}

\definecolor{mcgris}{RGB}{220, 220, 220}% ancien~; pour compatibilité
\definecolor{mcbleu}{RGB}{52, 152, 219}
\definecolor{mcvert}{RGB}{125, 194, 70}
\definecolor{mcmauve}{RGB}{154, 0, 215}
\definecolor{mcorange}{RGB}{255, 96, 0}
\definecolor{mcturquoise}{RGB}{0, 153, 153}
\definecolor{mcrouge}{RGB}{255, 0, 0}
\definecolor{mclightvert}{RGB}{205, 234, 190}

\definecolor{gris}{RGB}{220, 220, 220}
\definecolor{bleu}{RGB}{52, 152, 219}
\definecolor{vert}{RGB}{125, 194, 70}
\definecolor{mauve}{RGB}{154, 0, 215}
\definecolor{orange}{RGB}{255, 96, 0}
\definecolor{turquoise}{RGB}{0, 153, 153}
\definecolor{rouge}{RGB}{255, 0, 0}
\definecolor{lightvert}{RGB}{205, 234, 190}
\setitemize[0]{label=\color{lightvert}  $\bullet$}

\pagestyle{fancy}
\renewcommand{\headrulewidth}{0.2pt}
\fancyhead[L]{maths-cours.fr}
\fancyhead[R]{\thepage}
\renewcommand{\footrulewidth}{0.2pt}
\fancyfoot[C]{}

\newcolumntype{C}{>{\centering\arraybackslash}X}
\newcolumntype{s}{>{\hsize=.35\hsize\arraybackslash}X}

\setlength{\parindent}{0pt}		 
\setlength{\parskip}{3mm}
\setlength{\headheight}{1cm}

\def\ebook{ebook}
\def\book{book}
\def\web{web}
\def\type{web}

\newcommand{\vect}[1]{\overrightarrow{\,\mathstrut#1\,}}

\def\Oij{$\left(\text{O}~;~\vect{\imath},~\vect{\jmath}\right)$}
\def\Oijk{$\left(\text{O}~;~\vect{\imath},~\vect{\jmath},~\vect{k}\right)$}
\def\Ouv{$\left(\text{O}~;~\vect{u},~\vect{v}\right)$}

\hypersetup{breaklinks=true, colorlinks = true, linkcolor = OliveGreen, urlcolor = OliveGreen, citecolor = OliveGreen, pdfauthor={Didier BONNEL - https://www.maths-cours.fr} } % supprime les bordures autour des liens

\renewcommand{\arg}[0]{\text{arg}}

\everymath{\displaystyle}

%================================================================================================================================
%
% Macros - Commandes
%
%================================================================================================================================

\newcommand\meta[2]{    			% Utilisé pour créer le post HTML.
	\def\titre{titre}
	\def\url{url}
	\def\arg{#1}
	\ifx\titre\arg
		\newcommand\maintitle{#2}
		\fancyhead[L]{#2}
		{\Large\sffamily \MakeUppercase{#2}}
		\vspace{1mm}\textcolor{mcvert}{\hrule}
	\fi 
	\ifx\url\arg
		\fancyfoot[L]{\href{https://www.maths-cours.fr#2}{\black \footnotesize{https://www.maths-cours.fr#2}}}
	\fi 
}


\newcommand\TitreC[1]{    		% Titre centré
     \needspace{3\baselineskip}
     \begin{center}\textbf{#1}\end{center}
}

\newcommand\newpar{    		% paragraphe
     \par
}

\newcommand\nosp {    		% commande vide (pas d'espace)
}
\newcommand{\id}[1]{} %ignore

\newcommand\boite[2]{				% Boite simple sans titre
	\vspace{5mm}
	\setlength{\fboxrule}{0.2mm}
	\setlength{\fboxsep}{5mm}	
	\fcolorbox{#1}{#1!3}{\makebox[\linewidth-2\fboxrule-2\fboxsep]{
  		\begin{minipage}[t]{\linewidth-2\fboxrule-4\fboxsep}\setlength{\parskip}{3mm}
  			 #2
  		\end{minipage}
	}}
	\vspace{5mm}
}

\newcommand\CBox[4]{				% Boites
	\vspace{5mm}
	\setlength{\fboxrule}{0.2mm}
	\setlength{\fboxsep}{5mm}
	
	\fcolorbox{#1}{#1!3}{\makebox[\linewidth-2\fboxrule-2\fboxsep]{
		\begin{minipage}[t]{1cm}\setlength{\parskip}{3mm}
	  		\textcolor{#1}{\LARGE{#2}}    
 	 	\end{minipage}  
  		\begin{minipage}[t]{\linewidth-2\fboxrule-4\fboxsep}\setlength{\parskip}{3mm}
			\raisebox{1.2mm}{\normalsize\sffamily{\textcolor{#1}{#3}}}						
  			 #4
  		\end{minipage}
	}}
	\vspace{5mm}
}

\newcommand\cadre[3]{				% Boites convertible html
	\par
	\vspace{2mm}
	\setlength{\fboxrule}{0.1mm}
	\setlength{\fboxsep}{5mm}
	\fcolorbox{#1}{white}{\makebox[\linewidth-2\fboxrule-2\fboxsep]{
  		\begin{minipage}[t]{\linewidth-2\fboxrule-4\fboxsep}\setlength{\parskip}{3mm}
			\raisebox{-2.5mm}{\sffamily \small{\textcolor{#1}{\MakeUppercase{#2}}}}		
			\par		
  			 #3
 	 		\end{minipage}
	}}
		\vspace{2mm}
	\par
}

\newcommand\bloc[3]{				% Boites convertible html sans bordure
     \needspace{2\baselineskip}
     {\sffamily \small{\textcolor{#1}{\MakeUppercase{#2}}}}    
		\par		
  			 #3
		\par
}

\newcommand\CHelp[1]{
     \CBox{Plum}{\faInfoCircle}{À RETENIR}{#1}
}

\newcommand\CUp[1]{
     \CBox{NavyBlue}{\faThumbsOUp}{EN PRATIQUE}{#1}
}

\newcommand\CInfo[1]{
     \CBox{Sepia}{\faArrowCircleRight}{REMARQUE}{#1}
}

\newcommand\CRedac[1]{
     \CBox{PineGreen}{\faEdit}{BIEN R\'EDIGER}{#1}
}

\newcommand\CError[1]{
     \CBox{Red}{\faExclamationTriangle}{ATTENTION}{#1}
}

\newcommand\TitreExo[2]{
\needspace{4\baselineskip}
 {\sffamily\large EXERCICE #1\ (\emph{#2 points})}
\vspace{5mm}
}

\newcommand\img[2]{
          \includegraphics[width=#2\paperwidth]{\imgdir#1}
}

\newcommand\imgsvg[2]{
       \begin{center}   \includegraphics[width=#2\paperwidth]{\imgsvgdir#1} \end{center}
}


\newcommand\Lien[2]{
     \href{#1}{#2 \tiny \faExternalLink}
}
\newcommand\mcLien[2]{
     \href{https~://www.maths-cours.fr/#1}{#2 \tiny \faExternalLink}
}

\newcommand{\euro}{\eurologo{}}

%================================================================================================================================
%
% Macros - Environement
%
%================================================================================================================================

\newenvironment{tex}{ %
}
{%
}

\newenvironment{indente}{ %
	\setlength\parindent{10mm}
}

{
	\setlength\parindent{0mm}
}

\newenvironment{corrige}{%
     \needspace{3\baselineskip}
     \medskip
     \textbf{\textsc{Corrigé}}
     \medskip
}
{
}

\newenvironment{extern}{%
     \begin{center}
     }
     {
     \end{center}
}

\NewEnviron{code}{%
	\par
     \boite{gray}{\texttt{%
     \BODY
     }}
     \par
}

\newenvironment{vbloc}{% boite sans cadre empeche saut de page
     \begin{minipage}[t]{\linewidth}
     }
     {
     \end{minipage}
}
\NewEnviron{h2}{%
    \needspace{3\baselineskip}
    \vspace{0.6cm}
	\noindent \MakeUppercase{\sffamily \large \BODY}
	\vspace{1mm}\textcolor{mcgris}{\hrule}\vspace{0.4cm}
	\par
}{}

\NewEnviron{h3}{%
    \needspace{3\baselineskip}
	\vspace{5mm}
	\textsc{\BODY}
	\par
}

\NewEnviron{margeneg}{ %
\begin{addmargin}[-1cm]{0cm}
\BODY
\end{addmargin}
}

\NewEnviron{html}{%
}

\begin{document}
\begin{h2}I. Equations\end{h2}
\cadre{rouge}{Théorème}{%id="t10"
     \begin{itemize}
          \item Si l'on ajoute ou si l'on soustrait un même nombre à chaque membre d'une équation, on obtient une équation équivalente (c'est à dire qui possède les mêmes solutions).
          \item Si l'on multiplie ou si l'on divise chaque membre d'une équation par un même nombre \textbf{non nul}, on obtient une équation équivalente.
     \end{itemize}
}
\bloc{cyan}{Remarque}{%id="r10"
     Pour résoudre une équation du type $ax+b=0$ on soustrait $b$ à chaque membre de l'égalité:
     \par
     $ax+b-b=0-b$ c'est à dire $ax=-b$.
     \par
     Puis:
     \begin{itemize}
          \item si $a$ est \textbf{non nul} on divise chaque membre par $a$~:~$\frac{ax}{a}=-\frac{b}{a}$ soit $x=-\frac{b}{a}$ donc $S=\left\{-\frac{b}{a}\right\}$
          \item si $a=0$:
          \begin{itemize}
               \item si $b=0$ l'équation se réduit à $0=0$. Elle est toujours vérifiée donc $S=\mathbb{R}$
               \item si $b\neq 0$ l'équation se réduit à $b=0$. Elle n'est jamais vérifiée donc $S=\varnothing$
          \end{itemize}
     \end{itemize}
}
\cadre{rouge}{Théorème (Équation produit)}{%id="t20"
     Un produit de facteurs est nul si et seulement si au moins un des facteurs est nul.
     \par
     En particulier, une équation du type $A(x)\times B(x)=0$ est vérifiée si et seulement si :
     \par
     $A(x)=0$ \textbf{ou} $B(x)=0$
}
\bloc{orange}{Exemple}{%id="e20"
     Soit l'équation $(3x-5)(x+2)=0$
     \par
     Cette équation est équivalente à $3x-5=0$ \textbf{ou} $x+2=0$.
     \par
     C'est à dire $x=\frac{5}{3}$ \textbf{ou} $x=-2$.
     \par
     L'ensemble des solutions de l'équation est donc $S=\left\{-2;\frac{5}{3}\right\}$
}
\bloc{cyan}{Remarques}{%id="r20"
     \begin{itemize}
          \item Lorsqu'on a affaire à une équation du second degré (ou plus), on fait "passer" tous les termes dans le membre de gauche que l'on essaie de factoriser et on utilise le théorème précédent.
          \item On rappelle les identités remarquables qui peuvent être utiles dans ce genre de situations:
          \begin{center}$(a+b)^2=a^2+2ab+b^2$
               \par
               $(a-b)^2=a^2-2ab+b^2$
               \par
          $(a+b)(a-b)=a^2-b^2$\end{center}
     \end{itemize}
}
\cadre{rouge}{Théorème}{%id="t30"
     Un quotient est \textbf{défini} si et seulement si son \textbf{dénominateur} est \textbf{non nul}.
     \par
     S'il est défini, un quotient est \textbf{nul} si et seulement si son \textbf{numérateur} est \textbf{nul}.
}
\bloc{orange}{Exemple}{%id="r30"
     Soit l'équation $\frac{2x-4}{x+1}=0$
     \par
     Cette équation a un sens si $x+1 \neq 0$ donc si $x\neq -1$
     \par
     Sur l'ensemble $\mathbb{R}\backslash\left\{-1\right\}$ cette équation est équivalente à $2x-4=0$ donc à $x=2$. L'ensemble des solutions de l'équation est donc $S=\left\{2\right\}$
}
\cadre{vert}{Propriété}{%id="p40"
     Soit $f$ une fonction définie sur $D$ de courbe représentative $\mathscr{C}_f$.
     \par
     Les solutions de l'équation $f(x)=m$ sont les \textbf{abscisses} des points d'intersection de la courbe $\mathscr{C}_f$ et de la droite horizontale d'équation $y=m$
}
\bloc{orange}{Exemple}{%id="e40"
     \begin{center}
          \begin{extern}%width="320" alt="équation et graphique"
               \resizebox{7cm}{!}{
                    \psset{xunit=1.0cm,yunit=1.0cm,algebraic=true,dimen=middle,linewidth=0.6pt,arrowsize=3pt}
                    \begin{pspicture*}(-2.5,-2.5)(5.5,6.)
                         \psaxes[xAxis=true,yAxis=true,Dx=1.,Dy=1.]{->}(0,0)(-2.5,-2.5)(5.5,6.)
                         \psplot[linewidth=0.8pt,linecolor=blue,plotpoints=200]{-2.5}{5.5}{(x-1.0)^(2.0)-2.0}
                         \psplot[linewidth=0.8pt,linecolor=red,plotpoints=200]{-2.5}{5.5}{2.0}
                         \psline[linewidth=0.8pt,linestyle=dashed,dash=1pt 1pt](-1.,2.)(-1.,0.)
                         \psline[linewidth=0.8pt,linestyle=dashed,dash=1pt 1pt](3.,2.)(3.,0.)
                         \rput[tl](4.6,2.4){\red{$y=2$}}
                         \rput[tl](3.97,5.8){\blue{$\mathscr{C}_f$}}
                         \psdots[dotsize=1pt 0,dotstyle=*](0.,0.)
                         \rput[bl](-0.5,-0.5){$O$}
                    \end{pspicture*}
               }
          \end{extern}
     \end{center}
     Sur la figure ci-dessus, l'équation $f(x)=2$ possède deux solutions qui sont -1 et 3
}
\cadre{rouge}{Théorème}{%id="t45"
     L'équation $x^2=a$ :
     \begin{itemize}
          \item admet deux solutions $x=\sqrt{a}$ ou $x=-\sqrt{a}$ si $a > 0$
          \item admet une unique solution $x=0$ si $a=0$
          \item n'admet aucune solution réelle si $a < 0$
     \end{itemize}
}
\bloc{orange}{Exemple}{%id="e45"
     \begin{itemize}
          \item L'équation $x^2=1$ admet deux solutions qui sont $x=-1$ et $x=1$
          \item L'équation $x^2+1=0$ est équivalente à $x^2=-1$ et n'admet donc aucune solution
     \end{itemize}
}
\begin{h2}II. Inéquations\end{h2}
\cadre{rouge}{Théorème}{%id="t50"
     \begin{itemize}
          \item Si l'on ajoute ou si l'on soustrait un même nombre à chaque membre d'une inéquation, on obtient une inéquation équivalente (c'est à dire qui à les mêmes solutions).
          \item Si l'on multiplie ou si l'on divise chaque membre d'une inéquation par un même nombre \textbf{strictement positif}, on obtient une inéquation équivalente.
          \item Si l'on multiplie ou si l'on divise chaque membre d'une inéquation par un même nombre \textbf{strictement négatif}, on obtient une inéquation équivalente \textbf{en changeant le sens de l'inégalité}.
     \end{itemize}
}
\bloc{orange}{Exemple}{%id="e50"
     Pour résoudre l'inéquation $-3x+5 > 0$ on soustrait 5 à chaque membre de l'inéquation:
     \par
     $-3x+5-5 > 0-5$ c'est à dire $-3x > -5$.
     \par
     Puis comme -3 est négatif on divise chaque membre par -3 \textbf{en changeant le sens de l'inégalité :}
\par
     $\frac{-3x}{-3} < \frac{-5}{-3}$
     \par
     $x < \frac{5}{3}$
     \par
     Donc $S=\left]-\infty ;\frac{5}{3}\right[$
}
\bloc{cyan}{Remarques}{%id="r50"
     En appliquant le théorème précédent à l'expression $ax+b$ on obtient :
     \par
     $ax+b > 0  \Leftrightarrow   ax > -b \Leftrightarrow   x > -\frac{b}{a}$ si $a$ est strictement positif
     \par
     et $ax+b > 0  \Leftrightarrow   ax > -b \Leftrightarrow   x < -\frac{b}{a}$ si $a$ est strictement négatif.
     \par
     On peut alors regrouper ces deux cas dans le tableau de signe suivant :
     \begin{center}
          \begin{extern}%width="390" alt="Tableau de signe polynôme du premier degré"
               \resizebox{8cm}{!}{
                    \begin{tikzpicture}[scale=0.875]
                         % Styles
                         \tikzstyle{cadre}=[thin]
                         \tikzstyle{fleche}=[->,>=latex,thin]
                         \tikzstyle{nondefini}=[lightgray]
                         % Dimensions Modifiables
                         \def\Lrg{1.8}
                         \def\HtX{1.2}
                         \def\HtY{0.5}
                         % Dimensions Calculées
                         \def\lignex{-0.5*\HtX}
                         \def\lignef{-1.5*\HtX}
                         \def\separateur{-0.5*\Lrg}
                         % Largeur du tableau
                         \def\gauche{-1.5*\Lrg}
                         \def\droite{4.5*\Lrg}
                         % Hauteur du tableau
                         \def\haut{0.5*\HtX}
                         \def\bas{-2.5*\HtX-2*\HtY}
                         % Pointillés
                         \draw[gray] (2*\Lrg,\lignex) -- (2*\Lrg,\lignef);
                         % Ligne de l'abscisse : x
                         \node at (-1*\Lrg,0) {$x$};
                         \node at (0*\Lrg,0) {$-\infty$};
                         \node at (2*\Lrg,0) {$-\dfrac{b}{a}$};
                         \node at (4*\Lrg,0) {$+\infty$};
                         % Ligne de la dérivée : f'(x)
                         \node at (-1*\Lrg,-1*\HtX) {$ax+b$};
                         \node at (0*\Lrg,-1*\HtX) {$ $};
                         \node at (1*\Lrg,-1*\HtX) {signe de $-a$};
                         \node at (2*\Lrg,-1*\HtX) {$0$};
                         \node at (3*\Lrg,-1*\HtX) {signe de $a$};
                         \node at (4*\Lrg,-1*\HtX) {$ $};
                         % Ligne de la fonction : f(x)
                         % Encadrement
                         \draw[cadre] (\separateur,\haut) -- (\separateur, \lignef);
                         \draw[cadre] (\gauche,\haut) rectangle  (\droite, \lignef);
                         \draw[cadre] (\gauche,\lignex) -- (\droite,\lignex);
                    \end{tikzpicture}
               }
          \end{extern}
     \end{center}
}
\cadre{rouge}{Théorème (Inéquation produit)}{%id="e60"
     Un produit de facteurs $A(x)B(x)$ est \textbf{positif ou nul} si et seulement si les deux facteurs $A(x)$ et $B(x)$ sont de \textbf{même signe}.
     \par
     Ce produit est \textbf{négatif ou nul} si et seulement si les deux facteurs $A(x)$ et $B(x)$ sont de \textbf{signes contraires}.
}
\bloc{cyan}{Remarques}{%id="r60"
     Lorsqu'on a affaire à une inéquation du second degré (ou plus), on fait "passer" tous les termes dans le membre de gauche que l'on essaie de factoriser puis on utilise un tableau de signe.
}
\bloc{orange}{Exemple}{%id="e60"
     Soit l'inéquation $(x-5)(-3x+4)\geqslant 0$
     \par
     Le signe de $x-5$ est donné par le tableau:
     \begin{center}
          \begin{extern}%width="390" alt="Exemple tableau de signe 1"
               \resizebox{8cm}{!}{
                    \begin{tikzpicture}[scale=0.875]
                         % Styles
                         \tikzstyle{cadre}=[thin]
                         \tikzstyle{fleche}=[->,>=latex,thin]
                         \tikzstyle{nondefini}=[lightgray]
                         % Dimensions Modifiables
                         \def\Lrg{1.8}
                         \def\HtX{1.2}
                         \def\HtY{0.5}
                         % Dimensions Calculées
                         \def\lignex{-0.5*\HtX}
                         \def\lignef{-1.5*\HtX}
                         \def\separateur{-0.5*\Lrg}
                         % Largeur du tableau
                         \def\gauche{-1.5*\Lrg}
                         \def\droite{4.5*\Lrg}
                         % Hauteur du tableau
                         \def\haut{0.5*\HtX}
                         \def\bas{-2.5*\HtX-2*\HtY}
                         % Pointillés
                         \draw[gray] (2*\Lrg,\lignex) -- (2*\Lrg,\lignef);
                         % Ligne de l'abscisse : x
                         \node at (-1*\Lrg,0) {$x$};
                         \node at (0*\Lrg,0) {$-\infty$};
                         \node at (2*\Lrg,0) {$5$};
                         \node at (4*\Lrg,0) {$+\infty$};
                         % Ligne de la dérivée : f'(x)
                         \node at (-1*\Lrg,-1*\HtX) {$x-5$};
                         \node at (0*\Lrg,-1*\HtX) {$ $};
                         \node at (1*\Lrg,-1*\HtX) {$-$};
                         \node at (2*\Lrg,-1*\HtX) {$0$};
                         \node at (3*\Lrg,-1*\HtX) {$+$};
                         \node at (4*\Lrg,-1*\HtX) {$ $};
                         % Ligne de la fonction : f(x)
                         % Encadrement
                         \draw[cadre] (\separateur,\haut) -- (\separateur, \lignef);
                         \draw[cadre] (\gauche,\haut) rectangle  (\droite, \lignef);
                         \draw[cadre] (\gauche,\lignex) -- (\droite,\lignex);
                    \end{tikzpicture}
               }
          \end{extern}
     \end{center}
     Le signe de $-3x+4$ est donné par le tableau:
     \begin{center}
          \begin{extern}%width="390" alt="Exemple tableau de signe 2"
               \resizebox{8cm}{!}{
                    \begin{tikzpicture}[scale=0.875]
                         % Styles
                         \tikzstyle{cadre}=[thin]
                         \tikzstyle{fleche}=[->,>=latex,thin]
                         \tikzstyle{nondefini}=[lightgray]
                         % Dimensions Modifiables
                         \def\Lrg{1.8}
                         \def\HtX{1.2}
                         \def\HtY{0.5}
                         % Dimensions Calculées
                         \def\lignex{-0.5*\HtX}
                         \def\lignef{-1.5*\HtX}
                         \def\separateur{-0.5*\Lrg}
                         % Largeur du tableau
                         \def\gauche{-1.5*\Lrg}
                         \def\droite{4.5*\Lrg}
                         % Hauteur du tableau
                         \def\haut{0.5*\HtX}
                         \def\bas{-2.5*\HtX-2*\HtY}
                         % Pointillés
                         \draw[gray] (2*\Lrg,\lignex) -- (2*\Lrg,\lignef);
                         % Ligne de l'abscisse : x
                         \node at (-1*\Lrg,0) {$x$};
                         \node at (0*\Lrg,0) {$-\infty$};
                         \node at (2*\Lrg,0) {$\dfrac{4}{3}$};
                         \node at (4*\Lrg,0) {$+\infty$};
                         % Ligne de la dérivée : f'(x)
                         \node at (-1*\Lrg,-1*\HtX) {$-3x+4$};
                         \node at (0*\Lrg,-1*\HtX) {$ $};
                         \node at (1*\Lrg,-1*\HtX) {$+$};
                         \node at (2*\Lrg,-1*\HtX) {$0$};
                         \node at (3*\Lrg,-1*\HtX) {$-$};
                         \node at (4*\Lrg,-1*\HtX) {$ $};
                         % Ligne de la fonction : f(x)
                         % Encadrement
                         \draw[cadre] (\separateur,\haut) -- (\separateur, \lignef);
                         \draw[cadre] (\gauche,\haut) rectangle  (\droite, \lignef);
                         \draw[cadre] (\gauche,\lignex) -- (\droite,\lignex);
                    \end{tikzpicture}
               }
          \end{extern}
     \end{center}
     On regroupe ces résultats dans un unique tableau et on utilise la règle des signes pour obtenir le signe du produit:
     \begin{center}
          \begin{extern}%width="500" alt="Exemple tableau de signes d'un produit"
               \resizebox{11cm}{!}{
                    \begin{tikzpicture}[scale=0.875]
                         % Styles
                         \tikzstyle{cadre}=[thin]
                         \tikzstyle{fleche}=[->,>=latex,thin]
                         \tikzstyle{nondefini}=[lightgray]
                         % Dimensions Modifiables
                         \def\Lrg{1.5}
                         \def\HtX{1.2}
                         \def\HtY{0.5}
                         % Dimensions Calculées
                         \def\lignex{-0.5*\HtX}
                         \def\lignea{-1.5*\HtX}
                         \def\ligneb{-2.5*\HtX}
                         \def\lignec{-3.5*\HtX}
                         \def\separateur{-0.5*\Lrg}
                         % Largeur du tableau
                         \def\gauche{-3.1*\Lrg}
                         \def\droite{6.5*\Lrg}
                         % Hauteur du tableau
                         \def\haut{0.5*\HtX}
                         \def\bas{-2.5*\HtX-2*\HtY}
                         % Pointillés
                         \draw[gray] (2*\Lrg,\lignex) -- (2*\Lrg,\lignec);
                         \draw[gray] (4*\Lrg,\lignex) -- (4*\Lrg,\lignec);
                         % Ligne de l'abscisse : x
                         \node at (-1.8*\Lrg,0) {$x$};
                         \node at (0*\Lrg,0) {$-\infty$};
                         \node at (2*\Lrg,0) {$\dfrac{4}{3}$};
                         \node at (4*\Lrg,0) {$5$};
                         \node at (6*\Lrg,0) {$+\infty$};
                         % Ligne a
                         \node at (-1.8*\Lrg,-1*\HtX) {$x-5$};
                         \node at (0*\Lrg,-1*\HtX) {$ $};
                         \node at (1*\Lrg,-1*\HtX) {$-$};
                         \node at (2*\Lrg,-1*\HtX) {$ $};
                         \node at (3*\Lrg,-1*\HtX) {$-$};
                         \node at (4*\Lrg,-1*\HtX) {$0$};
                         \node at (5*\Lrg,-1*\HtX) {$+$};
                         \node at (6*\Lrg,-1*\HtX) {$ $};
                         % Ligne b
                         \node at (-1.8*\Lrg,-2*\HtX) {$-3x+4$};
                         \node at (2*\Lrg,-2*\HtX) {$ $};
                         \node at (1*\Lrg,-2*\HtX) {$+$};
                         \node at (2*\Lrg,-2*\HtX) {$0$};
                         \node at (3*\Lrg,-2*\HtX) {$-$};
                         \node at (4*\Lrg,-2*\HtX) {$ $};
                         \node at (5*\Lrg,-2*\HtX) {$-$};
                         \node at (6*\Lrg,-2*\HtX) {$ $};
                         % Ligne c
                         \node at (-1.8*\Lrg,-3*\HtX) {$(x-5)(-3x+4)$};
                         \node at (0*\Lrg,-3*\HtX) {$ $};
                         \node at (1*\Lrg,-3*\HtX) {$-$};
                         \node at (2*\Lrg,-3*\HtX) {$0$};
                         \node at (3*\Lrg,-3*\HtX) {$+$};
                         \node at (4*\Lrg,-3*\HtX) {$0$};
                         \node at (5*\Lrg,-3*\HtX) {$-$};
                         \node at (6*\Lrg,-3*\HtX) {$ $};
                         % Encadrement
                         \draw[cadre] (\separateur,\haut) -- (\separateur, \lignec);
                         \draw[cadre] (\gauche,\haut) rectangle  (\droite, \lignec);
                         \draw[cadre] (\gauche,\lignex) -- (\droite,\lignex);
                         \draw[cadre] (\gauche,\lignea) -- (\droite,\lignea);
                         \draw[cadre] (\gauche,\ligneb) -- (\droite,\ligneb);
                    \end{tikzpicture}
               }
          \end{extern}
     \end{center}
     $(x-5)(-3x+4)$ est positif ou nul sur l'intervalle $\left[\frac{4}{3}; 5\right]$
     \par
     Pour plus de détails et d'autres exemples, consulter la fiche méthode : \mcLien{/methode/dresser-tableau-de-signes/}{Dresser un tableau de signes}
}
\cadre{rouge}{Théorème (Inéquation quotient)}{%id="t70"
     Un quotient $\frac{A(x)}{B(x)}$ est \textbf{défini} si et seulement si son \textbf{dénominateur} $B(x)$ est \textbf{non nul}.
     \par
     S'il est défini, il est \textbf{positif ou nul} si et seulement si $A(x)$ et $B(x)$ sont de \textbf{même signe} et il est \textbf{négatif ou nul} si et seulement si les deux facteurs $A(x)$ et $B(x)$ sont de \textbf{signes contraires}.
}
\bloc{orange}{Exemple}{%id="e70"
     Soit l'inéquation $\frac{2x-5}{x+2}\geqslant 0$
     \par
     Cette inéquation a un sens si $x+2 \neq 0$ donc si $x\neq -2$
     \par
     Le tableau de signe de $\frac{2x-5}{x+2}$ est :
     \begin{center}
          \begin{extern}%width="500" alt="Exemple tableau de signes d'un quotient"
               \resizebox{11cm}{!}{
                    \begin{tikzpicture}[scale=0.875]
                         % Styles
                         \tikzstyle{cadre}=[thin]
                         \tikzstyle{fleche}=[->,>=latex,thin]
                         \tikzstyle{nondefini}=[lightgray]
                         % Dimensions Modifiables
                         \def\Lrg{1.5}
                         \def\HtX{1.2}
                         \def\HtY{0.5}
                         % Dimensions Calculées
                         \def\lignex{-0.5*\HtX}
                         \def\lignea{-1.5*\HtX}
                         \def\ligneb{-2.5*\HtX}
                         \def\lignec{-3.5*\HtX}
                         \def\separateur{-0.5*\Lrg}
                         % Largeur du tableau
                         \def\gauche{-3.1*\Lrg}
                         \def\droite{6.5*\Lrg}
                         % Hauteur du tableau
                         \def\haut{0.5*\HtX}
                         \def\bas{-2.5*\HtX-2*\HtY}
                         % Pointillés
                         \draw[gray] (2*\Lrg,\lignex) -- (2*\Lrg,\ligneb);
                         \draw[gray] (4*\Lrg,\lignex) -- (4*\Lrg,\lignec);
                         \draw[double distance=2pt] (2*\Lrg,\ligneb) -- (2*\Lrg,\lignec);
                         % Ligne de l'abscisse : x
                         \node at (-1.8*\Lrg,0) {$x$};
                         \node at (0*\Lrg,0) {$-\infty$};
                         \node at (2*\Lrg,0) {$-2$};
                         \node at (4*\Lrg,0) {$\dfrac{5}{2}$};
                         \node at (6*\Lrg,0) {$+\infty$};
                         % Ligne a
                         \node at (-1.8*\Lrg,-1*\HtX) {$2x-5$};
                         \node at (0*\Lrg,-1*\HtX) {$ $};
                         \node at (1*\Lrg,-1*\HtX) {$-$};
                         \node at (2*\Lrg,-1*\HtX) {$ $};
                         \node at (3*\Lrg,-1*\HtX) {$-$};
                         \node at (4*\Lrg,-1*\HtX) {$0$};
                         \node at (5*\Lrg,-1*\HtX) {$+$};
                         \node at (6*\Lrg,-1*\HtX) {$ $};
                         % Ligne b
                         \node at (-1.8*\Lrg,-2*\HtX) {$x+2$};
                         \node at (2*\Lrg,-2*\HtX) {$ $};
                         \node at (1*\Lrg,-2*\HtX) {$-$};
                         \node at (2*\Lrg,-2*\HtX) {$0$};
                         \node at (3*\Lrg,-2*\HtX) {$+$};
                         \node at (4*\Lrg,-2*\HtX) {$ $};
                         \node at (5*\Lrg,-2*\HtX) {$+$};
                         \node at (6*\Lrg,-2*\HtX) {$ $};
                         % Ligne c
                         \node at (-1.8*\Lrg,-3*\HtX) {$\dfrac{2x-5}{x+2}$};
                         \node at (0*\Lrg,-3*\HtX) {$ $};
                         \node at (1*\Lrg,-3*\HtX) {$+$};
                         \node at (2*\Lrg,-3*\HtX) {$ $};
                         \node at (3*\Lrg,-3*\HtX) {$-$};
                         \node at (4*\Lrg,-3*\HtX) {$0$};
                         \node at (5*\Lrg,-3*\HtX) {$+$};
                         \node at (6*\Lrg,-3*\HtX) {$ $};
                         % Encadrement
                         \draw[cadre] (\separateur,\haut) -- (\separateur, \lignec);
                         \draw[cadre] (\gauche,\haut) rectangle  (\droite, \lignec);
                         \draw[cadre] (\gauche,\lignex) -- (\droite,\lignex);
                         \draw[cadre] (\gauche,\lignea) -- (\droite,\lignea);
                         \draw[cadre] (\gauche,\ligneb) -- (\droite,\ligneb);
                    \end{tikzpicture}
               }
          \end{extern}
     \end{center}
     $\frac{2x-5}{x+2}$ est positif ou nul sur l'ensemble $\left]-\infty ;-2\right[ \cup \left[\frac{5}{2}; +\infty \right[$
}
\cadre{vert}{Propriété}{%id="p80"
     Soit $f$ une fonction définie sur $D$ de courbe représentative $\mathscr{C}_f$ et $m$ un nombre réel.
     \begin{itemize}
          \item Les solutions de l'inéquation $f(x)\leqslant m$ sont les \textbf{abscisses} des points de la courbe $\mathscr{C}_f$ situés \textbf{au dessous} de la droite horizontale d'équation $y=m$(On inclut les points d'intersection si l'inégalité est large, on les exclut si l'inégalité est stricte.)
          \item De même, les solutions de l'inéquation $f(x)\geqslant m$ sont les \textbf{abscisses} des points de la courbe $\mathscr{C}_f$ situés \textbf{au dessus} de droite horizontale d'équation $y=m$
     \end{itemize}
}
\bloc{orange}{Exemple}{%id="e80"
     \begin{center}
          \begin{extern}%width="320" alt="inéquation et graphique"
               \resizebox{7cm}{!}{
                    \psset{xunit=1.0cm,yunit=1.0cm,algebraic=true,dimen=middle,linewidth=0.6pt,arrowsize=3pt}
                    \begin{pspicture*}(-2.5,-2.5)(5.5,6.)
                         \psaxes[xAxis=true,yAxis=true,Dx=10.,Dy=10.]{->}(0,0)(-2.5,-2.5)(5.5,6.)
                         \psplot[linewidth=0.8pt,linecolor=mcvert,plotpoints=200]{-2.5}{5.5}{(x-1.0)^(2.0)-2.0}
                         \psplot[linewidth=1pt,linecolor=blue,plotpoints=200]{-1}{3}{(x-1.0)^(2.0)-2.0}
                         \psplot[linewidth=0.8pt,linecolor=red,plotpoints=200]{-2.5}{5.5}{2.0}
                         \psline[linewidth=0.8pt,linestyle=dashed,dash=1pt 1pt](-1.,2.)(-1.,0.)
                         \psline[linewidth=0.8pt,linestyle=dashed,dash=1pt 1pt](3.,2.)(3.,0.)
                         \rput[tl](4.6,2.4){\red{$y=m$}}
                         \rput[tl](3.97,5.8){\color{mcvert}{$\mathscr{C}_f$}}
                         \rput[t](-1,-0.3){\blue{$x_1$}}\rput[t](3,-0.3){\blue{$x_2$}}
                         \psdots[dotsize=3pt 0,dotstyle=*,linecolor=blue](-1.,2.)\psdots[dotsize=3pt 0,dotstyle=*,linecolor=blue](3.,2.)
                         \psdots[dotsize=3pt 0,dotstyle=*,linecolor=blue](-1.,0.)\psdots[dotsize=3pt 0,dotstyle=*,linecolor=blue](3.,0.)
                         \psline[linewidth=1pt,linecolor=blue](-1.,0)(3.,0.)
                         \rput[bl](-0.5,-0.5){$O$}
                    \end{pspicture*}
               }
          \end{extern}
     \end{center}
     \begin{center}Sur la figure ci-dessus, l'inéquation $f(x) \leqslant m$ a pour solution l'intervalle $\left[x_1;x_2\right]$\end{center}
}

\end{document}
