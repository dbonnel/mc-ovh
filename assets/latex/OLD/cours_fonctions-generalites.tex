\documentclass[a4paper]{article}

%================================================================================================================================
%
% Packages
%
%================================================================================================================================

\usepackage[T1]{fontenc} 	% pour caractères accentués
\usepackage[utf8]{inputenc}  % encodage utf8
\usepackage[french]{babel}	% langue : français
\usepackage{fourier}			% caractères plus lisibles
\usepackage[dvipsnames]{xcolor} % couleurs
\usepackage{fancyhdr}		% réglage header footer
\usepackage{needspace}		% empêcher sauts de page mal placés
\usepackage{graphicx}		% pour inclure des graphiques
\usepackage{enumitem,cprotect}		% personnalise les listes d'items (nécessaire pour ol, al ...)
\usepackage{hyperref}		% Liens hypertexte
\usepackage{pstricks,pst-all,pst-node,pstricks-add,pst-math,pst-plot,pst-tree,pst-eucl} % pstricks
\usepackage[a4paper,includeheadfoot,top=2cm,left=3cm, bottom=2cm,right=3cm]{geometry} % marges etc.
\usepackage{comment}			% commentaires multilignes
\usepackage{amsmath,environ} % maths (matrices, etc.)
\usepackage{amssymb,makeidx}
\usepackage{bm}				% bold maths
\usepackage{tabularx}		% tableaux
\usepackage{colortbl}		% tableaux en couleur
\usepackage{fontawesome}		% Fontawesome
\usepackage{environ}			% environment with command
\usepackage{fp}				% calculs pour ps-tricks
\usepackage{multido}			% pour ps tricks
\usepackage[np]{numprint}	% formattage nombre
\usepackage{tikz,tkz-tab} 			% package principal TikZ
\usepackage{pgfplots}   % axes
\usepackage{mathrsfs}    % cursives
\usepackage{calc}			% calcul taille boites
\usepackage[scaled=0.875]{helvet} % font sans serif
\usepackage{svg} % svg
\usepackage{scrextend} % local margin
\usepackage{scratch} %scratch
\usepackage{multicol} % colonnes
%\usepackage{infix-RPN,pst-func} % formule en notation polanaise inversée
\usepackage{listings}

%================================================================================================================================
%
% Réglages de base
%
%================================================================================================================================

\lstset{
language=Python,   % R code
literate=
{á}{{\'a}}1
{à}{{\`a}}1
{ã}{{\~a}}1
{é}{{\'e}}1
{è}{{\`e}}1
{ê}{{\^e}}1
{í}{{\'i}}1
{ó}{{\'o}}1
{õ}{{\~o}}1
{ú}{{\'u}}1
{ü}{{\"u}}1
{ç}{{\c{c}}}1
{~}{{ }}1
}


\definecolor{codegreen}{rgb}{0,0.6,0}
\definecolor{codegray}{rgb}{0.5,0.5,0.5}
\definecolor{codepurple}{rgb}{0.58,0,0.82}
\definecolor{backcolour}{rgb}{0.95,0.95,0.92}

\lstdefinestyle{mystyle}{
    backgroundcolor=\color{backcolour},   
    commentstyle=\color{codegreen},
    keywordstyle=\color{magenta},
    numberstyle=\tiny\color{codegray},
    stringstyle=\color{codepurple},
    basicstyle=\ttfamily\footnotesize,
    breakatwhitespace=false,         
    breaklines=true,                 
    captionpos=b,                    
    keepspaces=true,                 
    numbers=left,                    
xleftmargin=2em,
framexleftmargin=2em,            
    showspaces=false,                
    showstringspaces=false,
    showtabs=false,                  
    tabsize=2,
    upquote=true
}

\lstset{style=mystyle}


\lstset{style=mystyle}
\newcommand{\imgdir}{C:/laragon/www/newmc/assets/imgsvg/}
\newcommand{\imgsvgdir}{C:/laragon/www/newmc/assets/imgsvg/}

\definecolor{mcgris}{RGB}{220, 220, 220}% ancien~; pour compatibilité
\definecolor{mcbleu}{RGB}{52, 152, 219}
\definecolor{mcvert}{RGB}{125, 194, 70}
\definecolor{mcmauve}{RGB}{154, 0, 215}
\definecolor{mcorange}{RGB}{255, 96, 0}
\definecolor{mcturquoise}{RGB}{0, 153, 153}
\definecolor{mcrouge}{RGB}{255, 0, 0}
\definecolor{mclightvert}{RGB}{205, 234, 190}

\definecolor{gris}{RGB}{220, 220, 220}
\definecolor{bleu}{RGB}{52, 152, 219}
\definecolor{vert}{RGB}{125, 194, 70}
\definecolor{mauve}{RGB}{154, 0, 215}
\definecolor{orange}{RGB}{255, 96, 0}
\definecolor{turquoise}{RGB}{0, 153, 153}
\definecolor{rouge}{RGB}{255, 0, 0}
\definecolor{lightvert}{RGB}{205, 234, 190}
\setitemize[0]{label=\color{lightvert}  $\bullet$}

\pagestyle{fancy}
\renewcommand{\headrulewidth}{0.2pt}
\fancyhead[L]{maths-cours.fr}
\fancyhead[R]{\thepage}
\renewcommand{\footrulewidth}{0.2pt}
\fancyfoot[C]{}

\newcolumntype{C}{>{\centering\arraybackslash}X}
\newcolumntype{s}{>{\hsize=.35\hsize\arraybackslash}X}

\setlength{\parindent}{0pt}		 
\setlength{\parskip}{3mm}
\setlength{\headheight}{1cm}

\def\ebook{ebook}
\def\book{book}
\def\web{web}
\def\type{web}

\newcommand{\vect}[1]{\overrightarrow{\,\mathstrut#1\,}}

\def\Oij{$\left(\text{O}~;~\vect{\imath},~\vect{\jmath}\right)$}
\def\Oijk{$\left(\text{O}~;~\vect{\imath},~\vect{\jmath},~\vect{k}\right)$}
\def\Ouv{$\left(\text{O}~;~\vect{u},~\vect{v}\right)$}

\hypersetup{breaklinks=true, colorlinks = true, linkcolor = OliveGreen, urlcolor = OliveGreen, citecolor = OliveGreen, pdfauthor={Didier BONNEL - https://www.maths-cours.fr} } % supprime les bordures autour des liens

\renewcommand{\arg}[0]{\text{arg}}

\everymath{\displaystyle}

%================================================================================================================================
%
% Macros - Commandes
%
%================================================================================================================================

\newcommand\meta[2]{    			% Utilisé pour créer le post HTML.
	\def\titre{titre}
	\def\url{url}
	\def\arg{#1}
	\ifx\titre\arg
		\newcommand\maintitle{#2}
		\fancyhead[L]{#2}
		{\Large\sffamily \MakeUppercase{#2}}
		\vspace{1mm}\textcolor{mcvert}{\hrule}
	\fi 
	\ifx\url\arg
		\fancyfoot[L]{\href{https://www.maths-cours.fr#2}{\black \footnotesize{https://www.maths-cours.fr#2}}}
	\fi 
}


\newcommand\TitreC[1]{    		% Titre centré
     \needspace{3\baselineskip}
     \begin{center}\textbf{#1}\end{center}
}

\newcommand\newpar{    		% paragraphe
     \par
}

\newcommand\nosp {    		% commande vide (pas d'espace)
}
\newcommand{\id}[1]{} %ignore

\newcommand\boite[2]{				% Boite simple sans titre
	\vspace{5mm}
	\setlength{\fboxrule}{0.2mm}
	\setlength{\fboxsep}{5mm}	
	\fcolorbox{#1}{#1!3}{\makebox[\linewidth-2\fboxrule-2\fboxsep]{
  		\begin{minipage}[t]{\linewidth-2\fboxrule-4\fboxsep}\setlength{\parskip}{3mm}
  			 #2
  		\end{minipage}
	}}
	\vspace{5mm}
}

\newcommand\CBox[4]{				% Boites
	\vspace{5mm}
	\setlength{\fboxrule}{0.2mm}
	\setlength{\fboxsep}{5mm}
	
	\fcolorbox{#1}{#1!3}{\makebox[\linewidth-2\fboxrule-2\fboxsep]{
		\begin{minipage}[t]{1cm}\setlength{\parskip}{3mm}
	  		\textcolor{#1}{\LARGE{#2}}    
 	 	\end{minipage}  
  		\begin{minipage}[t]{\linewidth-2\fboxrule-4\fboxsep}\setlength{\parskip}{3mm}
			\raisebox{1.2mm}{\normalsize\sffamily{\textcolor{#1}{#3}}}						
  			 #4
  		\end{minipage}
	}}
	\vspace{5mm}
}

\newcommand\cadre[3]{				% Boites convertible html
	\par
	\vspace{2mm}
	\setlength{\fboxrule}{0.1mm}
	\setlength{\fboxsep}{5mm}
	\fcolorbox{#1}{white}{\makebox[\linewidth-2\fboxrule-2\fboxsep]{
  		\begin{minipage}[t]{\linewidth-2\fboxrule-4\fboxsep}\setlength{\parskip}{3mm}
			\raisebox{-2.5mm}{\sffamily \small{\textcolor{#1}{\MakeUppercase{#2}}}}		
			\par		
  			 #3
 	 		\end{minipage}
	}}
		\vspace{2mm}
	\par
}

\newcommand\bloc[3]{				% Boites convertible html sans bordure
     \needspace{2\baselineskip}
     {\sffamily \small{\textcolor{#1}{\MakeUppercase{#2}}}}    
		\par		
  			 #3
		\par
}

\newcommand\CHelp[1]{
     \CBox{Plum}{\faInfoCircle}{À RETENIR}{#1}
}

\newcommand\CUp[1]{
     \CBox{NavyBlue}{\faThumbsOUp}{EN PRATIQUE}{#1}
}

\newcommand\CInfo[1]{
     \CBox{Sepia}{\faArrowCircleRight}{REMARQUE}{#1}
}

\newcommand\CRedac[1]{
     \CBox{PineGreen}{\faEdit}{BIEN R\'EDIGER}{#1}
}

\newcommand\CError[1]{
     \CBox{Red}{\faExclamationTriangle}{ATTENTION}{#1}
}

\newcommand\TitreExo[2]{
\needspace{4\baselineskip}
 {\sffamily\large EXERCICE #1\ (\emph{#2 points})}
\vspace{5mm}
}

\newcommand\img[2]{
          \includegraphics[width=#2\paperwidth]{\imgdir#1}
}

\newcommand\imgsvg[2]{
       \begin{center}   \includegraphics[width=#2\paperwidth]{\imgsvgdir#1} \end{center}
}


\newcommand\Lien[2]{
     \href{#1}{#2 \tiny \faExternalLink}
}
\newcommand\mcLien[2]{
     \href{https~://www.maths-cours.fr/#1}{#2 \tiny \faExternalLink}
}

\newcommand{\euro}{\eurologo{}}

%================================================================================================================================
%
% Macros - Environement
%
%================================================================================================================================

\newenvironment{tex}{ %
}
{%
}

\newenvironment{indente}{ %
	\setlength\parindent{10mm}
}

{
	\setlength\parindent{0mm}
}

\newenvironment{corrige}{%
     \needspace{3\baselineskip}
     \medskip
     \textbf{\textsc{Corrigé}}
     \medskip
}
{
}

\newenvironment{extern}{%
     \begin{center}
     }
     {
     \end{center}
}

\NewEnviron{code}{%
	\par
     \boite{gray}{\texttt{%
     \BODY
     }}
     \par
}

\newenvironment{vbloc}{% boite sans cadre empeche saut de page
     \begin{minipage}[t]{\linewidth}
     }
     {
     \end{minipage}
}
\NewEnviron{h2}{%
    \needspace{3\baselineskip}
    \vspace{0.6cm}
	\noindent \MakeUppercase{\sffamily \large \BODY}
	\vspace{1mm}\textcolor{mcgris}{\hrule}\vspace{0.4cm}
	\par
}{}

\NewEnviron{h3}{%
    \needspace{3\baselineskip}
	\vspace{5mm}
	\textsc{\BODY}
	\par
}

\NewEnviron{margeneg}{ %
\begin{addmargin}[-1cm]{0cm}
\BODY
\end{addmargin}
}

\NewEnviron{html}{%
}

\begin{document}
\begin{h2}1. Notion de fonction\end{h2}
\cadre{bleu}{Définition}{% id="d10"
     Une \textbf{fonction} $f$ est un procédé qui à tout nombre réel $x$ d'une partie $D$ de $\mathbb{R}$ associe \textbf{ un seul }nombre réel $y$.
     \begin{itemize}
          \item $x$ s'appelle la \textbf{variable}.
          \item $y$ s'appelle l'\textbf{image} de $x$ par la fonction $f$ et se note $f\left(x\right)$
          \item $f$ est la \textbf{fonction} et se note: $f : x \mapsto  y=f\left(x\right)$.
     \end{itemize}
}
\bloc{cyan}{Remarque}{% id="r10"
     Les procédés permettant d'associer un nombre à un autre nombre peuvent être :
     \begin{itemize}
          \item des formules mathématiques (par exemple : $f\left(x\right)=\frac{1}{1+x^2}$)
          \item une courbe (par exemple : la courbe donnant le cours d'une action en Bourse en fonction du temps)
          \item un instrument de mesure ou de conversion (par exemple : le compteur d'un taxi qui donne le prix à payer en fonction du trajet parcouru)
          \item un tableau de valeurs, chaque élément de la seconde ligne étant associé à un élément de la première ligne
          \item une touche de calculatrice (par exemple: \textit{sin, cos, ln, log}, etc.) qui affiche un résultat dépendant du nombre saisi auparavant
          \item etc.
     \end{itemize}
}
\cadre{rouge}{Méthode (Calcul d'une image)}{% id="t30"
     Pour calculer l'image d'un nombre par une fonction définie par une formule on remplace $x$ par ce nombre dans l'expression de $f\left(x\right)$
}
\bloc{orange}{Exemple}{% id="e30"
     Soit la fonction $f$ définie par $f\left(x\right)=\frac{x^2+3}{x+1}$
     \begin{itemize}
          \item Pour calculer l'image de $1$ - notée $f\left(1\right)$ - on remplace $x$ par $1$ dans la formule donnant $f\left(x\right)$. On obtient alors :
          \par
          $f\left(1\right)=\frac{1^2+3}{1+1}=\frac{4}{2}=2$
          \item Pour calculer  l'image de $-2$, on remplace $x$ par $\left(-2\right)$ dans cette même formule. Pensez bien à ajouter une parenthèse lorsque $x$ est négatif ou lorsqu'il s'agit d'une expression fractionnaire. On obtient  :
          \par
          $f\left(-2\right)=\frac{\left(-2\right)^2+3}{\left(-2\right)+1}=\frac{7}{-1}=-7$
     \end{itemize}
}
\cadre{bleu}{Définition}{% id="d40"
     L'ensemble $\mathscr D$ des éléments $x$ de $\mathbb{R}$ qui possèdent une image par $f$ s'appelle l'\textbf{ensemble de définition} de $f$.
     \par
     On dit également que $f$ est \textbf{définie} sur $\mathscr D$
}
\bloc{cyan}{Remarque}{% id="r40"
     Certaines fonctions sont définies sur $\mathbb{R}$ en entier. Parfois, cependant, l'ensemble de définition est plus petit. C'est en particulier le cas:
     \begin{itemize}
          \item s'il est impossible de calculer $f\left(x\right)$ pour certaines valeurs de $x$ (par exemple la fonction $f : x \mapsto  \frac{1}{x}$ n'est pas définie pour $x=0$ car il est impossible de diviser par zéro
          \item si la fonction n'a aucune signification pour certaines valeurs de $x$; par exemple la fonction donnant l'aire d'un carré en fonction de la longueur $x$ de ses côtés n'a pas de sens pour $x$ négatif.
     \end{itemize}
}
\cadre{bleu}{Définition}{% id="d50"
     Soit $y$ un nombre réel. Les \textbf{antécédents} de $y$ par $f$ sont les nombres réels $x$ appartenant à $\mathscr D$ tels que $f\left(x\right)=y$. Un nombre peut avoir aucun, un ou plusieurs antécédent(s).
}
\cadre{rouge}{Méthode (Calcul des antécédents)}{% id="t60"
     Pour déterminer les antécédents d'un nombre $y$, on résout l'équation  $f\left(x\right)=y$ d'inconnue $x$.
}
\bloc{orange}{Exemple}{% id="e60"
     Soit la fonction $f$ définie par $f\left(x\right)=\frac{x+5}{x+1}$
     \par
     Pour déterminer le ou les antécédents du nombre $2$ on résout l'équation $f\left(x\right)=2$ c'est à dire :
     \begin{center}$\frac{x+5}{x+1}=2$\end{center}
     On obtient alors :
     \par
     $x+5=2\left(x+1\right)$ (« produit en croix »)
     \par
     $x+5=2x+2$
     \par
     $x-2x=2-5$
     \par
     $-x=-3$
     \par
     $x=3$
     \par
     Le nombre $2$ possède un unique antécédent qui est $x=3$.
}
\begin{h2}2. Représentation graphique\end{h2}
Dans cette section, on munit le plan $\mathscr P$ d'un repère orthogonal $\left(O, i, j\right)$
\cadre{bleu}{Définition}{% id="d80"
     Soit $f$ une fonction définie sur un ensemble $\mathscr D$.
     \par
     La\textbf{ représentation graphique} de $f$ est la courbe  $\mathscr C_f $  formée des points $M\left(x;y\right)$ où $x\in \mathscr D$ et $y=f\left(x\right)$
     \par
     On dit aussi que la courbe $\mathscr C_f $ \textbf{a pour équation} $y=f\left(x\right)$.
}
\bloc{orange}{Exemple}{% id="e80"
     \begin{center}
          \begin{extern} %width="300" alt=" représentation graphique d'une fonction"
               \resizebox{8cm}{!}{%
                    % -+-+-+ variables modifiables
                    \def\fonction{x^3-x+1}
                    \def\xmin{-1.2}
                    \def\xmax{1.2}
                    \def\ymin{-0.2}
                    \def\ymax{1.7}
                    \def\xunit{4}  % unités en cm
                    \def\yunit{4}
                    \psset{xunit=\xunit,yunit=\yunit,algebraic=true}
                    \fontsize{15pt}{15pt}\selectfont
                    \begin{pspicture*}[linewidth=1pt](\xmin,\ymin)(\xmax,\ymax)
                         %    \psgrid[gridcolor=lightgray, subgriddiv=1, gridlabels=0pt](-3,-1.8)(7,4)
                         \psaxes[linewidth=0.75pt]{->}(0,0)(\xmin,\ymin)(\xmax,\ymax)
                         \psplot[plotpoints=2000,linecolor=blue]{-1}{1}{\fonction}
                         \rput[tr](-0.07,-0.07){$O$}
                         \rput[l](-0.88,1.05){$\color{blue} \mathcal{C}_f$}
                    \end{pspicture*}
               }
          \end{extern}
     \end{center}
     \begin{center}\textit{Exemple de représentation graphique d'une fonction définie sur [-1;1]}\end{center}
}
\bloc{cyan}{Remarque}{% id="r80"
     Du fait qu'un nombre ne peut pas avoir plusieurs images, la courbe représentative d'une fonction \textbf{ne peut pas contenir plusieurs points situés sur la même "verticale"} (droite parallèle à l'axe des ordonnées).
     \par
     Par contre, il peut très bien y avoir plusieurs points situés sur une même horizontale comme dans l'exemple ci-dessus.
}
\bloc{orange}{Lecture graphique de l'image d'un nombre}{% id="e82"
     \begin{center}
          \begin{extern} %width="300" alt="Lecture graphique d'une image"
               \resizebox{8cm}{!}{%
                    % -+-+-+ variables modifiables
                    \def\fonction{x^3-x+1}
                    \def\xmin{-1.2}
                    \def\xmax{1.2}
                    \def\ymin{-0.2}
                    \def\ymax{1.7}
                    \def\xunit{4}  % unités en cm
                    \def\yunit{4}
                    \psset{xunit=\xunit,yunit=\yunit,algebraic=true}
                    \fontsize{15pt}{15pt}\selectfont
                    \begin{pspicture*}[linewidth=1pt](\xmin,\ymin)(\xmax,\ymax)
                         %    \psgrid[gridcolor=lightgray, subgriddiv=1, gridlabels=0pt](-3,-1.8)(7,4)
                         \psaxes[linewidth=0.75pt]{->}(0,0)(\xmin,\ymin)(\xmax,\ymax)
                         \psplot[plotpoints=2000,linecolor=blue]{-1}{1}{\fonction}
                         \rput[tr](-0.07,-0.07){$O$}
                         \rput[l](-0.88,1.05){$\color{blue} \mathcal{C}_f$}
                         \rput(0.58,0.72){$\color{blue} M$}
                         \rput[t](0.5,-0.07){$\red 0,5$}\rput[r](-0.05,0.62){$\red f(0,5)\approx0,6$}
                         \psline[linecolor=red,linewidth=1pt](0.5,0)(0.5,0.625)(0,0.625)
                         \psline[linecolor=red,linewidth=.9pt,arrowsize=6pt]{->}(0.5,0)(0.5,0.33)
                         \psline[linecolor=red,linewidth=.9pt,arrowsize=6pt]{->}(0.5,0.625)(0.23,0.625)
                    \end{pspicture*}
               }
          \end{extern}
     \end{center}
     Pour déterminer graphiquement l'\textbf{image} de $0,5$ par la fonction $f $:
     \begin{itemize}
          \item on place le point de d'\textbf{abscisse $0,5$} sur l'axe des abscisses
          \item on le relie au point $M$ de la courbe qui a la même abscisse
          \item l'\textbf{ordonnée} du point $M$ nous donne la valeur de $f\left(0,5\right)$; on trouve ici environ $0,6$.
     \end{itemize}
}
\bloc{orange}{Lecture graphique des antécédents d'un nombre}{% id="e84"
     \begin{center}
          \begin{extern} %width="300" alt="Lecture graphique d'antécédents "
               \resizebox{8cm}{!}{%
                    % -+-+-+ variables modifiables
                    \def\fonction{x^3-x+1}
                    \def\xmin{-1.2}
                    \def\xmax{1.2}
                    \def\ymin{-0.2}
                    \def\ymax{1.7}
                    \def\xunit{4}  % unités en cm
                    \def\yunit{4}
                    \psset{xunit=\xunit,yunit=\yunit,algebraic=true}
                    \fontsize{15pt}{15pt}\selectfont
                    \begin{pspicture*}[linewidth=1pt](\xmin,\ymin)(\xmax,\ymax)
                         %    \psgrid[gridcolor=lightgray, subgriddiv=1, gridlabels=0pt](-3,-1.8)(7,4)
                         \psaxes[linewidth=0.75pt]{->}(0,0)(\xmin,\ymin)(\xmax,\ymax)
                         \psplot[plotpoints=2000,linecolor=blue]{-1}{1}{\fonction}
                         \rput[tr](-0.07,-0.07){$O$}
                         \rput[l](-0.88,1.05){$\color{blue} \mathcal{C}_f$}
                         \rput[t](0.13,-0.08){$\red 0,1$} \rput[t](0.82,-0.08){$\red 0,95$} \rput[r](-0.05,0.8){$\red 0,9$}
                         \psline[linecolor=red,linewidth=1pt](0.1,0.9)(0.1,0.)
                         \psline[linecolor=red,linewidth=1pt](0.95,0.9)(0.95,0.)
                         \psline[linecolor=red,linewidth=1pt](-1.5,0.9)(1.5,0.9)
                         \psline[linecolor=red,linewidth=.9pt,arrowsize=6pt]{->}(0.1,0.9)(0.1,0.4)
                         \psline[linecolor=red,linewidth=.9pt,arrowsize=6pt]{->}(0.95,0.9)(0.95,0.4)
                    \end{pspicture*}
               }
          \end{extern}
     \end{center}
     Pour déterminer graphiquement les \textbf{antécédents} de $0,9$ par la fonction $f $:
     \begin{itemize}
          \item on place le point de d'\textbf{ordonnée} $0,9$ sur l'axe des ordonnées
          \item on trace la droite horizontale (d'équation $y=0,9$) qui passe par ce point
          \item on trace le(s) \textbf{point(s) d'intersection} de cette droite avec la courbe. Dans cet exemple on en trouve deux ; dans d'autres exemples on pourrait en trouver zéro, un, deux ou plus...
          \item les \textbf{abscisses} de ces points d'intersection nous donne les antécédents de $0,9$; on trouve ici deux antécédents qui valent environ $0,1$ et $0,95$.
     \end{itemize}
}
\begin{h2}3. Variations d'une fonction\end{h2}
\cadre{bleu}{Définition}{% id="d100"
     La fonction $f$ est \textbf{croissante} sur l'intervalle $I$ si pour tous réels $x_1$ et $x_2$  appartenant à $I$ tels que $x_1\leqslant  x_2$ on a $f\left(x_1\right)\leqslant f\left(x_2\right)$.
}
\bloc{cyan}{Remarque}{% id="r100"
     Intuitivement, cela se traduit par le fait que la courbe représentative de la fonction $f$ "monte" lorsqu'on la parcourt dans le sens de l'axe des abscisses (e.g. de gauche à droite)
}
\begin{center}
     \begin{extern}%width="250" alt="fonction croissante"
          \resizebox{6cm}{!}{%
               % -+-+-+ variables modifiables
               \def\fonction{1+0.2*x*x }
               \def\xmin{-1.2}
               \def\xmax{5}
               \def\ymin{-0.9}
               \def\ymax{5}
               \def\xunit{1}  % unités en cm
               \def\yunit{1}
               \psset{xunit=\xunit,yunit=\yunit,algebraic=true}
               \fontsize{12pt}{12pt}\selectfont
               \begin{pspicture*}[linewidth=1pt](\xmin,\ymin)(\xmax,\ymax)
                    %      \psgrid[gridcolor=mcgris, subgriddiv=5, gridlabels=0pt](\xmin,\ymin)(\xmax,\ymax)
                    \psaxes[linewidth=0.75pt,Dx=10,Dy=10]{->}(0,0)(\xmin,\ymin)(\xmax,\ymax)
                    \psplot[plotpoints=2000,linecolor=red]{0.2}{\xmax}{\fonction}
                    \psline[linewidth=0.75pt,linecolor=lightgray](1,0)(1,1.2)(0,1.2)
                    \psline[linewidth=0.75pt,linecolor=lightgray](4,0)(4,4.2)(0,4.2)
                    \rput[tr](-0.3,-0.3){$O$} \rput[t](1,-0.3){$x_1$} \rput[t](4,-0.3){$x_2$}
                    \rput[r](-0.1,1.2){$f(x_1)$} \rput[r](-0.1,4.2){$f(x_2)$}
                    \rput[tl](4.3,4.5){$\color{red} \mathcal{C}_f$}
               \end{pspicture*}
          }
     \end{extern}
\end{center}
\cadre{bleu}{Définition}{% id="d110"
     La fonction $f$ est \textbf{décroissante} sur l'intervalle $I$ si pour tous réels $x_1$ et $x_2$  appartenant à $I$ tels que $x_1 \leqslant x_2$ on a $f\left(x_1\right) \geqslant f\left(x_2\right)$.
}
\bloc{cyan}{Remarque}{% id="r110"
     Intuitivement, cela se traduit par le fait que la courbe représentative de la fonction $f$ "descend" lorsqu'on la parcourt dans le sens de l'axe des abscisses (e.g. de gauche à droite)
}
\begin{center}
     \begin{extern}%width="250" alt="fonction décroissante"
          \resizebox{6cm}{!}{%
               % -+-+-+ variables modifiables
               \def\fonction{4-0.1*x*x }
               \def\xmin{-1.2}
               \def\xmax{5}
               \def\ymin{-0.9}
               \def\ymax{5}
               \def\xunit{1}  % unités en cm
               \def\yunit{1}
               \psset{xunit=\xunit,yunit=\yunit,algebraic=true}
               \fontsize{12pt}{12pt}\selectfont
               \begin{pspicture*}[linewidth=1pt](\xmin,\ymin)(\xmax,\ymax)
                    %      \psgrid[gridcolor=mcgris, subgriddiv=5, gridlabels=0pt](\xmin,\ymin)(\xmax,\ymax)
                    \psaxes[linewidth=0.75pt,Dx=10,Dy=10]{->}(0,0)(\xmin,\ymin)(\xmax,\ymax)
                    \psplot[plotpoints=2000,linecolor=red]{0.2}{\xmax}{\fonction}
                    \psline[linewidth=0.75pt,linecolor=lightgray](1,0)(1,3.9)(0,3.9)
                    \psline[linewidth=0.75pt,linecolor=lightgray](4,0)(4,2.4)(0,2.4)
                    \rput[tr](-0.3,-0.3){$O$} \rput[t](1,-0.3){$x_1$} \rput[t](4,-0.3){$x_2$}
                    \rput[r](-0.1,3.9){$f(x_1)$} \rput[r](-0.1,2.4){$f(x_2)$}
                    \rput[tl](4.5,2.5){$\color{red} \mathcal{C}_f$}
               \end{pspicture*}
          }
     \end{extern}
\end{center}
\cadre{bleu}{Définition}{% id="d120"
     Soit $I$ un intervalle et $x_0 \in  I$.
     \par
     La fonction $f$ admet un \textbf{maximum} en $x_0$ sur l'intervalle $I$ si pour tout réel $x$ de I, $f\left(x\right)\leqslant f\left(x_0\right)$. Le maximum de la fonction $f$ sur $I$ est alors $M=f\left(x_0\right)$
}
\begin{center}
     \begin{extern}%width="250" alt="maximum d'une fonction"
          \resizebox{6cm}{!}{%
               % -+-+-+ variables modifiables
               \def\fonction{(x+1)*(5-x)*4/9}
               \def\xmin{-1.2}
               \def\xmax{5}
               \def\ymin{-0.9}
               \def\ymax{5}
               \def\xunit{1}  % unités en cm
               \def\yunit{1}
               \psset{xunit=\xunit,yunit=\yunit,algebraic=true}
               \fontsize{12pt}{12pt}\selectfont
               \begin{pspicture*}[linewidth=1pt](\xmin,\ymin)(\xmax,\ymax)
                    %      \psgrid[gridcolor=mcgris, subgriddiv=5, gridlabels=0pt](\xmin,\ymin)(\xmax,\ymax)
                    \psaxes[linewidth=0.75pt,Dx=10,Dy=10]{->}(0,0)(\xmin,\ymin)(\xmax,\ymax)
                    \psplot[plotpoints=2000,linecolor=red]{0.2}{4}{\fonction}
                    \psline[linewidth=0.75pt,linecolor=lightgray](2,0)(2,4)(0,4)
                    \rput[tr](-0.3,-0.3){$O$} \rput[t](2,-0.3){$x_0$}
                    \rput[r](-0.1,4){$f(x_0)$}
                    \rput[tl](4.,3){$\color{red} \mathcal{C}_f$}
               \end{pspicture*}
          }
     \end{extern}
\end{center}
\cadre{bleu}{Définition}{% id="d130"
     Soit $I$ un intervalle et $x_0 \in  I$.
     \par
     La fonction $f$ admet un \textbf{minimum} en $x_0$ sur l'intervalle $I$ si pour tout réel $x$ de I, $f\left(x\right)\geqslant f\left(x_0\right)$. Le minimum de la fonction $f$ sur $I$ est alors $m=f\left(x_0\right)$
}
\begin{center}
     \begin{extern}%width="250" alt="maximum d'une fonction"
          \resizebox{6cm}{!}{%
               % -+-+-+ variables modifiables
               \def\fonction{(x+1)*(x-5)*4/9+6}
               \def\xmin{-1.2}
               \def\xmax{5}
               \def\ymin{-0.9}
               \def\ymax{5}
               \def\xunit{1}  % unités en cm
               \def\yunit{1}
               \psset{xunit=\xunit,yunit=\yunit,algebraic=true}
               \fontsize{12pt}{12pt}\selectfont
               \begin{pspicture*}[linewidth=1pt](\xmin,\ymin)(\xmax,\ymax)
                    %      \psgrid[gridcolor=mcgris, subgriddiv=5, gridlabels=0pt](\xmin,\ymin)(\xmax,\ymax)
                    \psaxes[linewidth=0.75pt,Dx=10,Dy=10]{->}(0,0)(\xmin,\ymin)(\xmax,\ymax)
                    \psplot[plotpoints=2000,linecolor=red]{0.2}{4}{\fonction}
                    \psline[linewidth=0.75pt,linecolor=lightgray](2,0)(2,2)(0,2)
                    \rput[tr](-0.3,-0.3){$O$} \rput[t](2,-0.3){$x_0$}
                    \rput[r](-0.1,2){$f(x_0)$}
                    \rput[tl](4.,3.5){$\color{red} \mathcal{C}_f$}
               \end{pspicture*}
          }
     \end{extern}
\end{center}
\bloc{cyan}{Remarques}{% id="r130"
     \begin{itemize}
          \item Un \textbf{extremum} est un maximum ou un minimum
          \item \textbf{Attention à la rédaction :}
          Lorsqu'on dit que $f$ admet un maximum (\textit{resp.} minimum) \textbf{en} $x_0$ (ou \textbf{pour $x=x_0$}), $x_0$ correspond à la valeur de la \textbf{variable $x$} et non à la valeur du  maximum (\textit{resp.} minimum).
          \par
          Par exemple, dans le tableau de l'exemple ci-dessous, $f$ admet un maximum \textbf{en $0$}. Ce maximum \textbf{est égal à 6} (\textit{Ne pas écrire que le maximum est $0$ !}).
          \item Les variations d'une fonction peuvent être représentées par un \textbf{tableau de variations}
     \end{itemize}
}
\bloc{orange}{Exemple}{% id="e130"
     Soit $f$ une fonction définie sur $\left[-2;5\right]$, croissante sur $\left[-2;0\right]$ et décroissante sur $\left[0; 5\right]$ avec $f\left(-2\right)=-3$, $f\left(0\right)=6$ et $f\left(5\right)=1$
     \par
     Le tableau de variations de la fonction $f$ est :
     %:-+-+-+-+- Engendré par : http://math.et.info.free.fr/TikZ/TableauxVariations/
     \begin{center}
          \begin{extern}%width="350" alt=""
               \begin{tikzpicture}[scale=0.875]
                    % Styles
                    \tikzstyle{cadre}=[thin]
                    \tikzstyle{fleche}=[->,>=latex,thin]
                    \tikzstyle{nondefini}=[lightgray]
                    % Dimensions Modifiables
                    \def\Lrg{1.5}
                    \def\HtX{1}
                    \def\HtY{0.5}
                    % Dimensions Calculées
                    \def\lignex{-0.5*\HtX}
                    \def\lignef{-1.5*\HtX}
                    \def\separateur{-0.5*\Lrg}
                    % Largeur du tableau
                    \def\gauche{-1.5*\Lrg}
                    \def\droite{4.5*\Lrg}
                    % Hauteur du tableau
                    \def\haut{0.5*\HtX}
                    \def\bas{-1.5*\HtX-2*\HtY}
                    % Pointillés
                    \draw[lightgray] (2*\Lrg,\lignex) -- (2*\Lrg,\bas);
                    % Ligne de l'abscisse : x
                    \node at (-1*\Lrg,0) {$x$};
                    \node at (0*\Lrg,0) {$-2$};
                    \node at (2*\Lrg,0) {$0$};
                    \node at (4*\Lrg,0) {$5$};
                    % Ligne de la fonction : f(x)
                    \node  at (-1*\Lrg,{-1*\HtX+(-1)*\HtY}) {$f(x)$};
                    \node (f1) at (0*\Lrg,{-1*\HtX+(-2)*\HtY}) {$-3$};
                    \node (f2) at (2*\Lrg,{-1*\HtX+(0)*\HtY}) {$6$};
                    \node (f3) at (4*\Lrg,{-1*\HtX+(-2)*\HtY}) {$1$};
                    % Flèches
                    \draw[fleche] (f1) -- (f2);
                    \draw[fleche] (f2) -- (f3);
                    % Encadrement
                    \draw[cadre] (\separateur,\haut) -- (\separateur,\bas);
                    \draw[cadre] (\gauche,\haut) rectangle  (\droite,\bas);
                    \draw[cadre] (\gauche,\lignex) -- (\droite,\lignex);
               \end{tikzpicture}
          \end{extern}
     \end{center}
}

\end{document}