\documentclass[a4paper]{article}

%================================================================================================================================
%
% Packages
%
%================================================================================================================================

\usepackage[T1]{fontenc} 	% pour caractères accentués
\usepackage[utf8]{inputenc}  % encodage utf8
\usepackage[french]{babel}	% langue : français
\usepackage{fourier}			% caractères plus lisibles
\usepackage[dvipsnames]{xcolor} % couleurs
\usepackage{fancyhdr}		% réglage header footer
\usepackage{needspace}		% empêcher sauts de page mal placés
\usepackage{graphicx}		% pour inclure des graphiques
\usepackage{enumitem,cprotect}		% personnalise les listes d'items (nécessaire pour ol, al ...)
\usepackage{hyperref}		% Liens hypertexte
\usepackage{pstricks,pst-all,pst-node,pstricks-add,pst-math,pst-plot,pst-tree,pst-eucl} % pstricks
\usepackage[a4paper,includeheadfoot,top=2cm,left=3cm, bottom=2cm,right=3cm]{geometry} % marges etc.
\usepackage{comment}			% commentaires multilignes
\usepackage{amsmath,environ} % maths (matrices, etc.)
\usepackage{amssymb,makeidx}
\usepackage{bm}				% bold maths
\usepackage{tabularx}		% tableaux
\usepackage{colortbl}		% tableaux en couleur
\usepackage{fontawesome}		% Fontawesome
\usepackage{environ}			% environment with command
\usepackage{fp}				% calculs pour ps-tricks
\usepackage{multido}			% pour ps tricks
\usepackage[np]{numprint}	% formattage nombre
\usepackage{tikz,tkz-tab} 			% package principal TikZ
\usepackage{pgfplots}   % axes
\usepackage{mathrsfs}    % cursives
\usepackage{calc}			% calcul taille boites
\usepackage[scaled=0.875]{helvet} % font sans serif
\usepackage{svg} % svg
\usepackage{scrextend} % local margin
\usepackage{scratch} %scratch
\usepackage{multicol} % colonnes
%\usepackage{infix-RPN,pst-func} % formule en notation polanaise inversée
\usepackage{listings}

%================================================================================================================================
%
% Réglages de base
%
%================================================================================================================================

\lstset{
language=Python,   % R code
literate=
{á}{{\'a}}1
{à}{{\`a}}1
{ã}{{\~a}}1
{é}{{\'e}}1
{è}{{\`e}}1
{ê}{{\^e}}1
{í}{{\'i}}1
{ó}{{\'o}}1
{õ}{{\~o}}1
{ú}{{\'u}}1
{ü}{{\"u}}1
{ç}{{\c{c}}}1
{~}{{ }}1
}


\definecolor{codegreen}{rgb}{0,0.6,0}
\definecolor{codegray}{rgb}{0.5,0.5,0.5}
\definecolor{codepurple}{rgb}{0.58,0,0.82}
\definecolor{backcolour}{rgb}{0.95,0.95,0.92}

\lstdefinestyle{mystyle}{
    backgroundcolor=\color{backcolour},   
    commentstyle=\color{codegreen},
    keywordstyle=\color{magenta},
    numberstyle=\tiny\color{codegray},
    stringstyle=\color{codepurple},
    basicstyle=\ttfamily\footnotesize,
    breakatwhitespace=false,         
    breaklines=true,                 
    captionpos=b,                    
    keepspaces=true,                 
    numbers=left,                    
xleftmargin=2em,
framexleftmargin=2em,            
    showspaces=false,                
    showstringspaces=false,
    showtabs=false,                  
    tabsize=2,
    upquote=true
}

\lstset{style=mystyle}


\lstset{style=mystyle}
\newcommand{\imgdir}{C:/laragon/www/newmc/assets/imgsvg/}
\newcommand{\imgsvgdir}{C:/laragon/www/newmc/assets/imgsvg/}

\definecolor{mcgris}{RGB}{220, 220, 220}% ancien~; pour compatibilité
\definecolor{mcbleu}{RGB}{52, 152, 219}
\definecolor{mcvert}{RGB}{125, 194, 70}
\definecolor{mcmauve}{RGB}{154, 0, 215}
\definecolor{mcorange}{RGB}{255, 96, 0}
\definecolor{mcturquoise}{RGB}{0, 153, 153}
\definecolor{mcrouge}{RGB}{255, 0, 0}
\definecolor{mclightvert}{RGB}{205, 234, 190}

\definecolor{gris}{RGB}{220, 220, 220}
\definecolor{bleu}{RGB}{52, 152, 219}
\definecolor{vert}{RGB}{125, 194, 70}
\definecolor{mauve}{RGB}{154, 0, 215}
\definecolor{orange}{RGB}{255, 96, 0}
\definecolor{turquoise}{RGB}{0, 153, 153}
\definecolor{rouge}{RGB}{255, 0, 0}
\definecolor{lightvert}{RGB}{205, 234, 190}
\setitemize[0]{label=\color{lightvert}  $\bullet$}

\pagestyle{fancy}
\renewcommand{\headrulewidth}{0.2pt}
\fancyhead[L]{maths-cours.fr}
\fancyhead[R]{\thepage}
\renewcommand{\footrulewidth}{0.2pt}
\fancyfoot[C]{}

\newcolumntype{C}{>{\centering\arraybackslash}X}
\newcolumntype{s}{>{\hsize=.35\hsize\arraybackslash}X}

\setlength{\parindent}{0pt}		 
\setlength{\parskip}{3mm}
\setlength{\headheight}{1cm}

\def\ebook{ebook}
\def\book{book}
\def\web{web}
\def\type{web}

\newcommand{\vect}[1]{\overrightarrow{\,\mathstrut#1\,}}

\def\Oij{$\left(\text{O}~;~\vect{\imath},~\vect{\jmath}\right)$}
\def\Oijk{$\left(\text{O}~;~\vect{\imath},~\vect{\jmath},~\vect{k}\right)$}
\def\Ouv{$\left(\text{O}~;~\vect{u},~\vect{v}\right)$}

\hypersetup{breaklinks=true, colorlinks = true, linkcolor = OliveGreen, urlcolor = OliveGreen, citecolor = OliveGreen, pdfauthor={Didier BONNEL - https://www.maths-cours.fr} } % supprime les bordures autour des liens

\renewcommand{\arg}[0]{\text{arg}}

\everymath{\displaystyle}

%================================================================================================================================
%
% Macros - Commandes
%
%================================================================================================================================

\newcommand\meta[2]{    			% Utilisé pour créer le post HTML.
	\def\titre{titre}
	\def\url{url}
	\def\arg{#1}
	\ifx\titre\arg
		\newcommand\maintitle{#2}
		\fancyhead[L]{#2}
		{\Large\sffamily \MakeUppercase{#2}}
		\vspace{1mm}\textcolor{mcvert}{\hrule}
	\fi 
	\ifx\url\arg
		\fancyfoot[L]{\href{https://www.maths-cours.fr#2}{\black \footnotesize{https://www.maths-cours.fr#2}}}
	\fi 
}


\newcommand\TitreC[1]{    		% Titre centré
     \needspace{3\baselineskip}
     \begin{center}\textbf{#1}\end{center}
}

\newcommand\newpar{    		% paragraphe
     \par
}

\newcommand\nosp {    		% commande vide (pas d'espace)
}
\newcommand{\id}[1]{} %ignore

\newcommand\boite[2]{				% Boite simple sans titre
	\vspace{5mm}
	\setlength{\fboxrule}{0.2mm}
	\setlength{\fboxsep}{5mm}	
	\fcolorbox{#1}{#1!3}{\makebox[\linewidth-2\fboxrule-2\fboxsep]{
  		\begin{minipage}[t]{\linewidth-2\fboxrule-4\fboxsep}\setlength{\parskip}{3mm}
  			 #2
  		\end{minipage}
	}}
	\vspace{5mm}
}

\newcommand\CBox[4]{				% Boites
	\vspace{5mm}
	\setlength{\fboxrule}{0.2mm}
	\setlength{\fboxsep}{5mm}
	
	\fcolorbox{#1}{#1!3}{\makebox[\linewidth-2\fboxrule-2\fboxsep]{
		\begin{minipage}[t]{1cm}\setlength{\parskip}{3mm}
	  		\textcolor{#1}{\LARGE{#2}}    
 	 	\end{minipage}  
  		\begin{minipage}[t]{\linewidth-2\fboxrule-4\fboxsep}\setlength{\parskip}{3mm}
			\raisebox{1.2mm}{\normalsize\sffamily{\textcolor{#1}{#3}}}						
  			 #4
  		\end{minipage}
	}}
	\vspace{5mm}
}

\newcommand\cadre[3]{				% Boites convertible html
	\par
	\vspace{2mm}
	\setlength{\fboxrule}{0.1mm}
	\setlength{\fboxsep}{5mm}
	\fcolorbox{#1}{white}{\makebox[\linewidth-2\fboxrule-2\fboxsep]{
  		\begin{minipage}[t]{\linewidth-2\fboxrule-4\fboxsep}\setlength{\parskip}{3mm}
			\raisebox{-2.5mm}{\sffamily \small{\textcolor{#1}{\MakeUppercase{#2}}}}		
			\par		
  			 #3
 	 		\end{minipage}
	}}
		\vspace{2mm}
	\par
}

\newcommand\bloc[3]{				% Boites convertible html sans bordure
     \needspace{2\baselineskip}
     {\sffamily \small{\textcolor{#1}{\MakeUppercase{#2}}}}    
		\par		
  			 #3
		\par
}

\newcommand\CHelp[1]{
     \CBox{Plum}{\faInfoCircle}{À RETENIR}{#1}
}

\newcommand\CUp[1]{
     \CBox{NavyBlue}{\faThumbsOUp}{EN PRATIQUE}{#1}
}

\newcommand\CInfo[1]{
     \CBox{Sepia}{\faArrowCircleRight}{REMARQUE}{#1}
}

\newcommand\CRedac[1]{
     \CBox{PineGreen}{\faEdit}{BIEN R\'EDIGER}{#1}
}

\newcommand\CError[1]{
     \CBox{Red}{\faExclamationTriangle}{ATTENTION}{#1}
}

\newcommand\TitreExo[2]{
\needspace{4\baselineskip}
 {\sffamily\large EXERCICE #1\ (\emph{#2 points})}
\vspace{5mm}
}

\newcommand\img[2]{
          \includegraphics[width=#2\paperwidth]{\imgdir#1}
}

\newcommand\imgsvg[2]{
       \begin{center}   \includegraphics[width=#2\paperwidth]{\imgsvgdir#1} \end{center}
}


\newcommand\Lien[2]{
     \href{#1}{#2 \tiny \faExternalLink}
}
\newcommand\mcLien[2]{
     \href{https~://www.maths-cours.fr/#1}{#2 \tiny \faExternalLink}
}

\newcommand{\euro}{\eurologo{}}

%================================================================================================================================
%
% Macros - Environement
%
%================================================================================================================================

\newenvironment{tex}{ %
}
{%
}

\newenvironment{indente}{ %
	\setlength\parindent{10mm}
}

{
	\setlength\parindent{0mm}
}

\newenvironment{corrige}{%
     \needspace{3\baselineskip}
     \medskip
     \textbf{\textsc{Corrigé}}
     \medskip
}
{
}

\newenvironment{extern}{%
     \begin{center}
     }
     {
     \end{center}
}

\NewEnviron{code}{%
	\par
     \boite{gray}{\texttt{%
     \BODY
     }}
     \par
}

\newenvironment{vbloc}{% boite sans cadre empeche saut de page
     \begin{minipage}[t]{\linewidth}
     }
     {
     \end{minipage}
}
\NewEnviron{h2}{%
    \needspace{3\baselineskip}
    \vspace{0.6cm}
	\noindent \MakeUppercase{\sffamily \large \BODY}
	\vspace{1mm}\textcolor{mcgris}{\hrule}\vspace{0.4cm}
	\par
}{}

\NewEnviron{h3}{%
    \needspace{3\baselineskip}
	\vspace{5mm}
	\textsc{\BODY}
	\par
}

\NewEnviron{margeneg}{ %
\begin{addmargin}[-1cm]{0cm}
\BODY
\end{addmargin}
}

\NewEnviron{html}{%
}

\begin{document}
\begin{h2}1. Loi normale centrée réduite\end{h2}
\cadre{bleu}{Définition}{%id="d10"
     On dit qu'une variable aléatoire $X$ suit la \textbf{loi normale centrée réduite} sur $\mathbb{R}$ (notée $\mathscr N \left(0;1\right)$) si sa densité de probabilité $f$ est définie par :
     \begin{center}$f\left(x\right)=\frac{1}{\sqrt{2\pi }}e^{^{-\frac{x^{2}}{2}}}$\end{center}
     Cela signifie que, pour tous réels $a$ et $b$ tels que $a\leqslant b$:
     \begin{center}$p\left(a\leqslant X\leqslant b\right)=\int_{a}^{ b}\frac{1}{\sqrt{2\pi }}e^{^{-\frac{t^{2}}{2}}}dt$\end{center}
}
\bloc{cyan}{Remarques}{%id="r10"
     \begin{itemize}
          \item On admet que $f$ définit bien une densité, c'est à dire que l'aire comprise entre l'axe des abscisses et la courbe représentative de $f$ est égale à 1
          \item On a également :
          \par
          $p\left(X\geqslant a\right) =\lim\limits_{x\rightarrow +\infty }\int_{a}^{ x}\frac{1}{\sqrt{2\pi }}e^{^{-\frac{t^{2}}{2}}}dt $ (limite que l'on peut noter : $\int_{a}^{ +\infty }\frac{1}{\sqrt{2\pi }}e^{^{-\frac{t^{2}}{2}}}dt $)
          \par
          $p\left(X\leqslant b\right) =\lim\limits_{x\rightarrow -\infty }\int_{x}^{b}\frac{1}{\sqrt{2\pi }}e^{^{-\frac{t^{2}}{2}}}dt $ (limite que l'on peut noter : $\int_{-\infty }^{ b}\frac{1}{\sqrt{2\pi }}e^{^{-\frac{t^{2}}{2}}}dt $)
          \item La fonction $f : x\mapsto \frac{1}{\sqrt{2\pi }}e^{^{-\frac{x^{2}}{2}}}$ est dérivable sur $\mathbb{R}$, paire, positive, son tableau de variation est :
          %:-+-+-+-+- Engendré par : http://math.et.info.free.fr/TikZ/TableauxVariations/
          \begin{center}
               \begin{extern}%width="350"
                    \begin{tikzpicture}[scale=0.875]
                         % Styles
                         \tikzstyle{cadre}=[thin]
                         \tikzstyle{fleche}=[->,>=latex,thin]
                         \tikzstyle{nondefini}=[lightgray]
                         % Dimensions Modifiables
                         \def\Lrg{1.5}
                         \def\HtX{1}
                         \def\HtY{0.5}
                         % Dimensions Calculées
                         \def\lignex{-0.5*\HtX}
                         \def\lignef{-1.5*\HtX}
                         \def\separateur{-0.5*\Lrg}
                         % Largeur du tableau
                         \def\gauche{-1.5*\Lrg}
                         \def\droite{4.5*\Lrg}
                         % Hauteur du tableau
                         \def\haut{0.5*\HtX}
                         \def\bas{-2.5*\HtX-2*\HtY}
                         % Ligne de l'abscisse : x
                         \node at (-1*\Lrg,0) {$x$};
                         \node at (0*\Lrg,0) {$-\infty$};
                         \node at (2*\Lrg,0) {$0$};
                         \node at (4*\Lrg,0) {$+\infty$};
                         % Ligne de la dérivée : f'(x)
                         \node at (-1*\Lrg,-1*\HtX) {$f'(x)$};
                         \node at (0*\Lrg,-1*\HtX) {$$};
                         \node at (1*\Lrg,-1*\HtX) {$+$};
                         \node at (2*\Lrg,-1*\HtX) {$0$};
                         \node at (3*\Lrg,-1*\HtX) {$-$};
                         \node at (4*\Lrg,-1*\HtX) {$$};
                         % Ligne de la fonction : f(x)
                         \node  at (-1*\Lrg,{-2*\HtX+(-1)*\HtY}) {$f(x)$};
                         \node (f1) at (0*\Lrg,{-2*\HtX+(-2)*\HtY}) {$0$};
                         \node (f2) at (2*\Lrg,{-2*\HtX+(-0.5)*\HtY}) {$\dfrac{1}{\sqrt{2\pi}}$};
                         \node (f3) at (4*\Lrg,{-2*\HtX+(-2)*\HtY}) {$0$};
                         % Flèches
                         \draw[fleche] (f1) -- (f2);
                         \draw[fleche] (f2) -- (f3);
                         % Encadrement
                         \draw[cadre] (\separateur,\haut) -- (\separateur,\bas);
                         \draw[cadre] (\gauche,\haut) rectangle  (\droite,\bas);
                         \draw[cadre] (\gauche,\lignex) -- (\droite,\lignex);
                         \draw[cadre] (\gauche,\lignef) -- (\droite,\lignef);
                    \end{tikzpicture}
               \end{extern}
          \end{center}
          et sa courbe représentative :
          \begin{center}
               \begin{extern} %width="550" alt="loi normale centrée réduite"
                    % -+-+-+ variables modifiables
                    \def\e{2.7182818}
                    \def\pi{3.1415926}
                    \def\fonction{1/sqrt(2*\pi)*\e^(-x*x/2)  }
                    \def\xmin{-3.5}
                    \def\xmax{3.5}
                    \def\ymin{0}
                    \def\ymax{0.5}
                    \def\xunit{2}  % unités en cm
                    \def\yunit{12}
                    \psset{xunit=\xunit,yunit=\yunit,algebraic=true}
                    \begin{pspicture*}(\xmin,-0.08)(\xmax,\ymax)
                         %      \psgrid[gridcolor=mcgris, subgriddiv=5, gridlabels=0pt](-5,-0.3)(5,1)
                         \psaxes[Dy=0.1,linewidth=0.75pt]{->}(0,0)(\xmin,-0.08)(\xmax,\ymax)
                         \psplot[plotpoints=2000,linecolor=red,linewidth=0.75pt]{\xmin}{\xmax}{\fonction}
                       \rput[br](-0.1,-0.045){$O$}
                      \end{pspicture*}
               \end{extern}
          \end{center}
          \item $p\left(a\leqslant X\leqslant b\right)$ est l'aire du domaine coloré ci-dessous :
          \begin{center}
               \begin{extern} %width="550" alt="loi normale probabilité"
                    % -+-+-+ variables modifiables
                    \def\e{2.7182818}
                    \def\pi{3.1415926}
                    \def\fonction{1/sqrt(2*\pi)*\e^(-x*x/2)  }
                    \def\xmin{-3.5}
                    \def\xmax{3.5}
                    \def\ymin{0}
                    \def\ymax{0.5}
                    \def\xunit{2}  % unités en cm
                    \def\yunit{12}
                    \psset{xunit=\xunit,yunit=\yunit,algebraic=true}
                    \begin{pspicture*}(\xmin,-0.08)(\xmax,\ymax)
                         %      \psgrid[gridcolor=mcgris, subgriddiv=5, gridlabels=0pt](-5,-0.3)(5,1)
                         \psaxes[Dy=0.1,linewidth=0.75pt]{->}(0,0)(\xmin,-0.08)(\xmax,\ymax)
                         \psplot[plotpoints=2000,linecolor=red,linewidth=0.75pt]{\xmin}{\xmax}{\fonction}
                         \pscustom[fillcolor=blue,fillstyle=solid,opacity=0.2,linecolor=blue]{
                              \moveto(1.14,0)
                              \psplot[plotpoints=3000,linewidth=0.75pt]{1.14}{2.24}{\fonction}
                              \psline(2.24,0)(1.14,0)
                         }
                         \rput[b](1.14,-0.045){$\color{blue} a$}
                         \rput[b](2.24,-0.045){$\color{blue} b$}
                         \rput[br](-0.1,-0.045){$O$}
                    \end{pspicture*}
               \end{extern}
          \end{center}
          \item $p\left(X\leqslant a\right)$ est l'aire du domaine coloré ci-dessous :
          \begin{center}
               \begin{extern} %width="550" alt="loi normale probabilité"
                    % -+-+-+ variables modifiables
                    \def\e{2.7182818}
                    \def\pi{3.1415926}
                    \def\fonction{1/sqrt(2*\pi)*\e^(-x*x/2)  }
                    \def\xmin{-3.5}
                    \def\xmax{3.5}
                    \def\ymin{0}
                    \def\ymax{0.5}
                    \def\xunit{2}  % unités en cm
                    \def\yunit{12}
                    \psset{xunit=\xunit,yunit=\yunit,algebraic=true}
                    \begin{pspicture*}(\xmin,-0.08)(\xmax,\ymax)
                         %      \psgrid[gridcolor=mcgris, subgriddiv=5, gridlabels=0pt](-5,-0.3)(5,1)
                         \psaxes[Dy=0.1,linewidth=0.75pt]{->}(0,0)(\xmin,-0.08)(\xmax,\ymax)
                         \psplot[plotpoints=2000,linecolor=red,linewidth=0.75pt]{\xmin}{\xmax}{\fonction}
                         \pscustom[fillcolor=blue,fillstyle=solid,opacity=0.2,linecolor=blue]{
                              \psplot[plotpoints=3000,linewidth=0.75pt]{-4}{1.14}{\fonction}
                              \psline(1.14,0)(-4,0)
                         }
                         \rput[b](1.14,-0.045){$\color{blue} a$}
                         \rput[br](-0.1,-0.045){$O$}
                    \end{pspicture*}
               \end{extern}
          \end{center}
     \end{itemize}
}
\cadre{vert}{Propriétés}{%id="p20"
     Soit $X$ une variable aléatoire qui suit la loi normale centrée réduite :
     \begin{itemize}
          \item L'espérance mathématique de $X$ est $E\left(X\right)=0$ (loi \textit{centrée})~;
          \item La variance de $X$ est $\sigma \left(X\right)=1$ (loi \textit{réduite}).
     \end{itemize}
}
\cadre{vert}{Propriétés}{%id="p30"
     Soit $X$ une variable aléatoire qui suit la loi normale centrée réduite et $\alpha $ un réel quelconque :
     \begin{itemize}
          \item $p\left(X\leqslant 0\right)=p\left(X\geqslant 0\right)=0,5$
          \item $p\left(X\leqslant -a\right)=p\left(X\geqslant a\right)$
          \item $p\left(-\alpha \leqslant X\leqslant a\right)=1-2\times p\left(X\geqslant a\right)$\nosp$=2\times p\left(X\leqslant a\right)-1$
     \end{itemize}
}
\bloc{cyan}{Remarque}{%id="r30"
     Ces propriétés résultent du fait que :
     \begin{itemize}
          \item  la courbe de la fonction $x\mapsto \frac{1}{\sqrt{2\pi }}e^{^{-\frac{x^{2}}{2}}}$ est symétrique par rapport à l'axe des ordonnées
          \item l'aire comprise entre l'axe des abscisses et la courbe est égale à 1.
     \end{itemize}
     On retrouve facilement ces propriétés à l'aide d'une figure par exemple pour la seconde formule :
     \begin{center}
          \begin{extern} %width="550" alt="loi normale symétrie"
               % -+-+-+ variables modifiables
               \def\e{2.7182818}
               \def\pi{3.1415926}
               \def\fonction{1/sqrt(2*\pi)*\e^(-x*x/2)  }
               \def\xmin{-3.5}
               \def\xmax{3.5}
               \def\ymin{0}
               \def\ymax{0.5}
               \def\xunit{2}  % unités en cm
               \def\yunit{12}
               \psset{xunit=\xunit,yunit=\yunit,algebraic=true}
               \begin{pspicture*}(\xmin,-0.08)(\xmax,\ymax)
                    %      \psgrid[gridcolor=mcgris, subgriddiv=5, gridlabels=0pt](-5,-0.3)(5,1)
                    \psaxes[Dy=0.1,linewidth=0.75pt]{->}(0,0)(\xmin,-0.08)(\xmax,\ymax)
                    \psplot[plotpoints=2000,linecolor=red,linewidth=0.75pt]{\xmin}{\xmax}{\fonction}
                    \pscustom[fillcolor=vert,fillstyle=solid,opacity=0.2,linecolor=vert]{
                         \psplot[plotpoints=3000,linewidth=0.75pt]{-4}{-1.14}{\fonction}
                         \psline(-1.14,0)(-4,0)
                    }
                    \pscustom[fillcolor=blue,fillstyle=solid,opacity=0.2,linecolor=blue]{
                         \moveto(1.14,0)
                         \psplot[plotpoints=3000,linewidth=0.75pt]{1.14}{4}{\fonction}
                         \psline(4,0)(1.14,0)
                    }
                    \rput[br](-1.14,-0.045){$\color{vert} -a$}
                    \rput[b](1.14,-0.045){$\color{blue} a$}
                    \rput[br](-0.1,-0.045){$O$}
               \end{pspicture*}
          \end{extern}
     \end{center}
     \begin{center}$p\left(X\leqslant -a\right)=p\left(X\geqslant a\right)$\end{center}
}
\cadre{vert}{Propriété («Loi normale inverse»)}{%id="p50"
     Soient $X$ une variable aléatoire qui suit la loi normale centrée réduite et un réel $k \in \left]0;1\right[ $.
     \par
     Il existe un \textbf{unique} réel $m_{k}$ tel que $p\left(X\leqslant m_{k}\right)=k$.
}
\bloc{cyan}{Remarque}{%id="r50"
     On peut calculer les valeurs de $m_{k}$ à la calculatrice.
}
\cadre{rouge}{Théorème}{%id="t60"
     Soient $X$ une variable aléatoire qui suit la loi normale centrée réduite et un réel $\alpha \in \left]0;1\right[ $.
     \par
     Il existe un \textbf{unique} réel $u_\alpha $ tel que :
\begin{center}
     $p\left(-u_\alpha \leqslant X\leqslant u_\alpha \right)=1-\alpha$.
\end{center}
}
          \begin{center}
               \begin{extern} %width="550" alt="loi normale seuil"
                    % -+-+-+ variables modifiables
                    \def\e{2.7182818}
                    \def\pi{3.1415926}
                    \def\fonction{1/sqrt(2*\pi)*\e^(-x*x/2)  }
                    \def\xmin{-3.5}
                    \def\xmax{3.5}
                    \def\ymin{0}
                    \def\ymax{0.5}
                    \def\xunit{2}  % unités en cm
                    \def\yunit{12}
                    \psset{xunit=\xunit,yunit=\yunit,algebraic=true}
                    \begin{pspicture*}(\xmin,-0.08)(\xmax,\ymax)
                         %      \psgrid[gridcolor=mcgris, subgriddiv=5, gridlabels=0pt](-5,-0.3)(5,1)
                         \psaxes[Dy=0.1,linewidth=0.75pt]{->}(0,0)(\xmin,-0.08)(\xmax,\ymax)
                         \psplot[plotpoints=2000,linecolor=red,linewidth=0.75pt]{\xmin}{\xmax}{\fonction}
                         \pscustom[fillcolor=blue,fillstyle=solid,opacity=0.1,linecolor=blue]{
                              \moveto(-1.6,0)
                              \psplot[plotpoints=3000,linewidth=0.75pt]{-1.6}{1.6}{\fonction}
                              \psline(1.6,0)(-1.6,0)
                         }
                         \rput[b](-1.6,-0.045){$\color{blue} -u_{\alpha}$}
                         \rput[b](1.6,-0.045){$\color{blue} u_{\alpha}$}
                          \rput[b](0.6,0.1){$\color{blue} 1-\alpha$}
                           \rput[b](1.9,0.02){$\color{gray} \alpha/2$}
                           \rput[b](-1.9,0.02){$\color{gray} \alpha/2$}
                         \rput[br](-0.1,-0.045){$O$}
                    \end{pspicture*}
               \end{extern}               
          \end{center}
          \begin{center}
          $p\left(-u_\alpha \leqslant X\leqslant u_\alpha \right)=1-\alpha $
          \end{center}
\bloc{cyan}{Remarques}{%id="r60"
     \begin{itemize}
          \item En utilisant la formule $p\left(-\alpha \leqslant X\leqslant a\right)=2\times p\left(X\leqslant a\right)-1$ et la \textit{«loi normale inverse»} on peut calculer les valeurs de $u_\alpha $ à la calculatrice.
          \item Deux valeurs à retenir :
          \par
          $u_{0,05}=1,96 $ c'est à dire que $p\left(-1.96\leqslant X\leqslant 1.96\right)=0,95$
          \par
          $u_{0,01}=2,58 $ c'est à dire que $p\left(-2,58\leqslant X\leqslant 2,58\right)=0,99$
     \end{itemize}
}
\begin{h2}2. Loi normale générale\end{h2}
\cadre{bleu}{Définition et théorème}{%id="d70"
     Soient deux réels $\mu $ et $\sigma  > 0$.
     \par
     On dit qu'une variable aléatoire $X$ suit une \textbf{loi normale de paramètres $\mu $ et $\sigma ^{2}$} (notée $\mathscr N \left(\mu ; \sigma ^{2}\right)$) si la variable aléatoire $Y=\frac{X-\mu }{\sigma} $ suit la loi normale centrée réduite.
     \par
     L'espérance mathématique de $X$ est $\mu $ et son écart-type $\sigma $ (et donc sa variance $\sigma ^{2}$).
}
\bloc{cyan}{Remarque}{%id="r70"
     La courbe représentative de la distribution d'une loi $\mathscr N \left(\mu ; \sigma ^{2}\right)$ est une courbe «~en cloche~» qui admet la droite d'équation $x=\mu $ comme axe de symétrie. Elle est plus ou moins «~étirée~» selon les valeurs de $\sigma $
     \begin{center}
          \begin{extern} %width="500" alt="loi normale différents écarts-types"
               % -+-+-+ variables modifiables
               \def\e{2.7182818}
               \def\pi{3.1415926}
               \def\f{2/sqrt(2*\pi)*\e^(-(x-3)*(x-3)/0.5)  }
               \def\g{1/sqrt(2*\pi)*\e^(-(x-3)*(x-3)/2)  }
               \def\h{0.5/sqrt(2*\pi)*\e^(-(x-3)*(x-3)/8)  }
               \def\xmin{-2.3}
               \def\xmax{8.3}
               \def\ymin{0}
               \def\ymax{0.85}
               \def\xunit{1}  % unités en cm
               \def\yunit{10}
               \psset{xunit=\xunit,yunit=\yunit,algebraic=true}
               \begin{pspicture*}(\xmin,-0.08)(\xmax,\ymax)
                    %      \psgrid[gridcolor=mcgris, subgriddiv=5, gridlabels=0pt](-5,-0.3)(5,1)
                    \psaxes[Dy=0.1,linewidth=0.75pt]{->}(0,0)(\xmin,-0.08)(\xmax,\ymax)
                    \psplot[plotpoints=2000,linecolor=blue,linewidth=0.75pt]{\xmin}{\xmax}{\f}
                    \psplot[plotpoints=2000,linecolor=red,linewidth=0.75pt]{\xmin}{\xmax}{\g}
                    \psplot[plotpoints=2000,linecolor=vert,linewidth=0.75pt]{\xmin}{\xmax}{\h}
                    \psline[linecolor=mauve,linewidth=0.5pt](3,10)(3,-1)
                    \rput[br](2.9,0.02){$\color{mauve} \mu$}
                   \rput[br](5,0.5){$\color{blue} \sigma=0,5$}
                   \rput[br](5.2,0.3){$\color{red} \sigma=1$}
                   \rput[br](6,0.15){$\color{vert} \sigma=2$}
                   \rput[br](-0.1,-0.045){$O$}
                  \end{pspicture*}
          \end{extern}
     \end{center}
     \begin{center}$\mu =3$ et $ \sigma =0,5$ ; $1$ ; $2$\end{center}
}
\cadre{vert}{Propriété (Règle des trois sigmas)}{%id="p80"
     Si $X$ suit une loi normale $\mathscr N \left(\mu ; \sigma ^{2}\right)$ alors :
     \begin{itemize}
          \item
          $p\left(\mu -\sigma \leqslant X\leqslant \mu + \sigma \right)\approx 0,68$ (à $10^{-2}$ près)
          \item
          $p\left(\mu -2\sigma \leqslant X\leqslant \mu + 2\sigma \right)\approx 0,95$ (à $10^{-2}$ près)
          \item
          $p\left(\mu -3\sigma \leqslant X\leqslant \mu + 3\sigma \right)\approx 0,997$ (à $10^{-3}$ près)
     \end{itemize}
}
\bloc{orange}{Exemple}{%id="e80"
     Si $X$ suit une loi normale $\mathscr N \left(11 ; 3^{2}\right)$ alors :
     \par
     $p\left(5\leqslant X\leqslant 17\right)\approx 0,95$
}
\begin{h2}3. Théorème de Moivre-Laplace\end{h2}
\cadre{rouge}{Théorème (Moivre-Laplace)}{%id="t90"
     Soit $X_{n}$ une variable aléatoire qui suit une loi \textbf{binomiale} $\mathscr B \left(n;p\right)$.
     \par
     On pose $Z_{n}=\frac{X_{n}-E\left(X_{n}\right)}{\sigma \left(X_{n}\right)}$.
     \par
     Alors pour tous réels $a$ et $b$ :
     \begin{center}$\lim\limits_{n\rightarrow +\infty }p\left(a\leqslant Z_{n}\leqslant b\right)=\int_{a}^{ b}\frac{1}{\sqrt{2\pi }}e^{^{-\frac{t^{2}}{2}}}dt$\end{center}
}
\bloc{cyan}{Remarques}{%id="r90"
     \begin{itemize}
          \item On rappelle que pour une loi binomiale $X$ de paramètres $n$ et $p$ :$E\left(X\right)=np$ et $\sigma \left(X\right)^{2}=np\left(1-p\right)$. $Z_{n}$ peut donc aussi s'écrire : $Z_{n}=\frac{X_{n}-np}{\sqrt{np\left(1-p\right)}}$
          \item Ce théorème signifie que pour $n$ élevé, la loi de $Z_{n}$ est proche de la loi normale centrée réduite :
          \begin{center}
               \begin{extern} %width="550" alt="comparaison loi normale loi binomiale"
                    % -+-+-+ variables modifiables
                    \def\e{2.7182818}
                    \def\pi{3.1415926}
                    \def\fonction{1/sqrt(2*\pi)*\e^(-x*x/2)  }
                    \def\xmin{-3.5}
                    \def\xmax{3.5}
                    \def\ymin{0}
                    \def\ymax{0.5}
                    \def\xunit{2}  % unités en cm
                    \def\yunit{12}
                    \psset{xunit=\xunit,yunit=\yunit,algebraic=true,dimen=middle,}
                    \begin{pspicture*}(\xmin,-0.08)(\xmax,\ymax)
                         %      \psgrid[gridcolor=mcgris, subgriddiv=5, gridlabels=0pt](-5,-0.3)(5,1)
                         \psframe[linewidth=0.8pt,linecolor=blue,fillcolor=blue,fillstyle=solid,opacity=0.05](-0.612,0)(-1.021,0.2863)
                         \psframe[linewidth=0.8pt,linecolor=blue,fillcolor=blue,fillstyle=solid,opacity=0.05](-1.021,0)(-1.429,0.1909)
                         \psframe[linewidth=0.8pt,linecolor=blue,fillcolor=blue,fillstyle=solid,opacity=0.05](-1.429,0)(-1.837,0.1073)
                         \psframe[linewidth=0.8pt,linecolor=blue,fillcolor=blue,fillstyle=solid,opacity=0.05](-1.837,0)(-2.245,0.0505)
                         \psframe[linewidth=0.8pt,linecolor=blue,fillcolor=blue,fillstyle=solid,opacity=0.05](-2.245,0)(-2.654,0.0196)
                         \psframe[linewidth=0.8pt,linecolor=blue,fillcolor=blue,fillstyle=solid,opacity=0.05](-2.654,0)(-3.062,0.0062)
                         \psframe[linewidth=0.8pt,linecolor=blue,fillcolor=blue,fillstyle=solid,opacity=0.05](-3.062,0)(-3.47,0.0015)
                         \psframe[linewidth=0.8pt,linecolor=blue,fillcolor=blue,fillstyle=solid,opacity=0.05](-0.204,0)(-0.612,0.3644)
                         \psframe[linewidth=0.8pt,linecolor=blue,fillcolor=blue,fillstyle=solid,opacity=0.05](-0.204,0)(0.204,0.3948)
                         \psframe[linewidth=0.8pt,linecolor=blue,fillcolor=blue,fillstyle=solid,opacity=0.05](0.204,0)(0.612,0.3644)
                         \psframe[linewidth=0.8pt,linecolor=blue,fillcolor=blue,fillstyle=solid,opacity=0.05](0.612,0)(1.021,0.2863)
                         \psframe[linewidth=0.8pt,linecolor=blue,fillcolor=blue,fillstyle=solid,opacity=0.05](1.021,0)(1.429,0.1909)
                         \psframe[linewidth=0.8pt,linecolor=blue,fillcolor=blue,fillstyle=solid,opacity=0.05](1.429,0)(1.837,0.1073)
                         \psframe[linewidth=0.8pt,linecolor=blue,fillcolor=blue,fillstyle=solid,opacity=0.05](1.837,0)(2.245,0.0505)
                         \psframe[linewidth=0.8pt,linecolor=blue,fillcolor=blue,fillstyle=solid,opacity=0.05](2.245,0)(2.654,0.0196)
                         \psframe[linewidth=0.8pt,linecolor=blue,fillcolor=blue,fillstyle=solid,opacity=0.05](2.654,0)(3.062,0.0062)
                         \psframe[linewidth=0.8pt,linecolor=blue,fillcolor=blue,fillstyle=solid,opacity=0.05](3.062,0)(3.47,0.0015)
                         \psplot[plotpoints=2000,linecolor=red,linewidth=0.75pt]{\xmin}{\xmax}{\fonction}
                         \psaxes[Dy=0.1,linewidth=0.75pt]{->}(0,0)(\xmin,-0.08)(\xmax,\ymax)
                    \end{pspicture*}
               \end{extern}
               \par
               \textit{Histogramme de $Z_{n}$ pour $n=24$ et $p=0,5$ et loi $\mathscr N \left(0;1\right)$}
          \end{center}
          \item En pratique, on considèrera que «$n$ est suffisamment élevé» si $n\geqslant 30$ ; $np\geqslant 5$ ; $n\left(1-p\right)\geqslant 5$.
          \par
          La loi binomiale $X$ pourra alors être approximée par la loi normale $\mathscr N \left(E\left(X\right);\sigma \left(X\right)^{2}\right)$
     \end{itemize}
}
\bloc{orange}{Exemple}{%id="e90"
     $X$ suit une loi binomiale $\mathscr B \left(30 ; 0,4\right)$.
     \par
     On cherche à calculer $p\left(7 < X \leqslant 17\right)$.
     \par
     Posons $Z=\frac{X-30\times 0.4}{\sqrt{30\times 0.4\times 0.6}}=\frac{X-12}{\sqrt{7,2}}$.
     \bigskip
     Alors :
     \par
     $7 < X \leqslant 17 \Leftrightarrow -5 < X-12\leqslant 5 $
     \smallskip
     $\phantom{7 < X \leqslant 17} \Leftrightarrow -\frac{5}{\sqrt{7,2}} < \frac{X-12}{\sqrt{7,2}}\leqslant \frac{5}{\sqrt{7,2}}$
     \smallskip
     $\phantom{7 < X \leqslant 17}\Leftrightarrow -1,86 < Z\leqslant 1,86$
     \bigskip
     On a bien $n\geqslant 30$ ; $np\geqslant 5$ ; $n\left(1-p\right)\geqslant 5$. On peut donc approximer $Z$ par une loi normale centrée réduite.
     \par
     A la calculatrice on trouve alors :
     \par
     $p\left(-1,86 < Z \leqslant 1,86\right)\approx 0,937 $(un calcul direct avec la loi binomiale donne $0,935$)
}

\end{document}