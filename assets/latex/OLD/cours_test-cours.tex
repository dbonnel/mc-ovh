\documentclass[a4paper]{article}

%================================================================================================================================
%
% Packages
%
%================================================================================================================================

\usepackage[T1]{fontenc} 	% pour caractères accentués
\usepackage[utf8]{inputenc}  % encodage utf8
\usepackage[french]{babel}	% langue : français
\usepackage{fourier}			% caractères plus lisibles
\usepackage[dvipsnames]{xcolor} % couleurs
\usepackage{fancyhdr}		% réglage header footer
\usepackage{needspace}		% empêcher sauts de page mal placés
\usepackage{graphicx}		% pour inclure des graphiques
\usepackage{enumitem,cprotect}		% personnalise les listes d'items (nécessaire pour ol, al ...)
\usepackage{hyperref}		% Liens hypertexte
\usepackage{pstricks,pst-all,pst-node,pstricks-add,pst-math,pst-plot,pst-tree,pst-eucl} % pstricks
\usepackage[a4paper,includeheadfoot,top=2cm,left=3cm, bottom=2cm,right=3cm]{geometry} % marges etc.
\usepackage{comment}			% commentaires multilignes
\usepackage{amsmath,environ} % maths (matrices, etc.)
\usepackage{amssymb,makeidx}
\usepackage{bm}				% bold maths
\usepackage{tabularx}		% tableaux
\usepackage{colortbl}		% tableaux en couleur
\usepackage{fontawesome}		% Fontawesome
\usepackage{environ}			% environment with command
\usepackage{fp}				% calculs pour ps-tricks
\usepackage{multido}			% pour ps tricks
\usepackage[np]{numprint}	% formattage nombre
\usepackage{tikz,tkz-tab} 			% package principal TikZ
\usepackage{pgfplots}   % axes
\usepackage{mathrsfs}    % cursives
\usepackage{calc}			% calcul taille boites
\usepackage[scaled=0.875]{helvet} % font sans serif
\usepackage{svg} % svg
\usepackage{scrextend} % local margin
\usepackage{scratch} %scratch
\usepackage{multicol} % colonnes
%\usepackage{infix-RPN,pst-func} % formule en notation polanaise inversée
\usepackage{listings}

%================================================================================================================================
%
% Réglages de base
%
%================================================================================================================================

\lstset{
language=Python,   % R code
literate=
{á}{{\'a}}1
{à}{{\`a}}1
{ã}{{\~a}}1
{é}{{\'e}}1
{è}{{\`e}}1
{ê}{{\^e}}1
{í}{{\'i}}1
{ó}{{\'o}}1
{õ}{{\~o}}1
{ú}{{\'u}}1
{ü}{{\"u}}1
{ç}{{\c{c}}}1
{~}{{ }}1
}


\definecolor{codegreen}{rgb}{0,0.6,0}
\definecolor{codegray}{rgb}{0.5,0.5,0.5}
\definecolor{codepurple}{rgb}{0.58,0,0.82}
\definecolor{backcolour}{rgb}{0.95,0.95,0.92}

\lstdefinestyle{mystyle}{
    backgroundcolor=\color{backcolour},   
    commentstyle=\color{codegreen},
    keywordstyle=\color{magenta},
    numberstyle=\tiny\color{codegray},
    stringstyle=\color{codepurple},
    basicstyle=\ttfamily\footnotesize,
    breakatwhitespace=false,         
    breaklines=true,                 
    captionpos=b,                    
    keepspaces=true,                 
    numbers=left,                    
xleftmargin=2em,
framexleftmargin=2em,            
    showspaces=false,                
    showstringspaces=false,
    showtabs=false,                  
    tabsize=2,
    upquote=true
}

\lstset{style=mystyle}


\lstset{style=mystyle}
\newcommand{\imgdir}{C:/laragon/www/newmc/assets/imgsvg/}
\newcommand{\imgsvgdir}{C:/laragon/www/newmc/assets/imgsvg/}

\definecolor{mcgris}{RGB}{220, 220, 220}% ancien~; pour compatibilité
\definecolor{mcbleu}{RGB}{52, 152, 219}
\definecolor{mcvert}{RGB}{125, 194, 70}
\definecolor{mcmauve}{RGB}{154, 0, 215}
\definecolor{mcorange}{RGB}{255, 96, 0}
\definecolor{mcturquoise}{RGB}{0, 153, 153}
\definecolor{mcrouge}{RGB}{255, 0, 0}
\definecolor{mclightvert}{RGB}{205, 234, 190}

\definecolor{gris}{RGB}{220, 220, 220}
\definecolor{bleu}{RGB}{52, 152, 219}
\definecolor{vert}{RGB}{125, 194, 70}
\definecolor{mauve}{RGB}{154, 0, 215}
\definecolor{orange}{RGB}{255, 96, 0}
\definecolor{turquoise}{RGB}{0, 153, 153}
\definecolor{rouge}{RGB}{255, 0, 0}
\definecolor{lightvert}{RGB}{205, 234, 190}
\setitemize[0]{label=\color{lightvert}  $\bullet$}

\pagestyle{fancy}
\renewcommand{\headrulewidth}{0.2pt}
\fancyhead[L]{maths-cours.fr}
\fancyhead[R]{\thepage}
\renewcommand{\footrulewidth}{0.2pt}
\fancyfoot[C]{}

\newcolumntype{C}{>{\centering\arraybackslash}X}
\newcolumntype{s}{>{\hsize=.35\hsize\arraybackslash}X}

\setlength{\parindent}{0pt}		 
\setlength{\parskip}{3mm}
\setlength{\headheight}{1cm}

\def\ebook{ebook}
\def\book{book}
\def\web{web}
\def\type{web}

\newcommand{\vect}[1]{\overrightarrow{\,\mathstrut#1\,}}

\def\Oij{$\left(\text{O}~;~\vect{\imath},~\vect{\jmath}\right)$}
\def\Oijk{$\left(\text{O}~;~\vect{\imath},~\vect{\jmath},~\vect{k}\right)$}
\def\Ouv{$\left(\text{O}~;~\vect{u},~\vect{v}\right)$}

\hypersetup{breaklinks=true, colorlinks = true, linkcolor = OliveGreen, urlcolor = OliveGreen, citecolor = OliveGreen, pdfauthor={Didier BONNEL - https://www.maths-cours.fr} } % supprime les bordures autour des liens

\renewcommand{\arg}[0]{\text{arg}}

\everymath{\displaystyle}

%================================================================================================================================
%
% Macros - Commandes
%
%================================================================================================================================

\newcommand\meta[2]{    			% Utilisé pour créer le post HTML.
	\def\titre{titre}
	\def\url{url}
	\def\arg{#1}
	\ifx\titre\arg
		\newcommand\maintitle{#2}
		\fancyhead[L]{#2}
		{\Large\sffamily \MakeUppercase{#2}}
		\vspace{1mm}\textcolor{mcvert}{\hrule}
	\fi 
	\ifx\url\arg
		\fancyfoot[L]{\href{https://www.maths-cours.fr#2}{\black \footnotesize{https://www.maths-cours.fr#2}}}
	\fi 
}


\newcommand\TitreC[1]{    		% Titre centré
     \needspace{3\baselineskip}
     \begin{center}\textbf{#1}\end{center}
}

\newcommand\newpar{    		% paragraphe
     \par
}

\newcommand\nosp {    		% commande vide (pas d'espace)
}
\newcommand{\id}[1]{} %ignore

\newcommand\boite[2]{				% Boite simple sans titre
	\vspace{5mm}
	\setlength{\fboxrule}{0.2mm}
	\setlength{\fboxsep}{5mm}	
	\fcolorbox{#1}{#1!3}{\makebox[\linewidth-2\fboxrule-2\fboxsep]{
  		\begin{minipage}[t]{\linewidth-2\fboxrule-4\fboxsep}\setlength{\parskip}{3mm}
  			 #2
  		\end{minipage}
	}}
	\vspace{5mm}
}

\newcommand\CBox[4]{				% Boites
	\vspace{5mm}
	\setlength{\fboxrule}{0.2mm}
	\setlength{\fboxsep}{5mm}
	
	\fcolorbox{#1}{#1!3}{\makebox[\linewidth-2\fboxrule-2\fboxsep]{
		\begin{minipage}[t]{1cm}\setlength{\parskip}{3mm}
	  		\textcolor{#1}{\LARGE{#2}}    
 	 	\end{minipage}  
  		\begin{minipage}[t]{\linewidth-2\fboxrule-4\fboxsep}\setlength{\parskip}{3mm}
			\raisebox{1.2mm}{\normalsize\sffamily{\textcolor{#1}{#3}}}						
  			 #4
  		\end{minipage}
	}}
	\vspace{5mm}
}

\newcommand\cadre[3]{				% Boites convertible html
	\par
	\vspace{2mm}
	\setlength{\fboxrule}{0.1mm}
	\setlength{\fboxsep}{5mm}
	\fcolorbox{#1}{white}{\makebox[\linewidth-2\fboxrule-2\fboxsep]{
  		\begin{minipage}[t]{\linewidth-2\fboxrule-4\fboxsep}\setlength{\parskip}{3mm}
			\raisebox{-2.5mm}{\sffamily \small{\textcolor{#1}{\MakeUppercase{#2}}}}		
			\par		
  			 #3
 	 		\end{minipage}
	}}
		\vspace{2mm}
	\par
}

\newcommand\bloc[3]{				% Boites convertible html sans bordure
     \needspace{2\baselineskip}
     {\sffamily \small{\textcolor{#1}{\MakeUppercase{#2}}}}    
		\par		
  			 #3
		\par
}

\newcommand\CHelp[1]{
     \CBox{Plum}{\faInfoCircle}{À RETENIR}{#1}
}

\newcommand\CUp[1]{
     \CBox{NavyBlue}{\faThumbsOUp}{EN PRATIQUE}{#1}
}

\newcommand\CInfo[1]{
     \CBox{Sepia}{\faArrowCircleRight}{REMARQUE}{#1}
}

\newcommand\CRedac[1]{
     \CBox{PineGreen}{\faEdit}{BIEN R\'EDIGER}{#1}
}

\newcommand\CError[1]{
     \CBox{Red}{\faExclamationTriangle}{ATTENTION}{#1}
}

\newcommand\TitreExo[2]{
\needspace{4\baselineskip}
 {\sffamily\large EXERCICE #1\ (\emph{#2 points})}
\vspace{5mm}
}

\newcommand\img[2]{
          \includegraphics[width=#2\paperwidth]{\imgdir#1}
}

\newcommand\imgsvg[2]{
       \begin{center}   \includegraphics[width=#2\paperwidth]{\imgsvgdir#1} \end{center}
}


\newcommand\Lien[2]{
     \href{#1}{#2 \tiny \faExternalLink}
}
\newcommand\mcLien[2]{
     \href{https~://www.maths-cours.fr/#1}{#2 \tiny \faExternalLink}
}

\newcommand{\euro}{\eurologo{}}

%================================================================================================================================
%
% Macros - Environement
%
%================================================================================================================================

\newenvironment{tex}{ %
}
{%
}

\newenvironment{indente}{ %
	\setlength\parindent{10mm}
}

{
	\setlength\parindent{0mm}
}

\newenvironment{corrige}{%
     \needspace{3\baselineskip}
     \medskip
     \textbf{\textsc{Corrigé}}
     \medskip
}
{
}

\newenvironment{extern}{%
     \begin{center}
     }
     {
     \end{center}
}

\NewEnviron{code}{%
	\par
     \boite{gray}{\texttt{%
     \BODY
     }}
     \par
}

\newenvironment{vbloc}{% boite sans cadre empeche saut de page
     \begin{minipage}[t]{\linewidth}
     }
     {
     \end{minipage}
}
\NewEnviron{h2}{%
    \needspace{3\baselineskip}
    \vspace{0.6cm}
	\noindent \MakeUppercase{\sffamily \large \BODY}
	\vspace{1mm}\textcolor{mcgris}{\hrule}\vspace{0.4cm}
	\par
}{}

\NewEnviron{h3}{%
    \needspace{3\baselineskip}
	\vspace{5mm}
	\textsc{\BODY}
	\par
}

\NewEnviron{margeneg}{ %
\begin{addmargin}[-1cm]{0cm}
\BODY
\end{addmargin}
}

\NewEnviron{html}{%
}

\begin{document}
\begin{h2}1. Fonctions continues\end{h2}
\cadre{bleu}{Définition}{% id="d10"
     <p>
     Une fonction définie sur un intervalle $I$ est \textit{\textbf{continue}} sur $I$ si l'on peut tracer sa courbe représentative \textit{sans lever le crayon}
     </p>
}
\bloc{orange}{Exemples}{% id="e10"
     \begin{itemize}
          \item <p class="item"> Les fonctions polynômes sont continues sur $\mathbb{R}$.
          </p>
          \item <p class="item"> Les fonctions rationnelles sont continues sur chaque intervalle contenu dans leur ensemble de définition.
          </p>
          \item <p class="item"> La fonction \textit{racine carrée} est continue sur $\mathbb{R}^+$.
          </p>
          \item <p class="item"> Les fonctions \textit{sinus} et \textit{cosinus} sont continues sur $\mathbb{R}$
          </p>
     \end{itemize}
}
\cadre{rouge}{Théorème}{% id="t20"
     <p>
     Si $f$ et $g$ sont continues sur $I$, les fonctions $f+g$, $kf$ ( $k\in \mathbb{R}$ ) et $f\times g$ sont continues sur $I$.
     </p>
     <p>
     Si, de plus, $g$ ne s'annule pas sur $I$, la fonction $\frac{f}{g}$, est continue sur $I$.
     </p>
}
\cadre{rouge}{Théorème (lien entre continuité et dérivabilité)}{% id="t30"
     <p>
     Toute fonction \textbf{dérivable} sur un intervalle $I$ est \textbf{continue} sur $I$.
     </p>
}
\bloc{cyan}{Remarque}{% id="r30"
     <p>
     \textbf{Attention !} La réciproque est fausse.
     </p>
     <p>
     Par exemple, la fonction valeur absolue ($x\mapsto |x|$) est continue sur $\mathbb{R}$ tout entier mais n'est pas dérivable en 0.
     </p>
}
\cadre{vert}{Propriété (lien entre continuité et limite)}{% id="p40"
     <p>
     Si $f$ est une fonction continue sur un intervalle $\left[a ; b\right]$, alors pour tout $\alpha  \in  \left[a ; b\right]$ :
     </p>
     <p>
     $\lim\limits_{x\rightarrow \alpha }f\left(x\right)=\lim\limits_{x\rightarrow \alpha ^-}f\left(x\right)=\lim\limits_{x\rightarrow \alpha ^+}f\left(x\right)=f\left(\alpha \right)$.
     </p>
}
\bloc{orange}{Exemple}{% id="e40"
     <p>
     Montrons à l'aide de cette propriété que la fonction «partie entière» (notée $x\mapsto E\left(x\right)$), qui à tout réel $x$ associe le plus grand entier inférieur ou égal à $x$, n'est pas continue en $1$.
     </p>
     <p>
     Si $x$ est un réel positif et strictement inférieur à $1$, sa partie entière vaut $0$.
     </p>
     <p>
     Donc $\lim\limits_{x\rightarrow 1^-}E\left(x\right)=0$.
     </p>
     <p>
     Par ailleurs, la partie entière de $1$ vaut $1$ c'est à dire $E\left(1\right)=1$.
     </p>
     <p>
     Donc  $\lim\limits_{x\rightarrow 1^-}E\left(x\right)\neq E\left(1\right)$.
     </p>
     <p>
     La fonction «partie entière» n'est donc pas continue en $1$.
     \par
     <img src="/wp-content/uploads/mc-0328.png" alt="" class="aligncenter size-full  img-pc" />
</p><div class="center"><p>\textbf{\textit{Fonction «partie entière»}}</p>}
}
\begin{h2}2. Théorème des valeurs intermédiaires\end{h2}
\cadre{rouge}{Théorème des valeurs intermédiaires}{% id="t60"
     <p>
     Si $f$ est une fonction \textbf{continue} sur un intervalle $\left[a;b\right]$ et si $y_{0}$ est compris entre $f\left(a\right)$ et $f\left(b\right)$, alors l'équation $f\left(x\right)=y_{0}$ admet \textbf{au moins une} solution sur l'intervalle $\left[a ; b\right]$.
     </p>
}
\bloc{cyan}{Remarques}{% id="r60"
     \begin{itemize}
          \item <p class="item"> Ce théorème dit que l'équation $f\left(x\right)=y_{0}$ admet \textbf{une ou plusieurs solutions} mais ne permet pas de déterminer le nombre de ces solutions. Dans les exercices où l'on recherche le nombre de solutions, il faut utiliser le corollaire ci-dessous.
          </p>
          \item <p class="item"> \textbf{Cas particulier fréquent : } Si $f$ est continue et si $f\left(a\right)$ et $f\left(b\right)$ sont de signes contraires, l'équation $f\left(x\right)=0$ admet au moins une solution sur l'intervalle $\left[a ; b\right]$ (en effet,  si $f\left(a\right)$ et $f\left(b\right)$ sont de signes contraires, $0$ est compris entre $f\left(a\right)$ et $f\left(b\right)$)
          </p>
     \end{itemize}
}
\cadre{rouge}{Corollaire (du théorème des valeurs intermédiaires)}{% id="t70"
     <p>
     Si $f$ est une fonction \textbf{continue} et \textbf{strictement monotone} sur un intervalle $\left[a ; b\right]$ et si $y_{0}$ est compris entre $f\left(a\right)$ et $f\left(b\right)$, l'équation $f\left(x\right)=y_{0}$ admet une \textbf{unique} solution sur l'intervalle $\left[a ; b\right]$.
     </p>
}
\bloc{cyan}{Remarques}{% id="r70"
     \begin{itemize}
          \item <p class="item"> Ce dernier théorème est aussi parfois appelé \textbf{"Théorème de la bijection"}
          </p>
          \item <p class="item">
          Il faut vérifier \textbf{3 conditions} pour pouvoir appliquer ce corollaire:
          </p>
          <ul class="tiret">
          \item <p class="item">
          $f$ est continue sur $\left[a ; b\right]$
          </p>
          \item <p class="item">
          $f$ est strictement croissante ou strictement décroissante sur $\left[a ; b\right]$
          </p>
          \item <p class="item">
          $y_{0}$ est compris entre $f\left(a\right)$ et $f\left(b\right)$
          </p>
     \end{itemize}
     \item <p class="item"> Les deux théorèmes précédents se généralisent à un intervalle ouvert $\left]a ; b\right[$ où $a$ et $b$ sont éventuellement infinis. Il faut alors remplacer $f\left(a\right)$ et $f\left(b\right)$ (qui ne sont alors généralement pas définis) par $\lim\limits_{x\rightarrow a}f\left(x\right)$ et $\lim\limits_{x\rightarrow b}f\left(x\right)$
     </p>
\end{itemize}
}
\bloc{orange}{Exemple}{% id="e70"
     <p>
     Soit une fonction $f$ définie sur $\left]0 ; +\infty \right[$ dont le tableau de variation est fourni ci-dessous :
     \par
     <img src="/wp-content/uploads/mc-0329.png" alt="" class="aligncenter size-full  img-pc" />
     \par
     On cherche à déterminer le nombre de solutions de l'équation $f\left(x\right)=-1$
     </p>
     <p>
     L'unique flèche oblique montre que la fonction $f$ est \textbf{continue} et \textbf{strictement croissante} sur $\left]0;+\infty \right[$.
     </p>
     <p>
     $-1$ est compris entre $\lim\limits_{x\rightarrow 0}f\left(x\right)=-\infty $ et  $\lim\limits_{x\rightarrow +\infty }f\left(x\right)=1$.
     </p>
     <p>
     Par conséquent, l'équation $f\left(x\right)=-1$ admet une \textbf{unique} solution sur l'intervalle  $\left]0 ; +\infty \right[$.
     </p>
}
\begin{h2}3. Calcul de dérivées\end{h2}
<p>
Le tableau ci-dessous recense les dérivées usuelles à connaitre en Terminale S. Pour faciliter les révisions, toutes les formules du programme ont été recensées mais si certaines seront étudiées dans des chapitres ultérieurs.
</p>
\cadre{vert}{Dérivée des fonctions usuelles }{% id="p80"
     <p>
     \begin{tabularx}{0.8\linewidth}{|*{3}{>{\centering \arraybackslash }X|}}%class="compact" width="600"
          \hline
          \par
          <tr><td>\textbf{Fonction} & \textbf{Dérivée} & \textbf{Ensemble de dérivabilité}
          \\ \hline
          $k$  $\left(k\in \mathbb{R}\right)$  &  $0$  &  $\mathbb{R}$
          \par
          \\ \hline
          $x$ &  $1$  &  $\mathbb{R}$
          \par
          \\ \hline
          $x^{n}$ $\left(n\in \mathbb{N}\right)$  &  $nx^{n-1}$  &  $\mathbb{R}$
          \par
          \\ \hline
          $\frac{1}{x^{n}}$ $\left(n\in \mathbb{N}\right)$ &  $-\frac{n}{x^{n+1}}$  &  $\mathbb{R}-\left\{0\right\}$
          \par
          \\ \hline
          $\sqrt{x}$ &  $\frac{1}{2\sqrt{x}}$  &  $\left]0;+\infty \right[$
          \par
          \\ \hline
          $\sin\left(x\right)$ &  $\cos\left(x\right)$  &  $\mathbb{R}$
          \par
          \\ \hline
          $\cos\left(x\right)$ &  $-\sin\left(x\right)$  &  $\mathbb{R}$
          \par
          \\ \hline
          $e^{x}$ &  $e^{x}$  &  $\mathbb{R}$
          \par
          \\ \hline
          $\ln\left(x\right)$ &  $\frac{1}{x}$  &  $\left]0;-\infty \right[$
          \par
     </td></tr>\end{tabularx}
     </p>
}
\cadre{vert}{Propriété}{% id="p90"
     <p>
     Soient une fonction $f$ définie et dérivable sur un certain intervalle et $a$ et $b$ deux réels.
     </p>
     <p>
     Alors la fonction $g  : x\mapsto f\left(ax+b\right)$ est dérivable là où elle est définie et :
</p><div class="center"><p>$g^{\prime}\left(x\right)=af^{\prime}\left(ax+b\right)$</p>}
}
\bloc{orange}{Exemples}{% id="e90"
     \begin{itemize}
          \item <p class="item"> La fonction $f  :  x\mapsto \left(5x+2\right)^{3}$ est définie et dérivable sur $\mathbb{R}$ et :
          </p>
          <p>
          $f^{\prime}\left(x\right)=5\times 3\left(5x+2\right)^{2}=15\left(5x+2\right)^{2}$
          </p>
          \item <p class="item"> En particulier, si $g\left(x\right)=f\left(-x\right)$ on a $g^{\prime}\left(x\right)=-f^{\prime}\left(-x\right)$.
          </p>
          <p>
          Par exemple la dérivée de la fonction $x\mapsto e^{-x}$ est la fonction $x\mapsto -e^{-x}$
          </p>
     \end{itemize}
}
\bloc{cyan}{Remarque}{% id="r90"
     <p>
     Le résultat précédent se généralise à l'aide du théorème suivant :
     </p>
}
\cadre{rouge}{Théorème (dérivées des fonctions composées)}{% id="t100"
     <p>
     Soit $u$ une fonction dérivable sur un intervalle $I$ et  prenant ses valeurs dans un intervalle $J$ et soit  $f$ une fonction dérivable sur $J$.
     </p>
     <p>
     Alors la fonction $g  :  x\mapsto f\left(u\left(x\right)\right)$ est dérivable sur $I$ et :
</p><div class="center"><p>$g^{\prime}\left(x\right)=u^{\prime}\left(x\right)\times f^{\prime}\left(u\left(x\right)\right)$</p>}
}
\bloc{orange}{Exemples}{% id="e100"
     <p>
     Soit $u$ une fonction dérivable sur intervalle $I$ :
     </p>
     \begin{itemize}
          \item <p class="item"> la fonction $u^{n}$ est dérivable sur $I$ et sa dérivée est $u^{\prime}\times nu^{n-1}$
          </p>
          \item <p class="item"> la fonction $\frac{1}{u}$ est dérivable sur la partie de $I$ où $u\neq 0$ et sa dérivée est $-\frac{u^{\prime}}{u^{2}}$
          </p>
          \item <p class="item"> la fonction $\sqrt{u}$ est dérivable sur la partie de $I$ où $u > 0$ et sa dérivée est $\frac{u^{\prime}}{2\sqrt{u}}$
          </p>
          \item <p class="item"> la fonction $\sin\left(u\right)$ est dérivable sur $I$ et sa dérivée est $u^{\prime}\times \cos\left(u\right)$
          </p>
          \item <p class="item"> la fonction $\cos\left(u\right)$ est dérivable sur $I$ et sa dérivée est $-u^{\prime}\times \sin\left(u\right)$
          </p>
          \item <p class="item"> la fonction $e^{u}$ est dérivable sur $I$ et sa dérivée est $u^{\prime}\times e^{u}$
          </p>
          \item <p class="item"> la fonction $\ln\left(u\right)$ est dérivable sur la partie de $I$ où $u > 0$ et sa dérivée est $\frac{u^{\prime}}{u}$
          </p>
     \end{itemize}
}

\end{document}