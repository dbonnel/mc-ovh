\documentclass[a4paper]{article}

%================================================================================================================================
%
% Packages
%
%================================================================================================================================

\usepackage[T1]{fontenc} 	% pour caractères accentués
\usepackage[utf8]{inputenc}  % encodage utf8
\usepackage[french]{babel}	% langue : français
\usepackage{fourier}			% caractères plus lisibles
\usepackage[dvipsnames]{xcolor} % couleurs
\usepackage{fancyhdr}		% réglage header footer
\usepackage{needspace}		% empêcher sauts de page mal placés
\usepackage{graphicx}		% pour inclure des graphiques
\usepackage{enumitem,cprotect}		% personnalise les listes d'items (nécessaire pour ol, al ...)
\usepackage{hyperref}		% Liens hypertexte
\usepackage{pstricks,pst-all,pst-node,pstricks-add,pst-math,pst-plot,pst-tree,pst-eucl} % pstricks
\usepackage[a4paper,includeheadfoot,top=2cm,left=3cm, bottom=2cm,right=3cm]{geometry} % marges etc.
\usepackage{comment}			% commentaires multilignes
\usepackage{amsmath,environ} % maths (matrices, etc.)
\usepackage{amssymb,makeidx}
\usepackage{bm}				% bold maths
\usepackage{tabularx}		% tableaux
\usepackage{colortbl}		% tableaux en couleur
\usepackage{fontawesome}		% Fontawesome
\usepackage{environ}			% environment with command
\usepackage{fp}				% calculs pour ps-tricks
\usepackage{multido}			% pour ps tricks
\usepackage[np]{numprint}	% formattage nombre
\usepackage{tikz,tkz-tab} 			% package principal TikZ
\usepackage{pgfplots}   % axes
\usepackage{mathrsfs}    % cursives
\usepackage{calc}			% calcul taille boites
\usepackage[scaled=0.875]{helvet} % font sans serif
\usepackage{svg} % svg
\usepackage{scrextend} % local margin
\usepackage{scratch} %scratch
\usepackage{multicol} % colonnes
%\usepackage{infix-RPN,pst-func} % formule en notation polanaise inversée
\usepackage{listings}

%================================================================================================================================
%
% Réglages de base
%
%================================================================================================================================

\lstset{
language=Python,   % R code
literate=
{á}{{\'a}}1
{à}{{\`a}}1
{ã}{{\~a}}1
{é}{{\'e}}1
{è}{{\`e}}1
{ê}{{\^e}}1
{í}{{\'i}}1
{ó}{{\'o}}1
{õ}{{\~o}}1
{ú}{{\'u}}1
{ü}{{\"u}}1
{ç}{{\c{c}}}1
{~}{{ }}1
}


\definecolor{codegreen}{rgb}{0,0.6,0}
\definecolor{codegray}{rgb}{0.5,0.5,0.5}
\definecolor{codepurple}{rgb}{0.58,0,0.82}
\definecolor{backcolour}{rgb}{0.95,0.95,0.92}

\lstdefinestyle{mystyle}{
    backgroundcolor=\color{backcolour},   
    commentstyle=\color{codegreen},
    keywordstyle=\color{magenta},
    numberstyle=\tiny\color{codegray},
    stringstyle=\color{codepurple},
    basicstyle=\ttfamily\footnotesize,
    breakatwhitespace=false,         
    breaklines=true,                 
    captionpos=b,                    
    keepspaces=true,                 
    numbers=left,                    
xleftmargin=2em,
framexleftmargin=2em,            
    showspaces=false,                
    showstringspaces=false,
    showtabs=false,                  
    tabsize=2,
    upquote=true
}

\lstset{style=mystyle}


\lstset{style=mystyle}
\newcommand{\imgdir}{C:/laragon/www/newmc/assets/imgsvg/}
\newcommand{\imgsvgdir}{C:/laragon/www/newmc/assets/imgsvg/}

\definecolor{mcgris}{RGB}{220, 220, 220}% ancien~; pour compatibilité
\definecolor{mcbleu}{RGB}{52, 152, 219}
\definecolor{mcvert}{RGB}{125, 194, 70}
\definecolor{mcmauve}{RGB}{154, 0, 215}
\definecolor{mcorange}{RGB}{255, 96, 0}
\definecolor{mcturquoise}{RGB}{0, 153, 153}
\definecolor{mcrouge}{RGB}{255, 0, 0}
\definecolor{mclightvert}{RGB}{205, 234, 190}

\definecolor{gris}{RGB}{220, 220, 220}
\definecolor{bleu}{RGB}{52, 152, 219}
\definecolor{vert}{RGB}{125, 194, 70}
\definecolor{mauve}{RGB}{154, 0, 215}
\definecolor{orange}{RGB}{255, 96, 0}
\definecolor{turquoise}{RGB}{0, 153, 153}
\definecolor{rouge}{RGB}{255, 0, 0}
\definecolor{lightvert}{RGB}{205, 234, 190}
\setitemize[0]{label=\color{lightvert}  $\bullet$}

\pagestyle{fancy}
\renewcommand{\headrulewidth}{0.2pt}
\fancyhead[L]{maths-cours.fr}
\fancyhead[R]{\thepage}
\renewcommand{\footrulewidth}{0.2pt}
\fancyfoot[C]{}

\newcolumntype{C}{>{\centering\arraybackslash}X}
\newcolumntype{s}{>{\hsize=.35\hsize\arraybackslash}X}

\setlength{\parindent}{0pt}		 
\setlength{\parskip}{3mm}
\setlength{\headheight}{1cm}

\def\ebook{ebook}
\def\book{book}
\def\web{web}
\def\type{web}

\newcommand{\vect}[1]{\overrightarrow{\,\mathstrut#1\,}}

\def\Oij{$\left(\text{O}~;~\vect{\imath},~\vect{\jmath}\right)$}
\def\Oijk{$\left(\text{O}~;~\vect{\imath},~\vect{\jmath},~\vect{k}\right)$}
\def\Ouv{$\left(\text{O}~;~\vect{u},~\vect{v}\right)$}

\hypersetup{breaklinks=true, colorlinks = true, linkcolor = OliveGreen, urlcolor = OliveGreen, citecolor = OliveGreen, pdfauthor={Didier BONNEL - https://www.maths-cours.fr} } % supprime les bordures autour des liens

\renewcommand{\arg}[0]{\text{arg}}

\everymath{\displaystyle}

%================================================================================================================================
%
% Macros - Commandes
%
%================================================================================================================================

\newcommand\meta[2]{    			% Utilisé pour créer le post HTML.
	\def\titre{titre}
	\def\url{url}
	\def\arg{#1}
	\ifx\titre\arg
		\newcommand\maintitle{#2}
		\fancyhead[L]{#2}
		{\Large\sffamily \MakeUppercase{#2}}
		\vspace{1mm}\textcolor{mcvert}{\hrule}
	\fi 
	\ifx\url\arg
		\fancyfoot[L]{\href{https://www.maths-cours.fr#2}{\black \footnotesize{https://www.maths-cours.fr#2}}}
	\fi 
}


\newcommand\TitreC[1]{    		% Titre centré
     \needspace{3\baselineskip}
     \begin{center}\textbf{#1}\end{center}
}

\newcommand\newpar{    		% paragraphe
     \par
}

\newcommand\nosp {    		% commande vide (pas d'espace)
}
\newcommand{\id}[1]{} %ignore

\newcommand\boite[2]{				% Boite simple sans titre
	\vspace{5mm}
	\setlength{\fboxrule}{0.2mm}
	\setlength{\fboxsep}{5mm}	
	\fcolorbox{#1}{#1!3}{\makebox[\linewidth-2\fboxrule-2\fboxsep]{
  		\begin{minipage}[t]{\linewidth-2\fboxrule-4\fboxsep}\setlength{\parskip}{3mm}
  			 #2
  		\end{minipage}
	}}
	\vspace{5mm}
}

\newcommand\CBox[4]{				% Boites
	\vspace{5mm}
	\setlength{\fboxrule}{0.2mm}
	\setlength{\fboxsep}{5mm}
	
	\fcolorbox{#1}{#1!3}{\makebox[\linewidth-2\fboxrule-2\fboxsep]{
		\begin{minipage}[t]{1cm}\setlength{\parskip}{3mm}
	  		\textcolor{#1}{\LARGE{#2}}    
 	 	\end{minipage}  
  		\begin{minipage}[t]{\linewidth-2\fboxrule-4\fboxsep}\setlength{\parskip}{3mm}
			\raisebox{1.2mm}{\normalsize\sffamily{\textcolor{#1}{#3}}}						
  			 #4
  		\end{minipage}
	}}
	\vspace{5mm}
}

\newcommand\cadre[3]{				% Boites convertible html
	\par
	\vspace{2mm}
	\setlength{\fboxrule}{0.1mm}
	\setlength{\fboxsep}{5mm}
	\fcolorbox{#1}{white}{\makebox[\linewidth-2\fboxrule-2\fboxsep]{
  		\begin{minipage}[t]{\linewidth-2\fboxrule-4\fboxsep}\setlength{\parskip}{3mm}
			\raisebox{-2.5mm}{\sffamily \small{\textcolor{#1}{\MakeUppercase{#2}}}}		
			\par		
  			 #3
 	 		\end{minipage}
	}}
		\vspace{2mm}
	\par
}

\newcommand\bloc[3]{				% Boites convertible html sans bordure
     \needspace{2\baselineskip}
     {\sffamily \small{\textcolor{#1}{\MakeUppercase{#2}}}}    
		\par		
  			 #3
		\par
}

\newcommand\CHelp[1]{
     \CBox{Plum}{\faInfoCircle}{À RETENIR}{#1}
}

\newcommand\CUp[1]{
     \CBox{NavyBlue}{\faThumbsOUp}{EN PRATIQUE}{#1}
}

\newcommand\CInfo[1]{
     \CBox{Sepia}{\faArrowCircleRight}{REMARQUE}{#1}
}

\newcommand\CRedac[1]{
     \CBox{PineGreen}{\faEdit}{BIEN R\'EDIGER}{#1}
}

\newcommand\CError[1]{
     \CBox{Red}{\faExclamationTriangle}{ATTENTION}{#1}
}

\newcommand\TitreExo[2]{
\needspace{4\baselineskip}
 {\sffamily\large EXERCICE #1\ (\emph{#2 points})}
\vspace{5mm}
}

\newcommand\img[2]{
          \includegraphics[width=#2\paperwidth]{\imgdir#1}
}

\newcommand\imgsvg[2]{
       \begin{center}   \includegraphics[width=#2\paperwidth]{\imgsvgdir#1} \end{center}
}


\newcommand\Lien[2]{
     \href{#1}{#2 \tiny \faExternalLink}
}
\newcommand\mcLien[2]{
     \href{https~://www.maths-cours.fr/#1}{#2 \tiny \faExternalLink}
}

\newcommand{\euro}{\eurologo{}}

%================================================================================================================================
%
% Macros - Environement
%
%================================================================================================================================

\newenvironment{tex}{ %
}
{%
}

\newenvironment{indente}{ %
	\setlength\parindent{10mm}
}

{
	\setlength\parindent{0mm}
}

\newenvironment{corrige}{%
     \needspace{3\baselineskip}
     \medskip
     \textbf{\textsc{Corrigé}}
     \medskip
}
{
}

\newenvironment{extern}{%
     \begin{center}
     }
     {
     \end{center}
}

\NewEnviron{code}{%
	\par
     \boite{gray}{\texttt{%
     \BODY
     }}
     \par
}

\newenvironment{vbloc}{% boite sans cadre empeche saut de page
     \begin{minipage}[t]{\linewidth}
     }
     {
     \end{minipage}
}
\NewEnviron{h2}{%
    \needspace{3\baselineskip}
    \vspace{0.6cm}
	\noindent \MakeUppercase{\sffamily \large \BODY}
	\vspace{1mm}\textcolor{mcgris}{\hrule}\vspace{0.4cm}
	\par
}{}

\NewEnviron{h3}{%
    \needspace{3\baselineskip}
	\vspace{5mm}
	\textsc{\BODY}
	\par
}

\NewEnviron{margeneg}{ %
\begin{addmargin}[-1cm]{0cm}
\BODY
\end{addmargin}
}

\NewEnviron{html}{%
}

\begin{document}
\begin{h2}1. Définition de la fonction logarithme népérien\end{h2}
\cadre{bleu}{Théorème et définition}{%id="t10"
     Pour tout réel $x>0$, l'équation $e^{y}=x$, d'inconnue $y$, admet une \textbf{unique} solution.
     \par
     La fonction \textbf{logarithme népérien}, notée $\ln$, est la fonction définie sur $\left]0;+\infty \right[$ qui à $x > 0$, associe le réel $y$ solution de l'équation $e^{y}=x$.
}
\bloc{cyan}{Remarques}{%id="r10"
     \begin{itemize}
          \item Pour $x\leqslant 0$, par contre, l'équation $e^{y}=x$ n'a \textbf{pas de solution}.
     \end{itemize}
}
\cadre{vert}{Propriétés}{%id="p20"
     \begin{itemize}
          \item Pour tout réel $x > 0$ et tout $y \in \mathbb{R}$ : $ e^{y}=x  \Leftrightarrow y=\ln\left(x\right)$.
          \item Pour tout réel $x > 0$ : $e^{\ln\left(x\right)}=x$.
          \item Pour tout réel $x$ : $\ln\left(e^{x}\right)=x$.
     \end{itemize}
}
\bloc{cyan}{Remarques}{%id="r0"
     \begin{itemize}
          \item Ces propriétés se déduisent immédiatement de la définition.
          \item On dit que les fonctions «logarithme népérien» et «exponentielle» sont \textit{réciproques}.
          \item On en déduit immédiatement : $\ln\left(1\right)=0$ et $\ln\left(e\right)=1$.
     \end{itemize}
}
\begin{h2}2. Etude de la fonction logarithme népérien\end{h2}
\cadre{rouge}{Théorème}{%id="t30"
     La fonction logarithme népérien est dérivable sur $\left]0 ;+\infty \right[$ et sa dérivée est définie par :
     \begin{center}$\ln^{\prime}\left(x\right)=\frac{1}{x}.$\end{center}
}
\cadre{vert}{Propriété}{%id="p40"
     La fonction logarithme népérien est \textbf{strictement croissante} sur $\left]0;+\infty \right[$.
}
\bloc{cyan}{Démonstration}{%id="r40"
     Sa dérivée $\ln^{\prime}\left(x\right)=\frac{1}{x}$ est strictement positive sur $\left]0;+\infty \right[$.
}
\bloc{cyan}{Remarques}{%id="r40"
     \begin{itemize}
          \item Ces résultats permettent de tracer le tableau de variation et la courbe représentative de la fonction logarithme népérien :
     \end{itemize}
}
\begin{center}
     \begin{extern}%width="600" alt="Tableau de variation de la fonction logarithme népérien"
          \tikzset{double style/.style = {double,double distance=2pt}}
          \begin{tikzpicture}
               \tkzTabInit[lgt=3,espcl=10] {$x$ /1, $f'(x)=\dfrac{1}{x}$ /1.5,%
               $f(x)=\ln x$/2} {$0$ , $+\infty$}%
               \tkzTabLine{d,+,}%
               \tkzTabVar{ D- / $-\infty$, + / $+\infty$ }
               \tkzTabVal[draw]{1}{2}{0.33}{1}{0}
               \tkzTabVal[draw]{1}{2}{0.66}{e}{1}
          \end{tikzpicture}
     \end{extern}
\end{center}
\begin{center}
     \textit{Tableau de variation de la fonction logarithme népérien }
\end{center}
\par
\begin{center}
     \begin{extern} %width="400" alt="Représentation graphique de la fonction logarithme népérien"
          \resizebox{8cm}{!}{%
               % -+-+-+ variables modifiables
               \def\fonction{x ln }
               \def\xmin{-0.5}
               \def\xmax{5.5}
               \def\ymin{-3.5}
               \def\ymax{3}
               \def\xunit{2}  % unités en cm
               \def\yunit{2}
               \psset{xunit=\xunit,yunit=\yunit}
               \fontsize{15pt}{15pt}\selectfont
               \begin{pspicture*}[linewidth=1pt](\xmin,\ymin)(\xmax,\ymax)
                    %      \psgrid[gridcolor=mcgris, subgriddiv=5, gridlabels=0pt](\xmin,\ymin)(\xmax,\ymax)
                    \psaxes[linewidth=0.75pt]{->}(0,0)(\xmin,\ymin)(\xmax,\ymax)
                    \psplot[plotpoints=2000,linecolor=blue]{0.01}{\xmax}{\fonction}
                    \rput[tr](-0.1,-0.1){$O$}
                    \rput[tl](4.8,2){$\color{blue} \mathcal{C}_{\ln}$}
                    \psline(0,1)(2.7183,1)(2.7183,0)
                    \rput[t](2.7183,-0.25){$\text{e}$}
               \end{pspicture*}
          }
     \end{extern}
\end{center}
\begin{center}
     \textit{Représentation graphique de la fonction logarithme népérien }
\end{center}
\cadre{vert}{Propriété}{%id="p45"
     Soit $u$ une fonction dérivable et \textbf{strictement positive} sur un intervalle $I$.
     \par
     Alors la fonction $ f : x\mapsto \ln\left(u\left(x\right)\right)$ est dérivable sur $I$ et :
     \begin{center}$f^{\prime}=\frac{u^{\prime}}{u}$.\end{center}
}
\bloc{orange}{Exemple}{%id="e45"
     Soit $f$ définie sur $\mathbb{R}$ par $f\left(x\right)=\ln\left(x^{2}+1\right)$.
     \par
     $f$ est dérivable sur $\mathbb{R}$ et $f^{\prime}\left(x\right)=\frac{2x}{x^{2}+1}$.
}
\cadre{rouge}{Théorème}{%id="t70"
     Si $a$ et $b$ sont 2 réels strictement positifs :
     \begin{itemize}
          \item $\ln a=\ln b$ si et seulement si $a=b$.
          \item $\ln a < \ln b$ si et seulement si $a < b$.
     \end{itemize}
}
\bloc{cyan}{Remarques}{%id="r70"
     \begin{itemize}
          \item Le théorème précédent résulte de la stricte croissance de la fonction logarithme népérien.
          \item En particulier, comme $\ln\left(1\right)=0$ : $\ln x < 0 \Leftrightarrow x < 1$. N'oubliez donc pas que \textbf{$\ln\left(x\right)$ peut être négatif} (si $0 < x < 1$); c'est une cause d'erreurs fréquente dans les exercices notamment avec des inéquations !
     \end{itemize}
}
\begin{h2}3. Propriétés algébriques de la fonction logarithme népérien\end{h2}
\cadre{rouge}{Théorème}{%id="80"
     Si $a$ et $b$ sont 2 réels strictement positifs et si $n \in \mathbb{Z}$ :
     \begin{itemize}
          \item $\ln\left(ab\right)=\ln a+\ln b$.
          \item $\ln\left(\frac{1}{a}\right)=-\ln a$.
          \item $\ln\left(\frac{a}{b}\right)=\ln a-\ln b$.
          \item $\ln\left(a^{n}\right)=n \ln a $.
          \item $\ln\left(\sqrt{a}\right)=\frac{1}{2} \ln a $.
     \end{itemize}
}
\bloc{orange}{Exemples}{%id="e80"
     \begin{itemize}
          \item $\ln\left(4\right)=\ln\left(2^{2}\right)=2\ln\left(2\right)$.
          \item Pour $x > 1$ : $\ln\left(\frac{x+1}{x-1}\right)= \ln\left(x+1\right)-\ln\left(x-1\right)$.
          \par
          Cette égalité peut être intéressante (pour calculer la dérivée par exemple) mais il faut que $x > 1$.
          \par
          Si $x < -1$, l'expression $\ln\left(\frac{x+1}{x-1}\right)$ est définie mais pas $\ln\left(x+1\right)-\ln\left(x-1\right)$.
     \end{itemize}
}

\end{document}