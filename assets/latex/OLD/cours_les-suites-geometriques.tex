\documentclass[a4paper]{article}

%================================================================================================================================
%
% Packages
%
%================================================================================================================================

\usepackage[T1]{fontenc} 	% pour caractères accentués
\usepackage[utf8]{inputenc}  % encodage utf8
\usepackage[french]{babel}	% langue : français
\usepackage{fourier}			% caractères plus lisibles
\usepackage[dvipsnames]{xcolor} % couleurs
\usepackage{fancyhdr}		% réglage header footer
\usepackage{needspace}		% empêcher sauts de page mal placés
\usepackage{graphicx}		% pour inclure des graphiques
\usepackage{enumitem,cprotect}		% personnalise les listes d'items (nécessaire pour ol, al ...)
\usepackage{hyperref}		% Liens hypertexte
\usepackage{pstricks,pst-all,pst-node,pstricks-add,pst-math,pst-plot,pst-tree,pst-eucl} % pstricks
\usepackage[a4paper,includeheadfoot,top=2cm,left=3cm, bottom=2cm,right=3cm]{geometry} % marges etc.
\usepackage{comment}			% commentaires multilignes
\usepackage{amsmath,environ} % maths (matrices, etc.)
\usepackage{amssymb,makeidx}
\usepackage{bm}				% bold maths
\usepackage{tabularx}		% tableaux
\usepackage{colortbl}		% tableaux en couleur
\usepackage{fontawesome}		% Fontawesome
\usepackage{environ}			% environment with command
\usepackage{fp}				% calculs pour ps-tricks
\usepackage{multido}			% pour ps tricks
\usepackage[np]{numprint}	% formattage nombre
\usepackage{tikz,tkz-tab} 			% package principal TikZ
\usepackage{pgfplots}   % axes
\usepackage{mathrsfs}    % cursives
\usepackage{calc}			% calcul taille boites
\usepackage[scaled=0.875]{helvet} % font sans serif
\usepackage{svg} % svg
\usepackage{scrextend} % local margin
\usepackage{scratch} %scratch
\usepackage{multicol} % colonnes
%\usepackage{infix-RPN,pst-func} % formule en notation polanaise inversée
\usepackage{listings}

%================================================================================================================================
%
% Réglages de base
%
%================================================================================================================================

\lstset{
language=Python,   % R code
literate=
{á}{{\'a}}1
{à}{{\`a}}1
{ã}{{\~a}}1
{é}{{\'e}}1
{è}{{\`e}}1
{ê}{{\^e}}1
{í}{{\'i}}1
{ó}{{\'o}}1
{õ}{{\~o}}1
{ú}{{\'u}}1
{ü}{{\"u}}1
{ç}{{\c{c}}}1
{~}{{ }}1
}


\definecolor{codegreen}{rgb}{0,0.6,0}
\definecolor{codegray}{rgb}{0.5,0.5,0.5}
\definecolor{codepurple}{rgb}{0.58,0,0.82}
\definecolor{backcolour}{rgb}{0.95,0.95,0.92}

\lstdefinestyle{mystyle}{
    backgroundcolor=\color{backcolour},   
    commentstyle=\color{codegreen},
    keywordstyle=\color{magenta},
    numberstyle=\tiny\color{codegray},
    stringstyle=\color{codepurple},
    basicstyle=\ttfamily\footnotesize,
    breakatwhitespace=false,         
    breaklines=true,                 
    captionpos=b,                    
    keepspaces=true,                 
    numbers=left,                    
xleftmargin=2em,
framexleftmargin=2em,            
    showspaces=false,                
    showstringspaces=false,
    showtabs=false,                  
    tabsize=2,
    upquote=true
}

\lstset{style=mystyle}


\lstset{style=mystyle}
\newcommand{\imgdir}{C:/laragon/www/newmc/assets/imgsvg/}
\newcommand{\imgsvgdir}{C:/laragon/www/newmc/assets/imgsvg/}

\definecolor{mcgris}{RGB}{220, 220, 220}% ancien~; pour compatibilité
\definecolor{mcbleu}{RGB}{52, 152, 219}
\definecolor{mcvert}{RGB}{125, 194, 70}
\definecolor{mcmauve}{RGB}{154, 0, 215}
\definecolor{mcorange}{RGB}{255, 96, 0}
\definecolor{mcturquoise}{RGB}{0, 153, 153}
\definecolor{mcrouge}{RGB}{255, 0, 0}
\definecolor{mclightvert}{RGB}{205, 234, 190}

\definecolor{gris}{RGB}{220, 220, 220}
\definecolor{bleu}{RGB}{52, 152, 219}
\definecolor{vert}{RGB}{125, 194, 70}
\definecolor{mauve}{RGB}{154, 0, 215}
\definecolor{orange}{RGB}{255, 96, 0}
\definecolor{turquoise}{RGB}{0, 153, 153}
\definecolor{rouge}{RGB}{255, 0, 0}
\definecolor{lightvert}{RGB}{205, 234, 190}
\setitemize[0]{label=\color{lightvert}  $\bullet$}

\pagestyle{fancy}
\renewcommand{\headrulewidth}{0.2pt}
\fancyhead[L]{maths-cours.fr}
\fancyhead[R]{\thepage}
\renewcommand{\footrulewidth}{0.2pt}
\fancyfoot[C]{}

\newcolumntype{C}{>{\centering\arraybackslash}X}
\newcolumntype{s}{>{\hsize=.35\hsize\arraybackslash}X}

\setlength{\parindent}{0pt}		 
\setlength{\parskip}{3mm}
\setlength{\headheight}{1cm}

\def\ebook{ebook}
\def\book{book}
\def\web{web}
\def\type{web}

\newcommand{\vect}[1]{\overrightarrow{\,\mathstrut#1\,}}

\def\Oij{$\left(\text{O}~;~\vect{\imath},~\vect{\jmath}\right)$}
\def\Oijk{$\left(\text{O}~;~\vect{\imath},~\vect{\jmath},~\vect{k}\right)$}
\def\Ouv{$\left(\text{O}~;~\vect{u},~\vect{v}\right)$}

\hypersetup{breaklinks=true, colorlinks = true, linkcolor = OliveGreen, urlcolor = OliveGreen, citecolor = OliveGreen, pdfauthor={Didier BONNEL - https://www.maths-cours.fr} } % supprime les bordures autour des liens

\renewcommand{\arg}[0]{\text{arg}}

\everymath{\displaystyle}

%================================================================================================================================
%
% Macros - Commandes
%
%================================================================================================================================

\newcommand\meta[2]{    			% Utilisé pour créer le post HTML.
	\def\titre{titre}
	\def\url{url}
	\def\arg{#1}
	\ifx\titre\arg
		\newcommand\maintitle{#2}
		\fancyhead[L]{#2}
		{\Large\sffamily \MakeUppercase{#2}}
		\vspace{1mm}\textcolor{mcvert}{\hrule}
	\fi 
	\ifx\url\arg
		\fancyfoot[L]{\href{https://www.maths-cours.fr#2}{\black \footnotesize{https://www.maths-cours.fr#2}}}
	\fi 
}


\newcommand\TitreC[1]{    		% Titre centré
     \needspace{3\baselineskip}
     \begin{center}\textbf{#1}\end{center}
}

\newcommand\newpar{    		% paragraphe
     \par
}

\newcommand\nosp {    		% commande vide (pas d'espace)
}
\newcommand{\id}[1]{} %ignore

\newcommand\boite[2]{				% Boite simple sans titre
	\vspace{5mm}
	\setlength{\fboxrule}{0.2mm}
	\setlength{\fboxsep}{5mm}	
	\fcolorbox{#1}{#1!3}{\makebox[\linewidth-2\fboxrule-2\fboxsep]{
  		\begin{minipage}[t]{\linewidth-2\fboxrule-4\fboxsep}\setlength{\parskip}{3mm}
  			 #2
  		\end{minipage}
	}}
	\vspace{5mm}
}

\newcommand\CBox[4]{				% Boites
	\vspace{5mm}
	\setlength{\fboxrule}{0.2mm}
	\setlength{\fboxsep}{5mm}
	
	\fcolorbox{#1}{#1!3}{\makebox[\linewidth-2\fboxrule-2\fboxsep]{
		\begin{minipage}[t]{1cm}\setlength{\parskip}{3mm}
	  		\textcolor{#1}{\LARGE{#2}}    
 	 	\end{minipage}  
  		\begin{minipage}[t]{\linewidth-2\fboxrule-4\fboxsep}\setlength{\parskip}{3mm}
			\raisebox{1.2mm}{\normalsize\sffamily{\textcolor{#1}{#3}}}						
  			 #4
  		\end{minipage}
	}}
	\vspace{5mm}
}

\newcommand\cadre[3]{				% Boites convertible html
	\par
	\vspace{2mm}
	\setlength{\fboxrule}{0.1mm}
	\setlength{\fboxsep}{5mm}
	\fcolorbox{#1}{white}{\makebox[\linewidth-2\fboxrule-2\fboxsep]{
  		\begin{minipage}[t]{\linewidth-2\fboxrule-4\fboxsep}\setlength{\parskip}{3mm}
			\raisebox{-2.5mm}{\sffamily \small{\textcolor{#1}{\MakeUppercase{#2}}}}		
			\par		
  			 #3
 	 		\end{minipage}
	}}
		\vspace{2mm}
	\par
}

\newcommand\bloc[3]{				% Boites convertible html sans bordure
     \needspace{2\baselineskip}
     {\sffamily \small{\textcolor{#1}{\MakeUppercase{#2}}}}    
		\par		
  			 #3
		\par
}

\newcommand\CHelp[1]{
     \CBox{Plum}{\faInfoCircle}{À RETENIR}{#1}
}

\newcommand\CUp[1]{
     \CBox{NavyBlue}{\faThumbsOUp}{EN PRATIQUE}{#1}
}

\newcommand\CInfo[1]{
     \CBox{Sepia}{\faArrowCircleRight}{REMARQUE}{#1}
}

\newcommand\CRedac[1]{
     \CBox{PineGreen}{\faEdit}{BIEN R\'EDIGER}{#1}
}

\newcommand\CError[1]{
     \CBox{Red}{\faExclamationTriangle}{ATTENTION}{#1}
}

\newcommand\TitreExo[2]{
\needspace{4\baselineskip}
 {\sffamily\large EXERCICE #1\ (\emph{#2 points})}
\vspace{5mm}
}

\newcommand\img[2]{
          \includegraphics[width=#2\paperwidth]{\imgdir#1}
}

\newcommand\imgsvg[2]{
       \begin{center}   \includegraphics[width=#2\paperwidth]{\imgsvgdir#1} \end{center}
}


\newcommand\Lien[2]{
     \href{#1}{#2 \tiny \faExternalLink}
}
\newcommand\mcLien[2]{
     \href{https~://www.maths-cours.fr/#1}{#2 \tiny \faExternalLink}
}

\newcommand{\euro}{\eurologo{}}

%================================================================================================================================
%
% Macros - Environement
%
%================================================================================================================================

\newenvironment{tex}{ %
}
{%
}

\newenvironment{indente}{ %
	\setlength\parindent{10mm}
}

{
	\setlength\parindent{0mm}
}

\newenvironment{corrige}{%
     \needspace{3\baselineskip}
     \medskip
     \textbf{\textsc{Corrigé}}
     \medskip
}
{
}

\newenvironment{extern}{%
     \begin{center}
     }
     {
     \end{center}
}

\NewEnviron{code}{%
	\par
     \boite{gray}{\texttt{%
     \BODY
     }}
     \par
}

\newenvironment{vbloc}{% boite sans cadre empeche saut de page
     \begin{minipage}[t]{\linewidth}
     }
     {
     \end{minipage}
}
\NewEnviron{h2}{%
    \needspace{3\baselineskip}
    \vspace{0.6cm}
	\noindent \MakeUppercase{\sffamily \large \BODY}
	\vspace{1mm}\textcolor{mcgris}{\hrule}\vspace{0.4cm}
	\par
}{}

\NewEnviron{h3}{%
    \needspace{3\baselineskip}
	\vspace{5mm}
	\textsc{\BODY}
	\par
}

\NewEnviron{margeneg}{ %
\begin{addmargin}[-1cm]{0cm}
\BODY
\end{addmargin}
}

\NewEnviron{html}{%
}

\begin{document}
\begin{h2}1 - Caractéristiques d'une suite géométrique\end{h2}
\cadre{bleu}{Définition}{% id="d10"
     On dit qu'une suite $\left(u_{n}\right)_{n\in \mathbb{N}}$ est une \textbf{suite géométrique} s'il existe un nombre réel $q$ tel que :
     \par
     pour tout $n\in \mathbb{N}$,  $u_{n+1}=q \times  u_{n}$
     \par
     Le réel $q$ s'appelle la \textbf{raison} de la suite géométrique $\left(u_{n}\right)$.
}
\bloc{cyan}{Remarque}{% id="r10"
     Pour démontrer qu'une suite $\left(u_{n}\right)_{n\in \mathbb{N}}$ dont les termes sont non nuls est une suite géométrique, on pourra calculer le rapport $\frac{u_{n+1}}{u_{n}}$.
     \par
     Si ce rapport est une constante $q$, on pourra affirmer que la suite est une suite géométrique de raison $q$.
}
\bloc{orange}{Exemple}{% id="e10"
     Soit la suite $\left(u_{n}\right)_{n\in \mathbb{N}}$ définie par $u_{n}=\frac{3}{2^{n}}$.
     \par
     Les termes de la suite sont tous strictement positifs et
     \par
     $\frac{u_{n+1}}{u_{n}}=$$\frac{3}{2^{n+1}}\times \frac{2^{n}}{3}=\frac{2^{n}}{2^{n+1}}=$$\frac{2^{n}}{2\times 2^{n}}=\frac{1}{2}$
     \par
     La suite $\left(u_{n}\right)$ est une suite géométrique de raison $\frac{1}{2}$
}
\cadre{vert}{Propriété}{% id="p20"
     Pour $n$ et $k$ quelconques entiers naturels, si la suite $\left(u_{n}\right)$ est géométrique de raison $q$ :$u_{n}=u_{k}\times q^{n-k}$.
     \par
     En particulier $u_{n}=u_{0}\times q^{n}$.
}
\cadre{vert}{Propriété}{% id="p30"
     Réciproquement, soient $a$ et $b$ deux nombres réels. La suite $\left(u_{n}\right)$ définie par $u_{n}=a\times b^{n}$ suite est une suite géométrique de raison $q=b$ et de premier terme $u_{0}=a$.
}
\bloc{cyan}{Démonstration}{% id="m30"
     $u_{n+1}=a\times b^{n+1}=a\times b^{n}\times b=u_{n}\times b$
     \par
     et
     \par
     $u_{0}=a\times b^{0}=a\times 1=a$
}
\cadre{rouge}{Théorème}{% id="t40"
     Soit $\left(u_{n}\right) $une suite géométrique de raison $q  > 0$ et de premier terme strictement positif :
     \begin{itemize}
          \item Si q >1, la suite $\left(u_{n}\right) $est strictement croissante
          \item Si 0 < q <1, la suite $\left(u_{n}\right) $est strictement décroissante
          \item Si q=1, la suite $\left(u_{n}\right) $est constante
     \end{itemize}
}
\cadre{rouge}{Théorème}{% id="t50"
     Si $\left(u_{n}\right)$ et $\left(v_{n}\right)$ sont deux suites géométriques de raison respectives $q$ et $q^{\prime}$ alors le produit $\left(w_{n}\right)$ de ces deux suites défini par :
     \par
     $w_{n}=u_{n}\times v_{n}$
     \par
     est une suite géométrique de raison $q^{\prime\prime}=q\times q^{\prime}$
}
\begin{h2}2 - Somme des puissances successives d'un nombre\end{h2}
\cadre{rouge}{Théorème}{% id="t70"
     Soit $q$ un nombre réel \textbf{différent de 1}:
     \begin{center}
     $1+q+q^{2}+ . . . +q^{n} = \frac{1-q^{n+1}}{1-q}$
     \end{center}
}
\bloc{cyan}{Remarque}{% id="r70"
     Cette formule n'est pas valable pour $q=1$. Mais dans ce cas le calcul est immédiat car tous les termes sont égaux à 1.
}
\bloc{orange}{Exemple}{% id="e70"
     Soit à calculer la somme $S=1+2+4+8+16 + . . .+2^{n}$
     \par
     Donc:
     \par
     $S=\frac{1-2^{n+1}}{1-2}=\frac{1-2^{n+1}}{-1}=2^{n+1}-1$
}
\begin{h2}3 - Limite de la suite $\left(q^{n}\right)$ où $q\geqslant 0$\end{h2}
\cadre{rouge}{Théorème}{% id="t90"
     Soit $q$ un nombre réel positif.
     \begin{itemize}
          \item \textbf{Si $q > 1$ :} alors $q^{n}$ est aussi grand que l'on veut dès que $n$ est suffisamment grand. On dit que la suite $\left(q^{n}\right)$ tend vers $+\infty $ et on écrit :
          \begin{center}$\lim\limits_{n\rightarrow +\infty }  q^{n} = +\infty  $ ( ou $\lim\limits_{n\rightarrow +\infty }\left(q^{n}\right) = +\infty $)\end{center}
          \item \textbf{Si $ 0 \leqslant  q < 1$ :} alors $q^{n}$ est aussi proche de zéro que l'on veut dès que $n$ est suffisamment grand. On dit que la suite $\left(q^{n}\right)$ tend vers $0$ et on écrit :
          \begin{center}$\lim\limits_{n\rightarrow +\infty }  q^{n} = 0 $ ( ou $\lim\limits_{n\rightarrow +\infty }\left(q^{n}\right) = 0$)\end{center}
     \end{itemize}
}
\bloc{cyan}{Remarque}{% id="r90"
     Pour $q=1$ $q^{n}=1^{n}=1$; la suite est constante, égale à $1$, et tend donc vers $1$;
}
\begin{h2}4 - Suites arithmético-géométriques\end{h2}
\cadre{bleu}{Définition}{% id="d110"
     Une suite arithmético-géométrique $u_{n}$ est définie par son premier terme $u_{0}$ et une relation de  récurrence du type :
     \begin{center}$u_{n+1} = a\times u_{n}+b$ pour tout entier $n$\end{center}
     où $a$ et $b$ sont deux nombres réels.
}
\bloc{cyan}{Remarque}{% id="r110"
     \textbf{Attention} : Ces suites ne sont \textbf{ni arithmétiques} (sauf si $a=1$) \textbf{ni géométriques} (sauf si $b=0$).
}
\cadre{vert}{Propriété}{% id="p130"
     Il existe un nombre réel $k$ tel que la suite $v_{n}$ définie, pour tout entier $n$, par $v_{n}=u_{n}+k$ soit une suite géométrique de raison $a$.
}
\bloc{cyan}{Remarques}{% id="r130"
     \begin{itemize}
          \item En général, dans les exercices, le nombre $k$ vous sera donné (et si ce n'est pas le cas on vous indiquera une démarche pour le trouver). On vous demandera de prouver que $v_{n}$ est une suite géométrique de raison $a$.
          \item Puisque  $v_{n}=u_{n}+k$, pour tout entier $n$, on a en particulier $v_{0}=u_{0}+k$ ce qui permet de connaître le premier terme de la suite $v_{n}$.
          \item $v_{n}=u_{n}+k$ signifie aussi que $u_{n}=v_{n}-k$.
          \par
          Donc une fois que l'on connaît $v_{n}$ on peut trouver $u_{n}$ (voir exemple ci-dessous)
     \end{itemize}
}
\bloc{orange}{Exemple détaillé}{% id="e130"
     Soit la suite $\left(u_{n}\right)$ définie par $u_{0}=5$ et $u_{n+1}=0,6u_{n}+4$.
     \begin{enumerate}\item Montrer que la suite $\left(v_{n}\right)$ définie par $v_{n}=u_{n}-10$ est une suite géométrique.
          \item En déduire l'expression de $u_{n}$ en fonction de $n$.
                    \end{enumerate}
          \begin{enumerate}\item \textbf{Montrons que la suite $\left(v_{n}\right)$ est une suite géométrique}
               Pour montrer que la suite $\left(v_{n}\right)$ est géométrique on va calculer $v_{n+1}$ en fonction de $v_{n}$.
               \par
               $v_{n}=u_{n}-10$ pour tout entier $n$ donc :
               \par
               $v_{n+1}=u_{n+1}-10$
               \par
               or on sait que
               \par
               $u_{n+1}=0,6u_{n}+4$
               \par
               donc
               \par
               $v_{n+1}=0,6u_{n}+4-10 = 0,6u_{n}-6$
               \par
               Ici, une petite astuce consiste à mettre $0,6$ en facteur (on peut également dire que $u_{n}=v_{n}+10$ et remplacer $u_{n}$ par $v_{n}+10$)
               \par
               $v_{n+1}=0,6u_{n}-0,6\times 10=0,6\left(u_{n}-10\right)=0,6v_{n}$
               \par
               On a bien une relation du type $v_{n+1}=q\times v_{n}$ avec $q=0,6$ ce qui montre que \textbf{la suite $\left(v_{n}\right)$ est une suite géométrique de raison $0,6$}.
               \item \textbf{Expression de $u_{n}$ en fonction de $n$}
               Par ailleurs, $v_{0}=u_{0}-10=5-10=-5$
               \par
               $\left(v_{n}\right)$ est une suite géométrique de premier terme $v_{0}=5$ et de raison $q=0,6$ donc pour tout entier $n$:
               \par
               $v_{n}=v_{0}\times q^{n}=-5\times 0,6^{n}$
               \par
               Comme $u_{n}=v_{n}+10$, on obtient finalement :
               \begin{center}$u_{n}=-5\times 0,6^{n}+10$\end{center}
          \end{enumerate}
     }
     
\end{document}