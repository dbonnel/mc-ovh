\documentclass[a4paper]{article}

%================================================================================================================================
%
% Packages
%
%================================================================================================================================

\usepackage[T1]{fontenc} 	% pour caractères accentués
\usepackage[utf8]{inputenc}  % encodage utf8
\usepackage[french]{babel}	% langue : français
\usepackage{fourier}			% caractères plus lisibles
\usepackage[dvipsnames]{xcolor} % couleurs
\usepackage{fancyhdr}		% réglage header footer
\usepackage{needspace}		% empêcher sauts de page mal placés
\usepackage{graphicx}		% pour inclure des graphiques
\usepackage{enumitem,cprotect}		% personnalise les listes d'items (nécessaire pour ol, al ...)
\usepackage{hyperref}		% Liens hypertexte
\usepackage{pstricks,pst-all,pst-node,pstricks-add,pst-math,pst-plot,pst-tree,pst-eucl} % pstricks
\usepackage[a4paper,includeheadfoot,top=2cm,left=3cm, bottom=2cm,right=3cm]{geometry} % marges etc.
\usepackage{comment}			% commentaires multilignes
\usepackage{amsmath,environ} % maths (matrices, etc.)
\usepackage{amssymb,makeidx}
\usepackage{bm}				% bold maths
\usepackage{tabularx}		% tableaux
\usepackage{colortbl}		% tableaux en couleur
\usepackage{fontawesome}		% Fontawesome
\usepackage{environ}			% environment with command
\usepackage{fp}				% calculs pour ps-tricks
\usepackage{multido}			% pour ps tricks
\usepackage[np]{numprint}	% formattage nombre
\usepackage{tikz,tkz-tab} 			% package principal TikZ
\usepackage{pgfplots}   % axes
\usepackage{mathrsfs}    % cursives
\usepackage{calc}			% calcul taille boites
\usepackage[scaled=0.875]{helvet} % font sans serif
\usepackage{svg} % svg
\usepackage{scrextend} % local margin
\usepackage{scratch} %scratch
\usepackage{multicol} % colonnes
%\usepackage{infix-RPN,pst-func} % formule en notation polanaise inversée
\usepackage{listings}

%================================================================================================================================
%
% Réglages de base
%
%================================================================================================================================

\lstset{
language=Python,   % R code
literate=
{á}{{\'a}}1
{à}{{\`a}}1
{ã}{{\~a}}1
{é}{{\'e}}1
{è}{{\`e}}1
{ê}{{\^e}}1
{í}{{\'i}}1
{ó}{{\'o}}1
{õ}{{\~o}}1
{ú}{{\'u}}1
{ü}{{\"u}}1
{ç}{{\c{c}}}1
{~}{{ }}1
}


\definecolor{codegreen}{rgb}{0,0.6,0}
\definecolor{codegray}{rgb}{0.5,0.5,0.5}
\definecolor{codepurple}{rgb}{0.58,0,0.82}
\definecolor{backcolour}{rgb}{0.95,0.95,0.92}

\lstdefinestyle{mystyle}{
    backgroundcolor=\color{backcolour},   
    commentstyle=\color{codegreen},
    keywordstyle=\color{magenta},
    numberstyle=\tiny\color{codegray},
    stringstyle=\color{codepurple},
    basicstyle=\ttfamily\footnotesize,
    breakatwhitespace=false,         
    breaklines=true,                 
    captionpos=b,                    
    keepspaces=true,                 
    numbers=left,                    
xleftmargin=2em,
framexleftmargin=2em,            
    showspaces=false,                
    showstringspaces=false,
    showtabs=false,                  
    tabsize=2,
    upquote=true
}

\lstset{style=mystyle}


\lstset{style=mystyle}
\newcommand{\imgdir}{C:/laragon/www/newmc/assets/imgsvg/}
\newcommand{\imgsvgdir}{C:/laragon/www/newmc/assets/imgsvg/}

\definecolor{mcgris}{RGB}{220, 220, 220}% ancien~; pour compatibilité
\definecolor{mcbleu}{RGB}{52, 152, 219}
\definecolor{mcvert}{RGB}{125, 194, 70}
\definecolor{mcmauve}{RGB}{154, 0, 215}
\definecolor{mcorange}{RGB}{255, 96, 0}
\definecolor{mcturquoise}{RGB}{0, 153, 153}
\definecolor{mcrouge}{RGB}{255, 0, 0}
\definecolor{mclightvert}{RGB}{205, 234, 190}

\definecolor{gris}{RGB}{220, 220, 220}
\definecolor{bleu}{RGB}{52, 152, 219}
\definecolor{vert}{RGB}{125, 194, 70}
\definecolor{mauve}{RGB}{154, 0, 215}
\definecolor{orange}{RGB}{255, 96, 0}
\definecolor{turquoise}{RGB}{0, 153, 153}
\definecolor{rouge}{RGB}{255, 0, 0}
\definecolor{lightvert}{RGB}{205, 234, 190}
\setitemize[0]{label=\color{lightvert}  $\bullet$}

\pagestyle{fancy}
\renewcommand{\headrulewidth}{0.2pt}
\fancyhead[L]{maths-cours.fr}
\fancyhead[R]{\thepage}
\renewcommand{\footrulewidth}{0.2pt}
\fancyfoot[C]{}

\newcolumntype{C}{>{\centering\arraybackslash}X}
\newcolumntype{s}{>{\hsize=.35\hsize\arraybackslash}X}

\setlength{\parindent}{0pt}		 
\setlength{\parskip}{3mm}
\setlength{\headheight}{1cm}

\def\ebook{ebook}
\def\book{book}
\def\web{web}
\def\type{web}

\newcommand{\vect}[1]{\overrightarrow{\,\mathstrut#1\,}}

\def\Oij{$\left(\text{O}~;~\vect{\imath},~\vect{\jmath}\right)$}
\def\Oijk{$\left(\text{O}~;~\vect{\imath},~\vect{\jmath},~\vect{k}\right)$}
\def\Ouv{$\left(\text{O}~;~\vect{u},~\vect{v}\right)$}

\hypersetup{breaklinks=true, colorlinks = true, linkcolor = OliveGreen, urlcolor = OliveGreen, citecolor = OliveGreen, pdfauthor={Didier BONNEL - https://www.maths-cours.fr} } % supprime les bordures autour des liens

\renewcommand{\arg}[0]{\text{arg}}

\everymath{\displaystyle}

%================================================================================================================================
%
% Macros - Commandes
%
%================================================================================================================================

\newcommand\meta[2]{    			% Utilisé pour créer le post HTML.
	\def\titre{titre}
	\def\url{url}
	\def\arg{#1}
	\ifx\titre\arg
		\newcommand\maintitle{#2}
		\fancyhead[L]{#2}
		{\Large\sffamily \MakeUppercase{#2}}
		\vspace{1mm}\textcolor{mcvert}{\hrule}
	\fi 
	\ifx\url\arg
		\fancyfoot[L]{\href{https://www.maths-cours.fr#2}{\black \footnotesize{https://www.maths-cours.fr#2}}}
	\fi 
}


\newcommand\TitreC[1]{    		% Titre centré
     \needspace{3\baselineskip}
     \begin{center}\textbf{#1}\end{center}
}

\newcommand\newpar{    		% paragraphe
     \par
}

\newcommand\nosp {    		% commande vide (pas d'espace)
}
\newcommand{\id}[1]{} %ignore

\newcommand\boite[2]{				% Boite simple sans titre
	\vspace{5mm}
	\setlength{\fboxrule}{0.2mm}
	\setlength{\fboxsep}{5mm}	
	\fcolorbox{#1}{#1!3}{\makebox[\linewidth-2\fboxrule-2\fboxsep]{
  		\begin{minipage}[t]{\linewidth-2\fboxrule-4\fboxsep}\setlength{\parskip}{3mm}
  			 #2
  		\end{minipage}
	}}
	\vspace{5mm}
}

\newcommand\CBox[4]{				% Boites
	\vspace{5mm}
	\setlength{\fboxrule}{0.2mm}
	\setlength{\fboxsep}{5mm}
	
	\fcolorbox{#1}{#1!3}{\makebox[\linewidth-2\fboxrule-2\fboxsep]{
		\begin{minipage}[t]{1cm}\setlength{\parskip}{3mm}
	  		\textcolor{#1}{\LARGE{#2}}    
 	 	\end{minipage}  
  		\begin{minipage}[t]{\linewidth-2\fboxrule-4\fboxsep}\setlength{\parskip}{3mm}
			\raisebox{1.2mm}{\normalsize\sffamily{\textcolor{#1}{#3}}}						
  			 #4
  		\end{minipage}
	}}
	\vspace{5mm}
}

\newcommand\cadre[3]{				% Boites convertible html
	\par
	\vspace{2mm}
	\setlength{\fboxrule}{0.1mm}
	\setlength{\fboxsep}{5mm}
	\fcolorbox{#1}{white}{\makebox[\linewidth-2\fboxrule-2\fboxsep]{
  		\begin{minipage}[t]{\linewidth-2\fboxrule-4\fboxsep}\setlength{\parskip}{3mm}
			\raisebox{-2.5mm}{\sffamily \small{\textcolor{#1}{\MakeUppercase{#2}}}}		
			\par		
  			 #3
 	 		\end{minipage}
	}}
		\vspace{2mm}
	\par
}

\newcommand\bloc[3]{				% Boites convertible html sans bordure
     \needspace{2\baselineskip}
     {\sffamily \small{\textcolor{#1}{\MakeUppercase{#2}}}}    
		\par		
  			 #3
		\par
}

\newcommand\CHelp[1]{
     \CBox{Plum}{\faInfoCircle}{À RETENIR}{#1}
}

\newcommand\CUp[1]{
     \CBox{NavyBlue}{\faThumbsOUp}{EN PRATIQUE}{#1}
}

\newcommand\CInfo[1]{
     \CBox{Sepia}{\faArrowCircleRight}{REMARQUE}{#1}
}

\newcommand\CRedac[1]{
     \CBox{PineGreen}{\faEdit}{BIEN R\'EDIGER}{#1}
}

\newcommand\CError[1]{
     \CBox{Red}{\faExclamationTriangle}{ATTENTION}{#1}
}

\newcommand\TitreExo[2]{
\needspace{4\baselineskip}
 {\sffamily\large EXERCICE #1\ (\emph{#2 points})}
\vspace{5mm}
}

\newcommand\img[2]{
          \includegraphics[width=#2\paperwidth]{\imgdir#1}
}

\newcommand\imgsvg[2]{
       \begin{center}   \includegraphics[width=#2\paperwidth]{\imgsvgdir#1} \end{center}
}


\newcommand\Lien[2]{
     \href{#1}{#2 \tiny \faExternalLink}
}
\newcommand\mcLien[2]{
     \href{https~://www.maths-cours.fr/#1}{#2 \tiny \faExternalLink}
}

\newcommand{\euro}{\eurologo{}}

%================================================================================================================================
%
% Macros - Environement
%
%================================================================================================================================

\newenvironment{tex}{ %
}
{%
}

\newenvironment{indente}{ %
	\setlength\parindent{10mm}
}

{
	\setlength\parindent{0mm}
}

\newenvironment{corrige}{%
     \needspace{3\baselineskip}
     \medskip
     \textbf{\textsc{Corrigé}}
     \medskip
}
{
}

\newenvironment{extern}{%
     \begin{center}
     }
     {
     \end{center}
}

\NewEnviron{code}{%
	\par
     \boite{gray}{\texttt{%
     \BODY
     }}
     \par
}

\newenvironment{vbloc}{% boite sans cadre empeche saut de page
     \begin{minipage}[t]{\linewidth}
     }
     {
     \end{minipage}
}
\NewEnviron{h2}{%
    \needspace{3\baselineskip}
    \vspace{0.6cm}
	\noindent \MakeUppercase{\sffamily \large \BODY}
	\vspace{1mm}\textcolor{mcgris}{\hrule}\vspace{0.4cm}
	\par
}{}

\NewEnviron{h3}{%
    \needspace{3\baselineskip}
	\vspace{5mm}
	\textsc{\BODY}
	\par
}

\NewEnviron{margeneg}{ %
\begin{addmargin}[-1cm]{0cm}
\BODY
\end{addmargin}
}

\NewEnviron{html}{%
}

\begin{document}
\begin{h2}1. Loi de Bernoulli\end{h2}
\cadre{bleu}{Définition}{% id="d10"
     On appelle \textbf{épreuve de Bernoulli} de paramètre $p$ (avec $0 < p < 1$) une expérience aléatoire ayant deux issues :
     \begin{itemize}
          \item l'une appelée \textbf{succès} (généralement notée $S$) de probabilité $p$,
          \item l'autre appelée \textbf{échec} (généralement notée $\overline S$) de probabilité $1-p$.
     \end{itemize}
}
\cadre{bleu}{Définition}{% id="d20"
     On considère la variable aléatoire $X$ qui vaut $1$ en cas de succès et $0$ en cas d'échec.
     \par
     Cette variable aléatoire suit la \textbf{loi de Bernoulli de paramètre $p$}, définie par le tableau suivant:
     \begin{tabular}{|c|c|c|}%class="compact" width="250"
          \hline
          \textbf{$x_{i}$}  & $0$  & $1$
          \\ \hline
          \textbf{$p\left(X=x_{i}\right)$}    &  $1-p$  &  $p$
          \\ \hline
     \end{tabular}
}
\bloc{orange}{Exemple}{% id="e20"
     Au bonneteau, deux cartes noires et une carte rouge sont présentées, faces cachées, sur la table.
     \begin{center}
          \img{bonneteau}{0.25}%width="250" alt="Bonneteau"
     \end{center}
    Le maître du jeu manipule les cartes rapidement et un joueur doit retrouver la carte rouge.
     \par
     On suppose que le joueur choisit une carte complètement au hasard.
     \par
     On a affaire à une loi de Bernoulli de paramètre $p=\frac{1}{3}$.
     \par
     La probabilité de succès est : $p\left(S\right)=p=\frac{1}{3}$ et la probabilité d'échec $p\left(\overline S\right)=1-p=\frac{2}{3}$
}
\cadre{vert}{Propriété}{% id="p30"
     L'espérance mathématique d'une variable aléatoire $X$ qui suit une loi de Bernoulli de paramètre $p$ est :
     \begin{center}$E\left(X\right)=p$\end{center}
}
\bloc{cyan}{Démonstration}{% id="m30"
     D'après la définition de l'espérance mathématique :
     \par
     $E\left(X\right)=0\times \left(1-p\right)+1\times p=p$
}
\begin{h2}2. Schéma de Bernoulli - Loi binomiale\end{h2}
\cadre{bleu}{Définition}{% id="d40"
     On appelle \textbf{schéma de Bernoulli} la répétition d'épreuves de Bernoulli \textbf{identiques} et \textbf{indépendantes}.
}
\bloc{orange}{Exemple}{% id="e40"
     Une urne contient 2 boules rouges et 3 boules blanches. On tire 3 boules au hasard.
     \begin{itemize}
          \item Si l'épreuve s'effectue sans remise, les tirages ne sont ni identiques, ni indépendants. En effet, le fait d'avoir retiré une boule lors du premier tirage fait que le second tirage n'est pas identique au premier.
          \item Si l'épreuve s'effectue avec remise, on pourra, par contre, considérer que les tirages sont identiques et indépendants. On a donc bien, dans ce cas, un schéma de Bernoulli
     \end{itemize}
}
\cadre{bleu}{Définition}{% id="d50"
     Soit $X$ la variable aléatoire qui \textbf{compte le nombre de succès} dans un schéma de Bernoulli constitué de $n$ épreuves ayant chacune une probabilité de succès égale à $p$.
     \par
     La variable aléatoire X suit une loi appelée \textbf{loi binomiale de paramètres $n$ et $p$}, souvent notée $\mathscr B \left(n ; p\right)$ .
}
\bloc{orange}{Exemple}{% id="e50"
     On reprend l'exemple précédent : tirage au hasard et avec remise de 3 boules parmi 5 boules comportant 2 boules rouges et 3 boules blanches. On considère la variable aléatoire $X$ qui compte le nombre de boules rouges obtenues. La variable $X$ sur une loi binomiale de paramètres 3 (nombre d'épreuves) et $\frac{2}{5}$ (probabilité d'obtenir une boule rouge lors d'une épreuve).
     \par
     Ce schéma peut être représenté par l'arbre suivant :
      \begin{center}
          \begin{extern}%width="420" alt="Arbre schéma de Bernoulli"
               %:-+-+-+- Engendré par : http://math.et.info.free.fr/TikZ/Arbre/
               % Racine à Gauche, développement vers la droite
               \resizebox{8cm}{!}{
                    \begin{tikzpicture}[xscale=1,yscale=1]
                         % Styles (MODIFIABLES)
                         \tikzstyle{fleche}=[>=latex,thick]
                         \tikzstyle{noeud}=[circle,draw]
                         \tikzstyle{feuille}=[circle,draw]
                         \tikzstyle{etiquette}=[midway,fill=white]
                         % Dimensions (MODIFIABLES)
                         \def\DistanceInterNiveaux{3}
                         \def\DistanceInterFeuilles{2}
                         % Dimensions calculées (NON MODIFIABLES)
                         \def\NiveauA{(0)*\DistanceInterNiveaux}
                         \def\NiveauB{(1.6666666666666665)*\DistanceInterNiveaux}
                         \def\NiveauC{(3)*\DistanceInterNiveaux}
                         \def\NiveauD{(4)*\DistanceInterNiveaux}
                         \def\InterFeuilles{(-1)*\DistanceInterFeuilles}
                         % Noeuds (MODIFIABLES : Styles et Coefficients d'InterFeuilles)
                         \node[noeud] (R) at ({\NiveauA},{(3.5)*\InterFeuilles}) {$ $};
                         \node[noeud] (Ra) at ({\NiveauB},{(1.5)*\InterFeuilles}) {$S$};
                         \node[noeud] (Raa) at ({\NiveauC},{(0.5)*\InterFeuilles}) {$S$};
                         \node[feuille] (Raaa) at ({\NiveauD},{(0)*\InterFeuilles}) {$S$};
                         \node[feuille] (Raab) at ({\NiveauD},{(1)*\InterFeuilles}) {$\overline{S}$};
                         \node[noeud] (Rab) at ({\NiveauC},{(2.5)*\InterFeuilles}) {$\overline{S}$};
                         \node[feuille] (Raba) at ({\NiveauD},{(2)*\InterFeuilles}) {$S$};
                         \node[feuille] (Rabb) at ({\NiveauD},{(3)*\InterFeuilles}) {$\overline{S}$};
                         \node[noeud] (Rb) at ({\NiveauB},{(5.5)*\InterFeuilles}) {$\overline{S}$};
                         \node[noeud] (Rba) at ({\NiveauC},{(4.5)*\InterFeuilles}) {$S$};
                         \node[feuille] (Rbaa) at ({\NiveauD},{(4)*\InterFeuilles}) {$S$};
                         \node[feuille] (Rbab) at ({\NiveauD},{(5)*\InterFeuilles}) {$\overline{S}$};
                         \node[noeud] (Rbb) at ({\NiveauC},{(6.5)*\InterFeuilles}) {$\overline{S}$};
                         \node[feuille] (Rbba) at ({\NiveauD},{(6)*\InterFeuilles}) {$S$};
                         \node[feuille] (Rbbb) at ({\NiveauD},{(7)*\InterFeuilles}) {$\overline{S}$};
                         % Arcs (MODIFIABLES : Styles)
                         \draw[fleche] (R)--(Ra) node[etiquette] {$\dfrac{2}{5}$};
                         \draw[fleche] (Ra)--(Raa) node[etiquette] {$\dfrac{2}{5}$};
                         \draw[fleche] (Raa)--(Raaa) node[etiquette] {$\dfrac{2}{5}$};
                         \draw[fleche] (Raa)--(Raab) node[etiquette] {$\dfrac{3}{5}$};
                         \draw[fleche] (Ra)--(Rab) node[etiquette] {$\dfrac{3}{5}$};
                         \draw[fleche] (Rab)--(Raba) node[etiquette] {$\dfrac{2}{5}$};
                         \draw[fleche] (Rab)--(Rabb) node[etiquette] {$\dfrac{3}{5}$};
                         \draw[fleche] (R)--(Rb) node[etiquette] {$\dfrac{3}{5}$};
                         \draw[fleche] (Rb)--(Rba) node[etiquette] {$\dfrac{2}{5}$};
                         \draw[fleche] (Rba)--(Rbaa) node[etiquette] {$\dfrac{2}{5}$};
                         \draw[fleche] (Rba)--(Rbab) node[etiquette] {$\dfrac{3}{5}$};
                         \draw[fleche] (Rb)--(Rbb) node[etiquette] {$\dfrac{3}{5}$};
                         \draw[fleche] (Rbb)--(Rbba) node[etiquette] {$\dfrac{2}{5}$};
                         \draw[fleche] (Rbb)--(Rbbb) node[etiquette] {$\dfrac{3}{5}$};
                    \end{tikzpicture}
                    %:-+-+-+-+- Fin
               }
          \end{extern}
     \end{center}
     Grâce à l'arbre on voit que :
     \begin{itemize}
          \item la probabilité d'avoir 3 succès (c'est à dire 3 boules rouges) est  $p\left(X=3\right) =\left(\frac{2}{5}\right)^{3}=\frac{8}{125}$
          \item il y a 3 chemins qui correspondent à 2 succès ($SS\overline S, S\overline SS, \overline SSS$). La probabilité d'obtenir 2 boules rouges est donc :
          \par
          $p\left(X=2\right) =\left(\frac{2}{5}\right)^{2}\times \frac{3}{5}+\left(\frac{2}{5}\right)^{2}\times \frac{3}{5}$\nosp$+\left(\frac{2}{5}\right)^{2}\times \frac{3}{5}$\nosp$=3\times \left[\left(\frac{2}{5}\right)^{2}\times \frac{3}{5}\right]=\frac{36}{125}$
          \item il y a également 3 chemins qui correspondent à un unique succès ($S\overline S\overline S, \overline SS\overline S, \overline S\overline SS$). La probabilité d'obtenir une unique boule rouge est donc :
          \par
          $p\left(X=1\right) = \frac{2}{5}\times \left(\frac{3}{5}\right)^{2}+ \frac{2}{5}\times \left(\frac{3}{5}\right)^{2}$\nosp$+ \frac{2}{5}\times \left(\frac{3}{5}\right)^{2}$\nosp$=3\times \left[ \frac{2}{5}\times \left(\frac{3}{5}\right)^{2}\right]=\frac{54}{125}$
          \item la probabilité de n'avoir aucune boule rouge est  $p\left(X=0\right) =\left(\frac{3}{5}\right)^{3}=\frac{27}{125}$
     \end{itemize}
     La loi de $X$ est donc donnée par le tableau suivant :
 \begin{center}
     \begin{tabular}{|c|c|c|c|c|}%class="compact" width="600"
          \hline
          \textbf{$x_{i}$}  & $0$  & $1$ & $2$ & $3$
          \\ \hline
          \textbf{$p\left(X=x_{i}\right)$}    &  $\frac{27}{125}$  &   $\frac{54}{125}$ &   $\frac{36}{125}$ &   $\frac{8}{125}$
          \\ \hline
     \end{tabular}
 \end{center}
     On vérifie bien que $\frac{27}{125}+\frac{54}{125}+\frac{36}{125}+\frac{8}{125}=1$
}
\cadre{vert}{Propriété}{% id="p60"
     L'espérance mathématique d'une variable aléatoire $X$ qui suit une loi binomiale $\mathscr B \left(n ; p\right)$ est :
     \par
     $E\left(X\right)=np$
}
\bloc{orange}{Exemple}{% id="e60"
     Dans l'exemple précédent, la variable X suit une loi binomiale $\mathscr{B} \left(3 ; \dfrac{2}{5}\right)$.
     \par
     Son espérance mathématique est donc $E\left(X\right)=3\times \frac{2}{5}=\frac{6}{5}=1,2$
     \par
     On vérifie que l'on obtient bien le même résultat en utilisant le tableau de la loi de $X$ et la définition de l'espérance mathématique :
     \par
     $E\left(X\right)=0\times \frac{27}{125}+1\times \frac{54}{125}$\nosp$+2\times \frac{36}{125}+3\times \frac{8}{125}$\nosp$=\frac{150}{125}=1,2$
}
\begin{h2}3. Coefficients binomiaux\end{h2}
\cadre{bleu}{Définition}{% id="d70"
     On considère un arbre pondéré représentant une loi binomiale $\mathscr B \left(n ; p\right)$.
     \par
     Le \textbf{coefficient binomial} $\begin{pmatrix} n \\ k \end{pmatrix}$ (lire \textit{$k$ parmi $n$}) est le nombre de chemins qui correspondent à $k$ succès.
}
\bloc{orange}{Exemple}{% id="e70"
     On reprend le même exemple que précédemment. On a vu, par exemple,  qu'il y avait 3 chemins correspondant à 2 succès. On a donc $\begin{pmatrix} 3 \\ 2 \end{pmatrix}= 3$.
}
\bloc{cyan}{Remarques}{% id="r70"
     \begin{itemize}
          \item On peut aussi employer le mot \textbf{combinaisons} pour désigner un coefficient binomial;
          \item Pour calculer un coefficient binomial sur la plupart des calculatrices on utilise la commande \textbf{nCr}. Dans un tableur, on utilise la formule \textbf{=COMBIN(n;k)}.
     \end{itemize}
}
\cadre{rouge}{Théorème}{% id="t80"
     Soit $X$ une variable aléatoire de loi $\mathscr B \left(n ; p\right)$.
     \par
     Pour tout entier $k$ compris entre $0$ et $n$ :
     \begin{center}
     $p\left(X=k\right)=\begin{pmatrix} n \\ k \end{pmatrix}$\nosp$p^{k} \left(1-p\right)^{n-k}$
     \end{center}
}
\bloc{orange}{Exemple}{% id="e80"
     On lance 8 fois une pièce équilibrée et on appelle $X$ la variable aléatoire qui compte le nombre de fois où l'on obtient \textit{Pile}.
     \par
     $X$ suit une loi binomiale de paramètres $n=8$ et $p=\frac{1}{2}$.
     \par
     La probabilité d'obtenir \textbf{4 fois} \textit{Pile} (par exemple) est :
     \par
     $p\left(X=4\right) = \begin{pmatrix} 8 \\ 4 \end{pmatrix}\times \left(\frac{1}{2}\right)^{4}\times \left(\frac{1}{2}\right)^{4}$
     \par
     $\begin{pmatrix} 8 \\ 4 \end{pmatrix}= 70$ ~ (à la calculatrice)
     \par
     Donc :
     \par
     $p\left(X=4\right)=70\times \frac{1}{16}\times \frac{1}{16}=\frac{70}{256}=\frac{35}{128}$.
}

\end{document}