\documentclass[a4paper]{article}

%================================================================================================================================
%
% Packages
%
%================================================================================================================================

\usepackage[T1]{fontenc} 	% pour caractères accentués
\usepackage[utf8]{inputenc}  % encodage utf8
\usepackage[french]{babel}	% langue : français
\usepackage{fourier}			% caractères plus lisibles
\usepackage[dvipsnames]{xcolor} % couleurs
\usepackage{fancyhdr}		% réglage header footer
\usepackage{needspace}		% empêcher sauts de page mal placés
\usepackage{graphicx}		% pour inclure des graphiques
\usepackage{enumitem,cprotect}		% personnalise les listes d'items (nécessaire pour ol, al ...)
\usepackage{hyperref}		% Liens hypertexte
\usepackage{pstricks,pst-all,pst-node,pstricks-add,pst-math,pst-plot,pst-tree,pst-eucl} % pstricks
\usepackage[a4paper,includeheadfoot,top=2cm,left=3cm, bottom=2cm,right=3cm]{geometry} % marges etc.
\usepackage{comment}			% commentaires multilignes
\usepackage{amsmath,environ} % maths (matrices, etc.)
\usepackage{amssymb,makeidx}
\usepackage{bm}				% bold maths
\usepackage{tabularx}		% tableaux
\usepackage{colortbl}		% tableaux en couleur
\usepackage{fontawesome}		% Fontawesome
\usepackage{environ}			% environment with command
\usepackage{fp}				% calculs pour ps-tricks
\usepackage{multido}			% pour ps tricks
\usepackage[np]{numprint}	% formattage nombre
\usepackage{tikz,tkz-tab} 			% package principal TikZ
\usepackage{pgfplots}   % axes
\usepackage{mathrsfs}    % cursives
\usepackage{calc}			% calcul taille boites
\usepackage[scaled=0.875]{helvet} % font sans serif
\usepackage{svg} % svg
\usepackage{scrextend} % local margin
\usepackage{scratch} %scratch
\usepackage{multicol} % colonnes
%\usepackage{infix-RPN,pst-func} % formule en notation polanaise inversée
\usepackage{listings}

%================================================================================================================================
%
% Réglages de base
%
%================================================================================================================================

\lstset{
language=Python,   % R code
literate=
{á}{{\'a}}1
{à}{{\`a}}1
{ã}{{\~a}}1
{é}{{\'e}}1
{è}{{\`e}}1
{ê}{{\^e}}1
{í}{{\'i}}1
{ó}{{\'o}}1
{õ}{{\~o}}1
{ú}{{\'u}}1
{ü}{{\"u}}1
{ç}{{\c{c}}}1
{~}{{ }}1
}


\definecolor{codegreen}{rgb}{0,0.6,0}
\definecolor{codegray}{rgb}{0.5,0.5,0.5}
\definecolor{codepurple}{rgb}{0.58,0,0.82}
\definecolor{backcolour}{rgb}{0.95,0.95,0.92}

\lstdefinestyle{mystyle}{
    backgroundcolor=\color{backcolour},   
    commentstyle=\color{codegreen},
    keywordstyle=\color{magenta},
    numberstyle=\tiny\color{codegray},
    stringstyle=\color{codepurple},
    basicstyle=\ttfamily\footnotesize,
    breakatwhitespace=false,         
    breaklines=true,                 
    captionpos=b,                    
    keepspaces=true,                 
    numbers=left,                    
xleftmargin=2em,
framexleftmargin=2em,            
    showspaces=false,                
    showstringspaces=false,
    showtabs=false,                  
    tabsize=2,
    upquote=true
}

\lstset{style=mystyle}


\lstset{style=mystyle}
\newcommand{\imgdir}{C:/laragon/www/newmc/assets/imgsvg/}
\newcommand{\imgsvgdir}{C:/laragon/www/newmc/assets/imgsvg/}

\definecolor{mcgris}{RGB}{220, 220, 220}% ancien~; pour compatibilité
\definecolor{mcbleu}{RGB}{52, 152, 219}
\definecolor{mcvert}{RGB}{125, 194, 70}
\definecolor{mcmauve}{RGB}{154, 0, 215}
\definecolor{mcorange}{RGB}{255, 96, 0}
\definecolor{mcturquoise}{RGB}{0, 153, 153}
\definecolor{mcrouge}{RGB}{255, 0, 0}
\definecolor{mclightvert}{RGB}{205, 234, 190}

\definecolor{gris}{RGB}{220, 220, 220}
\definecolor{bleu}{RGB}{52, 152, 219}
\definecolor{vert}{RGB}{125, 194, 70}
\definecolor{mauve}{RGB}{154, 0, 215}
\definecolor{orange}{RGB}{255, 96, 0}
\definecolor{turquoise}{RGB}{0, 153, 153}
\definecolor{rouge}{RGB}{255, 0, 0}
\definecolor{lightvert}{RGB}{205, 234, 190}
\setitemize[0]{label=\color{lightvert}  $\bullet$}

\pagestyle{fancy}
\renewcommand{\headrulewidth}{0.2pt}
\fancyhead[L]{maths-cours.fr}
\fancyhead[R]{\thepage}
\renewcommand{\footrulewidth}{0.2pt}
\fancyfoot[C]{}

\newcolumntype{C}{>{\centering\arraybackslash}X}
\newcolumntype{s}{>{\hsize=.35\hsize\arraybackslash}X}

\setlength{\parindent}{0pt}		 
\setlength{\parskip}{3mm}
\setlength{\headheight}{1cm}

\def\ebook{ebook}
\def\book{book}
\def\web{web}
\def\type{web}

\newcommand{\vect}[1]{\overrightarrow{\,\mathstrut#1\,}}

\def\Oij{$\left(\text{O}~;~\vect{\imath},~\vect{\jmath}\right)$}
\def\Oijk{$\left(\text{O}~;~\vect{\imath},~\vect{\jmath},~\vect{k}\right)$}
\def\Ouv{$\left(\text{O}~;~\vect{u},~\vect{v}\right)$}

\hypersetup{breaklinks=true, colorlinks = true, linkcolor = OliveGreen, urlcolor = OliveGreen, citecolor = OliveGreen, pdfauthor={Didier BONNEL - https://www.maths-cours.fr} } % supprime les bordures autour des liens

\renewcommand{\arg}[0]{\text{arg}}

\everymath{\displaystyle}

%================================================================================================================================
%
% Macros - Commandes
%
%================================================================================================================================

\newcommand\meta[2]{    			% Utilisé pour créer le post HTML.
	\def\titre{titre}
	\def\url{url}
	\def\arg{#1}
	\ifx\titre\arg
		\newcommand\maintitle{#2}
		\fancyhead[L]{#2}
		{\Large\sffamily \MakeUppercase{#2}}
		\vspace{1mm}\textcolor{mcvert}{\hrule}
	\fi 
	\ifx\url\arg
		\fancyfoot[L]{\href{https://www.maths-cours.fr#2}{\black \footnotesize{https://www.maths-cours.fr#2}}}
	\fi 
}


\newcommand\TitreC[1]{    		% Titre centré
     \needspace{3\baselineskip}
     \begin{center}\textbf{#1}\end{center}
}

\newcommand\newpar{    		% paragraphe
     \par
}

\newcommand\nosp {    		% commande vide (pas d'espace)
}
\newcommand{\id}[1]{} %ignore

\newcommand\boite[2]{				% Boite simple sans titre
	\vspace{5mm}
	\setlength{\fboxrule}{0.2mm}
	\setlength{\fboxsep}{5mm}	
	\fcolorbox{#1}{#1!3}{\makebox[\linewidth-2\fboxrule-2\fboxsep]{
  		\begin{minipage}[t]{\linewidth-2\fboxrule-4\fboxsep}\setlength{\parskip}{3mm}
  			 #2
  		\end{minipage}
	}}
	\vspace{5mm}
}

\newcommand\CBox[4]{				% Boites
	\vspace{5mm}
	\setlength{\fboxrule}{0.2mm}
	\setlength{\fboxsep}{5mm}
	
	\fcolorbox{#1}{#1!3}{\makebox[\linewidth-2\fboxrule-2\fboxsep]{
		\begin{minipage}[t]{1cm}\setlength{\parskip}{3mm}
	  		\textcolor{#1}{\LARGE{#2}}    
 	 	\end{minipage}  
  		\begin{minipage}[t]{\linewidth-2\fboxrule-4\fboxsep}\setlength{\parskip}{3mm}
			\raisebox{1.2mm}{\normalsize\sffamily{\textcolor{#1}{#3}}}						
  			 #4
  		\end{minipage}
	}}
	\vspace{5mm}
}

\newcommand\cadre[3]{				% Boites convertible html
	\par
	\vspace{2mm}
	\setlength{\fboxrule}{0.1mm}
	\setlength{\fboxsep}{5mm}
	\fcolorbox{#1}{white}{\makebox[\linewidth-2\fboxrule-2\fboxsep]{
  		\begin{minipage}[t]{\linewidth-2\fboxrule-4\fboxsep}\setlength{\parskip}{3mm}
			\raisebox{-2.5mm}{\sffamily \small{\textcolor{#1}{\MakeUppercase{#2}}}}		
			\par		
  			 #3
 	 		\end{minipage}
	}}
		\vspace{2mm}
	\par
}

\newcommand\bloc[3]{				% Boites convertible html sans bordure
     \needspace{2\baselineskip}
     {\sffamily \small{\textcolor{#1}{\MakeUppercase{#2}}}}    
		\par		
  			 #3
		\par
}

\newcommand\CHelp[1]{
     \CBox{Plum}{\faInfoCircle}{À RETENIR}{#1}
}

\newcommand\CUp[1]{
     \CBox{NavyBlue}{\faThumbsOUp}{EN PRATIQUE}{#1}
}

\newcommand\CInfo[1]{
     \CBox{Sepia}{\faArrowCircleRight}{REMARQUE}{#1}
}

\newcommand\CRedac[1]{
     \CBox{PineGreen}{\faEdit}{BIEN R\'EDIGER}{#1}
}

\newcommand\CError[1]{
     \CBox{Red}{\faExclamationTriangle}{ATTENTION}{#1}
}

\newcommand\TitreExo[2]{
\needspace{4\baselineskip}
 {\sffamily\large EXERCICE #1\ (\emph{#2 points})}
\vspace{5mm}
}

\newcommand\img[2]{
          \includegraphics[width=#2\paperwidth]{\imgdir#1}
}

\newcommand\imgsvg[2]{
       \begin{center}   \includegraphics[width=#2\paperwidth]{\imgsvgdir#1} \end{center}
}


\newcommand\Lien[2]{
     \href{#1}{#2 \tiny \faExternalLink}
}
\newcommand\mcLien[2]{
     \href{https~://www.maths-cours.fr/#1}{#2 \tiny \faExternalLink}
}

\newcommand{\euro}{\eurologo{}}

%================================================================================================================================
%
% Macros - Environement
%
%================================================================================================================================

\newenvironment{tex}{ %
}
{%
}

\newenvironment{indente}{ %
	\setlength\parindent{10mm}
}

{
	\setlength\parindent{0mm}
}

\newenvironment{corrige}{%
     \needspace{3\baselineskip}
     \medskip
     \textbf{\textsc{Corrigé}}
     \medskip
}
{
}

\newenvironment{extern}{%
     \begin{center}
     }
     {
     \end{center}
}

\NewEnviron{code}{%
	\par
     \boite{gray}{\texttt{%
     \BODY
     }}
     \par
}

\newenvironment{vbloc}{% boite sans cadre empeche saut de page
     \begin{minipage}[t]{\linewidth}
     }
     {
     \end{minipage}
}
\NewEnviron{h2}{%
    \needspace{3\baselineskip}
    \vspace{0.6cm}
	\noindent \MakeUppercase{\sffamily \large \BODY}
	\vspace{1mm}\textcolor{mcgris}{\hrule}\vspace{0.4cm}
	\par
}{}

\NewEnviron{h3}{%
    \needspace{3\baselineskip}
	\vspace{5mm}
	\textsc{\BODY}
	\par
}

\NewEnviron{margeneg}{ %
\begin{addmargin}[-1cm]{0cm}
\BODY
\end{addmargin}
}

\NewEnviron{html}{%
}

\begin{document}
\begin{h2}1. Fonctions linéaires\end{h2}
\cadre{bleu}{Définition}{% id="d10"
     Une fonction \textbf{linéaire} est une fonction définie sur $\mathbb{R}$ par une formule du type : $x\mapsto ax$ où $a \in  \mathbb{R}$.
     \par
     $a$ s'appelle le \textbf{coefficient directeur}.
}
\bloc{cyan}{Remarque}{% id="r10"
     La définition ci-dessus indique que si $f$ est une fonction linéaire, les valeurs de $f\left(x\right)$ sont proportionnelles aux valeurs de $x$, le coefficient de proportionnalité étant le coefficient directeur $a$.
}
\cadre{vert}{Propriété}{% id="p20"
     La courbe représentative d'une fonction linéaire est une \textbf{droite qui passe par l'origine} du repère.
}
\bloc{orange}{Exemple}{% id="e20"
\begin{center}
\begin{extern}%width="230" alt="fonction linéaire"
               \resizebox{6cm}{!}{%
                    % -+-+-+ variables modifiables
                    \def\fonction{1.5*x}
                    \def\xmin{-2.5}
                    \def\xmax{2.5}
                    \def\ymin{-2.5}
                    \def\ymax{4.5}
                    \def\xunit{1.5}  % unités en cm
                    \def\yunit{1.5}
                    \psset{xunit=\xunit,yunit=\yunit,algebraic=true}
                    \fontsize{15pt}{15pt}\selectfont
                    \begin{pspicture*}[linewidth=1pt](\xmin,\ymin)(\xmax,\ymax)
                         %      \psgrid[gridcolor=mcgris, subgriddiv=5, gridlabels=0pt](\xmin,\ymin)(\xmax,\ymax)
                         \psaxes[linewidth=0.75pt,Dx=1,Dy=1]{->}(0,0)(\xmin,\ymin)(\xmax,\ymax)
                         \psplot[plotpoints=2000,linecolor=blue]{\xmin}{\xmax}{\fonction}
                           \rput[tr](-0.2,-0.2){$O$} 
                                     \rput[tl](1.9,4){$\color{blue} \mathcal{C}_f$}
                    \end{pspicture*}
               }
\end{extern}
\end{center}   
  \begin{center}\textit{Courbe représentative de la fonction linéaire $x\mapsto \frac{3}{2}x$}\end{center}
}
\cadre{vert}{Propriété}{% id="p30"
     Soit $f$ une fonction linéaire.
     \par
     Pour tous réels $x$ et $x^{\prime}$ : $ f\left(x+x^{\prime}\right)=f\left(x\right)+f\left(x^{\prime}\right)$
     \par
     Pour tous réels $k$ et $x$ : $ f\left(kx\right)=kf\left(x\right)$
}
\begin{h2}2. Fonctions affines\end{h2}
\cadre{bleu}{Définition}{% id="d50"
     Une fonction \textbf{affine} est une fonction définie sur $\mathbb{R}$ par une formule du type : $x\mapsto ax+b$ où $a \in  \mathbb{R}$ et $b \in  \mathbb{R}$.
     \par
     $a$ s'appelle le \textbf{coefficient directeur} et $b$ l'\textbf{ordonnée à l'origine}.
}
\bloc{cyan}{Remarque}{% id="r50"
     Si $b=0$, la fonction est linéaire. Les fonctions linéaires sont donc des cas particuliers des fonctions affines.
}
\cadre{vert}{Propriété}{% id="p60"
     La courbe représentative d'une fonction affine est une \textbf{droite}.
}
\bloc{orange}{Exemple}{% id="e60"
\begin{center}
\begin{extern}%width="400" alt="fonction linéaire"
               \resizebox{11cm}{!}{%
                    % -+-+-+ variables modifiables
                    \def\fonction{0.5*x+2}
                    \def\xmin{-4.5}
                    \def\xmax{4.5}
                    \def\ymin{-0.9}
                    \def\ymax{5.2}
                    \def\xunit{1.5}  % unités en cm
                    \def\yunit{1.5}
                    \psset{xunit=\xunit,yunit=\yunit,algebraic=true}
                    \fontsize{15pt}{15pt}\selectfont
                    \begin{pspicture*}[linewidth=1pt](\xmin,\ymin)(\xmax,\ymax)
                         %      \psgrid[gridcolor=mcgris, subgriddiv=5, gridlabels=0pt](\xmin,\ymin)(\xmax,\ymax)
                         \psaxes[linewidth=0.75pt,Dx=1,Dy=1]{->}(0,0)(\xmin,\ymin)(\xmax,\ymax)
                         \psplot[plotpoints=2000,linecolor=blue]{\xmin}{\xmax}{\fonction}
                           \rput[tr](-0.2,-0.2){$O$} 
                                     \rput[tl](4,4.7){$\color{blue} \mathcal{C}_f$}
                    \end{pspicture*}
               }
\end{extern}
\end{center}   
     \begin{center}\textit{Courbe représentative de la fonction affine $x\mapsto \frac{1}{2}x+2$}\end{center}
}
\cadre{vert}{Propriété}{% id="p70"
     Soit $f$ une fonction affine de représentation graphique $\mathscr D$ et soient $A$ et $B$ deux points de $\mathscr D$.
     \par
     Le rapport $\dfrac{y_B-y_A}{x_B-x_A}$ ne dépend pas des points $A$ et $B$ choisis et est égal au coefficient directeur de la droite $\mathscr D$ :
     \begin{center}$a = \dfrac{y_B-y_A}{x_B-x_A}$\end{center}
}
\begin{center}
\begin{extern}%width="400" alt="coefficient directeur"
               \resizebox{11cm}{!}{%
                    % -+-+-+ variables modifiables
                    \def\fonction{0.5*x+2}
                    \def\xmin{-4.5}
                    \def\xmax{4.5}
                    \def\ymin{-0.9}
                    \def\ymax{5.2}
                    \def\xunit{1.5}  % unités en cm
                    \def\yunit{1.5}
                    \psset{xunit=\xunit,yunit=\yunit,algebraic=true}
                    \fontsize{15pt}{15pt}\selectfont
                    \begin{pspicture*}[linewidth=1pt](\xmin,\ymin)(\xmax,\ymax)
                         %      \psgrid[gridcolor=mcgris, subgriddiv=5, gridlabels=0pt](\xmin,\ymin)(\xmax,\ymax)
                         \psaxes[linewidth=0.75pt,Dx=1,Dy=1]{->}(0,0)(\xmin,\ymin)(\xmax,\ymax)
                         \psplot[plotpoints=2000,linecolor=blue]{\xmin}{\xmax}{\fonction}
                           \rput[tr](-0.2,-0.2){$O$} 
                                     \rput[tl](4,4.7){$\color{blue} \mathcal{C}_f$}
                                     \psdots[linecolor=red](-2,1)\psdots[linecolor=red](1,2.5)
                                     \psline[linewidth=0.75pt,linecolor=mcvert,arrowsize=6pt]{->}(-2,1)(1,1)
                                      \psline[linewidth=0.75pt,linecolor=mcmauve,arrowsize=6pt]{->}(1,1)(1,2.5)
                                       \rput[tr](-0.2,0.65){$\color{mcvert} x_B-x_A$}
                                      \rput[br](-2.1,1.1){$\color{red} A$} \rput[br](0.9,2.6){$\color{red} B$}
                                       \rput[l](1.2,1.7){$\color{mcmauve} y_B-y_A$}
                    \end{pspicture*}
               }
\end{extern}
\end{center} 
\begin{center}\textit{Lecture graphique du coefficient directeur $a = \dfrac{y_B-y_A}{x_B-x_A}=\dfrac{1,5}{3}=0,5$}\end{center}
  \cadre{rouge}{Théorème}{% id="t80"
     Une fonction affine est :
     \begin{itemize}
          \item \textbf{strictement croissante} si son coefficient directeur est \textbf{strictement positif}.
          \item \textbf{strictement décroissante} si son coefficient directeur est \textbf{strictement négatif}.
          \item \textbf{constante} si son coefficient directeur est \textbf{nul}.
     \end{itemize}
}
\bloc{orange}{Démonstration}{% id="m80"
     Démontrons par exemple que la fonction $f : x\mapsto ax+b$ est strictement décroissante si $a < 0$
     \par
     Soient deux réels $x_1$ et $x_2$ tels que $x_1 < x_2$
     \par
     Alors $ax_1 > ax_2$ (on change le sens de l'inégalité car on multiplie par un réel négatif) donc
     \par
     $ax_1+b > ax_2+b$ c'est à dire :
     \par
     $f\left(x_1\right) > f\left(x_2\right)$
     \par
     Le sens de l'inégalité est inversé donc $f$ est décroissante sur $\mathbb{R}$.
}
\bloc{cyan}{Remarque}{% id="r80"
     Ce théorème s'applique aussi aux fonctions linéaires puisque les fonctions linéaires sont des fonctions affines particulières.
}

\end{document}