\documentclass[a4paper]{article}

%================================================================================================================================
%
% Packages
%
%================================================================================================================================

\usepackage[T1]{fontenc} 	% pour caractères accentués
\usepackage[utf8]{inputenc}  % encodage utf8
\usepackage[french]{babel}	% langue : français
\usepackage{fourier}			% caractères plus lisibles
\usepackage[dvipsnames]{xcolor} % couleurs
\usepackage{fancyhdr}		% réglage header footer
\usepackage{needspace}		% empêcher sauts de page mal placés
\usepackage{graphicx}		% pour inclure des graphiques
\usepackage{enumitem,cprotect}		% personnalise les listes d'items (nécessaire pour ol, al ...)
\usepackage{hyperref}		% Liens hypertexte
\usepackage{pstricks,pst-all,pst-node,pstricks-add,pst-math,pst-plot,pst-tree,pst-eucl} % pstricks
\usepackage[a4paper,includeheadfoot,top=2cm,left=3cm, bottom=2cm,right=3cm]{geometry} % marges etc.
\usepackage{comment}			% commentaires multilignes
\usepackage{amsmath,environ} % maths (matrices, etc.)
\usepackage{amssymb,makeidx}
\usepackage{bm}				% bold maths
\usepackage{tabularx}		% tableaux
\usepackage{colortbl}		% tableaux en couleur
\usepackage{fontawesome}		% Fontawesome
\usepackage{environ}			% environment with command
\usepackage{fp}				% calculs pour ps-tricks
\usepackage{multido}			% pour ps tricks
\usepackage[np]{numprint}	% formattage nombre
\usepackage{tikz,tkz-tab} 			% package principal TikZ
\usepackage{pgfplots}   % axes
\usepackage{mathrsfs}    % cursives
\usepackage{calc}			% calcul taille boites
\usepackage[scaled=0.875]{helvet} % font sans serif
\usepackage{svg} % svg
\usepackage{scrextend} % local margin
\usepackage{scratch} %scratch
\usepackage{multicol} % colonnes
%\usepackage{infix-RPN,pst-func} % formule en notation polanaise inversée
\usepackage{listings}

%================================================================================================================================
%
% Réglages de base
%
%================================================================================================================================

\lstset{
language=Python,   % R code
literate=
{á}{{\'a}}1
{à}{{\`a}}1
{ã}{{\~a}}1
{é}{{\'e}}1
{è}{{\`e}}1
{ê}{{\^e}}1
{í}{{\'i}}1
{ó}{{\'o}}1
{õ}{{\~o}}1
{ú}{{\'u}}1
{ü}{{\"u}}1
{ç}{{\c{c}}}1
{~}{{ }}1
}


\definecolor{codegreen}{rgb}{0,0.6,0}
\definecolor{codegray}{rgb}{0.5,0.5,0.5}
\definecolor{codepurple}{rgb}{0.58,0,0.82}
\definecolor{backcolour}{rgb}{0.95,0.95,0.92}

\lstdefinestyle{mystyle}{
    backgroundcolor=\color{backcolour},   
    commentstyle=\color{codegreen},
    keywordstyle=\color{magenta},
    numberstyle=\tiny\color{codegray},
    stringstyle=\color{codepurple},
    basicstyle=\ttfamily\footnotesize,
    breakatwhitespace=false,         
    breaklines=true,                 
    captionpos=b,                    
    keepspaces=true,                 
    numbers=left,                    
xleftmargin=2em,
framexleftmargin=2em,            
    showspaces=false,                
    showstringspaces=false,
    showtabs=false,                  
    tabsize=2,
    upquote=true
}

\lstset{style=mystyle}


\lstset{style=mystyle}
\newcommand{\imgdir}{C:/laragon/www/newmc/assets/imgsvg/}
\newcommand{\imgsvgdir}{C:/laragon/www/newmc/assets/imgsvg/}

\definecolor{mcgris}{RGB}{220, 220, 220}% ancien~; pour compatibilité
\definecolor{mcbleu}{RGB}{52, 152, 219}
\definecolor{mcvert}{RGB}{125, 194, 70}
\definecolor{mcmauve}{RGB}{154, 0, 215}
\definecolor{mcorange}{RGB}{255, 96, 0}
\definecolor{mcturquoise}{RGB}{0, 153, 153}
\definecolor{mcrouge}{RGB}{255, 0, 0}
\definecolor{mclightvert}{RGB}{205, 234, 190}

\definecolor{gris}{RGB}{220, 220, 220}
\definecolor{bleu}{RGB}{52, 152, 219}
\definecolor{vert}{RGB}{125, 194, 70}
\definecolor{mauve}{RGB}{154, 0, 215}
\definecolor{orange}{RGB}{255, 96, 0}
\definecolor{turquoise}{RGB}{0, 153, 153}
\definecolor{rouge}{RGB}{255, 0, 0}
\definecolor{lightvert}{RGB}{205, 234, 190}
\setitemize[0]{label=\color{lightvert}  $\bullet$}

\pagestyle{fancy}
\renewcommand{\headrulewidth}{0.2pt}
\fancyhead[L]{maths-cours.fr}
\fancyhead[R]{\thepage}
\renewcommand{\footrulewidth}{0.2pt}
\fancyfoot[C]{}

\newcolumntype{C}{>{\centering\arraybackslash}X}
\newcolumntype{s}{>{\hsize=.35\hsize\arraybackslash}X}

\setlength{\parindent}{0pt}		 
\setlength{\parskip}{3mm}
\setlength{\headheight}{1cm}

\def\ebook{ebook}
\def\book{book}
\def\web{web}
\def\type{web}

\newcommand{\vect}[1]{\overrightarrow{\,\mathstrut#1\,}}

\def\Oij{$\left(\text{O}~;~\vect{\imath},~\vect{\jmath}\right)$}
\def\Oijk{$\left(\text{O}~;~\vect{\imath},~\vect{\jmath},~\vect{k}\right)$}
\def\Ouv{$\left(\text{O}~;~\vect{u},~\vect{v}\right)$}

\hypersetup{breaklinks=true, colorlinks = true, linkcolor = OliveGreen, urlcolor = OliveGreen, citecolor = OliveGreen, pdfauthor={Didier BONNEL - https://www.maths-cours.fr} } % supprime les bordures autour des liens

\renewcommand{\arg}[0]{\text{arg}}

\everymath{\displaystyle}

%================================================================================================================================
%
% Macros - Commandes
%
%================================================================================================================================

\newcommand\meta[2]{    			% Utilisé pour créer le post HTML.
	\def\titre{titre}
	\def\url{url}
	\def\arg{#1}
	\ifx\titre\arg
		\newcommand\maintitle{#2}
		\fancyhead[L]{#2}
		{\Large\sffamily \MakeUppercase{#2}}
		\vspace{1mm}\textcolor{mcvert}{\hrule}
	\fi 
	\ifx\url\arg
		\fancyfoot[L]{\href{https://www.maths-cours.fr#2}{\black \footnotesize{https://www.maths-cours.fr#2}}}
	\fi 
}


\newcommand\TitreC[1]{    		% Titre centré
     \needspace{3\baselineskip}
     \begin{center}\textbf{#1}\end{center}
}

\newcommand\newpar{    		% paragraphe
     \par
}

\newcommand\nosp {    		% commande vide (pas d'espace)
}
\newcommand{\id}[1]{} %ignore

\newcommand\boite[2]{				% Boite simple sans titre
	\vspace{5mm}
	\setlength{\fboxrule}{0.2mm}
	\setlength{\fboxsep}{5mm}	
	\fcolorbox{#1}{#1!3}{\makebox[\linewidth-2\fboxrule-2\fboxsep]{
  		\begin{minipage}[t]{\linewidth-2\fboxrule-4\fboxsep}\setlength{\parskip}{3mm}
  			 #2
  		\end{minipage}
	}}
	\vspace{5mm}
}

\newcommand\CBox[4]{				% Boites
	\vspace{5mm}
	\setlength{\fboxrule}{0.2mm}
	\setlength{\fboxsep}{5mm}
	
	\fcolorbox{#1}{#1!3}{\makebox[\linewidth-2\fboxrule-2\fboxsep]{
		\begin{minipage}[t]{1cm}\setlength{\parskip}{3mm}
	  		\textcolor{#1}{\LARGE{#2}}    
 	 	\end{minipage}  
  		\begin{minipage}[t]{\linewidth-2\fboxrule-4\fboxsep}\setlength{\parskip}{3mm}
			\raisebox{1.2mm}{\normalsize\sffamily{\textcolor{#1}{#3}}}						
  			 #4
  		\end{minipage}
	}}
	\vspace{5mm}
}

\newcommand\cadre[3]{				% Boites convertible html
	\par
	\vspace{2mm}
	\setlength{\fboxrule}{0.1mm}
	\setlength{\fboxsep}{5mm}
	\fcolorbox{#1}{white}{\makebox[\linewidth-2\fboxrule-2\fboxsep]{
  		\begin{minipage}[t]{\linewidth-2\fboxrule-4\fboxsep}\setlength{\parskip}{3mm}
			\raisebox{-2.5mm}{\sffamily \small{\textcolor{#1}{\MakeUppercase{#2}}}}		
			\par		
  			 #3
 	 		\end{minipage}
	}}
		\vspace{2mm}
	\par
}

\newcommand\bloc[3]{				% Boites convertible html sans bordure
     \needspace{2\baselineskip}
     {\sffamily \small{\textcolor{#1}{\MakeUppercase{#2}}}}    
		\par		
  			 #3
		\par
}

\newcommand\CHelp[1]{
     \CBox{Plum}{\faInfoCircle}{À RETENIR}{#1}
}

\newcommand\CUp[1]{
     \CBox{NavyBlue}{\faThumbsOUp}{EN PRATIQUE}{#1}
}

\newcommand\CInfo[1]{
     \CBox{Sepia}{\faArrowCircleRight}{REMARQUE}{#1}
}

\newcommand\CRedac[1]{
     \CBox{PineGreen}{\faEdit}{BIEN R\'EDIGER}{#1}
}

\newcommand\CError[1]{
     \CBox{Red}{\faExclamationTriangle}{ATTENTION}{#1}
}

\newcommand\TitreExo[2]{
\needspace{4\baselineskip}
 {\sffamily\large EXERCICE #1\ (\emph{#2 points})}
\vspace{5mm}
}

\newcommand\img[2]{
          \includegraphics[width=#2\paperwidth]{\imgdir#1}
}

\newcommand\imgsvg[2]{
       \begin{center}   \includegraphics[width=#2\paperwidth]{\imgsvgdir#1} \end{center}
}


\newcommand\Lien[2]{
     \href{#1}{#2 \tiny \faExternalLink}
}
\newcommand\mcLien[2]{
     \href{https~://www.maths-cours.fr/#1}{#2 \tiny \faExternalLink}
}

\newcommand{\euro}{\eurologo{}}

%================================================================================================================================
%
% Macros - Environement
%
%================================================================================================================================

\newenvironment{tex}{ %
}
{%
}

\newenvironment{indente}{ %
	\setlength\parindent{10mm}
}

{
	\setlength\parindent{0mm}
}

\newenvironment{corrige}{%
     \needspace{3\baselineskip}
     \medskip
     \textbf{\textsc{Corrigé}}
     \medskip
}
{
}

\newenvironment{extern}{%
     \begin{center}
     }
     {
     \end{center}
}

\NewEnviron{code}{%
	\par
     \boite{gray}{\texttt{%
     \BODY
     }}
     \par
}

\newenvironment{vbloc}{% boite sans cadre empeche saut de page
     \begin{minipage}[t]{\linewidth}
     }
     {
     \end{minipage}
}
\NewEnviron{h2}{%
    \needspace{3\baselineskip}
    \vspace{0.6cm}
	\noindent \MakeUppercase{\sffamily \large \BODY}
	\vspace{1mm}\textcolor{mcgris}{\hrule}\vspace{0.4cm}
	\par
}{}

\NewEnviron{h3}{%
    \needspace{3\baselineskip}
	\vspace{5mm}
	\textsc{\BODY}
	\par
}

\NewEnviron{margeneg}{ %
\begin{addmargin}[-1cm]{0cm}
\BODY
\end{addmargin}
}

\NewEnviron{html}{%
}

\begin{document}
\begin{h2}Exercice 2 (4 points)\end{h2}
\textbf{Commun à  tous les candidats}
\medbreak
\emph{Cet exercice est un questionnaire à choix multiples. Pour chaque question, une seule des quatre réponses proposées est correcte.\\
     Reporter sur la copie le numéro de la question ainsi que la lettre correspondant à la réponse choisie.\\
     Une réponse exacte rapporte 1 point. Une réponse fausse, une réponse multiple ou l'absence de réponse ne rappoete ni n'enlève aucun point. Aucune justification n'est demandée.\\
Les partie \emph{A} et \emph{B} sont indépendantes.}
     \TitreC{Partie A}
Dans un établissement scolaire, 30\,\% des élèves sont inscrits dans un club de sport, et parmi eux, 40\,\% sont des filles. Parmi ceux n'étant pas inscrits dans un club de sport, 50\,\% sont des garçons.
\smallbreak
\emph{Pour tout événement $E$, on note $\overline{E}$ l'événement contraire de $E$ et $p(E)$ sa probabilité. Pour tout événement $F$ de probabilité non nulle, on note $P_F(E)$ la probabilité de $E$ sachant que $F$ est réalisé.}
\medbreak
On interroge un élève au hasard et on considère les événements suivants~:
\begin{itemize}
     \item $S$~: \og l'élève est inscrit dans un club de sport \fg{}
     \item $F$~: \og l'élève est une fille \fg{}
\end{itemize}
     La situation est représentée par l'arbre pondéré ci-dessous :
        %:-+-+-+- Engendré par : http://math.et.info.free.fr/TikZ/Arbre/
          \begin{center}
               \begin{extern}%width="320"
                    % Racine à Gauche, développement vers la droite
                    \begin{tikzpicture}[xscale=1,yscale=1]
                         % Styles (MODIFIABLES)
                         \tikzstyle{fleche}=[-,>=latex,thick]
                         \tikzstyle{noeud}=[circle,draw]
                         \tikzstyle{feuille}=[circle,draw]
                         \tikzstyle{etiquette}=[midway,fill=white]
                         % Dimensions (MODIFIABLES)
                         \def\DistanceInterNiveaux{3}
                         \def\DistanceInterFeuilles{2}
                         % Dimensions calculées (NON MODIFIABLES)
                         \def\NiveauA{(0)*\DistanceInterNiveaux}
                         \def\NiveauB{(1.5)*\DistanceInterNiveaux}
                         \def\NiveauC{(2.5)*\DistanceInterNiveaux}
                         \def\InterFeuilles{(-1)*\DistanceInterFeuilles}
                         % Noeuds (MODIFIABLES : Styles et Coefficients d'InterFeuilles)
                         \node[noeud] (R) at ({\NiveauA},{(1.5)*\InterFeuilles}) {$ $};
                         \node[noeud] (Ra) at ({\NiveauB},{(0.5)*\InterFeuilles}) {$S$};
                         \node[feuille] (Raa) at ({\NiveauC},{(0)*\InterFeuilles}) {$F$};
                         \node[feuille] (Rab) at ({\NiveauC},{(1)*\InterFeuilles}) {$\overline{F}$};
                         \node[noeud] (Rb) at ({\NiveauB},{(2.5)*\InterFeuilles}) {$\overline{S}$};
                         \node[feuille] (Rba) at ({\NiveauC},{(2)*\InterFeuilles}) {$F$};
                         \node[feuille] (Rbb) at ({\NiveauC},{(3)*\InterFeuilles}) {$\overline{F}$};
                         % Arcs (MODIFIABLES : Styles)
                         \draw[fleche] (R)--(Ra) node[etiquette] {$0,3$};
                         \draw[fleche] (Ra)--(Raa) node[etiquette] {$0,4$};
                         \draw[fleche] (Ra)--(Rab) ;
                         \draw[fleche] (R)--(Rb) ;
                         \draw[fleche] (Rb)--(Rba) ;
                         \draw[fleche] (Rb)--(Rbb) node[etiquette] {$0,5$};
                    \end{tikzpicture}
               \end{extern}
          \end{center}
               \begin{enumerate}
          \item La probabilité $p_{\overline{F}}(S)$ est la probabilité que l'élève soit~:
          \begin{enumerate}[label=\alph*.]
               \item inscrit dans un club de sport sachant que c'est un garçon~;
               \item un garçon inscrit dans un club de sport~;
               \item inscrit dans un club de sport ou un garçon~;
               \item un garçon sachant qu'il est inscrit dans un club de sport.
          \end{enumerate}

     \item On admet que $P(F)=0,47$. La valeur arrondie de $P_{F}(S)$ est~:
     \begin{tabularx}{\linewidth}{XX}%class="cel50 noborder"
          \textbf{a.}~~$0,141$ & \textbf{b.}~~$0,255$ \\
           \textbf{c.}~~$0,400$ & \textbf{d.}~~$0,638$
     \end{tabularx}
     \medbreak
     \end{enumerate}
     \TitreC{Partie B}
Soit $g$ la fonction définie sur $[-1~;~4]$ par $g(x) = - x^3 + 3x^2 - 1$ et $\mathscr{C}_g$ sa courbe représentative dans un repère.
\begin{enumerate}
     \item La tangente à la courbe $\mathscr{C}_g$ au point d'abscisse 1 a pour équation~:
     \begin{center}
          \begin{tabularx}{\linewidth}{XX}%class="cel50 noborder"
               \textbf{a.}~~$y = - 3x^2 + 6x$ & \textbf{b.}~~$y= 3x-2$ \\
                \textbf{c.}~~$y= 3x - 3$ & \textbf{d.}~~$y = 2x - 1$
          \end{tabularx}
     \end{center}
     \item La valeur moyenne de la fonction $g$ sur l'intervalle $[-1~; a]$ est nulle pour~:
     \begin{center}
           \begin{tabularx}{\linewidth}{XX}%class="cel50 noborder"
              \textbf{a.}~~$a=0$ & \textbf{b.}~~$a=1$\\
               \textbf{c.}~~$a=2$ & \textbf{d.}~~$a=3$
          \end{tabularx}
     \end{center}
     \end{enumerate}


\end{document}