\documentclass[a4paper]{article}

%================================================================================================================================
%
% Packages
%
%================================================================================================================================

\usepackage[T1]{fontenc} 	% pour caractères accentués
\usepackage[utf8]{inputenc}  % encodage utf8
\usepackage[french]{babel}	% langue : français
\usepackage{fourier}			% caractères plus lisibles
\usepackage[dvipsnames]{xcolor} % couleurs
\usepackage{fancyhdr}		% réglage header footer
\usepackage{needspace}		% empêcher sauts de page mal placés
\usepackage{graphicx}		% pour inclure des graphiques
\usepackage{enumitem,cprotect}		% personnalise les listes d'items (nécessaire pour ol, al ...)
\usepackage{hyperref}		% Liens hypertexte
\usepackage{pstricks,pst-all,pst-node,pstricks-add,pst-math,pst-plot,pst-tree,pst-eucl} % pstricks
\usepackage[a4paper,includeheadfoot,top=2cm,left=3cm, bottom=2cm,right=3cm]{geometry} % marges etc.
\usepackage{comment}			% commentaires multilignes
\usepackage{amsmath,environ} % maths (matrices, etc.)
\usepackage{amssymb,makeidx}
\usepackage{bm}				% bold maths
\usepackage{tabularx}		% tableaux
\usepackage{colortbl}		% tableaux en couleur
\usepackage{fontawesome}		% Fontawesome
\usepackage{environ}			% environment with command
\usepackage{fp}				% calculs pour ps-tricks
\usepackage{multido}			% pour ps tricks
\usepackage[np]{numprint}	% formattage nombre
\usepackage{tikz,tkz-tab} 			% package principal TikZ
\usepackage{pgfplots}   % axes
\usepackage{mathrsfs}    % cursives
\usepackage{calc}			% calcul taille boites
\usepackage[scaled=0.875]{helvet} % font sans serif
\usepackage{svg} % svg
\usepackage{scrextend} % local margin
\usepackage{scratch} %scratch
\usepackage{multicol} % colonnes
%\usepackage{infix-RPN,pst-func} % formule en notation polanaise inversée
\usepackage{listings}

%================================================================================================================================
%
% Réglages de base
%
%================================================================================================================================

\lstset{
language=Python,   % R code
literate=
{á}{{\'a}}1
{à}{{\`a}}1
{ã}{{\~a}}1
{é}{{\'e}}1
{è}{{\`e}}1
{ê}{{\^e}}1
{í}{{\'i}}1
{ó}{{\'o}}1
{õ}{{\~o}}1
{ú}{{\'u}}1
{ü}{{\"u}}1
{ç}{{\c{c}}}1
{~}{{ }}1
}


\definecolor{codegreen}{rgb}{0,0.6,0}
\definecolor{codegray}{rgb}{0.5,0.5,0.5}
\definecolor{codepurple}{rgb}{0.58,0,0.82}
\definecolor{backcolour}{rgb}{0.95,0.95,0.92}

\lstdefinestyle{mystyle}{
    backgroundcolor=\color{backcolour},   
    commentstyle=\color{codegreen},
    keywordstyle=\color{magenta},
    numberstyle=\tiny\color{codegray},
    stringstyle=\color{codepurple},
    basicstyle=\ttfamily\footnotesize,
    breakatwhitespace=false,         
    breaklines=true,                 
    captionpos=b,                    
    keepspaces=true,                 
    numbers=left,                    
xleftmargin=2em,
framexleftmargin=2em,            
    showspaces=false,                
    showstringspaces=false,
    showtabs=false,                  
    tabsize=2,
    upquote=true
}

\lstset{style=mystyle}


\lstset{style=mystyle}
\newcommand{\imgdir}{C:/laragon/www/newmc/assets/imgsvg/}
\newcommand{\imgsvgdir}{C:/laragon/www/newmc/assets/imgsvg/}

\definecolor{mcgris}{RGB}{220, 220, 220}% ancien~; pour compatibilité
\definecolor{mcbleu}{RGB}{52, 152, 219}
\definecolor{mcvert}{RGB}{125, 194, 70}
\definecolor{mcmauve}{RGB}{154, 0, 215}
\definecolor{mcorange}{RGB}{255, 96, 0}
\definecolor{mcturquoise}{RGB}{0, 153, 153}
\definecolor{mcrouge}{RGB}{255, 0, 0}
\definecolor{mclightvert}{RGB}{205, 234, 190}

\definecolor{gris}{RGB}{220, 220, 220}
\definecolor{bleu}{RGB}{52, 152, 219}
\definecolor{vert}{RGB}{125, 194, 70}
\definecolor{mauve}{RGB}{154, 0, 215}
\definecolor{orange}{RGB}{255, 96, 0}
\definecolor{turquoise}{RGB}{0, 153, 153}
\definecolor{rouge}{RGB}{255, 0, 0}
\definecolor{lightvert}{RGB}{205, 234, 190}
\setitemize[0]{label=\color{lightvert}  $\bullet$}

\pagestyle{fancy}
\renewcommand{\headrulewidth}{0.2pt}
\fancyhead[L]{maths-cours.fr}
\fancyhead[R]{\thepage}
\renewcommand{\footrulewidth}{0.2pt}
\fancyfoot[C]{}

\newcolumntype{C}{>{\centering\arraybackslash}X}
\newcolumntype{s}{>{\hsize=.35\hsize\arraybackslash}X}

\setlength{\parindent}{0pt}		 
\setlength{\parskip}{3mm}
\setlength{\headheight}{1cm}

\def\ebook{ebook}
\def\book{book}
\def\web{web}
\def\type{web}

\newcommand{\vect}[1]{\overrightarrow{\,\mathstrut#1\,}}

\def\Oij{$\left(\text{O}~;~\vect{\imath},~\vect{\jmath}\right)$}
\def\Oijk{$\left(\text{O}~;~\vect{\imath},~\vect{\jmath},~\vect{k}\right)$}
\def\Ouv{$\left(\text{O}~;~\vect{u},~\vect{v}\right)$}

\hypersetup{breaklinks=true, colorlinks = true, linkcolor = OliveGreen, urlcolor = OliveGreen, citecolor = OliveGreen, pdfauthor={Didier BONNEL - https://www.maths-cours.fr} } % supprime les bordures autour des liens

\renewcommand{\arg}[0]{\text{arg}}

\everymath{\displaystyle}

%================================================================================================================================
%
% Macros - Commandes
%
%================================================================================================================================

\newcommand\meta[2]{    			% Utilisé pour créer le post HTML.
	\def\titre{titre}
	\def\url{url}
	\def\arg{#1}
	\ifx\titre\arg
		\newcommand\maintitle{#2}
		\fancyhead[L]{#2}
		{\Large\sffamily \MakeUppercase{#2}}
		\vspace{1mm}\textcolor{mcvert}{\hrule}
	\fi 
	\ifx\url\arg
		\fancyfoot[L]{\href{https://www.maths-cours.fr#2}{\black \footnotesize{https://www.maths-cours.fr#2}}}
	\fi 
}


\newcommand\TitreC[1]{    		% Titre centré
     \needspace{3\baselineskip}
     \begin{center}\textbf{#1}\end{center}
}

\newcommand\newpar{    		% paragraphe
     \par
}

\newcommand\nosp {    		% commande vide (pas d'espace)
}
\newcommand{\id}[1]{} %ignore

\newcommand\boite[2]{				% Boite simple sans titre
	\vspace{5mm}
	\setlength{\fboxrule}{0.2mm}
	\setlength{\fboxsep}{5mm}	
	\fcolorbox{#1}{#1!3}{\makebox[\linewidth-2\fboxrule-2\fboxsep]{
  		\begin{minipage}[t]{\linewidth-2\fboxrule-4\fboxsep}\setlength{\parskip}{3mm}
  			 #2
  		\end{minipage}
	}}
	\vspace{5mm}
}

\newcommand\CBox[4]{				% Boites
	\vspace{5mm}
	\setlength{\fboxrule}{0.2mm}
	\setlength{\fboxsep}{5mm}
	
	\fcolorbox{#1}{#1!3}{\makebox[\linewidth-2\fboxrule-2\fboxsep]{
		\begin{minipage}[t]{1cm}\setlength{\parskip}{3mm}
	  		\textcolor{#1}{\LARGE{#2}}    
 	 	\end{minipage}  
  		\begin{minipage}[t]{\linewidth-2\fboxrule-4\fboxsep}\setlength{\parskip}{3mm}
			\raisebox{1.2mm}{\normalsize\sffamily{\textcolor{#1}{#3}}}						
  			 #4
  		\end{minipage}
	}}
	\vspace{5mm}
}

\newcommand\cadre[3]{				% Boites convertible html
	\par
	\vspace{2mm}
	\setlength{\fboxrule}{0.1mm}
	\setlength{\fboxsep}{5mm}
	\fcolorbox{#1}{white}{\makebox[\linewidth-2\fboxrule-2\fboxsep]{
  		\begin{minipage}[t]{\linewidth-2\fboxrule-4\fboxsep}\setlength{\parskip}{3mm}
			\raisebox{-2.5mm}{\sffamily \small{\textcolor{#1}{\MakeUppercase{#2}}}}		
			\par		
  			 #3
 	 		\end{minipage}
	}}
		\vspace{2mm}
	\par
}

\newcommand\bloc[3]{				% Boites convertible html sans bordure
     \needspace{2\baselineskip}
     {\sffamily \small{\textcolor{#1}{\MakeUppercase{#2}}}}    
		\par		
  			 #3
		\par
}

\newcommand\CHelp[1]{
     \CBox{Plum}{\faInfoCircle}{À RETENIR}{#1}
}

\newcommand\CUp[1]{
     \CBox{NavyBlue}{\faThumbsOUp}{EN PRATIQUE}{#1}
}

\newcommand\CInfo[1]{
     \CBox{Sepia}{\faArrowCircleRight}{REMARQUE}{#1}
}

\newcommand\CRedac[1]{
     \CBox{PineGreen}{\faEdit}{BIEN R\'EDIGER}{#1}
}

\newcommand\CError[1]{
     \CBox{Red}{\faExclamationTriangle}{ATTENTION}{#1}
}

\newcommand\TitreExo[2]{
\needspace{4\baselineskip}
 {\sffamily\large EXERCICE #1\ (\emph{#2 points})}
\vspace{5mm}
}

\newcommand\img[2]{
          \includegraphics[width=#2\paperwidth]{\imgdir#1}
}

\newcommand\imgsvg[2]{
       \begin{center}   \includegraphics[width=#2\paperwidth]{\imgsvgdir#1} \end{center}
}


\newcommand\Lien[2]{
     \href{#1}{#2 \tiny \faExternalLink}
}
\newcommand\mcLien[2]{
     \href{https~://www.maths-cours.fr/#1}{#2 \tiny \faExternalLink}
}

\newcommand{\euro}{\eurologo{}}

%================================================================================================================================
%
% Macros - Environement
%
%================================================================================================================================

\newenvironment{tex}{ %
}
{%
}

\newenvironment{indente}{ %
	\setlength\parindent{10mm}
}

{
	\setlength\parindent{0mm}
}

\newenvironment{corrige}{%
     \needspace{3\baselineskip}
     \medskip
     \textbf{\textsc{Corrigé}}
     \medskip
}
{
}

\newenvironment{extern}{%
     \begin{center}
     }
     {
     \end{center}
}

\NewEnviron{code}{%
	\par
     \boite{gray}{\texttt{%
     \BODY
     }}
     \par
}

\newenvironment{vbloc}{% boite sans cadre empeche saut de page
     \begin{minipage}[t]{\linewidth}
     }
     {
     \end{minipage}
}
\NewEnviron{h2}{%
    \needspace{3\baselineskip}
    \vspace{0.6cm}
	\noindent \MakeUppercase{\sffamily \large \BODY}
	\vspace{1mm}\textcolor{mcgris}{\hrule}\vspace{0.4cm}
	\par
}{}

\NewEnviron{h3}{%
    \needspace{3\baselineskip}
	\vspace{5mm}
	\textsc{\BODY}
	\par
}

\NewEnviron{margeneg}{ %
\begin{addmargin}[-1cm]{0cm}
\BODY
\end{addmargin}
}

\NewEnviron{html}{%
}

\begin{document}
\begin{h2} Méthode~: \end{h2}
$ ABC $ et $ DCE $ sont deux triangles tels que~:
\begin{itemize}
     \item
     les points $ C, A, E $ et les points $ C, B, D $ sont alignés
     \item
     les droites $ \left( AB \right) $ et $ \left( ED \right) $ sont \textbf{ parallèles}.
\end{itemize}
\medskip
Deux cas de figure sont possibles~:
\begin{multicols}{2}
     \begin{center}
          \begin{extern}%width="300" alt="configuration de Thalès n°1"
               \psset{xunit=1.0cm,yunit=1.0cm,algebraic=true,dimen=middle,dotstyle=o,dotsize=5pt 0,linewidth=1.6pt,arrowsize=3pt 2,arrowinset=0.25}
               \newrgbcolor{tttttt}{0.2 0.2 0.2}
               \psset{xunit=1.0cm,yunit=1.0cm,algebraic=true,dimen=middle,dotstyle=o,dotsize=5pt 0,linewidth=1.6pt,arrowsize=3pt 2,arrowinset=0.25}
               \begin{pspicture*}(5.,1.)(11.,9.)
                    \psline[linewidth=0.4pt,linecolor=tttttt](6.,5.)(8.,4.)
                    \psline[linewidth=0.4pt,linecolor=tttttt](6.,3.)(10.,5.)
                    \psline[linewidth=0.4pt,linecolor=tttttt](10.,5.)(6.,7.)
                    \psline[linewidth=0.4pt,linecolor=tttttt](6.,7.)(6.,3.)
                    \begin{scriptsize}
                         \psdots[dotsize=2pt 0,dotstyle=*,linecolor=tttttt](6.,5.)
                         \rput[bl](5.677819005765453,5.027022512141235){\tttttt{$A$}}
                         \psdots[dotsize=2pt 0,dotstyle=*,linecolor=tttttt](8.,4.)
                         \rput[bl](8.150994540838274,3.9073261256157292){\tttttt{$B$}}
                         \psdots[dotsize=2pt 0,dotstyle=*,linecolor=tttttt](6.,7.)
                         \rput[bl](5.727036429348992,7.06954559085809){\tttttt{$E$}}
                         \psdots[dotsize=2pt 0,dotstyle=*](10.,5.)
                         \rput[bl](10.045865348804515,5.051631223933004){\tttttt{$D$}}
                         \psdots[dotsize=2pt 0,dotstyle=*,linecolor=tttttt](6.,3.)
                         \rput[bl](5.628601582181914,2.7999340949861087){\tttttt{$C$}}
                    \end{scriptsize}
               \end{pspicture*}
          \end{extern}
     \end{center}
     \columnbreak
     \begin{center}
          \begin{extern}%width="300" alt="configuration de Thalès n°2"
               \psset{xunit=1.0cm,yunit=1.0cm,algebraic=true,dimen=middle,dotstyle=o,dotsize=5pt 0,linewidth=1.6pt,arrowsize=3pt 2,arrowinset=0.25}
               \newrgbcolor{tttttt}{0.2 0.2 0.2}
               \psset{xunit=1.0cm,yunit=1.0cm,algebraic=true,dimen=middle,dotstyle=o,dotsize=5pt 0,linewidth=1.6pt,arrowsize=3pt 2,arrowinset=0.25}
               \begin{pspicture*}(4.,1.)(11.,8.)
                    \psline[linewidth=0.4pt,linecolor=tttttt](5.,3.)(7.,2.)
                    \psline[linewidth=0.4pt,linecolor=tttttt](6.,7.)(7.,2.)
                    \psline[linewidth=0.4pt,linecolor=tttttt](5.,3.)(8.673155137107534,5.663422431446233)
                    \psline[linewidth=0.4pt,linecolor=tttttt](8.673155137107534,5.663422431446233)(6.,7.)
                    \begin{scriptsize}
                         \psdots[dotsize=2pt 0,dotstyle=*,linecolor=tttttt](5.,3.)
                         \rput[bl](4.681166178198794,3.021412501112033){\tttttt{$A$}}
                         \psdots[dotsize=2pt 0,dotstyle=*,linecolor=tttttt](7.,2.)
                         \rput[bl](7.14203735737573,1.9017161145865278){\tttttt{$B$}}
                         \psdots[dotsize=2pt 0,dotstyle=*,linecolor=tttttt](6.,7.)
                         \rput[bl](5.727036429348992,7.06954559085809){\tttttt{$D$}}
                         \psdots[dotsize=2pt 0,dotstyle=*](8.673155137107534,5.663422431446233)
                         \rput[bl](8.864647182799585,5.543805459768391){\tttttt{$E$}}
                         \psdots[dotsize=2pt 0,dotstyle=*,linecolor=tttttt](6.572023624013923,4.139881879930384)
                         \rput[bl](6.723689256915651,3.9196304815116143){\tttttt{$C$}}
                    \end{scriptsize}
               \end{pspicture*}
          \end{extern}
     \end{center}
\end{multicols}
Le \mcLien{https://www.maths-cours.fr/cours/theoreme-thales/\#d10}{ \textbf{théorème de Thalès}} conclut que les longueurs des côtés du triangle $ ABC $ et les longueurs des côtés du triangle $ CDE $ sont proportionnelles, c'est à dire que~:
\begin{center}
     $ \frac{ CA }{ CE } = \frac{ CB }{ CD } = \frac{ AB }{ DE } $
\end{center}
\medskip
\par \textbf{Attention} à l'ordre des côtés - voir \mcLien{https://www.maths-cours.fr/cours/theoreme-thales/\#r10}{cours}~!
\medskip
En général, l'un des trois rapports sera inutile pour résoudre l'exercice.
\par
À partir des deux autres rapports, on peut calculer le côté recherché en effectuant, par exemple, un produit en croix.
\par
\begin{h2}Exemple 1\end{h2}
\begin{center}
     \begin{extern}%width="500" alt="Thalès exercice 1"
          \newrgbcolor{grey}{0.2 0.2 0.2}
          \psset{xunit=1.0cm,yunit=1.0cm,algebraic=true,dimen=middle,dotstyle=o,dotsize=5pt 0,linewidth=1.6pt,arrowsize=3pt 2,arrowinset=0.25}
          \begin{pspicture*}(1.,4.)(10.,9.)
               \psline[linewidth=0.4pt,linecolor=grey](2.,8.)(6.,8.)
               \psline[linewidth=0.4pt,linecolor=grey](6.,8.)(3.,5.)
               \psline[linewidth=0.4pt,linecolor=grey](3.,5.)(8.,5.)
               \psline[linewidth=0.4pt,linecolor=grey](8.,5.)(2.,8.)
               \rput[tl](7.338907051709888,7.241806573400475){$\grey{(MN)//(RS)}$}
               \begin{scriptsize}
                    \psdots[dotsize=2pt 0,dotstyle=*,linecolor=grey](2.,8.)
                    \rput[bl](1.7281207631864715,8.053894062528864){\grey{$R$}}
                    \psdots[dotsize=2pt 0,dotstyle=*,linecolor=grey](6.,8.)
                    \rput[bl](6.046949682641996,8.053894062528864){\grey{$S$}}
                    \psdots[dotsize=2pt 0,dotstyle=*,linecolor=grey](3.,5.)
                    \rput[bl](2.675556167169592,5.076239935724772){\grey{$M$}}
                    \psdots[dotsize=2pt 0,dotstyle=*,linecolor=grey](8.,5.)
                    \rput[bl](8.0525596936712,5.051631223933002){\grey{$N$}}
                    \psdots[dotsize=2pt 0,dotstyle=*,linecolor=grey](4.666666666666667,6.666666666666667)
                    \rput[bl](4.71807924588645,6.318979881209124){\grey{$I$}}
               \end{scriptsize}
          \end{pspicture*}
     \end{extern}
\end{center}
Calculer $ MI $ sachant que $ RS = 4 \texttt{cm} $ , $ MN = 5 \texttt{cm} $ et $ IS = 2 \texttt{cm} . $
\bloc{orange}{Solution~:}{ % id=s010
     \par
     On se place dans les triangles $ RSI $ et $ MNI $~;
     \par
     On précise les conditions d'alignement et de parallélisme~:
     \begin{itemize}
          \item
          les points $ R, I, N $ ainsi que les points $ M, I, S $ sont alignés
          \item
          les droites $ \left( MN \right) $ et $ \left( RS \right) $ sont parallèles.
     \end{itemize}
     Par conséquent, d'après le théorème de Thalès, on peut écrire~:
     \begin{center}
          $ \frac{ RI }{ IN} = \frac{ SI }{ IM } = \frac{ RS }{ MN } $
     \end{center}
     Ici, le rapport $ \frac{ RI }{ IN } $ ne nous intéresse pas car on ne connait ni la longueur $ RI $ ni la longueur $ IN $ .
     \par
     On utilise les deux autres rapports en remplaçant les longueurs connues par leurs valeurs~:
     \par
     $ \frac{ 2 }{ MI } $ = $\frac{4 }{ 5 } $
     \par
     En effectuant le produit en croix, on obtient~:
     \par
     $ 4MI = 2 \times 5 $
     \\
     $ 4MI=10 $
     \\
     $ MI= \frac{ 10}{ 4 } = 2,5 \texttt{cm} $
} % fin solution
\par
\begin{h2}Exemple 2 \end{h2}
\begin{center}
     \begin{extern}%width="500" alt="Thalès exercice 2"
          \newrgbcolor{grey}{0.2 0.2 0.2}
          \psset{xunit=1.0cm,yunit=1.0cm,algebraic=true,dimen=middle,dotstyle=o,dotsize=5pt 0,linewidth=1.6pt,arrowsize=3pt 2,arrowinset=0.25}
          \newrgbcolor{grey}{0.2 0.2 0.2}
          \psset{xunit=1.0cm,yunit=1.0cm,algebraic=true,dimen=middle,dotstyle=o,dotsize=5pt 0,linewidth=1.6pt,arrowsize=3pt 2,arrowinset=0.25}
          \begin{pspicture*}(0.,3.)(10.,8.)
               \rput[tl](7.77199779622993,6.47589412117782){$\grey{(BC)//(DE)}$}
               \psline[linewidth=.4pt,linecolor=grey](0.9365246496171764,3.624349766411451)(6.,7.)
               \psline[linewidth=.4pt,linecolor=grey](6.,7.)(9.37565023358855,3.624349766411451)
               \psline[linewidth=.4pt,linecolor=grey](9.37565023358855,3.624349766411451)(0.9365246496171764,3.624349766411451)
               \psline[linewidth=.4pt,linecolor=grey](3.,5.)(8.,5.)
               \begin{scriptsize}
                    \psdots[dotsize=2pt 0,dotstyle=*,linecolor=grey](6.,7.)
                    \rput[bl](5.858511062628407,7.186290393532876){\grey{$A$}}
                    \psdots[dotsize=2pt 0,dotstyle=*,linecolor=grey](3.,5.)
                    \rput[bl](2.684643845816301,5.0207275632892365){\grey{$B$}}
                    \psdots[dotsize=2pt 0,dotstyle=*,linecolor=grey](8.,5.)
                    \rput[bl](8.046989901657694,5.043643572074884){\grey{$C$}}
                    \psdots[dotsize=2pt 0,dotstyle=*](0.9365246496171764,3.624349766411451)
                    \rput[bl](0.6565770682865436,3.4166069482939485){$D$}
                    \psdots[dotsize=2pt 0,dotstyle=*,linecolor=grey](9.37565023358855,3.624349766411451)
                    \rput[bl](9.5136144639391,3.4395229570795953){\grey{$E$}}
               \end{scriptsize}
          \end{pspicture*}
     \end{extern}
\end{center}
On donne~: $ AC =3 \texttt{cm} $, $ AD = 5 \texttt{cm} $ et $ AE = 4 \texttt{cm} $.
\\
Calculer $ BD. $
\bloc{orange}{Solution~:}{ % id=s040
     \par
     Les triangles $ ABC $ et $ ADE$ sont en situation de Thalès car~:
     \begin{itemize}
          \item
          les points $ A, B, D $ ainsi que les points $ A, C, E $ sont alignés
          \item
          les droites $ \left( BC \right) $ et $ \left( DE \right) $ sont parallèles.
          \par
          En utilisant le théorème de Thalès, on obtient~:
          \begin{center}
               $ \frac{ AB }{ AD } = \frac{ AC }{ AE } = \frac{ BC }{ DE } $
          \end{center}
          \bloc{cyan}{Remarque}{ % id=r020
               Il ne faut pas chercher à tout prix à faire figurer la longueur recherchée (ici $ BD $) dans l'égalité des rapports.
               \\
               En effet, $ \left[ BD \right] $ n'est \textbf{pas} un côté de l'un des triangles $ ABC $ ou $ ADE $. Toutefois, il sera facile de calculer $ BD $ une fois la longueur $ AB$ connue.
          } % fin remarque
          \par
          L'égalité des deux premiers quotients donne~:
          \par
          $ \frac{ AB }{ 5 } = \frac{ 3 }{ 4 } $
          \par
          Avec le produit en croix~:
          \par
          $ 4AB = 3 \times 5 $
          \\
          $ 4 AB = 15 $
          \par
          $ AB = \frac{ 15 }{ 4 } = 3,75 $
          \par
          Pour calculer la distance $ BD $, il suffit maintenant de faire~:
          \\
          $ BD = AD - AB = 5 - 3,75 = 1,25.$
          \par
          Donc le segment $ \left[ BD \right] $ mesure $ 1,25 $cm.
     \end{itemize}
} % fin solution
\par
\begin{h2}Exemple 3 \end{h2}
\begin{center}
     \begin{extern}%width="500" alt="Thalès exercice 2"
          \newrgbcolor{grey}{0.2 0.2 0.2}
          \psset{xunit=1.0cm,yunit=1.0cm,algebraic=true,dimen=middle,dotstyle=o,dotsize=5pt 0,linewidth=1.6pt,arrowsize=3pt 2,arrowinset=0.25}
          \newrgbcolor{grey}{0.2 0.2 0.2}
          \psset{xunit=1.0cm,yunit=1.0cm,algebraic=true,dimen=middle,dotstyle=o,dotsize=5pt 0,linewidth=1.6pt,arrowsize=3pt 2,arrowinset=0.25}
          \begin{pspicture*}(0.,3.)(10.,8.)
               \rput[tl](7.77199779622993,6.47589412117782){$\grey{(BC)//(DE)}$}
               \psline[linewidth=.4pt,linecolor=grey](0.9365246496171764,3.624349766411451)(6.,7.)
               \psline[linewidth=.4pt,linecolor=grey](6.,7.)(9.37565023358855,3.624349766411451)
               \psline[linewidth=.4pt,linecolor=grey](9.37565023358855,3.624349766411451)(0.9365246496171764,3.624349766411451)
               \psline[linewidth=.4pt,linecolor=grey](3.,5.)(8.,5.)
               \begin{scriptsize}
                    \psdots[dotsize=2pt 0,dotstyle=*,linecolor=grey](6.,7.)
                    \rput[bl](5.858511062628407,7.186290393532876){\grey{$A$}}
                    \psdots[dotsize=2pt 0,dotstyle=*,linecolor=grey](3.,5.)
                    \rput[bl](2.684643845816301,5.0207275632892365){\grey{$B$}}
                    \psdots[dotsize=2pt 0,dotstyle=*,linecolor=grey](8.,5.)
                    \rput[bl](8.046989901657694,5.043643572074884){\grey{$C$}}
                    \psdots[dotsize=2pt 0,dotstyle=*](0.9365246496171764,3.624349766411451)
                    \rput[bl](0.6565770682865436,3.4166069482939485){$D$}
                    \psdots[dotsize=2pt 0,dotstyle=*,linecolor=grey](9.37565023358855,3.624349766411451)
                    \rput[bl](9.5136144639391,3.4395229570795953){\grey{$E$}}
               \end{scriptsize}
          \end{pspicture*}
     \end{extern}
\end{center}
La figure est similaire à celle de ll'exercice précédent mais, cette fois, on connait~: $ BD = 8\texttt{cm} $, $ AE = 9\texttt{cm} $ et $ AC = 4 \texttt{cm} $.
\\
On cherche à calculer la longueur $ AB. $
\bloc{orange}{Solution~:}{ %
     \par
     Le raisonnement précédent donne, là aussi~:
     \begin{center}
          $ \frac{ AB }{ AD } = \frac{ AC }{ AE } = \frac{ BC }{ DE } $
     \end{center}
     Cette fois, le calcul est légèrement plus compliqué car on ne connait ni $ AD $ ni $ AB $ (que l'on recherche).
     \par
     On pose alors~: $ AB = x $. Par conséquent~:
     \\
     $ AD = AB + BD = x + 8 $
     \par
     À partir de l'égalité~:
     \par
     $ \frac{ AB }{ AD } = \frac{ AC }{ AE } $
     \par
     on obtient l'équation~:
     \par
     $ \frac{ x }{ x+8 } = \frac{ 4 }{ 9 } $
     \par
     et en effectuant le produit en croix~:
     \par
     $ 9x = 4 \left( x+8 \right) $
     \\
     $ 9x = 4x + 32 $
     \\
     $ 9x - 4x = 32 $
     \\
     $ 5x = 32 $
     \\
     $ x = \frac{ 32 }{ 5 } = 6,4$
     \par
     La longueur $ AB $ est donc $ 6,4 $ cm.
} % fin solution

\end{document}