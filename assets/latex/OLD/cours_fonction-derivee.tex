\documentclass[a4paper]{article}

%================================================================================================================================
%
% Packages
%
%================================================================================================================================

\usepackage[T1]{fontenc} 	% pour caractères accentués
\usepackage[utf8]{inputenc}  % encodage utf8
\usepackage[french]{babel}	% langue : français
\usepackage{fourier}			% caractères plus lisibles
\usepackage[dvipsnames]{xcolor} % couleurs
\usepackage{fancyhdr}		% réglage header footer
\usepackage{needspace}		% empêcher sauts de page mal placés
\usepackage{graphicx}		% pour inclure des graphiques
\usepackage{enumitem,cprotect}		% personnalise les listes d'items (nécessaire pour ol, al ...)
\usepackage{hyperref}		% Liens hypertexte
\usepackage{pstricks,pst-all,pst-node,pstricks-add,pst-math,pst-plot,pst-tree,pst-eucl} % pstricks
\usepackage[a4paper,includeheadfoot,top=2cm,left=3cm, bottom=2cm,right=3cm]{geometry} % marges etc.
\usepackage{comment}			% commentaires multilignes
\usepackage{amsmath,environ} % maths (matrices, etc.)
\usepackage{amssymb,makeidx}
\usepackage{bm}				% bold maths
\usepackage{tabularx}		% tableaux
\usepackage{colortbl}		% tableaux en couleur
\usepackage{fontawesome}		% Fontawesome
\usepackage{environ}			% environment with command
\usepackage{fp}				% calculs pour ps-tricks
\usepackage{multido}			% pour ps tricks
\usepackage[np]{numprint}	% formattage nombre
\usepackage{tikz,tkz-tab} 			% package principal TikZ
\usepackage{pgfplots}   % axes
\usepackage{mathrsfs}    % cursives
\usepackage{calc}			% calcul taille boites
\usepackage[scaled=0.875]{helvet} % font sans serif
\usepackage{svg} % svg
\usepackage{scrextend} % local margin
\usepackage{scratch} %scratch
\usepackage{multicol} % colonnes
%\usepackage{infix-RPN,pst-func} % formule en notation polanaise inversée
\usepackage{listings}

%================================================================================================================================
%
% Réglages de base
%
%================================================================================================================================

\lstset{
language=Python,   % R code
literate=
{á}{{\'a}}1
{à}{{\`a}}1
{ã}{{\~a}}1
{é}{{\'e}}1
{è}{{\`e}}1
{ê}{{\^e}}1
{í}{{\'i}}1
{ó}{{\'o}}1
{õ}{{\~o}}1
{ú}{{\'u}}1
{ü}{{\"u}}1
{ç}{{\c{c}}}1
{~}{{ }}1
}


\definecolor{codegreen}{rgb}{0,0.6,0}
\definecolor{codegray}{rgb}{0.5,0.5,0.5}
\definecolor{codepurple}{rgb}{0.58,0,0.82}
\definecolor{backcolour}{rgb}{0.95,0.95,0.92}

\lstdefinestyle{mystyle}{
    backgroundcolor=\color{backcolour},   
    commentstyle=\color{codegreen},
    keywordstyle=\color{magenta},
    numberstyle=\tiny\color{codegray},
    stringstyle=\color{codepurple},
    basicstyle=\ttfamily\footnotesize,
    breakatwhitespace=false,         
    breaklines=true,                 
    captionpos=b,                    
    keepspaces=true,                 
    numbers=left,                    
xleftmargin=2em,
framexleftmargin=2em,            
    showspaces=false,                
    showstringspaces=false,
    showtabs=false,                  
    tabsize=2,
    upquote=true
}

\lstset{style=mystyle}


\lstset{style=mystyle}
\newcommand{\imgdir}{C:/laragon/www/newmc/assets/imgsvg/}
\newcommand{\imgsvgdir}{C:/laragon/www/newmc/assets/imgsvg/}

\definecolor{mcgris}{RGB}{220, 220, 220}% ancien~; pour compatibilité
\definecolor{mcbleu}{RGB}{52, 152, 219}
\definecolor{mcvert}{RGB}{125, 194, 70}
\definecolor{mcmauve}{RGB}{154, 0, 215}
\definecolor{mcorange}{RGB}{255, 96, 0}
\definecolor{mcturquoise}{RGB}{0, 153, 153}
\definecolor{mcrouge}{RGB}{255, 0, 0}
\definecolor{mclightvert}{RGB}{205, 234, 190}

\definecolor{gris}{RGB}{220, 220, 220}
\definecolor{bleu}{RGB}{52, 152, 219}
\definecolor{vert}{RGB}{125, 194, 70}
\definecolor{mauve}{RGB}{154, 0, 215}
\definecolor{orange}{RGB}{255, 96, 0}
\definecolor{turquoise}{RGB}{0, 153, 153}
\definecolor{rouge}{RGB}{255, 0, 0}
\definecolor{lightvert}{RGB}{205, 234, 190}
\setitemize[0]{label=\color{lightvert}  $\bullet$}

\pagestyle{fancy}
\renewcommand{\headrulewidth}{0.2pt}
\fancyhead[L]{maths-cours.fr}
\fancyhead[R]{\thepage}
\renewcommand{\footrulewidth}{0.2pt}
\fancyfoot[C]{}

\newcolumntype{C}{>{\centering\arraybackslash}X}
\newcolumntype{s}{>{\hsize=.35\hsize\arraybackslash}X}

\setlength{\parindent}{0pt}		 
\setlength{\parskip}{3mm}
\setlength{\headheight}{1cm}

\def\ebook{ebook}
\def\book{book}
\def\web{web}
\def\type{web}

\newcommand{\vect}[1]{\overrightarrow{\,\mathstrut#1\,}}

\def\Oij{$\left(\text{O}~;~\vect{\imath},~\vect{\jmath}\right)$}
\def\Oijk{$\left(\text{O}~;~\vect{\imath},~\vect{\jmath},~\vect{k}\right)$}
\def\Ouv{$\left(\text{O}~;~\vect{u},~\vect{v}\right)$}

\hypersetup{breaklinks=true, colorlinks = true, linkcolor = OliveGreen, urlcolor = OliveGreen, citecolor = OliveGreen, pdfauthor={Didier BONNEL - https://www.maths-cours.fr} } % supprime les bordures autour des liens

\renewcommand{\arg}[0]{\text{arg}}

\everymath{\displaystyle}

%================================================================================================================================
%
% Macros - Commandes
%
%================================================================================================================================

\newcommand\meta[2]{    			% Utilisé pour créer le post HTML.
	\def\titre{titre}
	\def\url{url}
	\def\arg{#1}
	\ifx\titre\arg
		\newcommand\maintitle{#2}
		\fancyhead[L]{#2}
		{\Large\sffamily \MakeUppercase{#2}}
		\vspace{1mm}\textcolor{mcvert}{\hrule}
	\fi 
	\ifx\url\arg
		\fancyfoot[L]{\href{https://www.maths-cours.fr#2}{\black \footnotesize{https://www.maths-cours.fr#2}}}
	\fi 
}


\newcommand\TitreC[1]{    		% Titre centré
     \needspace{3\baselineskip}
     \begin{center}\textbf{#1}\end{center}
}

\newcommand\newpar{    		% paragraphe
     \par
}

\newcommand\nosp {    		% commande vide (pas d'espace)
}
\newcommand{\id}[1]{} %ignore

\newcommand\boite[2]{				% Boite simple sans titre
	\vspace{5mm}
	\setlength{\fboxrule}{0.2mm}
	\setlength{\fboxsep}{5mm}	
	\fcolorbox{#1}{#1!3}{\makebox[\linewidth-2\fboxrule-2\fboxsep]{
  		\begin{minipage}[t]{\linewidth-2\fboxrule-4\fboxsep}\setlength{\parskip}{3mm}
  			 #2
  		\end{minipage}
	}}
	\vspace{5mm}
}

\newcommand\CBox[4]{				% Boites
	\vspace{5mm}
	\setlength{\fboxrule}{0.2mm}
	\setlength{\fboxsep}{5mm}
	
	\fcolorbox{#1}{#1!3}{\makebox[\linewidth-2\fboxrule-2\fboxsep]{
		\begin{minipage}[t]{1cm}\setlength{\parskip}{3mm}
	  		\textcolor{#1}{\LARGE{#2}}    
 	 	\end{minipage}  
  		\begin{minipage}[t]{\linewidth-2\fboxrule-4\fboxsep}\setlength{\parskip}{3mm}
			\raisebox{1.2mm}{\normalsize\sffamily{\textcolor{#1}{#3}}}						
  			 #4
  		\end{minipage}
	}}
	\vspace{5mm}
}

\newcommand\cadre[3]{				% Boites convertible html
	\par
	\vspace{2mm}
	\setlength{\fboxrule}{0.1mm}
	\setlength{\fboxsep}{5mm}
	\fcolorbox{#1}{white}{\makebox[\linewidth-2\fboxrule-2\fboxsep]{
  		\begin{minipage}[t]{\linewidth-2\fboxrule-4\fboxsep}\setlength{\parskip}{3mm}
			\raisebox{-2.5mm}{\sffamily \small{\textcolor{#1}{\MakeUppercase{#2}}}}		
			\par		
  			 #3
 	 		\end{minipage}
	}}
		\vspace{2mm}
	\par
}

\newcommand\bloc[3]{				% Boites convertible html sans bordure
     \needspace{2\baselineskip}
     {\sffamily \small{\textcolor{#1}{\MakeUppercase{#2}}}}    
		\par		
  			 #3
		\par
}

\newcommand\CHelp[1]{
     \CBox{Plum}{\faInfoCircle}{À RETENIR}{#1}
}

\newcommand\CUp[1]{
     \CBox{NavyBlue}{\faThumbsOUp}{EN PRATIQUE}{#1}
}

\newcommand\CInfo[1]{
     \CBox{Sepia}{\faArrowCircleRight}{REMARQUE}{#1}
}

\newcommand\CRedac[1]{
     \CBox{PineGreen}{\faEdit}{BIEN R\'EDIGER}{#1}
}

\newcommand\CError[1]{
     \CBox{Red}{\faExclamationTriangle}{ATTENTION}{#1}
}

\newcommand\TitreExo[2]{
\needspace{4\baselineskip}
 {\sffamily\large EXERCICE #1\ (\emph{#2 points})}
\vspace{5mm}
}

\newcommand\img[2]{
          \includegraphics[width=#2\paperwidth]{\imgdir#1}
}

\newcommand\imgsvg[2]{
       \begin{center}   \includegraphics[width=#2\paperwidth]{\imgsvgdir#1} \end{center}
}


\newcommand\Lien[2]{
     \href{#1}{#2 \tiny \faExternalLink}
}
\newcommand\mcLien[2]{
     \href{https~://www.maths-cours.fr/#1}{#2 \tiny \faExternalLink}
}

\newcommand{\euro}{\eurologo{}}

%================================================================================================================================
%
% Macros - Environement
%
%================================================================================================================================

\newenvironment{tex}{ %
}
{%
}

\newenvironment{indente}{ %
	\setlength\parindent{10mm}
}

{
	\setlength\parindent{0mm}
}

\newenvironment{corrige}{%
     \needspace{3\baselineskip}
     \medskip
     \textbf{\textsc{Corrigé}}
     \medskip
}
{
}

\newenvironment{extern}{%
     \begin{center}
     }
     {
     \end{center}
}

\NewEnviron{code}{%
	\par
     \boite{gray}{\texttt{%
     \BODY
     }}
     \par
}

\newenvironment{vbloc}{% boite sans cadre empeche saut de page
     \begin{minipage}[t]{\linewidth}
     }
     {
     \end{minipage}
}
\NewEnviron{h2}{%
    \needspace{3\baselineskip}
    \vspace{0.6cm}
	\noindent \MakeUppercase{\sffamily \large \BODY}
	\vspace{1mm}\textcolor{mcgris}{\hrule}\vspace{0.4cm}
	\par
}{}

\NewEnviron{h3}{%
    \needspace{3\baselineskip}
	\vspace{5mm}
	\textsc{\BODY}
	\par
}

\NewEnviron{margeneg}{ %
\begin{addmargin}[-1cm]{0cm}
\BODY
\end{addmargin}
}

\NewEnviron{html}{%
}

\begin{document}
\begin{h2}1. Nombre dérivé\end{h2}
\cadre{bleu}{Définition}{% id="d10"
     Soit $f$ une fonction définie sur un intervalle $I$ et soient 2 réels $x_{0}$ et $h\neq 0$ tels que $x_{0} \in  I$ et $x_{0}+h \in  I$.
     \par
     Le \textbf{taux de variation} (ou \textbf{taux d'accroissement)} de la fonction $f$ entre $x_{0}$ et $x_{0}+h$ est le nombre :
     \par
     $T=\frac{f\left(x_{0}+h\right)-f\left(x_{0}\right)}{h}$
}
\cadre{bleu}{Définition}{% id="d20"
     Une fonction $f$ est \textbf{dérivable} en $x_{0}$ si et seulement si le nombre $\frac{f\left(x_{0}+h\right)-f\left(x_{0}\right)}{h}$ a pour limite un certain réel $l$ lorsque $h$ tend vers 0.
     \par
     $l$ est appelée \textbf{nombre dérivé} de $f$ en $x_{0}$, on le note $f^{\prime}\left(x_{0}\right)$.
     \par
     On écrit : $f^{\prime}\left(x_{0}\right)=\lim\limits_{h\rightarrow 0}\frac{f\left(x_{0}+h\right)-f\left(x_{0}\right)}{h}$.
}
\bloc{cyan}{Remarques}{% id="r20"
     \begin{itemize}
          \item Le quotient $\frac{f\left(x_{0}+h\right)-f\left(x_{0}\right)}{h}$ est le taux d'accroissement de $f$ entre $x_{0}$ et $x_{0}+h$.
          \item \textit{« le nombre $\frac{f\left(x_{0}+h\right)-f\left(x_{0}\right)}{h}$ a pour limite un certain réel $l$ lorsque $h$ tend vers 0 »} signifie que $\frac{f\left(x_{0}+h\right)-f\left(x_{0}\right)}{h}$ se rapproche de $l$ lorsque $h$ se rapproche de 0.
          \par
          Une définition plus rigoureuse de la notion de limite sera vue en Terminale.
          \item On peut également définir le nombre dérivé de la façon suivante:
          \par
          $f^{\prime}\left(x_{0}\right)=\lim\limits_{x\rightarrow  x_{0}}\frac{f\left(x\right)-f\left(x_{0}\right)}{x-x_{0}}$
          \par
          (cela correspond au changement de variable $x=x_{0}+h$)
     \end{itemize}
}
\bloc{orange}{Exemple}{% id="e20"
     Calculons le nombre dérivé de la fonction $f : x \mapsto x^{2}$ pour $x=1$.
     \par
     Ce nombre se note $f^{\prime}\left(1\right)$ et vaut :
     \par
     $f^{\prime}\left(1\right)=\lim\limits_{h\rightarrow 0}\frac{\left(1+h\right)^{2}-1^{2}}{h}=\lim\limits_{h\rightarrow 0}\frac{2h+h^{2}}{h}=\lim\limits_{h\rightarrow 0}2+h$
     \par
     Or quand $h$ tend vers 0, $2+h$ tend vers 2; donc $f^{\prime}\left(1\right)=2$.
}
\bloc{cyan}{Remarque:}{% id="r21" 
\textbf{Interprétation graphique du nombre dérivé :}
  \begin{center}
     \begin{extern}%width="450" alt="nombre dérivé"
          % -+-+-+ variables modifiables
          \resizebox{8cm}{!}{%
               \def\xmin{-2.5}
               \def\xmax{8.5}
               \def\ymin{-1.5}
               \def\ymax{8.5}
               \def\xunit{1}  % unités en cm
               \def\yunit{1}
               \psset{xunit=\xunit,yunit=\yunit,algebraic=true}
               \fontsize{12pt}{12pt}\selectfont
               \begin{pspicture*}[linewidth=1pt](\xmin,\ymin)(\xmax,\ymax)
              %      \psgrid[gridcolor=mcgris,subgriddiv=0](-4,-2)(9,9)
           %        \psaxes[linewidth=0.75pt]{->}(0,0)(\xmin,\ymin)(\xmax,\ymax)
        \psline[linecolor=gray]{->}(\xmin,0)(\xmax,0)
           \psline[linecolor=gray]{->}(0,\ymin)(0,\ymax)
                     \psline[linecolor=lightgray](\xmin,0)(\xmax,0)
           \psline[linecolor=lightgray](2,0)(2,1.69)
                    \psline[linecolor=lightgray](5,0)(5,3.713)
                    \rput[tr](-0.2,-0.2){$O$}\rput[br](1.8,1.8){$A$}\rput[br](4.8,3.693){$B$}\rput[t](3.5,-0.8){\color{mcmauve} $h$}
                    \rput[t](8,4){\color{red} $\mathscr{T}$}\rput[t](8,7.5){\color{blue} $\mathscr{C}_f$}\rput[t](2,-0.2){$x_0$}\rput[t](5,-0.1){$x_0+h$}
                    \psdots(2,1.69)\psdots(5,3.713)
                    \psplot[plotpoints=1000,linewidth=0.8pt,linecolor=blue]{\xmin}{\xmax}{1.3^x}
                    \psplot[plotpoints=1000,linewidth=0.8pt,linecolor=mcvert]{\xmin}{\xmax}{0.674*x+0.341}
                     \psplot[plotpoints=1000,linewidth=0.8pt,linecolor=red]{\xmin}{\xmax}{0.443*x+0.803}
                    \psline[linecolor=mcmauve]{<->}(2,-0.7)(5,-0.7)
               \end{pspicture*}
          }
     \end{extern}
\end{center}
       Soit $\mathscr{C}_f$ la courbe représentative de la fonction $f$.
     \par
     Lorsque $h$ tend vers 0, $B$ \textit{"se rapproche"} de $A$ et la droite $\left(AB\right)$ se rapproche de la tangente 
     $\mathscr{T}$.
     \par
     \textbf{Le nombre dérivée $f^{\prime}\left(x_{0}\right)$ est le coefficient directeur de la tangente à la courbe $\mathscr{C}_f$ au point d'abscisse $x_{0}$.}
}
\cadre{vert}{Propriété}{% id="p30"
     Soit $f$ une fonction dérivable en $x_{0}$ de courbe représentative $\mathscr{C}_f$, l'équation de la tangente à $\mathscr{C}_f$ au point d'abscisse $x_{0}$ est :
     \par
$y=f^{\prime}\left(x_{0}\right)\left(x-x_{0}\right)+f\left(x_{0}\right)$}
\bloc{orange}{Démonstration}{% id="m21"
     D'après la propriété précédente, la tangente à $\mathscr{C}_f$ au point d'abscisse $x_{0}$ est une droite de coefficient directeur $f^{\prime}\left(x_{0}\right)$. Son équation est donc de la forme :
     \par
     $y=f^{\prime}\left(x_{0}\right)x+b$
     \par
     On sait que la tangente passe par le point $A$ de coordonnées $\left(x_{0}; f\left(x_{0}\right)\right)$ donc :
     \par
     $f\left(x_{0}\right)=f^{\prime}\left(x_{0}\right)x_{0}+b$
     \par
     $b=-f^{\prime}\left(x_{0}\right)x_{0}+f\left(x_{0}\right)$
     \par
     L'équation de la tangente est donc :
     \par
     $y=f^{\prime}\left(x_{0}\right)x-f^{\prime}\left(x_{0}\right)x_{0}+f\left(x_{0}\right)$
     \par
     Soit :
     \par
     $y=f^{\prime}\left(x_{0}\right)\left(x-x_{0}\right)+f\left(x_{0}\right)$
}
\begin{h2}2. Fonction dérivée\end{h2}
\cadre{bleu}{Définition}{% id="d40"
     Soit $f$ une fonction définie sur un intervalle $I$. On dit que $f$ est \textbf{dérivable} sur $I$ si et seulement si pour tout $x \in  I$, le nombre dérivé $f^{\prime}\left(x\right)$ existe.
     \par
     La fonction qui à $x \in  I$ associe le nombre dérivé de $f$ en $x$ s'appelle la \textbf{fonction dérivée} et se note $f^{\prime}$
}
\cadre{vert}{Propriétés}{% id="p50"
     \textbf{Dérivée des fonctions usuelles :}
\begin{center}
     \begin{tabularx}{0.8\linewidth}{|X|X|X|}%class="compact" width="600"
          \hline
          \textbf{Fonction} & \textbf{Dérivée} & \textbf{Ensemble de dérivabilité}
\\ \hline
          $k$  $\left(k\in \mathbb{R}\right)$  &  $0$  &  $\mathbb{R}$
          \\ \hline
          $x$ &  $1$  &  $\mathbb{R}$
          \\ \hline
          $x^{n}$ $\left(n\in \mathbb{N}\right)$  &  $nx^{n-1}$  &  $\mathbb{R}$
          \\ \hline
          $\frac{1}{x^{n}}$ $\left(n\in \mathbb{N}\right)$ &  $-\frac{n}{x^{n+1}}$  &  $\mathbb{R}-\left\{0\right\}$
          \\ \hline
          $\sqrt{x}$ &  $\frac{1}{2\sqrt{x}}$  &  $\left]0;+\infty \right[$
  \\        \hline
     \end{tabularx}
\end{center}
}
\cadre{vert}{Propriétés}{% id="p60"
     \textbf{Formules de base :}
     \par
     Si $u$ et $v$ sont 2 fonctions dérivables sur un intervalle $I$. Sur cet intervalle :
\begin{center}
     \begin{tabularx}{0.8\linewidth}{|*{3}{>{\centering \arraybackslash }X|}}%class="compact" width="600"
          \hline
          \textbf{Fonction} & \textbf{Dérivée}
          \\ \hline
          $u+v$  &  $u^{\prime}+v^{\prime}$
          \\ \hline
          $ku$  $\left(k\in \mathbb{R}\right)$  &  $ku^{\prime} $
          \\ \hline
          $\frac{1}{u}$ (avec $u\left(x\right)\neq 0$ sur $I$)  &  $-\frac{u^{\prime} }{u^{2}} $
          \\ \hline
          $uv$  &  $u^{\prime}v+uv^{\prime}$
          \\ \hline
          $\frac{u}{v}$  (avec $v\left(x\right)\neq 0$ sur $I$) &  $\frac{u^{\prime}v-uv^{\prime}}{v^{2}} $
          \\ \hline
          $\sqrt{u}$  (avec $u\geqslant 0$ sur $I$) &  $\frac{u^{\prime}}{2\sqrt{u}}$ lorsque $u > 0$
         \\  \hline
     \end{tabularx}
\end{center}
}
\bloc{orange}{Exemple}{% id="e70"
     On cherche à calculer la dérivée de la fonction $f$ définie sur $\mathbb{R}$ par :
     \par
     $f\left(x\right)=\frac{x}{x^{2}+1}$
     \par
     On pose
     \par
     $u\left(x\right)=x$ et $v\left(x\right)=x^{2}+1$
     \par
     On a alors
     \par
     $u^{\prime}\left(x\right)=1$
     \par
     $v^{\prime}\left(x\right)=2x$
     \par
     car la dérivée de la fonction $x \mapsto  x^{2}$ est la fonction $x \mapsto  2x$  (formule $nx^{n-1}$ avec $n=2$) et la dérivée de la fonction constante $x \mapsto 1$ est la fonction nulle.
     \par
     La dérivée du quotient est donc :
     \par
     $f^{\prime}\left(x\right)=\frac{u^{\prime}\left(x\right)v\left(x\right)-u\left(x\right)v^{\prime}\left(x\right)}{v\left(x\right)^{2}}=\frac{1\times \left(x^{2}+1\right)-x\times 2x}{\left(x^{2}+1\right)^{2}}=\frac{1-x^{2}}{\left(x^{2}+1\right)^{2}}$
}
\bloc{cyan}{Remarques}{% id="r70"
     \begin{itemize}
          \item Si le dénominateur d'une fraction est constant, il est très maladroit d'utiliser la formule
          \par
          $\left(\frac{u}{v}\right)^{\prime}=\frac{u^{\prime}v-uv^{\prime}}{v^{2}}$.
          \par
          Par exemple pour dériver $f\left(x\right)=\frac{x^{2}+1}{5}$ on écrira :
          \par
          $f\left(x\right)=\frac{1}{5}\times \left(x^{2}+1\right)$
          \par
          donc $f^{\prime}\left(x\right)=\frac{1}{5}\times \left(2x\right)$ (formule $\left(ku\right)^{\prime}=ku^{\prime}$)
          \par
          $f^{\prime}\left(x\right)=\frac{2x}{5}$
          \item De même, si le numérateur d'une fraction est constant on utilisera, de préférence, la formule :
          \par
          $\left(\frac{1}{u}\right)^{\prime}=-\frac{u^{\prime}}{u^{2}}$
          \par
          Par exemple, si $f\left(x\right)=\frac{5}{x^{2}+1}$
          \par
          $f\left(x\right)=5\times \frac{1}{x^{2}+1}$ donc :
          \par
          $f^{\prime}\left(x\right)=5\times \left(-\frac{2x}{\left(x^{2}+1\right)^{2}}\right)=-\frac{10x}{\left(x^{2}+1\right)^{2}}$ (formule $\left(\frac{1}{u}\right)^{\prime}=-\frac{u^{\prime}}{u^{2}}$ avec $u\left(x\right)=x^{2}+1$ donc $u^{\prime}\left(x\right)=2x$)
     \end{itemize}
}
\begin{h2}3. Fonction dérivée et sens de variations\end{h2}
\cadre{rouge}{Théorème}{% id="t80"
     Soit $f$ une fonction définie sur un intervalle $I$.
     \begin{itemize}
          \item $f$ est croissante sur $I$ si et seulement si $f^{\prime}\left(x\right)\geqslant 0$ pour tout $x \in  I$
          \item $f$ est décroissante sur $I$ si et seulement si $f^{\prime}\left(x\right)\leqslant 0$ pour tout $x \in  I$
     \end{itemize}
}
\bloc{cyan}{Remarque}{% id="r80"
     Si $f^{\prime}\left(x\right) > 0$ (resp.  $f^{\prime}\left(x\right) < 0$) sur $I$, alors $f$ est \textbf{strictement} croissante (resp. décroissante) sur $I$.
     \par
     Mais la réciproque est fausse. Une fonction peut être strictement croissante sur $I$ alors que sa dérivée s'annule sur $I$. C'est le cas par exemple de la fonction $x \mapsto  x^{3}$ qui est strictement croissante sur $\mathbb{R}$ alors que sa dérivée $x \mapsto  3x^{2}$ s'annule pour $x=0$
}
\bloc{orange}{Exemple}{% id="e80"
     Reprenons la fonction de l'exemple précédent.
     \par
     $f\left(x\right)=\frac{x}{x^{2}+1}$
     \par
     $f^{\prime}\left(x\right)=\frac{1-x^{2}}{\left(x^{2}+1\right)^{2}}$
     \par
     Le dénominateur de $f^{\prime}\left(x\right)$ est toujours strictement positif.
     \par
     Le numérateur de $f^{\prime}\left(x\right)$ peut se factoriser : $1-x^{2}=\left(1-x\right)\left(1+x\right)$
     \par
     Une facile étude de signe montre que $f^{\prime}$ est strictement négative sur $\left]-\infty  ; -1\right[$ et $\left]1 ; +\infty \right[$ et est strictement positive sur $\left]-1 ; 1\right[$.
     \par
     Par ailleurs, $f\left(-1\right)=-\frac{1}{2}$ et $f\left(1\right)=\frac{1}{2}$
     \par
     On en déduit le tableau de variations de $f$ (que l'on regroupe habituellement avec le tableau de signe de $f^{\prime}$) :
\begin{center}
\begin{extern}%width="500" alt="Dérivée et tableau de variations "
\begin{tikzpicture}[scale=1]
% Styles 
\tikzstyle{cadre}=[thin]
\tikzstyle{fleche}=[->,>=latex,thin]
\tikzstyle{nondefini}=[lightgray]
% Dimensions Modifiables
\def\Lrg{1.5}
\def\HtX{1}
\def\HtY{0.5}
% Dimensions Calculées
\def\lignex{-0.5*\HtX}
\def\lignef{-1.5*\HtX}
\def\separateur{-0.5*\Lrg}
% Largeur du tableau
\def\gauche{-1.5*\Lrg}
\def\droite{6.5*\Lrg}
% Hauteur du tableau
\def\haut{0.5*\HtX}
\def\bas{-2.5*\HtX-2*\HtY}
% Ligne de l'abscisse : x
\node at (-1*\Lrg,0) {$x$};
\node at (0*\Lrg,0) {$-\infty$};
\node at (2*\Lrg,0) {$-1$};
\node at (4*\Lrg,0) {$1$};
\node at (6*\Lrg,0) {$+\infty$};
% Pointillés
\draw[lightgray] (2*\Lrg,\lignex-0) -- (2*\Lrg,\lignef+0);
\draw[lightgray] (2*\Lrg,\lignef-0) -- (2*\Lrg,\bas+0);
\draw[lightgray] (4*\Lrg,\lignex-0) -- (4*\Lrg,\lignef+0);
\draw[lightgray] (4*\Lrg,\lignef-0) -- (4*\Lrg,\bas+0);
% Ligne de la dérivée : f'(x)
\node at (-1*\Lrg,-1*\HtX) {$f'(x)$};
\node at (0*\Lrg,-1*\HtX) {$ $};
\node at (1*\Lrg,-1*\HtX) {$-$};
\node at (2*\Lrg,-1*\HtX) {$0$};
\node at (3*\Lrg,-1*\HtX) {$+$};
\node at (4*\Lrg,-1*\HtX) {$0$};
\node at (5*\Lrg,-1*\HtX) {$-$};
\node at (6*\Lrg,-1*\HtX) {$ $};
% Ligne de la fonction : f(x)
\node  at (-1*\Lrg,{-2*\HtX+(-1)*\HtY}) {$f(x)$};
\node (f1) at (0*\Lrg,{-2*\HtX+(0)*\HtY}) {$$};
\node (f2) at (2*\Lrg,{-2*\HtX+(-2)*\HtY}) {$-\dfrac{1}{2}$};
\node (f3) at (4*\Lrg,{-2*\HtX+(0)*\HtY}) {$\dfrac{1}{2}$};
\node (f4) at (6*\Lrg,{-2*\HtX+(-2)*\HtY}) {$$};
% Flèches
\draw[fleche] (f1) -- (f2);
\draw[fleche] (f2) -- (f3);
\draw[fleche] (f3) -- (f4);
% Encadrement
\draw[cadre] (\separateur,\haut) -- (\separateur,\bas);
\draw[cadre] (\gauche,\haut) rectangle  (\droite,\bas);
\draw[cadre] (\gauche,\lignex) -- (\droite,\lignex);
\draw[cadre] (\gauche,\lignef) -- (\droite,\lignef);
\end{tikzpicture}
\end{extern}
\end{center}
}

\end{document}