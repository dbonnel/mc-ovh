\documentclass[a4paper]{article}

%================================================================================================================================
%
% Packages
%
%================================================================================================================================

\usepackage[T1]{fontenc} 	% pour caractères accentués
\usepackage[utf8]{inputenc}  % encodage utf8
\usepackage[french]{babel}	% langue : français
\usepackage{fourier}			% caractères plus lisibles
\usepackage[dvipsnames]{xcolor} % couleurs
\usepackage{fancyhdr}		% réglage header footer
\usepackage{needspace}		% empêcher sauts de page mal placés
\usepackage{graphicx}		% pour inclure des graphiques
\usepackage{enumitem,cprotect}		% personnalise les listes d'items (nécessaire pour ol, al ...)
\usepackage{hyperref}		% Liens hypertexte
\usepackage{pstricks,pst-all,pst-node,pstricks-add,pst-math,pst-plot,pst-tree,pst-eucl} % pstricks
\usepackage[a4paper,includeheadfoot,top=2cm,left=3cm, bottom=2cm,right=3cm]{geometry} % marges etc.
\usepackage{comment}			% commentaires multilignes
\usepackage{amsmath,environ} % maths (matrices, etc.)
\usepackage{amssymb,makeidx}
\usepackage{bm}				% bold maths
\usepackage{tabularx}		% tableaux
\usepackage{colortbl}		% tableaux en couleur
\usepackage{fontawesome}		% Fontawesome
\usepackage{environ}			% environment with command
\usepackage{fp}				% calculs pour ps-tricks
\usepackage{multido}			% pour ps tricks
\usepackage[np]{numprint}	% formattage nombre
\usepackage{tikz,tkz-tab} 			% package principal TikZ
\usepackage{pgfplots}   % axes
\usepackage{mathrsfs}    % cursives
\usepackage{calc}			% calcul taille boites
\usepackage[scaled=0.875]{helvet} % font sans serif
\usepackage{svg} % svg
\usepackage{scrextend} % local margin
\usepackage{scratch} %scratch
\usepackage{multicol} % colonnes
%\usepackage{infix-RPN,pst-func} % formule en notation polanaise inversée
\usepackage{listings}

%================================================================================================================================
%
% Réglages de base
%
%================================================================================================================================

\lstset{
language=Python,   % R code
literate=
{á}{{\'a}}1
{à}{{\`a}}1
{ã}{{\~a}}1
{é}{{\'e}}1
{è}{{\`e}}1
{ê}{{\^e}}1
{í}{{\'i}}1
{ó}{{\'o}}1
{õ}{{\~o}}1
{ú}{{\'u}}1
{ü}{{\"u}}1
{ç}{{\c{c}}}1
{~}{{ }}1
}


\definecolor{codegreen}{rgb}{0,0.6,0}
\definecolor{codegray}{rgb}{0.5,0.5,0.5}
\definecolor{codepurple}{rgb}{0.58,0,0.82}
\definecolor{backcolour}{rgb}{0.95,0.95,0.92}

\lstdefinestyle{mystyle}{
    backgroundcolor=\color{backcolour},   
    commentstyle=\color{codegreen},
    keywordstyle=\color{magenta},
    numberstyle=\tiny\color{codegray},
    stringstyle=\color{codepurple},
    basicstyle=\ttfamily\footnotesize,
    breakatwhitespace=false,         
    breaklines=true,                 
    captionpos=b,                    
    keepspaces=true,                 
    numbers=left,                    
xleftmargin=2em,
framexleftmargin=2em,            
    showspaces=false,                
    showstringspaces=false,
    showtabs=false,                  
    tabsize=2,
    upquote=true
}

\lstset{style=mystyle}


\lstset{style=mystyle}
\newcommand{\imgdir}{C:/laragon/www/newmc/assets/imgsvg/}
\newcommand{\imgsvgdir}{C:/laragon/www/newmc/assets/imgsvg/}

\definecolor{mcgris}{RGB}{220, 220, 220}% ancien~; pour compatibilité
\definecolor{mcbleu}{RGB}{52, 152, 219}
\definecolor{mcvert}{RGB}{125, 194, 70}
\definecolor{mcmauve}{RGB}{154, 0, 215}
\definecolor{mcorange}{RGB}{255, 96, 0}
\definecolor{mcturquoise}{RGB}{0, 153, 153}
\definecolor{mcrouge}{RGB}{255, 0, 0}
\definecolor{mclightvert}{RGB}{205, 234, 190}

\definecolor{gris}{RGB}{220, 220, 220}
\definecolor{bleu}{RGB}{52, 152, 219}
\definecolor{vert}{RGB}{125, 194, 70}
\definecolor{mauve}{RGB}{154, 0, 215}
\definecolor{orange}{RGB}{255, 96, 0}
\definecolor{turquoise}{RGB}{0, 153, 153}
\definecolor{rouge}{RGB}{255, 0, 0}
\definecolor{lightvert}{RGB}{205, 234, 190}
\setitemize[0]{label=\color{lightvert}  $\bullet$}

\pagestyle{fancy}
\renewcommand{\headrulewidth}{0.2pt}
\fancyhead[L]{maths-cours.fr}
\fancyhead[R]{\thepage}
\renewcommand{\footrulewidth}{0.2pt}
\fancyfoot[C]{}

\newcolumntype{C}{>{\centering\arraybackslash}X}
\newcolumntype{s}{>{\hsize=.35\hsize\arraybackslash}X}

\setlength{\parindent}{0pt}		 
\setlength{\parskip}{3mm}
\setlength{\headheight}{1cm}

\def\ebook{ebook}
\def\book{book}
\def\web{web}
\def\type{web}

\newcommand{\vect}[1]{\overrightarrow{\,\mathstrut#1\,}}

\def\Oij{$\left(\text{O}~;~\vect{\imath},~\vect{\jmath}\right)$}
\def\Oijk{$\left(\text{O}~;~\vect{\imath},~\vect{\jmath},~\vect{k}\right)$}
\def\Ouv{$\left(\text{O}~;~\vect{u},~\vect{v}\right)$}

\hypersetup{breaklinks=true, colorlinks = true, linkcolor = OliveGreen, urlcolor = OliveGreen, citecolor = OliveGreen, pdfauthor={Didier BONNEL - https://www.maths-cours.fr} } % supprime les bordures autour des liens

\renewcommand{\arg}[0]{\text{arg}}

\everymath{\displaystyle}

%================================================================================================================================
%
% Macros - Commandes
%
%================================================================================================================================

\newcommand\meta[2]{    			% Utilisé pour créer le post HTML.
	\def\titre{titre}
	\def\url{url}
	\def\arg{#1}
	\ifx\titre\arg
		\newcommand\maintitle{#2}
		\fancyhead[L]{#2}
		{\Large\sffamily \MakeUppercase{#2}}
		\vspace{1mm}\textcolor{mcvert}{\hrule}
	\fi 
	\ifx\url\arg
		\fancyfoot[L]{\href{https://www.maths-cours.fr#2}{\black \footnotesize{https://www.maths-cours.fr#2}}}
	\fi 
}


\newcommand\TitreC[1]{    		% Titre centré
     \needspace{3\baselineskip}
     \begin{center}\textbf{#1}\end{center}
}

\newcommand\newpar{    		% paragraphe
     \par
}

\newcommand\nosp {    		% commande vide (pas d'espace)
}
\newcommand{\id}[1]{} %ignore

\newcommand\boite[2]{				% Boite simple sans titre
	\vspace{5mm}
	\setlength{\fboxrule}{0.2mm}
	\setlength{\fboxsep}{5mm}	
	\fcolorbox{#1}{#1!3}{\makebox[\linewidth-2\fboxrule-2\fboxsep]{
  		\begin{minipage}[t]{\linewidth-2\fboxrule-4\fboxsep}\setlength{\parskip}{3mm}
  			 #2
  		\end{minipage}
	}}
	\vspace{5mm}
}

\newcommand\CBox[4]{				% Boites
	\vspace{5mm}
	\setlength{\fboxrule}{0.2mm}
	\setlength{\fboxsep}{5mm}
	
	\fcolorbox{#1}{#1!3}{\makebox[\linewidth-2\fboxrule-2\fboxsep]{
		\begin{minipage}[t]{1cm}\setlength{\parskip}{3mm}
	  		\textcolor{#1}{\LARGE{#2}}    
 	 	\end{minipage}  
  		\begin{minipage}[t]{\linewidth-2\fboxrule-4\fboxsep}\setlength{\parskip}{3mm}
			\raisebox{1.2mm}{\normalsize\sffamily{\textcolor{#1}{#3}}}						
  			 #4
  		\end{minipage}
	}}
	\vspace{5mm}
}

\newcommand\cadre[3]{				% Boites convertible html
	\par
	\vspace{2mm}
	\setlength{\fboxrule}{0.1mm}
	\setlength{\fboxsep}{5mm}
	\fcolorbox{#1}{white}{\makebox[\linewidth-2\fboxrule-2\fboxsep]{
  		\begin{minipage}[t]{\linewidth-2\fboxrule-4\fboxsep}\setlength{\parskip}{3mm}
			\raisebox{-2.5mm}{\sffamily \small{\textcolor{#1}{\MakeUppercase{#2}}}}		
			\par		
  			 #3
 	 		\end{minipage}
	}}
		\vspace{2mm}
	\par
}

\newcommand\bloc[3]{				% Boites convertible html sans bordure
     \needspace{2\baselineskip}
     {\sffamily \small{\textcolor{#1}{\MakeUppercase{#2}}}}    
		\par		
  			 #3
		\par
}

\newcommand\CHelp[1]{
     \CBox{Plum}{\faInfoCircle}{À RETENIR}{#1}
}

\newcommand\CUp[1]{
     \CBox{NavyBlue}{\faThumbsOUp}{EN PRATIQUE}{#1}
}

\newcommand\CInfo[1]{
     \CBox{Sepia}{\faArrowCircleRight}{REMARQUE}{#1}
}

\newcommand\CRedac[1]{
     \CBox{PineGreen}{\faEdit}{BIEN R\'EDIGER}{#1}
}

\newcommand\CError[1]{
     \CBox{Red}{\faExclamationTriangle}{ATTENTION}{#1}
}

\newcommand\TitreExo[2]{
\needspace{4\baselineskip}
 {\sffamily\large EXERCICE #1\ (\emph{#2 points})}
\vspace{5mm}
}

\newcommand\img[2]{
          \includegraphics[width=#2\paperwidth]{\imgdir#1}
}

\newcommand\imgsvg[2]{
       \begin{center}   \includegraphics[width=#2\paperwidth]{\imgsvgdir#1} \end{center}
}


\newcommand\Lien[2]{
     \href{#1}{#2 \tiny \faExternalLink}
}
\newcommand\mcLien[2]{
     \href{https~://www.maths-cours.fr/#1}{#2 \tiny \faExternalLink}
}

\newcommand{\euro}{\eurologo{}}

%================================================================================================================================
%
% Macros - Environement
%
%================================================================================================================================

\newenvironment{tex}{ %
}
{%
}

\newenvironment{indente}{ %
	\setlength\parindent{10mm}
}

{
	\setlength\parindent{0mm}
}

\newenvironment{corrige}{%
     \needspace{3\baselineskip}
     \medskip
     \textbf{\textsc{Corrigé}}
     \medskip
}
{
}

\newenvironment{extern}{%
     \begin{center}
     }
     {
     \end{center}
}

\NewEnviron{code}{%
	\par
     \boite{gray}{\texttt{%
     \BODY
     }}
     \par
}

\newenvironment{vbloc}{% boite sans cadre empeche saut de page
     \begin{minipage}[t]{\linewidth}
     }
     {
     \end{minipage}
}
\NewEnviron{h2}{%
    \needspace{3\baselineskip}
    \vspace{0.6cm}
	\noindent \MakeUppercase{\sffamily \large \BODY}
	\vspace{1mm}\textcolor{mcgris}{\hrule}\vspace{0.4cm}
	\par
}{}

\NewEnviron{h3}{%
    \needspace{3\baselineskip}
	\vspace{5mm}
	\textsc{\BODY}
	\par
}

\NewEnviron{margeneg}{ %
\begin{addmargin}[-1cm]{0cm}
\BODY
\end{addmargin}
}

\NewEnviron{html}{%
}

\begin{document}
\begin{h2}1. Primitives d'une fonction\end{h2}
\cadre{bleu}{Définition}{% id="d10"
     Soit $f$ une fonction définie sur $I$.
     \par
     On dit que $F$  est une primitive de  $f$  sur l'intervalle $I$, si et seulement si $F$ est dérivable sur $I$ et pour tout $x$ de $I$, $F^{\prime}\left(x\right)=f\left(x\right)$.
}
\bloc{orange}{Exemple}{% id="e10"
     La fonction $F: ~x\mapsto x^{2}$ est une primitive de la fonction $f:~x\mapsto 2x$ sur $\mathbb{R}$.
     \par
     La fonction $G: ~x\mapsto x^{2}+1$ est aussi une primitive de cette même fonction $f$.
}
\cadre{vert}{Propriété}{% id="e20"
     Si $F$ est une primitive de $f$ sur $I$, alors les autres primitives de $f$ sur $I$ sont les fonctions de la forme $F+k$ où $k\in \mathbb{R}.$
}
\bloc{vert}{Remarque}{% id="r20"
     Une fonction continue ayant une infinité de primitives, il ne faut pas dire \textbf{la} primitive de $f$ mais \textbf{une} primitive de $f$.
}
\bloc{orange}{Exemple}{% id="e20"
     Les primitives de la fonction $f:~x\mapsto 2x$ sont les fonctions $F:~ x\mapsto x^{2}+k$ où $k\in \mathbb{R}.$
}
\cadre{vert}{Propriété}{% id="p30"
     Toute fonction \textbf{continue} sur un intervalle $I$ admet des primitives sur $I$.
}
\cadre{vert}{Propriétés}{% id="p40"
     \textbf{Primitives des fonctions usuelles :}
     \begin{tabularx}{0.8\linewidth}{|*{3}{>{\centering \arraybackslash }X|}}%class="compact" width="600"
          \hline
          \textbf{Fonction $f$} & \textbf{Primitives $F$} & \textbf{Ensemble de validité}
          \\ \hline
          $0$ & $k$ & $\mathbb{R}$
          \\ \hline
          $a$ & $ax+k$ & $\mathbb{R}$
          \\ \hline
          $x^{n} ~ \left(n\in \mathbb{N}\right)$ & $\frac{x^{n+1}}{n+1}+k$ & $\mathbb{R}$
          \\ \hline
          $\frac{1}{x^{n}} ~ \left(n\in \mathbb{N};~n>1\right)$ & $-\frac{1}{\left(n-1\right)x^{n-1}}+k$ & $\mathbb{R}-\left\{0\right\}$
          \\ \hline
          $\frac{1}{x}$ & $\ln x+k$ & $\left]0;+\infty \right[$
          \\ \hline
          $e^{x}$ & $e^{x}+k$ & $\mathbb{R}$
          \\ \hline
     \end{tabularx}
}
\cadre{vert}{Propriétés}{% id="p50"
     Si $f$ et $g$ sont deux fonctions définies sur $I$ et admettant respectivement $F$ et $G$ comme primitives sur $I$ et $k$ un réel quelconque.
     \begin{itemize}
          \item $F+G$ est une primitive de la fonction $f+g$ sur $I$.
          \item $kF$ est une primitive de la fonction $kf$ sur $I$.
     \end{itemize}
}
\cadre{vert}{Propriétés}{% id="p60"
     \textbf{Primitives et fonctions composées}
     \par
     Soit $u$ une fonction définie et dérivable sur un intervalle $I$.
     \begin{tabularx}{0.8\linewidth}{|*{3}{>{\centering \arraybackslash }X|}}%class="compact" width="600"
          \hline
          \textbf{Fonction $f$ }  &  \textbf{Primitives $F$}  &  \textbf{Condition}
          \\ \hline
          $u^{\prime}u^{n} ~ \left(n\in \mathbb{N}\right)$  &  $\frac{u^{n+1}}{n+1}+k$  &
          \\ \hline
          $\frac{u^{\prime}}{u}$  &  $\ln u+k$  &  si $u\left(x\right)>0$
          \\ \hline
          $\frac{u^{\prime}}{u^{n}} ~ \left(n\in \mathbb{N};~n>1\right)$  &  $-\frac{1}{\left(n-1\right)u^{n-1}}+k$  &  si $u\left(x\right)\neq 0$
          \\ \hline
          $\frac{u^{\prime}}{\sqrt{u}}$  &  $2\sqrt{u}+k$  &  si $u\left(x\right)>0$
          \\ \hline
          $u^{\prime}e^{u}$  &  $e^{u}+k$  &
          \\  \hline
     \end{tabularx}
}
\bloc{orange}{Exemple}{% id="e60"
     La fonction $x\mapsto \frac{2x}{x^{2}+1}$ admet comme primitives les fonctions de la forme $x\mapsto \ln\left(x^{2}+1\right)+k$ sur tout intervalle de $\mathbb{R}$ (forme $\frac{u^{\prime}}{u}$).
}
\begin{h2}2. Intégrales\end{h2}
\cadre{bleu}{Définition}{% id="d80"
     Soit $f$ une fonction continue sur un intervalle $\left[a;b\right]$ et $F$ une primitive de $f$ sur $\left[a;b\right]$.
     \textbf{L'intégrale de $a$ à $b$ de $f$} est le nombre réel noté $\int_{a}^{b}f\left(x\right)\text{d}x$ défini par:
     \par
     $\int_{a}^{b}f\left(x\right)\text{d}x=F\left(b\right)-F\left(a\right)$.
}
\bloc{vert}{Remarques}{% id="r80"
     \begin{itemize}
          \item L'intégrale ne dépend pas de la primitive de $f$ choisie.
          \par
          En effet si $G$ est une autre primitive de $f$, on a $G=F+k$ donc :
          \par
          $G\left(b\right)-G\left(a\right)=F\left(b\right)+k-\left(F\left(a\right)+k\right)=F\left(b\right)-F\left(a\right)$
          \item Dans l'expression  $\int_{a}^{b}f\left(x\right)\text{d}x$, $x$ est une variable \og~muette~\fg{}. C'est à dire que l'on ne change pas l'expression si on remplace $x$ par une autre lettre. En pratique, on emploie souvent la lettre $t$ notamment lorsque la lettre $x$ est employée par ailleurs.
     \end{itemize}
}
\bloc{vert}{Notations}{% id="n80"
     On note souvent : $F\left(b\right)-F\left(a\right)=\left[F\left(x\right)\right]_{a}^{b}$.
     \par
     On a avec cette notation :
     \par
     $\int_{a}^{b}f\left(x\right)\text{d}x=\left[F\left(x\right)\right]_{a}^{b}$.
}
\bloc{orange}{Exemple}{% id="e80"
     La fonction $F$ définie par $F\left(x\right)=\frac{x^{3}}{3}$ est une primitive de la fonction carré.
     \par
     On a donc :
     \par
     $\int_{0}^{1}x^{2}\text{d}x=\left[\frac{x^{3}}{3}\right]_{0}^{1}=\frac{1}{3}-\frac{0}{3}=\frac{1}{3}$.
}
\cadre{rouge}{Théorème (intégrale fonction de sa borne supérieure)}{% id="t90"
     Soit $f$ une fonction continue sur un intervalle $I$ et $a \in  I$; la fonction définie sur $I$ par~:
     \begin{center}$x\mapsto \int_{a}^{ x}f\left(t\right)\text{d}t$\end{center}
     est la primitive de $f$ qui s'annule pour $x=a$.
}
\bloc{orange}{Démonstration}{% id="m90"
     Soit $F$ une primitive (quelconque) de $f$. Posons $\Phi \left(x\right)=\int_{a}^{ x}f\left(t\right)\text{d}t$
     \par
     $\Phi \left(x\right)=\int_{a}^{ x}f\left(t\right)\text{d}t=F\left(x\right)-F\left(a\right)$
     \par
     donc:
     \par
     $\Phi ^{\prime}\left(x\right)=F^{\prime}\left(x\right)=f\left(x\right)$.
     \par
     Ce qui prouve que $\Phi $ est aussi une primitive de $f$.
     \par
     De plus $\Phi \left(a\right)=F\left(a\right)-F\left(a\right)=0$.
}
\bloc{cyan}{Remarque}{% id="r90"
     Notez bien la position du $x$ en borne supérieure de l'intégrale.
}
\bloc{orange}{Exemple}{% id="e90"
     La fonction définie sur $\left[0 ; +\infty \right[$ $x\mapsto \int_{1}^{ x}\frac{1}{t}\text{d}t$ (on peut aussi écrire $\int_{1}^{ x}\frac{\text{d}t}{t}$) est la primitive de la fonction inverse qui s'annule pour $x=1$. C'est donc la fonction logarithme népérien:
     \par
     $\ln\left(x\right)= \int_{1}^{ x}\frac{\text{d}t}{t}.$
}
\cadre{vert}{Propriété}{% id="p100"
     \textbf{Relation de Chasles}
     \par
     Soit $f$ une fonction continue sur $\left[a;b\right]$ et $c\in \left[a;b\right]$.
     \par
     $\int_{a}^{b}f\left(x\right)\text{d}x=\int_{a}^{c}f\left(x\right)\text{d}x+\int_{c}^{b}f\left(x\right)\text{d}x$.
}
\cadre{vert}{Propriété}{% id="p110"
     \textbf{Linéarité de l'intégrale}
     \par
     Soit $f$ et $g$ deux fonctions continues sur $\left[a;b\right]$ et $\lambda \in \mathbb{R}$.
     \begin{itemize}
          \item $\int_{a}^{b}f\left(x\right)+g\left(x\right)\text{d}x=\int_{a}^{b}f\left(x\right)\text{d}x+\int_{a}^{b}g\left(x\right)\text{d}x$
          \item $\int_{a}^{b} \lambda  f\left(x\right)\text{d}x=\lambda  \int_{a}^{b}f\left(x\right)\text{d}x$.
     \end{itemize}
}
\cadre{vert}{Propriété}{% id="p120"
     \textbf{Comparaison d'intégrales}
     \par
     Soit $f$ et $g$ deux fonctions continues sur $\left[a;b\right]$ telles que $f\geqslant g$ sur $\left[a;b\right]$.
     \par
     $\int_{a}^{b}f\left(x\right)\text{d}x\geqslant \int_{a}^{b}g\left(x\right)\text{d}x$.
}
\bloc{vert}{Remarque}{% id="r120"
     En particulier, en prenant pour $g$ la fonction nulle on obtient si $f\left(x\right)\geqslant 0$ sur $\left[a;b\right]$:
     \begin{center}
          $\int_{a}^{b}f\left(x\right)\text{d}x\geqslant 0$.
     \end{center}
}
\begin{h2}3. Interprétation graphique\end{h2}
\cadre{bleu}{Définition}{% id="d130"
     Le plan $P$ est rapporté à un repère orthogonal $\left(O,\vec{i},\vec{j}\right)$.
     \par
     On appelle \textbf{unité d'aire (u.a.)} l'aire d'un rectangle (qui est un carré si le repère est orthonormé) dont les côtés mesurent $||\vec{i}||$ et $||\vec{j}||$.
}
\begin{center}
     \begin{extern} %width="400" alt="unité d'aire"
          \resizebox{6cm}{!}{%
               % -+-+-+ variables modifiables
               \def\xmin{-1.2}
               \def\xmax{4.2}
               \def\ymin{-0.9}
               \def\ymax{1.8}
               \def\xunit{2}  % unités en cm
               \def\yunit{2}
               \psset{xunit=\xunit,yunit=\yunit,algebraic=true}
               \fontsize{15pt}{15pt}\selectfont
               \begin{pspicture*}[linewidth=1pt](\xmin,\ymin)(\xmax,\ymax)
                    \psgrid[gridcolor=mcgris, subgriddiv=1, gridlabels=0pt](-2,-1)(5,2)
                    \psaxes[linewidth=0.75pt]{->}(0,0)(\xmin,\ymin)(\xmax,\ymax)
                    \pscustom[fillstyle=solid,fillcolor=vert,linecolor=vert,linewidth=0.75pt,opacity=0.2]{%
                         \psline(0,0)(1,0)(1,1)(0,1)
                         \closepath
                    }
                    \rput[tr](-0.1,-0.1){$O$}
                    \psline[linewidth=1.25pt]{->}(0,0)(0,1)
                    \psline[linewidth=1.25pt]{->}(0,0)(1,0)
                    \rput[t](0.5,-0.03){$\vect{i}$}
                    \rput[r](-0.03,0.5){$\vect{j}$}
                    %                        \rput[l](1.1){ unité d'aire}
               \end{pspicture*}
          }
     \end{extern}
\end{center}
\begin{center}
     \textit{Unité d'aire dans le cas d'un repère orthonormé}
\end{center}
\cadre{vert}{Propriété}{% id="p130"
     Si $f$ est une fonction continue et \textbf{positive} sur $\left[a;b\right]$, alors l'intégrale $\int_{a}^{b}f\left(x\right)\text{d}x$ est l'aire, en unités d'aire, de la surface délimitée par :
     \begin{itemize}
          \item la courbe $C_{f}$,
          \item l'axe des abscisses,
          \item les droites (verticales) d'équations $x=a$ et $x=b$.
     \end{itemize}
}
\bloc{orange}{Exemple}{% id="e130"
     \begin{center}
          \begin{extern} %width="600" alt="aire et intégrale"
               \resizebox{8cm}{!}{%
                    % -+-+-+ variables modifiables
                    \def\fonction{ln(1+x*x) }
                    \def\xmin{-2.2}
                    \def\xmax{6.2}
                    \def\ymin{-1.8}
                    \def\ymax{3.8}
                    \def\xunit{2}  % unités en cm
                    \def\yunit{2}
                    \psset{xunit=\xunit,yunit=\yunit,algebraic=true}
                    \fontsize{15pt}{15pt}\selectfont
                    \begin{pspicture*}[linewidth=1pt](\xmin,\ymin)(\xmax,\ymax)
                         \psgrid[gridcolor=mcgris, subgriddiv=1, gridlabels=0pt](-3,-1.8)(7,4)
                         \psaxes[linewidth=0.75pt]{->}(0,0)(\xmin,\ymin)(\xmax,\ymax)
                         \pscustom[fillstyle=solid,fillcolor=vert,linecolor=vert,linewidth=0.75pt,opacity=0.2]{%
                              \psplot{1}{3}{\fonction}
                              \psline(3,0)(1,0)
                              \closepath
                         }
                         \psplot[plotpoints=2000,linecolor=blue]{\xmin}{\xmax}{\fonction}
                         \rput[tr](-0.1,-0.1){$O$}
                         \rput[l](5.5,3.2){$\color{blue} \mathcal{C}_f$}
                    \end{pspicture*}
               }
          \end{extern}
     \end{center}
     L'aire colorée ci-dessus est égale (en unités d'aire) à $\int_{1}^{3}f\left(x\right)\text{d}x$.
}
\bloc{vert}{Remarques}{% id="r130"
     \begin{itemize}
          \item Si $f$ est négative sur $\left[a;b\right]$, la propriété précédente appliquée à la fonction $-f$ montre que $\int_{a}^{b}f\left(x\right)\text{d}x$ est égale à l'\textbf{opposé} de l'aire délimitée par la courbe $C_{f}$, l'axe des abscisses, les droites d'équations $x=a$ et $x=b$.
          \item Si le signe de $f$ varie sur $\left[a;b\right]$, on découpe $\left[a;b\right]$ en sous-intervalles sur lesquels $f$ garde un signe constant.
     \end{itemize}
}
\cadre{vert}{Propriété}{% id="p140"
     Si $f$ et $g$ sont des fonctions continues et telles que $f\leqslant g$ sur $\left[a;b\right]$, alors l'aire de la surface délimitée par :
     \begin{itemize}
          \item la courbe $C_{f}$,
          \item la courbe $C_{g}$,
          \item les droites (verticales) d'équations $x=a$ et $x=b$.
     \end{itemize}
     est égale (en unités d'aire) à :
     \par
     $A=\int_{a}^{b}\left(g\left(x\right)-f\left(x\right)\right)\text{d}x$.
}
\bloc{orange}{Exemple}{% id="e140"
     $f$ et $g$ définies par $f\left(x\right)=x^{2}-x$ et $g\left(x\right)=3x-x^{2}$ sont représentées par les paraboles ci-dessous :
     \begin{center}
          \begin{extern} %width="600" alt="aire entre deux courbes"
               \resizebox{8cm}{!}{%
                    % -+-+-+ variables modifiables
                    \def\fonction{x*x-x }
                    \def\g{3*x-x*x }
                    \def\xmin{-2.2}
                    \def\xmax{6.2}
                    \def\ymin{-1.8}
                    \def\ymax{3.8}
                    \def\xunit{2}  % unités en cm
                    \def\yunit{2}
                    \psset{xunit=\xunit,yunit=\yunit,algebraic=true}
                    \fontsize{15pt}{15pt}\selectfont
                    \begin{pspicture*}[linewidth=1pt](\xmin,\ymin)(\xmax,\ymax)
                         \psgrid[gridcolor=mcgris, subgriddiv=1, gridlabels=0pt](-3,-3)(7,4)
                         \psaxes[linewidth=0.75pt]{->}(0,0)(\xmin,\ymin)(\xmax,\ymax)
                         \pscustom[fillstyle=solid,fillcolor=vert,linestyle=solid,linewidth=0.2pt,opacity=0.2]{%
                              \psplot{0}{2}{\fonction}
                              \psplot{2}{0}{\g}
                         }
                         \psplot[plotpoints=2000,linecolor=blue]{\xmin}{\xmax}{\fonction}
                         \psplot[plotpoints=2000,linecolor=red]{\xmin}{\xmax}{\g}
                         \rput[tr](-0.1,-0.1){$O$}
                         \rput(-1.6,3){$\color{blue} \mathcal{C}_f$}
                         \rput(3.6,-1){$\color{red} \mathcal{C}_g$}
                    \end{pspicture*}
               }
          \end{extern}
     \end{center}
     L'aire colorée est égale (en unités d'aire) à :
     \par
     $A=\int_{0}^{2}\left(g\left(x\right)-f\left(x\right)\right)\text{d}x=\int_{0}^{2} \left(4x-2x^{2}\right)\text{d}x=\left[2x^{2}-\frac{2}{3}x^{3}\right]_{0}^{2}=\frac{8}{3} \text{u.a.}$
}

\end{document}