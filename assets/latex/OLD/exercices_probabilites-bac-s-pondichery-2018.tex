\documentclass[a4paper]{article}

%================================================================================================================================
%
% Packages
%
%================================================================================================================================

\usepackage[T1]{fontenc} 	% pour caractères accentués
\usepackage[utf8]{inputenc}  % encodage utf8
\usepackage[french]{babel}	% langue : français
\usepackage{fourier}			% caractères plus lisibles
\usepackage[dvipsnames]{xcolor} % couleurs
\usepackage{fancyhdr}		% réglage header footer
\usepackage{needspace}		% empêcher sauts de page mal placés
\usepackage{graphicx}		% pour inclure des graphiques
\usepackage{enumitem,cprotect}		% personnalise les listes d'items (nécessaire pour ol, al ...)
\usepackage{hyperref}		% Liens hypertexte
\usepackage{pstricks,pst-all,pst-node,pstricks-add,pst-math,pst-plot,pst-tree,pst-eucl} % pstricks
\usepackage[a4paper,includeheadfoot,top=2cm,left=3cm, bottom=2cm,right=3cm]{geometry} % marges etc.
\usepackage{comment}			% commentaires multilignes
\usepackage{amsmath,environ} % maths (matrices, etc.)
\usepackage{amssymb,makeidx}
\usepackage{bm}				% bold maths
\usepackage{tabularx}		% tableaux
\usepackage{colortbl}		% tableaux en couleur
\usepackage{fontawesome}		% Fontawesome
\usepackage{environ}			% environment with command
\usepackage{fp}				% calculs pour ps-tricks
\usepackage{multido}			% pour ps tricks
\usepackage[np]{numprint}	% formattage nombre
\usepackage{tikz,tkz-tab} 			% package principal TikZ
\usepackage{pgfplots}   % axes
\usepackage{mathrsfs}    % cursives
\usepackage{calc}			% calcul taille boites
\usepackage[scaled=0.875]{helvet} % font sans serif
\usepackage{svg} % svg
\usepackage{scrextend} % local margin
\usepackage{scratch} %scratch
\usepackage{multicol} % colonnes
%\usepackage{infix-RPN,pst-func} % formule en notation polanaise inversée
\usepackage{listings}

%================================================================================================================================
%
% Réglages de base
%
%================================================================================================================================

\lstset{
language=Python,   % R code
literate=
{á}{{\'a}}1
{à}{{\`a}}1
{ã}{{\~a}}1
{é}{{\'e}}1
{è}{{\`e}}1
{ê}{{\^e}}1
{í}{{\'i}}1
{ó}{{\'o}}1
{õ}{{\~o}}1
{ú}{{\'u}}1
{ü}{{\"u}}1
{ç}{{\c{c}}}1
{~}{{ }}1
}


\definecolor{codegreen}{rgb}{0,0.6,0}
\definecolor{codegray}{rgb}{0.5,0.5,0.5}
\definecolor{codepurple}{rgb}{0.58,0,0.82}
\definecolor{backcolour}{rgb}{0.95,0.95,0.92}

\lstdefinestyle{mystyle}{
    backgroundcolor=\color{backcolour},   
    commentstyle=\color{codegreen},
    keywordstyle=\color{magenta},
    numberstyle=\tiny\color{codegray},
    stringstyle=\color{codepurple},
    basicstyle=\ttfamily\footnotesize,
    breakatwhitespace=false,         
    breaklines=true,                 
    captionpos=b,                    
    keepspaces=true,                 
    numbers=left,                    
xleftmargin=2em,
framexleftmargin=2em,            
    showspaces=false,                
    showstringspaces=false,
    showtabs=false,                  
    tabsize=2,
    upquote=true
}

\lstset{style=mystyle}


\lstset{style=mystyle}
\newcommand{\imgdir}{C:/laragon/www/newmc/assets/imgsvg/}
\newcommand{\imgsvgdir}{C:/laragon/www/newmc/assets/imgsvg/}

\definecolor{mcgris}{RGB}{220, 220, 220}% ancien~; pour compatibilité
\definecolor{mcbleu}{RGB}{52, 152, 219}
\definecolor{mcvert}{RGB}{125, 194, 70}
\definecolor{mcmauve}{RGB}{154, 0, 215}
\definecolor{mcorange}{RGB}{255, 96, 0}
\definecolor{mcturquoise}{RGB}{0, 153, 153}
\definecolor{mcrouge}{RGB}{255, 0, 0}
\definecolor{mclightvert}{RGB}{205, 234, 190}

\definecolor{gris}{RGB}{220, 220, 220}
\definecolor{bleu}{RGB}{52, 152, 219}
\definecolor{vert}{RGB}{125, 194, 70}
\definecolor{mauve}{RGB}{154, 0, 215}
\definecolor{orange}{RGB}{255, 96, 0}
\definecolor{turquoise}{RGB}{0, 153, 153}
\definecolor{rouge}{RGB}{255, 0, 0}
\definecolor{lightvert}{RGB}{205, 234, 190}
\setitemize[0]{label=\color{lightvert}  $\bullet$}

\pagestyle{fancy}
\renewcommand{\headrulewidth}{0.2pt}
\fancyhead[L]{maths-cours.fr}
\fancyhead[R]{\thepage}
\renewcommand{\footrulewidth}{0.2pt}
\fancyfoot[C]{}

\newcolumntype{C}{>{\centering\arraybackslash}X}
\newcolumntype{s}{>{\hsize=.35\hsize\arraybackslash}X}

\setlength{\parindent}{0pt}		 
\setlength{\parskip}{3mm}
\setlength{\headheight}{1cm}

\def\ebook{ebook}
\def\book{book}
\def\web{web}
\def\type{web}

\newcommand{\vect}[1]{\overrightarrow{\,\mathstrut#1\,}}

\def\Oij{$\left(\text{O}~;~\vect{\imath},~\vect{\jmath}\right)$}
\def\Oijk{$\left(\text{O}~;~\vect{\imath},~\vect{\jmath},~\vect{k}\right)$}
\def\Ouv{$\left(\text{O}~;~\vect{u},~\vect{v}\right)$}

\hypersetup{breaklinks=true, colorlinks = true, linkcolor = OliveGreen, urlcolor = OliveGreen, citecolor = OliveGreen, pdfauthor={Didier BONNEL - https://www.maths-cours.fr} } % supprime les bordures autour des liens

\renewcommand{\arg}[0]{\text{arg}}

\everymath{\displaystyle}

%================================================================================================================================
%
% Macros - Commandes
%
%================================================================================================================================

\newcommand\meta[2]{    			% Utilisé pour créer le post HTML.
	\def\titre{titre}
	\def\url{url}
	\def\arg{#1}
	\ifx\titre\arg
		\newcommand\maintitle{#2}
		\fancyhead[L]{#2}
		{\Large\sffamily \MakeUppercase{#2}}
		\vspace{1mm}\textcolor{mcvert}{\hrule}
	\fi 
	\ifx\url\arg
		\fancyfoot[L]{\href{https://www.maths-cours.fr#2}{\black \footnotesize{https://www.maths-cours.fr#2}}}
	\fi 
}


\newcommand\TitreC[1]{    		% Titre centré
     \needspace{3\baselineskip}
     \begin{center}\textbf{#1}\end{center}
}

\newcommand\newpar{    		% paragraphe
     \par
}

\newcommand\nosp {    		% commande vide (pas d'espace)
}
\newcommand{\id}[1]{} %ignore

\newcommand\boite[2]{				% Boite simple sans titre
	\vspace{5mm}
	\setlength{\fboxrule}{0.2mm}
	\setlength{\fboxsep}{5mm}	
	\fcolorbox{#1}{#1!3}{\makebox[\linewidth-2\fboxrule-2\fboxsep]{
  		\begin{minipage}[t]{\linewidth-2\fboxrule-4\fboxsep}\setlength{\parskip}{3mm}
  			 #2
  		\end{minipage}
	}}
	\vspace{5mm}
}

\newcommand\CBox[4]{				% Boites
	\vspace{5mm}
	\setlength{\fboxrule}{0.2mm}
	\setlength{\fboxsep}{5mm}
	
	\fcolorbox{#1}{#1!3}{\makebox[\linewidth-2\fboxrule-2\fboxsep]{
		\begin{minipage}[t]{1cm}\setlength{\parskip}{3mm}
	  		\textcolor{#1}{\LARGE{#2}}    
 	 	\end{minipage}  
  		\begin{minipage}[t]{\linewidth-2\fboxrule-4\fboxsep}\setlength{\parskip}{3mm}
			\raisebox{1.2mm}{\normalsize\sffamily{\textcolor{#1}{#3}}}						
  			 #4
  		\end{minipage}
	}}
	\vspace{5mm}
}

\newcommand\cadre[3]{				% Boites convertible html
	\par
	\vspace{2mm}
	\setlength{\fboxrule}{0.1mm}
	\setlength{\fboxsep}{5mm}
	\fcolorbox{#1}{white}{\makebox[\linewidth-2\fboxrule-2\fboxsep]{
  		\begin{minipage}[t]{\linewidth-2\fboxrule-4\fboxsep}\setlength{\parskip}{3mm}
			\raisebox{-2.5mm}{\sffamily \small{\textcolor{#1}{\MakeUppercase{#2}}}}		
			\par		
  			 #3
 	 		\end{minipage}
	}}
		\vspace{2mm}
	\par
}

\newcommand\bloc[3]{				% Boites convertible html sans bordure
     \needspace{2\baselineskip}
     {\sffamily \small{\textcolor{#1}{\MakeUppercase{#2}}}}    
		\par		
  			 #3
		\par
}

\newcommand\CHelp[1]{
     \CBox{Plum}{\faInfoCircle}{À RETENIR}{#1}
}

\newcommand\CUp[1]{
     \CBox{NavyBlue}{\faThumbsOUp}{EN PRATIQUE}{#1}
}

\newcommand\CInfo[1]{
     \CBox{Sepia}{\faArrowCircleRight}{REMARQUE}{#1}
}

\newcommand\CRedac[1]{
     \CBox{PineGreen}{\faEdit}{BIEN R\'EDIGER}{#1}
}

\newcommand\CError[1]{
     \CBox{Red}{\faExclamationTriangle}{ATTENTION}{#1}
}

\newcommand\TitreExo[2]{
\needspace{4\baselineskip}
 {\sffamily\large EXERCICE #1\ (\emph{#2 points})}
\vspace{5mm}
}

\newcommand\img[2]{
          \includegraphics[width=#2\paperwidth]{\imgdir#1}
}

\newcommand\imgsvg[2]{
       \begin{center}   \includegraphics[width=#2\paperwidth]{\imgsvgdir#1} \end{center}
}


\newcommand\Lien[2]{
     \href{#1}{#2 \tiny \faExternalLink}
}
\newcommand\mcLien[2]{
     \href{https~://www.maths-cours.fr/#1}{#2 \tiny \faExternalLink}
}

\newcommand{\euro}{\eurologo{}}

%================================================================================================================================
%
% Macros - Environement
%
%================================================================================================================================

\newenvironment{tex}{ %
}
{%
}

\newenvironment{indente}{ %
	\setlength\parindent{10mm}
}

{
	\setlength\parindent{0mm}
}

\newenvironment{corrige}{%
     \needspace{3\baselineskip}
     \medskip
     \textbf{\textsc{Corrigé}}
     \medskip
}
{
}

\newenvironment{extern}{%
     \begin{center}
     }
     {
     \end{center}
}

\NewEnviron{code}{%
	\par
     \boite{gray}{\texttt{%
     \BODY
     }}
     \par
}

\newenvironment{vbloc}{% boite sans cadre empeche saut de page
     \begin{minipage}[t]{\linewidth}
     }
     {
     \end{minipage}
}
\NewEnviron{h2}{%
    \needspace{3\baselineskip}
    \vspace{0.6cm}
	\noindent \MakeUppercase{\sffamily \large \BODY}
	\vspace{1mm}\textcolor{mcgris}{\hrule}\vspace{0.4cm}
	\par
}{}

\NewEnviron{h3}{%
    \needspace{3\baselineskip}
	\vspace{5mm}
	\textsc{\BODY}
	\par
}

\NewEnviron{margeneg}{ %
\begin{addmargin}[-1cm]{0cm}
\BODY
\end{addmargin}
}

\NewEnviron{html}{%
}

\begin{document}
\begin{h2}Exercice 3 (5 points)\end{h2}
\textbf{Commun  à tous les candidats}
\medskip
Une entreprise conditionne du sucre blanc provenant de deux exploitations U et V en paquets
de 1 kg et de différentes qualités.
\smallskip
Le sucre extra fin est conditionné séparément dans des paquets portant le label \og  extra fin \fg.
\smallskip
\emph{Les parties A, B et C peuvent être traitées de façon indépendante.}
\smallskip
Dans tout l'exercice, les résultats seront arrondis, si nécessaire, au millième.
\begin{center}\begin{h3}Partie A \end{h3}\end{center}
Pour calibrer le sucre en fonction de la taille de ses cristaux, on le fait passer au travers d'une
série de trois tamis positionnés les uns au-dessus des autres et posés sur un récipient à fond
étanche.
Les ouvertures des mailles sont les suivantes~:
\begin{center}
     \begin{extern}%alt="tamis"
          \psset{unit=1cm}
          \begin{pspicture}(12,4)
               %\psgrid
               \uput[r](0.5,3){Tamis 1 : 0,8 mm} \psline[linewidth=1pt]{->}(4,1)(6.8,1)
               \uput[r](0.5,2){Tamis 2 : 0,5 mm} \psline[linewidth=1pt]{->}(4,2)(6.8,2)
               \uput[r](0.5,1){Tamis 3 : 0,2 mm} \psline[linewidth=1pt]{->}(4,3)(6.8,3)
               \uput[r](-0.95,0.2){Récipient à fond étanche}  \psline[linewidth=1pt]{->}(4,0.2)(7,0.2)
               \psline[linewidth=1pt](7,3.75)(7,0)(12,0)(12,3.75)
               \multido{\n=7.00+0.13}{39}{\psframe(\n,2.94)(\n,3)}
               \multido{\n=7.00+0.12}{42}{\psframe(\n,1.96)(\n,2.02)}
               \multido{\n=7.00+0.1}{50}{\psframe(\n,0.97)(\n,1.03)}
          \end{pspicture}
     \end{extern}
\end{center}
Les cristaux de sucre dont la taille est inférieure à $0,2$ mm se trouvent dans le récipient à fond
étanche à la fin du calibrage. Ils seront conditionnés dans des paquets portant le label \og  sucre
extra fin \fg.
\medskip
\begin{enumerate}
     \item On prélève au hasard un cristal de sucre de l'exploitation U. La taille de ce cristal,
     exprimée en millimètre, est modélisée par la variable aléatoire $X_{\text{ U}}$ qui suit la loi normale
     de moyenne $\mu_{\text{ U}} = 0,58$~mm et d'écart type $\sigma_{\text{ U}} = 0,21$~mm.
     \begin{enumerate}[label=\alph*.]
          \item Calculer les probabilités des événements suivants~: $X_{\text{ U}} < 0,2 $ et $ 0,5 \leqslant X_{\text{ U}} < 0,8$.
          \item On fait passer 1~800 grammes de sucre provenant de l'exploitation U au travers de la
          série de tamis.
          \par
          Déduire de la question précédente une estimation de la masse de sucre récupérée dans
          le récipient à fond étanche et une estimation de la masse de sucre récupérée dans le
          tamis 2.
     \end{enumerate}
     \item On prélève au hasard un cristal de sucre de l'exploitation V. La taille de ce cristal,
     exprimée en millimètre, est modélisée par la variable aléatoire $X_{\text{V}}$ qui suit la loi normale
     de moyenne $\mu_{\text{V}} = 0,65$ mm et d'écart type $\sigma_{\text{V}}$ à déterminer.
     \par
     Lors du calibrage d'une grande quantité de cristaux de sucre provenant de l'exploitation V,
     on constate que 40\,\% de ces cristaux se retrouvent dans le tamis 2.
     \par
     Quelle est la valeur de l'écart type $\sigma_{\text{V}}$ de la variable aléatoire $X_{\text{V}}$~?
\end{enumerate}
\begin{center}\begin{h3}Partie B \end{h3}\end{center}
Dans cette partie, on admet que 3\,\% du sucre provenant de l'exploitation U est extra fin et que
5\,\% du sucre provenant de l'exploitation V est extra fin.
\par
On prélève au hasard un paquet de sucre dans la production de l'entreprise et, dans un souci
de traçabilité, on s'intéresse à la provenance de ce paquet.
\par
On considère les événements suivants~:
\begin{indent}
     \begin{itemize}
          \item $U$~: \og  Le paquet contient du sucre provenant de l'exploitation U \fg{}~;
          \item $V$~: \og Le paquet contient du sucre provenant de l'exploitation V \fg{}~;
          \item $E$~: \og Le paquet porte le label "extra fin" \fg{}.
     \end{itemize}
\end{indent}
\medskip
\begin{enumerate}
     \item Dans cette question, on admet que l'entreprise fabrique 30\,\% de ses paquets avec du sucre
     provenant de l'exploitation U et les autres avec du sucre provenant de l'exploitation V,
     sans mélanger les sucres des deux exploitations.
     \begin{enumerate}[label=\alph*.]
          \item Quelle est la probabilité que le paquet prélevé porte le label \og extra fin \fg{}~?
          \item Sachant qu'un paquet porte le label \og extra fin \fg, quelle est la probabilité que le sucre
          qu'il contient provienne de l'exploitation U~?
     \end{enumerate}
     \item L'entreprise souhaite modifier son approvisionnement auprès des deux exploitations afin
     que parmi les paquets portant le label « extra fin », 30\,\% d'entre eux contiennent du sucre
     provenant de l'exploitation U.
     \par
     Comment doit-elle s'approvisionner auprès des exploitations U et V~?
     \par
     \emph{Toute trace de recherche sera valorisée dans cette question}.
\end{enumerate}
\begin{center}\begin{h3}Partie C \end{h3}\end{center}
\begin{enumerate}
     \item L'entreprise annonce que 30\,\% des paquets de sucre portant le label « extra fin » qu'elle
     conditionne contiennent du sucre provenant de l'exploitation U.
     \par
     Avant de valider une commande, un acheteur veut vérifier cette proportion annoncée. Il
     prélève $150$ paquets pris au hasard dans la production de paquets labellisés \og extra fin \fg{} de
     l'entreprise. Parmi ces paquets, $30$ contiennent du sucre provenant de l'exploitation U.
     \par
     A-t-il des raisons de remettre en question l'annonce de l'entreprise~?
     \item  L'année suivante, l'entreprise déclare avoir modifié sa production. L'acheteur souhaite
     estimer la nouvelle proportion de paquets de sucre provenant de l'exploitation U parmi les
     paquets portant le label \og extra fin \fg.
     \par
     Il prélève 150 paquets pris au hasard dans la production de paquets labellisés \og extra fin \fg{} de l'entreprise. Parmi ces paquets 42\,\% contiennent du sucre provenant de l'exploitation U.
     \par
     Donner un intervalle de confiance, au niveau de confiance 95\,\%, de la nouvelle proportion
     de paquets labellisés \og extra fin \fg{} contenant du sucre provenant de l'exploitation U.
\end{enumerate}
\begin{corrige}
     \begin{center}\begin{h3}Partie A \end{h3}\end{center}
     \begin{enumerate}
          \item
          \begin{enumerate}[label=\alph*.]
               \item
               $X_{\text{ U}}$ suit la loi normale de moyenne $\mu_{\text{ U}} = 0,58$ et d'écart type $\sigma_{\text{ U}} = 0,21$.
               \par
               \`A la calculatrice, on trouve~:
               \par
               $p\left(X_{\text{ U}}<0,2 \right)\approx 0,035$  (au millième).
               \par
               $p\left(0,5 \leqslant  X_{\text{ U}} < 0,8 \right)\approx 0,501$  (au millième).
               \item
               $p\left(X_{\text{ U}}<0,2 \right)\approx 0,035$ donc 3,5\% des cristaux se retrouvent dans le récipient à fond étanche.
               \par
               Pour 1~800g de sucre, cela correspond à une masse d'environ $1~800 \times  0,035~=~63$ grammes (arrondie au gramme).
               \smallskip
               $p\left(0,5 \leqslant  X_{\text{ U}}<0,8 \right)\approx 0,501$ donc 50,1\% des cristaux se retrouvent dans le tamis 2 .
               \par
               Pour 1~800g de sucre, cela correspond à une masse d'environ $1~800 \times   0,501~=~902 $ grammes (arrondie au gramme).
          \end{enumerate}
          \item
          Posons  $Z=\dfrac{X_{\text{V}}-\mu_{\text{V}}}{\sigma_{\text{V}}} = \dfrac{X_{\text{V}}-0,65}{\sigma_{\text{V}}}$
          \par
          Puisque $X_{\text{V}}$ suit la loi normale de moyenne $\mu_{\text{ V}} = 0,58$ et d'écart type $\sigma_{\text{V}}$, $Z$ suit la loi normale centrée réduite.
          \par
          D'après l'énoncé, 40\,\% des cristaux se retrouvent dans le tamis 2 donc~:
          \par
          $p\left(0,5 \leqslant X_{\text{ V}}<0,8 \right)= 0,4$.
          \par
          Or~:
          \par
          $0,5 \leqslant X_{\text{ V}}<0,8~ \Leftrightarrow ~0,5 -0,65\leqslant X_{\text{ V}}-0,65<0,8-0,65$\\
          $\phantom{0,5 \leqslant X_{\text{ V}}<0,8}~ \Leftrightarrow  ~ -0,15\leqslant X_{\text{ V}}-0,65<0,15$\\
          $\phantom{0,5 \leqslant X_{\text{ V}}<0,8}~ \Leftrightarrow  ~ -\dfrac{0,15}{\sigma_{\text{V}}} \leqslant Z<\dfrac{0,15}{\sigma_{\text{V}}} $
          \par
          Par conséquent~:
          \par
          $p\left(-\dfrac{0,15}{\sigma_{\text{V}}} \leqslant Z<\dfrac{0,15}{\sigma_{\text{V}}} \right) =0,4$
          \par
          où $Z$ suit la loi normale centrée réduite.
          \par
          \`A  la calculatrice on trouve $p\left(-0,524 \leqslant Z<0,524\right) =0,4$.
          \par
          Donc $\dfrac{0,15}{\sigma_{\text{V}}}\approx 0,524$  et $\sigma_{\text{V}} \approx \dfrac{0,15}{0,524}\approx 0,286$.
     \end{enumerate}
     \begin{center}\begin{h3}Partie B \end{h3}\end{center}
     \begin{enumerate}
          \item
          \begin{enumerate}[label=\alph*.]
               \item
               On cherche à calculer $p(E)$.
               \par
               On peut modéliser la situation à l'aide de l'arbre ci-dessous~:
               \begin{center}
                    \begin{extern}%width="350" alt="arbre pondéré probabilités"
                         % Racine à Gauche, développement vers la droite
                         \begin{tikzpicture}[xscale=1,yscale=1]
                              % Styles (MODIFIABLES)
                              \tikzstyle{fleche}=[-,>=latex,thick]
                              \tikzstyle{noeud}=[fill=white]
                              \tikzstyle{feuille}=[fill=white]
                              \tikzstyle{etiquette}=[midway,fill=white]
                              % Dimensions (MODIFIABLES)
                              \def\DistanceInterNiveaux{3}
                              \def\DistanceInterFeuilles{2}
                              % Dimensions calculées (NON MODIFIABLES)
                              \def\NiveauA{(0)*\DistanceInterNiveaux}
                              \def\NiveauB{(1.5)*\DistanceInterNiveaux}
                              \def\NiveauC{(2.5)*\DistanceInterNiveaux}
                              \def\InterFeuilles{(-1)*\DistanceInterFeuilles}
                              % Noeuds (MODIFIABLES : Styles et Coefficients d'InterFeuilles)
                              \node[noeud] (R) at ({\NiveauA},{(1.5)*\InterFeuilles}) {$ $};
                              \node[noeud] (Ra) at ({\NiveauB},{(0.5)*\InterFeuilles}) {$U$};
                              \node[feuille] (Raa) at ({\NiveauC},{(0)*\InterFeuilles}) {$E$};
                              \node[feuille] (Rab) at ({\NiveauC},{(1)*\InterFeuilles}) {$\overline{E}$};
                              \node[noeud] (Rb) at ({\NiveauB},{(2.5)*\InterFeuilles}) {$V$};
                              \node[feuille] (Rba) at ({\NiveauC},{(2)*\InterFeuilles}) {$E$};
                              \node[feuille] (Rbb) at ({\NiveauC},{(3)*\InterFeuilles}) {$\overline{E}$};
                              % Arcs (MODIFIABLES : Styles)
                              \draw[fleche] (R)--(Ra) node[etiquette] {$0,3$};
                              \draw[fleche] (Ra)--(Raa) node[etiquette] {$0,03$};
                              \draw[fleche] (Ra)--(Rab) node[etiquette] {$0,97$};
                              \draw[fleche] (R)--(Rb) node[etiquette] {$0,7$};
                              \draw[fleche] (Rb)--(Rba) node[etiquette] {$0,05$};
                              \draw[fleche] (Rb)--(Rbb) node[etiquette] {$0,95$};
                         \end{tikzpicture}
                    \end{extern}
               \end{center}
               %~:-+-+-+-+- Fin
               \par
               Le sucre provenant soit de l'exploitation U, soit de l'exploitation V, les événements $U$ et $V$ forment une partition de l'univers.
               \par
               D'après la formule des probabilités totales~:
               \par
               $p(E)=p(U) \times  p_U(E) +p(V) \times  p_V(E)$\\
               $\phantom{P(E)}=0,3 \times 0,03 + 0,7 \times 0,05 $ $= 0,009 +0,035=0,044$
               \item
               La probabilité cherchée est $P_E(U)$.
               \par
               D'après la formule des probabilités conditionnelles~:
               \par
               $P_E(U)=\dfrac{P(E\cap U)}{P(E)}$ $=\dfrac{0,009}{0,044}=\dfrac{9}{44}\approx 0,205$
          \end{enumerate}
          \item
          Posons  $p(U)=x$. Alors $p(V)=1-x$.
          \par
          L'arbre obtenu est alors~:
          \par
          %~:-+-+-+- Engendré par~: http~://math.et.info.free.fr/TikZ/Arbre/
          \begin{center}
               \begin{extern}%width="350" alt="arbre pondéré Pondichery 2018"
                    % Racine à Gauche, développement vers la droite
                    \begin{tikzpicture}[xscale=1,yscale=1]
                         % Styles (MODIFIABLES)
                         \tikzstyle{fleche}=[-,>=latex,thick]
                         \tikzstyle{noeud}=[fill=white]
                         \tikzstyle{feuille}=[fill=white]
                         \tikzstyle{etiquette}=[midway,fill=white]
                         % Dimensions (MODIFIABLES)
                         \def\DistanceInterNiveaux{3}
                         \def\DistanceInterFeuilles{2}
                         % Dimensions calculées (NON MODIFIABLES)
                         \def\NiveauA{(0)*\DistanceInterNiveaux}
                         \def\NiveauB{(1.5)*\DistanceInterNiveaux}
                         \def\NiveauC{(2.5)*\DistanceInterNiveaux}
                         \def\InterFeuilles{(-1)*\DistanceInterFeuilles}
                         % Noeuds (MODIFIABLES : Styles et Coefficients d'InterFeuilles)
                         \node[noeud] (R) at ({\NiveauA},{(1.5)*\InterFeuilles}) {$ $};
                         \node[noeud] (Ra) at ({\NiveauB},{(0.5)*\InterFeuilles}) {$U$};
                         \node[feuille] (Raa) at ({\NiveauC},{(0)*\InterFeuilles}) {$E$};
                         \node[feuille] (Rab) at ({\NiveauC},{(1)*\InterFeuilles}) {$\overline{E}$};
                         \node[noeud] (Rb) at ({\NiveauB},{(2.5)*\InterFeuilles}) {$V$};
                         \node[feuille] (Rba) at ({\NiveauC},{(2)*\InterFeuilles}) {$E$};
                         \node[feuille] (Rbb) at ({\NiveauC},{(3)*\InterFeuilles}) {$\overline{E}$};
                         % Arcs (MODIFIABLES : Styles)
                         \draw[fleche] (R)--(Ra) node[etiquette] {$x$};
                         \draw[fleche] (Ra)--(Raa) node[etiquette] {$0,03$};
                         \draw[fleche] (Ra)--(Rab) node[etiquette] {$0,97$};
                         \draw[fleche] (R)--(Rb) node[etiquette] {$1-x$};
                         \draw[fleche] (Rb)--(Rba) node[etiquette] {$0,05$};
                         \draw[fleche] (Rb)--(Rbb) node[etiquette] {$0,95$};
                    \end{tikzpicture}
               \end{extern}
          \end{center}
          %~:-+-+-+-+- Fin
          \par
          On a donc~:
          \par
          $p_E(U)=\dfrac{p(E\cap U)}{p(E)} = \dfrac{x\times p_U(E)}{x\times p_U(E) +(1-x)\times p_V(E)}$\\
          $\phantom{p_E(U)}=\dfrac{0,03x}{0,03x +0,05(1-x)}=\dfrac{0,03x}{0,05 -0,02x}$
          \medskip
          Par conséquent~:
          \par
          $p_E(U)=0,3 \Leftrightarrow \dfrac{0,03x}{0,05 -0,02x}=0,3$\\
          $\phantom{p_E(U)=0,3} \Leftrightarrow 0,03x=0,015-0,006x$\\
          $\phantom{p_E(U)=0,3} \Leftrightarrow 0,036x=0,015$\\
          $\phantom{p_E(U)=0,3} \Leftrightarrow  x=\dfrac{0,015}{0,036}=\dfrac{5}{12}\approx 0,417 $
          \par
          Pour que, parmi les paquets portant le label « extra fin », 30\,\%  contiennent du sucre provenant de l'exploitation U, l'entreprise doit s'approvisionner à  41,7\% auprès de l'exploitation U.
     \end{enumerate}
     \begin{center}\begin{h3}Partie C \end{h3}\end{center}
     \begin{enumerate}
          \item
          D'après l'entreprise, la proportion de paquets de sucre portant le label «extra fin» provenant de l’exploitation U est $p = 0,3$.
          \par
          La taille  de l'échantillon est $n=150$.
          \par
          On vérifie que~:
          \begin{itbullet}
               \item $n=150 \geqslant 30$~;
               \item $np=150 \times 0,3=45\geqslant 5$~;
               \item $n(1-p)=150 \times 0,7=105\geqslant 5$.
          \end{itbullet}
          Les conditions de validité étant remplies, l'intervalle de fluctuation asymptotique au seuil de $95\%$ est~:
          \par
          \[ I=\left[p-1,96\dfrac{\sqrt{p(1-p)}}{\sqrt{n}}~;~p+1,96\dfrac{\sqrt{p(1-p)}}{\sqrt{n}}\right]. \]
          \medskip
          $p-1,96\dfrac{\sqrt{p(1-p)}}{\sqrt{n}}=0,3-1,96\dfrac{\sqrt{0,3(1-0,3)}}{\sqrt{150}}$\\
          $\phantom{p-1,96\dfrac{\sqrt{p(1-p)}}{\sqrt{n}}}  \approx 0,226$ (arrondi au millième par défaut).
          \medskip
          $p+1,96\dfrac{\sqrt{p(1-p)}}{\sqrt{n}}=0,3+1,96\dfrac{\sqrt{0,3(1-0,3)}}{\sqrt{150}}$\\
          $\phantom{p+1,96\dfrac{\sqrt{p(1-p)}}{\sqrt{n}}} \approx 0,374 $ (arrondi au millième par excès).
          \medskip
          L'intervalle de fluctuation asymptotique au seuil de $95\%$ de la proportion de paquets  provenant de l’exploitation U est donc~:
          \[ I=[0,226~;~0,374]. \]
          \par
          La fréquence observée des paquets  provenant de l’exploitation U  est $f=\dfrac{30}{150}=0,2$.
          \par
          Comme $0,2 \notin I$, on peut donc rejeter l'affirmation de l'entreprise avec un risque d'erreur inférieur à 5\%.
          \item
          La taille  de l'échantillon est $n=150$.
          \par
          La fréquence observée de paquets provenant de l'exploitation U est $f=0,42$
          \par
          On vérifie que~:
          \begin{itbullet}
               \item $n=150 \geqslant 30$~;
               \item $nf=150 \times 0,42=63\geqslant 5$~;
               \item $n(1-f)=150 \times 0,58=87\geqslant 5$.
          \end{itbullet}
          Les conditions de validité étant remplies, un intervalle de confiance, au niveau de confiance 95\,\% est~:
          \[ I=\left[ f-\dfrac{1}{\sqrt{n}}~; f+\dfrac{1}{\sqrt{n}}  \right]. \]
          \par
          $f-\dfrac{1}{\sqrt{n}} \approx 0,338$ (arrondi au millième par défaut).
          \par
          $f+\dfrac{1}{\sqrt{n}} \approx 0,502$ (arrondi au millième par excès).
          \par
          Un intervalle de confiance, au niveau de confiance 95\,\% est donc~:
          \[ I=[0,338~;~0,502] .\]
     \end{enumerate}
\end{corrige}

\end{document}