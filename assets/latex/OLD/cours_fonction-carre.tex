\documentclass[a4paper]{article}

%================================================================================================================================
%
% Packages
%
%================================================================================================================================

\usepackage[T1]{fontenc} 	% pour caractères accentués
\usepackage[utf8]{inputenc}  % encodage utf8
\usepackage[french]{babel}	% langue : français
\usepackage{fourier}			% caractères plus lisibles
\usepackage[dvipsnames]{xcolor} % couleurs
\usepackage{fancyhdr}		% réglage header footer
\usepackage{needspace}		% empêcher sauts de page mal placés
\usepackage{graphicx}		% pour inclure des graphiques
\usepackage{enumitem,cprotect}		% personnalise les listes d'items (nécessaire pour ol, al ...)
\usepackage{hyperref}		% Liens hypertexte
\usepackage{pstricks,pst-all,pst-node,pstricks-add,pst-math,pst-plot,pst-tree,pst-eucl} % pstricks
\usepackage[a4paper,includeheadfoot,top=2cm,left=3cm, bottom=2cm,right=3cm]{geometry} % marges etc.
\usepackage{comment}			% commentaires multilignes
\usepackage{amsmath,environ} % maths (matrices, etc.)
\usepackage{amssymb,makeidx}
\usepackage{bm}				% bold maths
\usepackage{tabularx}		% tableaux
\usepackage{colortbl}		% tableaux en couleur
\usepackage{fontawesome}		% Fontawesome
\usepackage{environ}			% environment with command
\usepackage{fp}				% calculs pour ps-tricks
\usepackage{multido}			% pour ps tricks
\usepackage[np]{numprint}	% formattage nombre
\usepackage{tikz,tkz-tab} 			% package principal TikZ
\usepackage{pgfplots}   % axes
\usepackage{mathrsfs}    % cursives
\usepackage{calc}			% calcul taille boites
\usepackage[scaled=0.875]{helvet} % font sans serif
\usepackage{svg} % svg
\usepackage{scrextend} % local margin
\usepackage{scratch} %scratch
\usepackage{multicol} % colonnes
%\usepackage{infix-RPN,pst-func} % formule en notation polanaise inversée
\usepackage{listings}

%================================================================================================================================
%
% Réglages de base
%
%================================================================================================================================

\lstset{
language=Python,   % R code
literate=
{á}{{\'a}}1
{à}{{\`a}}1
{ã}{{\~a}}1
{é}{{\'e}}1
{è}{{\`e}}1
{ê}{{\^e}}1
{í}{{\'i}}1
{ó}{{\'o}}1
{õ}{{\~o}}1
{ú}{{\'u}}1
{ü}{{\"u}}1
{ç}{{\c{c}}}1
{~}{{ }}1
}


\definecolor{codegreen}{rgb}{0,0.6,0}
\definecolor{codegray}{rgb}{0.5,0.5,0.5}
\definecolor{codepurple}{rgb}{0.58,0,0.82}
\definecolor{backcolour}{rgb}{0.95,0.95,0.92}

\lstdefinestyle{mystyle}{
    backgroundcolor=\color{backcolour},   
    commentstyle=\color{codegreen},
    keywordstyle=\color{magenta},
    numberstyle=\tiny\color{codegray},
    stringstyle=\color{codepurple},
    basicstyle=\ttfamily\footnotesize,
    breakatwhitespace=false,         
    breaklines=true,                 
    captionpos=b,                    
    keepspaces=true,                 
    numbers=left,                    
xleftmargin=2em,
framexleftmargin=2em,            
    showspaces=false,                
    showstringspaces=false,
    showtabs=false,                  
    tabsize=2,
    upquote=true
}

\lstset{style=mystyle}


\lstset{style=mystyle}
\newcommand{\imgdir}{C:/laragon/www/newmc/assets/imgsvg/}
\newcommand{\imgsvgdir}{C:/laragon/www/newmc/assets/imgsvg/}

\definecolor{mcgris}{RGB}{220, 220, 220}% ancien~; pour compatibilité
\definecolor{mcbleu}{RGB}{52, 152, 219}
\definecolor{mcvert}{RGB}{125, 194, 70}
\definecolor{mcmauve}{RGB}{154, 0, 215}
\definecolor{mcorange}{RGB}{255, 96, 0}
\definecolor{mcturquoise}{RGB}{0, 153, 153}
\definecolor{mcrouge}{RGB}{255, 0, 0}
\definecolor{mclightvert}{RGB}{205, 234, 190}

\definecolor{gris}{RGB}{220, 220, 220}
\definecolor{bleu}{RGB}{52, 152, 219}
\definecolor{vert}{RGB}{125, 194, 70}
\definecolor{mauve}{RGB}{154, 0, 215}
\definecolor{orange}{RGB}{255, 96, 0}
\definecolor{turquoise}{RGB}{0, 153, 153}
\definecolor{rouge}{RGB}{255, 0, 0}
\definecolor{lightvert}{RGB}{205, 234, 190}
\setitemize[0]{label=\color{lightvert}  $\bullet$}

\pagestyle{fancy}
\renewcommand{\headrulewidth}{0.2pt}
\fancyhead[L]{maths-cours.fr}
\fancyhead[R]{\thepage}
\renewcommand{\footrulewidth}{0.2pt}
\fancyfoot[C]{}

\newcolumntype{C}{>{\centering\arraybackslash}X}
\newcolumntype{s}{>{\hsize=.35\hsize\arraybackslash}X}

\setlength{\parindent}{0pt}		 
\setlength{\parskip}{3mm}
\setlength{\headheight}{1cm}

\def\ebook{ebook}
\def\book{book}
\def\web{web}
\def\type{web}

\newcommand{\vect}[1]{\overrightarrow{\,\mathstrut#1\,}}

\def\Oij{$\left(\text{O}~;~\vect{\imath},~\vect{\jmath}\right)$}
\def\Oijk{$\left(\text{O}~;~\vect{\imath},~\vect{\jmath},~\vect{k}\right)$}
\def\Ouv{$\left(\text{O}~;~\vect{u},~\vect{v}\right)$}

\hypersetup{breaklinks=true, colorlinks = true, linkcolor = OliveGreen, urlcolor = OliveGreen, citecolor = OliveGreen, pdfauthor={Didier BONNEL - https://www.maths-cours.fr} } % supprime les bordures autour des liens

\renewcommand{\arg}[0]{\text{arg}}

\everymath{\displaystyle}

%================================================================================================================================
%
% Macros - Commandes
%
%================================================================================================================================

\newcommand\meta[2]{    			% Utilisé pour créer le post HTML.
	\def\titre{titre}
	\def\url{url}
	\def\arg{#1}
	\ifx\titre\arg
		\newcommand\maintitle{#2}
		\fancyhead[L]{#2}
		{\Large\sffamily \MakeUppercase{#2}}
		\vspace{1mm}\textcolor{mcvert}{\hrule}
	\fi 
	\ifx\url\arg
		\fancyfoot[L]{\href{https://www.maths-cours.fr#2}{\black \footnotesize{https://www.maths-cours.fr#2}}}
	\fi 
}


\newcommand\TitreC[1]{    		% Titre centré
     \needspace{3\baselineskip}
     \begin{center}\textbf{#1}\end{center}
}

\newcommand\newpar{    		% paragraphe
     \par
}

\newcommand\nosp {    		% commande vide (pas d'espace)
}
\newcommand{\id}[1]{} %ignore

\newcommand\boite[2]{				% Boite simple sans titre
	\vspace{5mm}
	\setlength{\fboxrule}{0.2mm}
	\setlength{\fboxsep}{5mm}	
	\fcolorbox{#1}{#1!3}{\makebox[\linewidth-2\fboxrule-2\fboxsep]{
  		\begin{minipage}[t]{\linewidth-2\fboxrule-4\fboxsep}\setlength{\parskip}{3mm}
  			 #2
  		\end{minipage}
	}}
	\vspace{5mm}
}

\newcommand\CBox[4]{				% Boites
	\vspace{5mm}
	\setlength{\fboxrule}{0.2mm}
	\setlength{\fboxsep}{5mm}
	
	\fcolorbox{#1}{#1!3}{\makebox[\linewidth-2\fboxrule-2\fboxsep]{
		\begin{minipage}[t]{1cm}\setlength{\parskip}{3mm}
	  		\textcolor{#1}{\LARGE{#2}}    
 	 	\end{minipage}  
  		\begin{minipage}[t]{\linewidth-2\fboxrule-4\fboxsep}\setlength{\parskip}{3mm}
			\raisebox{1.2mm}{\normalsize\sffamily{\textcolor{#1}{#3}}}						
  			 #4
  		\end{minipage}
	}}
	\vspace{5mm}
}

\newcommand\cadre[3]{				% Boites convertible html
	\par
	\vspace{2mm}
	\setlength{\fboxrule}{0.1mm}
	\setlength{\fboxsep}{5mm}
	\fcolorbox{#1}{white}{\makebox[\linewidth-2\fboxrule-2\fboxsep]{
  		\begin{minipage}[t]{\linewidth-2\fboxrule-4\fboxsep}\setlength{\parskip}{3mm}
			\raisebox{-2.5mm}{\sffamily \small{\textcolor{#1}{\MakeUppercase{#2}}}}		
			\par		
  			 #3
 	 		\end{minipage}
	}}
		\vspace{2mm}
	\par
}

\newcommand\bloc[3]{				% Boites convertible html sans bordure
     \needspace{2\baselineskip}
     {\sffamily \small{\textcolor{#1}{\MakeUppercase{#2}}}}    
		\par		
  			 #3
		\par
}

\newcommand\CHelp[1]{
     \CBox{Plum}{\faInfoCircle}{À RETENIR}{#1}
}

\newcommand\CUp[1]{
     \CBox{NavyBlue}{\faThumbsOUp}{EN PRATIQUE}{#1}
}

\newcommand\CInfo[1]{
     \CBox{Sepia}{\faArrowCircleRight}{REMARQUE}{#1}
}

\newcommand\CRedac[1]{
     \CBox{PineGreen}{\faEdit}{BIEN R\'EDIGER}{#1}
}

\newcommand\CError[1]{
     \CBox{Red}{\faExclamationTriangle}{ATTENTION}{#1}
}

\newcommand\TitreExo[2]{
\needspace{4\baselineskip}
 {\sffamily\large EXERCICE #1\ (\emph{#2 points})}
\vspace{5mm}
}

\newcommand\img[2]{
          \includegraphics[width=#2\paperwidth]{\imgdir#1}
}

\newcommand\imgsvg[2]{
       \begin{center}   \includegraphics[width=#2\paperwidth]{\imgsvgdir#1} \end{center}
}


\newcommand\Lien[2]{
     \href{#1}{#2 \tiny \faExternalLink}
}
\newcommand\mcLien[2]{
     \href{https~://www.maths-cours.fr/#1}{#2 \tiny \faExternalLink}
}

\newcommand{\euro}{\eurologo{}}

%================================================================================================================================
%
% Macros - Environement
%
%================================================================================================================================

\newenvironment{tex}{ %
}
{%
}

\newenvironment{indente}{ %
	\setlength\parindent{10mm}
}

{
	\setlength\parindent{0mm}
}

\newenvironment{corrige}{%
     \needspace{3\baselineskip}
     \medskip
     \textbf{\textsc{Corrigé}}
     \medskip
}
{
}

\newenvironment{extern}{%
     \begin{center}
     }
     {
     \end{center}
}

\NewEnviron{code}{%
	\par
     \boite{gray}{\texttt{%
     \BODY
     }}
     \par
}

\newenvironment{vbloc}{% boite sans cadre empeche saut de page
     \begin{minipage}[t]{\linewidth}
     }
     {
     \end{minipage}
}
\NewEnviron{h2}{%
    \needspace{3\baselineskip}
    \vspace{0.6cm}
	\noindent \MakeUppercase{\sffamily \large \BODY}
	\vspace{1mm}\textcolor{mcgris}{\hrule}\vspace{0.4cm}
	\par
}{}

\NewEnviron{h3}{%
    \needspace{3\baselineskip}
	\vspace{5mm}
	\textsc{\BODY}
	\par
}

\NewEnviron{margeneg}{ %
\begin{addmargin}[-1cm]{0cm}
\BODY
\end{addmargin}
}

\NewEnviron{html}{%
}

\begin{document}
\begin{h2}I. La fonction «carré»\end{h2}
\cadre{bleu}{Définition}{% id="d10"
     La fonction "\textbf{carré}" est la fonction  définie sur $\mathbb{R}$ par : $x\mapsto x^2$.
     \par
     Sa courbe représentative est une \textbf{parabole}.
     \par
     Elle est symétrique par rapport à l'\textbf{axe des ordonnées}.
}
\begin{center}
\begin{extern}%width="230" alt="fonction carré"
               \resizebox{5.5cm}{!}{%
                    % -+-+-+ variables modifiables
                    \def\fonction{x*x}
                    \def\xmin{-2.5}
                    \def\xmax{2.5}
                    \def\ymin{-0.9}
                    \def\ymax{5.5}
                    \def\xunit{1.5}  % unités en cm
                    \def\yunit{1.5}
                    \psset{xunit=\xunit,yunit=\yunit,algebraic=true}
                    \fontsize{15pt}{15pt}\selectfont
                    \begin{pspicture*}[linewidth=1pt](\xmin,\ymin)(\xmax,\ymax)
                         %      \psgrid[gridcolor=mcgris, subgriddiv=5, gridlabels=0pt](\xmin,\ymin)(\xmax,\ymax)
                         \psaxes[linewidth=0.75pt,Dx=1,Dy=1]{->}(0,0)(\xmin,\ymin)(\xmax,\ymax)
                         \psplot[plotpoints=2000,linecolor=blue]{\xmin}{\xmax}{\fonction}
                           \rput[tr](-0.2,-0.2){$O$} 
                                     \rput[tr](2.1,5.2){$\color{blue} \mathcal{C}_f$}
                    \end{pspicture*}
               }
\end{extern}
\end{center}
\cadre{vert}{Propriété}{% id="p20"
     La fonction carré est strictement décroissante sur $\left]-\infty ; 0\right[$ et strictement croissante sur $\left]0;  \infty \right[$. Elle admet en 0 un minimum égal à 0.
}
          \begin{center}
               \begin{extern}%width="340" alt="Tableau de variation polynôme du second degré pour a < 0"
                    \begin{tikzpicture}[scale=0.875]
                         % Styles
                         \tikzstyle{cadre}=[thin]
                         \tikzstyle{fleche}=[->,>=latex,thin]
                         \tikzstyle{nondefini}=[lightgray]
                         % Dimensions Modifiables
                         \def\Lrg{1.5}
                         \def\HtX{1}
                         \def\HtY{0.5}
                         % Dimensions Calculées
                         \def\lignex{-0.5*\HtX}
                         \def\lignef{-1.5*\HtX}
                         \def\separateur{-0.5*\Lrg}
                         % Largeur du tableau
                         \def\gauche{-1.5*\Lrg}
                         \def\droite{4.5*\Lrg}
                         % Hauteur du tableau
                         \def\haut{0.5*\HtX}
                         \def\bas{-1.5*\HtX-2*\HtY}
                         % Pointillés
                         \draw[lightgray] (2*\Lrg,\lignex) -- (2*\Lrg,\lignef);
                         \draw[lightgray] (2*\Lrg,\lignef) -- (2*\Lrg,\bas);
                         % Ligne de l'abscisse : x
                         \node at (-1*\Lrg,0) {$x$};
                         \node at (0*\Lrg,0) {$ -\infty$};
                         \node at (2*\Lrg,0) {$0$};
                         \node at (4*\Lrg,0) {$+\infty $};
                         % Ligne de la fonction : f(x)
                         \node  at (-1*\Lrg,{-1*\HtX+(-1)*\HtY}) {$f(x)$};
                         \node (f1) at (0*\Lrg,{-1*\HtX+(0)*\HtY}) {$ $};
                         \node (f2) at (2*\Lrg,{-1*\HtX+(-2)*\HtY}) {$0$};
                         \node (f3) at (4*\Lrg,{-1*\HtX+(0)*\HtY}) {$ $};
                         % Flèches
                         \draw[fleche] (f1) -- (f2);
                         \draw[fleche] (f2) -- (f3);
                         % Encadrement
                         \draw[cadre] (\separateur,\haut) -- (\separateur,\bas);
                         \draw[cadre] (\gauche,\haut) rectangle  (\droite,\bas);
                         \draw[cadre] (\gauche,\lignex) -- (\droite,\lignex);
                   \end{tikzpicture}
               \end{extern}
          \end{center}
\begin{center}\textit{Tableau de variations de la fonction carrée}\end{center}

\bloc{cyan}{Démonstration}{% id="m20"
     Démontrons par exemple que la fonction carré est décroissante sur $\left]-\infty ; 0\right[$.
     \par
     Notons $f : x\mapsto x^2$ et soient $x_1$ et $x_2$, deux réels quelconques tels que $x_1 < x_2 < 0$.
     \par
     Alors :
     \par
     $f\left(x_1\right)-f\left(x_2\right)=x_1^2-x_2^2=\left(x_1-x_2\right)\left(x_1+x_2\right)$
     \par
     Or $x_1-x_2 < 0$ car $x_1 < x_2$
     \par
     et $x_1+x_2 < 0$  car $x_1$ et $x_2$ sont tous les deux négatifs.
     \par
     Donc le produit $\left(x_1-x_2\right)\left(x_1+x_2\right)$ est positif.
     \par
     On en déduit $f\left(x_1\right)-f\left(x_2\right) > 0$ donc $f\left(x_1\right) > f\left(x_2\right)$
     \par
     $x_1 < x_2 < 0 \Rightarrow  f\left(x_1\right) > f\left(x_2\right) $, donc la fonction $f$ est strictement décroissante sur $\left]-\infty ; 0\right[$.
}
\cadre{vert}{Propriété}{% id="p30"
     Soit $a$ un nombre réel. Dans $\mathbb{R}$, l'équation $x^2=a$
     \begin{itemize}
          \item n'admet \textbf{aucune} solution \textbf{si $a < 0$}
          \item admet $x=0$ comme \textbf{unique} solution \textbf{si $a=0$}
          \item admet \textbf{deux} solutions $\sqrt{a}$ et $-\sqrt{a}$ \textbf{ si $a > 0$}
     \end{itemize}
}
\bloc{orange}{Exemples}{% id="e30"
     \begin{itemize}
          \item L'équation $x^2=2$ admet deux solutions : $\sqrt{2}$ et $-\sqrt{2}$.
          \item L'équation $x^2+1=0$ est équivalente à $x^2=-1$. Elle n'admet donc aucune solution réelle.
     \end{itemize}
}
\begin{h2}II. Fonctions polynômes du second degré\end{h2}
\cadre{bleu}{Définition}{% id="d50"
     Une fonction \textbf{polynôme du second degré} est une fonction  définie sur $\mathbb{R}$ par : $x\mapsto ax^2+bx+c$.
     \par
     où $a$,$b$ et $c$ sont des réels appelés \textbf{coefficients} et $a\neq 0$
     \par
     Sa courbe représentative est une \textbf{parabole}, elle admet un axe de symétrie parallèle à l'axe des ordonnées.
}
\bloc{cyan}{Remarque}{% id="r50"
     Une expression de la forme $ax^2+bx+c$ avec $a\neq 0$ est la \textbf{forme développée} d'un polynôme du second degré.
     \par
     Une expression de la forme $a\left(x-x_1\right)\left(x-x_2\right)$ avec $a\neq 0$ est la \textbf{forme factorisée} d'un polynôme du second degré.
}
\cadre{rouge}{Théorème}{% id="t60"
     Une fonction polynôme du second degré est :
     \textbf{Si $a > 0$} :
     \par
     strictement décroissante sur $\left]-\infty ; \frac{-b}{2a}\right]$ et strictement croissante sur $\left[\frac{-b}{2a}; +\infty \right[$.
     \textbf{Si $a < 0$} :
     \par
     strictement croissante sur $\left]-\infty ; \frac{-b}{2a}\right]$ et strictement décroissante sur $\left[\frac{-b}{2a}; +\infty \right[$.
}
\bigbreak
          \begin{center}
               \begin{extern}%width="340" alt="Tableau de variation polynôme du second degré pour a < 0"
                    \begin{tikzpicture}[scale=0.875]
                         % Styles
                         \tikzstyle{cadre}=[thin]
                         \tikzstyle{fleche}=[->,>=latex,thin]
                         \tikzstyle{nondefini}=[lightgray]
                         % Dimensions Modifiables
                         \def\Lrg{1.5}
                         \def\HtX{1}
                         \def\HtY{0.5}
                         % Dimensions Calculées
                         \def\lignex{-0.7*\HtX}
                         \def\lignef{-1.5*\HtX}
                         \def\separateur{-0.5*\Lrg}
                         % Largeur du tableau
                         \def\gauche{-1.5*\Lrg}
                         \def\droite{4.5*\Lrg}
                         % Hauteur du tableau
                         \def\haut{0.5*\HtX}
                         \def\bas{-1.5*\HtX-2*\HtY}
                         % Pointillés
                         \draw[lightgray] (2*\Lrg,\lignex) -- (2*\Lrg,\lignef);
                         \draw[lightgray] (2*\Lrg,\lignef) -- (2*\Lrg,\bas);
                         % Ligne de l'abscisse : x
                         \node at (-1*\Lrg,-0.1) {$x$};
                         \node at (0*\Lrg,-0.1) {$ -\infty$};
                         \node at (2*\Lrg,-0.1) {$-\dfrac{b}{2a}$};
                         \node at (4*\Lrg,-0.1) {$+\infty $};
                         % Ligne de la fonction : f(x)
                         \node  at (-1*\Lrg,{-1*\HtX+(-1)*\HtY}) {$f(x)$};
                         \node (f1) at (0*\Lrg,{-1*\HtX+(0)*\HtY}) {$ $};
                         \node (f2) at (2*\Lrg,{-1*\HtX+(-2)*\HtY}) {$\text{min}$};
                         \node (f3) at (4*\Lrg,{-1*\HtX+(0)*\HtY}) {$ $};
                         % Flèches
                         \draw[fleche] (f1) -- (f2);
                         \draw[fleche] (f2) -- (f3);
                         % Encadrement
                         \draw[cadre] (\separateur,\haut) -- (\separateur,\bas);
                         \draw[cadre] (\gauche,\haut) rectangle  (\droite,\bas);
                         \draw[cadre] (\gauche,\lignex) -- (\droite,\lignex);
                   \end{tikzpicture}
               \end{extern}
          \end{center}
\begin{center}\textit{Tableau de variations d'une fonction polynôme du second degré pour $a > 0$ }\end{center}
\bigbreak
          \begin{center}
               \begin{extern}%width="340" alt="Tableau de variation polynôme du second degré pour a < 0"
                    \begin{tikzpicture}[scale=0.875]
                         % Styles
                         \tikzstyle{cadre}=[thin]
                         \tikzstyle{fleche}=[->,>=latex,thin]
                         \tikzstyle{nondefini}=[lightgray]
                         % Dimensions Modifiables
                         \def\Lrg{1.5}
                         \def\HtX{1}
                         \def\HtY{0.5}
                         % Dimensions Calculées
                         \def\lignex{-0.7*\HtX}
                         \def\lignef{-1.5*\HtX}
                         \def\separateur{-0.5*\Lrg}
                         % Largeur du tableau
                         \def\gauche{-1.5*\Lrg}
                         \def\droite{4.5*\Lrg}
                         % Hauteur du tableau
                         \def\haut{0.5*\HtX}
                         \def\bas{-1.5*\HtX-2*\HtY}
                         % Pointillés
                         \draw[lightgray] (2*\Lrg,\lignex) -- (2*\Lrg,\lignef);
                         \draw[lightgray] (2*\Lrg,\lignef) -- (2*\Lrg,\bas);
                         % Ligne de l'abscisse : x
                         \node at (-1*\Lrg,-0.1) {$x$};
                         \node at (0*\Lrg,-0.1) {$ -\infty$};
                         \node at (2*\Lrg,-0.1) {$-\dfrac{b}{2a}$};
                         \node at (4*\Lrg,-0.1) {$+\infty $};
                         % Ligne de la fonction : f(x)
                         \node  at (-1*\Lrg,{-1*\HtX+(-1)*\HtY}) {$f(x)$};
                         \node (f1) at (0*\Lrg,{-1*\HtX+(-2)*\HtY}) {$ $};
                         \node (f2) at (2*\Lrg,{-1*\HtX+(0)*\HtY}) {$\text{max}$};
                         \node (f3) at (4*\Lrg,{-1*\HtX+(-2)*\HtY}) {$ $};
                         % Flèches
                         \draw[fleche] (f1) -- (f2);
                         \draw[fleche] (f2) -- (f3);
                         % Encadrement
                         \draw[cadre] (\separateur,\haut) -- (\separateur,\bas);
                         \draw[cadre] (\gauche,\haut) rectangle  (\droite,\bas);
                         \draw[cadre] (\gauche,\lignex) -- (\droite,\lignex);
                    \end{tikzpicture}
               \end{extern}
          \end{center}
\begin{center}\textit{Tableau de variations d'une fonction polynôme du second degré pour $a < 0$ }\end{center}
\bloc{orange}{Exemple}{% id="e60"
     Soit $f\left(x\right)=x^2-4x+3$
\begin{center}
\begin{extern}%width="280" alt="Parabole - Courbe représentative d'une fonction polynôme du second degré "
               \resizebox{5.5cm}{!}{%
                    % -+-+-+ variables modifiables
                    \def\fonction{x*x-4*x+3}
                    \def\xmin{-1.5}
                    \def\xmax{4.5}
                    \def\ymin{-1.2}
                    \def\ymax{4.9}
                    \def\xunit{1.5}  % unités en cm
                    \def\yunit{1.5}
                    \psset{xunit=\xunit,yunit=\yunit,algebraic=true}
                    \fontsize{15pt}{15pt}\selectfont
                    \begin{pspicture*}[linewidth=1pt](\xmin,\ymin)(\xmax,\ymax)
                         %      \psgrid[gridcolor=mcgris, subgriddiv=5, gridlabels=0pt](\xmin,\ymin)(\xmax,\ymax)
                         \psaxes[linewidth=0.75pt,Dx=1,Dy=1]{->}(0,0)(\xmin,\ymin)(\xmax,\ymax)
                         \psplot[plotpoints=2000,linecolor=blue]{\xmin}{\xmax}{\fonction}
                           \rput[tr](-0.2,-0.2){$O$} 
                                     \rput[tr](4.1,4.5){$\color{blue} \mathcal{C}_f$}
                    \end{pspicture*}
               }
\end{extern}
\end{center}
     \begin{center}\textit{Courbe représentative de $f~:~x\longmapsto x^2-4x+3$}\end{center}
}
\cadre{vert}{Propriété et définition}{% id="p70"
     Soit $f$ une fonction polynôme du second degré définie sur $\mathbb{R}$ par : $f\left(x\right)=ax^2+bx+c$
     \par
     $f\left(x\right)$ peut s'écrire sous la forme :
     \begin{center}$f\left(x\right)=a\left(x-\alpha \right)^2+\beta $\end{center}
     avec $\alpha  = -\frac{b}{2a}$ et $\beta  = f\left(\alpha \right)$
     \par
     Cette écriture est appelée \textbf{forme canonique}.
     \par
     $\left(\alpha  ; \beta \right)$ sont les coordonnées du sommet de la parabole.
}
\bloc{cyan}{Remarque}{% id="r70"
     Une caractéristique de la forme canonique est que la variable $x$ n'apparaît qu'à un seul endroit dans l'écriture.
}
\bloc{orange}{Exemple}{% id="e70"
     Reprenons l'exemple $f\left(x\right)=x^2-4x+3$
     \par
     On a $\alpha =-\frac{b}{2a}=- \frac{-4}{2\times 1}=2$
     \par
     et $\beta =f\left(2\right)=2^2-4\times 2+3=-1$
     \par
     donc la forme canonique de $f$ est :
     \par
     $f\left(x\right)=\left(x-2\right)^2-1$
}

\end{document}