\documentclass[a4paper]{article}

%================================================================================================================================
%
% Packages
%
%================================================================================================================================

\usepackage[T1]{fontenc} 	% pour caractères accentués
\usepackage[utf8]{inputenc}  % encodage utf8
\usepackage[french]{babel}	% langue : français
\usepackage{fourier}			% caractères plus lisibles
\usepackage[dvipsnames]{xcolor} % couleurs
\usepackage{fancyhdr}		% réglage header footer
\usepackage{needspace}		% empêcher sauts de page mal placés
\usepackage{graphicx}		% pour inclure des graphiques
\usepackage{enumitem,cprotect}		% personnalise les listes d'items (nécessaire pour ol, al ...)
\usepackage{hyperref}		% Liens hypertexte
\usepackage{pstricks,pst-all,pst-node,pstricks-add,pst-math,pst-plot,pst-tree,pst-eucl} % pstricks
\usepackage[a4paper,includeheadfoot,top=2cm,left=3cm, bottom=2cm,right=3cm]{geometry} % marges etc.
\usepackage{comment}			% commentaires multilignes
\usepackage{amsmath,environ} % maths (matrices, etc.)
\usepackage{amssymb,makeidx}
\usepackage{bm}				% bold maths
\usepackage{tabularx}		% tableaux
\usepackage{colortbl}		% tableaux en couleur
\usepackage{fontawesome}		% Fontawesome
\usepackage{environ}			% environment with command
\usepackage{fp}				% calculs pour ps-tricks
\usepackage{multido}			% pour ps tricks
\usepackage[np]{numprint}	% formattage nombre
\usepackage{tikz,tkz-tab} 			% package principal TikZ
\usepackage{pgfplots}   % axes
\usepackage{mathrsfs}    % cursives
\usepackage{calc}			% calcul taille boites
\usepackage[scaled=0.875]{helvet} % font sans serif
\usepackage{svg} % svg
\usepackage{scrextend} % local margin
\usepackage{scratch} %scratch
\usepackage{multicol} % colonnes
%\usepackage{infix-RPN,pst-func} % formule en notation polanaise inversée
\usepackage{listings}

%================================================================================================================================
%
% Réglages de base
%
%================================================================================================================================

\lstset{
language=Python,   % R code
literate=
{á}{{\'a}}1
{à}{{\`a}}1
{ã}{{\~a}}1
{é}{{\'e}}1
{è}{{\`e}}1
{ê}{{\^e}}1
{í}{{\'i}}1
{ó}{{\'o}}1
{õ}{{\~o}}1
{ú}{{\'u}}1
{ü}{{\"u}}1
{ç}{{\c{c}}}1
{~}{{ }}1
}


\definecolor{codegreen}{rgb}{0,0.6,0}
\definecolor{codegray}{rgb}{0.5,0.5,0.5}
\definecolor{codepurple}{rgb}{0.58,0,0.82}
\definecolor{backcolour}{rgb}{0.95,0.95,0.92}

\lstdefinestyle{mystyle}{
    backgroundcolor=\color{backcolour},   
    commentstyle=\color{codegreen},
    keywordstyle=\color{magenta},
    numberstyle=\tiny\color{codegray},
    stringstyle=\color{codepurple},
    basicstyle=\ttfamily\footnotesize,
    breakatwhitespace=false,         
    breaklines=true,                 
    captionpos=b,                    
    keepspaces=true,                 
    numbers=left,                    
xleftmargin=2em,
framexleftmargin=2em,            
    showspaces=false,                
    showstringspaces=false,
    showtabs=false,                  
    tabsize=2,
    upquote=true
}

\lstset{style=mystyle}


\lstset{style=mystyle}
\newcommand{\imgdir}{C:/laragon/www/newmc/assets/imgsvg/}
\newcommand{\imgsvgdir}{C:/laragon/www/newmc/assets/imgsvg/}

\definecolor{mcgris}{RGB}{220, 220, 220}% ancien~; pour compatibilité
\definecolor{mcbleu}{RGB}{52, 152, 219}
\definecolor{mcvert}{RGB}{125, 194, 70}
\definecolor{mcmauve}{RGB}{154, 0, 215}
\definecolor{mcorange}{RGB}{255, 96, 0}
\definecolor{mcturquoise}{RGB}{0, 153, 153}
\definecolor{mcrouge}{RGB}{255, 0, 0}
\definecolor{mclightvert}{RGB}{205, 234, 190}

\definecolor{gris}{RGB}{220, 220, 220}
\definecolor{bleu}{RGB}{52, 152, 219}
\definecolor{vert}{RGB}{125, 194, 70}
\definecolor{mauve}{RGB}{154, 0, 215}
\definecolor{orange}{RGB}{255, 96, 0}
\definecolor{turquoise}{RGB}{0, 153, 153}
\definecolor{rouge}{RGB}{255, 0, 0}
\definecolor{lightvert}{RGB}{205, 234, 190}
\setitemize[0]{label=\color{lightvert}  $\bullet$}

\pagestyle{fancy}
\renewcommand{\headrulewidth}{0.2pt}
\fancyhead[L]{maths-cours.fr}
\fancyhead[R]{\thepage}
\renewcommand{\footrulewidth}{0.2pt}
\fancyfoot[C]{}

\newcolumntype{C}{>{\centering\arraybackslash}X}
\newcolumntype{s}{>{\hsize=.35\hsize\arraybackslash}X}

\setlength{\parindent}{0pt}		 
\setlength{\parskip}{3mm}
\setlength{\headheight}{1cm}

\def\ebook{ebook}
\def\book{book}
\def\web{web}
\def\type{web}

\newcommand{\vect}[1]{\overrightarrow{\,\mathstrut#1\,}}

\def\Oij{$\left(\text{O}~;~\vect{\imath},~\vect{\jmath}\right)$}
\def\Oijk{$\left(\text{O}~;~\vect{\imath},~\vect{\jmath},~\vect{k}\right)$}
\def\Ouv{$\left(\text{O}~;~\vect{u},~\vect{v}\right)$}

\hypersetup{breaklinks=true, colorlinks = true, linkcolor = OliveGreen, urlcolor = OliveGreen, citecolor = OliveGreen, pdfauthor={Didier BONNEL - https://www.maths-cours.fr} } % supprime les bordures autour des liens

\renewcommand{\arg}[0]{\text{arg}}

\everymath{\displaystyle}

%================================================================================================================================
%
% Macros - Commandes
%
%================================================================================================================================

\newcommand\meta[2]{    			% Utilisé pour créer le post HTML.
	\def\titre{titre}
	\def\url{url}
	\def\arg{#1}
	\ifx\titre\arg
		\newcommand\maintitle{#2}
		\fancyhead[L]{#2}
		{\Large\sffamily \MakeUppercase{#2}}
		\vspace{1mm}\textcolor{mcvert}{\hrule}
	\fi 
	\ifx\url\arg
		\fancyfoot[L]{\href{https://www.maths-cours.fr#2}{\black \footnotesize{https://www.maths-cours.fr#2}}}
	\fi 
}


\newcommand\TitreC[1]{    		% Titre centré
     \needspace{3\baselineskip}
     \begin{center}\textbf{#1}\end{center}
}

\newcommand\newpar{    		% paragraphe
     \par
}

\newcommand\nosp {    		% commande vide (pas d'espace)
}
\newcommand{\id}[1]{} %ignore

\newcommand\boite[2]{				% Boite simple sans titre
	\vspace{5mm}
	\setlength{\fboxrule}{0.2mm}
	\setlength{\fboxsep}{5mm}	
	\fcolorbox{#1}{#1!3}{\makebox[\linewidth-2\fboxrule-2\fboxsep]{
  		\begin{minipage}[t]{\linewidth-2\fboxrule-4\fboxsep}\setlength{\parskip}{3mm}
  			 #2
  		\end{minipage}
	}}
	\vspace{5mm}
}

\newcommand\CBox[4]{				% Boites
	\vspace{5mm}
	\setlength{\fboxrule}{0.2mm}
	\setlength{\fboxsep}{5mm}
	
	\fcolorbox{#1}{#1!3}{\makebox[\linewidth-2\fboxrule-2\fboxsep]{
		\begin{minipage}[t]{1cm}\setlength{\parskip}{3mm}
	  		\textcolor{#1}{\LARGE{#2}}    
 	 	\end{minipage}  
  		\begin{minipage}[t]{\linewidth-2\fboxrule-4\fboxsep}\setlength{\parskip}{3mm}
			\raisebox{1.2mm}{\normalsize\sffamily{\textcolor{#1}{#3}}}						
  			 #4
  		\end{minipage}
	}}
	\vspace{5mm}
}

\newcommand\cadre[3]{				% Boites convertible html
	\par
	\vspace{2mm}
	\setlength{\fboxrule}{0.1mm}
	\setlength{\fboxsep}{5mm}
	\fcolorbox{#1}{white}{\makebox[\linewidth-2\fboxrule-2\fboxsep]{
  		\begin{minipage}[t]{\linewidth-2\fboxrule-4\fboxsep}\setlength{\parskip}{3mm}
			\raisebox{-2.5mm}{\sffamily \small{\textcolor{#1}{\MakeUppercase{#2}}}}		
			\par		
  			 #3
 	 		\end{minipage}
	}}
		\vspace{2mm}
	\par
}

\newcommand\bloc[3]{				% Boites convertible html sans bordure
     \needspace{2\baselineskip}
     {\sffamily \small{\textcolor{#1}{\MakeUppercase{#2}}}}    
		\par		
  			 #3
		\par
}

\newcommand\CHelp[1]{
     \CBox{Plum}{\faInfoCircle}{À RETENIR}{#1}
}

\newcommand\CUp[1]{
     \CBox{NavyBlue}{\faThumbsOUp}{EN PRATIQUE}{#1}
}

\newcommand\CInfo[1]{
     \CBox{Sepia}{\faArrowCircleRight}{REMARQUE}{#1}
}

\newcommand\CRedac[1]{
     \CBox{PineGreen}{\faEdit}{BIEN R\'EDIGER}{#1}
}

\newcommand\CError[1]{
     \CBox{Red}{\faExclamationTriangle}{ATTENTION}{#1}
}

\newcommand\TitreExo[2]{
\needspace{4\baselineskip}
 {\sffamily\large EXERCICE #1\ (\emph{#2 points})}
\vspace{5mm}
}

\newcommand\img[2]{
          \includegraphics[width=#2\paperwidth]{\imgdir#1}
}

\newcommand\imgsvg[2]{
       \begin{center}   \includegraphics[width=#2\paperwidth]{\imgsvgdir#1} \end{center}
}


\newcommand\Lien[2]{
     \href{#1}{#2 \tiny \faExternalLink}
}
\newcommand\mcLien[2]{
     \href{https~://www.maths-cours.fr/#1}{#2 \tiny \faExternalLink}
}

\newcommand{\euro}{\eurologo{}}

%================================================================================================================================
%
% Macros - Environement
%
%================================================================================================================================

\newenvironment{tex}{ %
}
{%
}

\newenvironment{indente}{ %
	\setlength\parindent{10mm}
}

{
	\setlength\parindent{0mm}
}

\newenvironment{corrige}{%
     \needspace{3\baselineskip}
     \medskip
     \textbf{\textsc{Corrigé}}
     \medskip
}
{
}

\newenvironment{extern}{%
     \begin{center}
     }
     {
     \end{center}
}

\NewEnviron{code}{%
	\par
     \boite{gray}{\texttt{%
     \BODY
     }}
     \par
}

\newenvironment{vbloc}{% boite sans cadre empeche saut de page
     \begin{minipage}[t]{\linewidth}
     }
     {
     \end{minipage}
}
\NewEnviron{h2}{%
    \needspace{3\baselineskip}
    \vspace{0.6cm}
	\noindent \MakeUppercase{\sffamily \large \BODY}
	\vspace{1mm}\textcolor{mcgris}{\hrule}\vspace{0.4cm}
	\par
}{}

\NewEnviron{h3}{%
    \needspace{3\baselineskip}
	\vspace{5mm}
	\textsc{\BODY}
	\par
}

\NewEnviron{margeneg}{ %
\begin{addmargin}[-1cm]{0cm}
\BODY
\end{addmargin}
}

\NewEnviron{html}{%
}

\begin{document}
\begin{h2}Exercice 4 (5 points)\end{h2}
\textbf{Candidats n'ayant pas suivi l'enseignement de spécialité}
\medskip
Dans l'espace muni du repère orthonormé $(O~;~\overrightarrow{i},~\overrightarrow{j}~,~\overrightarrow{k})$ d'unité 1~cm, on considère les points
A, B, C et D de coordonnées respectives $(2~;~1~;~4)$, $(4~;~-1~;~0)$, $(0~;~3~;~2)$ et $(4~;~3~;~-2)$.
\medskip
\begin{enumerate}
     \item Déterminer une représentation paramétrique de la droite (CD).
     \item Soit M un point de la droite (CD).
     \begin{enumerate}[label=\alph*.]
          \item Déterminer les coordonnées du point M tel que la distance BM soit minimale.
          \item On note H le point de la droite (CD) ayant pour coordonnées $(3~;~3~;~- 1)$.
          Vérifier que les droites (BH) et (CD) sont perpendiculaires.
          \item Montrer que l'aire du triangle BCD est égale à 12 cm$^2$.
     \end{enumerate}
     \item
     \begin{enumerate}[label=\alph*.]
          \item Démontrer que le vecteur $\overrightarrow{n}\begin{pmatrix}2\\1\\2\end{pmatrix}$  est un vecteur normal au plan (BCD).
          \item Déterminer une équation cartésienne du plan (BCD).
          \item Déterminer une représentation paramétrique de la droite $\Delta$ passant par A et orthogonale
          au plan (BCD).
          \item Démontrer que le point I, intersection de la droite $\Delta$ et du plan (BCD) a pour
          coordonnées $\left(\dfrac{2}{3}~;~\dfrac{1}{3}~;~\dfrac{8}{3}\right)$.
     \end{enumerate}
     \item  Calculer le volume du tétraèdre ABCD.
\end{enumerate}
\begin{corrige}
     \begin{enumerate}
          \item
          \par
          Un vecteur directeur de la droite $(CD)$ est le vecteur $\overrightarrow{CD}$ de coordonnées $\begin{pmatrix} 4\\0\\-4 \end{pmatrix}$. Cette droite passe par le point  $C(0~;~3~;~2)$.
          \par
          Une représentation paramétrique de la droite $(CD)$ est donc~:
          \begin{center}
               $\begin{cases}
                    x=t\\y=3\\z=2-t
               \end{cases}~~(t \in  \mathbb{R})$
          \end{center}
          \item
          \begin{enumerate}[label=\alph*.]
               \item
               Si $M$ est un point de la droite $(CD)$, ses coordonnées sont de la forme $(t~;~3~;~2-t)$ où $t \in \mathbb{R}$
               \par
               La distance $BM$ vaut alors~:
               \par
               $BM=\sqrt{\left(t-4 \right)^2+\left(4 \right)^2+\left(2-t \right)^2}$\\
               $BM=\sqrt{2t^2-12t+36} $
               \par
               La distance $BM$ est minimale lorsque $2t^2-12t+36$ est minimal, c'est à dire pour $t= -\dfrac{b}{2a}=-\dfrac{-12}{4}=3$
               \par
               En remplaçant $t$ par $3$ dans les coordonnées du point $M$, on obtient que la distance  $BM$ est minimale pour $M(3~;~3~;~-1)$.
               \item
               Le vecteur $\overrightarrow{BH}$ a pour coordonnées $ \begin{pmatrix}-1\\4\\-1\end{pmatrix}$.
               \par
               Le vecteur  $\overrightarrow{CD}$ a pour coordonnées $\begin{pmatrix}4\\0\\-4\end{pmatrix}$.
               \par
               Le produit scalaire $\overrightarrow{HB} \cdot \overrightarrow{CD} $ vaut donc~:
               \par
               $\overrightarrow{HB}\cdot \overrightarrow{CD} = -1 \times 4+ 4 \times 0-1 \times (-4)= 0$
               \par
               Les droites $(BH)$ et $(CD)$ sont donc orthogonales et comme elles sont sécantes en $H$, elles sont perpendiculaires.
               \item
               D'après la question précédente, $(BH)$ est la hauteur issue de $B$ dans le triangle $BCD$.
               \par
               Par conséquent, l'aire du triangle $BCD$ est égale à~:
               \par
               $\mathscr{A}=\dfrac{1}{2} \times  CD \times  BH$\nosp$=\dfrac{1}{2}\times \sqrt{32} \times  \sqrt{18}$\nosp$=\dfrac{1}{2}\sqrt{576}=12$cm$^2$
          \end{enumerate}
          \item
          \begin{enumerate}[label=\alph*.]
               \item
               Le vecteur $\overrightarrow{n}$ est un vecteur normal au plan $(BCD)$ si et seulement s'il  est orthogonal à deux vecteurs non colinéaires de ce plan.
               \par
               Les vecteurs
               $\overrightarrow{BC}\begin{pmatrix}
                    -4\\4\\2
                    \end{pmatrix}$ et $\overrightarrow{CD}\begin{pmatrix}
                    4\\0\\-4
               \end{pmatrix}$ ne sont
               pas colinéaires et~:
               \par
               $\overrightarrow{n}\cdot\overrightarrow{BC}=-4 \times 2+4 \times 1+2\times 2=0$\\
               $\overrightarrow{n}\cdot\overrightarrow{CD}=4 \times 2+0\times 1-4\times 2=0$
               \par
               Le vecteur$\overrightarrow{n}$ est donc bien normal au plan $(BCD)$.
               \item
               Le vecteur $\overrightarrow{n}\begin{pmatrix}2\\1\\2\end{pmatrix}$ est normal au plan $(BCD)$ donc ce plan admet une équation cartésienne de la forme~: $2x+y+2z+d=0$ où $d \in \mathbb{R}$.
               \par
               Par ailleurs, le point $B(4~;~-1~;~0)$ appartient à ce plan donc ses coordonnées vérifient l'équation du plan.
               \par
               Par conséquent $2 \times 4 -1+2 \times 0+d=0$ donc $d=-7$.
               \par
               Une équation cartésienne du plan $(BCD)$ est donc $2x+y+2z-7=0$.
               \item
               $\Delta$ étant orthogonale au plan $(BCD)$, le vecteur $\overrightarrow{n}$ est un vecteur directeur de $\Delta$. Comme par ailleurs la droite $\Delta$ passe par le point $A(2~;~1~;~4)$, une représentation paramétrique de la droite  $\Delta$ est~:
               \begin{center}
                    $\begin{cases}
                         x=2+2t\\y=1+t\\z=4+2t
                    \end{cases}~~(t\in \mathbb{R})$
               \end{center}
               \item
               Soient $(x~;~y~;~z)$ les coordonnées du point $I$, intersection de la droite $\Delta$ et du plan $(BCD)$.
               \par
               Il existe une valeur de $t$ telle que les coordonnées de $I$ vérifient simultanément les équations~:
               \par
               \begin{center}
                    $\begin{cases}
                         x=2+2t\\y=1+t\\z=4+2t\\2x+y+2z-7=0
                    \end{cases}$
               \end{center}
               On a alors~:
               \par
               $2(2+2t)+(1+t)+2(4+2t)-7=0$
               \par
               soit $9t=-6$ et donc $t=-\dfrac{2}{3}$.
               \par
               Les coordonnées de $I$ sont donc~:
               \par
               $x=2+2t=\dfrac{2}{3}$\\
               $y=1+t=\dfrac{1}{3}$\\
               $z=4+2t=~\dfrac{8}{3}$\\
          \end{enumerate}
          \item
          D'après les questions précédentes, la droite $(AI)$ est la perpendiculaire au plan $(BCD)$ passant par $A$.
          \par
          Les coordonnées du vecteur $\overrightarrow{AI}$ sont $\begin{pmatrix}-4/3\\-2/3\\-4/3\end{pmatrix}$.
          \par
          La hauteur du tétraèdre $ABCD$ associée à la base $BCD$ est donc~:
          \par
          $AI=\sqrt{\left(-\dfrac{4}{3} \right)^2+\left(-\dfrac{2}{3} \right)^2+\left(-\dfrac{4}{3} \right)^2}=2$cm.
          \par
          Le volume du tétraèdre $ABCD$ est alors~:
          \par
          $\mathscr{V}=\dfrac{1}{3} \times \mathscr{A} \times  AI =\dfrac{1}{3} \times 12 \times 2=8$cm$^3$.
     \end{enumerate}
\end{corrige}
\par

\end{document}