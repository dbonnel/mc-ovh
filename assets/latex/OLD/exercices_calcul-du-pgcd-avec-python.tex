\documentclass[a4paper]{article}

%================================================================================================================================
%
% Packages
%
%================================================================================================================================

\usepackage[T1]{fontenc} 	% pour caractères accentués
\usepackage[utf8]{inputenc}  % encodage utf8
\usepackage[french]{babel}	% langue : français
\usepackage{fourier}			% caractères plus lisibles
\usepackage[dvipsnames]{xcolor} % couleurs
\usepackage{fancyhdr}		% réglage header footer
\usepackage{needspace}		% empêcher sauts de page mal placés
\usepackage{graphicx}		% pour inclure des graphiques
\usepackage{enumitem,cprotect}		% personnalise les listes d'items (nécessaire pour ol, al ...)
\usepackage{hyperref}		% Liens hypertexte
\usepackage{pstricks,pst-all,pst-node,pstricks-add,pst-math,pst-plot,pst-tree,pst-eucl} % pstricks
\usepackage[a4paper,includeheadfoot,top=2cm,left=3cm, bottom=2cm,right=3cm]{geometry} % marges etc.
\usepackage{comment}			% commentaires multilignes
\usepackage{amsmath,environ} % maths (matrices, etc.)
\usepackage{amssymb,makeidx}
\usepackage{bm}				% bold maths
\usepackage{tabularx}		% tableaux
\usepackage{colortbl}		% tableaux en couleur
\usepackage{fontawesome}		% Fontawesome
\usepackage{environ}			% environment with command
\usepackage{fp}				% calculs pour ps-tricks
\usepackage{multido}			% pour ps tricks
\usepackage[np]{numprint}	% formattage nombre
\usepackage{tikz,tkz-tab} 			% package principal TikZ
\usepackage{pgfplots}   % axes
\usepackage{mathrsfs}    % cursives
\usepackage{calc}			% calcul taille boites
\usepackage[scaled=0.875]{helvet} % font sans serif
\usepackage{svg} % svg
\usepackage{scrextend} % local margin
\usepackage{scratch} %scratch
\usepackage{multicol} % colonnes
%\usepackage{infix-RPN,pst-func} % formule en notation polanaise inversée
\usepackage{listings}

%================================================================================================================================
%
% Réglages de base
%
%================================================================================================================================

\lstset{
language=Python,   % R code
literate=
{á}{{\'a}}1
{à}{{\`a}}1
{ã}{{\~a}}1
{é}{{\'e}}1
{è}{{\`e}}1
{ê}{{\^e}}1
{í}{{\'i}}1
{ó}{{\'o}}1
{õ}{{\~o}}1
{ú}{{\'u}}1
{ü}{{\"u}}1
{ç}{{\c{c}}}1
{~}{{ }}1
}


\definecolor{codegreen}{rgb}{0,0.6,0}
\definecolor{codegray}{rgb}{0.5,0.5,0.5}
\definecolor{codepurple}{rgb}{0.58,0,0.82}
\definecolor{backcolour}{rgb}{0.95,0.95,0.92}

\lstdefinestyle{mystyle}{
    backgroundcolor=\color{backcolour},   
    commentstyle=\color{codegreen},
    keywordstyle=\color{magenta},
    numberstyle=\tiny\color{codegray},
    stringstyle=\color{codepurple},
    basicstyle=\ttfamily\footnotesize,
    breakatwhitespace=false,         
    breaklines=true,                 
    captionpos=b,                    
    keepspaces=true,                 
    numbers=left,                    
xleftmargin=2em,
framexleftmargin=2em,            
    showspaces=false,                
    showstringspaces=false,
    showtabs=false,                  
    tabsize=2,
    upquote=true
}

\lstset{style=mystyle}


\lstset{style=mystyle}
\newcommand{\imgdir}{C:/laragon/www/newmc/assets/imgsvg/}
\newcommand{\imgsvgdir}{C:/laragon/www/newmc/assets/imgsvg/}

\definecolor{mcgris}{RGB}{220, 220, 220}% ancien~; pour compatibilité
\definecolor{mcbleu}{RGB}{52, 152, 219}
\definecolor{mcvert}{RGB}{125, 194, 70}
\definecolor{mcmauve}{RGB}{154, 0, 215}
\definecolor{mcorange}{RGB}{255, 96, 0}
\definecolor{mcturquoise}{RGB}{0, 153, 153}
\definecolor{mcrouge}{RGB}{255, 0, 0}
\definecolor{mclightvert}{RGB}{205, 234, 190}

\definecolor{gris}{RGB}{220, 220, 220}
\definecolor{bleu}{RGB}{52, 152, 219}
\definecolor{vert}{RGB}{125, 194, 70}
\definecolor{mauve}{RGB}{154, 0, 215}
\definecolor{orange}{RGB}{255, 96, 0}
\definecolor{turquoise}{RGB}{0, 153, 153}
\definecolor{rouge}{RGB}{255, 0, 0}
\definecolor{lightvert}{RGB}{205, 234, 190}
\setitemize[0]{label=\color{lightvert}  $\bullet$}

\pagestyle{fancy}
\renewcommand{\headrulewidth}{0.2pt}
\fancyhead[L]{maths-cours.fr}
\fancyhead[R]{\thepage}
\renewcommand{\footrulewidth}{0.2pt}
\fancyfoot[C]{}

\newcolumntype{C}{>{\centering\arraybackslash}X}
\newcolumntype{s}{>{\hsize=.35\hsize\arraybackslash}X}

\setlength{\parindent}{0pt}		 
\setlength{\parskip}{3mm}
\setlength{\headheight}{1cm}

\def\ebook{ebook}
\def\book{book}
\def\web{web}
\def\type{web}

\newcommand{\vect}[1]{\overrightarrow{\,\mathstrut#1\,}}

\def\Oij{$\left(\text{O}~;~\vect{\imath},~\vect{\jmath}\right)$}
\def\Oijk{$\left(\text{O}~;~\vect{\imath},~\vect{\jmath},~\vect{k}\right)$}
\def\Ouv{$\left(\text{O}~;~\vect{u},~\vect{v}\right)$}

\hypersetup{breaklinks=true, colorlinks = true, linkcolor = OliveGreen, urlcolor = OliveGreen, citecolor = OliveGreen, pdfauthor={Didier BONNEL - https://www.maths-cours.fr} } % supprime les bordures autour des liens

\renewcommand{\arg}[0]{\text{arg}}

\everymath{\displaystyle}

%================================================================================================================================
%
% Macros - Commandes
%
%================================================================================================================================

\newcommand\meta[2]{    			% Utilisé pour créer le post HTML.
	\def\titre{titre}
	\def\url{url}
	\def\arg{#1}
	\ifx\titre\arg
		\newcommand\maintitle{#2}
		\fancyhead[L]{#2}
		{\Large\sffamily \MakeUppercase{#2}}
		\vspace{1mm}\textcolor{mcvert}{\hrule}
	\fi 
	\ifx\url\arg
		\fancyfoot[L]{\href{https://www.maths-cours.fr#2}{\black \footnotesize{https://www.maths-cours.fr#2}}}
	\fi 
}


\newcommand\TitreC[1]{    		% Titre centré
     \needspace{3\baselineskip}
     \begin{center}\textbf{#1}\end{center}
}

\newcommand\newpar{    		% paragraphe
     \par
}

\newcommand\nosp {    		% commande vide (pas d'espace)
}
\newcommand{\id}[1]{} %ignore

\newcommand\boite[2]{				% Boite simple sans titre
	\vspace{5mm}
	\setlength{\fboxrule}{0.2mm}
	\setlength{\fboxsep}{5mm}	
	\fcolorbox{#1}{#1!3}{\makebox[\linewidth-2\fboxrule-2\fboxsep]{
  		\begin{minipage}[t]{\linewidth-2\fboxrule-4\fboxsep}\setlength{\parskip}{3mm}
  			 #2
  		\end{minipage}
	}}
	\vspace{5mm}
}

\newcommand\CBox[4]{				% Boites
	\vspace{5mm}
	\setlength{\fboxrule}{0.2mm}
	\setlength{\fboxsep}{5mm}
	
	\fcolorbox{#1}{#1!3}{\makebox[\linewidth-2\fboxrule-2\fboxsep]{
		\begin{minipage}[t]{1cm}\setlength{\parskip}{3mm}
	  		\textcolor{#1}{\LARGE{#2}}    
 	 	\end{minipage}  
  		\begin{minipage}[t]{\linewidth-2\fboxrule-4\fboxsep}\setlength{\parskip}{3mm}
			\raisebox{1.2mm}{\normalsize\sffamily{\textcolor{#1}{#3}}}						
  			 #4
  		\end{minipage}
	}}
	\vspace{5mm}
}

\newcommand\cadre[3]{				% Boites convertible html
	\par
	\vspace{2mm}
	\setlength{\fboxrule}{0.1mm}
	\setlength{\fboxsep}{5mm}
	\fcolorbox{#1}{white}{\makebox[\linewidth-2\fboxrule-2\fboxsep]{
  		\begin{minipage}[t]{\linewidth-2\fboxrule-4\fboxsep}\setlength{\parskip}{3mm}
			\raisebox{-2.5mm}{\sffamily \small{\textcolor{#1}{\MakeUppercase{#2}}}}		
			\par		
  			 #3
 	 		\end{minipage}
	}}
		\vspace{2mm}
	\par
}

\newcommand\bloc[3]{				% Boites convertible html sans bordure
     \needspace{2\baselineskip}
     {\sffamily \small{\textcolor{#1}{\MakeUppercase{#2}}}}    
		\par		
  			 #3
		\par
}

\newcommand\CHelp[1]{
     \CBox{Plum}{\faInfoCircle}{À RETENIR}{#1}
}

\newcommand\CUp[1]{
     \CBox{NavyBlue}{\faThumbsOUp}{EN PRATIQUE}{#1}
}

\newcommand\CInfo[1]{
     \CBox{Sepia}{\faArrowCircleRight}{REMARQUE}{#1}
}

\newcommand\CRedac[1]{
     \CBox{PineGreen}{\faEdit}{BIEN R\'EDIGER}{#1}
}

\newcommand\CError[1]{
     \CBox{Red}{\faExclamationTriangle}{ATTENTION}{#1}
}

\newcommand\TitreExo[2]{
\needspace{4\baselineskip}
 {\sffamily\large EXERCICE #1\ (\emph{#2 points})}
\vspace{5mm}
}

\newcommand\img[2]{
          \includegraphics[width=#2\paperwidth]{\imgdir#1}
}

\newcommand\imgsvg[2]{
       \begin{center}   \includegraphics[width=#2\paperwidth]{\imgsvgdir#1} \end{center}
}


\newcommand\Lien[2]{
     \href{#1}{#2 \tiny \faExternalLink}
}
\newcommand\mcLien[2]{
     \href{https~://www.maths-cours.fr/#1}{#2 \tiny \faExternalLink}
}

\newcommand{\euro}{\eurologo{}}

%================================================================================================================================
%
% Macros - Environement
%
%================================================================================================================================

\newenvironment{tex}{ %
}
{%
}

\newenvironment{indente}{ %
	\setlength\parindent{10mm}
}

{
	\setlength\parindent{0mm}
}

\newenvironment{corrige}{%
     \needspace{3\baselineskip}
     \medskip
     \textbf{\textsc{Corrigé}}
     \medskip
}
{
}

\newenvironment{extern}{%
     \begin{center}
     }
     {
     \end{center}
}

\NewEnviron{code}{%
	\par
     \boite{gray}{\texttt{%
     \BODY
     }}
     \par
}

\newenvironment{vbloc}{% boite sans cadre empeche saut de page
     \begin{minipage}[t]{\linewidth}
     }
     {
     \end{minipage}
}
\NewEnviron{h2}{%
    \needspace{3\baselineskip}
    \vspace{0.6cm}
	\noindent \MakeUppercase{\sffamily \large \BODY}
	\vspace{1mm}\textcolor{mcgris}{\hrule}\vspace{0.4cm}
	\par
}{}

\NewEnviron{h3}{%
    \needspace{3\baselineskip}
	\vspace{5mm}
	\textsc{\BODY}
	\par
}

\NewEnviron{margeneg}{ %
\begin{addmargin}[-1cm]{0cm}
\BODY
\end{addmargin}
}

\NewEnviron{html}  renvoie le reste de la division euclidienne de deux entiers.  

\item
En utilisant la fonction \texttt{diviseurs} de la question précédente, écrire une fonction  \texttt{est_premier} qui prend un argument entier et retourne un booléen valant  \texttt{True} si l'argument est un nombre premier et  \texttt{False} dans le cas contraire.
\par
Tester votre fonction à l'aide des instructions suivantes :
\begin{lstlisting}[language=Python]
>>> est_premier(1)
>>> est_premier(2)
>>> est_premier(100)
>>> est_premier(101)
\end{lstlisting}

\item
L'opérateur Python  \texttt{in} permet de savoir si un élément appartient ou non à une liste. 
\\
On l'utilise de la manière suivante~: 
\begin{lstlisting}[language=Python]
liste = ['rouge', 'vert', 'jaune']
print('rouge' in liste) # affiche True
print('bleu' in liste) # affiche False
\end{lstlisting}
\\
 À l'aide de la fonction \texttt{diviseurs} de la première question et de l'opérateur  \texttt{in} écrire une fonction nommée  \texttt{inter} qui prend deux listes en arguments et qui renvoie une troisième liste contenant les éléments communs aux deux listes.
\par
Par exemple on souhaite que l'instruction \texttt{inter(['a', 'b', 'c'], ['b', 'c', 'd', 'e']) } retourne la liste  \texttt{['b', 'c']}.  
 \par
Tester votre fonction à l'aide des instructions suivantes :
\begin{lstlisting}[language=Python]
>>> inter(['a', 'b', 'c'], ['b', 'c', 'd', 'e'])
>>> inter([1,2,4], [1, 2, 3, 6])
>>> inter([1,2,3], [4,5,6])
\end{lstlisting}

\item
 À l'aide des fonctions définies aux questions  \textbf{1} et  \textbf{3},  écrire une fonction nommée  \texttt{pgcd} qui prend deux entiers naturels en arguments et qui retourne leur PGCD.
\par
Tester votre fonction à l'aide des instructions suivantes :
\begin{lstlisting}[language=Python]
>>> pgcd(10, 11)
>>> pgcd(15,20)
>>> pgcd(501,666)
>>> pgcd(110,121)
\end{lstlisting} 
\end{enumerate} 

\begin{corrige}

\begin{enumerate}
\item

On peut coder la fonction à l'aide d'une boucle  \texttt{for}.
\par
L'instruction  \texttt{range(1, n+1)} retourne les entiers naturels compris, au sens large, entre  \texttt{1} et  \texttt{n}. On teste ensuite si l'entier  \texttt{i} divise  \texttt{n} grâce à l'instruction   \texttt{if n % i == 0 :} et, si c'est le cas, on ajoute  \texttt{i} à la liste des diviseurs~: 
\\  
\begin{lstlisting}[language=Python]
 
def diviseurs(n) :
    liste = []
    for i in range(1, n+1) :
        if n % i == 0 :
            liste.append(i)
    return liste

\end{lstlisting} 
\\
Le mode de construction de cette liste fait que celle-ci est automatiquement triée par ordre croissant.
\par
Il est également possible et plus concis d'utiliser une définition de la liste en  \textit{compréhension}~:  
\begin{lstlisting}[language=Python]
def diviseurs(n) :
    return [i for i in range(1,n+1) if n % i == 0]
\end{lstlisting} 
\par
Les tests proposés dans l'énoncé donnent les résultats suivants~: 
\\
\begin{lstlisting}[language=Python]
>>> diviseurs(12)
[1, 2, 3, 4, 6, 12]
>>> diviseurs(1)
[1]
>>> diviseurs(2)
[1, 2]
>>> diviseurs(100)
[1, 2, 4, 5, 10, 20, 25, 50, 100]
>>> diviseurs(101)
[1, 101]
\end{lstlisting} 
\item
Un nombre entier naturel est premier si et seulement s'il possède exactement deux diviseurs~: 1 et lui-même. 
\par
Il suffit donc de tester la longueur de la liste retournée par la fonction  \texttt{diviseurs()} pour déterminer si l'argument est premier. La commande  \texttt{len(diviseurs(n)) == 2} renvoie  \texttt{True} si cette liste contient deux nombres et  \texttt{False} sinon~:   
 
\begin{lstlisting}[language=Python]
def est_premier(n) :
    return len(diviseurs(n)) == 2
\end{lstlisting} 
\par
Cette fonction donne les résultats suivants~: 
\begin{lstlisting}[language=Python]
>>> est_premier(1)
False
>>> est_premier(2)
True
>>> est_premier(100)
False
>>> est_premier(101)
True
\end{lstlisting}

\item
Voici un exemple de fonction qui donne le résultat souhaité (mais qui n'est pas le plus concis~: 
\\
\begin{lstlisting}[language=Python]
def inter(liste1, liste2) :
    liste3=[]
    for i in range(len(liste1)) :
        a = liste1[i]
        if a in liste2 :
            liste3.append(a)
    return liste3
\end{lstlisting} 
\\ 
Pour chaque élément de la première liste, on regarde s'il appartient à la seconde liste grâce à l'opérateur  \texttt{in}~; si c'est le cas, on ajoute cet élément à la réponse.  
\par
Voici le résultat des tests~: 
\begin{lstlisting}[language=Python]
>>> inter(['a', 'b', 'c'], ['b', 'c', 'd', 'e'])
['b', 'c']
>>> inter([1,2,4], [1, 2, 3, 6])
[1, 2]
>>> inter([1,2,3], [4,5,6])
[]
\end{lstlisting} 

\item
La liste des diviseurs étant ordonnées par ordre croissant, il suffit de retourner le dernier élément de la liste des diviseurs communs (obtenue en faisant l'intersection des deux listes de diviseurs)~:   
\par
\textbf{Attention :} les éléments d'une liste sont indexés de  \texttt{0} à  \texttt{len(liste)-1}.  L'indice du dernier élément est donc \texttt{len(liste)-1}. 
\par
\begin{lstlisting}[language=Python]
def pgcd(a, b) :
    diviseurs_communs = inter(diviseurs(a), diviseurs(b))
    return diviseurs_communs[len(diviseurs_communs)-1]
\end{lstlisting}  
\par
Les tests demandés donnent les résultats ci-dessous~: 
\\
\begin{lstlisting}[language=Python]
>>> pgcd(10, 11)
1
>>> pgcd(15,20)
5
>>> pgcd(501,666)
3
>>> pgcd(110,121)
11
\end{lstlisting} 
\end{enumerate} 
\end{corrige} 

\end{document}