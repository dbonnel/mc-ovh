\documentclass[a4paper]{article}

%================================================================================================================================
%
% Packages
%
%================================================================================================================================

\usepackage[T1]{fontenc} 	% pour caractères accentués
\usepackage[utf8]{inputenc}  % encodage utf8
\usepackage[french]{babel}	% langue : français
\usepackage{fourier}			% caractères plus lisibles
\usepackage[dvipsnames]{xcolor} % couleurs
\usepackage{fancyhdr}		% réglage header footer
\usepackage{needspace}		% empêcher sauts de page mal placés
\usepackage{graphicx}		% pour inclure des graphiques
\usepackage{enumitem,cprotect}		% personnalise les listes d'items (nécessaire pour ol, al ...)
\usepackage{hyperref}		% Liens hypertexte
\usepackage{pstricks,pst-all,pst-node,pstricks-add,pst-math,pst-plot,pst-tree,pst-eucl} % pstricks
\usepackage[a4paper,includeheadfoot,top=2cm,left=3cm, bottom=2cm,right=3cm]{geometry} % marges etc.
\usepackage{comment}			% commentaires multilignes
\usepackage{amsmath,environ} % maths (matrices, etc.)
\usepackage{amssymb,makeidx}
\usepackage{bm}				% bold maths
\usepackage{tabularx}		% tableaux
\usepackage{colortbl}		% tableaux en couleur
\usepackage{fontawesome}		% Fontawesome
\usepackage{environ}			% environment with command
\usepackage{fp}				% calculs pour ps-tricks
\usepackage{multido}			% pour ps tricks
\usepackage[np]{numprint}	% formattage nombre
\usepackage{tikz,tkz-tab} 			% package principal TikZ
\usepackage{pgfplots}   % axes
\usepackage{mathrsfs}    % cursives
\usepackage{calc}			% calcul taille boites
\usepackage[scaled=0.875]{helvet} % font sans serif
\usepackage{svg} % svg
\usepackage{scrextend} % local margin
\usepackage{scratch} %scratch
\usepackage{multicol} % colonnes
%\usepackage{infix-RPN,pst-func} % formule en notation polanaise inversée
\usepackage{listings}

%================================================================================================================================
%
% Réglages de base
%
%================================================================================================================================

\lstset{
language=Python,   % R code
literate=
{á}{{\'a}}1
{à}{{\`a}}1
{ã}{{\~a}}1
{é}{{\'e}}1
{è}{{\`e}}1
{ê}{{\^e}}1
{í}{{\'i}}1
{ó}{{\'o}}1
{õ}{{\~o}}1
{ú}{{\'u}}1
{ü}{{\"u}}1
{ç}{{\c{c}}}1
{~}{{ }}1
}


\definecolor{codegreen}{rgb}{0,0.6,0}
\definecolor{codegray}{rgb}{0.5,0.5,0.5}
\definecolor{codepurple}{rgb}{0.58,0,0.82}
\definecolor{backcolour}{rgb}{0.95,0.95,0.92}

\lstdefinestyle{mystyle}{
    backgroundcolor=\color{backcolour},   
    commentstyle=\color{codegreen},
    keywordstyle=\color{magenta},
    numberstyle=\tiny\color{codegray},
    stringstyle=\color{codepurple},
    basicstyle=\ttfamily\footnotesize,
    breakatwhitespace=false,         
    breaklines=true,                 
    captionpos=b,                    
    keepspaces=true,                 
    numbers=left,                    
xleftmargin=2em,
framexleftmargin=2em,            
    showspaces=false,                
    showstringspaces=false,
    showtabs=false,                  
    tabsize=2,
    upquote=true
}

\lstset{style=mystyle}


\lstset{style=mystyle}
\newcommand{\imgdir}{C:/laragon/www/newmc/assets/imgsvg/}
\newcommand{\imgsvgdir}{C:/laragon/www/newmc/assets/imgsvg/}

\definecolor{mcgris}{RGB}{220, 220, 220}% ancien~; pour compatibilité
\definecolor{mcbleu}{RGB}{52, 152, 219}
\definecolor{mcvert}{RGB}{125, 194, 70}
\definecolor{mcmauve}{RGB}{154, 0, 215}
\definecolor{mcorange}{RGB}{255, 96, 0}
\definecolor{mcturquoise}{RGB}{0, 153, 153}
\definecolor{mcrouge}{RGB}{255, 0, 0}
\definecolor{mclightvert}{RGB}{205, 234, 190}

\definecolor{gris}{RGB}{220, 220, 220}
\definecolor{bleu}{RGB}{52, 152, 219}
\definecolor{vert}{RGB}{125, 194, 70}
\definecolor{mauve}{RGB}{154, 0, 215}
\definecolor{orange}{RGB}{255, 96, 0}
\definecolor{turquoise}{RGB}{0, 153, 153}
\definecolor{rouge}{RGB}{255, 0, 0}
\definecolor{lightvert}{RGB}{205, 234, 190}
\setitemize[0]{label=\color{lightvert}  $\bullet$}

\pagestyle{fancy}
\renewcommand{\headrulewidth}{0.2pt}
\fancyhead[L]{maths-cours.fr}
\fancyhead[R]{\thepage}
\renewcommand{\footrulewidth}{0.2pt}
\fancyfoot[C]{}

\newcolumntype{C}{>{\centering\arraybackslash}X}
\newcolumntype{s}{>{\hsize=.35\hsize\arraybackslash}X}

\setlength{\parindent}{0pt}		 
\setlength{\parskip}{3mm}
\setlength{\headheight}{1cm}

\def\ebook{ebook}
\def\book{book}
\def\web{web}
\def\type{web}

\newcommand{\vect}[1]{\overrightarrow{\,\mathstrut#1\,}}

\def\Oij{$\left(\text{O}~;~\vect{\imath},~\vect{\jmath}\right)$}
\def\Oijk{$\left(\text{O}~;~\vect{\imath},~\vect{\jmath},~\vect{k}\right)$}
\def\Ouv{$\left(\text{O}~;~\vect{u},~\vect{v}\right)$}

\hypersetup{breaklinks=true, colorlinks = true, linkcolor = OliveGreen, urlcolor = OliveGreen, citecolor = OliveGreen, pdfauthor={Didier BONNEL - https://www.maths-cours.fr} } % supprime les bordures autour des liens

\renewcommand{\arg}[0]{\text{arg}}

\everymath{\displaystyle}

%================================================================================================================================
%
% Macros - Commandes
%
%================================================================================================================================

\newcommand\meta[2]{    			% Utilisé pour créer le post HTML.
	\def\titre{titre}
	\def\url{url}
	\def\arg{#1}
	\ifx\titre\arg
		\newcommand\maintitle{#2}
		\fancyhead[L]{#2}
		{\Large\sffamily \MakeUppercase{#2}}
		\vspace{1mm}\textcolor{mcvert}{\hrule}
	\fi 
	\ifx\url\arg
		\fancyfoot[L]{\href{https://www.maths-cours.fr#2}{\black \footnotesize{https://www.maths-cours.fr#2}}}
	\fi 
}


\newcommand\TitreC[1]{    		% Titre centré
     \needspace{3\baselineskip}
     \begin{center}\textbf{#1}\end{center}
}

\newcommand\newpar{    		% paragraphe
     \par
}

\newcommand\nosp {    		% commande vide (pas d'espace)
}
\newcommand{\id}[1]{} %ignore

\newcommand\boite[2]{				% Boite simple sans titre
	\vspace{5mm}
	\setlength{\fboxrule}{0.2mm}
	\setlength{\fboxsep}{5mm}	
	\fcolorbox{#1}{#1!3}{\makebox[\linewidth-2\fboxrule-2\fboxsep]{
  		\begin{minipage}[t]{\linewidth-2\fboxrule-4\fboxsep}\setlength{\parskip}{3mm}
  			 #2
  		\end{minipage}
	}}
	\vspace{5mm}
}

\newcommand\CBox[4]{				% Boites
	\vspace{5mm}
	\setlength{\fboxrule}{0.2mm}
	\setlength{\fboxsep}{5mm}
	
	\fcolorbox{#1}{#1!3}{\makebox[\linewidth-2\fboxrule-2\fboxsep]{
		\begin{minipage}[t]{1cm}\setlength{\parskip}{3mm}
	  		\textcolor{#1}{\LARGE{#2}}    
 	 	\end{minipage}  
  		\begin{minipage}[t]{\linewidth-2\fboxrule-4\fboxsep}\setlength{\parskip}{3mm}
			\raisebox{1.2mm}{\normalsize\sffamily{\textcolor{#1}{#3}}}						
  			 #4
  		\end{minipage}
	}}
	\vspace{5mm}
}

\newcommand\cadre[3]{				% Boites convertible html
	\par
	\vspace{2mm}
	\setlength{\fboxrule}{0.1mm}
	\setlength{\fboxsep}{5mm}
	\fcolorbox{#1}{white}{\makebox[\linewidth-2\fboxrule-2\fboxsep]{
  		\begin{minipage}[t]{\linewidth-2\fboxrule-4\fboxsep}\setlength{\parskip}{3mm}
			\raisebox{-2.5mm}{\sffamily \small{\textcolor{#1}{\MakeUppercase{#2}}}}		
			\par		
  			 #3
 	 		\end{minipage}
	}}
		\vspace{2mm}
	\par
}

\newcommand\bloc[3]{				% Boites convertible html sans bordure
     \needspace{2\baselineskip}
     {\sffamily \small{\textcolor{#1}{\MakeUppercase{#2}}}}    
		\par		
  			 #3
		\par
}

\newcommand\CHelp[1]{
     \CBox{Plum}{\faInfoCircle}{À RETENIR}{#1}
}

\newcommand\CUp[1]{
     \CBox{NavyBlue}{\faThumbsOUp}{EN PRATIQUE}{#1}
}

\newcommand\CInfo[1]{
     \CBox{Sepia}{\faArrowCircleRight}{REMARQUE}{#1}
}

\newcommand\CRedac[1]{
     \CBox{PineGreen}{\faEdit}{BIEN R\'EDIGER}{#1}
}

\newcommand\CError[1]{
     \CBox{Red}{\faExclamationTriangle}{ATTENTION}{#1}
}

\newcommand\TitreExo[2]{
\needspace{4\baselineskip}
 {\sffamily\large EXERCICE #1\ (\emph{#2 points})}
\vspace{5mm}
}

\newcommand\img[2]{
          \includegraphics[width=#2\paperwidth]{\imgdir#1}
}

\newcommand\imgsvg[2]{
       \begin{center}   \includegraphics[width=#2\paperwidth]{\imgsvgdir#1} \end{center}
}


\newcommand\Lien[2]{
     \href{#1}{#2 \tiny \faExternalLink}
}
\newcommand\mcLien[2]{
     \href{https~://www.maths-cours.fr/#1}{#2 \tiny \faExternalLink}
}

\newcommand{\euro}{\eurologo{}}

%================================================================================================================================
%
% Macros - Environement
%
%================================================================================================================================

\newenvironment{tex}{ %
}
{%
}

\newenvironment{indente}{ %
	\setlength\parindent{10mm}
}

{
	\setlength\parindent{0mm}
}

\newenvironment{corrige}{%
     \needspace{3\baselineskip}
     \medskip
     \textbf{\textsc{Corrigé}}
     \medskip
}
{
}

\newenvironment{extern}{%
     \begin{center}
     }
     {
     \end{center}
}

\NewEnviron{code}{%
	\par
     \boite{gray}{\texttt{%
     \BODY
     }}
     \par
}

\newenvironment{vbloc}{% boite sans cadre empeche saut de page
     \begin{minipage}[t]{\linewidth}
     }
     {
     \end{minipage}
}
\NewEnviron{h2}{%
    \needspace{3\baselineskip}
    \vspace{0.6cm}
	\noindent \MakeUppercase{\sffamily \large \BODY}
	\vspace{1mm}\textcolor{mcgris}{\hrule}\vspace{0.4cm}
	\par
}{}

\NewEnviron{h3}{%
    \needspace{3\baselineskip}
	\vspace{5mm}
	\textsc{\BODY}
	\par
}

\NewEnviron{margeneg}{ %
\begin{addmargin}[-1cm]{0cm}
\BODY
\end{addmargin}
}

\NewEnviron{html}{%
}

\begin{document}
\\
L'un des intérêts de la programmation est de pouvoir faire exécuter facilement à une machine des tâches \textbf{répétitives}.
\newpar
Le langage Python propose deux instructions~: « for » et « while » qui permettent de répéter automatiquement l'exécution de certains blocs de code.
\id{h05}\begin{h2}1. Les boucles \og for \fg{} (Boucles bornées) \end{h2}
Une boucle \og for \fg{} (ou boucle \og pour \fg{} ou boucle bornée) est généralement utilisée lorsque l'on connaît le nombre de répétitions que l'on souhaite exécuter.
\newpar
La syntaxe de cette instruction est la suivante~:
\begin{lstlisting}[language=Python]
for variable in [liste de valeurs] :
   # bloc d'instructions à répéter
# instructions à exécuter une fois la boucle terminée
\end{lstlisting}
Ce programme se déroule de la manière suivante~:
\begin{itemize}
     \item
     les \og instructions à répéter \fg{} sont exécutées en donnant à la \og variable \fg{} chacune des valeurs de la \og liste de valeurs \fg{}~; c''est l'indentation (écriture décalée vers la droite) qui détermine la taille du bloc d'instructions à répéter
     \item
     une fois que tous les items de la liste ont été parcourus, le programme passe au \og instructions à exécuter une fois la boucle terminée \fg{}.
\end{itemize}
Par exemple, le programme Python ci-dessous ~:
\begin{lstlisting}[language=Python]
for i in [1, 2, 5, 11] :
   j = i**2 # calcul du carré de i
   print(j, end=' - ') # affichage~; on sépare les valeurs par des tirets 
print("fin") 
\end{lstlisting}
affichera~:\\
1 - 4 - 25 - 121 - fin\\
ce qui correspond aux carrés des nombres de la liste.
\newpar
\textbf{Remarque~: } on aurait pu faire l'économie de la variable j, en écrivant plus simplement \og print(i**2, end=' - ') \fg{} mais le but, ici, était de montrer qu'un bloc pouvait comporter plusieurs lignes.
\newpar
Avec l'instruction \texttt{for} on utilise fréquemment la fonction \texttt{range}~;
en effet, \texttt{range(a,b)} renvoie la liste des entiers compris (au sens large) entre \texttt{a} et \texttt{b - 1}.
\cadre{rouge}{Attention}{ % id=a010
     L'instruction \texttt{range(a,b)} créer une liste qui s'arrête à l'entier \texttt{b -1} et non à l'entier \texttt{b}~!
}% fin théorème
\newpar
Par exemple, le programme suivant affiche les doubles des nombres entiers compris entre 3 et 5~:\\
\begin{lstlisting}[language=Python]
for i in range(3, 6) :
   print(2*i, end=' - ') # affiche 6 - 8 - 10 -
\end{lstlisting}
\textbf{Remarque~:} si l'on utilise l'instruction \texttt{range} avec un seul paramètre \texttt{b}, celle-ci retournera la liste des entiers compris entre 0 et \texttt{b - 1}~:
\begin{lstlisting}[language=Python]
for i in range(4):
   print(i, end=' - ') # affiche 0 - 1 - 2 - 3 -
\end{lstlisting}

\id{h10}\begin{h2}2. Les boucles « while » (Boucles non bornées) \end{h2}
On utilise une boucle \texttt{while} (ou boucle \og Tant que \fg{} ou boucle non bornée) lorsque l'on doit répéter l'exécution d'un bloc d'instructions \textbf{tant qu'une condition est vérifiée} (mais en général, on ne sait pas au préalable le nombre de répétitions que l'on devra effectuer). Là encore, c'est l'indentation qui détermine la fin du bloc d'instructions à répéter.
\newpar
La syntaxe de l'instruction « while » est~:
\begin{lstlisting}[language=Python]
while condition :
   #bloc d'instructions à répéter 
#instructions à exécuter une fois la boucle terminée
\end{lstlisting}
Le programme se déroule alors de la façon suivante~:
\begin{itemize}
     \item
     tant que la \og condition \fg{} de la ligne 1. est vraie, les \og instructions à répéter \fg{} de la ligne 2. sont exécutées \item
     dès que la \og condition \fg{} de la ligne 1. devient fausse, le programme passe aux \og instructions à exécuter une fois la boucle terminée \fg{} (ligne 3.).
\end{itemize}
Par exemple, le programme ci-dessous affiche la plus petite puissance de 2 qui est supérieure ou égale à 1~000~:
\begin{lstlisting}[language=Python]
i = 1 # initialisation de i
while i < 1000 :
   i = i * 2
print(i) # affiche 1024
\end{lstlisting}
À chaque passage à la ligne 3., on multiplie la valeur précédente de de \texttt{i} par 2. On obtient ainsi les puissances successives de 2.
\newpar
Pour bien comprendre comment fonctionne ce programme, il peut être utile de représenter les différentes valeurs de \texttt{i} et de la condition \texttt{i < 1000} dans un tableau~:
\begin{center}
     \begin{tabular}{|c|c|c|c|c|c|c|c|c|c|c|c|c|c|c|}%class="compact" width="600"
          \hline
          $i$ & 1 & 2 & 4 & 8 & 16 & 32 & 64 & 128 & 256 & 512 & 1024
          \\ \hline
          $ i < 1000 $ & vrai & vrai & vrai & vrai & vrai & vrai & vrai & vrai & vrai & vrai & faux
          \\ \hline
     \end{tabular}
\end{center}
L'utilisation de l'instruction \texttt{while} peut être assez délicate~; si, suite à une erreur de programmation, la condition figurant dans le \texttt{while} est toujours vérifiée, le programme ne sortira jamais de la boucle et tournera indéfiniment (sauf si un intervenant extérieur met fin à son exécution...)
\newpar
En particulier, il faut faire attention aux points suivants~:
\begin{itemize}
     \item
     penser à initialiser les variables avant l'instruction \texttt{while}~: dans notre exemple, python provoquerait une erreur si la ligne 1. (initialisation de i) était manquante.
     \item
     la condition figurant après l'instruction \texttt{while} est celle qui permet de \textbf{rester} dans la boucle~; c'est donc le \textbf{contraire} de la condition de \textbf{sortie} de boucle. Dans l'exemple précédent, on souhaitait \textbf{sortir} de la boucle lorsque la valeur de \texttt{i} était \textbf{supérieure ou égale à 1000}~; il fallait donc coder \texttt{i < 1000} à l'intérieur de l'instruction \texttt{while}.
\end{itemize}

\end{document}