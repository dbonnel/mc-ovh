\documentclass[a4paper]{article}

%================================================================================================================================
%
% Packages
%
%================================================================================================================================

\usepackage[T1]{fontenc} 	% pour caractères accentués
\usepackage[utf8]{inputenc}  % encodage utf8
\usepackage[french]{babel}	% langue : français
\usepackage{fourier}			% caractères plus lisibles
\usepackage[dvipsnames]{xcolor} % couleurs
\usepackage{fancyhdr}		% réglage header footer
\usepackage{needspace}		% empêcher sauts de page mal placés
\usepackage{graphicx}		% pour inclure des graphiques
\usepackage{enumitem,cprotect}		% personnalise les listes d'items (nécessaire pour ol, al ...)
\usepackage{hyperref}		% Liens hypertexte
\usepackage{pstricks,pst-all,pst-node,pstricks-add,pst-math,pst-plot,pst-tree,pst-eucl} % pstricks
\usepackage[a4paper,includeheadfoot,top=2cm,left=3cm, bottom=2cm,right=3cm]{geometry} % marges etc.
\usepackage{comment}			% commentaires multilignes
\usepackage{amsmath,environ} % maths (matrices, etc.)
\usepackage{amssymb,makeidx}
\usepackage{bm}				% bold maths
\usepackage{tabularx}		% tableaux
\usepackage{colortbl}		% tableaux en couleur
\usepackage{fontawesome}		% Fontawesome
\usepackage{environ}			% environment with command
\usepackage{fp}				% calculs pour ps-tricks
\usepackage{multido}			% pour ps tricks
\usepackage[np]{numprint}	% formattage nombre
\usepackage{tikz,tkz-tab} 			% package principal TikZ
\usepackage{pgfplots}   % axes
\usepackage{mathrsfs}    % cursives
\usepackage{calc}			% calcul taille boites
\usepackage[scaled=0.875]{helvet} % font sans serif
\usepackage{svg} % svg
\usepackage{scrextend} % local margin
\usepackage{scratch} %scratch
\usepackage{multicol} % colonnes
%\usepackage{infix-RPN,pst-func} % formule en notation polanaise inversée
\usepackage{listings}

%================================================================================================================================
%
% Réglages de base
%
%================================================================================================================================

\lstset{
language=Python,   % R code
literate=
{á}{{\'a}}1
{à}{{\`a}}1
{ã}{{\~a}}1
{é}{{\'e}}1
{è}{{\`e}}1
{ê}{{\^e}}1
{í}{{\'i}}1
{ó}{{\'o}}1
{õ}{{\~o}}1
{ú}{{\'u}}1
{ü}{{\"u}}1
{ç}{{\c{c}}}1
{~}{{ }}1
}


\definecolor{codegreen}{rgb}{0,0.6,0}
\definecolor{codegray}{rgb}{0.5,0.5,0.5}
\definecolor{codepurple}{rgb}{0.58,0,0.82}
\definecolor{backcolour}{rgb}{0.95,0.95,0.92}

\lstdefinestyle{mystyle}{
    backgroundcolor=\color{backcolour},   
    commentstyle=\color{codegreen},
    keywordstyle=\color{magenta},
    numberstyle=\tiny\color{codegray},
    stringstyle=\color{codepurple},
    basicstyle=\ttfamily\footnotesize,
    breakatwhitespace=false,         
    breaklines=true,                 
    captionpos=b,                    
    keepspaces=true,                 
    numbers=left,                    
xleftmargin=2em,
framexleftmargin=2em,            
    showspaces=false,                
    showstringspaces=false,
    showtabs=false,                  
    tabsize=2,
    upquote=true
}

\lstset{style=mystyle}


\lstset{style=mystyle}
\newcommand{\imgdir}{C:/laragon/www/newmc/assets/imgsvg/}
\newcommand{\imgsvgdir}{C:/laragon/www/newmc/assets/imgsvg/}

\definecolor{mcgris}{RGB}{220, 220, 220}% ancien~; pour compatibilité
\definecolor{mcbleu}{RGB}{52, 152, 219}
\definecolor{mcvert}{RGB}{125, 194, 70}
\definecolor{mcmauve}{RGB}{154, 0, 215}
\definecolor{mcorange}{RGB}{255, 96, 0}
\definecolor{mcturquoise}{RGB}{0, 153, 153}
\definecolor{mcrouge}{RGB}{255, 0, 0}
\definecolor{mclightvert}{RGB}{205, 234, 190}

\definecolor{gris}{RGB}{220, 220, 220}
\definecolor{bleu}{RGB}{52, 152, 219}
\definecolor{vert}{RGB}{125, 194, 70}
\definecolor{mauve}{RGB}{154, 0, 215}
\definecolor{orange}{RGB}{255, 96, 0}
\definecolor{turquoise}{RGB}{0, 153, 153}
\definecolor{rouge}{RGB}{255, 0, 0}
\definecolor{lightvert}{RGB}{205, 234, 190}
\setitemize[0]{label=\color{lightvert}  $\bullet$}

\pagestyle{fancy}
\renewcommand{\headrulewidth}{0.2pt}
\fancyhead[L]{maths-cours.fr}
\fancyhead[R]{\thepage}
\renewcommand{\footrulewidth}{0.2pt}
\fancyfoot[C]{}

\newcolumntype{C}{>{\centering\arraybackslash}X}
\newcolumntype{s}{>{\hsize=.35\hsize\arraybackslash}X}

\setlength{\parindent}{0pt}		 
\setlength{\parskip}{3mm}
\setlength{\headheight}{1cm}

\def\ebook{ebook}
\def\book{book}
\def\web{web}
\def\type{web}

\newcommand{\vect}[1]{\overrightarrow{\,\mathstrut#1\,}}

\def\Oij{$\left(\text{O}~;~\vect{\imath},~\vect{\jmath}\right)$}
\def\Oijk{$\left(\text{O}~;~\vect{\imath},~\vect{\jmath},~\vect{k}\right)$}
\def\Ouv{$\left(\text{O}~;~\vect{u},~\vect{v}\right)$}

\hypersetup{breaklinks=true, colorlinks = true, linkcolor = OliveGreen, urlcolor = OliveGreen, citecolor = OliveGreen, pdfauthor={Didier BONNEL - https://www.maths-cours.fr} } % supprime les bordures autour des liens

\renewcommand{\arg}[0]{\text{arg}}

\everymath{\displaystyle}

%================================================================================================================================
%
% Macros - Commandes
%
%================================================================================================================================

\newcommand\meta[2]{    			% Utilisé pour créer le post HTML.
	\def\titre{titre}
	\def\url{url}
	\def\arg{#1}
	\ifx\titre\arg
		\newcommand\maintitle{#2}
		\fancyhead[L]{#2}
		{\Large\sffamily \MakeUppercase{#2}}
		\vspace{1mm}\textcolor{mcvert}{\hrule}
	\fi 
	\ifx\url\arg
		\fancyfoot[L]{\href{https://www.maths-cours.fr#2}{\black \footnotesize{https://www.maths-cours.fr#2}}}
	\fi 
}


\newcommand\TitreC[1]{    		% Titre centré
     \needspace{3\baselineskip}
     \begin{center}\textbf{#1}\end{center}
}

\newcommand\newpar{    		% paragraphe
     \par
}

\newcommand\nosp {    		% commande vide (pas d'espace)
}
\newcommand{\id}[1]{} %ignore

\newcommand\boite[2]{				% Boite simple sans titre
	\vspace{5mm}
	\setlength{\fboxrule}{0.2mm}
	\setlength{\fboxsep}{5mm}	
	\fcolorbox{#1}{#1!3}{\makebox[\linewidth-2\fboxrule-2\fboxsep]{
  		\begin{minipage}[t]{\linewidth-2\fboxrule-4\fboxsep}\setlength{\parskip}{3mm}
  			 #2
  		\end{minipage}
	}}
	\vspace{5mm}
}

\newcommand\CBox[4]{				% Boites
	\vspace{5mm}
	\setlength{\fboxrule}{0.2mm}
	\setlength{\fboxsep}{5mm}
	
	\fcolorbox{#1}{#1!3}{\makebox[\linewidth-2\fboxrule-2\fboxsep]{
		\begin{minipage}[t]{1cm}\setlength{\parskip}{3mm}
	  		\textcolor{#1}{\LARGE{#2}}    
 	 	\end{minipage}  
  		\begin{minipage}[t]{\linewidth-2\fboxrule-4\fboxsep}\setlength{\parskip}{3mm}
			\raisebox{1.2mm}{\normalsize\sffamily{\textcolor{#1}{#3}}}						
  			 #4
  		\end{minipage}
	}}
	\vspace{5mm}
}

\newcommand\cadre[3]{				% Boites convertible html
	\par
	\vspace{2mm}
	\setlength{\fboxrule}{0.1mm}
	\setlength{\fboxsep}{5mm}
	\fcolorbox{#1}{white}{\makebox[\linewidth-2\fboxrule-2\fboxsep]{
  		\begin{minipage}[t]{\linewidth-2\fboxrule-4\fboxsep}\setlength{\parskip}{3mm}
			\raisebox{-2.5mm}{\sffamily \small{\textcolor{#1}{\MakeUppercase{#2}}}}		
			\par		
  			 #3
 	 		\end{minipage}
	}}
		\vspace{2mm}
	\par
}

\newcommand\bloc[3]{				% Boites convertible html sans bordure
     \needspace{2\baselineskip}
     {\sffamily \small{\textcolor{#1}{\MakeUppercase{#2}}}}    
		\par		
  			 #3
		\par
}

\newcommand\CHelp[1]{
     \CBox{Plum}{\faInfoCircle}{À RETENIR}{#1}
}

\newcommand\CUp[1]{
     \CBox{NavyBlue}{\faThumbsOUp}{EN PRATIQUE}{#1}
}

\newcommand\CInfo[1]{
     \CBox{Sepia}{\faArrowCircleRight}{REMARQUE}{#1}
}

\newcommand\CRedac[1]{
     \CBox{PineGreen}{\faEdit}{BIEN R\'EDIGER}{#1}
}

\newcommand\CError[1]{
     \CBox{Red}{\faExclamationTriangle}{ATTENTION}{#1}
}

\newcommand\TitreExo[2]{
\needspace{4\baselineskip}
 {\sffamily\large EXERCICE #1\ (\emph{#2 points})}
\vspace{5mm}
}

\newcommand\img[2]{
          \includegraphics[width=#2\paperwidth]{\imgdir#1}
}

\newcommand\imgsvg[2]{
       \begin{center}   \includegraphics[width=#2\paperwidth]{\imgsvgdir#1} \end{center}
}


\newcommand\Lien[2]{
     \href{#1}{#2 \tiny \faExternalLink}
}
\newcommand\mcLien[2]{
     \href{https~://www.maths-cours.fr/#1}{#2 \tiny \faExternalLink}
}

\newcommand{\euro}{\eurologo{}}

%================================================================================================================================
%
% Macros - Environement
%
%================================================================================================================================

\newenvironment{tex}{ %
}
{%
}

\newenvironment{indente}{ %
	\setlength\parindent{10mm}
}

{
	\setlength\parindent{0mm}
}

\newenvironment{corrige}{%
     \needspace{3\baselineskip}
     \medskip
     \textbf{\textsc{Corrigé}}
     \medskip
}
{
}

\newenvironment{extern}{%
     \begin{center}
     }
     {
     \end{center}
}

\NewEnviron{code}{%
	\par
     \boite{gray}{\texttt{%
     \BODY
     }}
     \par
}

\newenvironment{vbloc}{% boite sans cadre empeche saut de page
     \begin{minipage}[t]{\linewidth}
     }
     {
     \end{minipage}
}
\NewEnviron{h2}{%
    \needspace{3\baselineskip}
    \vspace{0.6cm}
	\noindent \MakeUppercase{\sffamily \large \BODY}
	\vspace{1mm}\textcolor{mcgris}{\hrule}\vspace{0.4cm}
	\par
}{}

\NewEnviron{h3}{%
    \needspace{3\baselineskip}
	\vspace{5mm}
	\textsc{\BODY}
	\par
}

\NewEnviron{margeneg}{ %
\begin{addmargin}[-1cm]{0cm}
\BODY
\end{addmargin}
}

\NewEnviron{html}{%
}

\begin{document}
\cadre{vert}{Introduction}{% id="i10"
     Il arrive qu'une variable aléatoire puisse prendre n'importe quelle valeur sur $\mathbb{R}$ ou sur un intervalle $I$ de $\mathbb{R}$. Dans ce cas, $X\left(\Omega \right)=I$.
     \par
     Une telle variable est dite variable aléatoire réelle continue.
     \par
     Pour une telle variable, les événements intéressants ne sont plus $X=5$ , $X=20$ , etc... , mais $X \leqslant  5$ , $5 \leqslant  X \leqslant  20$  etc...
}
\begin{h2}1. Généralités\end{h2}
\cadre{bleu}{Définition}{% id="d20"
     Une variable aléatoire réelle \textbf{continue} est une application définie sur un ensemble $\Omega $ et prenant toutes les valeurs d'un intervalle $I$ de $\mathbb{R}$.
}
\bloc{orange}{Exemple}{% id="e20"
     C'est le cas par exemple d'une variable aléatoire qui mesure la durée de vie d'un organisme ou la taille d'une personne.
     \par
     Plus généralement, les variables qui résultent d'une mesure peuvent généralement être considérées comme des variables aléatoires continues.
}
\cadre{bleu}{Définition}{% id="d30"
     Soit $f$ une fonction \textbf{continue} et \textbf{positive} sur un intervalle $I=\left[a;b\right]$ telle que
     \par
     $\int_{a}^{b}f\left(x\right)dx=1$
     \par
     On dit que X est une variable aléatoire réelle continue de \textbf{densité} $f$ si et seulement si pour tout $x_{1} \in  I$ et tout $x_{2} \in  I $ ($x_{1}\leqslant x_{2}$) :
     \par
     $P\left(x_{1}\leqslant X\leqslant x_{2}\right)=\int_{x_{1}}^{x_{2}}f\left(x\right)dx$
}
\bloc{orange}{Exemple}{% id="e30"
     La fonction $f$ définie sur $I=\left[0;2\right]$ par $f\left(x\right)=\frac{x}{2}$ est une fonction continue et positive sur $I$ et :
     \par
     $\int_{0}^{2}f\left(x\right)dx=\left[\frac{x^{2}}{4}\right]_{0}^{2}=1$
     \par
     Si X est la variable aléatoire réelle à valeur dans $I$ de \textbf{densité} $f$ on a, par exemple :
     \par
     $P\left(1\leqslant X\leqslant 1,5\right)=\int_{1}^{1,5}f\left(x\right)dx$
     \par
     donc $P\left(1\leqslant X\leqslant 1,5\right)$ est l'aire (en u.a.) colorée ci-dessous.
      \begin{center}
          \begin{extern} %width="250" alt="densité linéaire"
               \begin{pspicture}(0,-0.5)(5,3)
                    \psset{xunit=2 cm, yunit=2 cm, algebraic=true}
                    %\psgrid[gridcolor=mcgris, subgriddiv=0, gridlabels=0pt](0,0)(2,1)
                    \psaxes{->}(0,0)(0,0)(2.5,1.5)
                    \psplot[linecolor=red,linewidth=0.75pt]{0}{2}{x/2}
                    \pscustom[linecolor=mcvert,linewidth=0.75pt,fillstyle=solid,fillcolor=mcvert,opacity=0.1]{
                         \psplot{1}{1.5}{x/2}
                         \psline(1.5,0)(1,0)(1,0.5)
                    }
               \end{pspicture}
          \end{extern}
     \end{center}
     Un calcul simple montre que $P\left(1\leqslant X\leqslant 1,5\right)=\left[\frac{x^{2}}{4}\right]_{1}^{1,5}=0,3125$
}
\cadre{bleu}{Définition}{% id="d40"
     Soit $X$ une variable aléatoire de densité $f$ sur l'intervalle $\left[a;b\right]$.
     \par
     On appelle \textbf{espérance mathématique} de $X$ le nombre :
     \begin{center}$E\left(X\right)=\int_{a}^{b}xf\left(x\right)dx$\end{center}
}
\bloc{cyan}{Remarque}{% id="r40"
     Comme dans le cas d'une variable aléatoire discrète, l'espérance mathématique représente la valeur moyenne prise par la variable aléatoire $X$.
}
\begin{h2}2. Loi uniforme sur un intervalle\end{h2}
\cadre{bleu}{Définition}{% id="d50"
     On dit qu'une variable aléatoire $X$ suit une \textbf{loi uniforme} sur $\left[a;b\right]$ si sa densité de probabilité $f$ est définie sur $\left[a;b\right]$ par $f\left(x\right)=\frac{1}{b-a}$
}
     \begin{center}
          \begin{extern} %width="250" alt="loi uniforme"
               \begin{pspicture}(0,-0.5)(5,3)
                    \psset{xunit=2 cm, yunit=2 cm, algebraic=true}
                    %\psgrid[gridcolor=mcgris, subgriddiv=0, gridlabels=0pt](0,0)(2,1)
                    \psaxes{->}(0,0)(0,0)(2.5,1.5)
                    \psplot[linecolor=red,linewidth=0.75pt]{0}{2}{1/2}
               \end{pspicture}
          \end{extern}
     \end{center}
     \begin{center}\textit{Densité de la loi uniforme sur l'intervalle $\left[0, 2\right]$}\end{center}\bloc{vert}{Remarque}{% id="r50"
     On vérifie facilement que $\int_{a}^{b} \frac{1}{b-a} dx=\left[\frac{x}{b-a}\right]_{a}^{b}=1$
}
\cadre{vert}{Propriété}{% id="p60"
     Si $X$ suit une loi uniforme sur $\left[a;b\right]$ et si $c$ et $d$ sont deux réels tels que $a\leqslant c\leqslant d\leqslant b$, alors :
     \begin{center}$P\left(c\leqslant X\leqslant d\right) = \frac{d-c}{b-a}$\end{center}
}
\bloc{cyan}{Démonstration}{% id="r60"
     $P\left(c\leqslant X\leqslant d\right)=\int_{c}^{d} \frac{1}{b-a} dx=\left[\frac{x}{b-a}\right]_{c}^{d}=\frac{d}{b-a}-\frac{c}{b-a}=\frac{d-c}{b-a}$
}
\cadre{vert}{Propriété}{% id="p70"
     L'espérance mathématique d'une loi uniforme sur $\left[a;b\right]$ est :
     \begin{center}$E\left(X\right)=\frac{a+b}{2}$\end{center}
}
\begin{h2}3. Loi normale centrée réduite\end{h2}
\cadre{bleu}{Définition}{% id="d110"
     Une variable aléatoire $X$ suit la \textbf{loi normale centrée réduite} sur $\mathbb{R}$ (notée $\mathscr N \left(0;1\right)$) si sa densité de probabilité $f$ est définie par :
     \begin{center}$f\left(x\right)=\frac{1}{\sqrt{2\pi }}e^{^{-\frac{x^{2}}{2}}}$\end{center}
}
\bloc{cyan}{Remarques}{% id="r110"
     \begin{itemize}
          \item On admet que $f$ définit bien une densité, c'est à dire que l'aire comprise entre l'axe des abscisses et la courbe représentative de $f$ est égale à  1
          \item La fonction $f : x\mapsto \frac{1}{\sqrt{2\pi }}e^{^{-\frac{x^{2}}{2}}}$ est dérivable et positive sur $\mathbb{R}$ sa courbe représentative
          \par
          est symétrique par rapport à l'axe des ordonnées :
                    \begin{center}
               \begin{extern} %width="550" alt="loi normale centrée réduite"
                    % -+-+-+ variables modifiables
                    \def\e{2.7182818}
                    \def\pi{3.1415926}
                    \def\fonction{1/sqrt(2*\pi)*\e^(-x*x/2)  }
                    \def\xmin{-3.5}
                    \def\xmax{3.5}
                    \def\ymin{0}
                    \def\ymax{0.5}
                    \def\xunit{2}  % unités en cm
                    \def\yunit{12}
                    \psset{xunit=\xunit,yunit=\yunit,algebraic=true}
                    \begin{pspicture*}(\xmin,-0.08)(\xmax,\ymax)
                         %      \psgrid[gridcolor=mcgris, subgriddiv=5, gridlabels=0pt](-5,-0.3)(5,1)
                         \psaxes[Dy=0.1,linewidth=0.75pt]{->}(0,0)(\xmin,-0.08)(\xmax,\ymax)
                         \psplot[plotpoints=2000,linecolor=red,linewidth=0.75pt]{\xmin}{\xmax}{\fonction}
                         \rput[br](-0.1,-0.045){$O$}
                    \end{pspicture*}
               \end{extern}
          \end{center}
          \item $p\left(a\leqslant X\leqslant b\right)$ est l'aire du domaine coloré ci-dessous :
          \begin{center}
               \begin{extern} %width="550" alt="loi normale probabilité"
                    % -+-+-+ variables modifiables
                    \def\e{2.7182818}
                    \def\pi{3.1415926}
                    \def\fonction{1/sqrt(2*\pi)*\e^(-x*x/2)  }
                    \def\xmin{-3.5}
                    \def\xmax{3.5}
                    \def\ymin{0}
                    \def\ymax{0.5}
                    \def\xunit{2}  % unités en cm
                    \def\yunit{12}
                    \psset{xunit=\xunit,yunit=\yunit,algebraic=true}
                    \begin{pspicture*}(\xmin,-0.08)(\xmax,\ymax)
                         %      \psgrid[gridcolor=mcgris, subgriddiv=5, gridlabels=0pt](-5,-0.3)(5,1)
                         \psaxes[Dy=0.1,linewidth=0.75pt]{->}(0,0)(\xmin,-0.08)(\xmax,\ymax)
                         \psplot[plotpoints=2000,linecolor=red,linewidth=0.75pt]{\xmin}{\xmax}{\fonction}
                         \pscustom[fillcolor=blue,fillstyle=solid,opacity=0.2,linecolor=blue]{
                              \moveto(1.14,0)
                              \psplot[plotpoints=3000,linewidth=0.75pt]{1.14}{2.24}{\fonction}
                              \psline(2.24,0)(1.14,0)
                         }
                         \rput[b](1.14,-0.045){$\color{blue} a$}
                         \rput[b](2.24,-0.045){$\color{blue} b$}
                         \rput[br](-0.1,-0.045){$O$}
                    \end{pspicture*}
               \end{extern}
          \end{center}
          \item $p\left(X\leqslant a\right)$ est l'aire du domaine coloré ci-dessous :
          \begin{center}
               \begin{extern} %width="550" alt="loi normale probabilité"
                    % -+-+-+ variables modifiables
                    \def\e{2.7182818}
                    \def\pi{3.1415926}
                    \def\fonction{1/sqrt(2*\pi)*\e^(-x*x/2)  }
                    \def\xmin{-3.5}
                    \def\xmax{3.5}
                    \def\ymin{0}
                    \def\ymax{0.5}
                    \def\xunit{2}  % unités en cm
                    \def\yunit{12}
                    \psset{xunit=\xunit,yunit=\yunit,algebraic=true}
                    \begin{pspicture*}(\xmin,-0.08)(\xmax,\ymax)
                         %      \psgrid[gridcolor=mcgris, subgriddiv=5, gridlabels=0pt](-5,-0.3)(5,1)
                         \psaxes[Dy=0.1,linewidth=0.75pt]{->}(0,0)(\xmin,-0.08)(\xmax,\ymax)
                         \psplot[plotpoints=2000,linecolor=red,linewidth=0.75pt]{\xmin}{\xmax}{\fonction}
                         \pscustom[fillcolor=blue,fillstyle=solid,opacity=0.2,linecolor=blue]{
                              \psplot[plotpoints=3000,linewidth=0.75pt]{-4}{1.14}{\fonction}
                              \psline(1.14,0)(-4,0)
                         }
                         \rput[b](1.14,-0.045){$\color{blue} a$}
                         \rput[br](-0.1,-0.045){$O$}
                    \end{pspicture*}
               \end{extern}
          \end{center}
          \item Il n'est pas possible d'exprimer les primitives de la fonction $f : x\mapsto \frac{1}{\sqrt{2\pi }}e^{^{-\frac{x^{2}}{2}}}$ à l'aide des fonctions usuelles. Toutefois, les calculatrices scientifiques permettent de calculer directement $p\left(a\leqslant X\leqslant b\right)$ pour une loi normale.
     \end{itemize}
}
\cadre{vert}{Propriétés}{% id="p130"
     Soit $X$ une variable aléatoire qui suit la loi normale centrée réduite et $\alpha $ un réel quelconque :
     \begin{itemize}
          \item $p\left(X\leqslant 0\right)=p\left(X\geqslant 0\right)=0,5$
          \item $p\left(X\leqslant -a\right)=p\left(X\geqslant a\right)$
          \item $p\left(-\alpha \leqslant X\leqslant a\right)=1-2\times p\left(X\geqslant a\right)=2\times p\left(X\leqslant a\right)-1$
     \end{itemize}
}
\bloc{cyan}{Remarque}{% id="r130"
     Ces propriétés résultent du fait que :
     \begin{itemize}
     \item
     la courbe de la fonction $x\mapsto \frac{1}{\sqrt{2\pi }}e^{^{-\frac{x^{2}}{2}}}$ est symétrique par rapport à l'axe des ordonnées
     \item
     l'aire comprise entre l'axe des abscisses et la courbe est égale à  1.
     \end{itemize}
          On retrouve facilement ces propriétés à l'aide d'une figure par exemple pour la seconde formule :
       \begin{center}
          \begin{extern} %width="550" alt="loi normale symétrie"
               % -+-+-+ variables modifiables
               \def\e{2.7182818}
               \def\pi{3.1415926}
               \def\fonction{1/sqrt(2*\pi)*\e^(-x*x/2)  }
               \def\xmin{-3.5}
               \def\xmax{3.5}
               \def\ymin{0}
               \def\ymax{0.5}
               \def\xunit{2}  % unités en cm
               \def\yunit{12}
               \psset{xunit=\xunit,yunit=\yunit,algebraic=true}
               \begin{pspicture*}(\xmin,-0.08)(\xmax,\ymax)
                    %      \psgrid[gridcolor=mcgris, subgriddiv=5, gridlabels=0pt](-5,-0.3)(5,1)
                    \psaxes[Dy=0.1,linewidth=0.75pt]{->}(0,0)(\xmin,-0.08)(\xmax,\ymax)
                    \psplot[plotpoints=2000,linecolor=red,linewidth=0.75pt]{\xmin}{\xmax}{\fonction}
                    \pscustom[fillcolor=vert,fillstyle=solid,opacity=0.2,linecolor=vert]{
                         \psplot[plotpoints=3000,linewidth=0.75pt]{-4}{-1.14}{\fonction}
                         \psline(-1.14,0)(-4,0)
                    }
                    \pscustom[fillcolor=blue,fillstyle=solid,opacity=0.2,linecolor=blue]{
                         \moveto(1.14,0)
                         \psplot[plotpoints=3000,linewidth=0.75pt]{1.14}{4}{\fonction}
                         \psline(4,0)(1.14,0)
                    }
                    \rput[br](-1.14,-0.045){$\color{vert} -a$}
                    \rput[b](1.14,-0.045){$\color{blue} a$}
                    \rput[br](-0.1,-0.045){$O$}
               \end{pspicture*}
          \end{extern}
     \end{center}
     \begin{center}$p\left(X\leqslant -a\right)=p\left(X\geqslant a\right)$\end{center}
}
\begin{h2}4. Loi normale quelconque\end{h2}
\cadre{bleu}{Définition et théorème}{% id="d170"
     Soient deux réels $\mu $ et $\sigma  > 0$.
     \par
     On dit qu'une variable aléatoire $X$ suit une \textbf{loi normale de paramètres $\mu $ et $\sigma ^{2}$} (notée $\mathscr N \left(\mu  ; \sigma ^{2}\right)$) si la variable aléatoire $Y=\frac{X-\mu }{\sigma} $ suit la loi normale centrée réduite.
     \par
     L'espérance mathématique de $X$ est $\mu $ et son écart-type $\sigma $ (et donc sa variance $\sigma ^{2}$)
}
\bloc{cyan}{Remarque}{% id="r170"
     La courbe représentative de la distribution d'une loi $\mathscr N \left(\mu  ; \sigma ^{2}\right)$ est une courbe «en cloche» qui admet la droite d'équation $x=\mu $ comme axe de symétrie. Elle est plus ou moins «étirée» selon les valeurs de $\sigma $
      \begin{center}
          \begin{extern} %width="500" alt="loi normale différents écarts-types"
               % -+-+-+ variables modifiables
               \def\e{2.7182818}
               \def\pi{3.1415926}
               \def\f{2/sqrt(2*\pi)*\e^(-(x-3)*(x-3)/0.5)  }
               \def\g{1/sqrt(2*\pi)*\e^(-(x-3)*(x-3)/2)  }
               \def\h{0.5/sqrt(2*\pi)*\e^(-(x-3)*(x-3)/8)  }
               \def\xmin{-2.3}
               \def\xmax{8.3}
               \def\ymin{0}
               \def\ymax{0.85}
               \def\xunit{1}  % unités en cm
               \def\yunit{10}
               \psset{xunit=\xunit,yunit=\yunit,algebraic=true}
               \begin{pspicture*}(\xmin,-0.08)(\xmax,\ymax)
                    %      \psgrid[gridcolor=mcgris, subgriddiv=5, gridlabels=0pt](-5,-0.3)(5,1)
                    \psaxes[Dy=0.1,linewidth=0.75pt]{->}(0,0)(\xmin,-0.08)(\xmax,\ymax)
                    \psplot[plotpoints=2000,linecolor=blue,linewidth=0.75pt]{\xmin}{\xmax}{\f}
                    \psplot[plotpoints=2000,linecolor=red,linewidth=0.75pt]{\xmin}{\xmax}{\g}
                    \psplot[plotpoints=2000,linecolor=vert,linewidth=0.75pt]{\xmin}{\xmax}{\h}
                    \psline[linecolor=mauve,linewidth=0.5pt](3,10)(3,-1)
                    \rput[br](2.9,0.02){$\color{mauve} \mu$}
                    \rput[br](5,0.5){$\color{blue} \sigma=0,5$}
                    \rput[br](5.2,0.3){$\color{red} \sigma=1$}
                    \rput[br](6,0.15){$\color{vert} \sigma=2$}
                    \rput[br](-0.1,-0.045){$O$}
               \end{pspicture*}
          \end{extern}
     \end{center}
     \begin{center}$\mu =3$ et $ \sigma =0,5$ ; $1$ ; $2$\end{center}
 }
\cadre{vert}{Propriété (Règle des un-deux-trois sigmas)}{% id="p180"
     Si $X$ suit une loi normale $\mathscr N \left(\mu ; \sigma ^{2}\right)$ alors :
     \begin{itemize}
          \item %
          $p\left(\mu -\sigma \leqslant X\leqslant \mu + \sigma \right)\approx 0,68$ (à $10^{-2}$ près)
\item %
$p\left(\mu -2\sigma \leqslant X\leqslant \mu + 2\sigma \right)\approx 0,95$ (à $10^{-2}$ près)
\item %
 $p\left(\mu -3\sigma \leqslant X\leqslant \mu + 3\sigma \right)\approx 0,997$ (à $10^{-3}$ près)
     \end{itemize}

}
\bloc{orange}{Exemple}{% id="e180"
     Si $X$ suit une loi normale  $\mathscr N \left(11 ; 3^{2}\right)$ alors :
     \par
     $p\left(5\leqslant X\leqslant 17\right)\approx 0,95$
}

\end{document}