\documentclass[a4paper]{article}

%================================================================================================================================
%
% Packages
%
%================================================================================================================================

\usepackage[T1]{fontenc} 	% pour caractères accentués
\usepackage[utf8]{inputenc}  % encodage utf8
\usepackage[french]{babel}	% langue : français
\usepackage{fourier}			% caractères plus lisibles
\usepackage[dvipsnames]{xcolor} % couleurs
\usepackage{fancyhdr}		% réglage header footer
\usepackage{needspace}		% empêcher sauts de page mal placés
\usepackage{graphicx}		% pour inclure des graphiques
\usepackage{enumitem,cprotect}		% personnalise les listes d'items (nécessaire pour ol, al ...)
\usepackage{hyperref}		% Liens hypertexte
\usepackage{pstricks,pst-all,pst-node,pstricks-add,pst-math,pst-plot,pst-tree,pst-eucl} % pstricks
\usepackage[a4paper,includeheadfoot,top=2cm,left=3cm, bottom=2cm,right=3cm]{geometry} % marges etc.
\usepackage{comment}			% commentaires multilignes
\usepackage{amsmath,environ} % maths (matrices, etc.)
\usepackage{amssymb,makeidx}
\usepackage{bm}				% bold maths
\usepackage{tabularx}		% tableaux
\usepackage{colortbl}		% tableaux en couleur
\usepackage{fontawesome}		% Fontawesome
\usepackage{environ}			% environment with command
\usepackage{fp}				% calculs pour ps-tricks
\usepackage{multido}			% pour ps tricks
\usepackage[np]{numprint}	% formattage nombre
\usepackage{tikz,tkz-tab} 			% package principal TikZ
\usepackage{pgfplots}   % axes
\usepackage{mathrsfs}    % cursives
\usepackage{calc}			% calcul taille boites
\usepackage[scaled=0.875]{helvet} % font sans serif
\usepackage{svg} % svg
\usepackage{scrextend} % local margin
\usepackage{scratch} %scratch
\usepackage{multicol} % colonnes
%\usepackage{infix-RPN,pst-func} % formule en notation polanaise inversée
\usepackage{listings}

%================================================================================================================================
%
% Réglages de base
%
%================================================================================================================================

\lstset{
language=Python,   % R code
literate=
{á}{{\'a}}1
{à}{{\`a}}1
{ã}{{\~a}}1
{é}{{\'e}}1
{è}{{\`e}}1
{ê}{{\^e}}1
{í}{{\'i}}1
{ó}{{\'o}}1
{õ}{{\~o}}1
{ú}{{\'u}}1
{ü}{{\"u}}1
{ç}{{\c{c}}}1
{~}{{ }}1
}


\definecolor{codegreen}{rgb}{0,0.6,0}
\definecolor{codegray}{rgb}{0.5,0.5,0.5}
\definecolor{codepurple}{rgb}{0.58,0,0.82}
\definecolor{backcolour}{rgb}{0.95,0.95,0.92}

\lstdefinestyle{mystyle}{
    backgroundcolor=\color{backcolour},   
    commentstyle=\color{codegreen},
    keywordstyle=\color{magenta},
    numberstyle=\tiny\color{codegray},
    stringstyle=\color{codepurple},
    basicstyle=\ttfamily\footnotesize,
    breakatwhitespace=false,         
    breaklines=true,                 
    captionpos=b,                    
    keepspaces=true,                 
    numbers=left,                    
xleftmargin=2em,
framexleftmargin=2em,            
    showspaces=false,                
    showstringspaces=false,
    showtabs=false,                  
    tabsize=2,
    upquote=true
}

\lstset{style=mystyle}


\lstset{style=mystyle}
\newcommand{\imgdir}{C:/laragon/www/newmc/assets/imgsvg/}
\newcommand{\imgsvgdir}{C:/laragon/www/newmc/assets/imgsvg/}

\definecolor{mcgris}{RGB}{220, 220, 220}% ancien~; pour compatibilité
\definecolor{mcbleu}{RGB}{52, 152, 219}
\definecolor{mcvert}{RGB}{125, 194, 70}
\definecolor{mcmauve}{RGB}{154, 0, 215}
\definecolor{mcorange}{RGB}{255, 96, 0}
\definecolor{mcturquoise}{RGB}{0, 153, 153}
\definecolor{mcrouge}{RGB}{255, 0, 0}
\definecolor{mclightvert}{RGB}{205, 234, 190}

\definecolor{gris}{RGB}{220, 220, 220}
\definecolor{bleu}{RGB}{52, 152, 219}
\definecolor{vert}{RGB}{125, 194, 70}
\definecolor{mauve}{RGB}{154, 0, 215}
\definecolor{orange}{RGB}{255, 96, 0}
\definecolor{turquoise}{RGB}{0, 153, 153}
\definecolor{rouge}{RGB}{255, 0, 0}
\definecolor{lightvert}{RGB}{205, 234, 190}
\setitemize[0]{label=\color{lightvert}  $\bullet$}

\pagestyle{fancy}
\renewcommand{\headrulewidth}{0.2pt}
\fancyhead[L]{maths-cours.fr}
\fancyhead[R]{\thepage}
\renewcommand{\footrulewidth}{0.2pt}
\fancyfoot[C]{}

\newcolumntype{C}{>{\centering\arraybackslash}X}
\newcolumntype{s}{>{\hsize=.35\hsize\arraybackslash}X}

\setlength{\parindent}{0pt}		 
\setlength{\parskip}{3mm}
\setlength{\headheight}{1cm}

\def\ebook{ebook}
\def\book{book}
\def\web{web}
\def\type{web}

\newcommand{\vect}[1]{\overrightarrow{\,\mathstrut#1\,}}

\def\Oij{$\left(\text{O}~;~\vect{\imath},~\vect{\jmath}\right)$}
\def\Oijk{$\left(\text{O}~;~\vect{\imath},~\vect{\jmath},~\vect{k}\right)$}
\def\Ouv{$\left(\text{O}~;~\vect{u},~\vect{v}\right)$}

\hypersetup{breaklinks=true, colorlinks = true, linkcolor = OliveGreen, urlcolor = OliveGreen, citecolor = OliveGreen, pdfauthor={Didier BONNEL - https://www.maths-cours.fr} } % supprime les bordures autour des liens

\renewcommand{\arg}[0]{\text{arg}}

\everymath{\displaystyle}

%================================================================================================================================
%
% Macros - Commandes
%
%================================================================================================================================

\newcommand\meta[2]{    			% Utilisé pour créer le post HTML.
	\def\titre{titre}
	\def\url{url}
	\def\arg{#1}
	\ifx\titre\arg
		\newcommand\maintitle{#2}
		\fancyhead[L]{#2}
		{\Large\sffamily \MakeUppercase{#2}}
		\vspace{1mm}\textcolor{mcvert}{\hrule}
	\fi 
	\ifx\url\arg
		\fancyfoot[L]{\href{https://www.maths-cours.fr#2}{\black \footnotesize{https://www.maths-cours.fr#2}}}
	\fi 
}


\newcommand\TitreC[1]{    		% Titre centré
     \needspace{3\baselineskip}
     \begin{center}\textbf{#1}\end{center}
}

\newcommand\newpar{    		% paragraphe
     \par
}

\newcommand\nosp {    		% commande vide (pas d'espace)
}
\newcommand{\id}[1]{} %ignore

\newcommand\boite[2]{				% Boite simple sans titre
	\vspace{5mm}
	\setlength{\fboxrule}{0.2mm}
	\setlength{\fboxsep}{5mm}	
	\fcolorbox{#1}{#1!3}{\makebox[\linewidth-2\fboxrule-2\fboxsep]{
  		\begin{minipage}[t]{\linewidth-2\fboxrule-4\fboxsep}\setlength{\parskip}{3mm}
  			 #2
  		\end{minipage}
	}}
	\vspace{5mm}
}

\newcommand\CBox[4]{				% Boites
	\vspace{5mm}
	\setlength{\fboxrule}{0.2mm}
	\setlength{\fboxsep}{5mm}
	
	\fcolorbox{#1}{#1!3}{\makebox[\linewidth-2\fboxrule-2\fboxsep]{
		\begin{minipage}[t]{1cm}\setlength{\parskip}{3mm}
	  		\textcolor{#1}{\LARGE{#2}}    
 	 	\end{minipage}  
  		\begin{minipage}[t]{\linewidth-2\fboxrule-4\fboxsep}\setlength{\parskip}{3mm}
			\raisebox{1.2mm}{\normalsize\sffamily{\textcolor{#1}{#3}}}						
  			 #4
  		\end{minipage}
	}}
	\vspace{5mm}
}

\newcommand\cadre[3]{				% Boites convertible html
	\par
	\vspace{2mm}
	\setlength{\fboxrule}{0.1mm}
	\setlength{\fboxsep}{5mm}
	\fcolorbox{#1}{white}{\makebox[\linewidth-2\fboxrule-2\fboxsep]{
  		\begin{minipage}[t]{\linewidth-2\fboxrule-4\fboxsep}\setlength{\parskip}{3mm}
			\raisebox{-2.5mm}{\sffamily \small{\textcolor{#1}{\MakeUppercase{#2}}}}		
			\par		
  			 #3
 	 		\end{minipage}
	}}
		\vspace{2mm}
	\par
}

\newcommand\bloc[3]{				% Boites convertible html sans bordure
     \needspace{2\baselineskip}
     {\sffamily \small{\textcolor{#1}{\MakeUppercase{#2}}}}    
		\par		
  			 #3
		\par
}

\newcommand\CHelp[1]{
     \CBox{Plum}{\faInfoCircle}{À RETENIR}{#1}
}

\newcommand\CUp[1]{
     \CBox{NavyBlue}{\faThumbsOUp}{EN PRATIQUE}{#1}
}

\newcommand\CInfo[1]{
     \CBox{Sepia}{\faArrowCircleRight}{REMARQUE}{#1}
}

\newcommand\CRedac[1]{
     \CBox{PineGreen}{\faEdit}{BIEN R\'EDIGER}{#1}
}

\newcommand\CError[1]{
     \CBox{Red}{\faExclamationTriangle}{ATTENTION}{#1}
}

\newcommand\TitreExo[2]{
\needspace{4\baselineskip}
 {\sffamily\large EXERCICE #1\ (\emph{#2 points})}
\vspace{5mm}
}

\newcommand\img[2]{
          \includegraphics[width=#2\paperwidth]{\imgdir#1}
}

\newcommand\imgsvg[2]{
       \begin{center}   \includegraphics[width=#2\paperwidth]{\imgsvgdir#1} \end{center}
}


\newcommand\Lien[2]{
     \href{#1}{#2 \tiny \faExternalLink}
}
\newcommand\mcLien[2]{
     \href{https~://www.maths-cours.fr/#1}{#2 \tiny \faExternalLink}
}

\newcommand{\euro}{\eurologo{}}

%================================================================================================================================
%
% Macros - Environement
%
%================================================================================================================================

\newenvironment{tex}{ %
}
{%
}

\newenvironment{indente}{ %
	\setlength\parindent{10mm}
}

{
	\setlength\parindent{0mm}
}

\newenvironment{corrige}{%
     \needspace{3\baselineskip}
     \medskip
     \textbf{\textsc{Corrigé}}
     \medskip
}
{
}

\newenvironment{extern}{%
     \begin{center}
     }
     {
     \end{center}
}

\NewEnviron{code}{%
	\par
     \boite{gray}{\texttt{%
     \BODY
     }}
     \par
}

\newenvironment{vbloc}{% boite sans cadre empeche saut de page
     \begin{minipage}[t]{\linewidth}
     }
     {
     \end{minipage}
}
\NewEnviron{h2}{%
    \needspace{3\baselineskip}
    \vspace{0.6cm}
	\noindent \MakeUppercase{\sffamily \large \BODY}
	\vspace{1mm}\textcolor{mcgris}{\hrule}\vspace{0.4cm}
	\par
}{}

\NewEnviron{h3}{%
    \needspace{3\baselineskip}
	\vspace{5mm}
	\textsc{\BODY}
	\par
}

\NewEnviron{margeneg}{ %
\begin{addmargin}[-1cm]{0cm}
\BODY
\end{addmargin}
}

\NewEnviron{html}{%
}

\begin{document}
\begin{center}
     \begin{extern}%width="400" alt=""
          \newrgbcolor{grey}{0.2 0.2 0.2}
          \psset{xunit=1.0cm,yunit=1.0cm,algebraic=true,dimen=middle,dotstyle=o,dotsize=5pt 0,linewidth=1.6pt,arrowsize=3pt 2,arrowinset=0.25}
          \begin{pspicture*}(2.,0.)(9.,5.)
               \pspolygon[linewidth=0.4pt,linecolor=grey](5.738795805833042,3.3693979029165204)(5.8040177825740145,3.238953949434575)(5.93446173605596,3.3041759261755477)(5.869239759314987,3.434619879657493)
               \pspolygon[linewidth=0.4pt,linecolor=grey](7.142504831241246,1.101090032454455)(7.001129447156849,1.1369024887369967)(6.965316990874307,0.995527104652599)(7.106692374958705,0.9597146483700572)
               \pspolygon[linewidth=0.4pt,linecolor=grey](5.003467207433093,1.6429396046799178)(4.8620918233486945,1.6787520609624595)(4.826279367066153,1.5373766768780617)(4.967654751150551,1.50156422059552)
               \pspolygon[linewidth=0.4pt,linecolor=grey](5.267143958152722,3.133571979076361)(5.332365934893695,3.0031280255944157)(5.462809888375641,3.0683500023353885)(5.397587911634668,3.198793955817334)
               \psplot[linewidth=0.4pt,linecolor=grey]{2.}{9.}{(--1.--1.*x)/2.}
               \psplot[linewidth=0.4pt,linecolor=grey]{2.}{9.}{(--5.6380408873423296-0.5174736842105285*x)/2.042809917355372}
               \psline[linewidth=0.4pt,linecolor=blue](6.43162538544843,1.130719008189808)(5.397587911634668,3.198793955817334)
               \psline[linewidth=0.4pt,linecolor=blue](5.397587911634668,3.198793955817334)(4.967654751150551,1.50156422059552)
               \psline[linewidth=0.4pt,linecolor=blue](8.004005905632527,4.502002952816263)(7.106692374958705,0.9597146483700572)
               \psline[linewidth=0.4pt,linecolor=blue](7.106692374958705,0.9597146483700572)(5.869239759314987,3.434619879657493)
               \psplot[linewidth=0.4pt,linecolor=red]{2.}{9.}{(--8.248965338892633-1.9330556590619732*x)/-0.9015850081644361}
               \psplot[linewidth=0.4pt,linecolor=red]{2.}{9.}{(--19.904914857534692-3.3712839446264553*x)/-1.5723805201840966}
               \fontsize{9pt}{9pt}\selectfont
               \rput[tl](8.217190082644626,4.6){$\tiny\grey{\mathscr{D}}$}
               \rput[tl](8.065940082644625,0.5432786885245919){$\tiny\grey{\mathscr{D'}}$}
               \begin{scriptsize}
                    \psdots[dotsize=2pt 0,dotstyle=*,linecolor=grey](3.,2.)
                    \rput[bl](2.9440650826446273,2.1109387402933586){\grey{$O$}}
                    \psdots[dotsize=2pt 0,dotstyle=*](8.004005905632527,4.502002952816263)
                    \rput[bl](7.907815082644626,4.598381363244179){$I$}
                    \psdots[dotsize=2pt 0,dotstyle=*,linecolor=grey](7.106692374958705,0.9597146483700572)
                    \rput[bl](7.0759400826446255,0.6259637618636773){\grey{$K$}}
                    \psdots[dotsize=2pt 0,dotstyle=*](6.43162538544843,1.130719008189808)
                    \rput[bl](6.429690082644626,0.7844434857635911){$J$}
                    \psdots[dotsize=2pt 0,dotstyle=*,linecolor=grey](5.397587911634668,3.198793955817334)
                    \rput[bl](5.302190082644627,3.3787765314926688){\grey{$M$}}
                    \psdots[dotsize=2pt 0,dotstyle=*,linecolor=grey](4.967654751150551,1.50156422059552)
                    \rput[bl](4.9928150826446265,1.1703071613459898){\grey{$N$}}
                    \psdots[dotsize=2pt 0,dotstyle=*,linecolor=grey](5.869239759314987,3.434619879657493)
                    \rput[bl](5.797190082644626,3.5854892148403823){\grey{$L$}}
               \end{scriptsize}
          \end{pspicture*}
     \end{extern}
\end{center}
$\mathscr{D} $ et $\mathscr{D'} $ sont deux droites sécantes en $ O $.
\\
$ I $ est un point quelconque de $\mathscr{D} $ et $ J $ un point quelconque de $\mathscr{D'}. $
\par
$K$ est la projection orthogonale de $ I $ sur $\mathscr{D'} $
\\
(cela signifie que $ K \in \mathscr{D'} $ et que les droites $ \left( IJ \right) $ et $\mathscr{D'} $ sont perpendiculaires.)
\\
$L$ est la projection orthogonale de $ K $ sur $\mathscr{D} $
\\
$M$ est la projection orthogonale de $ J $ sur $\mathscr{D} $
\\
$N$ est la projection orthogonale de $M$ sur $\mathscr{D'}. $
\par
Démontrer que les droites $ \left( IJ \right) $ et $ \left( LN \right) $ sont parallèles.
\begin{corrige}
     Les droites $ \left( JM \right) $ et $ \left( KL \right) $ sont toutes les deux perpendiculaires à la droite $\mathscr{D} $, donc, elles sont parallèles entre elles.
     \par
     Par ailleurs, les points $ O, N, K $ sont alignés ainsi que les points $ O, L, M $~;
     \\
     par conséquent, d'après le théorème de Thalès~:
     \begin{center}
          $ \frac{ OJ }{ OK } = \frac{ OM }{ OL } = \frac{ JM }{ KL } $
     \end{center}
     L'égalité $ \frac{ OJ }{ OK } = \frac{ OM }{ OL } $ est équivalente à~:
     \begin{center}
          $ OM \times OK = OJ \times OL \quad \textbf{(1)} $
     \end{center}
\medskip
De même, les droites $  \left( IK \right)   $ et $  \left( NM \right)   $ sont parallèles puisqu'elles sont toutes les deux perpendiculaires à la droite $\mathscr{D'} $.
\par
Les points $ O, M, K $ et $ O, M, I $ sont alignés ;
\par
donc, d'après le théorème de Thalès :   
     \begin{center}
          $ \frac{ OK }{ ON } = \frac{ OI }{ OM } = \frac{ IK }{ NM } $
     \end{center}
 L'égalité $ \frac{ OK }{ ON } = \frac{ OI }{ OM } $ est équivalente à~:
     \begin{center}
          $ OM \times OK = OI \times ON \quad \textbf{(2)} $
     \end{center}
\medskip
Des égalités  \textbf{(1) } et  \textbf{(2)}  on en déduit que :
     \begin{center}
         $ OJ \times OL = OI \times ON  $
     \end{center}
En divisant chaque membre de l'égalité par $ OL  \times ON  $ on en déduit que :
\par
$ \frac{ OJ  \times OL  }{ OL  \times ON  } = \frac{ OI  \times ON  }{ OL  \times ON  }    $ 
\par
$ \frac{ OJ }{ ON } = \frac{ OI }{ OL }    $ 
\par
Donc, d'après \mcLien{https://www.maths-cours.fr/cours/theoreme-thales/\#t50}{la réciproque du théorème de Thalès}, les droites $  \left( NL \right)   $ et $  \left( IJ \right)   $ sont parallèles.   
\end{corrige}

\end{document}