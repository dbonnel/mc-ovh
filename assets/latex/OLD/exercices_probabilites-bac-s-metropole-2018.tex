\documentclass[a4paper]{article}

%================================================================================================================================
%
% Packages
%
%================================================================================================================================

\usepackage[T1]{fontenc} 	% pour caractères accentués
\usepackage[utf8]{inputenc}  % encodage utf8
\usepackage[french]{babel}	% langue : français
\usepackage{fourier}			% caractères plus lisibles
\usepackage[dvipsnames]{xcolor} % couleurs
\usepackage{fancyhdr}		% réglage header footer
\usepackage{needspace}		% empêcher sauts de page mal placés
\usepackage{graphicx}		% pour inclure des graphiques
\usepackage{enumitem,cprotect}		% personnalise les listes d'items (nécessaire pour ol, al ...)
\usepackage{hyperref}		% Liens hypertexte
\usepackage{pstricks,pst-all,pst-node,pstricks-add,pst-math,pst-plot,pst-tree,pst-eucl} % pstricks
\usepackage[a4paper,includeheadfoot,top=2cm,left=3cm, bottom=2cm,right=3cm]{geometry} % marges etc.
\usepackage{comment}			% commentaires multilignes
\usepackage{amsmath,environ} % maths (matrices, etc.)
\usepackage{amssymb,makeidx}
\usepackage{bm}				% bold maths
\usepackage{tabularx}		% tableaux
\usepackage{colortbl}		% tableaux en couleur
\usepackage{fontawesome}		% Fontawesome
\usepackage{environ}			% environment with command
\usepackage{fp}				% calculs pour ps-tricks
\usepackage{multido}			% pour ps tricks
\usepackage[np]{numprint}	% formattage nombre
\usepackage{tikz,tkz-tab} 			% package principal TikZ
\usepackage{pgfplots}   % axes
\usepackage{mathrsfs}    % cursives
\usepackage{calc}			% calcul taille boites
\usepackage[scaled=0.875]{helvet} % font sans serif
\usepackage{svg} % svg
\usepackage{scrextend} % local margin
\usepackage{scratch} %scratch
\usepackage{multicol} % colonnes
%\usepackage{infix-RPN,pst-func} % formule en notation polanaise inversée
\usepackage{listings}

%================================================================================================================================
%
% Réglages de base
%
%================================================================================================================================

\lstset{
language=Python,   % R code
literate=
{á}{{\'a}}1
{à}{{\`a}}1
{ã}{{\~a}}1
{é}{{\'e}}1
{è}{{\`e}}1
{ê}{{\^e}}1
{í}{{\'i}}1
{ó}{{\'o}}1
{õ}{{\~o}}1
{ú}{{\'u}}1
{ü}{{\"u}}1
{ç}{{\c{c}}}1
{~}{{ }}1
}


\definecolor{codegreen}{rgb}{0,0.6,0}
\definecolor{codegray}{rgb}{0.5,0.5,0.5}
\definecolor{codepurple}{rgb}{0.58,0,0.82}
\definecolor{backcolour}{rgb}{0.95,0.95,0.92}

\lstdefinestyle{mystyle}{
    backgroundcolor=\color{backcolour},   
    commentstyle=\color{codegreen},
    keywordstyle=\color{magenta},
    numberstyle=\tiny\color{codegray},
    stringstyle=\color{codepurple},
    basicstyle=\ttfamily\footnotesize,
    breakatwhitespace=false,         
    breaklines=true,                 
    captionpos=b,                    
    keepspaces=true,                 
    numbers=left,                    
xleftmargin=2em,
framexleftmargin=2em,            
    showspaces=false,                
    showstringspaces=false,
    showtabs=false,                  
    tabsize=2,
    upquote=true
}

\lstset{style=mystyle}


\lstset{style=mystyle}
\newcommand{\imgdir}{C:/laragon/www/newmc/assets/imgsvg/}
\newcommand{\imgsvgdir}{C:/laragon/www/newmc/assets/imgsvg/}

\definecolor{mcgris}{RGB}{220, 220, 220}% ancien~; pour compatibilité
\definecolor{mcbleu}{RGB}{52, 152, 219}
\definecolor{mcvert}{RGB}{125, 194, 70}
\definecolor{mcmauve}{RGB}{154, 0, 215}
\definecolor{mcorange}{RGB}{255, 96, 0}
\definecolor{mcturquoise}{RGB}{0, 153, 153}
\definecolor{mcrouge}{RGB}{255, 0, 0}
\definecolor{mclightvert}{RGB}{205, 234, 190}

\definecolor{gris}{RGB}{220, 220, 220}
\definecolor{bleu}{RGB}{52, 152, 219}
\definecolor{vert}{RGB}{125, 194, 70}
\definecolor{mauve}{RGB}{154, 0, 215}
\definecolor{orange}{RGB}{255, 96, 0}
\definecolor{turquoise}{RGB}{0, 153, 153}
\definecolor{rouge}{RGB}{255, 0, 0}
\definecolor{lightvert}{RGB}{205, 234, 190}
\setitemize[0]{label=\color{lightvert}  $\bullet$}

\pagestyle{fancy}
\renewcommand{\headrulewidth}{0.2pt}
\fancyhead[L]{maths-cours.fr}
\fancyhead[R]{\thepage}
\renewcommand{\footrulewidth}{0.2pt}
\fancyfoot[C]{}

\newcolumntype{C}{>{\centering\arraybackslash}X}
\newcolumntype{s}{>{\hsize=.35\hsize\arraybackslash}X}

\setlength{\parindent}{0pt}		 
\setlength{\parskip}{3mm}
\setlength{\headheight}{1cm}

\def\ebook{ebook}
\def\book{book}
\def\web{web}
\def\type{web}

\newcommand{\vect}[1]{\overrightarrow{\,\mathstrut#1\,}}

\def\Oij{$\left(\text{O}~;~\vect{\imath},~\vect{\jmath}\right)$}
\def\Oijk{$\left(\text{O}~;~\vect{\imath},~\vect{\jmath},~\vect{k}\right)$}
\def\Ouv{$\left(\text{O}~;~\vect{u},~\vect{v}\right)$}

\hypersetup{breaklinks=true, colorlinks = true, linkcolor = OliveGreen, urlcolor = OliveGreen, citecolor = OliveGreen, pdfauthor={Didier BONNEL - https://www.maths-cours.fr} } % supprime les bordures autour des liens

\renewcommand{\arg}[0]{\text{arg}}

\everymath{\displaystyle}

%================================================================================================================================
%
% Macros - Commandes
%
%================================================================================================================================

\newcommand\meta[2]{    			% Utilisé pour créer le post HTML.
	\def\titre{titre}
	\def\url{url}
	\def\arg{#1}
	\ifx\titre\arg
		\newcommand\maintitle{#2}
		\fancyhead[L]{#2}
		{\Large\sffamily \MakeUppercase{#2}}
		\vspace{1mm}\textcolor{mcvert}{\hrule}
	\fi 
	\ifx\url\arg
		\fancyfoot[L]{\href{https://www.maths-cours.fr#2}{\black \footnotesize{https://www.maths-cours.fr#2}}}
	\fi 
}


\newcommand\TitreC[1]{    		% Titre centré
     \needspace{3\baselineskip}
     \begin{center}\textbf{#1}\end{center}
}

\newcommand\newpar{    		% paragraphe
     \par
}

\newcommand\nosp {    		% commande vide (pas d'espace)
}
\newcommand{\id}[1]{} %ignore

\newcommand\boite[2]{				% Boite simple sans titre
	\vspace{5mm}
	\setlength{\fboxrule}{0.2mm}
	\setlength{\fboxsep}{5mm}	
	\fcolorbox{#1}{#1!3}{\makebox[\linewidth-2\fboxrule-2\fboxsep]{
  		\begin{minipage}[t]{\linewidth-2\fboxrule-4\fboxsep}\setlength{\parskip}{3mm}
  			 #2
  		\end{minipage}
	}}
	\vspace{5mm}
}

\newcommand\CBox[4]{				% Boites
	\vspace{5mm}
	\setlength{\fboxrule}{0.2mm}
	\setlength{\fboxsep}{5mm}
	
	\fcolorbox{#1}{#1!3}{\makebox[\linewidth-2\fboxrule-2\fboxsep]{
		\begin{minipage}[t]{1cm}\setlength{\parskip}{3mm}
	  		\textcolor{#1}{\LARGE{#2}}    
 	 	\end{minipage}  
  		\begin{minipage}[t]{\linewidth-2\fboxrule-4\fboxsep}\setlength{\parskip}{3mm}
			\raisebox{1.2mm}{\normalsize\sffamily{\textcolor{#1}{#3}}}						
  			 #4
  		\end{minipage}
	}}
	\vspace{5mm}
}

\newcommand\cadre[3]{				% Boites convertible html
	\par
	\vspace{2mm}
	\setlength{\fboxrule}{0.1mm}
	\setlength{\fboxsep}{5mm}
	\fcolorbox{#1}{white}{\makebox[\linewidth-2\fboxrule-2\fboxsep]{
  		\begin{minipage}[t]{\linewidth-2\fboxrule-4\fboxsep}\setlength{\parskip}{3mm}
			\raisebox{-2.5mm}{\sffamily \small{\textcolor{#1}{\MakeUppercase{#2}}}}		
			\par		
  			 #3
 	 		\end{minipage}
	}}
		\vspace{2mm}
	\par
}

\newcommand\bloc[3]{				% Boites convertible html sans bordure
     \needspace{2\baselineskip}
     {\sffamily \small{\textcolor{#1}{\MakeUppercase{#2}}}}    
		\par		
  			 #3
		\par
}

\newcommand\CHelp[1]{
     \CBox{Plum}{\faInfoCircle}{À RETENIR}{#1}
}

\newcommand\CUp[1]{
     \CBox{NavyBlue}{\faThumbsOUp}{EN PRATIQUE}{#1}
}

\newcommand\CInfo[1]{
     \CBox{Sepia}{\faArrowCircleRight}{REMARQUE}{#1}
}

\newcommand\CRedac[1]{
     \CBox{PineGreen}{\faEdit}{BIEN R\'EDIGER}{#1}
}

\newcommand\CError[1]{
     \CBox{Red}{\faExclamationTriangle}{ATTENTION}{#1}
}

\newcommand\TitreExo[2]{
\needspace{4\baselineskip}
 {\sffamily\large EXERCICE #1\ (\emph{#2 points})}
\vspace{5mm}
}

\newcommand\img[2]{
          \includegraphics[width=#2\paperwidth]{\imgdir#1}
}

\newcommand\imgsvg[2]{
       \begin{center}   \includegraphics[width=#2\paperwidth]{\imgsvgdir#1} \end{center}
}


\newcommand\Lien[2]{
     \href{#1}{#2 \tiny \faExternalLink}
}
\newcommand\mcLien[2]{
     \href{https~://www.maths-cours.fr/#1}{#2 \tiny \faExternalLink}
}

\newcommand{\euro}{\eurologo{}}

%================================================================================================================================
%
% Macros - Environement
%
%================================================================================================================================

\newenvironment{tex}{ %
}
{%
}

\newenvironment{indente}{ %
	\setlength\parindent{10mm}
}

{
	\setlength\parindent{0mm}
}

\newenvironment{corrige}{%
     \needspace{3\baselineskip}
     \medskip
     \textbf{\textsc{Corrigé}}
     \medskip
}
{
}

\newenvironment{extern}{%
     \begin{center}
     }
     {
     \end{center}
}

\NewEnviron{code}{%
	\par
     \boite{gray}{\texttt{%
     \BODY
     }}
     \par
}

\newenvironment{vbloc}{% boite sans cadre empeche saut de page
     \begin{minipage}[t]{\linewidth}
     }
     {
     \end{minipage}
}
\NewEnviron{h2}{%
    \needspace{3\baselineskip}
    \vspace{0.6cm}
	\noindent \MakeUppercase{\sffamily \large \BODY}
	\vspace{1mm}\textcolor{mcgris}{\hrule}\vspace{0.4cm}
	\par
}{}

\NewEnviron{h3}{%
    \needspace{3\baselineskip}
	\vspace{5mm}
	\textsc{\BODY}
	\par
}

\NewEnviron{margeneg}{ %
\begin{addmargin}[-1cm]{0cm}
\BODY
\end{addmargin}
}

\NewEnviron{html}{%
}

\begin{document}
\begin{h2}Exercice 2 (4 points)\end{h2}
\textbf{Commun à tous les candidats }
\bigbreak
\emph{Les parties A et B de cet exercice sont indépendantes.}
\medbreak
Le virus de la grippe atteint chaque année, en période hivernale, une partie de la population d'une ville.
\par
La vaccination contre la grippe est possible~; elle doit être renouvelée chaque année.
\bigbreak
\TitreC{Partie A}
\medbreak
L'efficacité du vaccin contre la grippe peut être diminuée en fonction des caractéristiques
individuelles des personnes vaccinées, ou en raison du vaccin, qui n'est pas toujours
totalement adapté aux souches du virus qui circulent. Il est donc possible de contracter la
grippe tout en étant vacciné.
\par
Une étude menée dans la population de la ville à l'issue de la période hivernale a permis de constater que~:
\begin{itemize}
     \item40\,\% de la population est vaccinée~;
     \item8\,\% des personnes vaccinées ont contracté la grippe~;
     \item20\,\% de la population a contracté la grippe.
\end{itemize}
\smallbreak
On choisit une personne au hasard dans la population de la ville et on considère les événements~:
\begin{itemize}[label=---]
     \item $V$~: \og la personne est vaccinée contre la grippe \fg{}~;
     \item $G$~: \og la personne a contracté la grippe \fg.
\end{itemize}
\medbreak
\begin{enumerate}
     \item
     \begin{enumerate}[label=\alph*.]
          \item Donner la probabilité de l'événement $G$.
          \item Reproduire l'arbre pondéré ci-dessous et compléter les pointillés indiqués sur quatre de ses branches.
         %:-+-+-+- Engendré par : http://math.et.info.free.fr/TikZ/Arbre/
          \begin{center}
               \begin{extern}%width="350"
                    % Racine à Gauche, développement vers la droite
                    \begin{tikzpicture}[xscale=1,yscale=1]
                         % Styles (MODIFIABLES)
                         \tikzstyle{fleche}=[-,>=latex,thick]
                         \tikzstyle{noeud}=[circle,draw]
                         \tikzstyle{feuille}=[circle,draw]
                         \tikzstyle{etiquette}=[midway,fill=white]
                         % Dimensions (MODIFIABLES)
                         \def\DistanceInterNiveaux{3}
                         \def\DistanceInterFeuilles{2}
                         % Dimensions calculées (NON MODIFIABLES)
                         \def\NiveauA{(0)*\DistanceInterNiveaux}
                         \def\NiveauB{(1.5)*\DistanceInterNiveaux}
                         \def\NiveauC{(2.5)*\DistanceInterNiveaux}
                         \def\InterFeuilles{(-1)*\DistanceInterFeuilles}
                         % Noeuds (MODIFIABLES : Styles et Coefficients d'InterFeuilles)
                         \node[noeud] (R) at ({\NiveauA},{(1.5)*\InterFeuilles}) {$ $};
                         \node[noeud] (Ra) at ({\NiveauB},{(0.5)*\InterFeuilles}) {$V$};
                         \node[feuille] (Raa) at ({\NiveauC},{(0)*\InterFeuilles}) {$G$};
                         \node[feuille] (Rab) at ({\NiveauC},{(1)*\InterFeuilles}) {$\overline{G}$};
                         \node[noeud] (Rb) at ({\NiveauB},{(2.5)*\InterFeuilles}) {$\overline{V}$};
                         \node[feuille] (Rba) at ({\NiveauC},{(2)*\InterFeuilles}) {$G$};
                         \node[feuille] (Rbb) at ({\NiveauC},{(3)*\InterFeuilles}) {$\overline{G}$};
                         % Arcs (MODIFIABLES : Styles)
                         \draw[fleche] (R)--(Ra) node[etiquette] {$\cdots$};
                         \draw[fleche] (Ra)--(Raa) node[etiquette] {$\cdots$};
                         \draw[fleche] (Ra)--(Rab) node[etiquette] {$\cdots$};
                         \draw[fleche] (R)--(Rb) node[etiquette] {$\cdots$};
                         \draw[fleche] (Rb)--(Rba) node[etiquette] {$\cdots$};
                         \draw[fleche] (Rb)--(Rbb) node[etiquette] {$\cdots$};
                    \end{tikzpicture}
               \end{extern}
          \end{center}
     \end{enumerate}
     \item Déterminer la probabilité que la personne choisie ait contracté la grippe et soit vaccinée.
     \item La personne choisie n'est pas vaccinée. Montrer que la probabilité qu'elle ait contracté la grippe est égale à $0,28$.
\end{enumerate}
\bigbreak
\TitreC{Partie B}
\medbreak
\emph{Dans cette partie, les probabilités demandées seront données à $10^{-3}$ près.}
\medbreak
Un laboratoire pharmaceutique mène une étude sur la vaccination contre la grippe dans cette
ville.
\medbreak
Après la période hivernale, on interroge au hasard $n$ habitants de la ville, en admettant que ce choix se ramène à $n$ tirages successifs indépendants et avec remise. On suppose que la probabilité qu'une personne choisie au hasard dans la ville soit vaccinée contre la grippe est égale à $0,4$.
\par
On note $X$ la variable aléatoire égale au nombre de personnes vaccinées parmi les $n$
interrogées.
\medbreak
\begin{enumerate}
     \item Quelle est la loi de probabilité suivie par la variable aléatoire $X$~?
     \item Dans cette question, on suppose que $n = 40$.
     \begin{enumerate}[label=\alph*.]
          \item Déterminer la probabilité qu'exactement $15$ des $40$ personnes interrogées soient vaccinées.
          \item Déterminer la probabilité qu'au moins la moitié des personnes interrogées soit vaccinée.
     \end{enumerate}
     \item  On interroge un échantillon de 3~750 habitants de la ville, c'est-à-dire que l'on suppose ici que $n = 3~750$.
     \par
     On note $Z$ la variable aléatoire définie par~: $Z = \dfrac{X - 1~500}{30}$.
     \par
     On admet que la loi de probabilité de la variable aléatoire $Z$ peut être approchée par la
     loi normale centrée réduite.
     \par
     En utilisant cette approximation, déterminer la probabilité qu'il y ait entre 1~450 et 1~550 individus vaccinés dans l'échantillon interrogé.
\end{enumerate}

\end{document}