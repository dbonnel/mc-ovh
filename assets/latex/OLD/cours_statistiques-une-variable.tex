\documentclass[a4paper]{article}

%================================================================================================================================
%
% Packages
%
%================================================================================================================================

\usepackage[T1]{fontenc} 	% pour caractères accentués
\usepackage[utf8]{inputenc}  % encodage utf8
\usepackage[french]{babel}	% langue : français
\usepackage{fourier}			% caractères plus lisibles
\usepackage[dvipsnames]{xcolor} % couleurs
\usepackage{fancyhdr}		% réglage header footer
\usepackage{needspace}		% empêcher sauts de page mal placés
\usepackage{graphicx}		% pour inclure des graphiques
\usepackage{enumitem,cprotect}		% personnalise les listes d'items (nécessaire pour ol, al ...)
\usepackage{hyperref}		% Liens hypertexte
\usepackage{pstricks,pst-all,pst-node,pstricks-add,pst-math,pst-plot,pst-tree,pst-eucl} % pstricks
\usepackage[a4paper,includeheadfoot,top=2cm,left=3cm, bottom=2cm,right=3cm]{geometry} % marges etc.
\usepackage{comment}			% commentaires multilignes
\usepackage{amsmath,environ} % maths (matrices, etc.)
\usepackage{amssymb,makeidx}
\usepackage{bm}				% bold maths
\usepackage{tabularx}		% tableaux
\usepackage{colortbl}		% tableaux en couleur
\usepackage{fontawesome}		% Fontawesome
\usepackage{environ}			% environment with command
\usepackage{fp}				% calculs pour ps-tricks
\usepackage{multido}			% pour ps tricks
\usepackage[np]{numprint}	% formattage nombre
\usepackage{tikz,tkz-tab} 			% package principal TikZ
\usepackage{pgfplots}   % axes
\usepackage{mathrsfs}    % cursives
\usepackage{calc}			% calcul taille boites
\usepackage[scaled=0.875]{helvet} % font sans serif
\usepackage{svg} % svg
\usepackage{scrextend} % local margin
\usepackage{scratch} %scratch
\usepackage{multicol} % colonnes
%\usepackage{infix-RPN,pst-func} % formule en notation polanaise inversée
\usepackage{listings}

%================================================================================================================================
%
% Réglages de base
%
%================================================================================================================================

\lstset{
language=Python,   % R code
literate=
{á}{{\'a}}1
{à}{{\`a}}1
{ã}{{\~a}}1
{é}{{\'e}}1
{è}{{\`e}}1
{ê}{{\^e}}1
{í}{{\'i}}1
{ó}{{\'o}}1
{õ}{{\~o}}1
{ú}{{\'u}}1
{ü}{{\"u}}1
{ç}{{\c{c}}}1
{~}{{ }}1
}


\definecolor{codegreen}{rgb}{0,0.6,0}
\definecolor{codegray}{rgb}{0.5,0.5,0.5}
\definecolor{codepurple}{rgb}{0.58,0,0.82}
\definecolor{backcolour}{rgb}{0.95,0.95,0.92}

\lstdefinestyle{mystyle}{
    backgroundcolor=\color{backcolour},   
    commentstyle=\color{codegreen},
    keywordstyle=\color{magenta},
    numberstyle=\tiny\color{codegray},
    stringstyle=\color{codepurple},
    basicstyle=\ttfamily\footnotesize,
    breakatwhitespace=false,         
    breaklines=true,                 
    captionpos=b,                    
    keepspaces=true,                 
    numbers=left,                    
xleftmargin=2em,
framexleftmargin=2em,            
    showspaces=false,                
    showstringspaces=false,
    showtabs=false,                  
    tabsize=2,
    upquote=true
}

\lstset{style=mystyle}


\lstset{style=mystyle}
\newcommand{\imgdir}{C:/laragon/www/newmc/assets/imgsvg/}
\newcommand{\imgsvgdir}{C:/laragon/www/newmc/assets/imgsvg/}

\definecolor{mcgris}{RGB}{220, 220, 220}% ancien~; pour compatibilité
\definecolor{mcbleu}{RGB}{52, 152, 219}
\definecolor{mcvert}{RGB}{125, 194, 70}
\definecolor{mcmauve}{RGB}{154, 0, 215}
\definecolor{mcorange}{RGB}{255, 96, 0}
\definecolor{mcturquoise}{RGB}{0, 153, 153}
\definecolor{mcrouge}{RGB}{255, 0, 0}
\definecolor{mclightvert}{RGB}{205, 234, 190}

\definecolor{gris}{RGB}{220, 220, 220}
\definecolor{bleu}{RGB}{52, 152, 219}
\definecolor{vert}{RGB}{125, 194, 70}
\definecolor{mauve}{RGB}{154, 0, 215}
\definecolor{orange}{RGB}{255, 96, 0}
\definecolor{turquoise}{RGB}{0, 153, 153}
\definecolor{rouge}{RGB}{255, 0, 0}
\definecolor{lightvert}{RGB}{205, 234, 190}
\setitemize[0]{label=\color{lightvert}  $\bullet$}

\pagestyle{fancy}
\renewcommand{\headrulewidth}{0.2pt}
\fancyhead[L]{maths-cours.fr}
\fancyhead[R]{\thepage}
\renewcommand{\footrulewidth}{0.2pt}
\fancyfoot[C]{}

\newcolumntype{C}{>{\centering\arraybackslash}X}
\newcolumntype{s}{>{\hsize=.35\hsize\arraybackslash}X}

\setlength{\parindent}{0pt}		 
\setlength{\parskip}{3mm}
\setlength{\headheight}{1cm}

\def\ebook{ebook}
\def\book{book}
\def\web{web}
\def\type{web}

\newcommand{\vect}[1]{\overrightarrow{\,\mathstrut#1\,}}

\def\Oij{$\left(\text{O}~;~\vect{\imath},~\vect{\jmath}\right)$}
\def\Oijk{$\left(\text{O}~;~\vect{\imath},~\vect{\jmath},~\vect{k}\right)$}
\def\Ouv{$\left(\text{O}~;~\vect{u},~\vect{v}\right)$}

\hypersetup{breaklinks=true, colorlinks = true, linkcolor = OliveGreen, urlcolor = OliveGreen, citecolor = OliveGreen, pdfauthor={Didier BONNEL - https://www.maths-cours.fr} } % supprime les bordures autour des liens

\renewcommand{\arg}[0]{\text{arg}}

\everymath{\displaystyle}

%================================================================================================================================
%
% Macros - Commandes
%
%================================================================================================================================

\newcommand\meta[2]{    			% Utilisé pour créer le post HTML.
	\def\titre{titre}
	\def\url{url}
	\def\arg{#1}
	\ifx\titre\arg
		\newcommand\maintitle{#2}
		\fancyhead[L]{#2}
		{\Large\sffamily \MakeUppercase{#2}}
		\vspace{1mm}\textcolor{mcvert}{\hrule}
	\fi 
	\ifx\url\arg
		\fancyfoot[L]{\href{https://www.maths-cours.fr#2}{\black \footnotesize{https://www.maths-cours.fr#2}}}
	\fi 
}


\newcommand\TitreC[1]{    		% Titre centré
     \needspace{3\baselineskip}
     \begin{center}\textbf{#1}\end{center}
}

\newcommand\newpar{    		% paragraphe
     \par
}

\newcommand\nosp {    		% commande vide (pas d'espace)
}
\newcommand{\id}[1]{} %ignore

\newcommand\boite[2]{				% Boite simple sans titre
	\vspace{5mm}
	\setlength{\fboxrule}{0.2mm}
	\setlength{\fboxsep}{5mm}	
	\fcolorbox{#1}{#1!3}{\makebox[\linewidth-2\fboxrule-2\fboxsep]{
  		\begin{minipage}[t]{\linewidth-2\fboxrule-4\fboxsep}\setlength{\parskip}{3mm}
  			 #2
  		\end{minipage}
	}}
	\vspace{5mm}
}

\newcommand\CBox[4]{				% Boites
	\vspace{5mm}
	\setlength{\fboxrule}{0.2mm}
	\setlength{\fboxsep}{5mm}
	
	\fcolorbox{#1}{#1!3}{\makebox[\linewidth-2\fboxrule-2\fboxsep]{
		\begin{minipage}[t]{1cm}\setlength{\parskip}{3mm}
	  		\textcolor{#1}{\LARGE{#2}}    
 	 	\end{minipage}  
  		\begin{minipage}[t]{\linewidth-2\fboxrule-4\fboxsep}\setlength{\parskip}{3mm}
			\raisebox{1.2mm}{\normalsize\sffamily{\textcolor{#1}{#3}}}						
  			 #4
  		\end{minipage}
	}}
	\vspace{5mm}
}

\newcommand\cadre[3]{				% Boites convertible html
	\par
	\vspace{2mm}
	\setlength{\fboxrule}{0.1mm}
	\setlength{\fboxsep}{5mm}
	\fcolorbox{#1}{white}{\makebox[\linewidth-2\fboxrule-2\fboxsep]{
  		\begin{minipage}[t]{\linewidth-2\fboxrule-4\fboxsep}\setlength{\parskip}{3mm}
			\raisebox{-2.5mm}{\sffamily \small{\textcolor{#1}{\MakeUppercase{#2}}}}		
			\par		
  			 #3
 	 		\end{minipage}
	}}
		\vspace{2mm}
	\par
}

\newcommand\bloc[3]{				% Boites convertible html sans bordure
     \needspace{2\baselineskip}
     {\sffamily \small{\textcolor{#1}{\MakeUppercase{#2}}}}    
		\par		
  			 #3
		\par
}

\newcommand\CHelp[1]{
     \CBox{Plum}{\faInfoCircle}{À RETENIR}{#1}
}

\newcommand\CUp[1]{
     \CBox{NavyBlue}{\faThumbsOUp}{EN PRATIQUE}{#1}
}

\newcommand\CInfo[1]{
     \CBox{Sepia}{\faArrowCircleRight}{REMARQUE}{#1}
}

\newcommand\CRedac[1]{
     \CBox{PineGreen}{\faEdit}{BIEN R\'EDIGER}{#1}
}

\newcommand\CError[1]{
     \CBox{Red}{\faExclamationTriangle}{ATTENTION}{#1}
}

\newcommand\TitreExo[2]{
\needspace{4\baselineskip}
 {\sffamily\large EXERCICE #1\ (\emph{#2 points})}
\vspace{5mm}
}

\newcommand\img[2]{
          \includegraphics[width=#2\paperwidth]{\imgdir#1}
}

\newcommand\imgsvg[2]{
       \begin{center}   \includegraphics[width=#2\paperwidth]{\imgsvgdir#1} \end{center}
}


\newcommand\Lien[2]{
     \href{#1}{#2 \tiny \faExternalLink}
}
\newcommand\mcLien[2]{
     \href{https~://www.maths-cours.fr/#1}{#2 \tiny \faExternalLink}
}

\newcommand{\euro}{\eurologo{}}

%================================================================================================================================
%
% Macros - Environement
%
%================================================================================================================================

\newenvironment{tex}{ %
}
{%
}

\newenvironment{indente}{ %
	\setlength\parindent{10mm}
}

{
	\setlength\parindent{0mm}
}

\newenvironment{corrige}{%
     \needspace{3\baselineskip}
     \medskip
     \textbf{\textsc{Corrigé}}
     \medskip
}
{
}

\newenvironment{extern}{%
     \begin{center}
     }
     {
     \end{center}
}

\NewEnviron{code}{%
	\par
     \boite{gray}{\texttt{%
     \BODY
     }}
     \par
}

\newenvironment{vbloc}{% boite sans cadre empeche saut de page
     \begin{minipage}[t]{\linewidth}
     }
     {
     \end{minipage}
}
\NewEnviron{h2}{%
    \needspace{3\baselineskip}
    \vspace{0.6cm}
	\noindent \MakeUppercase{\sffamily \large \BODY}
	\vspace{1mm}\textcolor{mcgris}{\hrule}\vspace{0.4cm}
	\par
}{}

\NewEnviron{h3}{%
    \needspace{3\baselineskip}
	\vspace{5mm}
	\textsc{\BODY}
	\par
}

\NewEnviron{margeneg}{ %
\begin{addmargin}[-1cm]{0cm}
\BODY
\end{addmargin}
}

\NewEnviron{html}{%
}

\begin{document}
\begin{h2}I - Rappels\end{h2}
\cadre{bleu}{Définitions}{% id="d10"
     Les statistiques permettent d'étudier un \textbf{caractère} d'une \textbf{population}.
     \par
     Le nombre d'éléments de la population s'appelle l'\textbf{effectif global} (ou l'\textbf{effectif total}).
     \par
     Pour une valeur de caractère donnée, l'\textbf{effectif} est le nombre d'éléments correspondant à cette valeur.
     \par
     Une \textbf{série statistique} est un tableau donnant les effectifs pour chacune des valeurs possibles du caractère.
}
\bloc{orange}{Exemple}{% id="e10"
     On effectue une étude portant sur l'âge des élèves d'un lycée.
     \begin{itemize}
          \item le \textbf{caractère} étudié est l'âge
          \item la \textbf{population} est l'ensemble des élèves du lycée
          \item l'\textbf{effectif global} est le nombre d'élèves du lycée
          \item le tableau ci-dessous est la \textbf{série statistique} pour ce caractère dans un lycée donné :
\begin{center}
          \begin{tabularx}{\linewidth}{|m{2.5cm}|*{7}{>{\centering \arraybackslash }X|}}%class="compact" width="600"
               \hline
               \textbf{âges (en années)}  &  14  &  15  &  16  &  17  &  18  &  19  &  20
               \\ \hline
               \textbf{effectifs}  &  3  &  22  &  65  &  82  &  59  &  35  &  2
               \\ \hline
          \end{tabularx}
\end{center}
               \end{itemize}
     }
     \begin{h2}II - Médiane - Quartiles - Déciles\end{h2}
     \cadre{bleu}{Définition}{% id="d20"
          La \textbf{médiane} d'une série statistique est la valeur du caractère qui partage la population en deux classes de même effectif.
     }
     \bloc{cyan}{Remarque}{% id="r20"
          En pratique pour trouver la médiane d'une série statistique d'effectif global $n$ :
          \begin{itemize}
               \item On ordonne les valeurs du caractère dans l'ordre croissant.
               \item Si $n$ est pair, la médiane sera la moyenne des valeurs du terme de rang $\frac{n}{2}$ et du terme de rang $\frac{n}{2}+1$.
               \item Si $n$ est impair, la médiane sera la valeur du terme de rang $\frac{n+1}{2}$.
               \item Lorsque l'effectif global est élevé, il est souvent utile de calculer les effectifs cumulés pour trouver cette valeur.
          \end{itemize}
     }
     \bloc{orange}{Exemple}{% id="e20"
          On lance 10 fois un dé à six faces. Les résultats obtenus sont : $1;5;6;6;3;2;3;1;4;1$
          \par
          On trie ces valeurs par ordre croissant : $1;1;1;2;3;3;4;5;6;6$
          \par
          $n=10$ étant pair on effectue la moyenne du cinquième et du sixième terme ($3$ et $3$) et la médiane est donc $3$.
     }
     \cadre{bleu}{Définitions}{% id="d30"
          \begin{itemize}
               \item Le \textbf{premier quartile} Q1 d'une série statistique est la plus petite valeur des termes de la série pour laquelle au moins un quart des données sont inférieures ou égales à Q1.
               \item Le \textbf{troisième quartile} Q3  d'une série statistique est la plus petite valeur des termes de la série pour laquelle au moins trois quarts des données sont inférieures ou égales à Q3.
               \item Le \textbf{premier décile} D1 d'une série statistique est la plus petite valeur  des termes de la série pour laquelle au moins 10\% des données sont inférieures ou égales à D1.
               \item Le \textbf{neuvième décile} D9 d'une série statistique est la plus petite valeur des termes de la série pour laquelle au moins 90\% des données sont inférieures ou égales à D9.
          \end{itemize}
     }
     \cadre{bleu}{Définition}{% id="d35"
          L'\textbf{écart interquartile} est la différence entre le troisième et le premier quartile $Q_{3}-Q_{1}$.
     }
     \bloc{cyan}{Remarque}{% id="r35"
          L'écart interquartile mesure la dispersion autour de la médiane.
     }
     \begin{h2}III - Diagramme en boîte\end{h2}
\begin{center}
\begin{extern}%width="320" alt="Diagramme en boîte"
\resizebox{6cm}{!}{
\psset{xunit=0.4cm,yunit=0.4cm,algebraic=true,dimen=middle,dotstyle=o,dotsize=5pt 0,linewidth=1pt,arrowsize=3pt 2,arrowinset=0.25}
\begin{pspicture*}(1.,0.5)(23.,5.5)
\psline(3.,2.5)(3.,1.5)
\psline(3.,2.)(6.,2.)
\psline(14.,2.)(18.,2.)
\psline(18.,2.5)(18.,1.5)
\psline[linecolor=blue](6.,3.)(6.,1.)
\psline[linecolor=blue](6.,1.)(14.,1.)
\psline[linecolor=blue](14.,1.)(14.,3.)
\psline[linecolor=blue](14.,3.)(6.,3.)
\psline[linecolor=red](11.,3.)(11.,1.)
\rput[tl](0.5,3.8){Minimum}
\rput[tl](5.56,4.2){\blue{Q1}}
\rput[tl](13.55,4.2){\blue{Q3}}
\rput[tl](16,3.8){Maximum}
\rput[tl](9.,4.2){\red{Médiane}}
\end{pspicture*}
}
\end{extern}
\end{center}

     On peut résumer un certain nombre d'informations relatives à une série statistique grâce à un \textbf{diagramme en boîte} (aussi appelé \textit{boîte à moustache}) qui fait apparaître (voir figure ci-dessus) :
     \begin{itemize}
          \item les valeurs minimum et maximum
          \item le premier et le troisième quartile (Q1 et Q3)
          \item la médiane
     \end{itemize}
     \par
     \bloc{orange}{Exemple}{% id="e40"
\begin{center}
\begin{extern}%width="400" alt="Exemple boîte à moustache"
\resizebox{7cm}{!}{
\psset{xunit=0.4cm,yunit=0.4cm,algebraic=true,dimen=middle,dotstyle=o,dotsize=5pt 0,linewidth=1pt,arrowsize=3pt 2,arrowinset=0.25}
\begin{pspicture*}(-1.,-1.5)(23.,4.)
\psaxes[linewidth=0.8pt,xAxis=true,yAxis=false,Dx=5.,Dy=5.,ticksize=-2pt 0,subticks=2]{->}(0,0)(-1.,-1.)(23.,4.)
\psline(3.,2.5)(3.,1.5)
\psline(3.,2.)(6.,2.)
\psline(14.,2.)(20.,2.)
\psline(20.,2.5)(20.,1.5)
\psline[linecolor=blue](6.,3.)(6.,1.)
\psline[linecolor=blue](6.,1.)(14.,1.)
\psline[linecolor=blue](14.,1.)(14.,3.)
\psline[linecolor=blue](14.,3.)(6.,3.)
\psline[linecolor=blue](9.5,3.)(9.5,1.)
\psline[linecolor=red](9.5,3.)(9.5,1.)
\end{pspicture*}
}
\end{extern}
\end{center}
          La figure ci-dessus représente une série statistique de valeurs extrêmes 3 et 20, de premier quartile 6, de troisième quartile 14 et de médiane 9,5.
     }
     
\end{document}