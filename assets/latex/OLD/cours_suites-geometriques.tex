\documentclass[a4paper]{article}

%================================================================================================================================
%
% Packages
%
%================================================================================================================================

\usepackage[T1]{fontenc} 	% pour caractères accentués
\usepackage[utf8]{inputenc}  % encodage utf8
\usepackage[french]{babel}	% langue : français
\usepackage{fourier}			% caractères plus lisibles
\usepackage[dvipsnames]{xcolor} % couleurs
\usepackage{fancyhdr}		% réglage header footer
\usepackage{needspace}		% empêcher sauts de page mal placés
\usepackage{graphicx}		% pour inclure des graphiques
\usepackage{enumitem,cprotect}		% personnalise les listes d'items (nécessaire pour ol, al ...)
\usepackage{hyperref}		% Liens hypertexte
\usepackage{pstricks,pst-all,pst-node,pstricks-add,pst-math,pst-plot,pst-tree,pst-eucl} % pstricks
\usepackage[a4paper,includeheadfoot,top=2cm,left=3cm, bottom=2cm,right=3cm]{geometry} % marges etc.
\usepackage{comment}			% commentaires multilignes
\usepackage{amsmath,environ} % maths (matrices, etc.)
\usepackage{amssymb,makeidx}
\usepackage{bm}				% bold maths
\usepackage{tabularx}		% tableaux
\usepackage{colortbl}		% tableaux en couleur
\usepackage{fontawesome}		% Fontawesome
\usepackage{environ}			% environment with command
\usepackage{fp}				% calculs pour ps-tricks
\usepackage{multido}			% pour ps tricks
\usepackage[np]{numprint}	% formattage nombre
\usepackage{tikz,tkz-tab} 			% package principal TikZ
\usepackage{pgfplots}   % axes
\usepackage{mathrsfs}    % cursives
\usepackage{calc}			% calcul taille boites
\usepackage[scaled=0.875]{helvet} % font sans serif
\usepackage{svg} % svg
\usepackage{scrextend} % local margin
\usepackage{scratch} %scratch
\usepackage{multicol} % colonnes
%\usepackage{infix-RPN,pst-func} % formule en notation polanaise inversée
\usepackage{listings}

%================================================================================================================================
%
% Réglages de base
%
%================================================================================================================================

\lstset{
language=Python,   % R code
literate=
{á}{{\'a}}1
{à}{{\`a}}1
{ã}{{\~a}}1
{é}{{\'e}}1
{è}{{\`e}}1
{ê}{{\^e}}1
{í}{{\'i}}1
{ó}{{\'o}}1
{õ}{{\~o}}1
{ú}{{\'u}}1
{ü}{{\"u}}1
{ç}{{\c{c}}}1
{~}{{ }}1
}


\definecolor{codegreen}{rgb}{0,0.6,0}
\definecolor{codegray}{rgb}{0.5,0.5,0.5}
\definecolor{codepurple}{rgb}{0.58,0,0.82}
\definecolor{backcolour}{rgb}{0.95,0.95,0.92}

\lstdefinestyle{mystyle}{
    backgroundcolor=\color{backcolour},   
    commentstyle=\color{codegreen},
    keywordstyle=\color{magenta},
    numberstyle=\tiny\color{codegray},
    stringstyle=\color{codepurple},
    basicstyle=\ttfamily\footnotesize,
    breakatwhitespace=false,         
    breaklines=true,                 
    captionpos=b,                    
    keepspaces=true,                 
    numbers=left,                    
xleftmargin=2em,
framexleftmargin=2em,            
    showspaces=false,                
    showstringspaces=false,
    showtabs=false,                  
    tabsize=2,
    upquote=true
}

\lstset{style=mystyle}


\lstset{style=mystyle}
\newcommand{\imgdir}{C:/laragon/www/newmc/assets/imgsvg/}
\newcommand{\imgsvgdir}{C:/laragon/www/newmc/assets/imgsvg/}

\definecolor{mcgris}{RGB}{220, 220, 220}% ancien~; pour compatibilité
\definecolor{mcbleu}{RGB}{52, 152, 219}
\definecolor{mcvert}{RGB}{125, 194, 70}
\definecolor{mcmauve}{RGB}{154, 0, 215}
\definecolor{mcorange}{RGB}{255, 96, 0}
\definecolor{mcturquoise}{RGB}{0, 153, 153}
\definecolor{mcrouge}{RGB}{255, 0, 0}
\definecolor{mclightvert}{RGB}{205, 234, 190}

\definecolor{gris}{RGB}{220, 220, 220}
\definecolor{bleu}{RGB}{52, 152, 219}
\definecolor{vert}{RGB}{125, 194, 70}
\definecolor{mauve}{RGB}{154, 0, 215}
\definecolor{orange}{RGB}{255, 96, 0}
\definecolor{turquoise}{RGB}{0, 153, 153}
\definecolor{rouge}{RGB}{255, 0, 0}
\definecolor{lightvert}{RGB}{205, 234, 190}
\setitemize[0]{label=\color{lightvert}  $\bullet$}

\pagestyle{fancy}
\renewcommand{\headrulewidth}{0.2pt}
\fancyhead[L]{maths-cours.fr}
\fancyhead[R]{\thepage}
\renewcommand{\footrulewidth}{0.2pt}
\fancyfoot[C]{}

\newcolumntype{C}{>{\centering\arraybackslash}X}
\newcolumntype{s}{>{\hsize=.35\hsize\arraybackslash}X}

\setlength{\parindent}{0pt}		 
\setlength{\parskip}{3mm}
\setlength{\headheight}{1cm}

\def\ebook{ebook}
\def\book{book}
\def\web{web}
\def\type{web}

\newcommand{\vect}[1]{\overrightarrow{\,\mathstrut#1\,}}

\def\Oij{$\left(\text{O}~;~\vect{\imath},~\vect{\jmath}\right)$}
\def\Oijk{$\left(\text{O}~;~\vect{\imath},~\vect{\jmath},~\vect{k}\right)$}
\def\Ouv{$\left(\text{O}~;~\vect{u},~\vect{v}\right)$}

\hypersetup{breaklinks=true, colorlinks = true, linkcolor = OliveGreen, urlcolor = OliveGreen, citecolor = OliveGreen, pdfauthor={Didier BONNEL - https://www.maths-cours.fr} } % supprime les bordures autour des liens

\renewcommand{\arg}[0]{\text{arg}}

\everymath{\displaystyle}

%================================================================================================================================
%
% Macros - Commandes
%
%================================================================================================================================

\newcommand\meta[2]{    			% Utilisé pour créer le post HTML.
	\def\titre{titre}
	\def\url{url}
	\def\arg{#1}
	\ifx\titre\arg
		\newcommand\maintitle{#2}
		\fancyhead[L]{#2}
		{\Large\sffamily \MakeUppercase{#2}}
		\vspace{1mm}\textcolor{mcvert}{\hrule}
	\fi 
	\ifx\url\arg
		\fancyfoot[L]{\href{https://www.maths-cours.fr#2}{\black \footnotesize{https://www.maths-cours.fr#2}}}
	\fi 
}


\newcommand\TitreC[1]{    		% Titre centré
     \needspace{3\baselineskip}
     \begin{center}\textbf{#1}\end{center}
}

\newcommand\newpar{    		% paragraphe
     \par
}

\newcommand\nosp {    		% commande vide (pas d'espace)
}
\newcommand{\id}[1]{} %ignore

\newcommand\boite[2]{				% Boite simple sans titre
	\vspace{5mm}
	\setlength{\fboxrule}{0.2mm}
	\setlength{\fboxsep}{5mm}	
	\fcolorbox{#1}{#1!3}{\makebox[\linewidth-2\fboxrule-2\fboxsep]{
  		\begin{minipage}[t]{\linewidth-2\fboxrule-4\fboxsep}\setlength{\parskip}{3mm}
  			 #2
  		\end{minipage}
	}}
	\vspace{5mm}
}

\newcommand\CBox[4]{				% Boites
	\vspace{5mm}
	\setlength{\fboxrule}{0.2mm}
	\setlength{\fboxsep}{5mm}
	
	\fcolorbox{#1}{#1!3}{\makebox[\linewidth-2\fboxrule-2\fboxsep]{
		\begin{minipage}[t]{1cm}\setlength{\parskip}{3mm}
	  		\textcolor{#1}{\LARGE{#2}}    
 	 	\end{minipage}  
  		\begin{minipage}[t]{\linewidth-2\fboxrule-4\fboxsep}\setlength{\parskip}{3mm}
			\raisebox{1.2mm}{\normalsize\sffamily{\textcolor{#1}{#3}}}						
  			 #4
  		\end{minipage}
	}}
	\vspace{5mm}
}

\newcommand\cadre[3]{				% Boites convertible html
	\par
	\vspace{2mm}
	\setlength{\fboxrule}{0.1mm}
	\setlength{\fboxsep}{5mm}
	\fcolorbox{#1}{white}{\makebox[\linewidth-2\fboxrule-2\fboxsep]{
  		\begin{minipage}[t]{\linewidth-2\fboxrule-4\fboxsep}\setlength{\parskip}{3mm}
			\raisebox{-2.5mm}{\sffamily \small{\textcolor{#1}{\MakeUppercase{#2}}}}		
			\par		
  			 #3
 	 		\end{minipage}
	}}
		\vspace{2mm}
	\par
}

\newcommand\bloc[3]{				% Boites convertible html sans bordure
     \needspace{2\baselineskip}
     {\sffamily \small{\textcolor{#1}{\MakeUppercase{#2}}}}    
		\par		
  			 #3
		\par
}

\newcommand\CHelp[1]{
     \CBox{Plum}{\faInfoCircle}{À RETENIR}{#1}
}

\newcommand\CUp[1]{
     \CBox{NavyBlue}{\faThumbsOUp}{EN PRATIQUE}{#1}
}

\newcommand\CInfo[1]{
     \CBox{Sepia}{\faArrowCircleRight}{REMARQUE}{#1}
}

\newcommand\CRedac[1]{
     \CBox{PineGreen}{\faEdit}{BIEN R\'EDIGER}{#1}
}

\newcommand\CError[1]{
     \CBox{Red}{\faExclamationTriangle}{ATTENTION}{#1}
}

\newcommand\TitreExo[2]{
\needspace{4\baselineskip}
 {\sffamily\large EXERCICE #1\ (\emph{#2 points})}
\vspace{5mm}
}

\newcommand\img[2]{
          \includegraphics[width=#2\paperwidth]{\imgdir#1}
}

\newcommand\imgsvg[2]{
       \begin{center}   \includegraphics[width=#2\paperwidth]{\imgsvgdir#1} \end{center}
}


\newcommand\Lien[2]{
     \href{#1}{#2 \tiny \faExternalLink}
}
\newcommand\mcLien[2]{
     \href{https~://www.maths-cours.fr/#1}{#2 \tiny \faExternalLink}
}

\newcommand{\euro}{\eurologo{}}

%================================================================================================================================
%
% Macros - Environement
%
%================================================================================================================================

\newenvironment{tex}{ %
}
{%
}

\newenvironment{indente}{ %
	\setlength\parindent{10mm}
}

{
	\setlength\parindent{0mm}
}

\newenvironment{corrige}{%
     \needspace{3\baselineskip}
     \medskip
     \textbf{\textsc{Corrigé}}
     \medskip
}
{
}

\newenvironment{extern}{%
     \begin{center}
     }
     {
     \end{center}
}

\NewEnviron{code}{%
	\par
     \boite{gray}{\texttt{%
     \BODY
     }}
     \par
}

\newenvironment{vbloc}{% boite sans cadre empeche saut de page
     \begin{minipage}[t]{\linewidth}
     }
     {
     \end{minipage}
}
\NewEnviron{h2}{%
    \needspace{3\baselineskip}
    \vspace{0.6cm}
	\noindent \MakeUppercase{\sffamily \large \BODY}
	\vspace{1mm}\textcolor{mcgris}{\hrule}\vspace{0.4cm}
	\par
}{}

\NewEnviron{h3}{%
    \needspace{3\baselineskip}
	\vspace{5mm}
	\textsc{\BODY}
	\par
}

\NewEnviron{margeneg}{ %
\begin{addmargin}[-1cm]{0cm}
\BODY
\end{addmargin}
}

\NewEnviron{html}{%
}

\begin{document}
\begin{h2}1. Suites arithmétiques\end{h2}
\cadre{bleu}{Définition}{%id="d10"
     On dit qu'une suite $\left(u_{n}\right)$ est une \textbf{suite arithmétique} s'il existe un nombre $r$ tel que, pour tout $n\in \mathbb{N}$ :
     \begin{center}$u_{n+1}=u_{n}+r$\end{center}
     Le réel $r$ s'appelle la \textbf{raison} de la suite arithmétique.
}
\bloc{cyan}{Remarque}{%id="r10"
     Pour démontrer qu'une suite $\left(u_{n}\right)$ est arithmétique, on pourra calculer la différence $u_{n+1}-u_{n}$.
     \par
     Si on constate que la différence est une constante $r$, on pourra affirmer que la suite est arithmétique de raison $r$.
}
\bloc{orange}{Exemple}{%id="e10"
     Soit la suite $\left(u_{n}\right)$ définie par $u_{n}=3n+5$.
     \par
     $u_{n+1}-u_{n}=3\left(n+1\right)+5-\left(3n+5\right)$\nosp$=3n+3+5-3n-5=3$
     \par
     La suite $\left(u_{n}\right)$ est une suite arithmétique de raison $r=3$
}
\cadre{vert}{Propriété}{%id="p20"
     Si la suite $\left(u_{n}\right)$ est arithmétique de raison $r$ alors pour tous entiers naturels $n$ et $k$ :
     \begin{center}$u_{n}=u_{k}+\left(n-k\right)\times r$\end{center}
     En particulier :
     \begin{center}$u_{n}=u_{0}+n\times r$\end{center}
}
\bloc{orange}{Exemple}{%id="e20"
     Soit $\left(u_{n}\right)$ la suite arithmétique de raison $2$ et de premier terme $u_{0}=5$.
     \par
     $u_{100}=5+2\times 100=205$
}
\cadre{vert}{Propriété}{%id="p30"
     Réciproquement, si $a$ et $b$ sont deux nombres réels et si la suite $\left(u_{n}\right)$ est définie par $u_{n}=a\times n+b$ alors cette suite est une suite arithmétique de raison $r=a$ et de premier terme $u_{0}=b$.
}
\bloc{cyan}{Démonstration}{%id="m30"
     $u_{n+1}-u_{n}=a\left(n+1\right)+b-\left(an+b\right)$\nosp$=an+a+b-an-b=a$
     \par
     et
     \par
     $u_{0}=a\times 0+b=b$
}
\cadre{vert}{Propriété}{%id="p40"
     La représentation graphique d'une suite arithmétique est formée de points alignés.
}
\bloc{cyan}{Remarque}{%id="r40"
     Cela se déduit immédiatement du fait que, pour tout $n \in \mathbb{N}$, $u_{n}=u_{0}+n\times r$ donc les points représentant la suite sont sur la droite d'équation $y=rx+u_{0}$
}
\bloc{orange}{Exemple}{%id="e40"
     \begin{center}
          \begin{extern}%width="300" alt="représentation graphique d'une suite"
               % -+-+-+ variables modifiables
               \resizebox{6cm}{!}{%
                    \def\xmin{-0.8}
                    \def\xmax{7.5}
                    \def\ymin{-0.8}
                    \def\ymax{4.8}
                    \def\xunit{1}  % unités en cm
                    \def\yunit{1}
                    \psset{xunit=\xunit,yunit=\yunit,algebraic=true}
                    \fontsize{15pt}{15pt}\selectfont
                    \begin{pspicture*}[linewidth=1pt](\xmin,\ymin)(\xmax,\ymax)
                         \psaxes[Dx=1,Dy=1,linewidth=0.75pt]{->}(0,0)(\xmin,\ymin)(\xmax,\ymax)
                         \rput[tr](-0.2,-0.3){$O$}
                         \multido{\n=0.0+1}{8}{
                              \FPeval{\suite}{1+n/2}
                              \psdots[linecolor=blue](\n,\suite)
                         }
                    \end{pspicture*}
               }
          \end{extern}
     \end{center}
     \begin{center}
          \textit{Suite arithmétique de premier terme $u_{0}=1$ et de raison $r=\frac{1}{2}$}
     \end{center}
}
\cadre{rouge}{Théorème}{%id="t50"
     Soit $\left(u_{n}\right)$ une suite arithmétique de raison $r$ :
     \begin{itemize}
          \item si $r > 0$ alors $\left(u_{n}\right)$ est strictement croissante
          \item si $r=0$ alors $\left(u_{n}\right)$ est constante
          \item si $r < 0$ alors $\left(u_{n}\right)$ est strictement décroissante.
     \end{itemize}
}
\bloc{cyan}{Démonstration}{%id="m50"
     Ce résultat découle immédiatement de $u_{n+1}-u_{n}=r$
}
\cadre{rouge}{Théorème (Somme des premiers entiers)}{%id="t60"
     Pour tout entier $n \in \mathbb{N}$ :
     \begin{center}$0+1+. . .+n=\frac{n\left(n+1\right)}{2}$\end{center}
}
\bloc{cyan}{Démonstration}{%id="m60"
     Une démonstration astucieuse consiste à réécrire la somme en inversant l'ordre des termes :
     \par
     $S = 0 + 1 + 2 + . . . + n $\textbf{(1)} \\
     $S = n + n-1 + n-2 + . . . + 0 $\textbf{(2)}
     \par
     Puis on additionne les lignes \textbf{(1)} et \textbf{(2)} termes à termes. Dans le membre de gauche on trouve que tous les termes sont égaux à $n$ ($0+n=n$ ; $1+n-1=n$ ; $2 + n-2=n$, etc.). Comme en tout il y a $n+1$ termes on trouve :
     \par
     $S+S = n + n + n + . . . + n$
     \par
     $2S = n\left(n+1\right)$
     \par
     $S = \frac{n\left(n+1\right)}{2}$
}
\bloc{orange}{Exemple}{%id="e60"
     Soit à calculer la somme $S_{100}=1+2+. . .+100$.
     \par
     $S_{100}=\frac{100\times 101}{2}=50\times 101=5050$
}
\begin{h2}2. Suites géométriques\end{h2}
\cadre{bleu}{Définition}{%id="d100"
     On dit qu'une suite $\left(u_{n}\right)$ est une \textbf{suite géométrique} s'il existe un nombre réel $q$ tel que, pour tout $n\in \mathbb{N}$ :
     \begin{center}$u_{n+1}=q \times u_{n}$\end{center}
     Le réel $q$ s'appelle la \textbf{raison} de la suite géométrique $\left(u_{n}\right)$.
}
\bloc{cyan}{Remarque}{%id="r100"
     Pour démontrer qu'une suite $\left(u_{n}\right)$ dont les termes sont non nuls est une suite géométrique, on pourra calculer le rapport $\frac{u_{n+1}}{u_{n}}$.
     \par
     Si ce rapport est une constante $q$, on pourra affirmer que la suite est une suite géométrique de raison $q$.
}
\bloc{orange}{Exemple}{%id="e100"
     Soit la suite $\left(u_{n}\right)_{n\in \mathbb{N}}$ définie par $u_{n}=\frac{3}{2^{n}}$.
     \par
     Les termes de la suite sont tous strictement positifs et
     \par
     $\frac{u_{n+1}}{u_{n}}=\frac{3}{2^{n+1}}$÷$\frac{3}{2^{n}}$\nosp$=\frac{3}{2^{n+1}}\times \frac{2^{n}}{3}$\nosp$=\frac{2^{n}}{2^{n+1}}$\nosp$=\frac{2^{n}}{2\times 2^{n}}=\frac{1}{2}$
     \par
     La suite $\left(u_{n}\right)$ est une suite géométrique de raison $\frac{1}{2}$
}
\cadre{vert}{Propriété}{%id="p110"
     Si la suite $\left(u_{n}\right)$ est géométrique de raison $q$, pour tous entiers naturels $n$ et $k$ :
     \begin{center}$u_{n}=u_{k}\times q^{n-k}$.\end{center}
     En particulier :
     \begin{center}$u_{n}=u_{0}\times q^{n}$.\end{center}
}
\cadre{vert}{Propriété}{%id="p120"
     Réciproquement, soient $a$ et $b$ deux nombres réels. La suite $\left(u_{n}\right)$ définie par $u_{n}=a\times b^{n}$ suite est une suite géométrique de raison $q=b$ et de premier terme $u_{0}=a$.
}
\bloc{cyan}{Démonstration}{%id="m120"
     $u_{n+1}=a\times b^{n+1}=a\times b^{n}\times b=u_{n}\times b$
     \par
     et
     \par
     $u_{0}=a\times b^{0}=a\times 1=a$
}
\cadre{rouge}{Théorème}{%id="t130"
     Soit $\left(u_{n}\right) $une suite géométrique de raison $q > 0$ et de premier terme strictement positif :
     \begin{itemize}
          \item Si q  > 1, la suite $\left(u_{n}\right) $ est strictement croissante
          \item Si 0 < q  < 1, la suite $\left(u_{n}\right) $ est strictement décroissante
          \item Si q=1, la suite $\left(u_{n}\right) $est constante
     \end{itemize}
}
\bloc{cyan}{Remarques}{%id="r130"
     \begin{itemize}
          \item Si le premier terme est strictement négatif, le sens de variation est inversé.
          \item Si la raison est strictement négative, la suite n'est ni croissante ni décroissante.
     \end{itemize}
}
\cadre{rouge}{Théorème}{%id="t140"
     Pour tout entier $n \in \mathbb{N}$ et tout réel $q\neq 1$
     \begin{center}$1+q+q^{2}+. . . +q^{n}=\frac{1-q^{n+1}}{1-q}$\end{center}
}
\bloc{cyan}{Remarque}{%id="r140"
     Cette formule n'est pas valable pour $q=1$. Mais dans ce cas tous les termes de la somme valent 1; la somme est donc égale au nombre de termes $n+1$
}
\bloc{cyan}{Démonstration}{%id="d140"
     On multiplie chaque membre par $q$. Cela incrémente chacun des exposants de $q$ :
     \par
     $S = 1 + q + q^{2} + . . . + q^{n} $\textbf{(1)} \\
     $qS = q + q^{2} + q^{3} + . . . + q^{n+1} $\textbf{(2)}
     \par
     On soustrait termes à termes les égalités \textbf{(1)} et \textbf{(2)}; tous les termes se simplifient sauf le premier et le dernier~:
     \par
     $S-qS = 1-q+q-q^{2}+q^{2}-q^{3}+ . . .$\nosp$ +q^{n}-q^{n+1} $
     \par
     $\left(1-q\right)S = 1-q^{n+1} $
     \par
     $S = \frac{1-q^{n+1}}{1-q}$
}
\bloc{orange}{Exemple}{%id="e140"
     Soit à calculer la somme $S=1+2+4+8+16. . .+2^{10}$
     \par
     $S=\frac{1-2^{10+1}}{1-2}=\frac{1-2048}{1-2}$\nosp$=\frac{-2047}{-1}=2047$
}

\end{document}