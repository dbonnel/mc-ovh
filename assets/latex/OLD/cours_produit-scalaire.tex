\documentclass[a4paper]{article}

%================================================================================================================================
%
% Packages
%
%================================================================================================================================

\usepackage[T1]{fontenc} 	% pour caractères accentués
\usepackage[utf8]{inputenc}  % encodage utf8
\usepackage[french]{babel}	% langue : français
\usepackage{fourier}			% caractères plus lisibles
\usepackage[dvipsnames]{xcolor} % couleurs
\usepackage{fancyhdr}		% réglage header footer
\usepackage{needspace}		% empêcher sauts de page mal placés
\usepackage{graphicx}		% pour inclure des graphiques
\usepackage{enumitem,cprotect}		% personnalise les listes d'items (nécessaire pour ol, al ...)
\usepackage{hyperref}		% Liens hypertexte
\usepackage{pstricks,pst-all,pst-node,pstricks-add,pst-math,pst-plot,pst-tree,pst-eucl} % pstricks
\usepackage[a4paper,includeheadfoot,top=2cm,left=3cm, bottom=2cm,right=3cm]{geometry} % marges etc.
\usepackage{comment}			% commentaires multilignes
\usepackage{amsmath,environ} % maths (matrices, etc.)
\usepackage{amssymb,makeidx}
\usepackage{bm}				% bold maths
\usepackage{tabularx}		% tableaux
\usepackage{colortbl}		% tableaux en couleur
\usepackage{fontawesome}		% Fontawesome
\usepackage{environ}			% environment with command
\usepackage{fp}				% calculs pour ps-tricks
\usepackage{multido}			% pour ps tricks
\usepackage[np]{numprint}	% formattage nombre
\usepackage{tikz,tkz-tab} 			% package principal TikZ
\usepackage{pgfplots}   % axes
\usepackage{mathrsfs}    % cursives
\usepackage{calc}			% calcul taille boites
\usepackage[scaled=0.875]{helvet} % font sans serif
\usepackage{svg} % svg
\usepackage{scrextend} % local margin
\usepackage{scratch} %scratch
\usepackage{multicol} % colonnes
%\usepackage{infix-RPN,pst-func} % formule en notation polanaise inversée
\usepackage{listings}

%================================================================================================================================
%
% Réglages de base
%
%================================================================================================================================

\lstset{
language=Python,   % R code
literate=
{á}{{\'a}}1
{à}{{\`a}}1
{ã}{{\~a}}1
{é}{{\'e}}1
{è}{{\`e}}1
{ê}{{\^e}}1
{í}{{\'i}}1
{ó}{{\'o}}1
{õ}{{\~o}}1
{ú}{{\'u}}1
{ü}{{\"u}}1
{ç}{{\c{c}}}1
{~}{{ }}1
}


\definecolor{codegreen}{rgb}{0,0.6,0}
\definecolor{codegray}{rgb}{0.5,0.5,0.5}
\definecolor{codepurple}{rgb}{0.58,0,0.82}
\definecolor{backcolour}{rgb}{0.95,0.95,0.92}

\lstdefinestyle{mystyle}{
    backgroundcolor=\color{backcolour},   
    commentstyle=\color{codegreen},
    keywordstyle=\color{magenta},
    numberstyle=\tiny\color{codegray},
    stringstyle=\color{codepurple},
    basicstyle=\ttfamily\footnotesize,
    breakatwhitespace=false,         
    breaklines=true,                 
    captionpos=b,                    
    keepspaces=true,                 
    numbers=left,                    
xleftmargin=2em,
framexleftmargin=2em,            
    showspaces=false,                
    showstringspaces=false,
    showtabs=false,                  
    tabsize=2,
    upquote=true
}

\lstset{style=mystyle}


\lstset{style=mystyle}
\newcommand{\imgdir}{C:/laragon/www/newmc/assets/imgsvg/}
\newcommand{\imgsvgdir}{C:/laragon/www/newmc/assets/imgsvg/}

\definecolor{mcgris}{RGB}{220, 220, 220}% ancien~; pour compatibilité
\definecolor{mcbleu}{RGB}{52, 152, 219}
\definecolor{mcvert}{RGB}{125, 194, 70}
\definecolor{mcmauve}{RGB}{154, 0, 215}
\definecolor{mcorange}{RGB}{255, 96, 0}
\definecolor{mcturquoise}{RGB}{0, 153, 153}
\definecolor{mcrouge}{RGB}{255, 0, 0}
\definecolor{mclightvert}{RGB}{205, 234, 190}

\definecolor{gris}{RGB}{220, 220, 220}
\definecolor{bleu}{RGB}{52, 152, 219}
\definecolor{vert}{RGB}{125, 194, 70}
\definecolor{mauve}{RGB}{154, 0, 215}
\definecolor{orange}{RGB}{255, 96, 0}
\definecolor{turquoise}{RGB}{0, 153, 153}
\definecolor{rouge}{RGB}{255, 0, 0}
\definecolor{lightvert}{RGB}{205, 234, 190}
\setitemize[0]{label=\color{lightvert}  $\bullet$}

\pagestyle{fancy}
\renewcommand{\headrulewidth}{0.2pt}
\fancyhead[L]{maths-cours.fr}
\fancyhead[R]{\thepage}
\renewcommand{\footrulewidth}{0.2pt}
\fancyfoot[C]{}

\newcolumntype{C}{>{\centering\arraybackslash}X}
\newcolumntype{s}{>{\hsize=.35\hsize\arraybackslash}X}

\setlength{\parindent}{0pt}		 
\setlength{\parskip}{3mm}
\setlength{\headheight}{1cm}

\def\ebook{ebook}
\def\book{book}
\def\web{web}
\def\type{web}

\newcommand{\vect}[1]{\overrightarrow{\,\mathstrut#1\,}}

\def\Oij{$\left(\text{O}~;~\vect{\imath},~\vect{\jmath}\right)$}
\def\Oijk{$\left(\text{O}~;~\vect{\imath},~\vect{\jmath},~\vect{k}\right)$}
\def\Ouv{$\left(\text{O}~;~\vect{u},~\vect{v}\right)$}

\hypersetup{breaklinks=true, colorlinks = true, linkcolor = OliveGreen, urlcolor = OliveGreen, citecolor = OliveGreen, pdfauthor={Didier BONNEL - https://www.maths-cours.fr} } % supprime les bordures autour des liens

\renewcommand{\arg}[0]{\text{arg}}

\everymath{\displaystyle}

%================================================================================================================================
%
% Macros - Commandes
%
%================================================================================================================================

\newcommand\meta[2]{    			% Utilisé pour créer le post HTML.
	\def\titre{titre}
	\def\url{url}
	\def\arg{#1}
	\ifx\titre\arg
		\newcommand\maintitle{#2}
		\fancyhead[L]{#2}
		{\Large\sffamily \MakeUppercase{#2}}
		\vspace{1mm}\textcolor{mcvert}{\hrule}
	\fi 
	\ifx\url\arg
		\fancyfoot[L]{\href{https://www.maths-cours.fr#2}{\black \footnotesize{https://www.maths-cours.fr#2}}}
	\fi 
}


\newcommand\TitreC[1]{    		% Titre centré
     \needspace{3\baselineskip}
     \begin{center}\textbf{#1}\end{center}
}

\newcommand\newpar{    		% paragraphe
     \par
}

\newcommand\nosp {    		% commande vide (pas d'espace)
}
\newcommand{\id}[1]{} %ignore

\newcommand\boite[2]{				% Boite simple sans titre
	\vspace{5mm}
	\setlength{\fboxrule}{0.2mm}
	\setlength{\fboxsep}{5mm}	
	\fcolorbox{#1}{#1!3}{\makebox[\linewidth-2\fboxrule-2\fboxsep]{
  		\begin{minipage}[t]{\linewidth-2\fboxrule-4\fboxsep}\setlength{\parskip}{3mm}
  			 #2
  		\end{minipage}
	}}
	\vspace{5mm}
}

\newcommand\CBox[4]{				% Boites
	\vspace{5mm}
	\setlength{\fboxrule}{0.2mm}
	\setlength{\fboxsep}{5mm}
	
	\fcolorbox{#1}{#1!3}{\makebox[\linewidth-2\fboxrule-2\fboxsep]{
		\begin{minipage}[t]{1cm}\setlength{\parskip}{3mm}
	  		\textcolor{#1}{\LARGE{#2}}    
 	 	\end{minipage}  
  		\begin{minipage}[t]{\linewidth-2\fboxrule-4\fboxsep}\setlength{\parskip}{3mm}
			\raisebox{1.2mm}{\normalsize\sffamily{\textcolor{#1}{#3}}}						
  			 #4
  		\end{minipage}
	}}
	\vspace{5mm}
}

\newcommand\cadre[3]{				% Boites convertible html
	\par
	\vspace{2mm}
	\setlength{\fboxrule}{0.1mm}
	\setlength{\fboxsep}{5mm}
	\fcolorbox{#1}{white}{\makebox[\linewidth-2\fboxrule-2\fboxsep]{
  		\begin{minipage}[t]{\linewidth-2\fboxrule-4\fboxsep}\setlength{\parskip}{3mm}
			\raisebox{-2.5mm}{\sffamily \small{\textcolor{#1}{\MakeUppercase{#2}}}}		
			\par		
  			 #3
 	 		\end{minipage}
	}}
		\vspace{2mm}
	\par
}

\newcommand\bloc[3]{				% Boites convertible html sans bordure
     \needspace{2\baselineskip}
     {\sffamily \small{\textcolor{#1}{\MakeUppercase{#2}}}}    
		\par		
  			 #3
		\par
}

\newcommand\CHelp[1]{
     \CBox{Plum}{\faInfoCircle}{À RETENIR}{#1}
}

\newcommand\CUp[1]{
     \CBox{NavyBlue}{\faThumbsOUp}{EN PRATIQUE}{#1}
}

\newcommand\CInfo[1]{
     \CBox{Sepia}{\faArrowCircleRight}{REMARQUE}{#1}
}

\newcommand\CRedac[1]{
     \CBox{PineGreen}{\faEdit}{BIEN R\'EDIGER}{#1}
}

\newcommand\CError[1]{
     \CBox{Red}{\faExclamationTriangle}{ATTENTION}{#1}
}

\newcommand\TitreExo[2]{
\needspace{4\baselineskip}
 {\sffamily\large EXERCICE #1\ (\emph{#2 points})}
\vspace{5mm}
}

\newcommand\img[2]{
          \includegraphics[width=#2\paperwidth]{\imgdir#1}
}

\newcommand\imgsvg[2]{
       \begin{center}   \includegraphics[width=#2\paperwidth]{\imgsvgdir#1} \end{center}
}


\newcommand\Lien[2]{
     \href{#1}{#2 \tiny \faExternalLink}
}
\newcommand\mcLien[2]{
     \href{https~://www.maths-cours.fr/#1}{#2 \tiny \faExternalLink}
}

\newcommand{\euro}{\eurologo{}}

%================================================================================================================================
%
% Macros - Environement
%
%================================================================================================================================

\newenvironment{tex}{ %
}
{%
}

\newenvironment{indente}{ %
	\setlength\parindent{10mm}
}

{
	\setlength\parindent{0mm}
}

\newenvironment{corrige}{%
     \needspace{3\baselineskip}
     \medskip
     \textbf{\textsc{Corrigé}}
     \medskip
}
{
}

\newenvironment{extern}{%
     \begin{center}
     }
     {
     \end{center}
}

\NewEnviron{code}{%
	\par
     \boite{gray}{\texttt{%
     \BODY
     }}
     \par
}

\newenvironment{vbloc}{% boite sans cadre empeche saut de page
     \begin{minipage}[t]{\linewidth}
     }
     {
     \end{minipage}
}
\NewEnviron{h2}{%
    \needspace{3\baselineskip}
    \vspace{0.6cm}
	\noindent \MakeUppercase{\sffamily \large \BODY}
	\vspace{1mm}\textcolor{mcgris}{\hrule}\vspace{0.4cm}
	\par
}{}

\NewEnviron{h3}{%
    \needspace{3\baselineskip}
	\vspace{5mm}
	\textsc{\BODY}
	\par
}

\NewEnviron{margeneg}{ %
\begin{addmargin}[-1cm]{0cm}
\BODY
\end{addmargin}
}

\NewEnviron{html}{%
}

\begin{document}
\begin{h2}1. Produit scalaire de deux vecteurs\end{h2}
\cadre{bleu}{Définition}{% id="d10"
     Soient $\vec{u}$ et $\vec{v}$ deux vecteurs non nuls du plan.
     \par
     On appelle \textbf{produit scalaire} de $\vec{u}$ et $\vec{v}$ le \textbf{nombre réel} noté $\vec{u}.\vec{v}$ défini par :
     \begin{center}$\vec{u}.\vec{v}=||\vec{u}||\times ||\vec{v}||\times \cos\left(\vec{u},\vec{v}\right)$\end{center}
}
\bloc{cyan}{Remarques}{% id="r10"
     \begin{itemize}
          \item \textbf{Attention : } le produit scalaire est un nombre réel et non un vecteur !
          \item On rappelle que $||\overrightarrow{AB}||$ (norme du vecteur $\overrightarrow{AB}$) désigne la longueur du segment $AB$.
          \item Si l'un des vecteurs $\vec{u}$ ou $\vec{v}$ est nul, $\cos\left(\vec{u},\vec{v}\right)$ n'est pas défini; on considèrera alors que le produit scalaire $\vec{u}.\vec{v}$ vaut $0$
          \item Le cosinus d'un angle étant égal au cosinus de l'angle opposé : $\cos\left(\vec{u}, \vec{v}\right)=\cos\left(\vec{v}, \vec{u}\right)$. Par conséquent $\vec{u}.\vec{v}=\vec{v}.\vec{u}$
     \end{itemize}
}
\bloc{orange}{Exemple}{% id="e10"
     \begin{center}
          \begin{extern}%width="200" alt="Triangle équilatéral"
               \newrgbcolor{mvert}{0. 0.4 0.}
               \psset{xunit=1cm,yunit=1cm,algebraic=true,dimen=middle,dotstyle=o,dotsize=5pt 0,linewidth=1pt}
               \begin{pspicture*}(-1.,-0.2)(5.,4.)
                    \psline[linewidth=1.pt](0.,0.)(2.,3.464)
                    \psline[linewidth=1.pt](2.,3.464)(4.,0.)
                    \psline[linewidth=1.pt](4.,0.)(0.,0.)
                    \parametricplot[arrows=->,linecolor=mvert]{0.0}{1.}{0.6*cos(t)+0.|0.6*sin(t)+0.}
                    \psdots[dotsize=2pt 0,dotstyle=*](0.,0.)
                    \rput[bl](-0.44,-0.08){$A$}
                    \psdots[dotsize=2pt 0,dotstyle=*](4.,0.)
                    \rput[bl](4.12,-0.08){$B$}
                    \psdots[dotsize=2pt 0,dotstyle=*](2.,3.464)
                    \rput[bl](1.82,3.7){$C$}
                    \rput[bl](0.58,0.18){\mvert{$\dfrac{\pi}{3}$}}
               \end{pspicture*}
          \end{extern}
     \end{center}
     $ABC$ est un triangle équilatéral dont le côté mesure $1$ unité.
     \par
     $\overrightarrow{AB}.\overrightarrow{AC}=AB\times AC\times \cos\left(\overrightarrow{AB}, \overrightarrow{AC}\right)=1\times 1\times \cos\frac{\pi }{3}=\frac{1}{2}$
}
\cadre{vert}{Propriété}{% id="p20"
     Deux vecteurs $\vec{u}$ et $\vec{v}$ sont orthogonaux si et seulement si : $\vec{u}.\vec{v}=0$
}
\bloc{cyan}{Démonstration}{% id="r10"
     Si l'un des vecteurs est nul le produit scalaire est nul et la propriété est vraie puisque, par convention, le vecteur nul est orthogonal à tout vecteur du plan.
     \par
     Si les deux vecteurs sont non nuls, leurs normes sont non nulles donc :
     \par
     $\vec{u}.\vec{v}=0 \Leftrightarrow ||\vec{u}||\times ||\vec{v}||\times \cos\left(\vec{u},\vec{v}\right)=0 \Leftrightarrow \cos\left(\vec{u},\vec{v}\right)=0 \Leftrightarrow \vec{u}$ et $\vec{v}$ sont orthogonaux
}
\cadre{vert}{Propriété}{% id="p30"
     Pour tous vecteurs $\vec{u}, \vec{v}, \vec{w}$ et tout réel $k$ :
     \begin{itemize}
          \item $\left(k\vec{u}\right).\vec{v}=k \left(\vec{u}.\vec{v}\right)$
          \item $\vec{u}.\left(\vec{v}+\vec{w}\right)=\vec{u}.\vec{v}+\vec{u}.\vec{w}$
     \end{itemize}
}
\cadre{bleu}{Définition et propriété}{% id="d30"
     Soit $\vec{u}$ un vecteur du plan. Le \textbf{carré scalaire} de $\vec{u}$ est le réel positif ou nul :
     \par
     $\vec{u}^{2}=\vec{u}.\vec{u}=||\vec{u}||^{2}$
}
\bloc{cyan}{Démonstration}{% id="m30"
     Le cosinus d'un angle nul vaut $1$ donc $\cos\left(\vec{u}, \vec{u}\right)=1$. Par conséquent :
     \par
     $\vec{u}.\vec{u}=||\vec{u}||\times ||\vec{u}||\times \cos\left(\vec{u},\vec{u}\right)=||\vec{u}||^{2}$
}
\cadre{rouge}{Théorème}{% id="t40"
     Pour tous vecteurs $\vec{u}$ et $\vec{v}$ :
     \begin{center}$\vec{u}.\vec{v}=\frac{1}{2} \left(||\vec{u}+\vec{v}||^{2}-||\vec{u}||^{2}-||\vec{v}||^{2}\right)$\end{center}
}
\bloc{cyan}{Démonstration}{% id="m40"
     $||\vec{u}+\vec{v}||^{2}=\left(\vec{u}+\vec{v}\right)^{2}=\vec{u}^{2}+2\left(\vec{u}.\vec{v}\right)+\vec{v}^{2}=||\vec{u}||^{2}+2\left(\vec{u}.\vec{v}\right)+||\vec{v}||^{2}$
     \par
     Par conséquent :
     \par
     $||\vec{u}+\vec{v}||^{2}-||\vec{u}||^{2}-||\vec{v}||^{2}=2\left(\vec{u}.\vec{v}\right)$
     \par
     et l'on obtient l'égalité souhaitée en divisant chaque membre par $2$.
}
\bloc{cyan}{Remarque}{ % id=r40
     De la même manière, en développant  $(\vec{u}-\vec{v})^{2}$ on obtient~:
     \newpar
     \begin{center}$\vec{u}.\vec{v}=\frac{1}{2} \left(||\vec{u}||^{2}+||\vec{v}||^{2}  - ||\vec{u}-\vec{v}||^{2}\right)$\end{center}
}% fin remarque
\bloc{orange}{Exemple}{% id="e40"
     \begin{center}
          \begin{extern}%width="300" alt="Triangle équilatéral"
               \newrgbcolor{mvert}{0. 0.4 0.}
               \psset{xunit=1cm,yunit=1cm,algebraic=true,dimen=middle,dotstyle=o,dotsize=5pt 0,linewidth=1pt}
               \begin{pspicture*}(-1.,-1.5)(8.,3.)
                    \psline[linewidth=1.pt](0.,0.)(4,-1)
                    \psline[linewidth=1.pt](4,-1)(7,1)
                    \psline[linewidth=1.pt](7,1)(3,2)
                    \psline[linewidth=1.pt](3,2)(0.,0.)
                    \psline[linewidth=1.pt](3,2)(4,-1)
                    \rput[l](-0.5,0){$B$}\rput[l](4,-1.3){$A$}
                    \rput[l](3,2.3){$C$}\rput[l](7.2,1){$D$}
                    \rput[b](2,-0.4){$6$}\rput[l](3.7,0.5){$4$}\rput[l](1.25,1.2){$5$}
               \end{pspicture*}
          \end{extern}
     \end{center}
     $ABCD$ est un parallélogramme tel que $AB=6$, $AC=4$ et $BC=5$.\\
     On souhaite calculer $\overrightarrow{AB}.\overrightarrow{AD}$.
     \par
     $\overrightarrow{AB}.\overrightarrow{AD}=\frac{1}{2}\left(||\overrightarrow{AB}+\overrightarrow{AD}||^{2}-||\overrightarrow{AB}||^{2}-||\overrightarrow{AD}||^{2}\right)$
     \par
     Or $\overrightarrow{AB}+\overrightarrow{AD}=\overrightarrow{AB}+\overrightarrow{BC}=\overrightarrow{AC}$ d'après la relation de Chasles. Par conséquent :
     \par
     $\overrightarrow{AB}.\overrightarrow{AD}=\frac{1}{2}\left(||\overrightarrow{AC}||^{2}-||\overrightarrow{AB}||^{2}-||\overrightarrow{AD}||^{2}\right)$\\
     $\phantom{{AB}.{AD}}=\frac{1}{2}\left(AC^{2}-AB^{2}-AD^{2}\right)$\\
     $\phantom{{AB}.{AD}}=\frac{1}{2}\left(16-36-25\right)=-\frac{45}{2}$
}
\cadre{rouge}{Théorème}{% id="t50"
     Soient $A, b, c$ trois points du plan et $H$ la projection orthogonale de $C$ sur la droite $\left(AB\right)$
     \par
     Alors :
     \begin{itemize}
          \item $\overrightarrow{AB}.\overrightarrow{AC}=AB\times AH   $ si l'angle $\widehat{BAC}$ est aigu
          \item $\overrightarrow{AB}.\overrightarrow{AC}=-AB\times AH   $ si l'angle $\widehat{BAC}$ est obtus
     \end{itemize}
}
\begin{center}
     \begin{extern}%width="300" alt="Produit scalaire et projection orthogonale"
          \newrgbcolor{mvert}{0. 0.4 0.}
          \psset{xunit=1cm,yunit=1cm,algebraic=true,dimen=middle,dotstyle=o,dotsize=5pt 0,linewidth=1pt}
          \begin{pspicture*}(-1.,-0.5)(8.,4.)
               \psline[linecolor=blue]{->}(0.,0.)(7,1)
               \psline[linecolor=blue]{->}(0,0)(2,3)
               \psline[linecolor=lightgray](2,3)(2.36,0.345)
               \psline[linecolor=lightgray](2.48,0.505)(2.34,0.485)
               \psline[linecolor=lightgray](2.5,0.365)(2.48,0.505)
               \psline[linecolor=red](0,0)(2.36,0.345)
               \rput[l](-0.5,0){\blue $A$}\rput[l](7.1,1){\blue $B$}\rput[l](2.1,3.2){\blue $C$}
               \rput[l](2.36,0.12){\red $H$}
          \end{pspicture*}
     \end{extern}
\end{center}
\begin{center}
     \textit{Ici : $\overrightarrow{AB}.\overrightarrow{AC}=AB\times AH$ }
\end{center}
\bloc{orange}{Exemple}{% id="e50"
     \begin{center}
          \begin{extern}%width="160" alt="Produit scalaire et projection orthogonale"
               \newrgbcolor{mvert}{0. 0.4 0.}
               \psset{xunit=1cm,yunit=1cm,algebraic=true,dimen=middle,dotstyle=o,dotsize=5pt 0,linewidth=1pt}
               \begin{pspicture*}(-0.5,-0.5)(3.5,2.5)
                    \psgrid[griddots=0, subgriddiv=0, gridlabels=0pt,gridcolor=lightgray](0,0)(3.,2.)
                    \psline[linecolor=blue]{->}(0.,0.)(3,0)
                    \psline[linecolor=blue]{->}(0,0)(2,2)
                    \rput[l](-0.4,0){\blue $A$}\rput[r](3.3,0){\blue $B$}\rput[l](2.1,2.3){\blue $C$}
               \end{pspicture*}
          \end{extern}
     \end{center}
     Sur la figure ci-dessus où l'unité est le carreau, le point $C$ se projette orthogonalement sur la droite $\left(AB\right)$ en un point $H$ (non représenté) tel que $AH=2$.
     \par
     Par conséquent, l'angle $\widehat{BAC}$ étant aigu :
     \par
     $\overrightarrow{AB}.\overrightarrow{AC}=AB\times AH=3\times 2=6$
}
\cadre{rouge}{Théorème}{% id="t60"
     Le plan étant rapporté à un repère orthonormé $\left(O; \vec{i}, \vec{j}\right)$, soient $\vec{u}\left(x; y\right)$ et $\vec{v}\left(x^{\prime}, y^{\prime}\right)$ deux vecteurs du plan; alors :
     \begin{center}$\vec{u}.\vec{v}=xx^{\prime}+yy^{\prime}$\end{center}
}
\bloc{cyan}{Démonstration}{% id="m60"
     Dire que $\vec{u}$ a pour coordonnées $\left(x ; y\right)$ signifie que $\vec{u}=x\vec{i}+y\vec{j}$. De même $\vec{v}=x^{\prime}\vec{i}+y^{\prime}\vec{j}$
     \par
     $\vec{u}.\vec{v}=\left(x\vec{i}+y\vec{j}\right).\left(x^{\prime}\vec{i}+y^{\prime}\vec{j}\right)=xx^{\prime}\vec{i}^{2}+xy^{\prime}\vec{i}.\vec{j}+x^{\prime}y\vec{i}.\vec{j}+yy^{\prime}\vec{j}^{2}$
     \par
     Or, comme le repère $\left(O; \vec{i}, \vec{j}\right)$ est orthonormé, $\vec{i}^{2}=||\vec{i}||^{2}=1$, $\vec{j}^{2}=||\vec{j}||^{2}=1$ et $\vec{i}.\vec{j}=0$. Par conséquent :
     \par
     $\vec{u}.\vec{v}=xx^{\prime}+yy^{\prime}$
}
\begin{h2}2. Applications du produit scalaire\end{h2}
\cadre{rouge}{Théorème (de la médiane)}{% id="t70"
     Soient $ABC$ un triangle quelconque et $I$ le milieu de $\left[BC\right]$. Alors :
     \begin{center}$AB^{2}+AC^{2}=2AI^{2}+\frac{BC^{2}}{2}$\end{center}
}
\begin{center}
     \begin{extern}%width="300" alt="Théorème de la médiane"
          \psset{xunit=1.0cm,yunit=1.0cm,algebraic=true,dimen=middle,dotstyle=o,dotsize=5pt 0,linewidth=0.8pt}
          \begin{pspicture*}(-1.,-1.)(7.,6.)
               \psline(0.,0.04)(3.,1.)
               \psline(1.46,0.63)(1.54,0.41)
               \psline(3.,1.)(6.,2.)
               \psline(4.46,1.61)(4.54,1.39)
               \psline(6.,2.)(3.,5.)
               \psline(3.,5.)(0.,0.04)
               \psline[linecolor=red](3.,5.)(3.,1.)
               \rput[bl](-0.3,-0.3){$B$}
               \rput[bl](2.9,5.1){$A$}
               \rput[bl](6.08,1.8){$C$}
               \rput[bl](3.,0.65){\red $I$}
          \end{pspicture*}
     \end{extern}
\end{center}
\begin{center}
     \textit{Médiane dans un triangle}
\end{center}
\bloc{cyan}{Remarque}{% id="r70"
     La démonstration est laissée en exercice : \mcLien{/exercices/geometrie-plan/theoreme-mediane}{Exercice théorème de la médiane}
}
\cadre{vert}{Propriété (Formule d'Al Kashi)}{% id="p80"
     Soit $ABC$ un triangle quelconque :
     \begin{center}$BC^{2}=AB^{2}+AC^{2}-2 AB\times AC \cos\left(\overrightarrow{AB}, \overrightarrow{AC}\right)$\end{center}
}
\bloc{cyan}{Remarque}{% id="r80"
     \begin{itemize}
          \item La démonstration est faite en exercice : \mcLien{/exercices/geometrie-plan/formule-al-kashi}{Exercice formule d'Al Kashi}
          \item Si le triangle $ABC$ est rectangle en $A$ alors $\cos\left(\overrightarrow{AB}, \overrightarrow{AC}\right)=0$. On retrouve alors le théorème de Pythagore.
     \end{itemize}
}
\cadre{bleu}{Définition (Vecteur normal à une droite)}{% id="d85"
     On dit qu'un vecteur $\vec{n}$ non nul est \textbf{normal} à la droite $d$ si et seulement si il est orthogonal à un vecteur directeur de $d$.
}
\begin{center}
     \begin{extern}%width="400" alt="Vecteur normal à une droite"
          \resizebox{8cm}{!}{
               \psset{xunit=1.0cm,yunit=1.0cm,algebraic=true,dimen=middle}
               \begin{pspicture*}(0.,0.)(10.,5.)
                    \psline[linewidth=1.pt](0.,0.)(10.,5.)
                    \psline[linewidth=1.pt,linecolor=blue]{->}(4.,2.)(3.,4.)
                    \rput[tl](8.74,5.){$d$}
                    \rput[bl](3.6,3){\blue{$\vec{n}$}}
               \end{pspicture*}
          }
     \end{extern}
\end{center}
\begin{center}
     \textit{Vecteur $\vec{n}$ normal à la droite $d$}
\end{center}
\cadre{rouge}{Théorème}{% id="t100"
     Le plan est rapporté à un repère orthonormé $\left(O, \vec{i}, \vec{j}\right)$
     \par
     La droite $d$ de vecteur normal $\vec{n} \left(a ; b\right)$ admet une équation cartésienne de la forme :
     \begin{center}$ax+by+c=0$\end{center}
     où $a$, $b$ sont les coordonnées de $\vec{n}$ et $c$ un nombre réel.
     \par
     \textbf{Réciproquement}, l'ensemble des points $M\left(x ; y\right)$ tels que $ax+by+c=0$ ($a, b, c$ étant des réels avec $a\neq 0$ ou $b\neq 0$)  est une droite dont un vecteur normal est  $\vec{n}\left(a ; b\right)$.
}
\bloc{cyan}{Remarque}{% id="r100"
     La démonstration est laissée en exercice : \mcLien{/exercices/geometrie-plan/vecteur-directeur-normal-droite}{Exercice vecteur normal à une droite}
}
\cadre{rouge}{Théorème (équation cartésienne d'un cercle)}{% id="t110"
     Le plan est rapporté à un repère orthonormé $\left(O, \vec{i}, \vec{j}\right)$.
     \par
     Soit $I \left(x_{I} ; y_{I}\right)$ un point quelconque du plan et $r$ un réel positif.
     \par
     Une équation du cercle de centre $I$ et de rayon $r$ est :
     \begin{center}$\left(x-x_{I}\right)^{2}+\left(y-y_{I}\right)^{2}=r^{2}$\end{center}
}
\bloc{cyan}{Démonstration}{% id="r110"
     Le point $M \left(x ; y\right)$ appartient au cercle si et seulement si $IM=r$. Comme $IM$ et $r$ sont positif cela équivaut à $IM^{2}=r^{2}$. Or $IM^{2}= \left(x-x_{I}\right)^{2}+\left(y-y_{I}\right)^{2}$; on obtient donc le résultat souhaité.
}
\bloc{orange}{Exemple}{% id="e110"
     Le cercle de centre $\Omega  \left(3;4\right)$ et de rayon $5$ a pour équation :
     \par
     $\left(x-3\right)^{2}+\left(y-4\right)^{2}=25$
     \par
     $x^{2}-6x+9+y^{2}-8y+16=25$
     \par
     $x^{2}-6x+y^{2}-8y=0$
     \par
     Ce cercle passe par $O$ car on obtient une égalité juste en remplaçant $x$ et $y$ par $0$.
}
Une autre utilisation du produit scalaire est la démonstration des formules d'addition des sinus et cosinus (voir \mcLien{/exercices/geometrie-plan/formule-soustraction-cosinus}{exercice soustraction des cosinus})

\end{document}