\documentclass[a4paper]{article}

%================================================================================================================================
%
% Packages
%
%================================================================================================================================

\usepackage[T1]{fontenc} 	% pour caractères accentués
\usepackage[utf8]{inputenc}  % encodage utf8
\usepackage[french]{babel}	% langue : français
\usepackage{fourier}			% caractères plus lisibles
\usepackage[dvipsnames]{xcolor} % couleurs
\usepackage{fancyhdr}		% réglage header footer
\usepackage{needspace}		% empêcher sauts de page mal placés
\usepackage{graphicx}		% pour inclure des graphiques
\usepackage{enumitem,cprotect}		% personnalise les listes d'items (nécessaire pour ol, al ...)
\usepackage{hyperref}		% Liens hypertexte
\usepackage{pstricks,pst-all,pst-node,pstricks-add,pst-math,pst-plot,pst-tree,pst-eucl} % pstricks
\usepackage[a4paper,includeheadfoot,top=2cm,left=3cm, bottom=2cm,right=3cm]{geometry} % marges etc.
\usepackage{comment}			% commentaires multilignes
\usepackage{amsmath,environ} % maths (matrices, etc.)
\usepackage{amssymb,makeidx}
\usepackage{bm}				% bold maths
\usepackage{tabularx}		% tableaux
\usepackage{colortbl}		% tableaux en couleur
\usepackage{fontawesome}		% Fontawesome
\usepackage{environ}			% environment with command
\usepackage{fp}				% calculs pour ps-tricks
\usepackage{multido}			% pour ps tricks
\usepackage[np]{numprint}	% formattage nombre
\usepackage{tikz,tkz-tab} 			% package principal TikZ
\usepackage{pgfplots}   % axes
\usepackage{mathrsfs}    % cursives
\usepackage{calc}			% calcul taille boites
\usepackage[scaled=0.875]{helvet} % font sans serif
\usepackage{svg} % svg
\usepackage{scrextend} % local margin
\usepackage{scratch} %scratch
\usepackage{multicol} % colonnes
%\usepackage{infix-RPN,pst-func} % formule en notation polanaise inversée
\usepackage{listings}

%================================================================================================================================
%
% Réglages de base
%
%================================================================================================================================

\lstset{
language=Python,   % R code
literate=
{á}{{\'a}}1
{à}{{\`a}}1
{ã}{{\~a}}1
{é}{{\'e}}1
{è}{{\`e}}1
{ê}{{\^e}}1
{í}{{\'i}}1
{ó}{{\'o}}1
{õ}{{\~o}}1
{ú}{{\'u}}1
{ü}{{\"u}}1
{ç}{{\c{c}}}1
{~}{{ }}1
}


\definecolor{codegreen}{rgb}{0,0.6,0}
\definecolor{codegray}{rgb}{0.5,0.5,0.5}
\definecolor{codepurple}{rgb}{0.58,0,0.82}
\definecolor{backcolour}{rgb}{0.95,0.95,0.92}

\lstdefinestyle{mystyle}{
    backgroundcolor=\color{backcolour},   
    commentstyle=\color{codegreen},
    keywordstyle=\color{magenta},
    numberstyle=\tiny\color{codegray},
    stringstyle=\color{codepurple},
    basicstyle=\ttfamily\footnotesize,
    breakatwhitespace=false,         
    breaklines=true,                 
    captionpos=b,                    
    keepspaces=true,                 
    numbers=left,                    
xleftmargin=2em,
framexleftmargin=2em,            
    showspaces=false,                
    showstringspaces=false,
    showtabs=false,                  
    tabsize=2,
    upquote=true
}

\lstset{style=mystyle}


\lstset{style=mystyle}
\newcommand{\imgdir}{C:/laragon/www/newmc/assets/imgsvg/}
\newcommand{\imgsvgdir}{C:/laragon/www/newmc/assets/imgsvg/}

\definecolor{mcgris}{RGB}{220, 220, 220}% ancien~; pour compatibilité
\definecolor{mcbleu}{RGB}{52, 152, 219}
\definecolor{mcvert}{RGB}{125, 194, 70}
\definecolor{mcmauve}{RGB}{154, 0, 215}
\definecolor{mcorange}{RGB}{255, 96, 0}
\definecolor{mcturquoise}{RGB}{0, 153, 153}
\definecolor{mcrouge}{RGB}{255, 0, 0}
\definecolor{mclightvert}{RGB}{205, 234, 190}

\definecolor{gris}{RGB}{220, 220, 220}
\definecolor{bleu}{RGB}{52, 152, 219}
\definecolor{vert}{RGB}{125, 194, 70}
\definecolor{mauve}{RGB}{154, 0, 215}
\definecolor{orange}{RGB}{255, 96, 0}
\definecolor{turquoise}{RGB}{0, 153, 153}
\definecolor{rouge}{RGB}{255, 0, 0}
\definecolor{lightvert}{RGB}{205, 234, 190}
\setitemize[0]{label=\color{lightvert}  $\bullet$}

\pagestyle{fancy}
\renewcommand{\headrulewidth}{0.2pt}
\fancyhead[L]{maths-cours.fr}
\fancyhead[R]{\thepage}
\renewcommand{\footrulewidth}{0.2pt}
\fancyfoot[C]{}

\newcolumntype{C}{>{\centering\arraybackslash}X}
\newcolumntype{s}{>{\hsize=.35\hsize\arraybackslash}X}

\setlength{\parindent}{0pt}		 
\setlength{\parskip}{3mm}
\setlength{\headheight}{1cm}

\def\ebook{ebook}
\def\book{book}
\def\web{web}
\def\type{web}

\newcommand{\vect}[1]{\overrightarrow{\,\mathstrut#1\,}}

\def\Oij{$\left(\text{O}~;~\vect{\imath},~\vect{\jmath}\right)$}
\def\Oijk{$\left(\text{O}~;~\vect{\imath},~\vect{\jmath},~\vect{k}\right)$}
\def\Ouv{$\left(\text{O}~;~\vect{u},~\vect{v}\right)$}

\hypersetup{breaklinks=true, colorlinks = true, linkcolor = OliveGreen, urlcolor = OliveGreen, citecolor = OliveGreen, pdfauthor={Didier BONNEL - https://www.maths-cours.fr} } % supprime les bordures autour des liens

\renewcommand{\arg}[0]{\text{arg}}

\everymath{\displaystyle}

%================================================================================================================================
%
% Macros - Commandes
%
%================================================================================================================================

\newcommand\meta[2]{    			% Utilisé pour créer le post HTML.
	\def\titre{titre}
	\def\url{url}
	\def\arg{#1}
	\ifx\titre\arg
		\newcommand\maintitle{#2}
		\fancyhead[L]{#2}
		{\Large\sffamily \MakeUppercase{#2}}
		\vspace{1mm}\textcolor{mcvert}{\hrule}
	\fi 
	\ifx\url\arg
		\fancyfoot[L]{\href{https://www.maths-cours.fr#2}{\black \footnotesize{https://www.maths-cours.fr#2}}}
	\fi 
}


\newcommand\TitreC[1]{    		% Titre centré
     \needspace{3\baselineskip}
     \begin{center}\textbf{#1}\end{center}
}

\newcommand\newpar{    		% paragraphe
     \par
}

\newcommand\nosp {    		% commande vide (pas d'espace)
}
\newcommand{\id}[1]{} %ignore

\newcommand\boite[2]{				% Boite simple sans titre
	\vspace{5mm}
	\setlength{\fboxrule}{0.2mm}
	\setlength{\fboxsep}{5mm}	
	\fcolorbox{#1}{#1!3}{\makebox[\linewidth-2\fboxrule-2\fboxsep]{
  		\begin{minipage}[t]{\linewidth-2\fboxrule-4\fboxsep}\setlength{\parskip}{3mm}
  			 #2
  		\end{minipage}
	}}
	\vspace{5mm}
}

\newcommand\CBox[4]{				% Boites
	\vspace{5mm}
	\setlength{\fboxrule}{0.2mm}
	\setlength{\fboxsep}{5mm}
	
	\fcolorbox{#1}{#1!3}{\makebox[\linewidth-2\fboxrule-2\fboxsep]{
		\begin{minipage}[t]{1cm}\setlength{\parskip}{3mm}
	  		\textcolor{#1}{\LARGE{#2}}    
 	 	\end{minipage}  
  		\begin{minipage}[t]{\linewidth-2\fboxrule-4\fboxsep}\setlength{\parskip}{3mm}
			\raisebox{1.2mm}{\normalsize\sffamily{\textcolor{#1}{#3}}}						
  			 #4
  		\end{minipage}
	}}
	\vspace{5mm}
}

\newcommand\cadre[3]{				% Boites convertible html
	\par
	\vspace{2mm}
	\setlength{\fboxrule}{0.1mm}
	\setlength{\fboxsep}{5mm}
	\fcolorbox{#1}{white}{\makebox[\linewidth-2\fboxrule-2\fboxsep]{
  		\begin{minipage}[t]{\linewidth-2\fboxrule-4\fboxsep}\setlength{\parskip}{3mm}
			\raisebox{-2.5mm}{\sffamily \small{\textcolor{#1}{\MakeUppercase{#2}}}}		
			\par		
  			 #3
 	 		\end{minipage}
	}}
		\vspace{2mm}
	\par
}

\newcommand\bloc[3]{				% Boites convertible html sans bordure
     \needspace{2\baselineskip}
     {\sffamily \small{\textcolor{#1}{\MakeUppercase{#2}}}}    
		\par		
  			 #3
		\par
}

\newcommand\CHelp[1]{
     \CBox{Plum}{\faInfoCircle}{À RETENIR}{#1}
}

\newcommand\CUp[1]{
     \CBox{NavyBlue}{\faThumbsOUp}{EN PRATIQUE}{#1}
}

\newcommand\CInfo[1]{
     \CBox{Sepia}{\faArrowCircleRight}{REMARQUE}{#1}
}

\newcommand\CRedac[1]{
     \CBox{PineGreen}{\faEdit}{BIEN R\'EDIGER}{#1}
}

\newcommand\CError[1]{
     \CBox{Red}{\faExclamationTriangle}{ATTENTION}{#1}
}

\newcommand\TitreExo[2]{
\needspace{4\baselineskip}
 {\sffamily\large EXERCICE #1\ (\emph{#2 points})}
\vspace{5mm}
}

\newcommand\img[2]{
          \includegraphics[width=#2\paperwidth]{\imgdir#1}
}

\newcommand\imgsvg[2]{
       \begin{center}   \includegraphics[width=#2\paperwidth]{\imgsvgdir#1} \end{center}
}


\newcommand\Lien[2]{
     \href{#1}{#2 \tiny \faExternalLink}
}
\newcommand\mcLien[2]{
     \href{https~://www.maths-cours.fr/#1}{#2 \tiny \faExternalLink}
}

\newcommand{\euro}{\eurologo{}}

%================================================================================================================================
%
% Macros - Environement
%
%================================================================================================================================

\newenvironment{tex}{ %
}
{%
}

\newenvironment{indente}{ %
	\setlength\parindent{10mm}
}

{
	\setlength\parindent{0mm}
}

\newenvironment{corrige}{%
     \needspace{3\baselineskip}
     \medskip
     \textbf{\textsc{Corrigé}}
     \medskip
}
{
}

\newenvironment{extern}{%
     \begin{center}
     }
     {
     \end{center}
}

\NewEnviron{code}{%
	\par
     \boite{gray}{\texttt{%
     \BODY
     }}
     \par
}

\newenvironment{vbloc}{% boite sans cadre empeche saut de page
     \begin{minipage}[t]{\linewidth}
     }
     {
     \end{minipage}
}
\NewEnviron{h2}{%
    \needspace{3\baselineskip}
    \vspace{0.6cm}
	\noindent \MakeUppercase{\sffamily \large \BODY}
	\vspace{1mm}\textcolor{mcgris}{\hrule}\vspace{0.4cm}
	\par
}{}

\NewEnviron{h3}{%
    \needspace{3\baselineskip}
	\vspace{5mm}
	\textsc{\BODY}
	\par
}

\NewEnviron{margeneg}{ %
\begin{addmargin}[-1cm]{0cm}
\BODY
\end{addmargin}
}

\NewEnviron{html}{%
}

\begin{document}
\begin{h2}1. Le type liste\end{h2}
\begin{h3} Définition d'une liste en Python\end{h3}
Une \textit{liste} est un type de valeur qui contient une suite ordonnée de valeurs.
\par
On dit qu'une liste est définie \textit{en extension} lorsque l'on énumère toutes les valeurs de la liste. Une liste est encadrée par des \textbf{crochets} et les différentes valeurs de la liste sont séparées par des virgules~; par exemple~:
\begin{lstlisting}[language=Python]
liste_de_nombres = [ 7, 12, 1.5, 9, 43]
liste_de_couleurs = [ "rouge", "vert", "bleu", "jaune", "orange" ]
\end{lstlisting}
Il est possible, en Python, de définir des listes comportant des valeurs de types différents (nombres, chaîne de caractères où même des sous-listes...)~:
 \begin{lstlisting}[language=Python]
liste_de_types_differents = [ 3, "Bonjour", 2.5, ["a", "b"], true ]
\end{lstlisting}
Il est également courant de définir une liste vide que l'on remplira par la suite (grâce à l'instruction \texttt{append} que l'on détaillera ultérieurement)~:
 \begin{lstlisting}[language=Python]
liste_vide = []
\end{lstlisting}
\begin{h3}Accès à un élément\end{h3}
Il est possible d'accéder à un élément d'une liste grâce à sa position appelée \textit{indice}.
\par
\textbf{Attention~: } Le premier élément d'une liste correspond à l'indice 0 et si la liste contient \texttt{n} éléments, l'indice du dernier élément est \texttt{n-1}~:
\begin{lstlisting}[language=Python]
liste = [ "un", "deux", "trois", "quatre"]
# indices : 0       1        2         3
\end{lstlisting}
Pour accéder à un élément d'une liste on utilise la syntaxe suivante~: \texttt{nom_de_la_liste[indice]}, par exemple~:
\begin{lstlisting}[language=Python]
liste = [ "rouge", "vert", "bleu", "jaune", "orange" ]
print(liste[0]) # affiche rouge
print(liste[4]) # affiche orange
liste[2] = "violet" # modifie le 3ème élément de la liste
print(liste) # affiche ['rouge', 'vert', 'violet', 'jaune', 'orange']
\end{lstlisting}
\begin{h3} Liste définie en compréhension \end{h3}
En mathématiques, il est possible de définir un ensemble en \textit{compréhension }. Par exemple, l'ensemble~:
\begin{center}
     $ E = \left\{ n \in \mathbb{N} ~|~ 5 \leqslant n < 10 \right\} $
\end{center}
représente l'ensemble des entiers naturels supérieurs ou égaux à $5$ et strictement inférieurs à $10 $, c'est à dire l'ensemble~:
\begin{center}
     $ E = \left\{ 5~;~ 6~;~ 7~;~ 8~;~ 9 \right\} $
\end{center}
En Python, il est également possible de définir une liste en \textit{compréhension } en utilisant une boucle \texttt{for} à l'intérieur de la définition de la liste~:
\begin{lstlisting}[language=Python]
liste = [ n for n in range(5, 10) ]
# équivaut à liste = [ 5, 6, 7, 8, 9 ]
\end{lstlisting}
Voici deux exemples plus avancés de listes définies en compréhension~:
\begin{lstlisting}[language=Python]
liste1 = [ n**2 for n in range(6) ]
# liste1 contient [ 0, 1, 4, 9, 16, 25 ]
liste2 = [ n for n in range(2, 7) if n!=4 ]
# liste2 contient [2, 3, 5, 6]
\end{lstlisting}
\begin{h2} 2. Opérations sur les listes \end{h2}
\begin{h3} La méthode \texttt{.append()} \end{h3}
En programmation, une \textit{ méthode} est une fonction qui agit sur un certain objet.
\par
La méthode \texttt{.append()} permet d'ajouter un élément à la fin d'une liste, par exemple~:
\begin{lstlisting}[language=Python]
mes_notes = [12, 14, 9, 16]
mes_notes.append(13)
print(mes_notes) # affiche [12, 14, 9, 16, 13]
\end{lstlisting}
Il est fréquent, lorsque l'on souhaite créer une liste pas à pas, de partir d'une liste vide et d'ajouter les éléments un par un. Par exemple, le programme suivant crée une liste en comptant de 3 en 3 en partant de 1 jusqu'à 16~:
\begin{lstlisting}[language=Python]
ma_liste = []
n = 1
while n <= 16 :
   ma_liste.append(n)
   n = n+3
print(ma_liste) # affiche [1, 4, 7, 10, 13, 16]
\end{lstlisting}
\begin{h3} Concaténation de listes \end{h3}
Comme pour les chaînes de caractères, l'opérateur \og + \fg{} permet de concaténer des listes~:
\begin{lstlisting}[language=Python]
couleurs1 = [ "rouge", "jaune", "vert"]
couleurs2 = ["orange", "bleu"]
couleurs = couleurs1 + couleurs2
print(couleurs) # affiche ['rouge', 'jaune', 'vert', 'orange', 'bleu']
\end{lstlisting}
\begin{h3} Longueur d'une liste \end{h3}
La fonction \texttt{len()} retourne la longueur de la liste passée en paramètre.
\par
Cette fonction peut être utilisée lorsque l'on souhaite parcourir les éléments d'une liste pour déterminer la fin de la boucle. Par exemple, le programme ci-dessous affiche les éléments de la liste séparés par des points-virgules~:
\begin{lstlisting}[language=Python]
liste = ["a", "b", "c", "d"]
for i in range(len(liste)) :
   print (liste[i], end=";") # affiche a;b;c;d;
\end{lstlisting}

\end{document}