\documentclass[a4paper]{article}

%================================================================================================================================
%
% Packages
%
%================================================================================================================================

\usepackage[T1]{fontenc} 	% pour caractères accentués
\usepackage[utf8]{inputenc}  % encodage utf8
\usepackage[french]{babel}	% langue : français
\usepackage{fourier}			% caractères plus lisibles
\usepackage[dvipsnames]{xcolor} % couleurs
\usepackage{fancyhdr}		% réglage header footer
\usepackage{needspace}		% empêcher sauts de page mal placés
\usepackage{graphicx}		% pour inclure des graphiques
\usepackage{enumitem,cprotect}		% personnalise les listes d'items (nécessaire pour ol, al ...)
\usepackage{hyperref}		% Liens hypertexte
\usepackage{pstricks,pst-all,pst-node,pstricks-add,pst-math,pst-plot,pst-tree,pst-eucl} % pstricks
\usepackage[a4paper,includeheadfoot,top=2cm,left=3cm, bottom=2cm,right=3cm]{geometry} % marges etc.
\usepackage{comment}			% commentaires multilignes
\usepackage{amsmath,environ} % maths (matrices, etc.)
\usepackage{amssymb,makeidx}
\usepackage{bm}				% bold maths
\usepackage{tabularx}		% tableaux
\usepackage{colortbl}		% tableaux en couleur
\usepackage{fontawesome}		% Fontawesome
\usepackage{environ}			% environment with command
\usepackage{fp}				% calculs pour ps-tricks
\usepackage{multido}			% pour ps tricks
\usepackage[np]{numprint}	% formattage nombre
\usepackage{tikz,tkz-tab} 			% package principal TikZ
\usepackage{pgfplots}   % axes
\usepackage{mathrsfs}    % cursives
\usepackage{calc}			% calcul taille boites
\usepackage[scaled=0.875]{helvet} % font sans serif
\usepackage{svg} % svg
\usepackage{scrextend} % local margin
\usepackage{scratch} %scratch
\usepackage{multicol} % colonnes
%\usepackage{infix-RPN,pst-func} % formule en notation polanaise inversée
\usepackage{listings}

%================================================================================================================================
%
% Réglages de base
%
%================================================================================================================================

\lstset{
language=Python,   % R code
literate=
{á}{{\'a}}1
{à}{{\`a}}1
{ã}{{\~a}}1
{é}{{\'e}}1
{è}{{\`e}}1
{ê}{{\^e}}1
{í}{{\'i}}1
{ó}{{\'o}}1
{õ}{{\~o}}1
{ú}{{\'u}}1
{ü}{{\"u}}1
{ç}{{\c{c}}}1
{~}{{ }}1
}


\definecolor{codegreen}{rgb}{0,0.6,0}
\definecolor{codegray}{rgb}{0.5,0.5,0.5}
\definecolor{codepurple}{rgb}{0.58,0,0.82}
\definecolor{backcolour}{rgb}{0.95,0.95,0.92}

\lstdefinestyle{mystyle}{
    backgroundcolor=\color{backcolour},   
    commentstyle=\color{codegreen},
    keywordstyle=\color{magenta},
    numberstyle=\tiny\color{codegray},
    stringstyle=\color{codepurple},
    basicstyle=\ttfamily\footnotesize,
    breakatwhitespace=false,         
    breaklines=true,                 
    captionpos=b,                    
    keepspaces=true,                 
    numbers=left,                    
xleftmargin=2em,
framexleftmargin=2em,            
    showspaces=false,                
    showstringspaces=false,
    showtabs=false,                  
    tabsize=2,
    upquote=true
}

\lstset{style=mystyle}


\lstset{style=mystyle}
\newcommand{\imgdir}{C:/laragon/www/newmc/assets/imgsvg/}
\newcommand{\imgsvgdir}{C:/laragon/www/newmc/assets/imgsvg/}

\definecolor{mcgris}{RGB}{220, 220, 220}% ancien~; pour compatibilité
\definecolor{mcbleu}{RGB}{52, 152, 219}
\definecolor{mcvert}{RGB}{125, 194, 70}
\definecolor{mcmauve}{RGB}{154, 0, 215}
\definecolor{mcorange}{RGB}{255, 96, 0}
\definecolor{mcturquoise}{RGB}{0, 153, 153}
\definecolor{mcrouge}{RGB}{255, 0, 0}
\definecolor{mclightvert}{RGB}{205, 234, 190}

\definecolor{gris}{RGB}{220, 220, 220}
\definecolor{bleu}{RGB}{52, 152, 219}
\definecolor{vert}{RGB}{125, 194, 70}
\definecolor{mauve}{RGB}{154, 0, 215}
\definecolor{orange}{RGB}{255, 96, 0}
\definecolor{turquoise}{RGB}{0, 153, 153}
\definecolor{rouge}{RGB}{255, 0, 0}
\definecolor{lightvert}{RGB}{205, 234, 190}
\setitemize[0]{label=\color{lightvert}  $\bullet$}

\pagestyle{fancy}
\renewcommand{\headrulewidth}{0.2pt}
\fancyhead[L]{maths-cours.fr}
\fancyhead[R]{\thepage}
\renewcommand{\footrulewidth}{0.2pt}
\fancyfoot[C]{}

\newcolumntype{C}{>{\centering\arraybackslash}X}
\newcolumntype{s}{>{\hsize=.35\hsize\arraybackslash}X}

\setlength{\parindent}{0pt}		 
\setlength{\parskip}{3mm}
\setlength{\headheight}{1cm}

\def\ebook{ebook}
\def\book{book}
\def\web{web}
\def\type{web}

\newcommand{\vect}[1]{\overrightarrow{\,\mathstrut#1\,}}

\def\Oij{$\left(\text{O}~;~\vect{\imath},~\vect{\jmath}\right)$}
\def\Oijk{$\left(\text{O}~;~\vect{\imath},~\vect{\jmath},~\vect{k}\right)$}
\def\Ouv{$\left(\text{O}~;~\vect{u},~\vect{v}\right)$}

\hypersetup{breaklinks=true, colorlinks = true, linkcolor = OliveGreen, urlcolor = OliveGreen, citecolor = OliveGreen, pdfauthor={Didier BONNEL - https://www.maths-cours.fr} } % supprime les bordures autour des liens

\renewcommand{\arg}[0]{\text{arg}}

\everymath{\displaystyle}

%================================================================================================================================
%
% Macros - Commandes
%
%================================================================================================================================

\newcommand\meta[2]{    			% Utilisé pour créer le post HTML.
	\def\titre{titre}
	\def\url{url}
	\def\arg{#1}
	\ifx\titre\arg
		\newcommand\maintitle{#2}
		\fancyhead[L]{#2}
		{\Large\sffamily \MakeUppercase{#2}}
		\vspace{1mm}\textcolor{mcvert}{\hrule}
	\fi 
	\ifx\url\arg
		\fancyfoot[L]{\href{https://www.maths-cours.fr#2}{\black \footnotesize{https://www.maths-cours.fr#2}}}
	\fi 
}


\newcommand\TitreC[1]{    		% Titre centré
     \needspace{3\baselineskip}
     \begin{center}\textbf{#1}\end{center}
}

\newcommand\newpar{    		% paragraphe
     \par
}

\newcommand\nosp {    		% commande vide (pas d'espace)
}
\newcommand{\id}[1]{} %ignore

\newcommand\boite[2]{				% Boite simple sans titre
	\vspace{5mm}
	\setlength{\fboxrule}{0.2mm}
	\setlength{\fboxsep}{5mm}	
	\fcolorbox{#1}{#1!3}{\makebox[\linewidth-2\fboxrule-2\fboxsep]{
  		\begin{minipage}[t]{\linewidth-2\fboxrule-4\fboxsep}\setlength{\parskip}{3mm}
  			 #2
  		\end{minipage}
	}}
	\vspace{5mm}
}

\newcommand\CBox[4]{				% Boites
	\vspace{5mm}
	\setlength{\fboxrule}{0.2mm}
	\setlength{\fboxsep}{5mm}
	
	\fcolorbox{#1}{#1!3}{\makebox[\linewidth-2\fboxrule-2\fboxsep]{
		\begin{minipage}[t]{1cm}\setlength{\parskip}{3mm}
	  		\textcolor{#1}{\LARGE{#2}}    
 	 	\end{minipage}  
  		\begin{minipage}[t]{\linewidth-2\fboxrule-4\fboxsep}\setlength{\parskip}{3mm}
			\raisebox{1.2mm}{\normalsize\sffamily{\textcolor{#1}{#3}}}						
  			 #4
  		\end{minipage}
	}}
	\vspace{5mm}
}

\newcommand\cadre[3]{				% Boites convertible html
	\par
	\vspace{2mm}
	\setlength{\fboxrule}{0.1mm}
	\setlength{\fboxsep}{5mm}
	\fcolorbox{#1}{white}{\makebox[\linewidth-2\fboxrule-2\fboxsep]{
  		\begin{minipage}[t]{\linewidth-2\fboxrule-4\fboxsep}\setlength{\parskip}{3mm}
			\raisebox{-2.5mm}{\sffamily \small{\textcolor{#1}{\MakeUppercase{#2}}}}		
			\par		
  			 #3
 	 		\end{minipage}
	}}
		\vspace{2mm}
	\par
}

\newcommand\bloc[3]{				% Boites convertible html sans bordure
     \needspace{2\baselineskip}
     {\sffamily \small{\textcolor{#1}{\MakeUppercase{#2}}}}    
		\par		
  			 #3
		\par
}

\newcommand\CHelp[1]{
     \CBox{Plum}{\faInfoCircle}{À RETENIR}{#1}
}

\newcommand\CUp[1]{
     \CBox{NavyBlue}{\faThumbsOUp}{EN PRATIQUE}{#1}
}

\newcommand\CInfo[1]{
     \CBox{Sepia}{\faArrowCircleRight}{REMARQUE}{#1}
}

\newcommand\CRedac[1]{
     \CBox{PineGreen}{\faEdit}{BIEN R\'EDIGER}{#1}
}

\newcommand\CError[1]{
     \CBox{Red}{\faExclamationTriangle}{ATTENTION}{#1}
}

\newcommand\TitreExo[2]{
\needspace{4\baselineskip}
 {\sffamily\large EXERCICE #1\ (\emph{#2 points})}
\vspace{5mm}
}

\newcommand\img[2]{
          \includegraphics[width=#2\paperwidth]{\imgdir#1}
}

\newcommand\imgsvg[2]{
       \begin{center}   \includegraphics[width=#2\paperwidth]{\imgsvgdir#1} \end{center}
}


\newcommand\Lien[2]{
     \href{#1}{#2 \tiny \faExternalLink}
}
\newcommand\mcLien[2]{
     \href{https~://www.maths-cours.fr/#1}{#2 \tiny \faExternalLink}
}

\newcommand{\euro}{\eurologo{}}

%================================================================================================================================
%
% Macros - Environement
%
%================================================================================================================================

\newenvironment{tex}{ %
}
{%
}

\newenvironment{indente}{ %
	\setlength\parindent{10mm}
}

{
	\setlength\parindent{0mm}
}

\newenvironment{corrige}{%
     \needspace{3\baselineskip}
     \medskip
     \textbf{\textsc{Corrigé}}
     \medskip
}
{
}

\newenvironment{extern}{%
     \begin{center}
     }
     {
     \end{center}
}

\NewEnviron{code}{%
	\par
     \boite{gray}{\texttt{%
     \BODY
     }}
     \par
}

\newenvironment{vbloc}{% boite sans cadre empeche saut de page
     \begin{minipage}[t]{\linewidth}
     }
     {
     \end{minipage}
}
\NewEnviron{h2}{%
    \needspace{3\baselineskip}
    \vspace{0.6cm}
	\noindent \MakeUppercase{\sffamily \large \BODY}
	\vspace{1mm}\textcolor{mcgris}{\hrule}\vspace{0.4cm}
	\par
}{}

\NewEnviron{h3}{%
    \needspace{3\baselineskip}
	\vspace{5mm}
	\textsc{\BODY}
	\par
}

\NewEnviron{margeneg}{ %
\begin{addmargin}[-1cm]{0cm}
\BODY
\end{addmargin}
}

\NewEnviron{html}{%
}

\begin{document}
\begin{h2}Exercice 1 (6 points)\end{h2}
\textbf{Commun à tous les candidats }
\bigbreak
\emph{Dans cet exercice, on munit le plan d'un repère orthonormé.}
\par
On a représenté ci-dessous la courbe d'équation~:
\[y = \dfrac{1}{2}\left(\text{e}^x + \text{e}^{-x} - 2\right).\]
\par
Cette courbe est appelée une \og chaînette \fg.
\par
On s'intéresse ici aux \og arcs de chaînette\fg{} délimités par deux points de cette courbe
symétriques par rapport à l'axe des ordonnées.
\par
Un tel arc est représenté sur le graphique ci-dessous en trait plein.
\par
On définit la \og largeur \fg{} et la \og hauteur \fg{} de l'arc de chaînette délimité par les points $M$ et $M'$ comme indiqué sur le graphique.
\begin{center}
     \begin{extern}%width="400" alt=""
          \psset{unit=1.5cm}
          \begin{pspicture}(-2,-0.8)(2,2)
               \psaxes[linewidth=1pt,Dx=4,Dy=4]{->}(0,0)(-2,0)(2,2)
               \psplot[plotpoints=3000,linewidth=1pt]{-1.8}{1.8}{2.71828 x exp 2.71828 x neg exp add 2 sub 0.5 mul}
               \psline[linestyle=dashed](1.4,0)(1.4,1.1509)(-1.4,1.1509)(-1.4,0)
               \uput[d](1.4,0){$x$} \uput[d](-1.4,0){$- x$}
               \psline{<->}(-1.4,-0.5)(1.4,-0.5)
               \uput[d](0,-0.5){largeur}
               \psline{<->}(-1.6,0)(-1.6,1.1509)
               \uput[l](-1.6,0.56){hauteur}
               \uput[ur](1.5,0.9){$M\left(x~;~\dfrac{1}{2}\left(\text{e}^x + \text{e}^{- x} - 2\right)\right)$}
               \uput[ur](-1.4,1.15){$M'$}
          \end{pspicture}
     \end{extern}
\end{center}
\medbreak
Le but de l'exercice est d'étudier les positions possibles sur la courbe du point $M$ d'abscisse $x$ strictement positive afin que la largeur de l'arc de chaînette soit égale à sa hauteur.
\medbreak
\begin{enumerate}
     \item Justifier que le problème étudié se ramène à la recherche des solutions strictement
     positives de l'équation
     \par
     \[(E)~: \text{e}^x + \text{e}^{- x} - 4x - 2 = 0.\]
     \item  On note $f$ la fonction définie sur l'intervalle $[0~;~+\infty[$ par~:
     \par
     \[f(x) = \text{e}^x + \text{e}^{- x} - 4x - 2.\]
     \begin{enumerate}[label=\alph*.]
          \item Vérifier que pour tout $x > 0,\: f(x) = x \left(\dfrac{\text{e}^x}{x}- 4\right) + \text{e}^{- x} - 2$.
          \item Déterminer $\displaystyle\lim_{x \to + \infty} f(x)$.
     \end{enumerate}
     \item
     \begin{enumerate}[label=\alph*.]
          \item On note $f'$ la fonction dérivée de la fonction $f$. Calculer $f'(x)$, où $x$ appartient à l'intervalle $[0~;~ +\infty[$.
          \item Montrer que l'équation $f'(x) = 0$ équivaut à l'équation~: $\left(\text{e}^x\right)^2 - 4\text{e}^x - 1 = 0$.
          \item En posant $X = \text{e}^x$, montrer que l'équation $f'(x) = 0$ admet pour unique solution réelle le nombre $\ln \left(2 + \sqrt{5}\right)$.
     \end{enumerate}
     \item  On donne ci-dessous le tableau de signes de la fonction dérivée $f'$ de $f$~:
     \begin{center}
          \begin{extern}%width="300" alt=""
               \psset{unit=1cm}
               \begin{pspicture}(7,1.5)
                    \psframe(7,1.5)\psline(0,0.75)(7,0.75)\psline(1,0)(1,1.5)
                    \uput[u](0.5,0.85){$x$}\uput[u](1.2,0.85){$0$}
                    \uput[u](4,0.7){$\ln \left(2 + \sqrt{5} \right)$}\uput[u](6.5,0.85){$+ \infty$}
                    \rput(0.5,0.375){$f'(x)$}\rput(2,0.375){$-$}
                    \rput(4,0.375){$0$}\rput(5,0.375){$+$}
               \end{pspicture}
          \end{extern}
     \end{center}
     \begin{enumerate}[label=\alph*.]
          \item Dresser le tableau de variations de la fonction $f$.
          \item Démontrer que l'équation $f(x) = 0$ admet une unique solution strictement positive que l'on notera $\alpha$.
     \end{enumerate}
     \item On considère l'algorithme suivant où les variables $a$, $b$ et $m$ sont des nombres réels~:
     \begin{center}
          \begin{extern}%width="350" alt=""
               \begin{tabularx}{0.5\linewidth}{|X|}\hline
                    Tant que $b - a > 0,1$ faire~:\\
                    \hspace{1cm}$m \gets \dfrac{a+b}{2}$\\
                    \hspace{1cm}Si $\text{e}^m + \text{e}^{-m} - 4m - 2 > 0$, alors~:\\
                    \hspace{2cm}$b \gets m$\\
                    \hspace{1cm}Sinon~:\\
                    \hspace{2cm}$a\gets m$\\
                    \hspace{1cm}Fin Si\\
                    Fin Tant que\\ \hline
               \end{tabularx}
          \end{extern}
     \end{center}
     \begin{enumerate}[label=\alph*.]
          \item Avant l'exécution de cet algorithme, les variables $a$ et $b$
          contiennent respectivement les valeurs $2$ et $3$.
          \par
          Que contiennent-elles à la fin de l'exécution de l'algorithme~?
          \par
          On justifiera la réponse en reproduisant et en complétant le tableau ci-contre avec les différentes valeurs prises par les variables, à chaque étape de l'algorithme.
          \item Comment peut-on utiliser les valeurs obtenues en fin d'algorithme à la question
          précédente~?
     \end{enumerate}
     \begin{center}
          \begin{extern}%width="550" alt=""
               \begin{tabularx}{0.8\linewidth}{|*{4}{>{\centering \arraybackslash}X|}}\hline
                    $m$ 	&$a$ &$b$ &$b - a$\\ \hline
                    \cellcolor{lightgray}	&2& 3 &1\\ \hline
                    2,5		&&&\\ \hline
                    \ldots	&\ldots&\ldots&\\ \hline
                    ~		&&&\\ \hline
               \end{tabularx}
          \end{extern}
     \end{center}
     \item La \emph{Gateway Arch}, édifiée dans la ville de Saint-Louis aux États-Unis, a l'allure ci-dessous.
     \begin{center}
          \begin{extern}%width="180" alt=""
               \psset{unit=1.3cm,arrowsize=2pt 3}
               \begin{pspicture}(-2,-1)(2,2)
                    %\psaxes[linewidth=1.25pt,Dx=2,Dy=2]{->}(0,0)(-2,0)(2,2)
                    \psplot[plotpoints=3000,linewidth=0.75pt]{-1.8}{1.8}{2.71828 x exp 2.71828 x neg exp add 2 sub 0.5 mul neg 1.5 add}
                    \psline{<->}(-1.8,-0.6)(1.8,-0.6)
                    \uput[d](0,-0.6){largeur}
                    \psline{<->}(0,-0.6)(0,1.5)
                    \uput[r](0,0.45){hauteur}
               \end{pspicture}
          \end{extern}
     \end{center}
     Son profil peut être approché par un arc de chaînette renversé dont la largeur est égale à  la hauteur.
     \par
     La largeur de cet arc, exprimée en mètre, est égale au double de la solution strictement
     positive de l'équation~:
     \par
     \[\left(E'\right)~: \text{e}^{\tfrac{t}{39}} + \text{e}^{-\tfrac{t}{39}} - 4\frac{t}{39} - 2 = 0.\]
     \par
     Donner un encadrement de la hauteur de la \emph{Gateway Arch}.
\end{enumerate}

\end{document}