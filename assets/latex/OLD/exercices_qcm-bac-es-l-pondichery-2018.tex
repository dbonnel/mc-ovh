\documentclass[a4paper]{article}

%================================================================================================================================
%
% Packages
%
%================================================================================================================================

\usepackage[T1]{fontenc} 	% pour caractères accentués
\usepackage[utf8]{inputenc}  % encodage utf8
\usepackage[french]{babel}	% langue : français
\usepackage{fourier}			% caractères plus lisibles
\usepackage[dvipsnames]{xcolor} % couleurs
\usepackage{fancyhdr}		% réglage header footer
\usepackage{needspace}		% empêcher sauts de page mal placés
\usepackage{graphicx}		% pour inclure des graphiques
\usepackage{enumitem,cprotect}		% personnalise les listes d'items (nécessaire pour ol, al ...)
\usepackage{hyperref}		% Liens hypertexte
\usepackage{pstricks,pst-all,pst-node,pstricks-add,pst-math,pst-plot,pst-tree,pst-eucl} % pstricks
\usepackage[a4paper,includeheadfoot,top=2cm,left=3cm, bottom=2cm,right=3cm]{geometry} % marges etc.
\usepackage{comment}			% commentaires multilignes
\usepackage{amsmath,environ} % maths (matrices, etc.)
\usepackage{amssymb,makeidx}
\usepackage{bm}				% bold maths
\usepackage{tabularx}		% tableaux
\usepackage{colortbl}		% tableaux en couleur
\usepackage{fontawesome}		% Fontawesome
\usepackage{environ}			% environment with command
\usepackage{fp}				% calculs pour ps-tricks
\usepackage{multido}			% pour ps tricks
\usepackage[np]{numprint}	% formattage nombre
\usepackage{tikz,tkz-tab} 			% package principal TikZ
\usepackage{pgfplots}   % axes
\usepackage{mathrsfs}    % cursives
\usepackage{calc}			% calcul taille boites
\usepackage[scaled=0.875]{helvet} % font sans serif
\usepackage{svg} % svg
\usepackage{scrextend} % local margin
\usepackage{scratch} %scratch
\usepackage{multicol} % colonnes
%\usepackage{infix-RPN,pst-func} % formule en notation polanaise inversée
\usepackage{listings}

%================================================================================================================================
%
% Réglages de base
%
%================================================================================================================================

\lstset{
language=Python,   % R code
literate=
{á}{{\'a}}1
{à}{{\`a}}1
{ã}{{\~a}}1
{é}{{\'e}}1
{è}{{\`e}}1
{ê}{{\^e}}1
{í}{{\'i}}1
{ó}{{\'o}}1
{õ}{{\~o}}1
{ú}{{\'u}}1
{ü}{{\"u}}1
{ç}{{\c{c}}}1
{~}{{ }}1
}


\definecolor{codegreen}{rgb}{0,0.6,0}
\definecolor{codegray}{rgb}{0.5,0.5,0.5}
\definecolor{codepurple}{rgb}{0.58,0,0.82}
\definecolor{backcolour}{rgb}{0.95,0.95,0.92}

\lstdefinestyle{mystyle}{
    backgroundcolor=\color{backcolour},   
    commentstyle=\color{codegreen},
    keywordstyle=\color{magenta},
    numberstyle=\tiny\color{codegray},
    stringstyle=\color{codepurple},
    basicstyle=\ttfamily\footnotesize,
    breakatwhitespace=false,         
    breaklines=true,                 
    captionpos=b,                    
    keepspaces=true,                 
    numbers=left,                    
xleftmargin=2em,
framexleftmargin=2em,            
    showspaces=false,                
    showstringspaces=false,
    showtabs=false,                  
    tabsize=2,
    upquote=true
}

\lstset{style=mystyle}


\lstset{style=mystyle}
\newcommand{\imgdir}{C:/laragon/www/newmc/assets/imgsvg/}
\newcommand{\imgsvgdir}{C:/laragon/www/newmc/assets/imgsvg/}

\definecolor{mcgris}{RGB}{220, 220, 220}% ancien~; pour compatibilité
\definecolor{mcbleu}{RGB}{52, 152, 219}
\definecolor{mcvert}{RGB}{125, 194, 70}
\definecolor{mcmauve}{RGB}{154, 0, 215}
\definecolor{mcorange}{RGB}{255, 96, 0}
\definecolor{mcturquoise}{RGB}{0, 153, 153}
\definecolor{mcrouge}{RGB}{255, 0, 0}
\definecolor{mclightvert}{RGB}{205, 234, 190}

\definecolor{gris}{RGB}{220, 220, 220}
\definecolor{bleu}{RGB}{52, 152, 219}
\definecolor{vert}{RGB}{125, 194, 70}
\definecolor{mauve}{RGB}{154, 0, 215}
\definecolor{orange}{RGB}{255, 96, 0}
\definecolor{turquoise}{RGB}{0, 153, 153}
\definecolor{rouge}{RGB}{255, 0, 0}
\definecolor{lightvert}{RGB}{205, 234, 190}
\setitemize[0]{label=\color{lightvert}  $\bullet$}

\pagestyle{fancy}
\renewcommand{\headrulewidth}{0.2pt}
\fancyhead[L]{maths-cours.fr}
\fancyhead[R]{\thepage}
\renewcommand{\footrulewidth}{0.2pt}
\fancyfoot[C]{}

\newcolumntype{C}{>{\centering\arraybackslash}X}
\newcolumntype{s}{>{\hsize=.35\hsize\arraybackslash}X}

\setlength{\parindent}{0pt}		 
\setlength{\parskip}{3mm}
\setlength{\headheight}{1cm}

\def\ebook{ebook}
\def\book{book}
\def\web{web}
\def\type{web}

\newcommand{\vect}[1]{\overrightarrow{\,\mathstrut#1\,}}

\def\Oij{$\left(\text{O}~;~\vect{\imath},~\vect{\jmath}\right)$}
\def\Oijk{$\left(\text{O}~;~\vect{\imath},~\vect{\jmath},~\vect{k}\right)$}
\def\Ouv{$\left(\text{O}~;~\vect{u},~\vect{v}\right)$}

\hypersetup{breaklinks=true, colorlinks = true, linkcolor = OliveGreen, urlcolor = OliveGreen, citecolor = OliveGreen, pdfauthor={Didier BONNEL - https://www.maths-cours.fr} } % supprime les bordures autour des liens

\renewcommand{\arg}[0]{\text{arg}}

\everymath{\displaystyle}

%================================================================================================================================
%
% Macros - Commandes
%
%================================================================================================================================

\newcommand\meta[2]{    			% Utilisé pour créer le post HTML.
	\def\titre{titre}
	\def\url{url}
	\def\arg{#1}
	\ifx\titre\arg
		\newcommand\maintitle{#2}
		\fancyhead[L]{#2}
		{\Large\sffamily \MakeUppercase{#2}}
		\vspace{1mm}\textcolor{mcvert}{\hrule}
	\fi 
	\ifx\url\arg
		\fancyfoot[L]{\href{https://www.maths-cours.fr#2}{\black \footnotesize{https://www.maths-cours.fr#2}}}
	\fi 
}


\newcommand\TitreC[1]{    		% Titre centré
     \needspace{3\baselineskip}
     \begin{center}\textbf{#1}\end{center}
}

\newcommand\newpar{    		% paragraphe
     \par
}

\newcommand\nosp {    		% commande vide (pas d'espace)
}
\newcommand{\id}[1]{} %ignore

\newcommand\boite[2]{				% Boite simple sans titre
	\vspace{5mm}
	\setlength{\fboxrule}{0.2mm}
	\setlength{\fboxsep}{5mm}	
	\fcolorbox{#1}{#1!3}{\makebox[\linewidth-2\fboxrule-2\fboxsep]{
  		\begin{minipage}[t]{\linewidth-2\fboxrule-4\fboxsep}\setlength{\parskip}{3mm}
  			 #2
  		\end{minipage}
	}}
	\vspace{5mm}
}

\newcommand\CBox[4]{				% Boites
	\vspace{5mm}
	\setlength{\fboxrule}{0.2mm}
	\setlength{\fboxsep}{5mm}
	
	\fcolorbox{#1}{#1!3}{\makebox[\linewidth-2\fboxrule-2\fboxsep]{
		\begin{minipage}[t]{1cm}\setlength{\parskip}{3mm}
	  		\textcolor{#1}{\LARGE{#2}}    
 	 	\end{minipage}  
  		\begin{minipage}[t]{\linewidth-2\fboxrule-4\fboxsep}\setlength{\parskip}{3mm}
			\raisebox{1.2mm}{\normalsize\sffamily{\textcolor{#1}{#3}}}						
  			 #4
  		\end{minipage}
	}}
	\vspace{5mm}
}

\newcommand\cadre[3]{				% Boites convertible html
	\par
	\vspace{2mm}
	\setlength{\fboxrule}{0.1mm}
	\setlength{\fboxsep}{5mm}
	\fcolorbox{#1}{white}{\makebox[\linewidth-2\fboxrule-2\fboxsep]{
  		\begin{minipage}[t]{\linewidth-2\fboxrule-4\fboxsep}\setlength{\parskip}{3mm}
			\raisebox{-2.5mm}{\sffamily \small{\textcolor{#1}{\MakeUppercase{#2}}}}		
			\par		
  			 #3
 	 		\end{minipage}
	}}
		\vspace{2mm}
	\par
}

\newcommand\bloc[3]{				% Boites convertible html sans bordure
     \needspace{2\baselineskip}
     {\sffamily \small{\textcolor{#1}{\MakeUppercase{#2}}}}    
		\par		
  			 #3
		\par
}

\newcommand\CHelp[1]{
     \CBox{Plum}{\faInfoCircle}{À RETENIR}{#1}
}

\newcommand\CUp[1]{
     \CBox{NavyBlue}{\faThumbsOUp}{EN PRATIQUE}{#1}
}

\newcommand\CInfo[1]{
     \CBox{Sepia}{\faArrowCircleRight}{REMARQUE}{#1}
}

\newcommand\CRedac[1]{
     \CBox{PineGreen}{\faEdit}{BIEN R\'EDIGER}{#1}
}

\newcommand\CError[1]{
     \CBox{Red}{\faExclamationTriangle}{ATTENTION}{#1}
}

\newcommand\TitreExo[2]{
\needspace{4\baselineskip}
 {\sffamily\large EXERCICE #1\ (\emph{#2 points})}
\vspace{5mm}
}

\newcommand\img[2]{
          \includegraphics[width=#2\paperwidth]{\imgdir#1}
}

\newcommand\imgsvg[2]{
       \begin{center}   \includegraphics[width=#2\paperwidth]{\imgsvgdir#1} \end{center}
}


\newcommand\Lien[2]{
     \href{#1}{#2 \tiny \faExternalLink}
}
\newcommand\mcLien[2]{
     \href{https~://www.maths-cours.fr/#1}{#2 \tiny \faExternalLink}
}

\newcommand{\euro}{\eurologo{}}

%================================================================================================================================
%
% Macros - Environement
%
%================================================================================================================================

\newenvironment{tex}{ %
}
{%
}

\newenvironment{indente}{ %
	\setlength\parindent{10mm}
}

{
	\setlength\parindent{0mm}
}

\newenvironment{corrige}{%
     \needspace{3\baselineskip}
     \medskip
     \textbf{\textsc{Corrigé}}
     \medskip
}
{
}

\newenvironment{extern}{%
     \begin{center}
     }
     {
     \end{center}
}

\NewEnviron{code}{%
	\par
     \boite{gray}{\texttt{%
     \BODY
     }}
     \par
}

\newenvironment{vbloc}{% boite sans cadre empeche saut de page
     \begin{minipage}[t]{\linewidth}
     }
     {
     \end{minipage}
}
\NewEnviron{h2}{%
    \needspace{3\baselineskip}
    \vspace{0.6cm}
	\noindent \MakeUppercase{\sffamily \large \BODY}
	\vspace{1mm}\textcolor{mcgris}{\hrule}\vspace{0.4cm}
	\par
}{}

\NewEnviron{h3}{%
    \needspace{3\baselineskip}
	\vspace{5mm}
	\textsc{\BODY}
	\par
}

\NewEnviron{margeneg}{ %
\begin{addmargin}[-1cm]{0cm}
\BODY
\end{addmargin}
}

\NewEnviron{html}{%
}

\begin{document}
\begin{h2}Exercice 1 (5 points)\end{h2}
\textbf{Commun à tous les candidats}
\medskip
\emph{Cet exercice est un QCM (questionnaire à choix multiples). Pour chacune des questions posées, une seule des trois réponses est exacte. Recopier le numéro de la question et la réponse exacte. Aucune justification n'est demandée. Une réponse exacte rapporte 1~point, une réponse fausse ou l'absence de réponse ne rapporte ni n'enlève de point. Une réponse multiple ne rapporte aucun point.}
\medskip
On considère la fonction $f$ définie sur l'intervalle [0,5~;~5] par~:
\par
\[f(x) = \dfrac{5 + 5\ln x}{x}.\]
\par
Sa représentation graphique est la courbe $\mathcal{C}$ donnée ci-dessous dans un repère d'origine O.
\par
On admet que le point A placé sur le graphique est le seul point d'inflexion de la courbe $\mathcal{C}$ sur l'intervalle [0,5~;~5].
\par
On note B le point de cette courbe d'abscisse e.
\par
On admet que la fonction $f$ est deux fois dérivable sur cet intervalle.
\par
On rappelle que $f'$ désigne la fonction dérivée de la fonction $f$ et $f''$ sa fonction dérivée seconde.
\begin{center}
     \begin{extern}%alt="courbe représentative fonction"
          \definecolor{darkgreen}{rgb}{0.,0.4,0.}
          \psset{xunit=2cm,yunit=1cm,arrowsize=2pt 3}
          \begin{pspicture}(-0.5,-1)(6,7)
               \psgrid[unit=1cm,gridwidth=0.1pt,gridcolor=darkgray,subgriddiv=0,gridlabels=0](0,0)(-1,-1)(12,7)
               \psaxes[linewidth=1pt,ticksize=-2pt 2pt,Dx=1,Dy=1](0,0)(0,0)(5.5,6.5)[$x$,-45][$y$,45]
               \psplot[plotpoints=500,linewidth=1.25pt, linecolor=darkgreen]{0.5}{5}{5 5 x ln mul add x div}
               \uput[dl](-0.05,-0.15){O}
               \psdots[linecolor=black](1.649,4.549)(2.718,3.679)
               \psline[linecolor=black,linestyle=dashed](2.718,3.679)(2.718,0)
               \uput[dl](1.649,4.549){\black A} \uput[dl](2.718,3.679){\black B}
               \uput[d](2.718,0){\black e}
               \uput[u](4.5,2.9){\color{darkgreen} \large $\mathcal{C}$}
          \end{pspicture}
     \end{extern}
\end{center}
On admet que pour tout $x$ de l'intervalle [0,5~;~5] on a~:
\par
$f'(x) = \dfrac{- 5\ln x}{x^2}$ $\qquad\qquad$ $f''(x) = \dfrac{10\ln x - 5}{x^3}$.
\medskip
\begin{enumerate}
     \item La fonction $f'$ est~:
     \begin{enumerate}[label=\alph*.]
          \item positive ou nulle sur l'intervalle [0,5~;~5]
          \item négative ou nulle sur l'intervalle [1~;~5]
          \item négative ou nulle sur l'intervalle [0,5~;~1]
     \end{enumerate}
     \medskip
     \item  Le coefficient directeur de la tangente à la courbe $\mathcal{C}$ au point B est égal à~:
     \begin{tabularx}{\linewidth}{*{3}X}  %class="noborder"
          \textbf{a.~~}$- \dfrac{5}{\text{e}^2}$&\textbf{b.~~}$\dfrac{10}{\text{e}}$&\textbf{c.~~}$ \dfrac{5}{\text{e}^3}$
     \end{tabularx}
     \medskip
     \item  La fonction $f'$ est~:
     \begin{enumerate}[label=\alph*.]
          \item croissante sur l'intervalle [0,5~;~1]
          \item décroissante sur l'intervalle [1~;~5]
          \item croissante sur l'intervalle [2~;~5]
     \end{enumerate}
     \medskip
     \item  La valeur exacte de l'abscisse du point A de la courbe $\mathcal{C}$ est égale à~:
     \begin{tabularx}{\linewidth}{*{3}X} %class="noborder"
          \textbf{a.~~} 1,65 &\textbf{b.~~} 1,6 &\textbf{c.~~} $\text{e}^{0,5}$
     \end{tabularx}
     \medskip
     \item  On note $\mathcal{A}$ l'aire, mesurée en unités d'aire, du domaine plan délimité par la courbe $\mathcal{C}$, l'axe des abscisses et les droites d'équation $x = 1$ et $x = 4$. Cette aire vérifie~:
     \begin{enumerate}[label=\alph*.]
          \item $20 \leqslant \mathcal{A} \leqslant 30$
          \item $10\leqslant \mathcal{A} \leqslant 15$
          \item $5 \leqslant \mathcal{A} \leqslant 8$
     \end{enumerate}
\end{enumerate}
\begin{corrige}
     \begin{enumerate}
          \item
          \textbf{Réponse correcte~:\quad b.}
          \par
          On peut utiliser le graphique ou la formule pour répondre à cette question.
          \begin{itemize}
               \item
               \textbf{\`A l'aide du graphique~:}
               \par
               On voit que la fonction $f$ est décroissante sur l'intervalle $[1~;~5]$. Sa dérivée $f'$ est donc négative ou nulle sur cet intervalle.
               \item
               \textbf{\`A partir de la formule~:}
               \par
               $f'(x) = \dfrac{- 5\ln x}{x^2}$ $\qquad\qquad$
               \par
               Le dénominateur est strictement positif sur l'intervalle $[1~;~5]$.
               \par
               Pour $x \geqslant 1$, $\ln x \geqslant \ln 1 = 0$, donc le numérateur est négatif ou nul.
               \par
               $f'$ est donc négative ou nulle sur l'intervalle $[1~;~5]$.
          \end{itemize}
          \item
          \textbf{Réponse correcte~:\quad a.}
          \par
          Le coefficient directeur de la tangente en $B$ d'abscisse $\text{e}$ à la courbe $\mathscr{C}$ est égal à $f'(e)$.
          \par
          $f'(e)== \dfrac{- 5\ln \text{e}}{\text{e}^2} = - \dfrac{5}{\text{e}^2}$
          \item
          \textbf{Réponse correcte~:\quad\textbf{ c.}}
          \par
          Là encore, on peut utiliser le graphique ou la formule pour répondre à cette question.
          \begin{itemize}
               \item
               \textbf{\`A l'aide du graphique~:}
               \par
               La fonction $f'$ est croissante sur l'intervalle $[2~;~5]$ si et seulement si la courbe $\mathscr{C}$ est convexe sur cet intervalle.
               \par
               On vérifie sur le graphique que c'est bien le cas.
               \item
               \textbf{\`A partir de la formule~:}
               \par
               La fonction $f'$ est croissante sur l'intervalle $[2~;~5]$ si et seulement si sa fonction dérivée $f''$ est positive ou nulle sur cet intervalle.
               \par
               Or pour $x \geqslant 2$~:
               \par
               $x \geqslant 2 \Leftrightarrow \ln x \geqslant \ln 2$ (car la fonction $\ln$ est croissante sur $]0~;~+oo[$)\\
               $\phantom{x \geqslant 2} \Leftrightarrow 10\ln x \geqslant 10\ln 2$\\
               $\phantom{x \geqslant 2} \Leftrightarrow 10\ln x-5 \geqslant 10\ln 2-5 $\\
               \par
               Or,  $10\ln 2-5 ~=1,9$ est positif donc le numérateur de $f''$ est positif sur l'intervalle $[2~;~5]$. Comme son dénominateur est également strictement positif sur cet intervalle, $f''$ est positive sur $[2~;~5]$.
               \par
               La fonction $f'$ est donc croissante sur l'intervalle $[2~;~5]$.
          \end{itemize}
          \item
          \textbf{Réponse correcte~:\quad c.}
          \par
          $A$ est l'unique point d'inflexion de la courbe $\mathscr{C}$. Son abscisse est donc la solution de l'équation $f''(x)=0$.
          \par
          $f''(x)=0 \Leftrightarrow  \dfrac{10\ln x - 5}{x^3}=0$\\
          $\phantom{f''(x)=0} \Leftrightarrow 10\ln x - 5=0$\\
          $\phantom{f''(x)=0} \Leftrightarrow \ln x =\dfrac{5}{10}=0,5$\\
          $\phantom{f''(x)=0} \Leftrightarrow x =\text{e}^{0,5}$\\
          \item
          Réponse correcte~:\quad\textbf{ b.}
          \par
          On compte le nombre de carreaux de la surface colorée ci-dessous.
          \begin{center}
               \begin{extern}%alt="aire sous la courbe"
                    \definecolor{darkgreen}{rgb}{0.,0.4,0.}
                    \psset{xunit=2cm,yunit=1cm,arrowsize=2pt 3}
                    \begin{pspicture}(-0.5,-1)(6,7)
                         \psgrid[unit=1cm,gridwidth=0.1pt,gridcolor=darkgray,subgriddiv=0,gridlabels=0](0,0)(-1,-1)(12,7)
                         \psaxes[linewidth=1pt,ticksize=-2pt 2pt,Dx=1,Dy=1](0,0)(0,0)(5.5,6.5)[$x$,-45][$y$,45]
                         \pscustom[fillstyle=solid,fillcolor=darkgreen,opacity=0.1]{
                              \psplot[plotpoints=500,linewidth=1.25pt, linecolor=darkgreen]{1}{4}{5 5 x ln mul add x div}
                         \psline[linewidth=0.75pt,linecolor=darkgreen](4,0)(1,0)(1,5)}
                         \psplot[plotpoints=500,linewidth=1.25pt, linecolor=darkgreen]{0.5}{5}{5 5 x ln mul add x div}
                         \uput[dl](-0.05,-0.15){O}
                         \psdots[linecolor=black](1.649,4.549)(2.718,3.679)
                         \psline[linecolor=black,linestyle=dashed](2.718,3.679)(2.718,0)
                         \uput[dl](1.649,4.549){\black A} \uput[dl](2.718,3.679){\black B}
                         \uput[d](2.718,0){\black e}
                         \uput[u](4.5,2.9){\color{darkgreen} \large $\mathcal{C}$}
                    \end{pspicture}
               \end{extern}
          \end{center}
          On trouve approximativement 24 carreaux.
          \par
          Chaque carreau a une aire de 0,5 unité d'aire.
          \par
          L'aire cherchée est donc approximativement égale à 12.
     \end{enumerate}
\end{corrige}

\end{document}