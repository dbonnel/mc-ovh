\documentclass[a4paper]{article}

%================================================================================================================================
%
% Packages
%
%================================================================================================================================

\usepackage[T1]{fontenc} 	% pour caractères accentués
\usepackage[utf8]{inputenc}  % encodage utf8
\usepackage[french]{babel}	% langue : français
\usepackage{fourier}			% caractères plus lisibles
\usepackage[dvipsnames]{xcolor} % couleurs
\usepackage{fancyhdr}		% réglage header footer
\usepackage{needspace}		% empêcher sauts de page mal placés
\usepackage{graphicx}		% pour inclure des graphiques
\usepackage{enumitem,cprotect}		% personnalise les listes d'items (nécessaire pour ol, al ...)
\usepackage{hyperref}		% Liens hypertexte
\usepackage{pstricks,pst-all,pst-node,pstricks-add,pst-math,pst-plot,pst-tree,pst-eucl} % pstricks
\usepackage[a4paper,includeheadfoot,top=2cm,left=3cm, bottom=2cm,right=3cm]{geometry} % marges etc.
\usepackage{comment}			% commentaires multilignes
\usepackage{amsmath,environ} % maths (matrices, etc.)
\usepackage{amssymb,makeidx}
\usepackage{bm}				% bold maths
\usepackage{tabularx}		% tableaux
\usepackage{colortbl}		% tableaux en couleur
\usepackage{fontawesome}		% Fontawesome
\usepackage{environ}			% environment with command
\usepackage{fp}				% calculs pour ps-tricks
\usepackage{multido}			% pour ps tricks
\usepackage[np]{numprint}	% formattage nombre
\usepackage{tikz,tkz-tab} 			% package principal TikZ
\usepackage{pgfplots}   % axes
\usepackage{mathrsfs}    % cursives
\usepackage{calc}			% calcul taille boites
\usepackage[scaled=0.875]{helvet} % font sans serif
\usepackage{svg} % svg
\usepackage{scrextend} % local margin
\usepackage{scratch} %scratch
\usepackage{multicol} % colonnes
%\usepackage{infix-RPN,pst-func} % formule en notation polanaise inversée
\usepackage{listings}

%================================================================================================================================
%
% Réglages de base
%
%================================================================================================================================

\lstset{
language=Python,   % R code
literate=
{á}{{\'a}}1
{à}{{\`a}}1
{ã}{{\~a}}1
{é}{{\'e}}1
{è}{{\`e}}1
{ê}{{\^e}}1
{í}{{\'i}}1
{ó}{{\'o}}1
{õ}{{\~o}}1
{ú}{{\'u}}1
{ü}{{\"u}}1
{ç}{{\c{c}}}1
{~}{{ }}1
}


\definecolor{codegreen}{rgb}{0,0.6,0}
\definecolor{codegray}{rgb}{0.5,0.5,0.5}
\definecolor{codepurple}{rgb}{0.58,0,0.82}
\definecolor{backcolour}{rgb}{0.95,0.95,0.92}

\lstdefinestyle{mystyle}{
    backgroundcolor=\color{backcolour},   
    commentstyle=\color{codegreen},
    keywordstyle=\color{magenta},
    numberstyle=\tiny\color{codegray},
    stringstyle=\color{codepurple},
    basicstyle=\ttfamily\footnotesize,
    breakatwhitespace=false,         
    breaklines=true,                 
    captionpos=b,                    
    keepspaces=true,                 
    numbers=left,                    
xleftmargin=2em,
framexleftmargin=2em,            
    showspaces=false,                
    showstringspaces=false,
    showtabs=false,                  
    tabsize=2,
    upquote=true
}

\lstset{style=mystyle}


\lstset{style=mystyle}
\newcommand{\imgdir}{C:/laragon/www/newmc/assets/imgsvg/}
\newcommand{\imgsvgdir}{C:/laragon/www/newmc/assets/imgsvg/}

\definecolor{mcgris}{RGB}{220, 220, 220}% ancien~; pour compatibilité
\definecolor{mcbleu}{RGB}{52, 152, 219}
\definecolor{mcvert}{RGB}{125, 194, 70}
\definecolor{mcmauve}{RGB}{154, 0, 215}
\definecolor{mcorange}{RGB}{255, 96, 0}
\definecolor{mcturquoise}{RGB}{0, 153, 153}
\definecolor{mcrouge}{RGB}{255, 0, 0}
\definecolor{mclightvert}{RGB}{205, 234, 190}

\definecolor{gris}{RGB}{220, 220, 220}
\definecolor{bleu}{RGB}{52, 152, 219}
\definecolor{vert}{RGB}{125, 194, 70}
\definecolor{mauve}{RGB}{154, 0, 215}
\definecolor{orange}{RGB}{255, 96, 0}
\definecolor{turquoise}{RGB}{0, 153, 153}
\definecolor{rouge}{RGB}{255, 0, 0}
\definecolor{lightvert}{RGB}{205, 234, 190}
\setitemize[0]{label=\color{lightvert}  $\bullet$}

\pagestyle{fancy}
\renewcommand{\headrulewidth}{0.2pt}
\fancyhead[L]{maths-cours.fr}
\fancyhead[R]{\thepage}
\renewcommand{\footrulewidth}{0.2pt}
\fancyfoot[C]{}

\newcolumntype{C}{>{\centering\arraybackslash}X}
\newcolumntype{s}{>{\hsize=.35\hsize\arraybackslash}X}

\setlength{\parindent}{0pt}		 
\setlength{\parskip}{3mm}
\setlength{\headheight}{1cm}

\def\ebook{ebook}
\def\book{book}
\def\web{web}
\def\type{web}

\newcommand{\vect}[1]{\overrightarrow{\,\mathstrut#1\,}}

\def\Oij{$\left(\text{O}~;~\vect{\imath},~\vect{\jmath}\right)$}
\def\Oijk{$\left(\text{O}~;~\vect{\imath},~\vect{\jmath},~\vect{k}\right)$}
\def\Ouv{$\left(\text{O}~;~\vect{u},~\vect{v}\right)$}

\hypersetup{breaklinks=true, colorlinks = true, linkcolor = OliveGreen, urlcolor = OliveGreen, citecolor = OliveGreen, pdfauthor={Didier BONNEL - https://www.maths-cours.fr} } % supprime les bordures autour des liens

\renewcommand{\arg}[0]{\text{arg}}

\everymath{\displaystyle}

%================================================================================================================================
%
% Macros - Commandes
%
%================================================================================================================================

\newcommand\meta[2]{    			% Utilisé pour créer le post HTML.
	\def\titre{titre}
	\def\url{url}
	\def\arg{#1}
	\ifx\titre\arg
		\newcommand\maintitle{#2}
		\fancyhead[L]{#2}
		{\Large\sffamily \MakeUppercase{#2}}
		\vspace{1mm}\textcolor{mcvert}{\hrule}
	\fi 
	\ifx\url\arg
		\fancyfoot[L]{\href{https://www.maths-cours.fr#2}{\black \footnotesize{https://www.maths-cours.fr#2}}}
	\fi 
}


\newcommand\TitreC[1]{    		% Titre centré
     \needspace{3\baselineskip}
     \begin{center}\textbf{#1}\end{center}
}

\newcommand\newpar{    		% paragraphe
     \par
}

\newcommand\nosp {    		% commande vide (pas d'espace)
}
\newcommand{\id}[1]{} %ignore

\newcommand\boite[2]{				% Boite simple sans titre
	\vspace{5mm}
	\setlength{\fboxrule}{0.2mm}
	\setlength{\fboxsep}{5mm}	
	\fcolorbox{#1}{#1!3}{\makebox[\linewidth-2\fboxrule-2\fboxsep]{
  		\begin{minipage}[t]{\linewidth-2\fboxrule-4\fboxsep}\setlength{\parskip}{3mm}
  			 #2
  		\end{minipage}
	}}
	\vspace{5mm}
}

\newcommand\CBox[4]{				% Boites
	\vspace{5mm}
	\setlength{\fboxrule}{0.2mm}
	\setlength{\fboxsep}{5mm}
	
	\fcolorbox{#1}{#1!3}{\makebox[\linewidth-2\fboxrule-2\fboxsep]{
		\begin{minipage}[t]{1cm}\setlength{\parskip}{3mm}
	  		\textcolor{#1}{\LARGE{#2}}    
 	 	\end{minipage}  
  		\begin{minipage}[t]{\linewidth-2\fboxrule-4\fboxsep}\setlength{\parskip}{3mm}
			\raisebox{1.2mm}{\normalsize\sffamily{\textcolor{#1}{#3}}}						
  			 #4
  		\end{minipage}
	}}
	\vspace{5mm}
}

\newcommand\cadre[3]{				% Boites convertible html
	\par
	\vspace{2mm}
	\setlength{\fboxrule}{0.1mm}
	\setlength{\fboxsep}{5mm}
	\fcolorbox{#1}{white}{\makebox[\linewidth-2\fboxrule-2\fboxsep]{
  		\begin{minipage}[t]{\linewidth-2\fboxrule-4\fboxsep}\setlength{\parskip}{3mm}
			\raisebox{-2.5mm}{\sffamily \small{\textcolor{#1}{\MakeUppercase{#2}}}}		
			\par		
  			 #3
 	 		\end{minipage}
	}}
		\vspace{2mm}
	\par
}

\newcommand\bloc[3]{				% Boites convertible html sans bordure
     \needspace{2\baselineskip}
     {\sffamily \small{\textcolor{#1}{\MakeUppercase{#2}}}}    
		\par		
  			 #3
		\par
}

\newcommand\CHelp[1]{
     \CBox{Plum}{\faInfoCircle}{À RETENIR}{#1}
}

\newcommand\CUp[1]{
     \CBox{NavyBlue}{\faThumbsOUp}{EN PRATIQUE}{#1}
}

\newcommand\CInfo[1]{
     \CBox{Sepia}{\faArrowCircleRight}{REMARQUE}{#1}
}

\newcommand\CRedac[1]{
     \CBox{PineGreen}{\faEdit}{BIEN R\'EDIGER}{#1}
}

\newcommand\CError[1]{
     \CBox{Red}{\faExclamationTriangle}{ATTENTION}{#1}
}

\newcommand\TitreExo[2]{
\needspace{4\baselineskip}
 {\sffamily\large EXERCICE #1\ (\emph{#2 points})}
\vspace{5mm}
}

\newcommand\img[2]{
          \includegraphics[width=#2\paperwidth]{\imgdir#1}
}

\newcommand\imgsvg[2]{
       \begin{center}   \includegraphics[width=#2\paperwidth]{\imgsvgdir#1} \end{center}
}


\newcommand\Lien[2]{
     \href{#1}{#2 \tiny \faExternalLink}
}
\newcommand\mcLien[2]{
     \href{https~://www.maths-cours.fr/#1}{#2 \tiny \faExternalLink}
}

\newcommand{\euro}{\eurologo{}}

%================================================================================================================================
%
% Macros - Environement
%
%================================================================================================================================

\newenvironment{tex}{ %
}
{%
}

\newenvironment{indente}{ %
	\setlength\parindent{10mm}
}

{
	\setlength\parindent{0mm}
}

\newenvironment{corrige}{%
     \needspace{3\baselineskip}
     \medskip
     \textbf{\textsc{Corrigé}}
     \medskip
}
{
}

\newenvironment{extern}{%
     \begin{center}
     }
     {
     \end{center}
}

\NewEnviron{code}{%
	\par
     \boite{gray}{\texttt{%
     \BODY
     }}
     \par
}

\newenvironment{vbloc}{% boite sans cadre empeche saut de page
     \begin{minipage}[t]{\linewidth}
     }
     {
     \end{minipage}
}
\NewEnviron{h2}{%
    \needspace{3\baselineskip}
    \vspace{0.6cm}
	\noindent \MakeUppercase{\sffamily \large \BODY}
	\vspace{1mm}\textcolor{mcgris}{\hrule}\vspace{0.4cm}
	\par
}{}

\NewEnviron{h3}{%
    \needspace{3\baselineskip}
	\vspace{5mm}
	\textsc{\BODY}
	\par
}

\NewEnviron{margeneg}{ %
\begin{addmargin}[-1cm]{0cm}
\BODY
\end{addmargin}
}

\NewEnviron{html}{%
}

\begin{document}
$ ABC $ est un triangle quelconque tel que $ AB = 7 $cm, $ AC= 5 $cm et $ BC = 4 $cm.
\\
$ M$ est un point du segment $ \left[ BC \right] . $
\\
La parallèle à la droite $ \left( AB \right) $ passant par $ M $ coupe le côté $ [AC] $ en $ N.$
\\
La parallèle à la droite $ \left( AC \right) $ passant par $ M $ coupe le côté $ [AB] $ en $ P.$
\begin{center}
     \begin{extern}%width="500" alt="Thalès et losange"
          \newrgbcolor{grey}{0.2 0.2 0.2}
          \psset{xunit=1.0cm,yunit=1.0cm,algebraic=true,dimen=middle,dotstyle=o,dotsize=5pt 0,linewidth=1.6pt,arrowsize=3pt 2,arrowinset=0.25}
          \begin{pspicture*}(0.,0.)(9.,5.)
               \psline[linewidth=0.4pt,linecolor=grey](1.,1.)(8.,1.)
               \psline[linewidth=0.4pt,linecolor=grey](8.,1.)(5.142857142857143,3.79941684889506)
               \psline[linewidth=0.4pt,linecolor=grey](5.142857142857143,3.79941684889506)(1.,1.)
               \psline[linewidth=0.4pt,linecolor=red](3.6425664347163558,2.785638448678848)(6.1775403898507895,2.7856384486788484)
               \psline[linewidth=0.4pt,linecolor=red](6.1775403898507895,2.7856384486788484)(3.5349739551344332,1.)
               \psline[linewidth=0.4pt,linecolor=red](3.5349739551344332,1.)(1.,1.)
               \psline[linewidth=0.4pt,linecolor=red](1.,1.)(3.6425664347163558,2.785638448678848)
               \begin{scriptsize}
                    \psdots[dotsize=2pt 0,dotstyle=*,linecolor=grey](1.,1.)
                    \rput[bl](0.855296988302761,0.7186720182506175){\grey{$A$}}
                    \psdots[dotsize=2pt 0,dotstyle=*,linecolor=grey](8.,1.)
                    \rput[bl](8.06583327519023,0.7186720182506175){\grey{$B$}}
                    \psdots[dotsize=2pt 0,dotstyle=*,linecolor=grey](5.142857142857143,3.79941684889506)
                    \rput[bl](5.1767631804440715,3.835885016525611){\grey{$C$}}
                    \psdots[dotsize=2pt 0,dotstyle=*](6.1775403898507895,2.7856384486788484)
                    \rput[bl](6.212620245227008,2.822036562804175){$M$}
                    \psdots[dotsize=2pt 0,dotstyle=*,linecolor=grey](3.6425664347163558,2.785638448678848)
                    \rput[bl](3.5096807168090343,2.854479713323261){\grey{$N$}}
                    \psdots[dotsize=2pt 0,dotstyle=*,linecolor=grey](3.5349739551344332,1.)
                    \rput[bl](3.533958616764884,0.7186720182506175){\grey{$P$}}
               \end{scriptsize}
          \end{pspicture*}
     \end{extern}
\end{center}
\begin{enumerate}
     \item
     À quelle distance du point $ C $ faut-il placer le point $ M $ pour que le quadrilatère $ APMN $ soit un parallélogramme~?
     \item
     Sans utiliser le résultat de la question précédente, construire le point $ M$ à l'aide d'une règle non graduée et d'un compas.
\end{enumerate}
\begin{corrige}
     \begin{enumerate}
          \item
          On sait déjà que $ APMN $ est un parallélogramme car les droites $ \left( MN \right) $ et $ \left( AP \right) $ sont parallèles ainsi que les droites $ \left( MP \right) et \left( AN \right).$
          \par
          Pour que ce soit un losange il faut et il suffit que $ MN = MP. $
          \par
          Calculons $ MP $ puis $ MN $ en utilisant le théorème de Thalès~:
          \begin{itemize}
               \item
               \textbf{Calcul de $ MP$ }
               \par
               Posons $ x = MC. $
               \par
               On a alors~:
               \\
               $ BM = BC - MC = 4 - x $
               \par
               Les droites $ \left( MP \right) $ et $ \left( AC \right) $ sont parallèles~; les points $ B, M, C $ et les points $ B, P, A $ sont alignés.
               \par
               Donc, d'après le théorème de Thalès~:
               \begin{center}
                    $ \frac{ BM }{ BC } = \frac{ MP }{ AC } = \frac{ BP }{ AB } $
               \end{center}
               La première égalité donne~:
               \par
               $ \frac{ 4-x }{ 4 } = \frac{ MP }{ 5 } $
               \par
               donc, avec un produit en croix~:
               \par
               $ 4 MP = 5 \left( 4-x \right) $
               \par
               $ MP = \frac{ 5 \left( 4-x \right) }{ 4 }. $
               \\
               \item
               \textbf{Calcul de MN}
               \par
               De même, les droites $ \left( MN \right) $ et $ \left( AB \right) $ sont parallèles et les points $ C, M, B$ et $ C, N, A $ sont alignés.
               \par
               Par conséquent~:
               \begin{center}
                    $ \frac{ CM }{ BC } = \frac{ MN }{ AB } = \frac{ CN }{ CA } $
               \end{center}
               L'égalité des deux premiers quotients équivaut à~:
               \par
               $ \frac{ x }{ 4 } = \frac{ MN }{ 7 } $
               \par
               soit~:
               $ 4 MN = 7x $
               \\
               $ MN = \frac{ 7x }{ 4 }. $
               \item
               \textbf{Conclusion }
               \par
               $ APMN $ est donc un losange si et seulement si~:
               \par
               $ MP = MN $
               \par
               $ \frac{ 5(4-x) }{ 4 } = \frac{ 7x }{ 4 } $
               \par
               $ 5 \left( 4 - x \right) = 7x $
               \\
               $ 20 - 5x = 7x $
               \\
               $ 20 = 7x + 5x $
               \\
               $ 12x = 20$
               \par
               $ x = \frac{ 5 }{ 3 } $
               \par
               Il faut placer le point $ M $ à $ \frac{ 5 }{ 3 } $cm ( $    \approx 1,67  $ cm) de $C$ pour que le quadrilatère $ APMN $ soit un parallélogramme.
          \end{itemize}
          \item
          Les diagonales d'un losange sont des axes de symétrie de ce losange.
          \par
          Donc, si $ APMN $ est un losange, la droite $ \left( AM \right) $ est un axe de symétrie donc une bissectrice de l'angle $ \widehat{ PAN } $ qui est aussi l'angle $ \widehat{ BAC }.$
          \par
          Pour placer le point $M$, il suffit donc de construire au compas la bissectrice de l'angle $ \widehat{ BAC }. $
          \par
          $M$ est alors le point d'intersection de cette bissectrice avec le côté $ \left[ BC \right] $~:
          \begin{center}
               \begin{extern}%width="500" alt=""
                    \newrgbcolor{grey}{0.2 0.2 0.2}
                    \psset{xunit=1.0cm,yunit=1.0cm,algebraic=true,dimen=middle,dotstyle=o,dotsize=5pt 0,linewidth=1.6pt,arrowsize=3pt 2,arrowinset=0.25}
                    \begin{pspicture*}(0.,0.)(9.,5.)
                         \psline[linewidth=0.4pt,linecolor=grey](1.,1.)(8.,1.)
                         \psline[linewidth=0.4pt,linecolor=grey](8.,1.)(5.142857142857143,3.79941684889506)
                         \psline[linewidth=0.4pt,linecolor=grey](5.142857142857143,3.79941684889506)(1.,1.)
                         \psline[linewidth=0.4pt,linecolor=red](3.382592586423977,2.609968579772903)(6.356832699017947,2.609968579772903)
                         \psline[linewidth=0.4pt,linecolor=red](6.356832699017947,2.609968579772903)(3.97424011259397,1.)
                         \psline[linewidth=0.4pt,linecolor=red](3.97424011259397,1.)(1.,1.)
                         \psline[linewidth=0.4pt,linecolor=red](1.,1.)(3.382592586423977,2.609968579772903)
                         \parametricplot[linewidth=0.4pt,linecolor=grey]{0.4430230091369528}{0.7809029873351472}{1.*1.500369888604726*cos(t)+0.*1.500369888604726*sin(t)+1.|0.*1.500369888604726*cos(t)+1.*1.500369888604726*sin(t)+1.}
                         \parametricplot[linewidth=0.4pt,linecolor=grey]{-0.18351779158694992}{0.18839081772518879}{1.*1.5003698886047259*cos(t)+0.*1.5003698886047259*sin(t)+1.|0.*1.5003698886047259*cos(t)+1.*1.5003698886047259*sin(t)+1.}
                         \parametricplot[linewidth=0.4pt,linecolor=grey]{-0.14893404425129297}{0.16320932683782904}{1.*1.5086156333068808*cos(t)+0.*1.5086156333068808*sin(t)+2.2499958104542728|0.*1.5086156333068808*cos(t)+1.*1.5086156333068808*sin(t)+1.8446488044771545}
                         \parametricplot[linewidth=0.4pt,linecolor=grey]{0.4243963995945869}{0.7436288249084765}{1.*1.506194185300565*cos(t)+0.*1.506194185300565*sin(t)+2.5061941853005654|0.*1.506194185300565*cos(t)+1.*1.506194185300565*sin(t)+1.}
                         \psplot[linewidth=0.4pt,linecolor=grey]{0.}{9.}{(--3.7468641192450445--1.6099685797729029*x)/5.356832699017947}
                         \begin{scriptsize}
                              \psdots[dotsize=2pt 0,dotstyle=*,linecolor=grey](1.,1.)
                              \rput[bl](0.855296988302761,0.7186720182506175){\grey{$A$}}
                              \psdots[dotsize=2pt 0,dotstyle=*,linecolor=grey](8.,1.)
                              \rput[bl](8.06583327519023,0.7186720182506175){\grey{$B$}}
                              \psdots[dotsize=2pt 0,dotstyle=*,linecolor=grey](5.142857142857143,3.79941684889506)
                              \rput[bl](5.1767631804440715,3.835885016525611){\grey{$C$}}
                              \psdots[dotsize=2pt 0,dotstyle=*](6.356832699017947,2.609968579772903)
                              \rput[bl](6.4,2.68){$M$}
                              \psdots[dotsize=2pt 0,dotstyle=*,linecolor=grey](3.382592586423977,2.609968579772903)
                              \rput[bl](3.2507164506133,2.676042385468288){\grey{$N$}}
                              \psdots[dotsize=2pt 0,dotstyle=*,linecolor=grey](3.97424011259397,1.)
                              \rput[bl](3.9709608159701855,0.8186720182506175){\grey{$P$}}
                         \end{scriptsize}
                    \end{pspicture*}
               \end{extern}
          \end{center}
     \end{enumerate}
\end{corrige}

\end{document}