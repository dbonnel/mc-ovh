\documentclass[a4paper]{article}

%================================================================================================================================
%
% Packages
%
%================================================================================================================================

\usepackage[T1]{fontenc} 	% pour caractères accentués
\usepackage[utf8]{inputenc}  % encodage utf8
\usepackage[french]{babel}	% langue : français
\usepackage{fourier}			% caractères plus lisibles
\usepackage[dvipsnames]{xcolor} % couleurs
\usepackage{fancyhdr}		% réglage header footer
\usepackage{needspace}		% empêcher sauts de page mal placés
\usepackage{graphicx}		% pour inclure des graphiques
\usepackage{enumitem,cprotect}		% personnalise les listes d'items (nécessaire pour ol, al ...)
\usepackage{hyperref}		% Liens hypertexte
\usepackage{pstricks,pst-all,pst-node,pstricks-add,pst-math,pst-plot,pst-tree,pst-eucl} % pstricks
\usepackage[a4paper,includeheadfoot,top=2cm,left=3cm, bottom=2cm,right=3cm]{geometry} % marges etc.
\usepackage{comment}			% commentaires multilignes
\usepackage{amsmath,environ} % maths (matrices, etc.)
\usepackage{amssymb,makeidx}
\usepackage{bm}				% bold maths
\usepackage{tabularx}		% tableaux
\usepackage{colortbl}		% tableaux en couleur
\usepackage{fontawesome}		% Fontawesome
\usepackage{environ}			% environment with command
\usepackage{fp}				% calculs pour ps-tricks
\usepackage{multido}			% pour ps tricks
\usepackage[np]{numprint}	% formattage nombre
\usepackage{tikz,tkz-tab} 			% package principal TikZ
\usepackage{pgfplots}   % axes
\usepackage{mathrsfs}    % cursives
\usepackage{calc}			% calcul taille boites
\usepackage[scaled=0.875]{helvet} % font sans serif
\usepackage{svg} % svg
\usepackage{scrextend} % local margin
\usepackage{scratch} %scratch
\usepackage{multicol} % colonnes
%\usepackage{infix-RPN,pst-func} % formule en notation polanaise inversée
\usepackage{listings}

%================================================================================================================================
%
% Réglages de base
%
%================================================================================================================================

\lstset{
language=Python,   % R code
literate=
{á}{{\'a}}1
{à}{{\`a}}1
{ã}{{\~a}}1
{é}{{\'e}}1
{è}{{\`e}}1
{ê}{{\^e}}1
{í}{{\'i}}1
{ó}{{\'o}}1
{õ}{{\~o}}1
{ú}{{\'u}}1
{ü}{{\"u}}1
{ç}{{\c{c}}}1
{~}{{ }}1
}


\definecolor{codegreen}{rgb}{0,0.6,0}
\definecolor{codegray}{rgb}{0.5,0.5,0.5}
\definecolor{codepurple}{rgb}{0.58,0,0.82}
\definecolor{backcolour}{rgb}{0.95,0.95,0.92}

\lstdefinestyle{mystyle}{
    backgroundcolor=\color{backcolour},   
    commentstyle=\color{codegreen},
    keywordstyle=\color{magenta},
    numberstyle=\tiny\color{codegray},
    stringstyle=\color{codepurple},
    basicstyle=\ttfamily\footnotesize,
    breakatwhitespace=false,         
    breaklines=true,                 
    captionpos=b,                    
    keepspaces=true,                 
    numbers=left,                    
xleftmargin=2em,
framexleftmargin=2em,            
    showspaces=false,                
    showstringspaces=false,
    showtabs=false,                  
    tabsize=2,
    upquote=true
}

\lstset{style=mystyle}


\lstset{style=mystyle}
\newcommand{\imgdir}{C:/laragon/www/newmc/assets/imgsvg/}
\newcommand{\imgsvgdir}{C:/laragon/www/newmc/assets/imgsvg/}

\definecolor{mcgris}{RGB}{220, 220, 220}% ancien~; pour compatibilité
\definecolor{mcbleu}{RGB}{52, 152, 219}
\definecolor{mcvert}{RGB}{125, 194, 70}
\definecolor{mcmauve}{RGB}{154, 0, 215}
\definecolor{mcorange}{RGB}{255, 96, 0}
\definecolor{mcturquoise}{RGB}{0, 153, 153}
\definecolor{mcrouge}{RGB}{255, 0, 0}
\definecolor{mclightvert}{RGB}{205, 234, 190}

\definecolor{gris}{RGB}{220, 220, 220}
\definecolor{bleu}{RGB}{52, 152, 219}
\definecolor{vert}{RGB}{125, 194, 70}
\definecolor{mauve}{RGB}{154, 0, 215}
\definecolor{orange}{RGB}{255, 96, 0}
\definecolor{turquoise}{RGB}{0, 153, 153}
\definecolor{rouge}{RGB}{255, 0, 0}
\definecolor{lightvert}{RGB}{205, 234, 190}
\setitemize[0]{label=\color{lightvert}  $\bullet$}

\pagestyle{fancy}
\renewcommand{\headrulewidth}{0.2pt}
\fancyhead[L]{maths-cours.fr}
\fancyhead[R]{\thepage}
\renewcommand{\footrulewidth}{0.2pt}
\fancyfoot[C]{}

\newcolumntype{C}{>{\centering\arraybackslash}X}
\newcolumntype{s}{>{\hsize=.35\hsize\arraybackslash}X}

\setlength{\parindent}{0pt}		 
\setlength{\parskip}{3mm}
\setlength{\headheight}{1cm}

\def\ebook{ebook}
\def\book{book}
\def\web{web}
\def\type{web}

\newcommand{\vect}[1]{\overrightarrow{\,\mathstrut#1\,}}

\def\Oij{$\left(\text{O}~;~\vect{\imath},~\vect{\jmath}\right)$}
\def\Oijk{$\left(\text{O}~;~\vect{\imath},~\vect{\jmath},~\vect{k}\right)$}
\def\Ouv{$\left(\text{O}~;~\vect{u},~\vect{v}\right)$}

\hypersetup{breaklinks=true, colorlinks = true, linkcolor = OliveGreen, urlcolor = OliveGreen, citecolor = OliveGreen, pdfauthor={Didier BONNEL - https://www.maths-cours.fr} } % supprime les bordures autour des liens

\renewcommand{\arg}[0]{\text{arg}}

\everymath{\displaystyle}

%================================================================================================================================
%
% Macros - Commandes
%
%================================================================================================================================

\newcommand\meta[2]{    			% Utilisé pour créer le post HTML.
	\def\titre{titre}
	\def\url{url}
	\def\arg{#1}
	\ifx\titre\arg
		\newcommand\maintitle{#2}
		\fancyhead[L]{#2}
		{\Large\sffamily \MakeUppercase{#2}}
		\vspace{1mm}\textcolor{mcvert}{\hrule}
	\fi 
	\ifx\url\arg
		\fancyfoot[L]{\href{https://www.maths-cours.fr#2}{\black \footnotesize{https://www.maths-cours.fr#2}}}
	\fi 
}


\newcommand\TitreC[1]{    		% Titre centré
     \needspace{3\baselineskip}
     \begin{center}\textbf{#1}\end{center}
}

\newcommand\newpar{    		% paragraphe
     \par
}

\newcommand\nosp {    		% commande vide (pas d'espace)
}
\newcommand{\id}[1]{} %ignore

\newcommand\boite[2]{				% Boite simple sans titre
	\vspace{5mm}
	\setlength{\fboxrule}{0.2mm}
	\setlength{\fboxsep}{5mm}	
	\fcolorbox{#1}{#1!3}{\makebox[\linewidth-2\fboxrule-2\fboxsep]{
  		\begin{minipage}[t]{\linewidth-2\fboxrule-4\fboxsep}\setlength{\parskip}{3mm}
  			 #2
  		\end{minipage}
	}}
	\vspace{5mm}
}

\newcommand\CBox[4]{				% Boites
	\vspace{5mm}
	\setlength{\fboxrule}{0.2mm}
	\setlength{\fboxsep}{5mm}
	
	\fcolorbox{#1}{#1!3}{\makebox[\linewidth-2\fboxrule-2\fboxsep]{
		\begin{minipage}[t]{1cm}\setlength{\parskip}{3mm}
	  		\textcolor{#1}{\LARGE{#2}}    
 	 	\end{minipage}  
  		\begin{minipage}[t]{\linewidth-2\fboxrule-4\fboxsep}\setlength{\parskip}{3mm}
			\raisebox{1.2mm}{\normalsize\sffamily{\textcolor{#1}{#3}}}						
  			 #4
  		\end{minipage}
	}}
	\vspace{5mm}
}

\newcommand\cadre[3]{				% Boites convertible html
	\par
	\vspace{2mm}
	\setlength{\fboxrule}{0.1mm}
	\setlength{\fboxsep}{5mm}
	\fcolorbox{#1}{white}{\makebox[\linewidth-2\fboxrule-2\fboxsep]{
  		\begin{minipage}[t]{\linewidth-2\fboxrule-4\fboxsep}\setlength{\parskip}{3mm}
			\raisebox{-2.5mm}{\sffamily \small{\textcolor{#1}{\MakeUppercase{#2}}}}		
			\par		
  			 #3
 	 		\end{minipage}
	}}
		\vspace{2mm}
	\par
}

\newcommand\bloc[3]{				% Boites convertible html sans bordure
     \needspace{2\baselineskip}
     {\sffamily \small{\textcolor{#1}{\MakeUppercase{#2}}}}    
		\par		
  			 #3
		\par
}

\newcommand\CHelp[1]{
     \CBox{Plum}{\faInfoCircle}{À RETENIR}{#1}
}

\newcommand\CUp[1]{
     \CBox{NavyBlue}{\faThumbsOUp}{EN PRATIQUE}{#1}
}

\newcommand\CInfo[1]{
     \CBox{Sepia}{\faArrowCircleRight}{REMARQUE}{#1}
}

\newcommand\CRedac[1]{
     \CBox{PineGreen}{\faEdit}{BIEN R\'EDIGER}{#1}
}

\newcommand\CError[1]{
     \CBox{Red}{\faExclamationTriangle}{ATTENTION}{#1}
}

\newcommand\TitreExo[2]{
\needspace{4\baselineskip}
 {\sffamily\large EXERCICE #1\ (\emph{#2 points})}
\vspace{5mm}
}

\newcommand\img[2]{
          \includegraphics[width=#2\paperwidth]{\imgdir#1}
}

\newcommand\imgsvg[2]{
       \begin{center}   \includegraphics[width=#2\paperwidth]{\imgsvgdir#1} \end{center}
}


\newcommand\Lien[2]{
     \href{#1}{#2 \tiny \faExternalLink}
}
\newcommand\mcLien[2]{
     \href{https~://www.maths-cours.fr/#1}{#2 \tiny \faExternalLink}
}

\newcommand{\euro}{\eurologo{}}

%================================================================================================================================
%
% Macros - Environement
%
%================================================================================================================================

\newenvironment{tex}{ %
}
{%
}

\newenvironment{indente}{ %
	\setlength\parindent{10mm}
}

{
	\setlength\parindent{0mm}
}

\newenvironment{corrige}{%
     \needspace{3\baselineskip}
     \medskip
     \textbf{\textsc{Corrigé}}
     \medskip
}
{
}

\newenvironment{extern}{%
     \begin{center}
     }
     {
     \end{center}
}

\NewEnviron{code}{%
	\par
     \boite{gray}{\texttt{%
     \BODY
     }}
     \par
}

\newenvironment{vbloc}{% boite sans cadre empeche saut de page
     \begin{minipage}[t]{\linewidth}
     }
     {
     \end{minipage}
}
\NewEnviron{h2}{%
    \needspace{3\baselineskip}
    \vspace{0.6cm}
	\noindent \MakeUppercase{\sffamily \large \BODY}
	\vspace{1mm}\textcolor{mcgris}{\hrule}\vspace{0.4cm}
	\par
}{}

\NewEnviron{h3}{%
    \needspace{3\baselineskip}
	\vspace{5mm}
	\textsc{\BODY}
	\par
}

\NewEnviron{margeneg}{ %
\begin{addmargin}[-1cm]{0cm}
\BODY
\end{addmargin}
}

\NewEnviron{html}{%
}

\begin{document}
\begin{h2}EXERCICE 3 (4 points)\end{h2}
\textbf{Commun à tous les candidats}
\medskip
L'objectif de cet exercice est d'étudier les trajectoires de deux sous-marins en phase de plongée.
\par
On considère que ces sous-marins se déplacent en ligne droite, chacun à vitesse constante.
\par
À chaque instant $t$, exprimé en minutes, le premier sous-marin est repéré par le point $S_1(t)$ et le second
sous-marin est repéré par le point $S_2(t)$ dans un repère orthonormé $\left(\text{O}~;~\vec{i},~\vec{j},~\vec{k}\right)$ dont l'unité est le mètre.
\par
Le plan défini par $\left(\text{O}~;~\vec{i},~\vec{j}\right)$ représente la surface de la mer. La cote $z$ est nulle au niveau de la
mer, négative sous l'eau.
\medskip
\begin{enumerate}
     \item On admet que, pour tout réel $t \geqslant 0$, le point $S_1(t)$ a pour coordonnées:
     \begin{center}
          $\begin{cases}
               x(t) = \phantom{-}140 - 60t\\
               y(t) = \phantom{-}105 - 90t\\
               z(t) = -170 - 30 t
          \end{cases}$
     \end{center}
     \begin{enumerate}[label=\alph*.]
          \item Donner les coordonnées du sous- marin au début de l'observation.
          \item Quelle est la vitesse du sous-marin ?
          \item  On se place dans le plan vertical contenant la trajectoire du premier sous-marin.
          \begin{center}
               \begin{extern}%width="420"
                    \psset{xunit=1.0cm,yunit=1.0cm,algebraic=true,dimen=middle,dotstyle=o,dotsize=5pt 0,linewidth=1.6pt,arrowsize=3pt 2,arrowinset=0.25}
                    \begin{pspicture*}(0,-3)(7,3)
                         \psscalebox{0.5}{
                              \psplot[linewidth=0.75pt,linecolor=blue]{-0.38}{13.42}{(--46.12-0.02*x)/11.48}
                              \pscustom[linewidth=0.75pt,linecolor=blue]{
                                   \parametricplot{-0.6324}{0.}{0.6*cos(t)+1.|0.6*sin(t)+4.}
                              \lineto(1.,4.)\closepath}
                              \psline[linewidth=1pt,linestyle=dashed,dash=2pt 2pt](1.,4.)(12,-4.)
                              \rput[tl](9.22,4.5){\fontsize{15pt}{15pt}\selectfont{\blue{niveau de la mer}}}
                              \rput[tl](1.82,3.78){$\blue{\alpha}$}
                              \def\sm{\pscustom[linestyle=none,fillstyle=solid,fillcolor=black]
                                   {
                                        \newpath
                                        \moveto(0.04237557,31.8862686)
                                        \curveto(-0.14149443,31.65805592)(0.28549936,31.08408035)(1.08023317,30.56566562)
                                        \curveto(1.81265886,29.99169006)(2.48583077,28.96037077)(2.48583077,28.21420253)
                                        \curveto(2.60739266,26.3193944)(3.28056457,25.45843106)(4.62385406,25.45843106)
                                        \curveto(5.23471786,25.45843106)(5.84558166,25.11289777)(6.02945167,24.71157406)
                                        \curveto(6.57800736,23.39097108)(33.57757641,3.76146599)(37.48893731,1.81132661)
                                        \curveto(40.6025101,0.31761261)(41.33799011,0.14496076)(42.12966959,0.71893632)
                                        \curveto(43.47601341,1.75255152)(43.35139719,3.18749043)(41.8865458,4.96819221)
                                        \curveto(41.21337389,5.77038046)(40.6648182,6.6901189)(40.6648182,7.03266752)
                                        \curveto(40.6648182,7.37820081)(40.42169441,7.66518859)(40.11626251,7.66518859)
                                        \curveto(39.81083061,7.66518859)(38.1590549,8.69903338)(36.45107972,9.96109084)
                                        \curveto(33.76144641,11.97000532)(33.39676072,12.42987454)(33.94531641,13.40494423)
                                        \curveto(34.74005022,14.78133763)(33.76144641,16.84879761)(32.41815692,16.84879761)
                                        \curveto(31.92885501,16.84879761)(31.31799122,17.30866683)(30.95330553,17.88264239)
                                        \curveto(30.52325741,18.5707243)(29.66926982,18.85771208)(27.59355463,18.79916657)
                                        \curveto(25.08779133,18.74360574)(24.47692753,19.03059352)(19.46845524,22.93362735)
                                        \curveto(16.41413625,25.34432472)(12.80820724,27.58145187)(11.22179396,28.15542744)
                                        \curveto(8.65677686,29.01639078)(8.35134497,29.30337856)(8.65677686,30.45132969)
                                        \curveto(9.14302445,32.11723637)(7.37274116,32.92011338)(6.27257546,31.5432608)
                                        \curveto(5.54014976,30.62421113)(5.23471786,30.62421113)(2.97513267,31.37106813)
                                        \curveto(1.56953507,31.83024858)(0.22624557,32.05915004)(0.04237557,31.8862686)
                                        \closepath
                              }}
                              \rput[tl](9.25,-3){\psscalebox{0.03}{\sm}}
                         }
                    \end{pspicture*}
               \end{extern}
          \end{center}
          Déterminer l'angle $\alpha$ que forme la trajectoire du sous-marin avec le plan horizontal.\\
          On donnera l'arrondi de $\alpha$ à $0,1$ degré près.
     \end{enumerate}
     \item  Au début de l'observation, le second sous-marin est situé au point $S_2(0)$ de coordonnées
     $(68~;~135~;~- 68)$ et atteint au bout de trois minutes le point $S_2(3)$ de coordonnées $(-202~;~-405~;~ - 248)$ avec une vitesse constante.
     \par
     \`A quel instant $t$, exprimé en minutes, les deux sous-marins sont-ils à la même profondeur ?
\end{enumerate}

\end{document}