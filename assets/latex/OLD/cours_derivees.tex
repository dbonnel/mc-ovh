\documentclass[a4paper]{article}

%================================================================================================================================
%
% Packages
%
%================================================================================================================================

\usepackage[T1]{fontenc} 	% pour caractères accentués
\usepackage[utf8]{inputenc}  % encodage utf8
\usepackage[french]{babel}	% langue : français
\usepackage{fourier}			% caractères plus lisibles
\usepackage[dvipsnames]{xcolor} % couleurs
\usepackage{fancyhdr}		% réglage header footer
\usepackage{needspace}		% empêcher sauts de page mal placés
\usepackage{graphicx}		% pour inclure des graphiques
\usepackage{enumitem,cprotect}		% personnalise les listes d'items (nécessaire pour ol, al ...)
\usepackage{hyperref}		% Liens hypertexte
\usepackage{pstricks,pst-all,pst-node,pstricks-add,pst-math,pst-plot,pst-tree,pst-eucl} % pstricks
\usepackage[a4paper,includeheadfoot,top=2cm,left=3cm, bottom=2cm,right=3cm]{geometry} % marges etc.
\usepackage{comment}			% commentaires multilignes
\usepackage{amsmath,environ} % maths (matrices, etc.)
\usepackage{amssymb,makeidx}
\usepackage{bm}				% bold maths
\usepackage{tabularx}		% tableaux
\usepackage{colortbl}		% tableaux en couleur
\usepackage{fontawesome}		% Fontawesome
\usepackage{environ}			% environment with command
\usepackage{fp}				% calculs pour ps-tricks
\usepackage{multido}			% pour ps tricks
\usepackage[np]{numprint}	% formattage nombre
\usepackage{tikz,tkz-tab} 			% package principal TikZ
\usepackage{pgfplots}   % axes
\usepackage{mathrsfs}    % cursives
\usepackage{calc}			% calcul taille boites
\usepackage[scaled=0.875]{helvet} % font sans serif
\usepackage{svg} % svg
\usepackage{scrextend} % local margin
\usepackage{scratch} %scratch
\usepackage{multicol} % colonnes
%\usepackage{infix-RPN,pst-func} % formule en notation polanaise inversée
\usepackage{listings}

%================================================================================================================================
%
% Réglages de base
%
%================================================================================================================================

\lstset{
language=Python,   % R code
literate=
{á}{{\'a}}1
{à}{{\`a}}1
{ã}{{\~a}}1
{é}{{\'e}}1
{è}{{\`e}}1
{ê}{{\^e}}1
{í}{{\'i}}1
{ó}{{\'o}}1
{õ}{{\~o}}1
{ú}{{\'u}}1
{ü}{{\"u}}1
{ç}{{\c{c}}}1
{~}{{ }}1
}


\definecolor{codegreen}{rgb}{0,0.6,0}
\definecolor{codegray}{rgb}{0.5,0.5,0.5}
\definecolor{codepurple}{rgb}{0.58,0,0.82}
\definecolor{backcolour}{rgb}{0.95,0.95,0.92}

\lstdefinestyle{mystyle}{
    backgroundcolor=\color{backcolour},   
    commentstyle=\color{codegreen},
    keywordstyle=\color{magenta},
    numberstyle=\tiny\color{codegray},
    stringstyle=\color{codepurple},
    basicstyle=\ttfamily\footnotesize,
    breakatwhitespace=false,         
    breaklines=true,                 
    captionpos=b,                    
    keepspaces=true,                 
    numbers=left,                    
xleftmargin=2em,
framexleftmargin=2em,            
    showspaces=false,                
    showstringspaces=false,
    showtabs=false,                  
    tabsize=2,
    upquote=true
}

\lstset{style=mystyle}


\lstset{style=mystyle}
\newcommand{\imgdir}{C:/laragon/www/newmc/assets/imgsvg/}
\newcommand{\imgsvgdir}{C:/laragon/www/newmc/assets/imgsvg/}

\definecolor{mcgris}{RGB}{220, 220, 220}% ancien~; pour compatibilité
\definecolor{mcbleu}{RGB}{52, 152, 219}
\definecolor{mcvert}{RGB}{125, 194, 70}
\definecolor{mcmauve}{RGB}{154, 0, 215}
\definecolor{mcorange}{RGB}{255, 96, 0}
\definecolor{mcturquoise}{RGB}{0, 153, 153}
\definecolor{mcrouge}{RGB}{255, 0, 0}
\definecolor{mclightvert}{RGB}{205, 234, 190}

\definecolor{gris}{RGB}{220, 220, 220}
\definecolor{bleu}{RGB}{52, 152, 219}
\definecolor{vert}{RGB}{125, 194, 70}
\definecolor{mauve}{RGB}{154, 0, 215}
\definecolor{orange}{RGB}{255, 96, 0}
\definecolor{turquoise}{RGB}{0, 153, 153}
\definecolor{rouge}{RGB}{255, 0, 0}
\definecolor{lightvert}{RGB}{205, 234, 190}
\setitemize[0]{label=\color{lightvert}  $\bullet$}

\pagestyle{fancy}
\renewcommand{\headrulewidth}{0.2pt}
\fancyhead[L]{maths-cours.fr}
\fancyhead[R]{\thepage}
\renewcommand{\footrulewidth}{0.2pt}
\fancyfoot[C]{}

\newcolumntype{C}{>{\centering\arraybackslash}X}
\newcolumntype{s}{>{\hsize=.35\hsize\arraybackslash}X}

\setlength{\parindent}{0pt}		 
\setlength{\parskip}{3mm}
\setlength{\headheight}{1cm}

\def\ebook{ebook}
\def\book{book}
\def\web{web}
\def\type{web}

\newcommand{\vect}[1]{\overrightarrow{\,\mathstrut#1\,}}

\def\Oij{$\left(\text{O}~;~\vect{\imath},~\vect{\jmath}\right)$}
\def\Oijk{$\left(\text{O}~;~\vect{\imath},~\vect{\jmath},~\vect{k}\right)$}
\def\Ouv{$\left(\text{O}~;~\vect{u},~\vect{v}\right)$}

\hypersetup{breaklinks=true, colorlinks = true, linkcolor = OliveGreen, urlcolor = OliveGreen, citecolor = OliveGreen, pdfauthor={Didier BONNEL - https://www.maths-cours.fr} } % supprime les bordures autour des liens

\renewcommand{\arg}[0]{\text{arg}}

\everymath{\displaystyle}

%================================================================================================================================
%
% Macros - Commandes
%
%================================================================================================================================

\newcommand\meta[2]{    			% Utilisé pour créer le post HTML.
	\def\titre{titre}
	\def\url{url}
	\def\arg{#1}
	\ifx\titre\arg
		\newcommand\maintitle{#2}
		\fancyhead[L]{#2}
		{\Large\sffamily \MakeUppercase{#2}}
		\vspace{1mm}\textcolor{mcvert}{\hrule}
	\fi 
	\ifx\url\arg
		\fancyfoot[L]{\href{https://www.maths-cours.fr#2}{\black \footnotesize{https://www.maths-cours.fr#2}}}
	\fi 
}


\newcommand\TitreC[1]{    		% Titre centré
     \needspace{3\baselineskip}
     \begin{center}\textbf{#1}\end{center}
}

\newcommand\newpar{    		% paragraphe
     \par
}

\newcommand\nosp {    		% commande vide (pas d'espace)
}
\newcommand{\id}[1]{} %ignore

\newcommand\boite[2]{				% Boite simple sans titre
	\vspace{5mm}
	\setlength{\fboxrule}{0.2mm}
	\setlength{\fboxsep}{5mm}	
	\fcolorbox{#1}{#1!3}{\makebox[\linewidth-2\fboxrule-2\fboxsep]{
  		\begin{minipage}[t]{\linewidth-2\fboxrule-4\fboxsep}\setlength{\parskip}{3mm}
  			 #2
  		\end{minipage}
	}}
	\vspace{5mm}
}

\newcommand\CBox[4]{				% Boites
	\vspace{5mm}
	\setlength{\fboxrule}{0.2mm}
	\setlength{\fboxsep}{5mm}
	
	\fcolorbox{#1}{#1!3}{\makebox[\linewidth-2\fboxrule-2\fboxsep]{
		\begin{minipage}[t]{1cm}\setlength{\parskip}{3mm}
	  		\textcolor{#1}{\LARGE{#2}}    
 	 	\end{minipage}  
  		\begin{minipage}[t]{\linewidth-2\fboxrule-4\fboxsep}\setlength{\parskip}{3mm}
			\raisebox{1.2mm}{\normalsize\sffamily{\textcolor{#1}{#3}}}						
  			 #4
  		\end{minipage}
	}}
	\vspace{5mm}
}

\newcommand\cadre[3]{				% Boites convertible html
	\par
	\vspace{2mm}
	\setlength{\fboxrule}{0.1mm}
	\setlength{\fboxsep}{5mm}
	\fcolorbox{#1}{white}{\makebox[\linewidth-2\fboxrule-2\fboxsep]{
  		\begin{minipage}[t]{\linewidth-2\fboxrule-4\fboxsep}\setlength{\parskip}{3mm}
			\raisebox{-2.5mm}{\sffamily \small{\textcolor{#1}{\MakeUppercase{#2}}}}		
			\par		
  			 #3
 	 		\end{minipage}
	}}
		\vspace{2mm}
	\par
}

\newcommand\bloc[3]{				% Boites convertible html sans bordure
     \needspace{2\baselineskip}
     {\sffamily \small{\textcolor{#1}{\MakeUppercase{#2}}}}    
		\par		
  			 #3
		\par
}

\newcommand\CHelp[1]{
     \CBox{Plum}{\faInfoCircle}{À RETENIR}{#1}
}

\newcommand\CUp[1]{
     \CBox{NavyBlue}{\faThumbsOUp}{EN PRATIQUE}{#1}
}

\newcommand\CInfo[1]{
     \CBox{Sepia}{\faArrowCircleRight}{REMARQUE}{#1}
}

\newcommand\CRedac[1]{
     \CBox{PineGreen}{\faEdit}{BIEN R\'EDIGER}{#1}
}

\newcommand\CError[1]{
     \CBox{Red}{\faExclamationTriangle}{ATTENTION}{#1}
}

\newcommand\TitreExo[2]{
\needspace{4\baselineskip}
 {\sffamily\large EXERCICE #1\ (\emph{#2 points})}
\vspace{5mm}
}

\newcommand\img[2]{
          \includegraphics[width=#2\paperwidth]{\imgdir#1}
}

\newcommand\imgsvg[2]{
       \begin{center}   \includegraphics[width=#2\paperwidth]{\imgsvgdir#1} \end{center}
}


\newcommand\Lien[2]{
     \href{#1}{#2 \tiny \faExternalLink}
}
\newcommand\mcLien[2]{
     \href{https~://www.maths-cours.fr/#1}{#2 \tiny \faExternalLink}
}

\newcommand{\euro}{\eurologo{}}

%================================================================================================================================
%
% Macros - Environement
%
%================================================================================================================================

\newenvironment{tex}{ %
}
{%
}

\newenvironment{indente}{ %
	\setlength\parindent{10mm}
}

{
	\setlength\parindent{0mm}
}

\newenvironment{corrige}{%
     \needspace{3\baselineskip}
     \medskip
     \textbf{\textsc{Corrigé}}
     \medskip
}
{
}

\newenvironment{extern}{%
     \begin{center}
     }
     {
     \end{center}
}

\NewEnviron{code}{%
	\par
     \boite{gray}{\texttt{%
     \BODY
     }}
     \par
}

\newenvironment{vbloc}{% boite sans cadre empeche saut de page
     \begin{minipage}[t]{\linewidth}
     }
     {
     \end{minipage}
}
\NewEnviron{h2}{%
    \needspace{3\baselineskip}
    \vspace{0.6cm}
	\noindent \MakeUppercase{\sffamily \large \BODY}
	\vspace{1mm}\textcolor{mcgris}{\hrule}\vspace{0.4cm}
	\par
}{}

\NewEnviron{h3}{%
    \needspace{3\baselineskip}
	\vspace{5mm}
	\textsc{\BODY}
	\par
}

\NewEnviron{margeneg}{ %
\begin{addmargin}[-1cm]{0cm}
\BODY
\end{addmargin}
}

\NewEnviron{html}{%
}

\begin{document}
\begin{h2}I - Nombre dérivé\end{h2}
\cadre{bleu}{Définition}{%id="d10"
     Soit $f$ une fonction définie sur un intervalle $I$ et $a$ et $b$ deux réels appartenant à $I$.
     \par
     On appelle \textbf{taux d'accroissement} de $f$ entre $a$ et $b$ le nombre :
     \par
     $T=\frac{f\left(b\right)-f\left(a\right)}{b-a}$
}
\bloc{cyan}{Remarque}{%id="r10"
     En faisant le changement de variable : $b=a+h$ ($h$ représente alors l'écart entre $b$ et $a$), ce taux s'écrit aussi :
     \par
     $T=\frac{f\left(a+h\right)-f\left(a\right)}{h}$
}
\bloc{cyan}{Interprétation graphique}{%id="i10"
     \begin{center}
          \begin{extern}%width="450" alt="taux d'accroissement"
               % -+-+-+ variables modifiables
               \resizebox{8cm}{!}{%
                    \def\xmin{-2.5}
                    \def\xmax{8.5}
                    \def\ymin{-1.5}
                    \def\ymax{8.5}
                    \def\xunit{1}  % unités en cm
                    \def\yunit{1}
                    \psset{xunit=\xunit,yunit=\yunit,algebraic=true}
                    \fontsize{12pt}{12pt}\selectfont
                    \begin{pspicture*}[linewidth=1pt](\xmin,\ymin)(\xmax,\ymax)
                         %      \psgrid[gridcolor=mcgris,subgriddiv=0](-4,-2)(9,9)
                         %        \psaxes[linewidth=0.75pt]{->}(0,0)(\xmin,\ymin)(\xmax,\ymax)
                         \psline[linecolor=gray]{->}(\xmin,0)(\xmax,0)
                         \psline[linecolor=gray]{->}(0,\ymin)(0,\ymax)
                         \psline[linecolor=lightgray](2,0)(2,1.69)
                         \psline[linecolor=lightgray](5,0)(5,3.713)
                         \rput[tr](-0.2,-0.2){$O$}\rput[br](1.8,1.8){$A$}\rput[br](4.8,3.693){$B$} \rput[t](3.5,-0.8){\color{mcmauve} $h$}
                         \rput[t](8,7.5){\color{blue} $\mathscr{C}_f$}\rput[t](2,-0.2){$a$}\rput[t](5,-0.1){$b$}
                         \psdots(2,1.69)\psdots(5,3.713)
                         \psplot[plotpoints=1000,linewidth=0.8pt,linecolor=blue]{\xmin}{\xmax}{1.3^x}
                         \psplot[plotpoints=1000,linewidth=0.8pt,linecolor=mcvert]{\xmin}{\xmax}{0.674*x+0.341}
                        \psline[linecolor=mcmauve]{<->}(2,-0.7)(5,-0.7)
                    \end{pspicture*}
               }
          \end{extern}
     \end{center}
     Le taux d'accroissement de $f$ entre $a$ et $b$ est le \textbf{coefficient directeur} de la droite $(AB)$.
}
\cadre{bleu}{Définition}{%id="d20"
     Soit $f$ une fonction définie sur un intervalle ouvert $I$ contenant $a$.
     \par
     On dit que $f$ est dérivable en $a$ si et seulement si le rapport $\frac{f\left(a+h\right)-f\left(a\right)}{h}$ tend vers un nombre réel lorsque $h$ tend vers zéro.
     \par
     Ce nombre s'appelle le nombre dérivé de $f$ en $a$ et se note $f^{\prime}\left(a\right)$.
}
\bloc{orange}{Exemple}{%id="e20"
     Calculons le nombre dérivé de la fonction $f : x\mapsto x^{2}$ pour $x=1$.
     \par
     $ \frac{f\left(1+h\right)-f\left(1\right)}{h}=\frac{\left(1+h\right)^{2}-1^{2}}{h}=\frac{1+2h+h^{2}-1^{2}}{h}=\frac{2h+h^{2}}{h}=2+h$
     \par
     Or quand $h$ tend vers $0$, $2+h$ tend vers 2; donc $f^{\prime}\left(1\right)=2$.
}
\bloc{cyan}{Interprétation graphique}{%id="i20"
      \begin{center}
          \begin{extern}%width="450" alt="nombre dérivé"
               % -+-+-+ variables modifiables
               \resizebox{8cm}{!}{%
                    \def\xmin{-2.5}
                    \def\xmax{8.5}
                    \def\ymin{-1.5}
                    \def\ymax{8.5}
                    \def\xunit{1}  % unités en cm
                    \def\yunit{1}
                    \psset{xunit=\xunit,yunit=\yunit,algebraic=true}
                    \fontsize{12pt}{12pt}\selectfont
                    \begin{pspicture*}[linewidth=1pt](\xmin,\ymin)(\xmax,\ymax)
                         %      \psgrid[gridcolor=mcgris,subgriddiv=0](-4,-2)(9,9)
                         %        \psaxes[linewidth=0.75pt]{->}(0,0)(\xmin,\ymin)(\xmax,\ymax)
                         \psline[linecolor=gray]{->}(\xmin,0)(\xmax,0)
                         \psline[linecolor=gray]{->}(0,\ymin)(0,\ymax)
                         \psline[linecolor=lightgray](2,0)(2,1.69)
                         \psline[linecolor=lightgray](5,0)(5,3.713)
                         \rput[tr](-0.2,-0.2){$O$}\rput[br](1.8,1.8){$A$}\rput[br](4.8,3.693){$B$}\rput[t](3.5,-0.8){\color{mcmauve} $h$}
                         \rput[t](8,4){\color{red} $\mathscr{T}$}\rput[t](8,7.5){\color{blue} $\mathscr{C}_f$}\rput[t](2,-0.2){$a$}\rput[t](5,-0.1){$a+h$}
                         \psdots(2,1.69)\psdots(5,3.713)
                         \psplot[plotpoints=1000,linewidth=0.8pt,linecolor=blue]{\xmin}{\xmax}{1.3^x}
                         \psplot[plotpoints=1000,linewidth=0.8pt,linecolor=mcvert]{\xmin}{\xmax}{0.674*x+0.341}
                         \psplot[plotpoints=1000,linewidth=0.8pt,linecolor=red]{\xmin}{\xmax}{0.443*x+0.803}
                         \psline[linecolor=mcmauve]{<->}(2,-0.7)(5,-0.7)
                    \end{pspicture*}
               }
          \end{extern}
     \end{center}
     Lorsque $h$ se rapproche de zéro, le point $B$ se rapproche du point $A$ et la droite $\left(AB\right)$ se rapproche de la tangente $\mathscr{T}$
}
\cadre{vert}{Propriété}{%id="p30"
     Soit $f$ une fonction dérivable en $a$ de courbe représentative $C_{f}$ . $f^{\prime}\left(a\right)$ représente le \textbf{coefficient directeur} de la \textbf{tangente} à la courbe $C_{f}$ au point d'abscisse $a$.
}
\cadre{vert}{Propriété}{%id="p40"
     Soit $f$ une fonction dérivable en $a$ de courbe représentative $C_{f}$ . L'équation de la tangente à $C_{f}$ au point d'abscisse $a$ est :
     \par
     $y=f^{\prime}\left(a\right)\left(x-a\right)+f\left(a\right)$
}
\bloc{orange}{Démonstration}{%id="m40"
     D'après la propriété précédente, la tangente à $C_{f}$ au point d'abscisse $a$ est une droite de coefficient directeur $f^{\prime}\left(a\right)$. Son équation est donc de la forme :
     \par
     $y=f^{\prime}\left(a\right)x+b$
     \par
     On sait que la tangente passe par le point $A$ de coordonnées $\left(a; f\left(a\right)\right)$ donc :
     \par
     $f\left(a\right)=f^{\prime}\left(a\right)a+b$
     \par
     $b=-f^{\prime}\left(a\right)a+f\left(a\right)$
     \par
     L'équation de la tangente est donc :
     \par
     $y=f^{\prime}\left(a\right)x-f^{\prime}\left(a\right)a+f\left(a\right)$
     \par
     Soit :
     \par
     $y=f^{\prime}\left(a\right)\left(x-a\right)+f\left(a\right)$
}
\bloc{orange}{Exemple}{%id="e40"
     Cherchons l'équation de la tangente à la courbe représentative de la fonction $f : x\mapsto x^{2}$ au point d'abscisse $a=1$.
     \par
     On a $f\left(1\right)=1^{2}=1$ et on a vu dans l'exemple précédent que $f^{\prime}\left(1\right)=2$.
     \par
     L'équation cherchée est donc :
     \par
     $y=2\left(x-1\right)+1$
     \par
     soit :
     \par
     $y=2x-1$
     \begin{center}
          \begin{extern}%width="380" alt="courbe et tangente"
               % -+-+-+ variables modifiables
               \resizebox{7cm}{!}{%
                    \def\xmin{-2.5}
                    \def\xmax{2.5}
                    \def\ymin{-1.5}
                    \def\ymax{3.8}
                    \def\xunit{2}  % unités en cm
                    \def\yunit{2}
                    \psset{xunit=\xunit,yunit=\yunit,algebraic=true}
                    \fontsize{12pt}{12pt}\selectfont
                    \begin{pspicture*}[linewidth=1pt](\xmin,\ymin)(\xmax,\ymax)
                         %      \psgrid[gridcolor=mcgris,subgriddiv=0](-4,-2)(9,9)
                               \psaxes[linewidth=0.75pt]{->}(0,0)(\xmin,\ymin)(\xmax,\ymax)
                           \rput[tr](-0.2,-0.2){$O$}  \rput[l](1.1,1){$A$}
                         \rput[t](1.5,3.3){\color{blue} $\mathcal{C}_f$}\rput[t](2.4,3.3){$\color{mcvert} \mathcal{T}$}
                         \psplot[plotpoints=1000,linecolor=blue]{\xmin}{\xmax}{x^2}
                         \psplot[plotpoints=1000,linecolor=mcvert]{\xmin}{\xmax}{2*x-1}
                         \psdots(1,1)
                         \psline[linecolor=lightgray](1,0)(1,1)
                     \end{pspicture*}
               }
          \end{extern}
     \end{center}
}
\begin{h2}II - Fonction dérivée\end{h2}
\cadre{bleu}{Définition}{%id="d50"
     Si $f$ est définie sur un intervalle $I$ et si le nombre dérivé existe en chaque point de $I$, on dit que $f$ est \textbf{dérivable} sur $I$. La fonction qui a $x$ associe le nombre dérivé de $f$ en $x$ s'appelle \textbf{fonction dérivée} de $f$ et se note $f^{\prime}$
}
\cadre{vert}{Dérivée des fonctions usuelles}{%id="p60"
     \begin{center}
          \begin{tabularx}{0.8\linewidth}{|X|X|X|}%class="compact" width="600"
               \hline
               \textbf{Fonction} & \textbf{Dérivée} & \textbf{Ensemble de dérivabilité}
               \\ \hline
               $k$  $\left(k\in \mathbb{R}\right)$  &  $0$  &  $\mathbb{R}$
               \\ \hline
               $x$ &  $1$  &  $\mathbb{R}$
               \\ \hline
               $x^{n}$ $\left(n\in \mathbb{N}\right)$  &  $nx^{n-1}$  &  $\mathbb{R}$
               \\ \hline
               $\frac{1}{x}$ &  $-\frac{1}{x^{2}}$  &  $\mathbb{R}-\left\{0\right\}$
               \\ \hline
               $\sqrt{x}$ &  $\frac{1}{2\sqrt{x}}$  &  $\left]0;+\infty \right[$
               \\        \hline
          \end{tabularx}
     \end{center}
}
\cadre{vert}{Opérations}{%id="p70"
     Si $u$ et $v$ sont 2 \textbf{fonctions} dérivables :
     \begin{center}
          \begin{tabularx}{0.8\linewidth}{|*{3}{>{\centering \arraybackslash }X|}}%class="compact" width="600"
               \hline
               \textbf{Fonction} & \textbf{Dérivée}
               \\ \hline
               $u+v$  &  $u^{\prime}+v^{\prime}$
               \\ \hline
               $ku$  $\left(k\in \mathbb{R}\right)$  &  $ku^{\prime} $
               \\ \hline
               $\frac{1}{u}$ (avec $u\left(x\right)\neq 0$ sur $I$)  &  $-\frac{u^{\prime} }{u^{2}} $
               \\ \hline
               $uv$  &  $u^{\prime}v+uv^{\prime}$
               \\ \hline
               $\frac{u}{v}$  (avec $v\left(x\right)\neq 0$ sur $I$) &  $\frac{u^{\prime}v-uv^{\prime}}{v^{2}} $
               \\ \hline
          \end{tabularx}
     \end{center}
}
\bloc{orange}{Exemples}{%id="e70"
     \begin{itemize}
          \item On cherche à calculer la dérivée de la fonction $f$ définie sur $\mathbb{R}\backslash\left\{0\right\}$ par $f\left(x\right)=x^{2}+\frac{1}{x}$
          \par
          $f$ est la somme des fonctions $u$ et $v$ définies par $u\left(x\right)=x^{2}$ et $v\left(x\right)=\frac{1}{x}$
          \par
          $u^{\prime}\left(x\right)=2x$ et $v^{\prime}\left(x\right)=-\frac{1}{x^{2}}$
          \par
          donc $f^{\prime}\left(x\right)=2x-\frac{1}{x^{2}}$
          \item Soit la fonction $g$ définie sur $\mathbb{R}$ par $g\left(x\right)=\frac{x^{3}-1}{x^{2}+1}$
          \par
          $g$ est le quotient des fonctions $u$ et $v$ définies par $u\left(x\right)=x^{3}-1$ et $v\left(x\right)=x^{2}+1$
          \par
          $u^{\prime}\left(x\right)=3x^{2}+0=3x^{2}$ et $v^{\prime}\left(x\right)=2x+0=2x$
          \par
          $g^{\prime}\left(x\right)=\frac{u^{\prime}\left(x\right)v\left(x\right)-u\left(x\right)v^{\prime}\left(x\right)}{v\left(x\right)^{2}}=\frac{\left(3x^{2}\right)\left(x^{2}+1\right)-\left(x^{3}-1\right)\times 2x}{\left(x^{2}+1\right)^{2}}=\frac{x^{4}+3x^{2}+2x}{\left(x^{2}+1\right)^{2}}$
          \item Soit enfin la fonction $h$ définie sur $\left]1;+\infty \right[$ par $h\left(x\right)=\frac{3}{x^{2}-1}$
          \par
          On pourrait utiliser la formule $\left(\frac{u}{v}\right)^{\prime}=\frac{u^{\prime}v-uv^{\prime}}{v^{2}}$ comme précédemment mais cela ne sera pas très judicieux. En effet, le numérateur étant constant, il y a une manière plus rapide de procéder. Il suffit d'écrire :
          \par
          $h\left(x\right)=3\times \frac{1}{x^{2}-1}$
          \par
          et d'appliquer la formule $\left(\frac{1}{v}\right)^{\prime} = -\frac{v^{\prime} }{v^{2}}$ avec $v\left(x\right)=x^{2}-1$ (donc $v^{\prime}\left(x\right)=2x$)
          \par
          On obtient :
          \par
          $h^{\prime}\left(x\right)=3\times -\frac{2x}{\left(x^{2}-1\right)^{2}}=-\frac{6x}{\left(x^{2}-1\right)^{2}}$
     \end{itemize}
}
\begin{h2}III - Applications de la dérivée\end{h2}
\boite{gray}{%
Si nécessaire, revoir la notion de \mcLien{/cours/premiere-es/etude-de-fonctions\#d40}{sens de variation d'une fonction}.
}
\cadre{rouge}{Théorème}{%id="t80"
     Soit $f$ une fonction dérivable sur un intervalle $I$, $f$ est \textbf{croissante} sur $I$ si et seulement si $f^{\prime}\left(x\right)$ est \textbf{positif ou nul} pour tout $x \in I$.
     \par
     De plus si $f^{\prime}\left(x\right)$ est \textbf{strictement positive} sur $I$, sauf éventuellement en quelques points, alors $f$ est \textbf{strictement croissante} sur $I$.
}
\bloc{orange}{Exemple}{%id="e80"
     Soit la fonction $f$ définie sur $\left[-1;1\right]$ par $f\left(x\right)=x^{3}$.
     \par
     $f^{\prime}\left(x\right)=3x^{2}$ est positive ou nulle sur $\left[-1;1\right]$, donc $f$ est \textbf{croissante} sur $\left[-1;1\right]$.
     \par
     Comme par ailleurs, $f^{\prime}$ est strictement positive sauf pour $x=0$, $f$ est \textbf{strictement croissante} sur $\left[-1;1\right]$.
 \begin{center}
     \begin{extern} %width="280" alt="Fonction  cube sur [-1;1]"
          \resizebox{6cm}{!}{%
               % -+-+-+ variables modifiables
               \def\fonction{x^3}
               \def\xmin{-1.2}
               \def\xmax{1.4}
               \def\ymin{-1.2}
               \def\ymax{1.2}
               \def\xunit{4}  % unités en cm
               \def\yunit{4}
               \psset{xunit=\xunit,yunit=\yunit,algebraic=true}
               \fontsize{18pt}{18pt}\selectfont
               \begin{pspicture*}[linewidth=1pt](\xmin,\ymin)(\xmax,\ymax)
                    %      \psgrid[gridcolor=mcgris, subgriddiv=5, gridlabels=0pt](\xmin,\ymin)(\xmax,\ymax)
                    \psaxes[linewidth=0.75pt]{->}(0,0)(\xmin,\ymin)(\xmax,\ymax)
                    \psplot[plotpoints=2000,linecolor=blue]{\xmin}{\xmax}{\fonction}
                    \rput[tr](-0.1,-0.2){$O$}
                    \rput[tl](1.1,1){$\color{blue} \mathcal{C}$}
               \end{pspicture*}
          }
     \end{extern}
\end{center}}
\begin{center}
\textit{Fonction  cube sur $[-1;1]$}
\end{center}
On a un théorème analogue si la dérivée est négative :
\cadre{rouge}{Théorème}{%id="t90"
     Soit $f$ une fonction dérivable sur un intervalle $I$, $f$ est \textbf{décroissante} sur $I$ si et seulement si $f^{\prime}\left(x\right)$ est \textbf{négatif ou nul} pour tout $x \in I$.
     \par
     De plus si $f^{\prime}\left(x\right)$ est \textbf{strictement négative} sur $I$, sauf éventuellement en quelques points, alors $f$ est \textbf{strictement décroissante} sur $I$.
}
\bloc{cyan}{Remarques}{%id="r90"
     \begin{itemize}
          \item Si $f$ est dérivable, les théorèmes précédents montre que l'étude des variations de $f$ se ramène à l'étude du \textbf{signe de la dérivée.}
          \item On regroupe couramment le tableau de signe de la dérivée et le tableau de variations de $f$ dans un même tableau à 3 lignes (voir exemple ci-dessous)
          \item Pour montrer qu'une fonction $f$ admet un maximum en $a$, on peut montrer que $f$ est croissante pour $x < a$ et décroissante pour $x > a$ ; c'est à dire, si $f$ est dérivable, que $f^{\prime}$ est positive pour $x < a$ et négative pour $x > a$.
\begin{center}
\begin{extern}%width="340" alt="Maximum en a"
\begin{tikzpicture}[scale=0.875]
% Styles 
\tikzstyle{cadre}=[thin]
\tikzstyle{fleche}=[->,>=latex,thin]
\tikzstyle{nondefini}=[lightgray]
% Dimensions Modifiables
\def\Lrg{1.5}
\def\HtX{1}
\def\HtY{0.5}
% Dimensions Calculées
\def\lignex{-0.5*\HtX}
\def\lignef{-1.5*\HtX}
\def\separateur{-0.5*\Lrg}
% Largeur du tableau
\def\gauche{-1.5*\Lrg}
\def\droite{4.5*\Lrg}
% Hauteur du tableau
\def\haut{0.5*\HtX}
\def\bas{-2.5*\HtX-2*\HtY}
% Pointillés
\draw[lightgray] (2*\Lrg,\lignex) -- (2*\Lrg,\lignef);
\draw[lightgray] (2*\Lrg,\lignef) -- (2*\Lrg,\bas);
% Ligne de l'abscisse : x
\node at (-1*\Lrg,0) {$x$};
\node at (0*\Lrg,0) {$ $};
\node at (2*\Lrg,0) {$a$};
\node at (4*\Lrg,0) {$ $};
% Ligne de la dérivée : f'(x)
\node at (-1*\Lrg,-1*\HtX) {$f'(x)$};
\node at (0*\Lrg,-1*\HtX) {$ $};
\node at (1*\Lrg,-1*\HtX) {$+$};
\node at (2*\Lrg,-1*\HtX) {$0$};
\node at (3*\Lrg,-1*\HtX) {$-$};
\node at (4*\Lrg,-1*\HtX) {$ $};
% Ligne de la fonction : f(x)
\node  at (-1*\Lrg,{-2*\HtX+(-1)*\HtY}) {$f(x)$};
\node (f1) at (0*\Lrg,{-2*\HtX+(-2)*\HtY}) {$ $};
\node (f2) at (2*\Lrg,{-2*\HtX+(0)*\HtY}) {$\text{max}$};
\node (f3) at (4*\Lrg,{-2*\HtX+(-2)*\HtY}) {$ $};
% Flèches
\draw[fleche] (f1) -- (f2);
\draw[fleche] (f2) -- (f3);
% Encadrement
\draw[cadre] (\separateur,\haut) -- (\separateur,\bas);
\draw[cadre] (\gauche,\haut) rectangle  (\droite,\bas);
\draw[cadre] (\gauche,\lignex) -- (\droite,\lignex);
\draw[cadre] (\gauche,\lignef) -- (\droite,\lignef);
\end{tikzpicture}
\end{extern}

\end{center}
     \end{itemize}
}

\end{document}