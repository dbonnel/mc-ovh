\documentclass[a4paper]{article}

%================================================================================================================================
%
% Packages
%
%================================================================================================================================

\usepackage[T1]{fontenc} 	% pour caractères accentués
\usepackage[utf8]{inputenc}  % encodage utf8
\usepackage[french]{babel}	% langue : français
\usepackage{fourier}			% caractères plus lisibles
\usepackage[dvipsnames]{xcolor} % couleurs
\usepackage{fancyhdr}		% réglage header footer
\usepackage{needspace}		% empêcher sauts de page mal placés
\usepackage{graphicx}		% pour inclure des graphiques
\usepackage{enumitem,cprotect}		% personnalise les listes d'items (nécessaire pour ol, al ...)
\usepackage{hyperref}		% Liens hypertexte
\usepackage{pstricks,pst-all,pst-node,pstricks-add,pst-math,pst-plot,pst-tree,pst-eucl} % pstricks
\usepackage[a4paper,includeheadfoot,top=2cm,left=3cm, bottom=2cm,right=3cm]{geometry} % marges etc.
\usepackage{comment}			% commentaires multilignes
\usepackage{amsmath,environ} % maths (matrices, etc.)
\usepackage{amssymb,makeidx}
\usepackage{bm}				% bold maths
\usepackage{tabularx}		% tableaux
\usepackage{colortbl}		% tableaux en couleur
\usepackage{fontawesome}		% Fontawesome
\usepackage{environ}			% environment with command
\usepackage{fp}				% calculs pour ps-tricks
\usepackage{multido}			% pour ps tricks
\usepackage[np]{numprint}	% formattage nombre
\usepackage{tikz,tkz-tab} 			% package principal TikZ
\usepackage{pgfplots}   % axes
\usepackage{mathrsfs}    % cursives
\usepackage{calc}			% calcul taille boites
\usepackage[scaled=0.875]{helvet} % font sans serif
\usepackage{svg} % svg
\usepackage{scrextend} % local margin
\usepackage{scratch} %scratch
\usepackage{multicol} % colonnes
%\usepackage{infix-RPN,pst-func} % formule en notation polanaise inversée
\usepackage{listings}

%================================================================================================================================
%
% Réglages de base
%
%================================================================================================================================

\lstset{
language=Python,   % R code
literate=
{á}{{\'a}}1
{à}{{\`a}}1
{ã}{{\~a}}1
{é}{{\'e}}1
{è}{{\`e}}1
{ê}{{\^e}}1
{í}{{\'i}}1
{ó}{{\'o}}1
{õ}{{\~o}}1
{ú}{{\'u}}1
{ü}{{\"u}}1
{ç}{{\c{c}}}1
{~}{{ }}1
}


\definecolor{codegreen}{rgb}{0,0.6,0}
\definecolor{codegray}{rgb}{0.5,0.5,0.5}
\definecolor{codepurple}{rgb}{0.58,0,0.82}
\definecolor{backcolour}{rgb}{0.95,0.95,0.92}

\lstdefinestyle{mystyle}{
    backgroundcolor=\color{backcolour},   
    commentstyle=\color{codegreen},
    keywordstyle=\color{magenta},
    numberstyle=\tiny\color{codegray},
    stringstyle=\color{codepurple},
    basicstyle=\ttfamily\footnotesize,
    breakatwhitespace=false,         
    breaklines=true,                 
    captionpos=b,                    
    keepspaces=true,                 
    numbers=left,                    
xleftmargin=2em,
framexleftmargin=2em,            
    showspaces=false,                
    showstringspaces=false,
    showtabs=false,                  
    tabsize=2,
    upquote=true
}

\lstset{style=mystyle}


\lstset{style=mystyle}
\newcommand{\imgdir}{C:/laragon/www/newmc/assets/imgsvg/}
\newcommand{\imgsvgdir}{C:/laragon/www/newmc/assets/imgsvg/}

\definecolor{mcgris}{RGB}{220, 220, 220}% ancien~; pour compatibilité
\definecolor{mcbleu}{RGB}{52, 152, 219}
\definecolor{mcvert}{RGB}{125, 194, 70}
\definecolor{mcmauve}{RGB}{154, 0, 215}
\definecolor{mcorange}{RGB}{255, 96, 0}
\definecolor{mcturquoise}{RGB}{0, 153, 153}
\definecolor{mcrouge}{RGB}{255, 0, 0}
\definecolor{mclightvert}{RGB}{205, 234, 190}

\definecolor{gris}{RGB}{220, 220, 220}
\definecolor{bleu}{RGB}{52, 152, 219}
\definecolor{vert}{RGB}{125, 194, 70}
\definecolor{mauve}{RGB}{154, 0, 215}
\definecolor{orange}{RGB}{255, 96, 0}
\definecolor{turquoise}{RGB}{0, 153, 153}
\definecolor{rouge}{RGB}{255, 0, 0}
\definecolor{lightvert}{RGB}{205, 234, 190}
\setitemize[0]{label=\color{lightvert}  $\bullet$}

\pagestyle{fancy}
\renewcommand{\headrulewidth}{0.2pt}
\fancyhead[L]{maths-cours.fr}
\fancyhead[R]{\thepage}
\renewcommand{\footrulewidth}{0.2pt}
\fancyfoot[C]{}

\newcolumntype{C}{>{\centering\arraybackslash}X}
\newcolumntype{s}{>{\hsize=.35\hsize\arraybackslash}X}

\setlength{\parindent}{0pt}		 
\setlength{\parskip}{3mm}
\setlength{\headheight}{1cm}

\def\ebook{ebook}
\def\book{book}
\def\web{web}
\def\type{web}

\newcommand{\vect}[1]{\overrightarrow{\,\mathstrut#1\,}}

\def\Oij{$\left(\text{O}~;~\vect{\imath},~\vect{\jmath}\right)$}
\def\Oijk{$\left(\text{O}~;~\vect{\imath},~\vect{\jmath},~\vect{k}\right)$}
\def\Ouv{$\left(\text{O}~;~\vect{u},~\vect{v}\right)$}

\hypersetup{breaklinks=true, colorlinks = true, linkcolor = OliveGreen, urlcolor = OliveGreen, citecolor = OliveGreen, pdfauthor={Didier BONNEL - https://www.maths-cours.fr} } % supprime les bordures autour des liens

\renewcommand{\arg}[0]{\text{arg}}

\everymath{\displaystyle}

%================================================================================================================================
%
% Macros - Commandes
%
%================================================================================================================================

\newcommand\meta[2]{    			% Utilisé pour créer le post HTML.
	\def\titre{titre}
	\def\url{url}
	\def\arg{#1}
	\ifx\titre\arg
		\newcommand\maintitle{#2}
		\fancyhead[L]{#2}
		{\Large\sffamily \MakeUppercase{#2}}
		\vspace{1mm}\textcolor{mcvert}{\hrule}
	\fi 
	\ifx\url\arg
		\fancyfoot[L]{\href{https://www.maths-cours.fr#2}{\black \footnotesize{https://www.maths-cours.fr#2}}}
	\fi 
}


\newcommand\TitreC[1]{    		% Titre centré
     \needspace{3\baselineskip}
     \begin{center}\textbf{#1}\end{center}
}

\newcommand\newpar{    		% paragraphe
     \par
}

\newcommand\nosp {    		% commande vide (pas d'espace)
}
\newcommand{\id}[1]{} %ignore

\newcommand\boite[2]{				% Boite simple sans titre
	\vspace{5mm}
	\setlength{\fboxrule}{0.2mm}
	\setlength{\fboxsep}{5mm}	
	\fcolorbox{#1}{#1!3}{\makebox[\linewidth-2\fboxrule-2\fboxsep]{
  		\begin{minipage}[t]{\linewidth-2\fboxrule-4\fboxsep}\setlength{\parskip}{3mm}
  			 #2
  		\end{minipage}
	}}
	\vspace{5mm}
}

\newcommand\CBox[4]{				% Boites
	\vspace{5mm}
	\setlength{\fboxrule}{0.2mm}
	\setlength{\fboxsep}{5mm}
	
	\fcolorbox{#1}{#1!3}{\makebox[\linewidth-2\fboxrule-2\fboxsep]{
		\begin{minipage}[t]{1cm}\setlength{\parskip}{3mm}
	  		\textcolor{#1}{\LARGE{#2}}    
 	 	\end{minipage}  
  		\begin{minipage}[t]{\linewidth-2\fboxrule-4\fboxsep}\setlength{\parskip}{3mm}
			\raisebox{1.2mm}{\normalsize\sffamily{\textcolor{#1}{#3}}}						
  			 #4
  		\end{minipage}
	}}
	\vspace{5mm}
}

\newcommand\cadre[3]{				% Boites convertible html
	\par
	\vspace{2mm}
	\setlength{\fboxrule}{0.1mm}
	\setlength{\fboxsep}{5mm}
	\fcolorbox{#1}{white}{\makebox[\linewidth-2\fboxrule-2\fboxsep]{
  		\begin{minipage}[t]{\linewidth-2\fboxrule-4\fboxsep}\setlength{\parskip}{3mm}
			\raisebox{-2.5mm}{\sffamily \small{\textcolor{#1}{\MakeUppercase{#2}}}}		
			\par		
  			 #3
 	 		\end{minipage}
	}}
		\vspace{2mm}
	\par
}

\newcommand\bloc[3]{				% Boites convertible html sans bordure
     \needspace{2\baselineskip}
     {\sffamily \small{\textcolor{#1}{\MakeUppercase{#2}}}}    
		\par		
  			 #3
		\par
}

\newcommand\CHelp[1]{
     \CBox{Plum}{\faInfoCircle}{À RETENIR}{#1}
}

\newcommand\CUp[1]{
     \CBox{NavyBlue}{\faThumbsOUp}{EN PRATIQUE}{#1}
}

\newcommand\CInfo[1]{
     \CBox{Sepia}{\faArrowCircleRight}{REMARQUE}{#1}
}

\newcommand\CRedac[1]{
     \CBox{PineGreen}{\faEdit}{BIEN R\'EDIGER}{#1}
}

\newcommand\CError[1]{
     \CBox{Red}{\faExclamationTriangle}{ATTENTION}{#1}
}

\newcommand\TitreExo[2]{
\needspace{4\baselineskip}
 {\sffamily\large EXERCICE #1\ (\emph{#2 points})}
\vspace{5mm}
}

\newcommand\img[2]{
          \includegraphics[width=#2\paperwidth]{\imgdir#1}
}

\newcommand\imgsvg[2]{
       \begin{center}   \includegraphics[width=#2\paperwidth]{\imgsvgdir#1} \end{center}
}


\newcommand\Lien[2]{
     \href{#1}{#2 \tiny \faExternalLink}
}
\newcommand\mcLien[2]{
     \href{https~://www.maths-cours.fr/#1}{#2 \tiny \faExternalLink}
}

\newcommand{\euro}{\eurologo{}}

%================================================================================================================================
%
% Macros - Environement
%
%================================================================================================================================

\newenvironment{tex}{ %
}
{%
}

\newenvironment{indente}{ %
	\setlength\parindent{10mm}
}

{
	\setlength\parindent{0mm}
}

\newenvironment{corrige}{%
     \needspace{3\baselineskip}
     \medskip
     \textbf{\textsc{Corrigé}}
     \medskip
}
{
}

\newenvironment{extern}{%
     \begin{center}
     }
     {
     \end{center}
}

\NewEnviron{code}{%
	\par
     \boite{gray}{\texttt{%
     \BODY
     }}
     \par
}

\newenvironment{vbloc}{% boite sans cadre empeche saut de page
     \begin{minipage}[t]{\linewidth}
     }
     {
     \end{minipage}
}
\NewEnviron{h2}{%
    \needspace{3\baselineskip}
    \vspace{0.6cm}
	\noindent \MakeUppercase{\sffamily \large \BODY}
	\vspace{1mm}\textcolor{mcgris}{\hrule}\vspace{0.4cm}
	\par
}{}

\NewEnviron{h3}{%
    \needspace{3\baselineskip}
	\vspace{5mm}
	\textsc{\BODY}
	\par
}

\NewEnviron{margeneg}{ %
\begin{addmargin}[-1cm]{0cm}
\BODY
\end{addmargin}
}

\NewEnviron{html}{%
}

\begin{document}
\begin{h2}I - Intervalle de fluctuation\end{h2}
Pour étudier un caractère présent dans une population, on prélève de façon aléatoire un échantillon dans cette population.
\par
On suppose connues :
\begin{itemize}
     \item la proportion $p$ du caractère \textbf{dans la population}
     \item la taille $n$ de l'échantillon
\end{itemize}
On cherche à évaluer :
\begin{itemize}
     \item la fréquence $f$ du caractère \textbf{dans l'échantillon}
\end{itemize}
\bloc{orange}{Exemple}{% id="e10"
     On sait que 48\% des élèves d'un lycée sont des garçons (et donc 52\% sont des filles...).
     \par
     Si l'on sélectionne au hasard 100 élèves dans l'établissement, on devrait obtenir \textbf{\textit{environ}} 52 filles et 48 garçons mais il n'est pas du tout certain que l'on obtienne \textbf{exactement} ces chiffres.
     \par
     Par contre, on pourra rechercher un intervalle dans lequel se situera \textit{"probablement"} la proportion de garçons dans cet échantillon.
}
Si $n$ est élevé, on peut assimiler la sélection de l'échantillon à un tirage avec remise. Le nombre d'individus présentant le caractère étudié suit alors une loi binomiale $\mathscr B\left(n,p\right)$. Pour $n$ élevé, on peut approximer cette loi binomiale par une loi normale. On obtient alors le résultat suivant :
\cadre{bleu}{Définition et propriété} l'intervalle :
     \begin{center}$I=\left[ p-1,96\times \frac{\sqrt{p\left(1-p\right)}}{\sqrt{n}} ;\right.$\nosp$\left. p+1,96\times \frac{\sqrt{p\left(1-p\right)}}{\sqrt{n}} \right]$\end{center}
     Cela s'interprète de la façon suivante :
     \par
     Pour $n$ élevé, la probabilité que la fréquence $f$ du caractère dans l'échantillon appartienne à $I$ est 0,95.
}
\bloc{orange}{Exemple}{% id="e20"
     Si l'on reprend l'exemple précédent, on a $n=100$ et $p=\frac{48}{100}$.
     \par
     On trouve $I= \left[0,38 ; 0,58\right]$.
     \par
     La proportion de garçons dans l'échantillon devrait être comprise entre 38\% et 58\% (avec une probabilité de 0,95)
}
\bloc{cyan}{Remarques}{% id="r20"
     \begin{itemize}
          \item On considèrera que $n$ est suffisamment élevé pour utiliser cet intervalle de fluctuation si $n\geqslant 30$, $np\geqslant 5$ et $n\left(1-p\right)\geqslant 5$
          \item L' intervalle de fluctuation peut être utilisé pour valider ou rejeter une hypothèse. On procède de la façon suivante :
\begin{itemize}[label=---]
\item %
On suppose que la proportion du caractère étudié est $p$.
\item %
 On prélève un échantillon de taille $n$.
\item %
 On regarde si la fréquence $f$ du caractère dans l'échantillon appartient à $I$. 
\item %
Si oui, l'hypothèse est validée ; si non, elle est rejetée.\\
Le risque de rejeter l'hypothèse à tort est alors inférieur à 5\%.
           \end{itemize}    
          
          \item Pour des valeurs moyennes de $p$ (par exemple $0,2\leqslant p\leqslant 0,8$),  $1,96\times \sqrt{p\left(1-p\right)}$ est proche de 1 (et légèrement inférieur). Si l'on arrondit  $1,96\times \sqrt{p\left(1-p\right)}$ à 1, on obtient :
          \begin{center}$I=\left[ p-\frac{1}{\sqrt{n}} ; p+\frac{1}{\sqrt{n}} \right]$\end{center}
          qui est l'intervalle vu en Seconde.
     \end{itemize}
}
\begin{h2}II - Intervalle de confiance\end{h2}
Dans cette partie (contrairement à la première partie), on suppose que l'\textbf{on connait la fréquence $f$ du caractère dans l'échantillon} mais que l' \textbf{on ne connait pas la proportion $p$ du caractère dans la population}.
\par
On cherche alors à évaluer $p$.
\cadre{bleu}{Définition et propriété} l'intervalle :
     \begin{center}$I=\left[ f-\frac{1}{\sqrt{n}} ; f+\frac{1}{\sqrt{n}} \right]$\end{center}
     Pour $n$ élevé, la proportion $p$ du caractère dans la population appartiendra à $I$ dans 95\% des cas.
}
\bloc{orange}{Exemple}{% id="e50"
     On recherche le pourcentage de truites femelles dans un élevage de truites.
     \par
     Pour cela, on a prélevé un échantillon de 50 truites et on a comptabilisé 28 femelles dans cet échantillon.
     \par
     Le pourcentage de truites femelles dans l'ensemble de l'élevage appartient donc à l'intervalle :
     \begin{center}$I=\left[ \frac{28}{50}-\frac{1}{\sqrt{50}} ; \frac{28}{50}+\frac{1}{\sqrt{50}} \right] \approx  \left[0,42 ; 0,70\right]$\end{center}
     avec un risque d'erreur inférieur à 5\%.
}
\bloc{cyan}{Remarque}{% id="r50"
     La longueur de l'intervalle $I$ est $\frac{2}{\sqrt{n}}$.
     \par
     Si l'on souhaite obtenir un intervalle d'amplitude maximale $a$, il faut choisir $n$ tel que $\frac{2}{\sqrt{n}}\leqslant a$ c'est à dire $n\geqslant \frac{4}{a^{2}}$.
}

\end{document}