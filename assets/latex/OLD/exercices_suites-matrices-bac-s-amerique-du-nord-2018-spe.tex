\documentclass[a4paper]{article}

%================================================================================================================================
%
% Packages
%
%================================================================================================================================

\usepackage[T1]{fontenc} 	% pour caractères accentués
\usepackage[utf8]{inputenc}  % encodage utf8
\usepackage[french]{babel}	% langue : français
\usepackage{fourier}			% caractères plus lisibles
\usepackage[dvipsnames]{xcolor} % couleurs
\usepackage{fancyhdr}		% réglage header footer
\usepackage{needspace}		% empêcher sauts de page mal placés
\usepackage{graphicx}		% pour inclure des graphiques
\usepackage{enumitem,cprotect}		% personnalise les listes d'items (nécessaire pour ol, al ...)
\usepackage{hyperref}		% Liens hypertexte
\usepackage{pstricks,pst-all,pst-node,pstricks-add,pst-math,pst-plot,pst-tree,pst-eucl} % pstricks
\usepackage[a4paper,includeheadfoot,top=2cm,left=3cm, bottom=2cm,right=3cm]{geometry} % marges etc.
\usepackage{comment}			% commentaires multilignes
\usepackage{amsmath,environ} % maths (matrices, etc.)
\usepackage{amssymb,makeidx}
\usepackage{bm}				% bold maths
\usepackage{tabularx}		% tableaux
\usepackage{colortbl}		% tableaux en couleur
\usepackage{fontawesome}		% Fontawesome
\usepackage{environ}			% environment with command
\usepackage{fp}				% calculs pour ps-tricks
\usepackage{multido}			% pour ps tricks
\usepackage[np]{numprint}	% formattage nombre
\usepackage{tikz,tkz-tab} 			% package principal TikZ
\usepackage{pgfplots}   % axes
\usepackage{mathrsfs}    % cursives
\usepackage{calc}			% calcul taille boites
\usepackage[scaled=0.875]{helvet} % font sans serif
\usepackage{svg} % svg
\usepackage{scrextend} % local margin
\usepackage{scratch} %scratch
\usepackage{multicol} % colonnes
%\usepackage{infix-RPN,pst-func} % formule en notation polanaise inversée
\usepackage{listings}

%================================================================================================================================
%
% Réglages de base
%
%================================================================================================================================

\lstset{
language=Python,   % R code
literate=
{á}{{\'a}}1
{à}{{\`a}}1
{ã}{{\~a}}1
{é}{{\'e}}1
{è}{{\`e}}1
{ê}{{\^e}}1
{í}{{\'i}}1
{ó}{{\'o}}1
{õ}{{\~o}}1
{ú}{{\'u}}1
{ü}{{\"u}}1
{ç}{{\c{c}}}1
{~}{{ }}1
}


\definecolor{codegreen}{rgb}{0,0.6,0}
\definecolor{codegray}{rgb}{0.5,0.5,0.5}
\definecolor{codepurple}{rgb}{0.58,0,0.82}
\definecolor{backcolour}{rgb}{0.95,0.95,0.92}

\lstdefinestyle{mystyle}{
    backgroundcolor=\color{backcolour},   
    commentstyle=\color{codegreen},
    keywordstyle=\color{magenta},
    numberstyle=\tiny\color{codegray},
    stringstyle=\color{codepurple},
    basicstyle=\ttfamily\footnotesize,
    breakatwhitespace=false,         
    breaklines=true,                 
    captionpos=b,                    
    keepspaces=true,                 
    numbers=left,                    
xleftmargin=2em,
framexleftmargin=2em,            
    showspaces=false,                
    showstringspaces=false,
    showtabs=false,                  
    tabsize=2,
    upquote=true
}

\lstset{style=mystyle}


\lstset{style=mystyle}
\newcommand{\imgdir}{C:/laragon/www/newmc/assets/imgsvg/}
\newcommand{\imgsvgdir}{C:/laragon/www/newmc/assets/imgsvg/}

\definecolor{mcgris}{RGB}{220, 220, 220}% ancien~; pour compatibilité
\definecolor{mcbleu}{RGB}{52, 152, 219}
\definecolor{mcvert}{RGB}{125, 194, 70}
\definecolor{mcmauve}{RGB}{154, 0, 215}
\definecolor{mcorange}{RGB}{255, 96, 0}
\definecolor{mcturquoise}{RGB}{0, 153, 153}
\definecolor{mcrouge}{RGB}{255, 0, 0}
\definecolor{mclightvert}{RGB}{205, 234, 190}

\definecolor{gris}{RGB}{220, 220, 220}
\definecolor{bleu}{RGB}{52, 152, 219}
\definecolor{vert}{RGB}{125, 194, 70}
\definecolor{mauve}{RGB}{154, 0, 215}
\definecolor{orange}{RGB}{255, 96, 0}
\definecolor{turquoise}{RGB}{0, 153, 153}
\definecolor{rouge}{RGB}{255, 0, 0}
\definecolor{lightvert}{RGB}{205, 234, 190}
\setitemize[0]{label=\color{lightvert}  $\bullet$}

\pagestyle{fancy}
\renewcommand{\headrulewidth}{0.2pt}
\fancyhead[L]{maths-cours.fr}
\fancyhead[R]{\thepage}
\renewcommand{\footrulewidth}{0.2pt}
\fancyfoot[C]{}

\newcolumntype{C}{>{\centering\arraybackslash}X}
\newcolumntype{s}{>{\hsize=.35\hsize\arraybackslash}X}

\setlength{\parindent}{0pt}		 
\setlength{\parskip}{3mm}
\setlength{\headheight}{1cm}

\def\ebook{ebook}
\def\book{book}
\def\web{web}
\def\type{web}

\newcommand{\vect}[1]{\overrightarrow{\,\mathstrut#1\,}}

\def\Oij{$\left(\text{O}~;~\vect{\imath},~\vect{\jmath}\right)$}
\def\Oijk{$\left(\text{O}~;~\vect{\imath},~\vect{\jmath},~\vect{k}\right)$}
\def\Ouv{$\left(\text{O}~;~\vect{u},~\vect{v}\right)$}

\hypersetup{breaklinks=true, colorlinks = true, linkcolor = OliveGreen, urlcolor = OliveGreen, citecolor = OliveGreen, pdfauthor={Didier BONNEL - https://www.maths-cours.fr} } % supprime les bordures autour des liens

\renewcommand{\arg}[0]{\text{arg}}

\everymath{\displaystyle}

%================================================================================================================================
%
% Macros - Commandes
%
%================================================================================================================================

\newcommand\meta[2]{    			% Utilisé pour créer le post HTML.
	\def\titre{titre}
	\def\url{url}
	\def\arg{#1}
	\ifx\titre\arg
		\newcommand\maintitle{#2}
		\fancyhead[L]{#2}
		{\Large\sffamily \MakeUppercase{#2}}
		\vspace{1mm}\textcolor{mcvert}{\hrule}
	\fi 
	\ifx\url\arg
		\fancyfoot[L]{\href{https://www.maths-cours.fr#2}{\black \footnotesize{https://www.maths-cours.fr#2}}}
	\fi 
}


\newcommand\TitreC[1]{    		% Titre centré
     \needspace{3\baselineskip}
     \begin{center}\textbf{#1}\end{center}
}

\newcommand\newpar{    		% paragraphe
     \par
}

\newcommand\nosp {    		% commande vide (pas d'espace)
}
\newcommand{\id}[1]{} %ignore

\newcommand\boite[2]{				% Boite simple sans titre
	\vspace{5mm}
	\setlength{\fboxrule}{0.2mm}
	\setlength{\fboxsep}{5mm}	
	\fcolorbox{#1}{#1!3}{\makebox[\linewidth-2\fboxrule-2\fboxsep]{
  		\begin{minipage}[t]{\linewidth-2\fboxrule-4\fboxsep}\setlength{\parskip}{3mm}
  			 #2
  		\end{minipage}
	}}
	\vspace{5mm}
}

\newcommand\CBox[4]{				% Boites
	\vspace{5mm}
	\setlength{\fboxrule}{0.2mm}
	\setlength{\fboxsep}{5mm}
	
	\fcolorbox{#1}{#1!3}{\makebox[\linewidth-2\fboxrule-2\fboxsep]{
		\begin{minipage}[t]{1cm}\setlength{\parskip}{3mm}
	  		\textcolor{#1}{\LARGE{#2}}    
 	 	\end{minipage}  
  		\begin{minipage}[t]{\linewidth-2\fboxrule-4\fboxsep}\setlength{\parskip}{3mm}
			\raisebox{1.2mm}{\normalsize\sffamily{\textcolor{#1}{#3}}}						
  			 #4
  		\end{minipage}
	}}
	\vspace{5mm}
}

\newcommand\cadre[3]{				% Boites convertible html
	\par
	\vspace{2mm}
	\setlength{\fboxrule}{0.1mm}
	\setlength{\fboxsep}{5mm}
	\fcolorbox{#1}{white}{\makebox[\linewidth-2\fboxrule-2\fboxsep]{
  		\begin{minipage}[t]{\linewidth-2\fboxrule-4\fboxsep}\setlength{\parskip}{3mm}
			\raisebox{-2.5mm}{\sffamily \small{\textcolor{#1}{\MakeUppercase{#2}}}}		
			\par		
  			 #3
 	 		\end{minipage}
	}}
		\vspace{2mm}
	\par
}

\newcommand\bloc[3]{				% Boites convertible html sans bordure
     \needspace{2\baselineskip}
     {\sffamily \small{\textcolor{#1}{\MakeUppercase{#2}}}}    
		\par		
  			 #3
		\par
}

\newcommand\CHelp[1]{
     \CBox{Plum}{\faInfoCircle}{À RETENIR}{#1}
}

\newcommand\CUp[1]{
     \CBox{NavyBlue}{\faThumbsOUp}{EN PRATIQUE}{#1}
}

\newcommand\CInfo[1]{
     \CBox{Sepia}{\faArrowCircleRight}{REMARQUE}{#1}
}

\newcommand\CRedac[1]{
     \CBox{PineGreen}{\faEdit}{BIEN R\'EDIGER}{#1}
}

\newcommand\CError[1]{
     \CBox{Red}{\faExclamationTriangle}{ATTENTION}{#1}
}

\newcommand\TitreExo[2]{
\needspace{4\baselineskip}
 {\sffamily\large EXERCICE #1\ (\emph{#2 points})}
\vspace{5mm}
}

\newcommand\img[2]{
          \includegraphics[width=#2\paperwidth]{\imgdir#1}
}

\newcommand\imgsvg[2]{
       \begin{center}   \includegraphics[width=#2\paperwidth]{\imgsvgdir#1} \end{center}
}


\newcommand\Lien[2]{
     \href{#1}{#2 \tiny \faExternalLink}
}
\newcommand\mcLien[2]{
     \href{https~://www.maths-cours.fr/#1}{#2 \tiny \faExternalLink}
}

\newcommand{\euro}{\eurologo{}}

%================================================================================================================================
%
% Macros - Environement
%
%================================================================================================================================

\newenvironment{tex}{ %
}
{%
}

\newenvironment{indente}{ %
	\setlength\parindent{10mm}
}

{
	\setlength\parindent{0mm}
}

\newenvironment{corrige}{%
     \needspace{3\baselineskip}
     \medskip
     \textbf{\textsc{Corrigé}}
     \medskip
}
{
}

\newenvironment{extern}{%
     \begin{center}
     }
     {
     \end{center}
}

\NewEnviron{code}{%
	\par
     \boite{gray}{\texttt{%
     \BODY
     }}
     \par
}

\newenvironment{vbloc}{% boite sans cadre empeche saut de page
     \begin{minipage}[t]{\linewidth}
     }
     {
     \end{minipage}
}
\NewEnviron{h2}{%
    \needspace{3\baselineskip}
    \vspace{0.6cm}
	\noindent \MakeUppercase{\sffamily \large \BODY}
	\vspace{1mm}\textcolor{mcgris}{\hrule}\vspace{0.4cm}
	\par
}{}

\NewEnviron{h3}{%
    \needspace{3\baselineskip}
	\vspace{5mm}
	\textsc{\BODY}
	\par
}

\NewEnviron{margeneg}{ %
\begin{addmargin}[-1cm]{0cm}
\BODY
\end{addmargin}
}

\NewEnviron{html}{%
}

\begin{document}
\begin{h2}Exercice 4 (5 points)\end{h2}
\textbf{Candidats ayant suivi l'enseignement de spécialité}
\medskip
Dans une région, on s'intéresse à la cohabitation de deux espèces animales~: les campagnols et les
renards, les renards étant les prédateurs des campagnols.
\par
Au 1$^{\text{er}}$ juillet 2012, on estime qu'il y a dans cette région approximativement deux millions de campagnols et cent-vingt renards.
\par
On note $u_n$ le nombre de campagnols et $v_n$ le nombre de renards au 1\up{er} juillet de l'année $2012+ n$.
\bigskip
\begin{center}\begin{h3}Partie A - Un modèle simple \end{h3}\end{center}
\medskip
On modélise l'évolution des populations par les relations suivantes~:
\par
\[\left\{\begin{array}{l}
          u_{n+1} = 1,1u_n - 2~000v_n\\
          v_{n+1}= 2 \times 10^{-5}u_n + 0,6v_n
\end{array}\right.\]
pour tout entier $n \geqslant 0$, avec $u_0 = 2~000~000$ et $v_0 = 120$.
\medskip
\begin{enumerate}
     \item
     \begin{enumerate}[label=\alph*.]
          \item On considère la matrice colonne $U_n = \begin{pmatrix}u_n\\v_n\end{pmatrix}$ pour tout entier $n \geqslant 0$.
          \par
          Déterminer la matrice $A$ telle que $U_{n+1} = A \times U_n$ pour tout entier $n$ et donner la matrice $U_0$.
          \item Calculer le nombre de campagnols et de renards estimés grâce à ce modèle au 1\up{er} juillet
          2018.
     \end{enumerate}
     \item Soit les matrices $P = \begin{pmatrix}20~000&5~000\\1&1\end{pmatrix}$, \:$D = \begin{pmatrix}1&0\\0&0,7\end{pmatrix}$ et $P^{- 1}  = \dfrac{1}{15~000}\begin{pmatrix}1& -5~000\\- 1&20~000\end{pmatrix}$.
     \par
     On admet que $P^{- 1}$ est la matrice inverse de la matrice $P$ et que $A = P \times D \times P^{- 1}$.
     \begin{enumerate}[label=\alph*.]
          \item Montrer que pour tout entier naturel $n$,\: $U_n = P \times D^n \times P^{- 1} \times U_0$.
          \item Donner sans justification l'expression de la matrice $D^n$ en fonction de $n$.
          \item On admet que, pour tout entier naturel $n$~:
          \par
          \[\left\{\begin{array}{l}
                    u_n = \dfrac{2,8 \times 10^7 + 2 \times 10^6 \times 0,7^n}{15}\\
                    \\
                    v_n =\dfrac{1~400 + 400 \times 0,7^n}{15}\\
          \end{array}\right.\]
          Décrire l'évolution des deux populations.
     \end{enumerate}
\end{enumerate}
\bigskip
\begin{center}\begin{h3}Partie B - Un modèle plus conforme à la réalité \end{h3}\end{center}
\medskip
Dans la réalité, on observe que si le nombre de renards a suffisamment baissé, alors le nombre de
campagnols augmente à nouveau, ce qui n'est pas le cas avec le modèle précédent.
\par
On construit donc un autre modèle, plus précis, qui tient compte de ce type d'observations à l'aide des relations suivantes~:
\par
\[\left\{\begin{array}{l}
          u_{n+1} = 1,1u_n - 0,001ru_n \times v_n\\
          v_{n+1} = 2 \times 10^{-7} u_n \times v_n + 0,6v_n
\end{array}\right.\]
pour tout entier $n \geqslant 0$, avec $u_0 = 2~000~000$ et $v_0 = 120$.
\medskip
Le tableau ci-dessous présente ce nouveau modèle sur les $25$ premières années en donnant les
effectifs des populations arrondis à l'unité~:
\begin{center}
     \begin{extern}%width="400"
          \begin{tabularx}{0.7\linewidth}{|>{\columncolor[gray]{0.7}}c|*{3}{>{\centering \arraybackslash}X|}}\hline
               \rowcolor[gray]{0.7}&A &B &C\\ \hline
               1& \multicolumn{3}{c|}{Modèle de la \textbf{partie B}}\\ \hline
               2& $n$ 	&$u_n$ 			&$v_n$\\ \hline
               3&0		& 2~000~000 	&120\\ \hline
               4&1		& 1~960~000 	&120\\ \hline
               5&2		& 1~920~800 	&119\\ \hline
               6&3		& 1~884~228 	&117\\ \hline
               7&4		& 1~851~905 	&114\\ \hline
               8&5		& 1~825~160 	&111\\ \hline
               9&6		& 1~804~988 	&107\\ \hline
               10&7	& 1~792~049 	&103\\ \hline
               11&8	& 1~786~692 	&99\\ \hline
               12&9	& 1~789~005 	&94\\ \hline
               13&10	& 1~798~854 	&91\\ \hline
               14&11	& 1~815~930 	&87\\ \hline
               15&12	& 1~839~780 	&84\\ \hline
               16&13	& 1~869~827 	&81\\ \hline
               17&14	& 1~905~378 	&79\\ \hline
               18&15	& 1~945~622 	&77\\ \hline
               19&16	& 1~989~620 	&77\\ \hline
               20&17	& 2~036~288 	&76\\ \hline
               21&18	& 2~084~374 	&77\\ \hline
               22&19	& 2~132~440 	&78\\ \hline
               23&20	& 2~178~846 	&80\\ \hline
               24&21	& 2~221~746 	&83\\ \hline
               25&22	& 2~259~109 	&87\\ \hline
               26&23	& 2~288~766 	&91\\ \hline
               27&24	& 2~308~508 	&97\\ \hline
          \end{tabularx}
     \end{extern}
\end{center}
\medskip
\begin{enumerate}
     \item Quelles formules faut-il écrire dans les cellules B4 et C4 et recopier vers le bas pour remplir
     les colonnes B et C~?
     \item  Avec le deuxième modèle, à partir de quelle année observe-t-on le phénomène décrit (baisse
     des renards et hausse des campagnols)~?
\end{enumerate}
\bigskip
\begin{center}\begin{h3}Partie C \end{h3}\end{center}
\medskip
Dans cette partie on utilise le modèle de la partie B.
\par
Est-il possible de donner à $u_0$ et $v_0$ des valeurs afin que les deux populations restent stables d'une
année sur l'autre, c'est-à-dire telles que pour tout entier naturel $n$ on ait $u_{n+1} = u_n$ et $v_{n+1} = v_n$~? (On parle alors d'état stable.)
\par
\par

\end{document}