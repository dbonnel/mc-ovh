\documentclass[a4paper]{article}

%================================================================================================================================
%
% Packages
%
%================================================================================================================================

\usepackage[T1]{fontenc} 	% pour caractères accentués
\usepackage[utf8]{inputenc}  % encodage utf8
\usepackage[french]{babel}	% langue : français
\usepackage{fourier}			% caractères plus lisibles
\usepackage[dvipsnames]{xcolor} % couleurs
\usepackage{fancyhdr}		% réglage header footer
\usepackage{needspace}		% empêcher sauts de page mal placés
\usepackage{graphicx}		% pour inclure des graphiques
\usepackage{enumitem,cprotect}		% personnalise les listes d'items (nécessaire pour ol, al ...)
\usepackage{hyperref}		% Liens hypertexte
\usepackage{pstricks,pst-all,pst-node,pstricks-add,pst-math,pst-plot,pst-tree,pst-eucl} % pstricks
\usepackage[a4paper,includeheadfoot,top=2cm,left=3cm, bottom=2cm,right=3cm]{geometry} % marges etc.
\usepackage{comment}			% commentaires multilignes
\usepackage{amsmath,environ} % maths (matrices, etc.)
\usepackage{amssymb,makeidx}
\usepackage{bm}				% bold maths
\usepackage{tabularx}		% tableaux
\usepackage{colortbl}		% tableaux en couleur
\usepackage{fontawesome}		% Fontawesome
\usepackage{environ}			% environment with command
\usepackage{fp}				% calculs pour ps-tricks
\usepackage{multido}			% pour ps tricks
\usepackage[np]{numprint}	% formattage nombre
\usepackage{tikz,tkz-tab} 			% package principal TikZ
\usepackage{pgfplots}   % axes
\usepackage{mathrsfs}    % cursives
\usepackage{calc}			% calcul taille boites
\usepackage[scaled=0.875]{helvet} % font sans serif
\usepackage{svg} % svg
\usepackage{scrextend} % local margin
\usepackage{scratch} %scratch
\usepackage{multicol} % colonnes
%\usepackage{infix-RPN,pst-func} % formule en notation polanaise inversée
\usepackage{listings}

%================================================================================================================================
%
% Réglages de base
%
%================================================================================================================================

\lstset{
language=Python,   % R code
literate=
{á}{{\'a}}1
{à}{{\`a}}1
{ã}{{\~a}}1
{é}{{\'e}}1
{è}{{\`e}}1
{ê}{{\^e}}1
{í}{{\'i}}1
{ó}{{\'o}}1
{õ}{{\~o}}1
{ú}{{\'u}}1
{ü}{{\"u}}1
{ç}{{\c{c}}}1
{~}{{ }}1
}


\definecolor{codegreen}{rgb}{0,0.6,0}
\definecolor{codegray}{rgb}{0.5,0.5,0.5}
\definecolor{codepurple}{rgb}{0.58,0,0.82}
\definecolor{backcolour}{rgb}{0.95,0.95,0.92}

\lstdefinestyle{mystyle}{
    backgroundcolor=\color{backcolour},   
    commentstyle=\color{codegreen},
    keywordstyle=\color{magenta},
    numberstyle=\tiny\color{codegray},
    stringstyle=\color{codepurple},
    basicstyle=\ttfamily\footnotesize,
    breakatwhitespace=false,         
    breaklines=true,                 
    captionpos=b,                    
    keepspaces=true,                 
    numbers=left,                    
xleftmargin=2em,
framexleftmargin=2em,            
    showspaces=false,                
    showstringspaces=false,
    showtabs=false,                  
    tabsize=2,
    upquote=true
}

\lstset{style=mystyle}


\lstset{style=mystyle}
\newcommand{\imgdir}{C:/laragon/www/newmc/assets/imgsvg/}
\newcommand{\imgsvgdir}{C:/laragon/www/newmc/assets/imgsvg/}

\definecolor{mcgris}{RGB}{220, 220, 220}% ancien~; pour compatibilité
\definecolor{mcbleu}{RGB}{52, 152, 219}
\definecolor{mcvert}{RGB}{125, 194, 70}
\definecolor{mcmauve}{RGB}{154, 0, 215}
\definecolor{mcorange}{RGB}{255, 96, 0}
\definecolor{mcturquoise}{RGB}{0, 153, 153}
\definecolor{mcrouge}{RGB}{255, 0, 0}
\definecolor{mclightvert}{RGB}{205, 234, 190}

\definecolor{gris}{RGB}{220, 220, 220}
\definecolor{bleu}{RGB}{52, 152, 219}
\definecolor{vert}{RGB}{125, 194, 70}
\definecolor{mauve}{RGB}{154, 0, 215}
\definecolor{orange}{RGB}{255, 96, 0}
\definecolor{turquoise}{RGB}{0, 153, 153}
\definecolor{rouge}{RGB}{255, 0, 0}
\definecolor{lightvert}{RGB}{205, 234, 190}
\setitemize[0]{label=\color{lightvert}  $\bullet$}

\pagestyle{fancy}
\renewcommand{\headrulewidth}{0.2pt}
\fancyhead[L]{maths-cours.fr}
\fancyhead[R]{\thepage}
\renewcommand{\footrulewidth}{0.2pt}
\fancyfoot[C]{}

\newcolumntype{C}{>{\centering\arraybackslash}X}
\newcolumntype{s}{>{\hsize=.35\hsize\arraybackslash}X}

\setlength{\parindent}{0pt}		 
\setlength{\parskip}{3mm}
\setlength{\headheight}{1cm}

\def\ebook{ebook}
\def\book{book}
\def\web{web}
\def\type{web}

\newcommand{\vect}[1]{\overrightarrow{\,\mathstrut#1\,}}

\def\Oij{$\left(\text{O}~;~\vect{\imath},~\vect{\jmath}\right)$}
\def\Oijk{$\left(\text{O}~;~\vect{\imath},~\vect{\jmath},~\vect{k}\right)$}
\def\Ouv{$\left(\text{O}~;~\vect{u},~\vect{v}\right)$}

\hypersetup{breaklinks=true, colorlinks = true, linkcolor = OliveGreen, urlcolor = OliveGreen, citecolor = OliveGreen, pdfauthor={Didier BONNEL - https://www.maths-cours.fr} } % supprime les bordures autour des liens

\renewcommand{\arg}[0]{\text{arg}}

\everymath{\displaystyle}

%================================================================================================================================
%
% Macros - Commandes
%
%================================================================================================================================

\newcommand\meta[2]{    			% Utilisé pour créer le post HTML.
	\def\titre{titre}
	\def\url{url}
	\def\arg{#1}
	\ifx\titre\arg
		\newcommand\maintitle{#2}
		\fancyhead[L]{#2}
		{\Large\sffamily \MakeUppercase{#2}}
		\vspace{1mm}\textcolor{mcvert}{\hrule}
	\fi 
	\ifx\url\arg
		\fancyfoot[L]{\href{https://www.maths-cours.fr#2}{\black \footnotesize{https://www.maths-cours.fr#2}}}
	\fi 
}


\newcommand\TitreC[1]{    		% Titre centré
     \needspace{3\baselineskip}
     \begin{center}\textbf{#1}\end{center}
}

\newcommand\newpar{    		% paragraphe
     \par
}

\newcommand\nosp {    		% commande vide (pas d'espace)
}
\newcommand{\id}[1]{} %ignore

\newcommand\boite[2]{				% Boite simple sans titre
	\vspace{5mm}
	\setlength{\fboxrule}{0.2mm}
	\setlength{\fboxsep}{5mm}	
	\fcolorbox{#1}{#1!3}{\makebox[\linewidth-2\fboxrule-2\fboxsep]{
  		\begin{minipage}[t]{\linewidth-2\fboxrule-4\fboxsep}\setlength{\parskip}{3mm}
  			 #2
  		\end{minipage}
	}}
	\vspace{5mm}
}

\newcommand\CBox[4]{				% Boites
	\vspace{5mm}
	\setlength{\fboxrule}{0.2mm}
	\setlength{\fboxsep}{5mm}
	
	\fcolorbox{#1}{#1!3}{\makebox[\linewidth-2\fboxrule-2\fboxsep]{
		\begin{minipage}[t]{1cm}\setlength{\parskip}{3mm}
	  		\textcolor{#1}{\LARGE{#2}}    
 	 	\end{minipage}  
  		\begin{minipage}[t]{\linewidth-2\fboxrule-4\fboxsep}\setlength{\parskip}{3mm}
			\raisebox{1.2mm}{\normalsize\sffamily{\textcolor{#1}{#3}}}						
  			 #4
  		\end{minipage}
	}}
	\vspace{5mm}
}

\newcommand\cadre[3]{				% Boites convertible html
	\par
	\vspace{2mm}
	\setlength{\fboxrule}{0.1mm}
	\setlength{\fboxsep}{5mm}
	\fcolorbox{#1}{white}{\makebox[\linewidth-2\fboxrule-2\fboxsep]{
  		\begin{minipage}[t]{\linewidth-2\fboxrule-4\fboxsep}\setlength{\parskip}{3mm}
			\raisebox{-2.5mm}{\sffamily \small{\textcolor{#1}{\MakeUppercase{#2}}}}		
			\par		
  			 #3
 	 		\end{minipage}
	}}
		\vspace{2mm}
	\par
}

\newcommand\bloc[3]{				% Boites convertible html sans bordure
     \needspace{2\baselineskip}
     {\sffamily \small{\textcolor{#1}{\MakeUppercase{#2}}}}    
		\par		
  			 #3
		\par
}

\newcommand\CHelp[1]{
     \CBox{Plum}{\faInfoCircle}{À RETENIR}{#1}
}

\newcommand\CUp[1]{
     \CBox{NavyBlue}{\faThumbsOUp}{EN PRATIQUE}{#1}
}

\newcommand\CInfo[1]{
     \CBox{Sepia}{\faArrowCircleRight}{REMARQUE}{#1}
}

\newcommand\CRedac[1]{
     \CBox{PineGreen}{\faEdit}{BIEN R\'EDIGER}{#1}
}

\newcommand\CError[1]{
     \CBox{Red}{\faExclamationTriangle}{ATTENTION}{#1}
}

\newcommand\TitreExo[2]{
\needspace{4\baselineskip}
 {\sffamily\large EXERCICE #1\ (\emph{#2 points})}
\vspace{5mm}
}

\newcommand\img[2]{
          \includegraphics[width=#2\paperwidth]{\imgdir#1}
}

\newcommand\imgsvg[2]{
       \begin{center}   \includegraphics[width=#2\paperwidth]{\imgsvgdir#1} \end{center}
}


\newcommand\Lien[2]{
     \href{#1}{#2 \tiny \faExternalLink}
}
\newcommand\mcLien[2]{
     \href{https~://www.maths-cours.fr/#1}{#2 \tiny \faExternalLink}
}

\newcommand{\euro}{\eurologo{}}

%================================================================================================================================
%
% Macros - Environement
%
%================================================================================================================================

\newenvironment{tex}{ %
}
{%
}

\newenvironment{indente}{ %
	\setlength\parindent{10mm}
}

{
	\setlength\parindent{0mm}
}

\newenvironment{corrige}{%
     \needspace{3\baselineskip}
     \medskip
     \textbf{\textsc{Corrigé}}
     \medskip
}
{
}

\newenvironment{extern}{%
     \begin{center}
     }
     {
     \end{center}
}

\NewEnviron{code}{%
	\par
     \boite{gray}{\texttt{%
     \BODY
     }}
     \par
}

\newenvironment{vbloc}{% boite sans cadre empeche saut de page
     \begin{minipage}[t]{\linewidth}
     }
     {
     \end{minipage}
}
\NewEnviron{h2}{%
    \needspace{3\baselineskip}
    \vspace{0.6cm}
	\noindent \MakeUppercase{\sffamily \large \BODY}
	\vspace{1mm}\textcolor{mcgris}{\hrule}\vspace{0.4cm}
	\par
}{}

\NewEnviron{h3}{%
    \needspace{3\baselineskip}
	\vspace{5mm}
	\textsc{\BODY}
	\par
}

\NewEnviron{margeneg}{ %
\begin{addmargin}[-1cm]{0cm}
\BODY
\end{addmargin}
}

\NewEnviron{html}{%
}

\begin{document}
\begin{h2}1. Fonctions exponentielles de base $q$\end{h2}
\cadre{rouge}{Théorème et définition}{% id="t10"
     Soit $q$ un réel strictement positif.
     \par
     Il existe une unique fonction $f$ définie et dérivable sur $\mathbb{R}$ telle que :
     \begin{itemize}
          \item pour tout entier $n \in  \mathbb{Z}$, $f\left(n\right)=q^{n}$
          \item pour tous réels $x$ et $y$ : $f\left(x+y\right)=f\left(x\right)\times f\left(y\right)  $ \textit{(relation fonctionnelle})
     \end{itemize}
     Cette fonction s'appelle fonction \textbf{exponentielle de base $q$} et on note $f\left(x\right)=q^{x}$
}
\bloc{cyan}{Remarques}{% id="r10"
     \begin{itemize}\item D'après la première propriété et les formules vues au collège, on a notamment : $q^{1}=q$, $q^{0}=1$, $q^{-1}=\frac{1}{q}$
          \item Avec la notation exponentielle, la seconde propriété  (relation fonctionnelle) s'écrit : $q^{x+y}=q^{x}\times q^{y}$.
          \par
          A partir de cette propriété on montre également que pour tout $q > 0$ et tous réels $x$ et $y$ :
          \par
          $q^{x-y}=\frac{q^{x}}{q^{y}} $ (en particulier $q^{-y}=\frac{1}{q^{y}}$)
          \par
          $\left[q^{x}\right] ^{y}=q^{xy}$
          \par
          ce qui généralise les propriétés vues au collège.
          \item La courbe de la fonction $x\mapsto q^{n}$ s'obtient en reliant les points de coordonnées $\left(n, q^{n}\right)$. Pour $n\geqslant 0$ ces points représentent la suite géométrique de premier terme $u_{0}=1$ et de raison $q$.
     \end{itemize}
     \begin{center}
          \begin{extern}%width="460" alt="fonction exponentielle et suite géométrique"
               % -+-+-+ variables modifiables
          \resizebox{8cm}{!}{%
                    \def\xmin{-2.5}
                    \def\xmax{8.5}
                    \def\ymin{-0.9}
                    \def\ymax{9.5}
                    \def\xunit{1}  % unités en cm
                    \def\yunit{1}
                    \psset{xunit=\xunit,yunit=\yunit,algebraic=true}
                    \fontsize{12pt}{12pt}\selectfont
                    \begin{pspicture*}[linewidth=1pt](\xmin,\ymin)(\xmax,\ymax)
                    \psgrid[gridcolor=mcgris,subgriddiv=0](-4,-1)(9,10)
                         \psaxes[linewidth=0.75pt]{->}(0,0)(\xmin,\ymin)(\xmax,\ymax)
                         \rput[tr](-0.2,-0.2){$O$}
                         \multido{\n=0.0+1}{7}{
                              \FPeval{\suite}{1.4^\n}
                              \psdots[linecolor=red,dotsize=4pt](\n,\suite)
                         }            
                           \psplot[plotpoints=1000,linewidth=0.8pt,linecolor=blue]{\xmin}{\xmax}{1.4^x}
                    \end{pspicture*}
           }
          \end{extern}
     \end{center}     \begin{center}\textit{Fonction exponentielle de base $q=1,4$}\\
     \textit{(les points correspondent à la suite géométrique $u_{0}=1$ et $q=1.4$)}\end{center}
}
\cadre{vert}{Propriété}{% id="p15"
     Pour tout réel $x$ et tout réel $q > 0$, $q^{x}$ est \textbf{strictement positif}.
}
\cadre{vert}{Propriété}{% id="p20"
     \begin{itemize}
          \item Pour $q > 1$, la fonction $x \mapsto  q^{x}$ est strictement croissante sur $\mathbb{R}$
          \item Pour $0 < q < 1$, la fonction $x \mapsto  q^{x}$ est strictement décroissante sur $\mathbb{R}$
     \end{itemize}
}
               % -+-+-+ variables modifiables
\begin{center}
          \begin{extern}%width="460" alt="fonction exponentielle de base supérieure à 1"
               % -+-+-+ variables modifiables
          \resizebox{8cm}{!}{%
                    \def\xmin{-2.5}
                    \def\xmax{8.5}
                    \def\ymin{-0.9}
                    \def\ymax{9.5}
                    \def\xunit{1}  % unités en cm
                    \def\yunit{1}
                    \psset{xunit=\xunit,yunit=\yunit,algebraic=true}
                    \fontsize{12pt}{12pt}\selectfont
                    \begin{pspicture*}[linewidth=1pt](\xmin,\ymin)(\xmax,\ymax)
                    \psgrid[gridcolor=mcgris,subgriddiv=0](-4,-1)(9,10)
                         \psaxes[linewidth=0.75pt]{->}(0,0)(\xmin,\ymin)(\xmax,\ymax)
                         \rput[tr](-0.2,-0.2){$O$}
                           \psplot[plotpoints=1000,linewidth=0.8pt,linecolor=blue]{\xmin}{\xmax}{1.4^x}
                    \end{pspicture*}
           }
          \end{extern}
\end{center}
         \begin{center}\textit{Fonction exponentielle de base $q > 1$}\end{center}
\begin{center}
          \begin{extern}%width="460" alt="fonction exponentielle de base inférieure à 1"
               % -+-+-+ variables modifiables
          \resizebox{8cm}{!}{%
                    \def\xmin{-5.5}
                    \def\xmax{6.5}
                    \def\ymin{-0.9}
                    \def\ymax{8.5}
                    \def\xunit{1}  % unités en cm
                    \def\yunit{1}
                    \psset{xunit=\xunit,yunit=\yunit,algebraic=true}
                    \fontsize{12pt}{12pt}\selectfont
                    \begin{pspicture*}[linewidth=1pt](\xmin,\ymin)(\xmax,\ymax)
                    \psgrid[gridcolor=mcgris,subgriddiv=0](-6,-1)(7,9)
                         \psaxes[linewidth=0.75pt]{->}(0,0)(\xmin,\ymin)(\xmax,\ymax)
                         \rput[tr](-0.2,-0.2){$O$}       
                           \psplot[plotpoints=1000,linewidth=0.8pt,linecolor=blue]{\xmin}{\xmax}{0.7^x}
                    \end{pspicture*}
           }
          \end{extern}
\end{center}
           \begin{center}\textit{Fonction exponentielle de base $0 < q < 1$}\end{center}
\bloc{cyan}{Remarque}{% id="r20"
     Pour $q=1$, la fonction $x \mapsto  q^{x}$ est constante et égale à $1$. Sa courbe représentative est une droite parallèle à l'axe des abscisses.
}
\begin{h2}2. Fonction exponentielle (de base $e$)\end{h2}
\cadre{rouge}{Théorème et Définition}{% id="t50"
     Il existe une valeur de $q$ pour laquelle la fonction $f : x\mapsto q^{x}$ vérifie $f^{\prime}\left(0\right)=1$.
     \par
     Cette valeur est notée $e$.
     \par
     La fonction  $x \mapsto  e^{x}$ (parfois notée $\text{exp}$) est appelée \textbf{fonction exponentielle}.
}
\bloc{cyan}{Remarque}{% id="r50"
     Le nombre $e$ est approximativement égal à $2,71828$ (on l'obtient à la calculatrice en faisant $e^{1}$ ou $\text{exp}\left(1\right)$.
}
\cadre{vert}{Propriété}{% id="p60"
     La fonction exponentielle est \textbf{strictement positive} et \textbf{strictement croissante} et sur $\mathbb{R}$.
}
\bloc{cyan}{Démonstration}{% id="m60"
     Cela résulte du fait que $e > 1$ et des résultats de la section précédente.
}
\begin{center}
          \begin{extern}%width="400" alt="fonction exponentielle de base e"
               % -+-+-+ variables modifiables
          \resizebox{8cm}{!}{%
                    \def\xmin{-4.5}
                    \def\xmax{4.5}
                    \def\ymin{-0.9}
                    \def\ymax{8.5}
                    \def\xunit{1}  % unités en cm
                    \def\yunit{1}
                    \psset{xunit=\xunit,yunit=\yunit,algebraic=true}
                    \fontsize{12pt}{12pt}\selectfont
                    \begin{pspicture*}[linewidth=1pt](\xmin,\ymin)(\xmax,\ymax)
                    \psgrid[gridcolor=mcgris,subgriddiv=1](-5,-1)(5,9)
                         \psaxes[linewidth=0.75pt]{->}(0,0)(-5,-1)(5,9)
                         \rput[tr](-0.2,-0.2){$O$}
               \psplot[plotpoints=1000,linewidth=0.8pt,linecolor=blue]{\xmin}{\xmax}{EXP(x)}
               \rput[r](-0.2,2.718){$\color{rouge} \text{e}$}
\psline[linewidth=0.8pt,linecolor=rouge](1,0)(1,2.71828)(0,2.71828)
                    \end{pspicture*}
           }
          \end{extern}
\end{center}
\begin{center}
\textit{Fonction exponentielle de base} $\text{e}$
\end{center}
          \bloc{cyan}{Remarque}{% id="r60"
     La stricte croissance de la fonction exponentielle entraîne que :
     \par
     $x < y \Leftrightarrow e^{x} < e^{y}$
     \par
     Cette propriété est fréquemment utilisée dans les exercices (inéquations notamment).
}
\cadre{rouge}{Théorème (dérivée de la fonction exponentielle}{% id="t70"
     La fonction exponentielle est égale à sa dérivée.
     \par
     Autrement dit, pour tout $x \in  \mathbb{R}$ : $\text{exp}^{\prime}\left(x\right)=\text{exp}\left(x\right)$
}
\bloc{cyan}{Démonstration}{% id="d70"
     Le taux d'accroissement de la fonction exponentielle sur l'intervalle $\left[x ; x+h\right]$ est égal à :
     \par
     $T=\frac{e^{x+h}-e^{x}}{h}=\frac{e^{x}\times e^{h}-e^{x}}{h}=e^{x}\times \frac{e^{h}-1}{h}$
     \par
     Par définition du nombre dérivé, le quotient $\frac{e^{h}-1}{h}$ tend vers $\text{exp}^{\prime}\left(0\right)=1$ quand $h$ tend vers $0$, donc $T$ tend vers $e^{x}$ quand $h$ tend vers 0.
}
\cadre{vert}{Propriété}{% id="p80"
     Soit $u$ une fonction dérivable sur un intervalle $I$.
     \par
     Alors la fonction $ f :  x\mapsto e^{u\left(x\right)}$ est dérivable sur $I$ et :
     \begin{center}$f^{\prime}\left(x\right)=u^{\prime}\left(x\right) e^{u\left(x\right)}$\end{center}
}
\bloc{orange}{Exemple}{% id="e80"
     Soit $f$ définie sur $\mathbb{R}$ par $f\left(x\right)=e^{-x}$
     \par
     $f$ est dérivable sur $\mathbb{R}$ et $f^{\prime}\left(x\right)=-e^{-x}$ (on pose $u\left(x\right)=-x$ donc $u^{\prime}\left(x\right)=-1$)
}

\end{document}