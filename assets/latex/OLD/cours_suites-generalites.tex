\documentclass[a4paper]{article}

%================================================================================================================================
%
% Packages
%
%================================================================================================================================

\usepackage[T1]{fontenc} 	% pour caractères accentués
\usepackage[utf8]{inputenc}  % encodage utf8
\usepackage[french]{babel}	% langue : français
\usepackage{fourier}			% caractères plus lisibles
\usepackage[dvipsnames]{xcolor} % couleurs
\usepackage{fancyhdr}		% réglage header footer
\usepackage{needspace}		% empêcher sauts de page mal placés
\usepackage{graphicx}		% pour inclure des graphiques
\usepackage{enumitem,cprotect}		% personnalise les listes d'items (nécessaire pour ol, al ...)
\usepackage{hyperref}		% Liens hypertexte
\usepackage{pstricks,pst-all,pst-node,pstricks-add,pst-math,pst-plot,pst-tree,pst-eucl} % pstricks
\usepackage[a4paper,includeheadfoot,top=2cm,left=3cm, bottom=2cm,right=3cm]{geometry} % marges etc.
\usepackage{comment}			% commentaires multilignes
\usepackage{amsmath,environ} % maths (matrices, etc.)
\usepackage{amssymb,makeidx}
\usepackage{bm}				% bold maths
\usepackage{tabularx}		% tableaux
\usepackage{colortbl}		% tableaux en couleur
\usepackage{fontawesome}		% Fontawesome
\usepackage{environ}			% environment with command
\usepackage{fp}				% calculs pour ps-tricks
\usepackage{multido}			% pour ps tricks
\usepackage[np]{numprint}	% formattage nombre
\usepackage{tikz,tkz-tab} 			% package principal TikZ
\usepackage{pgfplots}   % axes
\usepackage{mathrsfs}    % cursives
\usepackage{calc}			% calcul taille boites
\usepackage[scaled=0.875]{helvet} % font sans serif
\usepackage{svg} % svg
\usepackage{scrextend} % local margin
\usepackage{scratch} %scratch
\usepackage{multicol} % colonnes
%\usepackage{infix-RPN,pst-func} % formule en notation polanaise inversée
\usepackage{listings}

%================================================================================================================================
%
% Réglages de base
%
%================================================================================================================================

\lstset{
language=Python,   % R code
literate=
{á}{{\'a}}1
{à}{{\`a}}1
{ã}{{\~a}}1
{é}{{\'e}}1
{è}{{\`e}}1
{ê}{{\^e}}1
{í}{{\'i}}1
{ó}{{\'o}}1
{õ}{{\~o}}1
{ú}{{\'u}}1
{ü}{{\"u}}1
{ç}{{\c{c}}}1
{~}{{ }}1
}


\definecolor{codegreen}{rgb}{0,0.6,0}
\definecolor{codegray}{rgb}{0.5,0.5,0.5}
\definecolor{codepurple}{rgb}{0.58,0,0.82}
\definecolor{backcolour}{rgb}{0.95,0.95,0.92}

\lstdefinestyle{mystyle}{
    backgroundcolor=\color{backcolour},   
    commentstyle=\color{codegreen},
    keywordstyle=\color{magenta},
    numberstyle=\tiny\color{codegray},
    stringstyle=\color{codepurple},
    basicstyle=\ttfamily\footnotesize,
    breakatwhitespace=false,         
    breaklines=true,                 
    captionpos=b,                    
    keepspaces=true,                 
    numbers=left,                    
xleftmargin=2em,
framexleftmargin=2em,            
    showspaces=false,                
    showstringspaces=false,
    showtabs=false,                  
    tabsize=2,
    upquote=true
}

\lstset{style=mystyle}


\lstset{style=mystyle}
\newcommand{\imgdir}{C:/laragon/www/newmc/assets/imgsvg/}
\newcommand{\imgsvgdir}{C:/laragon/www/newmc/assets/imgsvg/}

\definecolor{mcgris}{RGB}{220, 220, 220}% ancien~; pour compatibilité
\definecolor{mcbleu}{RGB}{52, 152, 219}
\definecolor{mcvert}{RGB}{125, 194, 70}
\definecolor{mcmauve}{RGB}{154, 0, 215}
\definecolor{mcorange}{RGB}{255, 96, 0}
\definecolor{mcturquoise}{RGB}{0, 153, 153}
\definecolor{mcrouge}{RGB}{255, 0, 0}
\definecolor{mclightvert}{RGB}{205, 234, 190}

\definecolor{gris}{RGB}{220, 220, 220}
\definecolor{bleu}{RGB}{52, 152, 219}
\definecolor{vert}{RGB}{125, 194, 70}
\definecolor{mauve}{RGB}{154, 0, 215}
\definecolor{orange}{RGB}{255, 96, 0}
\definecolor{turquoise}{RGB}{0, 153, 153}
\definecolor{rouge}{RGB}{255, 0, 0}
\definecolor{lightvert}{RGB}{205, 234, 190}
\setitemize[0]{label=\color{lightvert}  $\bullet$}

\pagestyle{fancy}
\renewcommand{\headrulewidth}{0.2pt}
\fancyhead[L]{maths-cours.fr}
\fancyhead[R]{\thepage}
\renewcommand{\footrulewidth}{0.2pt}
\fancyfoot[C]{}

\newcolumntype{C}{>{\centering\arraybackslash}X}
\newcolumntype{s}{>{\hsize=.35\hsize\arraybackslash}X}

\setlength{\parindent}{0pt}		 
\setlength{\parskip}{3mm}
\setlength{\headheight}{1cm}

\def\ebook{ebook}
\def\book{book}
\def\web{web}
\def\type{web}

\newcommand{\vect}[1]{\overrightarrow{\,\mathstrut#1\,}}

\def\Oij{$\left(\text{O}~;~\vect{\imath},~\vect{\jmath}\right)$}
\def\Oijk{$\left(\text{O}~;~\vect{\imath},~\vect{\jmath},~\vect{k}\right)$}
\def\Ouv{$\left(\text{O}~;~\vect{u},~\vect{v}\right)$}

\hypersetup{breaklinks=true, colorlinks = true, linkcolor = OliveGreen, urlcolor = OliveGreen, citecolor = OliveGreen, pdfauthor={Didier BONNEL - https://www.maths-cours.fr} } % supprime les bordures autour des liens

\renewcommand{\arg}[0]{\text{arg}}

\everymath{\displaystyle}

%================================================================================================================================
%
% Macros - Commandes
%
%================================================================================================================================

\newcommand\meta[2]{    			% Utilisé pour créer le post HTML.
	\def\titre{titre}
	\def\url{url}
	\def\arg{#1}
	\ifx\titre\arg
		\newcommand\maintitle{#2}
		\fancyhead[L]{#2}
		{\Large\sffamily \MakeUppercase{#2}}
		\vspace{1mm}\textcolor{mcvert}{\hrule}
	\fi 
	\ifx\url\arg
		\fancyfoot[L]{\href{https://www.maths-cours.fr#2}{\black \footnotesize{https://www.maths-cours.fr#2}}}
	\fi 
}


\newcommand\TitreC[1]{    		% Titre centré
     \needspace{3\baselineskip}
     \begin{center}\textbf{#1}\end{center}
}

\newcommand\newpar{    		% paragraphe
     \par
}

\newcommand\nosp {    		% commande vide (pas d'espace)
}
\newcommand{\id}[1]{} %ignore

\newcommand\boite[2]{				% Boite simple sans titre
	\vspace{5mm}
	\setlength{\fboxrule}{0.2mm}
	\setlength{\fboxsep}{5mm}	
	\fcolorbox{#1}{#1!3}{\makebox[\linewidth-2\fboxrule-2\fboxsep]{
  		\begin{minipage}[t]{\linewidth-2\fboxrule-4\fboxsep}\setlength{\parskip}{3mm}
  			 #2
  		\end{minipage}
	}}
	\vspace{5mm}
}

\newcommand\CBox[4]{				% Boites
	\vspace{5mm}
	\setlength{\fboxrule}{0.2mm}
	\setlength{\fboxsep}{5mm}
	
	\fcolorbox{#1}{#1!3}{\makebox[\linewidth-2\fboxrule-2\fboxsep]{
		\begin{minipage}[t]{1cm}\setlength{\parskip}{3mm}
	  		\textcolor{#1}{\LARGE{#2}}    
 	 	\end{minipage}  
  		\begin{minipage}[t]{\linewidth-2\fboxrule-4\fboxsep}\setlength{\parskip}{3mm}
			\raisebox{1.2mm}{\normalsize\sffamily{\textcolor{#1}{#3}}}						
  			 #4
  		\end{minipage}
	}}
	\vspace{5mm}
}

\newcommand\cadre[3]{				% Boites convertible html
	\par
	\vspace{2mm}
	\setlength{\fboxrule}{0.1mm}
	\setlength{\fboxsep}{5mm}
	\fcolorbox{#1}{white}{\makebox[\linewidth-2\fboxrule-2\fboxsep]{
  		\begin{minipage}[t]{\linewidth-2\fboxrule-4\fboxsep}\setlength{\parskip}{3mm}
			\raisebox{-2.5mm}{\sffamily \small{\textcolor{#1}{\MakeUppercase{#2}}}}		
			\par		
  			 #3
 	 		\end{minipage}
	}}
		\vspace{2mm}
	\par
}

\newcommand\bloc[3]{				% Boites convertible html sans bordure
     \needspace{2\baselineskip}
     {\sffamily \small{\textcolor{#1}{\MakeUppercase{#2}}}}    
		\par		
  			 #3
		\par
}

\newcommand\CHelp[1]{
     \CBox{Plum}{\faInfoCircle}{À RETENIR}{#1}
}

\newcommand\CUp[1]{
     \CBox{NavyBlue}{\faThumbsOUp}{EN PRATIQUE}{#1}
}

\newcommand\CInfo[1]{
     \CBox{Sepia}{\faArrowCircleRight}{REMARQUE}{#1}
}

\newcommand\CRedac[1]{
     \CBox{PineGreen}{\faEdit}{BIEN R\'EDIGER}{#1}
}

\newcommand\CError[1]{
     \CBox{Red}{\faExclamationTriangle}{ATTENTION}{#1}
}

\newcommand\TitreExo[2]{
\needspace{4\baselineskip}
 {\sffamily\large EXERCICE #1\ (\emph{#2 points})}
\vspace{5mm}
}

\newcommand\img[2]{
          \includegraphics[width=#2\paperwidth]{\imgdir#1}
}

\newcommand\imgsvg[2]{
       \begin{center}   \includegraphics[width=#2\paperwidth]{\imgsvgdir#1} \end{center}
}


\newcommand\Lien[2]{
     \href{#1}{#2 \tiny \faExternalLink}
}
\newcommand\mcLien[2]{
     \href{https~://www.maths-cours.fr/#1}{#2 \tiny \faExternalLink}
}

\newcommand{\euro}{\eurologo{}}

%================================================================================================================================
%
% Macros - Environement
%
%================================================================================================================================

\newenvironment{tex}{ %
}
{%
}

\newenvironment{indente}{ %
	\setlength\parindent{10mm}
}

{
	\setlength\parindent{0mm}
}

\newenvironment{corrige}{%
     \needspace{3\baselineskip}
     \medskip
     \textbf{\textsc{Corrigé}}
     \medskip
}
{
}

\newenvironment{extern}{%
     \begin{center}
     }
     {
     \end{center}
}

\NewEnviron{code}{%
	\par
     \boite{gray}{\texttt{%
     \BODY
     }}
     \par
}

\newenvironment{vbloc}{% boite sans cadre empeche saut de page
     \begin{minipage}[t]{\linewidth}
     }
     {
     \end{minipage}
}
\NewEnviron{h2}{%
    \needspace{3\baselineskip}
    \vspace{0.6cm}
	\noindent \MakeUppercase{\sffamily \large \BODY}
	\vspace{1mm}\textcolor{mcgris}{\hrule}\vspace{0.4cm}
	\par
}{}

\NewEnviron{h3}{%
    \needspace{3\baselineskip}
	\vspace{5mm}
	\textsc{\BODY}
	\par
}

\NewEnviron{margeneg}{ %
\begin{addmargin}[-1cm]{0cm}
\BODY
\end{addmargin}
}

\NewEnviron{html}{%
}

\begin{document}
\begin{h2}I - Définition d'une suite\end{h2}
\cadre{bleu}{Définitions}{% id="d10"
     Une \textbf{suite} $u$ associe à tout entier naturel $n$ un nombre réel noté $u_{n}$.
     \par
     Les nombres réels $u_{n}$ sont les \textbf{termes} de la suite.
     \par
     Les nombres entiers $n$ sont les \textbf{indices} ou les \textbf{rangs}.
     \par
     La suite $u$ peut également se noter $\left(u_{n}\right)$ ou $\left(u_{n}\right)_{n\in \mathbb{N}}$.
}
\bloc{cyan}{Remarque}{% id="r10"
     Intuitivement, une suite est une liste infinie et ordonnée de nombres réels. Ces nombres réels sont les termes de la suite et les indices correspondent à la position du terme dans la liste.
}
\bloc{orange}{Exemple}{% id="e10"
     Par exemple la liste $1,6$ ; $2,4$ ; $3,2$ ; $5$ ; ... correspond à la suite $\left(u_{n}\right)$ suivante :
     \par
     $u_{0}=1,6$ (terme de rang 0)
     \par
     $u_{1}=2,4$ (terme de rang 1)
     \par
     $u_{2}=3,2$ (terme de rang 2)
     \par
     $u_{3}=5$ ...
}
\bloc{cyan}{Remarque}{% id="r11"
     Ne pas confondre l'écriture $\left(u_{n}\right)$ avec parenthèses qui désigne la suite et l'écriture $u_{n}$ sans parenthèse qui désigne le $n$-ième terme de la suite.
}
\cadre{bleu}{Définition}{% id="d20"
     Une suite est définie de façon \textbf{explicite} lorsqu'on dispose d'une formule du type $u_{n}=f\left(n\right)$ permettant de calculer chaque terme de la suite à partir de son rang.
}
\bloc{orange}{Exemple}{% id="e20"
     La suite $\left(u_{n}\right)$ définie par la formule explicite $u_{n}=\frac{2n+1}{3}$ est telle que
     \par
     $u_{0}=\frac{1}{3}$
     \par
     $u_{1}=\frac{3}{3}=1$ ...
     \par
     $u_{100}=\frac{201}{3}=67$
}
\cadre{bleu}{Définition}{% id="d30"
     Une suite est définie par une relation de \textbf{récurrence} lorsqu'on dispose du premier terme et d'une formule du type $u_{n+1}=f\left(u_{n}\right)$ permettant de calculer chaque terme de la suite à partir du terme précédent.
}
\bloc{cyan}{Remarque}{% id="r30"
     Il est possible de calculer un terme quelconque d'une suite définie par une relation de récurrence mais il faut au préalable calculer tout les termes précédents. Comme cela peut se révéler long,  on utilise parfois un algorithme pour faire ce calcul.
}
\bloc{orange}{Exemple}{% id="e30"
     La suite $\left(u_{n}\right)$ définie par la formule de récurrence
     \par
     $\left\{ \begin{matrix} u_{0}=1  \\  u_{n+1}=2u_{n}-3 \end{matrix}\right.$
          \par
          est telle que :
          \par
          $u_{0}=1$
          \par
          $u_{1}=2\times u_{0}-3=2\times 1-3=-1$
          \par
          $u_{2}=2\times u_{1}-3=2\times \left(-1\right)-3=-5$
          \par
          etc...
     }
     \begin{h2}II - Représentation graphique d'une suite\end{h2}
     \cadre{bleu}{Définition}{% id="d40"
          La représentation graphique d'une suite $\left(u_{n}\right)_{n \in  \mathbb{N}}$ dans un repère du plan, s'obtient en plaçant les points de coordonnées $\left(n ; u_{n}\right)$ lorsque $n$ parcourt $\mathbb{N}$.
     }
     \bloc{orange}{Exemple}{% id="e40"
          Pour représenter la suite définie par $u_{n}=1+\frac{3}{n+1}$ on calcule:
          \par
          $u_{0}=4$
          \par
          $u_{1}=\frac{5}{2}$
          \par
          $u_{2}=2$
          \par
          $u_{3}=\frac{7}{4}$
          \par
          etc.
          \par
          et on place les points de coordonnées : $\left(0 ; 4\right) ; \left(1 ; \frac{5}{2}\right) ; \left(2 ; 2\right) ; \left(3 ; \frac{7}{4}\right)$; etc.
     }
 \begin{center}
     \begin{extern}%width="300" alt="représentation graphique d'une suite"
          % -+-+-+ variables modifiables
          \resizebox{7cm}{!}{%
               \def\xmin{-0.8}
               \def\xmax{7.5}
               \def\ymin{-0.8}
               \def\ymax{4.8}
               \def\xunit{1}  % unités en cm
               \def\yunit{1}
               \psset{xunit=\xunit,yunit=\yunit,algebraic=true}
               \fontsize{15pt}{15pt}\selectfont
               \begin{pspicture*}[linewidth=1pt](\xmin,\ymin)(\xmax,\ymax)
                    \psaxes[Dx=1,Dy=1,linewidth=0.75pt]{->}(0,0)(\xmin,\ymin)(\xmax,\ymax)
                    \rput[tr](-0.2,-0.3){$O$}
                    \multido{\n=0.0+1}{8}{
                         \FPeval{\suite}{1+3/(\n+1)}
                         \psdots[linecolor=blue](\n,\suite)
                    }
               \end{pspicture*}
          }
     \end{extern}
\end{center}
\begin{center}
     \textit{Représentation graphique de la suite définie par $u_{n}=1+\frac{3}{n+1}$}
\end{center}
    \begin{h2}III - Sens de variation d'une suite\end{h2}
     \cadre{bleu}{Définitions}{% id="d50"
          On dit qu'une suite $\left(u_{n}\right)$ est \textbf{croissante} (\textit{resp.\textbf{décroissante})} si pour tout entier naturel $n$ :
          \begin{center}$u_{n+1} \geqslant  u_{n}   $  (\textit{resp. $u_{n+1} \leqslant  u_{n} $)}\end{center}
          On dit qu'une suite $\left(u_{n}\right)$ est \textbf{strictement croissante} (\textit{resp.\textbf{strictement décroissante})} si pour tout entier naturel $n$ :
          \begin{center}$u_{n+1} > u_{n}   $  (\textit{resp. $u_{n+1} < u_{n} $)}\end{center}
          On dit qu'une suite $\left(u_{n}\right)$ est \textbf{constante} si pour tout entier naturel  $n$ :
          \begin{center}$u_{n+1} = u_{n}  $\end{center}
     }
     \bloc{cyan}{Remarques}{% id="r50"
          \begin{itemize}
               \item Une suite peut n'être ni croissante,, ni décroissante, ni constante. C'est le cas, par exemple de la suite définie par $u_{n}=\left(-1\right)^{n}$ dont les termes valent successivement : $1; -1; 1; -1; 1; -1;$ etc.
               \item En pratique pour savoir si une suite $\left(u_{n}\right)$ est croissante ou décroissante, on calcule souvent $u_{n+1}-u_{n}$ :
               \begin{itemize}
               \item si $u_{n+1}-u_{n} \geqslant  0$ pour tout $n \in  \mathbb{N}$, la suite $u_{n}$ est croissante
              \item si $u_{n+1}-u_{n} \leqslant  0$ pour tout $n \in  \mathbb{N}$, la suite $u_{n}$ est décroissante
              \item  si $u_{n+1}-u_{n} = 0$ pour tout $n \in  \mathbb{N}$, la suite $u_{n}$ est constante.               
               \end{itemize}

          \end{itemize}
     }
     
\end{document}