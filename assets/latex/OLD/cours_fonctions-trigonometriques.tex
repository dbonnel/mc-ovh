\documentclass[a4paper]{article}

%================================================================================================================================
%
% Packages
%
%================================================================================================================================

\usepackage[T1]{fontenc} 	% pour caractères accentués
\usepackage[utf8]{inputenc}  % encodage utf8
\usepackage[french]{babel}	% langue : français
\usepackage{fourier}			% caractères plus lisibles
\usepackage[dvipsnames]{xcolor} % couleurs
\usepackage{fancyhdr}		% réglage header footer
\usepackage{needspace}		% empêcher sauts de page mal placés
\usepackage{graphicx}		% pour inclure des graphiques
\usepackage{enumitem,cprotect}		% personnalise les listes d'items (nécessaire pour ol, al ...)
\usepackage{hyperref}		% Liens hypertexte
\usepackage{pstricks,pst-all,pst-node,pstricks-add,pst-math,pst-plot,pst-tree,pst-eucl} % pstricks
\usepackage[a4paper,includeheadfoot,top=2cm,left=3cm, bottom=2cm,right=3cm]{geometry} % marges etc.
\usepackage{comment}			% commentaires multilignes
\usepackage{amsmath,environ} % maths (matrices, etc.)
\usepackage{amssymb,makeidx}
\usepackage{bm}				% bold maths
\usepackage{tabularx}		% tableaux
\usepackage{colortbl}		% tableaux en couleur
\usepackage{fontawesome}		% Fontawesome
\usepackage{environ}			% environment with command
\usepackage{fp}				% calculs pour ps-tricks
\usepackage{multido}			% pour ps tricks
\usepackage[np]{numprint}	% formattage nombre
\usepackage{tikz,tkz-tab} 			% package principal TikZ
\usepackage{pgfplots}   % axes
\usepackage{mathrsfs}    % cursives
\usepackage{calc}			% calcul taille boites
\usepackage[scaled=0.875]{helvet} % font sans serif
\usepackage{svg} % svg
\usepackage{scrextend} % local margin
\usepackage{scratch} %scratch
\usepackage{multicol} % colonnes
%\usepackage{infix-RPN,pst-func} % formule en notation polanaise inversée
\usepackage{listings}

%================================================================================================================================
%
% Réglages de base
%
%================================================================================================================================

\lstset{
language=Python,   % R code
literate=
{á}{{\'a}}1
{à}{{\`a}}1
{ã}{{\~a}}1
{é}{{\'e}}1
{è}{{\`e}}1
{ê}{{\^e}}1
{í}{{\'i}}1
{ó}{{\'o}}1
{õ}{{\~o}}1
{ú}{{\'u}}1
{ü}{{\"u}}1
{ç}{{\c{c}}}1
{~}{{ }}1
}


\definecolor{codegreen}{rgb}{0,0.6,0}
\definecolor{codegray}{rgb}{0.5,0.5,0.5}
\definecolor{codepurple}{rgb}{0.58,0,0.82}
\definecolor{backcolour}{rgb}{0.95,0.95,0.92}

\lstdefinestyle{mystyle}{
    backgroundcolor=\color{backcolour},   
    commentstyle=\color{codegreen},
    keywordstyle=\color{magenta},
    numberstyle=\tiny\color{codegray},
    stringstyle=\color{codepurple},
    basicstyle=\ttfamily\footnotesize,
    breakatwhitespace=false,         
    breaklines=true,                 
    captionpos=b,                    
    keepspaces=true,                 
    numbers=left,                    
xleftmargin=2em,
framexleftmargin=2em,            
    showspaces=false,                
    showstringspaces=false,
    showtabs=false,                  
    tabsize=2,
    upquote=true
}

\lstset{style=mystyle}


\lstset{style=mystyle}
\newcommand{\imgdir}{C:/laragon/www/newmc/assets/imgsvg/}
\newcommand{\imgsvgdir}{C:/laragon/www/newmc/assets/imgsvg/}

\definecolor{mcgris}{RGB}{220, 220, 220}% ancien~; pour compatibilité
\definecolor{mcbleu}{RGB}{52, 152, 219}
\definecolor{mcvert}{RGB}{125, 194, 70}
\definecolor{mcmauve}{RGB}{154, 0, 215}
\definecolor{mcorange}{RGB}{255, 96, 0}
\definecolor{mcturquoise}{RGB}{0, 153, 153}
\definecolor{mcrouge}{RGB}{255, 0, 0}
\definecolor{mclightvert}{RGB}{205, 234, 190}

\definecolor{gris}{RGB}{220, 220, 220}
\definecolor{bleu}{RGB}{52, 152, 219}
\definecolor{vert}{RGB}{125, 194, 70}
\definecolor{mauve}{RGB}{154, 0, 215}
\definecolor{orange}{RGB}{255, 96, 0}
\definecolor{turquoise}{RGB}{0, 153, 153}
\definecolor{rouge}{RGB}{255, 0, 0}
\definecolor{lightvert}{RGB}{205, 234, 190}
\setitemize[0]{label=\color{lightvert}  $\bullet$}

\pagestyle{fancy}
\renewcommand{\headrulewidth}{0.2pt}
\fancyhead[L]{maths-cours.fr}
\fancyhead[R]{\thepage}
\renewcommand{\footrulewidth}{0.2pt}
\fancyfoot[C]{}

\newcolumntype{C}{>{\centering\arraybackslash}X}
\newcolumntype{s}{>{\hsize=.35\hsize\arraybackslash}X}

\setlength{\parindent}{0pt}		 
\setlength{\parskip}{3mm}
\setlength{\headheight}{1cm}

\def\ebook{ebook}
\def\book{book}
\def\web{web}
\def\type{web}

\newcommand{\vect}[1]{\overrightarrow{\,\mathstrut#1\,}}

\def\Oij{$\left(\text{O}~;~\vect{\imath},~\vect{\jmath}\right)$}
\def\Oijk{$\left(\text{O}~;~\vect{\imath},~\vect{\jmath},~\vect{k}\right)$}
\def\Ouv{$\left(\text{O}~;~\vect{u},~\vect{v}\right)$}

\hypersetup{breaklinks=true, colorlinks = true, linkcolor = OliveGreen, urlcolor = OliveGreen, citecolor = OliveGreen, pdfauthor={Didier BONNEL - https://www.maths-cours.fr} } % supprime les bordures autour des liens

\renewcommand{\arg}[0]{\text{arg}}

\everymath{\displaystyle}

%================================================================================================================================
%
% Macros - Commandes
%
%================================================================================================================================

\newcommand\meta[2]{    			% Utilisé pour créer le post HTML.
	\def\titre{titre}
	\def\url{url}
	\def\arg{#1}
	\ifx\titre\arg
		\newcommand\maintitle{#2}
		\fancyhead[L]{#2}
		{\Large\sffamily \MakeUppercase{#2}}
		\vspace{1mm}\textcolor{mcvert}{\hrule}
	\fi 
	\ifx\url\arg
		\fancyfoot[L]{\href{https://www.maths-cours.fr#2}{\black \footnotesize{https://www.maths-cours.fr#2}}}
	\fi 
}


\newcommand\TitreC[1]{    		% Titre centré
     \needspace{3\baselineskip}
     \begin{center}\textbf{#1}\end{center}
}

\newcommand\newpar{    		% paragraphe
     \par
}

\newcommand\nosp {    		% commande vide (pas d'espace)
}
\newcommand{\id}[1]{} %ignore

\newcommand\boite[2]{				% Boite simple sans titre
	\vspace{5mm}
	\setlength{\fboxrule}{0.2mm}
	\setlength{\fboxsep}{5mm}	
	\fcolorbox{#1}{#1!3}{\makebox[\linewidth-2\fboxrule-2\fboxsep]{
  		\begin{minipage}[t]{\linewidth-2\fboxrule-4\fboxsep}\setlength{\parskip}{3mm}
  			 #2
  		\end{minipage}
	}}
	\vspace{5mm}
}

\newcommand\CBox[4]{				% Boites
	\vspace{5mm}
	\setlength{\fboxrule}{0.2mm}
	\setlength{\fboxsep}{5mm}
	
	\fcolorbox{#1}{#1!3}{\makebox[\linewidth-2\fboxrule-2\fboxsep]{
		\begin{minipage}[t]{1cm}\setlength{\parskip}{3mm}
	  		\textcolor{#1}{\LARGE{#2}}    
 	 	\end{minipage}  
  		\begin{minipage}[t]{\linewidth-2\fboxrule-4\fboxsep}\setlength{\parskip}{3mm}
			\raisebox{1.2mm}{\normalsize\sffamily{\textcolor{#1}{#3}}}						
  			 #4
  		\end{minipage}
	}}
	\vspace{5mm}
}

\newcommand\cadre[3]{				% Boites convertible html
	\par
	\vspace{2mm}
	\setlength{\fboxrule}{0.1mm}
	\setlength{\fboxsep}{5mm}
	\fcolorbox{#1}{white}{\makebox[\linewidth-2\fboxrule-2\fboxsep]{
  		\begin{minipage}[t]{\linewidth-2\fboxrule-4\fboxsep}\setlength{\parskip}{3mm}
			\raisebox{-2.5mm}{\sffamily \small{\textcolor{#1}{\MakeUppercase{#2}}}}		
			\par		
  			 #3
 	 		\end{minipage}
	}}
		\vspace{2mm}
	\par
}

\newcommand\bloc[3]{				% Boites convertible html sans bordure
     \needspace{2\baselineskip}
     {\sffamily \small{\textcolor{#1}{\MakeUppercase{#2}}}}    
		\par		
  			 #3
		\par
}

\newcommand\CHelp[1]{
     \CBox{Plum}{\faInfoCircle}{À RETENIR}{#1}
}

\newcommand\CUp[1]{
     \CBox{NavyBlue}{\faThumbsOUp}{EN PRATIQUE}{#1}
}

\newcommand\CInfo[1]{
     \CBox{Sepia}{\faArrowCircleRight}{REMARQUE}{#1}
}

\newcommand\CRedac[1]{
     \CBox{PineGreen}{\faEdit}{BIEN R\'EDIGER}{#1}
}

\newcommand\CError[1]{
     \CBox{Red}{\faExclamationTriangle}{ATTENTION}{#1}
}

\newcommand\TitreExo[2]{
\needspace{4\baselineskip}
 {\sffamily\large EXERCICE #1\ (\emph{#2 points})}
\vspace{5mm}
}

\newcommand\img[2]{
          \includegraphics[width=#2\paperwidth]{\imgdir#1}
}

\newcommand\imgsvg[2]{
       \begin{center}   \includegraphics[width=#2\paperwidth]{\imgsvgdir#1} \end{center}
}


\newcommand\Lien[2]{
     \href{#1}{#2 \tiny \faExternalLink}
}
\newcommand\mcLien[2]{
     \href{https~://www.maths-cours.fr/#1}{#2 \tiny \faExternalLink}
}

\newcommand{\euro}{\eurologo{}}

%================================================================================================================================
%
% Macros - Environement
%
%================================================================================================================================

\newenvironment{tex}{ %
}
{%
}

\newenvironment{indente}{ %
	\setlength\parindent{10mm}
}

{
	\setlength\parindent{0mm}
}

\newenvironment{corrige}{%
     \needspace{3\baselineskip}
     \medskip
     \textbf{\textsc{Corrigé}}
     \medskip
}
{
}

\newenvironment{extern}{%
     \begin{center}
     }
     {
     \end{center}
}

\NewEnviron{code}{%
	\par
     \boite{gray}{\texttt{%
     \BODY
     }}
     \par
}

\newenvironment{vbloc}{% boite sans cadre empeche saut de page
     \begin{minipage}[t]{\linewidth}
     }
     {
     \end{minipage}
}
\NewEnviron{h2}{%
    \needspace{3\baselineskip}
    \vspace{0.6cm}
	\noindent \MakeUppercase{\sffamily \large \BODY}
	\vspace{1mm}\textcolor{mcgris}{\hrule}\vspace{0.4cm}
	\par
}{}

\NewEnviron{h3}{%
    \needspace{3\baselineskip}
	\vspace{5mm}
	\textsc{\BODY}
	\par
}

\NewEnviron{margeneg}{ %
\begin{addmargin}[-1cm]{0cm}
\BODY
\end{addmargin}
}

\NewEnviron{html}{%
}

\begin{document}
\begin{h2}1. Rappels\end{h2}
Dans toute la suite, le plan est muni d'un repère orthonormé $\left(O ; \overrightarrow{OI} ,\overrightarrow{OJ}\right)$.\\
On oriente le \textbf{cercle trigonométrique} (cercle de centre $O$ et de rayon 1) dans le \textbf{sens direct} (sens inverse des aiguilles d'une montre).
\begin{center}
     \begin{extern}%width="480"
          \newrgbcolor{dblue}{0. 0. 0.7}
          \newrgbcolor{dvert}{0. 0.4 0.}
          \newrgbcolor{dmauve}{0.5 0. 0.5}
          \psset{xunit=5.0cm,yunit=5.0cm,algebraic=true,dimen=middle,dotstyle=o,dotsize=5pt 0,linewidth=0.8pt,arrowsize=3pt 2,arrowinset=0.25}
          \begin{pspicture*}(-1.2,-1.2)(1.2,1.2)
               \psaxes[linewidth=0.75pt,labelFontSize=\scriptstyle,xAxis=true,yAxis=true,Dx=10.,Dy=10.,ticksize=-2pt 0,subticks=1]{->}(0,0)(-1.2,-1.2)(1.2,1.2)
               \pscircle[linewidth=0.8pt](0.,0.){5.} %cercle trigo
               \parametricplot[linewidth=1.2pt,linecolor=red]{0.0}{0.698}{cos(t)|sin(t)}%arc angle
               \parametricplot[linewidth=0.8pt,arrows=->]{0.8}{1.3}{1.15*cos(t)|1.15*sin(t)}% sens trigo
               \rput[tl](0.58,1.07){+}
               \pscustom[linewidth=0.8pt,linecolor=dmauve,fillcolor=dmauve,fillstyle=solid,opacity=0.1]{ % color angle
                    \parametricplot{0.0}{0.698}{0.15*cos(t)|0.15*sin(t)}
               \lineto(0.,0.)\closepath}
               \psellipticarc[linewidth=0.8pt,linecolor=dmauve,arrows=->](0.,0.)(0.15,0.15){0.}{40} % fleche angle
               \psline[linewidth=0.8pt,linecolor=dmauve](0.,0.)(0.766,0.643)%rayon
               \psline[linewidth=0.8pt]{->}(0.,0.)(1.,0.) %vecteurs unités
               \psline[linewidth=0.8pt]{->}(0.,0.)(0,1)
               %\rput[tl](0.4,0.1){$\vec{i}$}
               %\rput[tl](-0.06,0.5){$\vec{j}$}
               \psdots[dotsize=2pt 0,dotstyle=*](0.,0.)
               \rput[bl](-0.09,-0.09){$O$}
               \psdots[dotsize=2pt 0,dotstyle=*,linecolor=dblue](1.,0.)
               \rput[bl](1.02,0.02){\dblue{$I$}}
               \psdots[dotsize=2pt 0,dotstyle=*,linecolor=dblue](0.766,0.643)
               \rput[bl](0.78,0.66){\dblue{$N$}}
               \psdots[dotsize=2pt 0,dotstyle=*,linecolor=dblue](0,1)
               \rput[bl](0.02,1.03){\dblue{$J$}}
               \rput[bl](0.19,0.05){\dmauve{$x$}}
               \psdots[dotsize=2pt 0,dotstyle=*,linecolor=dvert](0.766,0)
               \psdots[dotsize=2pt 0,dotstyle=*,linecolor=dvert](0,0.643)
               \psline[linewidth=1pt,linecolor=dvert](0.,0.)(0.766,0)
               \psline[linewidth=1pt,linecolor=dvert](0.,0.)(0,0.643)
               \psline[linewidth=0.4pt,linecolor=dvert](0.,0.643)(0.766,0.643)
               \psline[linewidth=0.4pt,linecolor=dvert](0.766,0.)(0.766,0.643)
               \rput(0.766,-0.05){\dvert{$\cos x$}}
               \rput(-0.10,0.643){\dvert{$\sin x$}}
          \end{pspicture*}
     \end{extern}
     \end{center}\cadre{bleu}{Définition}{%id="d10"
     Soit $N$ un point du cercle trigonométrique et $x$ une mesure en radians de l'angle $\left(\overrightarrow{OI},\overrightarrow{ON}\right)$.
     \par
     On appelle \textbf{cosinus} de $x$, noté\textbf{ $\cos x$} l'abscisse du point $N$.
     \par
     On appelle \textbf{sinus} de $x$, noté\textbf{ $\sin x$} l'ordonnée du point $N$.
}
\bloc{cyan}{Remarque}{%id="r10"
     Pour tout réel $x$ :
     \begin{itemize}
          \item $-1 \leqslant \cos x \leqslant 1$
          \item $-1 \leqslant \sin x \leqslant 1$
          \item $\left(\cos x\right)^{2} + \left(\sin x\right)^{2} = 1$ (d'après le théorème de Pythagore).
     \end{itemize}
}
\cadre{vert}{Quelques valeurs de sinus et de cosinus}{%id="p20"
     \begin{tabularx}{0.8\linewidth}{|*{7}{>{\centering \arraybackslash }X|}}%class="compact mw500"
          \hline
          \textbf{$x$} & $0$ & $\frac{\pi }{6}$ & $\frac{\pi }{4}$ & $\frac{\pi }{3}$ & $\frac{\pi }{2}$ & $\pi $\\ \hline
          \textbf{$\cos x$} & $1$ & $\frac{\sqrt{3}}{2}$ & $\frac{\sqrt{2}}{2}$ & $\frac{1}{2}$ & $0$ & $-1$\\ \hline
          \textbf{$\sin x$} & $0$ & $\frac{1}{2}$ & $\frac{\sqrt{2}}{2}$ & $\frac{\sqrt{3}}{2}$ & $1$ & $0$\\ \hline
     \end{tabularx}
}
\begin{center}
     \begin{extern}%width="600"
          \newrgbcolor{dblue}{0. 0. 0.7}
          \newrgbcolor{dvert}{0. 0.4 0.}
          \newrgbcolor{dmauve}{0.5 0. 0.5}
          \psset{xunit=5.0cm,yunit=5.0cm,algebraic=true,dimen=middle,dotstyle=o,dotsize=5pt 0,linewidth=0.8pt,arrowsize=3pt 2,arrowinset=0.25}
          \begin{pspicture*}(-1.2,-1.2)(1.2,1.2)
               \psaxes[linewidth=0.75pt,labelFontSize=\scriptstyle,xAxis=true,yAxis=true,Dx=10.,Dy=10.,ticksize=-2pt 0,subticks=1]{->}(0,0)(-1.2,-1.2)(1.2,1.2)
               \pscircle[linewidth=0.8pt](0.,0.){5.} %cercle trigo
               \psline[linewidth=0.8pt]{->}(0.,0.)(1.,0.) %vecteurs unités
               \psline[linewidth=0.8pt]{->}(0.,0.)(0,1)
               %\rput[tl](0.4,0.1){$\vec{i}$}
               %\rput[tl](-0.06,0.5){$\vec{j}$}
               \psdots[dotsize=2pt 0,dotstyle=*](0.,0.)
               %\rput[bl](-0.09,-0.09){$O$}
               \psframe[linewidth=0.4pt,linecolor=dvert](-0.707,-0.707)(0.707,0.707)
               \psline[linewidth=0.8pt,linecolor=dvert](-0.707,-0.707)(0.707,0.707)
               \psline[linewidth=0.8pt,linecolor=dvert](-0.707,0.707)(0.707,-0.707)
               \psframe[linewidth=0.4pt,linecolor=red](-0.866,-0.5)(0.866,0.5)
               \psline[linewidth=0.8pt,linecolor=red](-0.866,-0.5)(0.866,0.5)
               \psline[linewidth=0.8pt,linecolor=red](0.866,-0.5)(-0.866,0.5)
               \rput(0.943,0.55){$\red{\dfrac{\pi}{6}}$}
               \rput(-0.943,0.55){$\red{\dfrac{5\pi}{6}}$}
               \rput(-0.973,-0.55){$\red{-\dfrac{5\pi}{6}}$}
               \rput(0.943,-0.55){$\red{-\dfrac{\pi}{6}}$}
               \rput(0.05,-0.554){\fontsize{7 pt}{7 pt}\selectfont $\red{ -\dfrac{1}{2}}$}
               \rput(0.05,0.554){\fontsize{7 pt}{7 pt}\selectfont $\red{ \dfrac{1}{2}}$}
               \rput(0.93,0.07){\fontsize{7 pt}{7 pt}\selectfont $\red{ \dfrac{\sqrt{3}}{2}}$}
               \rput(-0.93,0.07){\fontsize{7 pt}{7 pt}\selectfont $\red{-\dfrac{\sqrt{3}}{2}}$}
               %
               \rput(0.777,0.777){$\dvert{\dfrac{\pi}{4}}$}
               \rput(-0.777,0.777){$\dvert{\dfrac{3\pi}{4}}$}
               \rput(-0.807,-0.777){$\dvert{-\dfrac{3\pi}{4}}$}
               \rput(0.777,-0.777){$\dvert{-\dfrac{\pi}{4}}$}
               \rput(0.06,-0.77){\fontsize{7 pt}{7 pt}\selectfont $\dvert{ -\dfrac{\sqrt{2}}{2}}$}
               \rput(0.06,0.77){\fontsize{7 pt}{7 pt}\selectfont $\dvert{ \dfrac{\sqrt{2}}{2}}$}
               \rput(0.764,0.07){\fontsize{7 pt}{7 pt}\selectfont $\dvert{ \dfrac{\sqrt{2}}{2}}$}
               \rput(-0.764,0.07){\fontsize{7 pt}{7 pt}\selectfont $\dvert{-\dfrac{\sqrt{2}}{2}}$}
               %
               \psframe[linewidth=0.4pt,linecolor=dblue](-0.5,-0.866)(0.5,0.866)
               \psline[linewidth=0.8pt,linecolor=dblue](-0.5,-0.866)(0.5,0.866)
               \psline[linewidth=0.8pt,linecolor=dblue](-0.5,0.866)(0.5,-0.866)
               \rput(0.55,0.943){$\dblue{\dfrac{\pi}{3}}$}
               \rput(-0.55,0.943){$\dblue{\dfrac{2\pi}{3}}$}
               \rput(-0.58,-0.943){$\dblue{-\dfrac{2\pi}{3}}$}
               \rput(0.55,-0.943){$\dblue{-\dfrac{\pi}{3}}$}
               \rput(0.06,-0.933){\fontsize{7 pt}{7 pt}\selectfont $\dblue{ -\dfrac{\sqrt{3}}{2}}$}
               \rput(0.06,0.933){\fontsize{7 pt}{7 pt}\selectfont $\dblue{ \dfrac{\sqrt{3}}{2}}$}
               \rput(0.538,0.07){\fontsize{7 pt}{7 pt}\selectfont $\dblue{ \dfrac{1}{2}}$}
               \rput(-0.538,0.07){\fontsize{7 pt}{7 pt}\selectfont $\dblue{-\dfrac{1}{2}}$}
               %
               \rput(1.06,0.06){$0$}
               \rput(0.06,1.1){$\dfrac{\pi}{2}$}
               \rput(0.06,-1.1){$-\dfrac{\pi}{2}$}
               \rput(-1.06,0.06){$\pi$}
          \end{pspicture*}
     \end{extern}
\end{center}
\cadre{rouge}{Théorème}{%id="t30"
     Soit $a$ un réel fixé.
     \par
     Les solutions de l'équation $\cos\left(x\right)=\cos\left(a\right)$ sont les réels de la forme :
     \begin{center}$a+2k\pi $ ou $ -a+2k\pi $ où $k$ décrit $\mathbb{Z}$\end{center}
     Les solutions de l'équation $\sin\left(x\right)=\sin\left(a\right)$ sont les réels de la forme :
     \begin{center}$a+2k\pi $ ou $ \pi -a+2k\pi $ où $k$ décrit $\mathbb{Z}$\end{center}
}
\bloc{orange}{Exemple}{%id="e30"
     Soit l'équation $\sin\left(x\right)=\frac{1}{2}$.
     \par
     Comme $\sin\frac{\pi }{6}=\frac{1}{2}$, l'équation peut s'écrire $\sin\left(x\right)=\sin\frac{\pi }{6}$.
     \par
     D'après le théorème précédent, l'ensemble des solutions est :
     \par
     $S=\left\{ \frac{\pi }{6}+2k\pi , \frac{5\pi }{6}+2k\pi | k\in \mathbb{Z} \right\}$.
}
\begin{h2}2. Fonctions sinus et cosinus\end{h2}
\cadre{bleu}{Définition}{%id="d40"
     La fonction, définie sur $\mathbb{R}$, qui à tout réel $x$ associe son cosinus : $x\mapsto \cos\left(x\right)$ est appelée \textbf{fonction cosinus}.
     \par
     La fonction, définie sur $\mathbb{R}$, qui à tout réel $x$ associe son sinus : $x\mapsto \sin\left(x\right)$ est appelée \textbf{fonction sinus}.
}
\cadre{rouge}{Formules de base}{%id="t50"
     Pour tout réel $x$ :
     \begin{itemize}
          \item $\cos\left(x+2\pi \right)=\cos\left(x\right)$
          \item $\sin\left(x+2\pi \right)=\sin\left(x\right)$.
     \end{itemize}
     On dit que les fonctions sinus et cosinus sont \textbf{périodiques} de période $2\pi $.
     \par
     \begin{itemize}
          \item $\cos\left(-x\right)=\cos\left(x\right) $ (la fonction cosinus est paire)
          \item $\sin\left(-x\right)=-\sin\left(x\right) $ (la fonction sinus est impaire)
     \end{itemize}
     \par
     \begin{itemize}
          \item $\cos\left(x+\pi \right)=-\cos\left(x\right)$
          \item $\sin\left(x+\pi \right)=-\sin\left(x\right)$
     \end{itemize}
     \par
     \begin{itemize}
          \item $\cos\left(x+\frac{\pi }{2}\right)=-\sin\left(x\right) $
          \item $\sin\left(x+\frac{\pi }{2}\right)=\cos\left(x\right) $.
     \end{itemize}
}
\bloc{cyan}{Remarque}{%id="r50"
     A partir des formules de base on peut montrer d'autres formules; par exemple :
     \par
     $\cos\left(\frac{\pi }{2}-x\right)=\cos\left(-x+\frac{\pi }{2}\right)=-\sin\left(-x\right)=\sin\left(x\right)$
     \par
     $\sin\left(\frac{\pi }{2}-x\right)=\sin\left(-x+\frac{\pi }{2}\right)=\cos\left(-x\right)=\cos\left(x\right)$
     \par
     etc.
}
\cadre{rouge}{Formules d'addition}{%id="t60"
     Pour tous réels $a$ et $b$ :
     \begin{itemize}
          \item $\cos\left(a+b\right)=\cos\left(a\right) \cos\left(b\right)-\sin\left(a\right) \sin\left(b\right)$
          \item $\sin\left(a+b\right)=\sin\left(a\right) \cos\left(b\right)+\cos\left(a\right) \sin\left(b\right)$
     \end{itemize}
}
\bloc{cyan}{Remarque}{%id="r60"
     En remplaçant $b$ par $-b$ et en utilisant la parité des fonctions sinus et cosinus on obtient les formules de soustraction:
     \begin{itemize}
          \item $\cos\left(a-b\right)=\cos\left(a\right) \cos\left(b\right)+\sin\left(a\right) \sin\left(b\right)$
          \item $\sin\left(a-b\right)=\sin\left(a\right) \cos\left(b\right)-\cos\left(a\right) \sin\left(b\right)$
     \end{itemize}
}
\cadre{vert}{Propriété (formules de duplication)}{%id="p70"
     Pour tout réel $a$ :
     \begin{itemize}
          \item $\cos\left(2a\right) = \cos^{2}\left(a\right)-\sin^{2}\left(a\right) = 2\cos^{2}\left(a\right)-1 = 1-2\sin^{2}\left(a\right)$
          \item $\sin\left(2a\right) = 2\sin\left(a\right) \cos\left(a\right)$.
     \end{itemize}
}
\bloc{cyan}{Remarques}{%id="r70"
     \begin{itemize}
          \item On démontre ces formules en posant $b=a$ dans les formules d'addition et en utilisant $\sin^{2}\left(a\right)+\cos^{2}\left(a\right)=1$.
          \item Rappel : $\sin^{2}\left(a\right)$ et $\cos^{2}\left(a\right)$ sont des écritures simplifiées pour $\left(\sin\left(a\right)\right)^{2}$ et $\left(\cos\left(a\right)\right)^{2}$.
     \end{itemize}
}
\begin{h2}3. Etude des fonctions sinus et cosinus\end{h2}
\cadre{rouge}{Théorème}{%id="t80"
     Les fonctions sinus et cosinus sont \textbf{dérivables} sur $\mathbb{R}$ et leurs dérivées sont :
     \begin{center}$\sin^{\prime}=\cos$\end{center}
     \begin{center}$\cos^{\prime}=-\sin$\end{center}
}
\cadre{vert}{Propriétés}{%id="p90"
     Soient $a$ et $b$ deux réels quelconques. Les fonctions $f$ et $g$ définies sur $\mathbb{R}$ par :
     \begin{itemize}
          \item $ f : x\mapsto \sin\left(ax+b\right)$
          \item $ g : x\mapsto \cos\left(ax+b\right)$
     \end{itemize}
     sont dérivables sur $\mathbb{R}$ et :
     \begin{itemize}
          \item $ f^{\prime}\left(x\right)=a \cos\left(ax+b\right)$
          \item $ g^{\prime}\left(x\right)=-a \sin\left(ax+b\right)$
     \end{itemize}
     Plus généralement, si $u$ est une fonction dérivable sur un intervalle $I$ et $f$ et $g$ définies sur $I$ par :
     \begin{itemize}
          \item $ f : x\mapsto \sin\left(u\left(x\right)\right)$
          \item $ g : x\mapsto \cos\left(u\left(x\right)\right)$
     \end{itemize}
     alors $f$ et $g$ sont dérivables sur $I$ et :
     \begin{itemize}
          \item $ f^{\prime}\left(x\right)=u^{\prime}\left(x\right)\times \cos\left(u\left(x\right)\right)$
          \item $ g^{\prime}\left(x\right)=-u^{\prime}\left(x\right)\times \sin\left(u\left(x\right)\right)$
     \end{itemize}
}
\bloc{cyan}{Remarque}{%id="r90"
     C'est un cas particulier du \mcLien{/cours/terminale-s/fonctions-continues\#t100}{ théorème de dérivation de fonctions composées}.
}
\cadre{vert}{Limites}{%id="p95"
     Les fonctions sinus et cosinus \textbf{ne possèdent pas de limite quand $x\rightarrow \pm\infty $}
     Par contre on démontre le résultat suivant :
     \par
     $\lim\limits_{x\rightarrow 0}\frac{\sin\left(x\right)}{x}=1$
}
\bloc{cyan}{Remarque}{%id="r95"
     Cette dernière limite peut s'obtenir en utilisant la définition du nombre dérivé de la fonction sinus pour $x=0$ (voir fiche méthode \mcLien{/methodes/limites/calcul-limite-nombre-derive}{Calculer une limite à l'aide du nombre dérivé}).
}
Les fonctions sinus et cosinus étant périodiques, il suffit de les étudier sur un intervalle d'amplitude $2\pi $, par exemple $\left[-\pi ; \pi \right]$.
\par
Pour obtenir la courbe complète, on effectue ensuite des translations de vecteurs $\pm2\pi \vec{i}$.
\bigskip
\begin{center}
\begin{h3}Fonction sinus\end{h3}
\end{center}
\begin{extern}
     \begin{center}
          \begin{tikzpicture}[scale=0.875]
               % Styles
               \tikzstyle{cadre}=[thin]
               \tikzstyle{fleche}=[->,>=latex,thin]
               \tikzstyle{nondefini}=[lightgray]
               % Dimensions Modifiables
               \def\Lrg{1.5}
               \def\HtX{1.2}
               \def\HtY{0.5}
               % Dimensions Calculées
               \def\lignex{-0.5*\HtX}
               \def\lignef{-1.5*\HtX}
               \def\separateur{-0.5*\Lrg}
               % Largeur du tableau
               \def\gauche{-3.5*\Lrg}
               \def\droite{6.5*\Lrg}
               % Hauteur du tableau
               \def\haut{0.5*\HtX}
               \def\bas{-2.5*\HtX-2*\HtY}
               % Ligne de l'abscisse : x
               \node at (-2*\Lrg,0) {$x$};
               \node at (0*\Lrg,0) {$-\pi$};
               \node at (2*\Lrg,0) {$-\dfrac{\pi}{2}$};
               \node at (4*\Lrg,0) {$\dfrac{\pi}{2}$};
               \node at (6*\Lrg,0) {$\pi$};
               % Ligne de la dérivée : f'(x)
               \node at (-2*\Lrg,-1*\HtX) {$f'(x)=\cos(x)$};
               \node at (0*\Lrg,-1*\HtX) {$ $};
               \node at (1*\Lrg,-1*\HtX) {$-$};
               \node at (2*\Lrg,-1*\HtX) {$0$};
               \node at (3*\Lrg,-1*\HtX) {$+$};
               \node at (4*\Lrg,-1*\HtX) {$0$};
               \node at (5*\Lrg,-1*\HtX) {$-$};
               \node at (6*\Lrg,-1*\HtX) {$ $};
               % Ligne de la fonction : f(x)
               \node  at (-2*\Lrg,{-2*\HtX+(-1)*\HtY}) {$f(x)=\sin(x)$};
               \node (f1) at (0*\Lrg,{-2*\HtX+(0)*\HtY}) {$0$};
               \node (f2) at (2*\Lrg,{-2*\HtX+(-2)*\HtY}) {$-1$};
               \node (f3) at (4*\Lrg,{-2*\HtX+(0)*\HtY}) {$1$};
               \node (f4) at (6*\Lrg,{-2*\HtX+(-2)*\HtY}) {$0$};
               % Flèches
               \draw[fleche] (f1) -- (f2);
               \draw[fleche] (f2) -- (f3);
               \draw[fleche] (f3) -- (f4);
               % Encadrement
               \draw[cadre] (\separateur,\haut) -- (\separateur,\bas);
               \draw[cadre] (\gauche,\haut) rectangle  (\droite,\bas);
               \draw[cadre] (\gauche,\lignex) -- (\droite,\lignex);
               \draw[cadre] (\gauche,\lignef) -- (\droite,\lignef);
          \end{tikzpicture}
     \end{center}
\end{extern}
\begin{center}\textit{Tableau de variation de la fonction sinus}\end{center}
\begin{center}
     \begin{extern} %width="450" alt="fonction sinus"
          \resizebox{8cm}{!}{%
               % -+-+-+ variables modifiables
               \def\fonction{SIN(x) }
               \def\xmin{-6.5}
               \def\xmax{6.5}
               \def\ymin{-1.5}
               \def\ymax{1.5}
               \def\xunit{1}  % unités en cm
               \def\yunit{1.5}
               \psset{xunit=\xunit,yunit=\yunit,algebraic=true}
               \fontsize{15pt}{15pt}\selectfont
               \begin{pspicture*}[linewidth=1pt](\xmin,\ymin)(\xmax,\ymax)
                    %      \psgrid[gridcolor=mcgris, subgriddiv=5, gridlabels=0pt](\xmin,\ymin)(\xmax,\ymax)
                    \psaxes[linewidth=0.75pt]{->}(0,0)(\xmin,\ymin)(\xmax,\ymax)
                    \psplot[linewidth=0.75pt,plotpoints=2000,linecolor=black]{\xmin}{\xmax}{\fonction}
                    \psplot[linewidth=1.2pt,plotpoints=2000,linecolor=red]{-3.14159}{3.14159}{\fonction}
                    \rput[tr](-0.1,-0.2){$O$}
                    \psline[linewidth=1.25pt]{->}(0,0)(0,1)
                         \psline[linewidth=1.25pt]{->}(0,0)(1,0)
                         \rput[t](0.5,-0.03){$\vect{i}$}
                         \rput[r](-0.03,0.5){$\vect{j}$}
               \end{pspicture*}
          }
     \end{extern}
\end{center}
\begin{center}\textit{Représentation graphique de la fonction sinus}\end{center}
\bigskip
\begin{center}
\begin{h3}Fonction cosinus\end{h3}
\end{center}
\begin{center}
     \begin{extern}%width="420"
          \begin{tikzpicture}[scale=0.875]
               % Styles
               \tikzstyle{cadre}=[thin]
               \tikzstyle{fleche}=[->,>=latex,thin]
               \tikzstyle{nondefini}=[lightgray]
               % Dimensions Modifiables
               \def\Lrg{1.5}
               \def\HtX{1}
               \def\HtY{0.5}
               % Dimensions Calculées
               \def\lignex{-0.5*\HtX}
               \def\lignef{-1.5*\HtX}
               \def\separateur{-0.5*\Lrg}
               % Largeur du tableau
               \def\gauche{-3.5*\Lrg}
               \def\droite{4.5*\Lrg}
               % Hauteur du tableau
               \def\haut{0.5*\HtX}
               \def\bas{-2.5*\HtX-2*\HtY}
               % Ligne de l'abscisse : x
               \node at (-2*\Lrg,0) {$x$};
               \node at (0*\Lrg,0) {$-\pi$};
               \node at (2*\Lrg,0) {$0$};
               \node at (4*\Lrg,0) {$\pi$};
               % Ligne de la dérivée : f'(x)
               \node at (-2*\Lrg,-1*\HtX) {$f'(x)=-\sin x$};
               \node at (0*\Lrg,-1*\HtX) {$ $};
               \node at (1*\Lrg,-1*\HtX) {$+$};
               \node at (2*\Lrg,-1*\HtX) {$0$};
               \node at (3*\Lrg,-1*\HtX) {$-$};
               \node at (4*\Lrg,-1*\HtX) {$0$};
               % Ligne de la fonction : f(x)
               \node  at (-2*\Lrg,{-2*\HtX+(-1)*\HtY}) {$f(x)=\cos x$};
               \node (f1) at (0*\Lrg,{-2*\HtX+(-2)*\HtY}) {$-1$};
               \node (f2) at (2*\Lrg,{-2*\HtX+(0)*\HtY}) {$1$};
               \node (f3) at (4*\Lrg,{-2*\HtX+(-2)*\HtY}) {$-1$};
               % Flèches
               \draw[fleche] (f1) -- (f2);
               \draw[fleche] (f2) -- (f3);
               % Encadrement
               \draw[cadre] (\separateur,\haut) -- (\separateur,\bas);
               \draw[cadre] (\gauche,\haut) rectangle  (\droite,\bas);
               \draw[cadre] (\gauche,\lignex) -- (\droite,\lignex);
               \draw[cadre] (\gauche,\lignef) -- (\droite,\lignef);
          \end{tikzpicture}
     \end{extern}
\end{center}
\begin{center}\textit{Tableau de variation de la fonction cosinus}\end{center}
\begin{center}
     \begin{extern} %width="450" alt="fonction cosinus"
          \resizebox{8cm}{!}{%
               % -+-+-+ variables modifiables
               \def\fonction{COS(x) }
               \def\xmin{-6.5}
               \def\xmax{6.5}
               \def\ymin{-1.5}
               \def\ymax{1.5}
               \def\xunit{1}  % unités en cm
               \def\yunit{1.5}
               \psset{xunit=\xunit,yunit=\yunit,algebraic=true}
               \fontsize{15pt}{15pt}\selectfont
               \begin{pspicture*}[linewidth=1pt](\xmin,\ymin)(\xmax,\ymax)
                    %      \psgrid[gridcolor=mcgris, subgriddiv=5, gridlabels=0pt](\xmin,\ymin)(\xmax,\ymax)
                    \psaxes[linewidth=0.75pt]{->}(0,0)(\xmin,\ymin)(\xmax,\ymax)
                    \psplot[linewidth=0.75pt,plotpoints=2000,linecolor=black]{\xmin}{\xmax}{\fonction}
                    \psplot[linewidth=1.2pt,plotpoints=2000,linecolor=blue]{-3.14159}{3.14159}{\fonction}
                    \rput[tr](-0.1,-0.2){$O$}
                    \psline[linewidth=1.25pt]{->}(0,0)(0,1)
                         \psline[linewidth=1.25pt]{->}(0,0)(1,0)
                         \rput[t](0.5,-0.03){$\vect{i}$}
                         \rput[r](-0.03,0.5){$\vect{j}$}
               \end{pspicture*}
          }
     \end{extern}
\end{center}\begin{center}\textit{Représentation graphique de la fonction cosinus}\end{center}
\bloc{cyan}{Remarque}{%id="r100"
     La relation $\sin\left(x+\frac{\pi }{2}\right)=\cos\left(x\right) $ montre que la courbe de la fonction sinus se déduit de la courbe de la fonction cosinus par une translation de vecteur $\frac{\pi }{2}\vec{i}$.
     \begin{center}
          \begin{extern} %width="450" alt="fonction cosinus"
               \resizebox{8cm}{!}{%
                    % -+-+-+ variables modifiables
                    \def\fonction{COS(x) }
                    \def\g{SIN(x) }
                    \def\xmin{-6.5}
                    \def\xmax{6.5}
                    \def\ymin{-1.5}
                    \def\ymax{1.5}
                    \def\xunit{1}  % unités en cm
                    \def\yunit{1.5}
                    \psset{xunit=\xunit,yunit=\yunit,algebraic=true}
                    \fontsize{15pt}{15pt}\selectfont
                    \begin{pspicture*}[linewidth=1pt](\xmin,\ymin)(\xmax,\ymax)
                         %      \psgrid[gridcolor=mcgris, subgriddiv=5, gridlabels=0pt](\xmin,\ymin)(\xmax,\ymax)
                         \psaxes[linewidth=0.75pt]{->}(0,0)(\xmin,\ymin)(\xmax,\ymax)
                         \psplot[linewidth=0.75pt,plotpoints=2000,linecolor=blue]{\xmin}{\xmax}{\fonction}
                         \psplot[linewidth=0.75pt,plotpoints=2000,linecolor=red]{\xmin}{\xmax}{\g}
                         \rput[tr](-0.1,-0.2){$O$}
                         \rput(4.5,0.7){$\blue \cos$}
                         \rput(-3,0.7){$\red \sin$}
                         \psline[linecolor=vert]{->}(-5,0.284)(-3.429,0.284)
                         \psline[linewidth=1.25pt]{->}(0,0)(0,1)
                         \psline[linewidth=1.25pt]{->}(0,0)(1,0)
                         \rput[t](0.5,-0.03){$\vect{i}$}
                         \rput[r](-0.03,0.5){$\vect{j}$}
                    \end{pspicture*}
               }
          \end{extern}
     \end{center}
     \begin{center}\textit{Position relative des deux courbes}\end{center}
}

\end{document}