\documentclass[a4paper]{article}

%================================================================================================================================
%
% Packages
%
%================================================================================================================================

\usepackage[T1]{fontenc} 	% pour caractères accentués
\usepackage[utf8]{inputenc}  % encodage utf8
\usepackage[french]{babel}	% langue : français
\usepackage{fourier}			% caractères plus lisibles
\usepackage[dvipsnames]{xcolor} % couleurs
\usepackage{fancyhdr}		% réglage header footer
\usepackage{needspace}		% empêcher sauts de page mal placés
\usepackage{graphicx}		% pour inclure des graphiques
\usepackage{enumitem,cprotect}		% personnalise les listes d'items (nécessaire pour ol, al ...)
\usepackage{hyperref}		% Liens hypertexte
\usepackage{pstricks,pst-all,pst-node,pstricks-add,pst-math,pst-plot,pst-tree,pst-eucl} % pstricks
\usepackage[a4paper,includeheadfoot,top=2cm,left=3cm, bottom=2cm,right=3cm]{geometry} % marges etc.
\usepackage{comment}			% commentaires multilignes
\usepackage{amsmath,environ} % maths (matrices, etc.)
\usepackage{amssymb,makeidx}
\usepackage{bm}				% bold maths
\usepackage{tabularx}		% tableaux
\usepackage{colortbl}		% tableaux en couleur
\usepackage{fontawesome}		% Fontawesome
\usepackage{environ}			% environment with command
\usepackage{fp}				% calculs pour ps-tricks
\usepackage{multido}			% pour ps tricks
\usepackage[np]{numprint}	% formattage nombre
\usepackage{tikz,tkz-tab} 			% package principal TikZ
\usepackage{pgfplots}   % axes
\usepackage{mathrsfs}    % cursives
\usepackage{calc}			% calcul taille boites
\usepackage[scaled=0.875]{helvet} % font sans serif
\usepackage{svg} % svg
\usepackage{scrextend} % local margin
\usepackage{scratch} %scratch
\usepackage{multicol} % colonnes
%\usepackage{infix-RPN,pst-func} % formule en notation polanaise inversée
\usepackage{listings}

%================================================================================================================================
%
% Réglages de base
%
%================================================================================================================================

\lstset{
language=Python,   % R code
literate=
{á}{{\'a}}1
{à}{{\`a}}1
{ã}{{\~a}}1
{é}{{\'e}}1
{è}{{\`e}}1
{ê}{{\^e}}1
{í}{{\'i}}1
{ó}{{\'o}}1
{õ}{{\~o}}1
{ú}{{\'u}}1
{ü}{{\"u}}1
{ç}{{\c{c}}}1
{~}{{ }}1
}


\definecolor{codegreen}{rgb}{0,0.6,0}
\definecolor{codegray}{rgb}{0.5,0.5,0.5}
\definecolor{codepurple}{rgb}{0.58,0,0.82}
\definecolor{backcolour}{rgb}{0.95,0.95,0.92}

\lstdefinestyle{mystyle}{
    backgroundcolor=\color{backcolour},   
    commentstyle=\color{codegreen},
    keywordstyle=\color{magenta},
    numberstyle=\tiny\color{codegray},
    stringstyle=\color{codepurple},
    basicstyle=\ttfamily\footnotesize,
    breakatwhitespace=false,         
    breaklines=true,                 
    captionpos=b,                    
    keepspaces=true,                 
    numbers=left,                    
xleftmargin=2em,
framexleftmargin=2em,            
    showspaces=false,                
    showstringspaces=false,
    showtabs=false,                  
    tabsize=2,
    upquote=true
}

\lstset{style=mystyle}


\lstset{style=mystyle}
\newcommand{\imgdir}{C:/laragon/www/newmc/assets/imgsvg/}
\newcommand{\imgsvgdir}{C:/laragon/www/newmc/assets/imgsvg/}

\definecolor{mcgris}{RGB}{220, 220, 220}% ancien~; pour compatibilité
\definecolor{mcbleu}{RGB}{52, 152, 219}
\definecolor{mcvert}{RGB}{125, 194, 70}
\definecolor{mcmauve}{RGB}{154, 0, 215}
\definecolor{mcorange}{RGB}{255, 96, 0}
\definecolor{mcturquoise}{RGB}{0, 153, 153}
\definecolor{mcrouge}{RGB}{255, 0, 0}
\definecolor{mclightvert}{RGB}{205, 234, 190}

\definecolor{gris}{RGB}{220, 220, 220}
\definecolor{bleu}{RGB}{52, 152, 219}
\definecolor{vert}{RGB}{125, 194, 70}
\definecolor{mauve}{RGB}{154, 0, 215}
\definecolor{orange}{RGB}{255, 96, 0}
\definecolor{turquoise}{RGB}{0, 153, 153}
\definecolor{rouge}{RGB}{255, 0, 0}
\definecolor{lightvert}{RGB}{205, 234, 190}
\setitemize[0]{label=\color{lightvert}  $\bullet$}

\pagestyle{fancy}
\renewcommand{\headrulewidth}{0.2pt}
\fancyhead[L]{maths-cours.fr}
\fancyhead[R]{\thepage}
\renewcommand{\footrulewidth}{0.2pt}
\fancyfoot[C]{}

\newcolumntype{C}{>{\centering\arraybackslash}X}
\newcolumntype{s}{>{\hsize=.35\hsize\arraybackslash}X}

\setlength{\parindent}{0pt}		 
\setlength{\parskip}{3mm}
\setlength{\headheight}{1cm}

\def\ebook{ebook}
\def\book{book}
\def\web{web}
\def\type{web}

\newcommand{\vect}[1]{\overrightarrow{\,\mathstrut#1\,}}

\def\Oij{$\left(\text{O}~;~\vect{\imath},~\vect{\jmath}\right)$}
\def\Oijk{$\left(\text{O}~;~\vect{\imath},~\vect{\jmath},~\vect{k}\right)$}
\def\Ouv{$\left(\text{O}~;~\vect{u},~\vect{v}\right)$}

\hypersetup{breaklinks=true, colorlinks = true, linkcolor = OliveGreen, urlcolor = OliveGreen, citecolor = OliveGreen, pdfauthor={Didier BONNEL - https://www.maths-cours.fr} } % supprime les bordures autour des liens

\renewcommand{\arg}[0]{\text{arg}}

\everymath{\displaystyle}

%================================================================================================================================
%
% Macros - Commandes
%
%================================================================================================================================

\newcommand\meta[2]{    			% Utilisé pour créer le post HTML.
	\def\titre{titre}
	\def\url{url}
	\def\arg{#1}
	\ifx\titre\arg
		\newcommand\maintitle{#2}
		\fancyhead[L]{#2}
		{\Large\sffamily \MakeUppercase{#2}}
		\vspace{1mm}\textcolor{mcvert}{\hrule}
	\fi 
	\ifx\url\arg
		\fancyfoot[L]{\href{https://www.maths-cours.fr#2}{\black \footnotesize{https://www.maths-cours.fr#2}}}
	\fi 
}


\newcommand\TitreC[1]{    		% Titre centré
     \needspace{3\baselineskip}
     \begin{center}\textbf{#1}\end{center}
}

\newcommand\newpar{    		% paragraphe
     \par
}

\newcommand\nosp {    		% commande vide (pas d'espace)
}
\newcommand{\id}[1]{} %ignore

\newcommand\boite[2]{				% Boite simple sans titre
	\vspace{5mm}
	\setlength{\fboxrule}{0.2mm}
	\setlength{\fboxsep}{5mm}	
	\fcolorbox{#1}{#1!3}{\makebox[\linewidth-2\fboxrule-2\fboxsep]{
  		\begin{minipage}[t]{\linewidth-2\fboxrule-4\fboxsep}\setlength{\parskip}{3mm}
  			 #2
  		\end{minipage}
	}}
	\vspace{5mm}
}

\newcommand\CBox[4]{				% Boites
	\vspace{5mm}
	\setlength{\fboxrule}{0.2mm}
	\setlength{\fboxsep}{5mm}
	
	\fcolorbox{#1}{#1!3}{\makebox[\linewidth-2\fboxrule-2\fboxsep]{
		\begin{minipage}[t]{1cm}\setlength{\parskip}{3mm}
	  		\textcolor{#1}{\LARGE{#2}}    
 	 	\end{minipage}  
  		\begin{minipage}[t]{\linewidth-2\fboxrule-4\fboxsep}\setlength{\parskip}{3mm}
			\raisebox{1.2mm}{\normalsize\sffamily{\textcolor{#1}{#3}}}						
  			 #4
  		\end{minipage}
	}}
	\vspace{5mm}
}

\newcommand\cadre[3]{				% Boites convertible html
	\par
	\vspace{2mm}
	\setlength{\fboxrule}{0.1mm}
	\setlength{\fboxsep}{5mm}
	\fcolorbox{#1}{white}{\makebox[\linewidth-2\fboxrule-2\fboxsep]{
  		\begin{minipage}[t]{\linewidth-2\fboxrule-4\fboxsep}\setlength{\parskip}{3mm}
			\raisebox{-2.5mm}{\sffamily \small{\textcolor{#1}{\MakeUppercase{#2}}}}		
			\par		
  			 #3
 	 		\end{minipage}
	}}
		\vspace{2mm}
	\par
}

\newcommand\bloc[3]{				% Boites convertible html sans bordure
     \needspace{2\baselineskip}
     {\sffamily \small{\textcolor{#1}{\MakeUppercase{#2}}}}    
		\par		
  			 #3
		\par
}

\newcommand\CHelp[1]{
     \CBox{Plum}{\faInfoCircle}{À RETENIR}{#1}
}

\newcommand\CUp[1]{
     \CBox{NavyBlue}{\faThumbsOUp}{EN PRATIQUE}{#1}
}

\newcommand\CInfo[1]{
     \CBox{Sepia}{\faArrowCircleRight}{REMARQUE}{#1}
}

\newcommand\CRedac[1]{
     \CBox{PineGreen}{\faEdit}{BIEN R\'EDIGER}{#1}
}

\newcommand\CError[1]{
     \CBox{Red}{\faExclamationTriangle}{ATTENTION}{#1}
}

\newcommand\TitreExo[2]{
\needspace{4\baselineskip}
 {\sffamily\large EXERCICE #1\ (\emph{#2 points})}
\vspace{5mm}
}

\newcommand\img[2]{
          \includegraphics[width=#2\paperwidth]{\imgdir#1}
}

\newcommand\imgsvg[2]{
       \begin{center}   \includegraphics[width=#2\paperwidth]{\imgsvgdir#1} \end{center}
}


\newcommand\Lien[2]{
     \href{#1}{#2 \tiny \faExternalLink}
}
\newcommand\mcLien[2]{
     \href{https~://www.maths-cours.fr/#1}{#2 \tiny \faExternalLink}
}

\newcommand{\euro}{\eurologo{}}

%================================================================================================================================
%
% Macros - Environement
%
%================================================================================================================================

\newenvironment{tex}{ %
}
{%
}

\newenvironment{indente}{ %
	\setlength\parindent{10mm}
}

{
	\setlength\parindent{0mm}
}

\newenvironment{corrige}{%
     \needspace{3\baselineskip}
     \medskip
     \textbf{\textsc{Corrigé}}
     \medskip
}
{
}

\newenvironment{extern}{%
     \begin{center}
     }
     {
     \end{center}
}

\NewEnviron{code}{%
	\par
     \boite{gray}{\texttt{%
     \BODY
     }}
     \par
}

\newenvironment{vbloc}{% boite sans cadre empeche saut de page
     \begin{minipage}[t]{\linewidth}
     }
     {
     \end{minipage}
}
\NewEnviron{h2}{%
    \needspace{3\baselineskip}
    \vspace{0.6cm}
	\noindent \MakeUppercase{\sffamily \large \BODY}
	\vspace{1mm}\textcolor{mcgris}{\hrule}\vspace{0.4cm}
	\par
}{}

\NewEnviron{h3}{%
    \needspace{3\baselineskip}
	\vspace{5mm}
	\textsc{\BODY}
	\par
}

\NewEnviron{margeneg}{ %
\begin{addmargin}[-1cm]{0cm}
\BODY
\end{addmargin}
}

\NewEnviron{html}{%
}

\begin{document}
\begin{h2}1. Vocabulaire\end{h2}
\cadre{bleu}{Définition}{%
     Un \textbf{graphe} est composé de \textbf{sommets} et d'\textbf{arêtes} (ou \textbf{arcs}) reliant certains de ces sommets.
}
\bloc{orange}{Exemple}{%
     Le diagramme ci-dessous représente un graphe comportant 4 sommets et 5 arêtes.
     \begin{center}
     \begin{extern}%width="350"
          \psset{unit=1.5cm}
          \begin{pspicture}(-1,0)(6,3)
               \rput(0,2){\circlenode{A}{A}}
               \rput(2,2){\circlenode{B}{B}}
               \rput(5,2){\circlenode{C}{C}}
               \rput(3.5,0.5){\circlenode{D}{D}}
               \ncarc[arcangle=-20]{A}{B}
                \ncarc[arcangle=-20]{B}{C}
                \ncarc[arcangle=-20]{B}{D}
                \ncarc[arcangle=20]{C}{D}
				\nccircle[angleA=-50]{C}{.5cm}
          \end{pspicture}
     \end{extern}
\end{center}
}
\cadre{bleu}{Définitions}{%
     \begin{itemize}
          \item L'\textbf{ordre} d'un graphe est le nombre de sommets de ce graphe.
          \item Le \textbf{degré} d'un sommet est le nombre d'arêtes dont ce sommet est une extrémité.
          \item Deux sommets reliés par une arête sont \textbf{adjacents}.
     \end{itemize}
}
\bloc{orange}{Exemple}{%
     \begin{itemize}
          \item Le graphe représenté ci-dessus est d'ordre 4.
          \item Le degré du sommet B est 3. Celui de C est 4 (la boucle compte 2 fois).
          \item A et B sont adjacents. A et D ne le sont pas.
     \end{itemize}
}
\cadre{bleu}{Définitions}{%
     Une \textbf{chaîne} (ou un \textbf{chemin}) est une suite de sommets telle que chaque sommet est relié au suivant par une arête.
     \par
     La \textbf{longueur} d'une chaîne est le nombre d'arêtes composant cette chaîne.
}
\bloc{orange}{Exemple}{%
       \begin{center}
     \begin{extern}%width="350"
          \psset{unit=1.5cm}
          \begin{pspicture}(-1,0)(6,3)
               \rput(0,2){\circlenode{A}{A}}
               \rput(2,2){\circlenode{B}{B}}
               \rput(5,2){\circlenode{C}{C}}
               \rput(3.5,0.5){\circlenode{D}{D}}
               \ncarc[linecolor=red,arcangle=-20]{A}{B}
                \ncarc[linecolor=red,arcangle=-20]{B}{C}
                \ncarc[arcangle=-20]{B}{D}
                \ncarc[linecolor=red,arcangle=20]{C}{D}
				\nccircle[angleA=-50]{C}{.5cm}
          \end{pspicture}
     \end{extern}
\end{center}
    (A; B; C; D) est une chaîne de longueur 3.
}
\cadre{bleu}{Définition}{%
     Un \textbf{cycle} est une chaîne \textbf{fermée} (c'est à dire dont l'origine et l'extrémité sont identiques) dont toutes les arêtes sont distinctes.
}
\bloc{orange}{Exemple}{%
       \begin{center}
     \begin{extern}%width="350"
          \psset{unit=1.5cm}
          \begin{pspicture}(-1,0)(6,3)
               \rput(0,2){\circlenode{A}{A}}
               \rput(2,2){\circlenode{B}{B}}
               \rput(5,2){\circlenode{C}{C}}
               \rput(3.5,0.5){\circlenode{D}{D}}
               \ncarc[arcangle=-20]{A}{B}
                \ncarc[linecolor=red,arcangle=-20]{B}{C}
                \ncarc[linecolor=red,arcangle=-20]{B}{D}
                \ncarc[linecolor=red,arcangle=20]{C}{D}
				\nccircle[linecolor=red,angleA=-50]{C}{.5cm}
          \end{pspicture}
     \end{extern}
\end{center}     (B; C; C; D; B) est un cycle.
}
\cadre{bleu}{Définition}{%
     On dit qu'un graphe est \textbf{connexe} si deux sommets quelconques peuvent être reliés par une chaîne.
}
\bloc{cyan}{Remarque}{%
     Intuitivement, cela signifie que le graphe comporte un seul "morceau"
}
\bloc{orange}{Exemple}{%
 \begin{vbloc}
      \begin{center}
     \begin{extern}%width="300" 
          \psset{unit=1.5cm}
          \begin{pspicture}(-1,0)(6,3)
               \rput(0,2){\circlenode{A}{A}}
               \rput(2,2){\circlenode{B}{B}}
               \rput(5,2){\circlenode{C}{C}}
               \rput(3.5,0.5){\circlenode{D}{D}}
               \ncarc[arcangle=-20]{A}{B}
                \ncarc[arcangle=-20]{B}{C}
                \ncarc[arcangle=-20]{B}{D}
                \ncarc[arcangle=20]{C}{D}
				\nccircle[angleA=-50]{C}{.5cm}
          \end{pspicture}
     \end{extern}
\end{center}
\begin{center}
Graphe connexe
\end{center}
 \end{vbloc}
  \begin{vbloc}
     \begin{center}
     \begin{extern}%width="300"
          \psset{unit=1.5cm}
          \begin{pspicture}(-1,0)(6,3)
               \rput(0,2){\circlenode{A}{A}}
               \rput(2,2){\circlenode{B}{B}}
               \rput(5,2){\circlenode{C}{C}}
               \rput(3.5,0.5){\circlenode{D}{D}}
               \ncarc[arcangle=-20]{A}{B}
                 \ncarc[arcangle=20]{C}{D}
				\nccircle[angleA=-50]{C}{.5cm}
          \end{pspicture}
     \end{extern}
\end{center}
\begin{center}
Graphe non connexe
\end{center}
 \end{vbloc}
}
\begin{h2}2. Chaînes et cycles eulériens\end{h2}
\cadre{bleu}{Définition}{%
     Une \textbf{chaîne eulérienne} est une chaîne qui contient une fois et une seule chacune des arêtes du graphe.
     \par
     Si cette chaîne est un cycle, on parle de \textbf{cycle eulérien}.
}
\bloc{orange}{Exemple}{%
     \begin{center}
     \begin{extern}%width="350"
          \psset{unit=1.5cm}
          \begin{pspicture}(-1,0)(6,3)
               \rput(0,2){\circlenode{A}{A}}
               \rput(2,2){\circlenode{B}{B}}
               \rput(5,2){\circlenode{C}{C}}
               \rput(3.5,0.5){\circlenode{D}{D}}
               \ncarc[arcangle=-20]{A}{B}
                \ncarc[arcangle=-20]{B}{C}
                \ncarc[arcangle=-20]{B}{D}
                \ncarc[arcangle=20]{C}{D}
				\nccircle[angleA=-50]{C}{.5cm}
          \end{pspicture}
     \end{extern}
\end{center}
     (A; B; C; C; D; B) est une chaîne eulérienne.
     \par
     Ce graphe ne contient aucun cycle eulérien.
}
\bloc{cyan}{Remarque}{%
     \begin{itemize}
          \item Un graphe connexe contient une chaîne eulérienne si et seulement si on peut le tracer "\textit{sans lever le crayon}". Le théorème d'Euler (ci-dessous) permet de déterminer facilement ce type de graphe.
          \item On ne peut jamais tracer un graphe \textbf{non connexe} sans lever le crayon !
     \end{itemize}
}
\cadre{rouge}{Théorème}{%
     \textbf{Théorème d'Euler.}
     Un graphe connexe contient une chaîne eulérienne si et seulement si il possède 0 ou 2 sommets de degré impair.
     \par
     Un graphe connexe contient un cycle eulérien si et seulement si il ne possède aucun sommet de degré impair (autrement dit tous ses sommets sont de degré pair)
}
\bloc{orange}{Exemples}{%
%     <img src="/wp-content/uploads/th_euler.png" alt="" class="aligncenter size-full  img-bpc" />

           \begin{vbloc}
               \begin{center}
     \begin{extern}%width="350"
          \psset{unit=1.5cm}
          \begin{pspicture}(-1,0)(6,3)
               \rput(0,2){\circlenode{A}{A}}
               \rput(2,2){\circlenode{B}{B}}
               \rput(5,2){\circlenode{C}{C}}
               \rput(3.5,0.5){\circlenode{D}{D}}
               \ncarc[arcangle=-20]{A}{B}
                \ncarc[arcangle=-20]{B}{C}
                \ncarc[arcangle=-20]{B}{D}
                \ncarc[arcangle=20]{C}{D}
				\nccircle[angleA=-50]{C}{.5cm}
          \end{pspicture}
     \end{extern}
\end{center}
\begin{center}
\textit{Exemple 1}
\end{center}
 \end{vbloc}
          Dans l'\textit{exemple 1}, il y a deux sommets de degré impair (A:1 et B:3). Le graphe contient une chaîne eulérienne, par exemple (A; B; C; C; D; B) mais pas de cycle eulérien.
          \par
                     \begin{vbloc}
               \begin{center}
     \begin{extern}%width="280"
          \psset{unit=1.5cm}
          \begin{pspicture}(-1,0)(6,3)
               \rput(0,0){\circlenode{A}{A}}
               \rput(0,2){\circlenode{B}{B}}
               \rput(2,3){\circlenode{C}{C}}
               \rput(4,2){\circlenode{D}{D}}
               \rput(4,0){\circlenode{E}{E}}
                \rput(2,1){\circlenode{F}{F}}
               \ncline{A}{B} \ncline{B}{C}\ncline{C}{D}\ncline{D}{E}\ncline{E}{A}\ncline{B}{D}
				\ncline{A}{F} \ncline{B}{F}\ncline{F}{D}\ncline{F}{E}
          \end{pspicture}
     \end{extern}
\end{center}
\begin{center}
\textit{Exemple 2}
\end{center}
 \end{vbloc}
          Dans l'\textit{exemple 2}, il y a deux sommets de degré impair (A:3 et E:3). Le graphe contient une chaîne eulérienne, par exemple (A; F; D; B; F; E; D; C; B; A; E) mais pas de cycle eulérien.

           \par
                                \begin{vbloc}
               \begin{center}
     \begin{extern}%width="280"
          \psset{unit=1.5cm}
          \begin{pspicture}(-1,0)(6,3)
               \rput(0,0){\circlenode{B}{B}}
               \rput(2,3.4){\circlenode{A}{A}}
               \rput(4,0){\circlenode{C}{C}}
               \rput(3,1.7){\circlenode{D}{D}}
               \rput(2,0){\circlenode{E}{E}}
                \rput(2,1.1){\circlenode{F}{F}}
               \ncline{A}{B} \ncline{B}{E}\ncline{D}{A}\ncline{D}{C}\ncline{E}{C}
				\ncline{A}{F} \ncline{B}{F}\ncline{F}{D}\ncline{F}{E}
          \end{pspicture}
     \end{extern}
\end{center}
\begin{center}
\textit{Exemple 3}
\end{center}
           Dans l'\textit{exemple 3}, il y a 4 sommets de degré impair (A:3, B:3, D:3 et E:3). Le graphe ne contient pas de chaîne eulérienne.
 \end{vbloc}      
           \par
                      \begin{vbloc}
               \begin{center}
     \begin{extern}%width="350"
          \psset{unit=1.5cm}
          \begin{pspicture}(-1,-2)(7,5)
               \rput(3,4){\circlenode{A}{A}}
               \rput(0,1){\circlenode{B}{B}}
               \rput(6,1){\circlenode{C}{C}}
               \rput(3,-0){\circlenode{D}{D}}
               \rput(0,3){\circlenode{E}{E}}
                \rput(6,3){\circlenode{F}{F}}
                \rput(2,3){\circlenode{G}{G}}
               \rput(4,3){\circlenode{H}{H}}
               \rput(5,2){\circlenode{I}{I}}
                 \rput(4,1){\circlenode{J}{J}}
                   \rput(2,1){\circlenode{K}{K}}
                \rput(1,2){\circlenode{L}{L}}
             \ncline{D}{K} \ncline{K}{J}\ncline{J}{D}
             \ncline{B}{K}\ncline{K}{L}\ncline{L}{B}
           \ncline{E}{G} \ncline{G}{L}\ncline{L}{E}
             \ncline{A}{G}\ncline{G}{H}\ncline{H}{A}
           \ncline{H}{F} \ncline{F}{I}\ncline{I}{H}
             \ncline{I}{C}\ncline{C}{J}\ncline{J}{I}
          \end{pspicture}
     \end{extern}
\end{center}
\begin{center}
\textit{Exemple 4}
\end{center}
 \end{vbloc}         
          Dans l'\textit{exemple 4}, tous les sommets sont de degré pair . Le graphe contient un cycle eulérien, par exemple: (G; A; H; F; I; C; J; D; K; B; L; E; G; H; I; J; K; L; G).

}
\begin{h2}3. Coloration d'un graphe\end{h2}
\cadre{bleu}{Définition}{%
     \textbf{Colorier un graphe} c'est associer à tout sommet une couleur telle que deux sommets adjacents n'aient pas la même couleur.
     \par
     Le plus petit nombre de couleurs nécessaire pour colorier un graphe s'appelle le \textbf{nombre chromatique} du graphe.
}
\bloc{orange}{Exemple}{%
                     \begin{vbloc}
               \begin{center}
     \begin{extern}%width="280"
          \psset{unit=1.5cm}
          \begin{pspicture}(-1,0)(6,3)
               \rput(0,0){\circlenode[fillstyle=solid,fillcolor=blue]{A}{}}
               \rput(0,2){\circlenode[fillstyle=solid,fillcolor=green]{B}{}}
               \rput(2,3){\circlenode[fillstyle=solid,fillcolor=red]{C}{}}
               \rput(4,2){\circlenode[fillstyle=solid,fillcolor=blue]{D}{}}
               \rput(4,0){\circlenode[fillstyle=solid,fillcolor=green]{E}{}}
                \rput(2,1){\circlenode[fillstyle=solid,fillcolor=red]{F}{}}
               \ncline{A}{B} \ncline{B}{C}\ncline{C}{D}\ncline{D}{E}\ncline{E}{A}\ncline{B}{D}
				\ncline{A}{F} \ncline{B}{F}\ncline{F}{D}\ncline{F}{E}
          \end{pspicture}
     \end{extern}
\end{center}
\begin{center}
\textit{Exemple 2}
\end{center}
 \end{vbloc}
     Le graphe ci-dessus a été colorié a l'aide de 3 couleurs différentes. Il n'est pas possible de le colorier avec seulement 2 couleurs. Le nombre chromatique du graphe est donc 3.
}
\cadre{rouge}{Théorème}{%
     Le nombre chromatique d'un graphe est inférieur ou égal à $d_{max}+1$ où $d_{max}$ est le plus grand degré des sommets.
}
\bloc{orange}{Exemple}{%
     Dans l'exemple précédent le plus grand degré est 4. Le nombre chromatique du graphe est donc inférieur ou égal à 5 (On a vu que c'était 3).
}
\begin{h2}4. Algorithme de Dijkstra\end{h2}
L'algorithme de Dijkstra (\textit{prononcer approximativement \og Dextra \fg{}}) permet de trouver \textbf{le plus court chemin entre deux sommets d'un graphe} (orienté ou non orienté).
\par
Le fonctionnement de l'algorithme de Dijkstra est généralement présenté sous forme d'un tableau dans lequel chaque ligne représente une étape.
\par
La construction d'un tel tableau est détaillée dans la fiche méthode~:  \mcLien{/methode/algorithme-de-dijkstra-etape-par-etape/}{Algorithme de Dijkstra - Étape par étape}.

\end{document}