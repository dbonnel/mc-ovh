\documentclass[a4paper]{article}

%================================================================================================================================
%
% Packages
%
%================================================================================================================================

\usepackage[T1]{fontenc} 	% pour caractères accentués
\usepackage[utf8]{inputenc}  % encodage utf8
\usepackage[french]{babel}	% langue : français
\usepackage{fourier}			% caractères plus lisibles
\usepackage[dvipsnames]{xcolor} % couleurs
\usepackage{fancyhdr}		% réglage header footer
\usepackage{needspace}		% empêcher sauts de page mal placés
\usepackage{graphicx}		% pour inclure des graphiques
\usepackage{enumitem,cprotect}		% personnalise les listes d'items (nécessaire pour ol, al ...)
\usepackage{hyperref}		% Liens hypertexte
\usepackage{pstricks,pst-all,pst-node,pstricks-add,pst-math,pst-plot,pst-tree,pst-eucl} % pstricks
\usepackage[a4paper,includeheadfoot,top=2cm,left=3cm, bottom=2cm,right=3cm]{geometry} % marges etc.
\usepackage{comment}			% commentaires multilignes
\usepackage{amsmath,environ} % maths (matrices, etc.)
\usepackage{amssymb,makeidx}
\usepackage{bm}				% bold maths
\usepackage{tabularx}		% tableaux
\usepackage{colortbl}		% tableaux en couleur
\usepackage{fontawesome}		% Fontawesome
\usepackage{environ}			% environment with command
\usepackage{fp}				% calculs pour ps-tricks
\usepackage{multido}			% pour ps tricks
\usepackage[np]{numprint}	% formattage nombre
\usepackage{tikz,tkz-tab} 			% package principal TikZ
\usepackage{pgfplots}   % axes
\usepackage{mathrsfs}    % cursives
\usepackage{calc}			% calcul taille boites
\usepackage[scaled=0.875]{helvet} % font sans serif
\usepackage{svg} % svg
\usepackage{scrextend} % local margin
\usepackage{scratch} %scratch
\usepackage{multicol} % colonnes
%\usepackage{infix-RPN,pst-func} % formule en notation polanaise inversée
\usepackage{listings}

%================================================================================================================================
%
% Réglages de base
%
%================================================================================================================================

\lstset{
language=Python,   % R code
literate=
{á}{{\'a}}1
{à}{{\`a}}1
{ã}{{\~a}}1
{é}{{\'e}}1
{è}{{\`e}}1
{ê}{{\^e}}1
{í}{{\'i}}1
{ó}{{\'o}}1
{õ}{{\~o}}1
{ú}{{\'u}}1
{ü}{{\"u}}1
{ç}{{\c{c}}}1
{~}{{ }}1
}


\definecolor{codegreen}{rgb}{0,0.6,0}
\definecolor{codegray}{rgb}{0.5,0.5,0.5}
\definecolor{codepurple}{rgb}{0.58,0,0.82}
\definecolor{backcolour}{rgb}{0.95,0.95,0.92}

\lstdefinestyle{mystyle}{
    backgroundcolor=\color{backcolour},   
    commentstyle=\color{codegreen},
    keywordstyle=\color{magenta},
    numberstyle=\tiny\color{codegray},
    stringstyle=\color{codepurple},
    basicstyle=\ttfamily\footnotesize,
    breakatwhitespace=false,         
    breaklines=true,                 
    captionpos=b,                    
    keepspaces=true,                 
    numbers=left,                    
xleftmargin=2em,
framexleftmargin=2em,            
    showspaces=false,                
    showstringspaces=false,
    showtabs=false,                  
    tabsize=2,
    upquote=true
}

\lstset{style=mystyle}


\lstset{style=mystyle}
\newcommand{\imgdir}{C:/laragon/www/newmc/assets/imgsvg/}
\newcommand{\imgsvgdir}{C:/laragon/www/newmc/assets/imgsvg/}

\definecolor{mcgris}{RGB}{220, 220, 220}% ancien~; pour compatibilité
\definecolor{mcbleu}{RGB}{52, 152, 219}
\definecolor{mcvert}{RGB}{125, 194, 70}
\definecolor{mcmauve}{RGB}{154, 0, 215}
\definecolor{mcorange}{RGB}{255, 96, 0}
\definecolor{mcturquoise}{RGB}{0, 153, 153}
\definecolor{mcrouge}{RGB}{255, 0, 0}
\definecolor{mclightvert}{RGB}{205, 234, 190}

\definecolor{gris}{RGB}{220, 220, 220}
\definecolor{bleu}{RGB}{52, 152, 219}
\definecolor{vert}{RGB}{125, 194, 70}
\definecolor{mauve}{RGB}{154, 0, 215}
\definecolor{orange}{RGB}{255, 96, 0}
\definecolor{turquoise}{RGB}{0, 153, 153}
\definecolor{rouge}{RGB}{255, 0, 0}
\definecolor{lightvert}{RGB}{205, 234, 190}
\setitemize[0]{label=\color{lightvert}  $\bullet$}

\pagestyle{fancy}
\renewcommand{\headrulewidth}{0.2pt}
\fancyhead[L]{maths-cours.fr}
\fancyhead[R]{\thepage}
\renewcommand{\footrulewidth}{0.2pt}
\fancyfoot[C]{}

\newcolumntype{C}{>{\centering\arraybackslash}X}
\newcolumntype{s}{>{\hsize=.35\hsize\arraybackslash}X}

\setlength{\parindent}{0pt}		 
\setlength{\parskip}{3mm}
\setlength{\headheight}{1cm}

\def\ebook{ebook}
\def\book{book}
\def\web{web}
\def\type{web}

\newcommand{\vect}[1]{\overrightarrow{\,\mathstrut#1\,}}

\def\Oij{$\left(\text{O}~;~\vect{\imath},~\vect{\jmath}\right)$}
\def\Oijk{$\left(\text{O}~;~\vect{\imath},~\vect{\jmath},~\vect{k}\right)$}
\def\Ouv{$\left(\text{O}~;~\vect{u},~\vect{v}\right)$}

\hypersetup{breaklinks=true, colorlinks = true, linkcolor = OliveGreen, urlcolor = OliveGreen, citecolor = OliveGreen, pdfauthor={Didier BONNEL - https://www.maths-cours.fr} } % supprime les bordures autour des liens

\renewcommand{\arg}[0]{\text{arg}}

\everymath{\displaystyle}

%================================================================================================================================
%
% Macros - Commandes
%
%================================================================================================================================

\newcommand\meta[2]{    			% Utilisé pour créer le post HTML.
	\def\titre{titre}
	\def\url{url}
	\def\arg{#1}
	\ifx\titre\arg
		\newcommand\maintitle{#2}
		\fancyhead[L]{#2}
		{\Large\sffamily \MakeUppercase{#2}}
		\vspace{1mm}\textcolor{mcvert}{\hrule}
	\fi 
	\ifx\url\arg
		\fancyfoot[L]{\href{https://www.maths-cours.fr#2}{\black \footnotesize{https://www.maths-cours.fr#2}}}
	\fi 
}


\newcommand\TitreC[1]{    		% Titre centré
     \needspace{3\baselineskip}
     \begin{center}\textbf{#1}\end{center}
}

\newcommand\newpar{    		% paragraphe
     \par
}

\newcommand\nosp {    		% commande vide (pas d'espace)
}
\newcommand{\id}[1]{} %ignore

\newcommand\boite[2]{				% Boite simple sans titre
	\vspace{5mm}
	\setlength{\fboxrule}{0.2mm}
	\setlength{\fboxsep}{5mm}	
	\fcolorbox{#1}{#1!3}{\makebox[\linewidth-2\fboxrule-2\fboxsep]{
  		\begin{minipage}[t]{\linewidth-2\fboxrule-4\fboxsep}\setlength{\parskip}{3mm}
  			 #2
  		\end{minipage}
	}}
	\vspace{5mm}
}

\newcommand\CBox[4]{				% Boites
	\vspace{5mm}
	\setlength{\fboxrule}{0.2mm}
	\setlength{\fboxsep}{5mm}
	
	\fcolorbox{#1}{#1!3}{\makebox[\linewidth-2\fboxrule-2\fboxsep]{
		\begin{minipage}[t]{1cm}\setlength{\parskip}{3mm}
	  		\textcolor{#1}{\LARGE{#2}}    
 	 	\end{minipage}  
  		\begin{minipage}[t]{\linewidth-2\fboxrule-4\fboxsep}\setlength{\parskip}{3mm}
			\raisebox{1.2mm}{\normalsize\sffamily{\textcolor{#1}{#3}}}						
  			 #4
  		\end{minipage}
	}}
	\vspace{5mm}
}

\newcommand\cadre[3]{				% Boites convertible html
	\par
	\vspace{2mm}
	\setlength{\fboxrule}{0.1mm}
	\setlength{\fboxsep}{5mm}
	\fcolorbox{#1}{white}{\makebox[\linewidth-2\fboxrule-2\fboxsep]{
  		\begin{minipage}[t]{\linewidth-2\fboxrule-4\fboxsep}\setlength{\parskip}{3mm}
			\raisebox{-2.5mm}{\sffamily \small{\textcolor{#1}{\MakeUppercase{#2}}}}		
			\par		
  			 #3
 	 		\end{minipage}
	}}
		\vspace{2mm}
	\par
}

\newcommand\bloc[3]{				% Boites convertible html sans bordure
     \needspace{2\baselineskip}
     {\sffamily \small{\textcolor{#1}{\MakeUppercase{#2}}}}    
		\par		
  			 #3
		\par
}

\newcommand\CHelp[1]{
     \CBox{Plum}{\faInfoCircle}{À RETENIR}{#1}
}

\newcommand\CUp[1]{
     \CBox{NavyBlue}{\faThumbsOUp}{EN PRATIQUE}{#1}
}

\newcommand\CInfo[1]{
     \CBox{Sepia}{\faArrowCircleRight}{REMARQUE}{#1}
}

\newcommand\CRedac[1]{
     \CBox{PineGreen}{\faEdit}{BIEN R\'EDIGER}{#1}
}

\newcommand\CError[1]{
     \CBox{Red}{\faExclamationTriangle}{ATTENTION}{#1}
}

\newcommand\TitreExo[2]{
\needspace{4\baselineskip}
 {\sffamily\large EXERCICE #1\ (\emph{#2 points})}
\vspace{5mm}
}

\newcommand\img[2]{
          \includegraphics[width=#2\paperwidth]{\imgdir#1}
}

\newcommand\imgsvg[2]{
       \begin{center}   \includegraphics[width=#2\paperwidth]{\imgsvgdir#1} \end{center}
}


\newcommand\Lien[2]{
     \href{#1}{#2 \tiny \faExternalLink}
}
\newcommand\mcLien[2]{
     \href{https~://www.maths-cours.fr/#1}{#2 \tiny \faExternalLink}
}

\newcommand{\euro}{\eurologo{}}

%================================================================================================================================
%
% Macros - Environement
%
%================================================================================================================================

\newenvironment{tex}{ %
}
{%
}

\newenvironment{indente}{ %
	\setlength\parindent{10mm}
}

{
	\setlength\parindent{0mm}
}

\newenvironment{corrige}{%
     \needspace{3\baselineskip}
     \medskip
     \textbf{\textsc{Corrigé}}
     \medskip
}
{
}

\newenvironment{extern}{%
     \begin{center}
     }
     {
     \end{center}
}

\NewEnviron{code}{%
	\par
     \boite{gray}{\texttt{%
     \BODY
     }}
     \par
}

\newenvironment{vbloc}{% boite sans cadre empeche saut de page
     \begin{minipage}[t]{\linewidth}
     }
     {
     \end{minipage}
}
\NewEnviron{h2}{%
    \needspace{3\baselineskip}
    \vspace{0.6cm}
	\noindent \MakeUppercase{\sffamily \large \BODY}
	\vspace{1mm}\textcolor{mcgris}{\hrule}\vspace{0.4cm}
	\par
}{}

\NewEnviron{h3}{%
    \needspace{3\baselineskip}
	\vspace{5mm}
	\textsc{\BODY}
	\par
}

\NewEnviron{margeneg}{ %
\begin{addmargin}[-1cm]{0cm}
\BODY
\end{addmargin}
}

\NewEnviron{html}{%
}

\begin{document}
\cadre{rouge}{Méthode}{ % id="m010"
     \begin{itemize}
          \item %
          \textbf{1ère étape~:} (facultative mais permet de simplifier les calculs)~:\\
          Rechercher l'équation dans laquelle il sera facile d'exprimer $y$ en fonction de $x$, ou $x$ en fonction de $y$.\\
          \textit{On supposera, dans l'explication qui suit, que l'on a choisi d'exprimer $y$ en fonction de $x$ dans la première équation.}
          \item %
          \textbf{2ème étape~:} \\
          Dans la première équation, exprimer $y$ en fonction de $x$.\\
          Ne pas modifier la seconde équation.
          \item %
          \textbf{3ème étape~:} \\
          Remplacer $y$ par l'expression trouvée précédemment dans la seconde équation.\\
          La seconde équation n'a alors plus qu'une seule inconnue $x.$
          \item %
          \textbf{4ème étape~:} \\
          Résoudre la seconde équation pour trouver $x.$
          \item %
          \textbf{5ème étape~:} \\
          Calculer $y$ en remplaçant $x$, dans la première équation,  par la valeur trouvée à l'étape précédente.
          \item %
          \textbf{6ème étape~:} \\
          Conclure en précisant la ou les couple(s) de solution(s).
     \end{itemize}
} % fin méth
\bloc{cyan}{Remarques}{ % id="r020"
     \begin{itemize}
          \item %
          Pour présenter les calculs, il est préférable de recopier à chaque étape un système équivalent au système de départ en réécrivant les deux équations, y compris celle que l'on n'a pas modifiée.
          \item %
          Un système admet souvent un unique couple solution mais peut aussi n'avoir aucune solution ou admettre une infinité de solutions (voir exemple 3 et 4).
     \end{itemize}
} % fin rem
\bloc{orange}{Exemple 1}{ % id="e030"
     Résoudre le système :
     \par
     $(S_1)~~\begin{cases} 3x+y=2 \\  5x+2y=3\end{cases}$
     \par
     \textbf{Solution~:}
     \begin{itemize}
          \item %
          \textbf{1ère étape~: Recherche de la méthode la plus rapide.}
          \par
          On remarque qu'ici,  il sera particulièrement simple d'exprimer $y$ en fonction de $x$ dans la première équation.
          \item %
          \textbf{2ème étape~: Expression de  $y$ en fonction de $x$.}
          \par
          Il suffit de \og faire passer \fg{} $3x$ dans l'autre membre dans la première équation~;
          \\on recopie la seconde équation sans y toucher.
          \par
          $(S_1)~\Leftrightarrow~\begin{cases} y=2-3x \\  5x+2y=3\end{cases}$
          \item %
          \textbf{3ème étape~: Remplacement de $y$.}
          \par
          On remplace $y$ par $(2-3x)$ dans la seconde équation (ne pas oublier la parenthèse~!).\\
          On ne touche pas à la première équation.
          \par
          $(S_1)~\Leftrightarrow~\begin{cases} y=2-3x \\  5x+2(2-3x)=3\end{cases}$
          \item %
          \textbf{4ème étape~: Calcul de $x.$}
          \par
          On résout la seconde équation (en recopiant à chaque fois la première à l'identique).
          \par
          $(S_1)~\Leftrightarrow~\begin{cases} y=2-3x \\  5x+4-6x=3\end{cases}$\\
          $(S_1)~\Leftrightarrow~\begin{cases} y=2-3x \\  -x=3-4\end{cases}$\\
          $(S_1)~\Leftrightarrow~\begin{cases} y=2-3x \\  x=1\end{cases}$\\
          \item %
          \textbf{5ème étape~: Calcul de $y.$}
          \par
          On  remplace $x$ par $1$ dans la première équation~:
          \par
          $(S_1)~\Leftrightarrow~\begin{cases} y=2-3\times 1\\  x=1\end{cases}$\\
          $(S_1)~\Leftrightarrow~\begin{cases} y=-1\\  x=1\end{cases}$\\
          \item %
          \textbf{6ème étape~: Conclusion. }
          \par
          Le couple $(1~;~-1)$ est l'unique couple solution du système $(S_1)$.
     \end{itemize}
} % fin ex
\bloc{orange}{Exemple 2}{ % id="e040"
     Résoudre le système :
     \par
     $(S_1)~~\begin{cases} 5x-2y=1 \\  x+3y=7\end{cases}$
     \par
     \textbf{Solution~:}
     \begin{itemize}
          \item %
          \textbf{1ère étape~: Recherche de la méthode la plus rapide.}
          \par
          On pourrait, comme dans l'exemple précédent, exprimer $y$ en fonction de $x$ dans la première équation. Toutefois, à cause du coefficient $-2$, cela entraînerait des calculs plus longs comportant des fractions (on trouverait $y=\dfrac{-1+5x}{2}$).
          \par
          Il est plus simple, ici, d'exprimer $x$ en fonction de $y$ dans la deuxième équation.
          \item %
          \textbf{2ème étape~: Expression de  $x$ en fonction de $y$.}
          \par
          $(S_2)~\Leftrightarrow~\begin{cases} 5x-2y=1 \\  x=7-3y\end{cases}$
          \item %
          \textbf{3ème étape~: Remplacement de $x$.}
          \par
          On remplace $x$ par $(7-3y)$ dans la première équation.
          \par
          $(S_2)~\Leftrightarrow~\begin{cases} 5(7-3y)-2y=1 \\  x=7-3y\end{cases}$
          \item %
          \textbf{4ème étape~: Calcul de $y.$}
          \par
          On résout la première équation.
          \par
          $(S_2)~\Leftrightarrow~\begin{cases} 35-15y-2y=1 \\  x=7-3y\end{cases}$\\
          $(S_2)~\Leftrightarrow~\begin{cases} -17y=-34 \\  x=7-3y\end{cases}$\\
          $(S_2)~\Leftrightarrow~\begin{cases} y=\dfrac{-34}{-17} \\  x=7-3y\end{cases}$\\
          $(S_2)~\Leftrightarrow~\begin{cases} y=2\\  x=7-3y\end{cases}$\\
          \item %
          \textbf{5ème étape~: Calcul de $x.$}
          \par
          On  remplace $y$ par $2$ dans la seconde équation~:
          \par
          $(S_2)~\Leftrightarrow~\begin{cases}  y=2\\  x=7-3 \times 2\end{cases}$\\
          $(S_2)~\Leftrightarrow~\begin{cases} y=2\\  x=1\end{cases}$\\
          \item %
          \textbf{6ème étape~: Conclusion. }
          \par
          Le couple $(1~;~2)$ est l'unique solution du système $(S_2)$.
     \end{itemize}
} % fin ex
\bloc{orange}{Exemple 3}{ % id="e050"
     Résoudre le système :
     \par
     $(S_3)~~\begin{cases} 6x-2y=3 \\  -3x+y=5\end{cases}$
     \par
     \textbf{Solution~:}
     \begin{itemize}
          \item %
          \textbf{1ère étape~: Recherche de la méthode la plus rapide.}
          \par
          Ici, il est facile d'exprimer $y$ en fonction de $x$ dans la seconde équation.
          \item %
          \textbf{2ème étape~: Expression de  $y$ en fonction de $x$.}
          \par
          $(S_3)~\Leftrightarrow~\begin{cases} 6x-2y=3 \\ y=5+3x\end{cases}$
          \item %
          \textbf{3ème étape~: Remplacement de $y$.}
          \par
          $(S_3)~\Leftrightarrow~\begin{cases}  6x-2(5+3x)=3 \\ y=5+3x\end{cases}$
          \item %
          \textbf{4ème étape~: Calcul de $x.$}
          \par
          $(S_3)~\Leftrightarrow~\begin{cases}  6x-10-6x=3 \\ y=5+3x\end{cases}$\\
          $(S_3)~\Leftrightarrow~\begin{cases}  -10=3 \\ y=5+3x\end{cases}$\\
          \par
          La première équation n'a pas de solution, donc le système n'en a pas non plus.
          \par
          On peut donc passer directement à la conclusion~:
          \item %
          \textbf{6ème étape~: Conclusion. }
          \par
          Le système $(S_3)$ n'admet aucune solution dans $\mathbb{R}.$
     \end{itemize}
} % fin ex
\bloc{orange}{Exemple 4}{ % id="e060"
     Résoudre le système :
     \par
     $(S_4)~~\begin{cases} 4x-2y=6 \\  -6x+3y=-9\end{cases}$
     \par
     \textbf{Solution~:}
     \begin{itemize}
          \item %
          \textbf{1ère étape~: Recherche de la méthode la plus rapide.}
          \par
          On choisit d'exprimer $y$ en fonction de $x$ dans la première équation.
          \item %
          \textbf{2ème étape~: Expression de  $y$ en fonction de $x$.}
          \par
          $(S_4)~\Leftrightarrow~\begin{cases}-2y=6 -4x\\  -6x+3y=-9\end{cases}$\\
          $(S_4)~\Leftrightarrow~\begin{cases}2y=-6 +4x\\  -6x+3y=-9\end{cases}$\\
          $(S_4)~\Leftrightarrow~\begin{cases}y=\dfrac{-6 +4x}{2}\\  -6x+3y=-9\end{cases}$\\
          $(S_4)~\Leftrightarrow~\begin{cases}y=\dfrac{-6}{2} +\dfrac{4x}{2}\\  -6x+3y=-9\end{cases}$\\
          $(S_4)~\Leftrightarrow~\begin{cases}y=-3+2x\\  -6x+3y=-9\end{cases}$\\
          \item %
          \textbf{3ème étape~: Remplacement de $y$.}
          \par
          $(S_4)~\Leftrightarrow~\begin{cases}  y=-3+2x\\  -6x+3(-3+2x)=-9\end{cases}$\\
          \item %
          \textbf{4ème étape~: Calcul de $x.$}
          \par
          $(S_4)~\Leftrightarrow~\begin{cases}  y=-3+2x\\  -6x-9+6x=-9\end{cases}$\\
          $(S_4)~\Leftrightarrow~\begin{cases}  y=-3+2x\\  -9=-9\end{cases}$\\
          \par
          La deuxième équation est toujours vérifiée. Il suffit donc qu'un couple soit solution de la première équation $y=-3+2x$ pour qu'il soit solution du système.
          \par
          Or, cette équation possède une infinité de solutions (par exemple $(0~;~-3)$, $(1~;~-1)$, etc.).
          \par
          On peut donc sauter la cinquième étape et passer à la conclusion.
          \item %
          \textbf{6ème étape~: Conclusion. }
          \par
          Le système $(S_4)$ admet une infinité de solutions dans $\mathbb{R}.$
     \end{itemize}
} % fin ex

\end{document}
