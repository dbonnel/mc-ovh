\documentclass[a4paper]{article}

%================================================================================================================================
%
% Packages
%
%================================================================================================================================

\usepackage[T1]{fontenc} 	% pour caractères accentués
\usepackage[utf8]{inputenc}  % encodage utf8
\usepackage[french]{babel}	% langue : français
\usepackage{fourier}			% caractères plus lisibles
\usepackage[dvipsnames]{xcolor} % couleurs
\usepackage{fancyhdr}		% réglage header footer
\usepackage{needspace}		% empêcher sauts de page mal placés
\usepackage{graphicx}		% pour inclure des graphiques
\usepackage{enumitem,cprotect}		% personnalise les listes d'items (nécessaire pour ol, al ...)
\usepackage{hyperref}		% Liens hypertexte
\usepackage{pstricks,pst-all,pst-node,pstricks-add,pst-math,pst-plot,pst-tree,pst-eucl} % pstricks
\usepackage[a4paper,includeheadfoot,top=2cm,left=3cm, bottom=2cm,right=3cm]{geometry} % marges etc.
\usepackage{comment}			% commentaires multilignes
\usepackage{amsmath,environ} % maths (matrices, etc.)
\usepackage{amssymb,makeidx}
\usepackage{bm}				% bold maths
\usepackage{tabularx}		% tableaux
\usepackage{colortbl}		% tableaux en couleur
\usepackage{fontawesome}		% Fontawesome
\usepackage{environ}			% environment with command
\usepackage{fp}				% calculs pour ps-tricks
\usepackage{multido}			% pour ps tricks
\usepackage[np]{numprint}	% formattage nombre
\usepackage{tikz,tkz-tab} 			% package principal TikZ
\usepackage{pgfplots}   % axes
\usepackage{mathrsfs}    % cursives
\usepackage{calc}			% calcul taille boites
\usepackage[scaled=0.875]{helvet} % font sans serif
\usepackage{svg} % svg
\usepackage{scrextend} % local margin
\usepackage{scratch} %scratch
\usepackage{multicol} % colonnes
%\usepackage{infix-RPN,pst-func} % formule en notation polanaise inversée
\usepackage{listings}

%================================================================================================================================
%
% Réglages de base
%
%================================================================================================================================

\lstset{
language=Python,   % R code
literate=
{á}{{\'a}}1
{à}{{\`a}}1
{ã}{{\~a}}1
{é}{{\'e}}1
{è}{{\`e}}1
{ê}{{\^e}}1
{í}{{\'i}}1
{ó}{{\'o}}1
{õ}{{\~o}}1
{ú}{{\'u}}1
{ü}{{\"u}}1
{ç}{{\c{c}}}1
{~}{{ }}1
}


\definecolor{codegreen}{rgb}{0,0.6,0}
\definecolor{codegray}{rgb}{0.5,0.5,0.5}
\definecolor{codepurple}{rgb}{0.58,0,0.82}
\definecolor{backcolour}{rgb}{0.95,0.95,0.92}

\lstdefinestyle{mystyle}{
    backgroundcolor=\color{backcolour},   
    commentstyle=\color{codegreen},
    keywordstyle=\color{magenta},
    numberstyle=\tiny\color{codegray},
    stringstyle=\color{codepurple},
    basicstyle=\ttfamily\footnotesize,
    breakatwhitespace=false,         
    breaklines=true,                 
    captionpos=b,                    
    keepspaces=true,                 
    numbers=left,                    
xleftmargin=2em,
framexleftmargin=2em,            
    showspaces=false,                
    showstringspaces=false,
    showtabs=false,                  
    tabsize=2,
    upquote=true
}

\lstset{style=mystyle}


\lstset{style=mystyle}
\newcommand{\imgdir}{C:/laragon/www/newmc/assets/imgsvg/}
\newcommand{\imgsvgdir}{C:/laragon/www/newmc/assets/imgsvg/}

\definecolor{mcgris}{RGB}{220, 220, 220}% ancien~; pour compatibilité
\definecolor{mcbleu}{RGB}{52, 152, 219}
\definecolor{mcvert}{RGB}{125, 194, 70}
\definecolor{mcmauve}{RGB}{154, 0, 215}
\definecolor{mcorange}{RGB}{255, 96, 0}
\definecolor{mcturquoise}{RGB}{0, 153, 153}
\definecolor{mcrouge}{RGB}{255, 0, 0}
\definecolor{mclightvert}{RGB}{205, 234, 190}

\definecolor{gris}{RGB}{220, 220, 220}
\definecolor{bleu}{RGB}{52, 152, 219}
\definecolor{vert}{RGB}{125, 194, 70}
\definecolor{mauve}{RGB}{154, 0, 215}
\definecolor{orange}{RGB}{255, 96, 0}
\definecolor{turquoise}{RGB}{0, 153, 153}
\definecolor{rouge}{RGB}{255, 0, 0}
\definecolor{lightvert}{RGB}{205, 234, 190}
\setitemize[0]{label=\color{lightvert}  $\bullet$}

\pagestyle{fancy}
\renewcommand{\headrulewidth}{0.2pt}
\fancyhead[L]{maths-cours.fr}
\fancyhead[R]{\thepage}
\renewcommand{\footrulewidth}{0.2pt}
\fancyfoot[C]{}

\newcolumntype{C}{>{\centering\arraybackslash}X}
\newcolumntype{s}{>{\hsize=.35\hsize\arraybackslash}X}

\setlength{\parindent}{0pt}		 
\setlength{\parskip}{3mm}
\setlength{\headheight}{1cm}

\def\ebook{ebook}
\def\book{book}
\def\web{web}
\def\type{web}

\newcommand{\vect}[1]{\overrightarrow{\,\mathstrut#1\,}}

\def\Oij{$\left(\text{O}~;~\vect{\imath},~\vect{\jmath}\right)$}
\def\Oijk{$\left(\text{O}~;~\vect{\imath},~\vect{\jmath},~\vect{k}\right)$}
\def\Ouv{$\left(\text{O}~;~\vect{u},~\vect{v}\right)$}

\hypersetup{breaklinks=true, colorlinks = true, linkcolor = OliveGreen, urlcolor = OliveGreen, citecolor = OliveGreen, pdfauthor={Didier BONNEL - https://www.maths-cours.fr} } % supprime les bordures autour des liens

\renewcommand{\arg}[0]{\text{arg}}

\everymath{\displaystyle}

%================================================================================================================================
%
% Macros - Commandes
%
%================================================================================================================================

\newcommand\meta[2]{    			% Utilisé pour créer le post HTML.
	\def\titre{titre}
	\def\url{url}
	\def\arg{#1}
	\ifx\titre\arg
		\newcommand\maintitle{#2}
		\fancyhead[L]{#2}
		{\Large\sffamily \MakeUppercase{#2}}
		\vspace{1mm}\textcolor{mcvert}{\hrule}
	\fi 
	\ifx\url\arg
		\fancyfoot[L]{\href{https://www.maths-cours.fr#2}{\black \footnotesize{https://www.maths-cours.fr#2}}}
	\fi 
}


\newcommand\TitreC[1]{    		% Titre centré
     \needspace{3\baselineskip}
     \begin{center}\textbf{#1}\end{center}
}

\newcommand\newpar{    		% paragraphe
     \par
}

\newcommand\nosp {    		% commande vide (pas d'espace)
}
\newcommand{\id}[1]{} %ignore

\newcommand\boite[2]{				% Boite simple sans titre
	\vspace{5mm}
	\setlength{\fboxrule}{0.2mm}
	\setlength{\fboxsep}{5mm}	
	\fcolorbox{#1}{#1!3}{\makebox[\linewidth-2\fboxrule-2\fboxsep]{
  		\begin{minipage}[t]{\linewidth-2\fboxrule-4\fboxsep}\setlength{\parskip}{3mm}
  			 #2
  		\end{minipage}
	}}
	\vspace{5mm}
}

\newcommand\CBox[4]{				% Boites
	\vspace{5mm}
	\setlength{\fboxrule}{0.2mm}
	\setlength{\fboxsep}{5mm}
	
	\fcolorbox{#1}{#1!3}{\makebox[\linewidth-2\fboxrule-2\fboxsep]{
		\begin{minipage}[t]{1cm}\setlength{\parskip}{3mm}
	  		\textcolor{#1}{\LARGE{#2}}    
 	 	\end{minipage}  
  		\begin{minipage}[t]{\linewidth-2\fboxrule-4\fboxsep}\setlength{\parskip}{3mm}
			\raisebox{1.2mm}{\normalsize\sffamily{\textcolor{#1}{#3}}}						
  			 #4
  		\end{minipage}
	}}
	\vspace{5mm}
}

\newcommand\cadre[3]{				% Boites convertible html
	\par
	\vspace{2mm}
	\setlength{\fboxrule}{0.1mm}
	\setlength{\fboxsep}{5mm}
	\fcolorbox{#1}{white}{\makebox[\linewidth-2\fboxrule-2\fboxsep]{
  		\begin{minipage}[t]{\linewidth-2\fboxrule-4\fboxsep}\setlength{\parskip}{3mm}
			\raisebox{-2.5mm}{\sffamily \small{\textcolor{#1}{\MakeUppercase{#2}}}}		
			\par		
  			 #3
 	 		\end{minipage}
	}}
		\vspace{2mm}
	\par
}

\newcommand\bloc[3]{				% Boites convertible html sans bordure
     \needspace{2\baselineskip}
     {\sffamily \small{\textcolor{#1}{\MakeUppercase{#2}}}}    
		\par		
  			 #3
		\par
}

\newcommand\CHelp[1]{
     \CBox{Plum}{\faInfoCircle}{À RETENIR}{#1}
}

\newcommand\CUp[1]{
     \CBox{NavyBlue}{\faThumbsOUp}{EN PRATIQUE}{#1}
}

\newcommand\CInfo[1]{
     \CBox{Sepia}{\faArrowCircleRight}{REMARQUE}{#1}
}

\newcommand\CRedac[1]{
     \CBox{PineGreen}{\faEdit}{BIEN R\'EDIGER}{#1}
}

\newcommand\CError[1]{
     \CBox{Red}{\faExclamationTriangle}{ATTENTION}{#1}
}

\newcommand\TitreExo[2]{
\needspace{4\baselineskip}
 {\sffamily\large EXERCICE #1\ (\emph{#2 points})}
\vspace{5mm}
}

\newcommand\img[2]{
          \includegraphics[width=#2\paperwidth]{\imgdir#1}
}

\newcommand\imgsvg[2]{
       \begin{center}   \includegraphics[width=#2\paperwidth]{\imgsvgdir#1} \end{center}
}


\newcommand\Lien[2]{
     \href{#1}{#2 \tiny \faExternalLink}
}
\newcommand\mcLien[2]{
     \href{https~://www.maths-cours.fr/#1}{#2 \tiny \faExternalLink}
}

\newcommand{\euro}{\eurologo{}}

%================================================================================================================================
%
% Macros - Environement
%
%================================================================================================================================

\newenvironment{tex}{ %
}
{%
}

\newenvironment{indente}{ %
	\setlength\parindent{10mm}
}

{
	\setlength\parindent{0mm}
}

\newenvironment{corrige}{%
     \needspace{3\baselineskip}
     \medskip
     \textbf{\textsc{Corrigé}}
     \medskip
}
{
}

\newenvironment{extern}{%
     \begin{center}
     }
     {
     \end{center}
}

\NewEnviron{code}{%
	\par
     \boite{gray}{\texttt{%
     \BODY
     }}
     \par
}

\newenvironment{vbloc}{% boite sans cadre empeche saut de page
     \begin{minipage}[t]{\linewidth}
     }
     {
     \end{minipage}
}
\NewEnviron{h2}{%
    \needspace{3\baselineskip}
    \vspace{0.6cm}
	\noindent \MakeUppercase{\sffamily \large \BODY}
	\vspace{1mm}\textcolor{mcgris}{\hrule}\vspace{0.4cm}
	\par
}{}

\NewEnviron{h3}{%
    \needspace{3\baselineskip}
	\vspace{5mm}
	\textsc{\BODY}
	\par
}

\NewEnviron{margeneg}{ %
\begin{addmargin}[-1cm]{0cm}
\BODY
\end{addmargin}
}

\NewEnviron{html}{%
}

\begin{document}
\begin{h2}Exercice 4 (5 points)\end{h2}
\textbf{Candidats n'ayant pas suivi l'enseignement de spécialité}
\medskip
\textbf{Les deux graphiques donnés en annexe seront à compléter et à rendre avec la copie}
\medskip
Un scooter radio commandé se déplace en ligne droite à la vitesse constante de 1 m.s$^{-1}$. Il est poursuivi
par un chien qui se déplace à la même vitesse.
\par
On représente la situation vue de dessus dans un repère orthonormé du plan d'unité 1 mètre. L'origine de ce repère est la position initiale du chien. Le scooter est représenté par un point appartenant à la droite d'équation $x = 5$. Il se déplace sur cette droite dans le sens des ordonnées croissantes.
\par
\smallskip
\par
Dans la suite de l'exercice, on étudie deux modélisations différentes de la trajectoire du chien.
\bigskip
\begin{center}\begin{h3}Partie A - Modélisation à l'aide d'une suite \end{h3}\end{center}
\medskip
La situation est représentée par le graphique n° 1 donné en annexe.
\par
À l'instant initial, le scooter est représenté par le point $S_0$. Le chien qui le poursuit est représenté
par le point $M_0$. On considère qu'à chaque seconde, le chien s'oriente instantanément en direction
du scooter et se déplace en ligne droite sur une distance de 1 mètre.
\par
Ainsi, à l'instant initial, le chien s'oriente en direction du point $S_0$, et une seconde plus tard il se
trouve un mètre plus loin au point $M_1$. À cet instant, le scooter est au point $S_1$. Le chien s'oriente
en direction de $S_1$ et se déplace en ligne droite en parcourant 1 mètre, et ainsi de suite.
\par
On modélise alors les trajectoires du chien et du scooter par deux suites de points notées $\left(M_n\right)$ et $\left(S_n\right)$.
\par
Au bout de $n$ secondes, les coordonnées du point $S_n$ sont $(5~:~n)$. On note $\left(x_n~:~y_n\right)$ les coordonnées du point $M_n$.
\medskip
\begin{enumerate}
     \item Construire sur le graphique n° 1 donné en annexe les points $M_2$ et $M_3$.
     \item  On note $d_n$ la distance entre le chien et le scooter $n$ secondes après le début de la poursuite.
     \par
     On a donc $d_n = M_nS_n$.
     \par
     Calculer $d_0$ et $d_1$.
     \item  Justifier que le point $M_2$ a pour coordonnées $\left(1 + \dfrac{4}{\sqrt{17}}~:~\dfrac{1}{\sqrt{17}}\right)$.
     \item  On admet que, pour tout entier naturel $n$~:
     \par
     \[\left\{\begin{array}{l c l}
               x_{n+1}& = &x_n + \dfrac{5 - x_n}{d_n}\\
               y_{n+1}&=&y_n + \dfrac{n - y_n}{d_n}
     \end{array}\right.\]
     \begin{enumerate}[label=\alph*.]
          \item Le tableau ci-dessous, obtenu à l'aide d'un tableur, donne les coordonnées des points $M_n$
          et $S_n$ ainsi que la distance $d_n$ en fonction de $n$. Quelles formules doit-on écrire dans les
          cellules C5 et F5 et recopier vers le bas pour remplir les colonnes C et F~?
          \begin{center}
               \begin{extern}%width="700" class="mw500"
                    \begin{tabularx}{\linewidth}{|>{\columncolor[gray]{0.9}}c|*{6}{>{\centering \arraybackslash}X|}}\hline
                         \rowcolor[gray]{0.9}&A &B &C &D &E &F\\ \hline
                         1 &$n$& \multicolumn{2}{|c|}{$M_n$} & \multicolumn{2}{|c|}{$S_n$} &$d_n$\\ \hline
                         2 &&$x_n$& $y_n$& 5 &n&\\ \hline
                         3 &0& 0& 0& 5 &0& 5\\ \hline
                         4 &1 &1 &0 &5 &1 &\np{4,12310563}\\ \hline
                         5 &2 &\np{1,9701425} &\np{0,24253563} &5 &2 &\np{3,50267291}\\ \hline
                         6 &3 &\np{2,83515547} &\np{0,74428512} &5 &3 &\np{3,12646789}\\ \hline
                         7 &4 &\np{3,52758047} &\np{1,46577498} &5 &4 &\np{2,93092404}\\ \hline
                         \ldots&\ldots&\ldots&\ldots&\ldots&\ldots&\ldots\\ \hline
                         28 &24 &\np{4,99979751} &\np{21,2268342} &5 &24 &\np{2,7731658}\\ \hline
                         29 &25 &\np{4,99987053} &\np{22,2268342} &5 &25 &\np{2,7731658}\\ \hline
                    \end{tabularx}
               \end{extern}
          \end{center}
          \medskip
          \item On admet que la suite $\left(d_n\right)$ est strictement décroissante.
          \par
          Justifier que cette suite est convergente et conjecturer sa limite à l'aide du tableau.
     \end{enumerate}
\end{enumerate}
\bigskip
\begin{center}\begin{h3}Partie B - Modélisation à l'aide d'une fonction \end{h3}\end{center}
\par
\medskip
\par
On modélise maintenant la trajectoire du chien à l'aide de la courbe $\mathscr{F}$ de la fonction $f$ définie
pour tout réel $x$ de l'intervalle [0~:~5[ par~:
\par
\[f(x) = -2,5\ln (1 - 0, 2x) - 0,5x + 0,05x^2.\]
\par
\medskip
\par
Cela signifie que le chien se déplace sur la courbe $\mathscr{F}$ de la fonction $f$.
\par
\medskip
\begin{enumerate}
     \item Lorsque le chien se trouve au point $M$ de coordonnées $(x~:~f(x))$ de la courbe $\mathscr{F}$, où $x$  appartient à l'intervalle [0~:~5[, le scooter se trouve au point $S$, d'ordonnée notée $y_S$. Ainsi le point $S$
     a pour coordonnées $\left(5~:~y_S\right)$. La tangente à la courbe $\mathscr{F}$ au point $M$ passe par le point $S$. Cela traduit le fait que le chien s'oriente toujours en direction du scooter. On note $d(x)$ la distance $MS$ entre le chien et le scooter lorsque $M$ a pour abscisse $x$.
     \begin{enumerate}[label=\alph*.]
          \item Sur le graphique n° 2 donné en annexe, construire, sans calcul, le point $S$ donnant la position du scooter lorsque le chien se trouve au point d'abscisse 3 de la courbe $\mathscr{F}$ et lire les
          coordonnées du point $S$.
          \item On note $f'$ la fonction dérivée de la fonction $f$ sur l'intervalle [0~:~5[ et on admet que, pour tout réel $x$ de l'intervalle [0~:~5[~:
          \par
          \[f'(x) = \dfrac{x(1  - 0,1x)}{5 - x}.\]
          \par
          Déterminer par le calcul une valeur approchée au centième de l'ordonnée du point $S$ lorsque
          le chien se trouve au point d'abscisse 3 de la courbe $\mathscr{F}$.
     \end{enumerate}
     \item  On admet que $d(x) = 0,1x^2 - x + 5$ pour tout réel $x$ de l'intervalle [0~:~5[.
     \par
     Justifier qu'au cours du temps la distance $MS$ se rapproche d'une valeur limite que l'on déterminera.
\end{enumerate}
\newpage
\begin{center}
     \begin{h3}Annexe\end{h3}
     \par
     \textbf{À rendre avec la copie}
     \bigskip
     \textbf{Partie A - question 1}
     \par
     Graphique n° 1
     \par          \begin{extern}
          \psset{unit=1.4cm}
          \begin{pspicture*}(-0.2,-0.2)(5.5,4.5)
               \psgrid[gridlabels=0pt,subgriddiv=1,gridwidth=0.3pt](0,0)(6,5)
               \psaxes[linewidth=0.75pt,labelFontSize=\scriptstyle]{->}(0,0)(0,0)(5.5,4.5)
               \psline[linecolor=blue](5,0)(5,6)
               \uput[dr](0,0){$M_0$}\uput[dr](1,0){$M_1$}\uput[ur](5,0){$S_0$}
               \uput[ur](5,1){$S_1$}\uput[ur](5,2){$S_2$}\uput[ur](5,3){$S_3$}
          \end{pspicture*}
     \end{extern}
     \bigskip
     \textbf{Partie B - question 1 }
     \par
     Graphique n° 2
     \par
     \begin{extern}
          \psset{unit=1.4cm}
          \begin{pspicture*}(-0.5,-0.5)(5.5,5.5)
               \psgrid[gridlabels=0pt,subgriddiv=1,gridwidth=0.3pt](0,0)(6,6)
               \psaxes[linewidth=0.75pt,labelFontSize=\scriptstyle]{->}(0,0)(0,0)(5.5,5.5)
               \psline[linecolor=blue](5,0)(5,6)
               \psplot[plotpoints=3000,linewidth=0.75pt,linecolor=red]{0}{4.8}{0.05 x dup mul mul 0.5 x mul sub 1 0.2 x mul sub ln 2.5 mul sub}
               \psdots[dotsize=3pt,dotstyle=+](3,1.24)\uput[ul](3,1.24){$M$}\uput[l](4.5,4.5){\red $\mathcal{F}$}
          \end{pspicture*}
     \end{extern}
\end{center}

\end{document}