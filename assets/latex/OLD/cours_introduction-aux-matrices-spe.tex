\documentclass[a4paper]{article}

%================================================================================================================================
%
% Packages
%
%================================================================================================================================

\usepackage[T1]{fontenc} 	% pour caractères accentués
\usepackage[utf8]{inputenc}  % encodage utf8
\usepackage[french]{babel}	% langue : français
\usepackage{fourier}			% caractères plus lisibles
\usepackage[dvipsnames]{xcolor} % couleurs
\usepackage{fancyhdr}		% réglage header footer
\usepackage{needspace}		% empêcher sauts de page mal placés
\usepackage{graphicx}		% pour inclure des graphiques
\usepackage{enumitem,cprotect}		% personnalise les listes d'items (nécessaire pour ol, al ...)
\usepackage{hyperref}		% Liens hypertexte
\usepackage{pstricks,pst-all,pst-node,pstricks-add,pst-math,pst-plot,pst-tree,pst-eucl} % pstricks
\usepackage[a4paper,includeheadfoot,top=2cm,left=3cm, bottom=2cm,right=3cm]{geometry} % marges etc.
\usepackage{comment}			% commentaires multilignes
\usepackage{amsmath,environ} % maths (matrices, etc.)
\usepackage{amssymb,makeidx}
\usepackage{bm}				% bold maths
\usepackage{tabularx}		% tableaux
\usepackage{colortbl}		% tableaux en couleur
\usepackage{fontawesome}		% Fontawesome
\usepackage{environ}			% environment with command
\usepackage{fp}				% calculs pour ps-tricks
\usepackage{multido}			% pour ps tricks
\usepackage[np]{numprint}	% formattage nombre
\usepackage{tikz,tkz-tab} 			% package principal TikZ
\usepackage{pgfplots}   % axes
\usepackage{mathrsfs}    % cursives
\usepackage{calc}			% calcul taille boites
\usepackage[scaled=0.875]{helvet} % font sans serif
\usepackage{svg} % svg
\usepackage{scrextend} % local margin
\usepackage{scratch} %scratch
\usepackage{multicol} % colonnes
%\usepackage{infix-RPN,pst-func} % formule en notation polanaise inversée
\usepackage{listings}

%================================================================================================================================
%
% Réglages de base
%
%================================================================================================================================

\lstset{
language=Python,   % R code
literate=
{á}{{\'a}}1
{à}{{\`a}}1
{ã}{{\~a}}1
{é}{{\'e}}1
{è}{{\`e}}1
{ê}{{\^e}}1
{í}{{\'i}}1
{ó}{{\'o}}1
{õ}{{\~o}}1
{ú}{{\'u}}1
{ü}{{\"u}}1
{ç}{{\c{c}}}1
{~}{{ }}1
}


\definecolor{codegreen}{rgb}{0,0.6,0}
\definecolor{codegray}{rgb}{0.5,0.5,0.5}
\definecolor{codepurple}{rgb}{0.58,0,0.82}
\definecolor{backcolour}{rgb}{0.95,0.95,0.92}

\lstdefinestyle{mystyle}{
    backgroundcolor=\color{backcolour},   
    commentstyle=\color{codegreen},
    keywordstyle=\color{magenta},
    numberstyle=\tiny\color{codegray},
    stringstyle=\color{codepurple},
    basicstyle=\ttfamily\footnotesize,
    breakatwhitespace=false,         
    breaklines=true,                 
    captionpos=b,                    
    keepspaces=true,                 
    numbers=left,                    
xleftmargin=2em,
framexleftmargin=2em,            
    showspaces=false,                
    showstringspaces=false,
    showtabs=false,                  
    tabsize=2,
    upquote=true
}

\lstset{style=mystyle}


\lstset{style=mystyle}
\newcommand{\imgdir}{C:/laragon/www/newmc/assets/imgsvg/}
\newcommand{\imgsvgdir}{C:/laragon/www/newmc/assets/imgsvg/}

\definecolor{mcgris}{RGB}{220, 220, 220}% ancien~; pour compatibilité
\definecolor{mcbleu}{RGB}{52, 152, 219}
\definecolor{mcvert}{RGB}{125, 194, 70}
\definecolor{mcmauve}{RGB}{154, 0, 215}
\definecolor{mcorange}{RGB}{255, 96, 0}
\definecolor{mcturquoise}{RGB}{0, 153, 153}
\definecolor{mcrouge}{RGB}{255, 0, 0}
\definecolor{mclightvert}{RGB}{205, 234, 190}

\definecolor{gris}{RGB}{220, 220, 220}
\definecolor{bleu}{RGB}{52, 152, 219}
\definecolor{vert}{RGB}{125, 194, 70}
\definecolor{mauve}{RGB}{154, 0, 215}
\definecolor{orange}{RGB}{255, 96, 0}
\definecolor{turquoise}{RGB}{0, 153, 153}
\definecolor{rouge}{RGB}{255, 0, 0}
\definecolor{lightvert}{RGB}{205, 234, 190}
\setitemize[0]{label=\color{lightvert}  $\bullet$}

\pagestyle{fancy}
\renewcommand{\headrulewidth}{0.2pt}
\fancyhead[L]{maths-cours.fr}
\fancyhead[R]{\thepage}
\renewcommand{\footrulewidth}{0.2pt}
\fancyfoot[C]{}

\newcolumntype{C}{>{\centering\arraybackslash}X}
\newcolumntype{s}{>{\hsize=.35\hsize\arraybackslash}X}

\setlength{\parindent}{0pt}		 
\setlength{\parskip}{3mm}
\setlength{\headheight}{1cm}

\def\ebook{ebook}
\def\book{book}
\def\web{web}
\def\type{web}

\newcommand{\vect}[1]{\overrightarrow{\,\mathstrut#1\,}}

\def\Oij{$\left(\text{O}~;~\vect{\imath},~\vect{\jmath}\right)$}
\def\Oijk{$\left(\text{O}~;~\vect{\imath},~\vect{\jmath},~\vect{k}\right)$}
\def\Ouv{$\left(\text{O}~;~\vect{u},~\vect{v}\right)$}

\hypersetup{breaklinks=true, colorlinks = true, linkcolor = OliveGreen, urlcolor = OliveGreen, citecolor = OliveGreen, pdfauthor={Didier BONNEL - https://www.maths-cours.fr} } % supprime les bordures autour des liens

\renewcommand{\arg}[0]{\text{arg}}

\everymath{\displaystyle}

%================================================================================================================================
%
% Macros - Commandes
%
%================================================================================================================================

\newcommand\meta[2]{    			% Utilisé pour créer le post HTML.
	\def\titre{titre}
	\def\url{url}
	\def\arg{#1}
	\ifx\titre\arg
		\newcommand\maintitle{#2}
		\fancyhead[L]{#2}
		{\Large\sffamily \MakeUppercase{#2}}
		\vspace{1mm}\textcolor{mcvert}{\hrule}
	\fi 
	\ifx\url\arg
		\fancyfoot[L]{\href{https://www.maths-cours.fr#2}{\black \footnotesize{https://www.maths-cours.fr#2}}}
	\fi 
}


\newcommand\TitreC[1]{    		% Titre centré
     \needspace{3\baselineskip}
     \begin{center}\textbf{#1}\end{center}
}

\newcommand\newpar{    		% paragraphe
     \par
}

\newcommand\nosp {    		% commande vide (pas d'espace)
}
\newcommand{\id}[1]{} %ignore

\newcommand\boite[2]{				% Boite simple sans titre
	\vspace{5mm}
	\setlength{\fboxrule}{0.2mm}
	\setlength{\fboxsep}{5mm}	
	\fcolorbox{#1}{#1!3}{\makebox[\linewidth-2\fboxrule-2\fboxsep]{
  		\begin{minipage}[t]{\linewidth-2\fboxrule-4\fboxsep}\setlength{\parskip}{3mm}
  			 #2
  		\end{minipage}
	}}
	\vspace{5mm}
}

\newcommand\CBox[4]{				% Boites
	\vspace{5mm}
	\setlength{\fboxrule}{0.2mm}
	\setlength{\fboxsep}{5mm}
	
	\fcolorbox{#1}{#1!3}{\makebox[\linewidth-2\fboxrule-2\fboxsep]{
		\begin{minipage}[t]{1cm}\setlength{\parskip}{3mm}
	  		\textcolor{#1}{\LARGE{#2}}    
 	 	\end{minipage}  
  		\begin{minipage}[t]{\linewidth-2\fboxrule-4\fboxsep}\setlength{\parskip}{3mm}
			\raisebox{1.2mm}{\normalsize\sffamily{\textcolor{#1}{#3}}}						
  			 #4
  		\end{minipage}
	}}
	\vspace{5mm}
}

\newcommand\cadre[3]{				% Boites convertible html
	\par
	\vspace{2mm}
	\setlength{\fboxrule}{0.1mm}
	\setlength{\fboxsep}{5mm}
	\fcolorbox{#1}{white}{\makebox[\linewidth-2\fboxrule-2\fboxsep]{
  		\begin{minipage}[t]{\linewidth-2\fboxrule-4\fboxsep}\setlength{\parskip}{3mm}
			\raisebox{-2.5mm}{\sffamily \small{\textcolor{#1}{\MakeUppercase{#2}}}}		
			\par		
  			 #3
 	 		\end{minipage}
	}}
		\vspace{2mm}
	\par
}

\newcommand\bloc[3]{				% Boites convertible html sans bordure
     \needspace{2\baselineskip}
     {\sffamily \small{\textcolor{#1}{\MakeUppercase{#2}}}}    
		\par		
  			 #3
		\par
}

\newcommand\CHelp[1]{
     \CBox{Plum}{\faInfoCircle}{À RETENIR}{#1}
}

\newcommand\CUp[1]{
     \CBox{NavyBlue}{\faThumbsOUp}{EN PRATIQUE}{#1}
}

\newcommand\CInfo[1]{
     \CBox{Sepia}{\faArrowCircleRight}{REMARQUE}{#1}
}

\newcommand\CRedac[1]{
     \CBox{PineGreen}{\faEdit}{BIEN R\'EDIGER}{#1}
}

\newcommand\CError[1]{
     \CBox{Red}{\faExclamationTriangle}{ATTENTION}{#1}
}

\newcommand\TitreExo[2]{
\needspace{4\baselineskip}
 {\sffamily\large EXERCICE #1\ (\emph{#2 points})}
\vspace{5mm}
}

\newcommand\img[2]{
          \includegraphics[width=#2\paperwidth]{\imgdir#1}
}

\newcommand\imgsvg[2]{
       \begin{center}   \includegraphics[width=#2\paperwidth]{\imgsvgdir#1} \end{center}
}


\newcommand\Lien[2]{
     \href{#1}{#2 \tiny \faExternalLink}
}
\newcommand\mcLien[2]{
     \href{https~://www.maths-cours.fr/#1}{#2 \tiny \faExternalLink}
}

\newcommand{\euro}{\eurologo{}}

%================================================================================================================================
%
% Macros - Environement
%
%================================================================================================================================

\newenvironment{tex}{ %
}
{%
}

\newenvironment{indente}{ %
	\setlength\parindent{10mm}
}

{
	\setlength\parindent{0mm}
}

\newenvironment{corrige}{%
     \needspace{3\baselineskip}
     \medskip
     \textbf{\textsc{Corrigé}}
     \medskip
}
{
}

\newenvironment{extern}{%
     \begin{center}
     }
     {
     \end{center}
}

\NewEnviron{code}{%
	\par
     \boite{gray}{\texttt{%
     \BODY
     }}
     \par
}

\newenvironment{vbloc}{% boite sans cadre empeche saut de page
     \begin{minipage}[t]{\linewidth}
     }
     {
     \end{minipage}
}
\NewEnviron{h2}{%
    \needspace{3\baselineskip}
    \vspace{0.6cm}
	\noindent \MakeUppercase{\sffamily \large \BODY}
	\vspace{1mm}\textcolor{mcgris}{\hrule}\vspace{0.4cm}
	\par
}{}

\NewEnviron{h3}{%
    \needspace{3\baselineskip}
	\vspace{5mm}
	\textsc{\BODY}
	\par
}

\NewEnviron{margeneg}{ %
\begin{addmargin}[-1cm]{0cm}
\BODY
\end{addmargin}
}

\NewEnviron{html}{%
}

\begin{document}
\begin{h2}1. Définitions\end{h2}
\cadre{bleu}{Définition}{%id="d10"
     Une \textbf{matrice} de dimension (ou d'\textit{ordre} or de \textit{taille}) $n\times p$ est un tableau de nombres réels (appelés coefficients ou termes) comportant $n$ lignes et $p$ colonnes.
     \par
     Si on désigne par $a_{ij}$ le coefficient situé à la $i$-ième ligne et la $j$-ième colonne la matrice s'écrira :
\begin{center}
$ A=\begin{pmatrix}  a_{11} & a_{12} & \ldots & a_{1p}\\ a_{21} & a_{22} & \ldots & a_{2p}  \\ \vdots & \vdots & \ddots & \vdots\\ a_{n1} & a_{n2} & \ldots & a_{np} \end{pmatrix}.$
\end{center}
}
\bloc{orange}{Exemple}{%id="e10"
     La matrice $A=\begin{pmatrix} 1 &amp; 2 & 3 \\ 4 & 5 &amp; 6 \end{pmatrix}$ est une matrice de dimension $2\times 3$.
}
\bloc{cyan}{Notations}{%id="r10"
     On notera, en abrégé, $A=\left(a_{ij}\right)$ la matrice dont le coefficient situé à la $i$-ème ligne et la $j$-ième colonne est $a_{ij}$.
}
\cadre{bleu}{Définitions}{%id="d20"
     \begin{itemize}
          \item Une matrice \textbf{carrée} est une matrice dont le nombre de lignes est égal au nombre de colonnes.
          \item Une matrice \textbf{ligne} est une matrice dont le nombre de lignes est égal à $1$.
          \item Une matrice \textbf{colonne} est une matrice dont le nombre de colonnes est égal à $1$.
     \end{itemize}
}
\bloc{orange}{Exemples}{%id="e20"
     \begin{itemize}
          \item La matrice $A=\begin{pmatrix} 1 & 2 \\ 1 & 2 \end{pmatrix}$ est une matrice carrée (de dimension $2\times 2$ - ou on peut dire, plus simplement, de dimension 2).
          \item La matrice $B=\begin{pmatrix}1 & 2 & 0,5 \end{pmatrix}$ est une matrice ligne (de dimension $1\times 3$).
          \item La matrice $C=\begin{pmatrix} 1 \\ 2 \\ 0 \\ 4 \end{pmatrix}$ est une matrice colonne (de dimension $4\times 1$).
     \end{itemize}
}
\bloc{cyan}{Remarque}{%id="r20"
     Pour une matrice carrée, on appelle \textbf{diagonale principale}, la diagonale qui relie le coin situé en haut à gauche au coin situé en bas à droite. Sur l'exemple ci-dessous, les coefficients de la diagonale principale sont marqués en rouge :
     \begin{center}$A=\begin{pmatrix} \color{red}{1} & 2 & 3 & 4 \\ 2 & \color{red}{3} & 4 & 5 \\ 3 & 4 & \color{red}{5} & 6 \\ 4 & 5 & 6 & \color{red}{7} \end{pmatrix}$.\end{center}
}
\cadre{bleu}{Définitions}{%id="d30"
     \begin{itemize}
          \item La matrice \textbf{nulle} de dimension $n\times p$ est la matrice de dimension $n\times p$ dont tous les coefficients sont nuls.
          \item Une matrice \textbf{diagonale} est une matrice carrée dont tout les coefficients situés en dehors de la diagonale principale sont nuls.
          \item La matrice \textbf{unité} de dimension $n$ est la matrice carrée de dimension $n$ qui contient des $1$ sur la diagonale principale et des $0$ ailleurs :
\begin{center}
$ A=\begin{pmatrix}  1 & 0 & \ldots & 0\\ 0 & 1 & \ldots & 0\\ \vdots & \vdots & \ddots & \vdots\\ 0 & 0 & \ldots & 1 \end{pmatrix}$.
\end{center}
\end{itemize}
}
\bloc{orange}{Exemples}{%id="e30"
     \begin{itemize}
          \item La matrice $A=\begin{pmatrix} 1 & 0 & 0 & 0 \\ 0 & 2 & 0 & 0 \\ 0 & 0 & 0 & 0 \\ 0 & 0 & 0 & 1 \end{pmatrix}$ est une matrice diagonale d'ordre 4.
          \item La matrice unité d'ordre 2 est $I_{2}=\begin{pmatrix} 1 & 0 \\ 0 & 1 \end{pmatrix}$.
     \end{itemize}
}
\begin{h2}2. Opérations sur les matrices\end{h2}
\cadre{bleu}{Définition (Somme de matrices)}{%id="d50"
     Soient $A$ et $B$ deux matrices de même dimension.
     \par
     La somme $A+B$ des matrices $A$ et $B$ s'obtient en ajoutant les coefficients de $A$ aux coefficients de $B$ situés \textbf{à la même position}.
}
\bloc{orange}{Exemple}{%id="e50"
     Soient $A=\begin{pmatrix} 2 & -2 & 1 \\ -1 & 1 & 0 \end{pmatrix}$ et $B=\begin{pmatrix} -1 & 1 & 1 \\ -2 & 2 & 0 \end{pmatrix}$.
     \par
     Alors :
     \par
     $A+B=\begin{pmatrix}2-1&-2+1&1+1\\-1-2&1+2&0+0\end{pmatrix}=\begin{pmatrix}1&-1&2\\-3&3&0\end{pmatrix}$.
}
\bloc{cyan}{Remarques}{%id="r50"
     \begin{itemize}
          \item On ne peut additionner deux matrices que si elles ont les même dimensions, c'est à dire le même nombre de lignes et le même nombre de colonnes.
          \item On définit de manière analogue la différence de deux matrices.
     \end{itemize}
}
\cadre{bleu}{Définition (Produit d'une matrice par un nombre réel)}{%id="d60"
     Soient $A$ une matrice et $k$ un nombre réel..
     \par
     Le produit $kA$ est la matrice obtenue en multipliant chacun des coefficients de $A$ par $k$.
}
\bloc{orange}{Exemple}{%id="e60"
     Si $A=\begin{pmatrix} 1 & 1 & 0 \\ 2 & 0 & 0 \end{pmatrix}$ alors :
     \begin{itemize}
          \item $2A=\begin{pmatrix} 2\times 1 & 2\times 1 & 2\times 0 \\ 2\times 2 & 2\times 0 & 2\times 0\end{pmatrix}=\begin{pmatrix}2 & 2 & 0 \\ 4 & 0 & 0\end{pmatrix}$.
          \item $-A=-1\times A=\begin{pmatrix} -1 & -1 & 0 \\ -2 & 0 & 0 \end{pmatrix}$.
     \end{itemize}
}
\cadre{vert}{Propriétés}{%id="p65"
     Soient $A$, $B$ et $C$ trois matrices de mêmes dimensions et $k$ et $k^{\prime}$ deux réels.
     \begin{itemize}
          \item $A+B = B+A $ (commutativité de l'addition) ;
          \item $\left(A+B\right)+C = A+\left(B+C\right)$ (associativité de l'addition) ;
          \item $k\left(A+B\right) = kA+kB$ ;
          \item $\left(k+k^{\prime}\right)A = kA+k^{\prime}A$ ;
          \item $k\left(k^{\prime}A\right) = \left(kk^{\prime}\right)A$.
     \end{itemize}
}
\cadre{bleu}{Définition (Produit d'une matrice ligne par une matrice colonne)}{%id="d70"
     Soient $A=\left(a_{1} a_{2} \cdots a_{n}\right)$ une matrice ligne $1\times n$ et $B=\begin{pmatrix} b_{1} \\ b_{2} \\ \cdots \\ b_{n} \end{pmatrix}$ une matrice colonne $n\times 1$. Le produit de $A$ par $B$ est le nombre réel :
     \par
     $A\times B = \left(a_{1} a_{2} \cdots a_{n}\right)\times \begin{pmatrix} b_{1} \\ b_{2} \\ \cdots \\ b_{n} \end{pmatrix} = a_{1}b_{1} + a_{2}b_{2} + \cdots + a_{n}b_{n}$.
}
\bloc{cyan}{Remarque}{%id="r70"
     \begin{itemize}
          \item Les deux matrices $A$ et $B$ doivent avoir le même nombre $n$ de coefficients.
          \item Pour cette formule, la matrice ligne doit être impérativement en premier !
     \end{itemize}
}
\bloc{orange}{Exemple}{%id="e70"
     Si $ A=\left(1 2 3 4\right) $ et $ B=\begin{pmatrix} 5 \\ 6 \\ 7 \\ 8 \end{pmatrix}$ :
     \par
     $A\times B = 1\times 5 + 2\times 6 + 3\times 7 + 4\times 8 = 5 + 12 + 21 + 32 = 70$.
}
\cadre{bleu}{Définition (Produit de deux matrices)}{%id="d80"
     Soient $A=\left(a_{ij}\right)$ une matrice $n\times p$ et $B=\left(b_{ij}\right)$ une matrice $p\times q$. Le produit de $A$ par $B$ est la matrice $C=\left(c_{ij}\right)$ à $n$ lignes et $q$ colonnes dont le coefficient situé à la $i$-ième ligne et la $j$-ième colonne est obtenu en multipliant la $i$-ième ligne de A par la $j$-ième colonne de B.
     \par
     C'est à dire que pour tout $1 \leqslant i \leqslant n$ et tout $1 \leqslant j \leqslant q$ :
     \begin{center}$c_{ij} = a_{i1}b_{1j} + a_{i2}b_{2j} + \cdots + a_{ip}b_{pj}$.\end{center}
}
\bloc{cyan}{Remarque}{%id="r80"
     Faites bien attention aux dimensions des matrices : Le nombre de colonnes de la première matrice doit être égal au nombre de lignes de la seconde pour que le calcul soit possible.
     \par
     Par exemple, le produit d'une matrice $2\times \color{red}{3}$ par une matrice $\color{red}{3}\times 4$ est possible et donnera une matrice $2\times 4$.
     \par
     Par contre, le produit d'une matrice $2\times \color{red}{3}$ par une matrice $\color{red}{2}\times 3$ n'est pas possible.
}
\bloc{orange}{Exemple}{%id="e80"
     Calculons le produit $C=A\times B$ avec :
     \par
     $A=\begin{pmatrix} 2 & 4 \\ 1 & 0 \end{pmatrix} $ et $ B=\begin{pmatrix} -1 & 0 & 2 \\ -2 & 1 & 0 \end{pmatrix}$.
     \par
     Ce calcul est possible car le nombre de colonnes de $A$ est égal au nombre de lignes de $B$. Le résultat $C$ sera une matrice $2\times 3$ ($\color{red}{2}\times 2 $par$ 2\times \color{red}{3} \rightarrow \color{red}{2}\times \color{red}{3}$).
     \par
     Notons $C=\begin{pmatrix} c_{11} & c_{12} & c_{13} \\ c_{21} & c_{22} & c_{23} \end{pmatrix}$.
     \par
     Pour calculer $c_{11}$ on multiplie la première ligne de $A$ et la première colonne de $B$ :
     \par
     $C=\begin{pmatrix} \color{red}{2} & \color{red}{4} \\ 1 & 0\end{pmatrix}\times \begin{pmatrix} \color{red}{-1} & 0 & 2 \\ \color{red}{-2} & 1 & 0\end{pmatrix}$ ;
     \par
     on a donc $c_{11}=2\times \left(-1\right)+4\times \left(-2\right)=-2-8=-10$.
     \par
     $C=\begin{pmatrix} \color{red}{2} & \color{red}{4} \\ 1 & 0 \end{pmatrix}\times \begin{pmatrix}\color{red}{-1} & 0 & 2 \\ \color{red}{-2} & 1 & 0 \end{pmatrix}=\begin{pmatrix}\color{red}{-10} & \cdots & \cdots \\ \cdots & \cdots & \cdots \end{pmatrix}$.
     \par
     Pour calculer $c_{12}$ on multiplie la première ligne de $A$ et la seconde colonne de $B$ :
     \par
     $C=\begin{pmatrix} \color{red}{2} & \color{red}{4} \\ 1 & 0\end{pmatrix}\times\begin{pmatrix}-1 & \color{red}{0} & 2 \\ -2 & \color{red}{1} & 0\end{pmatrix}$ ;
     \par
     on a donc $c_{12}=2\times 0+4\times 1=0+4=4$.
     \par
     $C=\begin{pmatrix} \color{red}{2} & \color{red}{4} \\ 1 & 0\end{pmatrix}\times \begin{pmatrix} -1 & \color{red}{0} & 2 \\ -2 & \color{red}{1} & 0\end{pmatrix}=\begin{pmatrix}-10 & \color{red}{4} & \cdots \\ \cdots & \cdots & \cdots \end{pmatrix}$.
     \par
     Et ainsi de suite...
     \par
     Au final on trouve :
     \par
     $C=\begin{pmatrix} 2 & 4 \\ 1 & 0\end{pmatrix}\times \begin{pmatrix}-1 & 0 & 2 \\ -2 & 1 & 0\end{pmatrix}=\begin{pmatrix}-10 & 4 & 4 \\ -1 & 0 & 2 \end{pmatrix}$.
}
Dans ce qui suit, on s'intéressera principalement à des matrices \textbf{carrées}.
\cadre{vert}{Propriété}{%id="p90"
     Soit $A, B$ et $C$, trois matrices carrées de même dimension.
     \begin{itemize}
          \item $A\times \left(B+C\right) = A\times B + A\times C$ (distributivité à gauche)
          \item $\left(A+B\right)\times C = A\times C + B\times C$ (distributivité à droite)
          \item $A\times \left(B\times C\right) = \left(A\times B\right)\times C$ (associativité de la multiplication)
     \end{itemize}
     Par contre en général : $A\times B\neq B\times A$ : la multiplication n'est \textbf{pas} commutative.
}
\bloc{orange}{Exemple}{%id="e90"
     Soit $A=\begin{pmatrix} 0 & 2 \\ 0 & 0 \end{pmatrix}$ et $B=\begin{pmatrix} 0 & 2 \\ 1 & 0 \end{pmatrix}$
     \par
     $A \times B=\begin{pmatrix} 2 & 0 \\ 0 & 0 \end{pmatrix}$
     \par
     tandis que :
     \par
     $B \times A=\begin{pmatrix} 0 & 0 \\ 0 & 2 \end{pmatrix}$
     \par
     Par conséquent $A\times B \neq B\times A$.
}
\cadre{bleu}{Définition (Puissance d'une matrice)}{%id="d100"
     Soit $A$ une matrice carrée et $n$ un entier naturel.
     \par
     On note $A^{n}$ la matrice :
     \begin{center}$A^{n}=A\times A\times \cdots.\times A$ ($n$ facteurs).\end{center}
}
\bloc{cyan}{Remarque}{%id="r100"
     Par convention, on considèrera que $A^{0}$ est la matrice unité de même taille que $A$.
}
\cadre{bleu}{Définition (Matrice inversible)}{%id="d110"
     Une matrice carrée A de dimension $n$ est \textbf{inversible} si et seulement si il existe une
     \par
     matrice $B$ telle que
     \begin{center}$A\times B = B\times A = I_{n}$\end{center}
     où $I_{n}$ est la matrice unité de dimension $n$.
     \par
     La matrice $B$ est appelée \textbf{matrice inverse} de $A$ et notée $A^{-1}$.
}
\begin{h2}3. Résolution de systèmes d'équations\end{h2}
Soit le système :
\par
$\left(S\right) \left\{ \begin{matrix} ax+by=s \\ cx+dy=t \end{matrix}\right.$
     \par
     d'inconnues $x$ et $y$.
     \par
     Si l'on pose $A=\begin{pmatrix} a & b \\ c & d \end{pmatrix}$, $X=\begin{pmatrix} x \\ y \end{pmatrix}$ et $B=\begin{pmatrix} s \\ t \end{pmatrix}$, le système $\left(S\right)$ peut s'écrire :
     \par
     $A\times X=B$. Le théorème ci-dessous permet alors de résoudre ce système.
     \cadre{rouge}{Théorème}{%id="t150"
          Soit $A$ une matrice carrée.
          \par
          Si $A$ est inversible, le système $A\times X=B$ admet une solution unique donnée par :
          \begin{center}$X=A^{-1}\times B$.\end{center}
     }
     \bloc{orange}{Exemple}{%id="e150"
          On cherche à résoudre le système :
          \par
          $\left(S\right) \left\{ \begin{matrix} 3x+4y=1 \\ 5x+7y=2 \end{matrix}\right.$
               \par
               Pour cela on pose : $A=\begin{pmatrix} 3 & 4 \\ 5 & 7 \end{pmatrix}$, $X=\begin{pmatrix} x \\ y \end{pmatrix}$ et $B=\begin{pmatrix} 1 \\ 2 \end{pmatrix}$.
               \par
               L'écriture matricielle est alors $A\times X=B$.
               \par
               A la calculatrice, on trouve que $A$ est inversible d'inverse $A^{-1}=\begin{pmatrix} 7 & -4 \\ -5 & 3 \end{pmatrix}$.
               \par
               La solution du système est donné par :
               \par
               $X=A^{-1}\times B=\begin{pmatrix} 7 & -4 \\ -5 & 3\end{pmatrix}\times \begin{pmatrix}1 \\ 2\end{pmatrix}=\begin{pmatrix}-1 \\ 1 \end{pmatrix}$.
               \par
               C'est à dire $x=-1$ et $y=1$.
          }
          
\end{document}