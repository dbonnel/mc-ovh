\documentclass[a4paper]{article}

%================================================================================================================================
%
% Packages
%
%================================================================================================================================

\usepackage[T1]{fontenc} 	% pour caractères accentués
\usepackage[utf8]{inputenc}  % encodage utf8
\usepackage[french]{babel}	% langue : français
\usepackage{fourier}			% caractères plus lisibles
\usepackage[dvipsnames]{xcolor} % couleurs
\usepackage{fancyhdr}		% réglage header footer
\usepackage{needspace}		% empêcher sauts de page mal placés
\usepackage{graphicx}		% pour inclure des graphiques
\usepackage{enumitem,cprotect}		% personnalise les listes d'items (nécessaire pour ol, al ...)
\usepackage{hyperref}		% Liens hypertexte
\usepackage{pstricks,pst-all,pst-node,pstricks-add,pst-math,pst-plot,pst-tree,pst-eucl} % pstricks
\usepackage[a4paper,includeheadfoot,top=2cm,left=3cm, bottom=2cm,right=3cm]{geometry} % marges etc.
\usepackage{comment}			% commentaires multilignes
\usepackage{amsmath,environ} % maths (matrices, etc.)
\usepackage{amssymb,makeidx}
\usepackage{bm}				% bold maths
\usepackage{tabularx}		% tableaux
\usepackage{colortbl}		% tableaux en couleur
\usepackage{fontawesome}		% Fontawesome
\usepackage{environ}			% environment with command
\usepackage{fp}				% calculs pour ps-tricks
\usepackage{multido}			% pour ps tricks
\usepackage[np]{numprint}	% formattage nombre
\usepackage{tikz,tkz-tab} 			% package principal TikZ
\usepackage{pgfplots}   % axes
\usepackage{mathrsfs}    % cursives
\usepackage{calc}			% calcul taille boites
\usepackage[scaled=0.875]{helvet} % font sans serif
\usepackage{svg} % svg
\usepackage{scrextend} % local margin
\usepackage{scratch} %scratch
\usepackage{multicol} % colonnes
%\usepackage{infix-RPN,pst-func} % formule en notation polanaise inversée
\usepackage{listings}

%================================================================================================================================
%
% Réglages de base
%
%================================================================================================================================

\lstset{
language=Python,   % R code
literate=
{á}{{\'a}}1
{à}{{\`a}}1
{ã}{{\~a}}1
{é}{{\'e}}1
{è}{{\`e}}1
{ê}{{\^e}}1
{í}{{\'i}}1
{ó}{{\'o}}1
{õ}{{\~o}}1
{ú}{{\'u}}1
{ü}{{\"u}}1
{ç}{{\c{c}}}1
{~}{{ }}1
}


\definecolor{codegreen}{rgb}{0,0.6,0}
\definecolor{codegray}{rgb}{0.5,0.5,0.5}
\definecolor{codepurple}{rgb}{0.58,0,0.82}
\definecolor{backcolour}{rgb}{0.95,0.95,0.92}

\lstdefinestyle{mystyle}{
    backgroundcolor=\color{backcolour},   
    commentstyle=\color{codegreen},
    keywordstyle=\color{magenta},
    numberstyle=\tiny\color{codegray},
    stringstyle=\color{codepurple},
    basicstyle=\ttfamily\footnotesize,
    breakatwhitespace=false,         
    breaklines=true,                 
    captionpos=b,                    
    keepspaces=true,                 
    numbers=left,                    
xleftmargin=2em,
framexleftmargin=2em,            
    showspaces=false,                
    showstringspaces=false,
    showtabs=false,                  
    tabsize=2,
    upquote=true
}

\lstset{style=mystyle}


\lstset{style=mystyle}
\newcommand{\imgdir}{C:/laragon/www/newmc/assets/imgsvg/}
\newcommand{\imgsvgdir}{C:/laragon/www/newmc/assets/imgsvg/}

\definecolor{mcgris}{RGB}{220, 220, 220}% ancien~; pour compatibilité
\definecolor{mcbleu}{RGB}{52, 152, 219}
\definecolor{mcvert}{RGB}{125, 194, 70}
\definecolor{mcmauve}{RGB}{154, 0, 215}
\definecolor{mcorange}{RGB}{255, 96, 0}
\definecolor{mcturquoise}{RGB}{0, 153, 153}
\definecolor{mcrouge}{RGB}{255, 0, 0}
\definecolor{mclightvert}{RGB}{205, 234, 190}

\definecolor{gris}{RGB}{220, 220, 220}
\definecolor{bleu}{RGB}{52, 152, 219}
\definecolor{vert}{RGB}{125, 194, 70}
\definecolor{mauve}{RGB}{154, 0, 215}
\definecolor{orange}{RGB}{255, 96, 0}
\definecolor{turquoise}{RGB}{0, 153, 153}
\definecolor{rouge}{RGB}{255, 0, 0}
\definecolor{lightvert}{RGB}{205, 234, 190}
\setitemize[0]{label=\color{lightvert}  $\bullet$}

\pagestyle{fancy}
\renewcommand{\headrulewidth}{0.2pt}
\fancyhead[L]{maths-cours.fr}
\fancyhead[R]{\thepage}
\renewcommand{\footrulewidth}{0.2pt}
\fancyfoot[C]{}

\newcolumntype{C}{>{\centering\arraybackslash}X}
\newcolumntype{s}{>{\hsize=.35\hsize\arraybackslash}X}

\setlength{\parindent}{0pt}		 
\setlength{\parskip}{3mm}
\setlength{\headheight}{1cm}

\def\ebook{ebook}
\def\book{book}
\def\web{web}
\def\type{web}

\newcommand{\vect}[1]{\overrightarrow{\,\mathstrut#1\,}}

\def\Oij{$\left(\text{O}~;~\vect{\imath},~\vect{\jmath}\right)$}
\def\Oijk{$\left(\text{O}~;~\vect{\imath},~\vect{\jmath},~\vect{k}\right)$}
\def\Ouv{$\left(\text{O}~;~\vect{u},~\vect{v}\right)$}

\hypersetup{breaklinks=true, colorlinks = true, linkcolor = OliveGreen, urlcolor = OliveGreen, citecolor = OliveGreen, pdfauthor={Didier BONNEL - https://www.maths-cours.fr} } % supprime les bordures autour des liens

\renewcommand{\arg}[0]{\text{arg}}

\everymath{\displaystyle}

%================================================================================================================================
%
% Macros - Commandes
%
%================================================================================================================================

\newcommand\meta[2]{    			% Utilisé pour créer le post HTML.
	\def\titre{titre}
	\def\url{url}
	\def\arg{#1}
	\ifx\titre\arg
		\newcommand\maintitle{#2}
		\fancyhead[L]{#2}
		{\Large\sffamily \MakeUppercase{#2}}
		\vspace{1mm}\textcolor{mcvert}{\hrule}
	\fi 
	\ifx\url\arg
		\fancyfoot[L]{\href{https://www.maths-cours.fr#2}{\black \footnotesize{https://www.maths-cours.fr#2}}}
	\fi 
}


\newcommand\TitreC[1]{    		% Titre centré
     \needspace{3\baselineskip}
     \begin{center}\textbf{#1}\end{center}
}

\newcommand\newpar{    		% paragraphe
     \par
}

\newcommand\nosp {    		% commande vide (pas d'espace)
}
\newcommand{\id}[1]{} %ignore

\newcommand\boite[2]{				% Boite simple sans titre
	\vspace{5mm}
	\setlength{\fboxrule}{0.2mm}
	\setlength{\fboxsep}{5mm}	
	\fcolorbox{#1}{#1!3}{\makebox[\linewidth-2\fboxrule-2\fboxsep]{
  		\begin{minipage}[t]{\linewidth-2\fboxrule-4\fboxsep}\setlength{\parskip}{3mm}
  			 #2
  		\end{minipage}
	}}
	\vspace{5mm}
}

\newcommand\CBox[4]{				% Boites
	\vspace{5mm}
	\setlength{\fboxrule}{0.2mm}
	\setlength{\fboxsep}{5mm}
	
	\fcolorbox{#1}{#1!3}{\makebox[\linewidth-2\fboxrule-2\fboxsep]{
		\begin{minipage}[t]{1cm}\setlength{\parskip}{3mm}
	  		\textcolor{#1}{\LARGE{#2}}    
 	 	\end{minipage}  
  		\begin{minipage}[t]{\linewidth-2\fboxrule-4\fboxsep}\setlength{\parskip}{3mm}
			\raisebox{1.2mm}{\normalsize\sffamily{\textcolor{#1}{#3}}}						
  			 #4
  		\end{minipage}
	}}
	\vspace{5mm}
}

\newcommand\cadre[3]{				% Boites convertible html
	\par
	\vspace{2mm}
	\setlength{\fboxrule}{0.1mm}
	\setlength{\fboxsep}{5mm}
	\fcolorbox{#1}{white}{\makebox[\linewidth-2\fboxrule-2\fboxsep]{
  		\begin{minipage}[t]{\linewidth-2\fboxrule-4\fboxsep}\setlength{\parskip}{3mm}
			\raisebox{-2.5mm}{\sffamily \small{\textcolor{#1}{\MakeUppercase{#2}}}}		
			\par		
  			 #3
 	 		\end{minipage}
	}}
		\vspace{2mm}
	\par
}

\newcommand\bloc[3]{				% Boites convertible html sans bordure
     \needspace{2\baselineskip}
     {\sffamily \small{\textcolor{#1}{\MakeUppercase{#2}}}}    
		\par		
  			 #3
		\par
}

\newcommand\CHelp[1]{
     \CBox{Plum}{\faInfoCircle}{À RETENIR}{#1}
}

\newcommand\CUp[1]{
     \CBox{NavyBlue}{\faThumbsOUp}{EN PRATIQUE}{#1}
}

\newcommand\CInfo[1]{
     \CBox{Sepia}{\faArrowCircleRight}{REMARQUE}{#1}
}

\newcommand\CRedac[1]{
     \CBox{PineGreen}{\faEdit}{BIEN R\'EDIGER}{#1}
}

\newcommand\CError[1]{
     \CBox{Red}{\faExclamationTriangle}{ATTENTION}{#1}
}

\newcommand\TitreExo[2]{
\needspace{4\baselineskip}
 {\sffamily\large EXERCICE #1\ (\emph{#2 points})}
\vspace{5mm}
}

\newcommand\img[2]{
          \includegraphics[width=#2\paperwidth]{\imgdir#1}
}

\newcommand\imgsvg[2]{
       \begin{center}   \includegraphics[width=#2\paperwidth]{\imgsvgdir#1} \end{center}
}


\newcommand\Lien[2]{
     \href{#1}{#2 \tiny \faExternalLink}
}
\newcommand\mcLien[2]{
     \href{https~://www.maths-cours.fr/#1}{#2 \tiny \faExternalLink}
}

\newcommand{\euro}{\eurologo{}}

%================================================================================================================================
%
% Macros - Environement
%
%================================================================================================================================

\newenvironment{tex}{ %
}
{%
}

\newenvironment{indente}{ %
	\setlength\parindent{10mm}
}

{
	\setlength\parindent{0mm}
}

\newenvironment{corrige}{%
     \needspace{3\baselineskip}
     \medskip
     \textbf{\textsc{Corrigé}}
     \medskip
}
{
}

\newenvironment{extern}{%
     \begin{center}
     }
     {
     \end{center}
}

\NewEnviron{code}{%
	\par
     \boite{gray}{\texttt{%
     \BODY
     }}
     \par
}

\newenvironment{vbloc}{% boite sans cadre empeche saut de page
     \begin{minipage}[t]{\linewidth}
     }
     {
     \end{minipage}
}
\NewEnviron{h2}{%
    \needspace{3\baselineskip}
    \vspace{0.6cm}
	\noindent \MakeUppercase{\sffamily \large \BODY}
	\vspace{1mm}\textcolor{mcgris}{\hrule}\vspace{0.4cm}
	\par
}{}

\NewEnviron{h3}{%
    \needspace{3\baselineskip}
	\vspace{5mm}
	\textsc{\BODY}
	\par
}

\NewEnviron{margeneg}{ %
\begin{addmargin}[-1cm]{0cm}
\BODY
\end{addmargin}
}

\NewEnviron{html}{%
}

\begin{document}
\cadre{bleu}{}{
     Le théorème de Thalès doit son nom au philosophe, astronome et mathématicien grec Thalès de Milet (env. 600 ans avant J.C.). S'il n'est pas l'\og inventeur \fg de ce théorème qui était déjà connu des babyloniens, Thalès l'aurait utilisé pour mesurer la hauteur de la grande pyramide de Kheops.
     \par
     Le \textbf{théorème de Thalès} permet de \textbf{calculer des distances} dans une configuration géométrique comportant des droites parallèles.
     \par
     La \textbf{réciproque} du théorème de Thalès sert à démontrer que \textbf{deux droites sont parallèles} en calculant des rapports de distances.
}
\begin{h2}1. Théorème de Thalès\end{h2}
\cadre{rouge}{Théorème de Thalès}{% id="d10"
     Si $A, B, C, D, E$ sont cinq points tels que~:
     \begin{itemize}
          \item les points $A, B, D$ et les points $A, C, E$ sont alignés
          \item les droites $\left(BC\right)$ et $\left(DE\right)$ sont parallèles
     \end{itemize}
     alors~:
     \begin{center}$\frac{AB}{AD}=\frac{AC}{AE}=\frac{BC}{DE}$\end{center}
}
\bloc{cyan}{Remarques}{% id="r05"
     Deux configurations différentes peuvent se présenter selon l'ordre des points $A, B, D$ et $A, C, E$. Il faut être capable de repérer chacune de ces configurations dans les exercices de géométrie.
}
\begin{center}
     \begin{extern}%width="550" alt="Théorème de Thalès"
          \begin{tabular}{ccc}
               \resizebox{5cm}{!}{
                    \psset{xunit=1.0cm,yunit=1.0cm,algebraic=true,dimen=middle,dotstyle=o,dotsize=5pt 0,linewidth=1pt,arrowsize=3pt 2,arrowinset=0.25}
                    \begin{pspicture*}(-2.,-2.)(8.,5.5)
                         \psplot{-3.}{6.}{(-26.+6.*x)/1.}
                         \psplot{-3.}{6.}{(12.+4.*x)/8.}
                         \psplot[linecolor=red]{-0.5}{6.}{(7.97-0.96*x)/3.51}
                         \psplot[linecolor=red]{-2.}{6.}{(2.55-0.96*x)/3.51}
                         \begin{Large}
                              \rput[bl](4.5,4.1){\blue{$A$}}
                              \rput[bl](4.75,1.1){\blue{$B$}}
                              \rput[bl](0.8,2.2){\blue{$C$}}
                              \rput[bl](-1.08,1.32){\blue{$E$}}
                              \rput[bl](4.41,-0.32){\blue{$D$}}
                         \end{Large}
                    \end{pspicture*}
               }
               &
               \resizebox{5cm}{!}{
                    \psset{xunit=1.0cm,yunit=1.0cm,algebraic=true,dimen=middle,dotstyle=o,dotsize=5pt 0,linewidth=1pt,arrowsize=3pt 2,arrowinset=0.25}
                    \begin{pspicture*}(0.,0.5)(10.,8.)
                         \psplot{0.}{10.}{(-26.+6.*x)/1.}
                         \psplot{0.}{10.}{(12.+4.*x)/8.}
                         \psplot[linecolor=red]{0.}{7.}{(7.97-0.96*x)/3.51}
                         \psplot[linecolor=red]{4.5}{10.}{(30-0.96*x)/3.51}
                         \begin{Large}
                              \rput[bl](4.5,4.1){\blue{$A$}}
                              \rput[bl](4.75,1.1){\blue{$B$}}
                              \rput[bl](0.8,2.2){\blue{$C$}}
                              \rput[bl](9.,6.3){\blue{$E$}}
                              \rput[bl](5.7,7.2){\blue{$D$}}
                         \end{Large}
                    \end{pspicture*}
               }
               \\
               \scriptsize Première configuration &\scriptsize Deuxième configuration
          \end{tabular}
     \end{extern}
\end{center}
\begin{center}\textit{Théorème de Thalès}\end{center}
\bloc{cyan}{Remarques}{% id="r10"
     \begin{itemize}
          \item Il est important de bien faire attention à l'ordre des points. On pourra s'aider en notant la correspondance entre les points. Dans les deux figures ci-dessus~:
          \par
          $A \rightarrow A$
          \par
          $B \rightarrow D$
          \par
          $C \rightarrow E$
          \par
          Par conséquent~:
          \par
          $AB \rightarrow AD$
          \par
          $AC \rightarrow AE$
          \par
          $BC \rightarrow DE$
     \end{itemize}
}
\bloc{orange}{Exemple}{% id="e10"
     \begin{center}
          \begin{extern}%width="350" alt="Exemple théorème de Thalès"
               \resizebox{6cm}{!}{
                    \psset{xunit=1.0cm,yunit=1.0cm,algebraic=true,dimen=middle,dotstyle=o,dotsize=5pt 0,linewidth=1.pt,arrowsize=3pt 2,arrowinset=0.25}
                    \begin{pspicture*}(0.,0.)(12.,4.5)
                         \psplot{0.}{12.}{(--3.--1.*x)/4.}
                         \psplot{0.}{12.}{(--11.-1.*x)/3.}
                         \psplot[linecolor=red]{0.}{12.}{(--1.-2.*x)/-1.}
                         \psplot[linecolor=red]{0.}{12.}{(--18.5-2.*x)/-1.}
                         \begin{Large}
                              \rput[bl](0.8,1.13){\blue{$I$}}
                              \rput[bl](4.9,2.14){\blue{$O$}}
                              \rput[bl](10.7,3.71){\blue{$L$}}
                              \rput[bl](1.8,3.20){\blue{$J$}}
                              \rput[bl](9.25,0.75){\blue{$K$}}
                              \rput[bl](7.8,2.9){$6$}
                              \rput[bl](2.7,1.6){$4$}
                              \rput[bl](1.1,2.){$2$}
                         \end{Large}
                    \end{pspicture*}
               }
          \end{extern}
     \end{center}
     Sur la figure ci-dessus, on sait que $OL=6$cm,$ OI=4$cm et $IJ=2$cm et que les droites $\left(IJ\right)$ et $\left(KL\right)$ sont parallèles.
     \par
     Quelle est la longueur du segment $\left[KL\right]$~?
     \begin{itemize}
          \item les points $O, J, K$ et les points $O, I, L$ sont alignés
          \item les droites $\left(IJ\right)$ et $\left(KL\right)$ sont parallèles
     \end{itemize}
     Par conséquent, d'après le théorème de Thalès~:
     \begin{center}$\frac{OJ}{OK}=\frac{OI}{OL}=\frac{IJ}{KL}$\end{center}
     On remplace les longueurs dont ont connait les mesures~:
     \begin{center}$\frac{OJ}{OK}=\frac{\color{red}{4}}{\color{red}{6}}=\frac{\color{red}{2}}{KL}$\end{center}
     L'égalité $\frac{4}{6}=\frac{2}{KL}$ nous permet de trouver $KL$ («quatrième proportionnelle»)~:
     \par
     $KL=\frac{2\times 6}{4}=3$cm.
}
\begin{h2}2. Réciproque du théorème de Thalès\end{h2}
\cadre{rouge}{Théorème (Réciproque du théorème de Thalès)}{% id="t50"
     Si $A, B, C, D, E$ sont cinq points tels que les points $A, B, D$ et les points $A, C, E$ sont alignés dans le même ordre. 
\par
Si $\frac{AB}{AD}=\frac{AC}{AE}$ alors, les droites $\left(BC\right)$ et $\left(DE\right)$ sont parallèles.

}
\begin{center}
     \begin{extern}%width="240" alt="Réciproque du théorème de Thalès"
          \resizebox{6cm}{!}{
               \psset{xunit=1.0cm,yunit=1.0cm,algebraic=true,dimen=middle,dotstyle=o,dotsize=5pt 0,linewidth=1pt,arrowsize=3pt 2,arrowinset=0.25}
               \begin{pspicture*}(-2.,-2.)(8.,5.5)
                    \psplot{-3.}{6.}{(-26.+6.*x)/1.}
                    \psplot{-3.}{6.}{(12.+4.*x)/8.}
                    \psplot[linecolor=red]{-0.5}{6.}{(7.97-0.96*x)/3.51}
                    \psplot[linecolor=red]{-2.}{6.}{(2.55-0.96*x)/3.51}
                    \begin{Large}
                         \rput[bl](4.5,4.1){\blue{$A$}}
                         \rput[bl](4.75,1.1){\blue{$B$}}
                         \rput[bl](0.8,2.2){\blue{$C$}}
                         \rput[bl](-1.08,1.32){\blue{$E$}}
                         \rput[bl](4.41,-0.32){\blue{$D$}}
                    \end{Large}
               \end{pspicture*}
          }
     \end{extern}
\end{center}
\bloc{cyan}{Remarques}{% id="r20" 
\begin{itemize}
\item
     Ce théorème sert à \textbf{démontrer que deux droites sont parallèles}. 
\item
Si $\frac{AB}{AD}\neq \frac{AC}{AE}$ alors, les droites $\left(BC\right)$ et $\left(DE\right)$ ne sont pas parallèles ; cela ne résulte toutefois pas de la réciproque du théorème de Thalès mais c'est une conséquence du théorème de Thalès lui-même (en effet d'après le théorème de Thalès si les droites $\left(BC\right)$ et $\left(DE\right)$ étaient parallèles on aurait $\frac{AB}{AD}= \frac{AC}{AE}$ - voir la fiche méthode  \og  \mcLien{https://www.maths-cours.fr/methode/determiner-si-deux-droites-sont-paralleles-thales}{Déterminer si deux droites sont parallèles} \fg{} ).  
     \end{itemize} 

}
\bloc{orange}{Exemple}{% id="e20"
     \begin{center}
          \begin{extern}%width="500" alt="Réciproque du théorème de Thalès~: exemple"
               \resizebox{10cm}{!}{
                    \psset{xunit=1.0cm,yunit=1.0cm,algebraic=true,dimen=middle,dotstyle=o,dotsize=5pt 0,linewidth=1.pt,arrowsize=3pt 2,arrowinset=0.25}
                    \begin{pspicture*}(-3.,-7.)(17.5,8.)
                         \psplot{-3.}{17.5}{(-16.9--5*x)/4.4}
                         \psplot{-3.}{17.5}{(--6.--2.*x)/8.}
                         \psplot[linecolor=red]{-3.}{17.5}{(--12.4--5.6*x)/10}
                         \psplot[linecolor=red]{-3.}{17.5}{(-47.4--5.8*x)/10}
                         \begin{Large}
                              \rput[bl](4.6,2.2){\blue{$O$}}
                              \rput[bl](8.6,6.6){\blue{$J$}}
                              \rput[bl](-1.8,0.6){\blue{$I$}}
                              \rput[bl](16.4,5.1){\blue{$L$}}
                              \rput[bl](-2,-5.5){\blue{$K$}}
                              \rput[bl](7.4,4){$5,6$}
                              \rput[bl](1.4,1.4){$6,2$}
                              \rput[bl](10.2,3.6){$7,2$}
                              \rput[bl](1.,-1.8){$6.8$}
                         \end{Large}
                    \end{pspicture*}
               }
          \end{extern}
     \end{center}
     Dans la figure ci-dessus, on sait que $OI=6,2$cm,$ OJ=5,6$cm, $OK=6,8$cm et $OL=7,2$cm.
     \par
     Les droites $\left(IJ\right)$ et $\left(KL\right)$ sont-elles parallèles~?
     \par
     \textbf{Méthode}~: On calcule séparément $\frac{OI}{OL}$ et $\frac{OJ}{OK}$
     \par
     $\frac{OI}{OL}=\frac{6,2}{7,2}=\frac{62}{72}=\frac{31}{36}$
     \par
     $\frac{OJ}{OK}=\frac{5,6}{6,8}=\frac{56}{68}=\frac{14}{17}$
     \par
     $\frac{OI}{OL} \neq \frac{OJ}{OK}$ (si vous n'êtes pas sûr, vérifiez à la calculatrice~!) donc les droites $\left(IJ\right)$ et $\left(KL\right)$ ne sont pas parallèles.
}

\end{document}