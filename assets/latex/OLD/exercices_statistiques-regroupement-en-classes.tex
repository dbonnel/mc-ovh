\documentclass[a4paper]{article}

%================================================================================================================================
%
% Packages
%
%================================================================================================================================

\usepackage[T1]{fontenc} 	% pour caractères accentués
\usepackage[utf8]{inputenc}  % encodage utf8
\usepackage[french]{babel}	% langue : français
\usepackage{fourier}			% caractères plus lisibles
\usepackage[dvipsnames]{xcolor} % couleurs
\usepackage{fancyhdr}		% réglage header footer
\usepackage{needspace}		% empêcher sauts de page mal placés
\usepackage{graphicx}		% pour inclure des graphiques
\usepackage{enumitem,cprotect}		% personnalise les listes d'items (nécessaire pour ol, al ...)
\usepackage{hyperref}		% Liens hypertexte
\usepackage{pstricks,pst-all,pst-node,pstricks-add,pst-math,pst-plot,pst-tree,pst-eucl} % pstricks
\usepackage[a4paper,includeheadfoot,top=2cm,left=3cm, bottom=2cm,right=3cm]{geometry} % marges etc.
\usepackage{comment}			% commentaires multilignes
\usepackage{amsmath,environ} % maths (matrices, etc.)
\usepackage{amssymb,makeidx}
\usepackage{bm}				% bold maths
\usepackage{tabularx}		% tableaux
\usepackage{colortbl}		% tableaux en couleur
\usepackage{fontawesome}		% Fontawesome
\usepackage{environ}			% environment with command
\usepackage{fp}				% calculs pour ps-tricks
\usepackage{multido}			% pour ps tricks
\usepackage[np]{numprint}	% formattage nombre
\usepackage{tikz,tkz-tab} 			% package principal TikZ
\usepackage{pgfplots}   % axes
\usepackage{mathrsfs}    % cursives
\usepackage{calc}			% calcul taille boites
\usepackage[scaled=0.875]{helvet} % font sans serif
\usepackage{svg} % svg
\usepackage{scrextend} % local margin
\usepackage{scratch} %scratch
\usepackage{multicol} % colonnes
%\usepackage{infix-RPN,pst-func} % formule en notation polanaise inversée
\usepackage{listings}

%================================================================================================================================
%
% Réglages de base
%
%================================================================================================================================

\lstset{
language=Python,   % R code
literate=
{á}{{\'a}}1
{à}{{\`a}}1
{ã}{{\~a}}1
{é}{{\'e}}1
{è}{{\`e}}1
{ê}{{\^e}}1
{í}{{\'i}}1
{ó}{{\'o}}1
{õ}{{\~o}}1
{ú}{{\'u}}1
{ü}{{\"u}}1
{ç}{{\c{c}}}1
{~}{{ }}1
}


\definecolor{codegreen}{rgb}{0,0.6,0}
\definecolor{codegray}{rgb}{0.5,0.5,0.5}
\definecolor{codepurple}{rgb}{0.58,0,0.82}
\definecolor{backcolour}{rgb}{0.95,0.95,0.92}

\lstdefinestyle{mystyle}{
    backgroundcolor=\color{backcolour},   
    commentstyle=\color{codegreen},
    keywordstyle=\color{magenta},
    numberstyle=\tiny\color{codegray},
    stringstyle=\color{codepurple},
    basicstyle=\ttfamily\footnotesize,
    breakatwhitespace=false,         
    breaklines=true,                 
    captionpos=b,                    
    keepspaces=true,                 
    numbers=left,                    
xleftmargin=2em,
framexleftmargin=2em,            
    showspaces=false,                
    showstringspaces=false,
    showtabs=false,                  
    tabsize=2,
    upquote=true
}

\lstset{style=mystyle}


\lstset{style=mystyle}
\newcommand{\imgdir}{C:/laragon/www/newmc/assets/imgsvg/}
\newcommand{\imgsvgdir}{C:/laragon/www/newmc/assets/imgsvg/}

\definecolor{mcgris}{RGB}{220, 220, 220}% ancien~; pour compatibilité
\definecolor{mcbleu}{RGB}{52, 152, 219}
\definecolor{mcvert}{RGB}{125, 194, 70}
\definecolor{mcmauve}{RGB}{154, 0, 215}
\definecolor{mcorange}{RGB}{255, 96, 0}
\definecolor{mcturquoise}{RGB}{0, 153, 153}
\definecolor{mcrouge}{RGB}{255, 0, 0}
\definecolor{mclightvert}{RGB}{205, 234, 190}

\definecolor{gris}{RGB}{220, 220, 220}
\definecolor{bleu}{RGB}{52, 152, 219}
\definecolor{vert}{RGB}{125, 194, 70}
\definecolor{mauve}{RGB}{154, 0, 215}
\definecolor{orange}{RGB}{255, 96, 0}
\definecolor{turquoise}{RGB}{0, 153, 153}
\definecolor{rouge}{RGB}{255, 0, 0}
\definecolor{lightvert}{RGB}{205, 234, 190}
\setitemize[0]{label=\color{lightvert}  $\bullet$}

\pagestyle{fancy}
\renewcommand{\headrulewidth}{0.2pt}
\fancyhead[L]{maths-cours.fr}
\fancyhead[R]{\thepage}
\renewcommand{\footrulewidth}{0.2pt}
\fancyfoot[C]{}

\newcolumntype{C}{>{\centering\arraybackslash}X}
\newcolumntype{s}{>{\hsize=.35\hsize\arraybackslash}X}

\setlength{\parindent}{0pt}		 
\setlength{\parskip}{3mm}
\setlength{\headheight}{1cm}

\def\ebook{ebook}
\def\book{book}
\def\web{web}
\def\type{web}

\newcommand{\vect}[1]{\overrightarrow{\,\mathstrut#1\,}}

\def\Oij{$\left(\text{O}~;~\vect{\imath},~\vect{\jmath}\right)$}
\def\Oijk{$\left(\text{O}~;~\vect{\imath},~\vect{\jmath},~\vect{k}\right)$}
\def\Ouv{$\left(\text{O}~;~\vect{u},~\vect{v}\right)$}

\hypersetup{breaklinks=true, colorlinks = true, linkcolor = OliveGreen, urlcolor = OliveGreen, citecolor = OliveGreen, pdfauthor={Didier BONNEL - https://www.maths-cours.fr} } % supprime les bordures autour des liens

\renewcommand{\arg}[0]{\text{arg}}

\everymath{\displaystyle}

%================================================================================================================================
%
% Macros - Commandes
%
%================================================================================================================================

\newcommand\meta[2]{    			% Utilisé pour créer le post HTML.
	\def\titre{titre}
	\def\url{url}
	\def\arg{#1}
	\ifx\titre\arg
		\newcommand\maintitle{#2}
		\fancyhead[L]{#2}
		{\Large\sffamily \MakeUppercase{#2}}
		\vspace{1mm}\textcolor{mcvert}{\hrule}
	\fi 
	\ifx\url\arg
		\fancyfoot[L]{\href{https://www.maths-cours.fr#2}{\black \footnotesize{https://www.maths-cours.fr#2}}}
	\fi 
}


\newcommand\TitreC[1]{    		% Titre centré
     \needspace{3\baselineskip}
     \begin{center}\textbf{#1}\end{center}
}

\newcommand\newpar{    		% paragraphe
     \par
}

\newcommand\nosp {    		% commande vide (pas d'espace)
}
\newcommand{\id}[1]{} %ignore

\newcommand\boite[2]{				% Boite simple sans titre
	\vspace{5mm}
	\setlength{\fboxrule}{0.2mm}
	\setlength{\fboxsep}{5mm}	
	\fcolorbox{#1}{#1!3}{\makebox[\linewidth-2\fboxrule-2\fboxsep]{
  		\begin{minipage}[t]{\linewidth-2\fboxrule-4\fboxsep}\setlength{\parskip}{3mm}
  			 #2
  		\end{minipage}
	}}
	\vspace{5mm}
}

\newcommand\CBox[4]{				% Boites
	\vspace{5mm}
	\setlength{\fboxrule}{0.2mm}
	\setlength{\fboxsep}{5mm}
	
	\fcolorbox{#1}{#1!3}{\makebox[\linewidth-2\fboxrule-2\fboxsep]{
		\begin{minipage}[t]{1cm}\setlength{\parskip}{3mm}
	  		\textcolor{#1}{\LARGE{#2}}    
 	 	\end{minipage}  
  		\begin{minipage}[t]{\linewidth-2\fboxrule-4\fboxsep}\setlength{\parskip}{3mm}
			\raisebox{1.2mm}{\normalsize\sffamily{\textcolor{#1}{#3}}}						
  			 #4
  		\end{minipage}
	}}
	\vspace{5mm}
}

\newcommand\cadre[3]{				% Boites convertible html
	\par
	\vspace{2mm}
	\setlength{\fboxrule}{0.1mm}
	\setlength{\fboxsep}{5mm}
	\fcolorbox{#1}{white}{\makebox[\linewidth-2\fboxrule-2\fboxsep]{
  		\begin{minipage}[t]{\linewidth-2\fboxrule-4\fboxsep}\setlength{\parskip}{3mm}
			\raisebox{-2.5mm}{\sffamily \small{\textcolor{#1}{\MakeUppercase{#2}}}}		
			\par		
  			 #3
 	 		\end{minipage}
	}}
		\vspace{2mm}
	\par
}

\newcommand\bloc[3]{				% Boites convertible html sans bordure
     \needspace{2\baselineskip}
     {\sffamily \small{\textcolor{#1}{\MakeUppercase{#2}}}}    
		\par		
  			 #3
		\par
}

\newcommand\CHelp[1]{
     \CBox{Plum}{\faInfoCircle}{À RETENIR}{#1}
}

\newcommand\CUp[1]{
     \CBox{NavyBlue}{\faThumbsOUp}{EN PRATIQUE}{#1}
}

\newcommand\CInfo[1]{
     \CBox{Sepia}{\faArrowCircleRight}{REMARQUE}{#1}
}

\newcommand\CRedac[1]{
     \CBox{PineGreen}{\faEdit}{BIEN R\'EDIGER}{#1}
}

\newcommand\CError[1]{
     \CBox{Red}{\faExclamationTriangle}{ATTENTION}{#1}
}

\newcommand\TitreExo[2]{
\needspace{4\baselineskip}
 {\sffamily\large EXERCICE #1\ (\emph{#2 points})}
\vspace{5mm}
}

\newcommand\img[2]{
          \includegraphics[width=#2\paperwidth]{\imgdir#1}
}

\newcommand\imgsvg[2]{
       \begin{center}   \includegraphics[width=#2\paperwidth]{\imgsvgdir#1} \end{center}
}


\newcommand\Lien[2]{
     \href{#1}{#2 \tiny \faExternalLink}
}
\newcommand\mcLien[2]{
     \href{https~://www.maths-cours.fr/#1}{#2 \tiny \faExternalLink}
}

\newcommand{\euro}{\eurologo{}}

%================================================================================================================================
%
% Macros - Environement
%
%================================================================================================================================

\newenvironment{tex}{ %
}
{%
}

\newenvironment{indente}{ %
	\setlength\parindent{10mm}
}

{
	\setlength\parindent{0mm}
}

\newenvironment{corrige}{%
     \needspace{3\baselineskip}
     \medskip
     \textbf{\textsc{Corrigé}}
     \medskip
}
{
}

\newenvironment{extern}{%
     \begin{center}
     }
     {
     \end{center}
}

\NewEnviron{code}{%
	\par
     \boite{gray}{\texttt{%
     \BODY
     }}
     \par
}

\newenvironment{vbloc}{% boite sans cadre empeche saut de page
     \begin{minipage}[t]{\linewidth}
     }
     {
     \end{minipage}
}
\NewEnviron{h2}{%
    \needspace{3\baselineskip}
    \vspace{0.6cm}
	\noindent \MakeUppercase{\sffamily \large \BODY}
	\vspace{1mm}\textcolor{mcgris}{\hrule}\vspace{0.4cm}
	\par
}{}

\NewEnviron{h3}{%
    \needspace{3\baselineskip}
	\vspace{5mm}
	\textsc{\BODY}
	\par
}

\NewEnviron{margeneg}{ %
\begin{addmargin}[-1cm]{0cm}
\BODY
\end{addmargin}
}

\NewEnviron{html}{%
}

\begin{document}
\par
%============================================================================================================================
\par
Le tableau ci-dessous (source INSEE) présente la répartition par âge de la population française métropolitaine au $1^\text{er}$ janvier 2018
\textit{(L'âge révolu est l'âge de la personne à son dernier anniversaire).}
\par
\`A cette date, la doyenne des français était âgée de 113 ans.
\begin{center}
     \begin{tabular}{|c|c|c|}\hline %class="compact"
          \textbf{Âge révolu}  &  \textbf{Effectifs}  &  \textbf{Effectifs cumulés croissants} \\ \hline
          0  &  691~165  &  691~165 \\ \hline
          1  &  710~534  &  1~401~699 \\ \hline
          2  &  728~579  &  2~130~278 \\ \hline
          3  &  749~270  &  2~879~548 \\ \hline
          4  &  763~228  &  3~642~776 \\ \hline
          5  &  782~484  &  4~425~260 \\ \hline
          6  &  792~558  &  5~217~818 \\ \hline
          7  &  813~001  &  6~030~819 \\ \hline
          8  &  808~393  &  6~839~212 \\ \hline
          9  &  813~680  &  7~652~892 \\ \hline
          10  &  807~548  &  8~460~440 \\ \hline
          11  &  822~302  &  9~282~742 \\ \hline
          12  &  802~674  &  10~085~416 \\ \hline
          13  &  800~480  &  10~885~896 \\ \hline
          14  &  796~320  &  11~682~216 \\ \hline
          15  &  800~560  &  12~482~776 \\ \hline
          16  &  816~021  &  13~298~797 \\ \hline
          17  &  828~193  &  14~126~990 \\ \hline
          18  &  785~471  &  14~912~461 \\ \hline
          19  &  775~524  &  15~687~985 \\ \hline
          20  &  750~885  &  16~438~870 \\ \hline
          21  &  751~084  &  17~189~954 \\ \hline
          22  &  734~838  &  17~924~792 \\ \hline
          23  &  705~808  &  18~630~600 \\ \hline
          24  &  698~780  &  19~329~380 \\ \hline
          25  &  732~693  &  20~062~073 \\ \hline
          26  &  742~199  &  20~804~272 \\ \hline
          27  &  758~458  &  21~562~730 \\ \hline
          28  &  763~258  &  22~325~988 \\ \hline
          29  &  774~435  &  23~100~423 \\ \hline
          30  &  778~738  &  23~879~161 \\ \hline
          31  &  793~507  &  24~672~668 \\ \hline
          32  &  794~025  &  25~466~693 \\ \hline
          33  &  789~604  &  26~256~297 \\ \hline
          34  &  781~093  &  27~037~390 \\ \hline
     \end{tabular}
\end{center}
\newpage
\begin{center}
     \begin{tabular}{|c|c|c|}\hline %class="compact"
          \textbf{Âge révolu}  &  \textbf{Effectifs}  &  \textbf{Effectifs cumulés croissants} \\ \hline
          35  &  829~365  &  27~866~755 \\ \hline
          36  &  837~426  &  28~704~181 \\ \hline
          37  &  849~108  &  29~553~289 \\ \hline
          38  &  804~184  &  30~357~473 \\ \hline
          39  &  788~264  &  31~145~737 \\ \hline
          40  &  794~640  &  31~940~377 \\ \hline
          41  &  773~344  &  32~713~721 \\ \hline
          42  &  793~019  &  33~506~740 \\ \hline
          43  &  836~502  &  34~343~242 \\ \hline
          44  &  885~498  &  35~228~740 \\ \hline
          45  &  903~921  &  36~132~661 \\ \hline
          46  &  897~885  &  37~030~546 \\ \hline
          47  &  881~680  &  37~912~226 \\ \hline
          48  &  868~783  &  38~781~009 \\ \hline
          49  &  860~664  &  39~641~673 \\ \hline
          50  &  856~564  &  40~498~237 \\ \hline
          51  &  873~805  &  41~372~042 \\ \hline
          52  &  875~149  &  42~247~191 \\ \hline
          53  &  884~018  &  43~131~209 \\ \hline
          54  &  874~390  &  44~005~599 \\ \hline
          55  &  842~410  &  44~848~009 \\ \hline
          56  &  844~453  &  45~692~462 \\ \hline
          57  &  840~074  &  46~532~536 \\ \hline
          58  &  833~430  &  47~365~966 \\ \hline
          59  &  813~824  &  48~179~790 \\ \hline
          60  &  809~540  &  48~989~330 \\ \hline
          61  &  801~271  &  49~790~601 \\ \hline
          62  &  790~864  &  50~581~465 \\ \hline
          63  &  788~149  &  51~369~614 \\ \hline
          64  &  769~877  &  52~139~491 \\ \hline
          65  &  780~203  &  52~919~694 \\ \hline
          66  &  760~764  &  53~680~458 \\ \hline
          67  &  788~609  &  54~469~067 \\ \hline
          68  &  771~267  &  55~240~334 \\ \hline
          69  &  763~386  &  56~003~720 \\ \hline
     \end{tabular}
\end{center}
\newpage
\begin{center}
     \begin{tabular}{|c|c|c|}\hline %class="compact"
          \textbf{Âge révolu}  &  \textbf{Effectifs}  &  \textbf{Effectifs cumulés croissants} \\ \hline
          70  &  746~205  &  56~749~925 \\ \hline
          71  &  703~078  &  57~453~003 \\ \hline
          72  &  526~166  &  57~979~169 \\ \hline
          73  &  510~477  &  58~489~646 \\ \hline
          74  &  493~523  &  58~983~169 \\ \hline
          75  &  452~195  &  59~435~364 \\ \hline
          76  &  399~640  &  59~835~004 \\ \hline
          77  &  410~887  &  60~245~891 \\ \hline
          78  &  424~148  &  60~670~039 \\ \hline
          79  &  410~734  &  61~080~773 \\ \hline
          80  &  394~185  &  61~474~958 \\ \hline
          81  &  385~845  &  61~860~803 \\ \hline
          82  &  365~057  &  62~225~860 \\ \hline
          83  &  356~869  &  62~582~729 \\ \hline
          84  &  327~225  &  62~909~954 \\ \hline
          85  &  319~458  &  63~229~412 \\ \hline
          86  &  290~749  &  63~520~161 \\ \hline
          87  &  268~489  &  63~788~650 \\ \hline
          88  &  227~255  &  64~015~905 \\ \hline
          89  &  201~758  &  64~217~663 \\ \hline
          90  &  171~893  &  64~389~556 \\ \hline
          91  &  147~011  &  64~536~567 \\ \hline
          92  &  123~524  &  64~660~091 \\ \hline
          93  &  98~697  &  64~758~788 \\ \hline
          94  &  78~283  &  64~837~071 \\ \hline
          95  &  61~359  &  64~898~430 \\ \hline
          96  &  46~186  &  64~944~616 \\ \hline
          97  &  34~225  &  64~978~841 \\ \hline
          98  &  14~599  &  64~993~440 \\ \hline
          99  &  8~401  &  65~001~841 \\ \hline
          100  &  5~174  &  65~007~015 \\ \hline
          101  &  3~135  &  65~010~150 \\ \hline
          102  &  2~297  &  65~012~447 \\ \hline
          103  &  2~281  &  65~014~728 \\ \hline
          104  &  1~208  &  65~015~936 \\ \hline
          105~ou~plus  &  2~160  &  65~018~096 \\ \hline
          \textbf{Total}  &  \textbf{65~018~096} &\\ \hline
     \end{tabular}
     \textbf{Tableau A}
\end{center}
%============================================================================================================================
%
\begin{center}\begin{h3}Partie A \end{h3}\end{center}
%
%============================================================================================================================
\begin{enumerate}
     \item %1
     Quel sont le mode et l'étendue de cette série statistique~?
     \item %2
     Calculer le pourcentage de personnes mineures (c'est à dire ayant un âge révolu inférieur ou égal à 17 ans) en métropole au $1^\text{er}$ janvier 2018.
     \item %3
     Déterminer la médiane ainsi que les premier et troisième quartiles de cette série.
     \item %4
     Représenter cette série par un diagramme en boîte.
\end{enumerate}
%============================================================================================================================
%
\begin{center}\begin{h3}Partie B \end{h3}\end{center}
%============================================================================================================================
\par
Le tableau \textbf{A} comporte trop de lignes pour être facilement exploitable.
\par
On décide donc de regrouper ces résultats en classes d'amplitude 10 ans (à l'exception de la dernière).
\par
On obtient alors le tableau suivant~:
\begin{center}
     \begin{tabular}{|c|c|c|c|c|c|c|}\hline %class="compact left"
          \textbf{Classes d'âge }& [0~;~10[ & [10~;~20[ & [20~;~30[ & [30~;~40[ & [40~;~50[ & [50~;~60[\\ \hline
          \textbf{Effectifs} & 7~652~892  &  8~035~093  &  7~412~438  &  8~045~314  &  8~495~936 &  8~538~117   \\ \hline
          \textbf{E.C.C} & 7~652~892  &  15~687~985  &  23~100~423  &  31~145~737  &  39~641~673  &  48~179~790  \\ \hline
          \textbf{Classes d'âge} &[60~;~70[ & [70~;~80[ & [80~;~90[ & [90~;~100[ & [100~;~114[ \\ \hline
          \textbf{Effectifs} & 7~823~930  &  5~077~053  &  3~136~890  &  784~178  &  16~255 \\ \hline
          \textbf{E.C.C} & 56~003~720  &  61~080~773  &  64~217~663  &  65~001~841  &  65~018~096 \\ \hline
     \end{tabular}
     \textbf{Tableau B}
\end{center}
\bigskip
\begin{enumerate}
     \item %1
     Expliquer comment, à partir du tableau \textbf{A}, on a obtenu le tableau \textbf{B}.
     \item %2
     Calculer la moyenne de cette série en utilisant le tableau \textbf{B}.
     \item %3
     Tracer le graphique des effectifs cumulés croissants en choisissant une échelle appropriée.
     \item %4
     \`A l'aide du graphique de la question précédente, déterminer des valeurs approchées à l'unité près de la médiane et des premier et troisième quartiles.\\
     Comparer ce résultat à celui de la question \textbf{3.} de la partie A.
\end{enumerate}

\end{document}