\documentclass[a4paper]{article}

%================================================================================================================================
%
% Packages
%
%================================================================================================================================

\usepackage[T1]{fontenc} 	% pour caractères accentués
\usepackage[utf8]{inputenc}  % encodage utf8
\usepackage[french]{babel}	% langue : français
\usepackage{fourier}			% caractères plus lisibles
\usepackage[dvipsnames]{xcolor} % couleurs
\usepackage{fancyhdr}		% réglage header footer
\usepackage{needspace}		% empêcher sauts de page mal placés
\usepackage{graphicx}		% pour inclure des graphiques
\usepackage{enumitem,cprotect}		% personnalise les listes d'items (nécessaire pour ol, al ...)
\usepackage{hyperref}		% Liens hypertexte
\usepackage{pstricks,pst-all,pst-node,pstricks-add,pst-math,pst-plot,pst-tree,pst-eucl} % pstricks
\usepackage[a4paper,includeheadfoot,top=2cm,left=3cm, bottom=2cm,right=3cm]{geometry} % marges etc.
\usepackage{comment}			% commentaires multilignes
\usepackage{amsmath,environ} % maths (matrices, etc.)
\usepackage{amssymb,makeidx}
\usepackage{bm}				% bold maths
\usepackage{tabularx}		% tableaux
\usepackage{colortbl}		% tableaux en couleur
\usepackage{fontawesome}		% Fontawesome
\usepackage{environ}			% environment with command
\usepackage{fp}				% calculs pour ps-tricks
\usepackage{multido}			% pour ps tricks
\usepackage[np]{numprint}	% formattage nombre
\usepackage{tikz,tkz-tab} 			% package principal TikZ
\usepackage{pgfplots}   % axes
\usepackage{mathrsfs}    % cursives
\usepackage{calc}			% calcul taille boites
\usepackage[scaled=0.875]{helvet} % font sans serif
\usepackage{svg} % svg
\usepackage{scrextend} % local margin
\usepackage{scratch} %scratch
\usepackage{multicol} % colonnes
%\usepackage{infix-RPN,pst-func} % formule en notation polanaise inversée
\usepackage{listings}

%================================================================================================================================
%
% Réglages de base
%
%================================================================================================================================

\lstset{
language=Python,   % R code
literate=
{á}{{\'a}}1
{à}{{\`a}}1
{ã}{{\~a}}1
{é}{{\'e}}1
{è}{{\`e}}1
{ê}{{\^e}}1
{í}{{\'i}}1
{ó}{{\'o}}1
{õ}{{\~o}}1
{ú}{{\'u}}1
{ü}{{\"u}}1
{ç}{{\c{c}}}1
{~}{{ }}1
}


\definecolor{codegreen}{rgb}{0,0.6,0}
\definecolor{codegray}{rgb}{0.5,0.5,0.5}
\definecolor{codepurple}{rgb}{0.58,0,0.82}
\definecolor{backcolour}{rgb}{0.95,0.95,0.92}

\lstdefinestyle{mystyle}{
    backgroundcolor=\color{backcolour},   
    commentstyle=\color{codegreen},
    keywordstyle=\color{magenta},
    numberstyle=\tiny\color{codegray},
    stringstyle=\color{codepurple},
    basicstyle=\ttfamily\footnotesize,
    breakatwhitespace=false,         
    breaklines=true,                 
    captionpos=b,                    
    keepspaces=true,                 
    numbers=left,                    
xleftmargin=2em,
framexleftmargin=2em,            
    showspaces=false,                
    showstringspaces=false,
    showtabs=false,                  
    tabsize=2,
    upquote=true
}

\lstset{style=mystyle}


\lstset{style=mystyle}
\newcommand{\imgdir}{C:/laragon/www/newmc/assets/imgsvg/}
\newcommand{\imgsvgdir}{C:/laragon/www/newmc/assets/imgsvg/}

\definecolor{mcgris}{RGB}{220, 220, 220}% ancien~; pour compatibilité
\definecolor{mcbleu}{RGB}{52, 152, 219}
\definecolor{mcvert}{RGB}{125, 194, 70}
\definecolor{mcmauve}{RGB}{154, 0, 215}
\definecolor{mcorange}{RGB}{255, 96, 0}
\definecolor{mcturquoise}{RGB}{0, 153, 153}
\definecolor{mcrouge}{RGB}{255, 0, 0}
\definecolor{mclightvert}{RGB}{205, 234, 190}

\definecolor{gris}{RGB}{220, 220, 220}
\definecolor{bleu}{RGB}{52, 152, 219}
\definecolor{vert}{RGB}{125, 194, 70}
\definecolor{mauve}{RGB}{154, 0, 215}
\definecolor{orange}{RGB}{255, 96, 0}
\definecolor{turquoise}{RGB}{0, 153, 153}
\definecolor{rouge}{RGB}{255, 0, 0}
\definecolor{lightvert}{RGB}{205, 234, 190}
\setitemize[0]{label=\color{lightvert}  $\bullet$}

\pagestyle{fancy}
\renewcommand{\headrulewidth}{0.2pt}
\fancyhead[L]{maths-cours.fr}
\fancyhead[R]{\thepage}
\renewcommand{\footrulewidth}{0.2pt}
\fancyfoot[C]{}

\newcolumntype{C}{>{\centering\arraybackslash}X}
\newcolumntype{s}{>{\hsize=.35\hsize\arraybackslash}X}

\setlength{\parindent}{0pt}		 
\setlength{\parskip}{3mm}
\setlength{\headheight}{1cm}

\def\ebook{ebook}
\def\book{book}
\def\web{web}
\def\type{web}

\newcommand{\vect}[1]{\overrightarrow{\,\mathstrut#1\,}}

\def\Oij{$\left(\text{O}~;~\vect{\imath},~\vect{\jmath}\right)$}
\def\Oijk{$\left(\text{O}~;~\vect{\imath},~\vect{\jmath},~\vect{k}\right)$}
\def\Ouv{$\left(\text{O}~;~\vect{u},~\vect{v}\right)$}

\hypersetup{breaklinks=true, colorlinks = true, linkcolor = OliveGreen, urlcolor = OliveGreen, citecolor = OliveGreen, pdfauthor={Didier BONNEL - https://www.maths-cours.fr} } % supprime les bordures autour des liens

\renewcommand{\arg}[0]{\text{arg}}

\everymath{\displaystyle}

%================================================================================================================================
%
% Macros - Commandes
%
%================================================================================================================================

\newcommand\meta[2]{    			% Utilisé pour créer le post HTML.
	\def\titre{titre}
	\def\url{url}
	\def\arg{#1}
	\ifx\titre\arg
		\newcommand\maintitle{#2}
		\fancyhead[L]{#2}
		{\Large\sffamily \MakeUppercase{#2}}
		\vspace{1mm}\textcolor{mcvert}{\hrule}
	\fi 
	\ifx\url\arg
		\fancyfoot[L]{\href{https://www.maths-cours.fr#2}{\black \footnotesize{https://www.maths-cours.fr#2}}}
	\fi 
}


\newcommand\TitreC[1]{    		% Titre centré
     \needspace{3\baselineskip}
     \begin{center}\textbf{#1}\end{center}
}

\newcommand\newpar{    		% paragraphe
     \par
}

\newcommand\nosp {    		% commande vide (pas d'espace)
}
\newcommand{\id}[1]{} %ignore

\newcommand\boite[2]{				% Boite simple sans titre
	\vspace{5mm}
	\setlength{\fboxrule}{0.2mm}
	\setlength{\fboxsep}{5mm}	
	\fcolorbox{#1}{#1!3}{\makebox[\linewidth-2\fboxrule-2\fboxsep]{
  		\begin{minipage}[t]{\linewidth-2\fboxrule-4\fboxsep}\setlength{\parskip}{3mm}
  			 #2
  		\end{minipage}
	}}
	\vspace{5mm}
}

\newcommand\CBox[4]{				% Boites
	\vspace{5mm}
	\setlength{\fboxrule}{0.2mm}
	\setlength{\fboxsep}{5mm}
	
	\fcolorbox{#1}{#1!3}{\makebox[\linewidth-2\fboxrule-2\fboxsep]{
		\begin{minipage}[t]{1cm}\setlength{\parskip}{3mm}
	  		\textcolor{#1}{\LARGE{#2}}    
 	 	\end{minipage}  
  		\begin{minipage}[t]{\linewidth-2\fboxrule-4\fboxsep}\setlength{\parskip}{3mm}
			\raisebox{1.2mm}{\normalsize\sffamily{\textcolor{#1}{#3}}}						
  			 #4
  		\end{minipage}
	}}
	\vspace{5mm}
}

\newcommand\cadre[3]{				% Boites convertible html
	\par
	\vspace{2mm}
	\setlength{\fboxrule}{0.1mm}
	\setlength{\fboxsep}{5mm}
	\fcolorbox{#1}{white}{\makebox[\linewidth-2\fboxrule-2\fboxsep]{
  		\begin{minipage}[t]{\linewidth-2\fboxrule-4\fboxsep}\setlength{\parskip}{3mm}
			\raisebox{-2.5mm}{\sffamily \small{\textcolor{#1}{\MakeUppercase{#2}}}}		
			\par		
  			 #3
 	 		\end{minipage}
	}}
		\vspace{2mm}
	\par
}

\newcommand\bloc[3]{				% Boites convertible html sans bordure
     \needspace{2\baselineskip}
     {\sffamily \small{\textcolor{#1}{\MakeUppercase{#2}}}}    
		\par		
  			 #3
		\par
}

\newcommand\CHelp[1]{
     \CBox{Plum}{\faInfoCircle}{À RETENIR}{#1}
}

\newcommand\CUp[1]{
     \CBox{NavyBlue}{\faThumbsOUp}{EN PRATIQUE}{#1}
}

\newcommand\CInfo[1]{
     \CBox{Sepia}{\faArrowCircleRight}{REMARQUE}{#1}
}

\newcommand\CRedac[1]{
     \CBox{PineGreen}{\faEdit}{BIEN R\'EDIGER}{#1}
}

\newcommand\CError[1]{
     \CBox{Red}{\faExclamationTriangle}{ATTENTION}{#1}
}

\newcommand\TitreExo[2]{
\needspace{4\baselineskip}
 {\sffamily\large EXERCICE #1\ (\emph{#2 points})}
\vspace{5mm}
}

\newcommand\img[2]{
          \includegraphics[width=#2\paperwidth]{\imgdir#1}
}

\newcommand\imgsvg[2]{
       \begin{center}   \includegraphics[width=#2\paperwidth]{\imgsvgdir#1} \end{center}
}


\newcommand\Lien[2]{
     \href{#1}{#2 \tiny \faExternalLink}
}
\newcommand\mcLien[2]{
     \href{https~://www.maths-cours.fr/#1}{#2 \tiny \faExternalLink}
}

\newcommand{\euro}{\eurologo{}}

%================================================================================================================================
%
% Macros - Environement
%
%================================================================================================================================

\newenvironment{tex}{ %
}
{%
}

\newenvironment{indente}{ %
	\setlength\parindent{10mm}
}

{
	\setlength\parindent{0mm}
}

\newenvironment{corrige}{%
     \needspace{3\baselineskip}
     \medskip
     \textbf{\textsc{Corrigé}}
     \medskip
}
{
}

\newenvironment{extern}{%
     \begin{center}
     }
     {
     \end{center}
}

\NewEnviron{code}{%
	\par
     \boite{gray}{\texttt{%
     \BODY
     }}
     \par
}

\newenvironment{vbloc}{% boite sans cadre empeche saut de page
     \begin{minipage}[t]{\linewidth}
     }
     {
     \end{minipage}
}
\NewEnviron{h2}{%
    \needspace{3\baselineskip}
    \vspace{0.6cm}
	\noindent \MakeUppercase{\sffamily \large \BODY}
	\vspace{1mm}\textcolor{mcgris}{\hrule}\vspace{0.4cm}
	\par
}{}

\NewEnviron{h3}{%
    \needspace{3\baselineskip}
	\vspace{5mm}
	\textsc{\BODY}
	\par
}

\NewEnviron{margeneg}{ %
\begin{addmargin}[-1cm]{0cm}
\BODY
\end{addmargin}
}

\NewEnviron{html}{%
}

\begin{document}
\meta{url}{/exercices/suites-bac-blanc-es-l-sujet-3-maths-cours-2018/}
\meta{pid}{10469}
\meta{titre}{Suites - Bac blanc ES/L Sujet 3 - Maths-cours 2018}
\meta{type}{exercices}
%
\begin{h2}Exercice 2 (6 points)\end{h2}
\par
L'objectif de ce problème est d'étudier la convergence de la suite $(u_n)$ définie par $u_0=2$ et pour tout entier naturel $n$ :
\[ u_{n+1} = 0,9u_n+2.\]
\par
%
\par
\TitreC{Partie A\\ \'Etude graphique}
\par
Sur le graphique fourni en Annexe (voir ci-dessous), on a représenté les droites $D$ et $\Delta$ d'équations respectives $y=0,9x+2$ et $y=x$.
\par
Ces deux droites se coupent en un point $M$.
\par
\begin{enumerate}
     \item
     Déterminer, par le calcul, les coordonnées exactes du point $M$.
     \item
     $A_0$ est le point de la droite $D$ d'abscisse $u_0=2$.
     \par
     Expliquer pourquoi l'ordonnée de $A_0$ est égale à $u_1$.
     \item
     $B_1$ est le point de la droite $\Delta$ tel que la droite $(A_0B_1)$ est parallèle à l'axe des abscisses.
     \par
     Exprimer, en fonction de $u_1$, les coordonnées de $B_1$.
     \item
     Compléter le graphique de l'annexe de manière à faire apparaître, sur l'axe des abscisses, les valeurs de $u_1,\ u_2,\ u_3,\ u_4,\ u_5$ et $u_6$.
     \item
     \`A l'aide du graphique, conjecturer la limite de la suite $(u_n)$.
     \par
\end{enumerate}
\par
%
\par
\TitreC{Partie B\\ Utilisation d'une suite annexe}
\par
Pour tout entier naturel $n$, on pose $v_n=u_n-20$.
\par
\begin{enumerate}
     \item
     Montrer que la suite $(v_n)$ est une suite géométrique dont on précisera le premier terme et la raison.
     \item
     Exprimer $v_n$ en fonction de $n$.
     \item
     Montrer que pour tout entier naturel $n$ :
     \par
     \[ u_n=20-18 \times 0,9^n. \]
     \item
     En déduire la limite de la suite $(u_n)$.
     \par
\end{enumerate}
\newpage
\begin{center}
     \textbf{ANNEXE}
\end{center}
\begin{center}
     \emph{\`A rendre avec la copie}
\end{center}

\begin{center}
\imgsvg{BBESL-s3-2-1}{0.3}% alt="Suite récurrente - Bac blanc" style="width:60rem"
\end{center}


\begin{corrige}
     %
     %
     \TitreC{Partie A}
     %
     %
     \par
     \begin{enumerate}
          \item %1
          Le point $M$ est le point d'intersection des droites $D$ et $\Delta$ d'équations $y=0,9x+2$ et $y=x$.
          \par
          Son abscisse $x_M$ est donc solution de l'équation $0,9x_M+2 = x_M$.
          \par
          $0,9x_M+2 = x_M\ \Leftrightarrow \ 2=x_M-0,9x_M$
          \par
          $\phantom{0,9x_M+2 = x_M}\ \Leftrightarrow \ 2=0,1x_M$
          \par
          $\phantom{0,9x_M+2 = x_M}\ \Leftrightarrow \ \dfrac{2}{0,1}=x_M$
          \par
          $\phantom{0,9x_M+2 = x_M}\ \Leftrightarrow \ x_M=20$.
          \par
          Comme le point $M$ est situé sur la droite $\Delta$ d'équation $y=x$ son ordonnée est $y_M=x_M=20$.
          \par
          Les coordonnées de $M$ sont donc $(20~;~20)$.
          \item %2
          Le point $A_0$ est situé sur la droite $D$ d'équation $y=0,9x+2$.
          \par
          Son abscisse est $u_0$ ; son ordonnée est donc :
          \par
          $y_{A_0}=0,9u_0+2$
          \par
          Or, d'après la définition de la suite $(u_n)$ : $u_1=0,9u_0+2$ ; par conséquent $y_{A_0}=u_1$.
          \par
          L'ordonnée de $A_0$ est donc $u_1$.
          \item %3
          La droite $(A_0B_1)$ est parallèle à l'axe des abscisses donc l'ordonnée de $B_1$ est égale à l'ordonnée de $A_0$ c'est à dire $u_1$.
          \par
          Comme le point $B_1$ appartient à la droite $\Delta$ d'équation $y=x$ :
          \par
          $y_{B_1}=x_{B_1}=u_1$
          \par
          Les coordonnées du point $B$ sont $(u_1~;~u_1)$.
          \par
          On réitère la procédure de la manière suivante :
          \par
          \begin{itemize}
               \item
               on trace la droite parallèle à l'axe des ordonnées passant par le point $B_1$ ; cette droite coupe $D$ en un point $A_1(u_1~;~u_2)$
               \item
               on trace la droite parallèle à l'axe des abscisses passant par le point  $A_1$ ; cette droite coupe $D$ en un point $B_2(u_2~;~u_2)$
               \par
          \end{itemize}
          \par
          et ainsi de suite...
          \par
          On obtient ainsi le graphique ci-après :

\begin{center}
\imgsvg{BBESL-s3-2-2}{0.3}% alt="Construction des termes d'une suite récurrente" style="width:60rem"
\end{center}

          \begin{center}
                \textit{(Les ordonnées des points n'ont pas été indiquées pour ne pas surcharger la figure)}
          \end{center}
          \item
          On conjecture que lorsque $n$ augmente, les points $A_n$ et $B_n$ se rapprochent du point $M$ et donc que :
          \[ \lim_{n \rightarrow +\infty} u_n =20. \]
          \par
     \end{enumerate}
     \par
     %
     %
     \TitreC{Partie B}
     %
     %
     \par
     \textit{Reportez-vous à la page  \og \'Etude d'une suite arithmético-géométrique étape par étape  \fg{} si vous souhaitez plus d'informations sur la méthode utilisée dans cette partie.
     }
     \par
     \begin{enumerate}
          \item %1
          Pour tout entier naturel $n$ :
          \par
          $v_{n+1}=u_{n+1}-20$
          \par
          $\phantom{v_{n+1}}=0,9u_n+2-20$
          \par
          $\phantom{v_{n+1}}=0,9u_n-18$.
          \par
          Or $v_n=u_n-20$ donc $u_n=v_n+20$ ; alors :
          \par
          $v_{n+1}=0,9(v_n+20)-18$
          \par
          $\phantom{v_{n+1}}=0,9v_n+18-18$
          \par
          $\phantom{v_{n+1}}=0,9v_n$.
          \par
          De plus ${v_0=u_0-20=2-20=-18}$ ; par conséquent, la suite $(v_n)$ est une suite géométrique de premier terme ${v_0=-18}$ et de raison ${q=0,9}$.
          \item %2
          On en déduit que :
          \par
          $v_n=v_0q^n=-18 \times 0,9^n$.
          \item %3
          En utilisant la question précédente et la relation $u_n=v_n+20$ on obtient, pour tout entier naturel $n$ :
          \par
          $u_n=v_n+20=20-18 \times 0,9^n$.
          \item %4
          ${0 \leqslant 0,9 < 1}\ $ donc $\ \lim\limits_{n \rightarrow +\infty } 0,9^n = 0$.
          \par
          Alors :
          \par
          $\lim\limits_{n \rightarrow +\infty}18 \times 0,9^n = 0\ $ et $\ \lim\limits_{n \rightarrow +\infty}20-18 \times 0,9^n = 20$.
          \par
          La suite $(u_n)$ converge vers 20.
          \par
          \cadre{rouge}{À retenir}{
               Soit $q$ un nombre réel positif ou nul.
               \par
               \begin{itemize}
                    \item %
                    Si $0 \leqslant q < 1$, alors $\lim\limits_{n \rightarrow +\infty}q^n=0$.
                    \item %
                    Si $  q > 1$, alors $\lim\limits_{n \rightarrow +\infty}q^n=+\infty$.
               \end{itemize}
               (Remarque : si $q=1$ alors $q^n=1$ pour tout entier naturel $n$, donc $\lim\limits_{n \rightarrow +\infty}q^n=1)$.
          }

     \end{enumerate}
\end{corrige}

\end{document}