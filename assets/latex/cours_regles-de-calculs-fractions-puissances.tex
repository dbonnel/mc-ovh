\documentclass[a4paper]{article}

%================================================================================================================================
%
% Packages
%
%================================================================================================================================

\usepackage[T1]{fontenc} 	% pour caractères accentués
\usepackage[utf8]{inputenc}  % encodage utf8
\usepackage[french]{babel}	% langue : français
\usepackage{fourier}			% caractères plus lisibles
\usepackage[dvipsnames]{xcolor} % couleurs
\usepackage{fancyhdr}		% réglage header footer
\usepackage{needspace}		% empêcher sauts de page mal placés
\usepackage{graphicx}		% pour inclure des graphiques
\usepackage{enumitem,cprotect}		% personnalise les listes d'items (nécessaire pour ol, al ...)
\usepackage{hyperref}		% Liens hypertexte
\usepackage{pstricks,pst-all,pst-node,pstricks-add,pst-math,pst-plot,pst-tree,pst-eucl} % pstricks
\usepackage[a4paper,includeheadfoot,top=2cm,left=3cm, bottom=2cm,right=3cm]{geometry} % marges etc.
\usepackage{comment}			% commentaires multilignes
\usepackage{amsmath,environ} % maths (matrices, etc.)
\usepackage{amssymb,makeidx}
\usepackage{bm}				% bold maths
\usepackage{tabularx}		% tableaux
\usepackage{colortbl}		% tableaux en couleur
\usepackage{fontawesome}		% Fontawesome
\usepackage{environ}			% environment with command
\usepackage{fp}				% calculs pour ps-tricks
\usepackage{multido}			% pour ps tricks
\usepackage[np]{numprint}	% formattage nombre
\usepackage{tikz,tkz-tab} 			% package principal TikZ
\usepackage{pgfplots}   % axes
\usepackage{mathrsfs}    % cursives
\usepackage{calc}			% calcul taille boites
\usepackage[scaled=0.875]{helvet} % font sans serif
\usepackage{svg} % svg
\usepackage{scrextend} % local margin
\usepackage{scratch} %scratch
\usepackage{multicol} % colonnes
%\usepackage{infix-RPN,pst-func} % formule en notation polanaise inversée
\usepackage{listings}

%================================================================================================================================
%
% Réglages de base
%
%================================================================================================================================

\lstset{
language=Python,   % R code
literate=
{á}{{\'a}}1
{à}{{\`a}}1
{ã}{{\~a}}1
{é}{{\'e}}1
{è}{{\`e}}1
{ê}{{\^e}}1
{í}{{\'i}}1
{ó}{{\'o}}1
{õ}{{\~o}}1
{ú}{{\'u}}1
{ü}{{\"u}}1
{ç}{{\c{c}}}1
{~}{{ }}1
}


\definecolor{codegreen}{rgb}{0,0.6,0}
\definecolor{codegray}{rgb}{0.5,0.5,0.5}
\definecolor{codepurple}{rgb}{0.58,0,0.82}
\definecolor{backcolour}{rgb}{0.95,0.95,0.92}

\lstdefinestyle{mystyle}{
    backgroundcolor=\color{backcolour},   
    commentstyle=\color{codegreen},
    keywordstyle=\color{magenta},
    numberstyle=\tiny\color{codegray},
    stringstyle=\color{codepurple},
    basicstyle=\ttfamily\footnotesize,
    breakatwhitespace=false,         
    breaklines=true,                 
    captionpos=b,                    
    keepspaces=true,                 
    numbers=left,                    
xleftmargin=2em,
framexleftmargin=2em,            
    showspaces=false,                
    showstringspaces=false,
    showtabs=false,                  
    tabsize=2,
    upquote=true
}

\lstset{style=mystyle}


\lstset{style=mystyle}
\newcommand{\imgdir}{C:/laragon/www/newmc/assets/imgsvg/}
\newcommand{\imgsvgdir}{C:/laragon/www/newmc/assets/imgsvg/}

\definecolor{mcgris}{RGB}{220, 220, 220}% ancien~; pour compatibilité
\definecolor{mcbleu}{RGB}{52, 152, 219}
\definecolor{mcvert}{RGB}{125, 194, 70}
\definecolor{mcmauve}{RGB}{154, 0, 215}
\definecolor{mcorange}{RGB}{255, 96, 0}
\definecolor{mcturquoise}{RGB}{0, 153, 153}
\definecolor{mcrouge}{RGB}{255, 0, 0}
\definecolor{mclightvert}{RGB}{205, 234, 190}

\definecolor{gris}{RGB}{220, 220, 220}
\definecolor{bleu}{RGB}{52, 152, 219}
\definecolor{vert}{RGB}{125, 194, 70}
\definecolor{mauve}{RGB}{154, 0, 215}
\definecolor{orange}{RGB}{255, 96, 0}
\definecolor{turquoise}{RGB}{0, 153, 153}
\definecolor{rouge}{RGB}{255, 0, 0}
\definecolor{lightvert}{RGB}{205, 234, 190}
\setitemize[0]{label=\color{lightvert}  $\bullet$}

\pagestyle{fancy}
\renewcommand{\headrulewidth}{0.2pt}
\fancyhead[L]{maths-cours.fr}
\fancyhead[R]{\thepage}
\renewcommand{\footrulewidth}{0.2pt}
\fancyfoot[C]{}

\newcolumntype{C}{>{\centering\arraybackslash}X}
\newcolumntype{s}{>{\hsize=.35\hsize\arraybackslash}X}

\setlength{\parindent}{0pt}		 
\setlength{\parskip}{3mm}
\setlength{\headheight}{1cm}

\def\ebook{ebook}
\def\book{book}
\def\web{web}
\def\type{web}

\newcommand{\vect}[1]{\overrightarrow{\,\mathstrut#1\,}}

\def\Oij{$\left(\text{O}~;~\vect{\imath},~\vect{\jmath}\right)$}
\def\Oijk{$\left(\text{O}~;~\vect{\imath},~\vect{\jmath},~\vect{k}\right)$}
\def\Ouv{$\left(\text{O}~;~\vect{u},~\vect{v}\right)$}

\hypersetup{breaklinks=true, colorlinks = true, linkcolor = OliveGreen, urlcolor = OliveGreen, citecolor = OliveGreen, pdfauthor={Didier BONNEL - https://www.maths-cours.fr} } % supprime les bordures autour des liens

\renewcommand{\arg}[0]{\text{arg}}

\everymath{\displaystyle}

%================================================================================================================================
%
% Macros - Commandes
%
%================================================================================================================================

\newcommand\meta[2]{    			% Utilisé pour créer le post HTML.
	\def\titre{titre}
	\def\url{url}
	\def\arg{#1}
	\ifx\titre\arg
		\newcommand\maintitle{#2}
		\fancyhead[L]{#2}
		{\Large\sffamily \MakeUppercase{#2}}
		\vspace{1mm}\textcolor{mcvert}{\hrule}
	\fi 
	\ifx\url\arg
		\fancyfoot[L]{\href{https://www.maths-cours.fr#2}{\black \footnotesize{https://www.maths-cours.fr#2}}}
	\fi 
}


\newcommand\TitreC[1]{    		% Titre centré
     \needspace{3\baselineskip}
     \begin{center}\textbf{#1}\end{center}
}

\newcommand\newpar{    		% paragraphe
     \par
}

\newcommand\nosp {    		% commande vide (pas d'espace)
}
\newcommand{\id}[1]{} %ignore

\newcommand\boite[2]{				% Boite simple sans titre
	\vspace{5mm}
	\setlength{\fboxrule}{0.2mm}
	\setlength{\fboxsep}{5mm}	
	\fcolorbox{#1}{#1!3}{\makebox[\linewidth-2\fboxrule-2\fboxsep]{
  		\begin{minipage}[t]{\linewidth-2\fboxrule-4\fboxsep}\setlength{\parskip}{3mm}
  			 #2
  		\end{minipage}
	}}
	\vspace{5mm}
}

\newcommand\CBox[4]{				% Boites
	\vspace{5mm}
	\setlength{\fboxrule}{0.2mm}
	\setlength{\fboxsep}{5mm}
	
	\fcolorbox{#1}{#1!3}{\makebox[\linewidth-2\fboxrule-2\fboxsep]{
		\begin{minipage}[t]{1cm}\setlength{\parskip}{3mm}
	  		\textcolor{#1}{\LARGE{#2}}    
 	 	\end{minipage}  
  		\begin{minipage}[t]{\linewidth-2\fboxrule-4\fboxsep}\setlength{\parskip}{3mm}
			\raisebox{1.2mm}{\normalsize\sffamily{\textcolor{#1}{#3}}}						
  			 #4
  		\end{minipage}
	}}
	\vspace{5mm}
}

\newcommand\cadre[3]{				% Boites convertible html
	\par
	\vspace{2mm}
	\setlength{\fboxrule}{0.1mm}
	\setlength{\fboxsep}{5mm}
	\fcolorbox{#1}{white}{\makebox[\linewidth-2\fboxrule-2\fboxsep]{
  		\begin{minipage}[t]{\linewidth-2\fboxrule-4\fboxsep}\setlength{\parskip}{3mm}
			\raisebox{-2.5mm}{\sffamily \small{\textcolor{#1}{\MakeUppercase{#2}}}}		
			\par		
  			 #3
 	 		\end{minipage}
	}}
		\vspace{2mm}
	\par
}

\newcommand\bloc[3]{				% Boites convertible html sans bordure
     \needspace{2\baselineskip}
     {\sffamily \small{\textcolor{#1}{\MakeUppercase{#2}}}}    
		\par		
  			 #3
		\par
}

\newcommand\CHelp[1]{
     \CBox{Plum}{\faInfoCircle}{À RETENIR}{#1}
}

\newcommand\CUp[1]{
     \CBox{NavyBlue}{\faThumbsOUp}{EN PRATIQUE}{#1}
}

\newcommand\CInfo[1]{
     \CBox{Sepia}{\faArrowCircleRight}{REMARQUE}{#1}
}

\newcommand\CRedac[1]{
     \CBox{PineGreen}{\faEdit}{BIEN R\'EDIGER}{#1}
}

\newcommand\CError[1]{
     \CBox{Red}{\faExclamationTriangle}{ATTENTION}{#1}
}

\newcommand\TitreExo[2]{
\needspace{4\baselineskip}
 {\sffamily\large EXERCICE #1\ (\emph{#2 points})}
\vspace{5mm}
}

\newcommand\img[2]{
          \includegraphics[width=#2\paperwidth]{\imgdir#1}
}

\newcommand\imgsvg[2]{
       \begin{center}   \includegraphics[width=#2\paperwidth]{\imgsvgdir#1} \end{center}
}


\newcommand\Lien[2]{
     \href{#1}{#2 \tiny \faExternalLink}
}
\newcommand\mcLien[2]{
     \href{https~://www.maths-cours.fr/#1}{#2 \tiny \faExternalLink}
}

\newcommand{\euro}{\eurologo{}}

%================================================================================================================================
%
% Macros - Environement
%
%================================================================================================================================

\newenvironment{tex}{ %
}
{%
}

\newenvironment{indente}{ %
	\setlength\parindent{10mm}
}

{
	\setlength\parindent{0mm}
}

\newenvironment{corrige}{%
     \needspace{3\baselineskip}
     \medskip
     \textbf{\textsc{Corrigé}}
     \medskip
}
{
}

\newenvironment{extern}{%
     \begin{center}
     }
     {
     \end{center}
}

\NewEnviron{code}{%
	\par
     \boite{gray}{\texttt{%
     \BODY
     }}
     \par
}

\newenvironment{vbloc}{% boite sans cadre empeche saut de page
     \begin{minipage}[t]{\linewidth}
     }
     {
     \end{minipage}
}
\NewEnviron{h2}{%
    \needspace{3\baselineskip}
    \vspace{0.6cm}
	\noindent \MakeUppercase{\sffamily \large \BODY}
	\vspace{1mm}\textcolor{mcgris}{\hrule}\vspace{0.4cm}
	\par
}{}

\NewEnviron{h3}{%
    \needspace{3\baselineskip}
	\vspace{5mm}
	\textsc{\BODY}
	\par
}

\NewEnviron{margeneg}{ %
\begin{addmargin}[-1cm]{0cm}
\BODY
\end{addmargin}
}

\NewEnviron{html}{%
}

\begin{document}
\meta{url}{/cours/regles-de-calculs-fractions-puissances/}
\meta{pid}{1559}
\meta{titre}{Les règles de calculs, fractions, puissances}
\meta{type}{cours}
\begin{h2}1 - Vocabulaire\end{h2}
\cadre{bleu}{Définitions}{% id="d010"
     \begin{itemize}
          \item La \textbf{somme} de deux \textbf{termes} est le résultat de l'\textbf{addition} de ces nombres.
          \item La \textbf{différence} de deux \textbf{termes} est le résultat de la \textbf{soustraction} de ces nombres.
          \item Le \textbf{produit} de deux \textbf{facteurs} est le résultat de la \textbf{multiplication} de ces nombres.
     \end{itemize}
}

\bloc{orange}{Exemples}{% id="e020"
     \begin{itemize}
          \item $5 = 3+2$ : $\quad 5$ est la \textbf{somme} des termes $3$ et $2$.
          \item $1 = 3-2$ : $\quad1$ est la \textbf{différence} des termes $3$ et $2$.
          \item $6 = 3\times 2$ : $\quad6$ est le \textbf{produit} des facteurs $3$ et $2$.
     \end{itemize}
}
\bloc{cyan}{Remarques}{% id="r040"
     On regroupe souvent \textbf{somme} et \textbf{différence} sous le même terme : \textbf{somme algébrique}. En effet, une soustraction d'un nombre positif correspond à une addition d'un nombre négatif.
     \par
     Lorsqu'une expression contient plusieurs opérations, il s'agit :
     \begin{itemize}
          \item d'une somme algébrique si la \textbf{dernière} opération effectuée (la \textbf{moins} prioritaire) est une addition ou une soustraction. Par exemple : $2x-3y$;
          \item d'un produit si la \textbf{dernière} opération effectuée (la \textbf{moins} prioritaire) est une multiplication.  Par exemple : $3x\left(y-3\right)$.
     \end{itemize}
}
\begin{h2}2 - Priorités de calculs\end{h2}
\cadre{vert}{Propriétés}{% id="p040"
     \begin{itemize}
          \item On effectue d'abord les calculs des expressions entre \textbf{parenthèses}, en commençant par les parenthèses les plus intérieures.
          \item Puis on effectue les \textbf{puissances} avant les multiplications, les divisions, les additions et les soustractions.
          \item Puis on effectue d'abord les \textbf{multiplications} et les \textbf{divisions} avant les additions et les soustractions.
          \item Enfin, on effectue les calculs de la gauche vers la droite.
          \par
          Dans une somme algébrique, on peut également regrouper ensemble les termes de même signe.
     \end{itemize}
}
\bloc{orange}{Exemples}{% id="e040"
     \begin{itemize}
          \item $A=5-3\times 7+2\times \left(4-1\right)$
          \par
          On effectue d'abord les parenthèses :
          \par
          $A=5-3\times 7+2\times 3$
          \par
          Puis les multiplications :
          \par
          $A=5-21+6$
          \par
          Puis les opérations restantes (en regroupant les termes positifs par exemple) :
          \par
          $A=5-21+6=11-21=-10$
          \item \textbf{Attention} à bien tenir compte de la priorité des opération même si l'expression contient des \textit{lettres}. Par exemple :
          \par
          $B=5+\left(7-4\right)\times x$
          \par
          On peut effectuer le calcul dans la parenthèse :
          \par
          $B=5+3\times x=5+3x$
          \par
          On ne peut pas effectuer l'addition $5+3$ car la multiplication $3\times x$ est prioritaire. On ne peut donc pas aller plus loin.
     \end{itemize}
}
\begin{h2}3 - Fractions\end{h2}
\cadre{vert}{Propriétés}{%  id="p060"
     \begin{itemize}
          \item Pour additionner (ou soustraire) des fractions, on ajoute (ou on soustrait) leurs numérateurs, après les avoir mises au \textbf{même dénominateur}.
          \item Pour multiplier des fractions, on multiplie les numérateurs entre eux et les dénominateurs entre eux, \textbf{en simplifiant} au maximum.
          \item Pour diviser par une fraction, on multiplie par son inverse.
     \end{itemize}
}
\bloc{orange}{Exemples}{% id="e060"
     \begin{itemize}
          \item %
          $A=\frac{3}{4}-\frac{2}{5}$
          \par
          $A=\frac{3\times 5}{4\times 5}-\frac{2\times 4}{5\times 4}$
          \par
          $A=\frac{15}{20}-\frac{8}{20}$
          \par
          $A=\frac{7}{20}$
          \item %
          $B=\frac{3}{4}\times \frac{2}{5}$
          \par
          $B=\frac{3\times 2}{4 \times 5}$
          \par
          $B=\frac{3\times 2}{2\times 2\times 5}$
          \par
          $B=\frac{3}{10}$
          \item %
          $C=\frac{3}{4} \div \frac{2}{5}$
          \par
          $C=\frac{3}{4}\times \frac{5}{2}$
          \par
          $C=\frac{3\times 5}{4\times 2}$
          \par
          $C=\frac{15}{8}$
     \end{itemize}
}
\begin{h2}4 - Puissances\end{h2}
\cadre{vert}{Propriétés}{% id="p080"
     \begin{itemize}
          \item Produit : $a^{n}\times a^{m}=a^{n+m}$
          \item Inverse : $\frac{1}{a^{m}}=a^{-m}$
          \item Quotient :$\frac{a^{n}}{a^{m}}=a^{n-m}$
          \item Puissance de puissance :$\left(a^{n}\right)^{m}=a^{n\times m}$
          \item Exposants identiques :$a^{n}\times b^{n}=\left(ab\right)^{n}$
     \end{itemize}
}
\bloc{orange}{Exemples}{% id="e080"
     \begin{itemize}
          \item $A=3^{2}\times 3^{3}=3^{2+3}=3^{5}$
          \item $B=\frac{2^{3}}{2^{-4}}=2^{3-\left(-4\right)}=2^{7}$
          \item $C=\left(10^{2}\right)^{-3}=10^{-6}$
     \end{itemize}
}
\bloc{cyan}{Remarques}{%  id="r090"
     \begin{itemize}
          \item Ces formules peuvent, bien sûr, être utilisées \textit{dans les deux sens}. Par exemple, pour passer de $\frac{1}{a^{m}}$ à $a^{-m}$ ou pour passer de $a^{-m}$ à $\frac{1}{a^{m}}$
          \item Cas particulier de la dernière formule :
          \par
          $\left(-a\right)^{n}=\left(-1\times a\right)^{n}=\left(-1\right)^{n}\times a^{n}$
          \par
          Donc pour $n$ \textbf{impair} : $\left(-a\right)^{n}=-a^{n}$ car alors $\left(-1\right)^{n}=-1$
          \par
          Pour $n$ \textbf{pair} : $\left(-a\right)^{n}=a^{n}$ car alors $\left(-1\right)^{n}=1$
     \end{itemize}
}
\cadre{bleu}{Définition}{% id="d100"
     On appelle \textbf{écriture scientifique} d'un  nombre positif, la notation $a\times 10^{n}$ avec $n$ entier relatif et $1 \leqslant  a < 10$.
}
\bloc{cyan}{Remarque}{% id="r100"
     L'encadrement $1 \leqslant  a < 10$ signifie que l'écriture décimale de $a$ comporte \textbf{un et un seul chiffre non nul avant la virgule}.
}
\bloc{orange}{Exemple}{% id="e100"
     $ D=\frac{5\times 10^{5}\times 10^{-2}\times 7}{2\times 10^{7}}$
     \par
     Donner l'écriture scientifique de $D$, puis son écriture décimale.
     \par
     On regroupe les puissances de 10 d'un coté et les nombres restants de l'autre:
     \par
     $ D=\frac{5\times 7}{2}\times \frac{10^{5}\times 10^{-2}}{10^{7}}$
     \par
     On simplifie :
     \par
     $ D=\frac{35}{2}\times \frac{10^{3}}{10^{7}}$
     \par
     $ D=17,5\times 10^{-4}$
     \par
     L'écriture scientifique de $D$ est :
     \par
     $ D=1,75\times 10^{-3}$
     \par
     L'écriture décimale de $D$ est :
     \par
     $ D=0,00175 $
}

\end{document}