\documentclass[a4paper]{article}

%================================================================================================================================
%
% Packages
%
%================================================================================================================================

\usepackage[T1]{fontenc} 	% pour caractères accentués
\usepackage[utf8]{inputenc}  % encodage utf8
\usepackage[french]{babel}	% langue : français
\usepackage{fourier}			% caractères plus lisibles
\usepackage[dvipsnames]{xcolor} % couleurs
\usepackage{fancyhdr}		% réglage header footer
\usepackage{needspace}		% empêcher sauts de page mal placés
\usepackage{graphicx}		% pour inclure des graphiques
\usepackage{enumitem,cprotect}		% personnalise les listes d'items (nécessaire pour ol, al ...)
\usepackage{hyperref}		% Liens hypertexte
\usepackage{pstricks,pst-all,pst-node,pstricks-add,pst-math,pst-plot,pst-tree,pst-eucl} % pstricks
\usepackage[a4paper,includeheadfoot,top=2cm,left=3cm, bottom=2cm,right=3cm]{geometry} % marges etc.
\usepackage{comment}			% commentaires multilignes
\usepackage{amsmath,environ} % maths (matrices, etc.)
\usepackage{amssymb,makeidx}
\usepackage{bm}				% bold maths
\usepackage{tabularx}		% tableaux
\usepackage{colortbl}		% tableaux en couleur
\usepackage{fontawesome}		% Fontawesome
\usepackage{environ}			% environment with command
\usepackage{fp}				% calculs pour ps-tricks
\usepackage{multido}			% pour ps tricks
\usepackage[np]{numprint}	% formattage nombre
\usepackage{tikz,tkz-tab} 			% package principal TikZ
\usepackage{pgfplots}   % axes
\usepackage{mathrsfs}    % cursives
\usepackage{calc}			% calcul taille boites
\usepackage[scaled=0.875]{helvet} % font sans serif
\usepackage{svg} % svg
\usepackage{scrextend} % local margin
\usepackage{scratch} %scratch
\usepackage{multicol} % colonnes
%\usepackage{infix-RPN,pst-func} % formule en notation polanaise inversée
\usepackage{listings}

%================================================================================================================================
%
% Réglages de base
%
%================================================================================================================================

\lstset{
language=Python,   % R code
literate=
{á}{{\'a}}1
{à}{{\`a}}1
{ã}{{\~a}}1
{é}{{\'e}}1
{è}{{\`e}}1
{ê}{{\^e}}1
{í}{{\'i}}1
{ó}{{\'o}}1
{õ}{{\~o}}1
{ú}{{\'u}}1
{ü}{{\"u}}1
{ç}{{\c{c}}}1
{~}{{ }}1
}


\definecolor{codegreen}{rgb}{0,0.6,0}
\definecolor{codegray}{rgb}{0.5,0.5,0.5}
\definecolor{codepurple}{rgb}{0.58,0,0.82}
\definecolor{backcolour}{rgb}{0.95,0.95,0.92}

\lstdefinestyle{mystyle}{
    backgroundcolor=\color{backcolour},   
    commentstyle=\color{codegreen},
    keywordstyle=\color{magenta},
    numberstyle=\tiny\color{codegray},
    stringstyle=\color{codepurple},
    basicstyle=\ttfamily\footnotesize,
    breakatwhitespace=false,         
    breaklines=true,                 
    captionpos=b,                    
    keepspaces=true,                 
    numbers=left,                    
xleftmargin=2em,
framexleftmargin=2em,            
    showspaces=false,                
    showstringspaces=false,
    showtabs=false,                  
    tabsize=2,
    upquote=true
}

\lstset{style=mystyle}


\lstset{style=mystyle}
\newcommand{\imgdir}{C:/laragon/www/newmc/assets/imgsvg/}
\newcommand{\imgsvgdir}{C:/laragon/www/newmc/assets/imgsvg/}

\definecolor{mcgris}{RGB}{220, 220, 220}% ancien~; pour compatibilité
\definecolor{mcbleu}{RGB}{52, 152, 219}
\definecolor{mcvert}{RGB}{125, 194, 70}
\definecolor{mcmauve}{RGB}{154, 0, 215}
\definecolor{mcorange}{RGB}{255, 96, 0}
\definecolor{mcturquoise}{RGB}{0, 153, 153}
\definecolor{mcrouge}{RGB}{255, 0, 0}
\definecolor{mclightvert}{RGB}{205, 234, 190}

\definecolor{gris}{RGB}{220, 220, 220}
\definecolor{bleu}{RGB}{52, 152, 219}
\definecolor{vert}{RGB}{125, 194, 70}
\definecolor{mauve}{RGB}{154, 0, 215}
\definecolor{orange}{RGB}{255, 96, 0}
\definecolor{turquoise}{RGB}{0, 153, 153}
\definecolor{rouge}{RGB}{255, 0, 0}
\definecolor{lightvert}{RGB}{205, 234, 190}
\setitemize[0]{label=\color{lightvert}  $\bullet$}

\pagestyle{fancy}
\renewcommand{\headrulewidth}{0.2pt}
\fancyhead[L]{maths-cours.fr}
\fancyhead[R]{\thepage}
\renewcommand{\footrulewidth}{0.2pt}
\fancyfoot[C]{}

\newcolumntype{C}{>{\centering\arraybackslash}X}
\newcolumntype{s}{>{\hsize=.35\hsize\arraybackslash}X}

\setlength{\parindent}{0pt}		 
\setlength{\parskip}{3mm}
\setlength{\headheight}{1cm}

\def\ebook{ebook}
\def\book{book}
\def\web{web}
\def\type{web}

\newcommand{\vect}[1]{\overrightarrow{\,\mathstrut#1\,}}

\def\Oij{$\left(\text{O}~;~\vect{\imath},~\vect{\jmath}\right)$}
\def\Oijk{$\left(\text{O}~;~\vect{\imath},~\vect{\jmath},~\vect{k}\right)$}
\def\Ouv{$\left(\text{O}~;~\vect{u},~\vect{v}\right)$}

\hypersetup{breaklinks=true, colorlinks = true, linkcolor = OliveGreen, urlcolor = OliveGreen, citecolor = OliveGreen, pdfauthor={Didier BONNEL - https://www.maths-cours.fr} } % supprime les bordures autour des liens

\renewcommand{\arg}[0]{\text{arg}}

\everymath{\displaystyle}

%================================================================================================================================
%
% Macros - Commandes
%
%================================================================================================================================

\newcommand\meta[2]{    			% Utilisé pour créer le post HTML.
	\def\titre{titre}
	\def\url{url}
	\def\arg{#1}
	\ifx\titre\arg
		\newcommand\maintitle{#2}
		\fancyhead[L]{#2}
		{\Large\sffamily \MakeUppercase{#2}}
		\vspace{1mm}\textcolor{mcvert}{\hrule}
	\fi 
	\ifx\url\arg
		\fancyfoot[L]{\href{https://www.maths-cours.fr#2}{\black \footnotesize{https://www.maths-cours.fr#2}}}
	\fi 
}


\newcommand\TitreC[1]{    		% Titre centré
     \needspace{3\baselineskip}
     \begin{center}\textbf{#1}\end{center}
}

\newcommand\newpar{    		% paragraphe
     \par
}

\newcommand\nosp {    		% commande vide (pas d'espace)
}
\newcommand{\id}[1]{} %ignore

\newcommand\boite[2]{				% Boite simple sans titre
	\vspace{5mm}
	\setlength{\fboxrule}{0.2mm}
	\setlength{\fboxsep}{5mm}	
	\fcolorbox{#1}{#1!3}{\makebox[\linewidth-2\fboxrule-2\fboxsep]{
  		\begin{minipage}[t]{\linewidth-2\fboxrule-4\fboxsep}\setlength{\parskip}{3mm}
  			 #2
  		\end{minipage}
	}}
	\vspace{5mm}
}

\newcommand\CBox[4]{				% Boites
	\vspace{5mm}
	\setlength{\fboxrule}{0.2mm}
	\setlength{\fboxsep}{5mm}
	
	\fcolorbox{#1}{#1!3}{\makebox[\linewidth-2\fboxrule-2\fboxsep]{
		\begin{minipage}[t]{1cm}\setlength{\parskip}{3mm}
	  		\textcolor{#1}{\LARGE{#2}}    
 	 	\end{minipage}  
  		\begin{minipage}[t]{\linewidth-2\fboxrule-4\fboxsep}\setlength{\parskip}{3mm}
			\raisebox{1.2mm}{\normalsize\sffamily{\textcolor{#1}{#3}}}						
  			 #4
  		\end{minipage}
	}}
	\vspace{5mm}
}

\newcommand\cadre[3]{				% Boites convertible html
	\par
	\vspace{2mm}
	\setlength{\fboxrule}{0.1mm}
	\setlength{\fboxsep}{5mm}
	\fcolorbox{#1}{white}{\makebox[\linewidth-2\fboxrule-2\fboxsep]{
  		\begin{minipage}[t]{\linewidth-2\fboxrule-4\fboxsep}\setlength{\parskip}{3mm}
			\raisebox{-2.5mm}{\sffamily \small{\textcolor{#1}{\MakeUppercase{#2}}}}		
			\par		
  			 #3
 	 		\end{minipage}
	}}
		\vspace{2mm}
	\par
}

\newcommand\bloc[3]{				% Boites convertible html sans bordure
     \needspace{2\baselineskip}
     {\sffamily \small{\textcolor{#1}{\MakeUppercase{#2}}}}    
		\par		
  			 #3
		\par
}

\newcommand\CHelp[1]{
     \CBox{Plum}{\faInfoCircle}{À RETENIR}{#1}
}

\newcommand\CUp[1]{
     \CBox{NavyBlue}{\faThumbsOUp}{EN PRATIQUE}{#1}
}

\newcommand\CInfo[1]{
     \CBox{Sepia}{\faArrowCircleRight}{REMARQUE}{#1}
}

\newcommand\CRedac[1]{
     \CBox{PineGreen}{\faEdit}{BIEN R\'EDIGER}{#1}
}

\newcommand\CError[1]{
     \CBox{Red}{\faExclamationTriangle}{ATTENTION}{#1}
}

\newcommand\TitreExo[2]{
\needspace{4\baselineskip}
 {\sffamily\large EXERCICE #1\ (\emph{#2 points})}
\vspace{5mm}
}

\newcommand\img[2]{
          \includegraphics[width=#2\paperwidth]{\imgdir#1}
}

\newcommand\imgsvg[2]{
       \begin{center}   \includegraphics[width=#2\paperwidth]{\imgsvgdir#1} \end{center}
}


\newcommand\Lien[2]{
     \href{#1}{#2 \tiny \faExternalLink}
}
\newcommand\mcLien[2]{
     \href{https~://www.maths-cours.fr/#1}{#2 \tiny \faExternalLink}
}

\newcommand{\euro}{\eurologo{}}

%================================================================================================================================
%
% Macros - Environement
%
%================================================================================================================================

\newenvironment{tex}{ %
}
{%
}

\newenvironment{indente}{ %
	\setlength\parindent{10mm}
}

{
	\setlength\parindent{0mm}
}

\newenvironment{corrige}{%
     \needspace{3\baselineskip}
     \medskip
     \textbf{\textsc{Corrigé}}
     \medskip
}
{
}

\newenvironment{extern}{%
     \begin{center}
     }
     {
     \end{center}
}

\NewEnviron{code}{%
	\par
     \boite{gray}{\texttt{%
     \BODY
     }}
     \par
}

\newenvironment{vbloc}{% boite sans cadre empeche saut de page
     \begin{minipage}[t]{\linewidth}
     }
     {
     \end{minipage}
}
\NewEnviron{h2}{%
    \needspace{3\baselineskip}
    \vspace{0.6cm}
	\noindent \MakeUppercase{\sffamily \large \BODY}
	\vspace{1mm}\textcolor{mcgris}{\hrule}\vspace{0.4cm}
	\par
}{}

\NewEnviron{h3}{%
    \needspace{3\baselineskip}
	\vspace{5mm}
	\textsc{\BODY}
	\par
}

\NewEnviron{margeneg}{ %
\begin{addmargin}[-1cm]{0cm}
\BODY
\end{addmargin}
}

\NewEnviron{html}{%
}

\begin{document}
\meta{url}{/methode/algorithme-de-dijkstra-etape-par-etape/}
\meta{pid}{7015}
\meta{titre}{Algorithme de Dijkstra - Etape par etape}
\meta{type}{methode}
L'algorithme de Dijkstra (\textit{prononcer approximativement « Dextra »}) permet de trouver \textbf{le plus court chemin entre deux sommets d'un graphe} (orienté ou non orienté).
Dans l'exemple du graphe ci-dessous, on va rechercher le chemin le plus court menant de M à S.
\begin{center}
     \begin{extern}%width="300"
          \psset{unit=0.7cm}
          \begin{pspicture}(8,18)(21,5)
               \rput(10,10){\circlenode{E}{E}}
               \rput(16,11){\circlenode{L}{L}}
               \rput(16,6){\circlenode{M}{M}}
               \rput(19,9){\circlenode{N}{N}}
               \rput(17,17){\circlenode{S}{S}}
               \rput(10,14){\circlenode{T}{T}}
               \ncarc[arcangle=20]{E}{L}\ncput*[nrot=:U]{8}
               \ncarc[arcangle=-30]{M}{L}\ncput*[nrot=:U]{7}
               \ncarc[arcangle=-30]{M}{N}\ncput*[nrot=:U]{4}
               \ncarc[arcangle=20]{L}{N}\ncput*[nrot=:U]{2}
               \ncarc[arcangle=20]{E}{S}\ncput*[nrot=:U]{10}
               \ncarc[arcangle=20]{L}{S}\ncput*[nrot=:U]{5}
               \ncarc[arcangle=-30]{T}{E}\ncput*[nrot=:U]{4}
               \ncarc[arcangle=20]{T}{S}\ncput*[nrot=:U]{8}
               \ncarc[arcangle=-50]{E}{M}\ncput*[nrot=:U]{10}
               \ncarc[arcangle=20]{S}{N}\ncput*[nrot=:U]{8}
          \end{pspicture}
     \end{extern}
\end{center}
\begin{h2}Initialisation :\end{h2}
On construit un tableau ayant pour colonnes chacun des sommets du graphe. On ajoute à gauche une colonne qui recensera les sommets choisis à chaque étape (cette colonne est facultative mais facilitera la compréhension de l'algorithme).
\par
Puisque l'on part du sommet M, on inscrit, sur la première ligne intitulée \og Départ \fg{}, $0_{\text{M}}$ \textbf{dans la colonne M et $\bm{\infty}$ dans les autres colonnes}.
\par
\textit{Cela signifie qu'à ce stade, on peut rejoindre M en 0 minute et on n'a rejoint aucun autre sommet puisque l'on n'a pas encore emprunté de chemin...}
\begin{center}
     \begin{extern}
          \begin{tabularx}{0.9\linewidth}{|c|C|C|C|C|C|C|}
               \hline
               \			&  E 						& L							& M							& N 							& S								& T  						\\ \hline
               Départ			&  $\infty$	 				& $\infty$					& $0_{\text{M}}$				& $\infty$					& $\infty$						& $\infty$	  				\\ \hline
               \ 				&  \ 						& \ 							& \							& \ 							& \ 								& \ 											\\
          \end{tabularx}
     \end{extern}
\end{center}
\begin{h2}\'Etape 1 :\end{h2}
On sélectionne \textbf{le plus petit résultat} de la dernière ligne. Ici, c'est \og $0_{\text{M}}$ \fg{} qui correspond au chemin menant au \textbf{sommet M} en 0 minute.
\begin{itemize}
     \item \textbf{On met en évidence cette sélection} (nous l'écrirons en rouge mais il est également possible de la souligner, de l'entourer, etc.).
     \item \textbf{On inscrit le sommet retenu et la durée correspondante dans la première colonne} (ici on écrit M(0)).
     \item \textbf{On désactive les cases situées en dessous de notre sélection} en les grisant par exemple. En effet, on a trouvé le trajet le plus court menant à M ; il sera inutile d'en chercher d'autres.
\end{itemize}
\begin{center}
     \begin{extern}
          \begin{tabularx}{0.9\linewidth}{|c|C|C|C|C|C|C|}
               \hline
               \			&  E 						& L							& M							& N 							& S								& T  						\\ \hline
               Départ			&  $\infty$	 				& $\infty$					& $\color{red}0_{\text{M}}$	& $\infty$					& $\infty$						& $\infty$	  				\\ \hline
               M (0) 			&  \ 						& \ 							& \cellcolor{black!20}		& \ 							& \ 								& \ 											\\ \hline
               \ 				&  \ 						& \ 							& \cellcolor{black!20}		& \ 							& \ 								& \ 											\\
          \end{tabularx}
     \end{extern}
\end{center}
\`A partir de M, on voit sur le graphe que l'on peut rejoindre E, L et N en respectivement 10, 7 et 4 minutes. Ces durées sont les durées les plus courtes ; elles sont inférieures au durées inscrites sur la ligne précédente qui étaient \og $\infty$ \fg{}.
\par
\textbf{On inscrit donc $\bm{10_{\text{M}}, 7_{\text{M}}}$ et $\bm{4_{\text{M}}}$ dans les colonnes E, L et N}. Le M situé en indice signifie que l'on vient du sommet M.
\par
Enfin on complète la ligne en recopiant dans les cellules vides les valeurs de la ligne précédente.
\begin{center}
     \begin{extern}
          \begin{tabularx}{0.9\linewidth}{|c|C|C|C|C|C|C|}
               \hline
               \			&  E 						& L							& M							& N 							& S								& T  						\\ \hline
               Départ			&  $\infty$	 				& $\infty$					&  $\color{red}0_{\text{M}}$	& $\infty$					& $\infty$						& $\infty$	  				\\ \hline
               M (0) 			&  $10_{\text{M}}$	 		& $7_{\text{M}}$	 			& \cellcolor{black!20}		& $4_{\text{M}}$	& $\infty$						& $\infty$ 					\\ \hline
               \ 				&  \ 						& \ 							& \cellcolor{black!20}		& \ 							& \ 								& \ 											\\
          \end{tabularx}
     \end{extern}
\end{center}
\begin{h2}\'Etape 2 :\end{h2}
On sélectionne \textbf{le plus petit résultat} de la dernière ligne. Ici, c'est \og $4_{\text{M}}$ \fg{} qui correspond au chemin menant au \textbf{sommet N} en 4 minutes.
\begin{itemize}
     \item \textbf{On met en évidence cette sélection}.
     \item \textbf{On inscrit le sommet retenu et la durée correspondante dans la première colonne} : N (4).
     \item \textbf{On désactive les cases situées en dessous de notre sélection}. On a trouvé le trajet le plus court menant à N ; il dure 4 minutes.
\end{itemize}
\begin{center}
     \begin{extern}
          \begin{tabularx}{0.9\linewidth}{|c|C|C|C|C|C|C|}
               \hline
               \			&  E 						& L							& M							& N 							& S								& T  						\\ \hline
               Départ			&  $\infty$	 				& $\infty$					& $\color{red}0_{\text{M}}$	& $\infty$					& $\infty$						& $\infty$	  				\\ \hline
               M (0) 			&  $10_{\text{M}}$	 		& $7_{\text{M}}$	 			& \cellcolor{black!20}		& $\color{red}4_{\text{M}}$	& $\infty$						& $\infty$ 					\\ \hline
               N (4)			&  \ 						& \ 							& \cellcolor{black!20}		& \ 				\cellcolor{black!20}			& \ 								& \ 											\\ \hline
               \ 				&  \ 						& \ 							& \cellcolor{black!20}		& \cellcolor{black!20}	 							& \
          \end{tabularx}
     \end{extern}
\end{center}
\`A partir de N, on peut rejoindre L et S (on ne se préoccupe plus de M qui a été \og désactivé \fg{}).
\begin{itemize}
     \item \textbf{Si l'on rejoint L :} On mettra 2 minutes pour aller de N à L et 4 minutes pour aller de M à N (ces 4 minutes sont inscrites dans la première colonne) soit au total 6 minutes. \textbf{Ce trajet est plus rapide que le précédent} qui durait 7 minutes. \textbf{On indique donc $\bm{6_{\text{N}}}$ dans la colonne L}. Le N situé en indice signifie que l'on vient du sommet N.
     \item \textbf{Si l'on rejoint S :} On mettra 8 minutes pour aller de N à S et 4 minutes pour aller de M à N soit au total 12 minutes. \textbf{Ce trajet est plus rapide que le précédent} qui était $\infty$. \textbf{On indique donc $\bm{12_{\text{N}}}$ dans la colonne S}.
\end{itemize}
Puis on complète la ligne en recopiant dans les cellules vides les valeurs de la ligne précédente.
\begin{center}
     \begin{extern}
          \begin{tabularx}{0.9\linewidth}{|c|C|C|C|C|C|C|}
               \hline
               \			&  E 						& L							& M							& N 							& S								& T  						\\ \hline
               Départ			&  $\infty$	 				& $\infty$					& $\color{red}0_{\text{M}}$	& $\infty$					& $\infty$						& $\infty$	  				\\ \hline
               M (0) 			&  $10_{\text{M}}$	 		& $7_{\text{M}}$	 			& \cellcolor{black!20}		& $\color{red}4_{\text{M}}$	& $\infty$						& $\infty$ 					\\ \hline
               N (4)			&  $10_{\text{M}}$	 		& $6_{\text{N}}$	& \cellcolor{black!20}		& \cellcolor{black!20}		& $12_{\text{N}}$				& $\infty$ 					\\ \hline
               \ 				&  \ 						& \ 							& \cellcolor{black!20}		& \cellcolor{black!20}	 							& \
          \end{tabularx}
     \end{extern}
\end{center}
\begin{h2}\'Etape 3 :\end{h2}
On sélectionne \textbf{le plus petit résultat} de la dernière ligne. Ici, c'est \og $6_{\text{N}}$ \fg{} qui correspond au chemin menant au \textbf{sommet L} en 6 minutes.
\begin{itemize}
     \item \textbf{On met en évidence cette sélection}.
     \item \textbf{On inscrit le sommet retenu et la durée correspondante dans la première colonne} : L (6).
     \item \textbf{On désactive les cases situées en dessous de notre sélection}. On a trouvé le trajet le plus court menant à L ; il dure 6 minutes.
\end{itemize}
\begin{center}
     \begin{extern}
          \begin{tabularx}{0.9\linewidth}{|c|C|C|C|C|C|C|}
               \hline
               \			&  E 						& L							& M							& N 							& S								& T  						\\ \hline
               Départ			&  $\infty$	 				& $\infty$					& $\color{red}0_{\text{M}}$	& $\infty$					& $\infty$						& $\infty$	  				\\ \hline
               M (0) 			&  $10_{\text{M}}$	 		& $7_{\text{M}}$	 			& \cellcolor{black!20}		& $\color{red}4_{\text{M}}$	& $\infty$						& $\infty$ 					\\ \hline
               N (4)			&  $10_{\text{M}}$	 		& $\color{red}6_{\text{N}}$	& \cellcolor{black!20}		& \cellcolor{black!20}		& $12_{\text{N}}$				& $\infty$ 					\\ \hline
               L (6)			&  \ 	& \cellcolor{black!20}		& \cellcolor{black!20}		& \cellcolor{black!20}		& \				& \ 					\\ \hline
               &  \ 	& \cellcolor{black!20}		& \cellcolor{black!20}		& \cellcolor{black!20}		& \				& \ 					\\
          \end{tabularx}
     \end{extern}
\end{center}
\`A partir de L, on peut rejoindre E et S (on ne se préoccupe plus de M ni de N qui ont été \og désactivés \fg{}).
\begin{itemize}
     \item \textbf{Si l'on rejoint E :} On mettra 8 minutes pour aller de L à E et 6 minutes pour aller de M à L soit, au total, 14 minutes. \textbf{Ce trajet N'EST PAS plus rapide que le précédent} qui durait 10 minutes.
     \par
     \textbf{On se contente donc de recopier le contenu précédent $\bm{10_{\text{M}}}$ dans la colonne E}.
     \item \textbf{Si l'on rejoint S :} On mettra 5 minutes pour aller de L à S et 6 minutes pour aller de M à L soit au total 11 minutes. \textbf{Ce trajet est plus rapide que le précédent} qui durait 12 minutes. \textbf{On indique donc $\bm{11_{\text{L}}}$ dans la colonne S}.
\end{itemize}
\cadre{vert}{Important !}{
     On inscrit la durée d'un trajet dans le tableau \textbf{uniquement si elle est inférieure} à la durée figurant sur la ligne précédente.
     Dans le cas contraire, on recopie la valeur précédente.
}
Puis on complète la ligne en recopiant dans les cellules vides les valeurs de la ligne précédente.
\begin{center}
     \begin{extern}
          \begin{tabularx}{0.9\linewidth}{|c|C|C|C|C|C|C|}
               \hline
               \			&  E 						& L							& M							& N 							& S								& T  						\\ \hline
               Départ			&  $\infty$	 				& $\infty$					& $\color{red}0_{\text{M}}$	& $\infty$					& $\infty$						& $\infty$	  				\\ \hline
               M (0) 			&  $10_{\text{M}}$	 		& $7_{\text{M}}$	 			& \cellcolor{black!20}		& $\color{red}4_{\text{M}}$	& $\infty$						& $\infty$ 					\\ \hline
               N (4)			&  $10_{\text{M}}$	 		& $\color{red}6_{\text{N}}$	& \cellcolor{black!20}		& \cellcolor{black!20}		& $12_{\text{N}}$				& $\infty$ 					\\ \hline
               L (6)			&  $10_{\text{M}}$	& \cellcolor{black!20}		& \cellcolor{black!20}		& \cellcolor{black!20}		& $11_{\text{L}}$				& $\infty$ 					\\ \hline
               &  \ 	& \cellcolor{black!20}		& \cellcolor{black!20}		& \cellcolor{black!20}		& \				& \ 					\\
          \end{tabularx}
     \end{extern}
\end{center}
\begin{h2}\'Etape 4 :\end{h2}
On sélectionne \textbf{le plus petit résultat}. C'est \og $10_{\text{M}}$ \fg{} qui correspond au chemin menant au \textbf{sommet E} en 10 minutes.
\begin{itemize}
     \item \textbf{On met en évidence cette sélection}.
     \item \textbf{On inscrit le sommet retenu et la durée correspondante dans la première colonne} : E (10).
     \item \textbf{On désactive les cases situées en dessous de notre sélection}. On a trouvé le trajet le plus court menant à E ; il dure 10 minutes.
\end{itemize}
\begin{center}
     \begin{extern}
          \begin{tabularx}{0.9\linewidth}{|c|C|C|C|C|C|C|}
               \hline
               \			&  E 						& L							& M							& N 							& S								& T  						\\ \hline
               Départ			&  $\infty$	 				& $\infty$					& $\color{red}0_{\text{M}}$	& $\infty$					& $\infty$						& $\infty$	  				\\ \hline
               M (0) 			&  $10_{\text{M}}$	 		& $7_{\text{M}}$	 			& \cellcolor{black!20}		& $\color{red}4_{\text{M}}$	& $\infty$						& $\infty$ 					\\ \hline
               N (4)			&  $10_{\text{M}}$	 		& $\color{red}6_{\text{N}}$	& \cellcolor{black!20}		& \cellcolor{black!20}		& $12_{\text{N}}$				& $\infty$ 					\\ \hline
               L (6)			&  $\color{red}10_{\text{M}}$	& \cellcolor{black!20}		& \cellcolor{black!20}		& \cellcolor{black!20}		& $11_{\text{L}}$				& $\infty$ 					\\ \hline
               E (10)			&  \cellcolor{black!20}		& \cellcolor{black!20}		& \cellcolor{black!20}		& \cellcolor{black!20}		& \		& \	 		\\ \hline
               &  \cellcolor{black!20}		& \cellcolor{black!20}		& \cellcolor{black!20}		& \cellcolor{black!20}		& \	& \	 		\\ \hline
          \end{tabularx}
     \end{extern}
\end{center}
\`A partir de E, on peut rejoindre S et T (on ne se préoccupe plus des autres sommets qui ont été \og désactivés \fg{}).
\begin{itemize}
     \item \textbf{Si l'on rejoint S :} On mettra 10 minutes pour aller de E à S et 10 minutes pour aller de M à E (ces 10 minutes sont inscrites dans la première colonne) soit au total 20 minutes.
     \par
     \textbf{Ce trajet N'EST PAS plus rapide que le précédent} qui durait 11 minutes. \textbf{On se contente donc de recopier le contenu précédent $\bm{11_{\text{L}}}$ dans la colonne S}.
     \item \textbf{Si l'on rejoint T :} On mettra 4 minutes pour aller de E à T et 10 minutes pour aller de M à E soit au total 14 minutes. \textbf{Ce trajet est plus rapide que le précédent} qui était $\infty$. \textbf{On indique donc $\bm{14_{\text{E}}}$ dans la colonne T}.
\end{itemize}
\begin{center}
     \begin{extern}
          \begin{tabularx}{0.9\linewidth}{|c|C|C|C|C|C|C|}
               \hline
               \			&  E 						& L							& M							& N 							& S								& T  						\\ \hline
               Départ			&  $\infty$	 				& $\infty$					& $\color{red}0_{\text{M}}$	& $\infty$					& $\infty$						& $\infty$	  				\\ \hline
               M (0) 			&  $10_{\text{M}}$	 		& $7_{\text{M}}$	 			& \cellcolor{black!20}		& $\color{red}4_{\text{M}}$	& $\infty$						& $\infty$ 					\\ \hline
               N (4)			&  $10_{\text{M}}$	 		& $\color{red}6_{\text{N}}$	& \cellcolor{black!20}		& \cellcolor{black!20}		& $12_{\text{N}}$				& $\infty$ 					\\ \hline
               L (6)			&  $\color{red}10_{\text{M}}$	& \cellcolor{black!20}		& \cellcolor{black!20}		& \cellcolor{black!20}		& $11_{\text{L}}$				& $\infty$ 					\\ \hline
               E (10)			&  \cellcolor{black!20}		& \cellcolor{black!20}		& \cellcolor{black!20}		& \cellcolor{black!20}		& $11_{\text{L}}$		& $14_{\text{E}}$	 		\\ \hline
               &  \cellcolor{black!20}		& \cellcolor{black!20}		& \cellcolor{black!20}		& \cellcolor{black!20}		& \	& \	 		\\ \hline
          \end{tabularx}
     \end{extern}
\end{center}
\begin{h2}\'Etape 5 :\end{h2}
On sélectionne \textbf{le plus petit résultat}. C'est \og $11_{\text{L}}$ \fg{} qui correspond au chemin menant au \textbf{sommet S} en 11 minutes.
\par
On a trouvé le trajet le plus court menant à S : il dure \textbf{11 minutes}. Comme c'est la question posée dans l'énoncé, il est inutile d'aller plus loin et le tableau est terminé !
\begin{center}
     \begin{extern}
          \begin{tabularx}{0.9\linewidth}{|c|C|C|C|C|C|C|}
               \hline
               \			&  E 						& L							& M							& N 							& S								& T  						\\ \hline
               Départ			&  $\infty$	 				& $\infty$					& $\color{red}0_{\text{M}}$	& $\infty$					& $\infty$						& $\infty$	  				\\ \hline
               M (0) 			&  $10_{\text{M}}$	 		& $7_{\text{M}}$	 			& \cellcolor{black!20}		& $\color{red}4_{\text{M}}$	& $\infty$						& $\infty$ 					\\ \hline
               N (4)			&  $10_{\text{M}}$	 		& $\color{red}6_{\text{N}}$	& \cellcolor{black!20}		& \cellcolor{black!20}		& $12_{\text{N}}$				& $\infty$ 					\\ \hline
               L (6)			&  $\color{red}10_{\text{M}}$	& \cellcolor{black!20}		& \cellcolor{black!20}		& \cellcolor{black!20}		& $11_{\text{L}}$				& $\infty$ 					\\ \hline
               E (10)			&  \cellcolor{black!20}		& \cellcolor{black!20}		& \cellcolor{black!20}		& \cellcolor{black!20}		& $\color{red}11_{\text{L}}$		& $14_{\text{E}}$	 		\\ \hline
          \end{tabularx}
     \end{extern}
\end{center}
Il reste toutefois à reconstituer le trajet qui correspond à cette durée de 11 minutes.
En pratique, il est plus facile de trouver le trajet en sens inverse en \og remontant \fg{} dans le tableau de la façon suivante :
\begin{itemize}
     \item On part de notre point d'arrivée : \textbf{S}
     \item On recherche la cellule marquée en rouge de la colonne \textbf{S} ; elle contient $\color{red}{11_{\text{L}}}$. On note la lettre écrite en indice : \textbf{L}.
     \item On recherche la cellule marquée en rouge de la colonne \textbf{L} ; elle contient $\color{red}{6_{\text{N}}}$. On note la lettre écrite en indice : \textbf{N}.
     \item On recherche la cellule marquée en rouge de la colonne \textbf{N} ; elle contient $\color{red}{4_{\text{M}}}$. On note la lettre écrite en indice : \textbf{M}.
\end{itemize}
On est arrivé à notre point de départ M après être passé par N et L et S (liste obtenue en listant les sommets en ordre inverse).
\par
Le trajet optimal est donc \textbf{M - N - L - S}.
\par
Enfin, on peut vérifier sur le graphe que ce trajet est correct et dure 11 minutes !

\end{document}