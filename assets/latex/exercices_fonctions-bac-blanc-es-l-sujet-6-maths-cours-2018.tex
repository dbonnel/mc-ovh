\documentclass[a4paper]{article}

%================================================================================================================================
%
% Packages
%
%================================================================================================================================

\usepackage[T1]{fontenc} 	% pour caractères accentués
\usepackage[utf8]{inputenc}  % encodage utf8
\usepackage[french]{babel}	% langue : français
\usepackage{fourier}			% caractères plus lisibles
\usepackage[dvipsnames]{xcolor} % couleurs
\usepackage{fancyhdr}		% réglage header footer
\usepackage{needspace}		% empêcher sauts de page mal placés
\usepackage{graphicx}		% pour inclure des graphiques
\usepackage{enumitem,cprotect}		% personnalise les listes d'items (nécessaire pour ol, al ...)
\usepackage{hyperref}		% Liens hypertexte
\usepackage{pstricks,pst-all,pst-node,pstricks-add,pst-math,pst-plot,pst-tree,pst-eucl} % pstricks
\usepackage[a4paper,includeheadfoot,top=2cm,left=3cm, bottom=2cm,right=3cm]{geometry} % marges etc.
\usepackage{comment}			% commentaires multilignes
\usepackage{amsmath,environ} % maths (matrices, etc.)
\usepackage{amssymb,makeidx}
\usepackage{bm}				% bold maths
\usepackage{tabularx}		% tableaux
\usepackage{colortbl}		% tableaux en couleur
\usepackage{fontawesome}		% Fontawesome
\usepackage{environ}			% environment with command
\usepackage{fp}				% calculs pour ps-tricks
\usepackage{multido}			% pour ps tricks
\usepackage[np]{numprint}	% formattage nombre
\usepackage{tikz,tkz-tab} 			% package principal TikZ
\usepackage{pgfplots}   % axes
\usepackage{mathrsfs}    % cursives
\usepackage{calc}			% calcul taille boites
\usepackage[scaled=0.875]{helvet} % font sans serif
\usepackage{svg} % svg
\usepackage{scrextend} % local margin
\usepackage{scratch} %scratch
\usepackage{multicol} % colonnes
%\usepackage{infix-RPN,pst-func} % formule en notation polanaise inversée
\usepackage{listings}

%================================================================================================================================
%
% Réglages de base
%
%================================================================================================================================

\lstset{
language=Python,   % R code
literate=
{á}{{\'a}}1
{à}{{\`a}}1
{ã}{{\~a}}1
{é}{{\'e}}1
{è}{{\`e}}1
{ê}{{\^e}}1
{í}{{\'i}}1
{ó}{{\'o}}1
{õ}{{\~o}}1
{ú}{{\'u}}1
{ü}{{\"u}}1
{ç}{{\c{c}}}1
{~}{{ }}1
}


\definecolor{codegreen}{rgb}{0,0.6,0}
\definecolor{codegray}{rgb}{0.5,0.5,0.5}
\definecolor{codepurple}{rgb}{0.58,0,0.82}
\definecolor{backcolour}{rgb}{0.95,0.95,0.92}

\lstdefinestyle{mystyle}{
    backgroundcolor=\color{backcolour},   
    commentstyle=\color{codegreen},
    keywordstyle=\color{magenta},
    numberstyle=\tiny\color{codegray},
    stringstyle=\color{codepurple},
    basicstyle=\ttfamily\footnotesize,
    breakatwhitespace=false,         
    breaklines=true,                 
    captionpos=b,                    
    keepspaces=true,                 
    numbers=left,                    
xleftmargin=2em,
framexleftmargin=2em,            
    showspaces=false,                
    showstringspaces=false,
    showtabs=false,                  
    tabsize=2,
    upquote=true
}

\lstset{style=mystyle}


\lstset{style=mystyle}
\newcommand{\imgdir}{C:/laragon/www/newmc/assets/imgsvg/}
\newcommand{\imgsvgdir}{C:/laragon/www/newmc/assets/imgsvg/}

\definecolor{mcgris}{RGB}{220, 220, 220}% ancien~; pour compatibilité
\definecolor{mcbleu}{RGB}{52, 152, 219}
\definecolor{mcvert}{RGB}{125, 194, 70}
\definecolor{mcmauve}{RGB}{154, 0, 215}
\definecolor{mcorange}{RGB}{255, 96, 0}
\definecolor{mcturquoise}{RGB}{0, 153, 153}
\definecolor{mcrouge}{RGB}{255, 0, 0}
\definecolor{mclightvert}{RGB}{205, 234, 190}

\definecolor{gris}{RGB}{220, 220, 220}
\definecolor{bleu}{RGB}{52, 152, 219}
\definecolor{vert}{RGB}{125, 194, 70}
\definecolor{mauve}{RGB}{154, 0, 215}
\definecolor{orange}{RGB}{255, 96, 0}
\definecolor{turquoise}{RGB}{0, 153, 153}
\definecolor{rouge}{RGB}{255, 0, 0}
\definecolor{lightvert}{RGB}{205, 234, 190}
\setitemize[0]{label=\color{lightvert}  $\bullet$}

\pagestyle{fancy}
\renewcommand{\headrulewidth}{0.2pt}
\fancyhead[L]{maths-cours.fr}
\fancyhead[R]{\thepage}
\renewcommand{\footrulewidth}{0.2pt}
\fancyfoot[C]{}

\newcolumntype{C}{>{\centering\arraybackslash}X}
\newcolumntype{s}{>{\hsize=.35\hsize\arraybackslash}X}

\setlength{\parindent}{0pt}		 
\setlength{\parskip}{3mm}
\setlength{\headheight}{1cm}

\def\ebook{ebook}
\def\book{book}
\def\web{web}
\def\type{web}

\newcommand{\vect}[1]{\overrightarrow{\,\mathstrut#1\,}}

\def\Oij{$\left(\text{O}~;~\vect{\imath},~\vect{\jmath}\right)$}
\def\Oijk{$\left(\text{O}~;~\vect{\imath},~\vect{\jmath},~\vect{k}\right)$}
\def\Ouv{$\left(\text{O}~;~\vect{u},~\vect{v}\right)$}

\hypersetup{breaklinks=true, colorlinks = true, linkcolor = OliveGreen, urlcolor = OliveGreen, citecolor = OliveGreen, pdfauthor={Didier BONNEL - https://www.maths-cours.fr} } % supprime les bordures autour des liens

\renewcommand{\arg}[0]{\text{arg}}

\everymath{\displaystyle}

%================================================================================================================================
%
% Macros - Commandes
%
%================================================================================================================================

\newcommand\meta[2]{    			% Utilisé pour créer le post HTML.
	\def\titre{titre}
	\def\url{url}
	\def\arg{#1}
	\ifx\titre\arg
		\newcommand\maintitle{#2}
		\fancyhead[L]{#2}
		{\Large\sffamily \MakeUppercase{#2}}
		\vspace{1mm}\textcolor{mcvert}{\hrule}
	\fi 
	\ifx\url\arg
		\fancyfoot[L]{\href{https://www.maths-cours.fr#2}{\black \footnotesize{https://www.maths-cours.fr#2}}}
	\fi 
}


\newcommand\TitreC[1]{    		% Titre centré
     \needspace{3\baselineskip}
     \begin{center}\textbf{#1}\end{center}
}

\newcommand\newpar{    		% paragraphe
     \par
}

\newcommand\nosp {    		% commande vide (pas d'espace)
}
\newcommand{\id}[1]{} %ignore

\newcommand\boite[2]{				% Boite simple sans titre
	\vspace{5mm}
	\setlength{\fboxrule}{0.2mm}
	\setlength{\fboxsep}{5mm}	
	\fcolorbox{#1}{#1!3}{\makebox[\linewidth-2\fboxrule-2\fboxsep]{
  		\begin{minipage}[t]{\linewidth-2\fboxrule-4\fboxsep}\setlength{\parskip}{3mm}
  			 #2
  		\end{minipage}
	}}
	\vspace{5mm}
}

\newcommand\CBox[4]{				% Boites
	\vspace{5mm}
	\setlength{\fboxrule}{0.2mm}
	\setlength{\fboxsep}{5mm}
	
	\fcolorbox{#1}{#1!3}{\makebox[\linewidth-2\fboxrule-2\fboxsep]{
		\begin{minipage}[t]{1cm}\setlength{\parskip}{3mm}
	  		\textcolor{#1}{\LARGE{#2}}    
 	 	\end{minipage}  
  		\begin{minipage}[t]{\linewidth-2\fboxrule-4\fboxsep}\setlength{\parskip}{3mm}
			\raisebox{1.2mm}{\normalsize\sffamily{\textcolor{#1}{#3}}}						
  			 #4
  		\end{minipage}
	}}
	\vspace{5mm}
}

\newcommand\cadre[3]{				% Boites convertible html
	\par
	\vspace{2mm}
	\setlength{\fboxrule}{0.1mm}
	\setlength{\fboxsep}{5mm}
	\fcolorbox{#1}{white}{\makebox[\linewidth-2\fboxrule-2\fboxsep]{
  		\begin{minipage}[t]{\linewidth-2\fboxrule-4\fboxsep}\setlength{\parskip}{3mm}
			\raisebox{-2.5mm}{\sffamily \small{\textcolor{#1}{\MakeUppercase{#2}}}}		
			\par		
  			 #3
 	 		\end{minipage}
	}}
		\vspace{2mm}
	\par
}

\newcommand\bloc[3]{				% Boites convertible html sans bordure
     \needspace{2\baselineskip}
     {\sffamily \small{\textcolor{#1}{\MakeUppercase{#2}}}}    
		\par		
  			 #3
		\par
}

\newcommand\CHelp[1]{
     \CBox{Plum}{\faInfoCircle}{À RETENIR}{#1}
}

\newcommand\CUp[1]{
     \CBox{NavyBlue}{\faThumbsOUp}{EN PRATIQUE}{#1}
}

\newcommand\CInfo[1]{
     \CBox{Sepia}{\faArrowCircleRight}{REMARQUE}{#1}
}

\newcommand\CRedac[1]{
     \CBox{PineGreen}{\faEdit}{BIEN R\'EDIGER}{#1}
}

\newcommand\CError[1]{
     \CBox{Red}{\faExclamationTriangle}{ATTENTION}{#1}
}

\newcommand\TitreExo[2]{
\needspace{4\baselineskip}
 {\sffamily\large EXERCICE #1\ (\emph{#2 points})}
\vspace{5mm}
}

\newcommand\img[2]{
          \includegraphics[width=#2\paperwidth]{\imgdir#1}
}

\newcommand\imgsvg[2]{
       \begin{center}   \includegraphics[width=#2\paperwidth]{\imgsvgdir#1} \end{center}
}


\newcommand\Lien[2]{
     \href{#1}{#2 \tiny \faExternalLink}
}
\newcommand\mcLien[2]{
     \href{https~://www.maths-cours.fr/#1}{#2 \tiny \faExternalLink}
}

\newcommand{\euro}{\eurologo{}}

%================================================================================================================================
%
% Macros - Environement
%
%================================================================================================================================

\newenvironment{tex}{ %
}
{%
}

\newenvironment{indente}{ %
	\setlength\parindent{10mm}
}

{
	\setlength\parindent{0mm}
}

\newenvironment{corrige}{%
     \needspace{3\baselineskip}
     \medskip
     \textbf{\textsc{Corrigé}}
     \medskip
}
{
}

\newenvironment{extern}{%
     \begin{center}
     }
     {
     \end{center}
}

\NewEnviron{code}{%
	\par
     \boite{gray}{\texttt{%
     \BODY
     }}
     \par
}

\newenvironment{vbloc}{% boite sans cadre empeche saut de page
     \begin{minipage}[t]{\linewidth}
     }
     {
     \end{minipage}
}
\NewEnviron{h2}{%
    \needspace{3\baselineskip}
    \vspace{0.6cm}
	\noindent \MakeUppercase{\sffamily \large \BODY}
	\vspace{1mm}\textcolor{mcgris}{\hrule}\vspace{0.4cm}
	\par
}{}

\NewEnviron{h3}{%
    \needspace{3\baselineskip}
	\vspace{5mm}
	\textsc{\BODY}
	\par
}

\NewEnviron{margeneg}{ %
\begin{addmargin}[-1cm]{0cm}
\BODY
\end{addmargin}
}

\NewEnviron{html}{%
}

\begin{document}
\meta{url}{/exercices/fonctions-bac-blanc-es-l-sujet-6-maths-cours-2018/}
\meta{pid}{10597}
\meta{titre}{Fonctions - Bac blanc ES/L Sujet 6 - Maths-cours 2018}
\meta{type}{exercices}
%
\begin{h2}Exercice 4 (5 points)\end{h2}
\par
On considère la fonction $f$ définie sur l'intervalle [0~;~5] par:
\par
\[ f(x) = 2\ln(x+1)-x+1. \]
\par
On a utilisé un logiciel de calcul formel pour déterminer la fonction dérivée $f'$, la fonction dérivée seconde $f''$ et une primitive $F$ de $f$ sur l'intervalle [0~;~5].
\par
On a obtenu les résultats suivants :
\par
\begin{center}
     \begin{extern}%width="300" alt="calcul formel "
          \begin{tabular}{|l|l|}\hline
               1&\textit{dériver} ( 2ln(x+1)-x+1 )\\
               & \\
               &$\textcolor{blue}{\rightarrow \quad \dfrac{-x+1}{x+1}}$ \\
               & \\ \hline
               2&\textit{dériver} ( (-x+1)/(x+1) )\\
               & \\
               &$\textcolor{blue}{\rightarrow \quad \dfrac{-2}{(x+1)^2}}$ \\
               & \\ \hline
               3&\textit{intégrer} ( 2ln(x+1)-x+1 )\\
               & \\
               &$\textcolor{blue}{\rightarrow \quad 2(x+1)\ln(x+1)-\dfrac{1}{2}x^2-x}$ \\
               & \\ \hline
          \end{tabular}
     \end{extern}
\end{center}
\par
\textbf{\textit{Dans les questions suivantes, on pourra utiliser tous les résultats fournis par le logiciel sans les avoir justifiés.}}
\par
\begin{enumerate}
     \item %1
     Dresser le tableau de variations de la fonction $f$ sur l'intervalle [0~;~5].
     \item %2
     Montrer que la fonction $f$ est concave sur l'intervalle [0~;~5].
     \item %3
     On rappelle que la valeur moyenne $m$ d'une fonction $f$ sur un intervalle [a~;~b] est donnée par la formule :
     \[ m=\dfrac{1}{b-a}\displaystyle\int_{a}^{b}f(t)\text{d}t. \]
     \par
     Déterminer la valeur exacte puis une valeur approchée à $10^{-3}$ de la valeur moyenne de la fonction $f$ sur l'intervalle [0~;~5].
     \item %4
     \begin{enumerate}[label=\alph*.]
          \item %4a
          Montrer que l'équation $f(x)=0$ admet une unique solution $\alpha$ sur l'intervalle [0~;~5] et que $4 < \alpha < 5$.
          \item %4b
          On a écrit l'algorithme suivant :
          \par
          \begin{center}
               \begin{extern}%width="440" alt="algorithme"
                    \begin{tabular}{|l|l|}\hline
                         Variables :	&$x$ est un nombre réel\\
                         &$y$ est un nombre réel\\
                         & \\
                         Initialisation: &$x$ prend la valeur $4$ \\
                         &$y$ prend la valeur $2\ln(x+1)-x+1$ \\
                         & \\
                         Traitement: &Tant que $y > 0$ faire : \\
                         &\quad$x$ prend la valeur $x+0,1$\\
                         &\quad$y$ prend la valeur $2\ln(x+1)-x+1$\\
                         &Fin Tant que\\
                         & \\
                         Sortie :	&Afficher $x$ \\
                         \hline
                    \end{tabular}
               \end{extern}
          \end{center}
          \par
          Recopier et compléter le tableau suivant, en ajoutant autant de colonnes que nécessaire. On arrondira les résultats au millième.
          \par
          \begin{center}
               \begin{tabular}{|l|c|c|c|}\hline %class="compact"
                    Valeur de $x$	&$4$	&	 $4,1$  &	 $\quad \cdots \quad$ \\ \hline
                    Valeur de $y$	&$0,219$	& $\quad \cdots \quad$ & $\quad \cdots \quad$ 	 \\ \hline
                    Condition $y > 0$	&vraie	& $\quad \cdots \quad$ & $\quad \cdots \quad$ 	\\ \hline
               \end{tabular}
          \end{center}
          \vspace{0.2cm}
          \item Quel est le résultat affiché par cet algorithme ? \\
          Interpréter ce résultat dans le cadre de l'exercice.
     \end{enumerate}
\end{enumerate}
\begin{corrige}
     \begin{enumerate}
          \item %1
          D'après les résultats fournis par le logiciel, la fonction $f$ est dérivable sur l'intervalle $[0~;~5]$ et :
          \[ f'(x)=\dfrac{-x+1}{x+1}. \]
          \par
          Le dénominateur est strictement positif sur l'intervalle $[0~;~5]$ (puisqu'alors $x+1 \geqslant 1$). $f'(x)$ est donc du signe de $-x+1$, c'est à dire nul pour ${x=1}$, strictement positif pour ${x < 1}$ et strictement négatif pour ${x > 1}$.
          \par
          Par ailleurs :
          \par
          $f(0)=2\ln1-0+1=1$ ;
          \par
          $f(1)=2\ln(1+1)-1+1=2\ln2$ ;
          \par
          $f(5)=2\ln(5+1)-5+1=2\ln6 - 4$.
          \par
          On obtient alors le tableau de variations suivant :
          \par
          %:-+-+-+-+- Engendré par : http://math.et.info.free.fr/TikZ/TableauxVariations/
          \begin{center}
               \begin{extern}%width="340" alt="tableau de variations"
                    \begin{tikzpicture}[scale=0.875]
                         % Styles
                         \tikzstyle{cadre}=[thin]
                         \tikzstyle{fleche}=[->,>=latex,thin]
                         \tikzstyle{nondefini}=[lightgray]
                         % Dimensions Modifiables
                         \def\Lrg{1.5}
                         \def\HtX{1}
                         \def\HtY{0.5}
                         % Dimensions Calculées
                         \def\lignex{-0.5*\HtX}
                         \def\lignef{-1.5*\HtX}
                         \def\separateur{-0.5*\Lrg}
                         % Largeur du tableau
                         \def\gauche{-1.5*\Lrg}
                         \def\droite{4.6*\Lrg}
                         % Hauteur du tableau
                         \def\haut{0.5*\HtX}
                         \def\bas{-2.5*\HtX-2*\HtY}
                         % Ligne de l'abscisse : x
                         \node at (-1*\Lrg,0) {$x$};
                         \node at (0*\Lrg,0) {$0$};
                         \node at (2*\Lrg,0) {$1$};
                         \node at (4*\Lrg,0) {$5$};
                         % Ligne de la dérivée : f'(x)
                         \node at (-1*\Lrg,-1*\HtX) {$f'(x)$};
                         \node at (0*\Lrg,-1*\HtX) {$\ $};
                         \node at (1*\Lrg,-1*\HtX) {$+$};
                         \node at (2*\Lrg,-1*\HtX) {$0$};
                         \node at (3*\Lrg,-1*\HtX) {$-$};
                         \node at (4*\Lrg,-1*\HtX) {$\ $};
                         % Ligne de la fonction : f(x)
                         \node  at (-1*\Lrg,{-2*\HtX+(-1)*\HtY}) {$f(x)$};
                         \node (f1) at (0*\Lrg,{-2*\HtX+(-2)*\HtY}) {$1$};
                         \node (f2) at (2*\Lrg,{-2*\HtX+(0)*\HtY}) {$2\ln2$};
                         \node (f3) at (3.9*\Lrg,{-2*\HtX+(-2)*\HtY}) {$2\ln6-4$};
                         % Flèches
                         \draw[fleche] (f1) -- (f2);
                         \draw[fleche] (f2) -- (f3);
                         % Encadrement
                         \draw[cadre] (\separateur,\haut) -- (\separateur,\bas);
                         \draw[cadre] (\gauche,\haut) rectangle  (\droite,\bas);
                         \draw[cadre] (\gauche,\lignex) -- (\droite,\lignex);
                         \draw[cadre] (\gauche,\lignef) -- (\droite,\lignef);
                    \end{tikzpicture}
               \end{extern}
          \end{center}
          %:-+-+-+-+- Fin
          \item %2
          Le logiciel de calcul formel indique que la fonction $f$ est deux fois dérivable et que :
          \[ f''(x)=\dfrac{-2}{(x+1)^2}. \]
          \par
          Le dénominateur est strictement positif sur l'intervalle $[0~;~5]$ et le numérateur strictement négatif.
          \par
          $f''$ est donc strictement négative sur l'intervalle $[0~;~5]$ et, par conséquent, la fonction $f$ est concave sur cet intervalle.
          \item %3
          D'après la formule rappelée dans l'énoncé, la valeur moyenne $m$ de la fonction $f$ sur l'intervalle [0~;~5] est :
          \[ m=\dfrac{1}{5}\displaystyle\int_{0}^{5}f(t)\text{d}t. \]
          \par
          Le logiciel de calcul formel indique que la fonction $F$ définie sur $[0~;~5]$ par :
          \[ F(x)=2(x+1)\ln(x+1)-\dfrac{1}{2}x^2-x \]
          est une primitive de la fonction $f$.
          \par
          Par conséquent :
          \par
          $m=\dfrac{1}{5}\left(F(5)-F(0)\right)$.
          \par
          Or :
          \par
          $F(5)=12\ln6-\dfrac{1}{2} \times 25 -5 =12\ln6-\dfrac{35}{2}$,
          \par
          $F(0)=2\ln1=0$.
          \par
          Donc :
          \par
          $m=\dfrac{1}{5}\left(12\ln6-\dfrac{35}{2}\right)=\dfrac{12}{5}\ln6-\dfrac{7}{2}$.
          \par
          Une valeur approchée de $m$ à $10^{-3}$ près est 0,8.
          \item %4
          \begin{enumerate}[label=\alph*.]
               \item %4a
               Remarquons d'abord que $f(1) \approx 1,386$ et $f(4) \approx 0,219$ sont strictement positifs alors que $f(5) \approx -0,416$ est strictement négatif.
               \par
               Sur l'intervalle $[0~;~1]$, $f(x)$ est supérieur ou égal à 1 donc strictement positif.
               L'équation $f(x)=0$ n'a donc aucune solution sur cet intervalle.
               \par
               Sur l'intervalle $[1~;~5]$, la fonction $f$ est \textbf{continue}, \textbf{strictement décroissante} et \textbf{change de signe}. Donc l'équation $f(x)=0$ admet une unique solution $\alpha$  sur cet intervalle.
               \par
               Comme $f(x)$ change de signe entre $x=4$ et $x=5$ : ${4 < \alpha < 5}$.
               \item %4b
               En faisant fonctionner l'algorithme on obtient le tableau suivant :
               \par
               \begin{center}
                    \begin{tabular}{|l|c|c|c|c|c|}\hline %class="compact"
                         $x$	&$4$	&	 $4,1$  &	 $4,2$ &	 $4,3$ &	 $4,4$ \\ \hline
                         $y$	&$0,219$ &$0,158$	& $0,097$ &$0,035$	&$-0,027$\\ \hline
                         $y > 0$	&vraie	&vraie	&vraie	&vraie	&fausse		\\ \hline
                    \end{tabular}
               \end{center}
               \item %4c
               D'après la question précédente, l'algorithme affiche la valeur 4,4.
               \par
               Dans cet algorithme, $x$ est incrémenté par pas de 0,1 et $y$ représente la valeur de $f(x)$.
               \par
               On sort de la boucle \og Tant que \fg{} quand $y \leqslant 0$, c'est à dire quand $f(x) \leqslant 0$.
               \par
               Le tableau de la question précédente montre que $f(4,3)$ est strictement positif tandis que $f(4,4)$ est strictement négatif. $f$ change donc de signe entre 4,3 et 4,4 ; par conséquent : ${4,3 < \alpha < 4,4}$.
               \par
               L'algorithme affiche donc une \textbf{valeur approchée à 0,1 près par excès de $\alpha$.}
               \par
          \end{enumerate}
          \par
     \end{enumerate}
\end{corrige}

\end{document}