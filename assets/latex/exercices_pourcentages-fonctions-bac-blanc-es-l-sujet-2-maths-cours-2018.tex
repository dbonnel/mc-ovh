\documentclass[a4paper]{article}

%================================================================================================================================
%
% Packages
%
%================================================================================================================================

\usepackage[T1]{fontenc} 	% pour caractères accentués
\usepackage[utf8]{inputenc}  % encodage utf8
\usepackage[french]{babel}	% langue : français
\usepackage{fourier}			% caractères plus lisibles
\usepackage[dvipsnames]{xcolor} % couleurs
\usepackage{fancyhdr}		% réglage header footer
\usepackage{needspace}		% empêcher sauts de page mal placés
\usepackage{graphicx}		% pour inclure des graphiques
\usepackage{enumitem,cprotect}		% personnalise les listes d'items (nécessaire pour ol, al ...)
\usepackage{hyperref}		% Liens hypertexte
\usepackage{pstricks,pst-all,pst-node,pstricks-add,pst-math,pst-plot,pst-tree,pst-eucl} % pstricks
\usepackage[a4paper,includeheadfoot,top=2cm,left=3cm, bottom=2cm,right=3cm]{geometry} % marges etc.
\usepackage{comment}			% commentaires multilignes
\usepackage{amsmath,environ} % maths (matrices, etc.)
\usepackage{amssymb,makeidx}
\usepackage{bm}				% bold maths
\usepackage{tabularx}		% tableaux
\usepackage{colortbl}		% tableaux en couleur
\usepackage{fontawesome}		% Fontawesome
\usepackage{environ}			% environment with command
\usepackage{fp}				% calculs pour ps-tricks
\usepackage{multido}			% pour ps tricks
\usepackage[np]{numprint}	% formattage nombre
\usepackage{tikz,tkz-tab} 			% package principal TikZ
\usepackage{pgfplots}   % axes
\usepackage{mathrsfs}    % cursives
\usepackage{calc}			% calcul taille boites
\usepackage[scaled=0.875]{helvet} % font sans serif
\usepackage{svg} % svg
\usepackage{scrextend} % local margin
\usepackage{scratch} %scratch
\usepackage{multicol} % colonnes
%\usepackage{infix-RPN,pst-func} % formule en notation polanaise inversée
\usepackage{listings}

%================================================================================================================================
%
% Réglages de base
%
%================================================================================================================================

\lstset{
language=Python,   % R code
literate=
{á}{{\'a}}1
{à}{{\`a}}1
{ã}{{\~a}}1
{é}{{\'e}}1
{è}{{\`e}}1
{ê}{{\^e}}1
{í}{{\'i}}1
{ó}{{\'o}}1
{õ}{{\~o}}1
{ú}{{\'u}}1
{ü}{{\"u}}1
{ç}{{\c{c}}}1
{~}{{ }}1
}


\definecolor{codegreen}{rgb}{0,0.6,0}
\definecolor{codegray}{rgb}{0.5,0.5,0.5}
\definecolor{codepurple}{rgb}{0.58,0,0.82}
\definecolor{backcolour}{rgb}{0.95,0.95,0.92}

\lstdefinestyle{mystyle}{
    backgroundcolor=\color{backcolour},   
    commentstyle=\color{codegreen},
    keywordstyle=\color{magenta},
    numberstyle=\tiny\color{codegray},
    stringstyle=\color{codepurple},
    basicstyle=\ttfamily\footnotesize,
    breakatwhitespace=false,         
    breaklines=true,                 
    captionpos=b,                    
    keepspaces=true,                 
    numbers=left,                    
xleftmargin=2em,
framexleftmargin=2em,            
    showspaces=false,                
    showstringspaces=false,
    showtabs=false,                  
    tabsize=2,
    upquote=true
}

\lstset{style=mystyle}


\lstset{style=mystyle}
\newcommand{\imgdir}{C:/laragon/www/newmc/assets/imgsvg/}
\newcommand{\imgsvgdir}{C:/laragon/www/newmc/assets/imgsvg/}

\definecolor{mcgris}{RGB}{220, 220, 220}% ancien~; pour compatibilité
\definecolor{mcbleu}{RGB}{52, 152, 219}
\definecolor{mcvert}{RGB}{125, 194, 70}
\definecolor{mcmauve}{RGB}{154, 0, 215}
\definecolor{mcorange}{RGB}{255, 96, 0}
\definecolor{mcturquoise}{RGB}{0, 153, 153}
\definecolor{mcrouge}{RGB}{255, 0, 0}
\definecolor{mclightvert}{RGB}{205, 234, 190}

\definecolor{gris}{RGB}{220, 220, 220}
\definecolor{bleu}{RGB}{52, 152, 219}
\definecolor{vert}{RGB}{125, 194, 70}
\definecolor{mauve}{RGB}{154, 0, 215}
\definecolor{orange}{RGB}{255, 96, 0}
\definecolor{turquoise}{RGB}{0, 153, 153}
\definecolor{rouge}{RGB}{255, 0, 0}
\definecolor{lightvert}{RGB}{205, 234, 190}
\setitemize[0]{label=\color{lightvert}  $\bullet$}

\pagestyle{fancy}
\renewcommand{\headrulewidth}{0.2pt}
\fancyhead[L]{maths-cours.fr}
\fancyhead[R]{\thepage}
\renewcommand{\footrulewidth}{0.2pt}
\fancyfoot[C]{}

\newcolumntype{C}{>{\centering\arraybackslash}X}
\newcolumntype{s}{>{\hsize=.35\hsize\arraybackslash}X}

\setlength{\parindent}{0pt}		 
\setlength{\parskip}{3mm}
\setlength{\headheight}{1cm}

\def\ebook{ebook}
\def\book{book}
\def\web{web}
\def\type{web}

\newcommand{\vect}[1]{\overrightarrow{\,\mathstrut#1\,}}

\def\Oij{$\left(\text{O}~;~\vect{\imath},~\vect{\jmath}\right)$}
\def\Oijk{$\left(\text{O}~;~\vect{\imath},~\vect{\jmath},~\vect{k}\right)$}
\def\Ouv{$\left(\text{O}~;~\vect{u},~\vect{v}\right)$}

\hypersetup{breaklinks=true, colorlinks = true, linkcolor = OliveGreen, urlcolor = OliveGreen, citecolor = OliveGreen, pdfauthor={Didier BONNEL - https://www.maths-cours.fr} } % supprime les bordures autour des liens

\renewcommand{\arg}[0]{\text{arg}}

\everymath{\displaystyle}

%================================================================================================================================
%
% Macros - Commandes
%
%================================================================================================================================

\newcommand\meta[2]{    			% Utilisé pour créer le post HTML.
	\def\titre{titre}
	\def\url{url}
	\def\arg{#1}
	\ifx\titre\arg
		\newcommand\maintitle{#2}
		\fancyhead[L]{#2}
		{\Large\sffamily \MakeUppercase{#2}}
		\vspace{1mm}\textcolor{mcvert}{\hrule}
	\fi 
	\ifx\url\arg
		\fancyfoot[L]{\href{https://www.maths-cours.fr#2}{\black \footnotesize{https://www.maths-cours.fr#2}}}
	\fi 
}


\newcommand\TitreC[1]{    		% Titre centré
     \needspace{3\baselineskip}
     \begin{center}\textbf{#1}\end{center}
}

\newcommand\newpar{    		% paragraphe
     \par
}

\newcommand\nosp {    		% commande vide (pas d'espace)
}
\newcommand{\id}[1]{} %ignore

\newcommand\boite[2]{				% Boite simple sans titre
	\vspace{5mm}
	\setlength{\fboxrule}{0.2mm}
	\setlength{\fboxsep}{5mm}	
	\fcolorbox{#1}{#1!3}{\makebox[\linewidth-2\fboxrule-2\fboxsep]{
  		\begin{minipage}[t]{\linewidth-2\fboxrule-4\fboxsep}\setlength{\parskip}{3mm}
  			 #2
  		\end{minipage}
	}}
	\vspace{5mm}
}

\newcommand\CBox[4]{				% Boites
	\vspace{5mm}
	\setlength{\fboxrule}{0.2mm}
	\setlength{\fboxsep}{5mm}
	
	\fcolorbox{#1}{#1!3}{\makebox[\linewidth-2\fboxrule-2\fboxsep]{
		\begin{minipage}[t]{1cm}\setlength{\parskip}{3mm}
	  		\textcolor{#1}{\LARGE{#2}}    
 	 	\end{minipage}  
  		\begin{minipage}[t]{\linewidth-2\fboxrule-4\fboxsep}\setlength{\parskip}{3mm}
			\raisebox{1.2mm}{\normalsize\sffamily{\textcolor{#1}{#3}}}						
  			 #4
  		\end{minipage}
	}}
	\vspace{5mm}
}

\newcommand\cadre[3]{				% Boites convertible html
	\par
	\vspace{2mm}
	\setlength{\fboxrule}{0.1mm}
	\setlength{\fboxsep}{5mm}
	\fcolorbox{#1}{white}{\makebox[\linewidth-2\fboxrule-2\fboxsep]{
  		\begin{minipage}[t]{\linewidth-2\fboxrule-4\fboxsep}\setlength{\parskip}{3mm}
			\raisebox{-2.5mm}{\sffamily \small{\textcolor{#1}{\MakeUppercase{#2}}}}		
			\par		
  			 #3
 	 		\end{minipage}
	}}
		\vspace{2mm}
	\par
}

\newcommand\bloc[3]{				% Boites convertible html sans bordure
     \needspace{2\baselineskip}
     {\sffamily \small{\textcolor{#1}{\MakeUppercase{#2}}}}    
		\par		
  			 #3
		\par
}

\newcommand\CHelp[1]{
     \CBox{Plum}{\faInfoCircle}{À RETENIR}{#1}
}

\newcommand\CUp[1]{
     \CBox{NavyBlue}{\faThumbsOUp}{EN PRATIQUE}{#1}
}

\newcommand\CInfo[1]{
     \CBox{Sepia}{\faArrowCircleRight}{REMARQUE}{#1}
}

\newcommand\CRedac[1]{
     \CBox{PineGreen}{\faEdit}{BIEN R\'EDIGER}{#1}
}

\newcommand\CError[1]{
     \CBox{Red}{\faExclamationTriangle}{ATTENTION}{#1}
}

\newcommand\TitreExo[2]{
\needspace{4\baselineskip}
 {\sffamily\large EXERCICE #1\ (\emph{#2 points})}
\vspace{5mm}
}

\newcommand\img[2]{
          \includegraphics[width=#2\paperwidth]{\imgdir#1}
}

\newcommand\imgsvg[2]{
       \begin{center}   \includegraphics[width=#2\paperwidth]{\imgsvgdir#1} \end{center}
}


\newcommand\Lien[2]{
     \href{#1}{#2 \tiny \faExternalLink}
}
\newcommand\mcLien[2]{
     \href{https~://www.maths-cours.fr/#1}{#2 \tiny \faExternalLink}
}

\newcommand{\euro}{\eurologo{}}

%================================================================================================================================
%
% Macros - Environement
%
%================================================================================================================================

\newenvironment{tex}{ %
}
{%
}

\newenvironment{indente}{ %
	\setlength\parindent{10mm}
}

{
	\setlength\parindent{0mm}
}

\newenvironment{corrige}{%
     \needspace{3\baselineskip}
     \medskip
     \textbf{\textsc{Corrigé}}
     \medskip
}
{
}

\newenvironment{extern}{%
     \begin{center}
     }
     {
     \end{center}
}

\NewEnviron{code}{%
	\par
     \boite{gray}{\texttt{%
     \BODY
     }}
     \par
}

\newenvironment{vbloc}{% boite sans cadre empeche saut de page
     \begin{minipage}[t]{\linewidth}
     }
     {
     \end{minipage}
}
\NewEnviron{h2}{%
    \needspace{3\baselineskip}
    \vspace{0.6cm}
	\noindent \MakeUppercase{\sffamily \large \BODY}
	\vspace{1mm}\textcolor{mcgris}{\hrule}\vspace{0.4cm}
	\par
}{}

\NewEnviron{h3}{%
    \needspace{3\baselineskip}
	\vspace{5mm}
	\textsc{\BODY}
	\par
}

\NewEnviron{margeneg}{ %
\begin{addmargin}[-1cm]{0cm}
\BODY
\end{addmargin}
}

\NewEnviron{html}{%
}

\begin{document}
\meta{url}{/exercices/pourcentages-fonctions-bac-blanc-es-l-sujet-2-maths-cours-2018/}
\meta{pid}{10442}
\meta{titre}{Pourcentages - Fonctions - Bac blanc ES/L Sujet 2 - Maths-cours 2018}
\meta{type}{exercices}
%
\begin{h2}Exercice 2 (6 points)\end{h2}
\par
Le propriétaire d'un restaurant a constaté que, lorsque le prix de son menu était fixé à 25~euros, il accueillait 20~clients, et qu'à chaque baisse de 1~euro, il attirait 2~clients supplémentaires.
%============================================================================================================================
%
\TitreC{Partie A}
%
%============================================================================================================================
\begin{enumerate}
     \item De quel pourcentage le prix a-t-il baissé lorsqu'il est passé de 25 à 24~euros ?
     \item Quel est le pourcentage d'augmentation du nombre de clients lorsque celui-ci passe de 20 à 22~clients ?
     \item L'\textbf{élasticité} de la demande par rapport au prix est le rapport :
     \[ e=\dfrac{V_c}{V_p} \]
     où $V_c$ désigne la variation en pourcentage du nombre de clients et $V_p$ la variation en pourcentage du prix du menu. \\
     Le nombre de clients variant en sens contraire du prix, ce rapport sera \textbf{négatif}.
     \par
     \begin{enumerate}[label=\alph*.]
          \item
          Montrer que l'élasticité de la demande par rapport au prix lorsque le prix du menu passe de 25 à 24 euros est égale à -2,5.
          \item
          Déterminer le nombre de clients pour un menu à 20 euros puis pour un menu à 19~euros.\\
          En déduire l'élasticité de la demande par rapport au prix lorsque le prix du menu passe de 20 à 19~euros.
     \end{enumerate}
     \item La recette est égale au produit du nombre de clients par le prix du menu.
     Calculer le montant de la recette lorsque le prix du menu est 25~euros puis lorsque le prix du menu est 20~euros.
\end{enumerate}
%============================================================================================================================
%
\TitreC{Partie B}
%
%============================================================================================================================
\begin{enumerate}
     \item On admet que la fonction $f$ donnant le nombre de clients en fonction du prix du menu $x$ est définie sur l'intervalle $[10~;~25]$ par :
     \[f(x)=70-2x.\]
     \par
     Exprimer la recette du restaurant en fonction de $x$.
     \par
     Pour quel prix $x_0$ la recette est-elle maximale ?
     \item Pour $x \in [10~;~25]$, on note $e(x)$ l'élasticité de la demande par rapport au prix du menu lorsque ce dernier passe de $x$ à $x-1$ euros.
     Montrer que $e(x)=x\ \dfrac{f'(x)}{f(x)}$.
     \item Calculer $e'(x)$ et en déduire le sens de variation de la fonction $e$ sur l'intervalle $[10~;~25]$.
     Montrer que $e(x_0)=-1$.
     \par
\end{enumerate}
\begin{corrige}
     %============================================================================================================================
     %
     \TitreC{Partie A}
     %
     %============================================================================================================================
     \par
     \begin{enumerate}
          \item %1
          \textbf{Lorsque le prix passe de 25 à 24 euros}, le pourcentage de variation est :
          \par
          $V_p=\dfrac{24-25}{25}=-\dfrac{1}{25}=-0,04=-4\%$.
          \par
          Le prix a \textbf{baissé de 4\%} lorsqu'il est passé de 25 à 24 euros.
          \cadre{rouge}{À retenir}{
               Lorsqu'une valeur passe de $V_0$ à $V_1$, le pourcentage de variation est :
               \[ t = \dfrac{V_1-V_0}{V_0}. \]
          }
          \item %2
          Lorsque le nombre de clients passe de 20 à 22, le pourcentage de variation est :
          \par
          $ V_c=\dfrac{22-20}{20}=\dfrac{2}{20}=0,1=10\%. $
          \par
          Le nombre de clients a \textbf{augmenté de 10\%} lorsqu'il est passé de 20 à 22.
          \item %3
          \begin{enumerate}[label=\alph*.]
               \item %3a
               D'après la définition de l'énoncé, l'élasticité $e$ de la demande par rapport au prix est le quotient de la variation en pourcentage du nombre de clients par la variation en pourcentage du prix du menu.
               \par
               C'est à dire :
               \par
               $e=\dfrac{V_c}{V_p}=\dfrac{0,1}{-0,04}=-\dfrac{10}{4}=-2,5$.
               \par
               L'élasticité de la demande par rapport au prix, lorsque le prix du menu passe de 25 à 24~euros, est de $-2,5$.
               \item %3b
               \`A chaque baisse de 1 euro, le restaurant attire 2 clients supplémentaires. Le tableau ci-après indique le nombre de clients en fonction du prix :
               \par
               \begin{center}
                    \begin{tabular}{|l|c|c|c|c|c|c|c|c|}%class="compact"
                         \hline
                         Prix & 25 & 24 & 23 & 22 & 21 & 20 & 19 & $\cdots$ \\
                         \hline
                         Nb clients & 20 & 22 & 24 & 26 & 28 & 30 & 32 & $\cdots$ \\
                         \hline
                    \end{tabular}
               \end{center}
               \par
               Pour un menu à 20 euros, le nombre de clients est \textbf{30}, et, pour un menu à 19 euros, le nombre de clients est \textbf{32}.
               \par
               \textbf{Lorsque le prix passe de 20 à 19 euros}, le pourcentage de variation du prix est :
               \par
               $ V_p=\dfrac{19-20}{20}=-\dfrac{1}{20} $.
               \par
               Le nombre de clients passe alors de 30 à 32 ; le pourcentage de variation du nombre de clients est donc :
               \par
               $V_c=\dfrac{32-30}{30}=\dfrac{2}{30}=\dfrac{1}{15}$.
               \par
               L'élasticité $e$ de la demande par rapport au prix vaut alors :
               \par
               $e=\dfrac{V_c}{V_p}=\dfrac{\dfrac{1}{15}}{-\dfrac{1}{20}}=-\dfrac{20}{15}=-\dfrac{4}{3}$.
               \par
               L'élasticité de la demande par rapport au prix lorsque le prix du menu passe de 20 à 19~euros est de $-\dfrac{4}{3}$.
               \par
          \end{enumerate}
          \item %4
          Lorsque le prix du menu est égal à 25~euros, le restaurant accueille {20~clients}.
          \par
          La recette vaut :
          \par
          $R_{25}=25 \times 20 = 500$ euros.
          \par
          La recette est de \textbf{500 euros} lorsque le prix du menu est 25~euros.
          \par
          Lorsque le prix du menu est égal à 20 euros, le restaurant accueille 30~clients.
          \par
          La recette vaut alors :
          \par
          $R_{20}=20 \times 30 = 600$ euros.
          \par
          La recette est de \textbf{600 euros} lorsque le prix du menu est 20~euros.
          \par
     \end{enumerate}
     \par
     %============================================================================================================================
     %
     \TitreC{Partie B}
     %
     %============================================================================================================================
     \par
     \begin{enumerate}
          \item %1
          La recette est égal au produit du nombre de clients par le prix du menu, par conséquent :
          \par
          $R(x)=xf(x)=x(70-2x)$\nosp$=-2x^2+70x$.
          \par
          La fonction $R$ est une fonction polynôme du second degré. Le coefficient de $x^2$ est -2 ; il est négatif.
          \par
          $R$ admet un maximum pour :
          \par
          $x_0=-\dfrac{b}{2a}=-\dfrac{70}{-4}=17,5$.
          \par

          La recette est maximale lorsque le prix du menu est égal à \textbf{17,50~euros}.
          \cadre{rouge}{À retenir}{
               Une fonction polynôme du second degré ${x \longmapsto ax^2+bx=c}$ ($a \neq 0$) admet un extremum pour :
               \[ x_0=-\dfrac{b}{2a}. \]
               \par
               Cet extremum est :
               \begin{itemize}
                    \item
                    un \textbf{minimum} si $a>0$,
                    \item
                    un \textbf{maximum} si $a<0$.
               \end{itemize}
          }
          \cadre{bleu}{Remarque}{
               Il est aussi possible de calculer $R'(x)$, d'étudier son signe et de dresser le tableau de variations de $R$.
               \par
               Toutefois, la propriété employée ici (et vue en classe de Seconde) permet d'aboutir au résultat plus rapidement.
          }
          \item %2
          La variation en pourcentage du prix du menu lorsqu'il passe de $x$ à $x-1$ euros est :
          \par
          $V_p(x)=\dfrac{(x-1)-x}{x}=-\dfrac{1}{x}$.
          \par
          Lorsque le prix du menu passe de $x$ à $x-1$ euros, le nombre de clients passe de $f(x)$ à $f(x-1)$.
          \par
          La variation en pourcentage du nombre de clients est alors :
          \par
          $V_c(x)=\dfrac{f(x-1)-f(x)}{f(x)}$\\
          $\phantom{V_c(x)}=\dfrac{70-2(x-1)-(70-2x)}{70-2x}$\\
          $\phantom{V_c(x)}=\dfrac{70-2x+2-70+2x}{70-2x}$\\
          $\phantom{V_c(x)}=\dfrac{2}{70-2x}.$
          \par
          L'élasticité vaut donc :
          \par
          $e(x)=\dfrac{V_c(x)}{V_p(x)}$//
          $\phantom{e(x)}=\dfrac{\dfrac{2}{70-2x}}{-\dfrac{1}{x}}$\\
          $\phantom{e(x)}=\dfrac{2}{70-2x} \times \dfrac{-x}{1}$\\
          $\phantom{e(x)}=\dfrac{-2x}{70-2x}.$
          \par

          Par ailleurs :
          \par
          $f(x)=70-2x$ donc $f'(x)=-2$ ;
          \par
          $x\ \dfrac{f'(x)}{f(x)} = x \times \dfrac{-2}{70-2x}=\dfrac{-2x}{70-2x}$.
          \par
          On a donc bien pour tout réel $x$ de l'intervalle $[10~;~25]$ :
          \[ e(x)=x\ \dfrac{f'(x)}{f(x)}. \]
          \cadre{rouge}{Bien rédiger}{
               Pour démontrer l'égalité $e(x)=x\ \dfrac{f'(x)}{f(x)}$, il est incorrect de présenter les calculs en partant de cette égalité (puisque l'on n'a pas encore démontré qu'elle était vraie...).
               \par
               Ici, on calcule \textbf{séparément} $e(x)$ et $x\ \dfrac{f'(x)}{f(x)}$ et on montre que l'on aboutit au même résultat.
          }
          \item
          D'après la question précédente, $e$ est une fonction rationnelle définie et dérivable sur l'intervalle $[10~;~25]$ telle que:
          \[ e(x)=\dfrac{-2x}{70-2x} \]
          Posons :
          \par
          $u(x)=-2x$ ;
          \par
          $v(x)=70-2x$ ;
          \par
          alors :
          \par
          $u'(x)=-2$ ;
          \par
          $v'(x)=-2$.
          \par
          Par conséquent :
          \par
          $e'(x)= \dfrac{u'(x)v(x)-u(x)v'(x)}{v(x)^2}$\\
          $\phantom{e'(x)}=\dfrac{-2(70-2x)-(-2x) \times (-2)}{(70-2x)^2}$\\
          $\phantom{e'(x)}= \dfrac{-140+4x-4x}{(70-2x)^2}$\\
          $\phantom{e'(x)}= \dfrac{-140}{(70-2x)^2}.$
          \par
          Le numérateur est strictement négatif et le dénominateur est strictement positif sur l'intervalle $[10~;~25]$ donc $e'$ est strictement négative et la fonction $e$ est strictement \textbf{décroissante} sur l'intervalle $[10~;~25]$.
          \par
          Enfin :
          \par
          $e(x_0)=e(17,5)=\dfrac{-2 \times 17,5}{70-2 \times 17,5}$\nosp$=\dfrac{-35}{35}=-1$.
          \par
     \end{enumerate}
\end{corrige}

\end{document}