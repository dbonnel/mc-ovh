\documentclass[a4paper]{article}

%================================================================================================================================
%
% Packages
%
%================================================================================================================================

\usepackage[T1]{fontenc} 	% pour caractères accentués
\usepackage[utf8]{inputenc}  % encodage utf8
\usepackage[french]{babel}	% langue : français
\usepackage{fourier}			% caractères plus lisibles
\usepackage[dvipsnames]{xcolor} % couleurs
\usepackage{fancyhdr}		% réglage header footer
\usepackage{needspace}		% empêcher sauts de page mal placés
\usepackage{graphicx}		% pour inclure des graphiques
\usepackage{enumitem,cprotect}		% personnalise les listes d'items (nécessaire pour ol, al ...)
\usepackage{hyperref}		% Liens hypertexte
\usepackage{pstricks,pst-all,pst-node,pstricks-add,pst-math,pst-plot,pst-tree,pst-eucl} % pstricks
\usepackage[a4paper,includeheadfoot,top=2cm,left=3cm, bottom=2cm,right=3cm]{geometry} % marges etc.
\usepackage{comment}			% commentaires multilignes
\usepackage{amsmath,environ} % maths (matrices, etc.)
\usepackage{amssymb,makeidx}
\usepackage{bm}				% bold maths
\usepackage{tabularx}		% tableaux
\usepackage{colortbl}		% tableaux en couleur
\usepackage{fontawesome}		% Fontawesome
\usepackage{environ}			% environment with command
\usepackage{fp}				% calculs pour ps-tricks
\usepackage{multido}			% pour ps tricks
\usepackage[np]{numprint}	% formattage nombre
\usepackage{tikz,tkz-tab} 			% package principal TikZ
\usepackage{pgfplots}   % axes
\usepackage{mathrsfs}    % cursives
\usepackage{calc}			% calcul taille boites
\usepackage[scaled=0.875]{helvet} % font sans serif
\usepackage{svg} % svg
\usepackage{scrextend} % local margin
\usepackage{scratch} %scratch
\usepackage{multicol} % colonnes
%\usepackage{infix-RPN,pst-func} % formule en notation polanaise inversée
\usepackage{listings}

%================================================================================================================================
%
% Réglages de base
%
%================================================================================================================================

\lstset{
language=Python,   % R code
literate=
{á}{{\'a}}1
{à}{{\`a}}1
{ã}{{\~a}}1
{é}{{\'e}}1
{è}{{\`e}}1
{ê}{{\^e}}1
{í}{{\'i}}1
{ó}{{\'o}}1
{õ}{{\~o}}1
{ú}{{\'u}}1
{ü}{{\"u}}1
{ç}{{\c{c}}}1
{~}{{ }}1
}


\definecolor{codegreen}{rgb}{0,0.6,0}
\definecolor{codegray}{rgb}{0.5,0.5,0.5}
\definecolor{codepurple}{rgb}{0.58,0,0.82}
\definecolor{backcolour}{rgb}{0.95,0.95,0.92}

\lstdefinestyle{mystyle}{
    backgroundcolor=\color{backcolour},   
    commentstyle=\color{codegreen},
    keywordstyle=\color{magenta},
    numberstyle=\tiny\color{codegray},
    stringstyle=\color{codepurple},
    basicstyle=\ttfamily\footnotesize,
    breakatwhitespace=false,         
    breaklines=true,                 
    captionpos=b,                    
    keepspaces=true,                 
    numbers=left,                    
xleftmargin=2em,
framexleftmargin=2em,            
    showspaces=false,                
    showstringspaces=false,
    showtabs=false,                  
    tabsize=2,
    upquote=true
}

\lstset{style=mystyle}


\lstset{style=mystyle}
\newcommand{\imgdir}{C:/laragon/www/newmc/assets/imgsvg/}
\newcommand{\imgsvgdir}{C:/laragon/www/newmc/assets/imgsvg/}

\definecolor{mcgris}{RGB}{220, 220, 220}% ancien~; pour compatibilité
\definecolor{mcbleu}{RGB}{52, 152, 219}
\definecolor{mcvert}{RGB}{125, 194, 70}
\definecolor{mcmauve}{RGB}{154, 0, 215}
\definecolor{mcorange}{RGB}{255, 96, 0}
\definecolor{mcturquoise}{RGB}{0, 153, 153}
\definecolor{mcrouge}{RGB}{255, 0, 0}
\definecolor{mclightvert}{RGB}{205, 234, 190}

\definecolor{gris}{RGB}{220, 220, 220}
\definecolor{bleu}{RGB}{52, 152, 219}
\definecolor{vert}{RGB}{125, 194, 70}
\definecolor{mauve}{RGB}{154, 0, 215}
\definecolor{orange}{RGB}{255, 96, 0}
\definecolor{turquoise}{RGB}{0, 153, 153}
\definecolor{rouge}{RGB}{255, 0, 0}
\definecolor{lightvert}{RGB}{205, 234, 190}
\setitemize[0]{label=\color{lightvert}  $\bullet$}

\pagestyle{fancy}
\renewcommand{\headrulewidth}{0.2pt}
\fancyhead[L]{maths-cours.fr}
\fancyhead[R]{\thepage}
\renewcommand{\footrulewidth}{0.2pt}
\fancyfoot[C]{}

\newcolumntype{C}{>{\centering\arraybackslash}X}
\newcolumntype{s}{>{\hsize=.35\hsize\arraybackslash}X}

\setlength{\parindent}{0pt}		 
\setlength{\parskip}{3mm}
\setlength{\headheight}{1cm}

\def\ebook{ebook}
\def\book{book}
\def\web{web}
\def\type{web}

\newcommand{\vect}[1]{\overrightarrow{\,\mathstrut#1\,}}

\def\Oij{$\left(\text{O}~;~\vect{\imath},~\vect{\jmath}\right)$}
\def\Oijk{$\left(\text{O}~;~\vect{\imath},~\vect{\jmath},~\vect{k}\right)$}
\def\Ouv{$\left(\text{O}~;~\vect{u},~\vect{v}\right)$}

\hypersetup{breaklinks=true, colorlinks = true, linkcolor = OliveGreen, urlcolor = OliveGreen, citecolor = OliveGreen, pdfauthor={Didier BONNEL - https://www.maths-cours.fr} } % supprime les bordures autour des liens

\renewcommand{\arg}[0]{\text{arg}}

\everymath{\displaystyle}

%================================================================================================================================
%
% Macros - Commandes
%
%================================================================================================================================

\newcommand\meta[2]{    			% Utilisé pour créer le post HTML.
	\def\titre{titre}
	\def\url{url}
	\def\arg{#1}
	\ifx\titre\arg
		\newcommand\maintitle{#2}
		\fancyhead[L]{#2}
		{\Large\sffamily \MakeUppercase{#2}}
		\vspace{1mm}\textcolor{mcvert}{\hrule}
	\fi 
	\ifx\url\arg
		\fancyfoot[L]{\href{https://www.maths-cours.fr#2}{\black \footnotesize{https://www.maths-cours.fr#2}}}
	\fi 
}


\newcommand\TitreC[1]{    		% Titre centré
     \needspace{3\baselineskip}
     \begin{center}\textbf{#1}\end{center}
}

\newcommand\newpar{    		% paragraphe
     \par
}

\newcommand\nosp {    		% commande vide (pas d'espace)
}
\newcommand{\id}[1]{} %ignore

\newcommand\boite[2]{				% Boite simple sans titre
	\vspace{5mm}
	\setlength{\fboxrule}{0.2mm}
	\setlength{\fboxsep}{5mm}	
	\fcolorbox{#1}{#1!3}{\makebox[\linewidth-2\fboxrule-2\fboxsep]{
  		\begin{minipage}[t]{\linewidth-2\fboxrule-4\fboxsep}\setlength{\parskip}{3mm}
  			 #2
  		\end{minipage}
	}}
	\vspace{5mm}
}

\newcommand\CBox[4]{				% Boites
	\vspace{5mm}
	\setlength{\fboxrule}{0.2mm}
	\setlength{\fboxsep}{5mm}
	
	\fcolorbox{#1}{#1!3}{\makebox[\linewidth-2\fboxrule-2\fboxsep]{
		\begin{minipage}[t]{1cm}\setlength{\parskip}{3mm}
	  		\textcolor{#1}{\LARGE{#2}}    
 	 	\end{minipage}  
  		\begin{minipage}[t]{\linewidth-2\fboxrule-4\fboxsep}\setlength{\parskip}{3mm}
			\raisebox{1.2mm}{\normalsize\sffamily{\textcolor{#1}{#3}}}						
  			 #4
  		\end{minipage}
	}}
	\vspace{5mm}
}

\newcommand\cadre[3]{				% Boites convertible html
	\par
	\vspace{2mm}
	\setlength{\fboxrule}{0.1mm}
	\setlength{\fboxsep}{5mm}
	\fcolorbox{#1}{white}{\makebox[\linewidth-2\fboxrule-2\fboxsep]{
  		\begin{minipage}[t]{\linewidth-2\fboxrule-4\fboxsep}\setlength{\parskip}{3mm}
			\raisebox{-2.5mm}{\sffamily \small{\textcolor{#1}{\MakeUppercase{#2}}}}		
			\par		
  			 #3
 	 		\end{minipage}
	}}
		\vspace{2mm}
	\par
}

\newcommand\bloc[3]{				% Boites convertible html sans bordure
     \needspace{2\baselineskip}
     {\sffamily \small{\textcolor{#1}{\MakeUppercase{#2}}}}    
		\par		
  			 #3
		\par
}

\newcommand\CHelp[1]{
     \CBox{Plum}{\faInfoCircle}{À RETENIR}{#1}
}

\newcommand\CUp[1]{
     \CBox{NavyBlue}{\faThumbsOUp}{EN PRATIQUE}{#1}
}

\newcommand\CInfo[1]{
     \CBox{Sepia}{\faArrowCircleRight}{REMARQUE}{#1}
}

\newcommand\CRedac[1]{
     \CBox{PineGreen}{\faEdit}{BIEN R\'EDIGER}{#1}
}

\newcommand\CError[1]{
     \CBox{Red}{\faExclamationTriangle}{ATTENTION}{#1}
}

\newcommand\TitreExo[2]{
\needspace{4\baselineskip}
 {\sffamily\large EXERCICE #1\ (\emph{#2 points})}
\vspace{5mm}
}

\newcommand\img[2]{
          \includegraphics[width=#2\paperwidth]{\imgdir#1}
}

\newcommand\imgsvg[2]{
       \begin{center}   \includegraphics[width=#2\paperwidth]{\imgsvgdir#1} \end{center}
}


\newcommand\Lien[2]{
     \href{#1}{#2 \tiny \faExternalLink}
}
\newcommand\mcLien[2]{
     \href{https~://www.maths-cours.fr/#1}{#2 \tiny \faExternalLink}
}

\newcommand{\euro}{\eurologo{}}

%================================================================================================================================
%
% Macros - Environement
%
%================================================================================================================================

\newenvironment{tex}{ %
}
{%
}

\newenvironment{indente}{ %
	\setlength\parindent{10mm}
}

{
	\setlength\parindent{0mm}
}

\newenvironment{corrige}{%
     \needspace{3\baselineskip}
     \medskip
     \textbf{\textsc{Corrigé}}
     \medskip
}
{
}

\newenvironment{extern}{%
     \begin{center}
     }
     {
     \end{center}
}

\NewEnviron{code}{%
	\par
     \boite{gray}{\texttt{%
     \BODY
     }}
     \par
}

\newenvironment{vbloc}{% boite sans cadre empeche saut de page
     \begin{minipage}[t]{\linewidth}
     }
     {
     \end{minipage}
}
\NewEnviron{h2}{%
    \needspace{3\baselineskip}
    \vspace{0.6cm}
	\noindent \MakeUppercase{\sffamily \large \BODY}
	\vspace{1mm}\textcolor{mcgris}{\hrule}\vspace{0.4cm}
	\par
}{}

\NewEnviron{h3}{%
    \needspace{3\baselineskip}
	\vspace{5mm}
	\textsc{\BODY}
	\par
}

\NewEnviron{margeneg}{ %
\begin{addmargin}[-1cm]{0cm}
\BODY
\end{addmargin}
}

\NewEnviron{html}{%
}

\begin{document}
\meta{url}{/cours/fonctions-generalites/}
\meta{pid}{86}
\meta{titre}{Fonctions - Généralités}
\meta{type}{cours}
\begin{h2}1. Notion de fonction\end{h2}
\cadre{bleu}{Définition}{% id="d10"
     Une \textbf{fonction} $f$ est un procédé qui à tout nombre réel $x$ d'une partie $D$ de $\mathbb{R}$ associe \textbf{ un seul }nombre réel $y$.
     \begin{itemize}
          \item $x$ s'appelle la \textbf{variable}.
          \item $y$ s'appelle l'\textbf{image} de $x$ par la fonction $f$ et se note $f\left(x\right)$
          \item $f$ est la \textbf{fonction} et se note: $f : x \mapsto  y=f\left(x\right)$.
     \end{itemize}
}
\bloc{cyan}{Remarque}{% id="r10"
     Les procédés permettant d'associer un nombre à un autre nombre peuvent être :
     \begin{itemize}
          \item des formules mathématiques (par exemple : $f\left(x\right)=\frac{1}{1+x^2}$)
          \item une courbe (par exemple : la courbe donnant le cours d'une action en Bourse en fonction du temps)
          \item un instrument de mesure ou de conversion (par exemple : le compteur d'un taxi qui donne le prix à payer en fonction du trajet parcouru)
          \item un tableau de valeurs, chaque élément de la seconde ligne étant associé à un élément de la première ligne
          \item une touche de calculatrice (par exemple: \textit{sin, cos, ln, log}, etc.) qui affiche un résultat dépendant du nombre saisi auparavant
          \item etc.
     \end{itemize}
}
\cadre{rouge}{Méthode (Calcul d'une image)}{% id="t30"
     Pour calculer l'image d'un nombre par une fonction définie par une formule on remplace $x$ par ce nombre dans l'expression de $f\left(x\right)$
}
\bloc{orange}{Exemple}{% id="e30"
     Soit la fonction $f$ définie par $f\left(x\right)=\frac{x^2+3}{x+1}$
     \begin{itemize}
          \item Pour calculer l'image de $1$ - notée $f\left(1\right)$ - on remplace $x$ par $1$ dans la formule donnant $f\left(x\right)$. On obtient alors :
          \par
          $f\left(1\right)=\frac{1^2+3}{1+1}=\frac{4}{2}=2$
          \item Pour calculer  l'image de $-2$, on remplace $x$ par $\left(-2\right)$ dans cette même formule. Pensez bien à ajouter une parenthèse lorsque $x$ est négatif ou lorsqu'il s'agit d'une expression fractionnaire. On obtient  :
          \par
          $f\left(-2\right)=\frac{\left(-2\right)^2+3}{\left(-2\right)+1}=\frac{7}{-1}=-7$
     \end{itemize}
}
\cadre{bleu}{Définition}{% id="d40"
     L'ensemble $\mathscr D$ des éléments $x$ de $\mathbb{R}$ qui possèdent une image par $f$ s'appelle l'\textbf{ensemble de définition} de $f$.
     \par
     On dit également que $f$ est \textbf{définie} sur $\mathscr D$
}
\bloc{cyan}{Remarque}{% id="r40"
     Certaines fonctions sont définies sur $\mathbb{R}$ en entier. Parfois, cependant, l'ensemble de définition est plus petit. C'est en particulier le cas:
     \begin{itemize}
          \item s'il est impossible de calculer $f\left(x\right)$ pour certaines valeurs de $x$ (par exemple la fonction $f : x \mapsto  \frac{1}{x}$ n'est pas définie pour $x=0$ car il est impossible de diviser par zéro
          \item si la fonction n'a aucune signification pour certaines valeurs de $x$; par exemple la fonction donnant l'aire d'un carré en fonction de la longueur $x$ de ses côtés n'a pas de sens pour $x$ négatif.
     \end{itemize}
}
\cadre{bleu}{Définition}{% id="d50"
     Soit $y$ un nombre réel. Les \textbf{antécédents} de $y$ par $f$ sont les nombres réels $x$ appartenant à $\mathscr D$ tels que $f\left(x\right)=y$. Un nombre peut avoir aucun, un ou plusieurs antécédent(s).
}
\cadre{rouge}{Méthode (Calcul des antécédents)}{% id="t60"
     Pour déterminer les antécédents d'un nombre $y$, on résout l'équation  $f\left(x\right)=y$ d'inconnue $x$.
}
\bloc{orange}{Exemple}{% id="e60"
     Soit la fonction $f$ définie par $f\left(x\right)=\frac{x+5}{x+1}$
     \par
     Pour déterminer le ou les antécédents du nombre $2$ on résout l'équation $f\left(x\right)=2$ c'est à dire :
     \begin{center}$\frac{x+5}{x+1}=2$\end{center}
     On obtient alors :
     \par
     $x+5=2\left(x+1\right)$ (« produit en croix »)
     \par
     $x+5=2x+2$
     \par
     $x-2x=2-5$
     \par
     $-x=-3$
     \par
     $x=3$
     \par
     Le nombre $2$ possède un unique antécédent qui est $x=3$.
}
\begin{h2}2. Représentation graphique\end{h2}
Dans cette section, on munit le plan $\mathscr P$ d'un repère orthogonal $\left(O, i, j\right)$
\cadre{bleu}{Définition}{% id="d80"
     Soit $f$ une fonction définie sur un ensemble $\mathscr D$.
     \par
     La\textbf{ représentation graphique} de $f$ est la courbe  $\mathscr C_f $  formée des points $M\left(x;y\right)$ où $x\in \mathscr D$ et $y=f\left(x\right)$
     \par
     On dit aussi que la courbe $\mathscr C_f $ \textbf{a pour équation} $y=f\left(x\right)$.
}
\bloc{orange}{Exemple}{% id="e80"
     \begin{center}
          \begin{extern} %width="300" alt=" représentation graphique d'une fonction"
               \resizebox{8cm}{!}{%
                    % -+-+-+ variables modifiables
                    \def\fonction{x^3-x+1}
                    \def\xmin{-1.2}
                    \def\xmax{1.2}
                    \def\ymin{-0.2}
                    \def\ymax{1.7}
                    \def\xunit{4}  % unités en cm
                    \def\yunit{4}
                    \psset{xunit=\xunit,yunit=\yunit,algebraic=true}
                    \fontsize{15pt}{15pt}\selectfont
                    \begin{pspicture*}[linewidth=1pt](\xmin,\ymin)(\xmax,\ymax)
                         %    \psgrid[gridcolor=lightgray, subgriddiv=1, gridlabels=0pt](-3,-1.8)(7,4)
                         \psaxes[linewidth=0.75pt]{->}(0,0)(\xmin,\ymin)(\xmax,\ymax)
                         \psplot[plotpoints=2000,linecolor=blue]{-1}{1}{\fonction}
                         \rput[tr](-0.07,-0.07){$O$}
                         \rput[l](-0.88,1.05){$\color{blue} \mathcal{C}_f$}
                    \end{pspicture*}
               }
          \end{extern}
     \end{center}
     \begin{center}\textit{Exemple de représentation graphique d'une fonction définie sur [-1;1]}\end{center}
}
\bloc{cyan}{Remarque}{% id="r80"
     Du fait qu'un nombre ne peut pas avoir plusieurs images, la courbe représentative d'une fonction \textbf{ne peut pas contenir plusieurs points situés sur la même "verticale"} (droite parallèle à l'axe des ordonnées).
     \par
     Par contre, il peut très bien y avoir plusieurs points situés sur une même horizontale comme dans l'exemple ci-dessus.
}
\bloc{orange}{Lecture graphique de l'image d'un nombre}{% id="e82"
     \begin{center}
          \begin{extern} %width="300" alt="Lecture graphique d'une image"
               \resizebox{8cm}{!}{%
                    % -+-+-+ variables modifiables
                    \def\fonction{x^3-x+1}
                    \def\xmin{-1.2}
                    \def\xmax{1.2}
                    \def\ymin{-0.2}
                    \def\ymax{1.7}
                    \def\xunit{4}  % unités en cm
                    \def\yunit{4}
                    \psset{xunit=\xunit,yunit=\yunit,algebraic=true}
                    \fontsize{15pt}{15pt}\selectfont
                    \begin{pspicture*}[linewidth=1pt](\xmin,\ymin)(\xmax,\ymax)
                         %    \psgrid[gridcolor=lightgray, subgriddiv=1, gridlabels=0pt](-3,-1.8)(7,4)
                         \psaxes[linewidth=0.75pt]{->}(0,0)(\xmin,\ymin)(\xmax,\ymax)
                         \psplot[plotpoints=2000,linecolor=blue]{-1}{1}{\fonction}
                         \rput[tr](-0.07,-0.07){$O$}
                         \rput[l](-0.88,1.05){$\color{blue} \mathcal{C}_f$}
                         \rput(0.58,0.72){$\color{blue} M$}
                         \rput[t](0.5,-0.07){$\red 0,5$}\rput[r](-0.05,0.62){$\red f(0,5)\approx0,6$}
                         \psline[linecolor=red,linewidth=1pt](0.5,0)(0.5,0.625)(0,0.625)
                         \psline[linecolor=red,linewidth=.9pt,arrowsize=6pt]{->}(0.5,0)(0.5,0.33)
                         \psline[linecolor=red,linewidth=.9pt,arrowsize=6pt]{->}(0.5,0.625)(0.23,0.625)
                    \end{pspicture*}
               }
          \end{extern}
     \end{center}
     Pour déterminer graphiquement l'\textbf{image} de $0,5$ par la fonction $f $:
     \begin{itemize}
          \item on place le point de d'\textbf{abscisse $0,5$} sur l'axe des abscisses
          \item on le relie au point $M$ de la courbe qui a la même abscisse
          \item l'\textbf{ordonnée} du point $M$ nous donne la valeur de $f\left(0,5\right)$; on trouve ici environ $0,6$.
     \end{itemize}
}
\bloc{orange}{Lecture graphique des antécédents d'un nombre}{% id="e84"
     \begin{center}
          \begin{extern} %width="300" alt="Lecture graphique d'antécédents "
               \resizebox{8cm}{!}{%
                    % -+-+-+ variables modifiables
                    \def\fonction{x^3-x+1}
                    \def\xmin{-1.2}
                    \def\xmax{1.2}
                    \def\ymin{-0.2}
                    \def\ymax{1.7}
                    \def\xunit{4}  % unités en cm
                    \def\yunit{4}
                    \psset{xunit=\xunit,yunit=\yunit,algebraic=true}
                    \fontsize{15pt}{15pt}\selectfont
                    \begin{pspicture*}[linewidth=1pt](\xmin,\ymin)(\xmax,\ymax)
                         %    \psgrid[gridcolor=lightgray, subgriddiv=1, gridlabels=0pt](-3,-1.8)(7,4)
                         \psaxes[linewidth=0.75pt]{->}(0,0)(\xmin,\ymin)(\xmax,\ymax)
                         \psplot[plotpoints=2000,linecolor=blue]{-1}{1}{\fonction}
                         \rput[tr](-0.07,-0.07){$O$}
                         \rput[l](-0.88,1.05){$\color{blue} \mathcal{C}_f$}
                         \rput[t](0.13,-0.08){$\red 0,1$} \rput[t](0.82,-0.08){$\red 0,95$} \rput[r](-0.05,0.8){$\red 0,9$}
                         \psline[linecolor=red,linewidth=1pt](0.1,0.9)(0.1,0.)
                         \psline[linecolor=red,linewidth=1pt](0.95,0.9)(0.95,0.)
                         \psline[linecolor=red,linewidth=1pt](-1.5,0.9)(1.5,0.9)
                         \psline[linecolor=red,linewidth=.9pt,arrowsize=6pt]{->}(0.1,0.9)(0.1,0.4)
                         \psline[linecolor=red,linewidth=.9pt,arrowsize=6pt]{->}(0.95,0.9)(0.95,0.4)
                    \end{pspicture*}
               }
          \end{extern}
     \end{center}
     Pour déterminer graphiquement les \textbf{antécédents} de $0,9$ par la fonction $f $:
     \begin{itemize}
          \item on place le point de d'\textbf{ordonnée} $0,9$ sur l'axe des ordonnées
          \item on trace la droite horizontale (d'équation $y=0,9$) qui passe par ce point
          \item on trace le(s) \textbf{point(s) d'intersection} de cette droite avec la courbe. Dans cet exemple on en trouve deux ; dans d'autres exemples on pourrait en trouver zéro, un, deux ou plus...
          \item les \textbf{abscisses} de ces points d'intersection nous donne les antécédents de $0,9$; on trouve ici deux antécédents qui valent environ $0,1$ et $0,95$.
     \end{itemize}
}
\begin{h2}3. Variations d'une fonction\end{h2}
\cadre{bleu}{Définition}{% id="d100"
     La fonction $f$ est \textbf{croissante} sur l'intervalle $I$ si pour tous réels $x_1$ et $x_2$  appartenant à $I$ tels que $x_1\leqslant  x_2$ on a $f\left(x_1\right)\leqslant f\left(x_2\right)$.
}
\bloc{cyan}{Remarque}{% id="r100"
     Intuitivement, cela se traduit par le fait que la courbe représentative de la fonction $f$ "monte" lorsqu'on la parcourt dans le sens de l'axe des abscisses (e.g. de gauche à droite)
}
\begin{center}
     \begin{extern}%width="250" alt="fonction croissante"
          \resizebox{6cm}{!}{%
               % -+-+-+ variables modifiables
               \def\fonction{1+0.2*x*x }
               \def\xmin{-1.2}
               \def\xmax{5}
               \def\ymin{-0.9}
               \def\ymax{5}
               \def\xunit{1}  % unités en cm
               \def\yunit{1}
               \psset{xunit=\xunit,yunit=\yunit,algebraic=true}
               \fontsize{12pt}{12pt}\selectfont
               \begin{pspicture*}[linewidth=1pt](\xmin,\ymin)(\xmax,\ymax)
                    %      \psgrid[gridcolor=mcgris, subgriddiv=5, gridlabels=0pt](\xmin,\ymin)(\xmax,\ymax)
                    \psaxes[linewidth=0.75pt,Dx=10,Dy=10]{->}(0,0)(\xmin,\ymin)(\xmax,\ymax)
                    \psplot[plotpoints=2000,linecolor=red]{0.2}{\xmax}{\fonction}
                    \psline[linewidth=0.75pt,linecolor=lightgray](1,0)(1,1.2)(0,1.2)
                    \psline[linewidth=0.75pt,linecolor=lightgray](4,0)(4,4.2)(0,4.2)
                    \rput[tr](-0.3,-0.3){$O$} \rput[t](1,-0.3){$x_1$} \rput[t](4,-0.3){$x_2$}
                    \rput[r](-0.1,1.2){$f(x_1)$} \rput[r](-0.1,4.2){$f(x_2)$}
                    \rput[tl](4.3,4.5){$\color{red} \mathcal{C}_f$}
               \end{pspicture*}
          }
     \end{extern}
\end{center}
\cadre{bleu}{Définition}{% id="d110"
     La fonction $f$ est \textbf{décroissante} sur l'intervalle $I$ si pour tous réels $x_1$ et $x_2$  appartenant à $I$ tels que $x_1 \leqslant x_2$ on a $f\left(x_1\right) \geqslant f\left(x_2\right)$.
}
\bloc{cyan}{Remarque}{% id="r110"
     Intuitivement, cela se traduit par le fait que la courbe représentative de la fonction $f$ "descend" lorsqu'on la parcourt dans le sens de l'axe des abscisses (e.g. de gauche à droite)
}
\begin{center}
     \begin{extern}%width="250" alt="fonction décroissante"
          \resizebox{6cm}{!}{%
               % -+-+-+ variables modifiables
               \def\fonction{4-0.1*x*x }
               \def\xmin{-1.2}
               \def\xmax{5}
               \def\ymin{-0.9}
               \def\ymax{5}
               \def\xunit{1}  % unités en cm
               \def\yunit{1}
               \psset{xunit=\xunit,yunit=\yunit,algebraic=true}
               \fontsize{12pt}{12pt}\selectfont
               \begin{pspicture*}[linewidth=1pt](\xmin,\ymin)(\xmax,\ymax)
                    %      \psgrid[gridcolor=mcgris, subgriddiv=5, gridlabels=0pt](\xmin,\ymin)(\xmax,\ymax)
                    \psaxes[linewidth=0.75pt,Dx=10,Dy=10]{->}(0,0)(\xmin,\ymin)(\xmax,\ymax)
                    \psplot[plotpoints=2000,linecolor=red]{0.2}{\xmax}{\fonction}
                    \psline[linewidth=0.75pt,linecolor=lightgray](1,0)(1,3.9)(0,3.9)
                    \psline[linewidth=0.75pt,linecolor=lightgray](4,0)(4,2.4)(0,2.4)
                    \rput[tr](-0.3,-0.3){$O$} \rput[t](1,-0.3){$x_1$} \rput[t](4,-0.3){$x_2$}
                    \rput[r](-0.1,3.9){$f(x_1)$} \rput[r](-0.1,2.4){$f(x_2)$}
                    \rput[tl](4.5,2.5){$\color{red} \mathcal{C}_f$}
               \end{pspicture*}
          }
     \end{extern}
\end{center}
\cadre{bleu}{Définition}{% id="d120"
     Soit $I$ un intervalle et $x_0 \in  I$.
     \par
     La fonction $f$ admet un \textbf{maximum} en $x_0$ sur l'intervalle $I$ si pour tout réel $x$ de I, $f\left(x\right)\leqslant f\left(x_0\right)$. Le maximum de la fonction $f$ sur $I$ est alors $M=f\left(x_0\right)$
}
\begin{center}
     \begin{extern}%width="250" alt="maximum d'une fonction"
          \resizebox{6cm}{!}{%
               % -+-+-+ variables modifiables
               \def\fonction{(x+1)*(5-x)*4/9}
               \def\xmin{-1.2}
               \def\xmax{5}
               \def\ymin{-0.9}
               \def\ymax{5}
               \def\xunit{1}  % unités en cm
               \def\yunit{1}
               \psset{xunit=\xunit,yunit=\yunit,algebraic=true}
               \fontsize{12pt}{12pt}\selectfont
               \begin{pspicture*}[linewidth=1pt](\xmin,\ymin)(\xmax,\ymax)
                    %      \psgrid[gridcolor=mcgris, subgriddiv=5, gridlabels=0pt](\xmin,\ymin)(\xmax,\ymax)
                    \psaxes[linewidth=0.75pt,Dx=10,Dy=10]{->}(0,0)(\xmin,\ymin)(\xmax,\ymax)
                    \psplot[plotpoints=2000,linecolor=red]{0.2}{4}{\fonction}
                    \psline[linewidth=0.75pt,linecolor=lightgray](2,0)(2,4)(0,4)
                    \rput[tr](-0.3,-0.3){$O$} \rput[t](2,-0.3){$x_0$}
                    \rput[r](-0.1,4){$f(x_0)$}
                    \rput[tl](4.,3){$\color{red} \mathcal{C}_f$}
               \end{pspicture*}
          }
     \end{extern}
\end{center}
\cadre{bleu}{Définition}{% id="d130"
     Soit $I$ un intervalle et $x_0 \in  I$.
     \par
     La fonction $f$ admet un \textbf{minimum} en $x_0$ sur l'intervalle $I$ si pour tout réel $x$ de I, $f\left(x\right)\geqslant f\left(x_0\right)$. Le minimum de la fonction $f$ sur $I$ est alors $m=f\left(x_0\right)$
}
\begin{center}
     \begin{extern}%width="250" alt="maximum d'une fonction"
          \resizebox{6cm}{!}{%
               % -+-+-+ variables modifiables
               \def\fonction{(x+1)*(x-5)*4/9+6}
               \def\xmin{-1.2}
               \def\xmax{5}
               \def\ymin{-0.9}
               \def\ymax{5}
               \def\xunit{1}  % unités en cm
               \def\yunit{1}
               \psset{xunit=\xunit,yunit=\yunit,algebraic=true}
               \fontsize{12pt}{12pt}\selectfont
               \begin{pspicture*}[linewidth=1pt](\xmin,\ymin)(\xmax,\ymax)
                    %      \psgrid[gridcolor=mcgris, subgriddiv=5, gridlabels=0pt](\xmin,\ymin)(\xmax,\ymax)
                    \psaxes[linewidth=0.75pt,Dx=10,Dy=10]{->}(0,0)(\xmin,\ymin)(\xmax,\ymax)
                    \psplot[plotpoints=2000,linecolor=red]{0.2}{4}{\fonction}
                    \psline[linewidth=0.75pt,linecolor=lightgray](2,0)(2,2)(0,2)
                    \rput[tr](-0.3,-0.3){$O$} \rput[t](2,-0.3){$x_0$}
                    \rput[r](-0.1,2){$f(x_0)$}
                    \rput[tl](4.,3.5){$\color{red} \mathcal{C}_f$}
               \end{pspicture*}
          }
     \end{extern}
\end{center}
\bloc{cyan}{Remarques}{% id="r130"
     \begin{itemize}
          \item Un \textbf{extremum} est un maximum ou un minimum
          \item \textbf{Attention à la rédaction :}
          Lorsqu'on dit que $f$ admet un maximum (\textit{resp.} minimum) \textbf{en} $x_0$ (ou \textbf{pour $x=x_0$}), $x_0$ correspond à la valeur de la \textbf{variable $x$} et non à la valeur du  maximum (\textit{resp.} minimum).
          \par
          Par exemple, dans le tableau de l'exemple ci-dessous, $f$ admet un maximum \textbf{en $0$}. Ce maximum \textbf{est égal à 6} (\textit{Ne pas écrire que le maximum est $0$ !}).
          \item Les variations d'une fonction peuvent être représentées par un \textbf{tableau de variations}
     \end{itemize}
}
\bloc{orange}{Exemple}{% id="e130"
     Soit $f$ une fonction définie sur $\left[-2;5\right]$, croissante sur $\left[-2;0\right]$ et décroissante sur $\left[0; 5\right]$ avec $f\left(-2\right)=-3$, $f\left(0\right)=6$ et $f\left(5\right)=1$
     \par
     Le tableau de variations de la fonction $f$ est :
     %:-+-+-+-+- Engendré par : http://math.et.info.free.fr/TikZ/TableauxVariations/
     \begin{center}
          \begin{extern}%width="350" alt=""
               \begin{tikzpicture}[scale=0.875]
                    % Styles
                    \tikzstyle{cadre}=[thin]
                    \tikzstyle{fleche}=[->,>=latex,thin]
                    \tikzstyle{nondefini}=[lightgray]
                    % Dimensions Modifiables
                    \def\Lrg{1.5}
                    \def\HtX{1}
                    \def\HtY{0.5}
                    % Dimensions Calculées
                    \def\lignex{-0.5*\HtX}
                    \def\lignef{-1.5*\HtX}
                    \def\separateur{-0.5*\Lrg}
                    % Largeur du tableau
                    \def\gauche{-1.5*\Lrg}
                    \def\droite{4.5*\Lrg}
                    % Hauteur du tableau
                    \def\haut{0.5*\HtX}
                    \def\bas{-1.5*\HtX-2*\HtY}
                    % Pointillés
                    \draw[lightgray] (2*\Lrg,\lignex) -- (2*\Lrg,\bas);
                    % Ligne de l'abscisse : x
                    \node at (-1*\Lrg,0) {$x$};
                    \node at (0*\Lrg,0) {$-2$};
                    \node at (2*\Lrg,0) {$0$};
                    \node at (4*\Lrg,0) {$5$};
                    % Ligne de la fonction : f(x)
                    \node  at (-1*\Lrg,{-1*\HtX+(-1)*\HtY}) {$f(x)$};
                    \node (f1) at (0*\Lrg,{-1*\HtX+(-2)*\HtY}) {$-3$};
                    \node (f2) at (2*\Lrg,{-1*\HtX+(0)*\HtY}) {$6$};
                    \node (f3) at (4*\Lrg,{-1*\HtX+(-2)*\HtY}) {$1$};
                    % Flèches
                    \draw[fleche] (f1) -- (f2);
                    \draw[fleche] (f2) -- (f3);
                    % Encadrement
                    \draw[cadre] (\separateur,\haut) -- (\separateur,\bas);
                    \draw[cadre] (\gauche,\haut) rectangle  (\droite,\bas);
                    \draw[cadre] (\gauche,\lignex) -- (\droite,\lignex);
               \end{tikzpicture}
          \end{extern}
     \end{center}
}

\end{document}
µ
\documentclass[a4paper]{article}

%================================================================================================================================
%
% Packages
%
%================================================================================================================================

\usepackage[T1]{fontenc} 	% pour caractères accentués
\usepackage[utf8]{inputenc}  % encodage utf8
\usepackage[french]{babel}	% langue : français
\usepackage{fourier}			% caractères plus lisibles
\usepackage[dvipsnames]{xcolor} % couleurs
\usepackage{fancyhdr}		% réglage header footer
\usepackage{needspace}		% empêcher sauts de page mal placés
\usepackage{graphicx}		% pour inclure des graphiques
\usepackage{enumitem,cprotect}		% personnalise les listes d'items (nécessaire pour ol, al ...)
\usepackage{hyperref}		% Liens hypertexte
\usepackage{pstricks,pst-all,pst-node,pstricks-add,pst-math,pst-plot,pst-tree,pst-eucl} % pstricks
\usepackage[a4paper,includeheadfoot,top=2cm,left=3cm, bottom=2cm,right=3cm]{geometry} % marges etc.
\usepackage{comment}			% commentaires multilignes
\usepackage{amsmath,environ} % maths (matrices, etc.)
\usepackage{amssymb,makeidx}
\usepackage{bm}				% bold maths
\usepackage{tabularx}		% tableaux
\usepackage{colortbl}		% tableaux en couleur
\usepackage{fontawesome}		% Fontawesome
\usepackage{environ}			% environment with command
\usepackage{fp}				% calculs pour ps-tricks
\usepackage{multido}			% pour ps tricks
\usepackage[np]{numprint}	% formattage nombre
\usepackage{tikz,tkz-tab} 			% package principal TikZ
\usepackage{pgfplots}   % axes
\usepackage{mathrsfs}    % cursives
\usepackage{calc}			% calcul taille boites
\usepackage[scaled=0.875]{helvet} % font sans serif
\usepackage{svg} % svg
\usepackage{scrextend} % local margin
\usepackage{scratch} %scratch
\usepackage{multicol} % colonnes
%\usepackage{infix-RPN,pst-func} % formule en notation polanaise inversée
\usepackage{listings}

%================================================================================================================================
%
% Réglages de base
%
%================================================================================================================================

\lstset{
language=Python,   % R code
literate=
{á}{{\'a}}1
{à}{{\`a}}1
{ã}{{\~a}}1
{é}{{\'e}}1
{è}{{\`e}}1
{ê}{{\^e}}1
{í}{{\'i}}1
{ó}{{\'o}}1
{õ}{{\~o}}1
{ú}{{\'u}}1
{ü}{{\"u}}1
{ç}{{\c{c}}}1
{~}{{ }}1
}


\definecolor{codegreen}{rgb}{0,0.6,0}
\definecolor{codegray}{rgb}{0.5,0.5,0.5}
\definecolor{codepurple}{rgb}{0.58,0,0.82}
\definecolor{backcolour}{rgb}{0.95,0.95,0.92}

\lstdefinestyle{mystyle}{
    backgroundcolor=\color{backcolour},   
    commentstyle=\color{codegreen},
    keywordstyle=\color{magenta},
    numberstyle=\tiny\color{codegray},
    stringstyle=\color{codepurple},
    basicstyle=\ttfamily\footnotesize,
    breakatwhitespace=false,         
    breaklines=true,                 
    captionpos=b,                    
    keepspaces=true,                 
    numbers=left,                    
xleftmargin=2em,
framexleftmargin=2em,            
    showspaces=false,                
    showstringspaces=false,
    showtabs=false,                  
    tabsize=2,
    upquote=true
}

\lstset{style=mystyle}


\lstset{style=mystyle}
\newcommand{\imgdir}{C:/laragon/www/newmc/assets/imgsvg/}
\newcommand{\imgsvgdir}{C:/laragon/www/newmc/assets/imgsvg/}

\definecolor{mcgris}{RGB}{220, 220, 220}% ancien~; pour compatibilité
\definecolor{mcbleu}{RGB}{52, 152, 219}
\definecolor{mcvert}{RGB}{125, 194, 70}
\definecolor{mcmauve}{RGB}{154, 0, 215}
\definecolor{mcorange}{RGB}{255, 96, 0}
\definecolor{mcturquoise}{RGB}{0, 153, 153}
\definecolor{mcrouge}{RGB}{255, 0, 0}
\definecolor{mclightvert}{RGB}{205, 234, 190}

\definecolor{gris}{RGB}{220, 220, 220}
\definecolor{bleu}{RGB}{52, 152, 219}
\definecolor{vert}{RGB}{125, 194, 70}
\definecolor{mauve}{RGB}{154, 0, 215}
\definecolor{orange}{RGB}{255, 96, 0}
\definecolor{turquoise}{RGB}{0, 153, 153}
\definecolor{rouge}{RGB}{255, 0, 0}
\definecolor{lightvert}{RGB}{205, 234, 190}
\setitemize[0]{label=\color{lightvert}  $\bullet$}

\pagestyle{fancy}
\renewcommand{\headrulewidth}{0.2pt}
\fancyhead[L]{maths-cours.fr}
\fancyhead[R]{\thepage}
\renewcommand{\footrulewidth}{0.2pt}
\fancyfoot[C]{}

\newcolumntype{C}{>{\centering\arraybackslash}X}
\newcolumntype{s}{>{\hsize=.35\hsize\arraybackslash}X}

\setlength{\parindent}{0pt}		 
\setlength{\parskip}{3mm}
\setlength{\headheight}{1cm}

\def\ebook{ebook}
\def\book{book}
\def\web{web}
\def\type{web}

\newcommand{\vect}[1]{\overrightarrow{\,\mathstrut#1\,}}

\def\Oij{$\left(\text{O}~;~\vect{\imath},~\vect{\jmath}\right)$}
\def\Oijk{$\left(\text{O}~;~\vect{\imath},~\vect{\jmath},~\vect{k}\right)$}
\def\Ouv{$\left(\text{O}~;~\vect{u},~\vect{v}\right)$}

\hypersetup{breaklinks=true, colorlinks = true, linkcolor = OliveGreen, urlcolor = OliveGreen, citecolor = OliveGreen, pdfauthor={Didier BONNEL - https://www.maths-cours.fr} } % supprime les bordures autour des liens

\renewcommand{\arg}[0]{\text{arg}}

\everymath{\displaystyle}

%================================================================================================================================
%
% Macros - Commandes
%
%================================================================================================================================

\newcommand\meta[2]{    			% Utilisé pour créer le post HTML.
	\def\titre{titre}
	\def\url{url}
	\def\arg{#1}
	\ifx\titre\arg
		\newcommand\maintitle{#2}
		\fancyhead[L]{#2}
		{\Large\sffamily \MakeUppercase{#2}}
		\vspace{1mm}\textcolor{mcvert}{\hrule}
	\fi 
	\ifx\url\arg
		\fancyfoot[L]{\href{https://www.maths-cours.fr#2}{\black \footnotesize{https://www.maths-cours.fr#2}}}
	\fi 
}


\newcommand\TitreC[1]{    		% Titre centré
     \needspace{3\baselineskip}
     \begin{center}\textbf{#1}\end{center}
}

\newcommand\newpar{    		% paragraphe
     \par
}

\newcommand\nosp {    		% commande vide (pas d'espace)
}
\newcommand{\id}[1]{} %ignore

\newcommand\boite[2]{				% Boite simple sans titre
	\vspace{5mm}
	\setlength{\fboxrule}{0.2mm}
	\setlength{\fboxsep}{5mm}	
	\fcolorbox{#1}{#1!3}{\makebox[\linewidth-2\fboxrule-2\fboxsep]{
  		\begin{minipage}[t]{\linewidth-2\fboxrule-4\fboxsep}\setlength{\parskip}{3mm}
  			 #2
  		\end{minipage}
	}}
	\vspace{5mm}
}

\newcommand\CBox[4]{				% Boites
	\vspace{5mm}
	\setlength{\fboxrule}{0.2mm}
	\setlength{\fboxsep}{5mm}
	
	\fcolorbox{#1}{#1!3}{\makebox[\linewidth-2\fboxrule-2\fboxsep]{
		\begin{minipage}[t]{1cm}\setlength{\parskip}{3mm}
	  		\textcolor{#1}{\LARGE{#2}}    
 	 	\end{minipage}  
  		\begin{minipage}[t]{\linewidth-2\fboxrule-4\fboxsep}\setlength{\parskip}{3mm}
			\raisebox{1.2mm}{\normalsize\sffamily{\textcolor{#1}{#3}}}						
  			 #4
  		\end{minipage}
	}}
	\vspace{5mm}
}

\newcommand\cadre[3]{				% Boites convertible html
	\par
	\vspace{2mm}
	\setlength{\fboxrule}{0.1mm}
	\setlength{\fboxsep}{5mm}
	\fcolorbox{#1}{white}{\makebox[\linewidth-2\fboxrule-2\fboxsep]{
  		\begin{minipage}[t]{\linewidth-2\fboxrule-4\fboxsep}\setlength{\parskip}{3mm}
			\raisebox{-2.5mm}{\sffamily \small{\textcolor{#1}{\MakeUppercase{#2}}}}		
			\par		
  			 #3
 	 		\end{minipage}
	}}
		\vspace{2mm}
	\par
}

\newcommand\bloc[3]{				% Boites convertible html sans bordure
     \needspace{2\baselineskip}
     {\sffamily \small{\textcolor{#1}{\MakeUppercase{#2}}}}    
		\par		
  			 #3
		\par
}

\newcommand\CHelp[1]{
     \CBox{Plum}{\faInfoCircle}{À RETENIR}{#1}
}

\newcommand\CUp[1]{
     \CBox{NavyBlue}{\faThumbsOUp}{EN PRATIQUE}{#1}
}

\newcommand\CInfo[1]{
     \CBox{Sepia}{\faArrowCircleRight}{REMARQUE}{#1}
}

\newcommand\CRedac[1]{
     \CBox{PineGreen}{\faEdit}{BIEN R\'EDIGER}{#1}
}

\newcommand\CError[1]{
     \CBox{Red}{\faExclamationTriangle}{ATTENTION}{#1}
}

\newcommand\TitreExo[2]{
\needspace{4\baselineskip}
 {\sffamily\large EXERCICE #1\ (\emph{#2 points})}
\vspace{5mm}
}

\newcommand\img[2]{
          \includegraphics[width=#2\paperwidth]{\imgdir#1}
}

\newcommand\imgsvg[2]{
       \begin{center}   \includegraphics[width=#2\paperwidth]{\imgsvgdir#1} \end{center}
}


\newcommand\Lien[2]{
     \href{#1}{#2 \tiny \faExternalLink}
}
\newcommand\mcLien[2]{
     \href{https~://www.maths-cours.fr/#1}{#2 \tiny \faExternalLink}
}

\newcommand{\euro}{\eurologo{}}

%================================================================================================================================
%
% Macros - Environement
%
%================================================================================================================================

\newenvironment{tex}{ %
}
{%
}

\newenvironment{indente}{ %
	\setlength\parindent{10mm}
}

{
	\setlength\parindent{0mm}
}

\newenvironment{corrige}{%
     \needspace{3\baselineskip}
     \medskip
     \textbf{\textsc{Corrigé}}
     \medskip
}
{
}

\newenvironment{extern}{%
     \begin{center}
     }
     {
     \end{center}
}

\NewEnviron{code}{%
	\par
     \boite{gray}{\texttt{%
     \BODY
     }}
     \par
}

\newenvironment{vbloc}{% boite sans cadre empeche saut de page
     \begin{minipage}[t]{\linewidth}
     }
     {
     \end{minipage}
}
\NewEnviron{h2}{%
    \needspace{3\baselineskip}
    \vspace{0.6cm}
	\noindent \MakeUppercase{\sffamily \large \BODY}
	\vspace{1mm}\textcolor{mcgris}{\hrule}\vspace{0.4cm}
	\par
}{}

\NewEnviron{h3}{%
    \needspace{3\baselineskip}
	\vspace{5mm}
	\textsc{\BODY}
	\par
}

\NewEnviron{margeneg}{ %
\begin{addmargin}[-1cm]{0cm}
\BODY
\end{addmargin}
}

\NewEnviron{html}{%
}

\begin{document}
\meta{url}{/cours/equations-et-inequations/}
\meta{pid}{90}
\meta{titre}{Equations et inéquations}
\meta{type}{cours}
\begin{h2}I. Equations\end{h2}
\cadre{rouge}{Théorème}{%id="t10"
     \begin{itemize}
          \item Si l'on ajoute ou si l'on soustrait un même nombre à chaque membre d'une équation, on obtient une équation équivalente (c'est à dire qui possède les mêmes solutions).
          \item Si l'on multiplie ou si l'on divise chaque membre d'une équation par un même nombre \textbf{non nul}, on obtient une équation équivalente.
     \end{itemize}
}
\bloc{cyan}{Remarque}{%id="r10"
     Pour résoudre une équation du type $ax+b=0$ on soustrait $b$ à chaque membre de l'égalité:
     \par
     $ax+b-b=0-b$ c'est à dire $ax=-b$.
     \par
     Puis:
     \begin{itemize}
          \item si $a$ est \textbf{non nul} on divise chaque membre par $a$~:~$\frac{ax}{a}=-\frac{b}{a}$ soit $x=-\frac{b}{a}$ donc $S=\left\{-\frac{b}{a}\right\}$
          \item si $a=0$:
          \begin{itemize}
               \item si $b=0$ l'équation se réduit à $0=0$. Elle est toujours vérifiée donc $S=\mathbb{R}$
               \item si $b\neq 0$ l'équation se réduit à $b=0$. Elle n'est jamais vérifiée donc $S=\varnothing$
          \end{itemize}
     \end{itemize}
}
\cadre{rouge}{Théorème (Équation produit)}{%id="t20"
     Un produit de facteurs est nul si et seulement si au moins un des facteurs est nul.
     \par
     En particulier, une équation du type $A(x)\times B(x)=0$ est vérifiée si et seulement si :
     \par
     $A(x)=0$ \textbf{ou} $B(x)=0$
}
\bloc{orange}{Exemple}{%id="e20"
     Soit l'équation $(3x-5)(x+2)=0$
     \par
     Cette équation est équivalente à $3x-5=0$ \textbf{ou} $x+2=0$.
     \par
     C'est à dire $x=\frac{5}{3}$ \textbf{ou} $x=-2$.
     \par
     L'ensemble des solutions de l'équation est donc $S=\left\{-2;\frac{5}{3}\right\}$
}
\bloc{cyan}{Remarques}{%id="r20"
     \begin{itemize}
          \item Lorsqu'on a affaire à une équation du second degré (ou plus), on fait "passer" tous les termes dans le membre de gauche que l'on essaie de factoriser et on utilise le théorème précédent.
          \item On rappelle les identités remarquables qui peuvent être utiles dans ce genre de situations:
          \begin{center}$(a+b)^2=a^2+2ab+b^2$
               \par
               $(a-b)^2=a^2-2ab+b^2$
               \par
          $(a+b)(a-b)=a^2-b^2$\end{center}
     \end{itemize}
}
\cadre{rouge}{Théorème}{%id="t30"
     Un quotient est \textbf{défini} si et seulement si son \textbf{dénominateur} est \textbf{non nul}.
     \par
     S'il est défini, un quotient est \textbf{nul} si et seulement si son \textbf{numérateur} est \textbf{nul}.
}
\bloc{orange}{Exemple}{%id="r30"
     Soit l'équation $\frac{2x-4}{x+1}=0$
     \par
     Cette équation a un sens si $x+1 \neq 0$ donc si $x\neq -1$
     \par
     Sur l'ensemble $\mathbb{R}\backslash\left\{-1\right\}$ cette équation est équivalente à $2x-4=0$ donc à $x=2$. L'ensemble des solutions de l'équation est donc $S=\left\{2\right\}$
}
\cadre{vert}{Propriété}{%id="p40"
     Soit $f$ une fonction définie sur $D$ de courbe représentative $\mathscr{C}_f$.
     \par
     Les solutions de l'équation $f(x)=m$ sont les \textbf{abscisses} des points d'intersection de la courbe $\mathscr{C}_f$ et de la droite horizontale d'équation $y=m$
}
\bloc{orange}{Exemple}{%id="e40"
     \begin{center}
          \begin{extern}%width="320" alt="équation et graphique"
               \resizebox{7cm}{!}{
                    \psset{xunit=1.0cm,yunit=1.0cm,algebraic=true,dimen=middle,linewidth=0.6pt,arrowsize=3pt}
                    \begin{pspicture*}(-2.5,-2.5)(5.5,6.)
                         \psaxes[xAxis=true,yAxis=true,Dx=1.,Dy=1.]{->}(0,0)(-2.5,-2.5)(5.5,6.)
                         \psplot[linewidth=0.8pt,linecolor=blue,plotpoints=200]{-2.5}{5.5}{(x-1.0)^(2.0)-2.0}
                         \psplot[linewidth=0.8pt,linecolor=red,plotpoints=200]{-2.5}{5.5}{2.0}
                         \psline[linewidth=0.8pt,linestyle=dashed,dash=1pt 1pt](-1.,2.)(-1.,0.)
                         \psline[linewidth=0.8pt,linestyle=dashed,dash=1pt 1pt](3.,2.)(3.,0.)
                         \rput[tl](4.6,2.4){\red{$y=2$}}
                         \rput[tl](3.97,5.8){\blue{$\mathscr{C}_f$}}
                         \psdots[dotsize=1pt 0,dotstyle=*](0.,0.)
                         \rput[bl](-0.5,-0.5){$O$}
                    \end{pspicture*}
               }
          \end{extern}
     \end{center}
     Sur la figure ci-dessus, l'équation $f(x)=2$ possède deux solutions qui sont -1 et 3
}
\cadre{rouge}{Théorème}{%id="t45"
     L'équation $x^2=a$ :
     \begin{itemize}
          \item admet deux solutions $x=\sqrt{a}$ ou $x=-\sqrt{a}$ si $a > 0$
          \item admet une unique solution $x=0$ si $a=0$
          \item n'admet aucune solution réelle si $a < 0$
     \end{itemize}
}
\bloc{orange}{Exemple}{%id="e45"
     \begin{itemize}
          \item L'équation $x^2=1$ admet deux solutions qui sont $x=-1$ et $x=1$
          \item L'équation $x^2+1=0$ est équivalente à $x^2=-1$ et n'admet donc aucune solution
     \end{itemize}
}
\begin{h2}II. Inéquations\end{h2}
\cadre{rouge}{Théorème}{%id="t50"
     \begin{itemize}
          \item Si l'on ajoute ou si l'on soustrait un même nombre à chaque membre d'une inéquation, on obtient une inéquation équivalente (c'est à dire qui à les mêmes solutions).
          \item Si l'on multiplie ou si l'on divise chaque membre d'une inéquation par un même nombre \textbf{strictement positif}, on obtient une inéquation équivalente.
          \item Si l'on multiplie ou si l'on divise chaque membre d'une inéquation par un même nombre \textbf{strictement négatif}, on obtient une inéquation équivalente \textbf{en changeant le sens de l'inégalité}.
     \end{itemize}
}
\bloc{orange}{Exemple}{%id="e50"
     Pour résoudre l'inéquation $-3x+5 > 0$ on soustrait 5 à chaque membre de l'inéquation:
     \par
     $-3x+5-5 > 0-5$ c'est à dire $-3x > -5$.
     \par
     Puis comme -3 est négatif on divise chaque membre par -3 \textbf{en changeant le sens de l'inégalité :}
\par
     $\frac{-3x}{-3} < \frac{-5}{-3}$
     \par
     $x < \frac{5}{3}$
     \par
     Donc $S=\left]-\infty ;\frac{5}{3}\right[$
}
\bloc{cyan}{Remarques}{%id="r50"
     En appliquant le théorème précédent à l'expression $ax+b$ on obtient :
     \par
     $ax+b > 0  \Leftrightarrow   ax > -b \Leftrightarrow   x > -\frac{b}{a}$ si $a$ est strictement positif
     \par
     et $ax+b > 0  \Leftrightarrow   ax > -b \Leftrightarrow   x < -\frac{b}{a}$ si $a$ est strictement négatif.
     \par
     On peut alors regrouper ces deux cas dans le tableau de signe suivant :
     \begin{center}
          \begin{extern}%width="390" alt="Tableau de signe polynôme du premier degré"
               \resizebox{8cm}{!}{
                    \begin{tikzpicture}[scale=0.875]
                         % Styles
                         \tikzstyle{cadre}=[thin]
                         \tikzstyle{fleche}=[->,>=latex,thin]
                         \tikzstyle{nondefini}=[lightgray]
                         % Dimensions Modifiables
                         \def\Lrg{1.8}
                         \def\HtX{1.2}
                         \def\HtY{0.5}
                         % Dimensions Calculées
                         \def\lignex{-0.5*\HtX}
                         \def\lignef{-1.5*\HtX}
                         \def\separateur{-0.5*\Lrg}
                         % Largeur du tableau
                         \def\gauche{-1.5*\Lrg}
                         \def\droite{4.5*\Lrg}
                         % Hauteur du tableau
                         \def\haut{0.5*\HtX}
                         \def\bas{-2.5*\HtX-2*\HtY}
                         % Pointillés
                         \draw[gray] (2*\Lrg,\lignex) -- (2*\Lrg,\lignef);
                         % Ligne de l'abscisse : x
                         \node at (-1*\Lrg,0) {$x$};
                         \node at (0*\Lrg,0) {$-\infty$};
                         \node at (2*\Lrg,0) {$-\dfrac{b}{a}$};
                         \node at (4*\Lrg,0) {$+\infty$};
                         % Ligne de la dérivée : f'(x)
                         \node at (-1*\Lrg,-1*\HtX) {$ax+b$};
                         \node at (0*\Lrg,-1*\HtX) {$ $};
                         \node at (1*\Lrg,-1*\HtX) {signe de $-a$};
                         \node at (2*\Lrg,-1*\HtX) {$0$};
                         \node at (3*\Lrg,-1*\HtX) {signe de $a$};
                         \node at (4*\Lrg,-1*\HtX) {$ $};
                         % Ligne de la fonction : f(x)
                         % Encadrement
                         \draw[cadre] (\separateur,\haut) -- (\separateur, \lignef);
                         \draw[cadre] (\gauche,\haut) rectangle  (\droite, \lignef);
                         \draw[cadre] (\gauche,\lignex) -- (\droite,\lignex);
                    \end{tikzpicture}
               }
          \end{extern}
     \end{center}
}
\cadre{rouge}{Théorème (Inéquation produit)}{%id="e60"
     Un produit de facteurs $A(x)B(x)$ est \textbf{positif ou nul} si et seulement si les deux facteurs $A(x)$ et $B(x)$ sont de \textbf{même signe}.
     \par
     Ce produit est \textbf{négatif ou nul} si et seulement si les deux facteurs $A(x)$ et $B(x)$ sont de \textbf{signes contraires}.
}
\bloc{cyan}{Remarques}{%id="r60"
     Lorsqu'on a affaire à une inéquation du second degré (ou plus), on fait "passer" tous les termes dans le membre de gauche que l'on essaie de factoriser puis on utilise un tableau de signe.
}
\bloc{orange}{Exemple}{%id="e60"
     Soit l'inéquation $(x-5)(-3x+4)\geqslant 0$
     \par
     Le signe de $x-5$ est donné par le tableau:
     \begin{center}
          \begin{extern}%width="390" alt="Exemple tableau de signe 1"
               \resizebox{8cm}{!}{
                    \begin{tikzpicture}[scale=0.875]
                         % Styles
                         \tikzstyle{cadre}=[thin]
                         \tikzstyle{fleche}=[->,>=latex,thin]
                         \tikzstyle{nondefini}=[lightgray]
                         % Dimensions Modifiables
                         \def\Lrg{1.8}
                         \def\HtX{1.2}
                         \def\HtY{0.5}
                         % Dimensions Calculées
                         \def\lignex{-0.5*\HtX}
                         \def\lignef{-1.5*\HtX}
                         \def\separateur{-0.5*\Lrg}
                         % Largeur du tableau
                         \def\gauche{-1.5*\Lrg}
                         \def\droite{4.5*\Lrg}
                         % Hauteur du tableau
                         \def\haut{0.5*\HtX}
                         \def\bas{-2.5*\HtX-2*\HtY}
                         % Pointillés
                         \draw[gray] (2*\Lrg,\lignex) -- (2*\Lrg,\lignef);
                         % Ligne de l'abscisse : x
                         \node at (-1*\Lrg,0) {$x$};
                         \node at (0*\Lrg,0) {$-\infty$};
                         \node at (2*\Lrg,0) {$5$};
                         \node at (4*\Lrg,0) {$+\infty$};
                         % Ligne de la dérivée : f'(x)
                         \node at (-1*\Lrg,-1*\HtX) {$x-5$};
                         \node at (0*\Lrg,-1*\HtX) {$ $};
                         \node at (1*\Lrg,-1*\HtX) {$-$};
                         \node at (2*\Lrg,-1*\HtX) {$0$};
                         \node at (3*\Lrg,-1*\HtX) {$+$};
                         \node at (4*\Lrg,-1*\HtX) {$ $};
                         % Ligne de la fonction : f(x)
                         % Encadrement
                         \draw[cadre] (\separateur,\haut) -- (\separateur, \lignef);
                         \draw[cadre] (\gauche,\haut) rectangle  (\droite, \lignef);
                         \draw[cadre] (\gauche,\lignex) -- (\droite,\lignex);
                    \end{tikzpicture}
               }
          \end{extern}
     \end{center}
     Le signe de $-3x+4$ est donné par le tableau:
     \begin{center}
          \begin{extern}%width="390" alt="Exemple tableau de signe 2"
               \resizebox{8cm}{!}{
                    \begin{tikzpicture}[scale=0.875]
                         % Styles
                         \tikzstyle{cadre}=[thin]
                         \tikzstyle{fleche}=[->,>=latex,thin]
                         \tikzstyle{nondefini}=[lightgray]
                         % Dimensions Modifiables
                         \def\Lrg{1.8}
                         \def\HtX{1.2}
                         \def\HtY{0.5}
                         % Dimensions Calculées
                         \def\lignex{-0.5*\HtX}
                         \def\lignef{-1.5*\HtX}
                         \def\separateur{-0.5*\Lrg}
                         % Largeur du tableau
                         \def\gauche{-1.5*\Lrg}
                         \def\droite{4.5*\Lrg}
                         % Hauteur du tableau
                         \def\haut{0.5*\HtX}
                         \def\bas{-2.5*\HtX-2*\HtY}
                         % Pointillés
                         \draw[gray] (2*\Lrg,\lignex) -- (2*\Lrg,\lignef);
                         % Ligne de l'abscisse : x
                         \node at (-1*\Lrg,0) {$x$};
                         \node at (0*\Lrg,0) {$-\infty$};
                         \node at (2*\Lrg,0) {$\dfrac{4}{3}$};
                         \node at (4*\Lrg,0) {$+\infty$};
                         % Ligne de la dérivée : f'(x)
                         \node at (-1*\Lrg,-1*\HtX) {$-3x+4$};
                         \node at (0*\Lrg,-1*\HtX) {$ $};
                         \node at (1*\Lrg,-1*\HtX) {$+$};
                         \node at (2*\Lrg,-1*\HtX) {$0$};
                         \node at (3*\Lrg,-1*\HtX) {$-$};
                         \node at (4*\Lrg,-1*\HtX) {$ $};
                         % Ligne de la fonction : f(x)
                         % Encadrement
                         \draw[cadre] (\separateur,\haut) -- (\separateur, \lignef);
                         \draw[cadre] (\gauche,\haut) rectangle  (\droite, \lignef);
                         \draw[cadre] (\gauche,\lignex) -- (\droite,\lignex);
                    \end{tikzpicture}
               }
          \end{extern}
     \end{center}
     On regroupe ces résultats dans un unique tableau et on utilise la règle des signes pour obtenir le signe du produit:
     \begin{center}
          \begin{extern}%width="500" alt="Exemple tableau de signes d'un produit"
               \resizebox{11cm}{!}{
                    \begin{tikzpicture}[scale=0.875]
                         % Styles
                         \tikzstyle{cadre}=[thin]
                         \tikzstyle{fleche}=[->,>=latex,thin]
                         \tikzstyle{nondefini}=[lightgray]
                         % Dimensions Modifiables
                         \def\Lrg{1.5}
                         \def\HtX{1.2}
                         \def\HtY{0.5}
                         % Dimensions Calculées
                         \def\lignex{-0.5*\HtX}
                         \def\lignea{-1.5*\HtX}
                         \def\ligneb{-2.5*\HtX}
                         \def\lignec{-3.5*\HtX}
                         \def\separateur{-0.5*\Lrg}
                         % Largeur du tableau
                         \def\gauche{-3.1*\Lrg}
                         \def\droite{6.5*\Lrg}
                         % Hauteur du tableau
                         \def\haut{0.5*\HtX}
                         \def\bas{-2.5*\HtX-2*\HtY}
                         % Pointillés
                         \draw[gray] (2*\Lrg,\lignex) -- (2*\Lrg,\lignec);
                         \draw[gray] (4*\Lrg,\lignex) -- (4*\Lrg,\lignec);
                         % Ligne de l'abscisse : x
                         \node at (-1.8*\Lrg,0) {$x$};
                         \node at (0*\Lrg,0) {$-\infty$};
                         \node at (2*\Lrg,0) {$\dfrac{4}{3}$};
                         \node at (4*\Lrg,0) {$5$};
                         \node at (6*\Lrg,0) {$+\infty$};
                         % Ligne a
                         \node at (-1.8*\Lrg,-1*\HtX) {$x-5$};
                         \node at (0*\Lrg,-1*\HtX) {$ $};
                         \node at (1*\Lrg,-1*\HtX) {$-$};
                         \node at (2*\Lrg,-1*\HtX) {$ $};
                         \node at (3*\Lrg,-1*\HtX) {$-$};
                         \node at (4*\Lrg,-1*\HtX) {$0$};
                         \node at (5*\Lrg,-1*\HtX) {$+$};
                         \node at (6*\Lrg,-1*\HtX) {$ $};
                         % Ligne b
                         \node at (-1.8*\Lrg,-2*\HtX) {$-3x+4$};
                         \node at (2*\Lrg,-2*\HtX) {$ $};
                         \node at (1*\Lrg,-2*\HtX) {$+$};
                         \node at (2*\Lrg,-2*\HtX) {$0$};
                         \node at (3*\Lrg,-2*\HtX) {$-$};
                         \node at (4*\Lrg,-2*\HtX) {$ $};
                         \node at (5*\Lrg,-2*\HtX) {$-$};
                         \node at (6*\Lrg,-2*\HtX) {$ $};
                         % Ligne c
                         \node at (-1.8*\Lrg,-3*\HtX) {$(x-5)(-3x+4)$};
                         \node at (0*\Lrg,-3*\HtX) {$ $};
                         \node at (1*\Lrg,-3*\HtX) {$-$};
                         \node at (2*\Lrg,-3*\HtX) {$0$};
                         \node at (3*\Lrg,-3*\HtX) {$+$};
                         \node at (4*\Lrg,-3*\HtX) {$0$};
                         \node at (5*\Lrg,-3*\HtX) {$-$};
                         \node at (6*\Lrg,-3*\HtX) {$ $};
                         % Encadrement
                         \draw[cadre] (\separateur,\haut) -- (\separateur, \lignec);
                         \draw[cadre] (\gauche,\haut) rectangle  (\droite, \lignec);
                         \draw[cadre] (\gauche,\lignex) -- (\droite,\lignex);
                         \draw[cadre] (\gauche,\lignea) -- (\droite,\lignea);
                         \draw[cadre] (\gauche,\ligneb) -- (\droite,\ligneb);
                    \end{tikzpicture}
               }
          \end{extern}
     \end{center}
     $(x-5)(-3x+4)$ est positif ou nul sur l'intervalle $\left[\frac{4}{3}; 5\right]$
     \par
     Pour plus de détails et d'autres exemples, consulter la fiche méthode : \mcLien{/methode/dresser-tableau-de-signes/}{Dresser un tableau de signes}
}
\cadre{rouge}{Théorème (Inéquation quotient)}{%id="t70"
     Un quotient $\frac{A(x)}{B(x)}$ est \textbf{défini} si et seulement si son \textbf{dénominateur} $B(x)$ est \textbf{non nul}.
     \par
     S'il est défini, il est \textbf{positif ou nul} si et seulement si $A(x)$ et $B(x)$ sont de \textbf{même signe} et il est \textbf{négatif ou nul} si et seulement si les deux facteurs $A(x)$ et $B(x)$ sont de \textbf{signes contraires}.
}
\bloc{orange}{Exemple}{%id="e70"
     Soit l'inéquation $\frac{2x-5}{x+2}\geqslant 0$
     \par
     Cette inéquation a un sens si $x+2 \neq 0$ donc si $x\neq -2$
     \par
     Le tableau de signe de $\frac{2x-5}{x+2}$ est :
     \begin{center}
          \begin{extern}%width="500" alt="Exemple tableau de signes d'un quotient"
               \resizebox{11cm}{!}{
                    \begin{tikzpicture}[scale=0.875]
                         % Styles
                         \tikzstyle{cadre}=[thin]
                         \tikzstyle{fleche}=[->,>=latex,thin]
                         \tikzstyle{nondefini}=[lightgray]
                         % Dimensions Modifiables
                         \def\Lrg{1.5}
                         \def\HtX{1.2}
                         \def\HtY{0.5}
                         % Dimensions Calculées
                         \def\lignex{-0.5*\HtX}
                         \def\lignea{-1.5*\HtX}
                         \def\ligneb{-2.5*\HtX}
                         \def\lignec{-3.5*\HtX}
                         \def\separateur{-0.5*\Lrg}
                         % Largeur du tableau
                         \def\gauche{-3.1*\Lrg}
                         \def\droite{6.5*\Lrg}
                         % Hauteur du tableau
                         \def\haut{0.5*\HtX}
                         \def\bas{-2.5*\HtX-2*\HtY}
                         % Pointillés
                         \draw[gray] (2*\Lrg,\lignex) -- (2*\Lrg,\ligneb);
                         \draw[gray] (4*\Lrg,\lignex) -- (4*\Lrg,\lignec);
                         \draw[double distance=2pt] (2*\Lrg,\ligneb) -- (2*\Lrg,\lignec);
                         % Ligne de l'abscisse : x
                         \node at (-1.8*\Lrg,0) {$x$};
                         \node at (0*\Lrg,0) {$-\infty$};
                         \node at (2*\Lrg,0) {$-2$};
                         \node at (4*\Lrg,0) {$\dfrac{5}{2}$};
                         \node at (6*\Lrg,0) {$+\infty$};
                         % Ligne a
                         \node at (-1.8*\Lrg,-1*\HtX) {$2x-5$};
                         \node at (0*\Lrg,-1*\HtX) {$ $};
                         \node at (1*\Lrg,-1*\HtX) {$-$};
                         \node at (2*\Lrg,-1*\HtX) {$ $};
                         \node at (3*\Lrg,-1*\HtX) {$-$};
                         \node at (4*\Lrg,-1*\HtX) {$0$};
                         \node at (5*\Lrg,-1*\HtX) {$+$};
                         \node at (6*\Lrg,-1*\HtX) {$ $};
                         % Ligne b
                         \node at (-1.8*\Lrg,-2*\HtX) {$x+2$};
                         \node at (2*\Lrg,-2*\HtX) {$ $};
                         \node at (1*\Lrg,-2*\HtX) {$-$};
                         \node at (2*\Lrg,-2*\HtX) {$0$};
                         \node at (3*\Lrg,-2*\HtX) {$+$};
                         \node at (4*\Lrg,-2*\HtX) {$ $};
                         \node at (5*\Lrg,-2*\HtX) {$+$};
                         \node at (6*\Lrg,-2*\HtX) {$ $};
                         % Ligne c
                         \node at (-1.8*\Lrg,-3*\HtX) {$\dfrac{2x-5}{x+2}$};
                         \node at (0*\Lrg,-3*\HtX) {$ $};
                         \node at (1*\Lrg,-3*\HtX) {$+$};
                         \node at (2*\Lrg,-3*\HtX) {$ $};
                         \node at (3*\Lrg,-3*\HtX) {$-$};
                         \node at (4*\Lrg,-3*\HtX) {$0$};
                         \node at (5*\Lrg,-3*\HtX) {$+$};
                         \node at (6*\Lrg,-3*\HtX) {$ $};
                         % Encadrement
                         \draw[cadre] (\separateur,\haut) -- (\separateur, \lignec);
                         \draw[cadre] (\gauche,\haut) rectangle  (\droite, \lignec);
                         \draw[cadre] (\gauche,\lignex) -- (\droite,\lignex);
                         \draw[cadre] (\gauche,\lignea) -- (\droite,\lignea);
                         \draw[cadre] (\gauche,\ligneb) -- (\droite,\ligneb);
                    \end{tikzpicture}
               }
          \end{extern}
     \end{center}
     $\frac{2x-5}{x+2}$ est positif ou nul sur l'ensemble $\left]-\infty ;-2\right[ \cup \left[\frac{5}{2}; +\infty \right[$
}
\cadre{vert}{Propriété}{%id="p80"
     Soit $f$ une fonction définie sur $D$ de courbe représentative $\mathscr{C}_f$ et $m$ un nombre réel.
     \begin{itemize}
          \item Les solutions de l'inéquation $f(x)\leqslant m$ sont les \textbf{abscisses} des points de la courbe $\mathscr{C}_f$ situés \textbf{au dessous} de la droite horizontale d'équation $y=m$(On inclut les points d'intersection si l'inégalité est large, on les exclut si l'inégalité est stricte.)
          \item De même, les solutions de l'inéquation $f(x)\geqslant m$ sont les \textbf{abscisses} des points de la courbe $\mathscr{C}_f$ situés \textbf{au dessus} de droite horizontale d'équation $y=m$
     \end{itemize}
}
\bloc{orange}{Exemple}{%id="e80"
     \begin{center}
          \begin{extern}%width="320" alt="inéquation et graphique"
               \resizebox{7cm}{!}{
                    \psset{xunit=1.0cm,yunit=1.0cm,algebraic=true,dimen=middle,linewidth=0.6pt,arrowsize=3pt}
                    \begin{pspicture*}(-2.5,-2.5)(5.5,6.)
                         \psaxes[xAxis=true,yAxis=true,Dx=10.,Dy=10.]{->}(0,0)(-2.5,-2.5)(5.5,6.)
                         \psplot[linewidth=0.8pt,linecolor=mcvert,plotpoints=200]{-2.5}{5.5}{(x-1.0)^(2.0)-2.0}
                         \psplot[linewidth=1pt,linecolor=blue,plotpoints=200]{-1}{3}{(x-1.0)^(2.0)-2.0}
                         \psplot[linewidth=0.8pt,linecolor=red,plotpoints=200]{-2.5}{5.5}{2.0}
                         \psline[linewidth=0.8pt,linestyle=dashed,dash=1pt 1pt](-1.,2.)(-1.,0.)
                         \psline[linewidth=0.8pt,linestyle=dashed,dash=1pt 1pt](3.,2.)(3.,0.)
                         \rput[tl](4.6,2.4){\red{$y=m$}}
                         \rput[tl](3.97,5.8){\color{mcvert}{$\mathscr{C}_f$}}
                         \rput[t](-1,-0.3){\blue{$x_1$}}\rput[t](3,-0.3){\blue{$x_2$}}
                         \psdots[dotsize=3pt 0,dotstyle=*,linecolor=blue](-1.,2.)\psdots[dotsize=3pt 0,dotstyle=*,linecolor=blue](3.,2.)
                         \psdots[dotsize=3pt 0,dotstyle=*,linecolor=blue](-1.,0.)\psdots[dotsize=3pt 0,dotstyle=*,linecolor=blue](3.,0.)
                         \psline[linewidth=1pt,linecolor=blue](-1.,0)(3.,0.)
                         \rput[bl](-0.5,-0.5){$O$}
                    \end{pspicture*}
               }
          \end{extern}
     \end{center}
     \begin{center}Sur la figure ci-dessus, l'inéquation $f(x) \leqslant m$ a pour solution l'intervalle $\left[x_1;x_2\right]$\end{center}
}

\end{document}

µ
\documentclass[a4paper]{article}

%================================================================================================================================
%
% Packages
%
%================================================================================================================================

\usepackage[T1]{fontenc} 	% pour caractères accentués
\usepackage[utf8]{inputenc}  % encodage utf8
\usepackage[french]{babel}	% langue : français
\usepackage{fourier}			% caractères plus lisibles
\usepackage[dvipsnames]{xcolor} % couleurs
\usepackage{fancyhdr}		% réglage header footer
\usepackage{needspace}		% empêcher sauts de page mal placés
\usepackage{graphicx}		% pour inclure des graphiques
\usepackage{enumitem,cprotect}		% personnalise les listes d'items (nécessaire pour ol, al ...)
\usepackage{hyperref}		% Liens hypertexte
\usepackage{pstricks,pst-all,pst-node,pstricks-add,pst-math,pst-plot,pst-tree,pst-eucl} % pstricks
\usepackage[a4paper,includeheadfoot,top=2cm,left=3cm, bottom=2cm,right=3cm]{geometry} % marges etc.
\usepackage{comment}			% commentaires multilignes
\usepackage{amsmath,environ} % maths (matrices, etc.)
\usepackage{amssymb,makeidx}
\usepackage{bm}				% bold maths
\usepackage{tabularx}		% tableaux
\usepackage{colortbl}		% tableaux en couleur
\usepackage{fontawesome}		% Fontawesome
\usepackage{environ}			% environment with command
\usepackage{fp}				% calculs pour ps-tricks
\usepackage{multido}			% pour ps tricks
\usepackage[np]{numprint}	% formattage nombre
\usepackage{tikz,tkz-tab} 			% package principal TikZ
\usepackage{pgfplots}   % axes
\usepackage{mathrsfs}    % cursives
\usepackage{calc}			% calcul taille boites
\usepackage[scaled=0.875]{helvet} % font sans serif
\usepackage{svg} % svg
\usepackage{scrextend} % local margin
\usepackage{scratch} %scratch
\usepackage{multicol} % colonnes
%\usepackage{infix-RPN,pst-func} % formule en notation polanaise inversée
\usepackage{listings}

%================================================================================================================================
%
% Réglages de base
%
%================================================================================================================================

\lstset{
language=Python,   % R code
literate=
{á}{{\'a}}1
{à}{{\`a}}1
{ã}{{\~a}}1
{é}{{\'e}}1
{è}{{\`e}}1
{ê}{{\^e}}1
{í}{{\'i}}1
{ó}{{\'o}}1
{õ}{{\~o}}1
{ú}{{\'u}}1
{ü}{{\"u}}1
{ç}{{\c{c}}}1
{~}{{ }}1
}


\definecolor{codegreen}{rgb}{0,0.6,0}
\definecolor{codegray}{rgb}{0.5,0.5,0.5}
\definecolor{codepurple}{rgb}{0.58,0,0.82}
\definecolor{backcolour}{rgb}{0.95,0.95,0.92}

\lstdefinestyle{mystyle}{
    backgroundcolor=\color{backcolour},   
    commentstyle=\color{codegreen},
    keywordstyle=\color{magenta},
    numberstyle=\tiny\color{codegray},
    stringstyle=\color{codepurple},
    basicstyle=\ttfamily\footnotesize,
    breakatwhitespace=false,         
    breaklines=true,                 
    captionpos=b,                    
    keepspaces=true,                 
    numbers=left,                    
xleftmargin=2em,
framexleftmargin=2em,            
    showspaces=false,                
    showstringspaces=false,
    showtabs=false,                  
    tabsize=2,
    upquote=true
}

\lstset{style=mystyle}


\lstset{style=mystyle}
\newcommand{\imgdir}{C:/laragon/www/newmc/assets/imgsvg/}
\newcommand{\imgsvgdir}{C:/laragon/www/newmc/assets/imgsvg/}

\definecolor{mcgris}{RGB}{220, 220, 220}% ancien~; pour compatibilité
\definecolor{mcbleu}{RGB}{52, 152, 219}
\definecolor{mcvert}{RGB}{125, 194, 70}
\definecolor{mcmauve}{RGB}{154, 0, 215}
\definecolor{mcorange}{RGB}{255, 96, 0}
\definecolor{mcturquoise}{RGB}{0, 153, 153}
\definecolor{mcrouge}{RGB}{255, 0, 0}
\definecolor{mclightvert}{RGB}{205, 234, 190}

\definecolor{gris}{RGB}{220, 220, 220}
\definecolor{bleu}{RGB}{52, 152, 219}
\definecolor{vert}{RGB}{125, 194, 70}
\definecolor{mauve}{RGB}{154, 0, 215}
\definecolor{orange}{RGB}{255, 96, 0}
\definecolor{turquoise}{RGB}{0, 153, 153}
\definecolor{rouge}{RGB}{255, 0, 0}
\definecolor{lightvert}{RGB}{205, 234, 190}
\setitemize[0]{label=\color{lightvert}  $\bullet$}

\pagestyle{fancy}
\renewcommand{\headrulewidth}{0.2pt}
\fancyhead[L]{maths-cours.fr}
\fancyhead[R]{\thepage}
\renewcommand{\footrulewidth}{0.2pt}
\fancyfoot[C]{}

\newcolumntype{C}{>{\centering\arraybackslash}X}
\newcolumntype{s}{>{\hsize=.35\hsize\arraybackslash}X}

\setlength{\parindent}{0pt}		 
\setlength{\parskip}{3mm}
\setlength{\headheight}{1cm}

\def\ebook{ebook}
\def\book{book}
\def\web{web}
\def\type{web}

\newcommand{\vect}[1]{\overrightarrow{\,\mathstrut#1\,}}

\def\Oij{$\left(\text{O}~;~\vect{\imath},~\vect{\jmath}\right)$}
\def\Oijk{$\left(\text{O}~;~\vect{\imath},~\vect{\jmath},~\vect{k}\right)$}
\def\Ouv{$\left(\text{O}~;~\vect{u},~\vect{v}\right)$}

\hypersetup{breaklinks=true, colorlinks = true, linkcolor = OliveGreen, urlcolor = OliveGreen, citecolor = OliveGreen, pdfauthor={Didier BONNEL - https://www.maths-cours.fr} } % supprime les bordures autour des liens

\renewcommand{\arg}[0]{\text{arg}}

\everymath{\displaystyle}

%================================================================================================================================
%
% Macros - Commandes
%
%================================================================================================================================

\newcommand\meta[2]{    			% Utilisé pour créer le post HTML.
	\def\titre{titre}
	\def\url{url}
	\def\arg{#1}
	\ifx\titre\arg
		\newcommand\maintitle{#2}
		\fancyhead[L]{#2}
		{\Large\sffamily \MakeUppercase{#2}}
		\vspace{1mm}\textcolor{mcvert}{\hrule}
	\fi 
	\ifx\url\arg
		\fancyfoot[L]{\href{https://www.maths-cours.fr#2}{\black \footnotesize{https://www.maths-cours.fr#2}}}
	\fi 
}


\newcommand\TitreC[1]{    		% Titre centré
     \needspace{3\baselineskip}
     \begin{center}\textbf{#1}\end{center}
}

\newcommand\newpar{    		% paragraphe
     \par
}

\newcommand\nosp {    		% commande vide (pas d'espace)
}
\newcommand{\id}[1]{} %ignore

\newcommand\boite[2]{				% Boite simple sans titre
	\vspace{5mm}
	\setlength{\fboxrule}{0.2mm}
	\setlength{\fboxsep}{5mm}	
	\fcolorbox{#1}{#1!3}{\makebox[\linewidth-2\fboxrule-2\fboxsep]{
  		\begin{minipage}[t]{\linewidth-2\fboxrule-4\fboxsep}\setlength{\parskip}{3mm}
  			 #2
  		\end{minipage}
	}}
	\vspace{5mm}
}

\newcommand\CBox[4]{				% Boites
	\vspace{5mm}
	\setlength{\fboxrule}{0.2mm}
	\setlength{\fboxsep}{5mm}
	
	\fcolorbox{#1}{#1!3}{\makebox[\linewidth-2\fboxrule-2\fboxsep]{
		\begin{minipage}[t]{1cm}\setlength{\parskip}{3mm}
	  		\textcolor{#1}{\LARGE{#2}}    
 	 	\end{minipage}  
  		\begin{minipage}[t]{\linewidth-2\fboxrule-4\fboxsep}\setlength{\parskip}{3mm}
			\raisebox{1.2mm}{\normalsize\sffamily{\textcolor{#1}{#3}}}						
  			 #4
  		\end{minipage}
	}}
	\vspace{5mm}
}

\newcommand\cadre[3]{				% Boites convertible html
	\par
	\vspace{2mm}
	\setlength{\fboxrule}{0.1mm}
	\setlength{\fboxsep}{5mm}
	\fcolorbox{#1}{white}{\makebox[\linewidth-2\fboxrule-2\fboxsep]{
  		\begin{minipage}[t]{\linewidth-2\fboxrule-4\fboxsep}\setlength{\parskip}{3mm}
			\raisebox{-2.5mm}{\sffamily \small{\textcolor{#1}{\MakeUppercase{#2}}}}		
			\par		
  			 #3
 	 		\end{minipage}
	}}
		\vspace{2mm}
	\par
}

\newcommand\bloc[3]{				% Boites convertible html sans bordure
     \needspace{2\baselineskip}
     {\sffamily \small{\textcolor{#1}{\MakeUppercase{#2}}}}    
		\par		
  			 #3
		\par
}

\newcommand\CHelp[1]{
     \CBox{Plum}{\faInfoCircle}{À RETENIR}{#1}
}

\newcommand\CUp[1]{
     \CBox{NavyBlue}{\faThumbsOUp}{EN PRATIQUE}{#1}
}

\newcommand\CInfo[1]{
     \CBox{Sepia}{\faArrowCircleRight}{REMARQUE}{#1}
}

\newcommand\CRedac[1]{
     \CBox{PineGreen}{\faEdit}{BIEN R\'EDIGER}{#1}
}

\newcommand\CError[1]{
     \CBox{Red}{\faExclamationTriangle}{ATTENTION}{#1}
}

\newcommand\TitreExo[2]{
\needspace{4\baselineskip}
 {\sffamily\large EXERCICE #1\ (\emph{#2 points})}
\vspace{5mm}
}

\newcommand\img[2]{
          \includegraphics[width=#2\paperwidth]{\imgdir#1}
}

\newcommand\imgsvg[2]{
       \begin{center}   \includegraphics[width=#2\paperwidth]{\imgsvgdir#1} \end{center}
}


\newcommand\Lien[2]{
     \href{#1}{#2 \tiny \faExternalLink}
}
\newcommand\mcLien[2]{
     \href{https~://www.maths-cours.fr/#1}{#2 \tiny \faExternalLink}
}

\newcommand{\euro}{\eurologo{}}

%================================================================================================================================
%
% Macros - Environement
%
%================================================================================================================================

\newenvironment{tex}{ %
}
{%
}

\newenvironment{indente}{ %
	\setlength\parindent{10mm}
}

{
	\setlength\parindent{0mm}
}

\newenvironment{corrige}{%
     \needspace{3\baselineskip}
     \medskip
     \textbf{\textsc{Corrigé}}
     \medskip
}
{
}

\newenvironment{extern}{%
     \begin{center}
     }
     {
     \end{center}
}

\NewEnviron{code}{%
	\par
     \boite{gray}{\texttt{%
     \BODY
     }}
     \par
}

\newenvironment{vbloc}{% boite sans cadre empeche saut de page
     \begin{minipage}[t]{\linewidth}
     }
     {
     \end{minipage}
}
\NewEnviron{h2}{%
    \needspace{3\baselineskip}
    \vspace{0.6cm}
	\noindent \MakeUppercase{\sffamily \large \BODY}
	\vspace{1mm}\textcolor{mcgris}{\hrule}\vspace{0.4cm}
	\par
}{}

\NewEnviron{h3}{%
    \needspace{3\baselineskip}
	\vspace{5mm}
	\textsc{\BODY}
	\par
}

\NewEnviron{margeneg}{ %
\begin{addmargin}[-1cm]{0cm}
\BODY
\end{addmargin}
}

\NewEnviron{html}{%
}

\begin{document}
\meta{url}{/cours/fonctions-lineaires-affines/}
\meta{pid}{123}
\meta{titre}{Fonctions linéaires et affines}
\meta{type}{cours}
\begin{h2}1. Fonctions linéaires\end{h2}
\cadre{bleu}{Définition}{% id="d10"
     Une fonction \textbf{linéaire} est une fonction $f$ définie sur $\mathbb{R}$ par une formule du type : $x\mapsto ax$ où $a \in  \mathbb{R}$.
     \par
     $a$ s'appelle le \textbf{coefficient de la fonction $f$}.
}
\bloc{cyan}{Remarque}{% id="r10"
     La définition ci-dessus indique que si $f$ est une fonction linéaire, les valeurs de $f\left(x\right)$ sont proportionnelles aux valeurs de $x$, le coefficient de proportionnalité étant le coefficient $a$ de la fonction $f$.
}
\cadre{vert}{Propriété}{% id="p20"
     La courbe représentative d'une fonction linéaire est une \textbf{droite qui passe par l'origine} du repère.
}
\bloc{orange}{Exemple}{% id="e20"
     \begin{center}
          \begin{extern}%width="230" alt="fonction linéaire"
               \resizebox{6cm}{!}{%
                    % -+-+-+ variables modifiables
                    \def\fonction{1.5*x}
                    \def\xmin{-2.5}
                    \def\xmax{2.5}
                    \def\ymin{-2.5}
                    \def\ymax{4.5}
                    \def\xunit{1.5}  % unités en cm
                    \def\yunit{1.5}
                    \psset{xunit=\xunit,yunit=\yunit,algebraic=true}
                    \fontsize{15pt}{15pt}\selectfont
                    \begin{pspicture*}[linewidth=1pt](\xmin,\ymin)(\xmax,\ymax)
                         %      \psgrid[gridcolor=mcgris, subgriddiv=5, gridlabels=0pt](\xmin,\ymin)(\xmax,\ymax)
                         \psaxes[linewidth=0.75pt,Dx=1,Dy=1]{->}(0,0)(\xmin,\ymin)(\xmax,\ymax)
                         \psplot[plotpoints=2000,linecolor=blue]{\xmin}{\xmax}{\fonction}
                         \rput[tr](-0.2,-0.2){$O$}
                         \rput[tl](1.9,4){$\color{blue} \mathcal{C}_f$}
                    \end{pspicture*}
               }
          \end{extern}
     \end{center}
     \begin{center}\textit{Représentation graphique de la fonction linéaire $x\mapsto \frac{3}{2}x$}\end{center}
}
\cadre{vert}{Propriété}{% id="p30"
     Soit $f$ une fonction linéaire.
     \par
     Pour tous réels $x$ et $x^{\prime}$ : $ f\left(x+x^{\prime}\right)=f\left(x\right)+f\left(x^{\prime}\right)$
     \par
     Pour tous réels $k$ et $x$ : $ f\left(kx\right)=kf\left(x\right)$
}
\begin{h2}2. Fonctions affines\end{h2}
\cadre{bleu}{Définition}{% id="d50"
     Une fonction \textbf{affine} est une fonction définie sur $\mathbb{R}$ par une formule du type : $x\mapsto ax+b$ où $a \in  \mathbb{R}$ et $b \in  \mathbb{R}$.
}
\bloc{cyan}{Remarque}{% id="r50"
     Si $b=0$, la fonction est linéaire. Les fonctions linéaires sont donc des cas particuliers des fonctions affines.
}
\cadre{vert}{Propriété}{% id="p60"
     La courbe représentative d'une fonction affine est une \textbf{droite}.
     \par
     $a$ est le \textbf{coefficient directeur} de la droite et $b$ son \textbf{ordonnée à l'origine}.
}
\bloc{orange}{Exemple}{% id="e60"
     \begin{center}
          \begin{extern}%width="400" alt="fonction linéaire"
               \resizebox{11cm}{!}{%
                    % -+-+-+ variables modifiables
                    \def\fonction{0.5*x+2}
                    \def\xmin{-4.5}
                    \def\xmax{4.5}
                    \def\ymin{-0.9}
                    \def\ymax{5.2}
                    \def\xunit{1.5}  % unités en cm
                    \def\yunit{1.5}
                    \psset{xunit=\xunit,yunit=\yunit,algebraic=true}
                    \fontsize{15pt}{15pt}\selectfont
                    \begin{pspicture*}[linewidth=1pt](\xmin,\ymin)(\xmax,\ymax)
                         %      \psgrid[gridcolor=mcgris, subgriddiv=5, gridlabels=0pt](\xmin,\ymin)(\xmax,\ymax)
                         \psaxes[linewidth=0.75pt,Dx=1,Dy=1]{->}(0,0)(\xmin,\ymin)(\xmax,\ymax)
                         \psplot[plotpoints=2000,linecolor=blue]{\xmin}{\xmax}{\fonction}
                         \rput[tr](-0.2,-0.2){$O$}
                         \rput[tl](4,4.7){$\color{blue} \mathcal{C}_f$}
                    \end{pspicture*}
               }
          \end{extern}
     \end{center}
     \begin{center}\textit{Représentation graphique de la fonction affine $x\mapsto \frac{1}{2}x+2$}\end{center}
}
\cadre{vert}{Propriété}{% id="p70"
     Soit $f$ une fonction affine de représentation graphique $\mathscr D$ et soient $A$ et $B$ deux points de $\mathscr D$.
     \par
     Le rapport $\dfrac{y_B-y_A}{x_B-x_A}$ ne dépend pas des points $A$ et $B$ choisis et est égal au coefficient directeur de la droite $\mathscr D$ :
     \begin{center}$a = \dfrac{y_B-y_A}{x_B-x_A}$\end{center}
}
\begin{center}
     \begin{extern}%width="400" alt="coefficient directeur"
          \resizebox{11cm}{!}{%
               % -+-+-+ variables modifiables
               \def\fonction{0.5*x+2}
               \def\xmin{-4.5}
               \def\xmax{4.5}
               \def\ymin{-0.9}
               \def\ymax{5.2}
               \def\xunit{1.5}  % unités en cm
               \def\yunit{1.5}
               \psset{xunit=\xunit,yunit=\yunit,algebraic=true}
               \fontsize{15pt}{15pt}\selectfont
               \begin{pspicture*}[linewidth=1pt](\xmin,\ymin)(\xmax,\ymax)
                    %      \psgrid[gridcolor=mcgris, subgriddiv=5, gridlabels=0pt](\xmin,\ymin)(\xmax,\ymax)
                    \psaxes[linewidth=0.75pt,Dx=1,Dy=1]{->}(0,0)(\xmin,\ymin)(\xmax,\ymax)
                    \psplot[plotpoints=2000,linecolor=blue]{\xmin}{\xmax}{\fonction}
                    \rput[tr](-0.2,-0.2){$O$}
                    \rput[tl](4,4.5){$\color{blue} \mathcal{D}$}
                    \psdots[linecolor=red](-2,1)\psdots[linecolor=red](1,2.5)
                    \psline[linewidth=0.75pt,linecolor=mcvert,arrowsize=6pt]{->}(-2,1)(1,1)
                    \psline[linewidth=0.75pt,linecolor=mcmauve,arrowsize=6pt]{->}(1,1)(1,2.5)
                    \rput[tr](-0.2,0.65){$\color{mcvert} x_B-x_A$}
                    \rput[br](-2.1,1.1){$\color{red} A$} \rput[br](0.9,2.6){$\color{red} B$}
                    \rput[l](1.2,1.7){$\color{mcmauve} y_B-y_A$}
               \end{pspicture*}
          }
     \end{extern}
\end{center}
\begin{center}\textit{Coefficient directeur de $\mathscr{D}$ : $a = \dfrac{y_B-y_A}{x_B-x_A}=\dfrac{1,5}{3}=0,5$}\end{center}
\cadre{rouge}{Théorème}{% id="t80"
     Une fonction affine $x \longmapsto ax+b$ est :
     \begin{itemize}
          \item \textbf{strictement croissante} si $a$ est \textbf{strictement positif}.
          \item \textbf{strictement décroissante} si $a$ est \textbf{strictement négatif}.
          \item \textbf{constante} si $a$ est \textbf{nul}.
     \end{itemize}
}
\bloc{orange}{Démonstration}{% id="m80"
     Démontrons, par exemple, que la fonction $f : x\mapsto ax+b$ est strictement décroissante si $a < 0$.
     \par
     Soient deux réels $x_1$ et $x_2$ tels que $x_1 < x_2$
     \par
     Alors $ax_1 > ax_2$ (on change le sens de l'inégalité car on multiplie par un réel négatif) donc
     \par
     $ax_1+b > ax_2+b$ c'est à dire :
     \par
     $f\left(x_1\right) > f\left(x_2\right)$
     \par
     Le sens de l'inégalité est inversé donc $f$ est strictement décroissante sur $\mathbb{R}$.
}
\bloc{cyan}{Remarque}{% id="r80"
     Ce théorème s'applique aussi aux fonctions linéaires puisque les fonctions linéaires sont des fonctions affines particulières.
}

\end{document}
µ
\documentclass[a4paper]{article}

%================================================================================================================================
%
% Packages
%
%================================================================================================================================

\usepackage[T1]{fontenc} 	% pour caractères accentués
\usepackage[utf8]{inputenc}  % encodage utf8
\usepackage[french]{babel}	% langue : français
\usepackage{fourier}			% caractères plus lisibles
\usepackage[dvipsnames]{xcolor} % couleurs
\usepackage{fancyhdr}		% réglage header footer
\usepackage{needspace}		% empêcher sauts de page mal placés
\usepackage{graphicx}		% pour inclure des graphiques
\usepackage{enumitem,cprotect}		% personnalise les listes d'items (nécessaire pour ol, al ...)
\usepackage{hyperref}		% Liens hypertexte
\usepackage{pstricks,pst-all,pst-node,pstricks-add,pst-math,pst-plot,pst-tree,pst-eucl} % pstricks
\usepackage[a4paper,includeheadfoot,top=2cm,left=3cm, bottom=2cm,right=3cm]{geometry} % marges etc.
\usepackage{comment}			% commentaires multilignes
\usepackage{amsmath,environ} % maths (matrices, etc.)
\usepackage{amssymb,makeidx}
\usepackage{bm}				% bold maths
\usepackage{tabularx}		% tableaux
\usepackage{colortbl}		% tableaux en couleur
\usepackage{fontawesome}		% Fontawesome
\usepackage{environ}			% environment with command
\usepackage{fp}				% calculs pour ps-tricks
\usepackage{multido}			% pour ps tricks
\usepackage[np]{numprint}	% formattage nombre
\usepackage{tikz,tkz-tab} 			% package principal TikZ
\usepackage{pgfplots}   % axes
\usepackage{mathrsfs}    % cursives
\usepackage{calc}			% calcul taille boites
\usepackage[scaled=0.875]{helvet} % font sans serif
\usepackage{svg} % svg
\usepackage{scrextend} % local margin
\usepackage{scratch} %scratch
\usepackage{multicol} % colonnes
%\usepackage{infix-RPN,pst-func} % formule en notation polanaise inversée
\usepackage{listings}

%================================================================================================================================
%
% Réglages de base
%
%================================================================================================================================

\lstset{
language=Python,   % R code
literate=
{á}{{\'a}}1
{à}{{\`a}}1
{ã}{{\~a}}1
{é}{{\'e}}1
{è}{{\`e}}1
{ê}{{\^e}}1
{í}{{\'i}}1
{ó}{{\'o}}1
{õ}{{\~o}}1
{ú}{{\'u}}1
{ü}{{\"u}}1
{ç}{{\c{c}}}1
{~}{{ }}1
}


\definecolor{codegreen}{rgb}{0,0.6,0}
\definecolor{codegray}{rgb}{0.5,0.5,0.5}
\definecolor{codepurple}{rgb}{0.58,0,0.82}
\definecolor{backcolour}{rgb}{0.95,0.95,0.92}

\lstdefinestyle{mystyle}{
    backgroundcolor=\color{backcolour},   
    commentstyle=\color{codegreen},
    keywordstyle=\color{magenta},
    numberstyle=\tiny\color{codegray},
    stringstyle=\color{codepurple},
    basicstyle=\ttfamily\footnotesize,
    breakatwhitespace=false,         
    breaklines=true,                 
    captionpos=b,                    
    keepspaces=true,                 
    numbers=left,                    
xleftmargin=2em,
framexleftmargin=2em,            
    showspaces=false,                
    showstringspaces=false,
    showtabs=false,                  
    tabsize=2,
    upquote=true
}

\lstset{style=mystyle}


\lstset{style=mystyle}
\newcommand{\imgdir}{C:/laragon/www/newmc/assets/imgsvg/}
\newcommand{\imgsvgdir}{C:/laragon/www/newmc/assets/imgsvg/}

\definecolor{mcgris}{RGB}{220, 220, 220}% ancien~; pour compatibilité
\definecolor{mcbleu}{RGB}{52, 152, 219}
\definecolor{mcvert}{RGB}{125, 194, 70}
\definecolor{mcmauve}{RGB}{154, 0, 215}
\definecolor{mcorange}{RGB}{255, 96, 0}
\definecolor{mcturquoise}{RGB}{0, 153, 153}
\definecolor{mcrouge}{RGB}{255, 0, 0}
\definecolor{mclightvert}{RGB}{205, 234, 190}

\definecolor{gris}{RGB}{220, 220, 220}
\definecolor{bleu}{RGB}{52, 152, 219}
\definecolor{vert}{RGB}{125, 194, 70}
\definecolor{mauve}{RGB}{154, 0, 215}
\definecolor{orange}{RGB}{255, 96, 0}
\definecolor{turquoise}{RGB}{0, 153, 153}
\definecolor{rouge}{RGB}{255, 0, 0}
\definecolor{lightvert}{RGB}{205, 234, 190}
\setitemize[0]{label=\color{lightvert}  $\bullet$}

\pagestyle{fancy}
\renewcommand{\headrulewidth}{0.2pt}
\fancyhead[L]{maths-cours.fr}
\fancyhead[R]{\thepage}
\renewcommand{\footrulewidth}{0.2pt}
\fancyfoot[C]{}

\newcolumntype{C}{>{\centering\arraybackslash}X}
\newcolumntype{s}{>{\hsize=.35\hsize\arraybackslash}X}

\setlength{\parindent}{0pt}		 
\setlength{\parskip}{3mm}
\setlength{\headheight}{1cm}

\def\ebook{ebook}
\def\book{book}
\def\web{web}
\def\type{web}

\newcommand{\vect}[1]{\overrightarrow{\,\mathstrut#1\,}}

\def\Oij{$\left(\text{O}~;~\vect{\imath},~\vect{\jmath}\right)$}
\def\Oijk{$\left(\text{O}~;~\vect{\imath},~\vect{\jmath},~\vect{k}\right)$}
\def\Ouv{$\left(\text{O}~;~\vect{u},~\vect{v}\right)$}

\hypersetup{breaklinks=true, colorlinks = true, linkcolor = OliveGreen, urlcolor = OliveGreen, citecolor = OliveGreen, pdfauthor={Didier BONNEL - https://www.maths-cours.fr} } % supprime les bordures autour des liens

\renewcommand{\arg}[0]{\text{arg}}

\everymath{\displaystyle}

%================================================================================================================================
%
% Macros - Commandes
%
%================================================================================================================================

\newcommand\meta[2]{    			% Utilisé pour créer le post HTML.
	\def\titre{titre}
	\def\url{url}
	\def\arg{#1}
	\ifx\titre\arg
		\newcommand\maintitle{#2}
		\fancyhead[L]{#2}
		{\Large\sffamily \MakeUppercase{#2}}
		\vspace{1mm}\textcolor{mcvert}{\hrule}
	\fi 
	\ifx\url\arg
		\fancyfoot[L]{\href{https://www.maths-cours.fr#2}{\black \footnotesize{https://www.maths-cours.fr#2}}}
	\fi 
}


\newcommand\TitreC[1]{    		% Titre centré
     \needspace{3\baselineskip}
     \begin{center}\textbf{#1}\end{center}
}

\newcommand\newpar{    		% paragraphe
     \par
}

\newcommand\nosp {    		% commande vide (pas d'espace)
}
\newcommand{\id}[1]{} %ignore

\newcommand\boite[2]{				% Boite simple sans titre
	\vspace{5mm}
	\setlength{\fboxrule}{0.2mm}
	\setlength{\fboxsep}{5mm}	
	\fcolorbox{#1}{#1!3}{\makebox[\linewidth-2\fboxrule-2\fboxsep]{
  		\begin{minipage}[t]{\linewidth-2\fboxrule-4\fboxsep}\setlength{\parskip}{3mm}
  			 #2
  		\end{minipage}
	}}
	\vspace{5mm}
}

\newcommand\CBox[4]{				% Boites
	\vspace{5mm}
	\setlength{\fboxrule}{0.2mm}
	\setlength{\fboxsep}{5mm}
	
	\fcolorbox{#1}{#1!3}{\makebox[\linewidth-2\fboxrule-2\fboxsep]{
		\begin{minipage}[t]{1cm}\setlength{\parskip}{3mm}
	  		\textcolor{#1}{\LARGE{#2}}    
 	 	\end{minipage}  
  		\begin{minipage}[t]{\linewidth-2\fboxrule-4\fboxsep}\setlength{\parskip}{3mm}
			\raisebox{1.2mm}{\normalsize\sffamily{\textcolor{#1}{#3}}}						
  			 #4
  		\end{minipage}
	}}
	\vspace{5mm}
}

\newcommand\cadre[3]{				% Boites convertible html
	\par
	\vspace{2mm}
	\setlength{\fboxrule}{0.1mm}
	\setlength{\fboxsep}{5mm}
	\fcolorbox{#1}{white}{\makebox[\linewidth-2\fboxrule-2\fboxsep]{
  		\begin{minipage}[t]{\linewidth-2\fboxrule-4\fboxsep}\setlength{\parskip}{3mm}
			\raisebox{-2.5mm}{\sffamily \small{\textcolor{#1}{\MakeUppercase{#2}}}}		
			\par		
  			 #3
 	 		\end{minipage}
	}}
		\vspace{2mm}
	\par
}

\newcommand\bloc[3]{				% Boites convertible html sans bordure
     \needspace{2\baselineskip}
     {\sffamily \small{\textcolor{#1}{\MakeUppercase{#2}}}}    
		\par		
  			 #3
		\par
}

\newcommand\CHelp[1]{
     \CBox{Plum}{\faInfoCircle}{À RETENIR}{#1}
}

\newcommand\CUp[1]{
     \CBox{NavyBlue}{\faThumbsOUp}{EN PRATIQUE}{#1}
}

\newcommand\CInfo[1]{
     \CBox{Sepia}{\faArrowCircleRight}{REMARQUE}{#1}
}

\newcommand\CRedac[1]{
     \CBox{PineGreen}{\faEdit}{BIEN R\'EDIGER}{#1}
}

\newcommand\CError[1]{
     \CBox{Red}{\faExclamationTriangle}{ATTENTION}{#1}
}

\newcommand\TitreExo[2]{
\needspace{4\baselineskip}
 {\sffamily\large EXERCICE #1\ (\emph{#2 points})}
\vspace{5mm}
}

\newcommand\img[2]{
          \includegraphics[width=#2\paperwidth]{\imgdir#1}
}

\newcommand\imgsvg[2]{
       \begin{center}   \includegraphics[width=#2\paperwidth]{\imgsvgdir#1} \end{center}
}


\newcommand\Lien[2]{
     \href{#1}{#2 \tiny \faExternalLink}
}
\newcommand\mcLien[2]{
     \href{https~://www.maths-cours.fr/#1}{#2 \tiny \faExternalLink}
}

\newcommand{\euro}{\eurologo{}}

%================================================================================================================================
%
% Macros - Environement
%
%================================================================================================================================

\newenvironment{tex}{ %
}
{%
}

\newenvironment{indente}{ %
	\setlength\parindent{10mm}
}

{
	\setlength\parindent{0mm}
}

\newenvironment{corrige}{%
     \needspace{3\baselineskip}
     \medskip
     \textbf{\textsc{Corrigé}}
     \medskip
}
{
}

\newenvironment{extern}{%
     \begin{center}
     }
     {
     \end{center}
}

\NewEnviron{code}{%
	\par
     \boite{gray}{\texttt{%
     \BODY
     }}
     \par
}

\newenvironment{vbloc}{% boite sans cadre empeche saut de page
     \begin{minipage}[t]{\linewidth}
     }
     {
     \end{minipage}
}
\NewEnviron{h2}{%
    \needspace{3\baselineskip}
    \vspace{0.6cm}
	\noindent \MakeUppercase{\sffamily \large \BODY}
	\vspace{1mm}\textcolor{mcgris}{\hrule}\vspace{0.4cm}
	\par
}{}

\NewEnviron{h3}{%
    \needspace{3\baselineskip}
	\vspace{5mm}
	\textsc{\BODY}
	\par
}

\NewEnviron{margeneg}{ %
\begin{addmargin}[-1cm]{0cm}
\BODY
\end{addmargin}
}

\NewEnviron{html}{%
}

\begin{document}
\meta{url}{/cours/fonction-carre/}
\meta{pid}{132}
\meta{titre}{Fonction « carré » et second degré}
\meta{type}{cours}
\begin{h2}I. La fonction «carré»\end{h2}
\cadre{bleu}{Définition}{% id="d10"
     La fonction "\textbf{carré}" est la fonction  définie sur $\mathbb{R}$ par : $x\mapsto x^2$.
     \par
     Sa courbe représentative est une \textbf{parabole}.
     \par
     Elle est symétrique par rapport à l'\textbf{axe des ordonnées}.
}
\begin{center}
     \begin{extern}%width="230" alt="fonction carré"
          \resizebox{5.5cm}{!}{%
               % -+-+-+ variables modifiables
               \def\fonction{x*x}
               \def\xmin{-2.5}
               \def\xmax{2.5}
               \def\ymin{-0.9}
               \def\ymax{5.5}
               \def\xunit{1.5}  % unités en cm
               \def\yunit{1.5}
               \psset{xunit=\xunit,yunit=\yunit,algebraic=true}
               \fontsize{15pt}{15pt}\selectfont
               \begin{pspicture*}[linewidth=1pt](\xmin,\ymin)(\xmax,\ymax)
                    %      \psgrid[gridcolor=mcgris, subgriddiv=5, gridlabels=0pt](\xmin,\ymin)(\xmax,\ymax)
                    \psaxes[linewidth=0.75pt,Dx=1,Dy=1]{->}(0,0)(\xmin,\ymin)(\xmax,\ymax)
                    \psplot[plotpoints=2000,linecolor=blue]{\xmin}{\xmax}{\fonction}
                    \rput[tr](-0.2,-0.2){$O$}
                    \rput[tr](2.1,5.2){$\color{blue} \mathcal{C}_f$}
               \end{pspicture*}
          }
     \end{extern}
\end{center}
\cadre{vert}{Propriété}{% id="p20"
     La fonction carré est strictement décroissante sur $\left]-\infty ; 0\right[$ et strictement croissante sur $\left]0;  \infty \right[$. Elle admet en 0 un minimum égal à 0.
}
\begin{center}
     \begin{extern}%width="340" alt="Tableau de variation polynôme du second degré"
          \begin{tikzpicture}[scale=0.875]
               % Styles
               \tikzstyle{cadre}=[thin]
               \tikzstyle{fleche}=[->,>=latex,thin]
               \tikzstyle{nondefini}=[lightgray]
               % Dimensions Modifiables
               \def\Lrg{1.5}
               \def\HtX{1}
               \def\HtY{0.5}
               % Dimensions Calculées
               \def\lignex{-0.5*\HtX}
               \def\lignef{-1.5*\HtX}
               \def\separateur{-0.5*\Lrg}
               % Largeur du tableau
               \def\gauche{-1.5*\Lrg}
               \def\droite{4.5*\Lrg}
               % Hauteur du tableau
               \def\haut{0.5*\HtX}
               \def\bas{-1.5*\HtX-2*\HtY}
               % Pointillés
               \draw[lightgray] (2*\Lrg,\lignex) -- (2*\Lrg,\lignef);
               \draw[lightgray] (2*\Lrg,\lignef) -- (2*\Lrg,\bas);
               % Ligne de l'abscisse : x
               \node at (-1*\Lrg,0) {$x$};
               \node at (0*\Lrg,0) {$ -\infty$};
               \node at (2*\Lrg,0) {$0$};
               \node at (4*\Lrg,0) {$+\infty $};
               % Ligne de la fonction : f(x)
               \node  at (-1*\Lrg,{-1*\HtX+(-1)*\HtY}) {$f(x)$};
               \node (f1) at (0*\Lrg,{-1*\HtX+(0)*\HtY}) {$ $};
               \node (f2) at (2*\Lrg,{-1*\HtX+(-2)*\HtY}) {$0$};
               \node (f3) at (4*\Lrg,{-1*\HtX+(0)*\HtY}) {$ $};
               % Flèches
               \draw[fleche] (f1) -- (f2);
               \draw[fleche] (f2) -- (f3);
               % Encadrement
               \draw[cadre] (\separateur,\haut) -- (\separateur,\bas);
               \draw[cadre] (\gauche,\haut) rectangle  (\droite,\bas);
               \draw[cadre] (\gauche,\lignex) -- (\droite,\lignex);
          \end{tikzpicture}
     \end{extern}
\end{center}
\begin{center}\textit{Tableau de variations de la fonction carrée}\end{center}
\bloc{cyan}{Démonstration}{% id="m20"
     Démontrons par exemple que la fonction carré est décroissante sur $\left]-\infty ; 0\right[$.
     \par
     Notons $f : x\mapsto x^2$ et soient $x_1$ et $x_2$, deux réels quelconques tels que $x_1 < x_2 < 0$.
     \par
     Alors :
     \par
     $f\left(x_1\right)-f\left(x_2\right)=x_1^2-x_2^2=\left(x_1-x_2\right)\left(x_1+x_2\right)$
     \par
     Or $x_1-x_2 < 0$ car $x_1 < x_2$
     \par
     et $x_1+x_2 < 0$  car $x_1$ et $x_2$ sont tous les deux négatifs.
     \par
     Donc le produit $\left(x_1-x_2\right)\left(x_1+x_2\right)$ est positif.
     \par
     On en déduit $f\left(x_1\right)-f\left(x_2\right) > 0$ donc $f\left(x_1\right) > f\left(x_2\right)$
     \par
     $x_1 < x_2 < 0 \Rightarrow  f\left(x_1\right) > f\left(x_2\right) $, donc la fonction $f$ est strictement décroissante sur $\left]-\infty ; 0\right[$.
}
\cadre{vert}{Propriété}{% id="p30"
     Soit $a$ un nombre réel. Dans $\mathbb{R}$, l'équation $x^2=a$
     \begin{itemize}
          \item n'admet \textbf{aucune} solution \textbf{si $a < 0$}
          \item admet $x=0$ comme \textbf{unique} solution \textbf{si $a=0$}
          \item admet \textbf{deux} solutions $\sqrt{a}$ et $-\sqrt{a}$ \textbf{ si $a > 0$}
     \end{itemize}
}
\bloc{orange}{Exemples}{% id="e30"
     \begin{itemize}
          \item L'équation $x^2=2$ admet deux solutions : $\sqrt{2}$ et $-\sqrt{2}$.
          \item L'équation $x^2+1=0$ est équivalente à $x^2=-1$. Elle n'admet donc aucune solution réelle.
     \end{itemize}
}
\begin{h2}II. Fonctions polynômes du second degré\end{h2}
\cadre{bleu}{Définition}{% id="d50"
     Une fonction \textbf{polynôme du second degré} est une fonction  définie sur $\mathbb{R}$ par : $x\mapsto ax^2+bx+c$.
     \par
     où $a$,$b$ et $c$ sont des réels appelés \textbf{coefficients} et $a\neq 0$
     \par
     Sa courbe représentative est une \textbf{parabole}, elle admet un axe de symétrie parallèle à l'axe des ordonnées.
}
\bloc{cyan}{Remarque}{% id="r50"
     Une expression de la forme $ax^2+bx+c$ avec $a\neq 0$ est la \textbf{forme développée} d'un polynôme du second degré.
     \par
     Une expression de la forme $a\left(x-x_1\right)\left(x-x_2\right)$ avec $a\neq 0$ est la \textbf{forme factorisée} d'un polynôme du second degré.
}
\cadre{rouge}{Théorème}{% id="t60"
     Une fonction polynôme du second degré est :
     \textbf{Si $a > 0$} :
     \par
     strictement décroissante sur $\left]-\infty ; \frac{-b}{2a}\right]$ et strictement croissante sur $\left[\frac{-b}{2a}; +\infty \right[$.
     \textbf{Si $a < 0$} :
     \par
     strictement croissante sur $\left]-\infty ; \frac{-b}{2a}\right]$ et strictement décroissante sur $\left[\frac{-b}{2a}; +\infty \right[$.
}
\bigbreak
\begin{center}
     \begin{extern}%width="340" alt="Tableau de variation polynôme du second degré pour a > 0"
          \begin{tikzpicture}[scale=0.875]
               % Styles
               \tikzstyle{cadre}=[thin]
               \tikzstyle{fleche}=[->,>=latex,thin]
               \tikzstyle{nondefini}=[lightgray]
               % Dimensions Modifiables
               \def\Lrg{1.5}
               \def\HtX{1}
               \def\HtY{0.5}
               % Dimensions Calculées
               \def\lignex{-0.7*\HtX}
               \def\lignef{-1.5*\HtX}
               \def\separateur{-0.5*\Lrg}
               % Largeur du tableau
               \def\gauche{-1.5*\Lrg}
               \def\droite{4.5*\Lrg}
               % Hauteur du tableau
               \def\haut{0.5*\HtX}
               \def\bas{-1.5*\HtX-2*\HtY}
               % Pointillés
               \draw[lightgray] (2*\Lrg,\lignex) -- (2*\Lrg,\lignef);
               \draw[lightgray] (2*\Lrg,\lignef) -- (2*\Lrg,\bas);
               % Ligne de l'abscisse : x
               \node at (-1*\Lrg,-0.1) {$x$};
               \node at (0*\Lrg,-0.1) {$ -\infty$};
               \node at (2*\Lrg,-0.1) {$-\dfrac{b}{2a}$};
               \node at (4*\Lrg,-0.1) {$+\infty $};
               % Ligne de la fonction : f(x)
               \node  at (-1*\Lrg,{-1*\HtX+(-1)*\HtY}) {$f(x)$};
               \node (f1) at (0*\Lrg,{-1*\HtX+(0)*\HtY}) {$ $};
               \node (f2) at (2*\Lrg,{-1*\HtX+(-2)*\HtY}) {$\text{min}$};
               \node (f3) at (4*\Lrg,{-1*\HtX+(0)*\HtY}) {$ $};
               % Flèches
               \draw[fleche] (f1) -- (f2);
               \draw[fleche] (f2) -- (f3);
               % Encadrement
               \draw[cadre] (\separateur,\haut) -- (\separateur,\bas);
               \draw[cadre] (\gauche,\haut) rectangle  (\droite,\bas);
               \draw[cadre] (\gauche,\lignex) -- (\droite,\lignex);
          \end{tikzpicture}
     \end{extern}
\end{center}
\begin{center}\textit{Tableau de variations d'une fonction polynôme du second degré pour $a > 0$ }\end{center}
\bigbreak
\begin{center}
     \begin{extern}%width="340" alt="Tableau de variation polynôme du second degré pour a < 0"
          \begin{tikzpicture}[scale=0.875]
               % Styles
               \tikzstyle{cadre}=[thin]
               \tikzstyle{fleche}=[->,>=latex,thin]
               \tikzstyle{nondefini}=[lightgray]
               % Dimensions Modifiables
               \def\Lrg{1.5}
               \def\HtX{1}
               \def\HtY{0.5}
               % Dimensions Calculées
               \def\lignex{-0.7*\HtX}
               \def\lignef{-1.5*\HtX}
               \def\separateur{-0.5*\Lrg}
               % Largeur du tableau
               \def\gauche{-1.5*\Lrg}
               \def\droite{4.5*\Lrg}
               % Hauteur du tableau
               \def\haut{0.5*\HtX}
               \def\bas{-1.5*\HtX-2*\HtY}
               % Pointillés
               \draw[lightgray] (2*\Lrg,\lignex) -- (2*\Lrg,\lignef);
               \draw[lightgray] (2*\Lrg,\lignef) -- (2*\Lrg,\bas);
               % Ligne de l'abscisse : x
               \node at (-1*\Lrg,-0.1) {$x$};
               \node at (0*\Lrg,-0.1) {$ -\infty$};
               \node at (2*\Lrg,-0.1) {$-\dfrac{b}{2a}$};
               \node at (4*\Lrg,-0.1) {$+\infty $};
               % Ligne de la fonction : f(x)
               \node  at (-1*\Lrg,{-1*\HtX+(-1)*\HtY}) {$f(x)$};
               \node (f1) at (0*\Lrg,{-1*\HtX+(-2)*\HtY}) {$ $};
               \node (f2) at (2*\Lrg,{-1*\HtX+(0)*\HtY}) {$\text{max}$};
               \node (f3) at (4*\Lrg,{-1*\HtX+(-2)*\HtY}) {$ $};
               % Flèches
               \draw[fleche] (f1) -- (f2);
               \draw[fleche] (f2) -- (f3);
               % Encadrement
               \draw[cadre] (\separateur,\haut) -- (\separateur,\bas);
               \draw[cadre] (\gauche,\haut) rectangle  (\droite,\bas);
               \draw[cadre] (\gauche,\lignex) -- (\droite,\lignex);
          \end{tikzpicture}
     \end{extern}
\end{center}
\begin{center}\textit{Tableau de variations d'une fonction polynôme du second degré pour $a < 0$ }\end{center}
\bloc{orange}{Exemple}{% id="e60"
     Soit $f\left(x\right)=x^2-4x+3$
     \begin{center}
          \begin{extern}%width="280" alt="Parabole - Courbe représentative d'une fonction polynôme du second degré "
               \resizebox{5.5cm}{!}{%
                    % -+-+-+ variables modifiables
                    \def\fonction{x*x-4*x+3}
                    \def\xmin{-1.5}
                    \def\xmax{4.5}
                    \def\ymin{-1.2}
                    \def\ymax{4.9}
                    \def\xunit{1.5}  % unités en cm
                    \def\yunit{1.5}
                    \psset{xunit=\xunit,yunit=\yunit,algebraic=true}
                    \fontsize{15pt}{15pt}\selectfont
                    \begin{pspicture*}[linewidth=1pt](\xmin,\ymin)(\xmax,\ymax)
                         %      \psgrid[gridcolor=mcgris, subgriddiv=5, gridlabels=0pt](\xmin,\ymin)(\xmax,\ymax)
                         \psaxes[linewidth=0.75pt,Dx=1,Dy=1]{->}(0,0)(\xmin,\ymin)(\xmax,\ymax)
                         \psplot[plotpoints=2000,linecolor=blue]{\xmin}{\xmax}{\fonction}
                         \rput[tr](-0.2,-0.2){$O$}
                         \rput[tr](4.1,4.5){$\color{blue} \mathcal{C}_f$}
                    \end{pspicture*}
               }
          \end{extern}
     \end{center}
     \begin{center}\textit{Courbe représentative de $f~:~x\longmapsto x^2-4x+3$}\end{center}
}
\cadre{vert}{Propriété et définition}{% id="p70"
     Soit $f$ une fonction polynôme du second degré définie sur $\mathbb{R}$ par : $f\left(x\right)=ax^2+bx+c$
     \par
     $f\left(x\right)$ peut s'écrire sous la forme :
     \begin{center}$f\left(x\right)=a\left(x-\alpha \right)^2+\beta $\end{center}
     avec $\alpha  = -\frac{b}{2a}$ et $\beta  = f\left(\alpha \right)$
     \par
     Cette écriture est appelée \textbf{forme canonique}.
     \par
     $\left(\alpha  ; \beta \right)$ sont les coordonnées du sommet de la parabole.
}
\bloc{cyan}{Remarque}{% id="r70"
     Une caractéristique de la forme canonique est que la variable $x$ n'apparaît qu'à un seul endroit dans l'écriture.
}
\bloc{orange}{Exemple}{% id="e70"
     Reprenons l'exemple $f\left(x\right)=x^2-4x+3$
     \par
     On a $\alpha =-\frac{b}{2a}=- \frac{-4}{2\times 1}=2$
     \par
     et $\beta =f\left(2\right)=2^2-4\times 2+3=-1$
     \par
     donc la forme canonique de $f$ est :
     \par
     $f\left(x\right)=\left(x-2\right)^2-1$
}

\end{document}
µ
\documentclass[a4paper]{article}

%================================================================================================================================
%
% Packages
%
%================================================================================================================================

\usepackage[T1]{fontenc} 	% pour caractères accentués
\usepackage[utf8]{inputenc}  % encodage utf8
\usepackage[french]{babel}	% langue : français
\usepackage{fourier}			% caractères plus lisibles
\usepackage[dvipsnames]{xcolor} % couleurs
\usepackage{fancyhdr}		% réglage header footer
\usepackage{needspace}		% empêcher sauts de page mal placés
\usepackage{graphicx}		% pour inclure des graphiques
\usepackage{enumitem,cprotect}		% personnalise les listes d'items (nécessaire pour ol, al ...)
\usepackage{hyperref}		% Liens hypertexte
\usepackage{pstricks,pst-all,pst-node,pstricks-add,pst-math,pst-plot,pst-tree,pst-eucl} % pstricks
\usepackage[a4paper,includeheadfoot,top=2cm,left=3cm, bottom=2cm,right=3cm]{geometry} % marges etc.
\usepackage{comment}			% commentaires multilignes
\usepackage{amsmath,environ} % maths (matrices, etc.)
\usepackage{amssymb,makeidx}
\usepackage{bm}				% bold maths
\usepackage{tabularx}		% tableaux
\usepackage{colortbl}		% tableaux en couleur
\usepackage{fontawesome}		% Fontawesome
\usepackage{environ}			% environment with command
\usepackage{fp}				% calculs pour ps-tricks
\usepackage{multido}			% pour ps tricks
\usepackage[np]{numprint}	% formattage nombre
\usepackage{tikz,tkz-tab} 			% package principal TikZ
\usepackage{pgfplots}   % axes
\usepackage{mathrsfs}    % cursives
\usepackage{calc}			% calcul taille boites
\usepackage[scaled=0.875]{helvet} % font sans serif
\usepackage{svg} % svg
\usepackage{scrextend} % local margin
\usepackage{scratch} %scratch
\usepackage{multicol} % colonnes
%\usepackage{infix-RPN,pst-func} % formule en notation polanaise inversée
\usepackage{listings}

%================================================================================================================================
%
% Réglages de base
%
%================================================================================================================================

\lstset{
language=Python,   % R code
literate=
{á}{{\'a}}1
{à}{{\`a}}1
{ã}{{\~a}}1
{é}{{\'e}}1
{è}{{\`e}}1
{ê}{{\^e}}1
{í}{{\'i}}1
{ó}{{\'o}}1
{õ}{{\~o}}1
{ú}{{\'u}}1
{ü}{{\"u}}1
{ç}{{\c{c}}}1
{~}{{ }}1
}


\definecolor{codegreen}{rgb}{0,0.6,0}
\definecolor{codegray}{rgb}{0.5,0.5,0.5}
\definecolor{codepurple}{rgb}{0.58,0,0.82}
\definecolor{backcolour}{rgb}{0.95,0.95,0.92}

\lstdefinestyle{mystyle}{
    backgroundcolor=\color{backcolour},   
    commentstyle=\color{codegreen},
    keywordstyle=\color{magenta},
    numberstyle=\tiny\color{codegray},
    stringstyle=\color{codepurple},
    basicstyle=\ttfamily\footnotesize,
    breakatwhitespace=false,         
    breaklines=true,                 
    captionpos=b,                    
    keepspaces=true,                 
    numbers=left,                    
xleftmargin=2em,
framexleftmargin=2em,            
    showspaces=false,                
    showstringspaces=false,
    showtabs=false,                  
    tabsize=2,
    upquote=true
}

\lstset{style=mystyle}


\lstset{style=mystyle}
\newcommand{\imgdir}{C:/laragon/www/newmc/assets/imgsvg/}
\newcommand{\imgsvgdir}{C:/laragon/www/newmc/assets/imgsvg/}

\definecolor{mcgris}{RGB}{220, 220, 220}% ancien~; pour compatibilité
\definecolor{mcbleu}{RGB}{52, 152, 219}
\definecolor{mcvert}{RGB}{125, 194, 70}
\definecolor{mcmauve}{RGB}{154, 0, 215}
\definecolor{mcorange}{RGB}{255, 96, 0}
\definecolor{mcturquoise}{RGB}{0, 153, 153}
\definecolor{mcrouge}{RGB}{255, 0, 0}
\definecolor{mclightvert}{RGB}{205, 234, 190}

\definecolor{gris}{RGB}{220, 220, 220}
\definecolor{bleu}{RGB}{52, 152, 219}
\definecolor{vert}{RGB}{125, 194, 70}
\definecolor{mauve}{RGB}{154, 0, 215}
\definecolor{orange}{RGB}{255, 96, 0}
\definecolor{turquoise}{RGB}{0, 153, 153}
\definecolor{rouge}{RGB}{255, 0, 0}
\definecolor{lightvert}{RGB}{205, 234, 190}
\setitemize[0]{label=\color{lightvert}  $\bullet$}

\pagestyle{fancy}
\renewcommand{\headrulewidth}{0.2pt}
\fancyhead[L]{maths-cours.fr}
\fancyhead[R]{\thepage}
\renewcommand{\footrulewidth}{0.2pt}
\fancyfoot[C]{}

\newcolumntype{C}{>{\centering\arraybackslash}X}
\newcolumntype{s}{>{\hsize=.35\hsize\arraybackslash}X}

\setlength{\parindent}{0pt}		 
\setlength{\parskip}{3mm}
\setlength{\headheight}{1cm}

\def\ebook{ebook}
\def\book{book}
\def\web{web}
\def\type{web}

\newcommand{\vect}[1]{\overrightarrow{\,\mathstrut#1\,}}

\def\Oij{$\left(\text{O}~;~\vect{\imath},~\vect{\jmath}\right)$}
\def\Oijk{$\left(\text{O}~;~\vect{\imath},~\vect{\jmath},~\vect{k}\right)$}
\def\Ouv{$\left(\text{O}~;~\vect{u},~\vect{v}\right)$}

\hypersetup{breaklinks=true, colorlinks = true, linkcolor = OliveGreen, urlcolor = OliveGreen, citecolor = OliveGreen, pdfauthor={Didier BONNEL - https://www.maths-cours.fr} } % supprime les bordures autour des liens

\renewcommand{\arg}[0]{\text{arg}}

\everymath{\displaystyle}

%================================================================================================================================
%
% Macros - Commandes
%
%================================================================================================================================

\newcommand\meta[2]{    			% Utilisé pour créer le post HTML.
	\def\titre{titre}
	\def\url{url}
	\def\arg{#1}
	\ifx\titre\arg
		\newcommand\maintitle{#2}
		\fancyhead[L]{#2}
		{\Large\sffamily \MakeUppercase{#2}}
		\vspace{1mm}\textcolor{mcvert}{\hrule}
	\fi 
	\ifx\url\arg
		\fancyfoot[L]{\href{https://www.maths-cours.fr#2}{\black \footnotesize{https://www.maths-cours.fr#2}}}
	\fi 
}


\newcommand\TitreC[1]{    		% Titre centré
     \needspace{3\baselineskip}
     \begin{center}\textbf{#1}\end{center}
}

\newcommand\newpar{    		% paragraphe
     \par
}

\newcommand\nosp {    		% commande vide (pas d'espace)
}
\newcommand{\id}[1]{} %ignore

\newcommand\boite[2]{				% Boite simple sans titre
	\vspace{5mm}
	\setlength{\fboxrule}{0.2mm}
	\setlength{\fboxsep}{5mm}	
	\fcolorbox{#1}{#1!3}{\makebox[\linewidth-2\fboxrule-2\fboxsep]{
  		\begin{minipage}[t]{\linewidth-2\fboxrule-4\fboxsep}\setlength{\parskip}{3mm}
  			 #2
  		\end{minipage}
	}}
	\vspace{5mm}
}

\newcommand\CBox[4]{				% Boites
	\vspace{5mm}
	\setlength{\fboxrule}{0.2mm}
	\setlength{\fboxsep}{5mm}
	
	\fcolorbox{#1}{#1!3}{\makebox[\linewidth-2\fboxrule-2\fboxsep]{
		\begin{minipage}[t]{1cm}\setlength{\parskip}{3mm}
	  		\textcolor{#1}{\LARGE{#2}}    
 	 	\end{minipage}  
  		\begin{minipage}[t]{\linewidth-2\fboxrule-4\fboxsep}\setlength{\parskip}{3mm}
			\raisebox{1.2mm}{\normalsize\sffamily{\textcolor{#1}{#3}}}						
  			 #4
  		\end{minipage}
	}}
	\vspace{5mm}
}

\newcommand\cadre[3]{				% Boites convertible html
	\par
	\vspace{2mm}
	\setlength{\fboxrule}{0.1mm}
	\setlength{\fboxsep}{5mm}
	\fcolorbox{#1}{white}{\makebox[\linewidth-2\fboxrule-2\fboxsep]{
  		\begin{minipage}[t]{\linewidth-2\fboxrule-4\fboxsep}\setlength{\parskip}{3mm}
			\raisebox{-2.5mm}{\sffamily \small{\textcolor{#1}{\MakeUppercase{#2}}}}		
			\par		
  			 #3
 	 		\end{minipage}
	}}
		\vspace{2mm}
	\par
}

\newcommand\bloc[3]{				% Boites convertible html sans bordure
     \needspace{2\baselineskip}
     {\sffamily \small{\textcolor{#1}{\MakeUppercase{#2}}}}    
		\par		
  			 #3
		\par
}

\newcommand\CHelp[1]{
     \CBox{Plum}{\faInfoCircle}{À RETENIR}{#1}
}

\newcommand\CUp[1]{
     \CBox{NavyBlue}{\faThumbsOUp}{EN PRATIQUE}{#1}
}

\newcommand\CInfo[1]{
     \CBox{Sepia}{\faArrowCircleRight}{REMARQUE}{#1}
}

\newcommand\CRedac[1]{
     \CBox{PineGreen}{\faEdit}{BIEN R\'EDIGER}{#1}
}

\newcommand\CError[1]{
     \CBox{Red}{\faExclamationTriangle}{ATTENTION}{#1}
}

\newcommand\TitreExo[2]{
\needspace{4\baselineskip}
 {\sffamily\large EXERCICE #1\ (\emph{#2 points})}
\vspace{5mm}
}

\newcommand\img[2]{
          \includegraphics[width=#2\paperwidth]{\imgdir#1}
}

\newcommand\imgsvg[2]{
       \begin{center}   \includegraphics[width=#2\paperwidth]{\imgsvgdir#1} \end{center}
}


\newcommand\Lien[2]{
     \href{#1}{#2 \tiny \faExternalLink}
}
\newcommand\mcLien[2]{
     \href{https~://www.maths-cours.fr/#1}{#2 \tiny \faExternalLink}
}

\newcommand{\euro}{\eurologo{}}

%================================================================================================================================
%
% Macros - Environement
%
%================================================================================================================================

\newenvironment{tex}{ %
}
{%
}

\newenvironment{indente}{ %
	\setlength\parindent{10mm}
}

{
	\setlength\parindent{0mm}
}

\newenvironment{corrige}{%
     \needspace{3\baselineskip}
     \medskip
     \textbf{\textsc{Corrigé}}
     \medskip
}
{
}

\newenvironment{extern}{%
     \begin{center}
     }
     {
     \end{center}
}

\NewEnviron{code}{%
	\par
     \boite{gray}{\texttt{%
     \BODY
     }}
     \par
}

\newenvironment{vbloc}{% boite sans cadre empeche saut de page
     \begin{minipage}[t]{\linewidth}
     }
     {
     \end{minipage}
}
\NewEnviron{h2}{%
    \needspace{3\baselineskip}
    \vspace{0.6cm}
	\noindent \MakeUppercase{\sffamily \large \BODY}
	\vspace{1mm}\textcolor{mcgris}{\hrule}\vspace{0.4cm}
	\par
}{}

\NewEnviron{h3}{%
    \needspace{3\baselineskip}
	\vspace{5mm}
	\textsc{\BODY}
	\par
}

\NewEnviron{margeneg}{ %
\begin{addmargin}[-1cm]{0cm}
\BODY
\end{addmargin}
}

\NewEnviron{html}{%
}

\begin{document}
\meta{url}{/cours/la-fonction-inverse-et-les-fonctions-homographiques/}
\meta{pid}{140}
\meta{titre}{La fonction « inverse » et les fonctions homographiques}
\meta{type}{cours}
\begin{h2}1. La fonction inverse\end{h2}
\cadre{bleu}{Définition}{% id="d10"
     La fonction \textbf{inverse} est la fonction  définie sur $\left]-\infty ; 0\right[ \cup  \left]0; +\infty \right[$ : $x \mapsto \frac{1}{x}$.
     \par
     Sa courbe représentative est une \textbf{hyperbole}.
}
\begin{center}
     \begin{extern}%width="330" alt="fonction inverse"
          \resizebox{7.5cm}{!}{%
               % -+-+-+ variables modifiables
               \def\fonction{1/x}
               \def\xmin{-3.5}
               \def\xmax{3.5}
               \def\ymin{-3.5}
               \def\ymax{3.5}
               \def\xunit{1.5}  % unités en cm
               \def\yunit{1.5}
               \psset{xunit=\xunit,yunit=\yunit,algebraic=true}
               \fontsize{15pt}{15pt}\selectfont
               \begin{pspicture*}[linewidth=1pt](\xmin,\ymin)(\xmax,\ymax)
                    %      \psgrid[gridcolor=mcgris, subgriddiv=5, gridlabels=0pt](\xmin,\ymin)(\xmax,\ymax)
                    \psaxes[linewidth=0.75pt,Dx=1,Dy=1]{->}(0,0)(\xmin,\ymin)(\xmax,\ymax)
                    \psplot[plotpoints=2000,linecolor=blue]{\xmin}{-0.01}{\fonction}
                    \psplot[plotpoints=2000,linecolor=blue]{0.01}{\xmax}{\fonction}
                    \rput[tr](-0.2,-0.2){$O$}
                    \rput[tr](3.1,0.8){$\color{blue} \mathcal{C}$}
               \end{pspicture*}
          }
     \end{extern}
\end{center}
\begin{center}L'hyperbole représentant la fonction $x \mapsto \frac{1}{x}$\end{center}
\cadre{rouge}{Théorème}{% id="t20"
     La courbe représentative de la fonction inverse est symétrique par rapport à l'origine du repère.
}
\cadre{rouge}{Théorème}{% id="t30"
     La fonction inverse est strictement décroissante sur $\left]-\infty ; 0\right[$ et sur $\left]0; +\infty \right[$.
}
\begin{center}
     \begin{extern}%width="350" alt="tableau de variation fonction inverse"
          \begin{center}
               \begin{tikzpicture}[scale=0.875]
                    % Styles
                    \tikzstyle{cadre}=[thin]
                    \tikzstyle{fleche}=[->,>=latex,thin]
                    \tikzstyle{nondefini}=[lightgray]
                    % Dimensions Modifiables
                    \def\Lrg{1.5}
                    \def\HtX{1}
                    \def\HtY{0.5}
                    % Dimensions Calculées
                    \def\lignex{-0.5*\HtX}
                    \def\lignef{-1.5*\HtX}
                    \def\separateur{-0.5*\Lrg}
                    % Largeur du tableau
                    \def\gauche{-1.5*\Lrg}
                    \def\droite{4.5*\Lrg}
                    % Hauteur du tableau
                    \def\haut{0.5*\HtX}
                    \def\bas{-1.5*\HtX-2*\HtY}
                    % Ligne de l'abscisse : x
                    \node at (-1*\Lrg,0) {$x$};
                    \node at (0*\Lrg,0) {$-\infty$};
                    \node at (2*\Lrg,0) {$0$};
                    \node at (4*\Lrg,0) {$+\infty$};
                    % Ligne de la fonction : f(x)
                    \node  at (-1*\Lrg,{-1*\HtX+(-1)*\HtY}) {$f(x)$};
                    \node (f1) at (0*\Lrg,{-1*\HtX+(0)*\HtY}) {$ $};
                    \node[left] (f2) at (2*\Lrg,{-1*\HtX+(-2)*\HtY}) {$ $};
                    \node[right] (f3) at (2*\Lrg,{-1*\HtX+(0)*\HtY}) {$ $};
                    \node (f4) at (4*\Lrg,{-1*\HtX+(-2)*\HtY}) {$ $};
                    % Flèches
                    \draw[fleche] (f1) -- (f2);
                    \draw[fleche] (f3) -- (f4);
                    % Doubles barres
                    \draw[double distance=2pt] (2*\Lrg,\lignex) -- (2*\Lrg,\bas);
                    % Encadrement
                    \draw[cadre] (\separateur,\haut) -- (\separateur,\bas);
                    \draw[cadre] (\gauche,\haut) rectangle  (\droite,\bas);
                    \draw[cadre] (\gauche,\lignex) -- (\droite,\lignex);
               \end{tikzpicture}
          \end{center}
     \end{extern}
\end{center}
\begin{center}Tableau de variation de la fonction "inverse"\end{center}
\bloc{orange}{Exemple d'application}{% id="e30"
     On veut comparer les nombres $\frac{1}{\pi }$ et $\frac{1}{3}$.
     \par
     On sait que $\pi  > 3$
     \par
     Comme les nombres $3$ et $\pi $ sont strictement positifs et que la fonction inverse est strictement décroissante sur $\left]0; +\infty \right[$ on en déduit que $\frac{1}{\pi } < \frac{1}{3}$
}
\begin{h2}2. Fonctions homographiques\end{h2}
\cadre{bleu}{Définition}{% id="d50"
     Soient $a, b, c, d$ quatre réels avec $c\neq 0$ et $ad-bc\neq 0$.
     \par
     La fonction $f$ définie sur $\mathbb{R}\backslash\left\{-\frac{d}{c}\right\}$ par :
     \begin{center}
          $f\left(x\right)=\frac{ax+b}{cx+d}$
     \end{center}
     s'appelle une \textbf{fonction homographique}.
     \par
     La courbe représentative d'une fonction homographique est une \textbf{hyperbole}.
}
\bloc{cyan}{Remarques}{% id="r50"
     \begin{itemize}
          \item La valeur « interdite » $-\frac{d}{c}$ est celle qui annule le dénominateur.
          \item Si $ad-bc=0$, la fraction se simplifie et dans ce cas la fonction $f$ est constante sur son ensemble de définition. Par exemple $f\left(x\right)=\frac{2x+1}{4x+2}=\frac{2x+1}{2\times \left(2x+1\right)}=\frac{1}{2}$ sur $\mathbb{R}\backslash\left\{-\frac{1}{2}\right\}$
     \end{itemize}
}
\bloc{orange}{Exemple}{% id="e50"
     La fonction $f $ telle que :
     \begin{center}
          $f\left(x\right)=\frac{3x + 2}{x + 1}$
     \end{center}
     est définie pour $x+1 \neq  0$ c'est à dire $x \neq  -1$.
     \par
     Son ensemble de définition est donc :
     \par
     $\mathscr D_f = \mathbb{R}\backslash\left\{-1\right\}   $( ou $  \mathscr D_f =\left]-\infty  ; -1\right[ \cup  \left]-1 ; +\infty \right[$)
     \par
     Elle est strictement croissante sur chacun des intervalles $\left]-\infty  ; -1\right[$ et  $\left]-1 ; +\infty \right[$ (pour cet exemple ; ce n'est pas le cas pour toutes les fonctions homographiques !).
     \begin{center}
          \begin{extern}%width="350" alt=""
               \begin{tikzpicture}[scale=0.875]
                    % Styles
                    \tikzstyle{cadre}=[thin]
                    \tikzstyle{fleche}=[->,>=latex,thin]
                    \tikzstyle{nondefini}=[lightgray]
                    % Dimensions Modifiables
                    \def\Lrg{1.5}
                    \def\HtX{1}
                    \def\HtY{0.5}
                    % Dimensions Calculées
                    \def\lignex{-0.5*\HtX}
                    \def\lignef{-1.5*\HtX}
                    \def\separateur{-0.5*\Lrg}
                    % Largeur du tableau
                    \def\gauche{-1.5*\Lrg}
                    \def\droite{4.5*\Lrg}
                    % Hauteur du tableau
                    \def\haut{0.5*\HtX}
                    \def\bas{-1.5*\HtX-2*\HtY}
                    % Ligne de l'abscisse : x
                    \node at (-1*\Lrg,0) {$x$};
                    \node at (0*\Lrg,0) {$-\infty$};
                    \node at (2*\Lrg,0) {$-1$};
                    \node at (4*\Lrg,0) {$+\infty$};
                    % Ligne de la fonction : f(x)
                    \node  at (-1*\Lrg,{-1*\HtX+(-1)*\HtY}) {$f(x)$};
                    \node (f1) at (0*\Lrg,{-1*\HtX+(-2)*\HtY}) {$ $};
                    \node[left] (f2) at (2*\Lrg,{-1*\HtX+(0)*\HtY}) {$ $};
                    \node[right] (f3) at (2*\Lrg,{-1*\HtX+(-2)*\HtY}) {$ $};
                    \node (f4) at (4*\Lrg,{-1*\HtX+(0)*\HtY}) {$ $};
                    % Flèches
                    \draw[fleche] (f1) -- (f2);
                    \draw[fleche] (f3) -- (f4);
                    % Doubles barres
                    \draw[double distance=2pt] (2*\Lrg,\lignex) -- (2*\Lrg,\bas);
                    % Encadrement
                    \draw[cadre] (\separateur,\haut) -- (\separateur,\bas);
                    \draw[cadre] (\gauche,\haut) rectangle  (\droite,\bas);
                    \draw[cadre] (\gauche,\lignex) -- (\droite,\lignex);
               \end{tikzpicture}
          \end{extern}
     \end{center}
     \begin{center}Tableau de variations de $f~:~x \longmapsto \frac{3x + 2}{x + 1} $ \end{center}
     \begin{center}
          \begin{extern}%width="330" alt="fonction homographique"
               \resizebox{7.5cm}{!}{%
                    % -+-+-+ variables modifiables
                    \def\fonction{(3*x+2)/(x+1)}
                    \def\xmin{-4.5}
                    \def\xmax{2.5}
                    \def\ymin{-0.9}
                    \def\ymax{6.5}
                    \def\xunit{1.5}  % unités en cm
                    \def\yunit{1.5}
                    \psset{xunit=\xunit,yunit=\yunit,algebraic=true}
                    \fontsize{15pt}{15pt}\selectfont
                    \begin{pspicture*}[linewidth=1pt](\xmin,\ymin)(\xmax,\ymax)
                         %      \psgrid[gridcolor=mcgris, subgriddiv=5, gridlabels=0pt](\xmin,\ymin)(\xmax,\ymax)
                         \psaxes[linewidth=0.75pt,Dx=1,Dy=1]{->}(0,0)(\xmin,\ymin)(\xmax,\ymax)
                         \psplot[plotpoints=2000,linecolor=blue]{\xmin}{-1.01}{\fonction}
                         \psplot[plotpoints=2000,linecolor=blue]{-0.99}{\xmax}{\fonction}
                         \rput[tr](-0.2,-0.2){$O$}
                         \rput[tr](2.4,2.5){$\color{blue} \mathcal{C}_f$}
                    \end{pspicture*}
               }
          \end{extern}
     \end{center}
     \begin{center}Courbe représentative de $f~:~x \longmapsto \frac{3x + 2}{x + 1} $\end{center}
}

\end{document}
µ
\documentclass[a4paper]{article}

%================================================================================================================================
%
% Packages
%
%================================================================================================================================

\usepackage[T1]{fontenc} 	% pour caractères accentués
\usepackage[utf8]{inputenc}  % encodage utf8
\usepackage[french]{babel}	% langue : français
\usepackage{fourier}			% caractères plus lisibles
\usepackage[dvipsnames]{xcolor} % couleurs
\usepackage{fancyhdr}		% réglage header footer
\usepackage{needspace}		% empêcher sauts de page mal placés
\usepackage{graphicx}		% pour inclure des graphiques
\usepackage{enumitem,cprotect}		% personnalise les listes d'items (nécessaire pour ol, al ...)
\usepackage{hyperref}		% Liens hypertexte
\usepackage{pstricks,pst-all,pst-node,pstricks-add,pst-math,pst-plot,pst-tree,pst-eucl} % pstricks
\usepackage[a4paper,includeheadfoot,top=2cm,left=3cm, bottom=2cm,right=3cm]{geometry} % marges etc.
\usepackage{comment}			% commentaires multilignes
\usepackage{amsmath,environ} % maths (matrices, etc.)
\usepackage{amssymb,makeidx}
\usepackage{bm}				% bold maths
\usepackage{tabularx}		% tableaux
\usepackage{colortbl}		% tableaux en couleur
\usepackage{fontawesome}		% Fontawesome
\usepackage{environ}			% environment with command
\usepackage{fp}				% calculs pour ps-tricks
\usepackage{multido}			% pour ps tricks
\usepackage[np]{numprint}	% formattage nombre
\usepackage{tikz,tkz-tab} 			% package principal TikZ
\usepackage{pgfplots}   % axes
\usepackage{mathrsfs}    % cursives
\usepackage{calc}			% calcul taille boites
\usepackage[scaled=0.875]{helvet} % font sans serif
\usepackage{svg} % svg
\usepackage{scrextend} % local margin
\usepackage{scratch} %scratch
\usepackage{multicol} % colonnes
%\usepackage{infix-RPN,pst-func} % formule en notation polanaise inversée
\usepackage{listings}

%================================================================================================================================
%
% Réglages de base
%
%================================================================================================================================

\lstset{
language=Python,   % R code
literate=
{á}{{\'a}}1
{à}{{\`a}}1
{ã}{{\~a}}1
{é}{{\'e}}1
{è}{{\`e}}1
{ê}{{\^e}}1
{í}{{\'i}}1
{ó}{{\'o}}1
{õ}{{\~o}}1
{ú}{{\'u}}1
{ü}{{\"u}}1
{ç}{{\c{c}}}1
{~}{{ }}1
}


\definecolor{codegreen}{rgb}{0,0.6,0}
\definecolor{codegray}{rgb}{0.5,0.5,0.5}
\definecolor{codepurple}{rgb}{0.58,0,0.82}
\definecolor{backcolour}{rgb}{0.95,0.95,0.92}

\lstdefinestyle{mystyle}{
    backgroundcolor=\color{backcolour},   
    commentstyle=\color{codegreen},
    keywordstyle=\color{magenta},
    numberstyle=\tiny\color{codegray},
    stringstyle=\color{codepurple},
    basicstyle=\ttfamily\footnotesize,
    breakatwhitespace=false,         
    breaklines=true,                 
    captionpos=b,                    
    keepspaces=true,                 
    numbers=left,                    
xleftmargin=2em,
framexleftmargin=2em,            
    showspaces=false,                
    showstringspaces=false,
    showtabs=false,                  
    tabsize=2,
    upquote=true
}

\lstset{style=mystyle}


\lstset{style=mystyle}
\newcommand{\imgdir}{C:/laragon/www/newmc/assets/imgsvg/}
\newcommand{\imgsvgdir}{C:/laragon/www/newmc/assets/imgsvg/}

\definecolor{mcgris}{RGB}{220, 220, 220}% ancien~; pour compatibilité
\definecolor{mcbleu}{RGB}{52, 152, 219}
\definecolor{mcvert}{RGB}{125, 194, 70}
\definecolor{mcmauve}{RGB}{154, 0, 215}
\definecolor{mcorange}{RGB}{255, 96, 0}
\definecolor{mcturquoise}{RGB}{0, 153, 153}
\definecolor{mcrouge}{RGB}{255, 0, 0}
\definecolor{mclightvert}{RGB}{205, 234, 190}

\definecolor{gris}{RGB}{220, 220, 220}
\definecolor{bleu}{RGB}{52, 152, 219}
\definecolor{vert}{RGB}{125, 194, 70}
\definecolor{mauve}{RGB}{154, 0, 215}
\definecolor{orange}{RGB}{255, 96, 0}
\definecolor{turquoise}{RGB}{0, 153, 153}
\definecolor{rouge}{RGB}{255, 0, 0}
\definecolor{lightvert}{RGB}{205, 234, 190}
\setitemize[0]{label=\color{lightvert}  $\bullet$}

\pagestyle{fancy}
\renewcommand{\headrulewidth}{0.2pt}
\fancyhead[L]{maths-cours.fr}
\fancyhead[R]{\thepage}
\renewcommand{\footrulewidth}{0.2pt}
\fancyfoot[C]{}

\newcolumntype{C}{>{\centering\arraybackslash}X}
\newcolumntype{s}{>{\hsize=.35\hsize\arraybackslash}X}

\setlength{\parindent}{0pt}		 
\setlength{\parskip}{3mm}
\setlength{\headheight}{1cm}

\def\ebook{ebook}
\def\book{book}
\def\web{web}
\def\type{web}

\newcommand{\vect}[1]{\overrightarrow{\,\mathstrut#1\,}}

\def\Oij{$\left(\text{O}~;~\vect{\imath},~\vect{\jmath}\right)$}
\def\Oijk{$\left(\text{O}~;~\vect{\imath},~\vect{\jmath},~\vect{k}\right)$}
\def\Ouv{$\left(\text{O}~;~\vect{u},~\vect{v}\right)$}

\hypersetup{breaklinks=true, colorlinks = true, linkcolor = OliveGreen, urlcolor = OliveGreen, citecolor = OliveGreen, pdfauthor={Didier BONNEL - https://www.maths-cours.fr} } % supprime les bordures autour des liens

\renewcommand{\arg}[0]{\text{arg}}

\everymath{\displaystyle}

%================================================================================================================================
%
% Macros - Commandes
%
%================================================================================================================================

\newcommand\meta[2]{    			% Utilisé pour créer le post HTML.
	\def\titre{titre}
	\def\url{url}
	\def\arg{#1}
	\ifx\titre\arg
		\newcommand\maintitle{#2}
		\fancyhead[L]{#2}
		{\Large\sffamily \MakeUppercase{#2}}
		\vspace{1mm}\textcolor{mcvert}{\hrule}
	\fi 
	\ifx\url\arg
		\fancyfoot[L]{\href{https://www.maths-cours.fr#2}{\black \footnotesize{https://www.maths-cours.fr#2}}}
	\fi 
}


\newcommand\TitreC[1]{    		% Titre centré
     \needspace{3\baselineskip}
     \begin{center}\textbf{#1}\end{center}
}

\newcommand\newpar{    		% paragraphe
     \par
}

\newcommand\nosp {    		% commande vide (pas d'espace)
}
\newcommand{\id}[1]{} %ignore

\newcommand\boite[2]{				% Boite simple sans titre
	\vspace{5mm}
	\setlength{\fboxrule}{0.2mm}
	\setlength{\fboxsep}{5mm}	
	\fcolorbox{#1}{#1!3}{\makebox[\linewidth-2\fboxrule-2\fboxsep]{
  		\begin{minipage}[t]{\linewidth-2\fboxrule-4\fboxsep}\setlength{\parskip}{3mm}
  			 #2
  		\end{minipage}
	}}
	\vspace{5mm}
}

\newcommand\CBox[4]{				% Boites
	\vspace{5mm}
	\setlength{\fboxrule}{0.2mm}
	\setlength{\fboxsep}{5mm}
	
	\fcolorbox{#1}{#1!3}{\makebox[\linewidth-2\fboxrule-2\fboxsep]{
		\begin{minipage}[t]{1cm}\setlength{\parskip}{3mm}
	  		\textcolor{#1}{\LARGE{#2}}    
 	 	\end{minipage}  
  		\begin{minipage}[t]{\linewidth-2\fboxrule-4\fboxsep}\setlength{\parskip}{3mm}
			\raisebox{1.2mm}{\normalsize\sffamily{\textcolor{#1}{#3}}}						
  			 #4
  		\end{minipage}
	}}
	\vspace{5mm}
}

\newcommand\cadre[3]{				% Boites convertible html
	\par
	\vspace{2mm}
	\setlength{\fboxrule}{0.1mm}
	\setlength{\fboxsep}{5mm}
	\fcolorbox{#1}{white}{\makebox[\linewidth-2\fboxrule-2\fboxsep]{
  		\begin{minipage}[t]{\linewidth-2\fboxrule-4\fboxsep}\setlength{\parskip}{3mm}
			\raisebox{-2.5mm}{\sffamily \small{\textcolor{#1}{\MakeUppercase{#2}}}}		
			\par		
  			 #3
 	 		\end{minipage}
	}}
		\vspace{2mm}
	\par
}

\newcommand\bloc[3]{				% Boites convertible html sans bordure
     \needspace{2\baselineskip}
     {\sffamily \small{\textcolor{#1}{\MakeUppercase{#2}}}}    
		\par		
  			 #3
		\par
}

\newcommand\CHelp[1]{
     \CBox{Plum}{\faInfoCircle}{À RETENIR}{#1}
}

\newcommand\CUp[1]{
     \CBox{NavyBlue}{\faThumbsOUp}{EN PRATIQUE}{#1}
}

\newcommand\CInfo[1]{
     \CBox{Sepia}{\faArrowCircleRight}{REMARQUE}{#1}
}

\newcommand\CRedac[1]{
     \CBox{PineGreen}{\faEdit}{BIEN R\'EDIGER}{#1}
}

\newcommand\CError[1]{
     \CBox{Red}{\faExclamationTriangle}{ATTENTION}{#1}
}

\newcommand\TitreExo[2]{
\needspace{4\baselineskip}
 {\sffamily\large EXERCICE #1\ (\emph{#2 points})}
\vspace{5mm}
}

\newcommand\img[2]{
          \includegraphics[width=#2\paperwidth]{\imgdir#1}
}

\newcommand\imgsvg[2]{
       \begin{center}   \includegraphics[width=#2\paperwidth]{\imgsvgdir#1} \end{center}
}


\newcommand\Lien[2]{
     \href{#1}{#2 \tiny \faExternalLink}
}
\newcommand\mcLien[2]{
     \href{https~://www.maths-cours.fr/#1}{#2 \tiny \faExternalLink}
}

\newcommand{\euro}{\eurologo{}}

%================================================================================================================================
%
% Macros - Environement
%
%================================================================================================================================

\newenvironment{tex}{ %
}
{%
}

\newenvironment{indente}{ %
	\setlength\parindent{10mm}
}

{
	\setlength\parindent{0mm}
}

\newenvironment{corrige}{%
     \needspace{3\baselineskip}
     \medskip
     \textbf{\textsc{Corrigé}}
     \medskip
}
{
}

\newenvironment{extern}{%
     \begin{center}
     }
     {
     \end{center}
}

\NewEnviron{code}{%
	\par
     \boite{gray}{\texttt{%
     \BODY
     }}
     \par
}

\newenvironment{vbloc}{% boite sans cadre empeche saut de page
     \begin{minipage}[t]{\linewidth}
     }
     {
     \end{minipage}
}
\NewEnviron{h2}{%
    \needspace{3\baselineskip}
    \vspace{0.6cm}
	\noindent \MakeUppercase{\sffamily \large \BODY}
	\vspace{1mm}\textcolor{mcgris}{\hrule}\vspace{0.4cm}
	\par
}{}

\NewEnviron{h3}{%
    \needspace{3\baselineskip}
	\vspace{5mm}
	\textsc{\BODY}
	\par
}

\NewEnviron{margeneg}{ %
\begin{addmargin}[-1cm]{0cm}
\BODY
\end{addmargin}
}

\NewEnviron{html}{%
}

\begin{document}
\meta{url}{/cours/statistiques-organisation-representation-donnees/}
\meta{pid}{144}
\meta{titre}{Statistiques en Seconde}
\meta{type}{cours}
\begin{h2}I. Organisation et représentation des données\end{h2}
\cadre{bleu}{Définitions}{%
     \begin{itemize}
          \item Les statistiques permettent d'étudier un \textbf{caractère} d'une \textbf{population}.
          \item Le nombre d'éléments de la population s'appelle l'\textbf{effectif global} (ou l'\textbf{effectif total}).
          \item Pour une valeur de caractère donnée, l'\textbf{effectif} est le nombre d'éléments correspondant à cette valeur.
          \item Une \textbf{série statistique} est un tableau donnant les effectifs pour chacune des valeurs possibles du caractère.
     \end{itemize}
}
\bloc{orange}{Exemple 1}{%
     On fait une étude portant sur l'âge des élèves d'un lycée.
     \begin{itemize}
          \item le \textbf{caractère} étudié est l'âge
          \item la \textbf{population} est l'ensemble des élèves du lycée
          \item l'\textbf{effectif global} est le nombre d'élèves du lycée
          \item le tableau ci-dessous est la \textbf{série statistique} pour ce caractère dans un lycée donné~:
          \begin{center}
               \begin{tabular}{|c|c|c|c|c|c|c|c|c|} %class="compact" width="600"
                    \hline
                    âges (en années)  &  14  &   15  &  16  &  17  &  18  &  19  &  20
                    \\ \hline
                    effectifs         &  3   &   22  &  65  &  82  &  59  &  35  &  2
                    \\ \hline
               \end{tabular}
          \end{center}
     \end{itemize}
}
\bloc{orange}{Exemple 2 : création d'un tableau pour une série statistique}{%
     On suppose que les notes à un contrôle dans une classe de 21 élèves sont les suivantes~:
     \begin{center}
          5 ; 14 ; 13 ; 16 ; 9 ; 8 ; 18 ; 2 ; 13 ; 12 ; 15 ; 12 ; 8 ; 6 ; 5 ; 17 ; 3 ; 19 ; 9 ; 13 ; 14
     \end{center}
     Ces données brutes sont assez peu pratiques à utiliser sous cette forme (notamment lorsqu'il y a beaucoup de valeurs).
     \par
     Pour commencer on commence à trier les notes de la plus petite à la plus grande :
     \begin{center}
          2 ; 3 ; 5 ; 5 ; 6 ; 8 ; 8 ; 9 ; 9 ; 12 ; 12 ; 13 ; 13 ; 13 ; 14 ; 14 ; 15 ; 16 ; 17 ; 18 ; 19
     \end{center}
     Ensuite, on va créer le tableau de cette série en indiquant pour chaque note son effectif c'est à dire le nombre d'élèves ayant obtenu cette note :
     \begin{center}
          \begin{tabular}{|c|c|c|c|c|c|c|c|c|c|c|c|c|c|c|}%class="compact" width="600"
               \hline
               notes       &  2  &   3  &   5  &  6  &  8  &  9  &  12  &  13  &  14  &  15  &  16  &  17  &  18  &  19
               \\ \hline
               effectifs   &  1   &  1  &  2   &  1  &  2  &  2  &  2   &  3   &  2  &  1   &  1  &  1  &  1  &  1
               \\ \hline
          \end{tabular}
     \end{center}
}
\begin{h2}II - Médiane - Quartiles\end{h2}
\cadre{bleu}{Définition}{%
     La \textbf{médiane} d'une série statistique est la valeur du caractère qui partage la population en deux classes de même effectif.
}
\bloc{cyan}{Remarque}{%
     En pratique pour trouver la médiane d'une série statistique d'effectif global $n$ :
     \begin{itemize}
          \item On ordonne les valeurs du caractère dans l'ordre croissant.
          \item Si $n$ est pair, la médiane sera la moyenne des valeurs du terme de rang $\frac{n}{2}$ et du terme de rang $\frac{n}{2}+1$.
          \item Si $n$ est impair, la médiane sera la valeur du terme de rang $\frac{n+1}{2}$.
          \item Lorsque l'effectif global est élevé, il est souvent utile de calculer les effectifs cumulés pour trouver cette valeur.
     \end{itemize}
}
\bloc{orange}{Exemple}{%
     Reprenons l'exemple 2 ci-dessus.
     \par
     Dans cet exemple, c'est la 11ème note ($11=\frac{21+1}{2}$) qui est la médiane. En effet, il y a 10 notes au dessous et 10 notes au dessus :
     \begin{center}
          2 ; 3 ; 5 ; 5 ; 6 ; 8 ; 8 ; 9 ; 9 ; 12 ; \textbf{$12$} ; 13 ; 13 ; 13 ; 14 ; 14 ; 15 ; 16 ; 17 ; 18 ; 19
     \end{center}
     \textbf{La médiane est donc 12.}
     \par
     Supposons qu'il n'y ait que 20 élèves (on enlève l'élève qui a eu 2) :
     \begin{center}
          3 ; 5 ; 5 ; 6 ; 8 ; 8 ; 9 ; 9 ; 12 ; 12 ; 13 ; 13 ; 13 ; 14 ; 14 ; 15 ; 16 ; 17 ; 18 ; 19
     \end{center}
     Il n'y a plus ici de note située "juste au milieu".
     \par
     Si on choisit la 10ème note (qui est 12) il y a 9 notes en dessous et 10 notes au dessus.
     \par
     Si on choisit la 11ème note (qui est 13) il y a 10 notes en dessous et 9 notes au dessus.
     \begin{center}
          3 ; 5 ; 5 ; 6 ; 8 ; 8 ; 9 ; 9 ; 12 ; \textbf{$12 ; 13 $}; 13 ; 13 ; 14 ; 14 ; 15 ; 16 ; 17 ; 18 ; 19
     \end{center}
     Dans ce cas, on prend comme médiane la moyenne de 12 et de 13 c'est à dire 12,5.
     \par
     \textbf{La médiane est donc 12,5.}
}
\cadre{bleu}{Définitions}{%
     \begin{itemize}
          \item Le \textbf{premier quartile} Q1 d'une série statistique est la plus petite valeur des termes de la série pour laquelle au moins un quart des données sont inférieures ou égales à Q1.
          \item Le \textbf{troisième quartile} Q3  d'une série statistique est la plus petite valeur des termes de la série pour laquelle au moins trois quarts des données sont inférieures ou égales à Q3.
     \end{itemize}
}
\bloc{orange}{Exemple}{%
     Reprenons l'exemple des notes ci-dessus (avec 21 élèves).
     \par
     Pour le \textbf{premier quartile} il faut qu'il y ait au moins 1/4 des notes qui soient inférieures ou égales. 1/4$\times $21=5,25. Le premier quartile est donc la 6ème note.
     \begin{center}
          2 ; 3 ; 5 ; 5 ; 6 ; \textbf{$8$} ; 8 ; 9 ; 9 ; 12 ; 12 ; 13 ; 13 ; 13 ; 14 ; 14 ; 15 ; 16 ; 17 ; 18 ; 19
     \end{center}
     \textbf{le premier quartile est 8.}
     \par
     Pour le \textbf{troisième quartile} il faut qu'il y ait au moins 3/4 des notes qui soient inférieures ou égales.3/4$\times $21=15,75.
     \par
     Le troisième quartile est donc la 16ème note.
     \begin{center}
          2 ; 3 ; 5 ; 5 ; 6 ; 8 ; 8 ; 9 ; 9 ; 12 ; 12 ; 13 ; 13 ; 13 ; 14 ; \textbf{$14$} ; 15 ; 16 ; 17 ; 18 ; 19
     \end{center}
     \textbf{le troisième quartile est 14.}
}

\end{document}
µ
\documentclass[a4paper]{article}

%================================================================================================================================
%
% Packages
%
%================================================================================================================================

\usepackage[T1]{fontenc} 	% pour caractères accentués
\usepackage[utf8]{inputenc}  % encodage utf8
\usepackage[french]{babel}	% langue : français
\usepackage{fourier}			% caractères plus lisibles
\usepackage[dvipsnames]{xcolor} % couleurs
\usepackage{fancyhdr}		% réglage header footer
\usepackage{needspace}		% empêcher sauts de page mal placés
\usepackage{graphicx}		% pour inclure des graphiques
\usepackage{enumitem,cprotect}		% personnalise les listes d'items (nécessaire pour ol, al ...)
\usepackage{hyperref}		% Liens hypertexte
\usepackage{pstricks,pst-all,pst-node,pstricks-add,pst-math,pst-plot,pst-tree,pst-eucl} % pstricks
\usepackage[a4paper,includeheadfoot,top=2cm,left=3cm, bottom=2cm,right=3cm]{geometry} % marges etc.
\usepackage{comment}			% commentaires multilignes
\usepackage{amsmath,environ} % maths (matrices, etc.)
\usepackage{amssymb,makeidx}
\usepackage{bm}				% bold maths
\usepackage{tabularx}		% tableaux
\usepackage{colortbl}		% tableaux en couleur
\usepackage{fontawesome}		% Fontawesome
\usepackage{environ}			% environment with command
\usepackage{fp}				% calculs pour ps-tricks
\usepackage{multido}			% pour ps tricks
\usepackage[np]{numprint}	% formattage nombre
\usepackage{tikz,tkz-tab} 			% package principal TikZ
\usepackage{pgfplots}   % axes
\usepackage{mathrsfs}    % cursives
\usepackage{calc}			% calcul taille boites
\usepackage[scaled=0.875]{helvet} % font sans serif
\usepackage{svg} % svg
\usepackage{scrextend} % local margin
\usepackage{scratch} %scratch
\usepackage{multicol} % colonnes
%\usepackage{infix-RPN,pst-func} % formule en notation polanaise inversée
\usepackage{listings}

%================================================================================================================================
%
% Réglages de base
%
%================================================================================================================================

\lstset{
language=Python,   % R code
literate=
{á}{{\'a}}1
{à}{{\`a}}1
{ã}{{\~a}}1
{é}{{\'e}}1
{è}{{\`e}}1
{ê}{{\^e}}1
{í}{{\'i}}1
{ó}{{\'o}}1
{õ}{{\~o}}1
{ú}{{\'u}}1
{ü}{{\"u}}1
{ç}{{\c{c}}}1
{~}{{ }}1
}


\definecolor{codegreen}{rgb}{0,0.6,0}
\definecolor{codegray}{rgb}{0.5,0.5,0.5}
\definecolor{codepurple}{rgb}{0.58,0,0.82}
\definecolor{backcolour}{rgb}{0.95,0.95,0.92}

\lstdefinestyle{mystyle}{
    backgroundcolor=\color{backcolour},   
    commentstyle=\color{codegreen},
    keywordstyle=\color{magenta},
    numberstyle=\tiny\color{codegray},
    stringstyle=\color{codepurple},
    basicstyle=\ttfamily\footnotesize,
    breakatwhitespace=false,         
    breaklines=true,                 
    captionpos=b,                    
    keepspaces=true,                 
    numbers=left,                    
xleftmargin=2em,
framexleftmargin=2em,            
    showspaces=false,                
    showstringspaces=false,
    showtabs=false,                  
    tabsize=2,
    upquote=true
}

\lstset{style=mystyle}


\lstset{style=mystyle}
\newcommand{\imgdir}{C:/laragon/www/newmc/assets/imgsvg/}
\newcommand{\imgsvgdir}{C:/laragon/www/newmc/assets/imgsvg/}

\definecolor{mcgris}{RGB}{220, 220, 220}% ancien~; pour compatibilité
\definecolor{mcbleu}{RGB}{52, 152, 219}
\definecolor{mcvert}{RGB}{125, 194, 70}
\definecolor{mcmauve}{RGB}{154, 0, 215}
\definecolor{mcorange}{RGB}{255, 96, 0}
\definecolor{mcturquoise}{RGB}{0, 153, 153}
\definecolor{mcrouge}{RGB}{255, 0, 0}
\definecolor{mclightvert}{RGB}{205, 234, 190}

\definecolor{gris}{RGB}{220, 220, 220}
\definecolor{bleu}{RGB}{52, 152, 219}
\definecolor{vert}{RGB}{125, 194, 70}
\definecolor{mauve}{RGB}{154, 0, 215}
\definecolor{orange}{RGB}{255, 96, 0}
\definecolor{turquoise}{RGB}{0, 153, 153}
\definecolor{rouge}{RGB}{255, 0, 0}
\definecolor{lightvert}{RGB}{205, 234, 190}
\setitemize[0]{label=\color{lightvert}  $\bullet$}

\pagestyle{fancy}
\renewcommand{\headrulewidth}{0.2pt}
\fancyhead[L]{maths-cours.fr}
\fancyhead[R]{\thepage}
\renewcommand{\footrulewidth}{0.2pt}
\fancyfoot[C]{}

\newcolumntype{C}{>{\centering\arraybackslash}X}
\newcolumntype{s}{>{\hsize=.35\hsize\arraybackslash}X}

\setlength{\parindent}{0pt}		 
\setlength{\parskip}{3mm}
\setlength{\headheight}{1cm}

\def\ebook{ebook}
\def\book{book}
\def\web{web}
\def\type{web}

\newcommand{\vect}[1]{\overrightarrow{\,\mathstrut#1\,}}

\def\Oij{$\left(\text{O}~;~\vect{\imath},~\vect{\jmath}\right)$}
\def\Oijk{$\left(\text{O}~;~\vect{\imath},~\vect{\jmath},~\vect{k}\right)$}
\def\Ouv{$\left(\text{O}~;~\vect{u},~\vect{v}\right)$}

\hypersetup{breaklinks=true, colorlinks = true, linkcolor = OliveGreen, urlcolor = OliveGreen, citecolor = OliveGreen, pdfauthor={Didier BONNEL - https://www.maths-cours.fr} } % supprime les bordures autour des liens

\renewcommand{\arg}[0]{\text{arg}}

\everymath{\displaystyle}

%================================================================================================================================
%
% Macros - Commandes
%
%================================================================================================================================

\newcommand\meta[2]{    			% Utilisé pour créer le post HTML.
	\def\titre{titre}
	\def\url{url}
	\def\arg{#1}
	\ifx\titre\arg
		\newcommand\maintitle{#2}
		\fancyhead[L]{#2}
		{\Large\sffamily \MakeUppercase{#2}}
		\vspace{1mm}\textcolor{mcvert}{\hrule}
	\fi 
	\ifx\url\arg
		\fancyfoot[L]{\href{https://www.maths-cours.fr#2}{\black \footnotesize{https://www.maths-cours.fr#2}}}
	\fi 
}


\newcommand\TitreC[1]{    		% Titre centré
     \needspace{3\baselineskip}
     \begin{center}\textbf{#1}\end{center}
}

\newcommand\newpar{    		% paragraphe
     \par
}

\newcommand\nosp {    		% commande vide (pas d'espace)
}
\newcommand{\id}[1]{} %ignore

\newcommand\boite[2]{				% Boite simple sans titre
	\vspace{5mm}
	\setlength{\fboxrule}{0.2mm}
	\setlength{\fboxsep}{5mm}	
	\fcolorbox{#1}{#1!3}{\makebox[\linewidth-2\fboxrule-2\fboxsep]{
  		\begin{minipage}[t]{\linewidth-2\fboxrule-4\fboxsep}\setlength{\parskip}{3mm}
  			 #2
  		\end{minipage}
	}}
	\vspace{5mm}
}

\newcommand\CBox[4]{				% Boites
	\vspace{5mm}
	\setlength{\fboxrule}{0.2mm}
	\setlength{\fboxsep}{5mm}
	
	\fcolorbox{#1}{#1!3}{\makebox[\linewidth-2\fboxrule-2\fboxsep]{
		\begin{minipage}[t]{1cm}\setlength{\parskip}{3mm}
	  		\textcolor{#1}{\LARGE{#2}}    
 	 	\end{minipage}  
  		\begin{minipage}[t]{\linewidth-2\fboxrule-4\fboxsep}\setlength{\parskip}{3mm}
			\raisebox{1.2mm}{\normalsize\sffamily{\textcolor{#1}{#3}}}						
  			 #4
  		\end{minipage}
	}}
	\vspace{5mm}
}

\newcommand\cadre[3]{				% Boites convertible html
	\par
	\vspace{2mm}
	\setlength{\fboxrule}{0.1mm}
	\setlength{\fboxsep}{5mm}
	\fcolorbox{#1}{white}{\makebox[\linewidth-2\fboxrule-2\fboxsep]{
  		\begin{minipage}[t]{\linewidth-2\fboxrule-4\fboxsep}\setlength{\parskip}{3mm}
			\raisebox{-2.5mm}{\sffamily \small{\textcolor{#1}{\MakeUppercase{#2}}}}		
			\par		
  			 #3
 	 		\end{minipage}
	}}
		\vspace{2mm}
	\par
}

\newcommand\bloc[3]{				% Boites convertible html sans bordure
     \needspace{2\baselineskip}
     {\sffamily \small{\textcolor{#1}{\MakeUppercase{#2}}}}    
		\par		
  			 #3
		\par
}

\newcommand\CHelp[1]{
     \CBox{Plum}{\faInfoCircle}{À RETENIR}{#1}
}

\newcommand\CUp[1]{
     \CBox{NavyBlue}{\faThumbsOUp}{EN PRATIQUE}{#1}
}

\newcommand\CInfo[1]{
     \CBox{Sepia}{\faArrowCircleRight}{REMARQUE}{#1}
}

\newcommand\CRedac[1]{
     \CBox{PineGreen}{\faEdit}{BIEN R\'EDIGER}{#1}
}

\newcommand\CError[1]{
     \CBox{Red}{\faExclamationTriangle}{ATTENTION}{#1}
}

\newcommand\TitreExo[2]{
\needspace{4\baselineskip}
 {\sffamily\large EXERCICE #1\ (\emph{#2 points})}
\vspace{5mm}
}

\newcommand\img[2]{
          \includegraphics[width=#2\paperwidth]{\imgdir#1}
}

\newcommand\imgsvg[2]{
       \begin{center}   \includegraphics[width=#2\paperwidth]{\imgsvgdir#1} \end{center}
}


\newcommand\Lien[2]{
     \href{#1}{#2 \tiny \faExternalLink}
}
\newcommand\mcLien[2]{
     \href{https~://www.maths-cours.fr/#1}{#2 \tiny \faExternalLink}
}

\newcommand{\euro}{\eurologo{}}

%================================================================================================================================
%
% Macros - Environement
%
%================================================================================================================================

\newenvironment{tex}{ %
}
{%
}

\newenvironment{indente}{ %
	\setlength\parindent{10mm}
}

{
	\setlength\parindent{0mm}
}

\newenvironment{corrige}{%
     \needspace{3\baselineskip}
     \medskip
     \textbf{\textsc{Corrigé}}
     \medskip
}
{
}

\newenvironment{extern}{%
     \begin{center}
     }
     {
     \end{center}
}

\NewEnviron{code}{%
	\par
     \boite{gray}{\texttt{%
     \BODY
     }}
     \par
}

\newenvironment{vbloc}{% boite sans cadre empeche saut de page
     \begin{minipage}[t]{\linewidth}
     }
     {
     \end{minipage}
}
\NewEnviron{h2}{%
    \needspace{3\baselineskip}
    \vspace{0.6cm}
	\noindent \MakeUppercase{\sffamily \large \BODY}
	\vspace{1mm}\textcolor{mcgris}{\hrule}\vspace{0.4cm}
	\par
}{}

\NewEnviron{h3}{%
    \needspace{3\baselineskip}
	\vspace{5mm}
	\textsc{\BODY}
	\par
}

\NewEnviron{margeneg}{ %
\begin{addmargin}[-1cm]{0cm}
\BODY
\end{addmargin}
}

\NewEnviron{html}{%
}

\begin{document}
\meta{url}{/cours/statistiques-echantillonnage/}
\meta{pid}{151}
\meta{titre}{Échantillonnage en Seconde}
\meta{type}{cours}
\begin{h2}1. Echantillons\end{h2}
Lorsqu'on travaille sur une population de grande taille, il est rarement possible d'avoir accès aux données relatives à l'ensemble de la population.
\par
On utilise alors un \textbf{échantillon} de cette population.
\cadre{bleu}{Définition}{%
     Un \textbf{échantillon de taille \textit{n}} est une sélection de \textit{n} individus choisis "au hasard" dans une population.
}
\bloc{orange}{Exemple}{%
     On étudie la répartition mâle/femelle d'une population de truites peuplant une rivière.
     \par
     Il est pratiquement impossible de recenser toutes les truites de la rivière. On décidera donc de travailler sur un échantillon en prélevant, par exemple, 100 truites.
     \par
     La taille de l'échantillon doit être suffisamment élevée pour fournir des résultats fiables ( mais pas trop pour ne pas entrainer un surcroit de travail important ! )
}
\bloc{cyan}{Remarque}{%
     Il existe deux manières d'effectuer un échantillonnage:
     \begin{itemize}
          \item \textbf{sans remise} : Dans l'exemple précédent, si l'on prélève les 100 truites simultanément, on obtient 100 individus différents
          \item \textbf{avec remise} : On prélève une truite au hasard, on note son sexe puis on la remet dans la rivière. Et on répète cette expérience 100 fois. Dans ce cas, il est possible de prélever plusieurs fois le même individu.
     \end{itemize}
     En pratique, si l'effectif global est nettement supérieur à la taille de l'échantillon ( c'est à dire, ici, si la rivière abrite beaucoup plus de 100 truites ) les deux méthodes donneront des résultats également satisfaisants.
}
\begin{h2}2. Intervalle de fluctuation\end{h2}
Si l'on effectue plusieurs échantillonnage de même taille sur une même population, on obtiendra en général des fréquences légèrement différentes pour un caractère donné.
\par
Voici, par exemple, les résultats que l'on pourrait obtenir en prélevant 5 échantillons de 100 truites :
\begin{center}
     \begin{tabular}{|c|c|c|c|c|c|}%class="compact" width="600"
          \hline
          Echantillons &  n°1  &  n°2  &  n°3  &  n°4 &  n°5
          \\ \hline
          Pourcentage de truites femelles &  52\%  &  55\%  &  42\%  &  50\%  &  48\%
          \\ \hline
     \end{tabular}
\end{center}
Ce phénomène s'appelle \textbf{fluctuation d'échantillonnage}.
\par
Le résultat suivant précise cette notion :
\cadre{rouge}{Théorème et définition}{%
     On note $p$ la proportion d'un caractère dans une population donnée.
     \par
     On prélève un échantillon de taille $n$ de cette population et on note $f $ la fréquence du caractère dans l'échantillon.
     \par
     Si $0,2 \leqslant p \leqslant 0,8$ et si $n\geqslant 25$ alors, dans au moins 95\% des cas, $f$ appartient à l'intervalle~:
     \begin{center}
          $I=\left[p-\frac{1}{\sqrt{n}}~;~p+\frac{1}{\sqrt{n}}\right]$.
     \end{center}
     $I$ est appelé \textbf{l'intervalle de fluctuation au seuil 95\%.}
}
\bloc{cyan}{Remarques}{%
     \begin{itemize}
          \item On applique le théorème ci-dessus si \textbf{on connaît la proportion $p$ du caractère dans la population}.
          \par
          On peut aussi utiliser ce théorème en \textbf{supposant} que le caractère est présent dans une proportion $p$. Suivant la (ou les) fréquence(s) observée(s) dans un (ou plusieurs) échantillon(s) on acceptera ou on rejettera l'hypothèse.
     \end{itemize}
     \begin{itemize}
          \item Bien retenir la signification de chacune des variables :
          \begin{itemize}
               \item $p$ = proportion du caractère dans l'\textbf{ensemble de la population}
               \item $f$ = fréquence du caractère dans l'\textbf{échantillon}
               \item $n$ = taille de l'échantillon
          \end{itemize}
          \item Au niveau Seconde, les intervalles de fluctuation seront toujours demandés au seuil de 95\%.
          \par
          Ce seuil a été choisi car :
          \begin{itemize}
               \item il conduit à une formule assez simple
               \item on peut considérer comme \textit{"raisonnablement fiable"} un résultat validé dans 95\% des cas
          \end{itemize}
     \end{itemize}
}
\bloc{orange}{Exemple}{%
     Supposons que notre rivière contienne 50\% de truites femelles (et donc 50\% de mâles...).
     \par
     Pour nos échantillons de taille 100, $n=100\geqslant 25$ ; par ailleurs $p=0,5 \in  \left[0,2 ; 0,8\right]$ Donc \textbf{l'intervalle de fluctuation au seuil de 95\%} sera $I=\left[0,5-\frac{1}{\sqrt{100}}~;~0,5+\frac{1}{\sqrt{100}}\right]$ c'est à dire $I=\left[0,4~;~0,6\right]$.
}

\end{document}
µ
\documentclass[a4paper]{article}

%================================================================================================================================
%
% Packages
%
%================================================================================================================================

\usepackage[T1]{fontenc} 	% pour caractères accentués
\usepackage[utf8]{inputenc}  % encodage utf8
\usepackage[french]{babel}	% langue : français
\usepackage{fourier}			% caractères plus lisibles
\usepackage[dvipsnames]{xcolor} % couleurs
\usepackage{fancyhdr}		% réglage header footer
\usepackage{needspace}		% empêcher sauts de page mal placés
\usepackage{graphicx}		% pour inclure des graphiques
\usepackage{enumitem,cprotect}		% personnalise les listes d'items (nécessaire pour ol, al ...)
\usepackage{hyperref}		% Liens hypertexte
\usepackage{pstricks,pst-all,pst-node,pstricks-add,pst-math,pst-plot,pst-tree,pst-eucl} % pstricks
\usepackage[a4paper,includeheadfoot,top=2cm,left=3cm, bottom=2cm,right=3cm]{geometry} % marges etc.
\usepackage{comment}			% commentaires multilignes
\usepackage{amsmath,environ} % maths (matrices, etc.)
\usepackage{amssymb,makeidx}
\usepackage{bm}				% bold maths
\usepackage{tabularx}		% tableaux
\usepackage{colortbl}		% tableaux en couleur
\usepackage{fontawesome}		% Fontawesome
\usepackage{environ}			% environment with command
\usepackage{fp}				% calculs pour ps-tricks
\usepackage{multido}			% pour ps tricks
\usepackage[np]{numprint}	% formattage nombre
\usepackage{tikz,tkz-tab} 			% package principal TikZ
\usepackage{pgfplots}   % axes
\usepackage{mathrsfs}    % cursives
\usepackage{calc}			% calcul taille boites
\usepackage[scaled=0.875]{helvet} % font sans serif
\usepackage{svg} % svg
\usepackage{scrextend} % local margin
\usepackage{scratch} %scratch
\usepackage{multicol} % colonnes
%\usepackage{infix-RPN,pst-func} % formule en notation polanaise inversée
\usepackage{listings}

%================================================================================================================================
%
% Réglages de base
%
%================================================================================================================================

\lstset{
language=Python,   % R code
literate=
{á}{{\'a}}1
{à}{{\`a}}1
{ã}{{\~a}}1
{é}{{\'e}}1
{è}{{\`e}}1
{ê}{{\^e}}1
{í}{{\'i}}1
{ó}{{\'o}}1
{õ}{{\~o}}1
{ú}{{\'u}}1
{ü}{{\"u}}1
{ç}{{\c{c}}}1
{~}{{ }}1
}


\definecolor{codegreen}{rgb}{0,0.6,0}
\definecolor{codegray}{rgb}{0.5,0.5,0.5}
\definecolor{codepurple}{rgb}{0.58,0,0.82}
\definecolor{backcolour}{rgb}{0.95,0.95,0.92}

\lstdefinestyle{mystyle}{
    backgroundcolor=\color{backcolour},   
    commentstyle=\color{codegreen},
    keywordstyle=\color{magenta},
    numberstyle=\tiny\color{codegray},
    stringstyle=\color{codepurple},
    basicstyle=\ttfamily\footnotesize,
    breakatwhitespace=false,         
    breaklines=true,                 
    captionpos=b,                    
    keepspaces=true,                 
    numbers=left,                    
xleftmargin=2em,
framexleftmargin=2em,            
    showspaces=false,                
    showstringspaces=false,
    showtabs=false,                  
    tabsize=2,
    upquote=true
}

\lstset{style=mystyle}


\lstset{style=mystyle}
\newcommand{\imgdir}{C:/laragon/www/newmc/assets/imgsvg/}
\newcommand{\imgsvgdir}{C:/laragon/www/newmc/assets/imgsvg/}

\definecolor{mcgris}{RGB}{220, 220, 220}% ancien~; pour compatibilité
\definecolor{mcbleu}{RGB}{52, 152, 219}
\definecolor{mcvert}{RGB}{125, 194, 70}
\definecolor{mcmauve}{RGB}{154, 0, 215}
\definecolor{mcorange}{RGB}{255, 96, 0}
\definecolor{mcturquoise}{RGB}{0, 153, 153}
\definecolor{mcrouge}{RGB}{255, 0, 0}
\definecolor{mclightvert}{RGB}{205, 234, 190}

\definecolor{gris}{RGB}{220, 220, 220}
\definecolor{bleu}{RGB}{52, 152, 219}
\definecolor{vert}{RGB}{125, 194, 70}
\definecolor{mauve}{RGB}{154, 0, 215}
\definecolor{orange}{RGB}{255, 96, 0}
\definecolor{turquoise}{RGB}{0, 153, 153}
\definecolor{rouge}{RGB}{255, 0, 0}
\definecolor{lightvert}{RGB}{205, 234, 190}
\setitemize[0]{label=\color{lightvert}  $\bullet$}

\pagestyle{fancy}
\renewcommand{\headrulewidth}{0.2pt}
\fancyhead[L]{maths-cours.fr}
\fancyhead[R]{\thepage}
\renewcommand{\footrulewidth}{0.2pt}
\fancyfoot[C]{}

\newcolumntype{C}{>{\centering\arraybackslash}X}
\newcolumntype{s}{>{\hsize=.35\hsize\arraybackslash}X}

\setlength{\parindent}{0pt}		 
\setlength{\parskip}{3mm}
\setlength{\headheight}{1cm}

\def\ebook{ebook}
\def\book{book}
\def\web{web}
\def\type{web}

\newcommand{\vect}[1]{\overrightarrow{\,\mathstrut#1\,}}

\def\Oij{$\left(\text{O}~;~\vect{\imath},~\vect{\jmath}\right)$}
\def\Oijk{$\left(\text{O}~;~\vect{\imath},~\vect{\jmath},~\vect{k}\right)$}
\def\Ouv{$\left(\text{O}~;~\vect{u},~\vect{v}\right)$}

\hypersetup{breaklinks=true, colorlinks = true, linkcolor = OliveGreen, urlcolor = OliveGreen, citecolor = OliveGreen, pdfauthor={Didier BONNEL - https://www.maths-cours.fr} } % supprime les bordures autour des liens

\renewcommand{\arg}[0]{\text{arg}}

\everymath{\displaystyle}

%================================================================================================================================
%
% Macros - Commandes
%
%================================================================================================================================

\newcommand\meta[2]{    			% Utilisé pour créer le post HTML.
	\def\titre{titre}
	\def\url{url}
	\def\arg{#1}
	\ifx\titre\arg
		\newcommand\maintitle{#2}
		\fancyhead[L]{#2}
		{\Large\sffamily \MakeUppercase{#2}}
		\vspace{1mm}\textcolor{mcvert}{\hrule}
	\fi 
	\ifx\url\arg
		\fancyfoot[L]{\href{https://www.maths-cours.fr#2}{\black \footnotesize{https://www.maths-cours.fr#2}}}
	\fi 
}


\newcommand\TitreC[1]{    		% Titre centré
     \needspace{3\baselineskip}
     \begin{center}\textbf{#1}\end{center}
}

\newcommand\newpar{    		% paragraphe
     \par
}

\newcommand\nosp {    		% commande vide (pas d'espace)
}
\newcommand{\id}[1]{} %ignore

\newcommand\boite[2]{				% Boite simple sans titre
	\vspace{5mm}
	\setlength{\fboxrule}{0.2mm}
	\setlength{\fboxsep}{5mm}	
	\fcolorbox{#1}{#1!3}{\makebox[\linewidth-2\fboxrule-2\fboxsep]{
  		\begin{minipage}[t]{\linewidth-2\fboxrule-4\fboxsep}\setlength{\parskip}{3mm}
  			 #2
  		\end{minipage}
	}}
	\vspace{5mm}
}

\newcommand\CBox[4]{				% Boites
	\vspace{5mm}
	\setlength{\fboxrule}{0.2mm}
	\setlength{\fboxsep}{5mm}
	
	\fcolorbox{#1}{#1!3}{\makebox[\linewidth-2\fboxrule-2\fboxsep]{
		\begin{minipage}[t]{1cm}\setlength{\parskip}{3mm}
	  		\textcolor{#1}{\LARGE{#2}}    
 	 	\end{minipage}  
  		\begin{minipage}[t]{\linewidth-2\fboxrule-4\fboxsep}\setlength{\parskip}{3mm}
			\raisebox{1.2mm}{\normalsize\sffamily{\textcolor{#1}{#3}}}						
  			 #4
  		\end{minipage}
	}}
	\vspace{5mm}
}

\newcommand\cadre[3]{				% Boites convertible html
	\par
	\vspace{2mm}
	\setlength{\fboxrule}{0.1mm}
	\setlength{\fboxsep}{5mm}
	\fcolorbox{#1}{white}{\makebox[\linewidth-2\fboxrule-2\fboxsep]{
  		\begin{minipage}[t]{\linewidth-2\fboxrule-4\fboxsep}\setlength{\parskip}{3mm}
			\raisebox{-2.5mm}{\sffamily \small{\textcolor{#1}{\MakeUppercase{#2}}}}		
			\par		
  			 #3
 	 		\end{minipage}
	}}
		\vspace{2mm}
	\par
}

\newcommand\bloc[3]{				% Boites convertible html sans bordure
     \needspace{2\baselineskip}
     {\sffamily \small{\textcolor{#1}{\MakeUppercase{#2}}}}    
		\par		
  			 #3
		\par
}

\newcommand\CHelp[1]{
     \CBox{Plum}{\faInfoCircle}{À RETENIR}{#1}
}

\newcommand\CUp[1]{
     \CBox{NavyBlue}{\faThumbsOUp}{EN PRATIQUE}{#1}
}

\newcommand\CInfo[1]{
     \CBox{Sepia}{\faArrowCircleRight}{REMARQUE}{#1}
}

\newcommand\CRedac[1]{
     \CBox{PineGreen}{\faEdit}{BIEN R\'EDIGER}{#1}
}

\newcommand\CError[1]{
     \CBox{Red}{\faExclamationTriangle}{ATTENTION}{#1}
}

\newcommand\TitreExo[2]{
\needspace{4\baselineskip}
 {\sffamily\large EXERCICE #1\ (\emph{#2 points})}
\vspace{5mm}
}

\newcommand\img[2]{
          \includegraphics[width=#2\paperwidth]{\imgdir#1}
}

\newcommand\imgsvg[2]{
       \begin{center}   \includegraphics[width=#2\paperwidth]{\imgsvgdir#1} \end{center}
}


\newcommand\Lien[2]{
     \href{#1}{#2 \tiny \faExternalLink}
}
\newcommand\mcLien[2]{
     \href{https~://www.maths-cours.fr/#1}{#2 \tiny \faExternalLink}
}

\newcommand{\euro}{\eurologo{}}

%================================================================================================================================
%
% Macros - Environement
%
%================================================================================================================================

\newenvironment{tex}{ %
}
{%
}

\newenvironment{indente}{ %
	\setlength\parindent{10mm}
}

{
	\setlength\parindent{0mm}
}

\newenvironment{corrige}{%
     \needspace{3\baselineskip}
     \medskip
     \textbf{\textsc{Corrigé}}
     \medskip
}
{
}

\newenvironment{extern}{%
     \begin{center}
     }
     {
     \end{center}
}

\NewEnviron{code}{%
	\par
     \boite{gray}{\texttt{%
     \BODY
     }}
     \par
}

\newenvironment{vbloc}{% boite sans cadre empeche saut de page
     \begin{minipage}[t]{\linewidth}
     }
     {
     \end{minipage}
}
\NewEnviron{h2}{%
    \needspace{3\baselineskip}
    \vspace{0.6cm}
	\noindent \MakeUppercase{\sffamily \large \BODY}
	\vspace{1mm}\textcolor{mcgris}{\hrule}\vspace{0.4cm}
	\par
}{}

\NewEnviron{h3}{%
    \needspace{3\baselineskip}
	\vspace{5mm}
	\textsc{\BODY}
	\par
}

\NewEnviron{margeneg}{ %
\begin{addmargin}[-1cm]{0cm}
\BODY
\end{addmargin}
}

\NewEnviron{html}{%
}

\begin{document}
\meta{url}{/cours/probabilites/}
\meta{pid}{155}
\meta{titre}{Probabilités en Seconde}
\meta{type}{cours}
\begin{h2}1. Expérience aléatoire\end{h2}
\cadre{bleu}{Définitions}{%
     Une expérience \textbf{aléatoire} est une expérience dont le résultat dépend du hasard.
     \par
     L'ensemble de tous les résultats possibles d'une expérience aléatoire s'appelle l'\textbf{univers} de l'expérience.
     \par
     On le note en général \textbf{$\Omega $}.
}
\cadre{bleu}{Définition}{%
     Soit une expérience aléatoire d'univers $\Omega $.
     \par
     Chacun des résultats possibles s'appelle une \textbf{éventualité} (ou un \textbf{événement élémentaire }ou une \textbf{issue}).
     \par
     On appelle \textbf{événement} tout sous ensemble de $\Omega $.
     \par
     Un événement est donc constitué de zéro, une ou plusieurs éventualités.
}
\bloc{orange}{Exemples}{%
     Le lancer d'un dé à six faces est une expérience aléatoire d'univers :
     \par
     $\Omega =\left\{1;2;3;4;5;6\right\}$
     \begin{itemize}
          \item L'ensemble $E_1=\left\{2;4;6\right\}$ est un événement. En français, cet événement peut se traduire par la phrase : « \textit{le résultat du dé est un nombre pair} »
          \item L'ensemble $E_2=\left\{1;2;3\right\}$ est un autre événement. Ce second événement peut se traduire par la phrase : « \textit{le résultat du dé est strictement inférieur à 4} »
     \end{itemize}
     Ces événements peuvent être représentés par un diagramme de Venn :
     \begin{center}
          \img{Venn}{0.33}%width="300" alt="Diagramme de Venn"
     \end{center}
}
\cadre{bleu}{Définition}{%
     \begin{itemize}
          \item l'\textbf{événement impossible} est la partie vide, noté $\varnothing $, lorsque aucune issue ne le réalise.
          \item l'\textbf{événement certain} est $\Omega $, lorsque toutes les issues le réalisent.
          \item l'\textbf{événement contraire} de $A$ noté $\overline A$ est l'ensemble des éventualités de $\Omega $ qui n'appartiennent pas à $A$.
          \item l'événement $A \cup  B$ (lire « $A$ union $B$ » ou « $A$ \textbf{ou} $B$ ») est constitué des éventualités qui appartiennent soit à $A$, soit à $B$, soit aux deux ensembles.
          \item l'événement $A \cap  B$ (lire « $A$ inter $B$ » ou « $A$ \textbf{et }$B$ ») est constitué des éventualités qui appartiennent à la fois à $A$ et à $B$.
     \end{itemize}
}
\bloc{orange}{Exemple}{%
     On reprend l'exemple précédent avec :
     \par
     $\Omega =\left\{1;2;3;4;5;6\right\}$
     \par
     $E_1=\left\{2;4;6\right\}$
     \par
     $E_2=\left\{1;2;3\right\}$
     \begin{itemize}
          \item  L'événement « obtenir un nombre supérieur à 7 » est l' événement impossible.
          \item  L'événement « obtenir un nombre entier » est l' événement certain.
          \item $\overline{E}_{1}=\left\{1; 3; 5\right\}$ : cet événement peut se traduire par \og le résultat est un nombre impair \fg{}~:
          \begin{center}
               \img{Venn-complementaire}{0.3}%width="300" alt="Diagramme de Venn - Complémentaire"
          \end{center}
          \item $E_{1} \cup  E_{2}=\left\{1; 2; 3; 4; 6\right\}$ : cet événement peut se traduire par \og le résultat est pair \textbf{ou} strictement inférieur à 4 \fg{}~:
          \begin{center}
               \img{Venn-union}{0.3}%width="300" alt="Diagramme de Venn - Union"
          \end{center}
          \item $E_{1} \cap  E_{2}=\left\{2\right\}$ : cet événement peut se traduire par \og le résultat est pair \textbf{et} strictement inférieur à 4 \fg{}~:
          \begin{center}
               \img{Venn-inter}{0.3}%width="300" alt="Diagramme de Venn - Intersection "
          \end{center}
     \end{itemize}
}
\cadre{bleu}{Définition}{%
     On dit que A et B sont \textbf{incompatibles} si et seulement si $A \cap  B=\varnothing$
     \par
     Deux événements sont incompatibles lorsqu'aucun événement ne les réalise simultanément.
}
\bloc{vert}{Remarque}{%
     Deux événements contraires sont incompatibles mais deux événements peuvent être incompatibles sans être contraires.
}
\bloc{orange}{Exemple}{%
     « Obtenir un chiffre inférieur à 2 » et « obtenir un chiffre supérieur à 4 » sont deux événements incompatibles.
}
\begin{h2}2. Probabilités\end{h2}
\cadre{bleu}{Définition}{%
     La probabilité d'un événement élémentaire est un nombre réel tel que:
     \begin{itemize}
          \item Ce nombre est compris entre 0 et 1
          \item La somme des probabilités de tous les événements élémentaires de l'univers vaut 1
     \end{itemize}
}
\cadre{vert}{Propriétés}{%
     \begin{itemize}
          \item $p\left(\varnothing\right)=0$
          \item $p\left(\Omega \right)=1$
          \item $p\left(\overline A\right)=1-p\left(A\right)$
     \end{itemize}
}
\bloc{orange}{Exemple}{%
     On lance un dé à six faces. On note $S$ l'événement : « obtenir un $6$. On suppose que le dé est bien équilibré et que la probabilité de $S$ est de $\frac{1}{6}$. La probabilité d'obtenir un résultat différent de $6$ est alors :
     \par
     $p\left(\overline S\right)=1-p\left(S\right)=1-\frac{1}{6}=\frac{5}{6}$
}
\cadre{rouge}{Théorème}{%
     Quels que soient les événements $A$ et $B$ de $\Omega $ :
     \par
     $p\left(A \cup  B\right)=p\left(A\right)+p\left(B\right)-p\left(A \cap  B\right)$
     \par
     En particulier, si $A$ et $B$ sont \textbf{incompatibles} :
     \par
     $p\left(A \cup  B\right)=p\left(A\right)+p\left(B\right)$
}
\cadre{bleu}{Définition}{%
     Deux événements qui ont la même probabilité sont dits \textbf{ équiprobables}.
     \par
     Lorsque tous les événements élémentaires sont équiprobables, on dit qu'il y a \textbf{équiprobabilité}.
}
\bloc{orange}{Exemple}{%
     Un lancer d'un dé non truqué est une situation d'équiprobabilité.
}
\cadre{vert}{Propriétés}{%
     On suppose que l'univers est composé de $n$ événements élémentaires
     \begin{itemize}
          \item Dans le cas d'équiprobabilité, chaque événement élémentaire a pour probabilité : $\frac{1}{n}$
          \item Si un événement $A$ de $ \Omega $ est composé de $m$ événements élémentaires, alors $P\left(A\right)=\frac{m}{n}$.
     \end{itemize}
}
\bloc{orange}{Exemple}{%
     On reprend l'exemple du lancer d'un dé avec $E_1$ : « le résultat du dé est un nombre pair »
     \par
     $P\left(E_1\right)=\frac{3}{6}=\frac{1}{2}$
}

\end{document}
µ
\documentclass[a4paper]{article}

%================================================================================================================================
%
% Packages
%
%================================================================================================================================

\usepackage[T1]{fontenc} 	% pour caractères accentués
\usepackage[utf8]{inputenc}  % encodage utf8
\usepackage[french]{babel}	% langue : français
\usepackage{fourier}			% caractères plus lisibles
\usepackage[dvipsnames]{xcolor} % couleurs
\usepackage{fancyhdr}		% réglage header footer
\usepackage{needspace}		% empêcher sauts de page mal placés
\usepackage{graphicx}		% pour inclure des graphiques
\usepackage{enumitem,cprotect}		% personnalise les listes d'items (nécessaire pour ol, al ...)
\usepackage{hyperref}		% Liens hypertexte
\usepackage{pstricks,pst-all,pst-node,pstricks-add,pst-math,pst-plot,pst-tree,pst-eucl} % pstricks
\usepackage[a4paper,includeheadfoot,top=2cm,left=3cm, bottom=2cm,right=3cm]{geometry} % marges etc.
\usepackage{comment}			% commentaires multilignes
\usepackage{amsmath,environ} % maths (matrices, etc.)
\usepackage{amssymb,makeidx}
\usepackage{bm}				% bold maths
\usepackage{tabularx}		% tableaux
\usepackage{colortbl}		% tableaux en couleur
\usepackage{fontawesome}		% Fontawesome
\usepackage{environ}			% environment with command
\usepackage{fp}				% calculs pour ps-tricks
\usepackage{multido}			% pour ps tricks
\usepackage[np]{numprint}	% formattage nombre
\usepackage{tikz,tkz-tab} 			% package principal TikZ
\usepackage{pgfplots}   % axes
\usepackage{mathrsfs}    % cursives
\usepackage{calc}			% calcul taille boites
\usepackage[scaled=0.875]{helvet} % font sans serif
\usepackage{svg} % svg
\usepackage{scrextend} % local margin
\usepackage{scratch} %scratch
\usepackage{multicol} % colonnes
%\usepackage{infix-RPN,pst-func} % formule en notation polanaise inversée
\usepackage{listings}

%================================================================================================================================
%
% Réglages de base
%
%================================================================================================================================

\lstset{
language=Python,   % R code
literate=
{á}{{\'a}}1
{à}{{\`a}}1
{ã}{{\~a}}1
{é}{{\'e}}1
{è}{{\`e}}1
{ê}{{\^e}}1
{í}{{\'i}}1
{ó}{{\'o}}1
{õ}{{\~o}}1
{ú}{{\'u}}1
{ü}{{\"u}}1
{ç}{{\c{c}}}1
{~}{{ }}1
}


\definecolor{codegreen}{rgb}{0,0.6,0}
\definecolor{codegray}{rgb}{0.5,0.5,0.5}
\definecolor{codepurple}{rgb}{0.58,0,0.82}
\definecolor{backcolour}{rgb}{0.95,0.95,0.92}

\lstdefinestyle{mystyle}{
    backgroundcolor=\color{backcolour},   
    commentstyle=\color{codegreen},
    keywordstyle=\color{magenta},
    numberstyle=\tiny\color{codegray},
    stringstyle=\color{codepurple},
    basicstyle=\ttfamily\footnotesize,
    breakatwhitespace=false,         
    breaklines=true,                 
    captionpos=b,                    
    keepspaces=true,                 
    numbers=left,                    
xleftmargin=2em,
framexleftmargin=2em,            
    showspaces=false,                
    showstringspaces=false,
    showtabs=false,                  
    tabsize=2,
    upquote=true
}

\lstset{style=mystyle}


\lstset{style=mystyle}
\newcommand{\imgdir}{C:/laragon/www/newmc/assets/imgsvg/}
\newcommand{\imgsvgdir}{C:/laragon/www/newmc/assets/imgsvg/}

\definecolor{mcgris}{RGB}{220, 220, 220}% ancien~; pour compatibilité
\definecolor{mcbleu}{RGB}{52, 152, 219}
\definecolor{mcvert}{RGB}{125, 194, 70}
\definecolor{mcmauve}{RGB}{154, 0, 215}
\definecolor{mcorange}{RGB}{255, 96, 0}
\definecolor{mcturquoise}{RGB}{0, 153, 153}
\definecolor{mcrouge}{RGB}{255, 0, 0}
\definecolor{mclightvert}{RGB}{205, 234, 190}

\definecolor{gris}{RGB}{220, 220, 220}
\definecolor{bleu}{RGB}{52, 152, 219}
\definecolor{vert}{RGB}{125, 194, 70}
\definecolor{mauve}{RGB}{154, 0, 215}
\definecolor{orange}{RGB}{255, 96, 0}
\definecolor{turquoise}{RGB}{0, 153, 153}
\definecolor{rouge}{RGB}{255, 0, 0}
\definecolor{lightvert}{RGB}{205, 234, 190}
\setitemize[0]{label=\color{lightvert}  $\bullet$}

\pagestyle{fancy}
\renewcommand{\headrulewidth}{0.2pt}
\fancyhead[L]{maths-cours.fr}
\fancyhead[R]{\thepage}
\renewcommand{\footrulewidth}{0.2pt}
\fancyfoot[C]{}

\newcolumntype{C}{>{\centering\arraybackslash}X}
\newcolumntype{s}{>{\hsize=.35\hsize\arraybackslash}X}

\setlength{\parindent}{0pt}		 
\setlength{\parskip}{3mm}
\setlength{\headheight}{1cm}

\def\ebook{ebook}
\def\book{book}
\def\web{web}
\def\type{web}

\newcommand{\vect}[1]{\overrightarrow{\,\mathstrut#1\,}}

\def\Oij{$\left(\text{O}~;~\vect{\imath},~\vect{\jmath}\right)$}
\def\Oijk{$\left(\text{O}~;~\vect{\imath},~\vect{\jmath},~\vect{k}\right)$}
\def\Ouv{$\left(\text{O}~;~\vect{u},~\vect{v}\right)$}

\hypersetup{breaklinks=true, colorlinks = true, linkcolor = OliveGreen, urlcolor = OliveGreen, citecolor = OliveGreen, pdfauthor={Didier BONNEL - https://www.maths-cours.fr} } % supprime les bordures autour des liens

\renewcommand{\arg}[0]{\text{arg}}

\everymath{\displaystyle}

%================================================================================================================================
%
% Macros - Commandes
%
%================================================================================================================================

\newcommand\meta[2]{    			% Utilisé pour créer le post HTML.
	\def\titre{titre}
	\def\url{url}
	\def\arg{#1}
	\ifx\titre\arg
		\newcommand\maintitle{#2}
		\fancyhead[L]{#2}
		{\Large\sffamily \MakeUppercase{#2}}
		\vspace{1mm}\textcolor{mcvert}{\hrule}
	\fi 
	\ifx\url\arg
		\fancyfoot[L]{\href{https://www.maths-cours.fr#2}{\black \footnotesize{https://www.maths-cours.fr#2}}}
	\fi 
}


\newcommand\TitreC[1]{    		% Titre centré
     \needspace{3\baselineskip}
     \begin{center}\textbf{#1}\end{center}
}

\newcommand\newpar{    		% paragraphe
     \par
}

\newcommand\nosp {    		% commande vide (pas d'espace)
}
\newcommand{\id}[1]{} %ignore

\newcommand\boite[2]{				% Boite simple sans titre
	\vspace{5mm}
	\setlength{\fboxrule}{0.2mm}
	\setlength{\fboxsep}{5mm}	
	\fcolorbox{#1}{#1!3}{\makebox[\linewidth-2\fboxrule-2\fboxsep]{
  		\begin{minipage}[t]{\linewidth-2\fboxrule-4\fboxsep}\setlength{\parskip}{3mm}
  			 #2
  		\end{minipage}
	}}
	\vspace{5mm}
}

\newcommand\CBox[4]{				% Boites
	\vspace{5mm}
	\setlength{\fboxrule}{0.2mm}
	\setlength{\fboxsep}{5mm}
	
	\fcolorbox{#1}{#1!3}{\makebox[\linewidth-2\fboxrule-2\fboxsep]{
		\begin{minipage}[t]{1cm}\setlength{\parskip}{3mm}
	  		\textcolor{#1}{\LARGE{#2}}    
 	 	\end{minipage}  
  		\begin{minipage}[t]{\linewidth-2\fboxrule-4\fboxsep}\setlength{\parskip}{3mm}
			\raisebox{1.2mm}{\normalsize\sffamily{\textcolor{#1}{#3}}}						
  			 #4
  		\end{minipage}
	}}
	\vspace{5mm}
}

\newcommand\cadre[3]{				% Boites convertible html
	\par
	\vspace{2mm}
	\setlength{\fboxrule}{0.1mm}
	\setlength{\fboxsep}{5mm}
	\fcolorbox{#1}{white}{\makebox[\linewidth-2\fboxrule-2\fboxsep]{
  		\begin{minipage}[t]{\linewidth-2\fboxrule-4\fboxsep}\setlength{\parskip}{3mm}
			\raisebox{-2.5mm}{\sffamily \small{\textcolor{#1}{\MakeUppercase{#2}}}}		
			\par		
  			 #3
 	 		\end{minipage}
	}}
		\vspace{2mm}
	\par
}

\newcommand\bloc[3]{				% Boites convertible html sans bordure
     \needspace{2\baselineskip}
     {\sffamily \small{\textcolor{#1}{\MakeUppercase{#2}}}}    
		\par		
  			 #3
		\par
}

\newcommand\CHelp[1]{
     \CBox{Plum}{\faInfoCircle}{À RETENIR}{#1}
}

\newcommand\CUp[1]{
     \CBox{NavyBlue}{\faThumbsOUp}{EN PRATIQUE}{#1}
}

\newcommand\CInfo[1]{
     \CBox{Sepia}{\faArrowCircleRight}{REMARQUE}{#1}
}

\newcommand\CRedac[1]{
     \CBox{PineGreen}{\faEdit}{BIEN R\'EDIGER}{#1}
}

\newcommand\CError[1]{
     \CBox{Red}{\faExclamationTriangle}{ATTENTION}{#1}
}

\newcommand\TitreExo[2]{
\needspace{4\baselineskip}
 {\sffamily\large EXERCICE #1\ (\emph{#2 points})}
\vspace{5mm}
}

\newcommand\img[2]{
          \includegraphics[width=#2\paperwidth]{\imgdir#1}
}

\newcommand\imgsvg[2]{
       \begin{center}   \includegraphics[width=#2\paperwidth]{\imgsvgdir#1} \end{center}
}


\newcommand\Lien[2]{
     \href{#1}{#2 \tiny \faExternalLink}
}
\newcommand\mcLien[2]{
     \href{https~://www.maths-cours.fr/#1}{#2 \tiny \faExternalLink}
}

\newcommand{\euro}{\eurologo{}}

%================================================================================================================================
%
% Macros - Environement
%
%================================================================================================================================

\newenvironment{tex}{ %
}
{%
}

\newenvironment{indente}{ %
	\setlength\parindent{10mm}
}

{
	\setlength\parindent{0mm}
}

\newenvironment{corrige}{%
     \needspace{3\baselineskip}
     \medskip
     \textbf{\textsc{Corrigé}}
     \medskip
}
{
}

\newenvironment{extern}{%
     \begin{center}
     }
     {
     \end{center}
}

\NewEnviron{code}{%
	\par
     \boite{gray}{\texttt{%
     \BODY
     }}
     \par
}

\newenvironment{vbloc}{% boite sans cadre empeche saut de page
     \begin{minipage}[t]{\linewidth}
     }
     {
     \end{minipage}
}
\NewEnviron{h2}{%
    \needspace{3\baselineskip}
    \vspace{0.6cm}
	\noindent \MakeUppercase{\sffamily \large \BODY}
	\vspace{1mm}\textcolor{mcgris}{\hrule}\vspace{0.4cm}
	\par
}{}

\NewEnviron{h3}{%
    \needspace{3\baselineskip}
	\vspace{5mm}
	\textsc{\BODY}
	\par
}

\NewEnviron{margeneg}{ %
\begin{addmargin}[-1cm]{0cm}
\BODY
\end{addmargin}
}

\NewEnviron{html}{%
}

\begin{document}
\meta{url}{/cours/droites/}
\meta{pid}{166}
\meta{titre}{Équations de droites}
\meta{type}{cours}
\begin{h2}1. Équation réduite d'une droite\end{h2}
\cadre{vert}{Propriété}{%
     Une droite du plan peut être caractérisée une équation de la forme :
     \begin{itemize}
          \item $x=c$ si cette droite est parallèle à l'axe des ordonnées (\textit{« verticale »})
          \item $y=mx+p$ si cette droite n'est pas parallèle à l'axe des ordonnées.
     \end{itemize}
     Dans le second cas, $m$ est appelé coefficient directeur et $p$ ordonnée à l'origine.
}
\bloc{orange}{Exemples}{%
     \begin{center}
          \begin{extern}%width="550" alt="équations de droites"
               \begin{tabular}{c c c}
                    \resizebox{5.5cm}{!}{%
                         % -+-+-+ variables modifiables
                         \def\fonction{1+0.2*x*x }
                         \def\xmin{-2.5}
                         \def\xmax{3.5}
                         \def\ymin{-2.5}
                         \def\ymax{4.5}
                         \def\xunit{1}  % unités en cm
                         \def\yunit{1}
                         \psset{xunit=\xunit,yunit=\yunit,algebraic=true}
                         \fontsize{12pt}{12pt}\selectfont
                         \begin{pspicture*}[linewidth=1pt](\xmin,\ymin)(\xmax,\ymax)
                              %      \psgrid[gridcolor=mcgris, subgriddiv=5, gridlabels=0pt](\xmin,\ymin)(\xmax,\ymax)
                              \psaxes[linewidth=0.75pt,Dx=1,Dy=1]{->}(0,0)(\xmin,\ymin)(\xmax,\ymax)
                              \psline[linewidth=0.75pt,linecolor=red](1,\ymin)(1,\ymax)
                              \rput[tr](-0.3,-0.3){$O$}
                              \rput[l](1.2,4){$\color{red} x=1$}
                         \end{pspicture*}
                    }
                    & ~~~~ &%
                    \resizebox{5.5cm}{!}{%
                         % -+-+-+ variables modifiables
                         \def\fonction{2*x-1 }
                         \def\xmin{-2.5}
                         \def\xmax{3.5}
                         \def\ymin{-2.5}
                         \def\ymax{4.5}
                         \def\xunit{1}  % unités en cm
                         \def\yunit{1}
                         \psset{xunit=\xunit,yunit=\yunit,algebraic=true}
                         \fontsize{12pt}{12pt}\selectfont
                         \begin{pspicture*}[linewidth=1pt](\xmin,\ymin)(\xmax,\ymax)
                              %      \psgrid[gridcolor=mcgris, subgriddiv=5, gridlabels=0pt](\xmin,\ymin)(\xmax,\ymax)
                              \psaxes[linewidth=0.75pt,Dx=1,Dy=1]{->}(0,0)(\xmin,\ymin)(\xmax,\ymax)
                              \psplot[plotpoints=2000,linecolor=red]{\xmin}{\xmax}{\fonction}
                              \rput[tr](-0.3,-0.3){$O$}
                              \rput[tl](2.7,4){$\color{red} y=2x+1$}
                         \end{pspicture*}
                    }
                    \\
                    Droite d'équation $x=1$ & ~~~~ & Droite d'équation $y=2x-1$ %
                    \\
               \end{tabular}
          \end{extern}
     \end{center}
}
\bloc{cyan}{Remarques}{%
     \begin{itemize}
          \item L'équation d'une droite peut s'écrire sous plusieurs formes. Par exemple $y=2x-1$ est équivalente à $y-2x+1=0$ ou $2y-4x+2=0$, etc.
          \par
          Les formes $x=c$ et $y=mx+p$ sont appelées \textbf{équation réduite} de la droite.
          \item Cette propriété indique que toute droite qui n'est pas parallèle à l'axe des ordonnées est la représentation graphique d'une fonction affine.(Voir chapitre \mcLien{/seconde/fonctions-lineaires-affines}{Fonctions linéaires et affines})
          \item Une droite parallèle à l'axe des abscisses a un coefficient direct $m$ égal à zéro. Son équation est donc de la forme $y=p$. C'est la représentation graphique d'une fonction constante.
     \end{itemize}
}
\cadre{vert}{Propriété}{%
     Soient $A$ et $B$ deux points du plan tels que $x_A\neq x_B$.
     \par
     Le coefficient directeur de la droite $\left(AB\right)$ est :
     \begin{center}$m = \frac{y_B-y_A}{x_B-x_A}$\end{center}
}
\bloc{cyan}{Remarque}{%
     Une fois que le coefficient directeur de la droite $\left(AB\right)$ est connu, on peut trouver l'ordonnée à l'origine en sachant que la droite $\left(AB\right)$ passe par le point $A$ donc que les coordonnées de $A$ vérifient l'équation de la droite.
}
\bloc{orange}{Exemple}{%
     On recherche l'équation de la droite passant par les points $A\left(1 ; 3\right)$ et $B\left(3 ; 5\right)$.
     \par
     Les points $A$ et $B$ n'ayant pas la même abscisse, cette équation est du type $y=mx+p$ avec :
     \par
     $m = \frac{y_B-y_A}{x_B-x_A}=\frac{5-3}{3-1}=\frac{2}{2}=1$
     \par
     Donc l'équation de $\left(AB\right)$ est de la forme $y=x+p$. Comme cette droite passe par $A$, l'équation est vérifiée si on remplace $x$ et $y$ par les coordonnées de $A$ donc :
     \par
     $3=1+p$ soit $p=2$.
     \par
     L'équation de $\left(AB\right)$ est donc $y=x+2$.
}
\begin{h2}2. Droites parallèles - Droites sécantes\end{h2}
\cadre{vert}{Propriété}{%
     Deux droites d'équations respectives $y=mx+p$ et $y=m^{\prime}x+p^{\prime}$ sont \textbf{parallèles} si et seulement si elles ont le même coefficient directeur : $m=m^{\prime}$.
}
\bloc{orange}{Exemple}{%
     \begin{center}
          \begin{extern}%width="320" alt="Droites parallèles"
               \resizebox{6cm}{!}{%
                    % -+-+-+ variables modifiables
                    \def\fonction{2*x-1 }
                    \def\g{2*x+3 }
                    \def\xmin{-4.5}
                    \def\xmax{3.5}
                    \def\ymin{-2.5}
                    \def\ymax{4.5}
                    \def\xunit{1}  % unités en cm
                    \def\yunit{1}
                    \psset{xunit=\xunit,yunit=\yunit,algebraic=true}
                    \fontsize{12pt}{12pt}\selectfont
                    \begin{pspicture*}[linewidth=1pt](\xmin,\ymin)(\xmax,\ymax)
                         %      \psgrid[gridcolor=mcgris, subgriddiv=5, gridlabels=0pt](\xmin,\ymin)(\xmax,\ymax)
                         \psaxes[linewidth=0.75pt,Dx=1,Dy=1]{->}(0,0)(\xmin,\ymin)(\xmax,\ymax)
                         \psplot[plotpoints=2000,linecolor=red]{\xmin}{\xmax}{\fonction}
                         \psplot[plotpoints=2000,linecolor=blue]{\xmin}{\xmax}{\g}
                         \rput[tr](-0.3,-0.3){$O$}
                         \rput[tl](2.6,4){$\color{red} y=2x-1$}
                         \rput[tr](-2.8,-2){$\color{blue} y=2x+3$}
                    \end{pspicture*}
               }
          \end{extern}
     \end{center}
     \begin{center}
          \'Equations de droites parallèles
     \end{center}
}
\cadre{vert}{Méthode}{%
     Soient $\mathscr D$ et $\mathscr D^{\prime}$ deux droites sécantes d'équations respectives $y=mx+p$ et $y=m^{\prime}x+p^{\prime}$.
     \par
     Les coordonnées $\left(x ; y\right)$ du point d'intersection des droites $\mathscr D$ et $\mathscr D^{\prime}$ s'obtiennent en résolvant le système :
     \par
     $\left\{ \begin{matrix} y=mx+p \\ y=m^{\prime}x+p^{\prime} \end{matrix}\right.$
}
\bloc{cyan}{Remarque}{%
     Ce système se résout simplement par substitution. Il est équivalent à :
     \par
     $\left\{ \begin{matrix} mx+p=m^{\prime}x+p^{\prime} \\ y=mx+p \end{matrix}\right.$
}
\bloc{orange}{Exemple}{%
     On cherche les coordonnées du point d'intersection des droites $\mathscr D$ et $\mathscr D^{\prime}$ d'équations respectives $y=2x+1$ et $y=3x-1$.
     \par
     Ces droites n'ont pas le même coefficient directeur donc elles sont sécantes.
     \par
     Les coordonnées du point d'intersection vérifient le système :
     \par
     $\left\{ \begin{matrix} y=2x+1 \\ y=3x-1 \end{matrix}\right.$
     \par
     qui équivaut à :
     \par
     $\left\{ \begin{matrix} y=2x+1 \\ y=3x-1 \end{matrix}\right. \Leftrightarrow \left\{ \begin{matrix} 2x+1=3x-1 \\ y=3x-1 \end{matrix}\right.$
     \par
     $\phantom{\left\{ \begin{matrix} y=2x+1 \\ y=3x-1 \end{matrix}\right.} \Leftrightarrow \left\{ \begin{matrix} x=2 \\ y=3x-1 \end{matrix}\right.$
     \par
     $\phantom{\left\{ \begin{matrix} y=2x+1 \\ y=3x-1 \end{matrix}\right.} \Leftrightarrow \left\{ \begin{matrix} x=2 \\ y=5 \end{matrix}\right.$
     \par
     Le point d'intersection a pour coordonnées $\left(2 ; 5\right)$.
}

\end{document}
µ
\documentclass[a4paper]{article}

%================================================================================================================================
%
% Packages
%
%================================================================================================================================

\usepackage[T1]{fontenc} 	% pour caractères accentués
\usepackage[utf8]{inputenc}  % encodage utf8
\usepackage[french]{babel}	% langue : français
\usepackage{fourier}			% caractères plus lisibles
\usepackage[dvipsnames]{xcolor} % couleurs
\usepackage{fancyhdr}		% réglage header footer
\usepackage{needspace}		% empêcher sauts de page mal placés
\usepackage{graphicx}		% pour inclure des graphiques
\usepackage{enumitem,cprotect}		% personnalise les listes d'items (nécessaire pour ol, al ...)
\usepackage{hyperref}		% Liens hypertexte
\usepackage{pstricks,pst-all,pst-node,pstricks-add,pst-math,pst-plot,pst-tree,pst-eucl} % pstricks
\usepackage[a4paper,includeheadfoot,top=2cm,left=3cm, bottom=2cm,right=3cm]{geometry} % marges etc.
\usepackage{comment}			% commentaires multilignes
\usepackage{amsmath,environ} % maths (matrices, etc.)
\usepackage{amssymb,makeidx}
\usepackage{bm}				% bold maths
\usepackage{tabularx}		% tableaux
\usepackage{colortbl}		% tableaux en couleur
\usepackage{fontawesome}		% Fontawesome
\usepackage{environ}			% environment with command
\usepackage{fp}				% calculs pour ps-tricks
\usepackage{multido}			% pour ps tricks
\usepackage[np]{numprint}	% formattage nombre
\usepackage{tikz,tkz-tab} 			% package principal TikZ
\usepackage{pgfplots}   % axes
\usepackage{mathrsfs}    % cursives
\usepackage{calc}			% calcul taille boites
\usepackage[scaled=0.875]{helvet} % font sans serif
\usepackage{svg} % svg
\usepackage{scrextend} % local margin
\usepackage{scratch} %scratch
\usepackage{multicol} % colonnes
%\usepackage{infix-RPN,pst-func} % formule en notation polanaise inversée
\usepackage{listings}

%================================================================================================================================
%
% Réglages de base
%
%================================================================================================================================

\lstset{
language=Python,   % R code
literate=
{á}{{\'a}}1
{à}{{\`a}}1
{ã}{{\~a}}1
{é}{{\'e}}1
{è}{{\`e}}1
{ê}{{\^e}}1
{í}{{\'i}}1
{ó}{{\'o}}1
{õ}{{\~o}}1
{ú}{{\'u}}1
{ü}{{\"u}}1
{ç}{{\c{c}}}1
{~}{{ }}1
}


\definecolor{codegreen}{rgb}{0,0.6,0}
\definecolor{codegray}{rgb}{0.5,0.5,0.5}
\definecolor{codepurple}{rgb}{0.58,0,0.82}
\definecolor{backcolour}{rgb}{0.95,0.95,0.92}

\lstdefinestyle{mystyle}{
    backgroundcolor=\color{backcolour},   
    commentstyle=\color{codegreen},
    keywordstyle=\color{magenta},
    numberstyle=\tiny\color{codegray},
    stringstyle=\color{codepurple},
    basicstyle=\ttfamily\footnotesize,
    breakatwhitespace=false,         
    breaklines=true,                 
    captionpos=b,                    
    keepspaces=true,                 
    numbers=left,                    
xleftmargin=2em,
framexleftmargin=2em,            
    showspaces=false,                
    showstringspaces=false,
    showtabs=false,                  
    tabsize=2,
    upquote=true
}

\lstset{style=mystyle}


\lstset{style=mystyle}
\newcommand{\imgdir}{C:/laragon/www/newmc/assets/imgsvg/}
\newcommand{\imgsvgdir}{C:/laragon/www/newmc/assets/imgsvg/}

\definecolor{mcgris}{RGB}{220, 220, 220}% ancien~; pour compatibilité
\definecolor{mcbleu}{RGB}{52, 152, 219}
\definecolor{mcvert}{RGB}{125, 194, 70}
\definecolor{mcmauve}{RGB}{154, 0, 215}
\definecolor{mcorange}{RGB}{255, 96, 0}
\definecolor{mcturquoise}{RGB}{0, 153, 153}
\definecolor{mcrouge}{RGB}{255, 0, 0}
\definecolor{mclightvert}{RGB}{205, 234, 190}

\definecolor{gris}{RGB}{220, 220, 220}
\definecolor{bleu}{RGB}{52, 152, 219}
\definecolor{vert}{RGB}{125, 194, 70}
\definecolor{mauve}{RGB}{154, 0, 215}
\definecolor{orange}{RGB}{255, 96, 0}
\definecolor{turquoise}{RGB}{0, 153, 153}
\definecolor{rouge}{RGB}{255, 0, 0}
\definecolor{lightvert}{RGB}{205, 234, 190}
\setitemize[0]{label=\color{lightvert}  $\bullet$}

\pagestyle{fancy}
\renewcommand{\headrulewidth}{0.2pt}
\fancyhead[L]{maths-cours.fr}
\fancyhead[R]{\thepage}
\renewcommand{\footrulewidth}{0.2pt}
\fancyfoot[C]{}

\newcolumntype{C}{>{\centering\arraybackslash}X}
\newcolumntype{s}{>{\hsize=.35\hsize\arraybackslash}X}

\setlength{\parindent}{0pt}		 
\setlength{\parskip}{3mm}
\setlength{\headheight}{1cm}

\def\ebook{ebook}
\def\book{book}
\def\web{web}
\def\type{web}

\newcommand{\vect}[1]{\overrightarrow{\,\mathstrut#1\,}}

\def\Oij{$\left(\text{O}~;~\vect{\imath},~\vect{\jmath}\right)$}
\def\Oijk{$\left(\text{O}~;~\vect{\imath},~\vect{\jmath},~\vect{k}\right)$}
\def\Ouv{$\left(\text{O}~;~\vect{u},~\vect{v}\right)$}

\hypersetup{breaklinks=true, colorlinks = true, linkcolor = OliveGreen, urlcolor = OliveGreen, citecolor = OliveGreen, pdfauthor={Didier BONNEL - https://www.maths-cours.fr} } % supprime les bordures autour des liens

\renewcommand{\arg}[0]{\text{arg}}

\everymath{\displaystyle}

%================================================================================================================================
%
% Macros - Commandes
%
%================================================================================================================================

\newcommand\meta[2]{    			% Utilisé pour créer le post HTML.
	\def\titre{titre}
	\def\url{url}
	\def\arg{#1}
	\ifx\titre\arg
		\newcommand\maintitle{#2}
		\fancyhead[L]{#2}
		{\Large\sffamily \MakeUppercase{#2}}
		\vspace{1mm}\textcolor{mcvert}{\hrule}
	\fi 
	\ifx\url\arg
		\fancyfoot[L]{\href{https://www.maths-cours.fr#2}{\black \footnotesize{https://www.maths-cours.fr#2}}}
	\fi 
}


\newcommand\TitreC[1]{    		% Titre centré
     \needspace{3\baselineskip}
     \begin{center}\textbf{#1}\end{center}
}

\newcommand\newpar{    		% paragraphe
     \par
}

\newcommand\nosp {    		% commande vide (pas d'espace)
}
\newcommand{\id}[1]{} %ignore

\newcommand\boite[2]{				% Boite simple sans titre
	\vspace{5mm}
	\setlength{\fboxrule}{0.2mm}
	\setlength{\fboxsep}{5mm}	
	\fcolorbox{#1}{#1!3}{\makebox[\linewidth-2\fboxrule-2\fboxsep]{
  		\begin{minipage}[t]{\linewidth-2\fboxrule-4\fboxsep}\setlength{\parskip}{3mm}
  			 #2
  		\end{minipage}
	}}
	\vspace{5mm}
}

\newcommand\CBox[4]{				% Boites
	\vspace{5mm}
	\setlength{\fboxrule}{0.2mm}
	\setlength{\fboxsep}{5mm}
	
	\fcolorbox{#1}{#1!3}{\makebox[\linewidth-2\fboxrule-2\fboxsep]{
		\begin{minipage}[t]{1cm}\setlength{\parskip}{3mm}
	  		\textcolor{#1}{\LARGE{#2}}    
 	 	\end{minipage}  
  		\begin{minipage}[t]{\linewidth-2\fboxrule-4\fboxsep}\setlength{\parskip}{3mm}
			\raisebox{1.2mm}{\normalsize\sffamily{\textcolor{#1}{#3}}}						
  			 #4
  		\end{minipage}
	}}
	\vspace{5mm}
}

\newcommand\cadre[3]{				% Boites convertible html
	\par
	\vspace{2mm}
	\setlength{\fboxrule}{0.1mm}
	\setlength{\fboxsep}{5mm}
	\fcolorbox{#1}{white}{\makebox[\linewidth-2\fboxrule-2\fboxsep]{
  		\begin{minipage}[t]{\linewidth-2\fboxrule-4\fboxsep}\setlength{\parskip}{3mm}
			\raisebox{-2.5mm}{\sffamily \small{\textcolor{#1}{\MakeUppercase{#2}}}}		
			\par		
  			 #3
 	 		\end{minipage}
	}}
		\vspace{2mm}
	\par
}

\newcommand\bloc[3]{				% Boites convertible html sans bordure
     \needspace{2\baselineskip}
     {\sffamily \small{\textcolor{#1}{\MakeUppercase{#2}}}}    
		\par		
  			 #3
		\par
}

\newcommand\CHelp[1]{
     \CBox{Plum}{\faInfoCircle}{À RETENIR}{#1}
}

\newcommand\CUp[1]{
     \CBox{NavyBlue}{\faThumbsOUp}{EN PRATIQUE}{#1}
}

\newcommand\CInfo[1]{
     \CBox{Sepia}{\faArrowCircleRight}{REMARQUE}{#1}
}

\newcommand\CRedac[1]{
     \CBox{PineGreen}{\faEdit}{BIEN R\'EDIGER}{#1}
}

\newcommand\CError[1]{
     \CBox{Red}{\faExclamationTriangle}{ATTENTION}{#1}
}

\newcommand\TitreExo[2]{
\needspace{4\baselineskip}
 {\sffamily\large EXERCICE #1\ (\emph{#2 points})}
\vspace{5mm}
}

\newcommand\img[2]{
          \includegraphics[width=#2\paperwidth]{\imgdir#1}
}

\newcommand\imgsvg[2]{
       \begin{center}   \includegraphics[width=#2\paperwidth]{\imgsvgdir#1} \end{center}
}


\newcommand\Lien[2]{
     \href{#1}{#2 \tiny \faExternalLink}
}
\newcommand\mcLien[2]{
     \href{https~://www.maths-cours.fr/#1}{#2 \tiny \faExternalLink}
}

\newcommand{\euro}{\eurologo{}}

%================================================================================================================================
%
% Macros - Environement
%
%================================================================================================================================

\newenvironment{tex}{ %
}
{%
}

\newenvironment{indente}{ %
	\setlength\parindent{10mm}
}

{
	\setlength\parindent{0mm}
}

\newenvironment{corrige}{%
     \needspace{3\baselineskip}
     \medskip
     \textbf{\textsc{Corrigé}}
     \medskip
}
{
}

\newenvironment{extern}{%
     \begin{center}
     }
     {
     \end{center}
}

\NewEnviron{code}{%
	\par
     \boite{gray}{\texttt{%
     \BODY
     }}
     \par
}

\newenvironment{vbloc}{% boite sans cadre empeche saut de page
     \begin{minipage}[t]{\linewidth}
     }
     {
     \end{minipage}
}
\NewEnviron{h2}{%
    \needspace{3\baselineskip}
    \vspace{0.6cm}
	\noindent \MakeUppercase{\sffamily \large \BODY}
	\vspace{1mm}\textcolor{mcgris}{\hrule}\vspace{0.4cm}
	\par
}{}

\NewEnviron{h3}{%
    \needspace{3\baselineskip}
	\vspace{5mm}
	\textsc{\BODY}
	\par
}

\NewEnviron{margeneg}{ %
\begin{addmargin}[-1cm]{0cm}
\BODY
\end{addmargin}
}

\NewEnviron{html}{%
}

\begin{document}
\meta{url}{/cours/vecteurs/}
\meta{pid}{194}
\meta{titre}{Généralités sur les vecteurs}
\meta{type}{cours}
\begin{h2}1. Notion de vecteur\end{h2}
\cadre{bleu}{Définition}{%id="d10"
     Un \textbf{vecteur} est défini par sa \textbf{direction}, son \textbf{sens} et sa \textbf{longueur}.
}
\bloc{vert}{Remarque}{%id="r10"
     Le mot \textbf{direction} désigne la direction de la droite qui "porte" ce vecteur; le mot \textbf{sens} permet de définir un sens de parcours sur cette droite parmi les deux possibles.
}
\bloc{orange}{Exemple}{%id="e10"
     \begin{center}
          \begin{extern}%width="220" alt="vecteurs égaux"
               \resizebox{5.5cm}{!}{
                    \psset{xunit=1.0cm,yunit=1.0cm,algebraic=true,dimen=middle,dotstyle=*,dotsize=3pt 0,linewidth=1pt,arrowsize=3pt 2,arrowinset=0.25}
                    \begin{pspicture*}(0.,0.5)(8.2,6.)
                         \begin{Large}
                              \psline{->}(1.,3.)(6.,5.)
                              \psline{->}(2.,1.)(7.,3.)
                              \psdots[dotstyle=*,linecolor=blue](1.,3.)
                              \rput[bl](0.38,2.94){\blue{$A$}}
                              \psdots[dotstyle=*,linecolor=blue](6.,5.)
                              \rput[bl](6.08,5.2){\blue{$B$}}
                              \psdots[dotstyle=*,linecolor=blue](2.,1.)
                              \rput[bl](1.42,0.88){\blue{$C$}}
                              \psdots[dotstyle=*,linecolor=blue](7.,3.)
                              \rput[bl](7.08,3.2){\blue{$D$}}
                         \end{Large}
                    \end{pspicture*}
               }
          \end{extern}
     \end{center}
     Les vecteurs $\overrightarrow{AB}$ et $\overrightarrow{CD}$ ont la même direction, le même sens, la même longueur. Ils sont égaux.
}
\bloc{vert}{Remarque}{%id="r20"
     \begin{center}
          \begin{extern}%width="240" alt="nommer un vecteur"
               \resizebox{6cm}{!}{
                    \psset{xunit=1.0cm,yunit=1.0cm,algebraic=true,dimen=middle,dotstyle=*,dotsize=3pt 0,linewidth=1pt,arrowsize=3pt 2,arrowinset=0.25}
                    \begin{pspicture*}(0.,0.5)(9.2,4.)
                         \begin{Large}
                              \psline{->}(1.,1.)(8.,3.)
                              \psdots[dotstyle=*,linecolor=blue](1.,1.)
                              \rput[bl](0.38,0.94){\blue{$A$}}
                              \psdots[dotstyle=*,linecolor=blue](8.,3.)
                              \rput[bl](8.08,3.2){\blue{$B$}}
                              \rput[bl](4.3,2.3){\red{$\vect{u}$}}
                         \end{Large}
                    \end{pspicture*}
               }
          \end{extern}
     \end{center}
     Pour nommer un vecteur on peut :
     \begin{itemize}
          \item utiliser l'origine et l'extrémité d'un représentant du vecteur : on parlera du vecteur $\overrightarrow{AB}$
          \item lui donner un nom à l'aide d'une lettre (en générale minuscule) : on parlera alors du vecteur $\vec{u}$
     \end{itemize}
}
\cadre{bleu}{Définition}{%id="d20"
     $\overrightarrow{AA}$, $\overrightarrow{BB}$, ... représentent un même vecteur de longueur nulle appelé \textbf{vecteur nul} et noté $\overrightarrow{0}$.
}
\bloc{cyan}{Remarque}{%id="r30"
     Le vecteur nul est assez particulier. En effet, contrairement aux autres vecteurs, il n'a ni direction, ni sens! Mais il intervient souvent dans les calculs.
}
\cadre{bleu}{Définition}{%id="d30"
     On appelle \textbf{norme} du vecteur $\overrightarrow{AB}$ et on note $||\overrightarrow{AB}||$ la longueur du segment $\left[AB\right]$ .
}
\bloc{vert}{Remarque}{%id="r40"
     On a donc $||\overrightarrow{AB}||=AB$.
}
\cadre{vert}{Propriété}{%id="p40"
     $M$ est le milieu du segment $\left[AB\right]$ si et seulement si $\overrightarrow{AM}=\overrightarrow{MB}$.
}
\begin{center}
     \begin{extern}%width="240" alt="nommer un vecteur"
          \resizebox{6cm}{!}{
               \psset{xunit=1.0cm,yunit=1.0cm,algebraic=true,dimen=middle,dotstyle=*,dotsize=3pt 0,linewidth=1pt,arrowsize=3pt 2,arrowinset=0.25}
               \begin{pspicture*}(0.,0.5)(9.2,4.)
                    \begin{Large}
                         \psline{->}(1.,1.)(8.,3.)
                         \psline{->}(1.,1.)(4.5,2.)
                         \psdots[dotstyle=*,linecolor=blue](1.,1.)
                         \rput[bl](0.38,0.94){\blue{$A$}}
                         \psdots[dotstyle=*,linecolor=blue](8.,3.)
                         \rput[bl](8.08,3.2){\blue{$B$}}
                         \rput[bl](4.3,2.3){\blue{$M$}}
                    \end{Large}
               \end{pspicture*}
          }
     \end{extern}
\end{center}
\bloc{cyan}{Remarque}{%id="r50"
     On rappelle que l'égalité de distance $AM=MB$ est insuffisante pour montrer que $M$ est le milieu de $\left[AB\right]$ (cette égalité montre seulement que M est équidistant de $A$ et $B$ c'est à dire est sur la médiatrice de $\left[AB\right]$). L'égalité de vecteurs $\overrightarrow{AM}=\overrightarrow{MB}$, par contre, suffit à montrer que $M$ est le milieu de $\left[AB\right]$.
}
\cadre{vert}{Propriété}{%id="p50"
     Le quadrilatère $\left(ABCD\right)$ est un parallélogramme si et seulement si $\overrightarrow{AB}=\overrightarrow{DC}$.
}
\begin{center}
     \begin{extern}%width="240" alt="vecteurs et parallélogramme"
          \resizebox{5.5cm}{!}{
               \psset{xunit=1.0cm,yunit=1.0cm,algebraic=true,dimen=middle,dotstyle=*,dotsize=3pt 0,linewidth=1pt,arrowsize=3pt 2,arrowinset=0.25}
               \begin{pspicture*}(0.,0.5)(9.2,6.)
                    \begin{Large}
                         \psline[linecolor=red]{->}(1.,3.)(6.,5.)
                         \psline[linecolor=red]{->}(3.,1.)(8.,3.)
                         \psline[linecolor=gray](1.,3.)(3.,1.)
                         \psline[linecolor=gray](6.,5.)(8.,3.)
                         \psdots[dotstyle=*,linecolor=blue](1.,3.)
                         \rput[bl](0.38,2.94){\blue{$A$}}
                         \psdots[dotstyle=*,linecolor=blue](6.,5.)
                         \rput[bl](6.08,5.2){\blue{$B$}}
                         \psdots[dotstyle=*,linecolor=blue](3.,1.)
                         \rput[bl](2.4,0.68){\blue{$D$}}
                         \psdots[dotstyle=*,linecolor=blue](8.,3.)
                         \rput[bl](8.08,3.2){\blue{$C$}}
                    \end{Large}
               \end{pspicture*}
          }
     \end{extern}
\end{center}
\bloc{cyan}{Remarques}{%id="r60"
     \begin{itemize}
          \item Attention à l'inversion des points $C$ et $D$ dans l'égalité $\overrightarrow{AB}=\overrightarrow{DC}$
          \item Avec cette propriété, il suffit de prouver \textbf{une seule égalité} pour montrer qu'un quadrilatère est un parallélogramme. C'est une méthode plus puissante que celles vues en 4ème qui nécessitaient de démontrer deux propriétés (double parallélisme ou parallélisme et égalité de longueurs, etc.)
     \end{itemize}
}
\cadre{bleu}{Définition}{%id="d60"
     La translation de vecteur $\vec{u}$ est la transformation du plan qui à tout point $M$ du plan associe l'unique point $M^{\prime}$ tel que $\overrightarrow{MM^{\prime}}=\vec{u}$
}
\begin{center}
     \begin{extern}%width="200" alt="translation de vecteur u"
          \resizebox{5.5cm}{!}{
               \psset{xunit=1.0cm,yunit=1.0cm,algebraic=true,dimen=middle,dotstyle=*,dotsize=3pt 0,linewidth=1pt,arrowsize=3pt 2,arrowinset=0.25}
               \newrgbcolor{lightblue}{0.8 0.8 1.}
               \begin{pspicture*}(0.,0.5)(8.2,6.)
                    \begin{Large}
                         \psline[linecolor=blue]{->}(1.,3.)(6.,5.)
                         \psline[linecolor=lightblue]{->}(2.,1.)(7.,3.)
                         \rput[bl](3.2,4.2){\blue{$\vect{u}$}}
                         \psdots[dotstyle=*,linecolor=red](2.,1.)
                         \rput[bl](1.3,0.88){\red{$M$}}
                         \psdots[dotstyle=*,linecolor=red](7.,3.)
                         \rput[bl](7.08,3.2){\red{$M'$}}
                    \end{Large}
               \end{pspicture*}
          }
     \end{extern}
\end{center}
\begin{center}\textit{Translation de vecteur $\vec{u}$}\end{center}
\begin{h2}2. Somme de vecteurs\end{h2}
On définit l'addition de deux vecteurs à l'aide de la relation de Chasles:
\cadre{vert}{Propriété}{%id="p70"
     Pour tous points $A$, $B$ et $C$ du plan : $\overrightarrow{AB}+\overrightarrow{BC}=\overrightarrow{AC}$ \textit{(Relation de Chasles)}
}
\begin{center}
     \begin{extern}%width="210" alt="relation de Chasles"
          \resizebox{5.5cm}{!}{
               \psset{xunit=1.0cm,yunit=1.0cm,algebraic=true,dimen=middle,dotstyle=*,dotsize=3pt 0,linewidth=1pt,arrowsize=3pt 2,arrowinset=0.25}
               \begin{pspicture*}(0.,1.5)(8.2,6.)
                    \begin{Large}
                         \psline[linecolor=blue]{->}(1.,3.)(4.,5.)
                         \psline[linecolor=blue]{->}(4,5)(7.,2.)
                         \psline[linecolor=red]{->}(1.,3.)(7.,2.)
                         \psdots[dotstyle=*,linecolor=blue](1.,3.)
                         \rput[bl](0.38,2.94){\blue{$A$}}
                         \psdots[dotstyle=*,linecolor=blue](4.,5.)
                         \rput[bl](4.08,5.2){\blue{$B$}}
                         \psdots[dotstyle=*,linecolor=blue](7.,2.)
                         \rput[bl](7.08,2.2){\blue{$C$}}
                    \end{Large}
               \end{pspicture*}
          }
     \end{extern}
\end{center}
\begin{center}\textit{Relation de Chasles}\end{center}
Pour appliquer la relation de Chasles, il faut que l'extrémité du premier vecteur coïncide avec l'origine du second. Pour additionner deux vecteurs qui ne sont pas dans cette configuration, on "reporte l'un des vecteurs à la suite de l'autre".
\bloc{orange}{Exemple}{%id="e62"
     \begin{center}
          \begin{extern}%width="260" alt="somme de vecteurs"
               \resizebox{5.5cm}{!}{
                    \psset{xunit=1.0cm,yunit=1.0cm,algebraic=true,dimen=middle,dotstyle=*,dotsize=3pt 0,linewidth=1pt,arrowsize=3pt 2,arrowinset=0.25}
                    \begin{pspicture*}(-1,0.5)(9.2,6.)
                         \begin{Large}
                              \psline[linecolor=red]{->}(0,2.)(8.,3.)
                              \psline[linecolor=blue]{->}(1.,3.)(6.,5.)
                              \psline[linecolor=gray]{->}(3.,1.)(8.,3.)
                              \psline[linecolor=blue]{->}(0,2.)(3.,1)
                              \psline[linecolor=lightgray,linestyle=dashed](1.,3.)(3.,1.)
                              \psline[linecolor=lightgray,linestyle=dashed](6.,5.)(8.,3.)
                              \psdots[dotstyle=*,linecolor=blue](1.,3.)
                              \rput[bl](0.38,2.94){\blue{$C$}}
                              \psdots[dotstyle=*,linecolor=blue](6.,5.)
                              \rput[bl](6.08,5.2){\blue{$D$}}
                              \psdots[dotstyle=*,linecolor=blue](0.,2.)
                              \rput[bl](-0.6,1.8){\blue{$A$}}
                              \psdots[dotstyle=*,linecolor=blue](3.,1.)
                              \rput[bl](2.4,0.6){\blue{$B$}}
                              \psdots[dotstyle=*,linecolor=blue](8.,3.)
                              \rput[bl](8.08,3.2){\blue{$E$}}
                         \end{Large}
                    \end{pspicture*}
               }
          \end{extern}
     \end{center}
     Pour tracer la somme des vecteurs $\overrightarrow{AB}$ et $\overrightarrow{CD}$ on reporte le vecteur $\overrightarrow{CD}$ à la suite du vecteur $\overrightarrow{AB}$; cela donne le vecteur $\overrightarrow{BE}$ qui est égal au vecteur $\overrightarrow{CD}.$ \\
     On applique alors la relation de Chasles : $\overrightarrow{AB}+\overrightarrow{BE}=\overrightarrow{AE}$ . La somme cherchée peut donc être représentée par le vecteur $\overrightarrow{AE}.$
}
\bloc{orange}{Cas particulier}{%id="d70"
     \begin{center}
          \begin{extern}%width="240" alt="somme de vecteurs de même origine"
               \resizebox{5.5cm}{!}{
                    \psset{xunit=1.0cm,yunit=1.0cm,algebraic=true,dimen=middle,dotstyle=*,dotsize=3pt 0,linewidth=1pt,arrowsize=3pt 2,arrowinset=0.25}
                    \begin{pspicture*}(-1,0.5)(9.2,6.)
                         \begin{Large}
                              \psline[linecolor=red]{->}(1,3)(8.,3.)
                              \psline[linecolor=blue]{->}(1.,3.)(6.,5.)
                              \psline[linecolor=gray]{->}(3.,1.)(8.,3.)
                              \psline[linecolor=blue]{->}(1.,3.)(3.,1)
                              \psline[linecolor=lightgray,linestyle=dashed](6.,5.)(8.,3.)
                              \psdots[dotstyle=*,linecolor=blue](1.,3.)
                              \rput[bl](0.38,2.94){\blue{$A$}}
                              \psdots[dotstyle=*,linecolor=blue](6.,5.)
                              \rput[bl](6.08,5.2){\blue{$B$}}
                              \psdots[dotstyle=*,linecolor=blue](3.,1.)
                              \rput[bl](2.4,0.6){\blue{$C$}}
                              \psdots[dotstyle=*,linecolor=blue](8.,3.)
                              \rput[bl](8.08,3.2){\blue{$D$}}
                         \end{Large}
                    \end{pspicture*}
               }
          \end{extern}
     \end{center}
     Si les vecteurs à additionner, ont la même origine, la méthode précédente aboutit à la construction d'un parallélogramme $\left(ABDC\right)$ :
     \par
     $\overrightarrow{AB}+\overrightarrow{AC}=\overrightarrow{AB}+\overrightarrow{BD}=\overrightarrow{AD}$
}
\cadre{bleu}{Propriété et définition}{%id="p80"
     Pour tout point $A$ et $B$ du plan : $\overrightarrow{AB}+\overrightarrow{BA}=\overrightarrow{AA}=\overrightarrow{0}$
     \par
     On dit que les vecteurs $\overrightarrow{AB}$ et $\overrightarrow{BA}$ sont \textbf{opposés} et l'on écrit  $\overrightarrow{AB}=-\overrightarrow{BA}$
}
\bloc{cyan}{Remarque}{%id="r80"
     Deux vecteurs opposés ont la même direction, la même longueur et des sens contraires.
}
\bloc{cyan}{Conséquence}{%id="c80"
     On peut donc définir la différence de 2 vecteurs par :
     \par
     $\overrightarrow{AB}-\overrightarrow{CD}=\overrightarrow{AB}+\overrightarrow{DC}$
}
\begin{h2}3. Produit d'un vecteur par un nombre réel\end{h2}
\cadre{bleu}{Définition}{%id="d90"
     Soit $\vec{u}$ un vecteur du plan et soit $k$ un nombre réel.
     \par
     On définit le vecteur $k\vec{u}$ de la manière suivante :
     \par
     Si $k$ est \textbf{strictement positif} :
     \begin{itemize}
          \item Les vecteurs $\vec{u}$ et $k\vec{u}$ ont la même direction
          \item Les vecteurs $\vec{u}$ et $k\vec{u}$ ont le même sens
          \item La norme de $k\vec{u}$ est $k ||\vec{u}||$
     \end{itemize}
     Si $k$ est \textbf{strictement négatif} :
     \begin{itemize}
          \item Les vecteurs $\vec{u}$ et $k\vec{u}$ ont la même direction
          \item Les vecteurs $\vec{u}$ et $k\vec{u}$ ont des sens opposés
          \item La norme de $k\vec{u}$ est $-k ||\vec{u}||$
     \end{itemize}
     Si $k$ est \textbf{nul} : $k\vec{u} = 0\vec{u}$ est le vecteur nul
}
\bloc{orange}{Exemple}{%id="e100"
     \begin{center}
          \begin{extern}%width="220" alt="Vecteurs colinéaires"
               \psset{xunit=0.5cm,yunit=0.5cm,algebraic=true,dimen=middle,linewidth=0.8pt}
               \begin{pspicture*}(-1,0.)(11,8)
                    \psline[linecolor=blue]{->}(3,3)(6,4)
                    \psline[linecolor=red]{->}(0,4)(9,7)
                    \psline[linecolor=mcvert]{->}(10,3)(4,1)
                    \rput[b](4.5,5.8){$\red 3\vec{u}$}
                    \rput[b](4.5,3.8){$\blue \vec{u}$}
                    \rput[b](7,2.3){$\color{mcvert}-2\vec{u}$}
               \end{pspicture*}
          \end{extern}
     \end{center}
     \begin{center}
          \textit{Vecteurs colinéaires}
     \end{center}
}
\cadre{bleu}{Définition}{%id="d110"
     On dit que deux vecteurs $\vec{u}$ et $\vec{v}$ sont \textbf{colinéaires} s'il existe un réel $k$ tel que $\vec{u} = k\vec{v}$ ou un réel $k^{\prime}$ tel que $\vec{v} = k^{\prime}\vec{u}$
}
\bloc{cyan}{Remarques}{%id="r110"
     \begin{itemize}
          \item Deux vecteurs non nuls sont colinéaires si et seulement si ils ont la même direction (mais ils peuvent avoir des sens opposés)
          \item Le vecteur nul est colinéaire à tout vecteur. En effet $\overrightarrow{0} = 0\vec{u}$
     \end{itemize}
}
\cadre{vert}{Propriétés}{%id="p120"
     Pour tous vecteurs $\vec{u}$ et $\vec{v}$ du plan et tous réels $k$ et $k^{\prime}$ :
     \begin{itemize}
          \item $k \left(\vec{u}+\vec{v}\right) = k\vec{u}+k\vec{v}$
          \item $\left(k+k^{\prime}\right) \vec{u} = k\vec{u}+k^{\prime}\vec{u}$
          \item $k \left(k^{\prime}\vec{u}\right) = \left(kk^{\prime}\right) \vec{u}$
     \end{itemize}
}
\bloc{orange}{Exemple}{%id="e130"
     $2 \left(\overrightarrow{AB}+3\overrightarrow{AC}\right) = 2\overrightarrow{AB} + 2 \left(3\overrightarrow{AC}\right) = 2\overrightarrow{AB} + 6\overrightarrow{AC}$
}

\end{document}

µ
\documentclass[a4paper]{article}

%================================================================================================================================
%
% Packages
%
%================================================================================================================================

\usepackage[T1]{fontenc} 	% pour caractères accentués
\usepackage[utf8]{inputenc}  % encodage utf8
\usepackage[french]{babel}	% langue : français
\usepackage{fourier}			% caractères plus lisibles
\usepackage[dvipsnames]{xcolor} % couleurs
\usepackage{fancyhdr}		% réglage header footer
\usepackage{needspace}		% empêcher sauts de page mal placés
\usepackage{graphicx}		% pour inclure des graphiques
\usepackage{enumitem,cprotect}		% personnalise les listes d'items (nécessaire pour ol, al ...)
\usepackage{hyperref}		% Liens hypertexte
\usepackage{pstricks,pst-all,pst-node,pstricks-add,pst-math,pst-plot,pst-tree,pst-eucl} % pstricks
\usepackage[a4paper,includeheadfoot,top=2cm,left=3cm, bottom=2cm,right=3cm]{geometry} % marges etc.
\usepackage{comment}			% commentaires multilignes
\usepackage{amsmath,environ} % maths (matrices, etc.)
\usepackage{amssymb,makeidx}
\usepackage{bm}				% bold maths
\usepackage{tabularx}		% tableaux
\usepackage{colortbl}		% tableaux en couleur
\usepackage{fontawesome}		% Fontawesome
\usepackage{environ}			% environment with command
\usepackage{fp}				% calculs pour ps-tricks
\usepackage{multido}			% pour ps tricks
\usepackage[np]{numprint}	% formattage nombre
\usepackage{tikz,tkz-tab} 			% package principal TikZ
\usepackage{pgfplots}   % axes
\usepackage{mathrsfs}    % cursives
\usepackage{calc}			% calcul taille boites
\usepackage[scaled=0.875]{helvet} % font sans serif
\usepackage{svg} % svg
\usepackage{scrextend} % local margin
\usepackage{scratch} %scratch
\usepackage{multicol} % colonnes
%\usepackage{infix-RPN,pst-func} % formule en notation polanaise inversée
\usepackage{listings}

%================================================================================================================================
%
% Réglages de base
%
%================================================================================================================================

\lstset{
language=Python,   % R code
literate=
{á}{{\'a}}1
{à}{{\`a}}1
{ã}{{\~a}}1
{é}{{\'e}}1
{è}{{\`e}}1
{ê}{{\^e}}1
{í}{{\'i}}1
{ó}{{\'o}}1
{õ}{{\~o}}1
{ú}{{\'u}}1
{ü}{{\"u}}1
{ç}{{\c{c}}}1
{~}{{ }}1
}


\definecolor{codegreen}{rgb}{0,0.6,0}
\definecolor{codegray}{rgb}{0.5,0.5,0.5}
\definecolor{codepurple}{rgb}{0.58,0,0.82}
\definecolor{backcolour}{rgb}{0.95,0.95,0.92}

\lstdefinestyle{mystyle}{
    backgroundcolor=\color{backcolour},   
    commentstyle=\color{codegreen},
    keywordstyle=\color{magenta},
    numberstyle=\tiny\color{codegray},
    stringstyle=\color{codepurple},
    basicstyle=\ttfamily\footnotesize,
    breakatwhitespace=false,         
    breaklines=true,                 
    captionpos=b,                    
    keepspaces=true,                 
    numbers=left,                    
xleftmargin=2em,
framexleftmargin=2em,            
    showspaces=false,                
    showstringspaces=false,
    showtabs=false,                  
    tabsize=2,
    upquote=true
}

\lstset{style=mystyle}


\lstset{style=mystyle}
\newcommand{\imgdir}{C:/laragon/www/newmc/assets/imgsvg/}
\newcommand{\imgsvgdir}{C:/laragon/www/newmc/assets/imgsvg/}

\definecolor{mcgris}{RGB}{220, 220, 220}% ancien~; pour compatibilité
\definecolor{mcbleu}{RGB}{52, 152, 219}
\definecolor{mcvert}{RGB}{125, 194, 70}
\definecolor{mcmauve}{RGB}{154, 0, 215}
\definecolor{mcorange}{RGB}{255, 96, 0}
\definecolor{mcturquoise}{RGB}{0, 153, 153}
\definecolor{mcrouge}{RGB}{255, 0, 0}
\definecolor{mclightvert}{RGB}{205, 234, 190}

\definecolor{gris}{RGB}{220, 220, 220}
\definecolor{bleu}{RGB}{52, 152, 219}
\definecolor{vert}{RGB}{125, 194, 70}
\definecolor{mauve}{RGB}{154, 0, 215}
\definecolor{orange}{RGB}{255, 96, 0}
\definecolor{turquoise}{RGB}{0, 153, 153}
\definecolor{rouge}{RGB}{255, 0, 0}
\definecolor{lightvert}{RGB}{205, 234, 190}
\setitemize[0]{label=\color{lightvert}  $\bullet$}

\pagestyle{fancy}
\renewcommand{\headrulewidth}{0.2pt}
\fancyhead[L]{maths-cours.fr}
\fancyhead[R]{\thepage}
\renewcommand{\footrulewidth}{0.2pt}
\fancyfoot[C]{}

\newcolumntype{C}{>{\centering\arraybackslash}X}
\newcolumntype{s}{>{\hsize=.35\hsize\arraybackslash}X}

\setlength{\parindent}{0pt}		 
\setlength{\parskip}{3mm}
\setlength{\headheight}{1cm}

\def\ebook{ebook}
\def\book{book}
\def\web{web}
\def\type{web}

\newcommand{\vect}[1]{\overrightarrow{\,\mathstrut#1\,}}

\def\Oij{$\left(\text{O}~;~\vect{\imath},~\vect{\jmath}\right)$}
\def\Oijk{$\left(\text{O}~;~\vect{\imath},~\vect{\jmath},~\vect{k}\right)$}
\def\Ouv{$\left(\text{O}~;~\vect{u},~\vect{v}\right)$}

\hypersetup{breaklinks=true, colorlinks = true, linkcolor = OliveGreen, urlcolor = OliveGreen, citecolor = OliveGreen, pdfauthor={Didier BONNEL - https://www.maths-cours.fr} } % supprime les bordures autour des liens

\renewcommand{\arg}[0]{\text{arg}}

\everymath{\displaystyle}

%================================================================================================================================
%
% Macros - Commandes
%
%================================================================================================================================

\newcommand\meta[2]{    			% Utilisé pour créer le post HTML.
	\def\titre{titre}
	\def\url{url}
	\def\arg{#1}
	\ifx\titre\arg
		\newcommand\maintitle{#2}
		\fancyhead[L]{#2}
		{\Large\sffamily \MakeUppercase{#2}}
		\vspace{1mm}\textcolor{mcvert}{\hrule}
	\fi 
	\ifx\url\arg
		\fancyfoot[L]{\href{https://www.maths-cours.fr#2}{\black \footnotesize{https://www.maths-cours.fr#2}}}
	\fi 
}


\newcommand\TitreC[1]{    		% Titre centré
     \needspace{3\baselineskip}
     \begin{center}\textbf{#1}\end{center}
}

\newcommand\newpar{    		% paragraphe
     \par
}

\newcommand\nosp {    		% commande vide (pas d'espace)
}
\newcommand{\id}[1]{} %ignore

\newcommand\boite[2]{				% Boite simple sans titre
	\vspace{5mm}
	\setlength{\fboxrule}{0.2mm}
	\setlength{\fboxsep}{5mm}	
	\fcolorbox{#1}{#1!3}{\makebox[\linewidth-2\fboxrule-2\fboxsep]{
  		\begin{minipage}[t]{\linewidth-2\fboxrule-4\fboxsep}\setlength{\parskip}{3mm}
  			 #2
  		\end{minipage}
	}}
	\vspace{5mm}
}

\newcommand\CBox[4]{				% Boites
	\vspace{5mm}
	\setlength{\fboxrule}{0.2mm}
	\setlength{\fboxsep}{5mm}
	
	\fcolorbox{#1}{#1!3}{\makebox[\linewidth-2\fboxrule-2\fboxsep]{
		\begin{minipage}[t]{1cm}\setlength{\parskip}{3mm}
	  		\textcolor{#1}{\LARGE{#2}}    
 	 	\end{minipage}  
  		\begin{minipage}[t]{\linewidth-2\fboxrule-4\fboxsep}\setlength{\parskip}{3mm}
			\raisebox{1.2mm}{\normalsize\sffamily{\textcolor{#1}{#3}}}						
  			 #4
  		\end{minipage}
	}}
	\vspace{5mm}
}

\newcommand\cadre[3]{				% Boites convertible html
	\par
	\vspace{2mm}
	\setlength{\fboxrule}{0.1mm}
	\setlength{\fboxsep}{5mm}
	\fcolorbox{#1}{white}{\makebox[\linewidth-2\fboxrule-2\fboxsep]{
  		\begin{minipage}[t]{\linewidth-2\fboxrule-4\fboxsep}\setlength{\parskip}{3mm}
			\raisebox{-2.5mm}{\sffamily \small{\textcolor{#1}{\MakeUppercase{#2}}}}		
			\par		
  			 #3
 	 		\end{minipage}
	}}
		\vspace{2mm}
	\par
}

\newcommand\bloc[3]{				% Boites convertible html sans bordure
     \needspace{2\baselineskip}
     {\sffamily \small{\textcolor{#1}{\MakeUppercase{#2}}}}    
		\par		
  			 #3
		\par
}

\newcommand\CHelp[1]{
     \CBox{Plum}{\faInfoCircle}{À RETENIR}{#1}
}

\newcommand\CUp[1]{
     \CBox{NavyBlue}{\faThumbsOUp}{EN PRATIQUE}{#1}
}

\newcommand\CInfo[1]{
     \CBox{Sepia}{\faArrowCircleRight}{REMARQUE}{#1}
}

\newcommand\CRedac[1]{
     \CBox{PineGreen}{\faEdit}{BIEN R\'EDIGER}{#1}
}

\newcommand\CError[1]{
     \CBox{Red}{\faExclamationTriangle}{ATTENTION}{#1}
}

\newcommand\TitreExo[2]{
\needspace{4\baselineskip}
 {\sffamily\large EXERCICE #1\ (\emph{#2 points})}
\vspace{5mm}
}

\newcommand\img[2]{
          \includegraphics[width=#2\paperwidth]{\imgdir#1}
}

\newcommand\imgsvg[2]{
       \begin{center}   \includegraphics[width=#2\paperwidth]{\imgsvgdir#1} \end{center}
}


\newcommand\Lien[2]{
     \href{#1}{#2 \tiny \faExternalLink}
}
\newcommand\mcLien[2]{
     \href{https~://www.maths-cours.fr/#1}{#2 \tiny \faExternalLink}
}

\newcommand{\euro}{\eurologo{}}

%================================================================================================================================
%
% Macros - Environement
%
%================================================================================================================================

\newenvironment{tex}{ %
}
{%
}

\newenvironment{indente}{ %
	\setlength\parindent{10mm}
}

{
	\setlength\parindent{0mm}
}

\newenvironment{corrige}{%
     \needspace{3\baselineskip}
     \medskip
     \textbf{\textsc{Corrigé}}
     \medskip
}
{
}

\newenvironment{extern}{%
     \begin{center}
     }
     {
     \end{center}
}

\NewEnviron{code}{%
	\par
     \boite{gray}{\texttt{%
     \BODY
     }}
     \par
}

\newenvironment{vbloc}{% boite sans cadre empeche saut de page
     \begin{minipage}[t]{\linewidth}
     }
     {
     \end{minipage}
}
\NewEnviron{h2}{%
    \needspace{3\baselineskip}
    \vspace{0.6cm}
	\noindent \MakeUppercase{\sffamily \large \BODY}
	\vspace{1mm}\textcolor{mcgris}{\hrule}\vspace{0.4cm}
	\par
}{}

\NewEnviron{h3}{%
    \needspace{3\baselineskip}
	\vspace{5mm}
	\textsc{\BODY}
	\par
}

\NewEnviron{margeneg}{ %
\begin{addmargin}[-1cm]{0cm}
\BODY
\end{addmargin}
}

\NewEnviron{html}{%
}

\begin{document}
\meta{url}{/cours/vecteurs-coordonnees/}
\meta{pid}{200}
\meta{titre}{Vecteurs et coordonnées}
\meta{type}{cours}
\cadre{bleu}{Définitions}{%
     Un \textbf{repère} du plan est déterminé par un point quelconque O, appelé \textbf{origine} du repère, et deux vecteurs $\vec{i}$ et $\vec{j}$ non colinéaires.
}
\cadre{bleu}{Définitions}{%
     On dit que le repère $\left(O;\vec{i},\vec{j}\right)$ est :
     \begin{itemize}
          \item \textbf{orthogonal} : si les vecteurs $\vec{i}$ et $\vec{j}$ sont orthogonaux
          \item \textbf{orthonormé} ou \textbf{orthonormal} : si le repère est orthogonal et si les vecteurs $\vec{i}$ et $\vec{j}$ ont la même norme.
     \end{itemize}
}
\begin{center}
     \begin{extern}%width="500" alt="repère orthonormé"
          \resizebox{10cm}{!}{%
               % -+-+-+ variables modifiables
               \def\xmin{-4.5}
               \def\xmax{4.5}
               \def\ymin{-3.5}
               \def\ymax{3.5}
               \def\xunit{1.5}  % unités en cm
               \def\yunit{1.5}
               \psset{xunit=\xunit,yunit=\yunit,algebraic=true}
               \fontsize{15pt}{15pt}\selectfont
               \begin{pspicture*}[linewidth=1pt](\xmin,\ymin)(\xmax,\ymax)
                    \psgrid[gridcolor=mcgris, subgriddiv=1, gridlabels=0pt](-5,-4)(5,4)
                    \psaxes[linewidth=0.75pt,Dx=1,Dy=1]{->}(0,0)(\xmin,\ymin)(\xmax,\ymax)
                    \rput[tr](-0.2,-0.2){$\blue O$}
                    \psline[linewidth=0.75pt,linecolor=blue,arrowsize=6pt]{->}(0,0)(0,1)
                    \psline[linewidth=0.75pt,linecolor=blue,arrowsize=6pt]{->}(0,0)(1,0)
                    \rput[t](0.5,-0.1){$\blue \vec{i}$}
                    \rput[r](-0.1,0.5){$\blue \vec{j}$}
               \end{pspicture*}
          }
     \end{extern}
\end{center}
\begin{center}
     \textit{Repère orthonormé}
\end{center}
\cadre{bleu}{Définitions}{%
     Soit $\left(O;\vec{i},\vec{j}\right)$ un repère du plan.
     \par
     On dit que $M$ a pour \textbf{coordonnées} $\left(x ; y\right)$ si et seulement si :
     \begin{center}
          $\overrightarrow{OM}=x\vec{i}+y\vec{j}$
     \end{center}
     On dit que $\vec{u}$ a pour \textbf{coordonnées} $\begin{pmatrix} x \\ y \end{pmatrix}$ si et seulement si :
     \begin{center}
          $\vec{u}=x\vec{i}+y\vec{j}$
\end{center}}
Par la suite, on considère que le plan P est muni d'un repère $\left(O;\vec{i},\vec{j}\right)$.
\cadre{vert}{Propriété}{%
     Deux vecteurs $\vec{u}$ et $\vec{v}$ sont \textbf{égaux} si et seulement si ils ont les \textbf{mêmes coordonnées}.
}
\cadre{vert}{Propriété}{%
     Soient $A\left(x_A;y_A\right)$ et $B\left(x_B;y_B\right)$. Le vecteur $\overrightarrow{AB}$ a pour coordonnées $\begin{pmatrix} x_B-x_A \\ y_B-y_A \end{pmatrix}$
}
\bloc{orange}{Exemple}{%
     Soient $A\left(1 ; 1\right), B\left(4 ; 2\right), C\left(5 ; 0\right), D\left(2 ; -1\right)$
     \par
     Les coordonnées de $\overrightarrow{AB}$ sont $\begin{pmatrix} 4-1 \\ 2-1 \end{pmatrix} = \begin{pmatrix} 3 \\ 1 \end{pmatrix}$
     \par
     Les coordonnées de $\overrightarrow{DC}$ sont $\begin{pmatrix} 5-2 \\ 0-\left(-1\right) \end{pmatrix} = \begin{pmatrix} 3 \\ 1 \end{pmatrix}$
     \par
     $\overrightarrow{AB} = \overrightarrow{DC}$ donc $ABCD$ est un parallélogramme. ( voir \mcLien{/seconde/vecteurs}{Généralités sur les vecteurs} )
}
\cadre{vert}{Propriétés}{%
     Soient deux vecteurs $\vec{u} \begin{pmatrix} x \\ y \end{pmatrix}$ et $\vec{v} \begin{pmatrix} x^{\prime} \\ y^{\prime} \end{pmatrix}$.
     \begin{itemize}
          \item Le vecteur $\vec{u}+\vec{v}$ a pour coordonnées $\begin{pmatrix} x+x^{\prime} \\ y+y^{\prime} \end{pmatrix}$
          \item Le vecteur $k\vec{u}$ a pour coordonnées $\begin{pmatrix} kx \\ ky \end{pmatrix}$
     \end{itemize}
}
\cadre{vert}{Propriété}{%
     \textbf{Colinéarité}
     \par
     Deux vecteurs non nuls $\vec{u} \begin{pmatrix} x \\ y \end{pmatrix}$ et $\vec{v} \begin{pmatrix} x^{\prime} \\ y^{\prime} \end{pmatrix}$ sont colinéaires si et seulement si:
     \par
     $xy^{\prime}-yx^{\prime}=0$
}
\cadre{vert}{Propriété}{%
     \textbf{Milieu d'un segment}
     \par
     Si $A\left(x_A;y_A\right)$ et $B\left(x_B;y_B\right)$, le milieu $M$ de $\left[AB\right]$ a pour coordonnées :
     \par
     $M \left(\frac{x_A+x_B}{2} ; \frac{y_A+y_B}{2}\right)$
}
\cadre{vert}{Propriété}{%
     \textbf{Norme et distance}
     \par
     Soit un vecteur $\vec{u} \begin{pmatrix} x \\ y \end{pmatrix}$. Alors :
     \par
     $||\vec{u}||=\sqrt{x^2+y^2}$
     \par
     On en déduit si $A\left(x_A;y_A\right)$ et $B\left(x_B;y_B\right)$ :
     \par
     $AB=||\overrightarrow{AB}||=\sqrt{\left(x_B-x_A\right)^2+\left(y_B-y_A\right)^2}$
}
\bloc{orange}{Exemple}{%
     Soient $A\left(1 ; 0\right), B\left(3 ; 1\right), C\left(0 ; 2\right)$. Que peut-on dire du triangle $ABC$ ?
     \par
     $AB=\sqrt{2^2+1^2}=\sqrt{5}$
     \par
     $AC=\sqrt{\left(-1\right)^2+2^2}=\sqrt{5}$
     \par
     $BC=\sqrt{\left(-3\right)^2+1^2}=\sqrt{10}$
     \par
     Donc $AB=AC$
     \par
     De plus :
     \par
     $AB^2+AC^2=5+5=10$
     \par
     $BC^2=10$
     \par
     Le triangle $ABC$ est donc rectangle en $A$ (réciproque du théorème de Pythagore) et isocèle.
}

\end{document}
µ
\documentclass[a4paper]{article}

%================================================================================================================================
%
% Packages
%
%================================================================================================================================

\usepackage[T1]{fontenc} 	% pour caractères accentués
\usepackage[utf8]{inputenc}  % encodage utf8
\usepackage[french]{babel}	% langue : français
\usepackage{fourier}			% caractères plus lisibles
\usepackage[dvipsnames]{xcolor} % couleurs
\usepackage{fancyhdr}		% réglage header footer
\usepackage{needspace}		% empêcher sauts de page mal placés
\usepackage{graphicx}		% pour inclure des graphiques
\usepackage{enumitem,cprotect}		% personnalise les listes d'items (nécessaire pour ol, al ...)
\usepackage{hyperref}		% Liens hypertexte
\usepackage{pstricks,pst-all,pst-node,pstricks-add,pst-math,pst-plot,pst-tree,pst-eucl} % pstricks
\usepackage[a4paper,includeheadfoot,top=2cm,left=3cm, bottom=2cm,right=3cm]{geometry} % marges etc.
\usepackage{comment}			% commentaires multilignes
\usepackage{amsmath,environ} % maths (matrices, etc.)
\usepackage{amssymb,makeidx}
\usepackage{bm}				% bold maths
\usepackage{tabularx}		% tableaux
\usepackage{colortbl}		% tableaux en couleur
\usepackage{fontawesome}		% Fontawesome
\usepackage{environ}			% environment with command
\usepackage{fp}				% calculs pour ps-tricks
\usepackage{multido}			% pour ps tricks
\usepackage[np]{numprint}	% formattage nombre
\usepackage{tikz,tkz-tab} 			% package principal TikZ
\usepackage{pgfplots}   % axes
\usepackage{mathrsfs}    % cursives
\usepackage{calc}			% calcul taille boites
\usepackage[scaled=0.875]{helvet} % font sans serif
\usepackage{svg} % svg
\usepackage{scrextend} % local margin
\usepackage{scratch} %scratch
\usepackage{multicol} % colonnes
%\usepackage{infix-RPN,pst-func} % formule en notation polanaise inversée
\usepackage{listings}

%================================================================================================================================
%
% Réglages de base
%
%================================================================================================================================

\lstset{
language=Python,   % R code
literate=
{á}{{\'a}}1
{à}{{\`a}}1
{ã}{{\~a}}1
{é}{{\'e}}1
{è}{{\`e}}1
{ê}{{\^e}}1
{í}{{\'i}}1
{ó}{{\'o}}1
{õ}{{\~o}}1
{ú}{{\'u}}1
{ü}{{\"u}}1
{ç}{{\c{c}}}1
{~}{{ }}1
}


\definecolor{codegreen}{rgb}{0,0.6,0}
\definecolor{codegray}{rgb}{0.5,0.5,0.5}
\definecolor{codepurple}{rgb}{0.58,0,0.82}
\definecolor{backcolour}{rgb}{0.95,0.95,0.92}

\lstdefinestyle{mystyle}{
    backgroundcolor=\color{backcolour},   
    commentstyle=\color{codegreen},
    keywordstyle=\color{magenta},
    numberstyle=\tiny\color{codegray},
    stringstyle=\color{codepurple},
    basicstyle=\ttfamily\footnotesize,
    breakatwhitespace=false,         
    breaklines=true,                 
    captionpos=b,                    
    keepspaces=true,                 
    numbers=left,                    
xleftmargin=2em,
framexleftmargin=2em,            
    showspaces=false,                
    showstringspaces=false,
    showtabs=false,                  
    tabsize=2,
    upquote=true
}

\lstset{style=mystyle}


\lstset{style=mystyle}
\newcommand{\imgdir}{C:/laragon/www/newmc/assets/imgsvg/}
\newcommand{\imgsvgdir}{C:/laragon/www/newmc/assets/imgsvg/}

\definecolor{mcgris}{RGB}{220, 220, 220}% ancien~; pour compatibilité
\definecolor{mcbleu}{RGB}{52, 152, 219}
\definecolor{mcvert}{RGB}{125, 194, 70}
\definecolor{mcmauve}{RGB}{154, 0, 215}
\definecolor{mcorange}{RGB}{255, 96, 0}
\definecolor{mcturquoise}{RGB}{0, 153, 153}
\definecolor{mcrouge}{RGB}{255, 0, 0}
\definecolor{mclightvert}{RGB}{205, 234, 190}

\definecolor{gris}{RGB}{220, 220, 220}
\definecolor{bleu}{RGB}{52, 152, 219}
\definecolor{vert}{RGB}{125, 194, 70}
\definecolor{mauve}{RGB}{154, 0, 215}
\definecolor{orange}{RGB}{255, 96, 0}
\definecolor{turquoise}{RGB}{0, 153, 153}
\definecolor{rouge}{RGB}{255, 0, 0}
\definecolor{lightvert}{RGB}{205, 234, 190}
\setitemize[0]{label=\color{lightvert}  $\bullet$}

\pagestyle{fancy}
\renewcommand{\headrulewidth}{0.2pt}
\fancyhead[L]{maths-cours.fr}
\fancyhead[R]{\thepage}
\renewcommand{\footrulewidth}{0.2pt}
\fancyfoot[C]{}

\newcolumntype{C}{>{\centering\arraybackslash}X}
\newcolumntype{s}{>{\hsize=.35\hsize\arraybackslash}X}

\setlength{\parindent}{0pt}		 
\setlength{\parskip}{3mm}
\setlength{\headheight}{1cm}

\def\ebook{ebook}
\def\book{book}
\def\web{web}
\def\type{web}

\newcommand{\vect}[1]{\overrightarrow{\,\mathstrut#1\,}}

\def\Oij{$\left(\text{O}~;~\vect{\imath},~\vect{\jmath}\right)$}
\def\Oijk{$\left(\text{O}~;~\vect{\imath},~\vect{\jmath},~\vect{k}\right)$}
\def\Ouv{$\left(\text{O}~;~\vect{u},~\vect{v}\right)$}

\hypersetup{breaklinks=true, colorlinks = true, linkcolor = OliveGreen, urlcolor = OliveGreen, citecolor = OliveGreen, pdfauthor={Didier BONNEL - https://www.maths-cours.fr} } % supprime les bordures autour des liens

\renewcommand{\arg}[0]{\text{arg}}

\everymath{\displaystyle}

%================================================================================================================================
%
% Macros - Commandes
%
%================================================================================================================================

\newcommand\meta[2]{    			% Utilisé pour créer le post HTML.
	\def\titre{titre}
	\def\url{url}
	\def\arg{#1}
	\ifx\titre\arg
		\newcommand\maintitle{#2}
		\fancyhead[L]{#2}
		{\Large\sffamily \MakeUppercase{#2}}
		\vspace{1mm}\textcolor{mcvert}{\hrule}
	\fi 
	\ifx\url\arg
		\fancyfoot[L]{\href{https://www.maths-cours.fr#2}{\black \footnotesize{https://www.maths-cours.fr#2}}}
	\fi 
}


\newcommand\TitreC[1]{    		% Titre centré
     \needspace{3\baselineskip}
     \begin{center}\textbf{#1}\end{center}
}

\newcommand\newpar{    		% paragraphe
     \par
}

\newcommand\nosp {    		% commande vide (pas d'espace)
}
\newcommand{\id}[1]{} %ignore

\newcommand\boite[2]{				% Boite simple sans titre
	\vspace{5mm}
	\setlength{\fboxrule}{0.2mm}
	\setlength{\fboxsep}{5mm}	
	\fcolorbox{#1}{#1!3}{\makebox[\linewidth-2\fboxrule-2\fboxsep]{
  		\begin{minipage}[t]{\linewidth-2\fboxrule-4\fboxsep}\setlength{\parskip}{3mm}
  			 #2
  		\end{minipage}
	}}
	\vspace{5mm}
}

\newcommand\CBox[4]{				% Boites
	\vspace{5mm}
	\setlength{\fboxrule}{0.2mm}
	\setlength{\fboxsep}{5mm}
	
	\fcolorbox{#1}{#1!3}{\makebox[\linewidth-2\fboxrule-2\fboxsep]{
		\begin{minipage}[t]{1cm}\setlength{\parskip}{3mm}
	  		\textcolor{#1}{\LARGE{#2}}    
 	 	\end{minipage}  
  		\begin{minipage}[t]{\linewidth-2\fboxrule-4\fboxsep}\setlength{\parskip}{3mm}
			\raisebox{1.2mm}{\normalsize\sffamily{\textcolor{#1}{#3}}}						
  			 #4
  		\end{minipage}
	}}
	\vspace{5mm}
}

\newcommand\cadre[3]{				% Boites convertible html
	\par
	\vspace{2mm}
	\setlength{\fboxrule}{0.1mm}
	\setlength{\fboxsep}{5mm}
	\fcolorbox{#1}{white}{\makebox[\linewidth-2\fboxrule-2\fboxsep]{
  		\begin{minipage}[t]{\linewidth-2\fboxrule-4\fboxsep}\setlength{\parskip}{3mm}
			\raisebox{-2.5mm}{\sffamily \small{\textcolor{#1}{\MakeUppercase{#2}}}}		
			\par		
  			 #3
 	 		\end{minipage}
	}}
		\vspace{2mm}
	\par
}

\newcommand\bloc[3]{				% Boites convertible html sans bordure
     \needspace{2\baselineskip}
     {\sffamily \small{\textcolor{#1}{\MakeUppercase{#2}}}}    
		\par		
  			 #3
		\par
}

\newcommand\CHelp[1]{
     \CBox{Plum}{\faInfoCircle}{À RETENIR}{#1}
}

\newcommand\CUp[1]{
     \CBox{NavyBlue}{\faThumbsOUp}{EN PRATIQUE}{#1}
}

\newcommand\CInfo[1]{
     \CBox{Sepia}{\faArrowCircleRight}{REMARQUE}{#1}
}

\newcommand\CRedac[1]{
     \CBox{PineGreen}{\faEdit}{BIEN R\'EDIGER}{#1}
}

\newcommand\CError[1]{
     \CBox{Red}{\faExclamationTriangle}{ATTENTION}{#1}
}

\newcommand\TitreExo[2]{
\needspace{4\baselineskip}
 {\sffamily\large EXERCICE #1\ (\emph{#2 points})}
\vspace{5mm}
}

\newcommand\img[2]{
          \includegraphics[width=#2\paperwidth]{\imgdir#1}
}

\newcommand\imgsvg[2]{
       \begin{center}   \includegraphics[width=#2\paperwidth]{\imgsvgdir#1} \end{center}
}


\newcommand\Lien[2]{
     \href{#1}{#2 \tiny \faExternalLink}
}
\newcommand\mcLien[2]{
     \href{https~://www.maths-cours.fr/#1}{#2 \tiny \faExternalLink}
}

\newcommand{\euro}{\eurologo{}}

%================================================================================================================================
%
% Macros - Environement
%
%================================================================================================================================

\newenvironment{tex}{ %
}
{%
}

\newenvironment{indente}{ %
	\setlength\parindent{10mm}
}

{
	\setlength\parindent{0mm}
}

\newenvironment{corrige}{%
     \needspace{3\baselineskip}
     \medskip
     \textbf{\textsc{Corrigé}}
     \medskip
}
{
}

\newenvironment{extern}{%
     \begin{center}
     }
     {
     \end{center}
}

\NewEnviron{code}{%
	\par
     \boite{gray}{\texttt{%
     \BODY
     }}
     \par
}

\newenvironment{vbloc}{% boite sans cadre empeche saut de page
     \begin{minipage}[t]{\linewidth}
     }
     {
     \end{minipage}
}
\NewEnviron{h2}{%
    \needspace{3\baselineskip}
    \vspace{0.6cm}
	\noindent \MakeUppercase{\sffamily \large \BODY}
	\vspace{1mm}\textcolor{mcgris}{\hrule}\vspace{0.4cm}
	\par
}{}

\NewEnviron{h3}{%
    \needspace{3\baselineskip}
	\vspace{5mm}
	\textsc{\BODY}
	\par
}

\NewEnviron{margeneg}{ %
\begin{addmargin}[-1cm]{0cm}
\BODY
\end{addmargin}
}

\NewEnviron{html}{%
}

\begin{document}
\meta{url}{/cours/trigonometrie/}
\meta{pid}{204}
\meta{titre}{Trigonométrie}
\meta{type}{cours}
\begin{h2}1. Angle dans le cercle trigonométrique\end{h2}
Dans tout le chapitre, le plan $P$ est muni d'un repère orthonormé $\left(O ; I , J\right)$
\cadre{bleu}{Définition}{%id="d10"
     On appelle \textbf{cercle trigonométrique} le cercle de centre $O$ et de rayon $1$ orienté dans le sens inverse des aiguilles d'une montre (aussi appelé \textit{« sens direct »} ou \textit{« sens trigonométrique»}).
}
\begin{center}
     \begin{extern}%width="250" alt="cercle trigonométrique"
          \resizebox{6cm}{!}{
               \psset{xunit=3.0cm,yunit=3.0cm,algebraic=true,dimen=middle,dotstyle=o,dotsize=5pt 0,linewidth=0.8pt,arrowsize=3pt 2,arrowinset=0.25}
               \begin{pspicture*}(-1.2,-1.2)(1.2,1.2)
                    \psaxes[linewidth=0.75pt,xAxis=true,yAxis=true,Dx=1.,Dy=1.,ticksize=-2pt 0,subticks=1]{->}(0,0)(-1.2,-1.2)(1.2,1.2)
                    \pscircle[linewidth=0.8pt,linecolor=blue](0.,0.){3.} %cercle trigo
                    \parametricplot[linewidth=0.8pt,arrows=->]{0.8}{1.3}{1.15*cos(t)|1.15*sin(t)}% sens trigo
                    \rput[tl](0.6,1.1){+}
                    \psline[linewidth=0.8pt]{->}(0.,0.)(1.,0.) %vecteurs unités
                    \psline[linewidth=0.8pt]{->}(0.,0.)(0,1)
                    \psdots[dotsize=2pt 0,dotstyle=*](0.,0.)
                    \rput[tr](-0.05,-0.05){$O$}
                    \psdots[dotsize=2pt 0,dotstyle=*](0,1)
                    \rput[bl](0.05,1.05){$J$}
                    \psdots[dotsize=2pt 0,dotstyle=*](1.,0.)
                    \rput[bl](1.05,0.05){$I$}
               \end{pspicture*}
          }
     \end{extern}
\end{center}
\bloc{cyan}{Mesure d'un angle en radians}{%
     Dans le plan $P$ muni d'un repère orthonormé $\left(O ; I , J\right)$, on trace le cercle trigonométrique et la droite d'équation $x=1$ qui est tangente à ce cercle.
     \begin{center}
          \begin{extern}%width="250" alt="cercle trigonométrique"
               \resizebox{6cm}{!}{
                    \psset{xunit=3.0cm,yunit=3.0cm,algebraic=true,dimen=middle,dotstyle=o,dotsize=5pt 0,linewidth=0.8pt,arrowsize=3pt 2,arrowinset=0.25}
                    \begin{pspicture*}(-1.2,-1.2)(1.2,1.2)
                         \psaxes[linewidth=0.75pt,,xAxis=true,yAxis=true,Dx=1.,Dy=1.,ticksize=-2pt 0,subticks=1]{->}(0,0)(-1.2,-1.2)(1.2,1.2)
                         \pscircle[linewidth=0.8pt,linecolor=blue](0.,0.){3.} %cercle trigo
                         \parametricplot[linewidth=0.8pt,arrows=->]{0.8}{1.3}{1.15*cos(t)|1.15*sin(t)}% sens trigo
                         \rput[tl](0.6,1.1){+}
                         \psline[linewidth=0.8pt]{->}(0.,0.)(1.,0.) %vecteurs unités
                         \psline[linewidth=0.8pt]{->}(0.,0.)(0,1)
                         \psdots[dotsize=2pt 0,dotstyle=*](0.,0.)
                         \rput[tr](-0.05,-0.05){$O$}
                         \psdots[dotsize=2pt 0,dotstyle=*](0,1)
                         \rput[bl](0.05,1.05){$J$}
                         \psdots[dotsize=2pt 0,dotstyle=*](1.,0.)
                         \rput[bl](1.05,0.05){$I$}
                         \psline[linewidth=0.8pt]{->}(1,-1.5)(1,1.5)
                    \end{pspicture*}
               }
          \end{extern}
     \end{center}
     Soit $N$ un point du cercle. Pour mesurer en radians l'angle $\widehat{ION}$ on mesure la longueur de l'arc $\left(IN\right)$.
     \begin{center}
          \begin{extern}%width="250" alt="longueur d'un arc en radians"
               \resizebox{6cm}{!}{
                    \newrgbcolor{dblue}{0. 0. 0.7}
                    \newrgbcolor{dvert}{0. 0.4 0.}
                    \newrgbcolor{dmauve}{0.5 0. 0.5}
                    \psset{xunit=3.0cm,yunit=3.0cm,algebraic=true,dimen=middle,dotstyle=o,dotsize=5pt 0,linewidth=0.8pt,arrowsize=3pt 2,arrowinset=0.25}
                    \begin{pspicture*}(-1.2,-1.2)(1.2,1.2)
                         \psaxes[linewidth=0.75pt,xAxis=true,yAxis=true,Dx=1.,Dy=1.,ticksize=-2pt 0,subticks=1]{->}(0,0)(-1.2,-1.2)(1.2,1.2)
                         \pscircle[linewidth=0.8pt,linecolor=blue](0.,0.){3.} %cercle trigo
                         \parametricplot[linewidth=0.8pt,arrows=->]{0.8}{1.3}{1.15*cos(t)|1.15*sin(t)}% sens trigo
                         \rput[tl](0.6,1.1){+}
                         \psline[linewidth=0.8pt]{->}(0.,0.)(1.,0.) %vecteurs unités
                         \psline[linewidth=0.8pt]{->}(0.,0.)(0,1)
                         \psdots[dotsize=2pt 0,dotstyle=*](0.,0.)
                         \rput[tr](-0.05,-0.05){$O$}
                         \psdots[dotsize=2pt 0,dotstyle=*](0,1)
                         \rput[bl](0.05,1.05){$J$}
                         \psdots[dotsize=2pt 0,dotstyle=*](1.,0.)
                         \rput[bl](1.05,0.05){$I$}
                         \psline[linewidth=0.8pt](1,-1.5)(1,1.5)
                         \pscustom[linewidth=0.8pt,linecolor=dmauve,fillcolor=dmauve,fillstyle=solid,opacity=0.1]					{ % color angle
                              \parametricplot{0.0}{1.92}{0.15*cos(t)|0.15*sin(t)}
                         \lineto(0.,0.)\closepath}
                         \psline[linewidth=0.8pt,linecolor=dmauve](0.,0.)(-0.342,0.94)%rayon
                         \parametricplot[linewidth=1.2pt,linecolor=red]{0.0}{1.92}{cos(t)|sin(t)}%arc angle
                         \psdots[dotsize=2pt 0,dotstyle=*,linecolor=dblue](-0.342,0.94)
                         \rput[br](-0.342,1){\dblue{$N$}}
                    \end{pspicture*}
               }
          \end{extern}
     \end{center}
     Pour cela on \textit{« enroule »} la tangente sur le cercle trigonométrique et on fait correspondre au point $N$ un point $M$ situé sur cette tangente.
     \begin{center}
          \begin{extern}%width="250" alt="mesure d'un angle"
               \resizebox{6cm}{!}{
                    \newrgbcolor{dblue}{0. 0. 0.7}
                    \newrgbcolor{dvert}{0. 0.4 0.}
                    \newrgbcolor{dmauve}{0.5 0. 0.5}
                    \psset{xunit=3.0cm,yunit=3.0cm,algebraic=true,dimen=middle,dotstyle=o,dotsize=5pt 0,linewidth=0.8pt,arrowsize=3pt 2,arrowinset=0.25}
                    \begin{pspicture*}(-1.2,-1.2)(1.2,2.2)
                         \psaxes[linewidth=0.75pt,xAxis=true,yAxis=true,Dx=1.,Dy=1.,ticksize=-2pt 0,subticks=1]{->}(0,0)(-1.2,-1.2)(1.2,2.2)
                         \pscircle[linewidth=0.8pt,linecolor=blue](0.,0.){3.} %cercle trigo
                         \parametricplot[linewidth=0.8pt,arrows=->]{0.8}{1.3}{1.15*cos(t)|1.15*sin(t)}% sens trigo
                         \rput[tl](0.6,1.1){+}
                         \psline[linewidth=0.8pt]{->}(0.,0.)(1.,0.) %vecteurs unités
                         \psline[linewidth=0.8pt]{->}(0.,0.)(0,1)
                         \psdots[dotsize=2pt 0,dotstyle=*](0.,0.)
                         \rput[tr](-0.05,-0.05){$O$}
                         \psdots[dotsize=2pt 0,dotstyle=*](0,1)
                         \rput[bl](0.05,1.05){$J$}
                         \psdots[dotsize=2pt 0,dotstyle=*](1.,0.)
                         \rput[bl](1.05,0.05){$I$}
                         \psline[linewidth=0.8pt](1,-1.5)(1,2.5)
                         \pscustom[linewidth=0.8pt,linecolor=dmauve,fillcolor=dmauve,fillstyle=solid,opacity=0.1]					{ % color angle
                              \parametricplot{0.0}{1.92}{0.15*cos(t)|0.15*sin(t)}
                         \lineto(0.,0.)\closepath}
                         \psline[linewidth=0.8pt,linecolor=dmauve](0.,0.)(-0.342,0.94)%rayon
                         \parametricplot[linewidth=1.2pt,linecolor=red]{0.0}{1.92}{cos(t)|sin(t)}%arc angle
                         \psdots[dotsize=2pt 0,dotstyle=*,linecolor=dblue](-0.342,0.94)
                         \rput[br](-0.342,1){\dblue{$N$}}
                         \psdots[dotsize=2pt 0,dotstyle=*,linecolor=dblue](1.0,1.92)
                         \rput[bl](1.03,1.92){\dblue{$M$}}
                         \psline[linewidth=1pt,linecolor=red](1.,0.)(1.,1.92) % segment IM
                         \psellipticarc[linewidth=0.8pt,linecolor=dvert,arrows=<->](1.373,0.)(1.956,1.956) {101.3}{151} % fleche MN
                    \end{pspicture*}
               }
          \end{extern}
     \end{center}
     L'ordonnée de $M$ est une \textit{mesure en radians} de l'angle $\widehat{ION}$ (sur la figure ci-dessus cette mesure vaut environ 1,9 radians).
     \par
     Cette mesure n'est pas unique.En effet, si l'on poursuit « l'enroulement » de la droite sur le cercle trigonométrique, on voit que plusieurs points de cette droite vont venir se positionner sur le point $N$.
     \par
     Il en est de même si l'on « enroule » la droite dans l'autre sens ; dans ce cas on obtiendra des mesures négatives de l'angle.
     \begin{center}
          \begin{extern}%width="320" alt="mesures négatives d'un angle"
               \resizebox{6cm}{!}{
                    \newrgbcolor{dblue}{0. 0. 0.7}
                    \newrgbcolor{dvert}{0. 0.4 0.}
                    \newrgbcolor{dmauve}{0.5 0. 0.5}
                    \psset{xunit=3.0cm,yunit=3.0cm,algebraic=true,dimen=middle,dotstyle=o,dotsize=5pt 0,linewidth=0.8pt,arrowsize=3pt 2,arrowinset=0.25}
                    \begin{pspicture*}(-2.6,-4.6)(1.2,1.2)
                         \psaxes[linewidth=0.75pt,xAxis=true,yAxis=true,Dx=1.,Dy=1.,ticksize=-2pt 0,subticks=1]{->}(0,0)(-2.6,-4.6)(1.2,1.2)
                         \pscircle[linewidth=0.8pt,linecolor=blue](0.,0.){3.} %cercle trigo
                         \parametricplot[linewidth=0.8pt,arrows=->]{0.8}{1.3}{1.15*cos(t)|1.15*sin(t)}% sens trigo
                         \rput[tl](0.6,1.1){+}
                         \psline[linewidth=0.8pt]{->}(0.,0.)(1.,0.) %vecteurs unités
                         \psline[linewidth=0.8pt]{->}(0.,0.)(0,1)
                         \psdots[dotsize=2pt 0,dotstyle=*](0.,0.)
                         \rput[tr](-0.05,-0.05){$O$}
                         \psdots[dotsize=2pt 0,dotstyle=*](0,1)
                         \rput[bl](0.05,1.05){$J$}
                         \psdots[dotsize=2pt 0,dotstyle=*](1.,0.)
                         \rput[bl](1.05,0.05){$I$}
                         \psline[linewidth=0.8pt](1,-4.9)(1,1.5)
                         \pscustom[linewidth=0.8pt,linecolor=dmauve,fillcolor=dmauve,fillstyle=solid,opacity=0.1]
                         { % color angle
                              \parametricplot{0.0}{1.92}{0.15*cos(t)|0.15*sin(t)}
                         \lineto(0.,0.)\closepath}
                         \psline[linewidth=0.8pt,linecolor=dmauve](0.,0.)(-0.342,0.94)%rayon
                         \parametricplot[linewidth=1.2pt,linecolor=red]{1.92}{6.283}{cos(t)|sin(t)}%arc angle
                         \psdots[dotsize=2pt 0,dotstyle=*,linecolor=dblue](-0.342,0.94)
                         \rput[br](-0.342,1){\dblue{$N$}}
                         \psdots[dotsize=2pt 0,dotstyle=*,linecolor=dblue](1.0,-4.36)
                         \rput[bl](1.03,-4.36){\dblue{$M$}}
                         \psline[linewidth=1pt,linecolor=red](1.,0.)(1.,-4.36) % segment IM
                         \psellipticarc[linewidth=0.8pt,linecolor=dvert,arrows=<->](0.329,-1.71)(2.73,2.73) {104}{284} % fleche MN
                    \end{pspicture*}
               }
          \end{extern}
\end{center}   }
\cadre{vert}{Propriété}{%
     Chaque angle possède une infinité de mesures (en radians) qui diffèrent d'un multiple de $2\pi $.
}
\bloc{cyan}{Remarques}{%
     \begin{itemize}
          \item Cela signifie que si $x$ est une mesure d'un angle, les autres mesures sont $x+2\pi , x+4\pi , $ etc. et $x-2\pi , x-4\pi , $ etc.
          \item Ces différentes mesures s'écrivent donc $x+2k\pi $ avec $k \in \mathbb{Z}$
     \end{itemize}
}
\bloc{orange}{Mesures d'angles à retenir}{%
     \begin{center}
          \begin{extern}%width="400" alt="Mesures d'angles remarquables"
               \newrgbcolor{dblue}{0. 0. 0.7}
               \newrgbcolor{dvert}{0. 0.4 0.}
               \newrgbcolor{dmauve}{0.5 0. 0.5}
               \psset{xunit=5.0cm,yunit=5.0cm,algebraic=true,dimen=middle,dotstyle=o,dotsize=5pt 0,linewidth=0.8pt,arrowsize=3pt 2,arrowinset=0.25}
               \resizebox{8cm}{!}{
                    \begin{pspicture*}(-1.2,-1.2)(1.2,1.2)
                         \psaxes[linewidth=0.75pt,labelFontSize=\scriptstyle,xAxis=true,yAxis=true,Dx=10.,Dy=10.,ticksize=-2pt 0,subticks=1]{->}(0,0)(-1.2,-1.2)(1.2,1.2)
                         \pscircle[linewidth=0.8pt](0.,0.){5.} %cercle trigo
                         \psline[linewidth=0.8pt]{->}(0.,0.)(1.,0.) %vecteurs unités
                         \psline[linewidth=0.8pt]{->}(0.,0.)(0,1)
                         %\rput[tl](0.4,0.1){$\vec{i}$}
                         %\rput[tl](-0.06,0.5){$\vec{j}$}
                         \psdots[dotsize=2pt 0,dotstyle=*](0.,0.)
                         %\rput[bl](-0.09,-0.09){$O$}
                         \psline[linewidth=0.8pt,linecolor=dvert](-0.707,-0.707)(0.707,0.707)
                         \psline[linewidth=0.8pt,linecolor=dvert](-0.707,0.707)(0.707,-0.707)
                         \psline[linewidth=0.8pt,linecolor=red](-0.866,-0.5)(0.866,0.5)
                         \psline[linewidth=0.8pt,linecolor=red](0.866,-0.5)(-0.866,0.5)
                         \rput(0.943,0.55){$\red{\dfrac{\pi}{6}}$}
                         \rput(-0.943,0.55){$\red{\dfrac{5\pi}{6}}$}
                         \rput(-0.973,-0.55){$\red{-\dfrac{5\pi}{6}}$}
                         \rput(0.943,-0.55){$\red{-\dfrac{\pi}{6}}$}
                         %
                         \rput(0.777,0.777){$\dvert{\dfrac{\pi}{4}}$}
                         \rput(-0.777,0.777){$\dvert{\dfrac{3\pi}{4}}$}
                         \rput(-0.807,-0.777){$\dvert{-\dfrac{3\pi}{4}}$}
                         \rput(0.777,-0.777){$\dvert{-\dfrac{\pi}{4}}$}
                         %
                         \psline[linewidth=0.8pt,linecolor=dblue](-0.5,-0.866)(0.5,0.866)
                         \psline[linewidth=0.8pt,linecolor=dblue](-0.5,0.866)(0.5,-0.866)
                         \rput(0.55,0.943){$\dblue{\dfrac{\pi}{3}}$}
                         \rput(-0.55,0.943){$\dblue{\dfrac{2\pi}{3}}$}
                         \rput(-0.58,-0.943){$\dblue{-\dfrac{2\pi}{3}}$}
                         \rput(0.55,-0.943){$\dblue{-\dfrac{\pi}{3}}$}
                         %
                         \rput(1.06,0.06){$0$}
                         \rput(0.06,1.1){$\dfrac{\pi}{2}$}
                         \rput(0.06,-1.1){$-\dfrac{\pi}{2}$}
                         \rput(-1.06,0.06){$\pi$}
                    \end{pspicture*}
               }
          \end{extern}
     \end{center}
     \begin{center}
          \textit{ Mesures d'angles remarquables}
     \end{center}
}
\begin{h2}2. Sinus et cosinus\end{h2}
\cadre{bleu}{Définition}{%
     Soit $N$ un point du cercle trigonométrique. On note $x$ une mesure de l'angle $\widehat{ION}$.
     \par
     On appelle \textbf{cosinus} de $x$, noté\textbf{ $\cos x$} l'abscisse du point $N$.
     \par
     On appelle \textbf{sinus} de $x$, noté\textbf{ $\sin x$} l'ordonnée du point $N$
}
\begin{center}
     \begin{extern}%width="330" alt="sinus et cosinus d'un angle"
          \resizebox{6cm}{!}{
               \newrgbcolor{dblue}{0. 0. 0.7}
               \newrgbcolor{dvert}{0. 0.4 0.}
               \newrgbcolor{dmauve}{0.5 0. 0.5}
               \psset{xunit=5.0cm,yunit=5.0cm,algebraic=true,dimen=middle,dotstyle=o,dotsize=5pt 0,linewidth=0.8pt,arrowsize=3pt 2,arrowinset=0.25}
               \begin{pspicture*}(-1.2,-1.2)(1.2,1.2)
                    \psaxes[linewidth=0.75pt,labelFontSize=\scriptstyle,xAxis=true,yAxis=true,Dx=10.,Dy=10.,ticksize=-2pt 0,subticks=1]{->}(0,0)(-1.2,-1.2)(1.2,1.2)
                    \pscircle[linewidth=0.8pt](0.,0.){5.} %cercle trigo
                    \parametricplot[linewidth=1.2pt,linecolor=red]{0.0}{0.698}{cos(t)|sin(t)}%arc angle
                    \pscustom[linewidth=0.8pt,linecolor=dmauve,fillcolor=dmauve,fillstyle=solid,opacity=0.1]{ % color angle
                         \parametricplot{0.0}{0.698}{0.15*cos(t)|0.15*sin(t)}
                    \lineto(0.,0.)\closepath}
                    \psellipticarc[linewidth=0.8pt,linecolor=dmauve](0.,0.)(0.15,0.15){0.}{40} % fleche angle
                    \psline[linewidth=0.8pt,linecolor=dmauve](0.,0.)(0.766,0.643)%rayon
                    \psline[linewidth=0.8pt]{->}(0.,0.)(1.,0.) %vecteurs unités
                    \psline[linewidth=0.8pt]{->}(0.,0.)(0,1)
                    %\rput[tl](0.4,0.1){$\vec{i}$}
                    %\rput[tl](-0.06,0.5){$\vec{j}$}
                    \psdots[dotsize=2pt 0,dotstyle=*](0.,0.)
                    \rput[bl](-0.09,-0.09){$O$}
                    \psdots[dotsize=2pt 0,dotstyle=*,linecolor=dblue](1.,0.)
                    \rput[bl](1.02,0.02){\dblue{$I$}}
                    \psdots[dotsize=2pt 0,dotstyle=*,linecolor=dblue](0.766,0.643)
                    \rput[bl](0.78,0.66){\dvert{$N$}}
                    \psdots[dotsize=2pt 0,dotstyle=*,linecolor=dblue](0,1)
                    \rput[bl](0.02,1.03){\dblue{$J$}}
                    \rput[bl](0.19,0.05){\dmauve{$x$}}
                    \psline[linewidth=0.8pt,linecolor=dvert](0.,0.643)(0.766,0.643)
                    \psline[linewidth=0.8pt,linecolor=dvert](0.766,0.)(0.766,0.643)
                    \rput(0.766,-0.05){\dvert{$\cos x$}}
                    \rput(-0.10,0.643){\dvert{$\sin x$}}
               \end{pspicture*}
          }
     \end{extern}
\end{center}
\bloc{cyan}{Remarque}{%
     Ces notions généralisent celles vues au collège.
     \par
     En effet si l'angle $\widehat{ION}$ est aigu :
     \begin{center}
          \begin{extern}%width="330" alt="sinus et cosinus d'un angle aigu"
               \resizebox{6cm}{!}{
                    \newrgbcolor{dblue}{0. 0. 0.7}
                    \newrgbcolor{dvert}{0. 0.4 0.}
                    \newrgbcolor{dmauve}{0.5 0. 0.5}
                    \psset{xunit=5.0cm,yunit=5.0cm,algebraic=true,dimen=middle,dotstyle=o,dotsize=5pt 0,linewidth=0.8pt,arrowsize=3pt 2,arrowinset=0.25}
                    \begin{pspicture*}(-1.2,-1.2)(1.2,1.2)
                         \psaxes[linewidth=0.75pt,labelFontSize=\scriptstyle,xAxis=true,yAxis=true,Dx=10.,Dy=10.,ticksize=-2pt 0,subticks=1]{->}(0,0)(-1.2,-1.2)(1.2,1.2)
                         \pscircle[linewidth=0.8pt](0.,0.){5.} %cercle trigo
                         \pscustom[linewidth=0.8pt,linecolor=dmauve,fillcolor=dmauve,fillstyle=solid,opacity=0.1]{ % color angle
                              \parametricplot{0.0}{0.698}{0.15*cos(t)|0.15*sin(t)}
                         \lineto(0.,0.)\closepath}
                         \psellipticarc[linewidth=0.8pt,linecolor=dmauve](0.,0.)(0.15,0.15){0.}{40} % fleche angle
                         \psline[linewidth=0.8pt,linecolor=dmauve](0.,0.)(0.766,0.643)%rayon
                         \psline[linewidth=0.8pt]{->}(0.,0.)(1.,0.) %vecteurs unités
                         \psline[linewidth=0.8pt]{->}(0.,0.)(0,1)
                         %\rput[tl](0.4,0.1){$\vec{i}$}
                         %\rput[tl](-0.06,0.5){$\vec{j}$}
                         \psdots[dotsize=2pt 0,dotstyle=*](0.,0.)
                         \rput[bl](-0.09,-0.09){$O$}
                         \psdots[dotsize=2pt 0,dotstyle=*,linecolor=dblue](1.,0.)
                         \rput[bl](1.02,0.02){\dblue{$I$}}
                         \psdots[dotsize=2pt 0,dotstyle=*,linecolor=dblue](0.766,0.643)
                         \rput[bl](0.78,0.66){\dvert{$N$}}
                         \psdots[dotsize=2pt 0,dotstyle=*,linecolor=dblue](0,1)
                         \rput[bl](0.02,1.03){\dblue{$J$}}
                         \psline[linewidth=0.8pt,linecolor=dvert](0.766,0.)(0.766,0.643)
                         \psframe[linewidth=0.4pt,linecolor=dvert](0.766,0.)(0.71,0.056)
                         \rput(0.766,-0.05){\dvert{$A$}}
                    \end{pspicture*}
               }
          \end{extern}
     \end{center}
     Le triangle $OAN$ est rectangle en $A$ et $ON=1$ car $\left[ON\right]$ est un rayon du cercle; par conséquent :
     \par
     $\cos\left(\widehat{ION}\right)=\cos\left(\widehat{AON}\right)=\frac{OA}{ON}=\frac{OA}{1}=OA$
     \par
     $\sin\left(\widehat{ION}\right)=\sin\left(\widehat{AON}\right)=\frac{AN}{ON}=\frac{AN}{1}=AN$
}
\bloc{orange}{Valeurs de sinus et de cosinus à retenir}{%
     \begin{center}
          \begin{extern}%width="450" alt="Valeurs de sinus et de cosinus"
               \newrgbcolor{dblue}{0. 0. 0.7}
               \newrgbcolor{dvert}{0. 0.4 0.}
               \newrgbcolor{dmauve}{0.5 0. 0.5}
               \psset{xunit=5.0cm,yunit=5.0cm,algebraic=true,dimen=middle,dotstyle=o,dotsize=5pt 0,linewidth=0.8pt,arrowsize=3pt 2,arrowinset=0.25}
               \begin{pspicture*}(-1.2,-1.2)(1.2,1.2)
                    \psaxes[linewidth=0.75pt,labelFontSize=\scriptstyle,xAxis=true,yAxis=true,Dx=10.,Dy=10.,ticksize=-2pt 0,subticks=1]{->}(0,0)(-1.2,-1.2)(1.2,1.2)
                    \pscircle[linewidth=0.8pt](0.,0.){5.} %cercle trigo
                    \psline[linewidth=0.8pt]{->}(0.,0.)(1.,0.) %vecteurs unités
                    \psline[linewidth=0.8pt]{->}(0.,0.)(0,1)
                    %\rput[tl](0.4,0.1){$\vec{i}$}
                    %\rput[tl](-0.06,0.5){$\vec{j}$}
                    \psdots[dotsize=2pt 0,dotstyle=*](0.,0.)
                    %\rput[bl](-0.09,-0.09){$O$}
                    \psframe[linewidth=0.4pt,linecolor=dvert](-0.707,-0.707)(0.707,0.707)
                    \psline[linewidth=0.8pt,linecolor=dvert](-0.707,-0.707)(0.707,0.707)
                    \psline[linewidth=0.8pt,linecolor=dvert](-0.707,0.707)(0.707,-0.707)
                    \psframe[linewidth=0.4pt,linecolor=red](-0.866,-0.5)(0.866,0.5)
                    \psline[linewidth=0.8pt,linecolor=red](-0.866,-0.5)(0.866,0.5)
                    \psline[linewidth=0.8pt,linecolor=red](0.866,-0.5)(-0.866,0.5)
                    \rput(0.943,0.55){$\red{\dfrac{\pi}{6}}$}
                    \rput(-0.943,0.55){$\red{\dfrac{5\pi}{6}}$}
                    \rput(-0.973,-0.55){$\red{-\dfrac{5\pi}{6}}$}
                    \rput(0.943,-0.55){$\red{-\dfrac{\pi}{6}}$}
                    \rput(0.05,-0.554){\fontsize{7 pt}{7 pt}\selectfont $\red{ -\dfrac{1}{2}}$}
                    \rput(0.05,0.554){\fontsize{7 pt}{7 pt}\selectfont $\red{ \dfrac{1}{2}}$}
                    \rput(0.93,0.07){\fontsize{7 pt}{7 pt}\selectfont $\red{ \dfrac{\sqrt{3}}{2}}$}
                    \rput(-0.93,0.07){\fontsize{7 pt}{7 pt}\selectfont $\red{-\dfrac{\sqrt{3}}{2}}$}
                    %
                    \rput(0.777,0.777){$\dvert{\dfrac{\pi}{4}}$}
                    \rput(-0.777,0.777){$\dvert{\dfrac{3\pi}{4}}$}
                    \rput(-0.807,-0.777){$\dvert{-\dfrac{3\pi}{4}}$}
                    \rput(0.777,-0.777){$\dvert{-\dfrac{\pi}{4}}$}
                    \rput(0.06,-0.77){\fontsize{7 pt}{7 pt}\selectfont $\dvert{ -\dfrac{\sqrt{2}}{2}}$}
                    \rput(0.06,0.77){\fontsize{7 pt}{7 pt}\selectfont $\dvert{ \dfrac{\sqrt{2}}{2}}$}
                    \rput(0.764,0.07){\fontsize{7 pt}{7 pt}\selectfont $\dvert{ \dfrac{\sqrt{2}}{2}}$}
                    \rput(-0.764,0.07){\fontsize{7 pt}{7 pt}\selectfont $\dvert{-\dfrac{\sqrt{2}}{2}}$}
                    %
                    \psframe[linewidth=0.4pt,linecolor=dblue](-0.5,-0.866)(0.5,0.866)
                    \psline[linewidth=0.8pt,linecolor=dblue](-0.5,-0.866)(0.5,0.866)
                    \psline[linewidth=0.8pt,linecolor=dblue](-0.5,0.866)(0.5,-0.866)
                    \rput(0.55,0.943){$\dblue{\dfrac{\pi}{3}}$}
                    \rput(-0.55,0.943){$\dblue{\dfrac{2\pi}{3}}$}
                    \rput(-0.58,-0.943){$\dblue{-\dfrac{2\pi}{3}}$}
                    \rput(0.55,-0.943){$\dblue{-\dfrac{\pi}{3}}$}
                    \rput(0.06,-0.933){\fontsize{7 pt}{7 pt}\selectfont $\dblue{ -\dfrac{\sqrt{3}}{2}}$}
                    \rput(0.06,0.933){\fontsize{7 pt}{7 pt}\selectfont $\dblue{ \dfrac{\sqrt{3}}{2}}$}
                    \rput(0.538,0.07){\fontsize{7 pt}{7 pt}\selectfont $\dblue{ \dfrac{1}{2}}$}
                    \rput(-0.538,0.07){\fontsize{7 pt}{7 pt}\selectfont $\dblue{-\dfrac{1}{2}}$}
                    %
                    \rput(1.06,0.06){$0$}
                    \rput(0.06,1.1){$\dfrac{\pi}{2}$}
                    \rput(0.06,-1.1){$-\dfrac{\pi}{2}$}
                    \rput(-1.06,0.06){$\pi$}
               \end{pspicture*}
          \end{extern}
     \end{center}
     \begin{center}
          \begin{tabularx}{0.8\linewidth}{|*{10}{>{\centering \arraybackslash }X|}}%class="compact" width="600"
               \hline
               \textbf{$x$} & $0$ & $\frac{\pi }{6}$ & $\frac{\pi }{4}$ & $\frac{\pi }{3}$ & $\frac{\pi }{2}$ & $\frac{2\pi }{3}$ & $\frac{3\pi }{4}$ & $\frac{5\pi }{6}$ & $\pi $\\ \hline
               \textbf{$\cos x$} & $1$ & $\frac{\sqrt{3}}{2}$ & $\frac{\sqrt{2}}{2}$ & $\frac{1}{2}$ & $0$ & $-\frac{1}{2}$ & $-\frac{\sqrt{2}}{2}$ & $-\frac{\sqrt{3}}{2}$ & $-1$\\ \hline
               \textbf{$\sin x$} & $0$ & $\frac{1}{2}$ & $\frac{\sqrt{2}}{2}$ & $\frac{\sqrt{3}}{2}$ & $1$ & $\frac{\sqrt{3}}{2}$ & $\frac{\sqrt{2}}{2}$ & $\frac{1}{2}$ & $0$\\ \hline
          \end{tabularx}
     \end{center}
     \begin{center}
          \begin{tabularx}{0.8\linewidth}{|*{8}{>{\centering \arraybackslash }X|}}%class="compact" width="600"
               \hline
               \textbf{$x$} & $-\frac{\pi }{6}$ & $-\frac{\pi }{4}$ & $-\frac{\pi }{3}$ & $-\frac{\pi }{2}$ & $-\frac{2\pi }{3}$ & $-\frac{3\pi }{4}$ & $-\frac{5\pi }{6}$\\ \hline
               \textbf{$\cos x$} & $\frac{\sqrt{3}}{2}$ & $\frac{\sqrt{2}}{2}$ & $\frac{1}{2}$ & $0$ & $-\frac{1}{2}$ & $-\frac{\sqrt{2}}{2}$ & $-\frac{\sqrt{3}}{2}$\\ \hline
               \textbf{$\sin x$} & $-\frac{1}{2}$ & $-\frac{\sqrt{2}}{2}$ & $-\frac{\sqrt{3}}{2}$ & $-1$ & $-\frac{\sqrt{3}}{2}$ & $-\frac{\sqrt{2}}{2}$ & $-\frac{1}{2}$\\ \hline
          \end{tabularx}
     \end{center}
}
\cadre{vert}{Propriétés}{%
     Pour tout réel $x$ :
     \begin{itemize}
          \item $-1 \leqslant \cos x \leqslant 1$
          \item $-1 \leqslant \sin x \leqslant 1$
          \item $\left(\cos x\right)^2 + \left(\sin x\right)^2 = 1$
     \end{itemize}
}
\bloc{cyan}{Remarque}{%
     On écrit souvent $\cos^2 x$ et $\sin^2 x$ à la place de $\left(\cos x\right)^2$ et $\left(\sin x\right)^2$ afin de simplifier les notations.
     \par
     La dernière propriété s'écrit alors :
     \par
     $\cos^2 x + \sin^2 x = 1$
}

\end{document}
µ
\documentclass[a4paper]{article}

%================================================================================================================================
%
% Packages
%
%================================================================================================================================

\usepackage[T1]{fontenc} 	% pour caractères accentués
\usepackage[utf8]{inputenc}  % encodage utf8
\usepackage[french]{babel}	% langue : français
\usepackage{fourier}			% caractères plus lisibles
\usepackage[dvipsnames]{xcolor} % couleurs
\usepackage{fancyhdr}		% réglage header footer
\usepackage{needspace}		% empêcher sauts de page mal placés
\usepackage{graphicx}		% pour inclure des graphiques
\usepackage{enumitem,cprotect}		% personnalise les listes d'items (nécessaire pour ol, al ...)
\usepackage{hyperref}		% Liens hypertexte
\usepackage{pstricks,pst-all,pst-node,pstricks-add,pst-math,pst-plot,pst-tree,pst-eucl} % pstricks
\usepackage[a4paper,includeheadfoot,top=2cm,left=3cm, bottom=2cm,right=3cm]{geometry} % marges etc.
\usepackage{comment}			% commentaires multilignes
\usepackage{amsmath,environ} % maths (matrices, etc.)
\usepackage{amssymb,makeidx}
\usepackage{bm}				% bold maths
\usepackage{tabularx}		% tableaux
\usepackage{colortbl}		% tableaux en couleur
\usepackage{fontawesome}		% Fontawesome
\usepackage{environ}			% environment with command
\usepackage{fp}				% calculs pour ps-tricks
\usepackage{multido}			% pour ps tricks
\usepackage[np]{numprint}	% formattage nombre
\usepackage{tikz,tkz-tab} 			% package principal TikZ
\usepackage{pgfplots}   % axes
\usepackage{mathrsfs}    % cursives
\usepackage{calc}			% calcul taille boites
\usepackage[scaled=0.875]{helvet} % font sans serif
\usepackage{svg} % svg
\usepackage{scrextend} % local margin
\usepackage{scratch} %scratch
\usepackage{multicol} % colonnes
%\usepackage{infix-RPN,pst-func} % formule en notation polanaise inversée
\usepackage{listings}

%================================================================================================================================
%
% Réglages de base
%
%================================================================================================================================

\lstset{
language=Python,   % R code
literate=
{á}{{\'a}}1
{à}{{\`a}}1
{ã}{{\~a}}1
{é}{{\'e}}1
{è}{{\`e}}1
{ê}{{\^e}}1
{í}{{\'i}}1
{ó}{{\'o}}1
{õ}{{\~o}}1
{ú}{{\'u}}1
{ü}{{\"u}}1
{ç}{{\c{c}}}1
{~}{{ }}1
}


\definecolor{codegreen}{rgb}{0,0.6,0}
\definecolor{codegray}{rgb}{0.5,0.5,0.5}
\definecolor{codepurple}{rgb}{0.58,0,0.82}
\definecolor{backcolour}{rgb}{0.95,0.95,0.92}

\lstdefinestyle{mystyle}{
    backgroundcolor=\color{backcolour},   
    commentstyle=\color{codegreen},
    keywordstyle=\color{magenta},
    numberstyle=\tiny\color{codegray},
    stringstyle=\color{codepurple},
    basicstyle=\ttfamily\footnotesize,
    breakatwhitespace=false,         
    breaklines=true,                 
    captionpos=b,                    
    keepspaces=true,                 
    numbers=left,                    
xleftmargin=2em,
framexleftmargin=2em,            
    showspaces=false,                
    showstringspaces=false,
    showtabs=false,                  
    tabsize=2,
    upquote=true
}

\lstset{style=mystyle}


\lstset{style=mystyle}
\newcommand{\imgdir}{C:/laragon/www/newmc/assets/imgsvg/}
\newcommand{\imgsvgdir}{C:/laragon/www/newmc/assets/imgsvg/}

\definecolor{mcgris}{RGB}{220, 220, 220}% ancien~; pour compatibilité
\definecolor{mcbleu}{RGB}{52, 152, 219}
\definecolor{mcvert}{RGB}{125, 194, 70}
\definecolor{mcmauve}{RGB}{154, 0, 215}
\definecolor{mcorange}{RGB}{255, 96, 0}
\definecolor{mcturquoise}{RGB}{0, 153, 153}
\definecolor{mcrouge}{RGB}{255, 0, 0}
\definecolor{mclightvert}{RGB}{205, 234, 190}

\definecolor{gris}{RGB}{220, 220, 220}
\definecolor{bleu}{RGB}{52, 152, 219}
\definecolor{vert}{RGB}{125, 194, 70}
\definecolor{mauve}{RGB}{154, 0, 215}
\definecolor{orange}{RGB}{255, 96, 0}
\definecolor{turquoise}{RGB}{0, 153, 153}
\definecolor{rouge}{RGB}{255, 0, 0}
\definecolor{lightvert}{RGB}{205, 234, 190}
\setitemize[0]{label=\color{lightvert}  $\bullet$}

\pagestyle{fancy}
\renewcommand{\headrulewidth}{0.2pt}
\fancyhead[L]{maths-cours.fr}
\fancyhead[R]{\thepage}
\renewcommand{\footrulewidth}{0.2pt}
\fancyfoot[C]{}

\newcolumntype{C}{>{\centering\arraybackslash}X}
\newcolumntype{s}{>{\hsize=.35\hsize\arraybackslash}X}

\setlength{\parindent}{0pt}		 
\setlength{\parskip}{3mm}
\setlength{\headheight}{1cm}

\def\ebook{ebook}
\def\book{book}
\def\web{web}
\def\type{web}

\newcommand{\vect}[1]{\overrightarrow{\,\mathstrut#1\,}}

\def\Oij{$\left(\text{O}~;~\vect{\imath},~\vect{\jmath}\right)$}
\def\Oijk{$\left(\text{O}~;~\vect{\imath},~\vect{\jmath},~\vect{k}\right)$}
\def\Ouv{$\left(\text{O}~;~\vect{u},~\vect{v}\right)$}

\hypersetup{breaklinks=true, colorlinks = true, linkcolor = OliveGreen, urlcolor = OliveGreen, citecolor = OliveGreen, pdfauthor={Didier BONNEL - https://www.maths-cours.fr} } % supprime les bordures autour des liens

\renewcommand{\arg}[0]{\text{arg}}

\everymath{\displaystyle}

%================================================================================================================================
%
% Macros - Commandes
%
%================================================================================================================================

\newcommand\meta[2]{    			% Utilisé pour créer le post HTML.
	\def\titre{titre}
	\def\url{url}
	\def\arg{#1}
	\ifx\titre\arg
		\newcommand\maintitle{#2}
		\fancyhead[L]{#2}
		{\Large\sffamily \MakeUppercase{#2}}
		\vspace{1mm}\textcolor{mcvert}{\hrule}
	\fi 
	\ifx\url\arg
		\fancyfoot[L]{\href{https://www.maths-cours.fr#2}{\black \footnotesize{https://www.maths-cours.fr#2}}}
	\fi 
}


\newcommand\TitreC[1]{    		% Titre centré
     \needspace{3\baselineskip}
     \begin{center}\textbf{#1}\end{center}
}

\newcommand\newpar{    		% paragraphe
     \par
}

\newcommand\nosp {    		% commande vide (pas d'espace)
}
\newcommand{\id}[1]{} %ignore

\newcommand\boite[2]{				% Boite simple sans titre
	\vspace{5mm}
	\setlength{\fboxrule}{0.2mm}
	\setlength{\fboxsep}{5mm}	
	\fcolorbox{#1}{#1!3}{\makebox[\linewidth-2\fboxrule-2\fboxsep]{
  		\begin{minipage}[t]{\linewidth-2\fboxrule-4\fboxsep}\setlength{\parskip}{3mm}
  			 #2
  		\end{minipage}
	}}
	\vspace{5mm}
}

\newcommand\CBox[4]{				% Boites
	\vspace{5mm}
	\setlength{\fboxrule}{0.2mm}
	\setlength{\fboxsep}{5mm}
	
	\fcolorbox{#1}{#1!3}{\makebox[\linewidth-2\fboxrule-2\fboxsep]{
		\begin{minipage}[t]{1cm}\setlength{\parskip}{3mm}
	  		\textcolor{#1}{\LARGE{#2}}    
 	 	\end{minipage}  
  		\begin{minipage}[t]{\linewidth-2\fboxrule-4\fboxsep}\setlength{\parskip}{3mm}
			\raisebox{1.2mm}{\normalsize\sffamily{\textcolor{#1}{#3}}}						
  			 #4
  		\end{minipage}
	}}
	\vspace{5mm}
}

\newcommand\cadre[3]{				% Boites convertible html
	\par
	\vspace{2mm}
	\setlength{\fboxrule}{0.1mm}
	\setlength{\fboxsep}{5mm}
	\fcolorbox{#1}{white}{\makebox[\linewidth-2\fboxrule-2\fboxsep]{
  		\begin{minipage}[t]{\linewidth-2\fboxrule-4\fboxsep}\setlength{\parskip}{3mm}
			\raisebox{-2.5mm}{\sffamily \small{\textcolor{#1}{\MakeUppercase{#2}}}}		
			\par		
  			 #3
 	 		\end{minipage}
	}}
		\vspace{2mm}
	\par
}

\newcommand\bloc[3]{				% Boites convertible html sans bordure
     \needspace{2\baselineskip}
     {\sffamily \small{\textcolor{#1}{\MakeUppercase{#2}}}}    
		\par		
  			 #3
		\par
}

\newcommand\CHelp[1]{
     \CBox{Plum}{\faInfoCircle}{À RETENIR}{#1}
}

\newcommand\CUp[1]{
     \CBox{NavyBlue}{\faThumbsOUp}{EN PRATIQUE}{#1}
}

\newcommand\CInfo[1]{
     \CBox{Sepia}{\faArrowCircleRight}{REMARQUE}{#1}
}

\newcommand\CRedac[1]{
     \CBox{PineGreen}{\faEdit}{BIEN R\'EDIGER}{#1}
}

\newcommand\CError[1]{
     \CBox{Red}{\faExclamationTriangle}{ATTENTION}{#1}
}

\newcommand\TitreExo[2]{
\needspace{4\baselineskip}
 {\sffamily\large EXERCICE #1\ (\emph{#2 points})}
\vspace{5mm}
}

\newcommand\img[2]{
          \includegraphics[width=#2\paperwidth]{\imgdir#1}
}

\newcommand\imgsvg[2]{
       \begin{center}   \includegraphics[width=#2\paperwidth]{\imgsvgdir#1} \end{center}
}


\newcommand\Lien[2]{
     \href{#1}{#2 \tiny \faExternalLink}
}
\newcommand\mcLien[2]{
     \href{https~://www.maths-cours.fr/#1}{#2 \tiny \faExternalLink}
}

\newcommand{\euro}{\eurologo{}}

%================================================================================================================================
%
% Macros - Environement
%
%================================================================================================================================

\newenvironment{tex}{ %
}
{%
}

\newenvironment{indente}{ %
	\setlength\parindent{10mm}
}

{
	\setlength\parindent{0mm}
}

\newenvironment{corrige}{%
     \needspace{3\baselineskip}
     \medskip
     \textbf{\textsc{Corrigé}}
     \medskip
}
{
}

\newenvironment{extern}{%
     \begin{center}
     }
     {
     \end{center}
}

\NewEnviron{code}{%
	\par
     \boite{gray}{\texttt{%
     \BODY
     }}
     \par
}

\newenvironment{vbloc}{% boite sans cadre empeche saut de page
     \begin{minipage}[t]{\linewidth}
     }
     {
     \end{minipage}
}
\NewEnviron{h2}{%
    \needspace{3\baselineskip}
    \vspace{0.6cm}
	\noindent \MakeUppercase{\sffamily \large \BODY}
	\vspace{1mm}\textcolor{mcgris}{\hrule}\vspace{0.4cm}
	\par
}{}

\NewEnviron{h3}{%
    \needspace{3\baselineskip}
	\vspace{5mm}
	\textsc{\BODY}
	\par
}

\NewEnviron{margeneg}{ %
\begin{addmargin}[-1cm]{0cm}
\BODY
\end{addmargin}
}

\NewEnviron{html}{%
}

\begin{document}
\meta{url}{/cours/geometrie-espace/}
\meta{pid}{219}
\meta{titre}{Géométrie dans l'espace}
\meta{type}{cours}
\begin{h2}1. Positions relatives de droites et de plans\end{h2}
\cadre{bleu}{Positions relatives de deux plans}{%
     Deux plans distincts de l'espace peuvent être :
     \begin{itemize}
          \item \textbf{strictement parallèles} : dans ce cas, ils n'ont aucun point commun
          \item \textbf{sécants} : dans ce cas, leur intersection est une droite
     \end{itemize}
}
\begin{vbloc}
     \begin{center}
          \begin{extern}%width="400" alt="Plans parallèles"
               \newrgbcolor{ttzzqq}{0.2 0.6 0.}
               \newrgbcolor{qqwwtt}{0. 0.4 0.2}
               \psset{xunit=1.0cm,yunit=1.0cm,algebraic=true,dimen=middle,dotstyle=o,dotsize=5pt 0,linewidth=1.6pt,arrowsize=3pt 2,arrowinset=0.25}
               \begin{pspicture*}(1.,2.8)(12,6.5)
                    \pspolygon[linewidth=0.8pt,linecolor=ttzzqq,fillcolor=ttzzqq,fillstyle=solid,opacity=0.1](4.,6.)(2.,5.)(9.,5.)(11.,6.)
                    \pspolygon[linewidth=0.8pt,linecolor=blue,fillcolor=blue,fillstyle=solid,opacity=0.1](4.,4.)(2.,3.)(9.,3.)(11.,4.)
                    \psline[linewidth=0.8pt,linecolor=ttzzqq](4.,6.)(2.,5.)
                    \psline[linewidth=0.8pt,linecolor=ttzzqq](2.,5.)(9.,5.)
                    \psline[linewidth=0.8pt,linecolor=ttzzqq](9.,5.)(11.,6.)
                    \psline[linewidth=0.8pt,linecolor=ttzzqq](11.,6.)(4.,6.)
                    \psline[linewidth=0.8pt,linecolor=blue](4.,4.)(2.,3.)
                    \psline[linewidth=0.8pt,linecolor=blue](2.,3.)(9.,3.)
                    \psline[linewidth=0.8pt,linecolor=blue](9.,3.)(11.,4.)
                    \psline[linewidth=0.8pt,linecolor=blue](11.,4.)(4.,4.)
                    \rput[tl](3,5.4){$\qqwwtt{\mathscr{P}'}$}
                    \rput[tl](3,3.4){$\blue{\mathscr{P}}$}
               \end{pspicture*}
          \end{extern}
     \end{center}
     \begin{center}
          \textit{Plans strictement parallèles}
     \end{center}
\end{vbloc}
\begin{center}
     \begin{extern}%width="400" alt=""
          \newrgbcolor{ttzzqq}{0.2 0.6 0.}
          \newrgbcolor{qqwwtt}{0. 0.4 0.2}
          \psset{xunit=1.0cm,yunit=1.0cm,algebraic=true,dimen=middle,dotstyle=o,dotsize=5pt 0,linewidth=1.6pt,arrowsize=3pt 2,arrowinset=0.25}
          \begin{pspicture*}(0.,2.5)(12,8.5)
               \pspolygon[linewidth=0.8pt,linecolor=ttzzqq,fillcolor=ttzzqq,fillstyle=solid,opacity=0.1](4.,6.)(2.,5.)(9.,5.)(11.,6.)
               \pspolygon[linewidth=0.8pt,linecolor=blue,fillcolor=blue,fillstyle=solid,opacity=0.1](4.,4.)(2.,3.)(8.,7.)(10.,8.)
               \psline[linewidth=0.8pt,linecolor=ttzzqq](4.,6.)(2.,5.)
               \psline[linewidth=0.8pt,linecolor=ttzzqq](2.,5.)(9.,5.)
               \psline[linewidth=0.8pt,linecolor=ttzzqq](9.,5.)(11.,6.)
               \psline[linewidth=0.8pt,linecolor=ttzzqq](11.,6.)(4.,6.)
               \psline[linewidth=0.8pt,linecolor=blue](4.,4.)(2.,3.)
               \psline[linewidth=0.8pt,linecolor=blue](2.,3.)(8.,7.)
               \psline[linewidth=0.8pt,linecolor=blue](8.,7.)(10.,8.)
               \psline[linewidth=0.8pt,linecolor=blue](10.,8.)(4.,4.)
               \rput[tl](3.,5.4){$\qqwwtt{\mathscr{P}}$}
               \rput[tl](2.6,3.2){$\blue{\mathscr{P}\ '}$}
               \psline[linewidth=0.8pt,linecolor=red](5.,5.)(7.,6.)
               \rput[tl](7.,5.9){$\red{\mathscr{D}}$}
          \end{pspicture*}
     \end{extern}
\end{center}
\begin{center}
     \textit{Plans sécants}
\end{center}
\cadre{bleu}{Positions relatives de d'une droite et d'un plan}{%
     Soient $\mathscr D$ une droite et $\mathscr P$ un plan de l'espace.
     \par
     La droite $\mathscr D$ peut être :
     \begin{itemize}
          \item \textbf{strictement parallèle} au plan  $\mathscr P$ : dans ce cas, $\mathscr D$  et $\mathscr P$ n'ont aucun point commun
          \item \textbf{sécante} avec le plan  $\mathscr P$ : dans ce cas, $\mathscr D$  et $\mathscr P$ ont un unique point commun
          \item \textbf{contenue} dans le plan  $\mathscr P$
     \end{itemize}
}
\begin{center}
     \begin{extern}%width="400" alt=""
          \newrgbcolor{ttzzqq}{0.2 0.6 0.}
          \newrgbcolor{qqwwtt}{0. 0.4 0.2}
          \psset{xunit=1.0cm,yunit=1.0cm,algebraic=true,dimen=middle,dotstyle=o,dotsize=5pt 0,linewidth=1.6pt,arrowsize=3pt 2,arrowinset=0.25}
          \begin{pspicture*}(1.5,5)(11.5,8.3)
               \pspolygon[linewidth=0.8pt,linecolor=ttzzqq,fillcolor=ttzzqq,fillstyle=solid,opacity=0.1](4.,6.)(2.,5.)(9.,5.)(11.,6.)
               \psline[linewidth=0.8pt,linecolor=ttzzqq](4.,6.)(2.,5.)
               \psline[linewidth=0.8pt,linecolor=ttzzqq](2.,5.)(9.,5.)
               \psline[linewidth=0.8pt,linecolor=ttzzqq](9.,5.)(11.,6.)
               \psline[linewidth=0.8pt,linecolor=ttzzqq](11.,6.)(4.,6.)
               \rput[tl](3.0,5.4){$\qqwwtt{\mathscr{P}}$}
               \rput[tl](2.25,7.5){$\red{\mathscr{D}}$}
               \psplot[linewidth=0.8pt,linecolor=red]{1.75}{11.2}{(--56.-0.*x)/8.}
          \end{pspicture*}
     \end{extern}
\end{center}
\begin{center}
     \textit{Droite strictement parallèle à un plan}
\end{center}
\begin{vbloc}
     \begin{center}
          \begin{extern}%width="400" alt=""
               \newrgbcolor{ttzzqq}{0.2 0.6 0.}
               \newrgbcolor{qqwwtt}{0. 0.4 0.2}
               \newrgbcolor{ttttff}{0.2 0.2 1.}
               \newrgbcolor{ffewdf}{1. 0.9 0.9}
               \newrgbcolor{qqqqcc}{0. 0. 0.8}
               \psset{xunit=1.0cm,yunit=1.0cm,algebraic=true,dimen=middle,dotstyle=o,dotsize=5pt 0,linewidth=1.6pt,arrowsize=3pt 2,arrowinset=0.25}
               \begin{pspicture*}(1.5,3.3)(11.5,8.5)
                    \pspolygon[linewidth=0.8pt,linecolor=ttzzqq,fillcolor=ttzzqq,fillstyle=solid,opacity=0.1](4.,6.)(2.,5.)(9.,5.)(11.,6.)
                    \psline[linewidth=0.8pt,linecolor=ttzzqq](4.,6.)(2.,5.)
                    \psline[linewidth=0.8pt,linecolor=ttzzqq](2.,5.)(9.,5.)
                    \psline[linewidth=0.8pt,linecolor=ttzzqq](9.,5.)(11.,6.)
                    \psline[linewidth=0.8pt,linecolor=ttzzqq](11.,6.)(4.,6.)
                    \rput[tl](3,5.4){$\qqwwtt{\mathscr{P}}$}
                    \rput[tl](7.5,8){$\red{\mathscr{D}}$}
                    \psplot[linewidth=0.8pt,linecolor=red]{1.5}{11.2}{(-7.2--2.17*x)/1.1}
                    \psline[linewidth=0.8pt,linecolor=ffewdf](6.15,5.555)(5.87,5.)
                    \rput[tl](5.9,5.9){$\qqqqcc{I}$}
                    \begin{scriptsize}
                         \psdots[dotsize=1pt 0,dotstyle=*,linecolor=ttttff](6.15,5.55)
                    \end{scriptsize}
               \end{pspicture*}
          \end{extern}
     \end{center}
     \begin{center}
          \textit{Droite sécante à un plan}
     \end{center}
\end{vbloc}
\begin{center}
     \begin{extern}%width="400" alt=""
          \newrgbcolor{ttzzqq}{0.2 0.6 0.}
          \newrgbcolor{qqwwtt}{0. 0.4 0.2}
          \psset{xunit=1.0cm,yunit=1.0cm,algebraic=true,dimen=middle,dotstyle=o,dotsize=5pt 0,linewidth=1.6pt,arrowsize=3pt 2,arrowinset=0.25}
          \begin{pspicture*}(2.,5)(12.,7.)
               \pspolygon[linewidth=0.8pt,linecolor=ttzzqq,fillcolor=ttzzqq,fillstyle=solid,opacity=0.1](4.,6.)(2.,5.)(9.,5.)(11.,6.)
               \psline[linewidth=0.8pt,linecolor=ttzzqq](4.,6.)(2.,5.)
               \psline[linewidth=0.8pt,linecolor=ttzzqq](2.,5.)(9.,5.)
               \psline[linewidth=0.8pt,linecolor=ttzzqq](9.,5.)(11.,6.)
               \psline[linewidth=0.8pt,linecolor=ttzzqq](11.,6.)(4.,6.)
               \rput[tl](3.,5.4){$\qqwwtt{\mathscr{P}}$}
               \rput[tl](6.405,5.4){$\red{\mathscr{D}}$}
               \psline[linewidth=0.8pt,linecolor=red](4.77,5.)(8.4,6.)
          \end{pspicture*}
     \end{extern}
\end{center}
\begin{center}
     \textit{Droite contenue (incluse) dans un plan}
\end{center}
\cadre{bleu}{Positions relatives de deux droites}{%
     Soient $\mathscr D$ et $\mathscr D^{\prime}$ deux droites distinctes de l'espace.
     \par
     Ces droites peuvent être :
     \begin{itemize}
          \item \textbf{strictement parallèles} : dans ce cas, elles n'ont aucun point commun
          \item \textbf{sécantes} : dans ce cas, leur intersection est un point
          \item \textbf{non coplanaires} : dans ce cas, elles n'ont aucun point commun
     \end{itemize}
}
\begin{center}
     \begin{extern}%width="400" alt=""
          \newrgbcolor{ttzzqq}{0.2 0.6 0.}
          \newrgbcolor{wwqqzz}{0.4 0. 0.6}
          \psset{xunit=1.0cm,yunit=1.0cm,algebraic=true,dimen=middle,dotstyle=o,dotsize=5pt 0,linewidth=1.6pt,arrowsize=3pt 2,arrowinset=0.25}
          \begin{pspicture*}(1.3,4.9)(12.,6.6)
               \pspolygon[linewidth=0.8pt,linecolor=ttzzqq,fillcolor=ttzzqq,fillstyle=solid,opacity=0.1](4.,6.)(2.,5.)(9.,5.)(11.,6.)
               \psline[linewidth=0.8pt,linecolor=ttzzqq](4.,6.)(2.,5.)
               \psline[linewidth=0.8pt,linecolor=ttzzqq](2.,5.)(9.,5.)
               \psline[linewidth=0.8pt,linecolor=ttzzqq](9.,5.)(11.,6.)
               \psline[linewidth=0.8pt,linecolor=ttzzqq](11.,6.)(4.,6.)
               \psline[linewidth=0.8pt,linecolor=red](3.8,5.)(7.5,6.)
               \rput[tl](4.8,5.7){$\red{\mathscr{D}}$}
               \rput[tl](8.0,5.5){$\wwqqzz{\mathscr{D}\ '}$}
               \psline[linewidth=0.8pt,linecolor=wwqqzz](6.0,5.)(9.75,6.)
          \end{pspicture*}
     \end{extern}
\end{center}
\begin{center}
     \textit{Droites strictement parallèles}
\end{center}
\begin{vbloc}
     \begin{center}
          \begin{extern}%width="400" alt=""
               \newrgbcolor{ttzzqq}{0.2 0.6 0.}
               \newrgbcolor{wwqqzz}{0.4 0. 0.6}
               \newrgbcolor{qqttcc}{0. 0.2 0.8}
               \psset{xunit=1.0cm,yunit=1.0cm,algebraic=true,dimen=middle,dotstyle=o,dotsize=5pt 0,linewidth=1.6pt,arrowsize=3pt 2,arrowinset=0.25}
               \begin{pspicture*}(1.3,4.7)(12.,6.6)
                    \pspolygon[linewidth=0.8pt,linecolor=ttzzqq,fillcolor=ttzzqq,fillstyle=solid,opacity=0.1](4.,6.)(2.,5.)(9.,5.)(11.,6.)
                    \psline[linewidth=0.8pt,linecolor=ttzzqq](4.,6.)(2.,5.)
                    \psline[linewidth=0.8pt,linecolor=ttzzqq](2.,5.)(9.,5.)
                    \psline[linewidth=0.8pt,linecolor=ttzzqq](9.,5.)(11.,6.)
                    \psline[linewidth=0.8pt,linecolor=ttzzqq](11.,6.)(4.,6.)
                    \psline[linewidth=0.8pt,linecolor=red](3.80,5.)(7.527,6.)
                    \rput[tl](4.846,5.7){$\red{\mathscr{D}}$}
                    \rput[tl](6.66,5.47){$\wwqqzz{\mathscr{D}\ '}$}
                    \psline[linewidth=0.8pt,linecolor=wwqqzz](5.36,6.)(6.758,5.)
                    \rput[tl](5.96,5.9){$\qqttcc{I}$}
                    \begin{scriptsize}
                    \end{scriptsize}
               \end{pspicture*}
          \end{extern}
     \end{center}
     \begin{center}
          \textit{Droites sécantes}
     \end{center}
\end{vbloc}
\begin{center}
     \begin{extern}%width="400" alt=""
          \newrgbcolor{ttzzqq}{0.2 0.6 0.}
          \newrgbcolor{xfqqff}{0.5 0. 1.}
          \newrgbcolor{wwqqzz}{0.4 0. 0.6}
          \psset{xunit=1.0cm,yunit=1.0cm,algebraic=true,dimen=middle,dotstyle=o,dotsize=5pt 0,linewidth=1.6pt,arrowsize=3pt 2,arrowinset=0.25}
          \begin{pspicture*}(1.5,3)(12.,6.5)
               \pspolygon[linewidth=0.8pt,linecolor=ttzzqq,fillcolor=ttzzqq,fillstyle=solid,opacity=0.1](4.,6.)(2.,5.)(9.,5.)(11.,6.)
               \pspolygon[linewidth=0.8pt,linecolor=blue,fillcolor=blue,fillstyle=solid,opacity=0.1](4.,4.)(2.,3.)(9.,3.)(11.,4.)
               \psline[linewidth=0.8pt,linecolor=ttzzqq](4.,6.)(2.,5.)
               \psline[linewidth=0.8pt,linecolor=ttzzqq](2.,5.)(9.,5.)
               \psline[linewidth=0.8pt,linecolor=ttzzqq](9.,5.)(11.,6.)
               \psline[linewidth=0.8pt,linecolor=ttzzqq](11.,6.)(4.,6.)
               \psline[linewidth=0.8pt,linecolor=blue](4.,4.)(2.,3.)
               \psline[linewidth=0.8pt,linecolor=blue](2.,3.)(9.,3.)
               \psline[linewidth=0.8pt,linecolor=blue](9.,3.)(11.,4.)
               \psline[linewidth=0.8pt,linecolor=blue](11.,4.)(4.,4.)
               \psline[linewidth=0.8pt,linecolor=red](3.8,5.)(7.5,6.)
               \psline[linewidth=0.8pt,linecolor=xfqqff](6.0,4.)(7.2,3.)
               \rput[tl](5,5.8){$\red{\mathscr{D}}$}
               \rput[tl](6.9,3.8){$\wwqqzz{\mathscr{D}\ '}$}
          \end{pspicture*}
     \end{extern}
\end{center}
\begin{center}
     \textit{Droites non coplanaires}
\end{center}
\bloc{cyan}{Remarque}{%
     Dans les deux premiers cas, les deux droites $\mathscr D$ et $\mathscr D^{\prime}$ appartiennent à un même plan $\mathscr P$; elles sont \textbf{coplanaires}.
}
\begin{h2}2. Solides et volumes\end{h2}
\cadre{bleu}{Définition}{%
     Un \textbf{prisme droit} est un solide ayant deux bases polygonales identiques et dont les faces latérales  sont des rectangles.
}
\begin{center}
     \begin{extern}%width="250" alt="prisme droit"
          \resizebox{7cm}{!}{
               \psset{xunit=1.0cm,yunit=1.0cm,algebraic=true,dimen=middle,dotstyle=o,dotsize=5pt 0,linewidth=1.pt,arrowsize=3pt 2,arrowinset=0.25}
               \begin{pspicture*}(-1.,0.)(9.,10.)
                    \pspolygon[linecolor=blue,fillcolor=blue,fillstyle=solid,opacity=0.1](1.,8.)(3.,6.5)(7.,7.)(8.,8.)(4.,9.)
                    \pspolygon[linecolor=blue,fillcolor=blue,fillstyle=solid,opacity=0.1](1.,2.)(3.,0.5)(7.,1.)(8.,2.)(4.,3.)
                    \psline(1.,8.)(1.,2.)
                    \psline(3.,6.5)(3.,0.5)
                    \psline(7.,7.)(7.,1.)
                    \psline(8.,8.)(8.,2.)
                    \psline[linestyle=dashed,dash=2pt 2pt](4.,9.)(4.,3.)
                    \rput(4.,2.){$\Large \blue{\mathscr{B}}$}
                    \psline{<->}(0.,8.)(0.,2.)
                    \rput(-0.5,5.){$\Large h$}
               \end{pspicture*}
          }
     \end{extern}
\end{center}
\begin{center}
     \textit{Prisme droit de hauteur $h$}
\end{center}
\cadre{vert}{Propriété}{%
     Si on désigne par $h$ la hauteur du prisme et $\mathscr B$ l'aire de la base, le volume du prisme est égal à :
     \begin{center}$V=\mathscr B\times h$\end{center}
}
\bloc{cyan}{Cas particuliers}{%
     \begin{itemize}
          \item Si le volume est un \textbf{pavé droit} (\textit{parallélépipède rectangle}) de dimensions $l, L, h$,  la base est un rectangle de largeur $l$ et de longueur $L$. Le volume vaut alors $V=L\times l\times h$
          \item Si le volume est un \textbf{cube} dont le côté mesure $c$, la base est un carré de côté $c$. Le volume vaut alors $V=c^3$
     \end{itemize}
}
\cadre{vert}{Propriété}{%
     Le volume d'un cylindre de révolution est égal à :
     \begin{center}$V=\mathscr B\times h$\end{center}
     où
     \begin{itemize}
          \item $h$ est la hauteur du cylindre de révolution
          \item $\mathscr B = \pi R^2$ est l'aire de la base de rayon $R$.
     \end{itemize}
}
\begin{center}
     \begin{extern}%width="250" alt="cylindre de révolution"
          \resizebox{7cm}{!}{
               \psset{xunit=1.0cm,yunit=1.0cm,algebraic=true,dimen=middle,dotstyle=o,dotsize=5pt 0,linewidth=1.pt,arrowsize=3pt 2,arrowinset=0.25}
               \begin{pspicture*}(-1.,1.)(9.,10.)
                    \psline(1.,8.)(1.,2.)
                    \psline(7.,8.)(7.,2.)
                    \rput{-0.172}(4,1.94){\psellipse[linecolor=blue,fillcolor=blue,fillstyle=solid,opacity=0.1](0,0)(3.,0.59)}
                    \rput{-0.172}(4,7.94){\psellipse[linecolor=blue,fillcolor=blue,fillstyle=solid,opacity=0.1](0,0)(3.,0.59)}
                    \rput(4.,2.){$\Large \blue{\mathscr{B}}$}
                    \psline{<->}(0.,8.)(0.,2.)
                    \rput(-0.5,5.){$\Large h$}
               \end{pspicture*}
          }
     \end{extern}
\end{center}
\begin{center}
     \textit{Cylindre de révolution de hauteur $h$}
\end{center}
\cadre{bleu}{Définition}{%
     Une \textbf{pyramide} est un solide ayant une base polygonale, un sommet et dont les faces latérales sont des triangles.
}
\begin{center}
     \begin{extern}%width="250" alt="pyramide"
          \resizebox{7cm}{!}{
               \psset{xunit=1.0cm,yunit=1.0cm,algebraic=true,dimen=middle,dotstyle=o,dotsize=5pt 0,linewidth=1.pt,arrowsize=3pt 2,arrowinset=0.25}
               \begin{pspicture*}(-1.,0.5)(9.,10.)
                    \pspolygon[linecolor=blue,fillcolor=blue,fillstyle=solid,opacity=0.1](1.,2.)(3.,0.5)(7.,1.)(8.,2.)(4.,3.)
                    \psline(4.,8.)(1.,2.)
                    \psline(4.,8.)(3.,0.5)
                    \psline(4.,8.)(7.,1.)
                    \psline(4.,8.)(8.,2.)
                    \psline[linestyle=dashed,dash=2pt 2pt](4.,8.)(4.,3.)
                    \rput(4.,2.){$\Large \blue{\mathscr{B}}$}
                    \psline{<->}(0.,8.)(0.,2.)
                    \rput(-0.5,5.){$\Large h$}
               \end{pspicture*}
          }
     \end{extern}
\end{center}
\begin{center}
     \textit{Pyramide de hauteur $h$}
\end{center}
\cadre{vert}{Propriété}{%
     Si on désigne par $h$ la hauteur de la pyramide et $\mathscr B$ l'aire de la base, le volume de la pyramide est égal à :
     \begin{center}$V=\frac{1}{3}\times \mathscr B\times h$\end{center}
}
\cadre{vert}{Propriété}{%
     Le volume d'un cône de révolution est égal à :
     \begin{center}$V=\frac{1}{3}\times \mathscr B\times h$\end{center}
     où
     \begin{itemize}
          \item $h$ est la hauteur du cône
          \item $\mathscr B = \pi R^2$ est l'aire de la base de rayon $R$.
     \end{itemize}
}
\begin{center}
     \begin{extern}%width="250" alt="cône de révolution"
          \resizebox{7cm}{!}{
               \psset{xunit=1.0cm,yunit=1.0cm,algebraic=true,dimen=middle,dotstyle=o,dotsize=5pt 0,linewidth=1.pt,arrowsize=3pt 2,arrowinset=0.25}
               \begin{pspicture*}(-1.,1)(9.,9.)
                    \psline(4.,8.)(1.,2.)
                    \psline(4.,8.)(7.,2.)
                    \rput{-0.172}(4,1.94){\psellipse[linecolor=blue,fillcolor=blue,fillstyle=solid,opacity=0.1](0,0)(3.,0.59)}
                    \rput(4.,2.){$\Large \blue{\mathscr{B}}$}
                    \psline{<->}(0.,8.)(0.,2.)
                    \rput(-0.5,5.){$\Large h$}
               \end{pspicture*}
          }
     \end{extern}
\end{center}
\begin{center}
     \textit{Cône de révolution de hauteur $h$}
\end{center}
\cadre{vert}{Propriété}{%
     Le volume d'une sphère de rayon $r$ est égal à :
     \begin{center}$V = \frac{4}{3}\times \pi \times r^3$ \end{center}
}
\begin{center}
     \begin{extern}%width="220" alt="sphère"
          \resizebox{7cm}{!}{
               \psset{xunit=1.0cm,yunit=1.0cm,algebraic=true,dimen=middle,dotstyle=o,dotsize=5pt 0,linewidth=1.pt,arrowsize=3pt 2,arrowinset=0.25}
               \begin{pspicture*}(0.,-1.2)(8.5,5.)
                    \newrgbcolor{lblue}{0.9 0.9 1}
                    \pscircle[linecolor=blue,GradientPos={(3,3.5)},fillstyle=gradient,gradmidpoint=0.01,GradientCircle=true, gradbegin=lblue, gradend=white](4.,2.){3}
                    \psline[linecolor=red](4.,2.)(4.28,1.36)
                    \rput[l](4.3,1.7){\red{$r$}}
                    \rput{-0.172}(4,1.94){\psellipse[linecolor=blue,fillcolor=blue,fillstyle=solid,opacity=0.05](0,0)(3.,0.59)}
                    \psdots[dotsize=3pt,dotstyle=*,linecolor=blue](4,2)
                    \rput(3.8,2.2){$\Large \blue{O}$}
               \end{pspicture*}
          }
     \end{extern}
\end{center}
\begin{center}
     \textit{Sphère de rayon $r$}
\end{center}

\end{document}
µ
\documentclass[a4paper]{article}

%================================================================================================================================
%
% Packages
%
%================================================================================================================================

\usepackage[T1]{fontenc} 	% pour caractères accentués
\usepackage[utf8]{inputenc}  % encodage utf8
\usepackage[french]{babel}	% langue : français
\usepackage{fourier}			% caractères plus lisibles
\usepackage[dvipsnames]{xcolor} % couleurs
\usepackage{fancyhdr}		% réglage header footer
\usepackage{needspace}		% empêcher sauts de page mal placés
\usepackage{graphicx}		% pour inclure des graphiques
\usepackage{enumitem,cprotect}		% personnalise les listes d'items (nécessaire pour ol, al ...)
\usepackage{hyperref}		% Liens hypertexte
\usepackage{pstricks,pst-all,pst-node,pstricks-add,pst-math,pst-plot,pst-tree,pst-eucl} % pstricks
\usepackage[a4paper,includeheadfoot,top=2cm,left=3cm, bottom=2cm,right=3cm]{geometry} % marges etc.
\usepackage{comment}			% commentaires multilignes
\usepackage{amsmath,environ} % maths (matrices, etc.)
\usepackage{amssymb,makeidx}
\usepackage{bm}				% bold maths
\usepackage{tabularx}		% tableaux
\usepackage{colortbl}		% tableaux en couleur
\usepackage{fontawesome}		% Fontawesome
\usepackage{environ}			% environment with command
\usepackage{fp}				% calculs pour ps-tricks
\usepackage{multido}			% pour ps tricks
\usepackage[np]{numprint}	% formattage nombre
\usepackage{tikz,tkz-tab} 			% package principal TikZ
\usepackage{pgfplots}   % axes
\usepackage{mathrsfs}    % cursives
\usepackage{calc}			% calcul taille boites
\usepackage[scaled=0.875]{helvet} % font sans serif
\usepackage{svg} % svg
\usepackage{scrextend} % local margin
\usepackage{scratch} %scratch
\usepackage{multicol} % colonnes
%\usepackage{infix-RPN,pst-func} % formule en notation polanaise inversée
\usepackage{listings}

%================================================================================================================================
%
% Réglages de base
%
%================================================================================================================================

\lstset{
language=Python,   % R code
literate=
{á}{{\'a}}1
{à}{{\`a}}1
{ã}{{\~a}}1
{é}{{\'e}}1
{è}{{\`e}}1
{ê}{{\^e}}1
{í}{{\'i}}1
{ó}{{\'o}}1
{õ}{{\~o}}1
{ú}{{\'u}}1
{ü}{{\"u}}1
{ç}{{\c{c}}}1
{~}{{ }}1
}


\definecolor{codegreen}{rgb}{0,0.6,0}
\definecolor{codegray}{rgb}{0.5,0.5,0.5}
\definecolor{codepurple}{rgb}{0.58,0,0.82}
\definecolor{backcolour}{rgb}{0.95,0.95,0.92}

\lstdefinestyle{mystyle}{
    backgroundcolor=\color{backcolour},   
    commentstyle=\color{codegreen},
    keywordstyle=\color{magenta},
    numberstyle=\tiny\color{codegray},
    stringstyle=\color{codepurple},
    basicstyle=\ttfamily\footnotesize,
    breakatwhitespace=false,         
    breaklines=true,                 
    captionpos=b,                    
    keepspaces=true,                 
    numbers=left,                    
xleftmargin=2em,
framexleftmargin=2em,            
    showspaces=false,                
    showstringspaces=false,
    showtabs=false,                  
    tabsize=2,
    upquote=true
}

\lstset{style=mystyle}


\lstset{style=mystyle}
\newcommand{\imgdir}{C:/laragon/www/newmc/assets/imgsvg/}
\newcommand{\imgsvgdir}{C:/laragon/www/newmc/assets/imgsvg/}

\definecolor{mcgris}{RGB}{220, 220, 220}% ancien~; pour compatibilité
\definecolor{mcbleu}{RGB}{52, 152, 219}
\definecolor{mcvert}{RGB}{125, 194, 70}
\definecolor{mcmauve}{RGB}{154, 0, 215}
\definecolor{mcorange}{RGB}{255, 96, 0}
\definecolor{mcturquoise}{RGB}{0, 153, 153}
\definecolor{mcrouge}{RGB}{255, 0, 0}
\definecolor{mclightvert}{RGB}{205, 234, 190}

\definecolor{gris}{RGB}{220, 220, 220}
\definecolor{bleu}{RGB}{52, 152, 219}
\definecolor{vert}{RGB}{125, 194, 70}
\definecolor{mauve}{RGB}{154, 0, 215}
\definecolor{orange}{RGB}{255, 96, 0}
\definecolor{turquoise}{RGB}{0, 153, 153}
\definecolor{rouge}{RGB}{255, 0, 0}
\definecolor{lightvert}{RGB}{205, 234, 190}
\setitemize[0]{label=\color{lightvert}  $\bullet$}

\pagestyle{fancy}
\renewcommand{\headrulewidth}{0.2pt}
\fancyhead[L]{maths-cours.fr}
\fancyhead[R]{\thepage}
\renewcommand{\footrulewidth}{0.2pt}
\fancyfoot[C]{}

\newcolumntype{C}{>{\centering\arraybackslash}X}
\newcolumntype{s}{>{\hsize=.35\hsize\arraybackslash}X}

\setlength{\parindent}{0pt}		 
\setlength{\parskip}{3mm}
\setlength{\headheight}{1cm}

\def\ebook{ebook}
\def\book{book}
\def\web{web}
\def\type{web}

\newcommand{\vect}[1]{\overrightarrow{\,\mathstrut#1\,}}

\def\Oij{$\left(\text{O}~;~\vect{\imath},~\vect{\jmath}\right)$}
\def\Oijk{$\left(\text{O}~;~\vect{\imath},~\vect{\jmath},~\vect{k}\right)$}
\def\Ouv{$\left(\text{O}~;~\vect{u},~\vect{v}\right)$}

\hypersetup{breaklinks=true, colorlinks = true, linkcolor = OliveGreen, urlcolor = OliveGreen, citecolor = OliveGreen, pdfauthor={Didier BONNEL - https://www.maths-cours.fr} } % supprime les bordures autour des liens

\renewcommand{\arg}[0]{\text{arg}}

\everymath{\displaystyle}

%================================================================================================================================
%
% Macros - Commandes
%
%================================================================================================================================

\newcommand\meta[2]{    			% Utilisé pour créer le post HTML.
	\def\titre{titre}
	\def\url{url}
	\def\arg{#1}
	\ifx\titre\arg
		\newcommand\maintitle{#2}
		\fancyhead[L]{#2}
		{\Large\sffamily \MakeUppercase{#2}}
		\vspace{1mm}\textcolor{mcvert}{\hrule}
	\fi 
	\ifx\url\arg
		\fancyfoot[L]{\href{https://www.maths-cours.fr#2}{\black \footnotesize{https://www.maths-cours.fr#2}}}
	\fi 
}


\newcommand\TitreC[1]{    		% Titre centré
     \needspace{3\baselineskip}
     \begin{center}\textbf{#1}\end{center}
}

\newcommand\newpar{    		% paragraphe
     \par
}

\newcommand\nosp {    		% commande vide (pas d'espace)
}
\newcommand{\id}[1]{} %ignore

\newcommand\boite[2]{				% Boite simple sans titre
	\vspace{5mm}
	\setlength{\fboxrule}{0.2mm}
	\setlength{\fboxsep}{5mm}	
	\fcolorbox{#1}{#1!3}{\makebox[\linewidth-2\fboxrule-2\fboxsep]{
  		\begin{minipage}[t]{\linewidth-2\fboxrule-4\fboxsep}\setlength{\parskip}{3mm}
  			 #2
  		\end{minipage}
	}}
	\vspace{5mm}
}

\newcommand\CBox[4]{				% Boites
	\vspace{5mm}
	\setlength{\fboxrule}{0.2mm}
	\setlength{\fboxsep}{5mm}
	
	\fcolorbox{#1}{#1!3}{\makebox[\linewidth-2\fboxrule-2\fboxsep]{
		\begin{minipage}[t]{1cm}\setlength{\parskip}{3mm}
	  		\textcolor{#1}{\LARGE{#2}}    
 	 	\end{minipage}  
  		\begin{minipage}[t]{\linewidth-2\fboxrule-4\fboxsep}\setlength{\parskip}{3mm}
			\raisebox{1.2mm}{\normalsize\sffamily{\textcolor{#1}{#3}}}						
  			 #4
  		\end{minipage}
	}}
	\vspace{5mm}
}

\newcommand\cadre[3]{				% Boites convertible html
	\par
	\vspace{2mm}
	\setlength{\fboxrule}{0.1mm}
	\setlength{\fboxsep}{5mm}
	\fcolorbox{#1}{white}{\makebox[\linewidth-2\fboxrule-2\fboxsep]{
  		\begin{minipage}[t]{\linewidth-2\fboxrule-4\fboxsep}\setlength{\parskip}{3mm}
			\raisebox{-2.5mm}{\sffamily \small{\textcolor{#1}{\MakeUppercase{#2}}}}		
			\par		
  			 #3
 	 		\end{minipage}
	}}
		\vspace{2mm}
	\par
}

\newcommand\bloc[3]{				% Boites convertible html sans bordure
     \needspace{2\baselineskip}
     {\sffamily \small{\textcolor{#1}{\MakeUppercase{#2}}}}    
		\par		
  			 #3
		\par
}

\newcommand\CHelp[1]{
     \CBox{Plum}{\faInfoCircle}{À RETENIR}{#1}
}

\newcommand\CUp[1]{
     \CBox{NavyBlue}{\faThumbsOUp}{EN PRATIQUE}{#1}
}

\newcommand\CInfo[1]{
     \CBox{Sepia}{\faArrowCircleRight}{REMARQUE}{#1}
}

\newcommand\CRedac[1]{
     \CBox{PineGreen}{\faEdit}{BIEN R\'EDIGER}{#1}
}

\newcommand\CError[1]{
     \CBox{Red}{\faExclamationTriangle}{ATTENTION}{#1}
}

\newcommand\TitreExo[2]{
\needspace{4\baselineskip}
 {\sffamily\large EXERCICE #1\ (\emph{#2 points})}
\vspace{5mm}
}

\newcommand\img[2]{
          \includegraphics[width=#2\paperwidth]{\imgdir#1}
}

\newcommand\imgsvg[2]{
       \begin{center}   \includegraphics[width=#2\paperwidth]{\imgsvgdir#1} \end{center}
}


\newcommand\Lien[2]{
     \href{#1}{#2 \tiny \faExternalLink}
}
\newcommand\mcLien[2]{
     \href{https~://www.maths-cours.fr/#1}{#2 \tiny \faExternalLink}
}

\newcommand{\euro}{\eurologo{}}

%================================================================================================================================
%
% Macros - Environement
%
%================================================================================================================================

\newenvironment{tex}{ %
}
{%
}

\newenvironment{indente}{ %
	\setlength\parindent{10mm}
}

{
	\setlength\parindent{0mm}
}

\newenvironment{corrige}{%
     \needspace{3\baselineskip}
     \medskip
     \textbf{\textsc{Corrigé}}
     \medskip
}
{
}

\newenvironment{extern}{%
     \begin{center}
     }
     {
     \end{center}
}

\NewEnviron{code}{%
	\par
     \boite{gray}{\texttt{%
     \BODY
     }}
     \par
}

\newenvironment{vbloc}{% boite sans cadre empeche saut de page
     \begin{minipage}[t]{\linewidth}
     }
     {
     \end{minipage}
}
\NewEnviron{h2}{%
    \needspace{3\baselineskip}
    \vspace{0.6cm}
	\noindent \MakeUppercase{\sffamily \large \BODY}
	\vspace{1mm}\textcolor{mcgris}{\hrule}\vspace{0.4cm}
	\par
}{}

\NewEnviron{h3}{%
    \needspace{3\baselineskip}
	\vspace{5mm}
	\textsc{\BODY}
	\par
}

\NewEnviron{margeneg}{ %
\begin{addmargin}[-1cm]{0cm}
\BODY
\end{addmargin}
}

\NewEnviron{html}{%
}

\begin{document}
\meta{url}{/cours/algorithmes/}
\meta{pid}{230}
\meta{titre}{Algorithmes : Présentation}
\meta{type}{cours}
\begin{h2}1. Notion d'algorithme\end{h2}
\cadre{bleu}{Définition}{%
     Un \textbf{algorithme} est une suite d'instructions détaillées qui, si elles sont correctement exécutées, conduit à un résultat donné.
}
\bloc{orange}{Exemples}{%
     \begin{itemize}
          \item une recette de cuisine, une notice de montage peuvent être considérées comme des algorithmes.
          \item la suite d'instructions suivantes :
          \begin{code}1. choisir un nombre entier\\
               2. le multiplier par lui-même\\
               3. énoncer le résultat obtenu
          \end{code}
          est un algorithme permettant d'obtenir le carré d'un nombre entier.
     \end{itemize}
}
\bloc{cyan}{Remarque}{%
     Dans la définition précédente, "détaillées" signifie que les instructions sont suffisamment précises pour pouvoir être mises en oeuvre correctement par l'exécutant (homme ou machine)
}
\begin{h2}2. Pseudo-code\end{h2}
Les instructions doivent être formulées dans un langage compréhensible par l'exécutant. Dans le cas d'un humain, il s'agira du langage courant (langue maternelle), ; dans le cas d'une machine, il faudra recourir à un langage de \textbf{programmation} (assembleur, basic, C, java, php ...).
\par
En algorithmique, nous utiliserons un langage situé à mi-chemin entre le langage courant et un langage de programmation appelé \textit{pseudo-code}. Il n'y a pas de norme concernant ce pseudo-code qui peut varier légèrement d'un enseignant à l'autre. Le but est surtout de mettre l'accent sur la logique de l'algorithme. L'avantage du pseudo-code est qu'il permet de rester proche d'un langage informatique sans qu'il soit nécessaire de connaître toutes les règles et spécificités d'un langage particulier.
\begin{h2}3. Les variables\end{h2}
Un algorithme (ou un programme informatique), agit sur des nombres, des textes, ... Ces différents éléments sont stockés dans des \textbf{variables}. On peut se représenter une variable comme une boîte portant une étiquette ("le nom de la variable") à l'intérieur de laquelle on peut placer un contenu.
\begin{center}
     \img{var-5}{0.15}%width="119" alt="contenu d'une variable"
\end{center}
En informatique, les variables sont des emplacements réservés dans la mémoire de l'ordinateur auxquels on attribue une étiquette.
\cadre{bleu}{Définition}{%
     \textbf{Déclarer une variable}  c'est indiquer le nom et le type (nombre texte, tableau,...) d'une variable que l'on utilisera dans l'algorithme.
     \par
     La déclaration des variables se fait au début de l'algorithme avant la première instruction.
}
\bloc{cyan}{Remarques}{%
     \begin{itemize}
          \item Pour reprendre l'image précédente, déclarer une variable consiste à "créer la boîte"
          \item Les principaux types de variables que nous utiliserons seront : entier, nombre (=réel), texte (=chaîne de caractères), tableau de nombres ou de textes, logique (=booléen -cf chapitre suivant)
          \item Lorsqu'on déclare une variable dans un programme informatique, l'ordinateur affecte une étiquette à une zone de mémoire et éventuellement réserve de l'espace pour le contenu de cette variable en fonction de son type.
     \end{itemize}
}
\bloc{orange}{Exemple}{%
     Dans notre pseudo-code, nous déclarerons les variables de la façon suivante :
     \begin{code}variables\\
          \hspace*{15px}  x : nombre\\
          \hspace*{15px}  y : texte\\
          \hspace*{15px}  a, b, c : entiers
     \end{code}
     (Dans l'exemple précédent on définit 5 variables : x du type nombre (réel), y du type texte, et a, b et c de type entier.)
     \par
     Nous distinguerons  la déclaration des variables en plaçant le reste de l'algorithme entre les instructions "début algorithme" et "fin algorithme".
}
\cadre{bleu}{Définition}{%
     \textbf{Affecter une variable}, c'est attribuer une valeur à cette variable. Si la variable contenait déjà une valeur, cette ancienne valeur est effacée.
}
\bloc{cyan}{Remarques}{%
     \begin{itemize}
          \item Affecter une variable revient à "remplir la boîte"
          <img src="/wp-content/uploads/affect-var5.png" alt="" class="aligncenter size-full  img-pc" />
          \item Dans notre pseudo-code, nous utiliserons l'expression \textit{«prend la valeur»} pour l'affectation. Voici la déclaration et l'affectation de la variable x :
          \begin{code}variables\\
               \hspace*{15px}  x : entier \\
               début algorithme \\
               \hspace*{15px}  x prend la valeur 5\\
               fin algorithme
          \end{code}
          \item on ne peut affecter à une variable qu'une valeur du type qui a été défini lors de la déclaration. Le code suivant est incorrect (le // indique le début d'un commentaire):
          \begin{code}variables\\
               \hspace*{15px}  x : entier\\
               début algorithme \\
               \hspace*{15px}  x prend la valeur "bonjour" // Erreur! x est de type entier !\\
               fin algorithme
          \end{code}
          \item Les textes (ou chaînes de caractères) doivent être entourés d'apostrophes afin de ne pas être confondus avec des noms de variables.
     \end{itemize}
}
Il est possible d'affecter à une variable le contenu d'une autre variable ou le résultat d'un calcul. Le contenu de l'autre variable n'est alors pas modifié. Par exemple :
\begin{code}variables \\
     \hspace*{15px}  x, y, z : entiers\\
     début algorithme   \\
     \hspace*{15px}  x prend la valeur 5\\
     \hspace*{15px}  y prend la valeur x\\
     \hspace*{15px}  z prend la valeur x+y+1\\
     fin algorithme
\end{code}
A la fin de cet algorithme, x et y contiennent la valeur 5 et z la valeur 11 ( = 5+5+1).
\begin{h2}4. Les instructions d'entrée-sortie\end{h2}
Faire effectuer un calcul à une machine c'est bien... Mais il faut au moins être capable d'entrer des valeurs et il faut aussi que la machine puisse afficher un résultat !
\par
Les instructions qui permettent de "dialoguer" avec une machine s'appellent les instructions "\textbf{d'entrée/sortie}" ou de "\textbf{lecture/écriture}"
\bloc{orange}{Lecture}{%
     Dans notre pseudo-code nous utiliserons l'instruction \textbf{\textit{lire}} (ou \textit{entrer}, ou \textit{saisir}, etc.) \textbf{suivie du nom d'une variable } pour pouvoir saisir une valeur (en anglais cette instruction se nomme généralement \textit{input}).
     \par
     Lorsqu'elle rencontre une telle instruction, la machine s'\textbf{arrête} et \textbf{attend que l'utilisateur entre une valeur}. Une fois la valeur saisie, la machine\textbf{ affecte la valeur saisie} à la variable dont le nom figure après \textit{lire}. Ensuite, elle passe à l'\textbf{instruction suivante}.
     \begin{code}variables \\
          \hspace*{15px}   x : entier\\
          début algorithme   \\
          \hspace*{15px}  lire x\\
          \hspace*{15px}  y prend la valeur 2*x\\
          fin algorithme
     \end{code}
     Cet algorithme demande d'\textbf{entrer} un nombre entier, \textbf{stocke} la valeur de ce nombre dans la variable x, puis \textbf{calcule le double} du nombre entré et \textbf{affecte ce double à la variable y}.
     \par
     Le résultat n'est pas affiché (d'où le paragraphe suivant...)
     \par
     Remarque : Dans un véritable programme, il faudrait vérifier que la valeur entrée est bien du type désiré (ici un entier). Dans un algorithme, on n'écrit pas cette vérification.
}
\bloc{orange}{Ecriture}{%
     Dans notre pseudo-code nous utiliserons l'instruction \textbf{\textit{afficher}} suivie du nom d'une variable ou d'une constante (nombre, texte ...) pour afficher une valeur (on peut également utiliser \textit{\textbf{"écrire"}} ou \textit{\textbf{"print"}} en anglais).
     \par
     Pour afficher un texte on utilise des \textbf{guillemets} (simples ou doubles) :
     \begin{code}
          ...\\
          afficher 'Ce texte sera affiché'\\
          ...
     \end{code}
     Il est fréquent d'afficher un texte pour donner des \textbf{consignes} ou des \textbf{informations} à l'utilisateur.
     \par
     Pour afficher le contenu d'une variable on fait suivre \textit{"afficher"} du nom de la variable \textbf{sans guillemet} :
     \begin{code}...\\
          afficher x\\
          ...
     \end{code}
     L'algorithme suivant calcule puis affiche l'âge qu'aura une personne en 2100 :
     \begin{code}
          variables \\
          \hspace*{15px}  annee, age, age\_en\_2100 : entiers  \\
          début algorithme   \\
          \hspace*{15px}  afficher "Entrer l'année actuelle"\\
          \hspace*{15px}  lire année\\
          \hspace*{15px}  afficher "Entrer votre âge"\\
          \hspace*{15px}  lire âge\\
          \hspace*{15px}  âge\_en\_2100 prend la valeur âge + 2100 - année \\
          \hspace*{15px}  afficher "En 2100, vous aurez ", âge\_en\_2100, " ans."\\
          fin algorithme
     \end{code}
     Si l'utilisateur entre \textit{"2014"} comme année et \textit{"16"} comme âge, l'algorithme affichera~:\\
     \textit{En 2100, vous aurez 102 ans.}
}

\end{document}
µ
\documentclass[a4paper]{article}

%================================================================================================================================
%
% Packages
%
%================================================================================================================================

\usepackage[T1]{fontenc} 	% pour caractères accentués
\usepackage[utf8]{inputenc}  % encodage utf8
\usepackage[french]{babel}	% langue : français
\usepackage{fourier}			% caractères plus lisibles
\usepackage[dvipsnames]{xcolor} % couleurs
\usepackage{fancyhdr}		% réglage header footer
\usepackage{needspace}		% empêcher sauts de page mal placés
\usepackage{graphicx}		% pour inclure des graphiques
\usepackage{enumitem,cprotect}		% personnalise les listes d'items (nécessaire pour ol, al ...)
\usepackage{hyperref}		% Liens hypertexte
\usepackage{pstricks,pst-all,pst-node,pstricks-add,pst-math,pst-plot,pst-tree,pst-eucl} % pstricks
\usepackage[a4paper,includeheadfoot,top=2cm,left=3cm, bottom=2cm,right=3cm]{geometry} % marges etc.
\usepackage{comment}			% commentaires multilignes
\usepackage{amsmath,environ} % maths (matrices, etc.)
\usepackage{amssymb,makeidx}
\usepackage{bm}				% bold maths
\usepackage{tabularx}		% tableaux
\usepackage{colortbl}		% tableaux en couleur
\usepackage{fontawesome}		% Fontawesome
\usepackage{environ}			% environment with command
\usepackage{fp}				% calculs pour ps-tricks
\usepackage{multido}			% pour ps tricks
\usepackage[np]{numprint}	% formattage nombre
\usepackage{tikz,tkz-tab} 			% package principal TikZ
\usepackage{pgfplots}   % axes
\usepackage{mathrsfs}    % cursives
\usepackage{calc}			% calcul taille boites
\usepackage[scaled=0.875]{helvet} % font sans serif
\usepackage{svg} % svg
\usepackage{scrextend} % local margin
\usepackage{scratch} %scratch
\usepackage{multicol} % colonnes
%\usepackage{infix-RPN,pst-func} % formule en notation polanaise inversée
\usepackage{listings}

%================================================================================================================================
%
% Réglages de base
%
%================================================================================================================================

\lstset{
language=Python,   % R code
literate=
{á}{{\'a}}1
{à}{{\`a}}1
{ã}{{\~a}}1
{é}{{\'e}}1
{è}{{\`e}}1
{ê}{{\^e}}1
{í}{{\'i}}1
{ó}{{\'o}}1
{õ}{{\~o}}1
{ú}{{\'u}}1
{ü}{{\"u}}1
{ç}{{\c{c}}}1
{~}{{ }}1
}


\definecolor{codegreen}{rgb}{0,0.6,0}
\definecolor{codegray}{rgb}{0.5,0.5,0.5}
\definecolor{codepurple}{rgb}{0.58,0,0.82}
\definecolor{backcolour}{rgb}{0.95,0.95,0.92}

\lstdefinestyle{mystyle}{
    backgroundcolor=\color{backcolour},   
    commentstyle=\color{codegreen},
    keywordstyle=\color{magenta},
    numberstyle=\tiny\color{codegray},
    stringstyle=\color{codepurple},
    basicstyle=\ttfamily\footnotesize,
    breakatwhitespace=false,         
    breaklines=true,                 
    captionpos=b,                    
    keepspaces=true,                 
    numbers=left,                    
xleftmargin=2em,
framexleftmargin=2em,            
    showspaces=false,                
    showstringspaces=false,
    showtabs=false,                  
    tabsize=2,
    upquote=true
}

\lstset{style=mystyle}


\lstset{style=mystyle}
\newcommand{\imgdir}{C:/laragon/www/newmc/assets/imgsvg/}
\newcommand{\imgsvgdir}{C:/laragon/www/newmc/assets/imgsvg/}

\definecolor{mcgris}{RGB}{220, 220, 220}% ancien~; pour compatibilité
\definecolor{mcbleu}{RGB}{52, 152, 219}
\definecolor{mcvert}{RGB}{125, 194, 70}
\definecolor{mcmauve}{RGB}{154, 0, 215}
\definecolor{mcorange}{RGB}{255, 96, 0}
\definecolor{mcturquoise}{RGB}{0, 153, 153}
\definecolor{mcrouge}{RGB}{255, 0, 0}
\definecolor{mclightvert}{RGB}{205, 234, 190}

\definecolor{gris}{RGB}{220, 220, 220}
\definecolor{bleu}{RGB}{52, 152, 219}
\definecolor{vert}{RGB}{125, 194, 70}
\definecolor{mauve}{RGB}{154, 0, 215}
\definecolor{orange}{RGB}{255, 96, 0}
\definecolor{turquoise}{RGB}{0, 153, 153}
\definecolor{rouge}{RGB}{255, 0, 0}
\definecolor{lightvert}{RGB}{205, 234, 190}
\setitemize[0]{label=\color{lightvert}  $\bullet$}

\pagestyle{fancy}
\renewcommand{\headrulewidth}{0.2pt}
\fancyhead[L]{maths-cours.fr}
\fancyhead[R]{\thepage}
\renewcommand{\footrulewidth}{0.2pt}
\fancyfoot[C]{}

\newcolumntype{C}{>{\centering\arraybackslash}X}
\newcolumntype{s}{>{\hsize=.35\hsize\arraybackslash}X}

\setlength{\parindent}{0pt}		 
\setlength{\parskip}{3mm}
\setlength{\headheight}{1cm}

\def\ebook{ebook}
\def\book{book}
\def\web{web}
\def\type{web}

\newcommand{\vect}[1]{\overrightarrow{\,\mathstrut#1\,}}

\def\Oij{$\left(\text{O}~;~\vect{\imath},~\vect{\jmath}\right)$}
\def\Oijk{$\left(\text{O}~;~\vect{\imath},~\vect{\jmath},~\vect{k}\right)$}
\def\Ouv{$\left(\text{O}~;~\vect{u},~\vect{v}\right)$}

\hypersetup{breaklinks=true, colorlinks = true, linkcolor = OliveGreen, urlcolor = OliveGreen, citecolor = OliveGreen, pdfauthor={Didier BONNEL - https://www.maths-cours.fr} } % supprime les bordures autour des liens

\renewcommand{\arg}[0]{\text{arg}}

\everymath{\displaystyle}

%================================================================================================================================
%
% Macros - Commandes
%
%================================================================================================================================

\newcommand\meta[2]{    			% Utilisé pour créer le post HTML.
	\def\titre{titre}
	\def\url{url}
	\def\arg{#1}
	\ifx\titre\arg
		\newcommand\maintitle{#2}
		\fancyhead[L]{#2}
		{\Large\sffamily \MakeUppercase{#2}}
		\vspace{1mm}\textcolor{mcvert}{\hrule}
	\fi 
	\ifx\url\arg
		\fancyfoot[L]{\href{https://www.maths-cours.fr#2}{\black \footnotesize{https://www.maths-cours.fr#2}}}
	\fi 
}


\newcommand\TitreC[1]{    		% Titre centré
     \needspace{3\baselineskip}
     \begin{center}\textbf{#1}\end{center}
}

\newcommand\newpar{    		% paragraphe
     \par
}

\newcommand\nosp {    		% commande vide (pas d'espace)
}
\newcommand{\id}[1]{} %ignore

\newcommand\boite[2]{				% Boite simple sans titre
	\vspace{5mm}
	\setlength{\fboxrule}{0.2mm}
	\setlength{\fboxsep}{5mm}	
	\fcolorbox{#1}{#1!3}{\makebox[\linewidth-2\fboxrule-2\fboxsep]{
  		\begin{minipage}[t]{\linewidth-2\fboxrule-4\fboxsep}\setlength{\parskip}{3mm}
  			 #2
  		\end{minipage}
	}}
	\vspace{5mm}
}

\newcommand\CBox[4]{				% Boites
	\vspace{5mm}
	\setlength{\fboxrule}{0.2mm}
	\setlength{\fboxsep}{5mm}
	
	\fcolorbox{#1}{#1!3}{\makebox[\linewidth-2\fboxrule-2\fboxsep]{
		\begin{minipage}[t]{1cm}\setlength{\parskip}{3mm}
	  		\textcolor{#1}{\LARGE{#2}}    
 	 	\end{minipage}  
  		\begin{minipage}[t]{\linewidth-2\fboxrule-4\fboxsep}\setlength{\parskip}{3mm}
			\raisebox{1.2mm}{\normalsize\sffamily{\textcolor{#1}{#3}}}						
  			 #4
  		\end{minipage}
	}}
	\vspace{5mm}
}

\newcommand\cadre[3]{				% Boites convertible html
	\par
	\vspace{2mm}
	\setlength{\fboxrule}{0.1mm}
	\setlength{\fboxsep}{5mm}
	\fcolorbox{#1}{white}{\makebox[\linewidth-2\fboxrule-2\fboxsep]{
  		\begin{minipage}[t]{\linewidth-2\fboxrule-4\fboxsep}\setlength{\parskip}{3mm}
			\raisebox{-2.5mm}{\sffamily \small{\textcolor{#1}{\MakeUppercase{#2}}}}		
			\par		
  			 #3
 	 		\end{minipage}
	}}
		\vspace{2mm}
	\par
}

\newcommand\bloc[3]{				% Boites convertible html sans bordure
     \needspace{2\baselineskip}
     {\sffamily \small{\textcolor{#1}{\MakeUppercase{#2}}}}    
		\par		
  			 #3
		\par
}

\newcommand\CHelp[1]{
     \CBox{Plum}{\faInfoCircle}{À RETENIR}{#1}
}

\newcommand\CUp[1]{
     \CBox{NavyBlue}{\faThumbsOUp}{EN PRATIQUE}{#1}
}

\newcommand\CInfo[1]{
     \CBox{Sepia}{\faArrowCircleRight}{REMARQUE}{#1}
}

\newcommand\CRedac[1]{
     \CBox{PineGreen}{\faEdit}{BIEN R\'EDIGER}{#1}
}

\newcommand\CError[1]{
     \CBox{Red}{\faExclamationTriangle}{ATTENTION}{#1}
}

\newcommand\TitreExo[2]{
\needspace{4\baselineskip}
 {\sffamily\large EXERCICE #1\ (\emph{#2 points})}
\vspace{5mm}
}

\newcommand\img[2]{
          \includegraphics[width=#2\paperwidth]{\imgdir#1}
}

\newcommand\imgsvg[2]{
       \begin{center}   \includegraphics[width=#2\paperwidth]{\imgsvgdir#1} \end{center}
}


\newcommand\Lien[2]{
     \href{#1}{#2 \tiny \faExternalLink}
}
\newcommand\mcLien[2]{
     \href{https~://www.maths-cours.fr/#1}{#2 \tiny \faExternalLink}
}

\newcommand{\euro}{\eurologo{}}

%================================================================================================================================
%
% Macros - Environement
%
%================================================================================================================================

\newenvironment{tex}{ %
}
{%
}

\newenvironment{indente}{ %
	\setlength\parindent{10mm}
}

{
	\setlength\parindent{0mm}
}

\newenvironment{corrige}{%
     \needspace{3\baselineskip}
     \medskip
     \textbf{\textsc{Corrigé}}
     \medskip
}
{
}

\newenvironment{extern}{%
     \begin{center}
     }
     {
     \end{center}
}

\NewEnviron{code}{%
	\par
     \boite{gray}{\texttt{%
     \BODY
     }}
     \par
}

\newenvironment{vbloc}{% boite sans cadre empeche saut de page
     \begin{minipage}[t]{\linewidth}
     }
     {
     \end{minipage}
}
\NewEnviron{h2}{%
    \needspace{3\baselineskip}
    \vspace{0.6cm}
	\noindent \MakeUppercase{\sffamily \large \BODY}
	\vspace{1mm}\textcolor{mcgris}{\hrule}\vspace{0.4cm}
	\par
}{}

\NewEnviron{h3}{%
    \needspace{3\baselineskip}
	\vspace{5mm}
	\textsc{\BODY}
	\par
}

\NewEnviron{margeneg}{ %
\begin{addmargin}[-1cm]{0cm}
\BODY
\end{addmargin}
}

\NewEnviron{html}{%
}

\begin{document}
\meta{url}{/cours/algorithmes-tests-boucles/}
\meta{pid}{237}
\meta{titre}{Algorithmes : Tests et boucles}
\meta{type}{cours}
Les algorithmes que nous avons utilisés dans le chapitre précédent exécutent toujours la même tâche ce qui limite leur intérêt. Les tests et les boucles vont enrichir nos algorithmes leur permettant d'agir différemment en fonction des données entrées par l'utilisateur.
\begin{h2}1. Conditions\end{h2}
Une condition est une expression qui peut prendre l'une des deux valeurs suivantes \textbf{vrai} ou \textbf{faux}. On dit également que c'est une valeur de type \textbf{"logique"} ou \textbf{"booléen"}.
\par
Les principaux opérateurs de comparaison que vous rencontrerez sont les suivants :
\begin{itemize}
     \item égal à ( = en pseudo code)
     \item différent de ( != en pseudo code)
     \item strictement supérieur (\textbf{  > } en pseudo code)
     \item strictement inférieur ( \textbf{ < } en pseudo code)
     \item supérieur ou égal ( \textbf{ > =} en pseudo code)
     \item inférieur ou égal (\textbf{  < =} en pseudo code)
\end{itemize}
Ces comparaisons n'ont un sens que si les variables que l'on compare sont de même type.
\bigbreak
\begin{h3}Conditions composées\end{h3}
On peut écrire des conditions plus complexes en reliant des comparaisons à l'aide des opérateurs logiques \textbf{ET},\textbf{ OU} et \textbf{NON}.
\begin{itemize}
     \item \textbf{Condition 1 ET condition 2} sera vraie si les deux conditions sont \textbf{toutes les deux vraies}.
     \par
     Par exemple, la condition : "\textbf{âge supérieur à 5 ET âge inférieur à 10}" sera vraie si la variable âge est \textbf{strictement comprise entre 5 et 10}.
     \item \textbf{Condition 1 OU condition 2} sera vraie si \textbf{l'une au moins} des deux conditions est vraie.
     \par
     Par exemple, la condition "\textbf{prénom=Jean OU nom=Dupont}" sera vraie pour :
     \begin{itemize}
          \item Jean Dupont (conditions 1 et 2 vraies)
          \item Jean Durand (condition 1 vraie)
          \item Pierre Dupont (condition 2 vraie)
     \end{itemize}
     mais fausse pour
     \begin{itemize}
          \item Pierre Durand (conditions 1 et 2 fausses)
     \end{itemize}
     \item \textbf{NON (condition 1)} sera vraie si et seulement si \textbf{condition 1 est fausse}.
     \par
     Par exemple : "\textbf{NON (x < 3)}" sera vraie si \textbf{x  > = 3}
\end{itemize}
\begin{h2}2. Tests\end{h2}
\cadre{bleu}{Définition}{%
     Un \textbf{test} est une instruction qui permet d'effectuer un traitement différent selon qu'une condition est vérifiée ou non.
}
\begin{h3}Première forme\end{h3}
La première forme possible est la suivante :
\begin{code}si \textit{condition} alors \\
     \hspace*{15px}\textit{instructions}
     fin si
\end{code}
Les instructions ne seront exécutées que \textbf{si la condition est vérifiée}. Par exemple :
\begin{code}\textbf{variable}\\
     \hspace*{15px}x : entier\\
     \textbf{début algorithme}\\
     \hspace*{15px}lire x\\
     \hspace*{15px}si x > 10 alors\\
     \hspace*{30px}x prend la valeur 10\\
     \hspace*{15px}fin si\\
     \hspace*{15px}afficher x\\
     \textbf{fin algorithme}
\end{code}
Si l'utilisateur entre un entier supérieur à 10 l'algorithme affichera 10 sinon il affichera le nombre saisi par l'utilisateur.
\begin{h3}Seconde forme\end{h3}
La seconde forme est légèrement plus complexe :
\begin{code}si \textit{condition} alors\\
     \hspace*{15px}\textit{instructions 1}\\
     sinon\\
     \hspace*{15px}\textit{instructions 2}\\
     fin si
\end{code}
\textbf{Si la condition est vraie}, l'algorithme effectuera les "instructions 1" puis passera aux instructions situées après le "fin si".
\textbf{Si la condition est fausse}, l'algorithme effectuera les "instructions 2" puis passera aux instructions situées après le "fin si".
\bloc{orange}{Exemple}{%
     \begin{code}\textbf{variables}\\
          \hspace*{15px}âge, prix : entier\\
          \textbf{début algorithme}\\
          \hspace*{15px}afficher "entrez votre âge :"\\
          \hspace*{15px}lire âge\\
          \hspace*{15px}si âge < 16 alors\\
          \hspace*{30px}afficher "vous bénéficiez du tarif réduit"\\
          \hspace*{30px}prix prend la valeur 10\\
          \hspace*{15px}sinon\\
          \hspace*{30px}afficher "vous ne bénéficiez pas du tarif réduit"\\
          \hspace*{30px}prix prend la valeur 15\\
          \hspace*{15px}fin si\\
          \hspace*{15px}afficher "vous devez payer", prix, "euros"\\
          \textbf{fin algorithme}
     \end{code}
     Si vous entrez \textbf{15} comme âge, vous obtiendrez le résultat suivant :
     \par
     \textit{\hspace*{15px}Vous bénéficiez du tarif réduit\\
     \hspace*{15px}Vous devez payer 10 euros}
     \par
     Si vous entrez \textbf{16} comme âge, vous obtiendrez :
     \par
     \textit{\hspace*{15px}Vous ne bénéficiez pas du tarif réduit\\
     \hspace*{15px}Vous devez payer 15 euros}
     \par
}
\begin{h2}3. Boucle\end{h2}
\cadre{bleu}{Définition}{%
     Une \textbf{boucle} permet de répéter un traitement un certain nombre de fois.
}
\begin{h3}Première forme\end{h3}
\textbf{ Boucle "Tant que" }
\begin{code}tant que \textit{condition}\\
     \hspace*{15px}\textit{instructions}\\
     fin tant que
\end{code}
L'algorithme ci-dessus effectuera les instructions tant que la condition sera vraie. Dès que la condition devient fausse, on se branchera sur l'instruction suivant le fin tant que.
\bloc{orange}{Exemple}{%
     \begin{code}\textbf{variables}\\
          \hspace*{15px}nombre, somme: nombres\\
          \hspace*{15px}continuer: texte\\
          \textbf{début algorithme}\\
          \hspace*{15px}continuer prend la valeur "oui" // initialisation\\
          \hspace*{15px}afficher 'entrez un nombre :'\\
          \hspace*{15px}lire nombre\\
          \hspace*{15px}somme prend la valeur nombre\\
          \hspace*{15px}tant que continuer="oui"\\
          \hspace*{30px}afficher "entrez le nombre suivant"\\
          \hspace*{30px}lire nombre\\
          \hspace*{30px}somme prend la valeur somme+nombre\\
          \hspace*{30px}afficher "voulez-vous continuer (oui/non)"\\
          \hspace*{30px}lire continuer\\
          \hspace*{15px}fin tant que\\
          \hspace*{15px}afficher "la somme des nombres entrés est" somme\\
          \textbf{fin algorithme}
     \end{code}
     L'algorithme précédent demande à l'utilisateur d'\textbf{entrer un premier nombre}.
     \par
     Puis il lui demande \textbf{s'il veut entrer un autre nombre}.
     \par
     Tant que l'utilisateur répond \textit{"oui"}, l'algorithme lui \textbf{demande un nouveau nombre} qu'il \textbf{additionne} au contenu de la variable "somme".
     \par
     Dès que l'utilisateur répond autre chose que \textit{"oui"}, l'algorithme \textbf{sort de la boucle}, \textbf{affiche le total} et se \textbf{termine}.
}
\begin{h3}Deuxième forme\end{h3}
\textbf{ Boucle "Pour" }
\bloc{orange}{Exemple}{%
     \begin{code}\textbf{variables}\\
          \hspace*{15px}i : nombre\\
          ...\\
          \textbf{début algorithme}\\
          ...\\
          pour i variant de 1 à 10\\
          \hspace*{15px}\textit{instructions}\\
          fin pour\\
          ...\\
          \textbf{fin algorithme}
     \end{code}
     L'algorithme ci-dessus va exécuter \textbf{dix fois} les \textit{instructions} situées dans la boucle.
     \par
     Plus précisément :
     \begin{itemize}
          \item \textbf{La première fois} que l'algorithme va rencontrer l'instruction \textit{"pour i variant de 1 à 10"}, il va affecter la valeur 1 à i; comme i est strictement inférieur à 10, il passe ensuite aux \textit{instructions} situées \textbf{à l'intérieur} de la boucle
          \item après les avoir exécutées, la ligne \textit{"fin pour"} va faire boucler l'algorithme et le faire revenir à l'instruction \textit{"pour i variant de 1 à 10"}
          \item \textbf{La seconde fois (et les fois suivantes...)} que l'algorithme va exécuter l'instruction \textit{"pour i variant de 1 à 10"}, il va :
          \begin{itemize}
               \item ajouter 1 à i (on dit \textbf{incrémenter i})
               \item si i est inférieur ou égal à 10, il passe aux \textit{instructions} situées \textbf{à l'intérieur} de la boucle
               \item si i est supérieur à 10, il passe aux instructions situées \textbf{après} la ligne \textit{"fin tant que"}
          \end{itemize}
     \end{itemize}
}
\bloc{cyan}{Remarque}{%
     Si l'on souhaite incrémenter l'indice avec une valeur différente de 1 on utilise l'instruction :
     \begin{code}pour i variant de ... à ... \textbf{avec un pas de} ...
     \end{code}
     Par exemple :
     \begin{code}pour i variant de 2 à 8 \textbf{avec un pas de 2}
     \end{code}
     \textit{i} va prendre successivement les valeurs : 2; 4 ; 6; 8 (et il quittera la boucle lorsqu'il vaudra 10)
}
\bloc{orange}{Exemple}{%
     L'algorithme ci-dessous affiche les carrés des 21 premiers nombres entiers naturels (de 0 à 20)
     \begin{code}\textbf{variables}\\
          \hspace*{15px}n : nombre\\
          \hspace*{15px}c : nombre\\
          \textbf{début algorithme}\\
          pour n variant de 0 à 20\\
          \hspace*{15px}c prend la valeur n*n\\
          \hspace*{15px}afficher "Le carré de ", n, " est ", c\\
          fin pour
          \textbf{fin algorithme}
     \end{code}
}
\bloc{cyan}{Remarque}{%
     On utilise généralement une instruction \textit{"pour"} lorsqu'on connaît le nombre d'itérations à réaliser dès le début de la boucle et une instruction \textit{"Tant que"} lorsque ce nombre est inconnu ou difficile à déterminer.
}

\end{document}
µ
\documentclass[a4paper]{article}

%================================================================================================================================
%
% Packages
%
%================================================================================================================================

\usepackage[T1]{fontenc} 	% pour caractères accentués
\usepackage[utf8]{inputenc}  % encodage utf8
\usepackage[french]{babel}	% langue : français
\usepackage{fourier}			% caractères plus lisibles
\usepackage[dvipsnames]{xcolor} % couleurs
\usepackage{fancyhdr}		% réglage header footer
\usepackage{needspace}		% empêcher sauts de page mal placés
\usepackage{graphicx}		% pour inclure des graphiques
\usepackage{enumitem,cprotect}		% personnalise les listes d'items (nécessaire pour ol, al ...)
\usepackage{hyperref}		% Liens hypertexte
\usepackage{pstricks,pst-all,pst-node,pstricks-add,pst-math,pst-plot,pst-tree,pst-eucl} % pstricks
\usepackage[a4paper,includeheadfoot,top=2cm,left=3cm, bottom=2cm,right=3cm]{geometry} % marges etc.
\usepackage{comment}			% commentaires multilignes
\usepackage{amsmath,environ} % maths (matrices, etc.)
\usepackage{amssymb,makeidx}
\usepackage{bm}				% bold maths
\usepackage{tabularx}		% tableaux
\usepackage{colortbl}		% tableaux en couleur
\usepackage{fontawesome}		% Fontawesome
\usepackage{environ}			% environment with command
\usepackage{fp}				% calculs pour ps-tricks
\usepackage{multido}			% pour ps tricks
\usepackage[np]{numprint}	% formattage nombre
\usepackage{tikz,tkz-tab} 			% package principal TikZ
\usepackage{pgfplots}   % axes
\usepackage{mathrsfs}    % cursives
\usepackage{calc}			% calcul taille boites
\usepackage[scaled=0.875]{helvet} % font sans serif
\usepackage{svg} % svg
\usepackage{scrextend} % local margin
\usepackage{scratch} %scratch
\usepackage{multicol} % colonnes
%\usepackage{infix-RPN,pst-func} % formule en notation polanaise inversée
\usepackage{listings}

%================================================================================================================================
%
% Réglages de base
%
%================================================================================================================================

\lstset{
language=Python,   % R code
literate=
{á}{{\'a}}1
{à}{{\`a}}1
{ã}{{\~a}}1
{é}{{\'e}}1
{è}{{\`e}}1
{ê}{{\^e}}1
{í}{{\'i}}1
{ó}{{\'o}}1
{õ}{{\~o}}1
{ú}{{\'u}}1
{ü}{{\"u}}1
{ç}{{\c{c}}}1
{~}{{ }}1
}


\definecolor{codegreen}{rgb}{0,0.6,0}
\definecolor{codegray}{rgb}{0.5,0.5,0.5}
\definecolor{codepurple}{rgb}{0.58,0,0.82}
\definecolor{backcolour}{rgb}{0.95,0.95,0.92}

\lstdefinestyle{mystyle}{
    backgroundcolor=\color{backcolour},   
    commentstyle=\color{codegreen},
    keywordstyle=\color{magenta},
    numberstyle=\tiny\color{codegray},
    stringstyle=\color{codepurple},
    basicstyle=\ttfamily\footnotesize,
    breakatwhitespace=false,         
    breaklines=true,                 
    captionpos=b,                    
    keepspaces=true,                 
    numbers=left,                    
xleftmargin=2em,
framexleftmargin=2em,            
    showspaces=false,                
    showstringspaces=false,
    showtabs=false,                  
    tabsize=2,
    upquote=true
}

\lstset{style=mystyle}


\lstset{style=mystyle}
\newcommand{\imgdir}{C:/laragon/www/newmc/assets/imgsvg/}
\newcommand{\imgsvgdir}{C:/laragon/www/newmc/assets/imgsvg/}

\definecolor{mcgris}{RGB}{220, 220, 220}% ancien~; pour compatibilité
\definecolor{mcbleu}{RGB}{52, 152, 219}
\definecolor{mcvert}{RGB}{125, 194, 70}
\definecolor{mcmauve}{RGB}{154, 0, 215}
\definecolor{mcorange}{RGB}{255, 96, 0}
\definecolor{mcturquoise}{RGB}{0, 153, 153}
\definecolor{mcrouge}{RGB}{255, 0, 0}
\definecolor{mclightvert}{RGB}{205, 234, 190}

\definecolor{gris}{RGB}{220, 220, 220}
\definecolor{bleu}{RGB}{52, 152, 219}
\definecolor{vert}{RGB}{125, 194, 70}
\definecolor{mauve}{RGB}{154, 0, 215}
\definecolor{orange}{RGB}{255, 96, 0}
\definecolor{turquoise}{RGB}{0, 153, 153}
\definecolor{rouge}{RGB}{255, 0, 0}
\definecolor{lightvert}{RGB}{205, 234, 190}
\setitemize[0]{label=\color{lightvert}  $\bullet$}

\pagestyle{fancy}
\renewcommand{\headrulewidth}{0.2pt}
\fancyhead[L]{maths-cours.fr}
\fancyhead[R]{\thepage}
\renewcommand{\footrulewidth}{0.2pt}
\fancyfoot[C]{}

\newcolumntype{C}{>{\centering\arraybackslash}X}
\newcolumntype{s}{>{\hsize=.35\hsize\arraybackslash}X}

\setlength{\parindent}{0pt}		 
\setlength{\parskip}{3mm}
\setlength{\headheight}{1cm}

\def\ebook{ebook}
\def\book{book}
\def\web{web}
\def\type{web}

\newcommand{\vect}[1]{\overrightarrow{\,\mathstrut#1\,}}

\def\Oij{$\left(\text{O}~;~\vect{\imath},~\vect{\jmath}\right)$}
\def\Oijk{$\left(\text{O}~;~\vect{\imath},~\vect{\jmath},~\vect{k}\right)$}
\def\Ouv{$\left(\text{O}~;~\vect{u},~\vect{v}\right)$}

\hypersetup{breaklinks=true, colorlinks = true, linkcolor = OliveGreen, urlcolor = OliveGreen, citecolor = OliveGreen, pdfauthor={Didier BONNEL - https://www.maths-cours.fr} } % supprime les bordures autour des liens

\renewcommand{\arg}[0]{\text{arg}}

\everymath{\displaystyle}

%================================================================================================================================
%
% Macros - Commandes
%
%================================================================================================================================

\newcommand\meta[2]{    			% Utilisé pour créer le post HTML.
	\def\titre{titre}
	\def\url{url}
	\def\arg{#1}
	\ifx\titre\arg
		\newcommand\maintitle{#2}
		\fancyhead[L]{#2}
		{\Large\sffamily \MakeUppercase{#2}}
		\vspace{1mm}\textcolor{mcvert}{\hrule}
	\fi 
	\ifx\url\arg
		\fancyfoot[L]{\href{https://www.maths-cours.fr#2}{\black \footnotesize{https://www.maths-cours.fr#2}}}
	\fi 
}


\newcommand\TitreC[1]{    		% Titre centré
     \needspace{3\baselineskip}
     \begin{center}\textbf{#1}\end{center}
}

\newcommand\newpar{    		% paragraphe
     \par
}

\newcommand\nosp {    		% commande vide (pas d'espace)
}
\newcommand{\id}[1]{} %ignore

\newcommand\boite[2]{				% Boite simple sans titre
	\vspace{5mm}
	\setlength{\fboxrule}{0.2mm}
	\setlength{\fboxsep}{5mm}	
	\fcolorbox{#1}{#1!3}{\makebox[\linewidth-2\fboxrule-2\fboxsep]{
  		\begin{minipage}[t]{\linewidth-2\fboxrule-4\fboxsep}\setlength{\parskip}{3mm}
  			 #2
  		\end{minipage}
	}}
	\vspace{5mm}
}

\newcommand\CBox[4]{				% Boites
	\vspace{5mm}
	\setlength{\fboxrule}{0.2mm}
	\setlength{\fboxsep}{5mm}
	
	\fcolorbox{#1}{#1!3}{\makebox[\linewidth-2\fboxrule-2\fboxsep]{
		\begin{minipage}[t]{1cm}\setlength{\parskip}{3mm}
	  		\textcolor{#1}{\LARGE{#2}}    
 	 	\end{minipage}  
  		\begin{minipage}[t]{\linewidth-2\fboxrule-4\fboxsep}\setlength{\parskip}{3mm}
			\raisebox{1.2mm}{\normalsize\sffamily{\textcolor{#1}{#3}}}						
  			 #4
  		\end{minipage}
	}}
	\vspace{5mm}
}

\newcommand\cadre[3]{				% Boites convertible html
	\par
	\vspace{2mm}
	\setlength{\fboxrule}{0.1mm}
	\setlength{\fboxsep}{5mm}
	\fcolorbox{#1}{white}{\makebox[\linewidth-2\fboxrule-2\fboxsep]{
  		\begin{minipage}[t]{\linewidth-2\fboxrule-4\fboxsep}\setlength{\parskip}{3mm}
			\raisebox{-2.5mm}{\sffamily \small{\textcolor{#1}{\MakeUppercase{#2}}}}		
			\par		
  			 #3
 	 		\end{minipage}
	}}
		\vspace{2mm}
	\par
}

\newcommand\bloc[3]{				% Boites convertible html sans bordure
     \needspace{2\baselineskip}
     {\sffamily \small{\textcolor{#1}{\MakeUppercase{#2}}}}    
		\par		
  			 #3
		\par
}

\newcommand\CHelp[1]{
     \CBox{Plum}{\faInfoCircle}{À RETENIR}{#1}
}

\newcommand\CUp[1]{
     \CBox{NavyBlue}{\faThumbsOUp}{EN PRATIQUE}{#1}
}

\newcommand\CInfo[1]{
     \CBox{Sepia}{\faArrowCircleRight}{REMARQUE}{#1}
}

\newcommand\CRedac[1]{
     \CBox{PineGreen}{\faEdit}{BIEN R\'EDIGER}{#1}
}

\newcommand\CError[1]{
     \CBox{Red}{\faExclamationTriangle}{ATTENTION}{#1}
}

\newcommand\TitreExo[2]{
\needspace{4\baselineskip}
 {\sffamily\large EXERCICE #1\ (\emph{#2 points})}
\vspace{5mm}
}

\newcommand\img[2]{
          \includegraphics[width=#2\paperwidth]{\imgdir#1}
}

\newcommand\imgsvg[2]{
       \begin{center}   \includegraphics[width=#2\paperwidth]{\imgsvgdir#1} \end{center}
}


\newcommand\Lien[2]{
     \href{#1}{#2 \tiny \faExternalLink}
}
\newcommand\mcLien[2]{
     \href{https~://www.maths-cours.fr/#1}{#2 \tiny \faExternalLink}
}

\newcommand{\euro}{\eurologo{}}

%================================================================================================================================
%
% Macros - Environement
%
%================================================================================================================================

\newenvironment{tex}{ %
}
{%
}

\newenvironment{indente}{ %
	\setlength\parindent{10mm}
}

{
	\setlength\parindent{0mm}
}

\newenvironment{corrige}{%
     \needspace{3\baselineskip}
     \medskip
     \textbf{\textsc{Corrigé}}
     \medskip
}
{
}

\newenvironment{extern}{%
     \begin{center}
     }
     {
     \end{center}
}

\NewEnviron{code}{%
	\par
     \boite{gray}{\texttt{%
     \BODY
     }}
     \par
}

\newenvironment{vbloc}{% boite sans cadre empeche saut de page
     \begin{minipage}[t]{\linewidth}
     }
     {
     \end{minipage}
}
\NewEnviron{h2}{%
    \needspace{3\baselineskip}
    \vspace{0.6cm}
	\noindent \MakeUppercase{\sffamily \large \BODY}
	\vspace{1mm}\textcolor{mcgris}{\hrule}\vspace{0.4cm}
	\par
}{}

\NewEnviron{h3}{%
    \needspace{3\baselineskip}
	\vspace{5mm}
	\textsc{\BODY}
	\par
}

\NewEnviron{margeneg}{ %
\begin{addmargin}[-1cm]{0cm}
\BODY
\end{addmargin}
}

\NewEnviron{html}{%
}

\begin{document}
\meta{url}{/cours/trinomes-du-second-degre/}
\meta{pid}{274}
\meta{titre}{Polynômes et équations du second degré}
\meta{type}{cours}
\begin{h2}1. Fonctions polynômes\end{h2}
\cadre{bleu}{Définition}{% id="d10"
     Une fonction $P$ est une \textbf{fonction polynôme} si elle est définie sur $\mathbb{R}$ et si on peut l'écrire sous la forme~:
     \begin{center}$P\left(x\right)=a_{n}x^{n}+a_{n-1}x^{n-1}+ . . . +a_{1}x+a_{0}$\end{center}
}
\bloc{vert}{Remarques}{% id="r10"
     \begin{itemize}
          \item par abus de langage, on dit souvent polynôme au lieu de fonction polynôme.
          \item les nombres $a_{i}$ s'appellent les \textbf{coefficients} du polynôme.
     \end{itemize}
}
\cadre{bleu}{Définition (Degré d'un polynôme)}{% id="d20"
     Si $P\left(x\right)=a_{n}x^{n}+a_{n-1}x^{n-1}+ . . . +a_{1}x+a_{0}$ (où le coefficient $ a_n $ est non nul), on dit que $P$ est une fonction polynôme de \textbf{degré} $n$.
}
\bloc{orange}{Cas particuliers}{% id="r20"
     \begin{itemize}
          \item la fonction nulle n'a pas de degré.
          \item une fonction constante non nulle définie par $f\left(x\right)=a$ avec $a\neq 0$ est une fonction polynôme de degré~0.
          \item une fonction affine $f\left(x\right)=ax+b$ avec $a\neq 0$ est une fonction polynôme de degré~1.
     \end{itemize}
}
\cadre{vert}{Propriété}{% id="p30"
     Le produit d'un polynôme de degré $n$ par un polynôme de degré $m$ est un polynôme de degré $m+n$.
}
\bloc{vert}{Remarque}{% id="r30"
     Il n'existe pas de formule donnant le degré d'une somme de polynôme. On peut tout au plus dire que le degré de $ P+Q $ est inférieur ou égal à la fois au degré de $P$ et au degré de $Q$.
}
\cadre{vert}{Propriété}{% id="p40"
     Deux polynômes sont égaux si et seulement si les coefficients des termes de même degré sont égaux.
}
\bloc{orange}{Cas particulier}{% id="r40"
     $P$ est le polynôme nul si et seulement si tous ses coefficients sont nuls.
}
\cadre{bleu}{Définition}{% id="d50"
     On dit que $a \in \mathbb{R}$ est une racine du polynôme $P$ si et seulement si $P\left(a\right)=0$.
}
\bloc{orange}{Exemple}{% id="e50"
     $1$ est racine du polynôme $P\left(x\right)=x^{3}-2x+1$ car $P\left(1\right)=0$
}
\cadre{rouge}{Théorème}{% id="t60"
     Si $P$ est un polynôme de degré $n\geqslant 1$ et si $a$ est une racine de $P$ alors $P\left(x\right)$ peut s'écrire sous la forme~:
     \begin{center}$P\left(x\right)=\left(x-a\right)Q\left(x\right)$\end{center}
     où $Q$ est un polynôme de degré $n-1$.
}
\begin{h2}2. Fonctions polynômes du second degré\end{h2}
\cadre{bleu}{Définition}{% id="d70"
     On appelle \textbf{polynôme (ou trinôme) du second degré} toute expression pouvant se mettre sous la forme~:
     \begin{center}$P\left(x\right)=ax^{2}+bx+c$\end{center}
     où $a$, $b$ et $c$ sont des réels avec $a \neq 0$.
}
\bloc{orange}{Exemples}{% id="e60"
     \begin{itemize}
          \item $P\left(x\right)=2x^{2}+3x-5$ est un polynôme du second degré.
          \item $P\left(x\right)=x^{2}-1$ est un polynôme du second degré avec $b=0$ mais $Q\left(x\right)=x-1$ n'en est pas un car $a$ n'est pas différent de zéro (c'est un polynôme du premier degré - ou une fonction affine).
          \item $P\left(x\right)=5\left(x-1\right)\left(3-2x\right)$ est un polynôme du second degré car en développant on obtient une expression du type souhaité.
     \end{itemize}
}
\cadre{rouge}{Théorème et définition}{% id="d70"
     Tout polynôme du second degré $P\left(x\right)=ax^{2}+bx+c$ peut s'écrire sous la forme~:
     \begin{center}$P\left(x\right)=a\left(x-\alpha \right)^{2}+ \beta $\end{center}
     avec $\alpha =-\frac{b}{2a}$ et $\beta =P\left(\alpha \right)$.
     \par
     Cette expression s'appelle \textbf{forme canonique} du polynôme $P$.
}
\cadre{bleu}{Définition}{% id="e70"
     Le nombre $\Delta =b^{2}-4ac$ s'appelle le \textbf{discriminant} du trinôme $ax^{2}+bx+c$.
}
\cadre{vert}{Propriété (Racines d'un polynôme du second degré)}{% id="p80"
     L'équation $ax^{2}+bx+c=0$~:
     \begin{itemize}
          \item n'a aucune solution réelle si $\Delta < 0$~;
          \item a une solution unique $x_{0}=\alpha =-\frac{b}{2a}$ si $\Delta =0$~;
          \item a deux solutions $x_{1}=\frac{-b+\sqrt{\Delta }}{2a}$ et $x_{2}=\frac{-b-\sqrt{\Delta }}{2a}$ si $\Delta > 0$.
     \end{itemize}
}
\bloc{orange}{Exemples}{% id="e80"
     \begin{itemize}
          \item \textbf{$P_{1}\left(x\right)=-x^{2}+3x-2$}~:
          \par
          $\Delta =9-4\times \left(-1\right)\times \left(-2\right)=1$.
          \par
          $P_{1}$ possède 2 racines~:
          \par
          $x_{1}=\frac{-3-1}{-2}=2$ et $x_{2}=\frac{-3+1}{-2}=1$
          \item \textbf{$P_{2}\left(x\right)=x^{2}-4x+4$}~:
          \par
          $\Delta =16-4\times 1\times 4=0$.
          \par
          $P_{2}$ possède une seule racine~:
          \par
          $x_{0}=-\frac{-4}{2}=2$.
          \item \textbf{$P_{3}\left(x\right)=x^{2}+x+1$}~:
          \par
          $\Delta =1-4\times 1\times 1=-3$.
          \par
          $P_{3}$ ne possède aucune racine.
     \end{itemize}
}
\cadre{vert}{Propriété (Somme et produit des racines)}{% id="p90"
     Soit un polynôme $P\left(x\right)=ax^{2}+bx+c$ dont le discriminant est strictement positif.
     \begin{itemize}
          \item La somme des racines vaut $x_{1}+x_{2}=-\frac{b}{a}$.
          \item Le produit des racines vaut $x_{1}x_{2}=\frac{c}{a}$.
     \end{itemize}
}
\bloc{cyan}{Remarque}{% id="r90"
     Ces propriétés sont souvent utilisées pour résoudre rapidement une équation qui possède une racine "évidente".
     \par
     Par exemple l'équation $x^{2}-4x+3=0$ admet $x_{1}=1$ comme racine puisque $1^{2}-4\times 1+3=0$~; comme $x_{1}\times x_{2}=\frac{c}{a}=3$ l'autre racine est $x_{2}=3$ .
}
\cadre{vert}{Propriété (Signe d'un polynôme du second degré)}{% id="p100"
     Le polynôme $P\left(x\right)=ax^{2}+bx+c$~:
     \begin{itemize}
          \item est toujours du signe de $a$ si $\Delta < 0$~;
          \item est toujours du signe de $a$ mais s'annule en $x_{0}=\alpha =-\frac{b}{2a}$ si $\Delta =0$~;
          \item est du signe de $a$ \og à l'extérieur des racines \fg{} (c'est à dire sur $\left]-\infty~; x_{1}\right[ \cup \left]x_{2}; +\infty \right[$) et du signe opposé \og entre les racines \fg{} ( sur $\left]x_{1}; x_{2}\right[$).
     \end{itemize}
}
\bloc{cyan}{Remarque}{% id="r100"
     Suivant chacun des cas on peut représenter le tableau de signe de $P$ de la façon suivante~:
     \begin{itemize}
          \item \textbf{Si $\Delta > 0$~:} $P\left(x\right)$ est du signe de $a$ à l'extérieur des racines (c'est à dire si $x < x_{1}$ ou $x > x_{2}$ ) et du signe opposé entre les racines (si $x_{1} < x < x_{2}$).
          %:-+-+-+-+- Engendré par~: http://math.et.info.free.fr/TikZ/TableauxVariations/
          \begin{center}
               \begin{extern}%width="500" alt="Tableau de signe plynôme du second degré delta positif"
                    \begin{tikzpicture}[scale=0.875]
                         % Styles
                         \tikzstyle{cadre}=[thin]
                         \tikzstyle{fleche}=[->,>=latex,thin]
                         \tikzstyle{nondefini}=[lightgray]
                         % Dimensions Modifiables
                         \def\Lrg{1.8}
                         \def\HtX{1.2}
                         \def\HtY{0.5}
                         % Dimensions Calculées
                         \def\lignex{-0.5*\HtX}
                         \def\lignef{-1.5*\HtX}
                         \def\separateur{-0.5*\Lrg}
                         % Largeur du tableau
                         \def\gauche{-1.5*\Lrg}
                         \def\droite{6.5*\Lrg}
                         % Hauteur du tableau
                         \def\haut{0.5*\HtX}
                         \def\bas{-2.5*\HtX-2*\HtY}
                         % Pointillés
                         \draw[gray] (2*\Lrg,\lignex) -- (2*\Lrg,\lignef);
                         \draw[gray] (4*\Lrg,\lignex) -- (4*\Lrg,\lignef);
                         % Ligne de l'abscisse~: x
                         \node at (-1*\Lrg,0) {$x$};
                         \node at (0*\Lrg,0) {$-\infty$};
                         \node at (2*\Lrg,0) {$x_1$};
                         \node at (4*\Lrg,0) {$x_2$};
                         \node at (6*\Lrg,0) {$+\infty$};
                         % Ligne de la dérivée~: f'(x)
                         \node at (-1*\Lrg,-1*\HtX) {$P(x)$};
                         \node at (0*\Lrg,-1*\HtX) {$ $};
                         \node at (1*\Lrg,-1*\HtX) {signe de $a$};
                         \node at (2*\Lrg,-1*\HtX) {$0$};
                         \node at (3*\Lrg,-1*\HtX) {signe de $-a$};
                         \node at (4*\Lrg,-1*\HtX) {$0$};
                         \node at (5*\Lrg,-1*\HtX) {signe de $a$};
                         \node at (6*\Lrg,-1*\HtX) {$ $};
                         % Ligne de la fonction~: f(x)
                         % Encadrement
                         \draw[cadre] (\separateur,\haut) -- (\separateur, \lignef);
                         \draw[cadre] (\gauche,\haut) rectangle (\droite, \lignef);
                         \draw[cadre] (\gauche,\lignex) -- (\droite,\lignex);
                    \end{tikzpicture}
               \end{extern}
          \end{center}
          %:-+-+-+-+- Fin
          \item \textbf{Si $\Delta =0$~:} $P\left(x\right)$ est toujours du signe de $a$ sauf en $x_{0}$ (où il s'annule).
          %:-+-+-+-+- Engendré par~: http://math.et.info.free.fr/TikZ/TableauxVariations/
          \begin{center}
               \begin{extern}%width="390" alt="Tableau de signe plynôme du second degré delta nul"
                    \begin{tikzpicture}[scale=0.875]
                         % Styles
                         \tikzstyle{cadre}=[thin]
                         \tikzstyle{fleche}=[->,>=latex,thin]
                         \tikzstyle{nondefini}=[lightgray]
                         % Dimensions Modifiables
                         \def\Lrg{1.8}
                         \def\HtX{1.2}
                         \def\HtY{0.5}
                         % Dimensions Calculées
                         \def\lignex{-0.5*\HtX}
                         \def\lignef{-1.5*\HtX}
                         \def\separateur{-0.5*\Lrg}
                         % Largeur du tableau
                         \def\gauche{-1.5*\Lrg}
                         \def\droite{4.5*\Lrg}
                         % Hauteur du tableau
                         \def\haut{0.5*\HtX}
                         \def\bas{-2.5*\HtX-2*\HtY}
                         % Pointillés
                         \draw[gray] (2*\Lrg,\lignex) -- (2*\Lrg,\lignef);
                         % Ligne de l'abscisse~: x
                         \node at (-1*\Lrg,0) {$x$};
                         \node at (0*\Lrg,0) {$-\infty$};
                         \node at (2*\Lrg,0) {$x_0$};
                         \node at (4*\Lrg,0) {$+\infty$};
                         % Ligne de la dérivée~: f'(x)
                         \node at (-1*\Lrg,-1*\HtX) {$P(x)$};
                         \node at (0*\Lrg,-1*\HtX) {$ $};
                         \node at (1*\Lrg,-1*\HtX) {signe de $a$};
                         \node at (2*\Lrg,-1*\HtX) {$0$};
                         \node at (3*\Lrg,-1*\HtX) {signe de $a$};
                         \node at (4*\Lrg,-1*\HtX) {$ $};
                         % Ligne de la fonction~: f(x)
                         % Encadrement
                         \draw[cadre] (\separateur,\haut) -- (\separateur, \lignef);
                         \draw[cadre] (\gauche,\haut) rectangle (\droite, \lignef);
                         \draw[cadre] (\gauche,\lignex) -- (\droite,\lignex);
                    \end{tikzpicture}
               \end{extern}
          \end{center}
          %:-+-+-+-+- Fin
          \item \textbf{Si $\Delta < 0$~:} $P\left(x\right)$ est toujours du signe de $a$.
          \begin{center}
               \begin{extern}%width="230" alt="Tableau de signe plynôme du second degré delta négatif"
                    \begin{tikzpicture}[scale=0.875]
                         % Styles
                         \tikzstyle{cadre}=[thin]
                         \tikzstyle{fleche}=[->,>=latex,thin]
                         \tikzstyle{nondefini}=[lightgray]
                         % Dimensions Modifiables
                         \def\Lrg{1.5}
                         \def\HtX{1.2}
                         \def\HtY{0.5}
                         % Dimensions Calculées
                         \def\lignex{-0.5*\HtX}
                         \def\lignef{-1.5*\HtX}
                         \def\separateur{-0.5*\Lrg}
                         % Largeur du tableau
                         \def\gauche{-1.5*\Lrg}
                         \def\droite{2.5*\Lrg}
                         % Hauteur du tableau
                         \def\haut{0.5*\HtX}
                         \def\bas{-2.5*\HtX-2*\HtY}
                         % Ligne de l'abscisse~: x
                         \node at (-1*\Lrg,0) {$x$};
                         \node at (0*\Lrg,0) {$-\infty$};
                         \node at (2*\Lrg,0) {$+\infty$};
                         % Ligne de la dérivée~: f'(x)
                         \node at (-1*\Lrg,-1*\HtX) {$P(x)$};
                         \node at (0*\Lrg,-1*\HtX) {$ $};
                         \node at (1*\Lrg,-1*\HtX) {signe de $a$};
                         \node at (2*\Lrg,-1*\HtX) {$ $};
                         % Ligne de la fonction~: f(x)
                         % Encadrement
                         \draw[cadre] (\separateur,\haut) -- (\separateur, \lignef);
                         \draw[cadre] (\gauche,\haut) rectangle (\droite, \lignef);
                         \draw[cadre] (\gauche,\lignex) -- (\droite,\lignex);
                    \end{tikzpicture}
               \end{extern}
          \end{center}
     \end{itemize}
}
\bloc{orange}{Exemples}{% id="e100"
     Si l'on reprend les exemples précédents~:
     \begin{itemize}
          \item $P_{1}\left(x\right)=-x^{2}+3x-2$~:
          \par
          $\Delta > 0$ et $a < 0$.
          %:-+-+-+-+- Engendré par~: http://math.et.info.free.fr/TikZ/TableauxVariations/
          \begin{center}
               \begin{extern}%width="430" alt="Exemple tableau de signe plynôme du second degré delta positif"
                    \begin{tikzpicture}[scale=0.875]
                         % Styles
                         \tikzstyle{cadre}=[thin]
                         \tikzstyle{fleche}=[->,>=latex,thin]
                         \tikzstyle{nondefini}=[lightgray]
                         % Dimensions Modifiables
                         \def\Lrg{1.5}
                         \def\HtX{1.2}
                         \def\HtY{0.5}
                         % Dimensions Calculées
                         \def\lignex{-0.5*\HtX}
                         \def\lignef{-1.5*\HtX}
                         \def\separateur{-0.5*\Lrg}
                         % Largeur du tableau
                         \def\gauche{-1.5*\Lrg}
                         \def\droite{6.5*\Lrg}
                         % Hauteur du tableau
                         \def\haut{0.5*\HtX}
                         \def\bas{-2.5*\HtX-2*\HtY}
                         % Pointillés
                         \draw[gray] (2*\Lrg,\lignex) -- (2*\Lrg,\lignef);
                         \draw[gray] (4*\Lrg,\lignex) -- (4*\Lrg,\lignef);
                         % Ligne de l'abscisse~: x
                         \node at (-1*\Lrg,0) {$x$};
                         \node at (0*\Lrg,0) {$-\infty$};
                         \node at (2*\Lrg,0) {$1$};
                         \node at (4*\Lrg,0) {$2$};
                         \node at (6*\Lrg,0) {$+\infty$};
                         % Ligne de la dérivée~: f'(x)
                         \node at (-1*\Lrg,-1*\HtX) {$P(x)$};
                         \node at (0*\Lrg,-1*\HtX) {$ $};
                         \node at (1*\Lrg,-1*\HtX) {$-$};
                         \node at (2*\Lrg,-1*\HtX) {$0$};
                         \node at (3*\Lrg,-1*\HtX) {$+$};
                         \node at (4*\Lrg,-1*\HtX) {$0$};
                         \node at (5*\Lrg,-1*\HtX) {$-$};
                         \node at (6*\Lrg,-1*\HtX) {$ $};
                         % Ligne de la fonction~: f(x)
                         % Encadrement
                         \draw[cadre] (\separateur,\haut) -- (\separateur, \lignef);
                         \draw[cadre] (\gauche,\haut) rectangle (\droite, \lignef);
                         \draw[cadre] (\gauche,\lignex) -- (\droite,\lignex);
                    \end{tikzpicture}
               \end{extern}
          \end{center}
          %:-+-+-+-+- Fin
          \item $P_{2}\left(x\right)=x^{2}-4x+4$~:
          \par
          $\Delta =0$ et $a > 0$.
          %:-+-+-+-+- Engendré par~: http://math.et.info.free.fr/TikZ/TableauxVariations/
          \begin{center}
               \begin{extern}%width="330" alt="Exemple tableau de signe plynôme du second degré delta nul"
                    \begin{tikzpicture}[scale=0.875]
                         % Styles
                         \tikzstyle{cadre}=[thin]
                         \tikzstyle{fleche}=[->,>=latex,thin]
                         \tikzstyle{nondefini}=[lightgray]
                         % Dimensions Modifiables
                         \def\Lrg{1.5}
                         \def\HtX{1.2}
                         \def\HtY{0.5}
                         % Dimensions Calculées
                         \def\lignex{-0.5*\HtX}
                         \def\lignef{-1.5*\HtX}
                         \def\separateur{-0.5*\Lrg}
                         % Largeur du tableau
                         \def\gauche{-1.5*\Lrg}
                         \def\droite{4.5*\Lrg}
                         % Hauteur du tableau
                         \def\haut{0.5*\HtX}
                         \def\bas{-2.5*\HtX-2*\HtY}
                         % Pointillés
                         \draw[gray] (2*\Lrg,\lignex) -- (2*\Lrg,\lignef);
                         % Ligne de l'abscisse~: x
                         \node at (-1*\Lrg,0) {$x$};
                         \node at (0*\Lrg,0) {$-\infty$};
                         \node at (2*\Lrg,0) {$2$};
                         \node at (4*\Lrg,0) {$+\infty$};
                         % Ligne de la dérivée~: f'(x)
                         \node at (-1*\Lrg,-1*\HtX) {$P(x)$};
                         \node at (0*\Lrg,-1*\HtX) {$ $};
                         \node at (1*\Lrg,-1*\HtX) {+};
                         \node at (2*\Lrg,-1*\HtX) {$0$};
                         \node at (3*\Lrg,-1*\HtX) {+};
                         \node at (4*\Lrg,-1*\HtX) {$ $};
                         % Ligne de la fonction~: f(x)
                         % Encadrement
                         \draw[cadre] (\separateur,\haut) -- (\separateur, \lignef);
                         \draw[cadre] (\gauche,\haut) rectangle (\droite, \lignef);
                         \draw[cadre] (\gauche,\lignex) -- (\droite,\lignex);
                    \end{tikzpicture}
               \end{extern}
          \end{center}
          %:-+-+-+-+- Fin
          \item $P_{3}\left(x\right)=x^{2}+x+1$~:
          \par
          $\Delta < 0$ et $a > 0$.
          \begin{center}
               \begin{extern}%width="230" alt="Exemple tableau de signe plynôme du second degré delta négatif"
                    \begin{tikzpicture}[scale=0.875]
                         % Styles
                         \tikzstyle{cadre}=[thin]
                         \tikzstyle{fleche}=[->,>=latex,thin]
                         \tikzstyle{nondefini}=[lightgray]
                         % Dimensions Modifiables
                         \def\Lrg{1.5}
                         \def\HtX{1.2}
                         \def\HtY{0.5}
                         % Dimensions Calculées
                         \def\lignex{-0.5*\HtX}
                         \def\lignef{-1.5*\HtX}
                         \def\separateur{-0.5*\Lrg}
                         % Largeur du tableau
                         \def\gauche{-1.5*\Lrg}
                         \def\droite{2.5*\Lrg}
                         % Hauteur du tableau
                         \def\haut{0.5*\HtX}
                         \def\bas{-2.5*\HtX-2*\HtY}
                         % Ligne de l'abscisse~: x
                         \node at (-1*\Lrg,0) {$x$};
                         \node at (0*\Lrg,0) {$-\infty$};
                         \node at (2*\Lrg,0) {$+\infty$};
                         % Ligne de la dérivée~: f'(x)
                         \node at (-1*\Lrg,-1*\HtX) {$P(x)$};
                         \node at (0*\Lrg,-1*\HtX) {$ $};
                         \node at (1*\Lrg,-1*\HtX) {$+$};
                         \node at (2*\Lrg,-1*\HtX) {$ $};
                         % Ligne de la fonction~: f(x)
                         % Encadrement
                         \draw[cadre] (\separateur,\haut) -- (\separateur, \lignef);
                         \draw[cadre] (\gauche,\haut) rectangle (\droite, \lignef);
                         \draw[cadre] (\gauche,\lignex) -- (\droite,\lignex);
                    \end{tikzpicture}
               \end{extern}
          \end{center}
     \end{itemize}
}
On rappelle que les solutions de l'équation $f\left(x\right)=0$ sont les abscisses des \textbf{points d'intersection de la courbe} $C_{f}$ et de l'\textbf{axe des abscisses}.
\par
En regroupant les propriétés de ce chapitre et celles vues en Seconde on peut résumer ces résultats dans le tableau~:
\begin{center}
     \begin{extern}%width="600" alt="Différentes paraboles"
          \begin{tabular}{|c|c|c|}
               \hline
               & $a > 0$ & $a < 0$\\ \hline
               $\Delta > 0$ & \img{parabole-1-1}{0.25}
               & \img{parabole-1-2}{0.25}
               \\
               & 2~racines~: $x_{1}$ et $x_{2}$
               & 2~racines~: $x_{1}$ et $x_{2}$\\ \hline
               $\Delta =0$ & \img{parabole-2-1}{0.25}
               & \img{parabole-2-2}{0.25}
               \\
               &1~racine~: $x_{0}$
               &1~racine~: $x_{0}$\\ \hline
               $\Delta < 0$ & \img{parabole-3-1}{0.25}
               & \img{parabole-3-2}{0.25}
               \\ &Pas~de racine
               & Pas~de racine\\ \hline
          \end{tabular}
     \end{extern}
\end{center}

\end{document}
µ
\documentclass[a4paper]{article}

%================================================================================================================================
%
% Packages
%
%================================================================================================================================

\usepackage[T1]{fontenc} 	% pour caractères accentués
\usepackage[utf8]{inputenc}  % encodage utf8
\usepackage[french]{babel}	% langue : français
\usepackage{fourier}			% caractères plus lisibles
\usepackage[dvipsnames]{xcolor} % couleurs
\usepackage{fancyhdr}		% réglage header footer
\usepackage{needspace}		% empêcher sauts de page mal placés
\usepackage{graphicx}		% pour inclure des graphiques
\usepackage{enumitem,cprotect}		% personnalise les listes d'items (nécessaire pour ol, al ...)
\usepackage{hyperref}		% Liens hypertexte
\usepackage{pstricks,pst-all,pst-node,pstricks-add,pst-math,pst-plot,pst-tree,pst-eucl} % pstricks
\usepackage[a4paper,includeheadfoot,top=2cm,left=3cm, bottom=2cm,right=3cm]{geometry} % marges etc.
\usepackage{comment}			% commentaires multilignes
\usepackage{amsmath,environ} % maths (matrices, etc.)
\usepackage{amssymb,makeidx}
\usepackage{bm}				% bold maths
\usepackage{tabularx}		% tableaux
\usepackage{colortbl}		% tableaux en couleur
\usepackage{fontawesome}		% Fontawesome
\usepackage{environ}			% environment with command
\usepackage{fp}				% calculs pour ps-tricks
\usepackage{multido}			% pour ps tricks
\usepackage[np]{numprint}	% formattage nombre
\usepackage{tikz,tkz-tab} 			% package principal TikZ
\usepackage{pgfplots}   % axes
\usepackage{mathrsfs}    % cursives
\usepackage{calc}			% calcul taille boites
\usepackage[scaled=0.875]{helvet} % font sans serif
\usepackage{svg} % svg
\usepackage{scrextend} % local margin
\usepackage{scratch} %scratch
\usepackage{multicol} % colonnes
%\usepackage{infix-RPN,pst-func} % formule en notation polanaise inversée
\usepackage{listings}

%================================================================================================================================
%
% Réglages de base
%
%================================================================================================================================

\lstset{
language=Python,   % R code
literate=
{á}{{\'a}}1
{à}{{\`a}}1
{ã}{{\~a}}1
{é}{{\'e}}1
{è}{{\`e}}1
{ê}{{\^e}}1
{í}{{\'i}}1
{ó}{{\'o}}1
{õ}{{\~o}}1
{ú}{{\'u}}1
{ü}{{\"u}}1
{ç}{{\c{c}}}1
{~}{{ }}1
}


\definecolor{codegreen}{rgb}{0,0.6,0}
\definecolor{codegray}{rgb}{0.5,0.5,0.5}
\definecolor{codepurple}{rgb}{0.58,0,0.82}
\definecolor{backcolour}{rgb}{0.95,0.95,0.92}

\lstdefinestyle{mystyle}{
    backgroundcolor=\color{backcolour},   
    commentstyle=\color{codegreen},
    keywordstyle=\color{magenta},
    numberstyle=\tiny\color{codegray},
    stringstyle=\color{codepurple},
    basicstyle=\ttfamily\footnotesize,
    breakatwhitespace=false,         
    breaklines=true,                 
    captionpos=b,                    
    keepspaces=true,                 
    numbers=left,                    
xleftmargin=2em,
framexleftmargin=2em,            
    showspaces=false,                
    showstringspaces=false,
    showtabs=false,                  
    tabsize=2,
    upquote=true
}

\lstset{style=mystyle}


\lstset{style=mystyle}
\newcommand{\imgdir}{C:/laragon/www/newmc/assets/imgsvg/}
\newcommand{\imgsvgdir}{C:/laragon/www/newmc/assets/imgsvg/}

\definecolor{mcgris}{RGB}{220, 220, 220}% ancien~; pour compatibilité
\definecolor{mcbleu}{RGB}{52, 152, 219}
\definecolor{mcvert}{RGB}{125, 194, 70}
\definecolor{mcmauve}{RGB}{154, 0, 215}
\definecolor{mcorange}{RGB}{255, 96, 0}
\definecolor{mcturquoise}{RGB}{0, 153, 153}
\definecolor{mcrouge}{RGB}{255, 0, 0}
\definecolor{mclightvert}{RGB}{205, 234, 190}

\definecolor{gris}{RGB}{220, 220, 220}
\definecolor{bleu}{RGB}{52, 152, 219}
\definecolor{vert}{RGB}{125, 194, 70}
\definecolor{mauve}{RGB}{154, 0, 215}
\definecolor{orange}{RGB}{255, 96, 0}
\definecolor{turquoise}{RGB}{0, 153, 153}
\definecolor{rouge}{RGB}{255, 0, 0}
\definecolor{lightvert}{RGB}{205, 234, 190}
\setitemize[0]{label=\color{lightvert}  $\bullet$}

\pagestyle{fancy}
\renewcommand{\headrulewidth}{0.2pt}
\fancyhead[L]{maths-cours.fr}
\fancyhead[R]{\thepage}
\renewcommand{\footrulewidth}{0.2pt}
\fancyfoot[C]{}

\newcolumntype{C}{>{\centering\arraybackslash}X}
\newcolumntype{s}{>{\hsize=.35\hsize\arraybackslash}X}

\setlength{\parindent}{0pt}		 
\setlength{\parskip}{3mm}
\setlength{\headheight}{1cm}

\def\ebook{ebook}
\def\book{book}
\def\web{web}
\def\type{web}

\newcommand{\vect}[1]{\overrightarrow{\,\mathstrut#1\,}}

\def\Oij{$\left(\text{O}~;~\vect{\imath},~\vect{\jmath}\right)$}
\def\Oijk{$\left(\text{O}~;~\vect{\imath},~\vect{\jmath},~\vect{k}\right)$}
\def\Ouv{$\left(\text{O}~;~\vect{u},~\vect{v}\right)$}

\hypersetup{breaklinks=true, colorlinks = true, linkcolor = OliveGreen, urlcolor = OliveGreen, citecolor = OliveGreen, pdfauthor={Didier BONNEL - https://www.maths-cours.fr} } % supprime les bordures autour des liens

\renewcommand{\arg}[0]{\text{arg}}

\everymath{\displaystyle}

%================================================================================================================================
%
% Macros - Commandes
%
%================================================================================================================================

\newcommand\meta[2]{    			% Utilisé pour créer le post HTML.
	\def\titre{titre}
	\def\url{url}
	\def\arg{#1}
	\ifx\titre\arg
		\newcommand\maintitle{#2}
		\fancyhead[L]{#2}
		{\Large\sffamily \MakeUppercase{#2}}
		\vspace{1mm}\textcolor{mcvert}{\hrule}
	\fi 
	\ifx\url\arg
		\fancyfoot[L]{\href{https://www.maths-cours.fr#2}{\black \footnotesize{https://www.maths-cours.fr#2}}}
	\fi 
}


\newcommand\TitreC[1]{    		% Titre centré
     \needspace{3\baselineskip}
     \begin{center}\textbf{#1}\end{center}
}

\newcommand\newpar{    		% paragraphe
     \par
}

\newcommand\nosp {    		% commande vide (pas d'espace)
}
\newcommand{\id}[1]{} %ignore

\newcommand\boite[2]{				% Boite simple sans titre
	\vspace{5mm}
	\setlength{\fboxrule}{0.2mm}
	\setlength{\fboxsep}{5mm}	
	\fcolorbox{#1}{#1!3}{\makebox[\linewidth-2\fboxrule-2\fboxsep]{
  		\begin{minipage}[t]{\linewidth-2\fboxrule-4\fboxsep}\setlength{\parskip}{3mm}
  			 #2
  		\end{minipage}
	}}
	\vspace{5mm}
}

\newcommand\CBox[4]{				% Boites
	\vspace{5mm}
	\setlength{\fboxrule}{0.2mm}
	\setlength{\fboxsep}{5mm}
	
	\fcolorbox{#1}{#1!3}{\makebox[\linewidth-2\fboxrule-2\fboxsep]{
		\begin{minipage}[t]{1cm}\setlength{\parskip}{3mm}
	  		\textcolor{#1}{\LARGE{#2}}    
 	 	\end{minipage}  
  		\begin{minipage}[t]{\linewidth-2\fboxrule-4\fboxsep}\setlength{\parskip}{3mm}
			\raisebox{1.2mm}{\normalsize\sffamily{\textcolor{#1}{#3}}}						
  			 #4
  		\end{minipage}
	}}
	\vspace{5mm}
}

\newcommand\cadre[3]{				% Boites convertible html
	\par
	\vspace{2mm}
	\setlength{\fboxrule}{0.1mm}
	\setlength{\fboxsep}{5mm}
	\fcolorbox{#1}{white}{\makebox[\linewidth-2\fboxrule-2\fboxsep]{
  		\begin{minipage}[t]{\linewidth-2\fboxrule-4\fboxsep}\setlength{\parskip}{3mm}
			\raisebox{-2.5mm}{\sffamily \small{\textcolor{#1}{\MakeUppercase{#2}}}}		
			\par		
  			 #3
 	 		\end{minipage}
	}}
		\vspace{2mm}
	\par
}

\newcommand\bloc[3]{				% Boites convertible html sans bordure
     \needspace{2\baselineskip}
     {\sffamily \small{\textcolor{#1}{\MakeUppercase{#2}}}}    
		\par		
  			 #3
		\par
}

\newcommand\CHelp[1]{
     \CBox{Plum}{\faInfoCircle}{À RETENIR}{#1}
}

\newcommand\CUp[1]{
     \CBox{NavyBlue}{\faThumbsOUp}{EN PRATIQUE}{#1}
}

\newcommand\CInfo[1]{
     \CBox{Sepia}{\faArrowCircleRight}{REMARQUE}{#1}
}

\newcommand\CRedac[1]{
     \CBox{PineGreen}{\faEdit}{BIEN R\'EDIGER}{#1}
}

\newcommand\CError[1]{
     \CBox{Red}{\faExclamationTriangle}{ATTENTION}{#1}
}

\newcommand\TitreExo[2]{
\needspace{4\baselineskip}
 {\sffamily\large EXERCICE #1\ (\emph{#2 points})}
\vspace{5mm}
}

\newcommand\img[2]{
          \includegraphics[width=#2\paperwidth]{\imgdir#1}
}

\newcommand\imgsvg[2]{
       \begin{center}   \includegraphics[width=#2\paperwidth]{\imgsvgdir#1} \end{center}
}


\newcommand\Lien[2]{
     \href{#1}{#2 \tiny \faExternalLink}
}
\newcommand\mcLien[2]{
     \href{https~://www.maths-cours.fr/#1}{#2 \tiny \faExternalLink}
}

\newcommand{\euro}{\eurologo{}}

%================================================================================================================================
%
% Macros - Environement
%
%================================================================================================================================

\newenvironment{tex}{ %
}
{%
}

\newenvironment{indente}{ %
	\setlength\parindent{10mm}
}

{
	\setlength\parindent{0mm}
}

\newenvironment{corrige}{%
     \needspace{3\baselineskip}
     \medskip
     \textbf{\textsc{Corrigé}}
     \medskip
}
{
}

\newenvironment{extern}{%
     \begin{center}
     }
     {
     \end{center}
}

\NewEnviron{code}{%
	\par
     \boite{gray}{\texttt{%
     \BODY
     }}
     \par
}

\newenvironment{vbloc}{% boite sans cadre empeche saut de page
     \begin{minipage}[t]{\linewidth}
     }
     {
     \end{minipage}
}
\NewEnviron{h2}{%
    \needspace{3\baselineskip}
    \vspace{0.6cm}
	\noindent \MakeUppercase{\sffamily \large \BODY}
	\vspace{1mm}\textcolor{mcgris}{\hrule}\vspace{0.4cm}
	\par
}{}

\NewEnviron{h3}{%
    \needspace{3\baselineskip}
	\vspace{5mm}
	\textsc{\BODY}
	\par
}

\NewEnviron{margeneg}{ %
\begin{addmargin}[-1cm]{0cm}
\BODY
\end{addmargin}
}

\NewEnviron{html}{%
}

\begin{document}
\meta{url}{/cours/les-statistiques/}
\meta{pid}{284}
\meta{titre}{Les statistiques en Première}
\meta{type}{cours}
Dans tout ce chapitre, on considère une série statistique représentée par le tableau :
\begin{center}
     \begin{tabularx}{0.7\linewidth}{|*{6}{>{\centering \arraybackslash }X|}}%class="compact" width="500"
          \hline
          \textbf{Valeurs} & $x_{1}$ & $x_{2}$ & ... & $x_{p}$ & \textbf{Total}
          \\ \hline
          \textbf{Effectifs} & $n_{1}$ & $n_{2}$ & ... & $n_{p}$ & $N$
          \\ \hline
     \end{tabularx}
\end{center}
\begin{h2}1. Paramètres de position\end{h2}
\cadre{bleu}{Définition}{% id="d10"
     La \textbf{moyenne} d'une série statistique est le nombre :
     \begin{center}
          $\overline x=\frac{n_{1}x_{1}+n_{2}x_{2}+. . .+n_{p}x_{p}}{N}$\nosp$=\frac{1}{N}\sum_{k=1}^{p}n_{k}x_{k}$
\end{center}}
\bloc{orange}{Exemple}{% id="e10"
     Les âges des élèves d'un lycée sont donnés par le tableau :
     \begin{center}
          \begin{tabularx}{0.95\linewidth}{|*{9}{>{\centering \arraybackslash }X|}}%class="compact" width="700"
               \hline
               \textbf{Ages} & 14 & 15 & 16 & 17 & 18 & 19 & 20 & \textbf{Total}
               \\ \hline
               \textbf{Effectifs} & 2 & 52 & 78 & 75 & 81 & 25 & 2 & 315
               \\ \hline
          \end{tabularx}
     \end{center}
     La moyenne des âges vaut:
     \par
$\overline x=\frac{1}{315}\left(2\times 14+52\times 15\right.$\nosp$\left.+78\times 16+75\times 17+81\times 18+25\times 19+2\times 20\right)$
\par
$\overline x=\frac{5304}{315} \approx  16,84 $ à $ 10^{-2} $ près.
}
\cadre{bleu}{Définition}{% id="d20"
     La \textbf{médiane} d'une série statistique est la valeur du caractère qui partage la population en deux classes de même effectif.
}
\bloc{cyan}{Remarque}{% id="r20"
     En pratique pour trouver la médiane d'une série statistique d'effectif global $N$ :
     \begin{itemize}
          \item On ordonne les valeurs du caractère dans l'ordre croissant.
          \item Si $N$ est pair, la médiane sera la moyenne des valeurs du terme de rang $\frac{N}{2}$ et du terme de rang $\frac{N}{2}+1$.
          \item Si $N$ est impair, la médiane sera la valeur du terme de rang $\frac{N+1}{2}$.
          \item Lorsque l'effectif global est élevé, il est souvent utile de calculer les effectifs cumulés pour trouver cette valeur.
     \end{itemize}
}
\bloc{orange}{Exemple}{% id="e20"
     On lance 10 fois un dé à six faces. Les résultats obtenus sont : 1~;~5~;~6~;~6~;~3~;~2~;~3~;~1~;~4~;~1
     \par
     On trie ces valeurs par ordre croissant : 1~;~1~;~1~;~2~;~3~;~3~;~4~;~5~;~6~;~6
     \par
     N=10 étant pair on effectue la moyenne du cinquième et du sixième terme (3 et 3) et on obtient donc 3.
}
\bloc{cyan}{Remarque}{% id="r21"
     Voir la fiche de \mcLien{/cours/seconde/statistiques-organisation-representation-donnees}{Statistiques en seconde} pour un exemple plus détaillé.
}
\begin{h2}2. Paramètres de dispersion\end{h2}
\cadre{bleu}{Définitions}{% id="d30"
     La \textbf{variance} d'une série statistique est le nombre :
     \begin{center}$V=\dfrac{1}{N}$\nosp$\left(n_{1}\left(x_{1}-\overline x\right)^{2}+n_{2}\left(x_{2}-\overline x\right)^{2}+. . .\right.$\nosp$\left.+n_{p}\left(x_{p}-\overline x\right)^{2}\right)$\\
$\phantom{V}=\frac{1}{N}\sum_{k=1}^{p}n_{k}\left(x_{k}-\overline x\right)^{2}$\end{center}
L'\textbf{écart-type} est la racine carrée de la variance :
\begin{center}$\sigma =\sqrt{V}$\end{center}
}
\cadre{vert}{Propriété}{% id="p40"
     La \textbf{variance} d'une série statistique est égale à :
     \begin{center}$V=\dfrac{n_{1}x_{1}^{2}+n_{2}x_{2}^{2}+. . .+n_{p}x_{p}^{2}}{N}-\overline x^{2}$\nosp$=\overline{x^{2}}-\overline x^{2}$\end{center}
}
\cadre{bleu}{Définitions}{% id="d50"
     \begin{itemize}
          \item Le \textbf{premier quartile} Q1 d'une série statistique est la plus petite valeur des termes de la série pour laquelle au moins un quart des données sont inférieures ou égales à Q1.
          \item Le \textbf{troisième quartile} Q3  d'une série statistique est la plus petite valeur des termes de la série pour laquelle au moins trois quarts des données sont inférieures ou égales à Q3.
          \item Le \textbf{premier décile} D1 d'une série statistique est la plus petite valeur  des termes de la série pour laquelle au moins 10\% des données sont inférieures ou égales à D1.
          \item Le \textbf{neuvième décile} D9 d'une série statistique est la plus petite valeur des termes de la série pour laquelle au moins 90\% des données sont inférieures ou égales à D9
     \end{itemize}
}
\cadre{bleu}{Définition}{% id="d60"
     L'\textbf{écart interquartile} est la différence entre le troisième et le premier quartile $Q_{3}-Q_{1}$.
}
\bloc{cyan}{Remarque}{% id="r60"
     L'écart interquartile mesure la dispersion autour de la médiane.
}
\begin{h2}3. Diagramme en boîte\end{h2}
\begin{center}
     \img{diag-boite-1}{0.4}%width="400" alt="boite à moustaches"
\end{center}
On peut résumer un certain nombre d'informations relatives à une série statistique grâce à un \textbf{diagramme en boîte} (aussi appelé \textit{boîte à moustache}) qui fait apparaître (voir figure ci-dessus) :
\begin{itemize}
     \item les valeurs minimum et maximum
     \item le premier et le troisième quartile (Q1 et Q3)
     \item la médiane
\end{itemize}
\bloc{orange}{Exemple}{% id="e70"
     \begin{center}
          \img{diag-boite-2}{0.6}%width="480" alt="exemple diagramme boite à moustaches"
     \end{center}
     Le figure ci-dessus représente une série statistique de valeurs extrêmes 3 et 20, de premier quartile 6, de troisième quartile 14 et de médiane 9,5.
}
\bloc{cyan}{Remarque}{% id="r70"
     Parfois, notamment lorsqu'on étudie des séries dont certaines valeurs peuvent être erronées, on remplace les valeurs minimum et maximum par les premier et neuvième déciles afin d'éliminer les valeurs aberrantes.
}

\end{document}
µ
\documentclass[a4paper]{article}

%================================================================================================================================
%
% Packages
%
%================================================================================================================================

\usepackage[T1]{fontenc} 	% pour caractères accentués
\usepackage[utf8]{inputenc}  % encodage utf8
\usepackage[french]{babel}	% langue : français
\usepackage{fourier}			% caractères plus lisibles
\usepackage[dvipsnames]{xcolor} % couleurs
\usepackage{fancyhdr}		% réglage header footer
\usepackage{needspace}		% empêcher sauts de page mal placés
\usepackage{graphicx}		% pour inclure des graphiques
\usepackage{enumitem,cprotect}		% personnalise les listes d'items (nécessaire pour ol, al ...)
\usepackage{hyperref}		% Liens hypertexte
\usepackage{pstricks,pst-all,pst-node,pstricks-add,pst-math,pst-plot,pst-tree,pst-eucl} % pstricks
\usepackage[a4paper,includeheadfoot,top=2cm,left=3cm, bottom=2cm,right=3cm]{geometry} % marges etc.
\usepackage{comment}			% commentaires multilignes
\usepackage{amsmath,environ} % maths (matrices, etc.)
\usepackage{amssymb,makeidx}
\usepackage{bm}				% bold maths
\usepackage{tabularx}		% tableaux
\usepackage{colortbl}		% tableaux en couleur
\usepackage{fontawesome}		% Fontawesome
\usepackage{environ}			% environment with command
\usepackage{fp}				% calculs pour ps-tricks
\usepackage{multido}			% pour ps tricks
\usepackage[np]{numprint}	% formattage nombre
\usepackage{tikz,tkz-tab} 			% package principal TikZ
\usepackage{pgfplots}   % axes
\usepackage{mathrsfs}    % cursives
\usepackage{calc}			% calcul taille boites
\usepackage[scaled=0.875]{helvet} % font sans serif
\usepackage{svg} % svg
\usepackage{scrextend} % local margin
\usepackage{scratch} %scratch
\usepackage{multicol} % colonnes
%\usepackage{infix-RPN,pst-func} % formule en notation polanaise inversée
\usepackage{listings}

%================================================================================================================================
%
% Réglages de base
%
%================================================================================================================================

\lstset{
language=Python,   % R code
literate=
{á}{{\'a}}1
{à}{{\`a}}1
{ã}{{\~a}}1
{é}{{\'e}}1
{è}{{\`e}}1
{ê}{{\^e}}1
{í}{{\'i}}1
{ó}{{\'o}}1
{õ}{{\~o}}1
{ú}{{\'u}}1
{ü}{{\"u}}1
{ç}{{\c{c}}}1
{~}{{ }}1
}


\definecolor{codegreen}{rgb}{0,0.6,0}
\definecolor{codegray}{rgb}{0.5,0.5,0.5}
\definecolor{codepurple}{rgb}{0.58,0,0.82}
\definecolor{backcolour}{rgb}{0.95,0.95,0.92}

\lstdefinestyle{mystyle}{
    backgroundcolor=\color{backcolour},   
    commentstyle=\color{codegreen},
    keywordstyle=\color{magenta},
    numberstyle=\tiny\color{codegray},
    stringstyle=\color{codepurple},
    basicstyle=\ttfamily\footnotesize,
    breakatwhitespace=false,         
    breaklines=true,                 
    captionpos=b,                    
    keepspaces=true,                 
    numbers=left,                    
xleftmargin=2em,
framexleftmargin=2em,            
    showspaces=false,                
    showstringspaces=false,
    showtabs=false,                  
    tabsize=2,
    upquote=true
}

\lstset{style=mystyle}


\lstset{style=mystyle}
\newcommand{\imgdir}{C:/laragon/www/newmc/assets/imgsvg/}
\newcommand{\imgsvgdir}{C:/laragon/www/newmc/assets/imgsvg/}

\definecolor{mcgris}{RGB}{220, 220, 220}% ancien~; pour compatibilité
\definecolor{mcbleu}{RGB}{52, 152, 219}
\definecolor{mcvert}{RGB}{125, 194, 70}
\definecolor{mcmauve}{RGB}{154, 0, 215}
\definecolor{mcorange}{RGB}{255, 96, 0}
\definecolor{mcturquoise}{RGB}{0, 153, 153}
\definecolor{mcrouge}{RGB}{255, 0, 0}
\definecolor{mclightvert}{RGB}{205, 234, 190}

\definecolor{gris}{RGB}{220, 220, 220}
\definecolor{bleu}{RGB}{52, 152, 219}
\definecolor{vert}{RGB}{125, 194, 70}
\definecolor{mauve}{RGB}{154, 0, 215}
\definecolor{orange}{RGB}{255, 96, 0}
\definecolor{turquoise}{RGB}{0, 153, 153}
\definecolor{rouge}{RGB}{255, 0, 0}
\definecolor{lightvert}{RGB}{205, 234, 190}
\setitemize[0]{label=\color{lightvert}  $\bullet$}

\pagestyle{fancy}
\renewcommand{\headrulewidth}{0.2pt}
\fancyhead[L]{maths-cours.fr}
\fancyhead[R]{\thepage}
\renewcommand{\footrulewidth}{0.2pt}
\fancyfoot[C]{}

\newcolumntype{C}{>{\centering\arraybackslash}X}
\newcolumntype{s}{>{\hsize=.35\hsize\arraybackslash}X}

\setlength{\parindent}{0pt}		 
\setlength{\parskip}{3mm}
\setlength{\headheight}{1cm}

\def\ebook{ebook}
\def\book{book}
\def\web{web}
\def\type{web}

\newcommand{\vect}[1]{\overrightarrow{\,\mathstrut#1\,}}

\def\Oij{$\left(\text{O}~;~\vect{\imath},~\vect{\jmath}\right)$}
\def\Oijk{$\left(\text{O}~;~\vect{\imath},~\vect{\jmath},~\vect{k}\right)$}
\def\Ouv{$\left(\text{O}~;~\vect{u},~\vect{v}\right)$}

\hypersetup{breaklinks=true, colorlinks = true, linkcolor = OliveGreen, urlcolor = OliveGreen, citecolor = OliveGreen, pdfauthor={Didier BONNEL - https://www.maths-cours.fr} } % supprime les bordures autour des liens

\renewcommand{\arg}[0]{\text{arg}}

\everymath{\displaystyle}

%================================================================================================================================
%
% Macros - Commandes
%
%================================================================================================================================

\newcommand\meta[2]{    			% Utilisé pour créer le post HTML.
	\def\titre{titre}
	\def\url{url}
	\def\arg{#1}
	\ifx\titre\arg
		\newcommand\maintitle{#2}
		\fancyhead[L]{#2}
		{\Large\sffamily \MakeUppercase{#2}}
		\vspace{1mm}\textcolor{mcvert}{\hrule}
	\fi 
	\ifx\url\arg
		\fancyfoot[L]{\href{https://www.maths-cours.fr#2}{\black \footnotesize{https://www.maths-cours.fr#2}}}
	\fi 
}


\newcommand\TitreC[1]{    		% Titre centré
     \needspace{3\baselineskip}
     \begin{center}\textbf{#1}\end{center}
}

\newcommand\newpar{    		% paragraphe
     \par
}

\newcommand\nosp {    		% commande vide (pas d'espace)
}
\newcommand{\id}[1]{} %ignore

\newcommand\boite[2]{				% Boite simple sans titre
	\vspace{5mm}
	\setlength{\fboxrule}{0.2mm}
	\setlength{\fboxsep}{5mm}	
	\fcolorbox{#1}{#1!3}{\makebox[\linewidth-2\fboxrule-2\fboxsep]{
  		\begin{minipage}[t]{\linewidth-2\fboxrule-4\fboxsep}\setlength{\parskip}{3mm}
  			 #2
  		\end{minipage}
	}}
	\vspace{5mm}
}

\newcommand\CBox[4]{				% Boites
	\vspace{5mm}
	\setlength{\fboxrule}{0.2mm}
	\setlength{\fboxsep}{5mm}
	
	\fcolorbox{#1}{#1!3}{\makebox[\linewidth-2\fboxrule-2\fboxsep]{
		\begin{minipage}[t]{1cm}\setlength{\parskip}{3mm}
	  		\textcolor{#1}{\LARGE{#2}}    
 	 	\end{minipage}  
  		\begin{minipage}[t]{\linewidth-2\fboxrule-4\fboxsep}\setlength{\parskip}{3mm}
			\raisebox{1.2mm}{\normalsize\sffamily{\textcolor{#1}{#3}}}						
  			 #4
  		\end{minipage}
	}}
	\vspace{5mm}
}

\newcommand\cadre[3]{				% Boites convertible html
	\par
	\vspace{2mm}
	\setlength{\fboxrule}{0.1mm}
	\setlength{\fboxsep}{5mm}
	\fcolorbox{#1}{white}{\makebox[\linewidth-2\fboxrule-2\fboxsep]{
  		\begin{minipage}[t]{\linewidth-2\fboxrule-4\fboxsep}\setlength{\parskip}{3mm}
			\raisebox{-2.5mm}{\sffamily \small{\textcolor{#1}{\MakeUppercase{#2}}}}		
			\par		
  			 #3
 	 		\end{minipage}
	}}
		\vspace{2mm}
	\par
}

\newcommand\bloc[3]{				% Boites convertible html sans bordure
     \needspace{2\baselineskip}
     {\sffamily \small{\textcolor{#1}{\MakeUppercase{#2}}}}    
		\par		
  			 #3
		\par
}

\newcommand\CHelp[1]{
     \CBox{Plum}{\faInfoCircle}{À RETENIR}{#1}
}

\newcommand\CUp[1]{
     \CBox{NavyBlue}{\faThumbsOUp}{EN PRATIQUE}{#1}
}

\newcommand\CInfo[1]{
     \CBox{Sepia}{\faArrowCircleRight}{REMARQUE}{#1}
}

\newcommand\CRedac[1]{
     \CBox{PineGreen}{\faEdit}{BIEN R\'EDIGER}{#1}
}

\newcommand\CError[1]{
     \CBox{Red}{\faExclamationTriangle}{ATTENTION}{#1}
}

\newcommand\TitreExo[2]{
\needspace{4\baselineskip}
 {\sffamily\large EXERCICE #1\ (\emph{#2 points})}
\vspace{5mm}
}

\newcommand\img[2]{
          \includegraphics[width=#2\paperwidth]{\imgdir#1}
}

\newcommand\imgsvg[2]{
       \begin{center}   \includegraphics[width=#2\paperwidth]{\imgsvgdir#1} \end{center}
}


\newcommand\Lien[2]{
     \href{#1}{#2 \tiny \faExternalLink}
}
\newcommand\mcLien[2]{
     \href{https~://www.maths-cours.fr/#1}{#2 \tiny \faExternalLink}
}

\newcommand{\euro}{\eurologo{}}

%================================================================================================================================
%
% Macros - Environement
%
%================================================================================================================================

\newenvironment{tex}{ %
}
{%
}

\newenvironment{indente}{ %
	\setlength\parindent{10mm}
}

{
	\setlength\parindent{0mm}
}

\newenvironment{corrige}{%
     \needspace{3\baselineskip}
     \medskip
     \textbf{\textsc{Corrigé}}
     \medskip
}
{
}

\newenvironment{extern}{%
     \begin{center}
     }
     {
     \end{center}
}

\NewEnviron{code}{%
	\par
     \boite{gray}{\texttt{%
     \BODY
     }}
     \par
}

\newenvironment{vbloc}{% boite sans cadre empeche saut de page
     \begin{minipage}[t]{\linewidth}
     }
     {
     \end{minipage}
}
\NewEnviron{h2}{%
    \needspace{3\baselineskip}
    \vspace{0.6cm}
	\noindent \MakeUppercase{\sffamily \large \BODY}
	\vspace{1mm}\textcolor{mcgris}{\hrule}\vspace{0.4cm}
	\par
}{}

\NewEnviron{h3}{%
    \needspace{3\baselineskip}
	\vspace{5mm}
	\textsc{\BODY}
	\par
}

\NewEnviron{margeneg}{ %
\begin{addmargin}[-1cm]{0cm}
\BODY
\end{addmargin}
}

\NewEnviron{html}{%
}

\begin{document}
\meta{url}{/cours/loi-probabilite/}
\meta{pid}{298}
\meta{titre}{Variable aléatoire - Loi de probabilité}
\meta{type}{cours}
\begin{h2}I - Rappels de probabilités\end{h2}
\cadre{bleu}{Définitions}{%id="d10"
     Une expérience \textbf{aléatoire} est une expérience dont le résultat dépend du hasard.
     \par
     Chacun des résultats possibles s'appelle une \textbf{éventualité} (ou une \textbf{issue} ou un \textbf{évènement élémentaire})
     \par
     L'ensemble de tous les résultats possibles d'une expérience aléatoire s'appelle l'\textbf{univers} de l'expérience.
}
\bloc{orange}{Exemple}{%id="e10"
     Par exemple, le lancer d'un dé à six faces est une expérience aléatoire. \textit{"Obtenir un 6 avec le dé"} est une éventualité. L'univers possède 6 éventualités; on peut le représenter par l'ensemble:
     \par
     $\Omega =\left\{1;2;3;4;5;6\right\}$
}
\cadre{bleu}{Définition}{%id="d20"
     Soit une expérience aléatoire ayant comme univers:
     \par
     $\Omega =\left\{x_{1}; x_{2};. . .; x_{n}\right\}$
     \par
     On définit une \textbf{probabilité} sur $\Omega $ en associant, à chaque éventualité $x_{i}$, un réel $p_{i}$ compris entre 0 et 1 tel que la somme de tous les $p_{i}$ soit égale à 1.
}
\bloc{cyan}{Remarques}{%id="r20"
     \begin{itemize}
          \item En pratique, pour définir les probabilités $p_{i}$ on peut effectuer un très grand nombre de fois l'expérience aléatoire. La fréquence des résultats obtenus permet d'obtenir une estimation de la loi de probabilité. Par exemple, si en lançant 1 000 000 de fois un dé, on obtient 166 724 fois la face "6" on considérera que la probabilité d'obtenir un "6" est d'environ $\frac{166\ 724}{1\ 000\ 000} \approx \frac{1}{6}$
          \item A condition de faire certaines hypothèses (par exemple : "\textit{le dé n'est pas truqué}") les théorèmes qui suivent permettent de calculer les lois de probabilité de certaines expériences sans avoir recours aux statistiques. Les statistiques peuvent alors servir à valider les hypothèses que l'on a faites au départ.
     \end{itemize}
}
\cadre{bleu}{Définition et propriété}{%id="d30"
     On dit que l'on est en situation d'\textbf{équiprobabilité} si toutes les éventualités on la même probabilité.
     \par
     Cette probabilité est alors $p=\frac{1}{n}$ où $n$ est le nombre total d'éventualités.
}
\bloc{cyan}{Remarque}{%id="r30"
     Dans les exercices, on considérera qu'il y a équiprobabilité si l'énoncé indique que l'on jette une pièce "\textit{équilibrée}", qu'on lance un dé "\textit{non truqué}", qu'on tire une carte "\textit{au hasard}" , etc.
}
\bloc{orange}{Exemples}{%id="e30"
     \begin{itemize}
          \item Si l'on jette une pièce non truquée, la probabilité d'obtenir \textit{pile} est $p=\frac{1}{2}$
          \item Pour un dé à six faces non truqué, la probabilité d'obtenir une face donnée est $p=\frac{1}{6}$
     \end{itemize}
}
\begin{h2}II - Variables aléatoires\end{h2}
\cadre{bleu}{Définition}{%id="d40"
     On définit une \textbf{variable aléatoire} en associant un nombre réel à chaque éventualité d'une expérience aléatoire.
}
\bloc{orange}{Exemples}{%id="e40"
     \begin{itemize}
          \item On mise 1€ sur le numéro 1 à la roulette. On gagne 35€ (36€ - la mise) si le numéro sort. On perd sa mise (soit 1€) dans les autres cas. On peut définir une variable aléatoire représentant le gain algébrique du joueur. Cette variable aléatoire peut prendre la valeur 35 (en cas de gain) ou -1 (en cas de perte).
          \item On lance 4 fois une pièce de monnaie. On peut définir une variable aléatoire égale au nombre de "\textit{faces}" obtenues.
          \par
          Les valeurs possibles pour cette variable sont : 0; 1; 2; 3 ou 4.
     \end{itemize}
}
\bloc{cyan}{Notations}{%id="n40"
     \begin{itemize}
          \item On note généralement une variable aléatoire à l'aide d'une lettre majuscule (le plus souvent $X$)
          \item Si la variable aléatoire $X$ peut prendre les valeurs $a_{1}, a_{2}, . . . a_{n}$, on note $\left(X=a_{i}\right)$ l'évènement : "$X$ prend la valeur $a_{i}$"
     \end{itemize}
}
\cadre{bleu}{Définition}{%id="d50"
     La loi de probabilité d'une variable aléatoire $X$ associe à chaque valeur $a_{i}$ prise par $X$ la probabilité de l'événement $\left(X = a_{i}\right)$.
     \par
     On la représente généralement sous forme de tableau.
}
\bloc{orange}{Exemples}{%id="e50"
     \begin{itemize}
          \item Si l'on reprend l'exemple de la roulette (ci-dessus) et si on suppose que la probabilité de sortie de chacun des 37 numéros (0 à 36) est égale, la probabilité de gain est de $\frac{1}{37}$ et la probabilité de perte $\frac{36}{37}$.
          \par
          La loi de probabilité est donnée par le tableau suivant :
          \begin{tabularx}{0.8\linewidth}{|*{3}{>{\centering \arraybackslash }X|}}%class="compact" width="600"
               \hline
               $a_{i}$ & $-1$ & $35$\\ \hline
               $p\left(X=a_{i}\right)$ & $\frac{36}{37}$ & $\frac{1}{37}$\\ \hline
          \end{tabularx}
          \item Si on lance 4 fois une pièce de monnaie équilibrée, on montre à l'aide d'un arbre que la variable aléatoire $X$ donnant le nombre de "\textit{faces}" obtenues suit la loi de probabilité donnée par le tableau ci-dessous :
          \begin{center}
               \begin{tabularx}{0.8\linewidth}{|*{6}{>{\centering \arraybackslash }X|}}%class="compact" width="600"
                    \hline
                    $a_{i}$ & $0$ & $1$ & $2$ & $3$ & $4$ \\ \hline
                    $p\left(X=a_{i}\right)$ & $\frac{1}{16}$ & $\frac{1}{4}$ & $\frac{3}{8}$ & $\frac{1}{4}$ & $\frac{1}{16}$  \\ \hline
               \end{tabularx}
          \end{center}
     \end{itemize}
}
\cadre{bleu}{Définition (Espérance mathématique)}{%id="d60"
     Soit $X$ une variable aléatoire qui prend les valeurs $x_{i}$ avec les probabilités $p_{i}=p\left(X=x_{i}\right)$.
     \par
     On appelle \textbf{espérance mathématique} de $X$ le nombre :
     \begin{center}
          $E\left(X\right)= x_{1}\times p_{1}+x_{2}\times p_{2}+. . . +x_{n}\times p_{n} $\nosp$= \sum_{i=1}^{n}p_{i} x_{i}$
     \end{center}
}
\bloc{cyan}{Remarque}{%id="r60"
     Ce nombre peut s'interpréter comme une valeur moyenne de $X$ si l'on répète un grand nombre de fois l'expérience.
}
\bloc{orange}{Exemple}{%id="e60"
     Pour l'exemple de la roulette on a :
     \par
     $E\left(X\right)=-1\times \frac{36}{37}+35\times \frac{1}{37}=-\frac{1}{37}$
     \par
     L'espérance est négative, ce qui signifie qu'en moyenne, le jeu n'est pas favorable au joueur.
}
\cadre{bleu}{Définition (Variance - Ecart-type)}{%id="d70"
     Soit $X$ une variable aléatoire d'espérance mathématique $\overline X$.
     \par
     La \textbf{variance} de la variable aléatoire $X$ est le nombre réel positif :
     \begin{center}
          $V\left(X\right)=E\left(\left(X-\overline X\right)^{2}\right)$
     \end{center}
     \smallskip
     L'\textbf{écart-type} est égal à la racine carrée de la variance :
     \begin{center}
          $\sigma \left(X\right)=\sqrt{V\left(X\right)}$
     \end{center}
}
\bloc{cyan}{Remarque}{%id="r70"
     D'après la définition de la variance, si $X$ les valeurs $x_{i}$ avec les probabilités $p_{i}$ :
     \par
     $V\left(X\right)=\sum_{i=1}^{n}p_{i}\left(x_{i}-\overline X\right)^{2}$
     \par
     En développant les carrés, on montre que la variance peut également s'écrire :
     \par
     $V\left(X\right) = E\left(X^{2}\right)-\overline X^{2} =\left(\sum_{i=1}^{n}p_{i} x_{i}^{2}\right)- \overline X^{2}$
}
\cadre{vert}{Propriétés}{%id="p80"
     Soit $X$ une variable aléatoire qui prend les valeurs $x_{i}$ avec les probabilités $p_{i}$. On note $aX+b$ la variable aléatoire qui prend les valeurs $ax_{i}+b$ avec les mêmes probabilités $p_{i}$.
     \par
     On a alors :
     \begin{itemize}
          \item $E\left(aX+b\right)=aE\left(X\right)+b$
          \item $V\left(aX+b\right)=a^{2}\times V\left(X\right)$
          \item $\sigma \left(aX+b\right)=|a|\times \sigma \left(X\right)$
     \end{itemize}
}
\bloc{orange}{Exemple}{%id="e80"
     Soit $X$ un variable aléatoire qui représente le gain algébrique en euro à un jeu d'argent.
     \begin{itemize}
          \item Si on augmente les gains de 1 euro, l'espérance mathématique augmentera de 1, la variance et l'écart-type ne seront pas modifiés ($a=1 ; b=1$).
          \item Si on double les gains, l'espérance mathématique et l'écart-type seront doublés, la variance sera quadruplée ($a=2 ; b=0$).
     \end{itemize}
}

\end{document}
µ
\documentclass[a4paper]{article}

%================================================================================================================================
%
% Packages
%
%================================================================================================================================

\usepackage[T1]{fontenc} 	% pour caractères accentués
\usepackage[utf8]{inputenc}  % encodage utf8
\usepackage[french]{babel}	% langue : français
\usepackage{fourier}			% caractères plus lisibles
\usepackage[dvipsnames]{xcolor} % couleurs
\usepackage{fancyhdr}		% réglage header footer
\usepackage{needspace}		% empêcher sauts de page mal placés
\usepackage{graphicx}		% pour inclure des graphiques
\usepackage{enumitem,cprotect}		% personnalise les listes d'items (nécessaire pour ol, al ...)
\usepackage{hyperref}		% Liens hypertexte
\usepackage{pstricks,pst-all,pst-node,pstricks-add,pst-math,pst-plot,pst-tree,pst-eucl} % pstricks
\usepackage[a4paper,includeheadfoot,top=2cm,left=3cm, bottom=2cm,right=3cm]{geometry} % marges etc.
\usepackage{comment}			% commentaires multilignes
\usepackage{amsmath,environ} % maths (matrices, etc.)
\usepackage{amssymb,makeidx}
\usepackage{bm}				% bold maths
\usepackage{tabularx}		% tableaux
\usepackage{colortbl}		% tableaux en couleur
\usepackage{fontawesome}		% Fontawesome
\usepackage{environ}			% environment with command
\usepackage{fp}				% calculs pour ps-tricks
\usepackage{multido}			% pour ps tricks
\usepackage[np]{numprint}	% formattage nombre
\usepackage{tikz,tkz-tab} 			% package principal TikZ
\usepackage{pgfplots}   % axes
\usepackage{mathrsfs}    % cursives
\usepackage{calc}			% calcul taille boites
\usepackage[scaled=0.875]{helvet} % font sans serif
\usepackage{svg} % svg
\usepackage{scrextend} % local margin
\usepackage{scratch} %scratch
\usepackage{multicol} % colonnes
%\usepackage{infix-RPN,pst-func} % formule en notation polanaise inversée
\usepackage{listings}

%================================================================================================================================
%
% Réglages de base
%
%================================================================================================================================

\lstset{
language=Python,   % R code
literate=
{á}{{\'a}}1
{à}{{\`a}}1
{ã}{{\~a}}1
{é}{{\'e}}1
{è}{{\`e}}1
{ê}{{\^e}}1
{í}{{\'i}}1
{ó}{{\'o}}1
{õ}{{\~o}}1
{ú}{{\'u}}1
{ü}{{\"u}}1
{ç}{{\c{c}}}1
{~}{{ }}1
}


\definecolor{codegreen}{rgb}{0,0.6,0}
\definecolor{codegray}{rgb}{0.5,0.5,0.5}
\definecolor{codepurple}{rgb}{0.58,0,0.82}
\definecolor{backcolour}{rgb}{0.95,0.95,0.92}

\lstdefinestyle{mystyle}{
    backgroundcolor=\color{backcolour},   
    commentstyle=\color{codegreen},
    keywordstyle=\color{magenta},
    numberstyle=\tiny\color{codegray},
    stringstyle=\color{codepurple},
    basicstyle=\ttfamily\footnotesize,
    breakatwhitespace=false,         
    breaklines=true,                 
    captionpos=b,                    
    keepspaces=true,                 
    numbers=left,                    
xleftmargin=2em,
framexleftmargin=2em,            
    showspaces=false,                
    showstringspaces=false,
    showtabs=false,                  
    tabsize=2,
    upquote=true
}

\lstset{style=mystyle}


\lstset{style=mystyle}
\newcommand{\imgdir}{C:/laragon/www/newmc/assets/imgsvg/}
\newcommand{\imgsvgdir}{C:/laragon/www/newmc/assets/imgsvg/}

\definecolor{mcgris}{RGB}{220, 220, 220}% ancien~; pour compatibilité
\definecolor{mcbleu}{RGB}{52, 152, 219}
\definecolor{mcvert}{RGB}{125, 194, 70}
\definecolor{mcmauve}{RGB}{154, 0, 215}
\definecolor{mcorange}{RGB}{255, 96, 0}
\definecolor{mcturquoise}{RGB}{0, 153, 153}
\definecolor{mcrouge}{RGB}{255, 0, 0}
\definecolor{mclightvert}{RGB}{205, 234, 190}

\definecolor{gris}{RGB}{220, 220, 220}
\definecolor{bleu}{RGB}{52, 152, 219}
\definecolor{vert}{RGB}{125, 194, 70}
\definecolor{mauve}{RGB}{154, 0, 215}
\definecolor{orange}{RGB}{255, 96, 0}
\definecolor{turquoise}{RGB}{0, 153, 153}
\definecolor{rouge}{RGB}{255, 0, 0}
\definecolor{lightvert}{RGB}{205, 234, 190}
\setitemize[0]{label=\color{lightvert}  $\bullet$}

\pagestyle{fancy}
\renewcommand{\headrulewidth}{0.2pt}
\fancyhead[L]{maths-cours.fr}
\fancyhead[R]{\thepage}
\renewcommand{\footrulewidth}{0.2pt}
\fancyfoot[C]{}

\newcolumntype{C}{>{\centering\arraybackslash}X}
\newcolumntype{s}{>{\hsize=.35\hsize\arraybackslash}X}

\setlength{\parindent}{0pt}		 
\setlength{\parskip}{3mm}
\setlength{\headheight}{1cm}

\def\ebook{ebook}
\def\book{book}
\def\web{web}
\def\type{web}

\newcommand{\vect}[1]{\overrightarrow{\,\mathstrut#1\,}}

\def\Oij{$\left(\text{O}~;~\vect{\imath},~\vect{\jmath}\right)$}
\def\Oijk{$\left(\text{O}~;~\vect{\imath},~\vect{\jmath},~\vect{k}\right)$}
\def\Ouv{$\left(\text{O}~;~\vect{u},~\vect{v}\right)$}

\hypersetup{breaklinks=true, colorlinks = true, linkcolor = OliveGreen, urlcolor = OliveGreen, citecolor = OliveGreen, pdfauthor={Didier BONNEL - https://www.maths-cours.fr} } % supprime les bordures autour des liens

\renewcommand{\arg}[0]{\text{arg}}

\everymath{\displaystyle}

%================================================================================================================================
%
% Macros - Commandes
%
%================================================================================================================================

\newcommand\meta[2]{    			% Utilisé pour créer le post HTML.
	\def\titre{titre}
	\def\url{url}
	\def\arg{#1}
	\ifx\titre\arg
		\newcommand\maintitle{#2}
		\fancyhead[L]{#2}
		{\Large\sffamily \MakeUppercase{#2}}
		\vspace{1mm}\textcolor{mcvert}{\hrule}
	\fi 
	\ifx\url\arg
		\fancyfoot[L]{\href{https://www.maths-cours.fr#2}{\black \footnotesize{https://www.maths-cours.fr#2}}}
	\fi 
}


\newcommand\TitreC[1]{    		% Titre centré
     \needspace{3\baselineskip}
     \begin{center}\textbf{#1}\end{center}
}

\newcommand\newpar{    		% paragraphe
     \par
}

\newcommand\nosp {    		% commande vide (pas d'espace)
}
\newcommand{\id}[1]{} %ignore

\newcommand\boite[2]{				% Boite simple sans titre
	\vspace{5mm}
	\setlength{\fboxrule}{0.2mm}
	\setlength{\fboxsep}{5mm}	
	\fcolorbox{#1}{#1!3}{\makebox[\linewidth-2\fboxrule-2\fboxsep]{
  		\begin{minipage}[t]{\linewidth-2\fboxrule-4\fboxsep}\setlength{\parskip}{3mm}
  			 #2
  		\end{minipage}
	}}
	\vspace{5mm}
}

\newcommand\CBox[4]{				% Boites
	\vspace{5mm}
	\setlength{\fboxrule}{0.2mm}
	\setlength{\fboxsep}{5mm}
	
	\fcolorbox{#1}{#1!3}{\makebox[\linewidth-2\fboxrule-2\fboxsep]{
		\begin{minipage}[t]{1cm}\setlength{\parskip}{3mm}
	  		\textcolor{#1}{\LARGE{#2}}    
 	 	\end{minipage}  
  		\begin{minipage}[t]{\linewidth-2\fboxrule-4\fboxsep}\setlength{\parskip}{3mm}
			\raisebox{1.2mm}{\normalsize\sffamily{\textcolor{#1}{#3}}}						
  			 #4
  		\end{minipage}
	}}
	\vspace{5mm}
}

\newcommand\cadre[3]{				% Boites convertible html
	\par
	\vspace{2mm}
	\setlength{\fboxrule}{0.1mm}
	\setlength{\fboxsep}{5mm}
	\fcolorbox{#1}{white}{\makebox[\linewidth-2\fboxrule-2\fboxsep]{
  		\begin{minipage}[t]{\linewidth-2\fboxrule-4\fboxsep}\setlength{\parskip}{3mm}
			\raisebox{-2.5mm}{\sffamily \small{\textcolor{#1}{\MakeUppercase{#2}}}}		
			\par		
  			 #3
 	 		\end{minipage}
	}}
		\vspace{2mm}
	\par
}

\newcommand\bloc[3]{				% Boites convertible html sans bordure
     \needspace{2\baselineskip}
     {\sffamily \small{\textcolor{#1}{\MakeUppercase{#2}}}}    
		\par		
  			 #3
		\par
}

\newcommand\CHelp[1]{
     \CBox{Plum}{\faInfoCircle}{À RETENIR}{#1}
}

\newcommand\CUp[1]{
     \CBox{NavyBlue}{\faThumbsOUp}{EN PRATIQUE}{#1}
}

\newcommand\CInfo[1]{
     \CBox{Sepia}{\faArrowCircleRight}{REMARQUE}{#1}
}

\newcommand\CRedac[1]{
     \CBox{PineGreen}{\faEdit}{BIEN R\'EDIGER}{#1}
}

\newcommand\CError[1]{
     \CBox{Red}{\faExclamationTriangle}{ATTENTION}{#1}
}

\newcommand\TitreExo[2]{
\needspace{4\baselineskip}
 {\sffamily\large EXERCICE #1\ (\emph{#2 points})}
\vspace{5mm}
}

\newcommand\img[2]{
          \includegraphics[width=#2\paperwidth]{\imgdir#1}
}

\newcommand\imgsvg[2]{
       \begin{center}   \includegraphics[width=#2\paperwidth]{\imgsvgdir#1} \end{center}
}


\newcommand\Lien[2]{
     \href{#1}{#2 \tiny \faExternalLink}
}
\newcommand\mcLien[2]{
     \href{https~://www.maths-cours.fr/#1}{#2 \tiny \faExternalLink}
}

\newcommand{\euro}{\eurologo{}}

%================================================================================================================================
%
% Macros - Environement
%
%================================================================================================================================

\newenvironment{tex}{ %
}
{%
}

\newenvironment{indente}{ %
	\setlength\parindent{10mm}
}

{
	\setlength\parindent{0mm}
}

\newenvironment{corrige}{%
     \needspace{3\baselineskip}
     \medskip
     \textbf{\textsc{Corrigé}}
     \medskip
}
{
}

\newenvironment{extern}{%
     \begin{center}
     }
     {
     \end{center}
}

\NewEnviron{code}{%
	\par
     \boite{gray}{\texttt{%
     \BODY
     }}
     \par
}

\newenvironment{vbloc}{% boite sans cadre empeche saut de page
     \begin{minipage}[t]{\linewidth}
     }
     {
     \end{minipage}
}
\NewEnviron{h2}{%
    \needspace{3\baselineskip}
    \vspace{0.6cm}
	\noindent \MakeUppercase{\sffamily \large \BODY}
	\vspace{1mm}\textcolor{mcgris}{\hrule}\vspace{0.4cm}
	\par
}{}

\NewEnviron{h3}{%
    \needspace{3\baselineskip}
	\vspace{5mm}
	\textsc{\BODY}
	\par
}

\NewEnviron{margeneg}{ %
\begin{addmargin}[-1cm]{0cm}
\BODY
\end{addmargin}
}

\NewEnviron{html}{%
}

\begin{document}
\meta{url}{/cours/loi-binomiale/}
\meta{pid}{306}
\meta{titre}{Schéma de Bernoulli - Loi binomiale}
\meta{type}{cours}
\begin{h2}1. Loi de Bernoulli\end{h2}
\cadre{bleu}{Définition}{%id="d10"
     On appelle \textbf{épreuve de Bernoulli} de paramètre $p$ (avec $0 < p < 1$) une expérience aléatoire ayant deux issues :
     \begin{itemize}
          \item l'une appelée \textbf{succès} (généralement notée $S$) de probabilité $p$,
          \item l'autre appelée \textbf{échec} (généralement notée $\overline S$) de probabilité $1-p$.
     \end{itemize}
}
\cadre{bleu}{Définition}{%id="d20"
     On considère la variable aléatoire $X$ qui vaut $1$ en cas de succès et $0$ en cas d'échec.
     \par
     Cette variable aléatoire suit la \textbf{loi de Bernoulli de paramètre $p$}, définie par le tableau suivant:
     \begin{center}
          \begin{tabularx}{0.4\linewidth}{|*{3}{>{\centering \arraybackslash }X|}}%class="compact" width="600"
               \hline
               \textbf{$x_{i}$} & $0$ & $1$\\ \hline
               \textbf{$p\left(X=x_{i}\right)$} & $1-p$ & $p$\\ \hline
          \end{tabularx}
     \end{center}
}
\bloc{orange}{Exemple}{%id="e20"
     Au bonneteau, deux cartes noires et une carte rouge sont présentées, faces cachées, sur la table.
     \begin{center}
          \img{bonneteau}{0.25}%width="250" alt="Bonneteau"
     \end{center}
     On suppose qu'un joueur choisit une carte complètement au hasard.
     \par
     On a affaire à une loi de Bernoulli de paramètre $p=\frac{1}{3}$.
     \par
     La probabilité de succès est : $p\left(S\right)=p=\frac{1}{3}$ et la probabilité d'échec $p\left(\overline S\right)=1-p=\frac{2}{3}$
}
\cadre{vert}{Propriété}{%id="p30"
     L'espérance mathématique d'une variable aléatoire $X$ qui suit une loi de Bernoulli de paramètre $p$ est :
     \begin{center}
          $E\left(X\right)=p.$
     \end{center}
}
\bloc{cyan}{Démonstration}{%id="m30"
     D'après la définition de l'espérance mathématique :
     \par
     $E\left(X\right)=0\times \left(1-p\right)+1\times p=p.$
}
\begin{h2}2. Schéma de Bernoulli - Loi binomiale\end{h2}
\cadre{bleu}{Définition}{%id="d40"
     On appelle \textbf{schéma de Bernoulli} la répétition d'épreuves de Bernoulli \textbf{identiques} et \textbf{indépendantes}.
}
\bloc{orange}{Exemple}{%id="e40"
     Une urne contient 2 boules rouges et 3 boules blanches. On tire 3 boules au hasard.
     \begin{itemize}
          \item Si l'épreuve s'effectue sans remise, les tirages ne sont ni identiques, ni indépendants. En effet, le fait d'avoir retiré une boule lors du premier tirage fait que le second tirage n'est pas identique au premier.
          \item Si l'épreuve s'effectue avec remise, on pourra, par contre, considérer que les tirages sont identiques et indépendants. On a donc bien, dans ce cas, un schéma de Bernoulli.
     \end{itemize}
}
\cadre{bleu}{Définition}{%id="d50"
     Soit $X$ la variable aléatoire qui \textbf{compte le nombre de succès} dans un schéma de Bernouilli constitué de $n$ épreuves ayant chacune une probabilité de succès égale à $p$.
     \par
     La variable aléatoire X suit une loi appelée \textbf{loi binomiale de paramètres $n$ et $p$}, souvent notée $\mathscr B \left(n ; p\right)$.
}
\bloc{orange}{Exemple}{%id="e50"
     On reprend l'exemple précédent : tirage au hasard et avec remise de 3 boules parmi 5 boules comportant 2 boules rouges et 3 boules blanches. On considère la variable aléatoire $X$ qui compte le nombre de boules rouges obtenues. La variable $X$ sur une loi binomiale de paramètres 3 (nombre d'épreuves) et $\frac{2}{5}$ (probabilité d'obtenir une boule rouge lors d'une épreuve).
     \par
     Ce schéma peut être représenté par l'arbre suivant :
     \begin{center}
          \begin{extern}%width="420" alt="Arbre schéma de Bernoulli"
               %:-+-+-+- Engendré par : http://math.et.info.free.fr/TikZ/Arbre/
               % Racine à Gauche, développement vers la droite
               \resizebox{8cm}{!}{
                    \begin{tikzpicture}[xscale=1,yscale=1]
                         % Styles (MODIFIABLES)
                         \tikzstyle{fleche}=[>=latex,thick]
                         \tikzstyle{noeud}=[circle,draw]
                         \tikzstyle{feuille}=[circle,draw]
                         \tikzstyle{etiquette}=[midway,fill=white]
                         % Dimensions (MODIFIABLES)
                         \def\DistanceInterNiveaux{3}
                         \def\DistanceInterFeuilles{2}
                         % Dimensions calculées (NON MODIFIABLES)
                         \def\NiveauA{(0)*\DistanceInterNiveaux}
                         \def\NiveauB{(1.6666666666666665)*\DistanceInterNiveaux}
                         \def\NiveauC{(3)*\DistanceInterNiveaux}
                         \def\NiveauD{(4)*\DistanceInterNiveaux}
                         \def\InterFeuilles{(-1)*\DistanceInterFeuilles}
                         % Noeuds (MODIFIABLES : Styles et Coefficients d'InterFeuilles)
                         \node[noeud] (R) at ({\NiveauA},{(3.5)*\InterFeuilles}) {$ $};
                         \node[noeud] (Ra) at ({\NiveauB},{(1.5)*\InterFeuilles}) {$S$};
                         \node[noeud] (Raa) at ({\NiveauC},{(0.5)*\InterFeuilles}) {$S$};
                         \node[feuille] (Raaa) at ({\NiveauD},{(0)*\InterFeuilles}) {$S$};
                         \node[feuille] (Raab) at ({\NiveauD},{(1)*\InterFeuilles}) {$\overline{S}$};
                         \node[noeud] (Rab) at ({\NiveauC},{(2.5)*\InterFeuilles}) {$\overline{S}$};
                         \node[feuille] (Raba) at ({\NiveauD},{(2)*\InterFeuilles}) {$S$};
                         \node[feuille] (Rabb) at ({\NiveauD},{(3)*\InterFeuilles}) {$\overline{S}$};
                         \node[noeud] (Rb) at ({\NiveauB},{(5.5)*\InterFeuilles}) {$\overline{S}$};
                         \node[noeud] (Rba) at ({\NiveauC},{(4.5)*\InterFeuilles}) {$S$};
                         \node[feuille] (Rbaa) at ({\NiveauD},{(4)*\InterFeuilles}) {$S$};
                         \node[feuille] (Rbab) at ({\NiveauD},{(5)*\InterFeuilles}) {$\overline{S}$};
                         \node[noeud] (Rbb) at ({\NiveauC},{(6.5)*\InterFeuilles}) {$\overline{S}$};
                         \node[feuille] (Rbba) at ({\NiveauD},{(6)*\InterFeuilles}) {$S$};
                         \node[feuille] (Rbbb) at ({\NiveauD},{(7)*\InterFeuilles}) {$\overline{S}$};
                         % Arcs (MODIFIABLES : Styles)
                         \draw[fleche] (R)--(Ra) node[etiquette] {$\dfrac{2}{5}$};
                         \draw[fleche] (Ra)--(Raa) node[etiquette] {$\dfrac{2}{5}$};
                         \draw[fleche] (Raa)--(Raaa) node[etiquette] {$\dfrac{2}{5}$};
                         \draw[fleche] (Raa)--(Raab) node[etiquette] {$\dfrac{3}{5}$};
                         \draw[fleche] (Ra)--(Rab) node[etiquette] {$\dfrac{3}{5}$};
                         \draw[fleche] (Rab)--(Raba) node[etiquette] {$\dfrac{2}{5}$};
                         \draw[fleche] (Rab)--(Rabb) node[etiquette] {$\dfrac{3}{5}$};
                         \draw[fleche] (R)--(Rb) node[etiquette] {$\dfrac{3}{5}$};
                         \draw[fleche] (Rb)--(Rba) node[etiquette] {$\dfrac{2}{5}$};
                         \draw[fleche] (Rba)--(Rbaa) node[etiquette] {$\dfrac{2}{5}$};
                         \draw[fleche] (Rba)--(Rbab) node[etiquette] {$\dfrac{3}{5}$};
                         \draw[fleche] (Rb)--(Rbb) node[etiquette] {$\dfrac{3}{5}$};
                         \draw[fleche] (Rbb)--(Rbba) node[etiquette] {$\dfrac{2}{5}$};
                         \draw[fleche] (Rbb)--(Rbbb) node[etiquette] {$\dfrac{3}{5}$};
                    \end{tikzpicture}
                    %:-+-+-+-+- Fin
               }
          \end{extern}
     \end{center}
     Grâce à l'arbre on voit que :
     \begin{itemize}
          \item la probabilité d'avoir 3 succès (c'est à dire 3 boules rouges) est $p\left(X=3\right) =\left(\frac{2}{5}\right)^{3}=\frac{8}{125}$ ;
          \item il y a 3 chemins qui correspondent à 2 succès ($SS\overline S, S\overline SS, \overline SSS$). La probabilité d'obtenir 2 boules rouges est donc :
          \par
          $p\left(X=2\right) =\left(\frac{2}{5}\right)^{2}\times \frac{3}{5}+\left(\frac{2}{5}\right)^{2}\times \frac{3}{5}+\left(\frac{2}{5}\right)^{2}\times \frac{3}{5}$\nosp$=3\times \left[\left(\frac{2}{5}\right)^{2}\times \frac{3}{5}\right]=\frac{36}{125}$ ;
          \item il y a également 3 chemins qui correspondent à un unique succès ($S\overline S\overline S, \overline SS\overline S, \overline S\overline SS$). La probabilité d'obtenir une unique boule rouge est donc :
          \par
          $p\left(X=1\right) = \frac{2}{5}\times \left(\frac{3}{5}\right)^{2}+ \frac{2}{5}\times \left(\frac{3}{5}\right)^{2}+ \frac{2}{5}\times \left(\frac{3}{5}\right)^{2}$\nosp$=3\times \left[ \frac{2}{5}\times \left(\frac{3}{5}\right)^{2}\right]=\frac{54}{125}$ ;
          \item la probabilité de n'avoir aucune boule rouge est $p\left(X=0\right) =\left(\frac{3}{5}\right)^{3}=\frac{27}{125}$.
     \end{itemize}
     La loi de $X$ est donc donnée par le tableau suivant :
     \begin{center}
          \begin{tabularx}{0.8\linewidth}{|*{5}{>{\centering \arraybackslash }X|}}%class="compact" width="600"
               \hline
               \textbf{$x_{i}$} & $0$ & $1$ & $2$ & $3$\\ \hline
               \textbf{$p\left(X=x_{i}\right)$} & $\frac{27}{125}$ & $\frac{54}{125}$ & $\frac{36}{125}$ & $\frac{8}{125}$\\ \hline
          \end{tabularx}
     \end{center}
     On vérifie bien que $\frac{27}{125}+\frac{54}{125}+\frac{36}{125}+\frac{8}{125}=1$.
}
\cadre{vert}{Propriété}{%id="p60"
     L'espérance mathématique d'une variable aléatoire $X$ qui suit une loi binomiale $\mathscr B \left(n ; p\right)$ est :
     \begin{center}
          $E\left(X\right)=np.$
     \end{center}
}
\bloc{orange}{Exemple}{%id="e60"
     Dans l'exemple précédent, la variable X suit une loi binomiale $\mathscr B (3 ; \dfrac{2}{5})$.
     \par
     Son espérance mathématique est donc $E\left(X\right)=3\times \frac{2}{5}=\frac{6}{5}=1,2$.
     \par
     On vérifie que l'on obtient bien le même résultat en utilisant le tableau de la loi de $X$ et la définition de l'espérance mathématique :
     \par
     $E\left(X\right)=0\times \frac{27}{125}+1\times \frac{54}{125}+2\times \frac{36}{125}+3\times \frac{8}{125}$\nosp$=\frac{150}{125}=1,2$.
}
\begin{h2}3. Coefficients binomiaux\end{h2}
\cadre{bleu}{Définition}{%id="d70"
     On considère un arbre pondéré représentant une loi binomiale $\mathscr B \left(n ; p\right)$.
     \par
     Le \textbf{coefficient binomial} $\begin{pmatrix} n \\ k \end{pmatrix}$ (lire \textit{$k$ parmi $n$}) est le nombre de chemins qui correspondent à $k$ succès.
}
\bloc{orange}{Exemple}{%id="e70"
     On reprend le même exemple que précédemment. On a vu, par exemple, qu'il y avait 3 chemins correspondant à 2 succès. On a donc $\begin{pmatrix} 3 \\ 2 \end{pmatrix}= 3$.
}
\bloc{cyan}{Remarques}{%id="r70"
     \begin{itemize}
          \item On peut aussi employer le mot \textbf{combinaisons} pour désigner un coefficient binomial;
          \item Pour calculer un coefficient binomial, sur la plupart des calculatrices, on utilise la commande \textbf{nCr}. Dans un tableur, on utilise la formule \textbf{=COMBIN(n;k)}.
     \end{itemize}
}
\cadre{vert}{Propriétés}{%id="p80"
     \begin{itemize}
          \item %
          Pour tout entier naturel $n$ :
          \begin{center}
               $\begin{pmatrix} n \\ 0 \end{pmatrix}=1$ et $\begin{pmatrix} n \\ n \end{pmatrix}=1$.
          \end{center}
          \item %
          Pour tout entier naturel $n$ et tout entier naturel $k$ ($0\leqslant k < n$) :
          \begin{center}$\begin{pmatrix} n \\ k \end{pmatrix}+\begin{pmatrix} n \\ k+1 \end{pmatrix}=\begin{pmatrix} n+1 \\ k+1 \end{pmatrix}$.\end{center}
     \end{itemize}
}
\bloc{cyan}{Remarque}{%id="r80"
     Ces propriétés permettent de calculer les coefficients binomiaux de proches en proches, grâce au \textit{Triangle de Pascal.}
     La figure ci-dessous représente ce triangle pour $n\leqslant 10$
     \begin{center}
          \img{triangle_pascal}{0.4}%width="450" alt="triangle de Pascal"
     \end{center}
     Pour construire ce triangle on procède de la manière suivante :
     \begin{itemize}
          \item On place des «1» dans la colonne $k=0$.
          \item On place des «1» sur la diagonale (qui correspond à $k=n$).
          \item On utilise la formule $\begin{pmatrix} n \\ k \end{pmatrix}+\begin{pmatrix} n \\ k+1 \end{pmatrix}=\begin{pmatrix} n+1 \\ k+1 \end{pmatrix}$ pour calculer les autres coefficients.
          \par
          Par exemple, pour trouver $\begin{pmatrix} 7 \\ 4 \end{pmatrix}=35$ on fait la somme des deux coefficients $\begin{pmatrix} 6 \\ 3 \end{pmatrix}=20$ et $\begin{pmatrix} 6 \\ 4 \end{pmatrix}=15$ de la ligne précédente.
     \end{itemize}
}
\cadre{rouge}{Théorème}{%id="t90"
     Soit $X$ une variable aléatoire de loi $\mathscr B \left(n ; p\right)$.
     \par
     Pour tout entier $k$ compris entre $0$ et $n$ :
     \begin{center}
          $p\left(X=k\right)=\begin{pmatrix} n \\ k \end{pmatrix} p^{k} \left(1-p\right)^{n-k}$.
     \end{center}
}
\bloc{orange}{Exemple}{%id="e90"
     On lance 8 fois une pièce équilibrée et on appelle $X$ la variable aléatoire qui compte le nombre de fois où l'on obtient \textit{Pile}.
     \par
     $X$ suit une loi binomiale de paramètres $n=8$ et $p=\frac{1}{2}$.
     \par
     La probabilité d'obtenir \textbf{4 fois} \textit{Pile} (par exemple) est :
     \par
     $p\left(X=4\right) = \begin{pmatrix} 8 \\ 4 \end{pmatrix}\times \left(\frac{1}{2}\right)^{4}\times \left(\frac{1}{2}\right)^{4}$.
     \par
     $\begin{pmatrix} 8 \\ 4 \end{pmatrix}= 70$ ~ (à la calculatrice).
     \par
     Donc :
     \par
     $p\left(X=4\right)=70\times \frac{1}{16}\times \frac{1}{16}=\frac{70}{256}=\frac{35}{128}$.
}

\end{document}
µ
\documentclass[a4paper]{article}

%================================================================================================================================
%
% Packages
%
%================================================================================================================================

\usepackage[T1]{fontenc} 	% pour caractères accentués
\usepackage[utf8]{inputenc}  % encodage utf8
\usepackage[french]{babel}	% langue : français
\usepackage{fourier}			% caractères plus lisibles
\usepackage[dvipsnames]{xcolor} % couleurs
\usepackage{fancyhdr}		% réglage header footer
\usepackage{needspace}		% empêcher sauts de page mal placés
\usepackage{graphicx}		% pour inclure des graphiques
\usepackage{enumitem,cprotect}		% personnalise les listes d'items (nécessaire pour ol, al ...)
\usepackage{hyperref}		% Liens hypertexte
\usepackage{pstricks,pst-all,pst-node,pstricks-add,pst-math,pst-plot,pst-tree,pst-eucl} % pstricks
\usepackage[a4paper,includeheadfoot,top=2cm,left=3cm, bottom=2cm,right=3cm]{geometry} % marges etc.
\usepackage{comment}			% commentaires multilignes
\usepackage{amsmath,environ} % maths (matrices, etc.)
\usepackage{amssymb,makeidx}
\usepackage{bm}				% bold maths
\usepackage{tabularx}		% tableaux
\usepackage{colortbl}		% tableaux en couleur
\usepackage{fontawesome}		% Fontawesome
\usepackage{environ}			% environment with command
\usepackage{fp}				% calculs pour ps-tricks
\usepackage{multido}			% pour ps tricks
\usepackage[np]{numprint}	% formattage nombre
\usepackage{tikz,tkz-tab} 			% package principal TikZ
\usepackage{pgfplots}   % axes
\usepackage{mathrsfs}    % cursives
\usepackage{calc}			% calcul taille boites
\usepackage[scaled=0.875]{helvet} % font sans serif
\usepackage{svg} % svg
\usepackage{scrextend} % local margin
\usepackage{scratch} %scratch
\usepackage{multicol} % colonnes
%\usepackage{infix-RPN,pst-func} % formule en notation polanaise inversée
\usepackage{listings}

%================================================================================================================================
%
% Réglages de base
%
%================================================================================================================================

\lstset{
language=Python,   % R code
literate=
{á}{{\'a}}1
{à}{{\`a}}1
{ã}{{\~a}}1
{é}{{\'e}}1
{è}{{\`e}}1
{ê}{{\^e}}1
{í}{{\'i}}1
{ó}{{\'o}}1
{õ}{{\~o}}1
{ú}{{\'u}}1
{ü}{{\"u}}1
{ç}{{\c{c}}}1
{~}{{ }}1
}


\definecolor{codegreen}{rgb}{0,0.6,0}
\definecolor{codegray}{rgb}{0.5,0.5,0.5}
\definecolor{codepurple}{rgb}{0.58,0,0.82}
\definecolor{backcolour}{rgb}{0.95,0.95,0.92}

\lstdefinestyle{mystyle}{
    backgroundcolor=\color{backcolour},   
    commentstyle=\color{codegreen},
    keywordstyle=\color{magenta},
    numberstyle=\tiny\color{codegray},
    stringstyle=\color{codepurple},
    basicstyle=\ttfamily\footnotesize,
    breakatwhitespace=false,         
    breaklines=true,                 
    captionpos=b,                    
    keepspaces=true,                 
    numbers=left,                    
xleftmargin=2em,
framexleftmargin=2em,            
    showspaces=false,                
    showstringspaces=false,
    showtabs=false,                  
    tabsize=2,
    upquote=true
}

\lstset{style=mystyle}


\lstset{style=mystyle}
\newcommand{\imgdir}{C:/laragon/www/newmc/assets/imgsvg/}
\newcommand{\imgsvgdir}{C:/laragon/www/newmc/assets/imgsvg/}

\definecolor{mcgris}{RGB}{220, 220, 220}% ancien~; pour compatibilité
\definecolor{mcbleu}{RGB}{52, 152, 219}
\definecolor{mcvert}{RGB}{125, 194, 70}
\definecolor{mcmauve}{RGB}{154, 0, 215}
\definecolor{mcorange}{RGB}{255, 96, 0}
\definecolor{mcturquoise}{RGB}{0, 153, 153}
\definecolor{mcrouge}{RGB}{255, 0, 0}
\definecolor{mclightvert}{RGB}{205, 234, 190}

\definecolor{gris}{RGB}{220, 220, 220}
\definecolor{bleu}{RGB}{52, 152, 219}
\definecolor{vert}{RGB}{125, 194, 70}
\definecolor{mauve}{RGB}{154, 0, 215}
\definecolor{orange}{RGB}{255, 96, 0}
\definecolor{turquoise}{RGB}{0, 153, 153}
\definecolor{rouge}{RGB}{255, 0, 0}
\definecolor{lightvert}{RGB}{205, 234, 190}
\setitemize[0]{label=\color{lightvert}  $\bullet$}

\pagestyle{fancy}
\renewcommand{\headrulewidth}{0.2pt}
\fancyhead[L]{maths-cours.fr}
\fancyhead[R]{\thepage}
\renewcommand{\footrulewidth}{0.2pt}
\fancyfoot[C]{}

\newcolumntype{C}{>{\centering\arraybackslash}X}
\newcolumntype{s}{>{\hsize=.35\hsize\arraybackslash}X}

\setlength{\parindent}{0pt}		 
\setlength{\parskip}{3mm}
\setlength{\headheight}{1cm}

\def\ebook{ebook}
\def\book{book}
\def\web{web}
\def\type{web}

\newcommand{\vect}[1]{\overrightarrow{\,\mathstrut#1\,}}

\def\Oij{$\left(\text{O}~;~\vect{\imath},~\vect{\jmath}\right)$}
\def\Oijk{$\left(\text{O}~;~\vect{\imath},~\vect{\jmath},~\vect{k}\right)$}
\def\Ouv{$\left(\text{O}~;~\vect{u},~\vect{v}\right)$}

\hypersetup{breaklinks=true, colorlinks = true, linkcolor = OliveGreen, urlcolor = OliveGreen, citecolor = OliveGreen, pdfauthor={Didier BONNEL - https://www.maths-cours.fr} } % supprime les bordures autour des liens

\renewcommand{\arg}[0]{\text{arg}}

\everymath{\displaystyle}

%================================================================================================================================
%
% Macros - Commandes
%
%================================================================================================================================

\newcommand\meta[2]{    			% Utilisé pour créer le post HTML.
	\def\titre{titre}
	\def\url{url}
	\def\arg{#1}
	\ifx\titre\arg
		\newcommand\maintitle{#2}
		\fancyhead[L]{#2}
		{\Large\sffamily \MakeUppercase{#2}}
		\vspace{1mm}\textcolor{mcvert}{\hrule}
	\fi 
	\ifx\url\arg
		\fancyfoot[L]{\href{https://www.maths-cours.fr#2}{\black \footnotesize{https://www.maths-cours.fr#2}}}
	\fi 
}


\newcommand\TitreC[1]{    		% Titre centré
     \needspace{3\baselineskip}
     \begin{center}\textbf{#1}\end{center}
}

\newcommand\newpar{    		% paragraphe
     \par
}

\newcommand\nosp {    		% commande vide (pas d'espace)
}
\newcommand{\id}[1]{} %ignore

\newcommand\boite[2]{				% Boite simple sans titre
	\vspace{5mm}
	\setlength{\fboxrule}{0.2mm}
	\setlength{\fboxsep}{5mm}	
	\fcolorbox{#1}{#1!3}{\makebox[\linewidth-2\fboxrule-2\fboxsep]{
  		\begin{minipage}[t]{\linewidth-2\fboxrule-4\fboxsep}\setlength{\parskip}{3mm}
  			 #2
  		\end{minipage}
	}}
	\vspace{5mm}
}

\newcommand\CBox[4]{				% Boites
	\vspace{5mm}
	\setlength{\fboxrule}{0.2mm}
	\setlength{\fboxsep}{5mm}
	
	\fcolorbox{#1}{#1!3}{\makebox[\linewidth-2\fboxrule-2\fboxsep]{
		\begin{minipage}[t]{1cm}\setlength{\parskip}{3mm}
	  		\textcolor{#1}{\LARGE{#2}}    
 	 	\end{minipage}  
  		\begin{minipage}[t]{\linewidth-2\fboxrule-4\fboxsep}\setlength{\parskip}{3mm}
			\raisebox{1.2mm}{\normalsize\sffamily{\textcolor{#1}{#3}}}						
  			 #4
  		\end{minipage}
	}}
	\vspace{5mm}
}

\newcommand\cadre[3]{				% Boites convertible html
	\par
	\vspace{2mm}
	\setlength{\fboxrule}{0.1mm}
	\setlength{\fboxsep}{5mm}
	\fcolorbox{#1}{white}{\makebox[\linewidth-2\fboxrule-2\fboxsep]{
  		\begin{minipage}[t]{\linewidth-2\fboxrule-4\fboxsep}\setlength{\parskip}{3mm}
			\raisebox{-2.5mm}{\sffamily \small{\textcolor{#1}{\MakeUppercase{#2}}}}		
			\par		
  			 #3
 	 		\end{minipage}
	}}
		\vspace{2mm}
	\par
}

\newcommand\bloc[3]{				% Boites convertible html sans bordure
     \needspace{2\baselineskip}
     {\sffamily \small{\textcolor{#1}{\MakeUppercase{#2}}}}    
		\par		
  			 #3
		\par
}

\newcommand\CHelp[1]{
     \CBox{Plum}{\faInfoCircle}{À RETENIR}{#1}
}

\newcommand\CUp[1]{
     \CBox{NavyBlue}{\faThumbsOUp}{EN PRATIQUE}{#1}
}

\newcommand\CInfo[1]{
     \CBox{Sepia}{\faArrowCircleRight}{REMARQUE}{#1}
}

\newcommand\CRedac[1]{
     \CBox{PineGreen}{\faEdit}{BIEN R\'EDIGER}{#1}
}

\newcommand\CError[1]{
     \CBox{Red}{\faExclamationTriangle}{ATTENTION}{#1}
}

\newcommand\TitreExo[2]{
\needspace{4\baselineskip}
 {\sffamily\large EXERCICE #1\ (\emph{#2 points})}
\vspace{5mm}
}

\newcommand\img[2]{
          \includegraphics[width=#2\paperwidth]{\imgdir#1}
}

\newcommand\imgsvg[2]{
       \begin{center}   \includegraphics[width=#2\paperwidth]{\imgsvgdir#1} \end{center}
}


\newcommand\Lien[2]{
     \href{#1}{#2 \tiny \faExternalLink}
}
\newcommand\mcLien[2]{
     \href{https~://www.maths-cours.fr/#1}{#2 \tiny \faExternalLink}
}

\newcommand{\euro}{\eurologo{}}

%================================================================================================================================
%
% Macros - Environement
%
%================================================================================================================================

\newenvironment{tex}{ %
}
{%
}

\newenvironment{indente}{ %
	\setlength\parindent{10mm}
}

{
	\setlength\parindent{0mm}
}

\newenvironment{corrige}{%
     \needspace{3\baselineskip}
     \medskip
     \textbf{\textsc{Corrigé}}
     \medskip
}
{
}

\newenvironment{extern}{%
     \begin{center}
     }
     {
     \end{center}
}

\NewEnviron{code}{%
	\par
     \boite{gray}{\texttt{%
     \BODY
     }}
     \par
}

\newenvironment{vbloc}{% boite sans cadre empeche saut de page
     \begin{minipage}[t]{\linewidth}
     }
     {
     \end{minipage}
}
\NewEnviron{h2}{%
    \needspace{3\baselineskip}
    \vspace{0.6cm}
	\noindent \MakeUppercase{\sffamily \large \BODY}
	\vspace{1mm}\textcolor{mcgris}{\hrule}\vspace{0.4cm}
	\par
}{}

\NewEnviron{h3}{%
    \needspace{3\baselineskip}
	\vspace{5mm}
	\textsc{\BODY}
	\par
}

\NewEnviron{margeneg}{ %
\begin{addmargin}[-1cm]{0cm}
\BODY
\end{addmargin}
}

\NewEnviron{html}{%
}

\begin{document}
\meta{url}{/cours/fonction-valeur-absolue/}
\meta{pid}{313}
\meta{titre}{Les fonctions valeur absolue et racine carrée}
\meta{type}{cours}
\begin{h2}I - La fonction valeur absolue\end{h2}
\cadre{bleu}{Définition}{%
     La fonction \textbf{valeur absolue} notée $ x \mapsto  |x|$ est définie sur $\mathbb{R}$ par
     \begin{itemize}
          \item $|x|$ = $x$ si $x$ est positif ou nul,
          \item $|x|$ = $-x$ si $x$ est négatif ou nul.
     \end{itemize}
}
\bloc{cyan}{Remarque}{%
     \begin{itemize}
          \item $-x$ est l'\textbf{opposé} de $x$.
          \textbf{Attention}, toutefois, à ne pas vous laisser abuser par cette notation: $-x$ n'est pas forcément négatif : $-x$ est négatif si $x$ est positif mais il est positif si $x$ est négatif. Par exemple $-\left(-5\right)$ est positif !
     \end{itemize}
}
\cadre{vert}{Propriété}{%
     La distance entre les nombres réels $x$ et $y$ est égale à $|y-x|$ (ou aussi à $|x-y|$).
}
\bloc{orange}{Exemple}{%
     \begin{center}
          \begin{extern} %width="500" alt="distance et valeur absolue"
               \resizebox{8.5cm}{!}{%
                    % -+-+-+ variables modifiables
                    \def\xmin{-4.5}
                    \def\xmax{6.5}
                    \def\ymin{-0.5}
                    \def\ymax{1}
                    \def\xunit{2}  % unités en cm
                    \def\yunit{2}
                    \psset{xunit=\xunit,yunit=\yunit,algebraic=true}
                    \fontsize{20pt}{20pt}\selectfont
                    \begin{pspicture*}[linewidth=1pt](\xmin,\ymin)(\xmax,\ymax)
                         %      \psgrid[gridcolor=mcgris, subgriddiv=5, gridlabels=0pt](\xmin,\ymin)(\xmax,\ymax)
                         \psaxes[yAxis=false,linewidth=0.75pt]{->}(0,0)(\xmin,\ymin)(\xmax,\ymax)
                         \rput[b](-3,0.3){$\red A$}
                         \rput[b](5,0.3){$\red B$}
                         \psline[linecolor=red,linewidth=1.25pt](-3,0)(5,0)
                         \psdots[linecolor=red](-3,0)(5,0)
                    \end{pspicture*}
               }
          \end{extern}
\end{center}}
\begin{center}
     $AB=|5-(-3)|=|8|=8$\\
     $BA=|-3-(+5)|=|-8|=8.$
\end{center}
\cadre{vert}{Propriété}{%
     La fonction \textbf{valeur absolue} est :
     \begin{itemize}
          \item strictement décroissante sur $\left]-\infty  ; 0\right]$~;
          \item strictement croissante sur $\left[0 ; +\infty \right[$.
     \end{itemize}
}
\bloc{cyan}{Tableau de variations}{%
     %:-+-+-+-+- Engendré par : http://math.et.info.free.fr/TikZ/TableauxVariations/
     \begin{center}
          \begin{extern}%width="400" alt="Tableau de variation de la fonction valeur absolue"
               \begin{tikzpicture}[scale=0.875]
                    % Styles
                    \tikzstyle{cadre}=[thin]
                    \tikzstyle{fleche}=[->,>=latex,thin]
                    \tikzstyle{nondefini}=[lightgray]
                    % Dimensions Modifiables
                    \def\Lrg{1.5}
                    \def\HtX{1}
                    \def\HtY{0.5}
                    % Dimensions Calculées
                    \def\lignex{-0.5*\HtX}
                    \def\lignef{-1.5*\HtX}
                    \def\separateur{-0.5*\Lrg}
                    % Largeur du tableau
                    \def\gauche{-2.5*\Lrg}
                    \def\droite{4.5*\Lrg}
                    % Hauteur du tableau
                    \def\haut{0.5*\HtX}
                    \def\bas{-1.5*\HtX-2*\HtY}
                    % Ligne de l'abscisse : x
                    \node at (-1.5*\Lrg,0) {$x$};
                    \node at (0*\Lrg,0) {$-\infty$};
                    \node at (2*\Lrg,0) {$0$};
                    \node at (4*\Lrg,0) {$+\infty$};
                    % Ligne de la fonction : f(x)
                    \node  at (-1.5*\Lrg,{-1*\HtX+(-1)*\HtY}) {$f(x)=|x|$};
                    \node (f1) at (0*\Lrg,{-1*\HtX+(0)*\HtY}) {$ $};
                    \node (f2) at (2*\Lrg,{-1*\HtX+(-2)*\HtY}) {$0$};
                    \node (f3) at (4*\Lrg,{-1*\HtX+(0)*\HtY}) {$ $};
                    % Pointillés
                    \draw[gray] (2*\Lrg,\lignex) -- (2*\Lrg,\bas);
                    % Flèches
                    \draw[fleche] (f1) -- (f2);
                    \draw[fleche] (f2) -- (f3);
                    % Encadrement
                    \draw[cadre] (\separateur,\haut) -- (\separateur,\bas);
                    \draw[cadre] (\gauche,\haut) rectangle  (\droite,\bas);
                    \draw[cadre] (\gauche,\lignex) -- (\droite,\lignex);
               \end{tikzpicture}
          \end{extern}
     \end{center}
     \begin{center}
          \textit{Tableau de variation de la fonction valeur absolue  }
     \end{center}
}
\bloc{cyan}{Courbe représentative}{%
     \begin{center}
          \begin{extern} %width="500" alt="Fonction valeur absolue  : graphique"
               \resizebox{8.5cm}{!}{%
                    % -+-+-+ variables modifiables
                    \def\fonction{abs(x) }
                    \def\xmin{-4.5}
                    \def\xmax{4.5}
                    \def\ymin{-0.5}
                    \def\ymax{5}
                    \def\xunit{2}  % unités en cm
                    \def\yunit{2}
                    \psset{xunit=\xunit,yunit=\yunit,algebraic=true}
                    \fontsize{18pt}{18pt}\selectfont
                    \begin{pspicture*}[linewidth=1pt](\xmin,\ymin)(\xmax,\ymax)
                         %      \psgrid[gridcolor=mcgris, subgriddiv=5, gridlabels=0pt](\xmin,\ymin)(\xmax,\ymax)
                         \psaxes[linewidth=0.75pt]{->}(0,0)(\xmin,\ymin)(\xmax,\ymax)
                         \psplot[plotpoints=2000,linecolor=blue]{\xmin}{\xmax}{\fonction}
                         \rput[tr](-0.1,-0.1){$O$}
                         \rput[tl](3.5,3.3){$\color{blue} \mathcal{C}$}
                    \end{pspicture*}
               }
          \end{extern}
     \end{center}
     \begin{center}
          \textit{Graphique de la fonction valeur absolue  }
     \end{center}
}
\cadre{vert}{Propriété}{%
     La courbe représentative de la fonction $x \mapsto  |x|$, dans un repère orthonormé, est \textbf{symétrique} par rapport à l'\textbf{axe des ordonnées}.
}
\begin{h2}II - La fonction racine carrée\end{h2}
\cadre{bleu}{Définition}{%
     La fonction racine carrée est la fonction définie sur $\left[0;+\infty \right[$ par $f\left(x\right)=\sqrt{x}$.
}
\cadre{vert}{Propriété}{%
     La fonction racine carrée est strictement croissante sur $\left[0;+\infty \right[$.
}
\bloc{cyan}{Tableau de variations}{%
     \begin{center}
          \begin{extern}%width="210" alt="Fonction racine carrée: tableau de variation"
               %:-+-+-+-+- Engendré par : http://math.et.info.free.fr/TikZ/TableauxVariations/
               \begin{tikzpicture}[scale=0.7]
                    % Styles
                    \tikzstyle{cadre}=[thin]
                    \tikzstyle{fleche}=[->,>=latex,thin]
                    \tikzstyle{nondefini}=[lightgray]
                    % Dimensions Modifiables
                    \def\Lrg{1.5}
                    \def\HtX{1}
                    \def\HtY{0.5}
                    % Dimensions Calculées
                    \def\lignex{-0.5*\HtX}
                    \def\lignef{-1.5*\HtX}
                    \def\separateur{-0.5*\Lrg}
                    % Largeur du tableau
                    \def\gauche{-1.5*\Lrg}
                    \def\droite{2.5*\Lrg}
                    % Hauteur du tableau
                    \def\haut{0.5*\HtX}
                    \def\bas{-1.5*\HtX-2*\HtY}
                    % Ligne de l'abscisse : x
                    \node at (-1*\Lrg,0) {$x$};
                    \node at (0*\Lrg,0) {$0$};
                    \node at (2*\Lrg,0) {$+\infty$};
                    % Ligne de la fonction : f(x)
                    \node  at (-1*\Lrg,{-1*\HtX+(-1)*\HtY}) {$\sqrt{x}$};
                    \node (f1) at (0*\Lrg,{-1*\HtX+(-2)*\HtY}) {$0$};
                    \node (f2) at (2*\Lrg,{-1*\HtX+(0)*\HtY}) {$ $};
                    % Flèches
                    \draw[fleche] (f1) -- (f2);
                    % Encadrement
                    \draw[cadre] (\separateur,\haut) -- (\separateur,\bas);
                    \draw[cadre] (\gauche,\haut) rectangle  (\droite,\bas);
                    \draw[cadre] (\gauche,\lignex) -- (\droite,\lignex);
               \end{tikzpicture}
               %:-+-+-+-+- Fin
          \end{extern}
     \end{center}
}
\begin{center}
     \textit{Tableau de variation de la fonction racine carrée }
\end{center}
\bloc{cyan}{Courbe représentative}{%
     \begin{center}
          \begin{extern} %width="500" alt="Fonction  racine carrée  : graphique"
               \resizebox{8.5cm}{!}{%
                    % -+-+-+ variables modifiables
                    \def\fonction{sqrt(x) }
                    \def\xmin{-0.5}
                    \def\xmax{8.5}
                    \def\ymin{-0.5}
                    \def\ymax{5}
                    \def\xunit{2}  % unités en cm
                    \def\yunit{2}
                    \psset{xunit=\xunit,yunit=\yunit,algebraic=true}
                    \fontsize{18pt}{18pt}\selectfont
                    \begin{pspicture*}[linewidth=1pt](\xmin,\ymin)(\xmax,\ymax)
                         %      \psgrid[gridcolor=mcgris, subgriddiv=5, gridlabels=0pt](\xmin,\ymin)(\xmax,\ymax)
                         \psaxes[linewidth=0.75pt]{->}(0,0)(\xmin,\ymin)(\xmax,\ymax)
                         \psplot[plotpoints=2000,linecolor=blue]{0.000001}{\xmax}{\fonction}
                         \rput[tr](-0.1,-0.1){$O$}
                         \rput[tl](7.5,3.3){$\color{blue} \mathcal{C}$}
                    \end{pspicture*}
               }
          \end{extern}
     \end{center}
     \begin{center}
          \textit{Graphique de la fonction  racine carrée  }
     \end{center}
}
\bloc{cyan}{Remarque}{%
     La courbe représentative de la fonction racine carrée est une demi-parabole.
}
\cadre{vert}{Propriété}{%
     Pour tout $x\in \mathbb{R}$ :
     \begin{center}$\sqrt{x^{2}}=|x|$.\end{center}
}
\bloc{orange}{Exemple}{%
     \begin{itemize}
          \item $\sqrt{3^{2}}=\sqrt{9}=3$ ;
          \item $\sqrt{\left(-3\right)^{2}}=\sqrt{9}=3$.
     \end{itemize}
}
\bloc{cyan}{Remarque}{%
     Ne pas confondre :
     \begin{itemize}
          \item $\sqrt{x^{2}}$ qui est défini pour tout $x\in \mathbb{R}$ (ce qui est sous le radical est $x^{2}$ donc toujours positif) et est égal à $|x|$ ;
          \item $\left(\sqrt{x}\right)^{2}$ qui n'est défini que pour $x \geqslant  0$ (ce qui est sous le radical est $x$).
     \end{itemize}
}

\end{document}
µ
\documentclass[a4paper]{article}

%================================================================================================================================
%
% Packages
%
%================================================================================================================================

\usepackage[T1]{fontenc} 	% pour caractères accentués
\usepackage[utf8]{inputenc}  % encodage utf8
\usepackage[french]{babel}	% langue : français
\usepackage{fourier}			% caractères plus lisibles
\usepackage[dvipsnames]{xcolor} % couleurs
\usepackage{fancyhdr}		% réglage header footer
\usepackage{needspace}		% empêcher sauts de page mal placés
\usepackage{graphicx}		% pour inclure des graphiques
\usepackage{enumitem,cprotect}		% personnalise les listes d'items (nécessaire pour ol, al ...)
\usepackage{hyperref}		% Liens hypertexte
\usepackage{pstricks,pst-all,pst-node,pstricks-add,pst-math,pst-plot,pst-tree,pst-eucl} % pstricks
\usepackage[a4paper,includeheadfoot,top=2cm,left=3cm, bottom=2cm,right=3cm]{geometry} % marges etc.
\usepackage{comment}			% commentaires multilignes
\usepackage{amsmath,environ} % maths (matrices, etc.)
\usepackage{amssymb,makeidx}
\usepackage{bm}				% bold maths
\usepackage{tabularx}		% tableaux
\usepackage{colortbl}		% tableaux en couleur
\usepackage{fontawesome}		% Fontawesome
\usepackage{environ}			% environment with command
\usepackage{fp}				% calculs pour ps-tricks
\usepackage{multido}			% pour ps tricks
\usepackage[np]{numprint}	% formattage nombre
\usepackage{tikz,tkz-tab} 			% package principal TikZ
\usepackage{pgfplots}   % axes
\usepackage{mathrsfs}    % cursives
\usepackage{calc}			% calcul taille boites
\usepackage[scaled=0.875]{helvet} % font sans serif
\usepackage{svg} % svg
\usepackage{scrextend} % local margin
\usepackage{scratch} %scratch
\usepackage{multicol} % colonnes
%\usepackage{infix-RPN,pst-func} % formule en notation polanaise inversée
\usepackage{listings}

%================================================================================================================================
%
% Réglages de base
%
%================================================================================================================================

\lstset{
language=Python,   % R code
literate=
{á}{{\'a}}1
{à}{{\`a}}1
{ã}{{\~a}}1
{é}{{\'e}}1
{è}{{\`e}}1
{ê}{{\^e}}1
{í}{{\'i}}1
{ó}{{\'o}}1
{õ}{{\~o}}1
{ú}{{\'u}}1
{ü}{{\"u}}1
{ç}{{\c{c}}}1
{~}{{ }}1
}


\definecolor{codegreen}{rgb}{0,0.6,0}
\definecolor{codegray}{rgb}{0.5,0.5,0.5}
\definecolor{codepurple}{rgb}{0.58,0,0.82}
\definecolor{backcolour}{rgb}{0.95,0.95,0.92}

\lstdefinestyle{mystyle}{
    backgroundcolor=\color{backcolour},   
    commentstyle=\color{codegreen},
    keywordstyle=\color{magenta},
    numberstyle=\tiny\color{codegray},
    stringstyle=\color{codepurple},
    basicstyle=\ttfamily\footnotesize,
    breakatwhitespace=false,         
    breaklines=true,                 
    captionpos=b,                    
    keepspaces=true,                 
    numbers=left,                    
xleftmargin=2em,
framexleftmargin=2em,            
    showspaces=false,                
    showstringspaces=false,
    showtabs=false,                  
    tabsize=2,
    upquote=true
}

\lstset{style=mystyle}


\lstset{style=mystyle}
\newcommand{\imgdir}{C:/laragon/www/newmc/assets/imgsvg/}
\newcommand{\imgsvgdir}{C:/laragon/www/newmc/assets/imgsvg/}

\definecolor{mcgris}{RGB}{220, 220, 220}% ancien~; pour compatibilité
\definecolor{mcbleu}{RGB}{52, 152, 219}
\definecolor{mcvert}{RGB}{125, 194, 70}
\definecolor{mcmauve}{RGB}{154, 0, 215}
\definecolor{mcorange}{RGB}{255, 96, 0}
\definecolor{mcturquoise}{RGB}{0, 153, 153}
\definecolor{mcrouge}{RGB}{255, 0, 0}
\definecolor{mclightvert}{RGB}{205, 234, 190}

\definecolor{gris}{RGB}{220, 220, 220}
\definecolor{bleu}{RGB}{52, 152, 219}
\definecolor{vert}{RGB}{125, 194, 70}
\definecolor{mauve}{RGB}{154, 0, 215}
\definecolor{orange}{RGB}{255, 96, 0}
\definecolor{turquoise}{RGB}{0, 153, 153}
\definecolor{rouge}{RGB}{255, 0, 0}
\definecolor{lightvert}{RGB}{205, 234, 190}
\setitemize[0]{label=\color{lightvert}  $\bullet$}

\pagestyle{fancy}
\renewcommand{\headrulewidth}{0.2pt}
\fancyhead[L]{maths-cours.fr}
\fancyhead[R]{\thepage}
\renewcommand{\footrulewidth}{0.2pt}
\fancyfoot[C]{}

\newcolumntype{C}{>{\centering\arraybackslash}X}
\newcolumntype{s}{>{\hsize=.35\hsize\arraybackslash}X}

\setlength{\parindent}{0pt}		 
\setlength{\parskip}{3mm}
\setlength{\headheight}{1cm}

\def\ebook{ebook}
\def\book{book}
\def\web{web}
\def\type{web}

\newcommand{\vect}[1]{\overrightarrow{\,\mathstrut#1\,}}

\def\Oij{$\left(\text{O}~;~\vect{\imath},~\vect{\jmath}\right)$}
\def\Oijk{$\left(\text{O}~;~\vect{\imath},~\vect{\jmath},~\vect{k}\right)$}
\def\Ouv{$\left(\text{O}~;~\vect{u},~\vect{v}\right)$}

\hypersetup{breaklinks=true, colorlinks = true, linkcolor = OliveGreen, urlcolor = OliveGreen, citecolor = OliveGreen, pdfauthor={Didier BONNEL - https://www.maths-cours.fr} } % supprime les bordures autour des liens

\renewcommand{\arg}[0]{\text{arg}}

\everymath{\displaystyle}

%================================================================================================================================
%
% Macros - Commandes
%
%================================================================================================================================

\newcommand\meta[2]{    			% Utilisé pour créer le post HTML.
	\def\titre{titre}
	\def\url{url}
	\def\arg{#1}
	\ifx\titre\arg
		\newcommand\maintitle{#2}
		\fancyhead[L]{#2}
		{\Large\sffamily \MakeUppercase{#2}}
		\vspace{1mm}\textcolor{mcvert}{\hrule}
	\fi 
	\ifx\url\arg
		\fancyfoot[L]{\href{https://www.maths-cours.fr#2}{\black \footnotesize{https://www.maths-cours.fr#2}}}
	\fi 
}


\newcommand\TitreC[1]{    		% Titre centré
     \needspace{3\baselineskip}
     \begin{center}\textbf{#1}\end{center}
}

\newcommand\newpar{    		% paragraphe
     \par
}

\newcommand\nosp {    		% commande vide (pas d'espace)
}
\newcommand{\id}[1]{} %ignore

\newcommand\boite[2]{				% Boite simple sans titre
	\vspace{5mm}
	\setlength{\fboxrule}{0.2mm}
	\setlength{\fboxsep}{5mm}	
	\fcolorbox{#1}{#1!3}{\makebox[\linewidth-2\fboxrule-2\fboxsep]{
  		\begin{minipage}[t]{\linewidth-2\fboxrule-4\fboxsep}\setlength{\parskip}{3mm}
  			 #2
  		\end{minipage}
	}}
	\vspace{5mm}
}

\newcommand\CBox[4]{				% Boites
	\vspace{5mm}
	\setlength{\fboxrule}{0.2mm}
	\setlength{\fboxsep}{5mm}
	
	\fcolorbox{#1}{#1!3}{\makebox[\linewidth-2\fboxrule-2\fboxsep]{
		\begin{minipage}[t]{1cm}\setlength{\parskip}{3mm}
	  		\textcolor{#1}{\LARGE{#2}}    
 	 	\end{minipage}  
  		\begin{minipage}[t]{\linewidth-2\fboxrule-4\fboxsep}\setlength{\parskip}{3mm}
			\raisebox{1.2mm}{\normalsize\sffamily{\textcolor{#1}{#3}}}						
  			 #4
  		\end{minipage}
	}}
	\vspace{5mm}
}

\newcommand\cadre[3]{				% Boites convertible html
	\par
	\vspace{2mm}
	\setlength{\fboxrule}{0.1mm}
	\setlength{\fboxsep}{5mm}
	\fcolorbox{#1}{white}{\makebox[\linewidth-2\fboxrule-2\fboxsep]{
  		\begin{minipage}[t]{\linewidth-2\fboxrule-4\fboxsep}\setlength{\parskip}{3mm}
			\raisebox{-2.5mm}{\sffamily \small{\textcolor{#1}{\MakeUppercase{#2}}}}		
			\par		
  			 #3
 	 		\end{minipage}
	}}
		\vspace{2mm}
	\par
}

\newcommand\bloc[3]{				% Boites convertible html sans bordure
     \needspace{2\baselineskip}
     {\sffamily \small{\textcolor{#1}{\MakeUppercase{#2}}}}    
		\par		
  			 #3
		\par
}

\newcommand\CHelp[1]{
     \CBox{Plum}{\faInfoCircle}{À RETENIR}{#1}
}

\newcommand\CUp[1]{
     \CBox{NavyBlue}{\faThumbsOUp}{EN PRATIQUE}{#1}
}

\newcommand\CInfo[1]{
     \CBox{Sepia}{\faArrowCircleRight}{REMARQUE}{#1}
}

\newcommand\CRedac[1]{
     \CBox{PineGreen}{\faEdit}{BIEN R\'EDIGER}{#1}
}

\newcommand\CError[1]{
     \CBox{Red}{\faExclamationTriangle}{ATTENTION}{#1}
}

\newcommand\TitreExo[2]{
\needspace{4\baselineskip}
 {\sffamily\large EXERCICE #1\ (\emph{#2 points})}
\vspace{5mm}
}

\newcommand\img[2]{
          \includegraphics[width=#2\paperwidth]{\imgdir#1}
}

\newcommand\imgsvg[2]{
       \begin{center}   \includegraphics[width=#2\paperwidth]{\imgsvgdir#1} \end{center}
}


\newcommand\Lien[2]{
     \href{#1}{#2 \tiny \faExternalLink}
}
\newcommand\mcLien[2]{
     \href{https~://www.maths-cours.fr/#1}{#2 \tiny \faExternalLink}
}

\newcommand{\euro}{\eurologo{}}

%================================================================================================================================
%
% Macros - Environement
%
%================================================================================================================================

\newenvironment{tex}{ %
}
{%
}

\newenvironment{indente}{ %
	\setlength\parindent{10mm}
}

{
	\setlength\parindent{0mm}
}

\newenvironment{corrige}{%
     \needspace{3\baselineskip}
     \medskip
     \textbf{\textsc{Corrigé}}
     \medskip
}
{
}

\newenvironment{extern}{%
     \begin{center}
     }
     {
     \end{center}
}

\NewEnviron{code}{%
	\par
     \boite{gray}{\texttt{%
     \BODY
     }}
     \par
}

\newenvironment{vbloc}{% boite sans cadre empeche saut de page
     \begin{minipage}[t]{\linewidth}
     }
     {
     \end{minipage}
}
\NewEnviron{h2}{%
    \needspace{3\baselineskip}
    \vspace{0.6cm}
	\noindent \MakeUppercase{\sffamily \large \BODY}
	\vspace{1mm}\textcolor{mcgris}{\hrule}\vspace{0.4cm}
	\par
}{}

\NewEnviron{h3}{%
    \needspace{3\baselineskip}
	\vspace{5mm}
	\textsc{\BODY}
	\par
}

\NewEnviron{margeneg}{ %
\begin{addmargin}[-1cm]{0cm}
\BODY
\end{addmargin}
}

\NewEnviron{html}{%
}

\begin{document}
\meta{url}{/cours/variations-une-fonction/}
\meta{pid}{315}
\meta{titre}{Variations d'une fonction - Fonctions associées}
\meta{type}{cours}
\begin{h2}I - Rappels\end{h2}
\cadre{bleu}{Définitions}{%
     On dit qu'une fonction $f$ définie sur un intervalle $I$ est :
     \begin{itemize}
          \item \textbf{croissante} sur l'intervalle $I$: si pour tous réels $x_{1}$ et $x_{2}$  appartenant à $I$ tels que $x_{1}\leqslant  x_{2}$ on a $f\left(x_{1}\right)\leqslant f\left(x_{2}\right)$.
          \item \textbf{décroissante} sur l'intervalle $I$: si pour tous réels $x_{1}$ et $x_{2}$  appartenant à $I$ tels que $x_{1} \leqslant x_{2}$ on a $f\left(x_{1}\right) \geqslant f\left(x_{2}\right)$.
          \item \textbf{strictement croissante} sur l'intervalle $I$: si pour tous réels $x_{1}$ et $x_{2}$  appartenant à $I$ tels que $x_{1} <   x_{2}$ on a $f\left(x_{1}\right) <  f\left(x_{2}\right)$.
          \item \textbf{strictement décroissante} sur l'intervalle $I$: si pour tous réels $x_{1}$ et $x_{2}$  appartenant à $I$ tels que $x_{1} <   x_{2}$ on a $f\left(x_{1}\right) >  f\left(x_{2}\right)$.
     \end{itemize}
}
\begin{center}
     \begin{extern}%width="550" alt="Fonctions croissante et décroissante"
          \begin{tabular}{c c c}
               \resizebox{5.5cm}{!}{%
                    % -+-+-+ variables modifiables
                    \def\fonction{1+0.2*x*x }
                    \def\xmin{-1.2}
                    \def\xmax{5}
                    \def\ymin{-0.9}
                    \def\ymax{5}
                    \def\xunit{1}  % unités en cm
                    \def\yunit{1}
                    \psset{xunit=\xunit,yunit=\yunit,algebraic=true}
                    \fontsize{12pt}{12pt}\selectfont
                    \begin{pspicture*}[linewidth=1pt](\xmin,\ymin)(\xmax,\ymax)
                         %      \psgrid[gridcolor=mcgris, subgriddiv=5, gridlabels=0pt](\xmin,\ymin)(\xmax,\ymax)
                         \psaxes[linewidth=0.75pt,Dx=10,Dy=10]{->}(0,0)(\xmin,\ymin)(\xmax,\ymax)
                         \psplot[plotpoints=2000,linecolor=red]{0.2}{\xmax}{\fonction}
                         \psline[linewidth=0.75pt,linecolor=lightgray](1,0)(1,1.2)(0,1.2)
                         \psline[linewidth=0.75pt,linecolor=lightgray](4,0)(4,4.2)(0,4.2)
                         \rput[tr](-0.3,-0.3){$O$} \rput[t](1,-0.3){$x_1$} \rput[t](4,-0.3){$x_2$}
                         \rput[r](-0.1,1.2){$f(x_1)$} \rput[r](-0.1,4.2){$f(x_2)$}
                         \rput[tl](4.3,4.5){$\color{red} \mathcal{C}_f$}
                    \end{pspicture*}
               }
               & ~~~~ &%
               \resizebox{5.5cm}{!}{%
                    % -+-+-+ variables modifiables
                    \def\fonction{4-0.1*x*x }
                    \def\xmin{-1.2}
                    \def\xmax{5}
                    \def\ymin{-0.9}
                    \def\ymax{5}
                    \def\xunit{1}  % unités en cm
                    \def\yunit{1}
                    \psset{xunit=\xunit,yunit=\yunit,algebraic=true}
                    \fontsize{12pt}{12pt}\selectfont
                    \begin{pspicture*}[linewidth=1pt](\xmin,\ymin)(\xmax,\ymax)
                         %      \psgrid[gridcolor=mcgris, subgriddiv=5, gridlabels=0pt](\xmin,\ymin)(\xmax,\ymax)
                         \psaxes[linewidth=0.75pt,Dx=10,Dy=10]{->}(0,0)(\xmin,\ymin)(\xmax,\ymax)
                         \psplot[plotpoints=2000,linecolor=red]{0.2}{\xmax}{\fonction}
                         \psline[linewidth=0.75pt,linecolor=lightgray](1,0)(1,3.9)(0,3.9)
                         \psline[linewidth=0.75pt,linecolor=lightgray](4,0)(4,2.4)(0,2.4)
                         \rput[tr](-0.3,-0.3){$O$} \rput[t](1,-0.3){$x_1$} \rput[t](4,-0.3){$x_2$}
                         \rput[r](-0.1,3.9){$f(x_1)$} \rput[r](-0.1,2.4){$f(x_2)$}
                         \rput[tl](4.5,2.5){$\color{red} \mathcal{C}_f$}
                    \end{pspicture*}
               }
               \\
               Fonction croissante & ~~~~ & Fonction décroissante %
               \\
          \end{tabular}
     \end{extern}
\end{center}
\bloc{cyan}{Remarques}{%
     \begin{itemize}
          \item Une fonction qui dont le sens de variations ne change pas sur $I$ (c'est à dire qui est soit croissante sur $I$ soit décroissante sur $I$) est dite \textbf{monotone} sur $I$.
          \item Une fonction constante ($x\mapsto k$ où $k$ est un réel fixé) est à la fois croissante et décroissante mais n'est ni strictement croissante, ni strictement décroissante.
     \end{itemize}
}
\cadre{vert}{Propriété}{%
     Une fonction affine $f : x\mapsto ax+b$ est croissante si son coefficient directeur $a$ est \textbf{positif ou nul}, et \textbf{décroissante} si son coefficient directeur est \textbf{négatif ou nul}.
}
\bloc{cyan}{Remarque}{%
     Si le coefficient directeur d'une fonction affine est nul la fonction est \textbf{constante}.
}
\begin{h2}II - Fonction associées\end{h2}
\cadre{bleu}{Fonctions $u+k$}{%
     Soit $u$ une fonction définie sur une partie $\mathscr D$ de $\mathbb{R}$ et $k \in  \mathbb{R}$
     \par
     On note $u+k$ la fonction définie sur $\mathscr D$ par :
     \begin{center}$u+k :  x\mapsto u\left(x\right)+k$\end{center}
}
\cadre{vert}{Propriété}{%
     Quel que soit $k \in  \mathbb{R}$, $u+k$ a le même sens de variation que $u$ sur $\mathscr D$.
}
\bloc{orange}{Exemple}{%
     Soit $f$ définie sur $\mathbb{R}$ par $f\left(x\right)=x^{2}-1$.
     \par
     Si on note $u$ la fonction \textit{carrée} définie sur $\mathbb{R}$ par  $u : x \mapsto x^{2}$
     \par
     on a $f = u-1$
     \par
     Le sens de variation de $f$ est donc identique à celui de $u$ d'après la propriété précédente.
     \par
     Donc
     \begin{itemize}
          \item $f$ est \textbf{décroissante} sur l'intervalle $\left]-\infty  ; 0\right]$
          \item $f$ est \textbf{croissante} sur l'intervalle $\left[0 ; +\infty \right[$
     \end{itemize}
     \begin{center}
          \begin{extern}%width="350" alt="Tableau de variation x²-1"
               \begin{tikzpicture}[scale=0.875]
                    % Styles
                    \tikzstyle{cadre}=[thin]
                    \tikzstyle{fleche}=[->,>=latex,thin]
                    \tikzstyle{nondefini}=[lightgray]
                    % Dimensions Modifiables
                    \def\Lrg{1.5}
                    \def\HtX{1}
                    \def\HtY{0.5}
                    % Dimensions Calculées
                    \def\lignex{-0.5*\HtX}
                    \def\lignef{-1.5*\HtX}
                    \def\separateur{-0.5*\Lrg}
                    % Largeur du tableau
                    \def\gauche{-1.5*\Lrg}
                    \def\droite{4.5*\Lrg}
                    % Hauteur du tableau
                    \def\haut{0.5*\HtX}
                    \def\bas{-1.5*\HtX-2*\HtY}
                    % Pointillés
                    \draw[lightgray] (2*\Lrg,\lignex) -- (2*\Lrg,\bas);
                    % Ligne de l'abscisse : x
                    \node at (-1*\Lrg,0) {$x$};
                    \node at (0*\Lrg,0) {$-\infty$};
                    \node at (2*\Lrg,0) {$0$};
                    \node at (4*\Lrg,0) {$+\infty$};
                    % Ligne de la fonction : f(x)
                    \node  at (-1*\Lrg,{-1*\HtX+(-1)*\HtY}) {$x^2-1$};
                    \node (f1) at (0*\Lrg,{-1*\HtX+(0)*\HtY}) {$ $};
                    \node (f2) at (2*\Lrg,{-1*\HtX+(-2)*\HtY}) {$-1$};
                    \node (f3) at (4*\Lrg,{-1*\HtX+(0)*\HtY}) {$ $};
                    % Flèches
                    \draw[fleche] (f1) -- (f2);
                    \draw[fleche] (f2) -- (f3);
                    % Encadrement
                    \draw[cadre] (\separateur,\haut) -- (\separateur,\bas);
                    \draw[cadre] (\gauche,\haut) rectangle  (\droite,\bas);
                    \draw[cadre] (\gauche,\lignex) -- (\droite,\lignex);
               \end{tikzpicture}
          \end{extern}
     \end{center}
     %:-+-+-+-+- Fin
}
\cadre{bleu}{Fonctions $k\times u$}{%
     Soit $u$ une fonction définie sur une partie $\mathscr D$ de $\mathbb{R}$ et $k \in  \mathbb{R}$
     \par
     On note $ku$ la fonction définie sur $\mathscr D$ par :
     \begin{center}$ku :  x\mapsto k\times u\left(x\right)$\end{center}
}
\cadre{vert}{Propriété}{%
     \begin{itemize}
          \item  si $k > 0$, $ku$ a le même sens de variation que $u$ sur $\mathscr D$.
          \item  si $k < 0$, le sens de variation de $ku$ est le contraire de celui de $u$ sur $\mathscr D$.
     \end{itemize}
}
\bloc{orange}{Exemple}{%
     Soit $f$ définie sur $\left]-\infty  ; 0\right[ \cup  \left]0 ; +\infty \right[$ par $f\left(x\right)=-\frac{1}{x}$.
     \par
     Si on note $u$ la fonction \textit{inverse} définie sur  $\left]-\infty  ; 0\right[ \cup  \left]0 ; +\infty \right[$ par  $u : x \mapsto \frac{1}{x}$
     \par
     on a $f = -1\times u$
     \par
     Comme $-1$ est négatif, le sens de variation de $f$ est inverse de celui de $u$ sur chacun des intervalles $\left]-\infty  ; 0\right[$ et $\left]0 ; +\infty \right[$
     \par
     Donc $f$ est \textbf{croissante} sur l'intervalle $\left]-\infty  ; 0\right]$ et sur l'intervalle $\left]0 ; +\infty \right[$
}
\begin{center}
     \begin{extern}%width="350" alt="tableau de variation fonction inverse négative"
          \begin{tikzpicture}[scale=0.875]
               % Styles
               \tikzstyle{cadre}=[thin]
               \tikzstyle{fleche}=[->,>=latex,thin]
               \tikzstyle{nondefini}=[lightgray]
               % Dimensions Modifiables
               \def\Lrg{1.5}
               \def\HtX{1}
               \def\HtY{0.5}
               % Dimensions Calculées
               \def\lignex{-0.5*\HtX}
               \def\lignef{-1.5*\HtX}
               \def\separateur{-0.5*\Lrg}
               % Largeur du tableau
               \def\gauche{-1.5*\Lrg}
               \def\droite{4.5*\Lrg}
               % Hauteur du tableau
               \def\haut{0.5*\HtX}
               \def\bas{-1.5*\HtX-2*\HtY}
               % Ligne de l'abscisse : x
               \node at (-1*\Lrg,0) {$x$};
               \node at (0*\Lrg,0) {$-\infty$};
               \node at (2*\Lrg,0) {$0$};
               \node at (4*\Lrg,0) {$+\infty$};
               % Ligne de la fonction : f(x)
               \node  at (-1*\Lrg,{-1*\HtX+(-1)*\HtY}) {$f$};
               \node (f1) at (0*\Lrg,{-1*\HtX+(-2)*\HtY}) {$ $};
               \node[left] (f2) at (2*\Lrg,{-1*\HtX+(0)*\HtY}) {$ $};
               \node[right] (f3) at (2*\Lrg,{-1*\HtX+(-2)*\HtY}) {$ $};
               \node (f4) at (4*\Lrg,{-1*\HtX+(0)*\HtY}) {$ $};
               % Flèches
               \draw[fleche] (f1) -- (f2);
               \draw[fleche] (f3) -- (f4);
               % Doubles barres
               \draw[double distance=2pt] (2*\Lrg,\lignef+0.9*\HtX) -- (2*\Lrg,\bas+0.1*\HtX);
               % Encadrement
               \draw[cadre] (\separateur,\haut) -- (\separateur,\bas);
               \draw[cadre] (\gauche,\haut) rectangle  (\droite,\bas);
               \draw[cadre] (\gauche,\lignex) -- (\droite,\lignex);
          \end{tikzpicture}
     \end{extern}
\end{center}
\cadre{bleu}{Fonctions $\sqrt{u}$}{%
     Soit $u$ une fonction définie sur une partie $\mathscr D$ de $\mathbb{R}$.
     \par
     On note $\sqrt{u}$ la fonction définie, pour tout $x$ de $\mathscr D$  tel que $u\left(x\right) \geqslant  0$, par~:
     \begin{center}$\sqrt{u} :  x\mapsto \sqrt{u\left(x\right)}$\end{center}
}
\cadre{vert}{Propriété}{%
     $\sqrt{u}$  a le \textbf{même sens de variation} que $u$ sur tout intervalle où $u$ est positive.
}
\bloc{orange}{Exemple}{%
     Soit $f : x \mapsto  \sqrt{x-2}$
     \par
     $f$ est définie si et seulement si $x-2 \geqslant  0$, c'est à dire sur $\mathscr D=\left[2 ; +\infty \right[$
     \par
     Sur l'intervalle $\mathscr D$ la fonction $f$ est croissante car la fonction $x \mapsto  x-2$ l'est (fonction affine dont le coefficient directeur est positif).
}
\begin{center}
     \begin{extern}%width="250" alt="Tableau de variation sqrt(x-2)"
          \begin{tikzpicture}[scale=0.875]
               % Styles
               \tikzstyle{cadre}=[thin]
               \tikzstyle{fleche}=[->,>=latex,thin]
               \tikzstyle{nondefini}=[lightgray]
               % Dimensions Modifiables
               \def\Lrg{1.5}
               \def\HtX{1}
               \def\HtY{0.5}
               % Dimensions Calculées
               \def\lignex{-0.5*\HtX}
               \def\lignef{-1.5*\HtX}
               \def\separateur{-0.5*\Lrg}
               % Largeur du tableau
               \def\gauche{-1.5*\Lrg}
               \def\droite{2.5*\Lrg}
               % Hauteur du tableau
               \def\haut{0.5*\HtX}
               \def\bas{-1.5*\HtX-2*\HtY}
               % Ligne de l'abscisse : x
               \node at (-1*\Lrg,0) {$x$};
               \node at (0*\Lrg,0) {$2$};
               \node at (2*\Lrg,0) {$+\infty$};
               % Ligne de la fonction : f(x)
               \node  at (-1*\Lrg,{-1*\HtX+(-1)*\HtY}) {$\sqrt{x-2}$};
               \node (f1) at (0*\Lrg,{-1*\HtX+(-2)*\HtY}) {$0$};
               \node (f2) at (2*\Lrg,{-1*\HtX+(0)*\HtY}) {$ $};
               % Flèches
               \draw[fleche] (f1) -- (f2);
               % Encadrement
               \draw[cadre] (\separateur,\haut) -- (\separateur,\bas);
               \draw[cadre] (\gauche,\haut) rectangle  (\droite,\bas);
               \draw[cadre] (\gauche,\lignex) -- (\droite,\lignex);
          \end{tikzpicture}
     \end{extern}
\end{center}
\cadre{bleu}{Fonctions $\frac{1}{u}$}{%
     Soit $u$ une fonction définie sur une partie $\mathscr D$ de $\mathbb{R}$.
     \par
     On note $\frac{1}{u}$ la fonction définie pour tout $x$ de $\mathscr D$ \textbf{ tel que $u\left(x\right) \neq  0$} par :
     \begin{center}$\frac{1}{u} :  x\mapsto \frac{1}{u\left(x\right)}$\end{center}
}
\cadre{vert}{Propriété}{%
     $\frac{1}{u}$  a le \textbf{sens de variation contraire} de $u$ sur tout intervalle où $u$ ne s'annule pas et garde un \textbf{signe constant}.
}
\bloc{orange}{Exemple}{%
     Soit $ f : x \mapsto  \frac{1}{x+1}$
     \par
     $f$ est définie si et seulement si $x+1 \neq  0$, c'est à dire sur $\mathscr D=\left]-\infty  ; -1\right[ \cup  \left]-1 ; +\infty \right[$
     \par
     La fonction $x \mapsto  x+1$ est croissante sur $\mathbb{R}$
     \par
     Sur l'intervalle $\left]-\infty  ; -1\right[ $ la fonction $x \mapsto  x+1$ est strictement négative (donc a un signe constant).
     \par
     Sur l'intervalle $\left]-1 ; +\infty \right[$ la fonction $x \mapsto  x+1$ est strictement positive (donc a un signe constant).
     \par
     Donc $f$ est strictement décroissante sur chacun des intervalles $\left]-\infty  ; -1\right[ $ et $ \left]-1 ; +\infty \right[$
     \par
}
\begin{center}
     \begin{extern}%width="350" alt="Tableau de variation 1/(x+1)"
          \begin{tikzpicture}[scale=0.875]
               % Styles
               \tikzstyle{cadre}=[thin]
               \tikzstyle{fleche}=[->,>=latex,thin]
               \tikzstyle{nondefini}=[lightgray]
               % Dimensions Modifiables
               \def\Lrg{1.5}
               \def\HtX{1}
               \def\HtY{0.5}
               % Dimensions Calculées
               \def\lignex{-0.5*\HtX}
               \def\lignef{-1.5*\HtX}
               \def\separateur{-0.5*\Lrg}
               % Largeur du tableau
               \def\gauche{-1.5*\Lrg}
               \def\droite{4.5*\Lrg}
               % Hauteur du tableau
               \def\haut{0.5*\HtX}
               \def\bas{-1.5*\HtX-2*\HtY}
               % Ligne de l'abscisse : x
               \node at (-1*\Lrg,0) {$x$};
               \node at (0*\Lrg,0) {$-\infty$};
               \node at (2*\Lrg,0) {$-1$};
               \node at (4*\Lrg,0) {$+\infty$};
               % Ligne de la fonction : f(x)
               \node  at (-1*\Lrg,{-1*\HtX+(-1)*\HtY}) {$\dfrac{1}{x+1}$};
               \node (f1) at (0*\Lrg,{-1*\HtX+(0)*\HtY}) {$ $};
               \node[left] (f2) at (2*\Lrg,{-1*\HtX+(-2)*\HtY}) {$~$};
               \node[right] (f3) at (2*\Lrg,{-1*\HtX+(0)*\HtY}) {$~$};
               \node (f4) at (4*\Lrg,{-1*\HtX+(-2)*\HtY}) {$ $};
               % Flèches
               \draw[fleche] (f1) -- (f2);
               \draw[fleche] (f3) -- (f4);
               % Doubles barres
               \draw[double distance=2pt] (2*\Lrg,\lignef+\HtX) -- (2*\Lrg,\bas+0);
               % Encadrement
               \draw[cadre] (\separateur,\haut) -- (\separateur,\bas);
               \draw[cadre] (\gauche,\haut) rectangle  (\droite,\bas);
               \draw[cadre] (\gauche,\lignex) -- (\droite,\lignex);
          \end{tikzpicture}
     \end{extern}
\end{center}

\end{document}
µ
\documentclass[a4paper]{article}

%================================================================================================================================
%
% Packages
%
%================================================================================================================================

\usepackage[T1]{fontenc} 	% pour caractères accentués
\usepackage[utf8]{inputenc}  % encodage utf8
\usepackage[french]{babel}	% langue : français
\usepackage{fourier}			% caractères plus lisibles
\usepackage[dvipsnames]{xcolor} % couleurs
\usepackage{fancyhdr}		% réglage header footer
\usepackage{needspace}		% empêcher sauts de page mal placés
\usepackage{graphicx}		% pour inclure des graphiques
\usepackage{enumitem,cprotect}		% personnalise les listes d'items (nécessaire pour ol, al ...)
\usepackage{hyperref}		% Liens hypertexte
\usepackage{pstricks,pst-all,pst-node,pstricks-add,pst-math,pst-plot,pst-tree,pst-eucl} % pstricks
\usepackage[a4paper,includeheadfoot,top=2cm,left=3cm, bottom=2cm,right=3cm]{geometry} % marges etc.
\usepackage{comment}			% commentaires multilignes
\usepackage{amsmath,environ} % maths (matrices, etc.)
\usepackage{amssymb,makeidx}
\usepackage{bm}				% bold maths
\usepackage{tabularx}		% tableaux
\usepackage{colortbl}		% tableaux en couleur
\usepackage{fontawesome}		% Fontawesome
\usepackage{environ}			% environment with command
\usepackage{fp}				% calculs pour ps-tricks
\usepackage{multido}			% pour ps tricks
\usepackage[np]{numprint}	% formattage nombre
\usepackage{tikz,tkz-tab} 			% package principal TikZ
\usepackage{pgfplots}   % axes
\usepackage{mathrsfs}    % cursives
\usepackage{calc}			% calcul taille boites
\usepackage[scaled=0.875]{helvet} % font sans serif
\usepackage{svg} % svg
\usepackage{scrextend} % local margin
\usepackage{scratch} %scratch
\usepackage{multicol} % colonnes
%\usepackage{infix-RPN,pst-func} % formule en notation polanaise inversée
\usepackage{listings}

%================================================================================================================================
%
% Réglages de base
%
%================================================================================================================================

\lstset{
language=Python,   % R code
literate=
{á}{{\'a}}1
{à}{{\`a}}1
{ã}{{\~a}}1
{é}{{\'e}}1
{è}{{\`e}}1
{ê}{{\^e}}1
{í}{{\'i}}1
{ó}{{\'o}}1
{õ}{{\~o}}1
{ú}{{\'u}}1
{ü}{{\"u}}1
{ç}{{\c{c}}}1
{~}{{ }}1
}


\definecolor{codegreen}{rgb}{0,0.6,0}
\definecolor{codegray}{rgb}{0.5,0.5,0.5}
\definecolor{codepurple}{rgb}{0.58,0,0.82}
\definecolor{backcolour}{rgb}{0.95,0.95,0.92}

\lstdefinestyle{mystyle}{
    backgroundcolor=\color{backcolour},   
    commentstyle=\color{codegreen},
    keywordstyle=\color{magenta},
    numberstyle=\tiny\color{codegray},
    stringstyle=\color{codepurple},
    basicstyle=\ttfamily\footnotesize,
    breakatwhitespace=false,         
    breaklines=true,                 
    captionpos=b,                    
    keepspaces=true,                 
    numbers=left,                    
xleftmargin=2em,
framexleftmargin=2em,            
    showspaces=false,                
    showstringspaces=false,
    showtabs=false,                  
    tabsize=2,
    upquote=true
}

\lstset{style=mystyle}


\lstset{style=mystyle}
\newcommand{\imgdir}{C:/laragon/www/newmc/assets/imgsvg/}
\newcommand{\imgsvgdir}{C:/laragon/www/newmc/assets/imgsvg/}

\definecolor{mcgris}{RGB}{220, 220, 220}% ancien~; pour compatibilité
\definecolor{mcbleu}{RGB}{52, 152, 219}
\definecolor{mcvert}{RGB}{125, 194, 70}
\definecolor{mcmauve}{RGB}{154, 0, 215}
\definecolor{mcorange}{RGB}{255, 96, 0}
\definecolor{mcturquoise}{RGB}{0, 153, 153}
\definecolor{mcrouge}{RGB}{255, 0, 0}
\definecolor{mclightvert}{RGB}{205, 234, 190}

\definecolor{gris}{RGB}{220, 220, 220}
\definecolor{bleu}{RGB}{52, 152, 219}
\definecolor{vert}{RGB}{125, 194, 70}
\definecolor{mauve}{RGB}{154, 0, 215}
\definecolor{orange}{RGB}{255, 96, 0}
\definecolor{turquoise}{RGB}{0, 153, 153}
\definecolor{rouge}{RGB}{255, 0, 0}
\definecolor{lightvert}{RGB}{205, 234, 190}
\setitemize[0]{label=\color{lightvert}  $\bullet$}

\pagestyle{fancy}
\renewcommand{\headrulewidth}{0.2pt}
\fancyhead[L]{maths-cours.fr}
\fancyhead[R]{\thepage}
\renewcommand{\footrulewidth}{0.2pt}
\fancyfoot[C]{}

\newcolumntype{C}{>{\centering\arraybackslash}X}
\newcolumntype{s}{>{\hsize=.35\hsize\arraybackslash}X}

\setlength{\parindent}{0pt}		 
\setlength{\parskip}{3mm}
\setlength{\headheight}{1cm}

\def\ebook{ebook}
\def\book{book}
\def\web{web}
\def\type{web}

\newcommand{\vect}[1]{\overrightarrow{\,\mathstrut#1\,}}

\def\Oij{$\left(\text{O}~;~\vect{\imath},~\vect{\jmath}\right)$}
\def\Oijk{$\left(\text{O}~;~\vect{\imath},~\vect{\jmath},~\vect{k}\right)$}
\def\Ouv{$\left(\text{O}~;~\vect{u},~\vect{v}\right)$}

\hypersetup{breaklinks=true, colorlinks = true, linkcolor = OliveGreen, urlcolor = OliveGreen, citecolor = OliveGreen, pdfauthor={Didier BONNEL - https://www.maths-cours.fr} } % supprime les bordures autour des liens

\renewcommand{\arg}[0]{\text{arg}}

\everymath{\displaystyle}

%================================================================================================================================
%
% Macros - Commandes
%
%================================================================================================================================

\newcommand\meta[2]{    			% Utilisé pour créer le post HTML.
	\def\titre{titre}
	\def\url{url}
	\def\arg{#1}
	\ifx\titre\arg
		\newcommand\maintitle{#2}
		\fancyhead[L]{#2}
		{\Large\sffamily \MakeUppercase{#2}}
		\vspace{1mm}\textcolor{mcvert}{\hrule}
	\fi 
	\ifx\url\arg
		\fancyfoot[L]{\href{https://www.maths-cours.fr#2}{\black \footnotesize{https://www.maths-cours.fr#2}}}
	\fi 
}


\newcommand\TitreC[1]{    		% Titre centré
     \needspace{3\baselineskip}
     \begin{center}\textbf{#1}\end{center}
}

\newcommand\newpar{    		% paragraphe
     \par
}

\newcommand\nosp {    		% commande vide (pas d'espace)
}
\newcommand{\id}[1]{} %ignore

\newcommand\boite[2]{				% Boite simple sans titre
	\vspace{5mm}
	\setlength{\fboxrule}{0.2mm}
	\setlength{\fboxsep}{5mm}	
	\fcolorbox{#1}{#1!3}{\makebox[\linewidth-2\fboxrule-2\fboxsep]{
  		\begin{minipage}[t]{\linewidth-2\fboxrule-4\fboxsep}\setlength{\parskip}{3mm}
  			 #2
  		\end{minipage}
	}}
	\vspace{5mm}
}

\newcommand\CBox[4]{				% Boites
	\vspace{5mm}
	\setlength{\fboxrule}{0.2mm}
	\setlength{\fboxsep}{5mm}
	
	\fcolorbox{#1}{#1!3}{\makebox[\linewidth-2\fboxrule-2\fboxsep]{
		\begin{minipage}[t]{1cm}\setlength{\parskip}{3mm}
	  		\textcolor{#1}{\LARGE{#2}}    
 	 	\end{minipage}  
  		\begin{minipage}[t]{\linewidth-2\fboxrule-4\fboxsep}\setlength{\parskip}{3mm}
			\raisebox{1.2mm}{\normalsize\sffamily{\textcolor{#1}{#3}}}						
  			 #4
  		\end{minipage}
	}}
	\vspace{5mm}
}

\newcommand\cadre[3]{				% Boites convertible html
	\par
	\vspace{2mm}
	\setlength{\fboxrule}{0.1mm}
	\setlength{\fboxsep}{5mm}
	\fcolorbox{#1}{white}{\makebox[\linewidth-2\fboxrule-2\fboxsep]{
  		\begin{minipage}[t]{\linewidth-2\fboxrule-4\fboxsep}\setlength{\parskip}{3mm}
			\raisebox{-2.5mm}{\sffamily \small{\textcolor{#1}{\MakeUppercase{#2}}}}		
			\par		
  			 #3
 	 		\end{minipage}
	}}
		\vspace{2mm}
	\par
}

\newcommand\bloc[3]{				% Boites convertible html sans bordure
     \needspace{2\baselineskip}
     {\sffamily \small{\textcolor{#1}{\MakeUppercase{#2}}}}    
		\par		
  			 #3
		\par
}

\newcommand\CHelp[1]{
     \CBox{Plum}{\faInfoCircle}{À RETENIR}{#1}
}

\newcommand\CUp[1]{
     \CBox{NavyBlue}{\faThumbsOUp}{EN PRATIQUE}{#1}
}

\newcommand\CInfo[1]{
     \CBox{Sepia}{\faArrowCircleRight}{REMARQUE}{#1}
}

\newcommand\CRedac[1]{
     \CBox{PineGreen}{\faEdit}{BIEN R\'EDIGER}{#1}
}

\newcommand\CError[1]{
     \CBox{Red}{\faExclamationTriangle}{ATTENTION}{#1}
}

\newcommand\TitreExo[2]{
\needspace{4\baselineskip}
 {\sffamily\large EXERCICE #1\ (\emph{#2 points})}
\vspace{5mm}
}

\newcommand\img[2]{
          \includegraphics[width=#2\paperwidth]{\imgdir#1}
}

\newcommand\imgsvg[2]{
       \begin{center}   \includegraphics[width=#2\paperwidth]{\imgsvgdir#1} \end{center}
}


\newcommand\Lien[2]{
     \href{#1}{#2 \tiny \faExternalLink}
}
\newcommand\mcLien[2]{
     \href{https~://www.maths-cours.fr/#1}{#2 \tiny \faExternalLink}
}

\newcommand{\euro}{\eurologo{}}

%================================================================================================================================
%
% Macros - Environement
%
%================================================================================================================================

\newenvironment{tex}{ %
}
{%
}

\newenvironment{indente}{ %
	\setlength\parindent{10mm}
}

{
	\setlength\parindent{0mm}
}

\newenvironment{corrige}{%
     \needspace{3\baselineskip}
     \medskip
     \textbf{\textsc{Corrigé}}
     \medskip
}
{
}

\newenvironment{extern}{%
     \begin{center}
     }
     {
     \end{center}
}

\NewEnviron{code}{%
	\par
     \boite{gray}{\texttt{%
     \BODY
     }}
     \par
}

\newenvironment{vbloc}{% boite sans cadre empeche saut de page
     \begin{minipage}[t]{\linewidth}
     }
     {
     \end{minipage}
}
\NewEnviron{h2}{%
    \needspace{3\baselineskip}
    \vspace{0.6cm}
	\noindent \MakeUppercase{\sffamily \large \BODY}
	\vspace{1mm}\textcolor{mcgris}{\hrule}\vspace{0.4cm}
	\par
}{}

\NewEnviron{h3}{%
    \needspace{3\baselineskip}
	\vspace{5mm}
	\textsc{\BODY}
	\par
}

\NewEnviron{margeneg}{ %
\begin{addmargin}[-1cm]{0cm}
\BODY
\end{addmargin}
}

\NewEnviron{html}{%
}

\begin{document}
\meta{url}{/cours/fonction-derivee/}
\meta{pid}{318}
\meta{titre}{Nombre dérivé - Fonction dérivée}
\meta{type}{cours}
\begin{h2}1. Nombre dérivé\end{h2}
\cadre{bleu}{Définition}{% id="d10"
     Soit $f$ une fonction définie sur un intervalle $I$ et soient 2 réels $x_{0}$ et $h\neq 0$ tels que $x_{0} \in  I$ et $x_{0}+h \in  I$.
     \par
     Le \textbf{taux de variation} (ou \textbf{taux d'accroissement)} de la fonction $f$ entre $x_{0}$ et $x_{0}+h$ est le nombre :
     \par
     $T=\frac{f\left(x_{0}+h\right)-f\left(x_{0}\right)}{h}$
}
\cadre{bleu}{Définition}{% id="d20"
     Une fonction $f$ est \textbf{dérivable} en $x_{0}$ si et seulement si le nombre $\frac{f\left(x_{0}+h\right)-f\left(x_{0}\right)}{h}$ a pour limite un certain réel $l$ lorsque $h$ tend vers 0.
     \par
     $l$ est appelée \textbf{nombre dérivé} de $f$ en $x_{0}$, on le note $f^{\prime}\left(x_{0}\right)$.
     \par
     On écrit : $f^{\prime}\left(x_{0}\right)=\lim\limits_{h\rightarrow 0}\frac{f\left(x_{0}+h\right)-f\left(x_{0}\right)}{h}$.
}
\bloc{cyan}{Remarques}{% id="r20"
     \begin{itemize}
          \item Le quotient $\frac{f\left(x_{0}+h\right)-f\left(x_{0}\right)}{h}$ est le taux d'accroissement de $f$ entre $x_{0}$ et $x_{0}+h$.
          \item \textit{« le nombre $\frac{f\left(x_{0}+h\right)-f\left(x_{0}\right)}{h}$ a pour limite un certain réel $l$ lorsque $h$ tend vers 0 »} signifie que $\frac{f\left(x_{0}+h\right)-f\left(x_{0}\right)}{h}$ se rapproche de $l$ lorsque $h$ se rapproche de 0.
          \par
          Une définition plus rigoureuse de la notion de limite sera vue en Terminale.
          \item On peut également définir le nombre dérivé de la façon suivante:
          \par
          $f^{\prime}\left(x_{0}\right)=\lim\limits_{x\rightarrow  x_{0}}\frac{f\left(x\right)-f\left(x_{0}\right)}{x-x_{0}}$
          \par
          (cela correspond au changement de variable $x=x_{0}+h$)
     \end{itemize}
}
\bloc{orange}{Exemple}{% id="e20"
     Calculons le nombre dérivé de la fonction $f : x \mapsto x^{2}$ pour $x=1$.
     \par
     Ce nombre se note $f^{\prime}\left(1\right)$ et vaut :
     \par
     $f^{\prime}\left(1\right)=\lim\limits_{h\rightarrow 0}\frac{\left(1+h\right)^{2}-1^{2}}{h}=\lim\limits_{h\rightarrow 0}\frac{2h+h^{2}}{h}=\lim\limits_{h\rightarrow 0}2+h$
     \par
     Or quand $h$ tend vers 0, $2+h$ tend vers 2; donc $f^{\prime}\left(1\right)=2$.
}
\bloc{cyan}{Remarque:}{% id="r21"
     \textbf{Interprétation graphique du nombre dérivé :}
     \begin{center}
          \begin{extern}%width="450" alt="nombre dérivé"
               % -+-+-+ variables modifiables
               \resizebox{8cm}{!}{%
                    \def\xmin{-2.5}
                    \def\xmax{8.5}
                    \def\ymin{-1.5}
                    \def\ymax{8.5}
                    \def\xunit{1}  % unités en cm
                    \def\yunit{1}
                    \psset{xunit=\xunit,yunit=\yunit,algebraic=true}
                    \fontsize{12pt}{12pt}\selectfont
                    \begin{pspicture*}[linewidth=1pt](\xmin,\ymin)(\xmax,\ymax)
                         %      \psgrid[gridcolor=mcgris,subgriddiv=0](-4,-2)(9,9)
                         %        \psaxes[linewidth=0.75pt]{->}(0,0)(\xmin,\ymin)(\xmax,\ymax)
                         \psline[linecolor=gray]{->}(\xmin,0)(\xmax,0)
                         \psline[linecolor=gray]{->}(0,\ymin)(0,\ymax)
                         \psline[linecolor=lightgray](\xmin,0)(\xmax,0)
                         \psline[linecolor=lightgray](2,0)(2,1.69)
                         \psline[linecolor=lightgray](5,0)(5,3.713)
                         \rput[tr](-0.2,-0.2){$O$}\rput[br](1.8,1.8){$A$}\rput[br](4.8,3.693){$B$}\rput[t](3.5,-0.8){\color{mcmauve} $h$}
                         \rput[t](8,4){\color{red} $\mathscr{T}$}\rput[t](8,7.5){\color{blue} $\mathscr{C}_f$}\rput[t](2,-0.2){$x_0$}\rput[t](5,-0.1){$x_0+h$}
                         \psdots(2,1.69)\psdots(5,3.713)
                         \psplot[plotpoints=1000,linewidth=0.8pt,linecolor=blue]{\xmin}{\xmax}{1.3^x}
                         \psplot[plotpoints=1000,linewidth=0.8pt,linecolor=mcvert]{\xmin}{\xmax}{0.674*x+0.341}
                         \psplot[plotpoints=1000,linewidth=0.8pt,linecolor=red]{\xmin}{\xmax}{0.443*x+0.803}
                         \psline[linecolor=mcmauve]{<->}(2,-0.7)(5,-0.7)
                    \end{pspicture*}
               }
          \end{extern}
     \end{center}
     Soit $\mathscr{C}_f$ la courbe représentative de la fonction $f$.
     \par
     Lorsque $h$ tend vers 0, $B$ \textit{"se rapproche"} de $A$ et la droite $\left(AB\right)$ se rapproche de la tangente
     $\mathscr{T}$.
     \par
     \textbf{Le nombre dérivée $f^{\prime}\left(x_{0}\right)$ est le coefficient directeur de la tangente à la courbe $\mathscr{C}_f$ au point d'abscisse $x_{0}$.}
}
\cadre{vert}{Propriété}{% id="p30"
     Soit $f$ une fonction dérivable en $x_{0}$ de courbe représentative $\mathscr{C}_f$, l'équation de la tangente à $\mathscr{C}_f$ au point d'abscisse $x_{0}$ est :
     \par
$y=f^{\prime}\left(x_{0}\right)\left(x-x_{0}\right)+f\left(x_{0}\right)$}
\bloc{orange}{Démonstration}{% id="m21"
     D'après la propriété précédente, la tangente à $\mathscr{C}_f$ au point d'abscisse $x_{0}$ est une droite de coefficient directeur $f^{\prime}\left(x_{0}\right)$. Son équation est donc de la forme :
     \par
     $y=f^{\prime}\left(x_{0}\right)x+b$
     \par
     On sait que la tangente passe par le point $A$ de coordonnées $\left(x_{0}; f\left(x_{0}\right)\right)$ donc :
     \par
     $f\left(x_{0}\right)=f^{\prime}\left(x_{0}\right)x_{0}+b$
     \par
     $b=-f^{\prime}\left(x_{0}\right)x_{0}+f\left(x_{0}\right)$
     \par
     L'équation de la tangente est donc :
     \par
     $y=f^{\prime}\left(x_{0}\right)x-f^{\prime}\left(x_{0}\right)x_{0}+f\left(x_{0}\right)$
     \par
     Soit :
     \par
     $y=f^{\prime}\left(x_{0}\right)\left(x-x_{0}\right)+f\left(x_{0}\right)$
}
\begin{h2}2. Fonction dérivée\end{h2}
\cadre{bleu}{Définition}{% id="d40"
     Soit $f$ une fonction définie sur un intervalle $I$. On dit que $f$ est \textbf{dérivable} sur $I$ si et seulement si pour tout $x \in  I$, le nombre dérivé $f^{\prime}\left(x\right)$ existe.
     \par
     La fonction qui à $x \in  I$ associe le nombre dérivé de $f$ en $x$ s'appelle la \textbf{fonction dérivée} et se note $f^{\prime}$
}
\cadre{vert}{Propriétés}{% id="p50"
     \textbf{Dérivée des fonctions usuelles :}
     \begin{center}
          \begin{tabularx}{0.8\linewidth}{|X|X|X|}%class="compact" width="600"
               \hline
               \textbf{Fonction} & \textbf{Dérivée} & \textbf{Ensemble de dérivabilité}
               \\ \hline
               $k$  $\left(k\in \mathbb{R}\right)$  &  $0$  &  $\mathbb{R}$
               \\ \hline
               $x$ &  $1$  &  $\mathbb{R}$
               \\ \hline
               $x^{n}$ $\left(n\in \mathbb{N}\right)$  &  $nx^{n-1}$  &  $\mathbb{R}$
               \\ \hline
               $\frac{1}{x^{n}}$ $\left(n\in \mathbb{N}\right)$ &  $-\frac{n}{x^{n+1}}$  &  $\mathbb{R}-\left\{0\right\}$
               \\ \hline
               $\sqrt{x}$ &  $\frac{1}{2\sqrt{x}}$  &  $\left]0;+\infty \right[$
               \\        \hline
          \end{tabularx}
     \end{center}
}
\cadre{vert}{Propriétés}{% id="p60"
     \textbf{Formules de base :}
     \par
     Si $u$ et $v$ sont 2 fonctions dérivables sur un intervalle $I$. Sur cet intervalle :
     \begin{center}
          \begin{tabularx}{0.8\linewidth}{|*{3}{>{\centering \arraybackslash }X|}}%class="compact" width="600"
               \hline
               \textbf{Fonction} & \textbf{Dérivée}
               \\ \hline
               $u+v$  &  $u^{\prime}+v^{\prime}$
               \\ \hline
               $ku$  $\left(k\in \mathbb{R}\right)$  &  $ku^{\prime} $
               \\ \hline
               $\frac{1}{u}$ (avec $u\left(x\right)\neq 0$ sur $I$)  &  $-\frac{u^{\prime} }{u^{2}} $
               \\ \hline
               $uv$  &  $u^{\prime}v+uv^{\prime}$
               \\ \hline
               $\frac{u}{v}$  (avec $v\left(x\right)\neq 0$ sur $I$) &  $\frac{u^{\prime}v-uv^{\prime}}{v^{2}} $
               \\ \hline
               $\sqrt{u}$  (avec $u\geqslant 0$ sur $I$) &  $\frac{u^{\prime}}{2\sqrt{u}}$ lorsque $u > 0$
               \\  \hline
          \end{tabularx}
     \end{center}
}
\bloc{orange}{Exemple}{% id="e70"
     On cherche à calculer la dérivée de la fonction $f$ définie sur $\mathbb{R}$ par :
     \par
     $f\left(x\right)=\frac{x}{x^{2}+1}$
     \par
     On pose
     \par
     $u\left(x\right)=x$ et $v\left(x\right)=x^{2}+1$
     \par
     On a alors
     \par
     $u^{\prime}\left(x\right)=1$
     \par
     $v^{\prime}\left(x\right)=2x$
     \par
     car la dérivée de la fonction $x \mapsto  x^{2}$ est la fonction $x \mapsto  2x$  (formule $nx^{n-1}$ avec $n=2$) et la dérivée de la fonction constante $x \mapsto 1$ est la fonction nulle.
     \par
     La dérivée du quotient est donc :
     \par
     $f^{\prime}\left(x\right)=\frac{u^{\prime}\left(x\right)v\left(x\right)-u\left(x\right)v^{\prime}\left(x\right)}{v\left(x\right)^{2}}=\frac{1\times \left(x^{2}+1\right)-x\times 2x}{\left(x^{2}+1\right)^{2}}=\frac{1-x^{2}}{\left(x^{2}+1\right)^{2}}$
}
\bloc{cyan}{Remarques}{% id="r70"
     \begin{itemize}
          \item Si le dénominateur d'une fraction est constant, il est très maladroit d'utiliser la formule
          \par
          $\left(\frac{u}{v}\right)^{\prime}=\frac{u^{\prime}v-uv^{\prime}}{v^{2}}$.
          \par
          Par exemple pour dériver $f\left(x\right)=\frac{x^{2}+1}{5}$ on écrira :
          \par
          $f\left(x\right)=\frac{1}{5}\times \left(x^{2}+1\right)$
          \par
          donc $f^{\prime}\left(x\right)=\frac{1}{5}\times \left(2x\right)$ (formule $\left(ku\right)^{\prime}=ku^{\prime}$)
          \par
          $f^{\prime}\left(x\right)=\frac{2x}{5}$
          \item De même, si le numérateur d'une fraction est constant on utilisera, de préférence, la formule :
          \par
          $\left(\frac{1}{u}\right)^{\prime}=-\frac{u^{\prime}}{u^{2}}$
          \par
          Par exemple, si $f\left(x\right)=\frac{5}{x^{2}+1}$
          \par
          $f\left(x\right)=5\times \frac{1}{x^{2}+1}$ donc :
          \par
          $f^{\prime}\left(x\right)=5\times \left(-\frac{2x}{\left(x^{2}+1\right)^{2}}\right)=-\frac{10x}{\left(x^{2}+1\right)^{2}}$ (formule $\left(\frac{1}{u}\right)^{\prime}=-\frac{u^{\prime}}{u^{2}}$ avec $u\left(x\right)=x^{2}+1$ donc $u^{\prime}\left(x\right)=2x$)
     \end{itemize}
}
\begin{h2}3. Fonction dérivée et sens de variations\end{h2}
\cadre{rouge}{Théorème}{% id="t80"
     Soit $f$ une fonction définie sur un intervalle $I$.
     \begin{itemize}
          \item $f$ est croissante sur $I$ si et seulement si $f^{\prime}\left(x\right)\geqslant 0$ pour tout $x \in  I$
          \item $f$ est décroissante sur $I$ si et seulement si $f^{\prime}\left(x\right)\leqslant 0$ pour tout $x \in  I$
     \end{itemize}
}
\bloc{cyan}{Remarque}{% id="r80"
     Si $f^{\prime}\left(x\right) > 0$ (resp.  $f^{\prime}\left(x\right) < 0$) sur $I$, alors $f$ est \textbf{strictement} croissante (resp. décroissante) sur $I$.
     \par
     Mais la réciproque est fausse. Une fonction peut être strictement croissante sur $I$ alors que sa dérivée s'annule sur $I$. C'est le cas par exemple de la fonction $x \mapsto  x^{3}$ qui est strictement croissante sur $\mathbb{R}$ alors que sa dérivée $x \mapsto  3x^{2}$ s'annule pour $x=0$
}
\bloc{orange}{Exemple}{% id="e80"
     Reprenons la fonction de l'exemple précédent.
     \par
     $f\left(x\right)=\frac{x}{x^{2}+1}$
     \par
     $f^{\prime}\left(x\right)=\frac{1-x^{2}}{\left(x^{2}+1\right)^{2}}$
     \par
     Le dénominateur de $f^{\prime}\left(x\right)$ est toujours strictement positif.
     \par
     Le numérateur de $f^{\prime}\left(x\right)$ peut se factoriser : $1-x^{2}=\left(1-x\right)\left(1+x\right)$
     \par
     Une facile étude de signe montre que $f^{\prime}$ est strictement négative sur $\left]-\infty  ; -1\right[$ et $\left]1 ; +\infty \right[$ et est strictement positive sur $\left]-1 ; 1\right[$.
     \par
     Par ailleurs, $f\left(-1\right)=-\frac{1}{2}$ et $f\left(1\right)=\frac{1}{2}$
     \par
     On en déduit le tableau de variations de $f$ (que l'on regroupe habituellement avec le tableau de signe de $f^{\prime}$) :
     \begin{center}
          \begin{extern}%width="500" alt="Dérivée et tableau de variations "
               \begin{tikzpicture}[scale=1]
                    % Styles
                    \tikzstyle{cadre}=[thin]
                    \tikzstyle{fleche}=[->,>=latex,thin]
                    \tikzstyle{nondefini}=[lightgray]
                    % Dimensions Modifiables
                    \def\Lrg{1.5}
                    \def\HtX{1}
                    \def\HtY{0.5}
                    % Dimensions Calculées
                    \def\lignex{-0.5*\HtX}
                    \def\lignef{-1.5*\HtX}
                    \def\separateur{-0.5*\Lrg}
                    % Largeur du tableau
                    \def\gauche{-1.5*\Lrg}
                    \def\droite{6.5*\Lrg}
                    % Hauteur du tableau
                    \def\haut{0.5*\HtX}
                    \def\bas{-2.5*\HtX-2*\HtY}
                    % Ligne de l'abscisse : x
                    \node at (-1*\Lrg,0) {$x$};
                    \node at (0*\Lrg,0) {$-\infty$};
                    \node at (2*\Lrg,0) {$-1$};
                    \node at (4*\Lrg,0) {$1$};
                    \node at (6*\Lrg,0) {$+\infty$};
                    % Pointillés
                    \draw[lightgray] (2*\Lrg,\lignex-0) -- (2*\Lrg,\lignef+0);
                    \draw[lightgray] (2*\Lrg,\lignef-0) -- (2*\Lrg,\bas+0);
                    \draw[lightgray] (4*\Lrg,\lignex-0) -- (4*\Lrg,\lignef+0);
                    \draw[lightgray] (4*\Lrg,\lignef-0) -- (4*\Lrg,\bas+0);
                    % Ligne de la dérivée : f'(x)
                    \node at (-1*\Lrg,-1*\HtX) {$f'(x)$};
                    \node at (0*\Lrg,-1*\HtX) {$ $};
                    \node at (1*\Lrg,-1*\HtX) {$-$};
                    \node at (2*\Lrg,-1*\HtX) {$0$};
                    \node at (3*\Lrg,-1*\HtX) {$+$};
                    \node at (4*\Lrg,-1*\HtX) {$0$};
                    \node at (5*\Lrg,-1*\HtX) {$-$};
                    \node at (6*\Lrg,-1*\HtX) {$ $};
                    % Ligne de la fonction : f(x)
                    \node  at (-1*\Lrg,{-2*\HtX+(-1)*\HtY}) {$f(x)$};
                    \node (f1) at (0*\Lrg,{-2*\HtX+(0)*\HtY}) {$$};
                    \node (f2) at (2*\Lrg,{-2*\HtX+(-2)*\HtY}) {$-\dfrac{1}{2}$};
                    \node (f3) at (4*\Lrg,{-2*\HtX+(0)*\HtY}) {$\dfrac{1}{2}$};
                    \node (f4) at (6*\Lrg,{-2*\HtX+(-2)*\HtY}) {$$};
                    % Flèches
                    \draw[fleche] (f1) -- (f2);
                    \draw[fleche] (f2) -- (f3);
                    \draw[fleche] (f3) -- (f4);
                    % Encadrement
                    \draw[cadre] (\separateur,\haut) -- (\separateur,\bas);
                    \draw[cadre] (\gauche,\haut) rectangle  (\droite,\bas);
                    \draw[cadre] (\gauche,\lignex) -- (\droite,\lignex);
                    \draw[cadre] (\gauche,\lignef) -- (\droite,\lignef);
               \end{tikzpicture}
          \end{extern}
     \end{center}
}

\end{document}
µ
\documentclass[a4paper]{article}

%================================================================================================================================
%
% Packages
%
%================================================================================================================================

\usepackage[T1]{fontenc} 	% pour caractères accentués
\usepackage[utf8]{inputenc}  % encodage utf8
\usepackage[french]{babel}	% langue : français
\usepackage{fourier}			% caractères plus lisibles
\usepackage[dvipsnames]{xcolor} % couleurs
\usepackage{fancyhdr}		% réglage header footer
\usepackage{needspace}		% empêcher sauts de page mal placés
\usepackage{graphicx}		% pour inclure des graphiques
\usepackage{enumitem,cprotect}		% personnalise les listes d'items (nécessaire pour ol, al ...)
\usepackage{hyperref}		% Liens hypertexte
\usepackage{pstricks,pst-all,pst-node,pstricks-add,pst-math,pst-plot,pst-tree,pst-eucl} % pstricks
\usepackage[a4paper,includeheadfoot,top=2cm,left=3cm, bottom=2cm,right=3cm]{geometry} % marges etc.
\usepackage{comment}			% commentaires multilignes
\usepackage{amsmath,environ} % maths (matrices, etc.)
\usepackage{amssymb,makeidx}
\usepackage{bm}				% bold maths
\usepackage{tabularx}		% tableaux
\usepackage{colortbl}		% tableaux en couleur
\usepackage{fontawesome}		% Fontawesome
\usepackage{environ}			% environment with command
\usepackage{fp}				% calculs pour ps-tricks
\usepackage{multido}			% pour ps tricks
\usepackage[np]{numprint}	% formattage nombre
\usepackage{tikz,tkz-tab} 			% package principal TikZ
\usepackage{pgfplots}   % axes
\usepackage{mathrsfs}    % cursives
\usepackage{calc}			% calcul taille boites
\usepackage[scaled=0.875]{helvet} % font sans serif
\usepackage{svg} % svg
\usepackage{scrextend} % local margin
\usepackage{scratch} %scratch
\usepackage{multicol} % colonnes
%\usepackage{infix-RPN,pst-func} % formule en notation polanaise inversée
\usepackage{listings}

%================================================================================================================================
%
% Réglages de base
%
%================================================================================================================================

\lstset{
language=Python,   % R code
literate=
{á}{{\'a}}1
{à}{{\`a}}1
{ã}{{\~a}}1
{é}{{\'e}}1
{è}{{\`e}}1
{ê}{{\^e}}1
{í}{{\'i}}1
{ó}{{\'o}}1
{õ}{{\~o}}1
{ú}{{\'u}}1
{ü}{{\"u}}1
{ç}{{\c{c}}}1
{~}{{ }}1
}


\definecolor{codegreen}{rgb}{0,0.6,0}
\definecolor{codegray}{rgb}{0.5,0.5,0.5}
\definecolor{codepurple}{rgb}{0.58,0,0.82}
\definecolor{backcolour}{rgb}{0.95,0.95,0.92}

\lstdefinestyle{mystyle}{
    backgroundcolor=\color{backcolour},   
    commentstyle=\color{codegreen},
    keywordstyle=\color{magenta},
    numberstyle=\tiny\color{codegray},
    stringstyle=\color{codepurple},
    basicstyle=\ttfamily\footnotesize,
    breakatwhitespace=false,         
    breaklines=true,                 
    captionpos=b,                    
    keepspaces=true,                 
    numbers=left,                    
xleftmargin=2em,
framexleftmargin=2em,            
    showspaces=false,                
    showstringspaces=false,
    showtabs=false,                  
    tabsize=2,
    upquote=true
}

\lstset{style=mystyle}


\lstset{style=mystyle}
\newcommand{\imgdir}{C:/laragon/www/newmc/assets/imgsvg/}
\newcommand{\imgsvgdir}{C:/laragon/www/newmc/assets/imgsvg/}

\definecolor{mcgris}{RGB}{220, 220, 220}% ancien~; pour compatibilité
\definecolor{mcbleu}{RGB}{52, 152, 219}
\definecolor{mcvert}{RGB}{125, 194, 70}
\definecolor{mcmauve}{RGB}{154, 0, 215}
\definecolor{mcorange}{RGB}{255, 96, 0}
\definecolor{mcturquoise}{RGB}{0, 153, 153}
\definecolor{mcrouge}{RGB}{255, 0, 0}
\definecolor{mclightvert}{RGB}{205, 234, 190}

\definecolor{gris}{RGB}{220, 220, 220}
\definecolor{bleu}{RGB}{52, 152, 219}
\definecolor{vert}{RGB}{125, 194, 70}
\definecolor{mauve}{RGB}{154, 0, 215}
\definecolor{orange}{RGB}{255, 96, 0}
\definecolor{turquoise}{RGB}{0, 153, 153}
\definecolor{rouge}{RGB}{255, 0, 0}
\definecolor{lightvert}{RGB}{205, 234, 190}
\setitemize[0]{label=\color{lightvert}  $\bullet$}

\pagestyle{fancy}
\renewcommand{\headrulewidth}{0.2pt}
\fancyhead[L]{maths-cours.fr}
\fancyhead[R]{\thepage}
\renewcommand{\footrulewidth}{0.2pt}
\fancyfoot[C]{}

\newcolumntype{C}{>{\centering\arraybackslash}X}
\newcolumntype{s}{>{\hsize=.35\hsize\arraybackslash}X}

\setlength{\parindent}{0pt}		 
\setlength{\parskip}{3mm}
\setlength{\headheight}{1cm}

\def\ebook{ebook}
\def\book{book}
\def\web{web}
\def\type{web}

\newcommand{\vect}[1]{\overrightarrow{\,\mathstrut#1\,}}

\def\Oij{$\left(\text{O}~;~\vect{\imath},~\vect{\jmath}\right)$}
\def\Oijk{$\left(\text{O}~;~\vect{\imath},~\vect{\jmath},~\vect{k}\right)$}
\def\Ouv{$\left(\text{O}~;~\vect{u},~\vect{v}\right)$}

\hypersetup{breaklinks=true, colorlinks = true, linkcolor = OliveGreen, urlcolor = OliveGreen, citecolor = OliveGreen, pdfauthor={Didier BONNEL - https://www.maths-cours.fr} } % supprime les bordures autour des liens

\renewcommand{\arg}[0]{\text{arg}}

\everymath{\displaystyle}

%================================================================================================================================
%
% Macros - Commandes
%
%================================================================================================================================

\newcommand\meta[2]{    			% Utilisé pour créer le post HTML.
	\def\titre{titre}
	\def\url{url}
	\def\arg{#1}
	\ifx\titre\arg
		\newcommand\maintitle{#2}
		\fancyhead[L]{#2}
		{\Large\sffamily \MakeUppercase{#2}}
		\vspace{1mm}\textcolor{mcvert}{\hrule}
	\fi 
	\ifx\url\arg
		\fancyfoot[L]{\href{https://www.maths-cours.fr#2}{\black \footnotesize{https://www.maths-cours.fr#2}}}
	\fi 
}


\newcommand\TitreC[1]{    		% Titre centré
     \needspace{3\baselineskip}
     \begin{center}\textbf{#1}\end{center}
}

\newcommand\newpar{    		% paragraphe
     \par
}

\newcommand\nosp {    		% commande vide (pas d'espace)
}
\newcommand{\id}[1]{} %ignore

\newcommand\boite[2]{				% Boite simple sans titre
	\vspace{5mm}
	\setlength{\fboxrule}{0.2mm}
	\setlength{\fboxsep}{5mm}	
	\fcolorbox{#1}{#1!3}{\makebox[\linewidth-2\fboxrule-2\fboxsep]{
  		\begin{minipage}[t]{\linewidth-2\fboxrule-4\fboxsep}\setlength{\parskip}{3mm}
  			 #2
  		\end{minipage}
	}}
	\vspace{5mm}
}

\newcommand\CBox[4]{				% Boites
	\vspace{5mm}
	\setlength{\fboxrule}{0.2mm}
	\setlength{\fboxsep}{5mm}
	
	\fcolorbox{#1}{#1!3}{\makebox[\linewidth-2\fboxrule-2\fboxsep]{
		\begin{minipage}[t]{1cm}\setlength{\parskip}{3mm}
	  		\textcolor{#1}{\LARGE{#2}}    
 	 	\end{minipage}  
  		\begin{minipage}[t]{\linewidth-2\fboxrule-4\fboxsep}\setlength{\parskip}{3mm}
			\raisebox{1.2mm}{\normalsize\sffamily{\textcolor{#1}{#3}}}						
  			 #4
  		\end{minipage}
	}}
	\vspace{5mm}
}

\newcommand\cadre[3]{				% Boites convertible html
	\par
	\vspace{2mm}
	\setlength{\fboxrule}{0.1mm}
	\setlength{\fboxsep}{5mm}
	\fcolorbox{#1}{white}{\makebox[\linewidth-2\fboxrule-2\fboxsep]{
  		\begin{minipage}[t]{\linewidth-2\fboxrule-4\fboxsep}\setlength{\parskip}{3mm}
			\raisebox{-2.5mm}{\sffamily \small{\textcolor{#1}{\MakeUppercase{#2}}}}		
			\par		
  			 #3
 	 		\end{minipage}
	}}
		\vspace{2mm}
	\par
}

\newcommand\bloc[3]{				% Boites convertible html sans bordure
     \needspace{2\baselineskip}
     {\sffamily \small{\textcolor{#1}{\MakeUppercase{#2}}}}    
		\par		
  			 #3
		\par
}

\newcommand\CHelp[1]{
     \CBox{Plum}{\faInfoCircle}{À RETENIR}{#1}
}

\newcommand\CUp[1]{
     \CBox{NavyBlue}{\faThumbsOUp}{EN PRATIQUE}{#1}
}

\newcommand\CInfo[1]{
     \CBox{Sepia}{\faArrowCircleRight}{REMARQUE}{#1}
}

\newcommand\CRedac[1]{
     \CBox{PineGreen}{\faEdit}{BIEN R\'EDIGER}{#1}
}

\newcommand\CError[1]{
     \CBox{Red}{\faExclamationTriangle}{ATTENTION}{#1}
}

\newcommand\TitreExo[2]{
\needspace{4\baselineskip}
 {\sffamily\large EXERCICE #1\ (\emph{#2 points})}
\vspace{5mm}
}

\newcommand\img[2]{
          \includegraphics[width=#2\paperwidth]{\imgdir#1}
}

\newcommand\imgsvg[2]{
       \begin{center}   \includegraphics[width=#2\paperwidth]{\imgsvgdir#1} \end{center}
}


\newcommand\Lien[2]{
     \href{#1}{#2 \tiny \faExternalLink}
}
\newcommand\mcLien[2]{
     \href{https~://www.maths-cours.fr/#1}{#2 \tiny \faExternalLink}
}

\newcommand{\euro}{\eurologo{}}

%================================================================================================================================
%
% Macros - Environement
%
%================================================================================================================================

\newenvironment{tex}{ %
}
{%
}

\newenvironment{indente}{ %
	\setlength\parindent{10mm}
}

{
	\setlength\parindent{0mm}
}

\newenvironment{corrige}{%
     \needspace{3\baselineskip}
     \medskip
     \textbf{\textsc{Corrigé}}
     \medskip
}
{
}

\newenvironment{extern}{%
     \begin{center}
     }
     {
     \end{center}
}

\NewEnviron{code}{%
	\par
     \boite{gray}{\texttt{%
     \BODY
     }}
     \par
}

\newenvironment{vbloc}{% boite sans cadre empeche saut de page
     \begin{minipage}[t]{\linewidth}
     }
     {
     \end{minipage}
}
\NewEnviron{h2}{%
    \needspace{3\baselineskip}
    \vspace{0.6cm}
	\noindent \MakeUppercase{\sffamily \large \BODY}
	\vspace{1mm}\textcolor{mcgris}{\hrule}\vspace{0.4cm}
	\par
}{}

\NewEnviron{h3}{%
    \needspace{3\baselineskip}
	\vspace{5mm}
	\textsc{\BODY}
	\par
}

\NewEnviron{margeneg}{ %
\begin{addmargin}[-1cm]{0cm}
\BODY
\end{addmargin}
}

\NewEnviron{html}{%
}

\begin{document}
\meta{url}{/cours/suites/}
\meta{pid}{323}
\meta{titre}{Les suites : Généralités}
\meta{type}{cours}
\begin{h2}I - Définition d'une suite\end{h2}
\cadre{bleu}{Définitions}{%id="d10"
     Une \textbf{suite} $u$ associe à tout entier naturel $n$ un nombre réel noté $u_{n}$.
     \par
     Les nombres réels $u_{n}$ sont les \textbf{termes} de la suite.
     \par
     Les nombres entiers $n$ sont les \textbf{indices} ou les \textbf{rangs}.
     \par
     La suite $u$ peut également se noter $\left(u_{n}\right)$ ou $\left(u_{n}\right)_{n\in \mathbb{N}}$
}
\bloc{cyan}{Remarque}{%id="r10"
     Intuitivement, une suite est une liste infinie et ordonnée de nombres réels. Ces nombres réels sont les termes de la suite et les indices correspondent à la position du terme dans la liste.
}
\bloc{orange}{Exemple}{%id="e10"
     Par exemple la liste 1,6 ; 2,4 ; 3,2 ; 5 ; ... correspond à la suite $\left(u_{n}\right)$ suivante :
     \par
     $u_{0}=1,6$ (terme de rang 0)
     \par
     $u_{1}=2,4$ (terme de rang 1)
     \par
     $u_{2}=3,2$ (terme de rang 2)
     \par
     $u_{3}=5$ ...
}
\bloc{cyan}{Remarque}{%id="r11"
     Ne pas confondre l'écriture $\left(u_{n}\right)$ avec parenthèses qui désigne la suite et l'écriture $u_{n}$ sans parenthèse qui désigne le $n$-ième terme de la suite.
}
\cadre{bleu}{Définition}{%id="d20"
     Une suite est définie de façon \textbf{explicite} lorsqu'on dispose d'une formule du type $u_{n}=f\left(n\right)$ permettant de calculer chaque terme de la suite à partir de son rang.
}
\bloc{orange}{Exemple}{%id="e20"
     La suite $\left(u_{n}\right)$ définie par la formule explicite $u_{n}=\frac{2n+1}{3}$ est telle que
     \par
     $u_{0}=\frac{1}{3}$
     \par
     $u_{1}=\frac{3}{3}=1$ ...
     \par
     $u_{100}=\frac{201}{3}=67$
}
\cadre{bleu}{Définition}{%id="d30"
     Une suite est définie par une relation de \textbf{récurrence} lorsqu'on dispose du premier terme et d'une formule du type $u_{n+1}=f\left(u_{n}\right)$ permettant de calculer chaque terme de la suite à partir du terme précédent..
}
\bloc{cyan}{Remarque}{%id="r30"
     Il est possible de calculer un terme quelconque d'une suite définie par une relation de récurrence mais il faut au préalable calculer tout les termes précédents. Comme cela peut se révéler long, on utilise parfois un algorithme pour faire ce calcul.
}
\bloc{orange}{Exemple}{%id="e30"
     La suite $\left(u_{n}\right)$ définie par la formule de récurrence
     \par
     $\left\{ \begin{matrix} u_{0}=1 \\ u_{n+1}=2u_{n}-3\end{matrix}\right.$
     \par
     est telle que :
     \par
     $u_{0}=1$
     \par
     $u_{1}=2\times u_{0}-3=2\times 1-3=-1$
     \par
     $u_{2}=2\times u_{1}-3=2\times \left(-1\right)-3=-5$
     \par
     etc...
}
\begin{h2}II - Représentation graphique d'une suite\end{h2}
\cadre{bleu}{Définition}{%id="d40"
     La représentation graphique d'une suite $\left(u_{n}\right)$ ($n \in \mathbb{N}$) dans un repère du plan, s'obtient en plaçant les points de coordonnées $\left(n ; u_{n}\right)$ lorsque $n$ parcourt $\mathbb{N}$
}
\bloc{orange}{Exemple}{%id="e40"
     Pour représenter la suite définie par $u_{n}=1+\frac{3}{n+1}$ on calcule:
     \par
     $u_{0}=4$
     \par
     $u_{1}=\frac{5}{2}$
     \par
     $u_{2}=2$
     \par
     $u_{3}=\frac{7}{4}$
     \par
     etc.
     \par
     et on place les points de coordonnées : $\left(0 ; 4\right) ; \left(1 ; \frac{5}{2}\right) ; \left(2 ; 2\right) ; \left(3 ; \frac{7}{4}\right)$; etc.
}
\begin{center}
     \begin{extern}%width="300" alt="représentation graphique d'une suite"
          % -+-+-+ variables modifiables
          \resizebox{7cm}{!}{%
               \def\xmin{-0.8}
               \def\xmax{7.5}
               \def\ymin{-0.8}
               \def\ymax{4.8}
               \def\xunit{1}  % unités en cm
               \def\yunit{1}
               \psset{xunit=\xunit,yunit=\yunit,algebraic=true}
               \fontsize{15pt}{15pt}\selectfont
               \begin{pspicture*}[linewidth=1pt](\xmin,\ymin)(\xmax,\ymax)
                    \psaxes[Dx=1,Dy=1,linewidth=0.75pt]{->}(0,0)(\xmin,\ymin)(\xmax,\ymax)
                    \rput[tr](-0.2,-0.3){$O$}
                    \multido{\n=0.0+1}{8}{
                         \FPeval{\suite}{1+3/(\n+1)}
                         \psdots[linecolor=blue](\n,\suite)
                    }
               \end{pspicture*}
          }
     \end{extern}
\end{center}
\begin{center}
     \textit{Représentation graphique de la suite définie par $u_{n}=1+\frac{3}{n+1}$}
\end{center}
\begin{h2}III - Sens de variation d'une suite\end{h2}
\cadre{bleu}{Définitions}{%id="d50"
     On dit qu'une suite $\left(u_{n}\right)$ est \textbf{croissante} (\textit{resp.\textbf{décroissante})} si pour tout entier naturel $n$ :
     \begin{center}$u_{n+1} \geqslant u_{n} $ (\textit{resp. $u_{n+1} \leqslant u_{n} $)}\end{center}
     On dit qu'une suite $\left(u_{n}\right)$ est \textbf{strictement croissante} (\textit{resp.\textbf{strictement décroissante})} si pour tout entier naturel $n$ :
     \begin{center}$u_{n+1} > u_{n} $ (\textit{resp. $u_{n+1} < u_{n} $)}\end{center}
     On dit qu'une suite $\left(u_{n}\right)$ est \textbf{constante} si pour tout entier naturel $n$ :
     \begin{center}$u_{n+1} = u_{n} $\end{center}
}
\bloc{cyan}{Remarques}{%id="r50"
     \begin{itemize}
          \item Une suite peut n'être ni croissante,, ni décroissante, ni constante. C'est le cas, par exemple de la suite définie par $u_{n}=\left(-1\right)^{n}$ dont les termes valent successivement : $1; -1; 1; -1; 1; -1;$ etc.
          \item En pratique pour savoir si une suite $\left(u_{n}\right)$ est croissante ou décroissante, on calcule souvent $u_{n+1}-u_{n}$ :
          \begin{itemize}[label=---]
               \item si $u_{n+1}-u_{n} \geqslant 0$ pour tout $n \in \mathbb{N}$, la suite $u_{n}$ est croissante
               \item  si $u_{n+1}-u_{n} \leqslant 0$ pour tout $n \in \mathbb{N}$, la suite $u_{n}$ est décroissante
               \item  si $u_{n+1}-u_{n} = 0$ pour tout $n \in \mathbb{N}$, la suite $u_{n}$ est constante.
          \end{itemize}
     \end{itemize}
}
\begin{h2}IV - Notion de limite\end{h2}
\cadre{bleu}{Définition}{%id="d60"
     On dit que la suite $u_{n}$ \textbf{converge} vers le nombre réel $l$ (ou \textbf{admet pour limite} le nombre réel $l$) si les termes de la suite se rapprochent de $l$ lorsque $n$ devient grand.
}
\begin{center}
     \begin{extern}%width="300" alt="représentation graphique d'une suite convergeant vers 3"
          % -+-+-+ variables modifiables
          \resizebox{7cm}{!}{%
               \def\xmin{-0.8}
               \def\xmax{7.5}
               \def\ymin{-0.8}
               \def\ymax{4.8}
               \def\xunit{1}  % unités en cm
               \def\yunit{1}
               \psset{xunit=\xunit,yunit=\yunit,algebraic=true}
               \fontsize{15pt}{15pt}\selectfont
               \begin{pspicture*}[linewidth=1pt](\xmin,\ymin)(\xmax,\ymax)
                    \psaxes[Dx=1,Dy=1,linewidth=0.75pt]{->}(0,0)(\xmin,\ymin)(\xmax,\ymax)
                    \rput[tr](-0.2,-0.3){$O$}
                    \multido{\n=0.0+1}{8}{
                         \FPeval{\suite}{3-3/(\n+1)}
                         \psdots[linecolor=blue](\n,\suite)
                    }
                    \psline[linecolor=red,linewidth=0.5pt](-1,3)(8,3)
               \end{pspicture*}
          }
     \end{extern}
\end{center}
\begin{center}
     \textit{Suite convergente vers 3}
\end{center}
\bloc{cyan}{Remarques}{%id="r60"
     \begin{itemize}
          \item Une suite qui n'est pas convergente est dite \textbf{divergente}.
          \item La limite, si elle existe, est \textbf{unique}.
     \end{itemize}
}
\bloc{orange}{Exemples}{%id="e60"
     \begin{itemize}
          \item La suite définie pour $n > 0$ par $u_{n}=\frac{1}{n}$ , \textbf{converge vers zéro}
          \begin{center}
               \begin{tabular}{|c|c|c|c|c|c|c|c|c|}%class="compact" width="600"
                    \hline
                    $n$ & 1 & 2 & 3 & 4 & 5 & 6 & 7 & ...\\ \hline
                    $u_{n}=\frac{1}{n}$ & 1 & 0,5 & 0,33 & 0,25 & 0,2 & 0,17 & 0,14 & ...\\ \hline
               \end{tabular}
          \end{center}
          \item La suite définie pour tout $n\in \mathbb{N}$ par $u_{n}=\left(-1\right)^{n}$ est \textbf{divergente}. En effet, les termes de la suite « oscillent » indéfiniment entre $1$ et $-1$
          \begin{center}
               \begin{tabular}{|c|c|c|c|c|c|c|c|c|}%class="compact" width="600"
                    \hline
                    $n$ & 0 & 1 & 2 & 3 & 4 & 5 & 6 & ...\\ \hline
                    $u_{n}=\left(-1\right)^{n}$ & 1 & -1 & 1 & -1 & 1 & -1 & 1 & ...\\ \hline
               \end{tabular}
          \end{center}               \item La suite définie pour tout $n\in \mathbb{N}$ par récurrence par~:
          \begin{center}
               $\left\{ \begin{matrix} u_{0}=1 \\ u_{n+1}=u_{n}+2\end{matrix}\right.$
          \end{center}
          est elle aussi \textbf{divergente}. Les termes de la suite croissent indéfiniment en ne se rapprochant d'aucun nombre réel.
          \begin{center}
               \begin{tabular}{|c|c|c|c|c|c|c|c|c|}%class="compact" width="600"
                    \hline
                    $n$ & 0 & 1 & 2 & 3 & 4 & 5 & 6 & ...\\ \hline
                    $u_{n}$ & 1 & 3 & 5 & 7 & 9 & 11 & 13 & ...\\ \hline
               \end{tabular}
     \end{center}               \end{itemize}
}

\end{document}
µ
\documentclass[a4paper]{article}

%================================================================================================================================
%
% Packages
%
%================================================================================================================================

\usepackage[T1]{fontenc} 	% pour caractères accentués
\usepackage[utf8]{inputenc}  % encodage utf8
\usepackage[french]{babel}	% langue : français
\usepackage{fourier}			% caractères plus lisibles
\usepackage[dvipsnames]{xcolor} % couleurs
\usepackage{fancyhdr}		% réglage header footer
\usepackage{needspace}		% empêcher sauts de page mal placés
\usepackage{graphicx}		% pour inclure des graphiques
\usepackage{enumitem,cprotect}		% personnalise les listes d'items (nécessaire pour ol, al ...)
\usepackage{hyperref}		% Liens hypertexte
\usepackage{pstricks,pst-all,pst-node,pstricks-add,pst-math,pst-plot,pst-tree,pst-eucl} % pstricks
\usepackage[a4paper,includeheadfoot,top=2cm,left=3cm, bottom=2cm,right=3cm]{geometry} % marges etc.
\usepackage{comment}			% commentaires multilignes
\usepackage{amsmath,environ} % maths (matrices, etc.)
\usepackage{amssymb,makeidx}
\usepackage{bm}				% bold maths
\usepackage{tabularx}		% tableaux
\usepackage{colortbl}		% tableaux en couleur
\usepackage{fontawesome}		% Fontawesome
\usepackage{environ}			% environment with command
\usepackage{fp}				% calculs pour ps-tricks
\usepackage{multido}			% pour ps tricks
\usepackage[np]{numprint}	% formattage nombre
\usepackage{tikz,tkz-tab} 			% package principal TikZ
\usepackage{pgfplots}   % axes
\usepackage{mathrsfs}    % cursives
\usepackage{calc}			% calcul taille boites
\usepackage[scaled=0.875]{helvet} % font sans serif
\usepackage{svg} % svg
\usepackage{scrextend} % local margin
\usepackage{scratch} %scratch
\usepackage{multicol} % colonnes
%\usepackage{infix-RPN,pst-func} % formule en notation polanaise inversée
\usepackage{listings}

%================================================================================================================================
%
% Réglages de base
%
%================================================================================================================================

\lstset{
language=Python,   % R code
literate=
{á}{{\'a}}1
{à}{{\`a}}1
{ã}{{\~a}}1
{é}{{\'e}}1
{è}{{\`e}}1
{ê}{{\^e}}1
{í}{{\'i}}1
{ó}{{\'o}}1
{õ}{{\~o}}1
{ú}{{\'u}}1
{ü}{{\"u}}1
{ç}{{\c{c}}}1
{~}{{ }}1
}


\definecolor{codegreen}{rgb}{0,0.6,0}
\definecolor{codegray}{rgb}{0.5,0.5,0.5}
\definecolor{codepurple}{rgb}{0.58,0,0.82}
\definecolor{backcolour}{rgb}{0.95,0.95,0.92}

\lstdefinestyle{mystyle}{
    backgroundcolor=\color{backcolour},   
    commentstyle=\color{codegreen},
    keywordstyle=\color{magenta},
    numberstyle=\tiny\color{codegray},
    stringstyle=\color{codepurple},
    basicstyle=\ttfamily\footnotesize,
    breakatwhitespace=false,         
    breaklines=true,                 
    captionpos=b,                    
    keepspaces=true,                 
    numbers=left,                    
xleftmargin=2em,
framexleftmargin=2em,            
    showspaces=false,                
    showstringspaces=false,
    showtabs=false,                  
    tabsize=2,
    upquote=true
}

\lstset{style=mystyle}


\lstset{style=mystyle}
\newcommand{\imgdir}{C:/laragon/www/newmc/assets/imgsvg/}
\newcommand{\imgsvgdir}{C:/laragon/www/newmc/assets/imgsvg/}

\definecolor{mcgris}{RGB}{220, 220, 220}% ancien~; pour compatibilité
\definecolor{mcbleu}{RGB}{52, 152, 219}
\definecolor{mcvert}{RGB}{125, 194, 70}
\definecolor{mcmauve}{RGB}{154, 0, 215}
\definecolor{mcorange}{RGB}{255, 96, 0}
\definecolor{mcturquoise}{RGB}{0, 153, 153}
\definecolor{mcrouge}{RGB}{255, 0, 0}
\definecolor{mclightvert}{RGB}{205, 234, 190}

\definecolor{gris}{RGB}{220, 220, 220}
\definecolor{bleu}{RGB}{52, 152, 219}
\definecolor{vert}{RGB}{125, 194, 70}
\definecolor{mauve}{RGB}{154, 0, 215}
\definecolor{orange}{RGB}{255, 96, 0}
\definecolor{turquoise}{RGB}{0, 153, 153}
\definecolor{rouge}{RGB}{255, 0, 0}
\definecolor{lightvert}{RGB}{205, 234, 190}
\setitemize[0]{label=\color{lightvert}  $\bullet$}

\pagestyle{fancy}
\renewcommand{\headrulewidth}{0.2pt}
\fancyhead[L]{maths-cours.fr}
\fancyhead[R]{\thepage}
\renewcommand{\footrulewidth}{0.2pt}
\fancyfoot[C]{}

\newcolumntype{C}{>{\centering\arraybackslash}X}
\newcolumntype{s}{>{\hsize=.35\hsize\arraybackslash}X}

\setlength{\parindent}{0pt}		 
\setlength{\parskip}{3mm}
\setlength{\headheight}{1cm}

\def\ebook{ebook}
\def\book{book}
\def\web{web}
\def\type{web}

\newcommand{\vect}[1]{\overrightarrow{\,\mathstrut#1\,}}

\def\Oij{$\left(\text{O}~;~\vect{\imath},~\vect{\jmath}\right)$}
\def\Oijk{$\left(\text{O}~;~\vect{\imath},~\vect{\jmath},~\vect{k}\right)$}
\def\Ouv{$\left(\text{O}~;~\vect{u},~\vect{v}\right)$}

\hypersetup{breaklinks=true, colorlinks = true, linkcolor = OliveGreen, urlcolor = OliveGreen, citecolor = OliveGreen, pdfauthor={Didier BONNEL - https://www.maths-cours.fr} } % supprime les bordures autour des liens

\renewcommand{\arg}[0]{\text{arg}}

\everymath{\displaystyle}

%================================================================================================================================
%
% Macros - Commandes
%
%================================================================================================================================

\newcommand\meta[2]{    			% Utilisé pour créer le post HTML.
	\def\titre{titre}
	\def\url{url}
	\def\arg{#1}
	\ifx\titre\arg
		\newcommand\maintitle{#2}
		\fancyhead[L]{#2}
		{\Large\sffamily \MakeUppercase{#2}}
		\vspace{1mm}\textcolor{mcvert}{\hrule}
	\fi 
	\ifx\url\arg
		\fancyfoot[L]{\href{https://www.maths-cours.fr#2}{\black \footnotesize{https://www.maths-cours.fr#2}}}
	\fi 
}


\newcommand\TitreC[1]{    		% Titre centré
     \needspace{3\baselineskip}
     \begin{center}\textbf{#1}\end{center}
}

\newcommand\newpar{    		% paragraphe
     \par
}

\newcommand\nosp {    		% commande vide (pas d'espace)
}
\newcommand{\id}[1]{} %ignore

\newcommand\boite[2]{				% Boite simple sans titre
	\vspace{5mm}
	\setlength{\fboxrule}{0.2mm}
	\setlength{\fboxsep}{5mm}	
	\fcolorbox{#1}{#1!3}{\makebox[\linewidth-2\fboxrule-2\fboxsep]{
  		\begin{minipage}[t]{\linewidth-2\fboxrule-4\fboxsep}\setlength{\parskip}{3mm}
  			 #2
  		\end{minipage}
	}}
	\vspace{5mm}
}

\newcommand\CBox[4]{				% Boites
	\vspace{5mm}
	\setlength{\fboxrule}{0.2mm}
	\setlength{\fboxsep}{5mm}
	
	\fcolorbox{#1}{#1!3}{\makebox[\linewidth-2\fboxrule-2\fboxsep]{
		\begin{minipage}[t]{1cm}\setlength{\parskip}{3mm}
	  		\textcolor{#1}{\LARGE{#2}}    
 	 	\end{minipage}  
  		\begin{minipage}[t]{\linewidth-2\fboxrule-4\fboxsep}\setlength{\parskip}{3mm}
			\raisebox{1.2mm}{\normalsize\sffamily{\textcolor{#1}{#3}}}						
  			 #4
  		\end{minipage}
	}}
	\vspace{5mm}
}

\newcommand\cadre[3]{				% Boites convertible html
	\par
	\vspace{2mm}
	\setlength{\fboxrule}{0.1mm}
	\setlength{\fboxsep}{5mm}
	\fcolorbox{#1}{white}{\makebox[\linewidth-2\fboxrule-2\fboxsep]{
  		\begin{minipage}[t]{\linewidth-2\fboxrule-4\fboxsep}\setlength{\parskip}{3mm}
			\raisebox{-2.5mm}{\sffamily \small{\textcolor{#1}{\MakeUppercase{#2}}}}		
			\par		
  			 #3
 	 		\end{minipage}
	}}
		\vspace{2mm}
	\par
}

\newcommand\bloc[3]{				% Boites convertible html sans bordure
     \needspace{2\baselineskip}
     {\sffamily \small{\textcolor{#1}{\MakeUppercase{#2}}}}    
		\par		
  			 #3
		\par
}

\newcommand\CHelp[1]{
     \CBox{Plum}{\faInfoCircle}{À RETENIR}{#1}
}

\newcommand\CUp[1]{
     \CBox{NavyBlue}{\faThumbsOUp}{EN PRATIQUE}{#1}
}

\newcommand\CInfo[1]{
     \CBox{Sepia}{\faArrowCircleRight}{REMARQUE}{#1}
}

\newcommand\CRedac[1]{
     \CBox{PineGreen}{\faEdit}{BIEN R\'EDIGER}{#1}
}

\newcommand\CError[1]{
     \CBox{Red}{\faExclamationTriangle}{ATTENTION}{#1}
}

\newcommand\TitreExo[2]{
\needspace{4\baselineskip}
 {\sffamily\large EXERCICE #1\ (\emph{#2 points})}
\vspace{5mm}
}

\newcommand\img[2]{
          \includegraphics[width=#2\paperwidth]{\imgdir#1}
}

\newcommand\imgsvg[2]{
       \begin{center}   \includegraphics[width=#2\paperwidth]{\imgsvgdir#1} \end{center}
}


\newcommand\Lien[2]{
     \href{#1}{#2 \tiny \faExternalLink}
}
\newcommand\mcLien[2]{
     \href{https~://www.maths-cours.fr/#1}{#2 \tiny \faExternalLink}
}

\newcommand{\euro}{\eurologo{}}

%================================================================================================================================
%
% Macros - Environement
%
%================================================================================================================================

\newenvironment{tex}{ %
}
{%
}

\newenvironment{indente}{ %
	\setlength\parindent{10mm}
}

{
	\setlength\parindent{0mm}
}

\newenvironment{corrige}{%
     \needspace{3\baselineskip}
     \medskip
     \textbf{\textsc{Corrigé}}
     \medskip
}
{
}

\newenvironment{extern}{%
     \begin{center}
     }
     {
     \end{center}
}

\NewEnviron{code}{%
	\par
     \boite{gray}{\texttt{%
     \BODY
     }}
     \par
}

\newenvironment{vbloc}{% boite sans cadre empeche saut de page
     \begin{minipage}[t]{\linewidth}
     }
     {
     \end{minipage}
}
\NewEnviron{h2}{%
    \needspace{3\baselineskip}
    \vspace{0.6cm}
	\noindent \MakeUppercase{\sffamily \large \BODY}
	\vspace{1mm}\textcolor{mcgris}{\hrule}\vspace{0.4cm}
	\par
}{}

\NewEnviron{h3}{%
    \needspace{3\baselineskip}
	\vspace{5mm}
	\textsc{\BODY}
	\par
}

\NewEnviron{margeneg}{ %
\begin{addmargin}[-1cm]{0cm}
\BODY
\end{addmargin}
}

\NewEnviron{html}{%
}

\begin{document}
\meta{url}{/cours/suites-geometriques/}
\meta{pid}{331}
\meta{titre}{Suites arithmétiques et géométriques}
\meta{type}{cours}
\begin{h2}1. Suites arithmétiques\end{h2}
\cadre{bleu}{Définition}{%id="d10"
     On dit qu'une suite $\left(u_{n}\right)$ est une \textbf{suite arithmétique} s'il existe un nombre $r$ tel que, pour tout $n\in \mathbb{N}$ :
     \begin{center}$u_{n+1}=u_{n}+r$\end{center}
     Le réel $r$ s'appelle la \textbf{raison} de la suite arithmétique.
}
\bloc{cyan}{Remarque}{%id="r10"
     Pour démontrer qu'une suite $\left(u_{n}\right)$ est arithmétique, on pourra calculer la différence $u_{n+1}-u_{n}$.
     \par
     Si on constate que la différence est une constante $r$, on pourra affirmer que la suite est arithmétique de raison $r$.
}
\bloc{orange}{Exemple}{%id="e10"
     Soit la suite $\left(u_{n}\right)$ définie par $u_{n}=3n+5$.
     \par
     $u_{n+1}-u_{n}=3\left(n+1\right)+5-\left(3n+5\right)$\nosp$=3n+3+5-3n-5=3$
     \par
     La suite $\left(u_{n}\right)$ est une suite arithmétique de raison $r=3$
}
\cadre{vert}{Propriété}{%id="p20"
     Si la suite $\left(u_{n}\right)$ est arithmétique de raison $r$ alors pour tous entiers naturels $n$ et $k$ :
     \begin{center}$u_{n}=u_{k}+\left(n-k\right)\times r$\end{center}
     En particulier :
     \begin{center}$u_{n}=u_{0}+n\times r$\end{center}
}
\bloc{orange}{Exemple}{%id="e20"
     Soit $\left(u_{n}\right)$ la suite arithmétique de raison $2$ et de premier terme $u_{0}=5$.
     \par
     $u_{100}=5+2\times 100=205$
}
\cadre{vert}{Propriété}{%id="p30"
     Réciproquement, si $a$ et $b$ sont deux nombres réels et si la suite $\left(u_{n}\right)$ est définie par $u_{n}=a\times n+b$ alors cette suite est une suite arithmétique de raison $r=a$ et de premier terme $u_{0}=b$.
}
\bloc{cyan}{Démonstration}{%id="m30"
     $u_{n+1}-u_{n}=a\left(n+1\right)+b-\left(an+b\right)$\nosp$=an+a+b-an-b=a$
     \par
     et
     \par
     $u_{0}=a\times 0+b=b$
}
\cadre{vert}{Propriété}{%id="p40"
     La représentation graphique d'une suite arithmétique est formée de points alignés.
}
\bloc{cyan}{Remarque}{%id="r40"
     Cela se déduit immédiatement du fait que, pour tout $n \in \mathbb{N}$, $u_{n}=u_{0}+n\times r$ donc les points représentant la suite sont sur la droite d'équation $y=rx+u_{0}$
}
\bloc{orange}{Exemple}{%id="e40"
     \begin{center}
          \begin{extern}%width="300" alt="représentation graphique d'une suite"
               % -+-+-+ variables modifiables
               \resizebox{6cm}{!}{%
                    \def\xmin{-0.8}
                    \def\xmax{7.5}
                    \def\ymin{-0.8}
                    \def\ymax{4.8}
                    \def\xunit{1}  % unités en cm
                    \def\yunit{1}
                    \psset{xunit=\xunit,yunit=\yunit,algebraic=true}
                    \fontsize{15pt}{15pt}\selectfont
                    \begin{pspicture*}[linewidth=1pt](\xmin,\ymin)(\xmax,\ymax)
                         \psaxes[Dx=1,Dy=1,linewidth=0.75pt]{->}(0,0)(\xmin,\ymin)(\xmax,\ymax)
                         \rput[tr](-0.2,-0.3){$O$}
                         \multido{\n=0.0+1}{8}{
                              \FPeval{\suite}{1+n/2}
                              \psdots[linecolor=blue](\n,\suite)
                         }
                    \end{pspicture*}
               }
          \end{extern}
     \end{center}
     \begin{center}
          \textit{Suite arithmétique de premier terme $u_{0}=1$ et de raison $r=\frac{1}{2}$}
     \end{center}
}
\cadre{rouge}{Théorème}{%id="t50"
     Soit $\left(u_{n}\right)$ une suite arithmétique de raison $r$ :
     \begin{itemize}
          \item si $r > 0$ alors $\left(u_{n}\right)$ est strictement croissante
          \item si $r=0$ alors $\left(u_{n}\right)$ est constante
          \item si $r < 0$ alors $\left(u_{n}\right)$ est strictement décroissante.
     \end{itemize}
}
\bloc{cyan}{Démonstration}{%id="m50"
     Ce résultat découle immédiatement de $u_{n+1}-u_{n}=r$
}
\cadre{rouge}{Théorème (Somme des premiers entiers)}{%id="t60"
     Pour tout entier $n \in \mathbb{N}$ :
     \begin{center}$0+1+. . .+n=\frac{n\left(n+1\right)}{2}$\end{center}
}
\bloc{cyan}{Démonstration}{%id="m60"
     Une démonstration astucieuse consiste à réécrire la somme en inversant l'ordre des termes :
     \par
     $S = 0 + 1 + 2 + . . . + n $\textbf{(1)} \\
     $S = n + n-1 + n-2 + . . . + 0 $\textbf{(2)}
     \par
     Puis on additionne les lignes \textbf{(1)} et \textbf{(2)} termes à termes. Dans le membre de gauche on trouve que tous les termes sont égaux à $n$ ($0+n=n$ ; $1+n-1=n$ ; $2 + n-2=n$, etc.). Comme en tout il y a $n+1$ termes on trouve :
     \par
     $S+S = n + n + n + . . . + n$
     \par
     $2S = n\left(n+1\right)$
     \par
     $S = \frac{n\left(n+1\right)}{2}$
}
\bloc{orange}{Exemple}{%id="e60"
     Soit à calculer la somme $S_{100}=1+2+. . .+100$.
     \par
     $S_{100}=\frac{100\times 101}{2}=50\times 101=5050$
}
\begin{h2}2. Suites géométriques\end{h2}
\cadre{bleu}{Définition}{%id="d100"
     On dit qu'une suite $\left(u_{n}\right)$ est une \textbf{suite géométrique} s'il existe un nombre réel $q$ tel que, pour tout $n\in \mathbb{N}$ :
     \begin{center}$u_{n+1}=q \times u_{n}$\end{center}
     Le réel $q$ s'appelle la \textbf{raison} de la suite géométrique $\left(u_{n}\right)$.
}
\bloc{cyan}{Remarque}{%id="r100"
     Pour démontrer qu'une suite $\left(u_{n}\right)$ dont les termes sont non nuls est une suite géométrique, on pourra calculer le rapport $\frac{u_{n+1}}{u_{n}}$.
     \par
     Si ce rapport est une constante $q$, on pourra affirmer que la suite est une suite géométrique de raison $q$.
}
\bloc{orange}{Exemple}{%id="e100"
     Soit la suite $\left(u_{n}\right)_{n\in \mathbb{N}}$ définie par $u_{n}=\frac{3}{2^{n}}$.
     \par
     Les termes de la suite sont tous strictement positifs et
     \par
     $\frac{u_{n+1}}{u_{n}}=\frac{3}{2^{n+1}}$÷$\frac{3}{2^{n}}$\nosp$=\frac{3}{2^{n+1}}\times \frac{2^{n}}{3}$\nosp$=\frac{2^{n}}{2^{n+1}}$\nosp$=\frac{2^{n}}{2\times 2^{n}}=\frac{1}{2}$
     \par
     La suite $\left(u_{n}\right)$ est une suite géométrique de raison $\frac{1}{2}$
}
\cadre{vert}{Propriété}{%id="p110"
     Si la suite $\left(u_{n}\right)$ est géométrique de raison $q$, pour tous entiers naturels $n$ et $k$ :
     \begin{center}$u_{n}=u_{k}\times q^{n-k}$.\end{center}
     En particulier :
     \begin{center}$u_{n}=u_{0}\times q^{n}$.\end{center}
}
\cadre{vert}{Propriété}{%id="p120"
     Réciproquement, soient $a$ et $b$ deux nombres réels. La suite $\left(u_{n}\right)$ définie par $u_{n}=a\times b^{n}$ suite est une suite géométrique de raison $q=b$ et de premier terme $u_{0}=a$.
}
\bloc{cyan}{Démonstration}{%id="m120"
     $u_{n+1}=a\times b^{n+1}=a\times b^{n}\times b=u_{n}\times b$
     \par
     et
     \par
     $u_{0}=a\times b^{0}=a\times 1=a$
}
\cadre{rouge}{Théorème}{%id="t130"
     Soit $\left(u_{n}\right) $une suite géométrique de raison $q > 0$ et de premier terme strictement positif :
     \begin{itemize}
          \item Si q  > 1, la suite $\left(u_{n}\right) $ est strictement croissante
          \item Si 0 < q  < 1, la suite $\left(u_{n}\right) $ est strictement décroissante
          \item Si q=1, la suite $\left(u_{n}\right) $est constante
     \end{itemize}
}
\bloc{cyan}{Remarques}{%id="r130"
     \begin{itemize}
          \item Si le premier terme est strictement négatif, le sens de variation est inversé.
          \item Si la raison est strictement négative, la suite n'est ni croissante ni décroissante.
     \end{itemize}
}
\cadre{rouge}{Théorème}{%id="t140"
     Pour tout entier $n \in \mathbb{N}$ et tout réel $q\neq 1$
     \begin{center}$1+q+q^{2}+. . . +q^{n}=\frac{1-q^{n+1}}{1-q}$\end{center}
}
\bloc{cyan}{Remarque}{%id="r140"
     Cette formule n'est pas valable pour $q=1$. Mais dans ce cas tous les termes de la somme valent 1; la somme est donc égale au nombre de termes $n+1$
}
\bloc{cyan}{Démonstration}{%id="d140"
     On multiplie chaque membre par $q$. Cela incrémente chacun des exposants de $q$ :
     \par
     $S = 1 + q + q^{2} + . . . + q^{n} $\textbf{(1)} \\
     $qS = q + q^{2} + q^{3} + . . . + q^{n+1} $\textbf{(2)}
     \par
     On soustrait termes à termes les égalités \textbf{(1)} et \textbf{(2)}; tous les termes se simplifient sauf le premier et le dernier~:
     \par
     $S-qS = 1-q+q-q^{2}+q^{2}-q^{3}+ . . .$\nosp$ +q^{n}-q^{n+1} $
     \par
     $\left(1-q\right)S = 1-q^{n+1} $
     \par
     $S = \frac{1-q^{n+1}}{1-q}$
}
\bloc{orange}{Exemple}{%id="e140"
     Soit à calculer la somme $S=1+2+4+8+16. . .+2^{10}$
     \par
     $S=\frac{1-2^{10+1}}{1-2}=\frac{1-2048}{1-2}$\nosp$=\frac{-2047}{-1}=2047$
}

\end{document}
µ
\documentclass[a4paper]{article}

%================================================================================================================================
%
% Packages
%
%================================================================================================================================

\usepackage[T1]{fontenc} 	% pour caractères accentués
\usepackage[utf8]{inputenc}  % encodage utf8
\usepackage[french]{babel}	% langue : français
\usepackage{fourier}			% caractères plus lisibles
\usepackage[dvipsnames]{xcolor} % couleurs
\usepackage{fancyhdr}		% réglage header footer
\usepackage{needspace}		% empêcher sauts de page mal placés
\usepackage{graphicx}		% pour inclure des graphiques
\usepackage{enumitem,cprotect}		% personnalise les listes d'items (nécessaire pour ol, al ...)
\usepackage{hyperref}		% Liens hypertexte
\usepackage{pstricks,pst-all,pst-node,pstricks-add,pst-math,pst-plot,pst-tree,pst-eucl} % pstricks
\usepackage[a4paper,includeheadfoot,top=2cm,left=3cm, bottom=2cm,right=3cm]{geometry} % marges etc.
\usepackage{comment}			% commentaires multilignes
\usepackage{amsmath,environ} % maths (matrices, etc.)
\usepackage{amssymb,makeidx}
\usepackage{bm}				% bold maths
\usepackage{tabularx}		% tableaux
\usepackage{colortbl}		% tableaux en couleur
\usepackage{fontawesome}		% Fontawesome
\usepackage{environ}			% environment with command
\usepackage{fp}				% calculs pour ps-tricks
\usepackage{multido}			% pour ps tricks
\usepackage[np]{numprint}	% formattage nombre
\usepackage{tikz,tkz-tab} 			% package principal TikZ
\usepackage{pgfplots}   % axes
\usepackage{mathrsfs}    % cursives
\usepackage{calc}			% calcul taille boites
\usepackage[scaled=0.875]{helvet} % font sans serif
\usepackage{svg} % svg
\usepackage{scrextend} % local margin
\usepackage{scratch} %scratch
\usepackage{multicol} % colonnes
%\usepackage{infix-RPN,pst-func} % formule en notation polanaise inversée
\usepackage{listings}

%================================================================================================================================
%
% Réglages de base
%
%================================================================================================================================

\lstset{
language=Python,   % R code
literate=
{á}{{\'a}}1
{à}{{\`a}}1
{ã}{{\~a}}1
{é}{{\'e}}1
{è}{{\`e}}1
{ê}{{\^e}}1
{í}{{\'i}}1
{ó}{{\'o}}1
{õ}{{\~o}}1
{ú}{{\'u}}1
{ü}{{\"u}}1
{ç}{{\c{c}}}1
{~}{{ }}1
}


\definecolor{codegreen}{rgb}{0,0.6,0}
\definecolor{codegray}{rgb}{0.5,0.5,0.5}
\definecolor{codepurple}{rgb}{0.58,0,0.82}
\definecolor{backcolour}{rgb}{0.95,0.95,0.92}

\lstdefinestyle{mystyle}{
    backgroundcolor=\color{backcolour},   
    commentstyle=\color{codegreen},
    keywordstyle=\color{magenta},
    numberstyle=\tiny\color{codegray},
    stringstyle=\color{codepurple},
    basicstyle=\ttfamily\footnotesize,
    breakatwhitespace=false,         
    breaklines=true,                 
    captionpos=b,                    
    keepspaces=true,                 
    numbers=left,                    
xleftmargin=2em,
framexleftmargin=2em,            
    showspaces=false,                
    showstringspaces=false,
    showtabs=false,                  
    tabsize=2,
    upquote=true
}

\lstset{style=mystyle}


\lstset{style=mystyle}
\newcommand{\imgdir}{C:/laragon/www/newmc/assets/imgsvg/}
\newcommand{\imgsvgdir}{C:/laragon/www/newmc/assets/imgsvg/}

\definecolor{mcgris}{RGB}{220, 220, 220}% ancien~; pour compatibilité
\definecolor{mcbleu}{RGB}{52, 152, 219}
\definecolor{mcvert}{RGB}{125, 194, 70}
\definecolor{mcmauve}{RGB}{154, 0, 215}
\definecolor{mcorange}{RGB}{255, 96, 0}
\definecolor{mcturquoise}{RGB}{0, 153, 153}
\definecolor{mcrouge}{RGB}{255, 0, 0}
\definecolor{mclightvert}{RGB}{205, 234, 190}

\definecolor{gris}{RGB}{220, 220, 220}
\definecolor{bleu}{RGB}{52, 152, 219}
\definecolor{vert}{RGB}{125, 194, 70}
\definecolor{mauve}{RGB}{154, 0, 215}
\definecolor{orange}{RGB}{255, 96, 0}
\definecolor{turquoise}{RGB}{0, 153, 153}
\definecolor{rouge}{RGB}{255, 0, 0}
\definecolor{lightvert}{RGB}{205, 234, 190}
\setitemize[0]{label=\color{lightvert}  $\bullet$}

\pagestyle{fancy}
\renewcommand{\headrulewidth}{0.2pt}
\fancyhead[L]{maths-cours.fr}
\fancyhead[R]{\thepage}
\renewcommand{\footrulewidth}{0.2pt}
\fancyfoot[C]{}

\newcolumntype{C}{>{\centering\arraybackslash}X}
\newcolumntype{s}{>{\hsize=.35\hsize\arraybackslash}X}

\setlength{\parindent}{0pt}		 
\setlength{\parskip}{3mm}
\setlength{\headheight}{1cm}

\def\ebook{ebook}
\def\book{book}
\def\web{web}
\def\type{web}

\newcommand{\vect}[1]{\overrightarrow{\,\mathstrut#1\,}}

\def\Oij{$\left(\text{O}~;~\vect{\imath},~\vect{\jmath}\right)$}
\def\Oijk{$\left(\text{O}~;~\vect{\imath},~\vect{\jmath},~\vect{k}\right)$}
\def\Ouv{$\left(\text{O}~;~\vect{u},~\vect{v}\right)$}

\hypersetup{breaklinks=true, colorlinks = true, linkcolor = OliveGreen, urlcolor = OliveGreen, citecolor = OliveGreen, pdfauthor={Didier BONNEL - https://www.maths-cours.fr} } % supprime les bordures autour des liens

\renewcommand{\arg}[0]{\text{arg}}

\everymath{\displaystyle}

%================================================================================================================================
%
% Macros - Commandes
%
%================================================================================================================================

\newcommand\meta[2]{    			% Utilisé pour créer le post HTML.
	\def\titre{titre}
	\def\url{url}
	\def\arg{#1}
	\ifx\titre\arg
		\newcommand\maintitle{#2}
		\fancyhead[L]{#2}
		{\Large\sffamily \MakeUppercase{#2}}
		\vspace{1mm}\textcolor{mcvert}{\hrule}
	\fi 
	\ifx\url\arg
		\fancyfoot[L]{\href{https://www.maths-cours.fr#2}{\black \footnotesize{https://www.maths-cours.fr#2}}}
	\fi 
}


\newcommand\TitreC[1]{    		% Titre centré
     \needspace{3\baselineskip}
     \begin{center}\textbf{#1}\end{center}
}

\newcommand\newpar{    		% paragraphe
     \par
}

\newcommand\nosp {    		% commande vide (pas d'espace)
}
\newcommand{\id}[1]{} %ignore

\newcommand\boite[2]{				% Boite simple sans titre
	\vspace{5mm}
	\setlength{\fboxrule}{0.2mm}
	\setlength{\fboxsep}{5mm}	
	\fcolorbox{#1}{#1!3}{\makebox[\linewidth-2\fboxrule-2\fboxsep]{
  		\begin{minipage}[t]{\linewidth-2\fboxrule-4\fboxsep}\setlength{\parskip}{3mm}
  			 #2
  		\end{minipage}
	}}
	\vspace{5mm}
}

\newcommand\CBox[4]{				% Boites
	\vspace{5mm}
	\setlength{\fboxrule}{0.2mm}
	\setlength{\fboxsep}{5mm}
	
	\fcolorbox{#1}{#1!3}{\makebox[\linewidth-2\fboxrule-2\fboxsep]{
		\begin{minipage}[t]{1cm}\setlength{\parskip}{3mm}
	  		\textcolor{#1}{\LARGE{#2}}    
 	 	\end{minipage}  
  		\begin{minipage}[t]{\linewidth-2\fboxrule-4\fboxsep}\setlength{\parskip}{3mm}
			\raisebox{1.2mm}{\normalsize\sffamily{\textcolor{#1}{#3}}}						
  			 #4
  		\end{minipage}
	}}
	\vspace{5mm}
}

\newcommand\cadre[3]{				% Boites convertible html
	\par
	\vspace{2mm}
	\setlength{\fboxrule}{0.1mm}
	\setlength{\fboxsep}{5mm}
	\fcolorbox{#1}{white}{\makebox[\linewidth-2\fboxrule-2\fboxsep]{
  		\begin{minipage}[t]{\linewidth-2\fboxrule-4\fboxsep}\setlength{\parskip}{3mm}
			\raisebox{-2.5mm}{\sffamily \small{\textcolor{#1}{\MakeUppercase{#2}}}}		
			\par		
  			 #3
 	 		\end{minipage}
	}}
		\vspace{2mm}
	\par
}

\newcommand\bloc[3]{				% Boites convertible html sans bordure
     \needspace{2\baselineskip}
     {\sffamily \small{\textcolor{#1}{\MakeUppercase{#2}}}}    
		\par		
  			 #3
		\par
}

\newcommand\CHelp[1]{
     \CBox{Plum}{\faInfoCircle}{À RETENIR}{#1}
}

\newcommand\CUp[1]{
     \CBox{NavyBlue}{\faThumbsOUp}{EN PRATIQUE}{#1}
}

\newcommand\CInfo[1]{
     \CBox{Sepia}{\faArrowCircleRight}{REMARQUE}{#1}
}

\newcommand\CRedac[1]{
     \CBox{PineGreen}{\faEdit}{BIEN R\'EDIGER}{#1}
}

\newcommand\CError[1]{
     \CBox{Red}{\faExclamationTriangle}{ATTENTION}{#1}
}

\newcommand\TitreExo[2]{
\needspace{4\baselineskip}
 {\sffamily\large EXERCICE #1\ (\emph{#2 points})}
\vspace{5mm}
}

\newcommand\img[2]{
          \includegraphics[width=#2\paperwidth]{\imgdir#1}
}

\newcommand\imgsvg[2]{
       \begin{center}   \includegraphics[width=#2\paperwidth]{\imgsvgdir#1} \end{center}
}


\newcommand\Lien[2]{
     \href{#1}{#2 \tiny \faExternalLink}
}
\newcommand\mcLien[2]{
     \href{https~://www.maths-cours.fr/#1}{#2 \tiny \faExternalLink}
}

\newcommand{\euro}{\eurologo{}}

%================================================================================================================================
%
% Macros - Environement
%
%================================================================================================================================

\newenvironment{tex}{ %
}
{%
}

\newenvironment{indente}{ %
	\setlength\parindent{10mm}
}

{
	\setlength\parindent{0mm}
}

\newenvironment{corrige}{%
     \needspace{3\baselineskip}
     \medskip
     \textbf{\textsc{Corrigé}}
     \medskip
}
{
}

\newenvironment{extern}{%
     \begin{center}
     }
     {
     \end{center}
}

\NewEnviron{code}{%
	\par
     \boite{gray}{\texttt{%
     \BODY
     }}
     \par
}

\newenvironment{vbloc}{% boite sans cadre empeche saut de page
     \begin{minipage}[t]{\linewidth}
     }
     {
     \end{minipage}
}
\NewEnviron{h2}{%
    \needspace{3\baselineskip}
    \vspace{0.6cm}
	\noindent \MakeUppercase{\sffamily \large \BODY}
	\vspace{1mm}\textcolor{mcgris}{\hrule}\vspace{0.4cm}
	\par
}{}

\NewEnviron{h3}{%
    \needspace{3\baselineskip}
	\vspace{5mm}
	\textsc{\BODY}
	\par
}

\NewEnviron{margeneg}{ %
\begin{addmargin}[-1cm]{0cm}
\BODY
\end{addmargin}
}

\NewEnviron{html}{%
}

\begin{document}
\meta{url}{/cours/vecteurs-droites/}
\meta{pid}{338}
\meta{titre}{Vecteurs et droites}
\meta{type}{cours}
\begin{h2}1. Vecteurs et repère cartésien\end{h2}
\cadre{bleu}{Définition (Vecteurs colinéaires)}{% id="d10"
On dit que deux vecteurs non nuls $\vec{u}$ et $\vec{v}$ sont \textbf{colinéaires} s'il existe un réel $k$ tel que $\vec{v} = k\vec{u}$}
\begin{center}
     \begin{extern}%width="220" alt="Vecteurs colinéaires"
          \psset{xunit=0.5cm,yunit=0.5cm,algebraic=true,dimen=middle,linewidth=1pt}
          \begin{pspicture*}(-1,0.)(11,8)
               \psline[linecolor=blue]{->}(3,3)(6,4)
               \psline[linecolor=red]{->}(0,4)(9,7)
               \psline[linecolor=mcvert]{->}(10,3)(4,1)
               \rput[b](4.5,5.8){$\red 3\vec{u}$}
               \rput[b](4.5,3.8){$\blue \vec{u}$}
               \rput[b](7,2.3){$\color{mcvert}-2\vec{u}$}
          \end{pspicture*}
     \end{extern}
\end{center}
\begin{center}
     \textit{Vecteurs colinéaires}
\end{center}
\bloc{cyan}{Remarques}{% id="r10"
     \begin{itemize}
          \item Par convention, on considère que le vecteur nul est colinéaire est tout vecteur du plan
          \item Deux vecteurs colinéaires ont la même «direction»~;~ils ont le même sens si $k > 0$ et sont de sens contraire si $k < 0$.
     \end{itemize}
}
\cadre{bleu}{Définition}{% id="d20"
     On dit que le vecteur non nul $\vec{u}$ est un \textbf{vecteur directeur} de la droite $d$ si et seulement si il existe deux points $A$ et $B$ de $d$ tels que $\vec{u}=\overrightarrow{AB}$.
}
\begin{center}
     \begin{extern}%width="250" alt="vecteur directeur"
          \psset{xunit=0.5cm,yunit=0.5cm,algebraic=true,dimen=middle,dotstyle=o,dotsize=5pt 0,linewidth=1.2pt}
          \begin{pspicture*}(-4.,-1.84)(7.36,3.8)
               \psplot[linewidth=0.8pt]{-4.}{7.36}{(-1.--2.*x)/5.}
               \psline[linewidth=0.8pt,linecolor=red]{->}(-1.,1.)(4.,3.)
               \psline[linewidth=0.8pt,linecolor=red]{->}(-2.,-1.)(3.,1.)
               \rput[tl](0.9,3){$\red{\vec{u}}$}
               \rput[tl](5.9,3){$d$}
               \psdots[dotsize=2pt 0,dotstyle=*,linecolor=blue](-2.,-1.)
               \rput[bl](-2.5,-0.8){\blue{$A$}}
               \psdots[dotsize=2pt 0,dotstyle=*,linecolor=blue](3.,1.)
               \rput[bl](2.6,1.3){\blue{$B$}}
          \end{pspicture*}
     \end{extern}
\end{center}
\begin{center}
     \textit{Vecteur directeur}
\end{center}
\cadre{vert}{Propriété}{% id="p25"
     Trois points distincts $A, B$ et $C$ sont alignés si et seulement si les vecteurs $\overrightarrow{AB}$ et $\overrightarrow{AC}$ sont colinéaires.
}
\cadre{vert}{Propriété}{% id="p30"
     Deux droites sont parallèles si et seulement si elles ont des vecteurs directeurs colinéaires.
}
\cadre{rouge}{Théorème et définitions}{% id="t40"
     Soient $O$ un point et $\vec{i}$ et $\vec{j}$ deux vecteurs \textbf{non colinéaires} du plan.
     \par
     Le triplet $\left(O ; \vec{i}, \vec{j}\right)$ s'appelle un \textbf{repère cartésien} du plan.
     \begin{itemize}
          \item Pour tout point $M$ du plan, il existe deux réels $x$ et $y$ tels que :
          \begin{center}$\overrightarrow{OM}=x\vec{i}+y\vec{j}$\end{center}
          \item Pour tout vecteur $\vec{u}$ du plan, il existe deux réels $x$ et $y$ tels que :
          \begin{center}$\vec{u}=x\vec{i}+y\vec{j}$\end{center}
     \end{itemize}
     Le couple $\left(x ; y\right)$ s'appelle le couple de \textbf{coordonnées} du point $M$ (ou du vecteur $\vec{u}$) dans le repère $\left(O ; \vec{i}, \vec{j}\right)$
}
\begin{center}
     \begin{extern}%width="330" alt="Coordonnées dans un repère cartésien"
          \psset{xunit=1.0cm,yunit=1.0cm,algebraic=true,dimen=middle,linewidth=1pt}
          \begin{pspicture*}(-2.5,0.)(6,5)
               \psplot[linewidth=0.8pt]{-2.5}{6}{3.+2.*x}
               \psplot[linewidth=0.8pt]{-2.5}{6}{1-0.*x}
               \psline[linewidth=0.8pt,linecolor=red]{->}(-1.,1.)(5.,4.)
               \psplot[linewidth=0.8pt,linecolor=lightgray]{-2.5}{6}{(-6.--2.*x)/1.}
               \psplot[linewidth=0.8pt,linecolor=lightgray]{-2.5}{6}{(--12.-0.*x)/3.}
               \psline[linewidth=0.8pt,linecolor=blue]{->}(-1.,1.)(1.,1.)
               \psline[linewidth=0.8pt,linecolor=blue]{->}(-1.,1.)(-0.5,2)
               \rput[tl](0,0.8){$\blue{\vec{i}}$}
               \rput[tl](-1.3,1.8){$\blue{\vec{j}}$}
               \rput[tl](3.5,0.8){$x$}
               \rput[tl](0.3,4.4){$y$}
               \rput[tl](1.8,3){$\red{\vec{u}}$}
               \psdots[dotsize=1pt 0,dotstyle=*,linecolor=blue](-1.,1.)
               \rput[bl](-1.6,0.6){\blue{$O$}}
               \psdots[dotsize=1pt 0,dotstyle=*,linecolor=blue](5.,4.)
               \rput[bl](4.7,4.2){\blue{$M$}}
          \end{pspicture*}
     \end{extern}
\end{center}
\begin{center}
     \textit{Coordonnées dans un repère cartésien}
\end{center}
\bloc{cyan}{Remarque}{% id="r40"
     Dans ce chapitre, les repères utilisés ne seront pas nécessairement orthonormés.
     \par
     L'étude spécifique des repères orthonormés sera détaillée dans le chapitre «produit scalaire»
}
\cadre{vert}{Propriétés}{% id="p50"
     On se place dans un repère $\left(O ; \vec{i}, \vec{j}\right)$.
     \par
     Soient deux points $A\left(x_{A} ; y_{A}\right)$ et $B\left(x_{B} ; y_{B}\right)$, alors :
     \begin{itemize}
          \item Le vecteur $\overrightarrow{AB}$ a pour coordonnées $\left(x_{B}-x_{A} ; y_{B}-y_{A}\right)$
          \item Le milieu $M$ de $\left[AB\right]$ a pour coordonnées $M \left(\frac{x_{A}+x_{B}}{2} ; \frac{y_{A}+y_{B}}{2}\right)$
     \end{itemize}
}
\cadre{rouge}{Théorème}{% id="t50"
     Soient $\vec{u}$ et $\vec{v}$ deux vecteurs de coordonnées respectives $\left(x ; y\right)$ et $\left(x^{\prime} ; y^{\prime}\right)$ dans un repère $\left(O ; \vec{i}, \vec{j}\right)$. Les vecteurs $\vec{u}$ et $\vec{v}$ sont colinéaires si et seulement si leurs coordonnées sont proportionnelles, c'est à dire si et seulement si :
     \begin{center}$xy^{\prime}-x^{\prime}y=0$\end{center}
}
\begin{h2}2. Équations de droites\end{h2}
Dans cette partie, on se place dans un repère $\left(O ; \vec{i}, \vec{j}\right)$ (non nécessairement orthonormé).
\cadre{rouge}{Théorème}{% id="t60"
     Soit $d$ une droite passant par un point $A$ et de vecteur directeur $\vec{u}$.
     \par
     Un point $M$ appartient à la droite $d$ si et seulement si les vecteurs $\overrightarrow{AM}$ et $\vec{u}$ sont colinéaires.
}
\bloc{orange}{Exemple}{% id="e60"
     Soient le point $A\left(0;1\right)$ et le vecteur $\vec{u}\left(1;-1\right)$. Le point $M\left(x ; y\right)$ appartient à la droite passant par $A$ et de vecteur directeur $\vec{u}$ si et seulement si $\overrightarrow{AM}$ et $\vec{u}$ sont colinéaires. Or les coordonnées de $\overrightarrow{AM}$ sont $\left(x ; y-1\right)$ donc :
     \par
     $M \in  d  \Leftrightarrow x\times \left(-1\right)-\left(y-1\right)\times 1=0 \Leftrightarrow -x-y+1=0$
     \par
     Cette dernière égalité s'appelle une équation cartésienne de la droite $d$.
}
\cadre{rouge}{Théorème}{% id="t70"
     Toute droite du plan possède une \textbf{équation cartésienne} du type :
     \[  ax+by+c=0 \]
     où $a, b$ et $c$ sont trois réels.
     \par
     \textbf{Réciproquement, }l'ensemble des points $M\left(x ; y\right)$ tels que $ax+by+c=0$ où $a, b$ et $c$ sont trois réels avec $a\neq 0$ ou $b\neq 0$ est une droite.
}
\bloc{cyan}{Remarques}{% id="r70"
     \begin{itemize}
          \item Une droite possède une infinité d'équation cartésienne (il suffit de multiplier une équation par un facteur non nul pour obtenir une équation équivalente).
          \item Si $b\neq 0$ l'équation peut s'écrire :
          \par
          $ax+by+c= 0 \Leftrightarrow by=-ax-c \Leftrightarrow y=-\frac{a}{b}x-\frac{c}{b}$
          \par
          qui est de la forme $y=mx+p$ (en posant $m=-\frac{a}{b}$ et $p=-\frac{c}{b}$).
          \par
          Cette forme est appelée \textbf{équation réduite} de la droite.
          \par
          Ce cas correspond à une droite qui n'est pas parallèle. à l'axe des ordonnées.
          \item Si $b=0$ et $a\neq 0$ l'équation peut s'écrire :
          \par
          $ax+c= 0 \Leftrightarrow ax=-c \Leftrightarrow x=-\frac{c}{a}$
          \par
          qui est du type $x=k$ (en posant $k=-\frac{c}{a}$)
          \par
          Ce cas correspond à une droite qui est parallèle. à l'axe des ordonnées.
     \end{itemize}
}
\cadre{vert}{Propriété}{% id="p80"
     Soit $d$ une droite d'équation $ax+by+c=0$.
     \par
     Le vecteur $\vec{u}$ de coordonnées $\left(-b ; a\right)$ est un vecteur directeur de la droite $d$.
}
\bloc{cyan}{Démonstration}{% id="r80"
     Voir exercice~: \mcLien{/exercices/geometrie-plan/equation-cartesienne-vecteur-directeur}{«~Equation cartésienne - Vecteur directeur~»}.
}

\end{document}
µ
\documentclass[a4paper]{article}

%================================================================================================================================
%
% Packages
%
%================================================================================================================================

\usepackage[T1]{fontenc} 	% pour caractères accentués
\usepackage[utf8]{inputenc}  % encodage utf8
\usepackage[french]{babel}	% langue : français
\usepackage{fourier}			% caractères plus lisibles
\usepackage[dvipsnames]{xcolor} % couleurs
\usepackage{fancyhdr}		% réglage header footer
\usepackage{needspace}		% empêcher sauts de page mal placés
\usepackage{graphicx}		% pour inclure des graphiques
\usepackage{enumitem,cprotect}		% personnalise les listes d'items (nécessaire pour ol, al ...)
\usepackage{hyperref}		% Liens hypertexte
\usepackage{pstricks,pst-all,pst-node,pstricks-add,pst-math,pst-plot,pst-tree,pst-eucl} % pstricks
\usepackage[a4paper,includeheadfoot,top=2cm,left=3cm, bottom=2cm,right=3cm]{geometry} % marges etc.
\usepackage{comment}			% commentaires multilignes
\usepackage{amsmath,environ} % maths (matrices, etc.)
\usepackage{amssymb,makeidx}
\usepackage{bm}				% bold maths
\usepackage{tabularx}		% tableaux
\usepackage{colortbl}		% tableaux en couleur
\usepackage{fontawesome}		% Fontawesome
\usepackage{environ}			% environment with command
\usepackage{fp}				% calculs pour ps-tricks
\usepackage{multido}			% pour ps tricks
\usepackage[np]{numprint}	% formattage nombre
\usepackage{tikz,tkz-tab} 			% package principal TikZ
\usepackage{pgfplots}   % axes
\usepackage{mathrsfs}    % cursives
\usepackage{calc}			% calcul taille boites
\usepackage[scaled=0.875]{helvet} % font sans serif
\usepackage{svg} % svg
\usepackage{scrextend} % local margin
\usepackage{scratch} %scratch
\usepackage{multicol} % colonnes
%\usepackage{infix-RPN,pst-func} % formule en notation polanaise inversée
\usepackage{listings}

%================================================================================================================================
%
% Réglages de base
%
%================================================================================================================================

\lstset{
language=Python,   % R code
literate=
{á}{{\'a}}1
{à}{{\`a}}1
{ã}{{\~a}}1
{é}{{\'e}}1
{è}{{\`e}}1
{ê}{{\^e}}1
{í}{{\'i}}1
{ó}{{\'o}}1
{õ}{{\~o}}1
{ú}{{\'u}}1
{ü}{{\"u}}1
{ç}{{\c{c}}}1
{~}{{ }}1
}


\definecolor{codegreen}{rgb}{0,0.6,0}
\definecolor{codegray}{rgb}{0.5,0.5,0.5}
\definecolor{codepurple}{rgb}{0.58,0,0.82}
\definecolor{backcolour}{rgb}{0.95,0.95,0.92}

\lstdefinestyle{mystyle}{
    backgroundcolor=\color{backcolour},   
    commentstyle=\color{codegreen},
    keywordstyle=\color{magenta},
    numberstyle=\tiny\color{codegray},
    stringstyle=\color{codepurple},
    basicstyle=\ttfamily\footnotesize,
    breakatwhitespace=false,         
    breaklines=true,                 
    captionpos=b,                    
    keepspaces=true,                 
    numbers=left,                    
xleftmargin=2em,
framexleftmargin=2em,            
    showspaces=false,                
    showstringspaces=false,
    showtabs=false,                  
    tabsize=2,
    upquote=true
}

\lstset{style=mystyle}


\lstset{style=mystyle}
\newcommand{\imgdir}{C:/laragon/www/newmc/assets/imgsvg/}
\newcommand{\imgsvgdir}{C:/laragon/www/newmc/assets/imgsvg/}

\definecolor{mcgris}{RGB}{220, 220, 220}% ancien~; pour compatibilité
\definecolor{mcbleu}{RGB}{52, 152, 219}
\definecolor{mcvert}{RGB}{125, 194, 70}
\definecolor{mcmauve}{RGB}{154, 0, 215}
\definecolor{mcorange}{RGB}{255, 96, 0}
\definecolor{mcturquoise}{RGB}{0, 153, 153}
\definecolor{mcrouge}{RGB}{255, 0, 0}
\definecolor{mclightvert}{RGB}{205, 234, 190}

\definecolor{gris}{RGB}{220, 220, 220}
\definecolor{bleu}{RGB}{52, 152, 219}
\definecolor{vert}{RGB}{125, 194, 70}
\definecolor{mauve}{RGB}{154, 0, 215}
\definecolor{orange}{RGB}{255, 96, 0}
\definecolor{turquoise}{RGB}{0, 153, 153}
\definecolor{rouge}{RGB}{255, 0, 0}
\definecolor{lightvert}{RGB}{205, 234, 190}
\setitemize[0]{label=\color{lightvert}  $\bullet$}

\pagestyle{fancy}
\renewcommand{\headrulewidth}{0.2pt}
\fancyhead[L]{maths-cours.fr}
\fancyhead[R]{\thepage}
\renewcommand{\footrulewidth}{0.2pt}
\fancyfoot[C]{}

\newcolumntype{C}{>{\centering\arraybackslash}X}
\newcolumntype{s}{>{\hsize=.35\hsize\arraybackslash}X}

\setlength{\parindent}{0pt}		 
\setlength{\parskip}{3mm}
\setlength{\headheight}{1cm}

\def\ebook{ebook}
\def\book{book}
\def\web{web}
\def\type{web}

\newcommand{\vect}[1]{\overrightarrow{\,\mathstrut#1\,}}

\def\Oij{$\left(\text{O}~;~\vect{\imath},~\vect{\jmath}\right)$}
\def\Oijk{$\left(\text{O}~;~\vect{\imath},~\vect{\jmath},~\vect{k}\right)$}
\def\Ouv{$\left(\text{O}~;~\vect{u},~\vect{v}\right)$}

\hypersetup{breaklinks=true, colorlinks = true, linkcolor = OliveGreen, urlcolor = OliveGreen, citecolor = OliveGreen, pdfauthor={Didier BONNEL - https://www.maths-cours.fr} } % supprime les bordures autour des liens

\renewcommand{\arg}[0]{\text{arg}}

\everymath{\displaystyle}

%================================================================================================================================
%
% Macros - Commandes
%
%================================================================================================================================

\newcommand\meta[2]{    			% Utilisé pour créer le post HTML.
	\def\titre{titre}
	\def\url{url}
	\def\arg{#1}
	\ifx\titre\arg
		\newcommand\maintitle{#2}
		\fancyhead[L]{#2}
		{\Large\sffamily \MakeUppercase{#2}}
		\vspace{1mm}\textcolor{mcvert}{\hrule}
	\fi 
	\ifx\url\arg
		\fancyfoot[L]{\href{https://www.maths-cours.fr#2}{\black \footnotesize{https://www.maths-cours.fr#2}}}
	\fi 
}


\newcommand\TitreC[1]{    		% Titre centré
     \needspace{3\baselineskip}
     \begin{center}\textbf{#1}\end{center}
}

\newcommand\newpar{    		% paragraphe
     \par
}

\newcommand\nosp {    		% commande vide (pas d'espace)
}
\newcommand{\id}[1]{} %ignore

\newcommand\boite[2]{				% Boite simple sans titre
	\vspace{5mm}
	\setlength{\fboxrule}{0.2mm}
	\setlength{\fboxsep}{5mm}	
	\fcolorbox{#1}{#1!3}{\makebox[\linewidth-2\fboxrule-2\fboxsep]{
  		\begin{minipage}[t]{\linewidth-2\fboxrule-4\fboxsep}\setlength{\parskip}{3mm}
  			 #2
  		\end{minipage}
	}}
	\vspace{5mm}
}

\newcommand\CBox[4]{				% Boites
	\vspace{5mm}
	\setlength{\fboxrule}{0.2mm}
	\setlength{\fboxsep}{5mm}
	
	\fcolorbox{#1}{#1!3}{\makebox[\linewidth-2\fboxrule-2\fboxsep]{
		\begin{minipage}[t]{1cm}\setlength{\parskip}{3mm}
	  		\textcolor{#1}{\LARGE{#2}}    
 	 	\end{minipage}  
  		\begin{minipage}[t]{\linewidth-2\fboxrule-4\fboxsep}\setlength{\parskip}{3mm}
			\raisebox{1.2mm}{\normalsize\sffamily{\textcolor{#1}{#3}}}						
  			 #4
  		\end{minipage}
	}}
	\vspace{5mm}
}

\newcommand\cadre[3]{				% Boites convertible html
	\par
	\vspace{2mm}
	\setlength{\fboxrule}{0.1mm}
	\setlength{\fboxsep}{5mm}
	\fcolorbox{#1}{white}{\makebox[\linewidth-2\fboxrule-2\fboxsep]{
  		\begin{minipage}[t]{\linewidth-2\fboxrule-4\fboxsep}\setlength{\parskip}{3mm}
			\raisebox{-2.5mm}{\sffamily \small{\textcolor{#1}{\MakeUppercase{#2}}}}		
			\par		
  			 #3
 	 		\end{minipage}
	}}
		\vspace{2mm}
	\par
}

\newcommand\bloc[3]{				% Boites convertible html sans bordure
     \needspace{2\baselineskip}
     {\sffamily \small{\textcolor{#1}{\MakeUppercase{#2}}}}    
		\par		
  			 #3
		\par
}

\newcommand\CHelp[1]{
     \CBox{Plum}{\faInfoCircle}{À RETENIR}{#1}
}

\newcommand\CUp[1]{
     \CBox{NavyBlue}{\faThumbsOUp}{EN PRATIQUE}{#1}
}

\newcommand\CInfo[1]{
     \CBox{Sepia}{\faArrowCircleRight}{REMARQUE}{#1}
}

\newcommand\CRedac[1]{
     \CBox{PineGreen}{\faEdit}{BIEN R\'EDIGER}{#1}
}

\newcommand\CError[1]{
     \CBox{Red}{\faExclamationTriangle}{ATTENTION}{#1}
}

\newcommand\TitreExo[2]{
\needspace{4\baselineskip}
 {\sffamily\large EXERCICE #1\ (\emph{#2 points})}
\vspace{5mm}
}

\newcommand\img[2]{
          \includegraphics[width=#2\paperwidth]{\imgdir#1}
}

\newcommand\imgsvg[2]{
       \begin{center}   \includegraphics[width=#2\paperwidth]{\imgsvgdir#1} \end{center}
}


\newcommand\Lien[2]{
     \href{#1}{#2 \tiny \faExternalLink}
}
\newcommand\mcLien[2]{
     \href{https~://www.maths-cours.fr/#1}{#2 \tiny \faExternalLink}
}

\newcommand{\euro}{\eurologo{}}

%================================================================================================================================
%
% Macros - Environement
%
%================================================================================================================================

\newenvironment{tex}{ %
}
{%
}

\newenvironment{indente}{ %
	\setlength\parindent{10mm}
}

{
	\setlength\parindent{0mm}
}

\newenvironment{corrige}{%
     \needspace{3\baselineskip}
     \medskip
     \textbf{\textsc{Corrigé}}
     \medskip
}
{
}

\newenvironment{extern}{%
     \begin{center}
     }
     {
     \end{center}
}

\NewEnviron{code}{%
	\par
     \boite{gray}{\texttt{%
     \BODY
     }}
     \par
}

\newenvironment{vbloc}{% boite sans cadre empeche saut de page
     \begin{minipage}[t]{\linewidth}
     }
     {
     \end{minipage}
}
\NewEnviron{h2}{%
    \needspace{3\baselineskip}
    \vspace{0.6cm}
	\noindent \MakeUppercase{\sffamily \large \BODY}
	\vspace{1mm}\textcolor{mcgris}{\hrule}\vspace{0.4cm}
	\par
}{}

\NewEnviron{h3}{%
    \needspace{3\baselineskip}
	\vspace{5mm}
	\textsc{\BODY}
	\par
}

\NewEnviron{margeneg}{ %
\begin{addmargin}[-1cm]{0cm}
\BODY
\end{addmargin}
}

\NewEnviron{html}{%
}

\begin{document}
\meta{url}{/cours/sinus-cosinus/}
\meta{pid}{341}
\meta{titre}{Trigonométrie}
\meta{type}{cours}
\begin{h2}1. Mesures en radians d'un angle orienté  \end{h2}
Dans tout le chapitre, le plan $\mathscr P$ est muni d'un repère orthonormé $\left(O~; \vec{i} , \vec{j}\right)$.
\cadre{bleu}{Définition}{% id="d10"
     Soit $I$ le point de coordonnées $\left(1~; 0\right)$ et $d$ la droite parallèle à l'axe des ordonnées passant par $I$.
     \par
     A tout réel $x$ on associe le point $N$ de la droite $d$ d'ordonnée $x$ puis le point $M$ obtenu en «~enroulant~» la droite $d$ sur le \mcLien{/cours/seconde/trigonometrie\#d10}{cercle trigonométrique} (voir figure ci-dessous).
     \par
     On dit que $x$ est une \textbf{mesure en radians} de l'angle orienté $\left(\overrightarrow{OI}, \overrightarrow{OM}\right)$
}
\begin{center}
     \begin{extern}%width="400" alt="Mesure en radians d'un angle orienté"
          \resizebox{8cm}{!}{
               \newrgbcolor{dblue}{0. 0. 0.7}
               \newrgbcolor{dvert}{0. 0.4 0.}
               \newrgbcolor{dmauve}{0.5 0. 0.5}
               \psset{xunit=5.0cm,yunit=5.0cm,algebraic=true,dimen=middle,dotstyle=o,dotsize=5pt 0,linewidth=0.8pt,arrowsize=3pt 2,arrowinset=0.25}
               \begin{pspicture*}(-1.2,-1.2)(1.2,2.2)
                    \psaxes[linewidth=0.75pt,labelFontSize=\scriptstyle,xAxis=true,yAxis=true,Dx=10.,Dy=10.,ticksize=-2pt 0,subticks=1]{->}(0,0)(-1.2,-1.2)(1.2,2.2)
                    \pscircle[linewidth=0.8pt](0.,0.){5.} %cercle trigo
                    %             \pscircle[linewidth=0.8pt](1.373,0.){9.779} %cercle trigo
                    \parametricplot[linewidth=1.2pt,linecolor=red]{0.0}{1.92}{cos(t)|sin(t)}%arc angle
                    \parametricplot[linewidth=0.8pt,arrows=->]{0.8}{1.3}{1.15*cos(t)|1.15*sin(t)}% sens trigo
                    \rput[tl](0.58,1.07){+}
                    \pscustom[linewidth=0.8pt,linecolor=dmauve,fillcolor=dmauve,fillstyle=solid,opacity=0.1]{ % color angle
                         \parametricplot{0.0}{1.92}{0.15*cos(t)|0.15*sin(t)}
                    \lineto(0.,0.)\closepath}
                    \psellipticarc[linewidth=0.8pt,linecolor=dmauve,arrows=->](0.,0.)(0.15,0.15){0.}{110} % fleche angle
                    \psellipticarc[linewidth=0.8pt,linecolor=dblue,arrows=->](1.373,0.)(1.956,1.956) {101.3}{151} % fleche mn
                    \psline[linewidth=0.8pt,linecolor=dmauve]{->}(0.,0.)(-0.342,0.94)%rayon
                    \psline[linewidth=0.8pt]{->}(0.,0.)(1.,0.) %vecteurs unités
                    \psline[linewidth=0.8pt]{->}(0.,0.)(0,1)
                    \psline[linewidth=0.8pt]{->}(1,-1.5)(1,2.5)
                    %\rput[tl](0.4,0.1){$\vec{i}$}
                    %\rput[tl](-0.06,0.5){$\vec{j}$}
                    \psdots[dotsize=2pt 0,dotstyle=*](0.,0.)
                    \rput[bl](-0.09,-0.09){$O$}
                    \psdots[dotsize=2pt 0,dotstyle=*,linecolor=dblue](1.,0.)
                    \rput[bl](1.02,0.02){\dblue{$I$}}
                    \psdots[dotsize=2pt 0,dotstyle=*,linecolor=dblue](-0.342,0.94)
                    \rput[br](-0.342,0.96){\dblue{$M$}}
                    \psdots[dotsize=2pt 0,dotstyle=*,linecolor=dblue](1.0,1.92)
                    \rput[bl](1.03,1.92){\dblue{$N$}}\rput[l](1.03,0.86){\red{$x$}} \rput[bl](1.03,2.12){$d$}
                    \rput[t](0.5,-0.04){$\overrightarrow{i}$}\rput[l](-0.1,0.5){$\overrightarrow{j}$}
                    \psline[linewidth=1pt,linecolor=red](1.,0.)(1.,1.92)
                    \psdots[dotsize=2pt 0,dotstyle=*,linecolor=dblue](0,1)
                    \rput[bl](0.02,1.03){\dblue{$J$}}
               \end{pspicture*}
          }
     \end{extern}
\end{center}
\begin{center}
     \begin{center}
          \textit{Mesures d'un angle orienté}
     \end{center}
\end{center}
\bloc{cyan}{Remarque}{% id="r10"
     \begin{itemize}
          \item Une infinités de points de la droite $d$ se superposent à $M$ par enroulement (en faisant plusieurs tours). Chaque angle possède une infinité de mesures qui diffèrent entre elles d'un multiple de $2\pi $. Si $x$ est une mesure d'un angle, les autres mesures sont $x+2\pi  , x+4\pi  ,$  etc. et $x-2\pi  , x-4\pi $ ,  etc.
          \par
          Ces différentes mesures s'écrivent $x+2k\pi $ avec $k \in  \mathbb{Z}$
          \item On note de la même façon $\left(\vec{u}, \vec{v}\right)$ l'angle orienté de $\vec{u}$ vers $\vec{v}$et la mesure en radians de cet angle.
     \end{itemize}
}
\cadre{bleu}{Propriété et définition (Mesure principale)}{% id="d20"
     Tout angle orienté $\left(\vec{u}, \vec{v}\right)$ possède une unique mesure dans l'intervalle $\left]-\pi ~; \pi \right]$.
     \par
     Cette mesure s'appelle \textbf{la mesure principale }de l'angle $\left(\vec{u}, \vec{v}\right)$.
}
\bloc{orange}{Exemple}{% id="e20"
     Soit un angle dont une mesure est $-\frac{5\pi }{2}$. Comme $-\frac{5\pi }{2} \notin  \left]-\pi ~; \pi \right]$, ce n'est pas la mesure principale. Comme~: $-\frac{5\pi }{2} = -\frac{\pi }{2}-\frac{4\pi }{2} = -\frac{\pi }{2}-2\pi $ et $-\frac{\pi }{2}\in  \left]-\pi ~; \pi \right]$, $-\frac{\pi }{2}$ est la mesure principale de cet angle.
}
\bloc{orange}{Mesures d'angles à connaitre}{% id="e30"
     \begin{center}
          \begin{extern}%width="400" alt="Mesures d'angles remarquables"
               \newrgbcolor{dblue}{0. 0. 0.7}
               \newrgbcolor{dvert}{0. 0.4 0.}
               \newrgbcolor{dmauve}{0.5 0. 0.5}
               \psset{xunit=5.0cm,yunit=5.0cm,algebraic=true,dimen=middle,dotstyle=o,dotsize=5pt 0,linewidth=0.8pt,arrowsize=3pt 2,arrowinset=0.25}
               \resizebox{8cm}{!}{
                    \begin{pspicture*}(-1.2,-1.2)(1.2,1.2)
                         \psaxes[linewidth=0.75pt,labelFontSize=\scriptstyle,xAxis=true,yAxis=true,Dx=10.,Dy=10.,ticksize=-2pt 0,subticks=1]{->}(0,0)(-1.2,-1.2)(1.2,1.2)
                         \pscircle[linewidth=0.8pt](0.,0.){5.} %cercle trigo
                         \psline[linewidth=0.8pt]{->}(0.,0.)(1.,0.) %vecteurs unités
                         \psline[linewidth=0.8pt]{->}(0.,0.)(0,1)
                         %\rput[tl](0.4,0.1){$\vec{i}$}
                         %\rput[tl](-0.06,0.5){$\vec{j}$}
                         \psdots[dotsize=2pt 0,dotstyle=*](0.,0.)
                         %\rput[bl](-0.09,-0.09){$O$}
                         \psline[linewidth=0.8pt,linecolor=dvert](-0.707,-0.707)(0.707,0.707)
                         \psline[linewidth=0.8pt,linecolor=dvert](-0.707,0.707)(0.707,-0.707)
                         \psline[linewidth=0.8pt,linecolor=red](-0.866,-0.5)(0.866,0.5)
                         \psline[linewidth=0.8pt,linecolor=red](0.866,-0.5)(-0.866,0.5)
                         \rput(0.943,0.55){$\red{\dfrac{\pi}{6}}$}
                         \rput(-0.943,0.55){$\red{\dfrac{5\pi}{6}}$}
                         \rput(-0.973,-0.55){$\red{-\dfrac{5\pi}{6}}$}
                         \rput(0.943,-0.55){$\red{-\dfrac{\pi}{6}}$}
                         %
                         \rput(0.777,0.777){$\dvert{\dfrac{\pi}{4}}$}
                         \rput(-0.777,0.777){$\dvert{\dfrac{3\pi}{4}}$}
                         \rput(-0.807,-0.777){$\dvert{-\dfrac{3\pi}{4}}$}
                         \rput(0.777,-0.777){$\dvert{-\dfrac{\pi}{4}}$}
                         %
                         \psline[linewidth=0.8pt,linecolor=dblue](-0.5,-0.866)(0.5,0.866)
                         \psline[linewidth=0.8pt,linecolor=dblue](-0.5,0.866)(0.5,-0.866)
                         \rput(0.55,0.943){$\dblue{\dfrac{\pi}{3}}$}
                         \rput(-0.55,0.943){$\dblue{\dfrac{2\pi}{3}}$}
                         \rput(-0.58,-0.943){$\dblue{-\dfrac{2\pi}{3}}$}
                         \rput(0.55,-0.943){$\dblue{-\dfrac{\pi}{3}}$}
                         %
                         \rput(1.06,0.06){$0$}
                         \rput(0.06,1.1){$\dfrac{\pi}{2}$}
                         \rput(0.06,-1.1){$-\dfrac{\pi}{2}$}
                         \rput(-1.06,0.06){$\pi$}
                    \end{pspicture*}
               }
          \end{extern}
\end{center}}
\begin{center}
     \textit{ Mesures d'angles remarquables}
\end{center}
\begin{h2}2. Sinus et cosinus - Équations trigonométriques\end{h2}
\cadre{bleu}{Définition}{% id="d40"
     Soit $M$ un point du cercle trigonométrique et $x$ une mesure de l'angle $\widehat{IOM}$.
     \par
     On appelle \textbf{cosinus} de $x$, noté\textbf{ $\cos x$} l'abscisse du point $M$.
     \par
     On appelle \textbf{sinus} de $x$, noté\textbf{ $\sin x$} l'ordonnée du point $M$
}
\begin{center}
     \begin{extern}%width="400" alt="sinus et cosinus d'un angle orienté"
          \resizebox{7cm}{!}{
               \newrgbcolor{dblue}{0. 0. 0.7}
               \newrgbcolor{dvert}{0. 0.4 0.}
               \newrgbcolor{dmauve}{0.5 0. 0.5}
               \psset{xunit=5.0cm,yunit=5.0cm,algebraic=true,dimen=middle,dotstyle=o,dotsize=5pt 0,linewidth=0.8pt,arrowsize=3pt 2,arrowinset=0.25}
               \begin{pspicture*}(-1.2,-1.2)(1.2,1.2)
                    \psaxes[linewidth=0.75pt,labelFontSize=\scriptstyle,xAxis=true,yAxis=true,Dx=10.,Dy=10.,ticksize=-2pt 0,subticks=1]{->}(0,0)(-1.2,-1.2)(1.2,1.2)
                    \pscircle[linewidth=0.8pt](0.,0.){5.} %cercle trigo
                    \parametricplot[linewidth=1.2pt,linecolor=red]{0.0}{0.698}{cos(t)|sin(t)}%arc angle
                    \pscustom[linewidth=0.8pt,linecolor=dmauve,fillcolor=dmauve,fillstyle=solid,opacity=0.1]{ % color angle
                         \parametricplot{0.0}{0.698}{0.15*cos(t)|0.15*sin(t)}
                    \lineto(0.,0.)\closepath}
                    \psellipticarc[linewidth=0.8pt,linecolor=dmauve,arrows=->](0.,0.)(0.15,0.15){0.}{40} % fleche angle
                    \psline[linewidth=0.8pt,linecolor=dmauve](0.,0.)(0.766,0.643)%rayon
                    \psline[linewidth=0.8pt]{->}(0.,0.)(1.,0.) %vecteurs unités
                    \psline[linewidth=0.8pt]{->}(0.,0.)(0,1)
                    %\rput[tl](0.4,0.1){$\vec{i}$}
                    %\rput[tl](-0.06,0.5){$\vec{j}$}
                    \psdots[dotsize=2pt 0,dotstyle=*](0.,0.)
                    \rput[bl](-0.09,-0.09){$O$}
                    \psdots[dotsize=2pt 0,dotstyle=*,linecolor=dblue](1.,0.)
                    \rput[bl](1.02,0.02){\dblue{$I$}}
                    \psdots[dotsize=2pt 0,dotstyle=*,linecolor=dblue](0.766,0.643)
                    \rput[bl](0.78,0.66){\dblue{$M$}}
                    \psdots[dotsize=2pt 0,dotstyle=*,linecolor=dblue](0,1)
                    \rput[bl](0.02,1.03){\dblue{$J$}}
                    \rput[bl](0.19,0.05){\dmauve{$x$}}
                    \psline[linewidth=1pt,linecolor=dvert](0.,0.)(0.766,0)
                    \psline[linewidth=1pt,linecolor=dvert](0.,0.)(0,0.643)
                    \psline[linewidth=0.4pt,linecolor=dvert](0.,0.643)(0.766,0.643)
                    \psline[linewidth=0.4pt,linecolor=dvert](0.766,0.)(0.766,0.643)
                    \rput(0.766,-0.05){\dvert{$\cos x$}}
                    \rput(-0.10,0.643){\dvert{$\sin x$}}
               \end{pspicture*}
          }
     \end{extern}
\end{center}
\begin{center}
     \textit{Sinus et cosinus}
\end{center}
\bloc{cyan}{Remarques}{% id="r40"
     Pour tout réel $x$~:
     \begin{itemize}
          \item $-1 \leqslant  \cos x \leqslant  1$
          \item $-1 \leqslant  \sin x \leqslant  1$
          \item Comme $M$ appartient au cercle trigonométrique, $OM=1$ donc $OM^{2}=1=1$ donc~:
          \par
          $\sin^{2}x+\cos^{2}x=1$ ($\sin^{2}x$ étant une écriture abrégée pour $\left(\sin x\right)^{2}$)
     \end{itemize}
}
\bloc{orange}{Valeurs de sinus et de cosinus à retenir}{% id="r45"
     \begin{center}
          \begin{extern}%width="450" alt="Valeurs de sinus et de cosinus"
               \newrgbcolor{dblue}{0. 0. 0.7}
               \newrgbcolor{dvert}{0. 0.4 0.}
               \newrgbcolor{dmauve}{0.5 0. 0.5}
               \psset{xunit=5.0cm,yunit=5.0cm,algebraic=true,dimen=middle,dotstyle=o,dotsize=5pt 0,linewidth=0.8pt,arrowsize=3pt 2,arrowinset=0.25}
               \begin{pspicture*}(-1.2,-1.2)(1.2,1.2)
                    \psaxes[linewidth=0.75pt,labelFontSize=\scriptstyle,xAxis=true,yAxis=true,Dx=10.,Dy=10.,ticksize=-2pt 0,subticks=1]{->}(0,0)(-1.2,-1.2)(1.2,1.2)
                    \pscircle[linewidth=0.8pt](0.,0.){5.} %cercle trigo
                    \psline[linewidth=0.8pt]{->}(0.,0.)(1.,0.) %vecteurs unités
                    \psline[linewidth=0.8pt]{->}(0.,0.)(0,1)
                    %\rput[tl](0.4,0.1){$\vec{i}$}
                    %\rput[tl](-0.06,0.5){$\vec{j}$}
                    \psdots[dotsize=2pt 0,dotstyle=*](0.,0.)
                    %\rput[bl](-0.09,-0.09){$O$}
                    \psframe[linewidth=0.4pt,linecolor=dvert](-0.707,-0.707)(0.707,0.707)
                    \psline[linewidth=0.8pt,linecolor=dvert](-0.707,-0.707)(0.707,0.707)
                    \psline[linewidth=0.8pt,linecolor=dvert](-0.707,0.707)(0.707,-0.707)
                    \psframe[linewidth=0.4pt,linecolor=red](-0.866,-0.5)(0.866,0.5)
                    \psline[linewidth=0.8pt,linecolor=red](-0.866,-0.5)(0.866,0.5)
                    \psline[linewidth=0.8pt,linecolor=red](0.866,-0.5)(-0.866,0.5)
                    \rput(0.943,0.55){$\red{\dfrac{\pi}{6}}$}
                    \rput(-0.943,0.55){$\red{\dfrac{5\pi}{6}}$}
                    \rput(-0.973,-0.55){$\red{-\dfrac{5\pi}{6}}$}
                    \rput(0.943,-0.55){$\red{-\dfrac{\pi}{6}}$}
                    \rput(0.05,-0.554){\fontsize{7 pt}{7 pt}\selectfont $\red{ -\dfrac{1}{2}}$}
                    \rput(0.05,0.554){\fontsize{7 pt}{7 pt}\selectfont $\red{ \dfrac{1}{2}}$}
                    \rput(0.93,0.07){\fontsize{7 pt}{7 pt}\selectfont $\red{ \dfrac{\sqrt{3}}{2}}$}
                    \rput(-0.93,0.07){\fontsize{7 pt}{7 pt}\selectfont $\red{-\dfrac{\sqrt{3}}{2}}$}
                    %
                    \rput(0.777,0.777){$\dvert{\dfrac{\pi}{4}}$}
                    \rput(-0.777,0.777){$\dvert{\dfrac{3\pi}{4}}$}
                    \rput(-0.807,-0.777){$\dvert{-\dfrac{3\pi}{4}}$}
                    \rput(0.777,-0.777){$\dvert{-\dfrac{\pi}{4}}$}
                    \rput(0.06,-0.77){\fontsize{7 pt}{7 pt}\selectfont $\dvert{ -\dfrac{\sqrt{2}}{2}}$}
                    \rput(0.06,0.77){\fontsize{7 pt}{7 pt}\selectfont $\dvert{ \dfrac{\sqrt{2}}{2}}$}
                    \rput(0.764,0.07){\fontsize{7 pt}{7 pt}\selectfont $\dvert{ \dfrac{\sqrt{2}}{2}}$}
                    \rput(-0.764,0.07){\fontsize{7 pt}{7 pt}\selectfont $\dvert{-\dfrac{\sqrt{2}}{2}}$}
                    %
                    \psframe[linewidth=0.4pt,linecolor=dblue](-0.5,-0.866)(0.5,0.866)
                    \psline[linewidth=0.8pt,linecolor=dblue](-0.5,-0.866)(0.5,0.866)
                    \psline[linewidth=0.8pt,linecolor=dblue](-0.5,0.866)(0.5,-0.866)
                    \rput(0.55,0.943){$\dblue{\dfrac{\pi}{3}}$}
                    \rput(-0.55,0.943){$\dblue{\dfrac{2\pi}{3}}$}
                    \rput(-0.58,-0.943){$\dblue{-\dfrac{2\pi}{3}}$}
                    \rput(0.55,-0.943){$\dblue{-\dfrac{\pi}{3}}$}
                    \rput(0.06,-0.933){\fontsize{7 pt}{7 pt}\selectfont $\dblue{ -\dfrac{\sqrt{3}}{2}}$}
                    \rput(0.06,0.933){\fontsize{7 pt}{7 pt}\selectfont $\dblue{ \dfrac{\sqrt{3}}{2}}$}
                    \rput(0.538,0.07){\fontsize{7 pt}{7 pt}\selectfont $\dblue{ \dfrac{1}{2}}$}
                    \rput(-0.538,0.07){\fontsize{7 pt}{7 pt}\selectfont $\dblue{-\dfrac{1}{2}}$}
                    %
                    \rput(1.06,0.06){$0$}
                    \rput(0.06,1.1){$\dfrac{\pi}{2}$}
                    \rput(0.06,-1.1){$-\dfrac{\pi}{2}$}
                    \rput(-1.06,0.06){$\pi$}
               \end{pspicture*}
          \end{extern}
     \end{center}
     \begin{tabularx}{0.8\linewidth}{|*{10}{>{\centering \arraybackslash }X|}}%class="compact" width="600"
          \hline
          \textbf{$x$}  & $0$ & $\frac{\pi }{6}$ & $\frac{\pi }{4}$ & $\frac{\pi }{3}$ & $\frac{\pi }{2}$ & $\frac{2\pi }{3}$ & $\frac{3\pi }{4}$ & $\frac{5\pi }{6}$ & $\pi $
          \\ \hline
          \textbf{$\cos x$} & $1$ & $\frac{\sqrt{3}}{2}$ & $\frac{\sqrt{2}}{2}$ & $\frac{1}{2}$ & $0$ &  $-\frac{1}{2}$ & $-\frac{\sqrt{2}}{2}$ & $-\frac{\sqrt{3}}{2}$ & $-1$
          \\ \hline
          \textbf{$\sin x$} & $0$ & $\frac{1}{2}$ & $\frac{\sqrt{2}}{2}$ & $\frac{\sqrt{3}}{2}$ & $1$ & $\frac{\sqrt{3}}{2}$ & $\frac{\sqrt{2}}{2}$ & $\frac{1}{2}$ & $0$
          \\  \hline
     \end{tabularx}
     \begin{tabularx}{0.8\linewidth}{|*{8}{>{\centering \arraybackslash }X|}}%class="compact" width="600"
          \hline
          \textbf{$x$} & $-\frac{\pi }{6}$ & $-\frac{\pi }{4}$ & $-\frac{\pi }{3}$ & $-\frac{\pi }{2}$ & $-\frac{2\pi }{3}$ & $-\frac{3\pi }{4}$ & $-\frac{5\pi }{6}$
          \\ \hline
          \textbf{$\cos x$} & $\frac{\sqrt{3}}{2}$ & $\frac{\sqrt{2}}{2}$ & $\frac{1}{2}$ & $0$ &  $-\frac{1}{2}$ & $-\frac{\sqrt{2}}{2}$ & $-\frac{\sqrt{3}}{2}$
          \\ \hline
          \textbf{$\sin x$} & $-\frac{1}{2}$ & $-\frac{\sqrt{2}}{2}$ & $-\frac{\sqrt{3}}{2}$ & $-1$ & $-\frac{\sqrt{3}}{2}$ & $-\frac{\sqrt{2}}{2}$ & $-\frac{1}{2}$
          \\    \hline
     \end{tabularx}
}
\cadre{vert}{Propriétés}{% id="p60"
     Pour tout réel $x$~:
     \begin{itemize}
          \item %
          $\sin\left(-x\right)=-\sin\left(x\right)$
          \item %
          $\cos\left(-x\right)=\cos\left(x\right)$
          \item %
          $\sin\left(\pi +x\right)=-\sin\left(x\right)$
          \item %
          $\cos\left(\pi +x\right)=-\cos\left(x\right)$
     \end{itemize}
}
\begin{center}
     \begin{extern}%width="400" alt=" Angles x, -x et pi+x"
          \resizebox{7cm}{!}{
               \newrgbcolor{dblue}{0. 0. 0.7}
               \newrgbcolor{dvert}{0. 0.4 0.}
               \newrgbcolor{dmauve}{0.5 0. 0.5}
               \psset{xunit=5.0cm,yunit=5.0cm,algebraic=true,dimen=middle,dotstyle=o,dotsize=5pt 0,linewidth=0.8pt,arrowsize=3pt 2,arrowinset=0.25}
               \begin{pspicture*}(-1.2,-1.2)(1.2,1.2)
                    \psaxes[linewidth=0.75pt,labelFontSize=\scriptstyle,xAxis=true,yAxis=true,Dx=10.,Dy=10.,ticksize=-2pt 0,subticks=1]{->}(0,0)(-1.2,-1.2)(1.2,1.2)
                    \pscircle[linewidth=0.8pt](0.,0.){5.} %cercle trigo
                    \psellipticarc[linewidth=0.8pt,linecolor=dmauve,arrows=->](0.,0.)(0.15,0.15){0.}{40} % fleche angle
                    \psellipticarc[linewidth=0.8pt,linecolor=dvert,arrows=<-](0.,0.)(0.15,0.15){-40}{0} % fleche angle -x
                    \psellipticarc[linewidth=0.8pt,linecolor=red,arrows=->](0.,0.)(0.1,0.1){0.}{220} % fleche angle pi+x
                    \psline[linewidth=0.8pt,linecolor=dmauve](0.,0.)(0.766,0.643)%rayon
                    \psline[linewidth=0.8pt,linecolor=dvert](0.,0.)(0.766,-0.643)%rayon -x
                    \psline[linewidth=0.8pt,linecolor=red](0.,0.)(-0.766,-0.643)%rayon -x
                    \psline[linewidth=0.8pt]{->}(0.,0.)(1.,0.) %vecteurs unités
                    \psline[linewidth=0.8pt]{->}(0.,0.)(0,1)
                    %\rput[tl](0.4,0.1){$\vec{i}$}
                    %\rput[tl](-0.06,0.5){$\vec{j}$}
                    \psdots[dotsize=2pt 0,dotstyle=*](0.,0.)
                    \rput[bl](-0.09,-0.09){$O$}
                    \psdots[dotsize=2pt 0,dotstyle=*,linecolor=dblue](1.,0.)
                    \rput[bl](1.02,0.02){\dblue{$I$}}
                    \psdots[dotsize=2pt 0,dotstyle=*,linecolor=dblue](0,1)
                    \rput[bl](0.02,1.03){\dblue{$J$}}
                    \rput[bl](0.19,0.05){\dmauve{$x$}}
                    \rput[tl](0.19,-0.05){\dvert{$-x$}}
                    \rput[br](-0.13,0.07){\red{$\pi+x$}}
                    \psframe[linewidth=0.4pt,linecolor=lightgray](-0.766,-0.643)(0.766,0.643)
                    \rput(0.766,-0.05){\dvert{$\cos x$}}
                    \rput(-0.766,-0.05){\dvert{$-\cos x$}}
                    \rput(-0.11,0.69){\dvert{$\sin x$}}
                    \rput(-0.14,-0.69){\dvert{$-\sin x$}}
               \end{pspicture*}
          }
     \end{extern}
\end{center}
\begin{center}
     \textit{   Angles $x$, $-x$ et $\pi+x$}
\end{center}
\cadre{vert}{Formules d'addition}{% id="p60"
     Pour tous réels $a$ et $b$~:
     \begin{itemize}
          \item $\cos\left(a+b\right)=\cos\left(a\right) \cos\left(b\right)-\sin\left(a\right) \sin\left(b\right)$
          \item $\sin\left(a+b\right)=\sin\left(a\right) \cos\left(b\right)+\cos\left(a\right) \sin\left(b\right)$
     \end{itemize}
}
\cadre{rouge}{Théorème}{% id="t70"
     Soit $a$ un réel fixé.
     \par
     Les solutions de l'équation $\cos\left(x\right)=\cos\left(a\right)$ sont les réels de la forme~:
     \begin{center}$a+2k\pi   $ ou $  -a+2k\pi       $ où $k$ décrit $\mathbb{Z}$\end{center}
}
\bloc{orange}{Exemple}{% id="e70"
     On cherche à résoudre l'équation $\cos\left(x\right)=0$
     \par
     On sait que $\cos\left(\frac{\pi }{2}\right)=0$ ce qui fournit une solution de l'équation mais permet aussi d'écrire l'équation sous la forme $\cos\left(x\right)=\cos\left(\frac{\pi }{2}\right)$
     \par
     D'après le théorème ci-dessus les solutions sont de la forme~:
     \par
     $x=\frac{\pi }{2}+2k\pi $ ou $x=-\frac{\pi }{2}+2k\pi $ avec $k \in  \mathbb{Z}$
}
\cadre{rouge}{Théorème}{% id="t80"
     Soit $a$ un réel fixé.
     \par
     Les solutions de l'équation $\sin\left(x\right)=\sin\left(a\right)$ sont les réels de la forme~:
     \begin{center}$a+2k\pi   $ ou $  \pi -a+2k\pi       $ où $k$ décrit $\mathbb{Z}$\end{center}
}

\end{document}
µ
\documentclass[a4paper]{article}

%================================================================================================================================
%
% Packages
%
%================================================================================================================================

\usepackage[T1]{fontenc} 	% pour caractères accentués
\usepackage[utf8]{inputenc}  % encodage utf8
\usepackage[french]{babel}	% langue : français
\usepackage{fourier}			% caractères plus lisibles
\usepackage[dvipsnames]{xcolor} % couleurs
\usepackage{fancyhdr}		% réglage header footer
\usepackage{needspace}		% empêcher sauts de page mal placés
\usepackage{graphicx}		% pour inclure des graphiques
\usepackage{enumitem,cprotect}		% personnalise les listes d'items (nécessaire pour ol, al ...)
\usepackage{hyperref}		% Liens hypertexte
\usepackage{pstricks,pst-all,pst-node,pstricks-add,pst-math,pst-plot,pst-tree,pst-eucl} % pstricks
\usepackage[a4paper,includeheadfoot,top=2cm,left=3cm, bottom=2cm,right=3cm]{geometry} % marges etc.
\usepackage{comment}			% commentaires multilignes
\usepackage{amsmath,environ} % maths (matrices, etc.)
\usepackage{amssymb,makeidx}
\usepackage{bm}				% bold maths
\usepackage{tabularx}		% tableaux
\usepackage{colortbl}		% tableaux en couleur
\usepackage{fontawesome}		% Fontawesome
\usepackage{environ}			% environment with command
\usepackage{fp}				% calculs pour ps-tricks
\usepackage{multido}			% pour ps tricks
\usepackage[np]{numprint}	% formattage nombre
\usepackage{tikz,tkz-tab} 			% package principal TikZ
\usepackage{pgfplots}   % axes
\usepackage{mathrsfs}    % cursives
\usepackage{calc}			% calcul taille boites
\usepackage[scaled=0.875]{helvet} % font sans serif
\usepackage{svg} % svg
\usepackage{scrextend} % local margin
\usepackage{scratch} %scratch
\usepackage{multicol} % colonnes
%\usepackage{infix-RPN,pst-func} % formule en notation polanaise inversée
\usepackage{listings}

%================================================================================================================================
%
% Réglages de base
%
%================================================================================================================================

\lstset{
language=Python,   % R code
literate=
{á}{{\'a}}1
{à}{{\`a}}1
{ã}{{\~a}}1
{é}{{\'e}}1
{è}{{\`e}}1
{ê}{{\^e}}1
{í}{{\'i}}1
{ó}{{\'o}}1
{õ}{{\~o}}1
{ú}{{\'u}}1
{ü}{{\"u}}1
{ç}{{\c{c}}}1
{~}{{ }}1
}


\definecolor{codegreen}{rgb}{0,0.6,0}
\definecolor{codegray}{rgb}{0.5,0.5,0.5}
\definecolor{codepurple}{rgb}{0.58,0,0.82}
\definecolor{backcolour}{rgb}{0.95,0.95,0.92}

\lstdefinestyle{mystyle}{
    backgroundcolor=\color{backcolour},   
    commentstyle=\color{codegreen},
    keywordstyle=\color{magenta},
    numberstyle=\tiny\color{codegray},
    stringstyle=\color{codepurple},
    basicstyle=\ttfamily\footnotesize,
    breakatwhitespace=false,         
    breaklines=true,                 
    captionpos=b,                    
    keepspaces=true,                 
    numbers=left,                    
xleftmargin=2em,
framexleftmargin=2em,            
    showspaces=false,                
    showstringspaces=false,
    showtabs=false,                  
    tabsize=2,
    upquote=true
}

\lstset{style=mystyle}


\lstset{style=mystyle}
\newcommand{\imgdir}{C:/laragon/www/newmc/assets/imgsvg/}
\newcommand{\imgsvgdir}{C:/laragon/www/newmc/assets/imgsvg/}

\definecolor{mcgris}{RGB}{220, 220, 220}% ancien~; pour compatibilité
\definecolor{mcbleu}{RGB}{52, 152, 219}
\definecolor{mcvert}{RGB}{125, 194, 70}
\definecolor{mcmauve}{RGB}{154, 0, 215}
\definecolor{mcorange}{RGB}{255, 96, 0}
\definecolor{mcturquoise}{RGB}{0, 153, 153}
\definecolor{mcrouge}{RGB}{255, 0, 0}
\definecolor{mclightvert}{RGB}{205, 234, 190}

\definecolor{gris}{RGB}{220, 220, 220}
\definecolor{bleu}{RGB}{52, 152, 219}
\definecolor{vert}{RGB}{125, 194, 70}
\definecolor{mauve}{RGB}{154, 0, 215}
\definecolor{orange}{RGB}{255, 96, 0}
\definecolor{turquoise}{RGB}{0, 153, 153}
\definecolor{rouge}{RGB}{255, 0, 0}
\definecolor{lightvert}{RGB}{205, 234, 190}
\setitemize[0]{label=\color{lightvert}  $\bullet$}

\pagestyle{fancy}
\renewcommand{\headrulewidth}{0.2pt}
\fancyhead[L]{maths-cours.fr}
\fancyhead[R]{\thepage}
\renewcommand{\footrulewidth}{0.2pt}
\fancyfoot[C]{}

\newcolumntype{C}{>{\centering\arraybackslash}X}
\newcolumntype{s}{>{\hsize=.35\hsize\arraybackslash}X}

\setlength{\parindent}{0pt}		 
\setlength{\parskip}{3mm}
\setlength{\headheight}{1cm}

\def\ebook{ebook}
\def\book{book}
\def\web{web}
\def\type{web}

\newcommand{\vect}[1]{\overrightarrow{\,\mathstrut#1\,}}

\def\Oij{$\left(\text{O}~;~\vect{\imath},~\vect{\jmath}\right)$}
\def\Oijk{$\left(\text{O}~;~\vect{\imath},~\vect{\jmath},~\vect{k}\right)$}
\def\Ouv{$\left(\text{O}~;~\vect{u},~\vect{v}\right)$}

\hypersetup{breaklinks=true, colorlinks = true, linkcolor = OliveGreen, urlcolor = OliveGreen, citecolor = OliveGreen, pdfauthor={Didier BONNEL - https://www.maths-cours.fr} } % supprime les bordures autour des liens

\renewcommand{\arg}[0]{\text{arg}}

\everymath{\displaystyle}

%================================================================================================================================
%
% Macros - Commandes
%
%================================================================================================================================

\newcommand\meta[2]{    			% Utilisé pour créer le post HTML.
	\def\titre{titre}
	\def\url{url}
	\def\arg{#1}
	\ifx\titre\arg
		\newcommand\maintitle{#2}
		\fancyhead[L]{#2}
		{\Large\sffamily \MakeUppercase{#2}}
		\vspace{1mm}\textcolor{mcvert}{\hrule}
	\fi 
	\ifx\url\arg
		\fancyfoot[L]{\href{https://www.maths-cours.fr#2}{\black \footnotesize{https://www.maths-cours.fr#2}}}
	\fi 
}


\newcommand\TitreC[1]{    		% Titre centré
     \needspace{3\baselineskip}
     \begin{center}\textbf{#1}\end{center}
}

\newcommand\newpar{    		% paragraphe
     \par
}

\newcommand\nosp {    		% commande vide (pas d'espace)
}
\newcommand{\id}[1]{} %ignore

\newcommand\boite[2]{				% Boite simple sans titre
	\vspace{5mm}
	\setlength{\fboxrule}{0.2mm}
	\setlength{\fboxsep}{5mm}	
	\fcolorbox{#1}{#1!3}{\makebox[\linewidth-2\fboxrule-2\fboxsep]{
  		\begin{minipage}[t]{\linewidth-2\fboxrule-4\fboxsep}\setlength{\parskip}{3mm}
  			 #2
  		\end{minipage}
	}}
	\vspace{5mm}
}

\newcommand\CBox[4]{				% Boites
	\vspace{5mm}
	\setlength{\fboxrule}{0.2mm}
	\setlength{\fboxsep}{5mm}
	
	\fcolorbox{#1}{#1!3}{\makebox[\linewidth-2\fboxrule-2\fboxsep]{
		\begin{minipage}[t]{1cm}\setlength{\parskip}{3mm}
	  		\textcolor{#1}{\LARGE{#2}}    
 	 	\end{minipage}  
  		\begin{minipage}[t]{\linewidth-2\fboxrule-4\fboxsep}\setlength{\parskip}{3mm}
			\raisebox{1.2mm}{\normalsize\sffamily{\textcolor{#1}{#3}}}						
  			 #4
  		\end{minipage}
	}}
	\vspace{5mm}
}

\newcommand\cadre[3]{				% Boites convertible html
	\par
	\vspace{2mm}
	\setlength{\fboxrule}{0.1mm}
	\setlength{\fboxsep}{5mm}
	\fcolorbox{#1}{white}{\makebox[\linewidth-2\fboxrule-2\fboxsep]{
  		\begin{minipage}[t]{\linewidth-2\fboxrule-4\fboxsep}\setlength{\parskip}{3mm}
			\raisebox{-2.5mm}{\sffamily \small{\textcolor{#1}{\MakeUppercase{#2}}}}		
			\par		
  			 #3
 	 		\end{minipage}
	}}
		\vspace{2mm}
	\par
}

\newcommand\bloc[3]{				% Boites convertible html sans bordure
     \needspace{2\baselineskip}
     {\sffamily \small{\textcolor{#1}{\MakeUppercase{#2}}}}    
		\par		
  			 #3
		\par
}

\newcommand\CHelp[1]{
     \CBox{Plum}{\faInfoCircle}{À RETENIR}{#1}
}

\newcommand\CUp[1]{
     \CBox{NavyBlue}{\faThumbsOUp}{EN PRATIQUE}{#1}
}

\newcommand\CInfo[1]{
     \CBox{Sepia}{\faArrowCircleRight}{REMARQUE}{#1}
}

\newcommand\CRedac[1]{
     \CBox{PineGreen}{\faEdit}{BIEN R\'EDIGER}{#1}
}

\newcommand\CError[1]{
     \CBox{Red}{\faExclamationTriangle}{ATTENTION}{#1}
}

\newcommand\TitreExo[2]{
\needspace{4\baselineskip}
 {\sffamily\large EXERCICE #1\ (\emph{#2 points})}
\vspace{5mm}
}

\newcommand\img[2]{
          \includegraphics[width=#2\paperwidth]{\imgdir#1}
}

\newcommand\imgsvg[2]{
       \begin{center}   \includegraphics[width=#2\paperwidth]{\imgsvgdir#1} \end{center}
}


\newcommand\Lien[2]{
     \href{#1}{#2 \tiny \faExternalLink}
}
\newcommand\mcLien[2]{
     \href{https~://www.maths-cours.fr/#1}{#2 \tiny \faExternalLink}
}

\newcommand{\euro}{\eurologo{}}

%================================================================================================================================
%
% Macros - Environement
%
%================================================================================================================================

\newenvironment{tex}{ %
}
{%
}

\newenvironment{indente}{ %
	\setlength\parindent{10mm}
}

{
	\setlength\parindent{0mm}
}

\newenvironment{corrige}{%
     \needspace{3\baselineskip}
     \medskip
     \textbf{\textsc{Corrigé}}
     \medskip
}
{
}

\newenvironment{extern}{%
     \begin{center}
     }
     {
     \end{center}
}

\NewEnviron{code}{%
	\par
     \boite{gray}{\texttt{%
     \BODY
     }}
     \par
}

\newenvironment{vbloc}{% boite sans cadre empeche saut de page
     \begin{minipage}[t]{\linewidth}
     }
     {
     \end{minipage}
}
\NewEnviron{h2}{%
    \needspace{3\baselineskip}
    \vspace{0.6cm}
	\noindent \MakeUppercase{\sffamily \large \BODY}
	\vspace{1mm}\textcolor{mcgris}{\hrule}\vspace{0.4cm}
	\par
}{}

\NewEnviron{h3}{%
    \needspace{3\baselineskip}
	\vspace{5mm}
	\textsc{\BODY}
	\par
}

\NewEnviron{margeneg}{ %
\begin{addmargin}[-1cm]{0cm}
\BODY
\end{addmargin}
}

\NewEnviron{html}{%
}

\begin{document}
\meta{url}{/cours/produit-scalaire/}
\meta{pid}{352}
\meta{titre}{Produit scalaire}
\meta{type}{cours}
\begin{h2}1. Produit scalaire de deux vecteurs\end{h2}
\cadre{bleu}{Définition}{% id="d10"
     Soient $\vec{u}$ et $\vec{v}$ deux vecteurs non nuls du plan.
     \par
     On appelle \textbf{produit scalaire} de $\vec{u}$ et $\vec{v}$ le \textbf{nombre réel} noté $\vec{u}.\vec{v}$ défini par :
     \begin{center}$\vec{u}.\vec{v}=||\vec{u}||\times ||\vec{v}||\times \cos\left(\vec{u},\vec{v}\right)$\end{center}
}
\bloc{cyan}{Remarques}{% id="r10"
     \begin{itemize}
          \item \textbf{Attention : } le produit scalaire est un nombre réel et non un vecteur !
          \item On rappelle que $||\overrightarrow{AB}||$ (norme du vecteur $\overrightarrow{AB}$) désigne la longueur du segment $AB$.
          \item Si l'un des vecteurs $\vec{u}$ ou $\vec{v}$ est nul, $\cos\left(\vec{u},\vec{v}\right)$ n'est pas défini; on considèrera alors que le produit scalaire $\vec{u}.\vec{v}$ vaut $0$
          \item Le cosinus d'un angle étant égal au cosinus de l'angle opposé : $\cos\left(\vec{u}, \vec{v}\right)=\cos\left(\vec{v}, \vec{u}\right)$. Par conséquent $\vec{u}.\vec{v}=\vec{v}.\vec{u}$
     \end{itemize}
}
\bloc{orange}{Exemple}{% id="e10"
     \begin{center}
          \begin{extern}%width="200" alt="Triangle équilatéral"
               \newrgbcolor{mvert}{0. 0.4 0.}
               \psset{xunit=1cm,yunit=1cm,algebraic=true,dimen=middle,dotstyle=o,dotsize=5pt 0,linewidth=1pt}
               \begin{pspicture*}(-1.,-0.2)(5.,4.)
                    \psline[linewidth=1.pt](0.,0.)(2.,3.464)
                    \psline[linewidth=1.pt](2.,3.464)(4.,0.)
                    \psline[linewidth=1.pt](4.,0.)(0.,0.)
                    \parametricplot[arrows=->,linecolor=mvert]{0.0}{1.}{0.6*cos(t)+0.|0.6*sin(t)+0.}
                    \psdots[dotsize=2pt 0,dotstyle=*](0.,0.)
                    \rput[bl](-0.44,-0.08){$A$}
                    \psdots[dotsize=2pt 0,dotstyle=*](4.,0.)
                    \rput[bl](4.12,-0.08){$B$}
                    \psdots[dotsize=2pt 0,dotstyle=*](2.,3.464)
                    \rput[bl](1.82,3.7){$C$}
                    \rput[bl](0.58,0.18){\mvert{$\dfrac{\pi}{3}$}}
               \end{pspicture*}
          \end{extern}
     \end{center}
     $ABC$ est un triangle équilatéral dont le côté mesure $1$ unité.
     \par
     $\overrightarrow{AB}.\overrightarrow{AC}=AB\times AC\times \cos\left(\overrightarrow{AB}, \overrightarrow{AC}\right)=1\times 1\times \cos\frac{\pi }{3}=\frac{1}{2}$
}
\cadre{vert}{Propriété}{% id="p20"
     Deux vecteurs $\vec{u}$ et $\vec{v}$ sont orthogonaux si et seulement si : $\vec{u}.\vec{v}=0$
}
\bloc{cyan}{Démonstration}{% id="r10"
     Si l'un des vecteurs est nul le produit scalaire est nul et la propriété est vraie puisque, par convention, le vecteur nul est orthogonal à tout vecteur du plan.
     \par
     Si les deux vecteurs sont non nuls, leurs normes sont non nulles donc :
     \par
     $\vec{u}.\vec{v}=0 \Leftrightarrow ||\vec{u}||\times ||\vec{v}||\times \cos\left(\vec{u},\vec{v}\right)=0 \Leftrightarrow \cos\left(\vec{u},\vec{v}\right)=0 \Leftrightarrow \vec{u}$ et $\vec{v}$ sont orthogonaux
}
\cadre{vert}{Propriété}{% id="p30"
     Pour tous vecteurs $\vec{u}, \vec{v}, \vec{w}$ et tout réel $k$ :
     \begin{itemize}
          \item $\left(k\vec{u}\right).\vec{v}=k \left(\vec{u}.\vec{v}\right)$
          \item $\vec{u}.\left(\vec{v}+\vec{w}\right)=\vec{u}.\vec{v}+\vec{u}.\vec{w}$
     \end{itemize}
}
\cadre{bleu}{Définition et propriété}{% id="d30"
     Soit $\vec{u}$ un vecteur du plan. Le \textbf{carré scalaire} de $\vec{u}$ est le réel positif ou nul :
     \par
     $\vec{u}^{2}=\vec{u}.\vec{u}=||\vec{u}||^{2}$
}
\bloc{cyan}{Démonstration}{% id="m30"
     Le cosinus d'un angle nul vaut $1$ donc $\cos\left(\vec{u}, \vec{u}\right)=1$. Par conséquent :
     \par
     $\vec{u}.\vec{u}=||\vec{u}||\times ||\vec{u}||\times \cos\left(\vec{u},\vec{u}\right)=||\vec{u}||^{2}$
}
\cadre{rouge}{Théorème}{% id="t40"
     Pour tous vecteurs $\vec{u}$ et $\vec{v}$ :
     \begin{center}$\vec{u}.\vec{v}=\frac{1}{2} \left(||\vec{u}+\vec{v}||^{2}-||\vec{u}||^{2}-||\vec{v}||^{2}\right)$\end{center}
}
\bloc{cyan}{Démonstration}{% id="m40"
     $||\vec{u}+\vec{v}||^{2}=\left(\vec{u}+\vec{v}\right)^{2}=\vec{u}^{2}+2\left(\vec{u}.\vec{v}\right)+\vec{v}^{2}=||\vec{u}||^{2}+2\left(\vec{u}.\vec{v}\right)+||\vec{v}||^{2}$
     \par
     Par conséquent :
     \par
     $||\vec{u}+\vec{v}||^{2}-||\vec{u}||^{2}-||\vec{v}||^{2}=2\left(\vec{u}.\vec{v}\right)$
     \par
     et l'on obtient l'égalité souhaitée en divisant chaque membre par $2$.
}
\bloc{cyan}{Remarque}{ % id=r40
     De la même manière, en développant  $(\vec{u}-\vec{v})^{2}$ on obtient~:
     \newpar
     \begin{center}$\vec{u}.\vec{v}=\frac{1}{2} \left(||\vec{u}||^{2}+||\vec{v}||^{2}  - ||\vec{u}-\vec{v}||^{2}\right)$\end{center}
}% fin remarque
\bloc{orange}{Exemple}{% id="e40"
     \begin{center}
          \begin{extern}%width="300" alt="Triangle équilatéral"
               \newrgbcolor{mvert}{0. 0.4 0.}
               \psset{xunit=1cm,yunit=1cm,algebraic=true,dimen=middle,dotstyle=o,dotsize=5pt 0,linewidth=1pt}
               \begin{pspicture*}(-1.,-1.5)(8.,3.)
                    \psline[linewidth=1.pt](0.,0.)(4,-1)
                    \psline[linewidth=1.pt](4,-1)(7,1)
                    \psline[linewidth=1.pt](7,1)(3,2)
                    \psline[linewidth=1.pt](3,2)(0.,0.)
                    \psline[linewidth=1.pt](3,2)(4,-1)
                    \rput[l](-0.5,0){$B$}\rput[l](4,-1.3){$A$}
                    \rput[l](3,2.3){$C$}\rput[l](7.2,1){$D$}
                    \rput[b](2,-0.4){$6$}\rput[l](3.7,0.5){$4$}\rput[l](1.25,1.2){$5$}
               \end{pspicture*}
          \end{extern}
     \end{center}
     $ABCD$ est un parallélogramme tel que $AB=6$, $AC=4$ et $BC=5$.\\
     On souhaite calculer $\overrightarrow{AB}.\overrightarrow{AD}$.
     \par
     $\overrightarrow{AB}.\overrightarrow{AD}=\frac{1}{2}\left(||\overrightarrow{AB}+\overrightarrow{AD}||^{2}-||\overrightarrow{AB}||^{2}-||\overrightarrow{AD}||^{2}\right)$
     \par
     Or $\overrightarrow{AB}+\overrightarrow{AD}=\overrightarrow{AB}+\overrightarrow{BC}=\overrightarrow{AC}$ d'après la relation de Chasles. Par conséquent :
     \par
     $\overrightarrow{AB}.\overrightarrow{AD}=\frac{1}{2}\left(||\overrightarrow{AC}||^{2}-||\overrightarrow{AB}||^{2}-||\overrightarrow{AD}||^{2}\right)$\\
     $\phantom{{AB}.{AD}}=\frac{1}{2}\left(AC^{2}-AB^{2}-AD^{2}\right)$\\
     $\phantom{{AB}.{AD}}=\frac{1}{2}\left(16-36-25\right)=-\frac{45}{2}$
}
\cadre{rouge}{Théorème}{% id="t50"
     Soient $A, B, C$ trois points du plan et $H$ la projection orthogonale de $C$ sur la droite $\left(AB\right)$
     \par
     Alors :
     \begin{itemize}
          \item $\overrightarrow{AB}.\overrightarrow{AC}=AB\times AH   $ si l'angle $\widehat{BAC}$ est aigu
          \item $\overrightarrow{AB}.\overrightarrow{AC}=-AB\times AH   $ si l'angle $\widehat{BAC}$ est obtus
     \end{itemize}
}
\begin{center}
     \begin{extern}%width="300" alt="Produit scalaire et projection orthogonale"
          \newrgbcolor{mvert}{0. 0.4 0.}
          \psset{xunit=1cm,yunit=1cm,algebraic=true,dimen=middle,dotstyle=o,dotsize=5pt 0,linewidth=1pt}
          \begin{pspicture*}(-1.,-0.5)(8.,4.)
               \psline[linecolor=blue]{->}(0.,0.)(7,1)
               \psline[linecolor=blue]{->}(0,0)(2,3)
               \psline[linecolor=lightgray](2,3)(2.36,0.345)
               \psline[linecolor=lightgray](2.48,0.505)(2.34,0.485)
               \psline[linecolor=lightgray](2.5,0.365)(2.48,0.505)
               \psline[linecolor=red](0,0)(2.36,0.345)
               \rput[l](-0.5,0){\blue $A$}\rput[l](7.1,1){\blue $B$}\rput[l](2.1,3.2){\blue $C$}
               \rput[l](2.36,0.12){\red $H$}
          \end{pspicture*}
     \end{extern}
\end{center}
\begin{center}
     \textit{Ici : $\overrightarrow{AB}.\overrightarrow{AC}=AB\times AH$ }
\end{center}
\bloc{orange}{Exemple}{% id="e50"
     \begin{center}
          \begin{extern}%width="160" alt="Produit scalaire et projection orthogonale"
               \newrgbcolor{mvert}{0. 0.4 0.}
               \psset{xunit=1cm,yunit=1cm,algebraic=true,dimen=middle,dotstyle=o,dotsize=5pt 0,linewidth=1pt}
               \begin{pspicture*}(-0.5,-0.5)(3.5,2.5)
                    \psgrid[griddots=0, subgriddiv=0, gridlabels=0pt,gridcolor=lightgray](0,0)(3.,2.)
                    \psline[linecolor=blue]{->}(0.,0.)(3,0)
                    \psline[linecolor=blue]{->}(0,0)(2,2)
                    \rput[l](-0.4,0){\blue $A$}\rput[r](3.3,0){\blue $B$}\rput[l](2.1,2.3){\blue $C$}
               \end{pspicture*}
          \end{extern}
     \end{center}
     Sur la figure ci-dessus où l'unité est le carreau, le point $C$ se projette orthogonalement sur la droite $\left(AB\right)$ en un point $H$ (non représenté) tel que $AH=2$.
     \par
     Par conséquent, l'angle $\widehat{BAC}$ étant aigu :
     \par
     $\overrightarrow{AB}.\overrightarrow{AC}=AB\times AH=3\times 2=6$
}
\cadre{rouge}{Théorème}{% id="t60"
     Le plan étant rapporté à un repère orthonormé $\left(O; \vec{i}, \vec{j}\right)$, soient $\vec{u}\left(x; y\right)$ et $\vec{v}\left(x^{\prime}, y^{\prime}\right)$ deux vecteurs du plan; alors :
     \begin{center}$\vec{u}.\vec{v}=xx^{\prime}+yy^{\prime}$\end{center}
}
\bloc{cyan}{Démonstration}{% id="m60"
     Dire que $\vec{u}$ a pour coordonnées $\left(x ; y\right)$ signifie que $\vec{u}=x\vec{i}+y\vec{j}$. De même $\vec{v}=x^{\prime}\vec{i}+y^{\prime}\vec{j}$
     \par
     $\vec{u}.\vec{v}=\left(x\vec{i}+y\vec{j}\right).\left(x^{\prime}\vec{i}+y^{\prime}\vec{j}\right)=xx^{\prime}\vec{i}^{2}+xy^{\prime}\vec{i}.\vec{j}+x^{\prime}y\vec{i}.\vec{j}+yy^{\prime}\vec{j}^{2}$
     \par
     Or, comme le repère $\left(O; \vec{i}, \vec{j}\right)$ est orthonormé, $\vec{i}^{2}=||\vec{i}||^{2}=1$, $\vec{j}^{2}=||\vec{j}||^{2}=1$ et $\vec{i}.\vec{j}=0$. Par conséquent :
     \par
     $\vec{u}.\vec{v}=xx^{\prime}+yy^{\prime}$
}
\begin{h2}2. Applications du produit scalaire\end{h2}
\cadre{rouge}{Théorème (de la médiane)}{% id="t70"
     Soient $ABC$ un triangle quelconque et $I$ le milieu de $\left[BC\right]$. Alors :
     \begin{center}$AB^{2}+AC^{2}=2AI^{2}+\frac{BC^{2}}{2}$\end{center}
}
\begin{center}
     \begin{extern}%width="300" alt="Théorème de la médiane"
          \psset{xunit=1.0cm,yunit=1.0cm,algebraic=true,dimen=middle,dotstyle=o,dotsize=5pt 0,linewidth=0.8pt}
          \begin{pspicture*}(-1.,-1.)(7.,6.)
               \psline(0.,0.04)(3.,1.)
               \psline(1.46,0.63)(1.54,0.41)
               \psline(3.,1.)(6.,2.)
               \psline(4.46,1.61)(4.54,1.39)
               \psline(6.,2.)(3.,5.)
               \psline(3.,5.)(0.,0.04)
               \psline[linecolor=red](3.,5.)(3.,1.)
               \rput[bl](-0.3,-0.3){$B$}
               \rput[bl](2.9,5.1){$A$}
               \rput[bl](6.08,1.8){$C$}
               \rput[bl](3.,0.65){\red $I$}
          \end{pspicture*}
     \end{extern}
\end{center}
\begin{center}
     \textit{Médiane dans un triangle}
\end{center}
\bloc{cyan}{Remarque}{% id="r70"
     La démonstration est laissée en exercice : \mcLien{/exercices/geometrie-plan/theoreme-mediane}{Exercice théorème de la médiane}
}
\cadre{vert}{Propriété (Formule d'Al Kashi)}{% id="p80"
     Soit $ABC$ un triangle quelconque :
     \begin{center}$BC^{2}=AB^{2}+AC^{2}-2 AB\times AC \cos\left(\overrightarrow{AB}, \overrightarrow{AC}\right)$\end{center}
}
\bloc{cyan}{Remarque}{% id="r80"
     \begin{itemize}
          \item La démonstration est faite en exercice : \mcLien{/exercices/geometrie-plan/formule-al-kashi}{Exercice formule d'Al Kashi}
          \item Si le triangle $ABC$ est rectangle en $A$ alors $\cos\left(\overrightarrow{AB}, \overrightarrow{AC}\right)=0$. On retrouve alors le théorème de Pythagore.
     \end{itemize}
}
\cadre{bleu}{Définition (Vecteur normal à une droite)}{% id="d85"
     On dit qu'un vecteur $\vec{n}$ non nul est \textbf{normal} à la droite $d$ si et seulement si il est orthogonal à un vecteur directeur de $d$.
}
\begin{center}
     \begin{extern}%width="400" alt="Vecteur normal à une droite"
          \resizebox{8cm}{!}{
               \psset{xunit=1.0cm,yunit=1.0cm,algebraic=true,dimen=middle}
               \begin{pspicture*}(0.,0.)(10.,5.)
                    \psline[linewidth=1.pt](0.,0.)(10.,5.)
                    \psline[linewidth=1.pt,linecolor=blue]{->}(4.,2.)(3.,4.)
                    \rput[tl](8.74,5.){$d$}
                    \rput[bl](3.6,3){\blue{$\vec{n}$}}
               \end{pspicture*}
          }
     \end{extern}
\end{center}
\begin{center}
     \textit{Vecteur $\vec{n}$ normal à la droite $d$}
\end{center}
\cadre{rouge}{Théorème}{% id="t100"
     Le plan est rapporté à un repère orthonormé $\left(O, \vec{i}, \vec{j}\right)$
     \par
     La droite $d$ de vecteur normal $\vec{n} \left(a ; b\right)$ admet une équation cartésienne de la forme :
     \begin{center}$ax+by+c=0$\end{center}
     où $a$, $b$ sont les coordonnées de $\vec{n}$ et $c$ un nombre réel.
     \par
     \textbf{Réciproquement}, l'ensemble des points $M\left(x ; y\right)$ tels que $ax+by+c=0$ ($a, b, c$ étant des réels avec $a\neq 0$ ou $b\neq 0$)  est une droite dont un vecteur normal est  $\vec{n}\left(a ; b\right)$.
}
\bloc{cyan}{Remarque}{% id="r100"
     La démonstration est laissée en exercice : \mcLien{/exercices/geometrie-plan/vecteur-directeur-normal-droite}{Exercice vecteur normal à une droite}
}
\cadre{rouge}{Théorème (équation cartésienne d'un cercle)}{% id="t110"
     Le plan est rapporté à un repère orthonormé $\left(O, \vec{i}, \vec{j}\right)$.
     \par
     Soit $I \left(x_{I} ; y_{I}\right)$ un point quelconque du plan et $r$ un réel positif.
     \par
     Une équation du cercle de centre $I$ et de rayon $r$ est :
     \begin{center}$\left(x-x_{I}\right)^{2}+\left(y-y_{I}\right)^{2}=r^{2}$\end{center}
}
\bloc{cyan}{Démonstration}{% id="r110"
     Le point $M \left(x ; y\right)$ appartient au cercle si et seulement si $IM=r$. Comme $IM$ et $r$ sont positif cela équivaut à $IM^{2}=r^{2}$. Or $IM^{2}= \left(x-x_{I}\right)^{2}+\left(y-y_{I}\right)^{2}$; on obtient donc le résultat souhaité.
}
\bloc{orange}{Exemple}{% id="e110"
     Le cercle de centre $\Omega  \left(3;4\right)$ et de rayon $5$ a pour équation :
     \par
     $\left(x-3\right)^{2}+\left(y-4\right)^{2}=25$
     \par
     $x^{2}-6x+9+y^{2}-8y+16=25$
     \par
     $x^{2}-6x+y^{2}-8y=0$
     \par
     Ce cercle passe par $O$ car on obtient une égalité juste en remplaçant $x$ et $y$ par $0$.
}
Une autre utilisation du produit scalaire est la démonstration des formules d'addition des sinus et cosinus (voir \mcLien{/exercices/geometrie-plan/formule-soustraction-cosinus}{exercice soustraction des cosinus})

\end{document}
µ
\documentclass[a4paper]{article}

%================================================================================================================================
%
% Packages
%
%================================================================================================================================

\usepackage[T1]{fontenc} 	% pour caractères accentués
\usepackage[utf8]{inputenc}  % encodage utf8
\usepackage[french]{babel}	% langue : français
\usepackage{fourier}			% caractères plus lisibles
\usepackage[dvipsnames]{xcolor} % couleurs
\usepackage{fancyhdr}		% réglage header footer
\usepackage{needspace}		% empêcher sauts de page mal placés
\usepackage{graphicx}		% pour inclure des graphiques
\usepackage{enumitem,cprotect}		% personnalise les listes d'items (nécessaire pour ol, al ...)
\usepackage{hyperref}		% Liens hypertexte
\usepackage{pstricks,pst-all,pst-node,pstricks-add,pst-math,pst-plot,pst-tree,pst-eucl} % pstricks
\usepackage[a4paper,includeheadfoot,top=2cm,left=3cm, bottom=2cm,right=3cm]{geometry} % marges etc.
\usepackage{comment}			% commentaires multilignes
\usepackage{amsmath,environ} % maths (matrices, etc.)
\usepackage{amssymb,makeidx}
\usepackage{bm}				% bold maths
\usepackage{tabularx}		% tableaux
\usepackage{colortbl}		% tableaux en couleur
\usepackage{fontawesome}		% Fontawesome
\usepackage{environ}			% environment with command
\usepackage{fp}				% calculs pour ps-tricks
\usepackage{multido}			% pour ps tricks
\usepackage[np]{numprint}	% formattage nombre
\usepackage{tikz,tkz-tab} 			% package principal TikZ
\usepackage{pgfplots}   % axes
\usepackage{mathrsfs}    % cursives
\usepackage{calc}			% calcul taille boites
\usepackage[scaled=0.875]{helvet} % font sans serif
\usepackage{svg} % svg
\usepackage{scrextend} % local margin
\usepackage{scratch} %scratch
\usepackage{multicol} % colonnes
%\usepackage{infix-RPN,pst-func} % formule en notation polanaise inversée
\usepackage{listings}

%================================================================================================================================
%
% Réglages de base
%
%================================================================================================================================

\lstset{
language=Python,   % R code
literate=
{á}{{\'a}}1
{à}{{\`a}}1
{ã}{{\~a}}1
{é}{{\'e}}1
{è}{{\`e}}1
{ê}{{\^e}}1
{í}{{\'i}}1
{ó}{{\'o}}1
{õ}{{\~o}}1
{ú}{{\'u}}1
{ü}{{\"u}}1
{ç}{{\c{c}}}1
{~}{{ }}1
}


\definecolor{codegreen}{rgb}{0,0.6,0}
\definecolor{codegray}{rgb}{0.5,0.5,0.5}
\definecolor{codepurple}{rgb}{0.58,0,0.82}
\definecolor{backcolour}{rgb}{0.95,0.95,0.92}

\lstdefinestyle{mystyle}{
    backgroundcolor=\color{backcolour},   
    commentstyle=\color{codegreen},
    keywordstyle=\color{magenta},
    numberstyle=\tiny\color{codegray},
    stringstyle=\color{codepurple},
    basicstyle=\ttfamily\footnotesize,
    breakatwhitespace=false,         
    breaklines=true,                 
    captionpos=b,                    
    keepspaces=true,                 
    numbers=left,                    
xleftmargin=2em,
framexleftmargin=2em,            
    showspaces=false,                
    showstringspaces=false,
    showtabs=false,                  
    tabsize=2,
    upquote=true
}

\lstset{style=mystyle}


\lstset{style=mystyle}
\newcommand{\imgdir}{C:/laragon/www/newmc/assets/imgsvg/}
\newcommand{\imgsvgdir}{C:/laragon/www/newmc/assets/imgsvg/}

\definecolor{mcgris}{RGB}{220, 220, 220}% ancien~; pour compatibilité
\definecolor{mcbleu}{RGB}{52, 152, 219}
\definecolor{mcvert}{RGB}{125, 194, 70}
\definecolor{mcmauve}{RGB}{154, 0, 215}
\definecolor{mcorange}{RGB}{255, 96, 0}
\definecolor{mcturquoise}{RGB}{0, 153, 153}
\definecolor{mcrouge}{RGB}{255, 0, 0}
\definecolor{mclightvert}{RGB}{205, 234, 190}

\definecolor{gris}{RGB}{220, 220, 220}
\definecolor{bleu}{RGB}{52, 152, 219}
\definecolor{vert}{RGB}{125, 194, 70}
\definecolor{mauve}{RGB}{154, 0, 215}
\definecolor{orange}{RGB}{255, 96, 0}
\definecolor{turquoise}{RGB}{0, 153, 153}
\definecolor{rouge}{RGB}{255, 0, 0}
\definecolor{lightvert}{RGB}{205, 234, 190}
\setitemize[0]{label=\color{lightvert}  $\bullet$}

\pagestyle{fancy}
\renewcommand{\headrulewidth}{0.2pt}
\fancyhead[L]{maths-cours.fr}
\fancyhead[R]{\thepage}
\renewcommand{\footrulewidth}{0.2pt}
\fancyfoot[C]{}

\newcolumntype{C}{>{\centering\arraybackslash}X}
\newcolumntype{s}{>{\hsize=.35\hsize\arraybackslash}X}

\setlength{\parindent}{0pt}		 
\setlength{\parskip}{3mm}
\setlength{\headheight}{1cm}

\def\ebook{ebook}
\def\book{book}
\def\web{web}
\def\type{web}

\newcommand{\vect}[1]{\overrightarrow{\,\mathstrut#1\,}}

\def\Oij{$\left(\text{O}~;~\vect{\imath},~\vect{\jmath}\right)$}
\def\Oijk{$\left(\text{O}~;~\vect{\imath},~\vect{\jmath},~\vect{k}\right)$}
\def\Ouv{$\left(\text{O}~;~\vect{u},~\vect{v}\right)$}

\hypersetup{breaklinks=true, colorlinks = true, linkcolor = OliveGreen, urlcolor = OliveGreen, citecolor = OliveGreen, pdfauthor={Didier BONNEL - https://www.maths-cours.fr} } % supprime les bordures autour des liens

\renewcommand{\arg}[0]{\text{arg}}

\everymath{\displaystyle}

%================================================================================================================================
%
% Macros - Commandes
%
%================================================================================================================================

\newcommand\meta[2]{    			% Utilisé pour créer le post HTML.
	\def\titre{titre}
	\def\url{url}
	\def\arg{#1}
	\ifx\titre\arg
		\newcommand\maintitle{#2}
		\fancyhead[L]{#2}
		{\Large\sffamily \MakeUppercase{#2}}
		\vspace{1mm}\textcolor{mcvert}{\hrule}
	\fi 
	\ifx\url\arg
		\fancyfoot[L]{\href{https://www.maths-cours.fr#2}{\black \footnotesize{https://www.maths-cours.fr#2}}}
	\fi 
}


\newcommand\TitreC[1]{    		% Titre centré
     \needspace{3\baselineskip}
     \begin{center}\textbf{#1}\end{center}
}

\newcommand\newpar{    		% paragraphe
     \par
}

\newcommand\nosp {    		% commande vide (pas d'espace)
}
\newcommand{\id}[1]{} %ignore

\newcommand\boite[2]{				% Boite simple sans titre
	\vspace{5mm}
	\setlength{\fboxrule}{0.2mm}
	\setlength{\fboxsep}{5mm}	
	\fcolorbox{#1}{#1!3}{\makebox[\linewidth-2\fboxrule-2\fboxsep]{
  		\begin{minipage}[t]{\linewidth-2\fboxrule-4\fboxsep}\setlength{\parskip}{3mm}
  			 #2
  		\end{minipage}
	}}
	\vspace{5mm}
}

\newcommand\CBox[4]{				% Boites
	\vspace{5mm}
	\setlength{\fboxrule}{0.2mm}
	\setlength{\fboxsep}{5mm}
	
	\fcolorbox{#1}{#1!3}{\makebox[\linewidth-2\fboxrule-2\fboxsep]{
		\begin{minipage}[t]{1cm}\setlength{\parskip}{3mm}
	  		\textcolor{#1}{\LARGE{#2}}    
 	 	\end{minipage}  
  		\begin{minipage}[t]{\linewidth-2\fboxrule-4\fboxsep}\setlength{\parskip}{3mm}
			\raisebox{1.2mm}{\normalsize\sffamily{\textcolor{#1}{#3}}}						
  			 #4
  		\end{minipage}
	}}
	\vspace{5mm}
}

\newcommand\cadre[3]{				% Boites convertible html
	\par
	\vspace{2mm}
	\setlength{\fboxrule}{0.1mm}
	\setlength{\fboxsep}{5mm}
	\fcolorbox{#1}{white}{\makebox[\linewidth-2\fboxrule-2\fboxsep]{
  		\begin{minipage}[t]{\linewidth-2\fboxrule-4\fboxsep}\setlength{\parskip}{3mm}
			\raisebox{-2.5mm}{\sffamily \small{\textcolor{#1}{\MakeUppercase{#2}}}}		
			\par		
  			 #3
 	 		\end{minipage}
	}}
		\vspace{2mm}
	\par
}

\newcommand\bloc[3]{				% Boites convertible html sans bordure
     \needspace{2\baselineskip}
     {\sffamily \small{\textcolor{#1}{\MakeUppercase{#2}}}}    
		\par		
  			 #3
		\par
}

\newcommand\CHelp[1]{
     \CBox{Plum}{\faInfoCircle}{À RETENIR}{#1}
}

\newcommand\CUp[1]{
     \CBox{NavyBlue}{\faThumbsOUp}{EN PRATIQUE}{#1}
}

\newcommand\CInfo[1]{
     \CBox{Sepia}{\faArrowCircleRight}{REMARQUE}{#1}
}

\newcommand\CRedac[1]{
     \CBox{PineGreen}{\faEdit}{BIEN R\'EDIGER}{#1}
}

\newcommand\CError[1]{
     \CBox{Red}{\faExclamationTriangle}{ATTENTION}{#1}
}

\newcommand\TitreExo[2]{
\needspace{4\baselineskip}
 {\sffamily\large EXERCICE #1\ (\emph{#2 points})}
\vspace{5mm}
}

\newcommand\img[2]{
          \includegraphics[width=#2\paperwidth]{\imgdir#1}
}

\newcommand\imgsvg[2]{
       \begin{center}   \includegraphics[width=#2\paperwidth]{\imgsvgdir#1} \end{center}
}


\newcommand\Lien[2]{
     \href{#1}{#2 \tiny \faExternalLink}
}
\newcommand\mcLien[2]{
     \href{https~://www.maths-cours.fr/#1}{#2 \tiny \faExternalLink}
}

\newcommand{\euro}{\eurologo{}}

%================================================================================================================================
%
% Macros - Environement
%
%================================================================================================================================

\newenvironment{tex}{ %
}
{%
}

\newenvironment{indente}{ %
	\setlength\parindent{10mm}
}

{
	\setlength\parindent{0mm}
}

\newenvironment{corrige}{%
     \needspace{3\baselineskip}
     \medskip
     \textbf{\textsc{Corrigé}}
     \medskip
}
{
}

\newenvironment{extern}{%
     \begin{center}
     }
     {
     \end{center}
}

\NewEnviron{code}{%
	\par
     \boite{gray}{\texttt{%
     \BODY
     }}
     \par
}

\newenvironment{vbloc}{% boite sans cadre empeche saut de page
     \begin{minipage}[t]{\linewidth}
     }
     {
     \end{minipage}
}
\NewEnviron{h2}{%
    \needspace{3\baselineskip}
    \vspace{0.6cm}
	\noindent \MakeUppercase{\sffamily \large \BODY}
	\vspace{1mm}\textcolor{mcgris}{\hrule}\vspace{0.4cm}
	\par
}{}

\NewEnviron{h3}{%
    \needspace{3\baselineskip}
	\vspace{5mm}
	\textsc{\BODY}
	\par
}

\NewEnviron{margeneg}{ %
\begin{addmargin}[-1cm]{0cm}
\BODY
\end{addmargin}
}

\NewEnviron{html}{%
}

\begin{document}
\meta{url}{/cours/polynomes-du-second-degre/}
\meta{pid}{356}
\meta{titre}{Polynômes et équations du second degré}
\meta{type}{cours}
\begin{h2}1. Polynômes du second degré\end{h2}
\cadre{bleu}{Définition}{%id="d10"
     On appelle \textbf{polynôme (ou trinôme) du second degré} toute expression pouvant se mettre sous la forme :
     \par
     $P\left(x\right)=ax^{2}+bx+c$
     \par
     où $a$, $b$ et $c$ sont des réels avec $a \neq 0$
}
\bloc{orange}{Exemples}{%id="e10"
     \begin{itemize}
          \item $P\left(x\right)=2x^{2}+3x-5$ est un polynôme du second degré.
          \item $P\left(x\right)=x^{2}-1$ est un polynôme du second degré avec $b=0$ mais $Q\left(x\right)=x-1$ n'en est pas un car $a$ n'est pas différent de zéro : c'est un polynôme du premier degré (ou une fonction affine)
          \item $P\left(x\right)=5\left(x-1\right)\left(3-2x\right)$ est un polynôme du second degré car en développant on obtient une expression du type souhaité.
     \end{itemize}
}
\cadre{rouge}{Théorème et définition}{%id="d20"
     Tout polynôme du second degré peut s'écrire sous la forme :
     \par
     $P\left(x\right)=a\left(x-\alpha \right)^{2}+ \beta $
     \par
     avec $\alpha =-\frac{b}{2a}$ et $\beta =P\left(\alpha \right)$
     \par
     Cette expression s'appelle \textbf{forme canonique} du polynôme $P$.
}
\bloc{orange}{Exemple}{%id="e20"
     Soit $P\left(x\right)=2x^{2}+4x+5$
     \par
     $\alpha =-\frac{b}{2a}=-\frac{4}{2\times 2}=-1$
     \par
     $\beta =P\left(\alpha \right)=P\left(-1\right)=2\times \left(-1\right)^{2}+4\times \left(-1\right)+5=2-4+5=3$
     \par
     La forme canonique de $P\left(x\right)$ est donc :
     \par
     $P\left(x\right)=2\left(x+1\right)^{2}+3$
}
\begin{h2}2. Equations du second degré\end{h2}
\cadre{bleu}{Définition}{%id="d30"
     On appelle \textbf{racine} d'un polynôme $P\left(x\right)$ une solution de l'équation $P\left(x\right)=0$
}
\bloc{cyan}{Remarque}{%id="r30"
     Ne pas confondre les mots \textit{"racine"} et \textit{"racine carrée"} !
}
\cadre{bleu}{Définition}{%id="d40"
     On appelle \textbf{discriminant} du polynôme $P\left(x\right)=ax^{2}+bx+c$ le nombre :
     \begin{center}$\Delta =b^{2}-4ac$\end{center}
}
\cadre{rouge}{Théorème}{%id="t50"
     \begin{itemize}
          \item Si $\Delta > 0$, le polynôme $P$ admet \textbf{deux racines distinctes} : $x_{1}=\frac{-b-\sqrt{\Delta }}{2a}$ et $x_{2}=\frac{-b+\sqrt{\Delta }}{2a}$
          \item Si $\Delta =0$, le polynôme $P$ admet \textbf{une racine unique} : $x_{0}=\frac{-b}{2a}$
          \item Si $\Delta < 0$, le polynôme $P$ n'admet \textbf{aucune racine} réelle.
     \end{itemize}
}
\bloc{orange}{Exemples}{%id="e50"
     \begin{itemize}
          \item $P_{1}\left(x\right)=-x^{2}+3x-2$
          \par
          $\Delta =9-4\times \left(-1\right)\times \left(-2\right)=1$
          \par
          $P_{1}$ possède 2 racines :
          \par
          $x_{1}=\frac{-3-1}{-2}=2$ et $x_{2}=\frac{-3+1}{-2}=1$
          \item $P_{2}\left(x\right)=x^{2}-4x+4$
          \par
          $\Delta =16-4\times 1\times 4=0$
          \par
          $P_{2}$ possède une seule racine :
          \par
          $x_{0}=-\frac{-4}{2}=2$
          \item $P_{3}\left(x\right)=x^{2}+x+1$
          \par
          $\Delta =1-4\times 1\times 1=-3$
          \par
          $P_{3}$ ne possède aucune racine.
     \end{itemize}
}
\begin{h2}3. Inéquations du second degré\end{h2}
\cadre{rouge}{Théorème}{%id="t60"
     Soit P(x) un trinôme du second degré de discriminant $\Delta $.
     \begin{itemize}
          \item \textbf{Si $\Delta  > 0$ :} $P\left(x\right)$ est du signe de $a$ à l'extérieur des racines (c'est à dire si $x < x_{1}$ ou $x > x_{2}$ ) et du signe opposé entre les racines (si $x_{1} < x < x_{2}$).
          %:-+-+-+-+- Engendré par : http://math.et.info.free.fr/TikZ/TableauxVariations/
          \begin{center}
               \begin{extern}%width="500" alt="Tableau de signe plynôme du second degré delta positif"
                    \begin{tikzpicture}[scale=0.875]
                         % Styles
                         \tikzstyle{cadre}=[thin]
                         \tikzstyle{fleche}=[->,>=latex,thin]
                         \tikzstyle{nondefini}=[lightgray]
                         % Dimensions Modifiables
                         \def\Lrg{1.8}
                         \def\HtX{1.2}
                         \def\HtY{0.5}
                         % Dimensions Calculées
                         \def\lignex{-0.5*\HtX}
                         \def\lignef{-1.5*\HtX}
                         \def\separateur{-0.5*\Lrg}
                         % Largeur du tableau
                         \def\gauche{-1.5*\Lrg}
                         \def\droite{6.5*\Lrg}
                         % Hauteur du tableau
                         \def\haut{0.5*\HtX}
                         \def\bas{-2.5*\HtX-2*\HtY}
                         % Pointillés
                         \draw[gray] (2*\Lrg,\lignex) -- (2*\Lrg,\lignef);
                         \draw[gray] (4*\Lrg,\lignex) -- (4*\Lrg,\lignef);
                         % Ligne de l'abscisse : x
                         \node at (-1*\Lrg,0) {$x$};
                         \node at (0*\Lrg,0) {$-\infty$};
                         \node at (2*\Lrg,0) {$x_1$};
                         \node at (4*\Lrg,0) {$x_2$};
                         \node at (6*\Lrg,0) {$+\infty$};
                         % Ligne de la dérivée : f'(x)
                         \node at (-1*\Lrg,-1*\HtX) {$P(x)$};
                         \node at (0*\Lrg,-1*\HtX) {$ $};
                         \node at (1*\Lrg,-1*\HtX) {signe de $a$};
                         \node at (2*\Lrg,-1*\HtX) {$0$};
                         \node at (3*\Lrg,-1*\HtX) {signe de $-a$};
                         \node at (4*\Lrg,-1*\HtX) {$0$};
                         \node at (5*\Lrg,-1*\HtX) {signe de $a$};
                         \node at (6*\Lrg,-1*\HtX) {$ $};
                         % Ligne de la fonction : f(x)
                         % Encadrement
                         \draw[cadre] (\separateur,\haut) -- (\separateur, \lignef);
                         \draw[cadre] (\gauche,\haut) rectangle  (\droite, \lignef);
                         \draw[cadre] (\gauche,\lignex) -- (\droite,\lignex);
                    \end{tikzpicture}
               \end{extern}
          \end{center}
          %:-+-+-+-+- Fin
          \item \textbf{Si $\Delta =0$ :} $P\left(x\right)$ est toujours du signe de $a$ sauf en $x_{0}$ (où il s'annule).
          %:-+-+-+-+- Engendré par : http://math.et.info.free.fr/TikZ/TableauxVariations/
          \begin{center}
               \begin{extern}%width="390" alt="Tableau de signe plynôme du second degré delta nul"
                    \begin{tikzpicture}[scale=0.875]
                         % Styles
                         \tikzstyle{cadre}=[thin]
                         \tikzstyle{fleche}=[->,>=latex,thin]
                         \tikzstyle{nondefini}=[lightgray]
                         % Dimensions Modifiables
                         \def\Lrg{1.8}
                         \def\HtX{1.2}
                         \def\HtY{0.5}
                         % Dimensions Calculées
                         \def\lignex{-0.5*\HtX}
                         \def\lignef{-1.5*\HtX}
                         \def\separateur{-0.5*\Lrg}
                         % Largeur du tableau
                         \def\gauche{-1.5*\Lrg}
                         \def\droite{4.5*\Lrg}
                         % Hauteur du tableau
                         \def\haut{0.5*\HtX}
                         \def\bas{-2.5*\HtX-2*\HtY}
                         % Pointillés
                         \draw[gray] (2*\Lrg,\lignex) -- (2*\Lrg,\lignef);
                         % Ligne de l'abscisse : x
                         \node at (-1*\Lrg,0) {$x$};
                         \node at (0*\Lrg,0) {$-\infty$};
                         \node at (2*\Lrg,0) {$x_0$};
                         \node at (4*\Lrg,0) {$+\infty$};
                         % Ligne de la dérivée : f'(x)
                         \node at (-1*\Lrg,-1*\HtX) {$P(x)$};
                         \node at (0*\Lrg,-1*\HtX) {$ $};
                         \node at (1*\Lrg,-1*\HtX) {signe de $a$};
                         \node at (2*\Lrg,-1*\HtX) {$0$};
                         \node at (3*\Lrg,-1*\HtX) {signe de $a$};
                         \node at (4*\Lrg,-1*\HtX) {$ $};
                         % Ligne de la fonction : f(x)
                         % Encadrement
                         \draw[cadre] (\separateur,\haut) -- (\separateur, \lignef);
                         \draw[cadre] (\gauche,\haut) rectangle  (\droite, \lignef);
                         \draw[cadre] (\gauche,\lignex) -- (\droite,\lignex);
                    \end{tikzpicture}
               \end{extern}
          \end{center}
          %:-+-+-+-+- Fin
          \item \textbf{Si $\Delta < 0$ :} $P\left(x\right)$ est toujours du signe de $a$.
          \begin{center}
               \begin{extern}%width="230" alt="Tableau de signe plynôme du second degré delta négatif"
                    \begin{tikzpicture}[scale=0.875]
                         % Styles
                         \tikzstyle{cadre}=[thin]
                         \tikzstyle{fleche}=[->,>=latex,thin]
                         \tikzstyle{nondefini}=[lightgray]
                         % Dimensions Modifiables
                         \def\Lrg{1.5}
                         \def\HtX{1.2}
                         \def\HtY{0.5}
                         % Dimensions Calculées
                         \def\lignex{-0.5*\HtX}
                         \def\lignef{-1.5*\HtX}
                         \def\separateur{-0.5*\Lrg}
                         % Largeur du tableau
                         \def\gauche{-1.5*\Lrg}
                         \def\droite{2.5*\Lrg}
                         % Hauteur du tableau
                         \def\haut{0.5*\HtX}
                         \def\bas{-2.5*\HtX-2*\HtY}
                         % Ligne de l'abscisse : x
                         \node at (-1*\Lrg,0) {$x$};
                         \node at (0*\Lrg,0) {$-\infty$};
                         \node at (2*\Lrg,0) {$+\infty$};
                         % Ligne de la dérivée : f'(x)
                         \node at (-1*\Lrg,-1*\HtX) {$P(x)$};
                         \node at (0*\Lrg,-1*\HtX) {$ $};
                         \node at (1*\Lrg,-1*\HtX) {signe de $a$};
                         \node at (2*\Lrg,-1*\HtX) {$ $};
                         % Ligne de la fonction : f(x)
                         % Encadrement
                         \draw[cadre] (\separateur,\haut) -- (\separateur, \lignef);
                         \draw[cadre] (\gauche,\haut) rectangle  (\droite, \lignef);
                         \draw[cadre] (\gauche,\lignex) -- (\droite,\lignex);
                    \end{tikzpicture}
               \end{extern}
          \end{center}
     \end{itemize}
}
\bloc{orange}{Exemples}{%id="e60"
     Si l'on reprend les exemples précédents :
     \begin{itemize}
          \item $P_{1}\left(x\right)=-x^{2}+3x-2$ :
          \par
          $\Delta  > 0$ et $a < 0$.
          %:-+-+-+-+- Engendré par : http://math.et.info.free.fr/TikZ/TableauxVariations/
          \begin{center}
               \begin{extern}%width="430" alt="Exemple tableau de signe plynôme du second degré delta positif"
                    \begin{tikzpicture}[scale=0.875]
                         % Styles
                         \tikzstyle{cadre}=[thin]
                         \tikzstyle{fleche}=[->,>=latex,thin]
                         \tikzstyle{nondefini}=[lightgray]
                         % Dimensions Modifiables
                         \def\Lrg{1.5}
                         \def\HtX{1.2}
                         \def\HtY{0.5}
                         % Dimensions Calculées
                         \def\lignex{-0.5*\HtX}
                         \def\lignef{-1.5*\HtX}
                         \def\separateur{-0.5*\Lrg}
                         % Largeur du tableau
                         \def\gauche{-1.5*\Lrg}
                         \def\droite{6.5*\Lrg}
                         % Hauteur du tableau
                         \def\haut{0.5*\HtX}
                         \def\bas{-2.5*\HtX-2*\HtY}
                         % Pointillés
                         \draw[gray] (2*\Lrg,\lignex) -- (2*\Lrg,\lignef);
                         \draw[gray] (4*\Lrg,\lignex) -- (4*\Lrg,\lignef);
                         % Ligne de l'abscisse : x
                         \node at (-1*\Lrg,0) {$x$};
                         \node at (0*\Lrg,0) {$-\infty$};
                         \node at (2*\Lrg,0) {$1$};
                         \node at (4*\Lrg,0) {$2$};
                         \node at (6*\Lrg,0) {$+\infty$};
                         % Ligne de la dérivée : f'(x)
                         \node at (-1*\Lrg,-1*\HtX) {$P(x)$};
                         \node at (0*\Lrg,-1*\HtX) {$ $};
                         \node at (1*\Lrg,-1*\HtX) {$-$};
                         \node at (2*\Lrg,-1*\HtX) {$0$};
                         \node at (3*\Lrg,-1*\HtX) {$+$};
                         \node at (4*\Lrg,-1*\HtX) {$0$};
                         \node at (5*\Lrg,-1*\HtX) {$-$};
                         \node at (6*\Lrg,-1*\HtX) {$ $};
                         % Ligne de la fonction : f(x)
                         % Encadrement
                         \draw[cadre] (\separateur,\haut) -- (\separateur, \lignef);
                         \draw[cadre] (\gauche,\haut) rectangle  (\droite, \lignef);
                         \draw[cadre] (\gauche,\lignex) -- (\droite,\lignex);
                    \end{tikzpicture}
               \end{extern}
          \end{center}
          %:-+-+-+-+- Fin
          \item $P_{2}\left(x\right)=x^{2}-4x+4$ :
          \par
          $\Delta =0$ et $a > 0$.
          %:-+-+-+-+- Engendré par : http://math.et.info.free.fr/TikZ/TableauxVariations/
          \begin{center}
               \begin{extern}%width="330" alt="Exemple tableau de signe plynôme du second degré delta nul"
                    \begin{tikzpicture}[scale=0.875]
                         % Styles
                         \tikzstyle{cadre}=[thin]
                         \tikzstyle{fleche}=[->,>=latex,thin]
                         \tikzstyle{nondefini}=[lightgray]
                         % Dimensions Modifiables
                         \def\Lrg{1.5}
                         \def\HtX{1.2}
                         \def\HtY{0.5}
                         % Dimensions Calculées
                         \def\lignex{-0.5*\HtX}
                         \def\lignef{-1.5*\HtX}
                         \def\separateur{-0.5*\Lrg}
                         % Largeur du tableau
                         \def\gauche{-1.5*\Lrg}
                         \def\droite{4.5*\Lrg}
                         % Hauteur du tableau
                         \def\haut{0.5*\HtX}
                         \def\bas{-2.5*\HtX-2*\HtY}
                         % Pointillés
                         \draw[gray] (2*\Lrg,\lignex) -- (2*\Lrg,\lignef);
                         % Ligne de l'abscisse : x
                         \node at (-1*\Lrg,0) {$x$};
                         \node at (0*\Lrg,0) {$-\infty$};
                         \node at (2*\Lrg,0) {$2$};
                         \node at (4*\Lrg,0) {$+\infty$};
                         % Ligne de la dérivée : f'(x)
                         \node at (-1*\Lrg,-1*\HtX) {$P(x)$};
                         \node at (0*\Lrg,-1*\HtX) {$ $};
                         \node at (1*\Lrg,-1*\HtX) {+};
                         \node at (2*\Lrg,-1*\HtX) {$0$};
                         \node at (3*\Lrg,-1*\HtX) {+};
                         \node at (4*\Lrg,-1*\HtX) {$ $};
                         % Ligne de la fonction : f(x)
                         % Encadrement
                         \draw[cadre] (\separateur,\haut) -- (\separateur, \lignef);
                         \draw[cadre] (\gauche,\haut) rectangle  (\droite, \lignef);
                         \draw[cadre] (\gauche,\lignex) -- (\droite,\lignex);
                    \end{tikzpicture}
               \end{extern}
          \end{center}
          %:-+-+-+-+- Fin
          \item $P_{3}\left(x\right)=x^{2}+x+1$ :
          \par
          $\Delta  < 0$ et $a > 0$.
          \begin{center}
               \begin{extern}%width="230" alt="Exemple tableau de signe plynôme du second degré delta négatif"
                    \begin{tikzpicture}[scale=0.875]
                         % Styles
                         \tikzstyle{cadre}=[thin]
                         \tikzstyle{fleche}=[->,>=latex,thin]
                         \tikzstyle{nondefini}=[lightgray]
                         % Dimensions Modifiables
                         \def\Lrg{1.5}
                         \def\HtX{1.2}
                         \def\HtY{0.5}
                         % Dimensions Calculées
                         \def\lignex{-0.5*\HtX}
                         \def\lignef{-1.5*\HtX}
                         \def\separateur{-0.5*\Lrg}
                         % Largeur du tableau
                         \def\gauche{-1.5*\Lrg}
                         \def\droite{2.5*\Lrg}
                         % Hauteur du tableau
                         \def\haut{0.5*\HtX}
                         \def\bas{-2.5*\HtX-2*\HtY}
                         % Ligne de l'abscisse : x
                         \node at (-1*\Lrg,0) {$x$};
                         \node at (0*\Lrg,0) {$-\infty$};
                         \node at (2*\Lrg,0) {$+\infty$};
                         % Ligne de la dérivée : f'(x)
                         \node at (-1*\Lrg,-1*\HtX) {$P(x)$};
                         \node at (0*\Lrg,-1*\HtX) {$ $};
                         \node at (1*\Lrg,-1*\HtX) {$+$};
                         \node at (2*\Lrg,-1*\HtX) {$ $};
                         % Ligne de la fonction : f(x)
                         % Encadrement
                         \draw[cadre] (\separateur,\haut) -- (\separateur, \lignef);
                         \draw[cadre] (\gauche,\haut) rectangle  (\droite, \lignef);
                         \draw[cadre] (\gauche,\lignex) -- (\droite,\lignex);
                    \end{tikzpicture}
               \end{extern}
          \end{center}
     \end{itemize}
}
\begin{h2}4. Interprétation graphique\end{h2}
On rappelle que les solutions de l'équation $f\left(x\right)=0$ sont les abscisses des \textbf{points d'intersection de la courbe} $C_{f}$ et de l'\textbf{axe des abscisses}.
\par
En regroupant les propriétés de ce chapitre et celles vues en Seconde on peut résumer ces résultats dans le tableau :
\begin{center}
     \begin{extern}%width="600" alt="Différentes paraboles"
          \begin{tabular}{|c|c|c|}
               \hline
               & $a > 0$ & $a < 0$\\ \hline
               $\Delta  > 0$ & \img{parabole-1-1}{0.25}
               & \img{parabole-1-2}{0.25}
               \\
               & 2~racines : $x_{1}$ et $x_{2}$
               & 2~racines : $x_{1}$ et $x_{2}$\\ \hline
               $\Delta =0$ &  \img{parabole-2-1}{0.25}
               &  \img{parabole-2-2}{0.25}
               \\
               &1~racine : $x_{0}$
               &1~racine : $x_{0}$\\ \hline
               $\Delta  < 0$ &  \img{parabole-3-1}{0.25}
               & \img{parabole-3-2}{0.25}
               \\    &Pas~de racine
               & Pas~de racine\\ \hline
          \end{tabular}
     \end{extern}
\end{center}

\end{document}
µ
\documentclass[a4paper]{article}

%================================================================================================================================
%
% Packages
%
%================================================================================================================================

\usepackage[T1]{fontenc} 	% pour caractères accentués
\usepackage[utf8]{inputenc}  % encodage utf8
\usepackage[french]{babel}	% langue : français
\usepackage{fourier}			% caractères plus lisibles
\usepackage[dvipsnames]{xcolor} % couleurs
\usepackage{fancyhdr}		% réglage header footer
\usepackage{needspace}		% empêcher sauts de page mal placés
\usepackage{graphicx}		% pour inclure des graphiques
\usepackage{enumitem,cprotect}		% personnalise les listes d'items (nécessaire pour ol, al ...)
\usepackage{hyperref}		% Liens hypertexte
\usepackage{pstricks,pst-all,pst-node,pstricks-add,pst-math,pst-plot,pst-tree,pst-eucl} % pstricks
\usepackage[a4paper,includeheadfoot,top=2cm,left=3cm, bottom=2cm,right=3cm]{geometry} % marges etc.
\usepackage{comment}			% commentaires multilignes
\usepackage{amsmath,environ} % maths (matrices, etc.)
\usepackage{amssymb,makeidx}
\usepackage{bm}				% bold maths
\usepackage{tabularx}		% tableaux
\usepackage{colortbl}		% tableaux en couleur
\usepackage{fontawesome}		% Fontawesome
\usepackage{environ}			% environment with command
\usepackage{fp}				% calculs pour ps-tricks
\usepackage{multido}			% pour ps tricks
\usepackage[np]{numprint}	% formattage nombre
\usepackage{tikz,tkz-tab} 			% package principal TikZ
\usepackage{pgfplots}   % axes
\usepackage{mathrsfs}    % cursives
\usepackage{calc}			% calcul taille boites
\usepackage[scaled=0.875]{helvet} % font sans serif
\usepackage{svg} % svg
\usepackage{scrextend} % local margin
\usepackage{scratch} %scratch
\usepackage{multicol} % colonnes
%\usepackage{infix-RPN,pst-func} % formule en notation polanaise inversée
\usepackage{listings}

%================================================================================================================================
%
% Réglages de base
%
%================================================================================================================================

\lstset{
language=Python,   % R code
literate=
{á}{{\'a}}1
{à}{{\`a}}1
{ã}{{\~a}}1
{é}{{\'e}}1
{è}{{\`e}}1
{ê}{{\^e}}1
{í}{{\'i}}1
{ó}{{\'o}}1
{õ}{{\~o}}1
{ú}{{\'u}}1
{ü}{{\"u}}1
{ç}{{\c{c}}}1
{~}{{ }}1
}


\definecolor{codegreen}{rgb}{0,0.6,0}
\definecolor{codegray}{rgb}{0.5,0.5,0.5}
\definecolor{codepurple}{rgb}{0.58,0,0.82}
\definecolor{backcolour}{rgb}{0.95,0.95,0.92}

\lstdefinestyle{mystyle}{
    backgroundcolor=\color{backcolour},   
    commentstyle=\color{codegreen},
    keywordstyle=\color{magenta},
    numberstyle=\tiny\color{codegray},
    stringstyle=\color{codepurple},
    basicstyle=\ttfamily\footnotesize,
    breakatwhitespace=false,         
    breaklines=true,                 
    captionpos=b,                    
    keepspaces=true,                 
    numbers=left,                    
xleftmargin=2em,
framexleftmargin=2em,            
    showspaces=false,                
    showstringspaces=false,
    showtabs=false,                  
    tabsize=2,
    upquote=true
}

\lstset{style=mystyle}


\lstset{style=mystyle}
\newcommand{\imgdir}{C:/laragon/www/newmc/assets/imgsvg/}
\newcommand{\imgsvgdir}{C:/laragon/www/newmc/assets/imgsvg/}

\definecolor{mcgris}{RGB}{220, 220, 220}% ancien~; pour compatibilité
\definecolor{mcbleu}{RGB}{52, 152, 219}
\definecolor{mcvert}{RGB}{125, 194, 70}
\definecolor{mcmauve}{RGB}{154, 0, 215}
\definecolor{mcorange}{RGB}{255, 96, 0}
\definecolor{mcturquoise}{RGB}{0, 153, 153}
\definecolor{mcrouge}{RGB}{255, 0, 0}
\definecolor{mclightvert}{RGB}{205, 234, 190}

\definecolor{gris}{RGB}{220, 220, 220}
\definecolor{bleu}{RGB}{52, 152, 219}
\definecolor{vert}{RGB}{125, 194, 70}
\definecolor{mauve}{RGB}{154, 0, 215}
\definecolor{orange}{RGB}{255, 96, 0}
\definecolor{turquoise}{RGB}{0, 153, 153}
\definecolor{rouge}{RGB}{255, 0, 0}
\definecolor{lightvert}{RGB}{205, 234, 190}
\setitemize[0]{label=\color{lightvert}  $\bullet$}

\pagestyle{fancy}
\renewcommand{\headrulewidth}{0.2pt}
\fancyhead[L]{maths-cours.fr}
\fancyhead[R]{\thepage}
\renewcommand{\footrulewidth}{0.2pt}
\fancyfoot[C]{}

\newcolumntype{C}{>{\centering\arraybackslash}X}
\newcolumntype{s}{>{\hsize=.35\hsize\arraybackslash}X}

\setlength{\parindent}{0pt}		 
\setlength{\parskip}{3mm}
\setlength{\headheight}{1cm}

\def\ebook{ebook}
\def\book{book}
\def\web{web}
\def\type{web}

\newcommand{\vect}[1]{\overrightarrow{\,\mathstrut#1\,}}

\def\Oij{$\left(\text{O}~;~\vect{\imath},~\vect{\jmath}\right)$}
\def\Oijk{$\left(\text{O}~;~\vect{\imath},~\vect{\jmath},~\vect{k}\right)$}
\def\Ouv{$\left(\text{O}~;~\vect{u},~\vect{v}\right)$}

\hypersetup{breaklinks=true, colorlinks = true, linkcolor = OliveGreen, urlcolor = OliveGreen, citecolor = OliveGreen, pdfauthor={Didier BONNEL - https://www.maths-cours.fr} } % supprime les bordures autour des liens

\renewcommand{\arg}[0]{\text{arg}}

\everymath{\displaystyle}

%================================================================================================================================
%
% Macros - Commandes
%
%================================================================================================================================

\newcommand\meta[2]{    			% Utilisé pour créer le post HTML.
	\def\titre{titre}
	\def\url{url}
	\def\arg{#1}
	\ifx\titre\arg
		\newcommand\maintitle{#2}
		\fancyhead[L]{#2}
		{\Large\sffamily \MakeUppercase{#2}}
		\vspace{1mm}\textcolor{mcvert}{\hrule}
	\fi 
	\ifx\url\arg
		\fancyfoot[L]{\href{https://www.maths-cours.fr#2}{\black \footnotesize{https://www.maths-cours.fr#2}}}
	\fi 
}


\newcommand\TitreC[1]{    		% Titre centré
     \needspace{3\baselineskip}
     \begin{center}\textbf{#1}\end{center}
}

\newcommand\newpar{    		% paragraphe
     \par
}

\newcommand\nosp {    		% commande vide (pas d'espace)
}
\newcommand{\id}[1]{} %ignore

\newcommand\boite[2]{				% Boite simple sans titre
	\vspace{5mm}
	\setlength{\fboxrule}{0.2mm}
	\setlength{\fboxsep}{5mm}	
	\fcolorbox{#1}{#1!3}{\makebox[\linewidth-2\fboxrule-2\fboxsep]{
  		\begin{minipage}[t]{\linewidth-2\fboxrule-4\fboxsep}\setlength{\parskip}{3mm}
  			 #2
  		\end{minipage}
	}}
	\vspace{5mm}
}

\newcommand\CBox[4]{				% Boites
	\vspace{5mm}
	\setlength{\fboxrule}{0.2mm}
	\setlength{\fboxsep}{5mm}
	
	\fcolorbox{#1}{#1!3}{\makebox[\linewidth-2\fboxrule-2\fboxsep]{
		\begin{minipage}[t]{1cm}\setlength{\parskip}{3mm}
	  		\textcolor{#1}{\LARGE{#2}}    
 	 	\end{minipage}  
  		\begin{minipage}[t]{\linewidth-2\fboxrule-4\fboxsep}\setlength{\parskip}{3mm}
			\raisebox{1.2mm}{\normalsize\sffamily{\textcolor{#1}{#3}}}						
  			 #4
  		\end{minipage}
	}}
	\vspace{5mm}
}

\newcommand\cadre[3]{				% Boites convertible html
	\par
	\vspace{2mm}
	\setlength{\fboxrule}{0.1mm}
	\setlength{\fboxsep}{5mm}
	\fcolorbox{#1}{white}{\makebox[\linewidth-2\fboxrule-2\fboxsep]{
  		\begin{minipage}[t]{\linewidth-2\fboxrule-4\fboxsep}\setlength{\parskip}{3mm}
			\raisebox{-2.5mm}{\sffamily \small{\textcolor{#1}{\MakeUppercase{#2}}}}		
			\par		
  			 #3
 	 		\end{minipage}
	}}
		\vspace{2mm}
	\par
}

\newcommand\bloc[3]{				% Boites convertible html sans bordure
     \needspace{2\baselineskip}
     {\sffamily \small{\textcolor{#1}{\MakeUppercase{#2}}}}    
		\par		
  			 #3
		\par
}

\newcommand\CHelp[1]{
     \CBox{Plum}{\faInfoCircle}{À RETENIR}{#1}
}

\newcommand\CUp[1]{
     \CBox{NavyBlue}{\faThumbsOUp}{EN PRATIQUE}{#1}
}

\newcommand\CInfo[1]{
     \CBox{Sepia}{\faArrowCircleRight}{REMARQUE}{#1}
}

\newcommand\CRedac[1]{
     \CBox{PineGreen}{\faEdit}{BIEN R\'EDIGER}{#1}
}

\newcommand\CError[1]{
     \CBox{Red}{\faExclamationTriangle}{ATTENTION}{#1}
}

\newcommand\TitreExo[2]{
\needspace{4\baselineskip}
 {\sffamily\large EXERCICE #1\ (\emph{#2 points})}
\vspace{5mm}
}

\newcommand\img[2]{
          \includegraphics[width=#2\paperwidth]{\imgdir#1}
}

\newcommand\imgsvg[2]{
       \begin{center}   \includegraphics[width=#2\paperwidth]{\imgsvgdir#1} \end{center}
}


\newcommand\Lien[2]{
     \href{#1}{#2 \tiny \faExternalLink}
}
\newcommand\mcLien[2]{
     \href{https~://www.maths-cours.fr/#1}{#2 \tiny \faExternalLink}
}

\newcommand{\euro}{\eurologo{}}

%================================================================================================================================
%
% Macros - Environement
%
%================================================================================================================================

\newenvironment{tex}{ %
}
{%
}

\newenvironment{indente}{ %
	\setlength\parindent{10mm}
}

{
	\setlength\parindent{0mm}
}

\newenvironment{corrige}{%
     \needspace{3\baselineskip}
     \medskip
     \textbf{\textsc{Corrigé}}
     \medskip
}
{
}

\newenvironment{extern}{%
     \begin{center}
     }
     {
     \end{center}
}

\NewEnviron{code}{%
	\par
     \boite{gray}{\texttt{%
     \BODY
     }}
     \par
}

\newenvironment{vbloc}{% boite sans cadre empeche saut de page
     \begin{minipage}[t]{\linewidth}
     }
     {
     \end{minipage}
}
\NewEnviron{h2}{%
    \needspace{3\baselineskip}
    \vspace{0.6cm}
	\noindent \MakeUppercase{\sffamily \large \BODY}
	\vspace{1mm}\textcolor{mcgris}{\hrule}\vspace{0.4cm}
	\par
}{}

\NewEnviron{h3}{%
    \needspace{3\baselineskip}
	\vspace{5mm}
	\textsc{\BODY}
	\par
}

\NewEnviron{margeneg}{ %
\begin{addmargin}[-1cm]{0cm}
\BODY
\end{addmargin}
}

\NewEnviron{html}{%
}

\begin{document}
\meta{url}{/cours/etude-de-fonctions/}
\meta{pid}{358}
\meta{titre}{Etude de fonctions}
\meta{type}{cours}
\begin{h2}I - Généralités sur les fonctions\end{h2}
\cadre{bleu}{Définition}{% id="d10"
     Une \textbf{fonction} $f$ associe, à tout nombre réel $x$ d'une partie \textit{$D$} de $\mathbb{R}$, un unique nombre réel $y$. $y$ s'appelle l'\textbf{image} de $x$ par la fonction $f$ et se note $f\left(x\right)$
}
\cadre{bleu}{Définition}{% id="d20"
     L'ensemble \textit{$D$} des éléments $x$ de $\mathbb{R}$ qui possèdent une image par $f$ s'appelle l'\textbf{ensemble de définition} de $f$.
}
\cadre{bleu}{Définition}{% id="d30"
     La \textbf{courbe représentative} de la fonction $f$ est l'ensemble des points du plan $\mathscr P$ dont les coordonnées $\left(x; y\right)$ vérifient l'égalité $y=f\left(x\right)$.
}
\bloc{cyan}{Remarque}{% id="r30"
     Cette définition est importante car elle établit un lien entre la courbe représentative d'une fonction et la formule définissant la fonction. Elle permet de déterminer notamment si un point appartient à la courbe représentative d'une fonction.
     \par
     Par exemple si $f$ est la fonction définie sur $\mathbb{R}$ par $f\left(x\right)=x^{2}+1$, le point $A\left(1;2\right)$ appartient à courbe représentative de $f$ car $f\left(1\right)=1^{2}+1=2$. Par contre le point $B\left(2;4\right)$ n'y appartient pas car $f\left(2\right)=5\neq 4$.
}
\cadre{bleu}{Définitions}{% id="d40"
     \begin{itemize}
          \item La fonction $f$ est \textbf{croissante} sur l'intervalle $I$ si et seulement si pour tous réels $x_{1}$ et $x_{2}$ de $I$ tels que $x_{1} \leqslant  x_{2}$ on a $f\left(x_{1}\right) \leqslant f\left(x_{2}\right)$.
          \item La fonction $f$ est \textbf{décroissante} sur l'intervalle $I$ si et seulement si  pour tous réels $x_{1}$ et $x_{2}$ de $I$ tels que $x_{1} \leqslant x_{2}$ on a $f\left(x_{1}\right) \geqslant f\left(x_{2}\right)$.
     \end{itemize}
}
\begin{center}
     \begin{extern}%width="550" alt="Fonctions croissante et décroissante"
          \begin{tabular}{c c c}
               \resizebox{5.5cm}{!}{%
                    % -+-+-+ variables modifiables
                    \def\fonction{1+0.2*x*x }
                    \def\xmin{-1.2}
                    \def\xmax{5}
                    \def\ymin{-0.9}
                    \def\ymax{5}
                    \def\xunit{1}  % unités en cm
                    \def\yunit{1}
                    \psset{xunit=\xunit,yunit=\yunit,algebraic=true}
                    \fontsize{12pt}{12pt}\selectfont
                    \begin{pspicture*}[linewidth=1pt](\xmin,\ymin)(\xmax,\ymax)
                         %      \psgrid[gridcolor=mcgris, subgriddiv=5, gridlabels=0pt](\xmin,\ymin)(\xmax,\ymax)
                         \psaxes[linewidth=0.75pt,Dx=10,Dy=10]{->}(0,0)(\xmin,\ymin)(\xmax,\ymax)
                         \psplot[plotpoints=2000,linecolor=red]{0.2}{\xmax}{\fonction}
                         \psline[linewidth=0.75pt,linecolor=lightgray](1,0)(1,1.2)(0,1.2)
                         \psline[linewidth=0.75pt,linecolor=lightgray](4,0)(4,4.2)(0,4.2)
                         \rput[tr](-0.3,-0.3){$O$} \rput[t](1,-0.3){$x_1$} \rput[t](4,-0.3){$x_2$}
                         \rput[r](-0.1,1.2){$f(x_1)$} \rput[r](-0.1,4.2){$f(x_2)$}
                         \rput[tl](4.3,4.5){$\color{red} \mathcal{C}_f$}
                    \end{pspicture*}
               }
               & ~~~~ &%
               \resizebox{5.5cm}{!}{%
                    % -+-+-+ variables modifiables
                    \def\fonction{4-0.1*x*x }
                    \def\xmin{-1.2}
                    \def\xmax{5}
                    \def\ymin{-0.9}
                    \def\ymax{5}
                    \def\xunit{1}  % unités en cm
                    \def\yunit{1}
                    \psset{xunit=\xunit,yunit=\yunit,algebraic=true}
                    \fontsize{12pt}{12pt}\selectfont
                    \begin{pspicture*}[linewidth=1pt](\xmin,\ymin)(\xmax,\ymax)
                         %      \psgrid[gridcolor=mcgris, subgriddiv=5, gridlabels=0pt](\xmin,\ymin)(\xmax,\ymax)
                         \psaxes[linewidth=0.75pt,Dx=10,Dy=10]{->}(0,0)(\xmin,\ymin)(\xmax,\ymax)
                         \psplot[plotpoints=2000,linecolor=red]{0.2}{\xmax}{\fonction}
                         \psline[linewidth=0.75pt,linecolor=lightgray](1,0)(1,3.9)(0,3.9)
                         \psline[linewidth=0.75pt,linecolor=lightgray](4,0)(4,2.4)(0,2.4)
                         \rput[tr](-0.3,-0.3){$O$} \rput[t](1,-0.3){$x_1$} \rput[t](4,-0.3){$x_2$}
                         \rput[r](-0.1,3.9){$f(x_1)$} \rput[r](-0.1,2.4){$f(x_2)$}
                         \rput[tl](4.5,2.5){$\color{red} \mathcal{C}_f$}
                    \end{pspicture*}
               }
               \\
               Fonction croissante & ~~~~ & Fonction décroissante %
               \\
          \end{tabular}
     \end{extern}
\end{center}
\cadre{bleu}{Définitions}{% id="d50"
     Soit $I$ un intervalle et $x_{0} \in  I$.
     \begin{itemize}
          \item La fonction $f$ admet un \textbf{maximum} en $x_{0}$ sur l'intervalle $I$ si et seulement si pour tout réel $x$ de $I$, $f\left(x\right) \leqslant f\left(x_{0}\right)$. Le maximum de la fonction $f$ sur $I$ est alors $M=f\left(x_{0}\right)$
          \item La fonction $f$ admet un \textbf{minimum} en $x_{0}$ sur l'intervalle $I$ si et seulement si pour tout réel $x$ de $I$, $f\left(x\right) \geqslant f\left(x_{0}\right)$. Le minimum de la fonction $f$ sur $I$ est alors $m=f\left(x_{0}\right)$
     \end{itemize}
}
\begin{h2}II - La fonction racine carrée\end{h2}
\cadre{bleu}{Définition}{% id="d60"
     La fonction racine carrée est la fonction définie sur $\left[0;+\infty \right[$ par $f\left(x\right)=\sqrt{x}$
}
\cadre{vert}{Propriété}{% id="p70"
     La fonction racine carrée est strictement croissante sur $\left[0;+\infty \right[$
}
\begin{center}
     \begin{extern}%width="200" alt="Fonction racine carrée: tableau de variation"
          %:-+-+-+-+- Engendré par : http://math.et.info.free.fr/TikZ/TableauxVariations/
          \begin{tikzpicture}[scale=0.7]
               % Styles
               \tikzstyle{cadre}=[thin]
               \tikzstyle{fleche}=[->,>=latex,thin]
               \tikzstyle{nondefini}=[lightgray]
               % Dimensions Modifiables
               \def\Lrg{1.5}
               \def\HtX{1}
               \def\HtY{0.5}
               % Dimensions Calculées
               \def\lignex{-0.5*\HtX}
               \def\lignef{-1.5*\HtX}
               \def\separateur{-0.5*\Lrg}
               % Largeur du tableau
               \def\gauche{-1.5*\Lrg}
               \def\droite{2.5*\Lrg}
               % Hauteur du tableau
               \def\haut{0.5*\HtX}
               \def\bas{-1.5*\HtX-2*\HtY}
               % Ligne de l'abscisse : x
               \node at (-1*\Lrg,0) {$x$};
               \node at (0*\Lrg,0) {$0$};
               \node at (2*\Lrg,0) {$+\infty$};
               % Ligne de la fonction : f(x)
               \node  at (-1*\Lrg,{-1*\HtX+(-1)*\HtY}) {$\sqrt{x}$};
               \node (f1) at (0*\Lrg,{-1*\HtX+(-2)*\HtY}) {$0$};
               \node (f2) at (2*\Lrg,{-1*\HtX+(0)*\HtY}) {$ $};
               % Flèches
               \draw[fleche] (f1) -- (f2);
               % Encadrement
               \draw[cadre] (\separateur,\haut) -- (\separateur,\bas);
               \draw[cadre] (\gauche,\haut) rectangle  (\droite,\bas);
               \draw[cadre] (\gauche,\lignex) -- (\droite,\lignex);
          \end{tikzpicture}
          %:-+-+-+-+- Fin
     \end{extern}
\end{center}
\begin{center}
     \textit{Tableau de variation de la fonction racine carrée }
\end{center}
\begin{center}
     \begin{extern} %width="450" alt="Fonction  racine carrée  : graphique"
          \resizebox{8.5cm}{!}{%
               % -+-+-+ variables modifiables
               \def\fonction{sqrt(x) }
               \def\xmin{-0.5}
               \def\xmax{8.5}
               \def\ymin{-0.5}
               \def\ymax{5}
               \def\xunit{2}  % unités en cm
               \def\yunit{2}
               \psset{xunit=\xunit,yunit=\yunit,algebraic=true}
               \fontsize{18pt}{18pt}\selectfont
               \begin{pspicture*}[linewidth=1pt](\xmin,\ymin)(\xmax,\ymax)
                    %      \psgrid[gridcolor=mcgris, subgriddiv=5, gridlabels=0pt](\xmin,\ymin)(\xmax,\ymax)
                    \psaxes[linewidth=0.75pt]{->}(0,0)(\xmin,\ymin)(\xmax,\ymax)
                    \psplot[plotpoints=2000,linecolor=blue]{0.000001}{\xmax}{\fonction}
                    \rput[tr](-0.1,-0.1){$O$}
                    \rput[tl](7.5,3.3){$\color{blue} \mathcal{C}$}
               \end{pspicture*}
          }
     \end{extern}
\end{center}
\begin{center}
     \textit{Graphique de la fonction  racine carrée  }
\end{center}
\bloc{cyan}{Remarque}{% id="p80"
     La courbe représentative de la fonction racine carrée est une demi-parabole
}
\begin{h2}III - La fonction cube\end{h2}
\cadre{bleu}{Définition}{% id="d90"
     La fonction cube est la fonction définie sur $\left]-\infty ;+\infty \right[$ par $f\left(x\right)=x^{3}$
}
\cadre{vert}{Propriété}{% id="p100"
     La fonction cube est strictement croissante sur $\left]-\infty ;+\infty \right[$
}
\begin{center}
     \begin{extern}%width="350" alt="tableau de variation de la fonction cube"
          \begin{tikzpicture}[scale=0.875]
               % Styles
               \tikzstyle{cadre}=[thin]
               \tikzstyle{fleche}=[->,>=latex,thin]
               \tikzstyle{nondefini}=[lightgray]
               % Dimensions Modifiables
               \def\Lrg{1.5}
               \def\HtX{1}
               \def\HtY{0.5}
               % Dimensions Calculées
               \def\lignex{-0.5*\HtX}
               \def\lignef{-1.5*\HtX}
               \def\separateur{-0.5*\Lrg}
               % Largeur du tableau
               \def\gauche{-1.5*\Lrg}
               \def\droite{4.5*\Lrg}
               % Hauteur du tableau
               \def\haut{0.5*\HtX}
               \def\bas{-1.5*\HtX-2*\HtY}
               % Pointillés
               \draw[lightgray] (2*\Lrg,\lignex) -- (2*\Lrg,\bas);
               % Ligne de l'abscisse : x
               \node at (-1*\Lrg,0) {$x$};
               \node at (0*\Lrg,0) {$-\infty$};
               \node at (2*\Lrg,0) {$0$};
               \node at (4*\Lrg,0) {$+\infty$};
               % Ligne de la fonction : f(x)
               \node  at (-1*\Lrg,{-1*\HtX+(-1)*\HtY}) {$x^3$};
               \node (f1) at (0*\Lrg,{-1*\HtX+(-2)*\HtY}) {$ $};
               \node (f2) at (2*\Lrg,{-1*\HtX+(-1)*\HtY}) {$0$};
               \node (f3) at (4*\Lrg,{-1*\HtX+(0)*\HtY}) {$ $};
               % Flèches
               \draw (f1) -- (f2);
               \draw[fleche] (f2) -- (f3);
               % Encadrement
               \draw[cadre] (\separateur,\haut) -- (\separateur,\bas);
               \draw[cadre] (\gauche,\haut) rectangle  (\droite,\bas);
               \draw[cadre] (\gauche,\lignex) -- (\droite,\lignex);
          \end{tikzpicture}
     \end{extern}
\end{center}
\begin{center}
     \textit{Tableau de variation de la fonction cube }
\end{center}
\begin{center}
     \begin{extern} %width="400" alt="Fonction  cube : graphique"
          \resizebox{8.5cm}{!}{%
               % -+-+-+ variables modifiables
               \def\fonction{x^3}
               \def\xmin{-4}
               \def\xmax{4}
               \def\ymin{-8}
               \def\ymax{8}
               \def\xunit{2}  % unités en cm
               \def\yunit{1}
               \psset{xunit=\xunit,yunit=\yunit,algebraic=true}
               \fontsize{18pt}{18pt}\selectfont
               \begin{pspicture*}[linewidth=1pt](\xmin,\ymin)(\xmax,\ymax)
                    %      \psgrid[gridcolor=mcgris, subgriddiv=5, gridlabels=0pt](\xmin,\ymin)(\xmax,\ymax)
                    \psaxes[Dy=2,linewidth=0.75pt]{->}(0,0)(\xmin,\ymin)(\xmax,\ymax)
                    \psplot[plotpoints=2000,linecolor=blue]{\xmin}{\xmax}{\fonction}
                    \rput[tr](-0.1,-0.2){$O$}
                    \rput[tl](2.1,7){$\color{blue} \mathcal{C}$}
               \end{pspicture*}
          }
     \end{extern}
\end{center}
\begin{center}
     \textit{Graphique de la fonction  cube }
\end{center}
\bloc{cyan}{Remarques}{% id="r100"
     \begin{itemize}
          \item Comme la fonction $x\mapsto x^{3}$ est strictement croissante sur $\mathbb{R}$ :
          \par
          $a > b  \Leftrightarrow a^{3} > b^{3}$
          \item En particulier $x > 0 \Leftrightarrow x^{3} > 0$
          \par
          Autrement dit, le cube d'un nombre positif est positif et le cube d'un nombre négatif est négatif.
     \end{itemize}
}

\end{document}
µ
\documentclass[a4paper]{article}

%================================================================================================================================
%
% Packages
%
%================================================================================================================================

\usepackage[T1]{fontenc} 	% pour caractères accentués
\usepackage[utf8]{inputenc}  % encodage utf8
\usepackage[french]{babel}	% langue : français
\usepackage{fourier}			% caractères plus lisibles
\usepackage[dvipsnames]{xcolor} % couleurs
\usepackage{fancyhdr}		% réglage header footer
\usepackage{needspace}		% empêcher sauts de page mal placés
\usepackage{graphicx}		% pour inclure des graphiques
\usepackage{enumitem,cprotect}		% personnalise les listes d'items (nécessaire pour ol, al ...)
\usepackage{hyperref}		% Liens hypertexte
\usepackage{pstricks,pst-all,pst-node,pstricks-add,pst-math,pst-plot,pst-tree,pst-eucl} % pstricks
\usepackage[a4paper,includeheadfoot,top=2cm,left=3cm, bottom=2cm,right=3cm]{geometry} % marges etc.
\usepackage{comment}			% commentaires multilignes
\usepackage{amsmath,environ} % maths (matrices, etc.)
\usepackage{amssymb,makeidx}
\usepackage{bm}				% bold maths
\usepackage{tabularx}		% tableaux
\usepackage{colortbl}		% tableaux en couleur
\usepackage{fontawesome}		% Fontawesome
\usepackage{environ}			% environment with command
\usepackage{fp}				% calculs pour ps-tricks
\usepackage{multido}			% pour ps tricks
\usepackage[np]{numprint}	% formattage nombre
\usepackage{tikz,tkz-tab} 			% package principal TikZ
\usepackage{pgfplots}   % axes
\usepackage{mathrsfs}    % cursives
\usepackage{calc}			% calcul taille boites
\usepackage[scaled=0.875]{helvet} % font sans serif
\usepackage{svg} % svg
\usepackage{scrextend} % local margin
\usepackage{scratch} %scratch
\usepackage{multicol} % colonnes
%\usepackage{infix-RPN,pst-func} % formule en notation polanaise inversée
\usepackage{listings}

%================================================================================================================================
%
% Réglages de base
%
%================================================================================================================================

\lstset{
language=Python,   % R code
literate=
{á}{{\'a}}1
{à}{{\`a}}1
{ã}{{\~a}}1
{é}{{\'e}}1
{è}{{\`e}}1
{ê}{{\^e}}1
{í}{{\'i}}1
{ó}{{\'o}}1
{õ}{{\~o}}1
{ú}{{\'u}}1
{ü}{{\"u}}1
{ç}{{\c{c}}}1
{~}{{ }}1
}


\definecolor{codegreen}{rgb}{0,0.6,0}
\definecolor{codegray}{rgb}{0.5,0.5,0.5}
\definecolor{codepurple}{rgb}{0.58,0,0.82}
\definecolor{backcolour}{rgb}{0.95,0.95,0.92}

\lstdefinestyle{mystyle}{
    backgroundcolor=\color{backcolour},   
    commentstyle=\color{codegreen},
    keywordstyle=\color{magenta},
    numberstyle=\tiny\color{codegray},
    stringstyle=\color{codepurple},
    basicstyle=\ttfamily\footnotesize,
    breakatwhitespace=false,         
    breaklines=true,                 
    captionpos=b,                    
    keepspaces=true,                 
    numbers=left,                    
xleftmargin=2em,
framexleftmargin=2em,            
    showspaces=false,                
    showstringspaces=false,
    showtabs=false,                  
    tabsize=2,
    upquote=true
}

\lstset{style=mystyle}


\lstset{style=mystyle}
\newcommand{\imgdir}{C:/laragon/www/newmc/assets/imgsvg/}
\newcommand{\imgsvgdir}{C:/laragon/www/newmc/assets/imgsvg/}

\definecolor{mcgris}{RGB}{220, 220, 220}% ancien~; pour compatibilité
\definecolor{mcbleu}{RGB}{52, 152, 219}
\definecolor{mcvert}{RGB}{125, 194, 70}
\definecolor{mcmauve}{RGB}{154, 0, 215}
\definecolor{mcorange}{RGB}{255, 96, 0}
\definecolor{mcturquoise}{RGB}{0, 153, 153}
\definecolor{mcrouge}{RGB}{255, 0, 0}
\definecolor{mclightvert}{RGB}{205, 234, 190}

\definecolor{gris}{RGB}{220, 220, 220}
\definecolor{bleu}{RGB}{52, 152, 219}
\definecolor{vert}{RGB}{125, 194, 70}
\definecolor{mauve}{RGB}{154, 0, 215}
\definecolor{orange}{RGB}{255, 96, 0}
\definecolor{turquoise}{RGB}{0, 153, 153}
\definecolor{rouge}{RGB}{255, 0, 0}
\definecolor{lightvert}{RGB}{205, 234, 190}
\setitemize[0]{label=\color{lightvert}  $\bullet$}

\pagestyle{fancy}
\renewcommand{\headrulewidth}{0.2pt}
\fancyhead[L]{maths-cours.fr}
\fancyhead[R]{\thepage}
\renewcommand{\footrulewidth}{0.2pt}
\fancyfoot[C]{}

\newcolumntype{C}{>{\centering\arraybackslash}X}
\newcolumntype{s}{>{\hsize=.35\hsize\arraybackslash}X}

\setlength{\parindent}{0pt}		 
\setlength{\parskip}{3mm}
\setlength{\headheight}{1cm}

\def\ebook{ebook}
\def\book{book}
\def\web{web}
\def\type{web}

\newcommand{\vect}[1]{\overrightarrow{\,\mathstrut#1\,}}

\def\Oij{$\left(\text{O}~;~\vect{\imath},~\vect{\jmath}\right)$}
\def\Oijk{$\left(\text{O}~;~\vect{\imath},~\vect{\jmath},~\vect{k}\right)$}
\def\Ouv{$\left(\text{O}~;~\vect{u},~\vect{v}\right)$}

\hypersetup{breaklinks=true, colorlinks = true, linkcolor = OliveGreen, urlcolor = OliveGreen, citecolor = OliveGreen, pdfauthor={Didier BONNEL - https://www.maths-cours.fr} } % supprime les bordures autour des liens

\renewcommand{\arg}[0]{\text{arg}}

\everymath{\displaystyle}

%================================================================================================================================
%
% Macros - Commandes
%
%================================================================================================================================

\newcommand\meta[2]{    			% Utilisé pour créer le post HTML.
	\def\titre{titre}
	\def\url{url}
	\def\arg{#1}
	\ifx\titre\arg
		\newcommand\maintitle{#2}
		\fancyhead[L]{#2}
		{\Large\sffamily \MakeUppercase{#2}}
		\vspace{1mm}\textcolor{mcvert}{\hrule}
	\fi 
	\ifx\url\arg
		\fancyfoot[L]{\href{https://www.maths-cours.fr#2}{\black \footnotesize{https://www.maths-cours.fr#2}}}
	\fi 
}


\newcommand\TitreC[1]{    		% Titre centré
     \needspace{3\baselineskip}
     \begin{center}\textbf{#1}\end{center}
}

\newcommand\newpar{    		% paragraphe
     \par
}

\newcommand\nosp {    		% commande vide (pas d'espace)
}
\newcommand{\id}[1]{} %ignore

\newcommand\boite[2]{				% Boite simple sans titre
	\vspace{5mm}
	\setlength{\fboxrule}{0.2mm}
	\setlength{\fboxsep}{5mm}	
	\fcolorbox{#1}{#1!3}{\makebox[\linewidth-2\fboxrule-2\fboxsep]{
  		\begin{minipage}[t]{\linewidth-2\fboxrule-4\fboxsep}\setlength{\parskip}{3mm}
  			 #2
  		\end{minipage}
	}}
	\vspace{5mm}
}

\newcommand\CBox[4]{				% Boites
	\vspace{5mm}
	\setlength{\fboxrule}{0.2mm}
	\setlength{\fboxsep}{5mm}
	
	\fcolorbox{#1}{#1!3}{\makebox[\linewidth-2\fboxrule-2\fboxsep]{
		\begin{minipage}[t]{1cm}\setlength{\parskip}{3mm}
	  		\textcolor{#1}{\LARGE{#2}}    
 	 	\end{minipage}  
  		\begin{minipage}[t]{\linewidth-2\fboxrule-4\fboxsep}\setlength{\parskip}{3mm}
			\raisebox{1.2mm}{\normalsize\sffamily{\textcolor{#1}{#3}}}						
  			 #4
  		\end{minipage}
	}}
	\vspace{5mm}
}

\newcommand\cadre[3]{				% Boites convertible html
	\par
	\vspace{2mm}
	\setlength{\fboxrule}{0.1mm}
	\setlength{\fboxsep}{5mm}
	\fcolorbox{#1}{white}{\makebox[\linewidth-2\fboxrule-2\fboxsep]{
  		\begin{minipage}[t]{\linewidth-2\fboxrule-4\fboxsep}\setlength{\parskip}{3mm}
			\raisebox{-2.5mm}{\sffamily \small{\textcolor{#1}{\MakeUppercase{#2}}}}		
			\par		
  			 #3
 	 		\end{minipage}
	}}
		\vspace{2mm}
	\par
}

\newcommand\bloc[3]{				% Boites convertible html sans bordure
     \needspace{2\baselineskip}
     {\sffamily \small{\textcolor{#1}{\MakeUppercase{#2}}}}    
		\par		
  			 #3
		\par
}

\newcommand\CHelp[1]{
     \CBox{Plum}{\faInfoCircle}{À RETENIR}{#1}
}

\newcommand\CUp[1]{
     \CBox{NavyBlue}{\faThumbsOUp}{EN PRATIQUE}{#1}
}

\newcommand\CInfo[1]{
     \CBox{Sepia}{\faArrowCircleRight}{REMARQUE}{#1}
}

\newcommand\CRedac[1]{
     \CBox{PineGreen}{\faEdit}{BIEN R\'EDIGER}{#1}
}

\newcommand\CError[1]{
     \CBox{Red}{\faExclamationTriangle}{ATTENTION}{#1}
}

\newcommand\TitreExo[2]{
\needspace{4\baselineskip}
 {\sffamily\large EXERCICE #1\ (\emph{#2 points})}
\vspace{5mm}
}

\newcommand\img[2]{
          \includegraphics[width=#2\paperwidth]{\imgdir#1}
}

\newcommand\imgsvg[2]{
       \begin{center}   \includegraphics[width=#2\paperwidth]{\imgsvgdir#1} \end{center}
}


\newcommand\Lien[2]{
     \href{#1}{#2 \tiny \faExternalLink}
}
\newcommand\mcLien[2]{
     \href{https~://www.maths-cours.fr/#1}{#2 \tiny \faExternalLink}
}

\newcommand{\euro}{\eurologo{}}

%================================================================================================================================
%
% Macros - Environement
%
%================================================================================================================================

\newenvironment{tex}{ %
}
{%
}

\newenvironment{indente}{ %
	\setlength\parindent{10mm}
}

{
	\setlength\parindent{0mm}
}

\newenvironment{corrige}{%
     \needspace{3\baselineskip}
     \medskip
     \textbf{\textsc{Corrigé}}
     \medskip
}
{
}

\newenvironment{extern}{%
     \begin{center}
     }
     {
     \end{center}
}

\NewEnviron{code}{%
	\par
     \boite{gray}{\texttt{%
     \BODY
     }}
     \par
}

\newenvironment{vbloc}{% boite sans cadre empeche saut de page
     \begin{minipage}[t]{\linewidth}
     }
     {
     \end{minipage}
}
\NewEnviron{h2}{%
    \needspace{3\baselineskip}
    \vspace{0.6cm}
	\noindent \MakeUppercase{\sffamily \large \BODY}
	\vspace{1mm}\textcolor{mcgris}{\hrule}\vspace{0.4cm}
	\par
}{}

\NewEnviron{h3}{%
    \needspace{3\baselineskip}
	\vspace{5mm}
	\textsc{\BODY}
	\par
}

\NewEnviron{margeneg}{ %
\begin{addmargin}[-1cm]{0cm}
\BODY
\end{addmargin}
}

\NewEnviron{html}{%
}

\begin{document}
\meta{url}{/cours/derivees/}
\meta{pid}{360}
\meta{titre}{Dérivées en Première ES et L}
\meta{type}{cours}
\begin{h2}I - Nombre dérivé\end{h2}
\cadre{bleu}{Définition}{%id="d10"
     Soit $f$ une fonction définie sur un intervalle $I$ et $a$ et $b$ deux réels appartenant à $I$.
     \par
     On appelle \textbf{taux d'accroissement} de $f$ entre $a$ et $b$ le nombre :
     \par
     $T=\frac{f\left(b\right)-f\left(a\right)}{b-a}$
}
\bloc{cyan}{Remarque}{%id="r10"
     En faisant le changement de variable : $b=a+h$ ($h$ représente alors l'écart entre $b$ et $a$), ce taux s'écrit aussi :
     \par
     $T=\frac{f\left(a+h\right)-f\left(a\right)}{h}$
}
\bloc{cyan}{Interprétation graphique}{%id="i10"
     \begin{center}
          \begin{extern}%width="450" alt="taux d'accroissement"
               % -+-+-+ variables modifiables
               \resizebox{8cm}{!}{%
                    \def\xmin{-2.5}
                    \def\xmax{8.5}
                    \def\ymin{-1.5}
                    \def\ymax{8.5}
                    \def\xunit{1}  % unités en cm
                    \def\yunit{1}
                    \psset{xunit=\xunit,yunit=\yunit,algebraic=true}
                    \fontsize{12pt}{12pt}\selectfont
                    \begin{pspicture*}[linewidth=1pt](\xmin,\ymin)(\xmax,\ymax)
                         %      \psgrid[gridcolor=mcgris,subgriddiv=0](-4,-2)(9,9)
                         %        \psaxes[linewidth=0.75pt]{->}(0,0)(\xmin,\ymin)(\xmax,\ymax)
                         \psline[linecolor=gray]{->}(\xmin,0)(\xmax,0)
                         \psline[linecolor=gray]{->}(0,\ymin)(0,\ymax)
                         \psline[linecolor=lightgray](2,0)(2,1.69)
                         \psline[linecolor=lightgray](5,0)(5,3.713)
                         \rput[tr](-0.2,-0.2){$O$}\rput[br](1.8,1.8){$A$}\rput[br](4.8,3.693){$B$} \rput[t](3.5,-0.8){\color{mcmauve} $h$}
                         \rput[t](8,7.5){\color{blue} $\mathscr{C}_f$}\rput[t](2,-0.2){$a$}\rput[t](5,-0.1){$b$}
                         \psdots(2,1.69)\psdots(5,3.713)
                         \psplot[plotpoints=1000,linewidth=0.8pt,linecolor=blue]{\xmin}{\xmax}{1.3^x}
                         \psplot[plotpoints=1000,linewidth=0.8pt,linecolor=mcvert]{\xmin}{\xmax}{0.674*x+0.341}
                         \psline[linecolor=mcmauve]{<->}(2,-0.7)(5,-0.7)
                    \end{pspicture*}
               }
          \end{extern}
     \end{center}
     Le taux d'accroissement de $f$ entre $a$ et $b$ est le \textbf{coefficient directeur} de la droite $(AB)$.
}
\cadre{bleu}{Définition}{%id="d20"
     Soit $f$ une fonction définie sur un intervalle ouvert $I$ contenant $a$.
     \par
     On dit que $f$ est dérivable en $a$ si et seulement si le rapport $\frac{f\left(a+h\right)-f\left(a\right)}{h}$ tend vers un nombre réel lorsque $h$ tend vers zéro.
     \par
     Ce nombre s'appelle le nombre dérivé de $f$ en $a$ et se note $f^{\prime}\left(a\right)$.
}
\bloc{orange}{Exemple}{%id="e20"
     Calculons le nombre dérivé de la fonction $f : x\mapsto x^{2}$ pour $x=1$.
     \par
     $ \frac{f\left(1+h\right)-f\left(1\right)}{h}=\frac{\left(1+h\right)^{2}-1^{2}}{h}=\frac{1+2h+h^{2}-1^{2}}{h}=\frac{2h+h^{2}}{h}=2+h$
     \par
     Or quand $h$ tend vers $0$, $2+h$ tend vers 2; donc $f^{\prime}\left(1\right)=2$.
}
\bloc{cyan}{Interprétation graphique}{%id="i20"
     \begin{center}
          \begin{extern}%width="450" alt="nombre dérivé"
               % -+-+-+ variables modifiables
               \resizebox{8cm}{!}{%
                    \def\xmin{-2.5}
                    \def\xmax{8.5}
                    \def\ymin{-1.5}
                    \def\ymax{8.5}
                    \def\xunit{1}  % unités en cm
                    \def\yunit{1}
                    \psset{xunit=\xunit,yunit=\yunit,algebraic=true}
                    \fontsize{12pt}{12pt}\selectfont
                    \begin{pspicture*}[linewidth=1pt](\xmin,\ymin)(\xmax,\ymax)
                         %      \psgrid[gridcolor=mcgris,subgriddiv=0](-4,-2)(9,9)
                         %        \psaxes[linewidth=0.75pt]{->}(0,0)(\xmin,\ymin)(\xmax,\ymax)
                         \psline[linecolor=gray]{->}(\xmin,0)(\xmax,0)
                         \psline[linecolor=gray]{->}(0,\ymin)(0,\ymax)
                         \psline[linecolor=lightgray](2,0)(2,1.69)
                         \psline[linecolor=lightgray](5,0)(5,3.713)
                         \rput[tr](-0.2,-0.2){$O$}\rput[br](1.8,1.8){$A$}\rput[br](4.8,3.693){$B$}\rput[t](3.5,-0.8){\color{mcmauve} $h$}
                         \rput[t](8,4){\color{red} $\mathscr{T}$}\rput[t](8,7.5){\color{blue} $\mathscr{C}_f$}\rput[t](2,-0.2){$a$}\rput[t](5,-0.1){$a+h$}
                         \psdots(2,1.69)\psdots(5,3.713)
                         \psplot[plotpoints=1000,linewidth=0.8pt,linecolor=blue]{\xmin}{\xmax}{1.3^x}
                         \psplot[plotpoints=1000,linewidth=0.8pt,linecolor=mcvert]{\xmin}{\xmax}{0.674*x+0.341}
                         \psplot[plotpoints=1000,linewidth=0.8pt,linecolor=red]{\xmin}{\xmax}{0.443*x+0.803}
                         \psline[linecolor=mcmauve]{<->}(2,-0.7)(5,-0.7)
                    \end{pspicture*}
               }
          \end{extern}
     \end{center}
     Lorsque $h$ se rapproche de zéro, le point $B$ se rapproche du point $A$ et la droite $\left(AB\right)$ se rapproche de la tangente $\mathscr{T}$
}
\cadre{vert}{Propriété}{%id="p30"
     Soit $f$ une fonction dérivable en $a$ de courbe représentative $C_{f}$ . $f^{\prime}\left(a\right)$ représente le \textbf{coefficient directeur} de la \textbf{tangente} à la courbe $C_{f}$ au point d'abscisse $a$.
}
\cadre{vert}{Propriété}{%id="p40"
     Soit $f$ une fonction dérivable en $a$ de courbe représentative $C_{f}$ . L'équation de la tangente à $C_{f}$ au point d'abscisse $a$ est :
     \par
     $y=f^{\prime}\left(a\right)\left(x-a\right)+f\left(a\right)$
}
\bloc{orange}{Démonstration}{%id="m40"
     D'après la propriété précédente, la tangente à $C_{f}$ au point d'abscisse $a$ est une droite de coefficient directeur $f^{\prime}\left(a\right)$. Son équation est donc de la forme :
     \par
     $y=f^{\prime}\left(a\right)x+b$
     \par
     On sait que la tangente passe par le point $A$ de coordonnées $\left(a; f\left(a\right)\right)$ donc :
     \par
     $f\left(a\right)=f^{\prime}\left(a\right)a+b$
     \par
     $b=-f^{\prime}\left(a\right)a+f\left(a\right)$
     \par
     L'équation de la tangente est donc :
     \par
     $y=f^{\prime}\left(a\right)x-f^{\prime}\left(a\right)a+f\left(a\right)$
     \par
     Soit :
     \par
     $y=f^{\prime}\left(a\right)\left(x-a\right)+f\left(a\right)$
}
\bloc{orange}{Exemple}{%id="e40"
     Cherchons l'équation de la tangente à la courbe représentative de la fonction $f : x\mapsto x^{2}$ au point d'abscisse $a=1$.
     \par
     On a $f\left(1\right)=1^{2}=1$ et on a vu dans l'exemple précédent que $f^{\prime}\left(1\right)=2$.
     \par
     L'équation cherchée est donc :
     \par
     $y=2\left(x-1\right)+1$
     \par
     soit :
     \par
     $y=2x-1$
     \begin{center}
          \begin{extern}%width="380" alt="courbe et tangente"
               % -+-+-+ variables modifiables
               \resizebox{7cm}{!}{%
                    \def\xmin{-2.5}
                    \def\xmax{2.5}
                    \def\ymin{-1.5}
                    \def\ymax{3.8}
                    \def\xunit{2}  % unités en cm
                    \def\yunit{2}
                    \psset{xunit=\xunit,yunit=\yunit,algebraic=true}
                    \fontsize{12pt}{12pt}\selectfont
                    \begin{pspicture*}[linewidth=1pt](\xmin,\ymin)(\xmax,\ymax)
                         %      \psgrid[gridcolor=mcgris,subgriddiv=0](-4,-2)(9,9)
                         \psaxes[linewidth=0.75pt]{->}(0,0)(\xmin,\ymin)(\xmax,\ymax)
                         \rput[tr](-0.2,-0.2){$O$}  \rput[l](1.1,1){$A$}
                         \rput[t](1.5,3.3){\color{blue} $\mathcal{C}_f$}\rput[t](2.4,3.3){$\color{mcvert} \mathcal{T}$}
                         \psplot[plotpoints=1000,linecolor=blue]{\xmin}{\xmax}{x^2}
                         \psplot[plotpoints=1000,linecolor=mcvert]{\xmin}{\xmax}{2*x-1}
                         \psdots(1,1)
                         \psline[linecolor=lightgray](1,0)(1,1)
                    \end{pspicture*}
               }
          \end{extern}
     \end{center}
}
\begin{h2}II - Fonction dérivée\end{h2}
\cadre{bleu}{Définition}{%id="d50"
     Si $f$ est définie sur un intervalle $I$ et si le nombre dérivé existe en chaque point de $I$, on dit que $f$ est \textbf{dérivable} sur $I$. La fonction qui a $x$ associe le nombre dérivé de $f$ en $x$ s'appelle \textbf{fonction dérivée} de $f$ et se note $f^{\prime}$
}
\cadre{vert}{Dérivée des fonctions usuelles}{%id="p60"
     \begin{center}
          \begin{tabularx}{0.8\linewidth}{|X|X|X|}%class="compact" width="600"
               \hline
               \textbf{Fonction} & \textbf{Dérivée} & \textbf{Ensemble de dérivabilité}
               \\ \hline
               $k$  $\left(k\in \mathbb{R}\right)$  &  $0$  &  $\mathbb{R}$
               \\ \hline
               $x$ &  $1$  &  $\mathbb{R}$
               \\ \hline
               $x^{n}$ $\left(n\in \mathbb{N}\right)$  &  $nx^{n-1}$  &  $\mathbb{R}$
               \\ \hline
               $\frac{1}{x}$ &  $-\frac{1}{x^{2}}$  &  $\mathbb{R}-\left\{0\right\}$
               \\ \hline
               $\sqrt{x}$ &  $\frac{1}{2\sqrt{x}}$  &  $\left]0;+\infty \right[$
               \\        \hline
          \end{tabularx}
     \end{center}
}
\cadre{vert}{Opérations}{%id="p70"
     Si $u$ et $v$ sont 2 \textbf{fonctions} dérivables :
     \begin{center}
          \begin{tabularx}{0.8\linewidth}{|*{3}{>{\centering \arraybackslash }X|}}%class="compact" width="600"
               \hline
               \textbf{Fonction} & \textbf{Dérivée}
               \\ \hline
               $u+v$  &  $u^{\prime}+v^{\prime}$
               \\ \hline
               $ku$  $\left(k\in \mathbb{R}\right)$  &  $ku^{\prime} $
               \\ \hline
               $\frac{1}{u}$ (avec $u\left(x\right)\neq 0$ sur $I$)  &  $-\frac{u^{\prime} }{u^{2}} $
               \\ \hline
               $uv$  &  $u^{\prime}v+uv^{\prime}$
               \\ \hline
               $\frac{u}{v}$  (avec $v\left(x\right)\neq 0$ sur $I$) &  $\frac{u^{\prime}v-uv^{\prime}}{v^{2}} $
               \\ \hline
          \end{tabularx}
     \end{center}
}
\bloc{orange}{Exemples}{%id="e70"
     \begin{itemize}
          \item On cherche à calculer la dérivée de la fonction $f$ définie sur $\mathbb{R}\backslash\left\{0\right\}$ par $f\left(x\right)=x^{2}+\frac{1}{x}$
          \par
          $f$ est la somme des fonctions $u$ et $v$ définies par $u\left(x\right)=x^{2}$ et $v\left(x\right)=\frac{1}{x}$
          \par
          $u^{\prime}\left(x\right)=2x$ et $v^{\prime}\left(x\right)=-\frac{1}{x^{2}}$
          \par
          donc $f^{\prime}\left(x\right)=2x-\frac{1}{x^{2}}$
          \item Soit la fonction $g$ définie sur $\mathbb{R}$ par $g\left(x\right)=\frac{x^{3}-1}{x^{2}+1}$
          \par
          $g$ est le quotient des fonctions $u$ et $v$ définies par $u\left(x\right)=x^{3}-1$ et $v\left(x\right)=x^{2}+1$
          \par
          $u^{\prime}\left(x\right)=3x^{2}+0=3x^{2}$ et $v^{\prime}\left(x\right)=2x+0=2x$
          \par
          $g^{\prime}\left(x\right)=\frac{u^{\prime}\left(x\right)v\left(x\right)-u\left(x\right)v^{\prime}\left(x\right)}{v\left(x\right)^{2}}=\frac{\left(3x^{2}\right)\left(x^{2}+1\right)-\left(x^{3}-1\right)\times 2x}{\left(x^{2}+1\right)^{2}}=\frac{x^{4}+3x^{2}+2x}{\left(x^{2}+1\right)^{2}}$
          \item Soit enfin la fonction $h$ définie sur $\left]1;+\infty \right[$ par $h\left(x\right)=\frac{3}{x^{2}-1}$
          \par
          On pourrait utiliser la formule $\left(\frac{u}{v}\right)^{\prime}=\frac{u^{\prime}v-uv^{\prime}}{v^{2}}$ comme précédemment mais cela ne sera pas très judicieux. En effet, le numérateur étant constant, il y a une manière plus rapide de procéder. Il suffit d'écrire :
          \par
          $h\left(x\right)=3\times \frac{1}{x^{2}-1}$
          \par
          et d'appliquer la formule $\left(\frac{1}{v}\right)^{\prime} = -\frac{v^{\prime} }{v^{2}}$ avec $v\left(x\right)=x^{2}-1$ (donc $v^{\prime}\left(x\right)=2x$)
          \par
          On obtient :
          \par
          $h^{\prime}\left(x\right)=3\times -\frac{2x}{\left(x^{2}-1\right)^{2}}=-\frac{6x}{\left(x^{2}-1\right)^{2}}$
     \end{itemize}
}
\begin{h2}III - Applications de la dérivée\end{h2}
\boite{gray}{%
     Si nécessaire, revoir la notion de \mcLien{/cours/premiere-es/etude-de-fonctions\#d40}{sens de variation d'une fonction}.
}
\cadre{rouge}{Théorème}{%id="t80"
     Soit $f$ une fonction dérivable sur un intervalle $I$, $f$ est \textbf{croissante} sur $I$ si et seulement si $f^{\prime}\left(x\right)$ est \textbf{positif ou nul} pour tout $x \in I$.
     \par
     De plus si $f^{\prime}\left(x\right)$ est \textbf{strictement positive} sur $I$, sauf éventuellement en quelques points, alors $f$ est \textbf{strictement croissante} sur $I$.
}
\bloc{orange}{Exemple}{%id="e80"
     Soit la fonction $f$ définie sur $\left[-1;1\right]$ par $f\left(x\right)=x^{3}$.
     \par
     $f^{\prime}\left(x\right)=3x^{2}$ est positive ou nulle sur $\left[-1;1\right]$, donc $f$ est \textbf{croissante} sur $\left[-1;1\right]$.
     \par
     Comme par ailleurs, $f^{\prime}$ est strictement positive sauf pour $x=0$, $f$ est \textbf{strictement croissante} sur $\left[-1;1\right]$.
     \begin{center}
          \begin{extern} %width="280" alt="Fonction  cube sur [-1;1]"
               \resizebox{6cm}{!}{%
                    % -+-+-+ variables modifiables
                    \def\fonction{x^3}
                    \def\xmin{-1.2}
                    \def\xmax{1.4}
                    \def\ymin{-1.2}
                    \def\ymax{1.2}
                    \def\xunit{4}  % unités en cm
                    \def\yunit{4}
                    \psset{xunit=\xunit,yunit=\yunit,algebraic=true}
                    \fontsize{18pt}{18pt}\selectfont
                    \begin{pspicture*}[linewidth=1pt](\xmin,\ymin)(\xmax,\ymax)
                         %      \psgrid[gridcolor=mcgris, subgriddiv=5, gridlabels=0pt](\xmin,\ymin)(\xmax,\ymax)
                         \psaxes[linewidth=0.75pt]{->}(0,0)(\xmin,\ymin)(\xmax,\ymax)
                         \psplot[plotpoints=2000,linecolor=blue]{\xmin}{\xmax}{\fonction}
                         \rput[tr](-0.1,-0.2){$O$}
                         \rput[tl](1.1,1){$\color{blue} \mathcal{C}$}
                    \end{pspicture*}
               }
          \end{extern}
\end{center}}
\begin{center}
     \textit{Fonction  cube sur $[-1;1]$}
\end{center}
On a un théorème analogue si la dérivée est négative :
\cadre{rouge}{Théorème}{%id="t90"
     Soit $f$ une fonction dérivable sur un intervalle $I$, $f$ est \textbf{décroissante} sur $I$ si et seulement si $f^{\prime}\left(x\right)$ est \textbf{négatif ou nul} pour tout $x \in I$.
     \par
     De plus si $f^{\prime}\left(x\right)$ est \textbf{strictement négative} sur $I$, sauf éventuellement en quelques points, alors $f$ est \textbf{strictement décroissante} sur $I$.
}
\bloc{cyan}{Remarques}{%id="r90"
     \begin{itemize}
          \item Si $f$ est dérivable, les théorèmes précédents montre que l'étude des variations de $f$ se ramène à l'étude du \textbf{signe de la dérivée.}
          \item On regroupe couramment le tableau de signe de la dérivée et le tableau de variations de $f$ dans un même tableau à 3 lignes (voir exemple ci-dessous)
          \item Pour montrer qu'une fonction $f$ admet un maximum en $a$, on peut montrer que $f$ est croissante pour $x < a$ et décroissante pour $x > a$ ; c'est à dire, si $f$ est dérivable, que $f^{\prime}$ est positive pour $x < a$ et négative pour $x > a$.
          \begin{center}
               \begin{extern}%width="340" alt="Maximum en a"
                    \begin{tikzpicture}[scale=0.875]
                         % Styles
                         \tikzstyle{cadre}=[thin]
                         \tikzstyle{fleche}=[->,>=latex,thin]
                         \tikzstyle{nondefini}=[lightgray]
                         % Dimensions Modifiables
                         \def\Lrg{1.5}
                         \def\HtX{1}
                         \def\HtY{0.5}
                         % Dimensions Calculées
                         \def\lignex{-0.5*\HtX}
                         \def\lignef{-1.5*\HtX}
                         \def\separateur{-0.5*\Lrg}
                         % Largeur du tableau
                         \def\gauche{-1.5*\Lrg}
                         \def\droite{4.5*\Lrg}
                         % Hauteur du tableau
                         \def\haut{0.5*\HtX}
                         \def\bas{-2.5*\HtX-2*\HtY}
                         % Pointillés
                         \draw[lightgray] (2*\Lrg,\lignex) -- (2*\Lrg,\lignef);
                         \draw[lightgray] (2*\Lrg,\lignef) -- (2*\Lrg,\bas);
                         % Ligne de l'abscisse : x
                         \node at (-1*\Lrg,0) {$x$};
                         \node at (0*\Lrg,0) {$ $};
                         \node at (2*\Lrg,0) {$a$};
                         \node at (4*\Lrg,0) {$ $};
                         % Ligne de la dérivée : f'(x)
                         \node at (-1*\Lrg,-1*\HtX) {$f'(x)$};
                         \node at (0*\Lrg,-1*\HtX) {$ $};
                         \node at (1*\Lrg,-1*\HtX) {$+$};
                         \node at (2*\Lrg,-1*\HtX) {$0$};
                         \node at (3*\Lrg,-1*\HtX) {$-$};
                         \node at (4*\Lrg,-1*\HtX) {$ $};
                         % Ligne de la fonction : f(x)
                         \node  at (-1*\Lrg,{-2*\HtX+(-1)*\HtY}) {$f(x)$};
                         \node (f1) at (0*\Lrg,{-2*\HtX+(-2)*\HtY}) {$ $};
                         \node (f2) at (2*\Lrg,{-2*\HtX+(0)*\HtY}) {$\text{max}$};
                         \node (f3) at (4*\Lrg,{-2*\HtX+(-2)*\HtY}) {$ $};
                         % Flèches
                         \draw[fleche] (f1) -- (f2);
                         \draw[fleche] (f2) -- (f3);
                         % Encadrement
                         \draw[cadre] (\separateur,\haut) -- (\separateur,\bas);
                         \draw[cadre] (\gauche,\haut) rectangle  (\droite,\bas);
                         \draw[cadre] (\gauche,\lignex) -- (\droite,\lignex);
                         \draw[cadre] (\gauche,\lignef) -- (\droite,\lignef);
                    \end{tikzpicture}
               \end{extern}
          \end{center}
     \end{itemize}
}

\end{document}
µ
\documentclass[a4paper]{article}

%================================================================================================================================
%
% Packages
%
%================================================================================================================================

\usepackage[T1]{fontenc} 	% pour caractères accentués
\usepackage[utf8]{inputenc}  % encodage utf8
\usepackage[french]{babel}	% langue : français
\usepackage{fourier}			% caractères plus lisibles
\usepackage[dvipsnames]{xcolor} % couleurs
\usepackage{fancyhdr}		% réglage header footer
\usepackage{needspace}		% empêcher sauts de page mal placés
\usepackage{graphicx}		% pour inclure des graphiques
\usepackage{enumitem,cprotect}		% personnalise les listes d'items (nécessaire pour ol, al ...)
\usepackage{hyperref}		% Liens hypertexte
\usepackage{pstricks,pst-all,pst-node,pstricks-add,pst-math,pst-plot,pst-tree,pst-eucl} % pstricks
\usepackage[a4paper,includeheadfoot,top=2cm,left=3cm, bottom=2cm,right=3cm]{geometry} % marges etc.
\usepackage{comment}			% commentaires multilignes
\usepackage{amsmath,environ} % maths (matrices, etc.)
\usepackage{amssymb,makeidx}
\usepackage{bm}				% bold maths
\usepackage{tabularx}		% tableaux
\usepackage{colortbl}		% tableaux en couleur
\usepackage{fontawesome}		% Fontawesome
\usepackage{environ}			% environment with command
\usepackage{fp}				% calculs pour ps-tricks
\usepackage{multido}			% pour ps tricks
\usepackage[np]{numprint}	% formattage nombre
\usepackage{tikz,tkz-tab} 			% package principal TikZ
\usepackage{pgfplots}   % axes
\usepackage{mathrsfs}    % cursives
\usepackage{calc}			% calcul taille boites
\usepackage[scaled=0.875]{helvet} % font sans serif
\usepackage{svg} % svg
\usepackage{scrextend} % local margin
\usepackage{scratch} %scratch
\usepackage{multicol} % colonnes
%\usepackage{infix-RPN,pst-func} % formule en notation polanaise inversée
\usepackage{listings}

%================================================================================================================================
%
% Réglages de base
%
%================================================================================================================================

\lstset{
language=Python,   % R code
literate=
{á}{{\'a}}1
{à}{{\`a}}1
{ã}{{\~a}}1
{é}{{\'e}}1
{è}{{\`e}}1
{ê}{{\^e}}1
{í}{{\'i}}1
{ó}{{\'o}}1
{õ}{{\~o}}1
{ú}{{\'u}}1
{ü}{{\"u}}1
{ç}{{\c{c}}}1
{~}{{ }}1
}


\definecolor{codegreen}{rgb}{0,0.6,0}
\definecolor{codegray}{rgb}{0.5,0.5,0.5}
\definecolor{codepurple}{rgb}{0.58,0,0.82}
\definecolor{backcolour}{rgb}{0.95,0.95,0.92}

\lstdefinestyle{mystyle}{
    backgroundcolor=\color{backcolour},   
    commentstyle=\color{codegreen},
    keywordstyle=\color{magenta},
    numberstyle=\tiny\color{codegray},
    stringstyle=\color{codepurple},
    basicstyle=\ttfamily\footnotesize,
    breakatwhitespace=false,         
    breaklines=true,                 
    captionpos=b,                    
    keepspaces=true,                 
    numbers=left,                    
xleftmargin=2em,
framexleftmargin=2em,            
    showspaces=false,                
    showstringspaces=false,
    showtabs=false,                  
    tabsize=2,
    upquote=true
}

\lstset{style=mystyle}


\lstset{style=mystyle}
\newcommand{\imgdir}{C:/laragon/www/newmc/assets/imgsvg/}
\newcommand{\imgsvgdir}{C:/laragon/www/newmc/assets/imgsvg/}

\definecolor{mcgris}{RGB}{220, 220, 220}% ancien~; pour compatibilité
\definecolor{mcbleu}{RGB}{52, 152, 219}
\definecolor{mcvert}{RGB}{125, 194, 70}
\definecolor{mcmauve}{RGB}{154, 0, 215}
\definecolor{mcorange}{RGB}{255, 96, 0}
\definecolor{mcturquoise}{RGB}{0, 153, 153}
\definecolor{mcrouge}{RGB}{255, 0, 0}
\definecolor{mclightvert}{RGB}{205, 234, 190}

\definecolor{gris}{RGB}{220, 220, 220}
\definecolor{bleu}{RGB}{52, 152, 219}
\definecolor{vert}{RGB}{125, 194, 70}
\definecolor{mauve}{RGB}{154, 0, 215}
\definecolor{orange}{RGB}{255, 96, 0}
\definecolor{turquoise}{RGB}{0, 153, 153}
\definecolor{rouge}{RGB}{255, 0, 0}
\definecolor{lightvert}{RGB}{205, 234, 190}
\setitemize[0]{label=\color{lightvert}  $\bullet$}

\pagestyle{fancy}
\renewcommand{\headrulewidth}{0.2pt}
\fancyhead[L]{maths-cours.fr}
\fancyhead[R]{\thepage}
\renewcommand{\footrulewidth}{0.2pt}
\fancyfoot[C]{}

\newcolumntype{C}{>{\centering\arraybackslash}X}
\newcolumntype{s}{>{\hsize=.35\hsize\arraybackslash}X}

\setlength{\parindent}{0pt}		 
\setlength{\parskip}{3mm}
\setlength{\headheight}{1cm}

\def\ebook{ebook}
\def\book{book}
\def\web{web}
\def\type{web}

\newcommand{\vect}[1]{\overrightarrow{\,\mathstrut#1\,}}

\def\Oij{$\left(\text{O}~;~\vect{\imath},~\vect{\jmath}\right)$}
\def\Oijk{$\left(\text{O}~;~\vect{\imath},~\vect{\jmath},~\vect{k}\right)$}
\def\Ouv{$\left(\text{O}~;~\vect{u},~\vect{v}\right)$}

\hypersetup{breaklinks=true, colorlinks = true, linkcolor = OliveGreen, urlcolor = OliveGreen, citecolor = OliveGreen, pdfauthor={Didier BONNEL - https://www.maths-cours.fr} } % supprime les bordures autour des liens

\renewcommand{\arg}[0]{\text{arg}}

\everymath{\displaystyle}

%================================================================================================================================
%
% Macros - Commandes
%
%================================================================================================================================

\newcommand\meta[2]{    			% Utilisé pour créer le post HTML.
	\def\titre{titre}
	\def\url{url}
	\def\arg{#1}
	\ifx\titre\arg
		\newcommand\maintitle{#2}
		\fancyhead[L]{#2}
		{\Large\sffamily \MakeUppercase{#2}}
		\vspace{1mm}\textcolor{mcvert}{\hrule}
	\fi 
	\ifx\url\arg
		\fancyfoot[L]{\href{https://www.maths-cours.fr#2}{\black \footnotesize{https://www.maths-cours.fr#2}}}
	\fi 
}


\newcommand\TitreC[1]{    		% Titre centré
     \needspace{3\baselineskip}
     \begin{center}\textbf{#1}\end{center}
}

\newcommand\newpar{    		% paragraphe
     \par
}

\newcommand\nosp {    		% commande vide (pas d'espace)
}
\newcommand{\id}[1]{} %ignore

\newcommand\boite[2]{				% Boite simple sans titre
	\vspace{5mm}
	\setlength{\fboxrule}{0.2mm}
	\setlength{\fboxsep}{5mm}	
	\fcolorbox{#1}{#1!3}{\makebox[\linewidth-2\fboxrule-2\fboxsep]{
  		\begin{minipage}[t]{\linewidth-2\fboxrule-4\fboxsep}\setlength{\parskip}{3mm}
  			 #2
  		\end{minipage}
	}}
	\vspace{5mm}
}

\newcommand\CBox[4]{				% Boites
	\vspace{5mm}
	\setlength{\fboxrule}{0.2mm}
	\setlength{\fboxsep}{5mm}
	
	\fcolorbox{#1}{#1!3}{\makebox[\linewidth-2\fboxrule-2\fboxsep]{
		\begin{minipage}[t]{1cm}\setlength{\parskip}{3mm}
	  		\textcolor{#1}{\LARGE{#2}}    
 	 	\end{minipage}  
  		\begin{minipage}[t]{\linewidth-2\fboxrule-4\fboxsep}\setlength{\parskip}{3mm}
			\raisebox{1.2mm}{\normalsize\sffamily{\textcolor{#1}{#3}}}						
  			 #4
  		\end{minipage}
	}}
	\vspace{5mm}
}

\newcommand\cadre[3]{				% Boites convertible html
	\par
	\vspace{2mm}
	\setlength{\fboxrule}{0.1mm}
	\setlength{\fboxsep}{5mm}
	\fcolorbox{#1}{white}{\makebox[\linewidth-2\fboxrule-2\fboxsep]{
  		\begin{minipage}[t]{\linewidth-2\fboxrule-4\fboxsep}\setlength{\parskip}{3mm}
			\raisebox{-2.5mm}{\sffamily \small{\textcolor{#1}{\MakeUppercase{#2}}}}		
			\par		
  			 #3
 	 		\end{minipage}
	}}
		\vspace{2mm}
	\par
}

\newcommand\bloc[3]{				% Boites convertible html sans bordure
     \needspace{2\baselineskip}
     {\sffamily \small{\textcolor{#1}{\MakeUppercase{#2}}}}    
		\par		
  			 #3
		\par
}

\newcommand\CHelp[1]{
     \CBox{Plum}{\faInfoCircle}{À RETENIR}{#1}
}

\newcommand\CUp[1]{
     \CBox{NavyBlue}{\faThumbsOUp}{EN PRATIQUE}{#1}
}

\newcommand\CInfo[1]{
     \CBox{Sepia}{\faArrowCircleRight}{REMARQUE}{#1}
}

\newcommand\CRedac[1]{
     \CBox{PineGreen}{\faEdit}{BIEN R\'EDIGER}{#1}
}

\newcommand\CError[1]{
     \CBox{Red}{\faExclamationTriangle}{ATTENTION}{#1}
}

\newcommand\TitreExo[2]{
\needspace{4\baselineskip}
 {\sffamily\large EXERCICE #1\ (\emph{#2 points})}
\vspace{5mm}
}

\newcommand\img[2]{
          \includegraphics[width=#2\paperwidth]{\imgdir#1}
}

\newcommand\imgsvg[2]{
       \begin{center}   \includegraphics[width=#2\paperwidth]{\imgsvgdir#1} \end{center}
}


\newcommand\Lien[2]{
     \href{#1}{#2 \tiny \faExternalLink}
}
\newcommand\mcLien[2]{
     \href{https~://www.maths-cours.fr/#1}{#2 \tiny \faExternalLink}
}

\newcommand{\euro}{\eurologo{}}

%================================================================================================================================
%
% Macros - Environement
%
%================================================================================================================================

\newenvironment{tex}{ %
}
{%
}

\newenvironment{indente}{ %
	\setlength\parindent{10mm}
}

{
	\setlength\parindent{0mm}
}

\newenvironment{corrige}{%
     \needspace{3\baselineskip}
     \medskip
     \textbf{\textsc{Corrigé}}
     \medskip
}
{
}

\newenvironment{extern}{%
     \begin{center}
     }
     {
     \end{center}
}

\NewEnviron{code}{%
	\par
     \boite{gray}{\texttt{%
     \BODY
     }}
     \par
}

\newenvironment{vbloc}{% boite sans cadre empeche saut de page
     \begin{minipage}[t]{\linewidth}
     }
     {
     \end{minipage}
}
\NewEnviron{h2}{%
    \needspace{3\baselineskip}
    \vspace{0.6cm}
	\noindent \MakeUppercase{\sffamily \large \BODY}
	\vspace{1mm}\textcolor{mcgris}{\hrule}\vspace{0.4cm}
	\par
}{}

\NewEnviron{h3}{%
    \needspace{3\baselineskip}
	\vspace{5mm}
	\textsc{\BODY}
	\par
}

\NewEnviron{margeneg}{ %
\begin{addmargin}[-1cm]{0cm}
\BODY
\end{addmargin}
}

\NewEnviron{html}{%
}

\begin{document}
\meta{url}{/cours/pourcentages/}
\meta{pid}{365}
\meta{titre}{Pourcentages}
\meta{type}{cours}
\begin{h2}1. Part en pourcentage\end{h2}
\cadre{bleu}{Définition}{% id="d10"
     Soit $E$ un ensemble fini (que l'on appellera \textbf{ensemble de référence}) et $F$ une partie de l'ensemble $E$. La \textbf{part en pourcentage} de $F$ par rapport à  $E$ est le nombre :
     \begin{center}$t \% =\frac{t}{100}= \frac{card \left(F\right)}{card \left(E\right)}$\end{center}
     où $card \left(E\right)$ (cardinal de $E$) désigne le nombre d'éléments de $E$ et $card \left(F\right)$ le nombre d'éléments de $F$.
     \par
     On dit également que $F$ représente $t\%$ de $E$.
}
\bloc{cyan}{Remarques}{% id="r10"
     \begin{itemize}
          \item $5\%$, $\frac{5}{100}$ et $0,05$ sont trois écritures différentes du \textbf{même nombre} (pourcentage, fraction, écriture décimale).
          \item On est en présence d'une situation de proportionnalité que l'on peut représenter par le tableau suivant :
          \begin{tabularx}{0.8\linewidth}{|*{3}{>{\centering \arraybackslash }X|}}%class="compact" width="600"
               \hline
               $t$ & nombre d'éléments de $F$
               \\ \hline
               $100$ & nombre d'éléments de $E$
               \\ \hline
          \end{tabularx}
          \item Ceci peut également s'écrire : nombre d'éléments de $F =\frac{t}{100} \times $ nombre d'élements de $E$.
          \par
          Cette dernière égalité permet de calculer le nombre d'éléments de $F$ connaissant sa part en pourcentage par rapport à $E$
     \end{itemize}
}
\bloc{orange}{Exemples}{% id="e10"
     \begin{itemize}
          \item Dans une classe de $25$ élèves qui compte $15$ garçons le pourcentage de garçons est :
          \par
          $\frac{15}{25}=0,6=\frac{60}{100}=60\%$
          \item $16\%$ de $75$€ font : $\frac{16}{100}\times 75=12$€
     \end{itemize}
}
\cadre{vert}{Propriété}{% id="p20"
     \textbf{Pourcentages de pourcentages}
     Soit 3 ensembles $E, F, G$ tels que $G \subset F \subset E$.
     \par
     Si $G$ représente $t_{1}$\% de $F$ et si $F$ représente $t_{2}$\% de $E$, la part en pourcentage de $G$ par rapport à $E$ est :
     \begin{center}$\frac{t}{100}=\frac{t_{1}}{100}\times \frac{t_{2}}{100}$\end{center}
}
\bloc{orange}{Exemple}{% id="e20"
     Dans un lycée de $800$ élèves :
     \begin{itemize}
          \item $25$ \% des élèves sont en Seconde;
          \item $45$ \% des élèves de Seconde sont des filles.
     \end{itemize}
     La part des filles de Seconde dans le lycée est :
     \par
     $\frac{t}{100}=\frac{25}{100}\times \frac{45}{100}=\frac{1125}{10000}=\frac{11,25}{100}=11,25\%$
     \par
     Le nombre de filles en Seconde est $\frac{11,25}{100}\times 800=90$
}
\begin{h2}2. Pourcentages d'évolution\end{h2}
\cadre{bleu}{Définition}{% id="d40"
     On considère une quantité passant d'une valeur $V_{0}$ à une valeur $V_{1}$.
     \par
     Le pourcentage d'évolution de cette quantité est le nombre
     \begin{center}$\frac{t}{100}=\frac{V_{1}-V_{0}}{V_{0}}$\end{center}
}
\bloc{cyan}{Remarques}{% id="r40"
     Le pourcentage d'évolution est \textbf{positif} dans le cas d'une \textbf{augmentation} et \textbf{négatif} dans le cas d'une \textbf{diminution}.
}
\bloc{orange}{Exemple}{% id="e40"
     Le prix d'un article passe de 80€ à 76€. Le pourcentage d'évolution est :
     \begin{center}$\frac{t}{100}=\frac{76-80}{80}=-\frac{4}{80}=-0,05=-5\%$\end{center}
     Le prix de l'article a diminué de 5\%
}    
\cadre{bleu}{Définition}{% id="d30"
     On considère une quantité passant d'une valeur $V_{0}$ à une valeur $V_{1}$.
     \par
     Le \textbf{coefficient multiplicateur} est le nombre par lequel il faut multiplier $V_{0}$ pour obtenir $V_{1}$ :
     \par
     $V_{1}=CM \times  V_{0}$
}
\bloc{cyan}{Remarques}{% id="r30"
     \begin{itemize}
          \item On a donc  $CM=\frac{V_{1}}{V_{0}}$
          \item Le coefficient multiplicateur est \textbf{supérieur à 1} dans le cas d'une \textbf{augmentation} et \textbf{inférieur à 1} dans le cas d'une \textbf{diminution}.
          \item La fonction qui à l'ancienne valeur associe la nouvelle valeur est : $x\mapsto CM\times x$
          \par
          C'est une \textbf{fonction linéaire} de coefficient directeur $CM$
     \end{itemize}
}
\cadre{vert}{Propriété}{% id="p50"
     Le coefficient multiplicateur s'exprime en fonction du pourcentage d'évolution par:
     \par
     $CM=1+\frac{t}{100}$
     \par
     (où $t$ est positif en cas d'augmentation, négatif en cas de diminution)
}
\bloc{cyan}{Remarques}{% id="r50"
     \begin{itemize}
          \item On a donc : $V_{1}=\left(1+\frac{t}{100}\right)V_{0}$.
          \item Dans le cas d'une diminution de $5$\%, par exemple, on pourra au choix considérer que :
          \par
          $CM=1+\frac{t}{100}$ avec $t=-5$
          \par
          ou
          \par
          $CM=1-\frac{t}{100}$ avec $t=5$
          \par
          Dans les deux raisonnements, on obtient évidemment le même coefficient multiplicateur $0,95$.
          \item Connaissant le coefficient multiplicateur, on a facilement le pourcentage d'évolution grâce à la relation : $\frac{t}{100}=CM-1$
          \item Le tableau ci-dessous résume les différents cas :
          \begin{tabularx}{0.8\linewidth}{|*{3}{>{\centering \arraybackslash }X|}}%class="compact" width="600"
               \hline
               &  Prendre $t\%$ de $x$                &   Augmenter $x$ de $t\%$                 & Diminuer $x$ de $t\%$
               \\ \hline
               Calculs à effectuer  &  Multiplier $x$ par \textbf{$\frac{t}{100}$}  &   Multiplier $x$ par <br />\textbf{$1+\frac{t}{100}$} & Multiplier $x$ par<br /> \textbf{$1-\frac{t}{100}$}
               \\ \hline
               Fonction linéaire    & $x\mapsto \frac{t}{100}\times x$                    &  $x\mapsto \left(1+\frac{t}{100}\right)\times x$                    & $x\mapsto \left(1-\frac{t}{100}\right)\times x$
               \\ \hline
          \end{tabularx}
     \end{itemize}
}
\bloc{orange}{Exemple}{% id="e50"
     \begin{tabularx}{0.8\linewidth}{|*{3}{>{\centering \arraybackslash }X|}}%class="compact" width="600"
          \hline
          &  Prendre $25\%$ de $x$         &   Augmenter $x$ de $25\%$        & Diminuer $x$ de $25\%$
          \\ \hline
          Calculs à effectuer  &  Multiplier $x$ par $\frac{25}{100}$  &   Multiplier $x$ par $1,25$     & Multiplier $x$ par $0,75$
          \\ \hline
          Fonction linéaire    & $x\mapsto 0,25\times x$                 &   $x\mapsto 1,25\times x$                 & $x\mapsto 0,75\times x$
          \\ \hline
          Exemples             & Prendre $25\%$ de 200          &   Augmenter 50 de $25\%$         & Diminuer 50 de $25\%$
          \\ \hline
          Résultat             & $0,25\times 200=50$          &   $1,25\times 50=62,5$          & $0,75\times 50=37,5$
          \\ \hline
     \end{tabularx}
}
\cadre{vert}{Propriété (Évolutions successives)}{% id="p60"
     Lors d'évolutions successives, le coefficient multiplicateur global est égal au \textbf{produit} des coefficients multiplicateurs de chaque évolution
}
\bloc{orange}{Exemple}{% id="e60"
     Le prix d'un objet augmente de $10\%$ puis diminue de $10\%$.
     \par
     Le coefficient multiplicateur global est :
     \begin{center}$CM=\left(1+\frac{10}{100}\right)\left(1-\frac{10}{100}\right)=0,99$ \end{center}
     Si $t$ désigne le pourcentage d'évolution global en \%, on a donc :
     \par
     $1+ \frac{t}{100}=0,99 $
     \par
     $\frac{t}{100}=0,99-1=-0,01=-\frac{1}{100}$
     \par
     Le prix de l'objet a globalement \textbf{diminué} de $1\%$.
}
\bloc{cyan}{Remarques}{% id="r60"
     \begin{itemize}
          \item \textbf{Une hausse de $t\%$ ne "compense" pas une baisse de $t\%$}. C'est dû au fait que les deux pourcentages ne portent pas sur le même montant.
          \par
          En effet, si un objet coûtant 100 euros subit une augmentation de $10\%$ son prix passera à $110$€ (les $10\%$ ont été calculé par rapport à $100$€).
          \par
          Si son prix subit ensuite une diminution de $10\%$, le montant de la baisse sera calculé par rapport au prix de $110$€ et non plus de $100$€. La baisse sera donc de $11$€ et non $10$€.
          \item En cas d'évolution successives,\textbf{ les pourcentages d'évolutions ne s'ajoutent (ni ne soustraient) jamais}.
     \end{itemize}
}
\cadre{vert}{Définition et propriété (Taux d'évolution réciproque)}{% id="p70"
     Si le taux d'évolution $t \%$ fait passer de $V_{0}$ à $V_{1}$, on appelle taux d'évolution réciproque $t^{\prime} \%$, le taux d'évolution qui fait passer de $V_{1}$ à $V_{0}$.
     \par
     On a alors la relation suivante :
     \begin{center}$\left(1+\frac{t}{100}\right)\left(1+\frac{t^{\prime}}{100}\right)=1$\end{center}
}
\bloc{orange}{Exemple}{% id="e70"
     Le prix d'un article augmente de 60\%. Pour qu'il revienne à son prix de départ, il faut qu'ensuite il varie de $t^{\prime} \%$ tel que :
     \par
     $\left(1+\frac{60}{100}\right)\left(1+\frac{t^{\prime}}{100}\right)=1$
     \par
     $1,6\times \left(1+\frac{t^{\prime}}{100}\right)=1$
     \par
     $1+\frac{t^{\prime}}{100}=\frac{1}{1,6}$
     \par
     $1+\frac{t^{\prime}}{100}=0,625$
     \par
     $\frac{t^{\prime}}{100}=-0,375$
     \par
     $t^{\prime}=-37,5$
     \par
     Il faut donc que le prix diminue de 37,5\% pour compenser la hausse de 60\%.
}

\end{document}
µ
\documentclass[a4paper]{article}

%================================================================================================================================
%
% Packages
%
%================================================================================================================================

\usepackage[T1]{fontenc} 	% pour caractères accentués
\usepackage[utf8]{inputenc}  % encodage utf8
\usepackage[french]{babel}	% langue : français
\usepackage{fourier}			% caractères plus lisibles
\usepackage[dvipsnames]{xcolor} % couleurs
\usepackage{fancyhdr}		% réglage header footer
\usepackage{needspace}		% empêcher sauts de page mal placés
\usepackage{graphicx}		% pour inclure des graphiques
\usepackage{enumitem,cprotect}		% personnalise les listes d'items (nécessaire pour ol, al ...)
\usepackage{hyperref}		% Liens hypertexte
\usepackage{pstricks,pst-all,pst-node,pstricks-add,pst-math,pst-plot,pst-tree,pst-eucl} % pstricks
\usepackage[a4paper,includeheadfoot,top=2cm,left=3cm, bottom=2cm,right=3cm]{geometry} % marges etc.
\usepackage{comment}			% commentaires multilignes
\usepackage{amsmath,environ} % maths (matrices, etc.)
\usepackage{amssymb,makeidx}
\usepackage{bm}				% bold maths
\usepackage{tabularx}		% tableaux
\usepackage{colortbl}		% tableaux en couleur
\usepackage{fontawesome}		% Fontawesome
\usepackage{environ}			% environment with command
\usepackage{fp}				% calculs pour ps-tricks
\usepackage{multido}			% pour ps tricks
\usepackage[np]{numprint}	% formattage nombre
\usepackage{tikz,tkz-tab} 			% package principal TikZ
\usepackage{pgfplots}   % axes
\usepackage{mathrsfs}    % cursives
\usepackage{calc}			% calcul taille boites
\usepackage[scaled=0.875]{helvet} % font sans serif
\usepackage{svg} % svg
\usepackage{scrextend} % local margin
\usepackage{scratch} %scratch
\usepackage{multicol} % colonnes
%\usepackage{infix-RPN,pst-func} % formule en notation polanaise inversée
\usepackage{listings}

%================================================================================================================================
%
% Réglages de base
%
%================================================================================================================================

\lstset{
language=Python,   % R code
literate=
{á}{{\'a}}1
{à}{{\`a}}1
{ã}{{\~a}}1
{é}{{\'e}}1
{è}{{\`e}}1
{ê}{{\^e}}1
{í}{{\'i}}1
{ó}{{\'o}}1
{õ}{{\~o}}1
{ú}{{\'u}}1
{ü}{{\"u}}1
{ç}{{\c{c}}}1
{~}{{ }}1
}


\definecolor{codegreen}{rgb}{0,0.6,0}
\definecolor{codegray}{rgb}{0.5,0.5,0.5}
\definecolor{codepurple}{rgb}{0.58,0,0.82}
\definecolor{backcolour}{rgb}{0.95,0.95,0.92}

\lstdefinestyle{mystyle}{
    backgroundcolor=\color{backcolour},   
    commentstyle=\color{codegreen},
    keywordstyle=\color{magenta},
    numberstyle=\tiny\color{codegray},
    stringstyle=\color{codepurple},
    basicstyle=\ttfamily\footnotesize,
    breakatwhitespace=false,         
    breaklines=true,                 
    captionpos=b,                    
    keepspaces=true,                 
    numbers=left,                    
xleftmargin=2em,
framexleftmargin=2em,            
    showspaces=false,                
    showstringspaces=false,
    showtabs=false,                  
    tabsize=2,
    upquote=true
}

\lstset{style=mystyle}


\lstset{style=mystyle}
\newcommand{\imgdir}{C:/laragon/www/newmc/assets/imgsvg/}
\newcommand{\imgsvgdir}{C:/laragon/www/newmc/assets/imgsvg/}

\definecolor{mcgris}{RGB}{220, 220, 220}% ancien~; pour compatibilité
\definecolor{mcbleu}{RGB}{52, 152, 219}
\definecolor{mcvert}{RGB}{125, 194, 70}
\definecolor{mcmauve}{RGB}{154, 0, 215}
\definecolor{mcorange}{RGB}{255, 96, 0}
\definecolor{mcturquoise}{RGB}{0, 153, 153}
\definecolor{mcrouge}{RGB}{255, 0, 0}
\definecolor{mclightvert}{RGB}{205, 234, 190}

\definecolor{gris}{RGB}{220, 220, 220}
\definecolor{bleu}{RGB}{52, 152, 219}
\definecolor{vert}{RGB}{125, 194, 70}
\definecolor{mauve}{RGB}{154, 0, 215}
\definecolor{orange}{RGB}{255, 96, 0}
\definecolor{turquoise}{RGB}{0, 153, 153}
\definecolor{rouge}{RGB}{255, 0, 0}
\definecolor{lightvert}{RGB}{205, 234, 190}
\setitemize[0]{label=\color{lightvert}  $\bullet$}

\pagestyle{fancy}
\renewcommand{\headrulewidth}{0.2pt}
\fancyhead[L]{maths-cours.fr}
\fancyhead[R]{\thepage}
\renewcommand{\footrulewidth}{0.2pt}
\fancyfoot[C]{}

\newcolumntype{C}{>{\centering\arraybackslash}X}
\newcolumntype{s}{>{\hsize=.35\hsize\arraybackslash}X}

\setlength{\parindent}{0pt}		 
\setlength{\parskip}{3mm}
\setlength{\headheight}{1cm}

\def\ebook{ebook}
\def\book{book}
\def\web{web}
\def\type{web}

\newcommand{\vect}[1]{\overrightarrow{\,\mathstrut#1\,}}

\def\Oij{$\left(\text{O}~;~\vect{\imath},~\vect{\jmath}\right)$}
\def\Oijk{$\left(\text{O}~;~\vect{\imath},~\vect{\jmath},~\vect{k}\right)$}
\def\Ouv{$\left(\text{O}~;~\vect{u},~\vect{v}\right)$}

\hypersetup{breaklinks=true, colorlinks = true, linkcolor = OliveGreen, urlcolor = OliveGreen, citecolor = OliveGreen, pdfauthor={Didier BONNEL - https://www.maths-cours.fr} } % supprime les bordures autour des liens

\renewcommand{\arg}[0]{\text{arg}}

\everymath{\displaystyle}

%================================================================================================================================
%
% Macros - Commandes
%
%================================================================================================================================

\newcommand\meta[2]{    			% Utilisé pour créer le post HTML.
	\def\titre{titre}
	\def\url{url}
	\def\arg{#1}
	\ifx\titre\arg
		\newcommand\maintitle{#2}
		\fancyhead[L]{#2}
		{\Large\sffamily \MakeUppercase{#2}}
		\vspace{1mm}\textcolor{mcvert}{\hrule}
	\fi 
	\ifx\url\arg
		\fancyfoot[L]{\href{https://www.maths-cours.fr#2}{\black \footnotesize{https://www.maths-cours.fr#2}}}
	\fi 
}


\newcommand\TitreC[1]{    		% Titre centré
     \needspace{3\baselineskip}
     \begin{center}\textbf{#1}\end{center}
}

\newcommand\newpar{    		% paragraphe
     \par
}

\newcommand\nosp {    		% commande vide (pas d'espace)
}
\newcommand{\id}[1]{} %ignore

\newcommand\boite[2]{				% Boite simple sans titre
	\vspace{5mm}
	\setlength{\fboxrule}{0.2mm}
	\setlength{\fboxsep}{5mm}	
	\fcolorbox{#1}{#1!3}{\makebox[\linewidth-2\fboxrule-2\fboxsep]{
  		\begin{minipage}[t]{\linewidth-2\fboxrule-4\fboxsep}\setlength{\parskip}{3mm}
  			 #2
  		\end{minipage}
	}}
	\vspace{5mm}
}

\newcommand\CBox[4]{				% Boites
	\vspace{5mm}
	\setlength{\fboxrule}{0.2mm}
	\setlength{\fboxsep}{5mm}
	
	\fcolorbox{#1}{#1!3}{\makebox[\linewidth-2\fboxrule-2\fboxsep]{
		\begin{minipage}[t]{1cm}\setlength{\parskip}{3mm}
	  		\textcolor{#1}{\LARGE{#2}}    
 	 	\end{minipage}  
  		\begin{minipage}[t]{\linewidth-2\fboxrule-4\fboxsep}\setlength{\parskip}{3mm}
			\raisebox{1.2mm}{\normalsize\sffamily{\textcolor{#1}{#3}}}						
  			 #4
  		\end{minipage}
	}}
	\vspace{5mm}
}

\newcommand\cadre[3]{				% Boites convertible html
	\par
	\vspace{2mm}
	\setlength{\fboxrule}{0.1mm}
	\setlength{\fboxsep}{5mm}
	\fcolorbox{#1}{white}{\makebox[\linewidth-2\fboxrule-2\fboxsep]{
  		\begin{minipage}[t]{\linewidth-2\fboxrule-4\fboxsep}\setlength{\parskip}{3mm}
			\raisebox{-2.5mm}{\sffamily \small{\textcolor{#1}{\MakeUppercase{#2}}}}		
			\par		
  			 #3
 	 		\end{minipage}
	}}
		\vspace{2mm}
	\par
}

\newcommand\bloc[3]{				% Boites convertible html sans bordure
     \needspace{2\baselineskip}
     {\sffamily \small{\textcolor{#1}{\MakeUppercase{#2}}}}    
		\par		
  			 #3
		\par
}

\newcommand\CHelp[1]{
     \CBox{Plum}{\faInfoCircle}{À RETENIR}{#1}
}

\newcommand\CUp[1]{
     \CBox{NavyBlue}{\faThumbsOUp}{EN PRATIQUE}{#1}
}

\newcommand\CInfo[1]{
     \CBox{Sepia}{\faArrowCircleRight}{REMARQUE}{#1}
}

\newcommand\CRedac[1]{
     \CBox{PineGreen}{\faEdit}{BIEN R\'EDIGER}{#1}
}

\newcommand\CError[1]{
     \CBox{Red}{\faExclamationTriangle}{ATTENTION}{#1}
}

\newcommand\TitreExo[2]{
\needspace{4\baselineskip}
 {\sffamily\large EXERCICE #1\ (\emph{#2 points})}
\vspace{5mm}
}

\newcommand\img[2]{
          \includegraphics[width=#2\paperwidth]{\imgdir#1}
}

\newcommand\imgsvg[2]{
       \begin{center}   \includegraphics[width=#2\paperwidth]{\imgsvgdir#1} \end{center}
}


\newcommand\Lien[2]{
     \href{#1}{#2 \tiny \faExternalLink}
}
\newcommand\mcLien[2]{
     \href{https~://www.maths-cours.fr/#1}{#2 \tiny \faExternalLink}
}

\newcommand{\euro}{\eurologo{}}

%================================================================================================================================
%
% Macros - Environement
%
%================================================================================================================================

\newenvironment{tex}{ %
}
{%
}

\newenvironment{indente}{ %
	\setlength\parindent{10mm}
}

{
	\setlength\parindent{0mm}
}

\newenvironment{corrige}{%
     \needspace{3\baselineskip}
     \medskip
     \textbf{\textsc{Corrigé}}
     \medskip
}
{
}

\newenvironment{extern}{%
     \begin{center}
     }
     {
     \end{center}
}

\NewEnviron{code}{%
	\par
     \boite{gray}{\texttt{%
     \BODY
     }}
     \par
}

\newenvironment{vbloc}{% boite sans cadre empeche saut de page
     \begin{minipage}[t]{\linewidth}
     }
     {
     \end{minipage}
}
\NewEnviron{h2}{%
    \needspace{3\baselineskip}
    \vspace{0.6cm}
	\noindent \MakeUppercase{\sffamily \large \BODY}
	\vspace{1mm}\textcolor{mcgris}{\hrule}\vspace{0.4cm}
	\par
}{}

\NewEnviron{h3}{%
    \needspace{3\baselineskip}
	\vspace{5mm}
	\textsc{\BODY}
	\par
}

\NewEnviron{margeneg}{ %
\begin{addmargin}[-1cm]{0cm}
\BODY
\end{addmargin}
}

\NewEnviron{html}{%
}

\begin{document}
\meta{url}{/cours/suites-generalites/}
\meta{pid}{377}
\meta{titre}{Les suites : Généralités}
\meta{type}{cours}
\begin{h2}I - Définition d'une suite\end{h2}
\cadre{bleu}{Définitions}{% id="d10"
     Une \textbf{suite} $u$ associe à tout entier naturel $n$ un nombre réel noté $u_{n}$.
     \par
     Les nombres réels $u_{n}$ sont les \textbf{termes} de la suite.
     \par
     Les nombres entiers $n$ sont les \textbf{indices} ou les \textbf{rangs}.
     \par
     La suite $u$ peut également se noter $\left(u_{n}\right)$ ou $\left(u_{n}\right)_{n\in \mathbb{N}}$.
}
\bloc{cyan}{Remarque}{% id="r10"
     Intuitivement, une suite est une liste infinie et ordonnée de nombres réels. Ces nombres réels sont les termes de la suite et les indices correspondent à la position du terme dans la liste.
}
\bloc{orange}{Exemple}{% id="e10"
     Par exemple la liste $1,6$ ; $2,4$ ; $3,2$ ; $5$ ; ... correspond à la suite $\left(u_{n}\right)$ suivante :
     \par
     $u_{0}=1,6$ (terme de rang 0)
     \par
     $u_{1}=2,4$ (terme de rang 1)
     \par
     $u_{2}=3,2$ (terme de rang 2)
     \par
     $u_{3}=5$ ...
}
\bloc{cyan}{Remarque}{% id="r11"
     Ne pas confondre l'écriture $\left(u_{n}\right)$ avec parenthèses qui désigne la suite et l'écriture $u_{n}$ sans parenthèse qui désigne le $n$-ième terme de la suite.
}
\cadre{bleu}{Définition}{% id="d20"
     Une suite est définie de façon \textbf{explicite} lorsqu'on dispose d'une formule du type $u_{n}=f\left(n\right)$ permettant de calculer chaque terme de la suite à partir de son rang.
}
\bloc{orange}{Exemple}{% id="e20"
     La suite $\left(u_{n}\right)$ définie par la formule explicite $u_{n}=\frac{2n+1}{3}$ est telle que
     \par
     $u_{0}=\frac{1}{3}$
     \par
     $u_{1}=\frac{3}{3}=1$ ...
     \par
     $u_{100}=\frac{201}{3}=67$
}
\cadre{bleu}{Définition}{% id="d30"
     Une suite est définie par une relation de \textbf{récurrence} lorsqu'on dispose du premier terme et d'une formule du type $u_{n+1}=f\left(u_{n}\right)$ permettant de calculer chaque terme de la suite à partir du terme précédent.
}
\bloc{cyan}{Remarque}{% id="r30"
     Il est possible de calculer un terme quelconque d'une suite définie par une relation de récurrence mais il faut au préalable calculer tout les termes précédents. Comme cela peut se révéler long,  on utilise parfois un algorithme pour faire ce calcul.
}
\bloc{orange}{Exemple}{% id="e30"
     La suite $\left(u_{n}\right)$ définie par la formule de récurrence
     \par
     $\left\{ \begin{matrix} u_{0}=1  \\  u_{n+1}=2u_{n}-3 \end{matrix}\right.$
     \par
     est telle que :
     \par
     $u_{0}=1$
     \par
     $u_{1}=2\times u_{0}-3=2\times 1-3=-1$
     \par
     $u_{2}=2\times u_{1}-3=2\times \left(-1\right)-3=-5$
     \par
     etc...
}
\begin{h2}II - Représentation graphique d'une suite\end{h2}
\cadre{bleu}{Définition}{% id="d40"
     La représentation graphique d'une suite $\left(u_{n}\right)_{n \in  \mathbb{N}}$ dans un repère du plan, s'obtient en plaçant les points de coordonnées $\left(n ; u_{n}\right)$ lorsque $n$ parcourt $\mathbb{N}$.
}
\bloc{orange}{Exemple}{% id="e40"
     Pour représenter la suite définie par $u_{n}=1+\frac{3}{n+1}$ on calcule:
     \par
     $u_{0}=4$
     \par
     $u_{1}=\frac{5}{2}$
     \par
     $u_{2}=2$
     \par
     $u_{3}=\frac{7}{4}$
     \par
     etc.
     \par
     et on place les points de coordonnées : $\left(0 ; 4\right) ; \left(1 ; \frac{5}{2}\right) ; \left(2 ; 2\right) ; \left(3 ; \frac{7}{4}\right)$; etc.
}
\begin{center}
     \begin{extern}%width="300" alt="représentation graphique d'une suite"
          % -+-+-+ variables modifiables
          \resizebox{7cm}{!}{%
               \def\xmin{-0.8}
               \def\xmax{7.5}
               \def\ymin{-0.8}
               \def\ymax{4.8}
               \def\xunit{1}  % unités en cm
               \def\yunit{1}
               \psset{xunit=\xunit,yunit=\yunit,algebraic=true}
               \fontsize{15pt}{15pt}\selectfont
               \begin{pspicture*}[linewidth=1pt](\xmin,\ymin)(\xmax,\ymax)
                    \psaxes[Dx=1,Dy=1,linewidth=0.75pt]{->}(0,0)(\xmin,\ymin)(\xmax,\ymax)
                    \rput[tr](-0.2,-0.3){$O$}
                    \multido{\n=0.0+1}{8}{
                         \FPeval{\suite}{1+3/(\n+1)}
                         \psdots[linecolor=blue](\n,\suite)
                    }
               \end{pspicture*}
          }
     \end{extern}
\end{center}
\begin{center}
     \textit{Représentation graphique de la suite définie par $u_{n}=1+\frac{3}{n+1}$}
\end{center}
\begin{h2}III - Sens de variation d'une suite\end{h2}
\cadre{bleu}{Définitions}{% id="d50"
     On dit qu'une suite $\left(u_{n}\right)$ est \textbf{croissante} (\textit{resp.\textbf{décroissante})} si pour tout entier naturel $n$ :
     \begin{center}$u_{n+1} \geqslant  u_{n}   $  (\textit{resp. $u_{n+1} \leqslant  u_{n} $)}\end{center}
     On dit qu'une suite $\left(u_{n}\right)$ est \textbf{strictement croissante} (\textit{resp.\textbf{strictement décroissante})} si pour tout entier naturel $n$ :
     \begin{center}$u_{n+1} > u_{n}   $  (\textit{resp. $u_{n+1} < u_{n} $)}\end{center}
     On dit qu'une suite $\left(u_{n}\right)$ est \textbf{constante} si pour tout entier naturel  $n$ :
     \begin{center}$u_{n+1} = u_{n}  $\end{center}
}
\bloc{cyan}{Remarques}{% id="r50"
     \begin{itemize}
          \item Une suite peut n'être ni croissante,, ni décroissante, ni constante. C'est le cas, par exemple de la suite définie par $u_{n}=\left(-1\right)^{n}$ dont les termes valent successivement : $1; -1; 1; -1; 1; -1;$ etc.
          \item En pratique pour savoir si une suite $\left(u_{n}\right)$ est croissante ou décroissante, on calcule souvent $u_{n+1}-u_{n}$ :
          \begin{itemize}
               \item si $u_{n+1}-u_{n} \geqslant  0$ pour tout $n \in  \mathbb{N}$, la suite $u_{n}$ est croissante
               \item si $u_{n+1}-u_{n} \leqslant  0$ pour tout $n \in  \mathbb{N}$, la suite $u_{n}$ est décroissante
               \item  si $u_{n+1}-u_{n} = 0$ pour tout $n \in  \mathbb{N}$, la suite $u_{n}$ est constante.
          \end{itemize}
     \end{itemize}
}

\end{document}
µ
\documentclass[a4paper]{article}

%================================================================================================================================
%
% Packages
%
%================================================================================================================================

\usepackage[T1]{fontenc} 	% pour caractères accentués
\usepackage[utf8]{inputenc}  % encodage utf8
\usepackage[french]{babel}	% langue : français
\usepackage{fourier}			% caractères plus lisibles
\usepackage[dvipsnames]{xcolor} % couleurs
\usepackage{fancyhdr}		% réglage header footer
\usepackage{needspace}		% empêcher sauts de page mal placés
\usepackage{graphicx}		% pour inclure des graphiques
\usepackage{enumitem,cprotect}		% personnalise les listes d'items (nécessaire pour ol, al ...)
\usepackage{hyperref}		% Liens hypertexte
\usepackage{pstricks,pst-all,pst-node,pstricks-add,pst-math,pst-plot,pst-tree,pst-eucl} % pstricks
\usepackage[a4paper,includeheadfoot,top=2cm,left=3cm, bottom=2cm,right=3cm]{geometry} % marges etc.
\usepackage{comment}			% commentaires multilignes
\usepackage{amsmath,environ} % maths (matrices, etc.)
\usepackage{amssymb,makeidx}
\usepackage{bm}				% bold maths
\usepackage{tabularx}		% tableaux
\usepackage{colortbl}		% tableaux en couleur
\usepackage{fontawesome}		% Fontawesome
\usepackage{environ}			% environment with command
\usepackage{fp}				% calculs pour ps-tricks
\usepackage{multido}			% pour ps tricks
\usepackage[np]{numprint}	% formattage nombre
\usepackage{tikz,tkz-tab} 			% package principal TikZ
\usepackage{pgfplots}   % axes
\usepackage{mathrsfs}    % cursives
\usepackage{calc}			% calcul taille boites
\usepackage[scaled=0.875]{helvet} % font sans serif
\usepackage{svg} % svg
\usepackage{scrextend} % local margin
\usepackage{scratch} %scratch
\usepackage{multicol} % colonnes
%\usepackage{infix-RPN,pst-func} % formule en notation polanaise inversée
\usepackage{listings}

%================================================================================================================================
%
% Réglages de base
%
%================================================================================================================================

\lstset{
language=Python,   % R code
literate=
{á}{{\'a}}1
{à}{{\`a}}1
{ã}{{\~a}}1
{é}{{\'e}}1
{è}{{\`e}}1
{ê}{{\^e}}1
{í}{{\'i}}1
{ó}{{\'o}}1
{õ}{{\~o}}1
{ú}{{\'u}}1
{ü}{{\"u}}1
{ç}{{\c{c}}}1
{~}{{ }}1
}


\definecolor{codegreen}{rgb}{0,0.6,0}
\definecolor{codegray}{rgb}{0.5,0.5,0.5}
\definecolor{codepurple}{rgb}{0.58,0,0.82}
\definecolor{backcolour}{rgb}{0.95,0.95,0.92}

\lstdefinestyle{mystyle}{
    backgroundcolor=\color{backcolour},   
    commentstyle=\color{codegreen},
    keywordstyle=\color{magenta},
    numberstyle=\tiny\color{codegray},
    stringstyle=\color{codepurple},
    basicstyle=\ttfamily\footnotesize,
    breakatwhitespace=false,         
    breaklines=true,                 
    captionpos=b,                    
    keepspaces=true,                 
    numbers=left,                    
xleftmargin=2em,
framexleftmargin=2em,            
    showspaces=false,                
    showstringspaces=false,
    showtabs=false,                  
    tabsize=2,
    upquote=true
}

\lstset{style=mystyle}


\lstset{style=mystyle}
\newcommand{\imgdir}{C:/laragon/www/newmc/assets/imgsvg/}
\newcommand{\imgsvgdir}{C:/laragon/www/newmc/assets/imgsvg/}

\definecolor{mcgris}{RGB}{220, 220, 220}% ancien~; pour compatibilité
\definecolor{mcbleu}{RGB}{52, 152, 219}
\definecolor{mcvert}{RGB}{125, 194, 70}
\definecolor{mcmauve}{RGB}{154, 0, 215}
\definecolor{mcorange}{RGB}{255, 96, 0}
\definecolor{mcturquoise}{RGB}{0, 153, 153}
\definecolor{mcrouge}{RGB}{255, 0, 0}
\definecolor{mclightvert}{RGB}{205, 234, 190}

\definecolor{gris}{RGB}{220, 220, 220}
\definecolor{bleu}{RGB}{52, 152, 219}
\definecolor{vert}{RGB}{125, 194, 70}
\definecolor{mauve}{RGB}{154, 0, 215}
\definecolor{orange}{RGB}{255, 96, 0}
\definecolor{turquoise}{RGB}{0, 153, 153}
\definecolor{rouge}{RGB}{255, 0, 0}
\definecolor{lightvert}{RGB}{205, 234, 190}
\setitemize[0]{label=\color{lightvert}  $\bullet$}

\pagestyle{fancy}
\renewcommand{\headrulewidth}{0.2pt}
\fancyhead[L]{maths-cours.fr}
\fancyhead[R]{\thepage}
\renewcommand{\footrulewidth}{0.2pt}
\fancyfoot[C]{}

\newcolumntype{C}{>{\centering\arraybackslash}X}
\newcolumntype{s}{>{\hsize=.35\hsize\arraybackslash}X}

\setlength{\parindent}{0pt}		 
\setlength{\parskip}{3mm}
\setlength{\headheight}{1cm}

\def\ebook{ebook}
\def\book{book}
\def\web{web}
\def\type{web}

\newcommand{\vect}[1]{\overrightarrow{\,\mathstrut#1\,}}

\def\Oij{$\left(\text{O}~;~\vect{\imath},~\vect{\jmath}\right)$}
\def\Oijk{$\left(\text{O}~;~\vect{\imath},~\vect{\jmath},~\vect{k}\right)$}
\def\Ouv{$\left(\text{O}~;~\vect{u},~\vect{v}\right)$}

\hypersetup{breaklinks=true, colorlinks = true, linkcolor = OliveGreen, urlcolor = OliveGreen, citecolor = OliveGreen, pdfauthor={Didier BONNEL - https://www.maths-cours.fr} } % supprime les bordures autour des liens

\renewcommand{\arg}[0]{\text{arg}}

\everymath{\displaystyle}

%================================================================================================================================
%
% Macros - Commandes
%
%================================================================================================================================

\newcommand\meta[2]{    			% Utilisé pour créer le post HTML.
	\def\titre{titre}
	\def\url{url}
	\def\arg{#1}
	\ifx\titre\arg
		\newcommand\maintitle{#2}
		\fancyhead[L]{#2}
		{\Large\sffamily \MakeUppercase{#2}}
		\vspace{1mm}\textcolor{mcvert}{\hrule}
	\fi 
	\ifx\url\arg
		\fancyfoot[L]{\href{https://www.maths-cours.fr#2}{\black \footnotesize{https://www.maths-cours.fr#2}}}
	\fi 
}


\newcommand\TitreC[1]{    		% Titre centré
     \needspace{3\baselineskip}
     \begin{center}\textbf{#1}\end{center}
}

\newcommand\newpar{    		% paragraphe
     \par
}

\newcommand\nosp {    		% commande vide (pas d'espace)
}
\newcommand{\id}[1]{} %ignore

\newcommand\boite[2]{				% Boite simple sans titre
	\vspace{5mm}
	\setlength{\fboxrule}{0.2mm}
	\setlength{\fboxsep}{5mm}	
	\fcolorbox{#1}{#1!3}{\makebox[\linewidth-2\fboxrule-2\fboxsep]{
  		\begin{minipage}[t]{\linewidth-2\fboxrule-4\fboxsep}\setlength{\parskip}{3mm}
  			 #2
  		\end{minipage}
	}}
	\vspace{5mm}
}

\newcommand\CBox[4]{				% Boites
	\vspace{5mm}
	\setlength{\fboxrule}{0.2mm}
	\setlength{\fboxsep}{5mm}
	
	\fcolorbox{#1}{#1!3}{\makebox[\linewidth-2\fboxrule-2\fboxsep]{
		\begin{minipage}[t]{1cm}\setlength{\parskip}{3mm}
	  		\textcolor{#1}{\LARGE{#2}}    
 	 	\end{minipage}  
  		\begin{minipage}[t]{\linewidth-2\fboxrule-4\fboxsep}\setlength{\parskip}{3mm}
			\raisebox{1.2mm}{\normalsize\sffamily{\textcolor{#1}{#3}}}						
  			 #4
  		\end{minipage}
	}}
	\vspace{5mm}
}

\newcommand\cadre[3]{				% Boites convertible html
	\par
	\vspace{2mm}
	\setlength{\fboxrule}{0.1mm}
	\setlength{\fboxsep}{5mm}
	\fcolorbox{#1}{white}{\makebox[\linewidth-2\fboxrule-2\fboxsep]{
  		\begin{minipage}[t]{\linewidth-2\fboxrule-4\fboxsep}\setlength{\parskip}{3mm}
			\raisebox{-2.5mm}{\sffamily \small{\textcolor{#1}{\MakeUppercase{#2}}}}		
			\par		
  			 #3
 	 		\end{minipage}
	}}
		\vspace{2mm}
	\par
}

\newcommand\bloc[3]{				% Boites convertible html sans bordure
     \needspace{2\baselineskip}
     {\sffamily \small{\textcolor{#1}{\MakeUppercase{#2}}}}    
		\par		
  			 #3
		\par
}

\newcommand\CHelp[1]{
     \CBox{Plum}{\faInfoCircle}{À RETENIR}{#1}
}

\newcommand\CUp[1]{
     \CBox{NavyBlue}{\faThumbsOUp}{EN PRATIQUE}{#1}
}

\newcommand\CInfo[1]{
     \CBox{Sepia}{\faArrowCircleRight}{REMARQUE}{#1}
}

\newcommand\CRedac[1]{
     \CBox{PineGreen}{\faEdit}{BIEN R\'EDIGER}{#1}
}

\newcommand\CError[1]{
     \CBox{Red}{\faExclamationTriangle}{ATTENTION}{#1}
}

\newcommand\TitreExo[2]{
\needspace{4\baselineskip}
 {\sffamily\large EXERCICE #1\ (\emph{#2 points})}
\vspace{5mm}
}

\newcommand\img[2]{
          \includegraphics[width=#2\paperwidth]{\imgdir#1}
}

\newcommand\imgsvg[2]{
       \begin{center}   \includegraphics[width=#2\paperwidth]{\imgsvgdir#1} \end{center}
}


\newcommand\Lien[2]{
     \href{#1}{#2 \tiny \faExternalLink}
}
\newcommand\mcLien[2]{
     \href{https~://www.maths-cours.fr/#1}{#2 \tiny \faExternalLink}
}

\newcommand{\euro}{\eurologo{}}

%================================================================================================================================
%
% Macros - Environement
%
%================================================================================================================================

\newenvironment{tex}{ %
}
{%
}

\newenvironment{indente}{ %
	\setlength\parindent{10mm}
}

{
	\setlength\parindent{0mm}
}

\newenvironment{corrige}{%
     \needspace{3\baselineskip}
     \medskip
     \textbf{\textsc{Corrigé}}
     \medskip
}
{
}

\newenvironment{extern}{%
     \begin{center}
     }
     {
     \end{center}
}

\NewEnviron{code}{%
	\par
     \boite{gray}{\texttt{%
     \BODY
     }}
     \par
}

\newenvironment{vbloc}{% boite sans cadre empeche saut de page
     \begin{minipage}[t]{\linewidth}
     }
     {
     \end{minipage}
}
\NewEnviron{h2}{%
    \needspace{3\baselineskip}
    \vspace{0.6cm}
	\noindent \MakeUppercase{\sffamily \large \BODY}
	\vspace{1mm}\textcolor{mcgris}{\hrule}\vspace{0.4cm}
	\par
}{}

\NewEnviron{h3}{%
    \needspace{3\baselineskip}
	\vspace{5mm}
	\textsc{\BODY}
	\par
}

\NewEnviron{margeneg}{ %
\begin{addmargin}[-1cm]{0cm}
\BODY
\end{addmargin}
}

\NewEnviron{html}{%
}

\begin{document}
\meta{url}{/cours/suites-arithmetiques-geometriques/}
\meta{pid}{384}
\meta{titre}{Suites arithmétiques et géométriques}
\meta{type}{cours}
\begin{h2}I - Suites arithmétiques\end{h2}
\cadre{bleu}{Définition}{% id="d10"
     On dit qu'une suite $\left(u_{n}\right)$ est une \textbf{suite arithmétique} s'il existe un nombre $r$ tel que :
     \par
     pour tout $n\in \mathbb{N}$,  $u_{n+1}=u_{n}+r$
     \par
     Le réel $r$ s'appelle la \textbf{raison} de la suite arithmétique.
}
\bloc{cyan}{Remarque}{% id="r10"
     Pour démontrer qu'une suite $\left(u_{n}\right)_{n\in \mathbb{N}}$ est arithmétique, on pourra calculer la différence $u_{n+1}-u_{n}$.
     \par
     Si on constate que la différence est une constante $r$, on pourra affirmer que la suite est arithmétique de raison $r$.
}
\bloc{orange}{Exemple}{% id="e10"
     Soit la suite $\left(u_{n}\right)$ définie par $u_{n}=3n+5$.
     \par
     $u_{n+1}-u_{n}=3\left(n+1\right)+5-\left(3n+5\right)=3$
     \par
     La suite $\left(u_{n}\right)$ est une suite arithmétique de raison $r=3$
}
\cadre{vert}{Propriété}{% id="p20"
     Pour $n$ et $k$ quelconques entiers naturels, si la suite $\left(u_{n}\right)$ est arithmétique de raison $r$ alors \begin{center}$u_{n}=u_{k}+\left(n-k\right)\times r$\end{center}
     En particulier pour $k=0$ :
     \begin{center}$u_{n}=u_{0}+n\times r$\end{center}
}
\bloc{orange}{Exemple}{% id="e20"
     Soit $\left(u_{n}\right)$ la suite arithmétique de premier terme $u_{0}=500$ et de raison $r=3$.
     \par
     La formule précédente permet de calculer directement $u_{100}$ (par exemple) :
     \par
     $u_{100}=u_{0}+100\times r=500+100\times 3=800$
}
\cadre{vert}{Propriété}{% id="p30"
     Réciproquement, si $a$ et $b$ sont deux nombres réels et si la suite $\left(u_{n}\right)$ est définie par $u_{n}=a\times n+b$ alors cette suite est une suite arithmétique de raison $r=a$ et de premier terme $u_{0}=b$.
}
\bloc{cyan}{Démonstration}{% id="m30"
     $u_{n+1}-u_{n}=a\left(n+1\right)+b-\left(an+b\right)=an+a+b-an-b=a$
     \par
     et
     \par
     $u_{0}=a\times 0+b=b$
}
\cadre{vert}{Propriété}{% id="p40"
     Les points de coordonnées $\left(n; u_{n}\right)$ représentant une suite arithmétique $\left(u_{n}\right)$ sont \textbf{alignés}.
}
\bloc{orange}{Exemple}{% id="e40"
     Le graphique ci-dessous représente les premiers termes de la suite arithmétique de raison $r=0,5$ et de premier terme $u_{0}=-1$.
     \begin{center}
          \begin{extern}%width="300" alt="suite arithmétique de raison positive"
               % -+-+-+ variables modifiables
               \resizebox{7cm}{!}{%
                    \def\fonction{0.5*x-1}
                    \def\xmin{-0.8}
                    \def\xmax{7.5}
                    \def\ymin{-1.8}
                    \def\ymax{4.8}
                    \def\xunit{1}  % unités en cm
                    \def\yunit{1}
                    \psset{xunit=\xunit,yunit=\yunit,algebraic=true}
                    \fontsize{15pt}{15pt}\selectfont
                    \begin{pspicture*}[linewidth=1pt](\xmin,\ymin)(\xmax,\ymax)
                         \newrgbcolor{lightblue}{0.8 0.8 1}
                         \psaxes[Dx=1,Dy=1,linewidth=0.75pt]{->}(0,0)(\xmin,\ymin)(\xmax,\ymax)
                         \rput[tr](-0.2,-0.3){$O$}
                         \psplot[plotpoints=1000,linecolor=lightblue]{\xmin}{\xmax}{\fonction}
                         \psdots[linecolor=blue](0,-1) \psdots[linecolor=blue](1,-0.5) \psdots[linecolor=blue](2,0) \psdots[linecolor=blue](3,0.5) \psdots[linecolor=blue](4,1)
                         \psdots[linecolor=blue](5,1.5) \psdots[linecolor=blue](6,2) \psdots[linecolor=blue](7,2.5)
                    \end{pspicture*}
               }
          \end{extern}
     \end{center}
     \begin{center}
          \textit{Suite arithmétique de raison $r=0,5$ et de premier terme $u_{0}=-1$}
\end{center}}
\cadre{rouge}{Théorème}{% id="t40"
     Soit $\left(u_{n}\right)$ une suite arithmétique de raison $r$ :
     \begin{itemize}
          \item si $r > 0$ alors $\left(u_{n}\right)$ est strictement croissante
          \item si $r=0$ alors $\left(u_{n}\right)$ est constante
          \item si $r < 0$ alors $\left(u_{n}\right)$ est strictement décroissante.
     \end{itemize}
}
\bloc{orange}{Exemples}{% id="e40"
     \begin{itemize}
          \item Le graphique de la partie II (ci-dessus) représente les premiers termes d'une suite arithmétique de raison $r=0,5$ \textbf{positive}. Cette suite est \textbf{croissante}.
          \item Le graphique ci-dessous représente les premiers termes d'une suite arithmétique de raison $r=-1$ \textbf{négative}. Cette suite est \textbf{décroissante}.
          \begin{center}
               \begin{extern}%width="300" alt="suite arithmétique de raison négative"
                    % -+-+-+ variables modifiables
                    \resizebox{7cm}{!}{%
                         \def\fonction{-x+3}
                         \def\xmin{-0.8}
                         \def\xmax{7.5}
                         \def\ymin{-4.8}
                         \def\ymax{4.8}
                         \def\xunit{1}  % unités en cm
                         \def\yunit{1}
                         \psset{xunit=\xunit,yunit=\yunit,algebraic=true}
                         \fontsize{15pt}{15pt}\selectfont
                         \begin{pspicture*}[linewidth=1pt](\xmin,\ymin)(\xmax,\ymax)
                              \newrgbcolor{lightred}{1 0.8 0.8}
                              \psaxes[Dx=1,Dy=1,linewidth=0.75pt]{->}(0,0)(\xmin,\ymin)(\xmax,\ymax)
                              \rput[tr](-0.2,-0.3){$O$}
                              \psplot[plotpoints=1000,linecolor=lightred]{\xmin}{\xmax}{\fonction}
                              \psdots[linecolor=red](0,3) \psdots[linecolor=red](1,2) \psdots[linecolor=red](2,1) \psdots[linecolor=red](3,0) \psdots[linecolor=red](4,-1)
                              \psdots[linecolor=red](5,-2) \psdots[linecolor=red](6,-3) \psdots[linecolor=red](7,-4)
                         \end{pspicture*}
                    }
               \end{extern}
          \end{center}
          \begin{center}
               \textit{Suite arithmétique de raison $r=-1$ et de premier terme $u_{0}=3$}
          \end{center}
     \end{itemize}
}
\begin{h2}II - Suites géométriques\end{h2}
\cadre{bleu}{Définition}{% id="d60"
     On dit qu'une suite $\left(u_{n}\right)$ est une \textbf{suite géométrique} s'il existe un nombre réel $q$ tel que, pour tout $n \in  \mathbb{N}$ :
     \begin{center}$u_{n+1}=q \times  u_{n}$\end{center}
     Le réel $q$ s'appelle la \textbf{raison} de la suite géométrique $\left(u_{n}\right)$.
}
\bloc{cyan}{Remarque}{% id="r60"
     Pour démontrer qu'une suite $\left(u_{n}\right)$ dont les termes sont \textbf{non nuls} est une suite géométrique, on pourra calculer le rapport $\frac{u_{n+1}}{u_{n}}$.
     \par
     Si ce rapport est une constante $q$, on pourra affirmer que la suite est une suite géométrique de raison $q$.
}
\bloc{orange}{Exemple}{% id="e60"
     \boite{gray}{Bien revoir les règles de calcul sur les puissances qui servent énormément pour les suites géométriques}
     \par
     Soit la suite $\left(u_{n}\right)$ définie par $u_{n}=\frac{3}{2^{n}}$.
     \par
     Les termes de la suite sont tous strictement positifs et
     \par
     $\frac{u_{n+1}}{u_{n}}=$$\frac{3}{2^{n+1}}\times \frac{2^{n}}{3}=\frac{2^{n}}{2^{n+1}}=$$\frac{2^{n}}{2\times 2^{n}}=\frac{1}{2}$
     \par
     La suite $\left(u_{n}\right)$ est une suite géométrique de raison $\frac{1}{2}$
}
\cadre{vert}{Propriété}{% id="p70"
     Pour $n$ et $k$ quelconques entiers naturels, si la suite $\left(u_{n}\right)$ est géométrique de raison $q$ \begin{center}$u_{n}=u_{k}\times q^{n-k}$.\end{center}
     En particulier pour $k=0$
     \begin{center} $u_{n}=u_{0}\times q^{n}$.\end{center}
}
\cadre{vert}{Propriété}{% id="p80"
     Réciproquement, soient $a$ et $b$ deux nombres réels. La suite $\left(u_{n}\right)$ définie par $u_{n}=a\times b^{n}$ suite est une suite géométrique de raison $q=b$ et de premier terme $u_{0}=a$.
}
\bloc{cyan}{Démonstration}{% id="m80"
     $u_{n+1}=a\times b^{n+1}=a\times b^{n}\times b=u_{n}\times b$
     \par
     $\left(u_{n}\right)$ est donc une suite géométrique de raison $q$.
     \par
     Le premier terme est
     \par
     $u_{0}=a\times b^{0}=a\times 1=a$
}
\cadre{rouge}{Théorème}{% id="t90"
     Soit $\left(u_{n}\right) $une suite géométrique de raison $q  > 0$ et de premier terme strictement positif :
     \begin{itemize}
          \item Si $q > 1$, la suite $\left(u_{n}\right) $est strictement croissante
          \item Si $0 < q < 1$, la suite $\left(u_{n}\right) $est strictement décroissante
          \item Si $q=1$, la suite $\left(u_{n}\right) $est constante
     \end{itemize}
}
\bloc{orange}{Exemples}{% id="e90"
     \begin{center}
          \begin{extern}%width="500" alt="Suites géométriques et raison"
               \begin{tabular}{c c c}
                    \resizebox{5.5cm}{!}{%
                         \def\xmin{-0.8}
                         \def\xmax{4.5}
                         \def\ymin{-0.8}
                         \def\ymax{5.5}
                         \begin{pspicture*}[linewidth=1pt](\xmin,\ymin)(\xmax,\ymax)
                              \psaxes[Dx=1,Dy=1,linewidth=0.75pt]{->}(0,0)(\xmin,\ymin)(\xmax,\ymax)
                              \rput[tr](-0.2,-0.3){$O$}
                              \psdots[linecolor=blue](0,1) \psdots[linecolor=blue](1,1.5) \psdots[linecolor=blue](2,2.25) \psdots[linecolor=blue](3,3.375) \psdots[linecolor=blue](4,5.063)
                         \end{pspicture*}
                    }
                    & ~~~~ &%
                    \resizebox{5.5cm}{!}{%
                         \def\xmin{-0.8}
                         \def\xmax{4.5}
                         \def\ymin{-0.8}
                         \def\ymax{5.5}
                         \begin{pspicture*}[linewidth=1pt](\xmin,\ymin)(\xmax,\ymax)
                              \psaxes[Dx=1,Dy=1,linewidth=0.75pt]{->}(0,0)(\xmin,\ymin)(\xmax,\ymax)
                              \rput[tr](-0.2,-0.3){$O$}
                              \psdots[linecolor=blue](0,3) \psdots[linecolor=blue](1,1.5) \psdots[linecolor=blue](2,0.75) \psdots[linecolor=blue](3,0.375) \psdots[linecolor=blue](4,0.188)
                         \end{pspicture*}
                    }
                    \\
                    Figure 1 & ~~~~ & Figure 2 %
                    \\
               \end{tabular}
          \end{extern}
     \end{center}
     \begin{itemize}
          \item La figure 1 représente une suite géométrique  de raison $q=1,5 > 1$
          \item La figure 2 représente une suite géométrique de raison $q=0,5 < 1$
     \end{itemize}
}

\end{document}
µ
\documentclass[a4paper]{article}

%================================================================================================================================
%
% Packages
%
%================================================================================================================================

\usepackage[T1]{fontenc} 	% pour caractères accentués
\usepackage[utf8]{inputenc}  % encodage utf8
\usepackage[french]{babel}	% langue : français
\usepackage{fourier}			% caractères plus lisibles
\usepackage[dvipsnames]{xcolor} % couleurs
\usepackage{fancyhdr}		% réglage header footer
\usepackage{needspace}		% empêcher sauts de page mal placés
\usepackage{graphicx}		% pour inclure des graphiques
\usepackage{enumitem,cprotect}		% personnalise les listes d'items (nécessaire pour ol, al ...)
\usepackage{hyperref}		% Liens hypertexte
\usepackage{pstricks,pst-all,pst-node,pstricks-add,pst-math,pst-plot,pst-tree,pst-eucl} % pstricks
\usepackage[a4paper,includeheadfoot,top=2cm,left=3cm, bottom=2cm,right=3cm]{geometry} % marges etc.
\usepackage{comment}			% commentaires multilignes
\usepackage{amsmath,environ} % maths (matrices, etc.)
\usepackage{amssymb,makeidx}
\usepackage{bm}				% bold maths
\usepackage{tabularx}		% tableaux
\usepackage{colortbl}		% tableaux en couleur
\usepackage{fontawesome}		% Fontawesome
\usepackage{environ}			% environment with command
\usepackage{fp}				% calculs pour ps-tricks
\usepackage{multido}			% pour ps tricks
\usepackage[np]{numprint}	% formattage nombre
\usepackage{tikz,tkz-tab} 			% package principal TikZ
\usepackage{pgfplots}   % axes
\usepackage{mathrsfs}    % cursives
\usepackage{calc}			% calcul taille boites
\usepackage[scaled=0.875]{helvet} % font sans serif
\usepackage{svg} % svg
\usepackage{scrextend} % local margin
\usepackage{scratch} %scratch
\usepackage{multicol} % colonnes
%\usepackage{infix-RPN,pst-func} % formule en notation polanaise inversée
\usepackage{listings}

%================================================================================================================================
%
% Réglages de base
%
%================================================================================================================================

\lstset{
language=Python,   % R code
literate=
{á}{{\'a}}1
{à}{{\`a}}1
{ã}{{\~a}}1
{é}{{\'e}}1
{è}{{\`e}}1
{ê}{{\^e}}1
{í}{{\'i}}1
{ó}{{\'o}}1
{õ}{{\~o}}1
{ú}{{\'u}}1
{ü}{{\"u}}1
{ç}{{\c{c}}}1
{~}{{ }}1
}


\definecolor{codegreen}{rgb}{0,0.6,0}
\definecolor{codegray}{rgb}{0.5,0.5,0.5}
\definecolor{codepurple}{rgb}{0.58,0,0.82}
\definecolor{backcolour}{rgb}{0.95,0.95,0.92}

\lstdefinestyle{mystyle}{
    backgroundcolor=\color{backcolour},   
    commentstyle=\color{codegreen},
    keywordstyle=\color{magenta},
    numberstyle=\tiny\color{codegray},
    stringstyle=\color{codepurple},
    basicstyle=\ttfamily\footnotesize,
    breakatwhitespace=false,         
    breaklines=true,                 
    captionpos=b,                    
    keepspaces=true,                 
    numbers=left,                    
xleftmargin=2em,
framexleftmargin=2em,            
    showspaces=false,                
    showstringspaces=false,
    showtabs=false,                  
    tabsize=2,
    upquote=true
}

\lstset{style=mystyle}


\lstset{style=mystyle}
\newcommand{\imgdir}{C:/laragon/www/newmc/assets/imgsvg/}
\newcommand{\imgsvgdir}{C:/laragon/www/newmc/assets/imgsvg/}

\definecolor{mcgris}{RGB}{220, 220, 220}% ancien~; pour compatibilité
\definecolor{mcbleu}{RGB}{52, 152, 219}
\definecolor{mcvert}{RGB}{125, 194, 70}
\definecolor{mcmauve}{RGB}{154, 0, 215}
\definecolor{mcorange}{RGB}{255, 96, 0}
\definecolor{mcturquoise}{RGB}{0, 153, 153}
\definecolor{mcrouge}{RGB}{255, 0, 0}
\definecolor{mclightvert}{RGB}{205, 234, 190}

\definecolor{gris}{RGB}{220, 220, 220}
\definecolor{bleu}{RGB}{52, 152, 219}
\definecolor{vert}{RGB}{125, 194, 70}
\definecolor{mauve}{RGB}{154, 0, 215}
\definecolor{orange}{RGB}{255, 96, 0}
\definecolor{turquoise}{RGB}{0, 153, 153}
\definecolor{rouge}{RGB}{255, 0, 0}
\definecolor{lightvert}{RGB}{205, 234, 190}
\setitemize[0]{label=\color{lightvert}  $\bullet$}

\pagestyle{fancy}
\renewcommand{\headrulewidth}{0.2pt}
\fancyhead[L]{maths-cours.fr}
\fancyhead[R]{\thepage}
\renewcommand{\footrulewidth}{0.2pt}
\fancyfoot[C]{}

\newcolumntype{C}{>{\centering\arraybackslash}X}
\newcolumntype{s}{>{\hsize=.35\hsize\arraybackslash}X}

\setlength{\parindent}{0pt}		 
\setlength{\parskip}{3mm}
\setlength{\headheight}{1cm}

\def\ebook{ebook}
\def\book{book}
\def\web{web}
\def\type{web}

\newcommand{\vect}[1]{\overrightarrow{\,\mathstrut#1\,}}

\def\Oij{$\left(\text{O}~;~\vect{\imath},~\vect{\jmath}\right)$}
\def\Oijk{$\left(\text{O}~;~\vect{\imath},~\vect{\jmath},~\vect{k}\right)$}
\def\Ouv{$\left(\text{O}~;~\vect{u},~\vect{v}\right)$}

\hypersetup{breaklinks=true, colorlinks = true, linkcolor = OliveGreen, urlcolor = OliveGreen, citecolor = OliveGreen, pdfauthor={Didier BONNEL - https://www.maths-cours.fr} } % supprime les bordures autour des liens

\renewcommand{\arg}[0]{\text{arg}}

\everymath{\displaystyle}

%================================================================================================================================
%
% Macros - Commandes
%
%================================================================================================================================

\newcommand\meta[2]{    			% Utilisé pour créer le post HTML.
	\def\titre{titre}
	\def\url{url}
	\def\arg{#1}
	\ifx\titre\arg
		\newcommand\maintitle{#2}
		\fancyhead[L]{#2}
		{\Large\sffamily \MakeUppercase{#2}}
		\vspace{1mm}\textcolor{mcvert}{\hrule}
	\fi 
	\ifx\url\arg
		\fancyfoot[L]{\href{https://www.maths-cours.fr#2}{\black \footnotesize{https://www.maths-cours.fr#2}}}
	\fi 
}


\newcommand\TitreC[1]{    		% Titre centré
     \needspace{3\baselineskip}
     \begin{center}\textbf{#1}\end{center}
}

\newcommand\newpar{    		% paragraphe
     \par
}

\newcommand\nosp {    		% commande vide (pas d'espace)
}
\newcommand{\id}[1]{} %ignore

\newcommand\boite[2]{				% Boite simple sans titre
	\vspace{5mm}
	\setlength{\fboxrule}{0.2mm}
	\setlength{\fboxsep}{5mm}	
	\fcolorbox{#1}{#1!3}{\makebox[\linewidth-2\fboxrule-2\fboxsep]{
  		\begin{minipage}[t]{\linewidth-2\fboxrule-4\fboxsep}\setlength{\parskip}{3mm}
  			 #2
  		\end{minipage}
	}}
	\vspace{5mm}
}

\newcommand\CBox[4]{				% Boites
	\vspace{5mm}
	\setlength{\fboxrule}{0.2mm}
	\setlength{\fboxsep}{5mm}
	
	\fcolorbox{#1}{#1!3}{\makebox[\linewidth-2\fboxrule-2\fboxsep]{
		\begin{minipage}[t]{1cm}\setlength{\parskip}{3mm}
	  		\textcolor{#1}{\LARGE{#2}}    
 	 	\end{minipage}  
  		\begin{minipage}[t]{\linewidth-2\fboxrule-4\fboxsep}\setlength{\parskip}{3mm}
			\raisebox{1.2mm}{\normalsize\sffamily{\textcolor{#1}{#3}}}						
  			 #4
  		\end{minipage}
	}}
	\vspace{5mm}
}

\newcommand\cadre[3]{				% Boites convertible html
	\par
	\vspace{2mm}
	\setlength{\fboxrule}{0.1mm}
	\setlength{\fboxsep}{5mm}
	\fcolorbox{#1}{white}{\makebox[\linewidth-2\fboxrule-2\fboxsep]{
  		\begin{minipage}[t]{\linewidth-2\fboxrule-4\fboxsep}\setlength{\parskip}{3mm}
			\raisebox{-2.5mm}{\sffamily \small{\textcolor{#1}{\MakeUppercase{#2}}}}		
			\par		
  			 #3
 	 		\end{minipage}
	}}
		\vspace{2mm}
	\par
}

\newcommand\bloc[3]{				% Boites convertible html sans bordure
     \needspace{2\baselineskip}
     {\sffamily \small{\textcolor{#1}{\MakeUppercase{#2}}}}    
		\par		
  			 #3
		\par
}

\newcommand\CHelp[1]{
     \CBox{Plum}{\faInfoCircle}{À RETENIR}{#1}
}

\newcommand\CUp[1]{
     \CBox{NavyBlue}{\faThumbsOUp}{EN PRATIQUE}{#1}
}

\newcommand\CInfo[1]{
     \CBox{Sepia}{\faArrowCircleRight}{REMARQUE}{#1}
}

\newcommand\CRedac[1]{
     \CBox{PineGreen}{\faEdit}{BIEN R\'EDIGER}{#1}
}

\newcommand\CError[1]{
     \CBox{Red}{\faExclamationTriangle}{ATTENTION}{#1}
}

\newcommand\TitreExo[2]{
\needspace{4\baselineskip}
 {\sffamily\large EXERCICE #1\ (\emph{#2 points})}
\vspace{5mm}
}

\newcommand\img[2]{
          \includegraphics[width=#2\paperwidth]{\imgdir#1}
}

\newcommand\imgsvg[2]{
       \begin{center}   \includegraphics[width=#2\paperwidth]{\imgsvgdir#1} \end{center}
}


\newcommand\Lien[2]{
     \href{#1}{#2 \tiny \faExternalLink}
}
\newcommand\mcLien[2]{
     \href{https~://www.maths-cours.fr/#1}{#2 \tiny \faExternalLink}
}

\newcommand{\euro}{\eurologo{}}

%================================================================================================================================
%
% Macros - Environement
%
%================================================================================================================================

\newenvironment{tex}{ %
}
{%
}

\newenvironment{indente}{ %
	\setlength\parindent{10mm}
}

{
	\setlength\parindent{0mm}
}

\newenvironment{corrige}{%
     \needspace{3\baselineskip}
     \medskip
     \textbf{\textsc{Corrigé}}
     \medskip
}
{
}

\newenvironment{extern}{%
     \begin{center}
     }
     {
     \end{center}
}

\NewEnviron{code}{%
	\par
     \boite{gray}{\texttt{%
     \BODY
     }}
     \par
}

\newenvironment{vbloc}{% boite sans cadre empeche saut de page
     \begin{minipage}[t]{\linewidth}
     }
     {
     \end{minipage}
}
\NewEnviron{h2}{%
    \needspace{3\baselineskip}
    \vspace{0.6cm}
	\noindent \MakeUppercase{\sffamily \large \BODY}
	\vspace{1mm}\textcolor{mcgris}{\hrule}\vspace{0.4cm}
	\par
}{}

\NewEnviron{h3}{%
    \needspace{3\baselineskip}
	\vspace{5mm}
	\textsc{\BODY}
	\par
}

\NewEnviron{margeneg}{ %
\begin{addmargin}[-1cm]{0cm}
\BODY
\end{addmargin}
}

\NewEnviron{html}{%
}

\begin{document}
\meta{url}{/cours/statistiques-une-variable/}
\meta{pid}{387}
\meta{titre}{Statistiques en Première ES et L}
\meta{type}{cours}
\begin{h2}I - Rappels\end{h2}
\cadre{bleu}{Définitions}{% id="d10"
     Les statistiques permettent d'étudier un \textbf{caractère} d'une \textbf{population}.
     \par
     Le nombre d'éléments de la population s'appelle l'\textbf{effectif global} (ou l'\textbf{effectif total}).
     \par
     Pour une valeur de caractère donnée, l'\textbf{effectif} est le nombre d'éléments correspondant à cette valeur.
     \par
     Une \textbf{série statistique} est un tableau donnant les effectifs pour chacune des valeurs possibles du caractère.
}
\bloc{orange}{Exemple}{% id="e10"
     On effectue une étude portant sur l'âge des élèves d'un lycée.
     \begin{itemize}
          \item le \textbf{caractère} étudié est l'âge
          \item la \textbf{population} est l'ensemble des élèves du lycée
          \item l'\textbf{effectif global} est le nombre d'élèves du lycée
          \item le tableau ci-dessous est la \textbf{série statistique} pour ce caractère dans un lycée donné :
          \begin{center}
               \begin{tabularx}{\linewidth}{|m{2.5cm}|*{7}{>{\centering \arraybackslash }X|}}%class="compact" width="600"
                    \hline
                    \textbf{âges (en années)}  &  14  &  15  &  16  &  17  &  18  &  19  &  20
                    \\ \hline
                    \textbf{effectifs}  &  3  &  22  &  65  &  82  &  59  &  35  &  2
                    \\ \hline
               \end{tabularx}
          \end{center}
     \end{itemize}
}
\begin{h2}II - Médiane - Quartiles - Déciles\end{h2}
\cadre{bleu}{Définition}{% id="d20"
     La \textbf{médiane} d'une série statistique est la valeur du caractère qui partage la population en deux classes de même effectif.
}
\bloc{cyan}{Remarque}{% id="r20"
     En pratique pour trouver la médiane d'une série statistique d'effectif global $n$ :
     \begin{itemize}
          \item On ordonne les valeurs du caractère dans l'ordre croissant.
          \item Si $n$ est pair, la médiane sera la moyenne des valeurs du terme de rang $\frac{n}{2}$ et du terme de rang $\frac{n}{2}+1$.
          \item Si $n$ est impair, la médiane sera la valeur du terme de rang $\frac{n+1}{2}$.
          \item Lorsque l'effectif global est élevé, il est souvent utile de calculer les effectifs cumulés pour trouver cette valeur.
     \end{itemize}
}
\bloc{orange}{Exemple}{% id="e20"
     On lance 10 fois un dé à six faces. Les résultats obtenus sont : $1;5;6;6;3;2;3;1;4;1$
     \par
     On trie ces valeurs par ordre croissant : $1;1;1;2;3;3;4;5;6;6$
     \par
     $n=10$ étant pair on effectue la moyenne du cinquième et du sixième terme ($3$ et $3$) et la médiane est donc $3$.
}
\cadre{bleu}{Définitions}{% id="d30"
     \begin{itemize}
          \item Le \textbf{premier quartile} Q1 d'une série statistique est la plus petite valeur des termes de la série pour laquelle au moins un quart des données sont inférieures ou égales à Q1.
          \item Le \textbf{troisième quartile} Q3  d'une série statistique est la plus petite valeur des termes de la série pour laquelle au moins trois quarts des données sont inférieures ou égales à Q3.
          \item Le \textbf{premier décile} D1 d'une série statistique est la plus petite valeur  des termes de la série pour laquelle au moins 10\% des données sont inférieures ou égales à D1.
          \item Le \textbf{neuvième décile} D9 d'une série statistique est la plus petite valeur des termes de la série pour laquelle au moins 90\% des données sont inférieures ou égales à D9.
     \end{itemize}
}
\cadre{bleu}{Définition}{% id="d35"
     L'\textbf{écart interquartile} est la différence entre le troisième et le premier quartile $Q_{3}-Q_{1}$.
}
\bloc{cyan}{Remarque}{% id="r35"
     L'écart interquartile mesure la dispersion autour de la médiane.
}
\begin{h2}III - Diagramme en boîte\end{h2}
\begin{center}
     \begin{extern}%width="320" alt="Diagramme en boîte"
          \resizebox{6cm}{!}{
               \psset{xunit=0.4cm,yunit=0.4cm,algebraic=true,dimen=middle,dotstyle=o,dotsize=5pt 0,linewidth=1pt,arrowsize=3pt 2,arrowinset=0.25}
               \begin{pspicture*}(1.,0.5)(23.,5.5)
                    \psline(3.,2.5)(3.,1.5)
                    \psline(3.,2.)(6.,2.)
                    \psline(14.,2.)(18.,2.)
                    \psline(18.,2.5)(18.,1.5)
                    \psline[linecolor=blue](6.,3.)(6.,1.)
                    \psline[linecolor=blue](6.,1.)(14.,1.)
                    \psline[linecolor=blue](14.,1.)(14.,3.)
                    \psline[linecolor=blue](14.,3.)(6.,3.)
                    \psline[linecolor=red](11.,3.)(11.,1.)
                    \rput[tl](0.5,3.8){Minimum}
                    \rput[tl](5.56,4.2){\blue{Q1}}
                    \rput[tl](13.55,4.2){\blue{Q3}}
                    \rput[tl](16,3.8){Maximum}
                    \rput[tl](9.,4.2){\red{Médiane}}
               \end{pspicture*}
          }
     \end{extern}
\end{center}
On peut résumer un certain nombre d'informations relatives à une série statistique grâce à un \textbf{diagramme en boîte} (aussi appelé \textit{boîte à moustache}) qui fait apparaître (voir figure ci-dessus) :
\begin{itemize}
     \item les valeurs minimum et maximum
     \item le premier et le troisième quartile (Q1 et Q3)
     \item la médiane
\end{itemize}
\par
\bloc{orange}{Exemple}{% id="e40"
     \begin{center}
          \begin{extern}%width="400" alt="Exemple boîte à moustache"
               \resizebox{7cm}{!}{
                    \psset{xunit=0.4cm,yunit=0.4cm,algebraic=true,dimen=middle,dotstyle=o,dotsize=5pt 0,linewidth=1pt,arrowsize=3pt 2,arrowinset=0.25}
                    \begin{pspicture*}(-1.,-1.5)(23.,4.)
                         \psaxes[linewidth=0.8pt,xAxis=true,yAxis=false,Dx=5.,Dy=5.,ticksize=-2pt 0,subticks=2]{->}(0,0)(-1.,-1.)(23.,4.)
                         \psline(3.,2.5)(3.,1.5)
                         \psline(3.,2.)(6.,2.)
                         \psline(14.,2.)(20.,2.)
                         \psline(20.,2.5)(20.,1.5)
                         \psline[linecolor=blue](6.,3.)(6.,1.)
                         \psline[linecolor=blue](6.,1.)(14.,1.)
                         \psline[linecolor=blue](14.,1.)(14.,3.)
                         \psline[linecolor=blue](14.,3.)(6.,3.)
                         \psline[linecolor=blue](9.5,3.)(9.5,1.)
                         \psline[linecolor=red](9.5,3.)(9.5,1.)
                    \end{pspicture*}
               }
          \end{extern}
     \end{center}
     La figure ci-dessus représente une série statistique de valeurs extrêmes 3 et 20, de premier quartile 6, de troisième quartile 14 et de médiane 9,5.
}

\end{document}
µ
\documentclass[a4paper]{article}

%================================================================================================================================
%
% Packages
%
%================================================================================================================================

\usepackage[T1]{fontenc} 	% pour caractères accentués
\usepackage[utf8]{inputenc}  % encodage utf8
\usepackage[french]{babel}	% langue : français
\usepackage{fourier}			% caractères plus lisibles
\usepackage[dvipsnames]{xcolor} % couleurs
\usepackage{fancyhdr}		% réglage header footer
\usepackage{needspace}		% empêcher sauts de page mal placés
\usepackage{graphicx}		% pour inclure des graphiques
\usepackage{enumitem,cprotect}		% personnalise les listes d'items (nécessaire pour ol, al ...)
\usepackage{hyperref}		% Liens hypertexte
\usepackage{pstricks,pst-all,pst-node,pstricks-add,pst-math,pst-plot,pst-tree,pst-eucl} % pstricks
\usepackage[a4paper,includeheadfoot,top=2cm,left=3cm, bottom=2cm,right=3cm]{geometry} % marges etc.
\usepackage{comment}			% commentaires multilignes
\usepackage{amsmath,environ} % maths (matrices, etc.)
\usepackage{amssymb,makeidx}
\usepackage{bm}				% bold maths
\usepackage{tabularx}		% tableaux
\usepackage{colortbl}		% tableaux en couleur
\usepackage{fontawesome}		% Fontawesome
\usepackage{environ}			% environment with command
\usepackage{fp}				% calculs pour ps-tricks
\usepackage{multido}			% pour ps tricks
\usepackage[np]{numprint}	% formattage nombre
\usepackage{tikz,tkz-tab} 			% package principal TikZ
\usepackage{pgfplots}   % axes
\usepackage{mathrsfs}    % cursives
\usepackage{calc}			% calcul taille boites
\usepackage[scaled=0.875]{helvet} % font sans serif
\usepackage{svg} % svg
\usepackage{scrextend} % local margin
\usepackage{scratch} %scratch
\usepackage{multicol} % colonnes
%\usepackage{infix-RPN,pst-func} % formule en notation polanaise inversée
\usepackage{listings}

%================================================================================================================================
%
% Réglages de base
%
%================================================================================================================================

\lstset{
language=Python,   % R code
literate=
{á}{{\'a}}1
{à}{{\`a}}1
{ã}{{\~a}}1
{é}{{\'e}}1
{è}{{\`e}}1
{ê}{{\^e}}1
{í}{{\'i}}1
{ó}{{\'o}}1
{õ}{{\~o}}1
{ú}{{\'u}}1
{ü}{{\"u}}1
{ç}{{\c{c}}}1
{~}{{ }}1
}


\definecolor{codegreen}{rgb}{0,0.6,0}
\definecolor{codegray}{rgb}{0.5,0.5,0.5}
\definecolor{codepurple}{rgb}{0.58,0,0.82}
\definecolor{backcolour}{rgb}{0.95,0.95,0.92}

\lstdefinestyle{mystyle}{
    backgroundcolor=\color{backcolour},   
    commentstyle=\color{codegreen},
    keywordstyle=\color{magenta},
    numberstyle=\tiny\color{codegray},
    stringstyle=\color{codepurple},
    basicstyle=\ttfamily\footnotesize,
    breakatwhitespace=false,         
    breaklines=true,                 
    captionpos=b,                    
    keepspaces=true,                 
    numbers=left,                    
xleftmargin=2em,
framexleftmargin=2em,            
    showspaces=false,                
    showstringspaces=false,
    showtabs=false,                  
    tabsize=2,
    upquote=true
}

\lstset{style=mystyle}


\lstset{style=mystyle}
\newcommand{\imgdir}{C:/laragon/www/newmc/assets/imgsvg/}
\newcommand{\imgsvgdir}{C:/laragon/www/newmc/assets/imgsvg/}

\definecolor{mcgris}{RGB}{220, 220, 220}% ancien~; pour compatibilité
\definecolor{mcbleu}{RGB}{52, 152, 219}
\definecolor{mcvert}{RGB}{125, 194, 70}
\definecolor{mcmauve}{RGB}{154, 0, 215}
\definecolor{mcorange}{RGB}{255, 96, 0}
\definecolor{mcturquoise}{RGB}{0, 153, 153}
\definecolor{mcrouge}{RGB}{255, 0, 0}
\definecolor{mclightvert}{RGB}{205, 234, 190}

\definecolor{gris}{RGB}{220, 220, 220}
\definecolor{bleu}{RGB}{52, 152, 219}
\definecolor{vert}{RGB}{125, 194, 70}
\definecolor{mauve}{RGB}{154, 0, 215}
\definecolor{orange}{RGB}{255, 96, 0}
\definecolor{turquoise}{RGB}{0, 153, 153}
\definecolor{rouge}{RGB}{255, 0, 0}
\definecolor{lightvert}{RGB}{205, 234, 190}
\setitemize[0]{label=\color{lightvert}  $\bullet$}

\pagestyle{fancy}
\renewcommand{\headrulewidth}{0.2pt}
\fancyhead[L]{maths-cours.fr}
\fancyhead[R]{\thepage}
\renewcommand{\footrulewidth}{0.2pt}
\fancyfoot[C]{}

\newcolumntype{C}{>{\centering\arraybackslash}X}
\newcolumntype{s}{>{\hsize=.35\hsize\arraybackslash}X}

\setlength{\parindent}{0pt}		 
\setlength{\parskip}{3mm}
\setlength{\headheight}{1cm}

\def\ebook{ebook}
\def\book{book}
\def\web{web}
\def\type{web}

\newcommand{\vect}[1]{\overrightarrow{\,\mathstrut#1\,}}

\def\Oij{$\left(\text{O}~;~\vect{\imath},~\vect{\jmath}\right)$}
\def\Oijk{$\left(\text{O}~;~\vect{\imath},~\vect{\jmath},~\vect{k}\right)$}
\def\Ouv{$\left(\text{O}~;~\vect{u},~\vect{v}\right)$}

\hypersetup{breaklinks=true, colorlinks = true, linkcolor = OliveGreen, urlcolor = OliveGreen, citecolor = OliveGreen, pdfauthor={Didier BONNEL - https://www.maths-cours.fr} } % supprime les bordures autour des liens

\renewcommand{\arg}[0]{\text{arg}}

\everymath{\displaystyle}

%================================================================================================================================
%
% Macros - Commandes
%
%================================================================================================================================

\newcommand\meta[2]{    			% Utilisé pour créer le post HTML.
	\def\titre{titre}
	\def\url{url}
	\def\arg{#1}
	\ifx\titre\arg
		\newcommand\maintitle{#2}
		\fancyhead[L]{#2}
		{\Large\sffamily \MakeUppercase{#2}}
		\vspace{1mm}\textcolor{mcvert}{\hrule}
	\fi 
	\ifx\url\arg
		\fancyfoot[L]{\href{https://www.maths-cours.fr#2}{\black \footnotesize{https://www.maths-cours.fr#2}}}
	\fi 
}


\newcommand\TitreC[1]{    		% Titre centré
     \needspace{3\baselineskip}
     \begin{center}\textbf{#1}\end{center}
}

\newcommand\newpar{    		% paragraphe
     \par
}

\newcommand\nosp {    		% commande vide (pas d'espace)
}
\newcommand{\id}[1]{} %ignore

\newcommand\boite[2]{				% Boite simple sans titre
	\vspace{5mm}
	\setlength{\fboxrule}{0.2mm}
	\setlength{\fboxsep}{5mm}	
	\fcolorbox{#1}{#1!3}{\makebox[\linewidth-2\fboxrule-2\fboxsep]{
  		\begin{minipage}[t]{\linewidth-2\fboxrule-4\fboxsep}\setlength{\parskip}{3mm}
  			 #2
  		\end{minipage}
	}}
	\vspace{5mm}
}

\newcommand\CBox[4]{				% Boites
	\vspace{5mm}
	\setlength{\fboxrule}{0.2mm}
	\setlength{\fboxsep}{5mm}
	
	\fcolorbox{#1}{#1!3}{\makebox[\linewidth-2\fboxrule-2\fboxsep]{
		\begin{minipage}[t]{1cm}\setlength{\parskip}{3mm}
	  		\textcolor{#1}{\LARGE{#2}}    
 	 	\end{minipage}  
  		\begin{minipage}[t]{\linewidth-2\fboxrule-4\fboxsep}\setlength{\parskip}{3mm}
			\raisebox{1.2mm}{\normalsize\sffamily{\textcolor{#1}{#3}}}						
  			 #4
  		\end{minipage}
	}}
	\vspace{5mm}
}

\newcommand\cadre[3]{				% Boites convertible html
	\par
	\vspace{2mm}
	\setlength{\fboxrule}{0.1mm}
	\setlength{\fboxsep}{5mm}
	\fcolorbox{#1}{white}{\makebox[\linewidth-2\fboxrule-2\fboxsep]{
  		\begin{minipage}[t]{\linewidth-2\fboxrule-4\fboxsep}\setlength{\parskip}{3mm}
			\raisebox{-2.5mm}{\sffamily \small{\textcolor{#1}{\MakeUppercase{#2}}}}		
			\par		
  			 #3
 	 		\end{minipage}
	}}
		\vspace{2mm}
	\par
}

\newcommand\bloc[3]{				% Boites convertible html sans bordure
     \needspace{2\baselineskip}
     {\sffamily \small{\textcolor{#1}{\MakeUppercase{#2}}}}    
		\par		
  			 #3
		\par
}

\newcommand\CHelp[1]{
     \CBox{Plum}{\faInfoCircle}{À RETENIR}{#1}
}

\newcommand\CUp[1]{
     \CBox{NavyBlue}{\faThumbsOUp}{EN PRATIQUE}{#1}
}

\newcommand\CInfo[1]{
     \CBox{Sepia}{\faArrowCircleRight}{REMARQUE}{#1}
}

\newcommand\CRedac[1]{
     \CBox{PineGreen}{\faEdit}{BIEN R\'EDIGER}{#1}
}

\newcommand\CError[1]{
     \CBox{Red}{\faExclamationTriangle}{ATTENTION}{#1}
}

\newcommand\TitreExo[2]{
\needspace{4\baselineskip}
 {\sffamily\large EXERCICE #1\ (\emph{#2 points})}
\vspace{5mm}
}

\newcommand\img[2]{
          \includegraphics[width=#2\paperwidth]{\imgdir#1}
}

\newcommand\imgsvg[2]{
       \begin{center}   \includegraphics[width=#2\paperwidth]{\imgsvgdir#1} \end{center}
}


\newcommand\Lien[2]{
     \href{#1}{#2 \tiny \faExternalLink}
}
\newcommand\mcLien[2]{
     \href{https~://www.maths-cours.fr/#1}{#2 \tiny \faExternalLink}
}

\newcommand{\euro}{\eurologo{}}

%================================================================================================================================
%
% Macros - Environement
%
%================================================================================================================================

\newenvironment{tex}{ %
}
{%
}

\newenvironment{indente}{ %
	\setlength\parindent{10mm}
}

{
	\setlength\parindent{0mm}
}

\newenvironment{corrige}{%
     \needspace{3\baselineskip}
     \medskip
     \textbf{\textsc{Corrigé}}
     \medskip
}
{
}

\newenvironment{extern}{%
     \begin{center}
     }
     {
     \end{center}
}

\NewEnviron{code}{%
	\par
     \boite{gray}{\texttt{%
     \BODY
     }}
     \par
}

\newenvironment{vbloc}{% boite sans cadre empeche saut de page
     \begin{minipage}[t]{\linewidth}
     }
     {
     \end{minipage}
}
\NewEnviron{h2}{%
    \needspace{3\baselineskip}
    \vspace{0.6cm}
	\noindent \MakeUppercase{\sffamily \large \BODY}
	\vspace{1mm}\textcolor{mcgris}{\hrule}\vspace{0.4cm}
	\par
}{}

\NewEnviron{h3}{%
    \needspace{3\baselineskip}
	\vspace{5mm}
	\textsc{\BODY}
	\par
}

\NewEnviron{margeneg}{ %
\begin{addmargin}[-1cm]{0cm}
\BODY
\end{addmargin}
}

\NewEnviron{html}{%
}

\begin{document}
\meta{url}{/cours/variable-aleatoire-loi-probabilite/}
\meta{pid}{389}
\meta{titre}{Variable aléatoire - Loi de probabilité}
\meta{type}{cours}
\begin{h2}I - Rappels de probabilités\end{h2}
\cadre{bleu}{Définitions}{% id="d10"
     Une expérience \textbf{aléatoire} est une expérience dont le résultat dépend du hasard.
     \par
     Chacun des résultats possibles s'appelle une \textbf{éventualité} (ou une \textbf{issue} ou un \textbf{évènement élémentaire})
     \par
     L'ensemble de tous les résultats possibles d'une expérience aléatoire s'appelle l'\textbf{univers} de l'expérience.
}
\bloc{orange}{Exemple}{% id="e10"
     Par exemple, le lancer d'un dé à six faces est une expérience aléatoire. \textit{"Obtenir un 6 avec le dé"} est une éventualité. L'univers possède 6 éventualités; on peut le représenter par l'ensemble:
     \par
     $\Omega =\left\{1;2;3;4;5;6\right\}$
}
\cadre{bleu}{Définition}{% id="d20"
     Soit une expérience aléatoire ayant comme univers:
     \par
     $\Omega =\left\{x_{1}; x_{2};. . .; x_{n}\right\}$
     \par
     On définit une \textbf{probabilité} sur $\Omega $ en associant, à chaque éventualité $x_{i}$, un réel $p_{i}$ compris entre 0 et 1 tel que la somme de tous les $p_{i}$ soit égale à 1.
}
\bloc{cyan}{Remarques}{% id="r20"
     \begin{itemize}
          \item En pratique, pour définir les probabilités $p_{i}$ on peut effectuer un très grand nombre de fois l'expérience aléatoire. La fréquence des résultats obtenus permet d'obtenir une estimation de la loi de probabilité. Par exemple, si en lançant 1 000 000 de fois un dé, on obtient 166 724 fois la face "6" on considérera que la probabilité d'obtenir un "6" est d'environ $\frac{166~724}{1~000~000} \approx \frac{1}{6}$
          \item A condition de faire certaines hypothèses (par exemple : "\textit{le dé n'est pas truqué}") les théorèmes qui suivent permettent de calculer les lois de probabilité de certaines expériences sans avoir recours aux statistiques. Les statistiques peuvent alors servir à valider les hypothèses que l'on a faites au départ.
     \end{itemize}
}
\cadre{bleu}{Définition et propriété}{% id="d30"
     On dit que l'on est en situation d'\textbf{équiprobabilité} si toutes les éventualités on la même probabilité.
     \par
     Cette probabilité est alors $p=\frac{1}{n}$ où $n$ est le nombre total d'éventualités.
}
\bloc{cyan}{Remarque}{% id="r30"
     Dans les exercices, on considérera qu'il y a équiprobabilité si l'énoncé indique que l'on jette une pièce "\textit{équilibrée}", qu'on lance un dé "\textit{non truqué}", qu'on tire une carte "\textit{au hasard}" , etc.
}
\bloc{orange}{Exemples}{% id="e30"
     \begin{itemize}\item Si l'on jette une pièce non truquée, la probabilité d'obtenir \textit{pile} est $p=\frac{1}{2}$
     \item Pour un dé à six faces non truqué, la probabilité d'obtenir une face donnée est $p=\frac{1}{6}$ \end{itemize}
}
\begin{h2}II - Variables aléatoires\end{h2}
\cadre{bleu}{Définition}{% id="d50"
     On définit une \textbf{variable aléatoire} en associant un nombre réel à chaque éventualité d'une expérience aléatoire.
}
\bloc{orange}{Exemples}{% id="e50"
     \begin{itemize}\item On mise 1€  sur le numéro 1  à la roulette. On gagne 35€ (36€ - la mise) si le numéro sort. On perd sa mise (soit 1€) dans les autres cas. On peut définir une variable aléatoire représentant le gain algébrique du joueur. Cette variable aléatoire peut prendre la valeur 35 (en cas de gain) ou -1 (en cas de perte).
          \item On lance 4 fois une pièce de monnaie. On peut définir une variable aléatoire égale au nombre de "\textit{faces}" obtenues.
          \par
          Les valeurs possibles pour cette variable sont : 0; 1; 2; 3 ou 4.
     \end{itemize}
}
\bloc{cyan}{Notations}{% id="n50"
     \begin{itemize}\item On note généralement une variable aléatoire à l'aide d'une lettre majuscule (le plus souvent $X$)
     \item Si la variable aléatoire $X$ peut prendre les valeurs $a_{1}, a_{2}, . . . a_{n}$, on note $\left(X=a_{i}\right)$ l'évènement : "$X$ prend la valeur $a_{i}$"\end{itemize}
}
\cadre{bleu}{Définition}{% id="d60"
     La loi de probabilité d'une variable aléatoire $X$ associe à chaque valeur $a_{i}$ prise par $X$ la probabilité de l'événement $\left(X = a_{i}\right)$.
     \par
     On la représente généralement sous forme de tableau.
}
\bloc{orange}{Exemples}{% id="e60"
     \begin{itemize}\item Si l'on reprend l'exemple de la roulette (ci-dessus) et si on suppose que la probabilité de sortie de chacun des 37 numéros (0 à 36) est égale, la probabilité de gain est de $\frac{1}{37}$ et la probabilité de perte $\frac{36}{37}$.
          \par
          La loi de probabilité est donnée par le tableau suivant :
          \begin{center}
               \begin{tabular}{|c|c|c|}%class="compact" width="600"
                    \hline
                    $a_{i}$  &  $-1$  &  $35$
                    \\ \hline
                    $p\left(X=a_{i}\right)$  &  $\frac{36}{37}$  &  $\frac{1}{37}$
                    \\ \hline
               \end{tabular}
          \end{center}
          \item Si on lance 4 fois une pièce de monnaie équilibrée, on montre à l'aide d'un arbre que la variable aléatoire $X$ donnant le nombre de "\textit{faces}" obtenues suit la loi de probabilité donnée par le tableau ci-dessous :
          \begin{center}
               \begin{tabular}{|c|c|c|c|c|c|}%class="compact" width="600"
                    \hline
                    $a_{i}$  &  $0$  &  $1$  &  $2$  &  $3$  &  $4$
                    \\ \hline
                    $p\left(X=a_{i}\right)$  &  $\frac{1}{16}$  &  $\frac{1}{4}$ &  $\frac{3}{8}$  &  $\frac{1}{4}$ &   $\frac{1}{16}$
                    \\ \hline
               \end{tabular}
          \end{center}
     \end{itemize}
}
\cadre{bleu}{Définition}{% id="d70"
     Soit $X$ une variable aléatoire qui prend les valeurs $a_{i}$ avec les probabilités $p_{i}=p\left(X=a_{i}\right)$.
     \par
     On appelle \textbf{espérance mathématique} de $X$ le nombre :
     \par
     $E\left(X\right)= a_{1}\times p_{1}+a_{2}\times p_{2}+. . . +a_{n}\times p_{n}$
}
\bloc{cyan}{Remarque}{% id="r70"
     Ce nombre peut s'interpréter comme une valeur moyenne de $X$ si l'on répète un grand nombre de fois l'expérience.
}
\bloc{orange}{Exemple}{% id="e70"
     Pour l'exemple de la roulette on a :
     \par
     $E\left(X\right)=-1\times \frac{36}{37}+35\times \frac{1}{37}=-\frac{1}{37}$
     \par
     L'espérance est négative, ce qui signifie qu'en moyenne, le jeu n'est pas favorable au joueur.
}

\end{document}
µ
\documentclass[a4paper]{article}

%================================================================================================================================
%
% Packages
%
%================================================================================================================================

\usepackage[T1]{fontenc} 	% pour caractères accentués
\usepackage[utf8]{inputenc}  % encodage utf8
\usepackage[french]{babel}	% langue : français
\usepackage{fourier}			% caractères plus lisibles
\usepackage[dvipsnames]{xcolor} % couleurs
\usepackage{fancyhdr}		% réglage header footer
\usepackage{needspace}		% empêcher sauts de page mal placés
\usepackage{graphicx}		% pour inclure des graphiques
\usepackage{enumitem,cprotect}		% personnalise les listes d'items (nécessaire pour ol, al ...)
\usepackage{hyperref}		% Liens hypertexte
\usepackage{pstricks,pst-all,pst-node,pstricks-add,pst-math,pst-plot,pst-tree,pst-eucl} % pstricks
\usepackage[a4paper,includeheadfoot,top=2cm,left=3cm, bottom=2cm,right=3cm]{geometry} % marges etc.
\usepackage{comment}			% commentaires multilignes
\usepackage{amsmath,environ} % maths (matrices, etc.)
\usepackage{amssymb,makeidx}
\usepackage{bm}				% bold maths
\usepackage{tabularx}		% tableaux
\usepackage{colortbl}		% tableaux en couleur
\usepackage{fontawesome}		% Fontawesome
\usepackage{environ}			% environment with command
\usepackage{fp}				% calculs pour ps-tricks
\usepackage{multido}			% pour ps tricks
\usepackage[np]{numprint}	% formattage nombre
\usepackage{tikz,tkz-tab} 			% package principal TikZ
\usepackage{pgfplots}   % axes
\usepackage{mathrsfs}    % cursives
\usepackage{calc}			% calcul taille boites
\usepackage[scaled=0.875]{helvet} % font sans serif
\usepackage{svg} % svg
\usepackage{scrextend} % local margin
\usepackage{scratch} %scratch
\usepackage{multicol} % colonnes
%\usepackage{infix-RPN,pst-func} % formule en notation polanaise inversée
\usepackage{listings}

%================================================================================================================================
%
% Réglages de base
%
%================================================================================================================================

\lstset{
language=Python,   % R code
literate=
{á}{{\'a}}1
{à}{{\`a}}1
{ã}{{\~a}}1
{é}{{\'e}}1
{è}{{\`e}}1
{ê}{{\^e}}1
{í}{{\'i}}1
{ó}{{\'o}}1
{õ}{{\~o}}1
{ú}{{\'u}}1
{ü}{{\"u}}1
{ç}{{\c{c}}}1
{~}{{ }}1
}


\definecolor{codegreen}{rgb}{0,0.6,0}
\definecolor{codegray}{rgb}{0.5,0.5,0.5}
\definecolor{codepurple}{rgb}{0.58,0,0.82}
\definecolor{backcolour}{rgb}{0.95,0.95,0.92}

\lstdefinestyle{mystyle}{
    backgroundcolor=\color{backcolour},   
    commentstyle=\color{codegreen},
    keywordstyle=\color{magenta},
    numberstyle=\tiny\color{codegray},
    stringstyle=\color{codepurple},
    basicstyle=\ttfamily\footnotesize,
    breakatwhitespace=false,         
    breaklines=true,                 
    captionpos=b,                    
    keepspaces=true,                 
    numbers=left,                    
xleftmargin=2em,
framexleftmargin=2em,            
    showspaces=false,                
    showstringspaces=false,
    showtabs=false,                  
    tabsize=2,
    upquote=true
}

\lstset{style=mystyle}


\lstset{style=mystyle}
\newcommand{\imgdir}{C:/laragon/www/newmc/assets/imgsvg/}
\newcommand{\imgsvgdir}{C:/laragon/www/newmc/assets/imgsvg/}

\definecolor{mcgris}{RGB}{220, 220, 220}% ancien~; pour compatibilité
\definecolor{mcbleu}{RGB}{52, 152, 219}
\definecolor{mcvert}{RGB}{125, 194, 70}
\definecolor{mcmauve}{RGB}{154, 0, 215}
\definecolor{mcorange}{RGB}{255, 96, 0}
\definecolor{mcturquoise}{RGB}{0, 153, 153}
\definecolor{mcrouge}{RGB}{255, 0, 0}
\definecolor{mclightvert}{RGB}{205, 234, 190}

\definecolor{gris}{RGB}{220, 220, 220}
\definecolor{bleu}{RGB}{52, 152, 219}
\definecolor{vert}{RGB}{125, 194, 70}
\definecolor{mauve}{RGB}{154, 0, 215}
\definecolor{orange}{RGB}{255, 96, 0}
\definecolor{turquoise}{RGB}{0, 153, 153}
\definecolor{rouge}{RGB}{255, 0, 0}
\definecolor{lightvert}{RGB}{205, 234, 190}
\setitemize[0]{label=\color{lightvert}  $\bullet$}

\pagestyle{fancy}
\renewcommand{\headrulewidth}{0.2pt}
\fancyhead[L]{maths-cours.fr}
\fancyhead[R]{\thepage}
\renewcommand{\footrulewidth}{0.2pt}
\fancyfoot[C]{}

\newcolumntype{C}{>{\centering\arraybackslash}X}
\newcolumntype{s}{>{\hsize=.35\hsize\arraybackslash}X}

\setlength{\parindent}{0pt}		 
\setlength{\parskip}{3mm}
\setlength{\headheight}{1cm}

\def\ebook{ebook}
\def\book{book}
\def\web{web}
\def\type{web}

\newcommand{\vect}[1]{\overrightarrow{\,\mathstrut#1\,}}

\def\Oij{$\left(\text{O}~;~\vect{\imath},~\vect{\jmath}\right)$}
\def\Oijk{$\left(\text{O}~;~\vect{\imath},~\vect{\jmath},~\vect{k}\right)$}
\def\Ouv{$\left(\text{O}~;~\vect{u},~\vect{v}\right)$}

\hypersetup{breaklinks=true, colorlinks = true, linkcolor = OliveGreen, urlcolor = OliveGreen, citecolor = OliveGreen, pdfauthor={Didier BONNEL - https://www.maths-cours.fr} } % supprime les bordures autour des liens

\renewcommand{\arg}[0]{\text{arg}}

\everymath{\displaystyle}

%================================================================================================================================
%
% Macros - Commandes
%
%================================================================================================================================

\newcommand\meta[2]{    			% Utilisé pour créer le post HTML.
	\def\titre{titre}
	\def\url{url}
	\def\arg{#1}
	\ifx\titre\arg
		\newcommand\maintitle{#2}
		\fancyhead[L]{#2}
		{\Large\sffamily \MakeUppercase{#2}}
		\vspace{1mm}\textcolor{mcvert}{\hrule}
	\fi 
	\ifx\url\arg
		\fancyfoot[L]{\href{https://www.maths-cours.fr#2}{\black \footnotesize{https://www.maths-cours.fr#2}}}
	\fi 
}


\newcommand\TitreC[1]{    		% Titre centré
     \needspace{3\baselineskip}
     \begin{center}\textbf{#1}\end{center}
}

\newcommand\newpar{    		% paragraphe
     \par
}

\newcommand\nosp {    		% commande vide (pas d'espace)
}
\newcommand{\id}[1]{} %ignore

\newcommand\boite[2]{				% Boite simple sans titre
	\vspace{5mm}
	\setlength{\fboxrule}{0.2mm}
	\setlength{\fboxsep}{5mm}	
	\fcolorbox{#1}{#1!3}{\makebox[\linewidth-2\fboxrule-2\fboxsep]{
  		\begin{minipage}[t]{\linewidth-2\fboxrule-4\fboxsep}\setlength{\parskip}{3mm}
  			 #2
  		\end{minipage}
	}}
	\vspace{5mm}
}

\newcommand\CBox[4]{				% Boites
	\vspace{5mm}
	\setlength{\fboxrule}{0.2mm}
	\setlength{\fboxsep}{5mm}
	
	\fcolorbox{#1}{#1!3}{\makebox[\linewidth-2\fboxrule-2\fboxsep]{
		\begin{minipage}[t]{1cm}\setlength{\parskip}{3mm}
	  		\textcolor{#1}{\LARGE{#2}}    
 	 	\end{minipage}  
  		\begin{minipage}[t]{\linewidth-2\fboxrule-4\fboxsep}\setlength{\parskip}{3mm}
			\raisebox{1.2mm}{\normalsize\sffamily{\textcolor{#1}{#3}}}						
  			 #4
  		\end{minipage}
	}}
	\vspace{5mm}
}

\newcommand\cadre[3]{				% Boites convertible html
	\par
	\vspace{2mm}
	\setlength{\fboxrule}{0.1mm}
	\setlength{\fboxsep}{5mm}
	\fcolorbox{#1}{white}{\makebox[\linewidth-2\fboxrule-2\fboxsep]{
  		\begin{minipage}[t]{\linewidth-2\fboxrule-4\fboxsep}\setlength{\parskip}{3mm}
			\raisebox{-2.5mm}{\sffamily \small{\textcolor{#1}{\MakeUppercase{#2}}}}		
			\par		
  			 #3
 	 		\end{minipage}
	}}
		\vspace{2mm}
	\par
}

\newcommand\bloc[3]{				% Boites convertible html sans bordure
     \needspace{2\baselineskip}
     {\sffamily \small{\textcolor{#1}{\MakeUppercase{#2}}}}    
		\par		
  			 #3
		\par
}

\newcommand\CHelp[1]{
     \CBox{Plum}{\faInfoCircle}{À RETENIR}{#1}
}

\newcommand\CUp[1]{
     \CBox{NavyBlue}{\faThumbsOUp}{EN PRATIQUE}{#1}
}

\newcommand\CInfo[1]{
     \CBox{Sepia}{\faArrowCircleRight}{REMARQUE}{#1}
}

\newcommand\CRedac[1]{
     \CBox{PineGreen}{\faEdit}{BIEN R\'EDIGER}{#1}
}

\newcommand\CError[1]{
     \CBox{Red}{\faExclamationTriangle}{ATTENTION}{#1}
}

\newcommand\TitreExo[2]{
\needspace{4\baselineskip}
 {\sffamily\large EXERCICE #1\ (\emph{#2 points})}
\vspace{5mm}
}

\newcommand\img[2]{
          \includegraphics[width=#2\paperwidth]{\imgdir#1}
}

\newcommand\imgsvg[2]{
       \begin{center}   \includegraphics[width=#2\paperwidth]{\imgsvgdir#1} \end{center}
}


\newcommand\Lien[2]{
     \href{#1}{#2 \tiny \faExternalLink}
}
\newcommand\mcLien[2]{
     \href{https~://www.maths-cours.fr/#1}{#2 \tiny \faExternalLink}
}

\newcommand{\euro}{\eurologo{}}

%================================================================================================================================
%
% Macros - Environement
%
%================================================================================================================================

\newenvironment{tex}{ %
}
{%
}

\newenvironment{indente}{ %
	\setlength\parindent{10mm}
}

{
	\setlength\parindent{0mm}
}

\newenvironment{corrige}{%
     \needspace{3\baselineskip}
     \medskip
     \textbf{\textsc{Corrigé}}
     \medskip
}
{
}

\newenvironment{extern}{%
     \begin{center}
     }
     {
     \end{center}
}

\NewEnviron{code}{%
	\par
     \boite{gray}{\texttt{%
     \BODY
     }}
     \par
}

\newenvironment{vbloc}{% boite sans cadre empeche saut de page
     \begin{minipage}[t]{\linewidth}
     }
     {
     \end{minipage}
}
\NewEnviron{h2}{%
    \needspace{3\baselineskip}
    \vspace{0.6cm}
	\noindent \MakeUppercase{\sffamily \large \BODY}
	\vspace{1mm}\textcolor{mcgris}{\hrule}\vspace{0.4cm}
	\par
}{}

\NewEnviron{h3}{%
    \needspace{3\baselineskip}
	\vspace{5mm}
	\textsc{\BODY}
	\par
}

\NewEnviron{margeneg}{ %
\begin{addmargin}[-1cm]{0cm}
\BODY
\end{addmargin}
}

\NewEnviron{html}{%
}

\begin{document}
\meta{url}{/cours/schema-bernouilli-loi-binomiale/}
\meta{pid}{393}
\meta{titre}{Loi binomiale en Première ES et L}
\meta{type}{cours}
\begin{h2}1. Loi de Bernoulli\end{h2}
\cadre{bleu}{Définition}{% id="d10"
     On appelle \textbf{épreuve de Bernoulli} de paramètre $p$ (avec $0 < p < 1$) une expérience aléatoire ayant deux issues :
     \begin{itemize}
          \item l'une appelée \textbf{succès} (généralement notée $S$) de probabilité $p$,
          \item l'autre appelée \textbf{échec} (généralement notée $\overline S$) de probabilité $1-p$.
     \end{itemize}
}
\cadre{bleu}{Définition}{% id="d20"
     On considère la variable aléatoire $X$ qui vaut $1$ en cas de succès et $0$ en cas d'échec.
     \par
     Cette variable aléatoire suit la \textbf{loi de Bernoulli de paramètre $p$}, définie par le tableau suivant:
     \begin{tabular}{|c|c|c|}%class="compact" width="250"
          \hline
          \textbf{$x_{i}$}  & $0$  & $1$
          \\ \hline
          \textbf{$p\left(X=x_{i}\right)$}    &  $1-p$  &  $p$
          \\ \hline
     \end{tabular}
}
\bloc{orange}{Exemple}{% id="e20"
     Au bonneteau, deux cartes noires et une carte rouge sont présentées, faces cachées, sur la table.
     \begin{center}
          \img{bonneteau}{0.25}%width="250" alt="Bonneteau"
     \end{center}
     Le maître du jeu manipule les cartes rapidement et un joueur doit retrouver la carte rouge.
     \par
     On suppose que le joueur choisit une carte complètement au hasard.
     \par
     On a affaire à une loi de Bernoulli de paramètre $p=\frac{1}{3}$.
     \par
     La probabilité de succès est : $p\left(S\right)=p=\frac{1}{3}$ et la probabilité d'échec $p\left(\overline S\right)=1-p=\frac{2}{3}$
}
\cadre{vert}{Propriété}{% id="p30"
     L'espérance mathématique d'une variable aléatoire $X$ qui suit une loi de Bernoulli de paramètre $p$ est :
     \begin{center}$E\left(X\right)=p$\end{center}
}
\bloc{cyan}{Démonstration}{% id="m30"
     D'après la définition de l'espérance mathématique :
     \par
     $E\left(X\right)=0\times \left(1-p\right)+1\times p=p$
}
\begin{h2}2. Schéma de Bernoulli - Loi binomiale\end{h2}
\cadre{bleu}{Définition}{% id="d40"
     On appelle \textbf{schéma de Bernoulli} la répétition d'épreuves de Bernoulli \textbf{identiques} et \textbf{indépendantes}.
}
\bloc{orange}{Exemple}{% id="e40"
     Une urne contient 2 boules rouges et 3 boules blanches. On tire 3 boules au hasard.
     \begin{itemize}
          \item Si l'épreuve s'effectue sans remise, les tirages ne sont ni identiques, ni indépendants. En effet, le fait d'avoir retiré une boule lors du premier tirage fait que le second tirage n'est pas identique au premier.
          \item Si l'épreuve s'effectue avec remise, on pourra, par contre, considérer que les tirages sont identiques et indépendants. On a donc bien, dans ce cas, un schéma de Bernoulli
     \end{itemize}
}
\cadre{bleu}{Définition}{% id="d50"
     Soit $X$ la variable aléatoire qui \textbf{compte le nombre de succès} dans un schéma de Bernoulli constitué de $n$ épreuves ayant chacune une probabilité de succès égale à $p$.
     \par
     La variable aléatoire X suit une loi appelée \textbf{loi binomiale de paramètres $n$ et $p$}, souvent notée $\mathscr B \left(n ; p\right)$ .
}
\bloc{orange}{Exemple}{% id="e50"
     On reprend l'exemple précédent : tirage au hasard et avec remise de 3 boules parmi 5 boules comportant 2 boules rouges et 3 boules blanches. On considère la variable aléatoire $X$ qui compte le nombre de boules rouges obtenues. La variable $X$ sur une loi binomiale de paramètres 3 (nombre d'épreuves) et $\frac{2}{5}$ (probabilité d'obtenir une boule rouge lors d'une épreuve).
     \par
     Ce schéma peut être représenté par l'arbre suivant :
     \begin{center}
          \begin{extern}%width="420" alt="Arbre schéma de Bernoulli"
               %:-+-+-+- Engendré par : http://math.et.info.free.fr/TikZ/Arbre/
               % Racine à Gauche, développement vers la droite
               \resizebox{8cm}{!}{
                    \begin{tikzpicture}[xscale=1,yscale=1]
                         % Styles (MODIFIABLES)
                         \tikzstyle{fleche}=[>=latex,thick]
                         \tikzstyle{noeud}=[circle,draw]
                         \tikzstyle{feuille}=[circle,draw]
                         \tikzstyle{etiquette}=[midway,fill=white]
                         % Dimensions (MODIFIABLES)
                         \def\DistanceInterNiveaux{3}
                         \def\DistanceInterFeuilles{2}
                         % Dimensions calculées (NON MODIFIABLES)
                         \def\NiveauA{(0)*\DistanceInterNiveaux}
                         \def\NiveauB{(1.6666666666666665)*\DistanceInterNiveaux}
                         \def\NiveauC{(3)*\DistanceInterNiveaux}
                         \def\NiveauD{(4)*\DistanceInterNiveaux}
                         \def\InterFeuilles{(-1)*\DistanceInterFeuilles}
                         % Noeuds (MODIFIABLES : Styles et Coefficients d'InterFeuilles)
                         \node[noeud] (R) at ({\NiveauA},{(3.5)*\InterFeuilles}) {$ $};
                         \node[noeud] (Ra) at ({\NiveauB},{(1.5)*\InterFeuilles}) {$S$};
                         \node[noeud] (Raa) at ({\NiveauC},{(0.5)*\InterFeuilles}) {$S$};
                         \node[feuille] (Raaa) at ({\NiveauD},{(0)*\InterFeuilles}) {$S$};
                         \node[feuille] (Raab) at ({\NiveauD},{(1)*\InterFeuilles}) {$\overline{S}$};
                         \node[noeud] (Rab) at ({\NiveauC},{(2.5)*\InterFeuilles}) {$\overline{S}$};
                         \node[feuille] (Raba) at ({\NiveauD},{(2)*\InterFeuilles}) {$S$};
                         \node[feuille] (Rabb) at ({\NiveauD},{(3)*\InterFeuilles}) {$\overline{S}$};
                         \node[noeud] (Rb) at ({\NiveauB},{(5.5)*\InterFeuilles}) {$\overline{S}$};
                         \node[noeud] (Rba) at ({\NiveauC},{(4.5)*\InterFeuilles}) {$S$};
                         \node[feuille] (Rbaa) at ({\NiveauD},{(4)*\InterFeuilles}) {$S$};
                         \node[feuille] (Rbab) at ({\NiveauD},{(5)*\InterFeuilles}) {$\overline{S}$};
                         \node[noeud] (Rbb) at ({\NiveauC},{(6.5)*\InterFeuilles}) {$\overline{S}$};
                         \node[feuille] (Rbba) at ({\NiveauD},{(6)*\InterFeuilles}) {$S$};
                         \node[feuille] (Rbbb) at ({\NiveauD},{(7)*\InterFeuilles}) {$\overline{S}$};
                         % Arcs (MODIFIABLES : Styles)
                         \draw[fleche] (R)--(Ra) node[etiquette] {$\dfrac{2}{5}$};
                         \draw[fleche] (Ra)--(Raa) node[etiquette] {$\dfrac{2}{5}$};
                         \draw[fleche] (Raa)--(Raaa) node[etiquette] {$\dfrac{2}{5}$};
                         \draw[fleche] (Raa)--(Raab) node[etiquette] {$\dfrac{3}{5}$};
                         \draw[fleche] (Ra)--(Rab) node[etiquette] {$\dfrac{3}{5}$};
                         \draw[fleche] (Rab)--(Raba) node[etiquette] {$\dfrac{2}{5}$};
                         \draw[fleche] (Rab)--(Rabb) node[etiquette] {$\dfrac{3}{5}$};
                         \draw[fleche] (R)--(Rb) node[etiquette] {$\dfrac{3}{5}$};
                         \draw[fleche] (Rb)--(Rba) node[etiquette] {$\dfrac{2}{5}$};
                         \draw[fleche] (Rba)--(Rbaa) node[etiquette] {$\dfrac{2}{5}$};
                         \draw[fleche] (Rba)--(Rbab) node[etiquette] {$\dfrac{3}{5}$};
                         \draw[fleche] (Rb)--(Rbb) node[etiquette] {$\dfrac{3}{5}$};
                         \draw[fleche] (Rbb)--(Rbba) node[etiquette] {$\dfrac{2}{5}$};
                         \draw[fleche] (Rbb)--(Rbbb) node[etiquette] {$\dfrac{3}{5}$};
                    \end{tikzpicture}
                    %:-+-+-+-+- Fin
               }
          \end{extern}
     \end{center}
     Grâce à l'arbre on voit que :
     \begin{itemize}
          \item la probabilité d'avoir 3 succès (c'est à dire 3 boules rouges) est  $p\left(X=3\right) =\left(\frac{2}{5}\right)^{3}=\frac{8}{125}$
          \item il y a 3 chemins qui correspondent à 2 succès ($SS\overline S, S\overline SS, \overline SSS$). La probabilité d'obtenir 2 boules rouges est donc :
          \par
          $p\left(X=2\right) =\left(\frac{2}{5}\right)^{2}\times \frac{3}{5}+\left(\frac{2}{5}\right)^{2}\times \frac{3}{5}$\nosp$+\left(\frac{2}{5}\right)^{2}\times \frac{3}{5}$\nosp$=3\times \left[\left(\frac{2}{5}\right)^{2}\times \frac{3}{5}\right]=\frac{36}{125}$
          \item il y a également 3 chemins qui correspondent à un unique succès ($S\overline S\overline S, \overline SS\overline S, \overline S\overline SS$). La probabilité d'obtenir une unique boule rouge est donc :
          \par
          $p\left(X=1\right) = \frac{2}{5}\times \left(\frac{3}{5}\right)^{2}+ \frac{2}{5}\times \left(\frac{3}{5}\right)^{2}$\nosp$+ \frac{2}{5}\times \left(\frac{3}{5}\right)^{2}$\nosp$=3\times \left[ \frac{2}{5}\times \left(\frac{3}{5}\right)^{2}\right]=\frac{54}{125}$
          \item la probabilité de n'avoir aucune boule rouge est  $p\left(X=0\right) =\left(\frac{3}{5}\right)^{3}=\frac{27}{125}$
     \end{itemize}
     La loi de $X$ est donc donnée par le tableau suivant :
     \begin{center}
          \begin{tabular}{|c|c|c|c|c|}%class="compact" width="600"
               \hline
               \textbf{$x_{i}$}  & $0$  & $1$ & $2$ & $3$
               \\ \hline
               \textbf{$p\left(X=x_{i}\right)$}    &  $\frac{27}{125}$  &   $\frac{54}{125}$ &   $\frac{36}{125}$ &   $\frac{8}{125}$
               \\ \hline
          \end{tabular}
     \end{center}
     On vérifie bien que $\frac{27}{125}+\frac{54}{125}+\frac{36}{125}+\frac{8}{125}=1$
}
\cadre{vert}{Propriété}{% id="p60"
     L'espérance mathématique d'une variable aléatoire $X$ qui suit une loi binomiale $\mathscr B \left(n ; p\right)$ est :
     \par
     $E\left(X\right)=np$
}
\bloc{orange}{Exemple}{% id="e60"
     Dans l'exemple précédent, la variable X suit une loi binomiale $\mathscr{B} \left(3 ; \dfrac{2}{5}\right)$.
     \par
     Son espérance mathématique est donc $E\left(X\right)=3\times \frac{2}{5}=\frac{6}{5}=1,2$
     \par
     On vérifie que l'on obtient bien le même résultat en utilisant le tableau de la loi de $X$ et la définition de l'espérance mathématique :
     \par
     $E\left(X\right)=0\times \frac{27}{125}+1\times \frac{54}{125}$\nosp$+2\times \frac{36}{125}+3\times \frac{8}{125}$\nosp$=\frac{150}{125}=1,2$
}
\begin{h2}3. Coefficients binomiaux\end{h2}
\cadre{bleu}{Définition}{% id="d70"
     On considère un arbre pondéré représentant une loi binomiale $\mathscr B \left(n ; p\right)$.
     \par
     Le \textbf{coefficient binomial} $\begin{pmatrix} n \\ k \end{pmatrix}$ (lire \textit{$k$ parmi $n$}) est le nombre de chemins qui correspondent à $k$ succès.
}
\bloc{orange}{Exemple}{% id="e70"
     On reprend le même exemple que précédemment. On a vu, par exemple,  qu'il y avait 3 chemins correspondant à 2 succès. On a donc $\begin{pmatrix} 3 \\ 2 \end{pmatrix}= 3$.
}
\bloc{cyan}{Remarques}{% id="r70"
     \begin{itemize}
          \item On peut aussi employer le mot \textbf{combinaisons} pour désigner un coefficient binomial;
          \item Pour calculer un coefficient binomial sur la plupart des calculatrices on utilise la commande \textbf{nCr}. Dans un tableur, on utilise la formule \textbf{=COMBIN(n;k)}.
     \end{itemize}
}
\cadre{rouge}{Théorème}{% id="t80"
     Soit $X$ une variable aléatoire de loi $\mathscr B \left(n ; p\right)$.
     \par
     Pour tout entier $k$ compris entre $0$ et $n$ :
     \begin{center}
          $p\left(X=k\right)=\begin{pmatrix} n \\ k \end{pmatrix}$\nosp$p^{k} \left(1-p\right)^{n-k}$
     \end{center}
}
\bloc{orange}{Exemple}{% id="e80"
     On lance 8 fois une pièce équilibrée et on appelle $X$ la variable aléatoire qui compte le nombre de fois où l'on obtient \textit{Pile}.
     \par
     $X$ suit une loi binomiale de paramètres $n=8$ et $p=\frac{1}{2}$.
     \par
     La probabilité d'obtenir \textbf{4 fois} \textit{Pile} (par exemple) est :
     \par
     $p\left(X=4\right) = \begin{pmatrix} 8 \\ 4 \end{pmatrix}\times \left(\frac{1}{2}\right)^{4}\times \left(\frac{1}{2}\right)^{4}$
     \par
     $\begin{pmatrix} 8 \\ 4 \end{pmatrix}= 70$ ~ (à la calculatrice)
     \par
     Donc :
     \par
     $p\left(X=4\right)=70\times \frac{1}{16}\times \frac{1}{16}=\frac{70}{256}=\frac{35}{128}$.
}

\end{document}
µ
\documentclass[a4paper]{article}

%================================================================================================================================
%
% Packages
%
%================================================================================================================================

\usepackage[T1]{fontenc} 	% pour caractères accentués
\usepackage[utf8]{inputenc}  % encodage utf8
\usepackage[french]{babel}	% langue : français
\usepackage{fourier}			% caractères plus lisibles
\usepackage[dvipsnames]{xcolor} % couleurs
\usepackage{fancyhdr}		% réglage header footer
\usepackage{needspace}		% empêcher sauts de page mal placés
\usepackage{graphicx}		% pour inclure des graphiques
\usepackage{enumitem,cprotect}		% personnalise les listes d'items (nécessaire pour ol, al ...)
\usepackage{hyperref}		% Liens hypertexte
\usepackage{pstricks,pst-all,pst-node,pstricks-add,pst-math,pst-plot,pst-tree,pst-eucl} % pstricks
\usepackage[a4paper,includeheadfoot,top=2cm,left=3cm, bottom=2cm,right=3cm]{geometry} % marges etc.
\usepackage{comment}			% commentaires multilignes
\usepackage{amsmath,environ} % maths (matrices, etc.)
\usepackage{amssymb,makeidx}
\usepackage{bm}				% bold maths
\usepackage{tabularx}		% tableaux
\usepackage{colortbl}		% tableaux en couleur
\usepackage{fontawesome}		% Fontawesome
\usepackage{environ}			% environment with command
\usepackage{fp}				% calculs pour ps-tricks
\usepackage{multido}			% pour ps tricks
\usepackage[np]{numprint}	% formattage nombre
\usepackage{tikz,tkz-tab} 			% package principal TikZ
\usepackage{pgfplots}   % axes
\usepackage{mathrsfs}    % cursives
\usepackage{calc}			% calcul taille boites
\usepackage[scaled=0.875]{helvet} % font sans serif
\usepackage{svg} % svg
\usepackage{scrextend} % local margin
\usepackage{scratch} %scratch
\usepackage{multicol} % colonnes
%\usepackage{infix-RPN,pst-func} % formule en notation polanaise inversée
\usepackage{listings}

%================================================================================================================================
%
% Réglages de base
%
%================================================================================================================================

\lstset{
language=Python,   % R code
literate=
{á}{{\'a}}1
{à}{{\`a}}1
{ã}{{\~a}}1
{é}{{\'e}}1
{è}{{\`e}}1
{ê}{{\^e}}1
{í}{{\'i}}1
{ó}{{\'o}}1
{õ}{{\~o}}1
{ú}{{\'u}}1
{ü}{{\"u}}1
{ç}{{\c{c}}}1
{~}{{ }}1
}


\definecolor{codegreen}{rgb}{0,0.6,0}
\definecolor{codegray}{rgb}{0.5,0.5,0.5}
\definecolor{codepurple}{rgb}{0.58,0,0.82}
\definecolor{backcolour}{rgb}{0.95,0.95,0.92}

\lstdefinestyle{mystyle}{
    backgroundcolor=\color{backcolour},   
    commentstyle=\color{codegreen},
    keywordstyle=\color{magenta},
    numberstyle=\tiny\color{codegray},
    stringstyle=\color{codepurple},
    basicstyle=\ttfamily\footnotesize,
    breakatwhitespace=false,         
    breaklines=true,                 
    captionpos=b,                    
    keepspaces=true,                 
    numbers=left,                    
xleftmargin=2em,
framexleftmargin=2em,            
    showspaces=false,                
    showstringspaces=false,
    showtabs=false,                  
    tabsize=2,
    upquote=true
}

\lstset{style=mystyle}


\lstset{style=mystyle}
\newcommand{\imgdir}{C:/laragon/www/newmc/assets/imgsvg/}
\newcommand{\imgsvgdir}{C:/laragon/www/newmc/assets/imgsvg/}

\definecolor{mcgris}{RGB}{220, 220, 220}% ancien~; pour compatibilité
\definecolor{mcbleu}{RGB}{52, 152, 219}
\definecolor{mcvert}{RGB}{125, 194, 70}
\definecolor{mcmauve}{RGB}{154, 0, 215}
\definecolor{mcorange}{RGB}{255, 96, 0}
\definecolor{mcturquoise}{RGB}{0, 153, 153}
\definecolor{mcrouge}{RGB}{255, 0, 0}
\definecolor{mclightvert}{RGB}{205, 234, 190}

\definecolor{gris}{RGB}{220, 220, 220}
\definecolor{bleu}{RGB}{52, 152, 219}
\definecolor{vert}{RGB}{125, 194, 70}
\definecolor{mauve}{RGB}{154, 0, 215}
\definecolor{orange}{RGB}{255, 96, 0}
\definecolor{turquoise}{RGB}{0, 153, 153}
\definecolor{rouge}{RGB}{255, 0, 0}
\definecolor{lightvert}{RGB}{205, 234, 190}
\setitemize[0]{label=\color{lightvert}  $\bullet$}

\pagestyle{fancy}
\renewcommand{\headrulewidth}{0.2pt}
\fancyhead[L]{maths-cours.fr}
\fancyhead[R]{\thepage}
\renewcommand{\footrulewidth}{0.2pt}
\fancyfoot[C]{}

\newcolumntype{C}{>{\centering\arraybackslash}X}
\newcolumntype{s}{>{\hsize=.35\hsize\arraybackslash}X}

\setlength{\parindent}{0pt}		 
\setlength{\parskip}{3mm}
\setlength{\headheight}{1cm}

\def\ebook{ebook}
\def\book{book}
\def\web{web}
\def\type{web}

\newcommand{\vect}[1]{\overrightarrow{\,\mathstrut#1\,}}

\def\Oij{$\left(\text{O}~;~\vect{\imath},~\vect{\jmath}\right)$}
\def\Oijk{$\left(\text{O}~;~\vect{\imath},~\vect{\jmath},~\vect{k}\right)$}
\def\Ouv{$\left(\text{O}~;~\vect{u},~\vect{v}\right)$}

\hypersetup{breaklinks=true, colorlinks = true, linkcolor = OliveGreen, urlcolor = OliveGreen, citecolor = OliveGreen, pdfauthor={Didier BONNEL - https://www.maths-cours.fr} } % supprime les bordures autour des liens

\renewcommand{\arg}[0]{\text{arg}}

\everymath{\displaystyle}

%================================================================================================================================
%
% Macros - Commandes
%
%================================================================================================================================

\newcommand\meta[2]{    			% Utilisé pour créer le post HTML.
	\def\titre{titre}
	\def\url{url}
	\def\arg{#1}
	\ifx\titre\arg
		\newcommand\maintitle{#2}
		\fancyhead[L]{#2}
		{\Large\sffamily \MakeUppercase{#2}}
		\vspace{1mm}\textcolor{mcvert}{\hrule}
	\fi 
	\ifx\url\arg
		\fancyfoot[L]{\href{https://www.maths-cours.fr#2}{\black \footnotesize{https://www.maths-cours.fr#2}}}
	\fi 
}


\newcommand\TitreC[1]{    		% Titre centré
     \needspace{3\baselineskip}
     \begin{center}\textbf{#1}\end{center}
}

\newcommand\newpar{    		% paragraphe
     \par
}

\newcommand\nosp {    		% commande vide (pas d'espace)
}
\newcommand{\id}[1]{} %ignore

\newcommand\boite[2]{				% Boite simple sans titre
	\vspace{5mm}
	\setlength{\fboxrule}{0.2mm}
	\setlength{\fboxsep}{5mm}	
	\fcolorbox{#1}{#1!3}{\makebox[\linewidth-2\fboxrule-2\fboxsep]{
  		\begin{minipage}[t]{\linewidth-2\fboxrule-4\fboxsep}\setlength{\parskip}{3mm}
  			 #2
  		\end{minipage}
	}}
	\vspace{5mm}
}

\newcommand\CBox[4]{				% Boites
	\vspace{5mm}
	\setlength{\fboxrule}{0.2mm}
	\setlength{\fboxsep}{5mm}
	
	\fcolorbox{#1}{#1!3}{\makebox[\linewidth-2\fboxrule-2\fboxsep]{
		\begin{minipage}[t]{1cm}\setlength{\parskip}{3mm}
	  		\textcolor{#1}{\LARGE{#2}}    
 	 	\end{minipage}  
  		\begin{minipage}[t]{\linewidth-2\fboxrule-4\fboxsep}\setlength{\parskip}{3mm}
			\raisebox{1.2mm}{\normalsize\sffamily{\textcolor{#1}{#3}}}						
  			 #4
  		\end{minipage}
	}}
	\vspace{5mm}
}

\newcommand\cadre[3]{				% Boites convertible html
	\par
	\vspace{2mm}
	\setlength{\fboxrule}{0.1mm}
	\setlength{\fboxsep}{5mm}
	\fcolorbox{#1}{white}{\makebox[\linewidth-2\fboxrule-2\fboxsep]{
  		\begin{minipage}[t]{\linewidth-2\fboxrule-4\fboxsep}\setlength{\parskip}{3mm}
			\raisebox{-2.5mm}{\sffamily \small{\textcolor{#1}{\MakeUppercase{#2}}}}		
			\par		
  			 #3
 	 		\end{minipage}
	}}
		\vspace{2mm}
	\par
}

\newcommand\bloc[3]{				% Boites convertible html sans bordure
     \needspace{2\baselineskip}
     {\sffamily \small{\textcolor{#1}{\MakeUppercase{#2}}}}    
		\par		
  			 #3
		\par
}

\newcommand\CHelp[1]{
     \CBox{Plum}{\faInfoCircle}{À RETENIR}{#1}
}

\newcommand\CUp[1]{
     \CBox{NavyBlue}{\faThumbsOUp}{EN PRATIQUE}{#1}
}

\newcommand\CInfo[1]{
     \CBox{Sepia}{\faArrowCircleRight}{REMARQUE}{#1}
}

\newcommand\CRedac[1]{
     \CBox{PineGreen}{\faEdit}{BIEN R\'EDIGER}{#1}
}

\newcommand\CError[1]{
     \CBox{Red}{\faExclamationTriangle}{ATTENTION}{#1}
}

\newcommand\TitreExo[2]{
\needspace{4\baselineskip}
 {\sffamily\large EXERCICE #1\ (\emph{#2 points})}
\vspace{5mm}
}

\newcommand\img[2]{
          \includegraphics[width=#2\paperwidth]{\imgdir#1}
}

\newcommand\imgsvg[2]{
       \begin{center}   \includegraphics[width=#2\paperwidth]{\imgsvgdir#1} \end{center}
}


\newcommand\Lien[2]{
     \href{#1}{#2 \tiny \faExternalLink}
}
\newcommand\mcLien[2]{
     \href{https~://www.maths-cours.fr/#1}{#2 \tiny \faExternalLink}
}

\newcommand{\euro}{\eurologo{}}

%================================================================================================================================
%
% Macros - Environement
%
%================================================================================================================================

\newenvironment{tex}{ %
}
{%
}

\newenvironment{indente}{ %
	\setlength\parindent{10mm}
}

{
	\setlength\parindent{0mm}
}

\newenvironment{corrige}{%
     \needspace{3\baselineskip}
     \medskip
     \textbf{\textsc{Corrigé}}
     \medskip
}
{
}

\newenvironment{extern}{%
     \begin{center}
     }
     {
     \end{center}
}

\NewEnviron{code}{%
	\par
     \boite{gray}{\texttt{%
     \BODY
     }}
     \par
}

\newenvironment{vbloc}{% boite sans cadre empeche saut de page
     \begin{minipage}[t]{\linewidth}
     }
     {
     \end{minipage}
}
\NewEnviron{h2}{%
    \needspace{3\baselineskip}
    \vspace{0.6cm}
	\noindent \MakeUppercase{\sffamily \large \BODY}
	\vspace{1mm}\textcolor{mcgris}{\hrule}\vspace{0.4cm}
	\par
}{}

\NewEnviron{h3}{%
    \needspace{3\baselineskip}
	\vspace{5mm}
	\textsc{\BODY}
	\par
}

\NewEnviron{margeneg}{ %
\begin{addmargin}[-1cm]{0cm}
\BODY
\end{addmargin}
}

\NewEnviron{html}{%
}

\begin{document}
\meta{url}{/cours/echantillonnage/}
\meta{pid}{395}
\meta{titre}{Echantillonnage en Première ES et L}
\meta{type}{cours}
\cadre{bleu}{Définition}{% id="d10"
     Pour réaliser un \textbf{échantillon} de taille $n$ dans une population, on sélectionne "au hasard" $n$ individus avec ou sans remise.
}
\bloc{cyan}{Remarque}{% id="r10"
     \textit{"Avec remise"} signifie qu'un même individu peut être sélectionné plusieurs fois
}
\cadre{vert}{Propriété}{% id="p20"
     Dans un échantillon avec remise, la variable aléatoire $X$ qui compte le nombre d'individus présentant un certain caractère suit une loi binomiale $\mathscr B\left(n;p\right)$ où $n$ est la taille de l'échantillon et $p$ la proportion du caractère étudié dans la population.
}
\bloc{cyan}{Remarque}{% id="r20"
     Si l'échantillon est réalisé \textit{sans remise} on peut encore considérer que la propriété précédente est vraie si la taille de l'échantillon est nettement inférieure à l'effectif total.
}
\bloc{orange}{Exemple}{% id="e20"
     La proportion des Français ayant des yeux bleus est estimée à 25\%. On choisit 100 personnes au hasard dans la population française. La variable aléatoire comptant le nombre de personnes aux yeux bleus dans un tel échantillon suit une loi binomiale $\mathscr B\left(100 ; 0,25\right)$
}
\cadre{bleu}{Définition} d'une fréquence correspondant à la réalisation, sur un échantillon de taille $n$, d'une variable aléatoire $X$ de loi binomiale $\mathscr B\left(n;p\right)$, est l'intervalle   $\left[\frac{a}{n} ; \frac{b}{n}\right]$, défini par :
     \begin{itemize}
          \item $a$ est le plus petit entier tel que $p\left(X \leqslant  a\right) > 0,025$;
          \item $b$ est le plus petit entier tel que $p\left(X \leqslant  b\right) \geqslant  0,975$.
     \end{itemize}
}
\bloc{cyan}{Remarque}{% id="r30"
     Pour $n\geqslant 30, np \geqslant  5$ et $n\left(1-p\right)\geqslant 5$, on observe que l'intervalle trouvé est sensiblement le même que l'intervalle $\left[p-\frac{1}{\sqrt{n}} ; p+\frac{1}{\sqrt{n}}\right]$ vu en classe de Seconde.
}
\bloc{orange}{Exemple}{% id="e30"
     On lance 100 fois une pièce équilibrée. La variable aléatoire correspondant au nombre de \textit{Pile} obtenus suit la loi binomiale $\mathscr B\left(100 ; 0.5\right)$. En consultant \mcLien{/solver/loi-binomiale-cumulee}{une table de loi binomiale} ou
     \mcLien{/methodes/statistiques/intervalle-fluctuation-tableur}{à l'aide d'un tableur} ou d'une calculatrice on trouve $a=40$ et $b=60$ donc l'intervalle
     \par
     $I=\left[\frac{40}{100} ; \frac{60}{100}\right]=\left[ 0,4 ; 0,6 \right]$.
     \par
     La formule $I=\left[p-\frac{1}{\sqrt{n}} ; p+\frac{1}{\sqrt{n}}\right]$\nosp$=\left[0,5-\frac{1}{\sqrt{100}} ; 0,5+\frac{1}{\sqrt{100}}\right]$ donne le même résultat.
}
\bloc{cyan}{Validation d'une hypothèse}{% id="r40"
     On fait l'hypothèse qu'un caractère est présent dans une proportion $p$ de la population.
     \par
     Pour valider cette hypothèse, on prélève un échantillon de taille $n$ et on note $f $ la fréquence du caractère étudié dans cet échantillon.
     \textbf{La règle de décision} est alors la suivante : si la fréquence observée $f$ appartient à l'intervalle de fluctuation à 95 \%, on \textbf{accepte} l'hypothèse selon laquelle la proportion est $p$ dans la population avec un risque d'erreur inférieur à 5 \%; sinon, on \textbf{rejette} cette hypothèse (avec un risque d'erreur inférieur à 5 \%).
}
\bloc{cyan}{Intervalle de fluctuation pour un seuil différent de 95 \%}{% id="r50"
     On peut généraliser le résultat précédent pour un seuil différent de 95\% :
     \textbf{L'intervalle de fluctuation à t \%} est alors l'intervalle $\left[\frac{a}{n} ; \frac{b}{n}\right]$, défini par :
     \begin{itemize}
          \item $a$ est le plus petit entier tel que $p\left(X \leqslant  a\right) > \frac{1}{2}\left(1-\frac{t}{100}\right)$;
          \item $b$ est le plus petit entier tel que $p\left(X \leqslant  b\right) \geqslant  \frac{1}{2}\left(1+\frac{t}{100}\right)$.
     \end{itemize}
}
\bloc{orange}{Exemple}{% id="e50"
     Une variable aléatoire $X$ suit une loi binomiale de paramètres $n=500$ et $p=0,30$.
     \par
     Pour trouver l'intervalle de fluctuation au seuil de 99\% on recherche les valeurs de $a$ et $b$ telles que :
     \begin{itemize}
          \item $a$ est le plus petit entier vérifiant $p\left(X \leqslant  a\right) > 0,005$;
          \item $b$ est le plus petit entier vérifiant $p\left(X \leqslant  b\right) \geqslant  0,995$.
     \end{itemize}
     \mcLien{/solver/loi-binomiale-cumulee?val_n=500&val_p=0.3&val_s=99}{La table de loi binomiale $\mathscr B\left(500 ; 0,30\right)$} (ou une calculatrice) donne au seuil de 99\% : $a=124$ et $b =177$ donc l'intervalle de fluctuation à 99 \% est $I=\left[\frac{124}{500} ; \frac{177}{500}\right]$
}

\end{document}
µ
\documentclass[a4paper]{article}

%================================================================================================================================
%
% Packages
%
%================================================================================================================================

\usepackage[T1]{fontenc} 	% pour caractères accentués
\usepackage[utf8]{inputenc}  % encodage utf8
\usepackage[french]{babel}	% langue : français
\usepackage{fourier}			% caractères plus lisibles
\usepackage[dvipsnames]{xcolor} % couleurs
\usepackage{fancyhdr}		% réglage header footer
\usepackage{needspace}		% empêcher sauts de page mal placés
\usepackage{graphicx}		% pour inclure des graphiques
\usepackage{enumitem,cprotect}		% personnalise les listes d'items (nécessaire pour ol, al ...)
\usepackage{hyperref}		% Liens hypertexte
\usepackage{pstricks,pst-all,pst-node,pstricks-add,pst-math,pst-plot,pst-tree,pst-eucl} % pstricks
\usepackage[a4paper,includeheadfoot,top=2cm,left=3cm, bottom=2cm,right=3cm]{geometry} % marges etc.
\usepackage{comment}			% commentaires multilignes
\usepackage{amsmath,environ} % maths (matrices, etc.)
\usepackage{amssymb,makeidx}
\usepackage{bm}				% bold maths
\usepackage{tabularx}		% tableaux
\usepackage{colortbl}		% tableaux en couleur
\usepackage{fontawesome}		% Fontawesome
\usepackage{environ}			% environment with command
\usepackage{fp}				% calculs pour ps-tricks
\usepackage{multido}			% pour ps tricks
\usepackage[np]{numprint}	% formattage nombre
\usepackage{tikz,tkz-tab} 			% package principal TikZ
\usepackage{pgfplots}   % axes
\usepackage{mathrsfs}    % cursives
\usepackage{calc}			% calcul taille boites
\usepackage[scaled=0.875]{helvet} % font sans serif
\usepackage{svg} % svg
\usepackage{scrextend} % local margin
\usepackage{scratch} %scratch
\usepackage{multicol} % colonnes
%\usepackage{infix-RPN,pst-func} % formule en notation polanaise inversée
\usepackage{listings}

%================================================================================================================================
%
% Réglages de base
%
%================================================================================================================================

\lstset{
language=Python,   % R code
literate=
{á}{{\'a}}1
{à}{{\`a}}1
{ã}{{\~a}}1
{é}{{\'e}}1
{è}{{\`e}}1
{ê}{{\^e}}1
{í}{{\'i}}1
{ó}{{\'o}}1
{õ}{{\~o}}1
{ú}{{\'u}}1
{ü}{{\"u}}1
{ç}{{\c{c}}}1
{~}{{ }}1
}


\definecolor{codegreen}{rgb}{0,0.6,0}
\definecolor{codegray}{rgb}{0.5,0.5,0.5}
\definecolor{codepurple}{rgb}{0.58,0,0.82}
\definecolor{backcolour}{rgb}{0.95,0.95,0.92}

\lstdefinestyle{mystyle}{
    backgroundcolor=\color{backcolour},   
    commentstyle=\color{codegreen},
    keywordstyle=\color{magenta},
    numberstyle=\tiny\color{codegray},
    stringstyle=\color{codepurple},
    basicstyle=\ttfamily\footnotesize,
    breakatwhitespace=false,         
    breaklines=true,                 
    captionpos=b,                    
    keepspaces=true,                 
    numbers=left,                    
xleftmargin=2em,
framexleftmargin=2em,            
    showspaces=false,                
    showstringspaces=false,
    showtabs=false,                  
    tabsize=2,
    upquote=true
}

\lstset{style=mystyle}


\lstset{style=mystyle}
\newcommand{\imgdir}{C:/laragon/www/newmc/assets/imgsvg/}
\newcommand{\imgsvgdir}{C:/laragon/www/newmc/assets/imgsvg/}

\definecolor{mcgris}{RGB}{220, 220, 220}% ancien~; pour compatibilité
\definecolor{mcbleu}{RGB}{52, 152, 219}
\definecolor{mcvert}{RGB}{125, 194, 70}
\definecolor{mcmauve}{RGB}{154, 0, 215}
\definecolor{mcorange}{RGB}{255, 96, 0}
\definecolor{mcturquoise}{RGB}{0, 153, 153}
\definecolor{mcrouge}{RGB}{255, 0, 0}
\definecolor{mclightvert}{RGB}{205, 234, 190}

\definecolor{gris}{RGB}{220, 220, 220}
\definecolor{bleu}{RGB}{52, 152, 219}
\definecolor{vert}{RGB}{125, 194, 70}
\definecolor{mauve}{RGB}{154, 0, 215}
\definecolor{orange}{RGB}{255, 96, 0}
\definecolor{turquoise}{RGB}{0, 153, 153}
\definecolor{rouge}{RGB}{255, 0, 0}
\definecolor{lightvert}{RGB}{205, 234, 190}
\setitemize[0]{label=\color{lightvert}  $\bullet$}

\pagestyle{fancy}
\renewcommand{\headrulewidth}{0.2pt}
\fancyhead[L]{maths-cours.fr}
\fancyhead[R]{\thepage}
\renewcommand{\footrulewidth}{0.2pt}
\fancyfoot[C]{}

\newcolumntype{C}{>{\centering\arraybackslash}X}
\newcolumntype{s}{>{\hsize=.35\hsize\arraybackslash}X}

\setlength{\parindent}{0pt}		 
\setlength{\parskip}{3mm}
\setlength{\headheight}{1cm}

\def\ebook{ebook}
\def\book{book}
\def\web{web}
\def\type{web}

\newcommand{\vect}[1]{\overrightarrow{\,\mathstrut#1\,}}

\def\Oij{$\left(\text{O}~;~\vect{\imath},~\vect{\jmath}\right)$}
\def\Oijk{$\left(\text{O}~;~\vect{\imath},~\vect{\jmath},~\vect{k}\right)$}
\def\Ouv{$\left(\text{O}~;~\vect{u},~\vect{v}\right)$}

\hypersetup{breaklinks=true, colorlinks = true, linkcolor = OliveGreen, urlcolor = OliveGreen, citecolor = OliveGreen, pdfauthor={Didier BONNEL - https://www.maths-cours.fr} } % supprime les bordures autour des liens

\renewcommand{\arg}[0]{\text{arg}}

\everymath{\displaystyle}

%================================================================================================================================
%
% Macros - Commandes
%
%================================================================================================================================

\newcommand\meta[2]{    			% Utilisé pour créer le post HTML.
	\def\titre{titre}
	\def\url{url}
	\def\arg{#1}
	\ifx\titre\arg
		\newcommand\maintitle{#2}
		\fancyhead[L]{#2}
		{\Large\sffamily \MakeUppercase{#2}}
		\vspace{1mm}\textcolor{mcvert}{\hrule}
	\fi 
	\ifx\url\arg
		\fancyfoot[L]{\href{https://www.maths-cours.fr#2}{\black \footnotesize{https://www.maths-cours.fr#2}}}
	\fi 
}


\newcommand\TitreC[1]{    		% Titre centré
     \needspace{3\baselineskip}
     \begin{center}\textbf{#1}\end{center}
}

\newcommand\newpar{    		% paragraphe
     \par
}

\newcommand\nosp {    		% commande vide (pas d'espace)
}
\newcommand{\id}[1]{} %ignore

\newcommand\boite[2]{				% Boite simple sans titre
	\vspace{5mm}
	\setlength{\fboxrule}{0.2mm}
	\setlength{\fboxsep}{5mm}	
	\fcolorbox{#1}{#1!3}{\makebox[\linewidth-2\fboxrule-2\fboxsep]{
  		\begin{minipage}[t]{\linewidth-2\fboxrule-4\fboxsep}\setlength{\parskip}{3mm}
  			 #2
  		\end{minipage}
	}}
	\vspace{5mm}
}

\newcommand\CBox[4]{				% Boites
	\vspace{5mm}
	\setlength{\fboxrule}{0.2mm}
	\setlength{\fboxsep}{5mm}
	
	\fcolorbox{#1}{#1!3}{\makebox[\linewidth-2\fboxrule-2\fboxsep]{
		\begin{minipage}[t]{1cm}\setlength{\parskip}{3mm}
	  		\textcolor{#1}{\LARGE{#2}}    
 	 	\end{minipage}  
  		\begin{minipage}[t]{\linewidth-2\fboxrule-4\fboxsep}\setlength{\parskip}{3mm}
			\raisebox{1.2mm}{\normalsize\sffamily{\textcolor{#1}{#3}}}						
  			 #4
  		\end{minipage}
	}}
	\vspace{5mm}
}

\newcommand\cadre[3]{				% Boites convertible html
	\par
	\vspace{2mm}
	\setlength{\fboxrule}{0.1mm}
	\setlength{\fboxsep}{5mm}
	\fcolorbox{#1}{white}{\makebox[\linewidth-2\fboxrule-2\fboxsep]{
  		\begin{minipage}[t]{\linewidth-2\fboxrule-4\fboxsep}\setlength{\parskip}{3mm}
			\raisebox{-2.5mm}{\sffamily \small{\textcolor{#1}{\MakeUppercase{#2}}}}		
			\par		
  			 #3
 	 		\end{minipage}
	}}
		\vspace{2mm}
	\par
}

\newcommand\bloc[3]{				% Boites convertible html sans bordure
     \needspace{2\baselineskip}
     {\sffamily \small{\textcolor{#1}{\MakeUppercase{#2}}}}    
		\par		
  			 #3
		\par
}

\newcommand\CHelp[1]{
     \CBox{Plum}{\faInfoCircle}{À RETENIR}{#1}
}

\newcommand\CUp[1]{
     \CBox{NavyBlue}{\faThumbsOUp}{EN PRATIQUE}{#1}
}

\newcommand\CInfo[1]{
     \CBox{Sepia}{\faArrowCircleRight}{REMARQUE}{#1}
}

\newcommand\CRedac[1]{
     \CBox{PineGreen}{\faEdit}{BIEN R\'EDIGER}{#1}
}

\newcommand\CError[1]{
     \CBox{Red}{\faExclamationTriangle}{ATTENTION}{#1}
}

\newcommand\TitreExo[2]{
\needspace{4\baselineskip}
 {\sffamily\large EXERCICE #1\ (\emph{#2 points})}
\vspace{5mm}
}

\newcommand\img[2]{
          \includegraphics[width=#2\paperwidth]{\imgdir#1}
}

\newcommand\imgsvg[2]{
       \begin{center}   \includegraphics[width=#2\paperwidth]{\imgsvgdir#1} \end{center}
}


\newcommand\Lien[2]{
     \href{#1}{#2 \tiny \faExternalLink}
}
\newcommand\mcLien[2]{
     \href{https~://www.maths-cours.fr/#1}{#2 \tiny \faExternalLink}
}

\newcommand{\euro}{\eurologo{}}

%================================================================================================================================
%
% Macros - Environement
%
%================================================================================================================================

\newenvironment{tex}{ %
}
{%
}

\newenvironment{indente}{ %
	\setlength\parindent{10mm}
}

{
	\setlength\parindent{0mm}
}

\newenvironment{corrige}{%
     \needspace{3\baselineskip}
     \medskip
     \textbf{\textsc{Corrigé}}
     \medskip
}
{
}

\newenvironment{extern}{%
     \begin{center}
     }
     {
     \end{center}
}

\NewEnviron{code}{%
	\par
     \boite{gray}{\texttt{%
     \BODY
     }}
     \par
}

\newenvironment{vbloc}{% boite sans cadre empeche saut de page
     \begin{minipage}[t]{\linewidth}
     }
     {
     \end{minipage}
}
\NewEnviron{h2}{%
    \needspace{3\baselineskip}
    \vspace{0.6cm}
	\noindent \MakeUppercase{\sffamily \large \BODY}
	\vspace{1mm}\textcolor{mcgris}{\hrule}\vspace{0.4cm}
	\par
}{}

\NewEnviron{h3}{%
    \needspace{3\baselineskip}
	\vspace{5mm}
	\textsc{\BODY}
	\par
}

\NewEnviron{margeneg}{ %
\begin{addmargin}[-1cm]{0cm}
\BODY
\end{addmargin}
}

\NewEnviron{html}{%
}

\begin{document}
\meta{url}{/cours/probabilites-conditionnelles/}
\meta{pid}{400}
\meta{titre}{Probabilités conditionnelles}
\meta{type}{cours}
\begin{h2}I - Conditionnement\end{h2}
\cadre{bleu}{Définition}{% id="d10"
     $A$ et $B$ étant deux événements tels que $p\left(A\right)\neq 0$, la \textbf{probabilité de $B$ sachant $A$} est le nombre réel :
     \par
     $p_{A}\left(B\right)=\frac{p\left(A \cap  B\right)}{p\left(A\right)}$
}
\bloc{cyan}{Remarques}{% id="r10"
     \begin{itemize}
          \item On note parfois $p\left(B/A\right)$ au lieu de $p_{A}\left(B\right)$.
          \item \textbf{Rappel} : Le signe $\cap $ (intersection) correspond à \textbf{"et"}.
          \item De même si $p\left(B\right)\neq 0$, la \textbf{probabilité de $A$ sachant $B$} est $p_{B}\left(A\right)=\frac{p\left(A \cap  B\right)}{p\left(B\right)}$.
     \end{itemize}
}
\bloc{orange}{Exemple}{% id="e10"
     Une urne contient 3 boules blanches et 4 boules rouges indiscernables au toucher. On tire successivement 2 boules \textbf{sans remise}
     On note :
     \begin{itemize}
          \item $B_{1}$  l'événement \textit{"la première boule tirée est blanche"}
          \item $B_{2}$  l'événement \textit{"la seconde boule tirée est blanche"}
     \end{itemize}
     la probabilité $p_{B_{1}}\left(B_{2}\right)$ est la probabilité que la seconde boule soit blanche sachant que la première était blanche. Pour la calculer, on se place dans la situation où l'on se trouve après avoir obtenu une boule blanche au premier tirage. Il reste alors 6 boules dans l'urne; 2 sont blanches et 4 sont rouges.
     \par
     La probabilité de tirer une boule blanche au second tirage est donc :
     \par
     $p_{B_{1}}\left(B_{2}\right)=\frac{2}{6}=\frac{1}{3}$
     \par
     Cette probabilité se place sur l'arbre de la façon suivante :
     \begin{extern} %width="350" alt="arbre pondéré" class="aligncenter"
          % Racine à Gauche, développement vers la droite
          \begin{tikzpicture}[xscale=1,yscale=1]
               % Styles (MODIFIABLES)
               \tikzstyle{fleche}=[-,>=latex,thick]
               \tikzstyle{noeud}=[fill=white,circle,draw]
               \tikzstyle{feuille}=[fill=white,circle,draw]
               \tikzstyle{etiquette}=[midway,fill=white]
               % Dimensions (MODIFIABLES)
               \def\DistanceInterNiveaux{3}
               \def\DistanceInterFeuilles{2}
               % Dimensions calculées (NON MODIFIABLES)
               \def\NiveauA{(0)*\DistanceInterNiveaux}
               \def\NiveauB{(1.5)*\DistanceInterNiveaux}
               \def\NiveauC{(2.5)*\DistanceInterNiveaux}
               \def\InterFeuilles{(-1)*\DistanceInterFeuilles}
               % Noeuds (MODIFIABLES : Styles et Coefficients d'InterFeuilles)
               \node[noeud] (R) at ({\NiveauA},{(1.5)*\InterFeuilles}) {$\ $};
               \node[noeud] (Ra) at ({\NiveauB},{(0.5)*\InterFeuilles}) {$B_1$};
               \node[feuille] (Raa) at ({\NiveauC},{(0)*\InterFeuilles}) {$B_2$};
               \node[feuille] (Rab) at ({\NiveauC},{(1)*\InterFeuilles}) {$\overline{B_2}$};
               \node[noeud] (Rb) at ({\NiveauB},{(2.5)*\InterFeuilles}) {$\overline{B_1}$};
               \node[feuille] (Rba) at ({\NiveauC},{(2)*\InterFeuilles}) {$B_2$};
               \node[feuille] (Rbb) at ({\NiveauC},{(3)*\InterFeuilles}) {$\overline{B_2}$};
               % Arcs (MODIFIABLES : Styles)
               \draw[fleche] (R)--(Ra) node[etiquette] {$3/7$};
               \draw[fleche] (Ra)--(Raa) node[etiquette] {$\color{red} 1/3$};
               \draw[fleche] (Ra)--(Rab) node[etiquette] {$2/3$};
               \draw[fleche] (R)--(Rb) node[etiquette] {$4/7$};
               \draw[fleche] (Rb)--(Rba) node[etiquette] {$\cdots$};
               \draw[fleche] (Rb)--(Rbb) node[etiquette] {$\cdots$};
          \end{tikzpicture}
     \end{extern}
     On peut calculer de même $p_{\overline{B_{1}}}\left(B_{2}\right)$ est la probabilité que la seconde boule soit blanche sachant que la première était rouge. Il reste alors 3 boules blanches et 3 boules rouges après le premier tirage donc :
     \par
     $p_{\overline{B_{1}}}\left(B_{2}\right)=\frac{3}{6}=\frac{1}{2}$
     \par
     et on peut compléter l'arbre :
     \begin{extern} %width="350" alt="arbre pondéré" class="aligncenter"
          % Racine à Gauche, développement vers la droite
          \begin{tikzpicture}[xscale=1,yscale=1]
               % Styles (MODIFIABLES)
               \tikzstyle{fleche}=[-,>=latex,thick]
               \tikzstyle{noeud}=[fill=white,circle,draw]
               \tikzstyle{feuille}=[fill=white,circle,draw]
               \tikzstyle{etiquette}=[midway,fill=white]
               % Dimensions (MODIFIABLES)
               \def\DistanceInterNiveaux{3}
               \def\DistanceInterFeuilles{2}
               % Dimensions calculées (NON MODIFIABLES)
               \def\NiveauA{(0)*\DistanceInterNiveaux}
               \def\NiveauB{(1.5)*\DistanceInterNiveaux}
               \def\NiveauC{(2.5)*\DistanceInterNiveaux}
               \def\InterFeuilles{(-1)*\DistanceInterFeuilles}
               % Noeuds (MODIFIABLES : Styles et Coefficients d'InterFeuilles)
               \node[noeud] (R) at ({\NiveauA},{(1.5)*\InterFeuilles}) {$\ $};
               \node[noeud] (Ra) at ({\NiveauB},{(0.5)*\InterFeuilles}) {$B_1$};
               \node[feuille] (Raa) at ({\NiveauC},{(0)*\InterFeuilles}) {$B_2$};
               \node[feuille] (Rab) at ({\NiveauC},{(1)*\InterFeuilles}) {$\overline{B_2}$};
               \node[noeud] (Rb) at ({\NiveauB},{(2.5)*\InterFeuilles}) {$\overline{B_1}$};
               \node[feuille] (Rba) at ({\NiveauC},{(2)*\InterFeuilles}) {$B_2$};
               \node[feuille] (Rbb) at ({\NiveauC},{(3)*\InterFeuilles}) {$\overline{B_2}$};
               % Arcs (MODIFIABLES : Styles)
               \draw[fleche] (R)--(Ra) node[etiquette] {$3/7$};
               \draw[fleche] (Ra)--(Raa) node[etiquette] {$1/3$};
               \draw[fleche] (Ra)--(Rab) node[etiquette] {$2/3$};
               \draw[fleche] (R)--(Rb) node[etiquette] {$4/7$};
               \draw[fleche] (Rb)--(Rba) node[etiquette] {$1/2$};
               \draw[fleche] (Rb)--(Rbb) node[etiquette] {$1/2$};
          \end{tikzpicture}
     \end{extern}
}
\cadre{vert}{Propriété}{% id="p20"
     De la définition précédente, on déduit immédiatement que :
     \par
     $p\left(A \cap  B\right)=p\left(A\right)\times p_{A}\left(B\right)$
}
\bloc{cyan}{Remarque}{% id="r20"
     Attention à ne pas confondre :
     \begin{itemize}
          \item  $p\left(A \cap  B\right)$ qui est la probabilité que $A$ \textbf{et} $B$ soient réalisés alors qu'on ne possède \textbf{aucune indication} sur la réalisation de $A$ ou de $B$
          \item  $p_{A}\left(B\right)$ qui est la probabilité que $B$ soit réalisé alors qu'\textbf{on sait déjà} que $A$ est réalisé.
     \end{itemize}
}
\bloc{orange}{Exemple}{% id="e20"
     Si l'on reprend l'exemple précédent, la probabilité de tirer 2 boules blanches est $p\left(B_{1} \cap  B_{2}\right)$ \textit{(il faut que la première boule soit blanche \textbf{et} que la seconde boule soit blanche)}.
     \par
     D'après la formule précédente :
     \par
     $p\left(B_{1} \cap  B_{2}\right)=p\left(B_{1}\right)\times p_{B_{1}}\left(B_{2}\right)=\frac{3}{7}\times \frac{1}{3}=\frac{1}{7}$
}
\begin{h2}II - Formule des probabilités totales\end{h2}
\cadre{bleu}{Définition}{% id="d40"
     On dit que les événements $A_{1}, A_{2}, . . . , A_{n}$ forment une \textbf{partition} de l'univers $\Omega $ si chaque élément de $\Omega $ appartient à un et un seul des $A_{i}$
}
\bloc{orange}{Exemple}{% id="e40"
     On lance un dé à  6 faces. On peut modéliser cette expérience par l'univers $\Omega  = \left\{1; 2; 3; 4; 5; 6\right\}$.
     \par
     Les événements :
     \begin{itemize}
          \item $A_{1}=\left\{1; 2\right\}$ \textit{(le résultat est inférieur à 3)}
          \item $A_{2}=\left\{3\right\}$ \textit{(le résultat est égal à 3)}
          \item $A_{3}=\left\{4; 5; 6\right\}$ \textit{(le résultat est supérieur à 3)}
     \end{itemize}
     forment une partition de $\Omega $. En effet, chacune des six éventualités $1, 2, 3, 4, 5, 6$ appartient à et à un seul des $A_{i}$.
}
\bloc{cyan}{Remarque}{% id="r40"
     $A$ et $\overline{A}$ forment une partition de l'univers, quel que soit l'événement $A$. En effet, toute éventualité appartient soit à un événement, soit à son contraire et ne peut appartenir au deux en même temps.
}
\cadre{rouge}{Théorème (Formule des probabilités totales)}{% id="t50"
     Soit  $A_{1}, A_{2}, . . . , A_{n}$ une \textbf{partition} de l'univers $\Omega $. Pour tout événement $B$ :
     \par
     $p\left(B\right)=p\left(A_{1} \cap  B\right) + p\left(A_{2} \cap  B\right) + . . . + p\left(A_{n} \cap  B\right)$
     \par
     $p\left(B\right)=p\left(A_{1}\right)\times p_{A_{1}}\left(B\right) + p\left(A_{2}\right)\times p_{A_{2}}\left(B\right) + . . . + p\left(A_{n}\right)\times p_{A_{n}}\left(B\right)$
}
\cadre{vert}{Cas particulier fréquent}{% id="p55"
     Comme $A$ et $\overline{A}$ forme une partition de l'univers :
     \par
     $p\left(B\right)=p\left(A \cap  B\right) + p\left(\overline{A} \cap  B\right)=p\left(A\right)\times p_{A}\left(B\right) + p\left(\overline{A}\right)\times p_{\overline{A}}\left(B\right)$
}
\bloc{cyan}{Remarque}{% id="r55"
     Le diagramme ci-dessous montre, sur un arbre,  les chemins à prendre en compte pour calculer $p\left(B\right)$. Ce sont les chemins qui aboutissent à $B$.
     \begin{extern} %width="350" alt="arbre pondéré" class="aligncenter"
          % Racine à Gauche, développement vers la droite
          \begin{tikzpicture}[xscale=1,yscale=1]
               % Styles (MODIFIABLES)
               \tikzstyle{fleche}=[-,>=latex,thick]
               \tikzstyle{noeud}=[fill=white,circle,draw]
               \tikzstyle{feuille}=[fill=white,circle,draw]
               \tikzstyle{etiquette}=[midway,fill=white]
               % Dimensions (MODIFIABLES)
               \def\DistanceInterNiveaux{3}
               \def\DistanceInterFeuilles{2}
               % Dimensions calculées (NON MODIFIABLES)
               \def\NiveauA{(0)*\DistanceInterNiveaux}
               \def\NiveauB{(1.5)*\DistanceInterNiveaux}
               \def\NiveauC{(2.5)*\DistanceInterNiveaux}
               \def\InterFeuilles{(-1)*\DistanceInterFeuilles}
               % Noeuds (MODIFIABLES : Styles et Coefficients d'InterFeuilles)
               \node[red,noeud] (R) at ({\NiveauA},{(1.5)*\InterFeuilles}) {$\ $};
               \node[red,noeud] (Ra) at ({\NiveauB},{(0.5)*\InterFeuilles}) {$A$};
               \node[red,feuille] (Raa) at ({\NiveauC},{(0)*\InterFeuilles}) {$B$};
               \node[feuille] (Rab) at ({\NiveauC},{(1)*\InterFeuilles}) {$\overline{B}$};
               \node[red,noeud] (Rb) at ({\NiveauB},{(2.5)*\InterFeuilles}) {$\overline{A}$};
               \node[red,feuille] (Rba) at ({\NiveauC},{(2)*\InterFeuilles}) {$B$};
               \node[feuille] (Rbb) at ({\NiveauC},{(3)*\InterFeuilles}) {$\overline{B}$};
               % Arcs (MODIFIABLES : Styles)
               \draw[red,fleche] (R)--(Ra) node[etiquette] {$\color{red} p(A)$};
               \draw[red,fleche] (Ra)--(Raa) node[etiquette] {$\color{red} p_A(B)$};
               \draw[fleche] (Ra)--(Rab) node[etiquette] {$p_A(\overline{B})$};
               \draw[red,fleche] (R)--(Rb) node[etiquette] {$\color{red} p(\overline{A})$};
               \draw[red,fleche] (Rb)--(Rba) node[etiquette] {$\color{red} p_{\overline{A}}(B)$};
               \draw[fleche] (Rb)--(Rbb) node[etiquette]  {$p_{\overline{A}}(\overline{B})$};
          \end{tikzpicture}
\end{extern}}
\bloc{orange}{Exemple}{% id="e55"
     Si on reprend l'exemple ci-dessus, la probabilité que la seconde boule soit blanche est :
     \par
     $p\left(B_{2}\right)=p\left(B_{1} \cap  B_{2}\right) + p\left(\overline{B_{1}} \cap  B_{2}\right)$
     \par
     $p\left(B_{2}\right)=p\left(B_{1}\right)\times p_{B_{1}}\left(B_{2}\right) + p\left(\overline{B_{1}}\right)\times p_{\overline{B_{1}}}\left(B_{2}\right)$
     \begin{extern} %width="350" alt="arbre pondéré" class="aligncenter"
          % Racine à Gauche, développement vers la droite
          \begin{tikzpicture}[xscale=1,yscale=1]
               % Styles (MODIFIABLES)
               \tikzstyle{fleche}=[-,>=latex,thick]
               \tikzstyle{noeud}=[fill=white,circle,draw]
               \tikzstyle{feuille}=[fill=white,circle,draw]
               \tikzstyle{etiquette}=[midway,fill=white]
               % Dimensions (MODIFIABLES)
               \def\DistanceInterNiveaux{3}
               \def\DistanceInterFeuilles{2}
               % Dimensions calculées (NON MODIFIABLES)
               \def\NiveauA{(0)*\DistanceInterNiveaux}
               \def\NiveauB{(1.5)*\DistanceInterNiveaux}
               \def\NiveauC{(2.5)*\DistanceInterNiveaux}
               \def\InterFeuilles{(-1)*\DistanceInterFeuilles}
               % Noeuds (MODIFIABLES : Styles et Coefficients d'InterFeuilles)
               \node[red,noeud] (R) at ({\NiveauA},{(1.5)*\InterFeuilles}) {$\ $};
               \node[red,noeud] (Ra) at ({\NiveauB},{(0.5)*\InterFeuilles}) {$B_1$};
               \node[red,feuille] (Raa) at ({\NiveauC},{(0)*\InterFeuilles}) {$B_2$};
               \node[feuille] (Rab) at ({\NiveauC},{(1)*\InterFeuilles}) {$\overline{B_2}$};
               \node[red,noeud] (Rb) at ({\NiveauB},{(2.5)*\InterFeuilles}) {$\overline{B_1}$};
               \node[red,feuille] (Rba) at ({\NiveauC},{(2)*\InterFeuilles}) {$B_2$};
               \node[feuille] (Rbb) at ({\NiveauC},{(3)*\InterFeuilles}) {$\overline{B_2}$};
               % Arcs (MODIFIABLES : Styles)
               \draw[red,fleche] (R)--(Ra) node[etiquette] {$\color{red} 3/7$};
               \draw[red,fleche] (Ra)--(Raa) node[etiquette] {$\color{red} 1/3$};
               \draw[fleche] (Ra)--(Rab) node[etiquette] {$2/3$};
               \draw[red,fleche] (R)--(Rb) node[etiquette] {$\color{red} 4/7$};
               \draw[red,fleche] (Rb)--(Rba) node[etiquette] {$\color{red} 1/2$};
               \draw[fleche] (Rb)--(Rbb) node[etiquette] {$1/2$};
          \end{tikzpicture}
     \end{extern}
     $p\left(B_{2}\right)=\frac{3}{7}\times \frac{1}{3}+\frac{4}{7}\times \frac{1}{2}=\frac{1}{7}+\frac{2}{7}=\frac{3}{7}$
}

\end{document}
µ
\documentclass[a4paper]{article}

%================================================================================================================================
%
% Packages
%
%================================================================================================================================

\usepackage[T1]{fontenc} 	% pour caractères accentués
\usepackage[utf8]{inputenc}  % encodage utf8
\usepackage[french]{babel}	% langue : français
\usepackage{fourier}			% caractères plus lisibles
\usepackage[dvipsnames]{xcolor} % couleurs
\usepackage{fancyhdr}		% réglage header footer
\usepackage{needspace}		% empêcher sauts de page mal placés
\usepackage{graphicx}		% pour inclure des graphiques
\usepackage{enumitem,cprotect}		% personnalise les listes d'items (nécessaire pour ol, al ...)
\usepackage{hyperref}		% Liens hypertexte
\usepackage{pstricks,pst-all,pst-node,pstricks-add,pst-math,pst-plot,pst-tree,pst-eucl} % pstricks
\usepackage[a4paper,includeheadfoot,top=2cm,left=3cm, bottom=2cm,right=3cm]{geometry} % marges etc.
\usepackage{comment}			% commentaires multilignes
\usepackage{amsmath,environ} % maths (matrices, etc.)
\usepackage{amssymb,makeidx}
\usepackage{bm}				% bold maths
\usepackage{tabularx}		% tableaux
\usepackage{colortbl}		% tableaux en couleur
\usepackage{fontawesome}		% Fontawesome
\usepackage{environ}			% environment with command
\usepackage{fp}				% calculs pour ps-tricks
\usepackage{multido}			% pour ps tricks
\usepackage[np]{numprint}	% formattage nombre
\usepackage{tikz,tkz-tab} 			% package principal TikZ
\usepackage{pgfplots}   % axes
\usepackage{mathrsfs}    % cursives
\usepackage{calc}			% calcul taille boites
\usepackage[scaled=0.875]{helvet} % font sans serif
\usepackage{svg} % svg
\usepackage{scrextend} % local margin
\usepackage{scratch} %scratch
\usepackage{multicol} % colonnes
%\usepackage{infix-RPN,pst-func} % formule en notation polanaise inversée
\usepackage{listings}

%================================================================================================================================
%
% Réglages de base
%
%================================================================================================================================

\lstset{
language=Python,   % R code
literate=
{á}{{\'a}}1
{à}{{\`a}}1
{ã}{{\~a}}1
{é}{{\'e}}1
{è}{{\`e}}1
{ê}{{\^e}}1
{í}{{\'i}}1
{ó}{{\'o}}1
{õ}{{\~o}}1
{ú}{{\'u}}1
{ü}{{\"u}}1
{ç}{{\c{c}}}1
{~}{{ }}1
}


\definecolor{codegreen}{rgb}{0,0.6,0}
\definecolor{codegray}{rgb}{0.5,0.5,0.5}
\definecolor{codepurple}{rgb}{0.58,0,0.82}
\definecolor{backcolour}{rgb}{0.95,0.95,0.92}

\lstdefinestyle{mystyle}{
    backgroundcolor=\color{backcolour},   
    commentstyle=\color{codegreen},
    keywordstyle=\color{magenta},
    numberstyle=\tiny\color{codegray},
    stringstyle=\color{codepurple},
    basicstyle=\ttfamily\footnotesize,
    breakatwhitespace=false,         
    breaklines=true,                 
    captionpos=b,                    
    keepspaces=true,                 
    numbers=left,                    
xleftmargin=2em,
framexleftmargin=2em,            
    showspaces=false,                
    showstringspaces=false,
    showtabs=false,                  
    tabsize=2,
    upquote=true
}

\lstset{style=mystyle}


\lstset{style=mystyle}
\newcommand{\imgdir}{C:/laragon/www/newmc/assets/imgsvg/}
\newcommand{\imgsvgdir}{C:/laragon/www/newmc/assets/imgsvg/}

\definecolor{mcgris}{RGB}{220, 220, 220}% ancien~; pour compatibilité
\definecolor{mcbleu}{RGB}{52, 152, 219}
\definecolor{mcvert}{RGB}{125, 194, 70}
\definecolor{mcmauve}{RGB}{154, 0, 215}
\definecolor{mcorange}{RGB}{255, 96, 0}
\definecolor{mcturquoise}{RGB}{0, 153, 153}
\definecolor{mcrouge}{RGB}{255, 0, 0}
\definecolor{mclightvert}{RGB}{205, 234, 190}

\definecolor{gris}{RGB}{220, 220, 220}
\definecolor{bleu}{RGB}{52, 152, 219}
\definecolor{vert}{RGB}{125, 194, 70}
\definecolor{mauve}{RGB}{154, 0, 215}
\definecolor{orange}{RGB}{255, 96, 0}
\definecolor{turquoise}{RGB}{0, 153, 153}
\definecolor{rouge}{RGB}{255, 0, 0}
\definecolor{lightvert}{RGB}{205, 234, 190}
\setitemize[0]{label=\color{lightvert}  $\bullet$}

\pagestyle{fancy}
\renewcommand{\headrulewidth}{0.2pt}
\fancyhead[L]{maths-cours.fr}
\fancyhead[R]{\thepage}
\renewcommand{\footrulewidth}{0.2pt}
\fancyfoot[C]{}

\newcolumntype{C}{>{\centering\arraybackslash}X}
\newcolumntype{s}{>{\hsize=.35\hsize\arraybackslash}X}

\setlength{\parindent}{0pt}		 
\setlength{\parskip}{3mm}
\setlength{\headheight}{1cm}

\def\ebook{ebook}
\def\book{book}
\def\web{web}
\def\type{web}

\newcommand{\vect}[1]{\overrightarrow{\,\mathstrut#1\,}}

\def\Oij{$\left(\text{O}~;~\vect{\imath},~\vect{\jmath}\right)$}
\def\Oijk{$\left(\text{O}~;~\vect{\imath},~\vect{\jmath},~\vect{k}\right)$}
\def\Ouv{$\left(\text{O}~;~\vect{u},~\vect{v}\right)$}

\hypersetup{breaklinks=true, colorlinks = true, linkcolor = OliveGreen, urlcolor = OliveGreen, citecolor = OliveGreen, pdfauthor={Didier BONNEL - https://www.maths-cours.fr} } % supprime les bordures autour des liens

\renewcommand{\arg}[0]{\text{arg}}

\everymath{\displaystyle}

%================================================================================================================================
%
% Macros - Commandes
%
%================================================================================================================================

\newcommand\meta[2]{    			% Utilisé pour créer le post HTML.
	\def\titre{titre}
	\def\url{url}
	\def\arg{#1}
	\ifx\titre\arg
		\newcommand\maintitle{#2}
		\fancyhead[L]{#2}
		{\Large\sffamily \MakeUppercase{#2}}
		\vspace{1mm}\textcolor{mcvert}{\hrule}
	\fi 
	\ifx\url\arg
		\fancyfoot[L]{\href{https://www.maths-cours.fr#2}{\black \footnotesize{https://www.maths-cours.fr#2}}}
	\fi 
}


\newcommand\TitreC[1]{    		% Titre centré
     \needspace{3\baselineskip}
     \begin{center}\textbf{#1}\end{center}
}

\newcommand\newpar{    		% paragraphe
     \par
}

\newcommand\nosp {    		% commande vide (pas d'espace)
}
\newcommand{\id}[1]{} %ignore

\newcommand\boite[2]{				% Boite simple sans titre
	\vspace{5mm}
	\setlength{\fboxrule}{0.2mm}
	\setlength{\fboxsep}{5mm}	
	\fcolorbox{#1}{#1!3}{\makebox[\linewidth-2\fboxrule-2\fboxsep]{
  		\begin{minipage}[t]{\linewidth-2\fboxrule-4\fboxsep}\setlength{\parskip}{3mm}
  			 #2
  		\end{minipage}
	}}
	\vspace{5mm}
}

\newcommand\CBox[4]{				% Boites
	\vspace{5mm}
	\setlength{\fboxrule}{0.2mm}
	\setlength{\fboxsep}{5mm}
	
	\fcolorbox{#1}{#1!3}{\makebox[\linewidth-2\fboxrule-2\fboxsep]{
		\begin{minipage}[t]{1cm}\setlength{\parskip}{3mm}
	  		\textcolor{#1}{\LARGE{#2}}    
 	 	\end{minipage}  
  		\begin{minipage}[t]{\linewidth-2\fboxrule-4\fboxsep}\setlength{\parskip}{3mm}
			\raisebox{1.2mm}{\normalsize\sffamily{\textcolor{#1}{#3}}}						
  			 #4
  		\end{minipage}
	}}
	\vspace{5mm}
}

\newcommand\cadre[3]{				% Boites convertible html
	\par
	\vspace{2mm}
	\setlength{\fboxrule}{0.1mm}
	\setlength{\fboxsep}{5mm}
	\fcolorbox{#1}{white}{\makebox[\linewidth-2\fboxrule-2\fboxsep]{
  		\begin{minipage}[t]{\linewidth-2\fboxrule-4\fboxsep}\setlength{\parskip}{3mm}
			\raisebox{-2.5mm}{\sffamily \small{\textcolor{#1}{\MakeUppercase{#2}}}}		
			\par		
  			 #3
 	 		\end{minipage}
	}}
		\vspace{2mm}
	\par
}

\newcommand\bloc[3]{				% Boites convertible html sans bordure
     \needspace{2\baselineskip}
     {\sffamily \small{\textcolor{#1}{\MakeUppercase{#2}}}}    
		\par		
  			 #3
		\par
}

\newcommand\CHelp[1]{
     \CBox{Plum}{\faInfoCircle}{À RETENIR}{#1}
}

\newcommand\CUp[1]{
     \CBox{NavyBlue}{\faThumbsOUp}{EN PRATIQUE}{#1}
}

\newcommand\CInfo[1]{
     \CBox{Sepia}{\faArrowCircleRight}{REMARQUE}{#1}
}

\newcommand\CRedac[1]{
     \CBox{PineGreen}{\faEdit}{BIEN R\'EDIGER}{#1}
}

\newcommand\CError[1]{
     \CBox{Red}{\faExclamationTriangle}{ATTENTION}{#1}
}

\newcommand\TitreExo[2]{
\needspace{4\baselineskip}
 {\sffamily\large EXERCICE #1\ (\emph{#2 points})}
\vspace{5mm}
}

\newcommand\img[2]{
          \includegraphics[width=#2\paperwidth]{\imgdir#1}
}

\newcommand\imgsvg[2]{
       \begin{center}   \includegraphics[width=#2\paperwidth]{\imgsvgdir#1} \end{center}
}


\newcommand\Lien[2]{
     \href{#1}{#2 \tiny \faExternalLink}
}
\newcommand\mcLien[2]{
     \href{https~://www.maths-cours.fr/#1}{#2 \tiny \faExternalLink}
}

\newcommand{\euro}{\eurologo{}}

%================================================================================================================================
%
% Macros - Environement
%
%================================================================================================================================

\newenvironment{tex}{ %
}
{%
}

\newenvironment{indente}{ %
	\setlength\parindent{10mm}
}

{
	\setlength\parindent{0mm}
}

\newenvironment{corrige}{%
     \needspace{3\baselineskip}
     \medskip
     \textbf{\textsc{Corrigé}}
     \medskip
}
{
}

\newenvironment{extern}{%
     \begin{center}
     }
     {
     \end{center}
}

\NewEnviron{code}{%
	\par
     \boite{gray}{\texttt{%
     \BODY
     }}
     \par
}

\newenvironment{vbloc}{% boite sans cadre empeche saut de page
     \begin{minipage}[t]{\linewidth}
     }
     {
     \end{minipage}
}
\NewEnviron{h2}{%
    \needspace{3\baselineskip}
    \vspace{0.6cm}
	\noindent \MakeUppercase{\sffamily \large \BODY}
	\vspace{1mm}\textcolor{mcgris}{\hrule}\vspace{0.4cm}
	\par
}{}

\NewEnviron{h3}{%
    \needspace{3\baselineskip}
	\vspace{5mm}
	\textsc{\BODY}
	\par
}

\NewEnviron{margeneg}{ %
\begin{addmargin}[-1cm]{0cm}
\BODY
\end{addmargin}
}

\NewEnviron{html}{%
}

\begin{document}
\meta{url}{/cours/estimation/}
\meta{pid}{417}
\meta{titre}{Estimation en Terminale ES et L}
\meta{type}{cours}
\begin{h2}I - Intervalle de fluctuation\end{h2}
Pour étudier un caractère présent dans une population, on prélève de façon aléatoire un échantillon dans cette population.
\par
On suppose connues :
\begin{itemize}
     \item la proportion $p$ du caractère \textbf{dans la population}
     \item la taille $n$ de l'échantillon
\end{itemize}
On cherche à évaluer :
\begin{itemize}
     \item la fréquence $f$ du caractère \textbf{dans l'échantillon}
\end{itemize}
\bloc{orange}{Exemple}{% id="e10"
     On sait que 48\% des élèves d'un lycée sont des garçons (et donc 52\% sont des filles...).
     \par
     Si l'on sélectionne au hasard 100 élèves dans l'établissement, on devrait obtenir \textbf{\textit{environ}} 52 filles et 48 garçons mais il n'est pas du tout certain que l'on obtienne \textbf{exactement} ces chiffres.
     \par
     Par contre, on pourra rechercher un intervalle dans lequel se situera \textit{"probablement"} la proportion de garçons dans cet échantillon.
}
Si $n$ est élevé, on peut assimiler la sélection de l'échantillon à un tirage avec remise. Le nombre d'individus présentant le caractère étudié suit alors une loi binomiale $\mathscr B\left(n,p\right)$. Pour $n$ élevé, on peut approximer cette loi binomiale par une loi normale. On obtient alors le résultat suivant :
\cadre{bleu}{Définition et propriété} l'intervalle :
\begin{center}$I=\left[ p-1,96\times \frac{\sqrt{p\left(1-p\right)}}{\sqrt{n}} ;\right.$\nosp$\left. p+1,96\times \frac{\sqrt{p\left(1-p\right)}}{\sqrt{n}} \right]$\end{center}
Cela s'interprète de la façon suivante :
\par
Pour $n$ élevé, la probabilité que la fréquence $f$ du caractère dans l'échantillon appartienne à $I$ est 0,95.
}
\bloc{orange}{Exemple}{% id="e20"
     Si l'on reprend l'exemple précédent, on a $n=100$ et $p=\frac{48}{100}$.
     \par
     On trouve $I= \left[0,38 ; 0,58\right]$.
     \par
     La proportion de garçons dans l'échantillon devrait être comprise entre 38\% et 58\% (avec une probabilité de 0,95)
}
\bloc{cyan}{Remarques}{% id="r20"
     \begin{itemize}
          \item On considèrera que $n$ est suffisamment élevé pour utiliser cet intervalle de fluctuation si $n\geqslant 30$, $np\geqslant 5$ et $n\left(1-p\right)\geqslant 5$
          \item L' intervalle de fluctuation peut être utilisé pour valider ou rejeter une hypothèse. On procède de la façon suivante :
          \begin{itemize}[label=---]
               \item %
               On suppose que la proportion du caractère étudié est $p$.
               \item %
               On prélève un échantillon de taille $n$.
               \item %
               On regarde si la fréquence $f$ du caractère dans l'échantillon appartient à $I$.
               \item %
               Si oui, l'hypothèse est validée ; si non, elle est rejetée.\\
               Le risque de rejeter l'hypothèse à tort est alors inférieur à 5\%.
          \end{itemize}
          \item Pour des valeurs moyennes de $p$ (par exemple $0,2\leqslant p\leqslant 0,8$),  $1,96\times \sqrt{p\left(1-p\right)}$ est proche de 1 (et légèrement inférieur). Si l'on arrondit  $1,96\times \sqrt{p\left(1-p\right)}$ à 1, on obtient :
          \begin{center}$I=\left[ p-\frac{1}{\sqrt{n}} ; p+\frac{1}{\sqrt{n}} \right]$\end{center}
          qui est l'intervalle vu en Seconde.
     \end{itemize}
}
\begin{h2}II - Intervalle de confiance\end{h2}
Dans cette partie (contrairement à la première partie), on suppose que l'\textbf{on connait la fréquence $f$ du caractère dans l'échantillon} mais que l' \textbf{on ne connait pas la proportion $p$ du caractère dans la population}.
\par
On cherche alors à évaluer $p$.
\cadre{bleu}{Définition et propriété} l'intervalle :
     \begin{center}$I=\left[ f-\frac{1}{\sqrt{n}} ; f+\frac{1}{\sqrt{n}} \right]$\end{center}
     Pour $n$ élevé, la proportion $p$ du caractère dans la population appartiendra à $I$ dans 95\% des cas.
}
\bloc{orange}{Exemple}{% id="e50"
     On recherche le pourcentage de truites femelles dans un élevage de truites.
     \par
     Pour cela, on a prélevé un échantillon de 50 truites et on a comptabilisé 28 femelles dans cet échantillon.
     \par
     Le pourcentage de truites femelles dans l'ensemble de l'élevage appartient donc à l'intervalle :
     \begin{center}$I=\left[ \frac{28}{50}-\frac{1}{\sqrt{50}} ; \frac{28}{50}+\frac{1}{\sqrt{50}} \right] \approx  \left[0,42 ; 0,70\right]$\end{center}
     avec un risque d'erreur inférieur à 5\%.
}
\bloc{cyan}{Remarque}{% id="r50"
     La longueur de l'intervalle $I$ est $\frac{2}{\sqrt{n}}$.
     \par
     Si l'on souhaite obtenir un intervalle d'amplitude maximale $a$, il faut choisir $n$ tel que $\frac{2}{\sqrt{n}}\leqslant a$ c'est à dire $n\geqslant \frac{4}{a^{2}}$.
}

\end{document}
µ
\documentclass[a4paper]{article}

%================================================================================================================================
%
% Packages
%
%================================================================================================================================

\usepackage[T1]{fontenc} 	% pour caractères accentués
\usepackage[utf8]{inputenc}  % encodage utf8
\usepackage[french]{babel}	% langue : français
\usepackage{fourier}			% caractères plus lisibles
\usepackage[dvipsnames]{xcolor} % couleurs
\usepackage{fancyhdr}		% réglage header footer
\usepackage{needspace}		% empêcher sauts de page mal placés
\usepackage{graphicx}		% pour inclure des graphiques
\usepackage{enumitem,cprotect}		% personnalise les listes d'items (nécessaire pour ol, al ...)
\usepackage{hyperref}		% Liens hypertexte
\usepackage{pstricks,pst-all,pst-node,pstricks-add,pst-math,pst-plot,pst-tree,pst-eucl} % pstricks
\usepackage[a4paper,includeheadfoot,top=2cm,left=3cm, bottom=2cm,right=3cm]{geometry} % marges etc.
\usepackage{comment}			% commentaires multilignes
\usepackage{amsmath,environ} % maths (matrices, etc.)
\usepackage{amssymb,makeidx}
\usepackage{bm}				% bold maths
\usepackage{tabularx}		% tableaux
\usepackage{colortbl}		% tableaux en couleur
\usepackage{fontawesome}		% Fontawesome
\usepackage{environ}			% environment with command
\usepackage{fp}				% calculs pour ps-tricks
\usepackage{multido}			% pour ps tricks
\usepackage[np]{numprint}	% formattage nombre
\usepackage{tikz,tkz-tab} 			% package principal TikZ
\usepackage{pgfplots}   % axes
\usepackage{mathrsfs}    % cursives
\usepackage{calc}			% calcul taille boites
\usepackage[scaled=0.875]{helvet} % font sans serif
\usepackage{svg} % svg
\usepackage{scrextend} % local margin
\usepackage{scratch} %scratch
\usepackage{multicol} % colonnes
%\usepackage{infix-RPN,pst-func} % formule en notation polanaise inversée
\usepackage{listings}

%================================================================================================================================
%
% Réglages de base
%
%================================================================================================================================

\lstset{
language=Python,   % R code
literate=
{á}{{\'a}}1
{à}{{\`a}}1
{ã}{{\~a}}1
{é}{{\'e}}1
{è}{{\`e}}1
{ê}{{\^e}}1
{í}{{\'i}}1
{ó}{{\'o}}1
{õ}{{\~o}}1
{ú}{{\'u}}1
{ü}{{\"u}}1
{ç}{{\c{c}}}1
{~}{{ }}1
}


\definecolor{codegreen}{rgb}{0,0.6,0}
\definecolor{codegray}{rgb}{0.5,0.5,0.5}
\definecolor{codepurple}{rgb}{0.58,0,0.82}
\definecolor{backcolour}{rgb}{0.95,0.95,0.92}

\lstdefinestyle{mystyle}{
    backgroundcolor=\color{backcolour},   
    commentstyle=\color{codegreen},
    keywordstyle=\color{magenta},
    numberstyle=\tiny\color{codegray},
    stringstyle=\color{codepurple},
    basicstyle=\ttfamily\footnotesize,
    breakatwhitespace=false,         
    breaklines=true,                 
    captionpos=b,                    
    keepspaces=true,                 
    numbers=left,                    
xleftmargin=2em,
framexleftmargin=2em,            
    showspaces=false,                
    showstringspaces=false,
    showtabs=false,                  
    tabsize=2,
    upquote=true
}

\lstset{style=mystyle}


\lstset{style=mystyle}
\newcommand{\imgdir}{C:/laragon/www/newmc/assets/imgsvg/}
\newcommand{\imgsvgdir}{C:/laragon/www/newmc/assets/imgsvg/}

\definecolor{mcgris}{RGB}{220, 220, 220}% ancien~; pour compatibilité
\definecolor{mcbleu}{RGB}{52, 152, 219}
\definecolor{mcvert}{RGB}{125, 194, 70}
\definecolor{mcmauve}{RGB}{154, 0, 215}
\definecolor{mcorange}{RGB}{255, 96, 0}
\definecolor{mcturquoise}{RGB}{0, 153, 153}
\definecolor{mcrouge}{RGB}{255, 0, 0}
\definecolor{mclightvert}{RGB}{205, 234, 190}

\definecolor{gris}{RGB}{220, 220, 220}
\definecolor{bleu}{RGB}{52, 152, 219}
\definecolor{vert}{RGB}{125, 194, 70}
\definecolor{mauve}{RGB}{154, 0, 215}
\definecolor{orange}{RGB}{255, 96, 0}
\definecolor{turquoise}{RGB}{0, 153, 153}
\definecolor{rouge}{RGB}{255, 0, 0}
\definecolor{lightvert}{RGB}{205, 234, 190}
\setitemize[0]{label=\color{lightvert}  $\bullet$}

\pagestyle{fancy}
\renewcommand{\headrulewidth}{0.2pt}
\fancyhead[L]{maths-cours.fr}
\fancyhead[R]{\thepage}
\renewcommand{\footrulewidth}{0.2pt}
\fancyfoot[C]{}

\newcolumntype{C}{>{\centering\arraybackslash}X}
\newcolumntype{s}{>{\hsize=.35\hsize\arraybackslash}X}

\setlength{\parindent}{0pt}		 
\setlength{\parskip}{3mm}
\setlength{\headheight}{1cm}

\def\ebook{ebook}
\def\book{book}
\def\web{web}
\def\type{web}

\newcommand{\vect}[1]{\overrightarrow{\,\mathstrut#1\,}}

\def\Oij{$\left(\text{O}~;~\vect{\imath},~\vect{\jmath}\right)$}
\def\Oijk{$\left(\text{O}~;~\vect{\imath},~\vect{\jmath},~\vect{k}\right)$}
\def\Ouv{$\left(\text{O}~;~\vect{u},~\vect{v}\right)$}

\hypersetup{breaklinks=true, colorlinks = true, linkcolor = OliveGreen, urlcolor = OliveGreen, citecolor = OliveGreen, pdfauthor={Didier BONNEL - https://www.maths-cours.fr} } % supprime les bordures autour des liens

\renewcommand{\arg}[0]{\text{arg}}

\everymath{\displaystyle}

%================================================================================================================================
%
% Macros - Commandes
%
%================================================================================================================================

\newcommand\meta[2]{    			% Utilisé pour créer le post HTML.
	\def\titre{titre}
	\def\url{url}
	\def\arg{#1}
	\ifx\titre\arg
		\newcommand\maintitle{#2}
		\fancyhead[L]{#2}
		{\Large\sffamily \MakeUppercase{#2}}
		\vspace{1mm}\textcolor{mcvert}{\hrule}
	\fi 
	\ifx\url\arg
		\fancyfoot[L]{\href{https://www.maths-cours.fr#2}{\black \footnotesize{https://www.maths-cours.fr#2}}}
	\fi 
}


\newcommand\TitreC[1]{    		% Titre centré
     \needspace{3\baselineskip}
     \begin{center}\textbf{#1}\end{center}
}

\newcommand\newpar{    		% paragraphe
     \par
}

\newcommand\nosp {    		% commande vide (pas d'espace)
}
\newcommand{\id}[1]{} %ignore

\newcommand\boite[2]{				% Boite simple sans titre
	\vspace{5mm}
	\setlength{\fboxrule}{0.2mm}
	\setlength{\fboxsep}{5mm}	
	\fcolorbox{#1}{#1!3}{\makebox[\linewidth-2\fboxrule-2\fboxsep]{
  		\begin{minipage}[t]{\linewidth-2\fboxrule-4\fboxsep}\setlength{\parskip}{3mm}
  			 #2
  		\end{minipage}
	}}
	\vspace{5mm}
}

\newcommand\CBox[4]{				% Boites
	\vspace{5mm}
	\setlength{\fboxrule}{0.2mm}
	\setlength{\fboxsep}{5mm}
	
	\fcolorbox{#1}{#1!3}{\makebox[\linewidth-2\fboxrule-2\fboxsep]{
		\begin{minipage}[t]{1cm}\setlength{\parskip}{3mm}
	  		\textcolor{#1}{\LARGE{#2}}    
 	 	\end{minipage}  
  		\begin{minipage}[t]{\linewidth-2\fboxrule-4\fboxsep}\setlength{\parskip}{3mm}
			\raisebox{1.2mm}{\normalsize\sffamily{\textcolor{#1}{#3}}}						
  			 #4
  		\end{minipage}
	}}
	\vspace{5mm}
}

\newcommand\cadre[3]{				% Boites convertible html
	\par
	\vspace{2mm}
	\setlength{\fboxrule}{0.1mm}
	\setlength{\fboxsep}{5mm}
	\fcolorbox{#1}{white}{\makebox[\linewidth-2\fboxrule-2\fboxsep]{
  		\begin{minipage}[t]{\linewidth-2\fboxrule-4\fboxsep}\setlength{\parskip}{3mm}
			\raisebox{-2.5mm}{\sffamily \small{\textcolor{#1}{\MakeUppercase{#2}}}}		
			\par		
  			 #3
 	 		\end{minipage}
	}}
		\vspace{2mm}
	\par
}

\newcommand\bloc[3]{				% Boites convertible html sans bordure
     \needspace{2\baselineskip}
     {\sffamily \small{\textcolor{#1}{\MakeUppercase{#2}}}}    
		\par		
  			 #3
		\par
}

\newcommand\CHelp[1]{
     \CBox{Plum}{\faInfoCircle}{À RETENIR}{#1}
}

\newcommand\CUp[1]{
     \CBox{NavyBlue}{\faThumbsOUp}{EN PRATIQUE}{#1}
}

\newcommand\CInfo[1]{
     \CBox{Sepia}{\faArrowCircleRight}{REMARQUE}{#1}
}

\newcommand\CRedac[1]{
     \CBox{PineGreen}{\faEdit}{BIEN R\'EDIGER}{#1}
}

\newcommand\CError[1]{
     \CBox{Red}{\faExclamationTriangle}{ATTENTION}{#1}
}

\newcommand\TitreExo[2]{
\needspace{4\baselineskip}
 {\sffamily\large EXERCICE #1\ (\emph{#2 points})}
\vspace{5mm}
}

\newcommand\img[2]{
          \includegraphics[width=#2\paperwidth]{\imgdir#1}
}

\newcommand\imgsvg[2]{
       \begin{center}   \includegraphics[width=#2\paperwidth]{\imgsvgdir#1} \end{center}
}


\newcommand\Lien[2]{
     \href{#1}{#2 \tiny \faExternalLink}
}
\newcommand\mcLien[2]{
     \href{https~://www.maths-cours.fr/#1}{#2 \tiny \faExternalLink}
}

\newcommand{\euro}{\eurologo{}}

%================================================================================================================================
%
% Macros - Environement
%
%================================================================================================================================

\newenvironment{tex}{ %
}
{%
}

\newenvironment{indente}{ %
	\setlength\parindent{10mm}
}

{
	\setlength\parindent{0mm}
}

\newenvironment{corrige}{%
     \needspace{3\baselineskip}
     \medskip
     \textbf{\textsc{Corrigé}}
     \medskip
}
{
}

\newenvironment{extern}{%
     \begin{center}
     }
     {
     \end{center}
}

\NewEnviron{code}{%
	\par
     \boite{gray}{\texttt{%
     \BODY
     }}
     \par
}

\newenvironment{vbloc}{% boite sans cadre empeche saut de page
     \begin{minipage}[t]{\linewidth}
     }
     {
     \end{minipage}
}
\NewEnviron{h2}{%
    \needspace{3\baselineskip}
    \vspace{0.6cm}
	\noindent \MakeUppercase{\sffamily \large \BODY}
	\vspace{1mm}\textcolor{mcgris}{\hrule}\vspace{0.4cm}
	\par
}{}

\NewEnviron{h3}{%
    \needspace{3\baselineskip}
	\vspace{5mm}
	\textsc{\BODY}
	\par
}

\NewEnviron{margeneg}{ %
\begin{addmargin}[-1cm]{0cm}
\BODY
\end{addmargin}
}

\NewEnviron{html}{%
}

\begin{document}
\meta{url}{/cours/fonctions-continues/}
\meta{pid}{421}
\meta{titre}{Fonctions continues}
\meta{type}{cours}
\begin{h2}1. Fonctions continues\end{h2}
\cadre{bleu}{Définition}{% id="d10"
     Une fonction définie sur un intervalle $I$ est \textit{\textbf{continue}} sur $I$ si l'on peut tracer sa courbe représentative \textit{sans lever le crayon}
}
\bloc{orange}{Exemples}{% id="e10"
     \begin{itemize}
          \item Les fonctions polynômes sont continues sur $\mathbb{R}$.
          \item Les fonctions rationnelles sont continues sur chaque intervalle contenu dans leur ensemble de définition.
          \item La fonction \textit{racine carrée} est continue sur $\mathbb{R}^+$.
          \item Les fonctions \textit{sinus} et \textit{cosinus} sont continues sur $\mathbb{R}$
     \end{itemize}
}
\cadre{rouge}{Théorème}{% id="t20"
     Si $f$ et $g$ sont continues sur $I$, les fonctions $f+g$, $kf$ ( $k\in \mathbb{R}$ ) et $f\times g$ sont continues sur $I$.
     \par
     Si, de plus, $g$ ne s'annule pas sur $I$, la fonction $\frac{f}{g}$, est continue sur $I$.
}
\cadre{rouge}{Théorème (lien entre continuité et dérivabilité)}{% id="t30"
     Toute fonction \textbf{dérivable} sur un intervalle $I$ est \textbf{continue} sur $I$.
}
\bloc{cyan}{Remarque}{% id="r30"
     \textbf{Attention !} La réciproque est fausse.
     \par
     Par exemple, la fonction valeur absolue ($x\mapsto |x|$) est continue sur $\mathbb{R}$ tout entier mais n'est pas dérivable en 0.
}
\begin{h2}2. Théorème des valeurs intermédiaires\end{h2}
\cadre{rouge}{Théorème des valeurs intermédiaires}{% id="t60"
     Si $f$ est une fonction \textbf{continue} sur un intervalle $\left[a;b\right]$ et si $y_{0}$ est compris entre $f\left(a\right)$ et $f\left(b\right)$, alors l'équation $f\left(x\right)=y_{0}$ admet \textbf{au moins une} solution sur l'intervalle $\left[a;b\right]$.
}
\bloc{cyan}{Remarques}{% id="r60"
     \begin{itemize}
          \item Ce théorème dit que l'équation $f\left(x\right)=y_{0}$ admet \textbf{une ou plusieurs solutions} mais ne permet pas de déterminer le nombre de ces solutions. Dans les exercices où l'on recherche le nombre de solutions, il faut utiliser le corollaire ci-dessous.
          \item \textbf{Cas particulier fréquent : } Si $f$ est continue et si $f\left(a\right)$ et $f\left(b\right)$ sont de signes contraires, l'équation $f\left(x\right)=0$ admet au moins une solution sur l'intervalle $\left[a;b\right]$ (en effet,  si $f\left(a\right)$ et $f\left(b\right)$ sont de signes contraires, $0$ est compris entre $f\left(a\right)$ et $f\left(b\right)$)
     \end{itemize}
}
\cadre{rouge}{Corollaire (du théorème des valeurs intermédiaires)}{% id="t70"
     Si $f$ est une fonction \textbf{continue} et \textbf{strictement monotone} sur un intervalle $\left[a;b\right]$ et si $y_{0}$ est compris entre $f\left(a\right)$ et $f\left(b\right)$, l'équation $f\left(x\right)=y_{0}$ admet une \textbf{unique} solution sur l'intervalle $\left[a;b\right]$.
}
\bloc{cyan}{Remarques}{% id="r70"
     \begin{itemize}
          \item
          Il faut vérifier \textbf{3 conditions} pour pouvoir appliquer ce corollaire:
          \begin{itemize}
               \item $f$ est continue sur $\left[a;b\right]$
               \item $f$ est strictement croissante ou strictement décroissante sur $\left[a;b\right]$
               \item $y_{0}$ est compris entre $f\left(a\right)$ et $f\left(b\right)$
          \end{itemize}
     \end{itemize}
}
\bloc{orange}{Exemple}{% id="e70"
     Soit une fonction $f$ définie sur $[0 ; 4]$ dont le tableau de variations est fourni ci-dessous :
     \begin{center}
          \begin{extern}%width="230" alt="tableau de variations - théorème des valeurs intermédiaires"
               \begin{tikzpicture}[scale=0.875]
                    % Styles
                    \tikzstyle{cadre}=[thin]
                    \tikzstyle{fleche}=[->,>=latex,thin]
                    \tikzstyle{nondefini}=[lightgray]
                    % Dimensions Modifiables
                    \def\Lrg{1.5}
                    \def\HtX{1}
                    \def\HtY{0.5}
                    % Dimensions Calculées
                    \def\lignex{-0.5*\HtX}
                    \def\lignef{-1.5*\HtX}
                    \def\separateur{-0.5*\Lrg}
                    % Largeur du tableau
                    \def\gauche{-1.5*\Lrg}
                    \def\droite{2.5*\Lrg}
                    % Hauteur du tableau
                    \def\haut{0.5*\HtX}
                    \def\bas{-2.5*\HtX-2*\HtY}
                    % Ligne de l'abscisse : x
                    \node at (-1*\Lrg,0) {$x$};
                    \node at (0*\Lrg,0) {$0$};
                    \node at (2*\Lrg,0) {$4$};
                    % Ligne de la dérivée : f'(x)
                    \node at (-1*\Lrg,-1*\HtX) {$f'(x)$};
                    \node at (0*\Lrg,-1*\HtX) {$$};
                    \node at (1*\Lrg,-1*\HtX) {$+$};
                    \node at (2*\Lrg,-1*\HtX) {$$};
                    % Ligne de la fonction : f(x)
                    \node  at (-1*\Lrg,{-2*\HtX+(-1)*\HtY}) {$f(x)$};
                    \node (f1) at (0*\Lrg,{-2*\HtX+(-2)*\HtY}) {$-3$};
                    \node (f2) at (2*\Lrg,{-2*\HtX+(0)*\HtY}) {$1$};
                    % Flèches
                    \draw[fleche] (f1) -- (f2);
                    % Encadrement
                    \draw[cadre] (\separateur,\haut) -- (\separateur,\bas);
                    \draw[cadre] (\gauche,\haut) rectangle  (\droite,\bas);
                    \draw[cadre] (\gauche,\lignex) -- (\droite,\lignex);
                    \draw[cadre] (\gauche,\lignef) -- (\droite,\lignef);
               \end{tikzpicture}
          \end{extern}
     \end{center}
     On cherche à déterminer le nombre de solutions de l'équation $f\left(x\right)=-1$
     \par
     L'unique flèche oblique montre que la fonction $f$ est \textbf{continue} et \textbf{strictement croissante} sur $\left[0;4\right]$.
     \par
     $-1$ est compris entre $f\left(0\right)=-3 $ et  $f\left(4\right)=1 $ .
     \par
     Par conséquent, l'équation $f\left(x\right)=-1$ admet une \textbf{unique} solution sur l'intervalle $[0 ; 4]$.
}

\end{document}
µ
\documentclass[a4paper]{article}

%================================================================================================================================
%
% Packages
%
%================================================================================================================================

\usepackage[T1]{fontenc} 	% pour caractères accentués
\usepackage[utf8]{inputenc}  % encodage utf8
\usepackage[french]{babel}	% langue : français
\usepackage{fourier}			% caractères plus lisibles
\usepackage[dvipsnames]{xcolor} % couleurs
\usepackage{fancyhdr}		% réglage header footer
\usepackage{needspace}		% empêcher sauts de page mal placés
\usepackage{graphicx}		% pour inclure des graphiques
\usepackage{enumitem,cprotect}		% personnalise les listes d'items (nécessaire pour ol, al ...)
\usepackage{hyperref}		% Liens hypertexte
\usepackage{pstricks,pst-all,pst-node,pstricks-add,pst-math,pst-plot,pst-tree,pst-eucl} % pstricks
\usepackage[a4paper,includeheadfoot,top=2cm,left=3cm, bottom=2cm,right=3cm]{geometry} % marges etc.
\usepackage{comment}			% commentaires multilignes
\usepackage{amsmath,environ} % maths (matrices, etc.)
\usepackage{amssymb,makeidx}
\usepackage{bm}				% bold maths
\usepackage{tabularx}		% tableaux
\usepackage{colortbl}		% tableaux en couleur
\usepackage{fontawesome}		% Fontawesome
\usepackage{environ}			% environment with command
\usepackage{fp}				% calculs pour ps-tricks
\usepackage{multido}			% pour ps tricks
\usepackage[np]{numprint}	% formattage nombre
\usepackage{tikz,tkz-tab} 			% package principal TikZ
\usepackage{pgfplots}   % axes
\usepackage{mathrsfs}    % cursives
\usepackage{calc}			% calcul taille boites
\usepackage[scaled=0.875]{helvet} % font sans serif
\usepackage{svg} % svg
\usepackage{scrextend} % local margin
\usepackage{scratch} %scratch
\usepackage{multicol} % colonnes
%\usepackage{infix-RPN,pst-func} % formule en notation polanaise inversée
\usepackage{listings}

%================================================================================================================================
%
% Réglages de base
%
%================================================================================================================================

\lstset{
language=Python,   % R code
literate=
{á}{{\'a}}1
{à}{{\`a}}1
{ã}{{\~a}}1
{é}{{\'e}}1
{è}{{\`e}}1
{ê}{{\^e}}1
{í}{{\'i}}1
{ó}{{\'o}}1
{õ}{{\~o}}1
{ú}{{\'u}}1
{ü}{{\"u}}1
{ç}{{\c{c}}}1
{~}{{ }}1
}


\definecolor{codegreen}{rgb}{0,0.6,0}
\definecolor{codegray}{rgb}{0.5,0.5,0.5}
\definecolor{codepurple}{rgb}{0.58,0,0.82}
\definecolor{backcolour}{rgb}{0.95,0.95,0.92}

\lstdefinestyle{mystyle}{
    backgroundcolor=\color{backcolour},   
    commentstyle=\color{codegreen},
    keywordstyle=\color{magenta},
    numberstyle=\tiny\color{codegray},
    stringstyle=\color{codepurple},
    basicstyle=\ttfamily\footnotesize,
    breakatwhitespace=false,         
    breaklines=true,                 
    captionpos=b,                    
    keepspaces=true,                 
    numbers=left,                    
xleftmargin=2em,
framexleftmargin=2em,            
    showspaces=false,                
    showstringspaces=false,
    showtabs=false,                  
    tabsize=2,
    upquote=true
}

\lstset{style=mystyle}


\lstset{style=mystyle}
\newcommand{\imgdir}{C:/laragon/www/newmc/assets/imgsvg/}
\newcommand{\imgsvgdir}{C:/laragon/www/newmc/assets/imgsvg/}

\definecolor{mcgris}{RGB}{220, 220, 220}% ancien~; pour compatibilité
\definecolor{mcbleu}{RGB}{52, 152, 219}
\definecolor{mcvert}{RGB}{125, 194, 70}
\definecolor{mcmauve}{RGB}{154, 0, 215}
\definecolor{mcorange}{RGB}{255, 96, 0}
\definecolor{mcturquoise}{RGB}{0, 153, 153}
\definecolor{mcrouge}{RGB}{255, 0, 0}
\definecolor{mclightvert}{RGB}{205, 234, 190}

\definecolor{gris}{RGB}{220, 220, 220}
\definecolor{bleu}{RGB}{52, 152, 219}
\definecolor{vert}{RGB}{125, 194, 70}
\definecolor{mauve}{RGB}{154, 0, 215}
\definecolor{orange}{RGB}{255, 96, 0}
\definecolor{turquoise}{RGB}{0, 153, 153}
\definecolor{rouge}{RGB}{255, 0, 0}
\definecolor{lightvert}{RGB}{205, 234, 190}
\setitemize[0]{label=\color{lightvert}  $\bullet$}

\pagestyle{fancy}
\renewcommand{\headrulewidth}{0.2pt}
\fancyhead[L]{maths-cours.fr}
\fancyhead[R]{\thepage}
\renewcommand{\footrulewidth}{0.2pt}
\fancyfoot[C]{}

\newcolumntype{C}{>{\centering\arraybackslash}X}
\newcolumntype{s}{>{\hsize=.35\hsize\arraybackslash}X}

\setlength{\parindent}{0pt}		 
\setlength{\parskip}{3mm}
\setlength{\headheight}{1cm}

\def\ebook{ebook}
\def\book{book}
\def\web{web}
\def\type{web}

\newcommand{\vect}[1]{\overrightarrow{\,\mathstrut#1\,}}

\def\Oij{$\left(\text{O}~;~\vect{\imath},~\vect{\jmath}\right)$}
\def\Oijk{$\left(\text{O}~;~\vect{\imath},~\vect{\jmath},~\vect{k}\right)$}
\def\Ouv{$\left(\text{O}~;~\vect{u},~\vect{v}\right)$}

\hypersetup{breaklinks=true, colorlinks = true, linkcolor = OliveGreen, urlcolor = OliveGreen, citecolor = OliveGreen, pdfauthor={Didier BONNEL - https://www.maths-cours.fr} } % supprime les bordures autour des liens

\renewcommand{\arg}[0]{\text{arg}}

\everymath{\displaystyle}

%================================================================================================================================
%
% Macros - Commandes
%
%================================================================================================================================

\newcommand\meta[2]{    			% Utilisé pour créer le post HTML.
	\def\titre{titre}
	\def\url{url}
	\def\arg{#1}
	\ifx\titre\arg
		\newcommand\maintitle{#2}
		\fancyhead[L]{#2}
		{\Large\sffamily \MakeUppercase{#2}}
		\vspace{1mm}\textcolor{mcvert}{\hrule}
	\fi 
	\ifx\url\arg
		\fancyfoot[L]{\href{https://www.maths-cours.fr#2}{\black \footnotesize{https://www.maths-cours.fr#2}}}
	\fi 
}


\newcommand\TitreC[1]{    		% Titre centré
     \needspace{3\baselineskip}
     \begin{center}\textbf{#1}\end{center}
}

\newcommand\newpar{    		% paragraphe
     \par
}

\newcommand\nosp {    		% commande vide (pas d'espace)
}
\newcommand{\id}[1]{} %ignore

\newcommand\boite[2]{				% Boite simple sans titre
	\vspace{5mm}
	\setlength{\fboxrule}{0.2mm}
	\setlength{\fboxsep}{5mm}	
	\fcolorbox{#1}{#1!3}{\makebox[\linewidth-2\fboxrule-2\fboxsep]{
  		\begin{minipage}[t]{\linewidth-2\fboxrule-4\fboxsep}\setlength{\parskip}{3mm}
  			 #2
  		\end{minipage}
	}}
	\vspace{5mm}
}

\newcommand\CBox[4]{				% Boites
	\vspace{5mm}
	\setlength{\fboxrule}{0.2mm}
	\setlength{\fboxsep}{5mm}
	
	\fcolorbox{#1}{#1!3}{\makebox[\linewidth-2\fboxrule-2\fboxsep]{
		\begin{minipage}[t]{1cm}\setlength{\parskip}{3mm}
	  		\textcolor{#1}{\LARGE{#2}}    
 	 	\end{minipage}  
  		\begin{minipage}[t]{\linewidth-2\fboxrule-4\fboxsep}\setlength{\parskip}{3mm}
			\raisebox{1.2mm}{\normalsize\sffamily{\textcolor{#1}{#3}}}						
  			 #4
  		\end{minipage}
	}}
	\vspace{5mm}
}

\newcommand\cadre[3]{				% Boites convertible html
	\par
	\vspace{2mm}
	\setlength{\fboxrule}{0.1mm}
	\setlength{\fboxsep}{5mm}
	\fcolorbox{#1}{white}{\makebox[\linewidth-2\fboxrule-2\fboxsep]{
  		\begin{minipage}[t]{\linewidth-2\fboxrule-4\fboxsep}\setlength{\parskip}{3mm}
			\raisebox{-2.5mm}{\sffamily \small{\textcolor{#1}{\MakeUppercase{#2}}}}		
			\par		
  			 #3
 	 		\end{minipage}
	}}
		\vspace{2mm}
	\par
}

\newcommand\bloc[3]{				% Boites convertible html sans bordure
     \needspace{2\baselineskip}
     {\sffamily \small{\textcolor{#1}{\MakeUppercase{#2}}}}    
		\par		
  			 #3
		\par
}

\newcommand\CHelp[1]{
     \CBox{Plum}{\faInfoCircle}{À RETENIR}{#1}
}

\newcommand\CUp[1]{
     \CBox{NavyBlue}{\faThumbsOUp}{EN PRATIQUE}{#1}
}

\newcommand\CInfo[1]{
     \CBox{Sepia}{\faArrowCircleRight}{REMARQUE}{#1}
}

\newcommand\CRedac[1]{
     \CBox{PineGreen}{\faEdit}{BIEN R\'EDIGER}{#1}
}

\newcommand\CError[1]{
     \CBox{Red}{\faExclamationTriangle}{ATTENTION}{#1}
}

\newcommand\TitreExo[2]{
\needspace{4\baselineskip}
 {\sffamily\large EXERCICE #1\ (\emph{#2 points})}
\vspace{5mm}
}

\newcommand\img[2]{
          \includegraphics[width=#2\paperwidth]{\imgdir#1}
}

\newcommand\imgsvg[2]{
       \begin{center}   \includegraphics[width=#2\paperwidth]{\imgsvgdir#1} \end{center}
}


\newcommand\Lien[2]{
     \href{#1}{#2 \tiny \faExternalLink}
}
\newcommand\mcLien[2]{
     \href{https~://www.maths-cours.fr/#1}{#2 \tiny \faExternalLink}
}

\newcommand{\euro}{\eurologo{}}

%================================================================================================================================
%
% Macros - Environement
%
%================================================================================================================================

\newenvironment{tex}{ %
}
{%
}

\newenvironment{indente}{ %
	\setlength\parindent{10mm}
}

{
	\setlength\parindent{0mm}
}

\newenvironment{corrige}{%
     \needspace{3\baselineskip}
     \medskip
     \textbf{\textsc{Corrigé}}
     \medskip
}
{
}

\newenvironment{extern}{%
     \begin{center}
     }
     {
     \end{center}
}

\NewEnviron{code}{%
	\par
     \boite{gray}{\texttt{%
     \BODY
     }}
     \par
}

\newenvironment{vbloc}{% boite sans cadre empeche saut de page
     \begin{minipage}[t]{\linewidth}
     }
     {
     \end{minipage}
}
\NewEnviron{h2}{%
    \needspace{3\baselineskip}
    \vspace{0.6cm}
	\noindent \MakeUppercase{\sffamily \large \BODY}
	\vspace{1mm}\textcolor{mcgris}{\hrule}\vspace{0.4cm}
	\par
}{}

\NewEnviron{h3}{%
    \needspace{3\baselineskip}
	\vspace{5mm}
	\textsc{\BODY}
	\par
}

\NewEnviron{margeneg}{ %
\begin{addmargin}[-1cm]{0cm}
\BODY
\end{addmargin}
}

\NewEnviron{html}{%
}

\begin{document}
\meta{url}{/cours/fonctions-exponentielles/}
\meta{pid}{429}
\meta{titre}{Fonctions exponentielles en Terminale ES et L}
\meta{type}{cours}
\begin{h2}1. Fonctions exponentielles de base $q$\end{h2}
\cadre{rouge}{Théorème et définition}{% id="t10"
     Soit $q$ un réel strictement positif.
     \par
     Il existe une unique fonction $f$ définie et dérivable sur $\mathbb{R}$ telle que :
     \begin{itemize}
          \item pour tout entier $n \in  \mathbb{Z}$, $f\left(n\right)=q^{n}$
          \item pour tous réels $x$ et $y$ : $f\left(x+y\right)=f\left(x\right)\times f\left(y\right)  $ \textit{(relation fonctionnelle})
     \end{itemize}
     Cette fonction s'appelle fonction \textbf{exponentielle de base $q$} et on note $f\left(x\right)=q^{x}$
}
\bloc{cyan}{Remarques}{% id="r10"
     \begin{itemize}\item D'après la première propriété et les formules vues au collège, on a notamment : $q^{1}=q$, $q^{0}=1$, $q^{-1}=\frac{1}{q}$
          \item Avec la notation exponentielle, la seconde propriété  (relation fonctionnelle) s'écrit : $q^{x+y}=q^{x}\times q^{y}$.
          \par
          A partir de cette propriété on montre également que pour tout $q > 0$ et tous réels $x$ et $y$ :
          \par
          $q^{x-y}=\frac{q^{x}}{q^{y}} $ (en particulier $q^{-y}=\frac{1}{q^{y}}$)
          \par
          $\left[q^{x}\right] ^{y}=q^{xy}$
          \par
          ce qui généralise les propriétés vues au collège.
          \item La courbe de la fonction $x\mapsto q^{n}$ s'obtient en reliant les points de coordonnées $\left(n, q^{n}\right)$. Pour $n\geqslant 0$ ces points représentent la suite géométrique de premier terme $u_{0}=1$ et de raison $q$.
     \end{itemize}
     \begin{center}
          \begin{extern}%width="460" alt="fonction exponentielle et suite géométrique"
               % -+-+-+ variables modifiables
               \resizebox{8cm}{!}{%
                    \def\xmin{-2.5}
                    \def\xmax{8.5}
                    \def\ymin{-0.9}
                    \def\ymax{9.5}
                    \def\xunit{1}  % unités en cm
                    \def\yunit{1}
                    \psset{xunit=\xunit,yunit=\yunit,algebraic=true}
                    \fontsize{12pt}{12pt}\selectfont
                    \begin{pspicture*}[linewidth=1pt](\xmin,\ymin)(\xmax,\ymax)
                         \psgrid[gridcolor=mcgris,subgriddiv=0](-4,-1)(9,10)
                         \psaxes[linewidth=0.75pt]{->}(0,0)(\xmin,\ymin)(\xmax,\ymax)
                         \rput[tr](-0.2,-0.2){$O$}
                         \multido{\n=0.0+1}{7}{
                              \FPeval{\suite}{1.4^\n}
                              \psdots[linecolor=red,dotsize=4pt](\n,\suite)
                         }
                         \psplot[plotpoints=1000,linewidth=0.8pt,linecolor=blue]{\xmin}{\xmax}{1.4^x}
                    \end{pspicture*}
               }
          \end{extern}
          \end{center}     \begin{center}\textit{Fonction exponentielle de base $q=1,4$}\\
     \textit{(les points correspondent à la suite géométrique $u_{0}=1$ et $q=1.4$)}\end{center}
}
\cadre{vert}{Propriété}{% id="p15"
     Pour tout réel $x$ et tout réel $q > 0$, $q^{x}$ est \textbf{strictement positif}.
}
\cadre{vert}{Propriété}{% id="p20"
     \begin{itemize}
          \item Pour $q > 1$, la fonction $x \mapsto  q^{x}$ est strictement croissante sur $\mathbb{R}$
          \item Pour $0 < q < 1$, la fonction $x \mapsto  q^{x}$ est strictement décroissante sur $\mathbb{R}$
     \end{itemize}
}
% -+-+-+ variables modifiables
\begin{center}
     \begin{extern}%width="460" alt="fonction exponentielle de base supérieure à 1"
          % -+-+-+ variables modifiables
          \resizebox{8cm}{!}{%
               \def\xmin{-2.5}
               \def\xmax{8.5}
               \def\ymin{-0.9}
               \def\ymax{9.5}
               \def\xunit{1}  % unités en cm
               \def\yunit{1}
               \psset{xunit=\xunit,yunit=\yunit,algebraic=true}
               \fontsize{12pt}{12pt}\selectfont
               \begin{pspicture*}[linewidth=1pt](\xmin,\ymin)(\xmax,\ymax)
                    \psgrid[gridcolor=mcgris,subgriddiv=0](-4,-1)(9,10)
                    \psaxes[linewidth=0.75pt]{->}(0,0)(\xmin,\ymin)(\xmax,\ymax)
                    \rput[tr](-0.2,-0.2){$O$}
                    \psplot[plotpoints=1000,linewidth=0.8pt,linecolor=blue]{\xmin}{\xmax}{1.4^x}
               \end{pspicture*}
          }
     \end{extern}
\end{center}
\begin{center}\textit{Fonction exponentielle de base $q > 1$}\end{center}
\begin{center}
     \begin{extern}%width="460" alt="fonction exponentielle de base inférieure à 1"
          % -+-+-+ variables modifiables
          \resizebox{8cm}{!}{%
               \def\xmin{-5.5}
               \def\xmax{6.5}
               \def\ymin{-0.9}
               \def\ymax{8.5}
               \def\xunit{1}  % unités en cm
               \def\yunit{1}
               \psset{xunit=\xunit,yunit=\yunit,algebraic=true}
               \fontsize{12pt}{12pt}\selectfont
               \begin{pspicture*}[linewidth=1pt](\xmin,\ymin)(\xmax,\ymax)
                    \psgrid[gridcolor=mcgris,subgriddiv=0](-6,-1)(7,9)
                    \psaxes[linewidth=0.75pt]{->}(0,0)(\xmin,\ymin)(\xmax,\ymax)
                    \rput[tr](-0.2,-0.2){$O$}
                    \psplot[plotpoints=1000,linewidth=0.8pt,linecolor=blue]{\xmin}{\xmax}{0.7^x}
               \end{pspicture*}
          }
     \end{extern}
\end{center}
\begin{center}\textit{Fonction exponentielle de base $0 < q < 1$}\end{center}
\bloc{cyan}{Remarque}{% id="r20"
     Pour $q=1$, la fonction $x \mapsto  q^{x}$ est constante et égale à $1$. Sa courbe représentative est une droite parallèle à l'axe des abscisses.
}
\begin{h2}2. Fonction exponentielle (de base $e$)\end{h2}
\cadre{rouge}{Théorème et Définition}{% id="t50"
     Il existe une valeur de $q$ pour laquelle la fonction $f : x\mapsto q^{x}$ vérifie $f^{\prime}\left(0\right)=1$.
     \par
     Cette valeur est notée $e$.
     \par
     La fonction  $x \mapsto  e^{x}$ (parfois notée $\text{exp}$) est appelée \textbf{fonction exponentielle}.
}
\bloc{cyan}{Remarque}{% id="r50"
     Le nombre $e$ est approximativement égal à $2,71828$ (on l'obtient à la calculatrice en faisant $e^{1}$ ou $\text{exp}\left(1\right)$.
}
\cadre{vert}{Propriété}{% id="p60"
     La fonction exponentielle est \textbf{strictement positive} et \textbf{strictement croissante} et sur $\mathbb{R}$.
}
\bloc{cyan}{Démonstration}{% id="m60"
     Cela résulte du fait que $e > 1$ et des résultats de la section précédente.
}
\begin{center}
     \begin{extern}%width="400" alt="fonction exponentielle de base e"
          % -+-+-+ variables modifiables
          \resizebox{8cm}{!}{%
               \def\xmin{-4.5}
               \def\xmax{4.5}
               \def\ymin{-0.9}
               \def\ymax{8.5}
               \def\xunit{1}  % unités en cm
               \def\yunit{1}
               \psset{xunit=\xunit,yunit=\yunit,algebraic=true}
               \fontsize{12pt}{12pt}\selectfont
               \begin{pspicture*}[linewidth=1pt](\xmin,\ymin)(\xmax,\ymax)
                    \psgrid[gridcolor=mcgris,subgriddiv=1](-5,-1)(5,9)
                    \psaxes[linewidth=0.75pt]{->}(0,0)(-5,-1)(5,9)
                    \rput[tr](-0.2,-0.2){$O$}
                    \psplot[plotpoints=1000,linewidth=0.8pt,linecolor=blue]{\xmin}{\xmax}{EXP(x)}
                    \rput[r](-0.2,2.718){$\color{rouge} \text{e}$}
                    \psline[linewidth=0.8pt,linecolor=rouge](1,0)(1,2.71828)(0,2.71828)
               \end{pspicture*}
          }
     \end{extern}
\end{center}
\begin{center}
     \textit{Fonction exponentielle de base} $\text{e}$
\end{center}
\bloc{cyan}{Remarque}{% id="r60"
     La stricte croissance de la fonction exponentielle entraîne que :
     \par
     $x < y \Leftrightarrow e^{x} < e^{y}$
     \par
     Cette propriété est fréquemment utilisée dans les exercices (inéquations notamment).
}
\cadre{rouge}{Théorème (dérivée de la fonction exponentielle}{% id="t70"
     La fonction exponentielle est égale à sa dérivée.
     \par
     Autrement dit, pour tout $x \in  \mathbb{R}$ : $\text{exp}^{\prime}\left(x\right)=\text{exp}\left(x\right)$
}
\bloc{cyan}{Démonstration}{% id="d70"
     Le taux d'accroissement de la fonction exponentielle sur l'intervalle $\left[x ; x+h\right]$ est égal à :
     \par
     $T=\frac{e^{x+h}-e^{x}}{h}=\frac{e^{x}\times e^{h}-e^{x}}{h}=e^{x}\times \frac{e^{h}-1}{h}$
     \par
     Par définition du nombre dérivé, le quotient $\frac{e^{h}-1}{h}$ tend vers $\text{exp}^{\prime}\left(0\right)=1$ quand $h$ tend vers $0$, donc $T$ tend vers $e^{x}$ quand $h$ tend vers 0.
}
\cadre{vert}{Propriété}{% id="p80"
     Soit $u$ une fonction dérivable sur un intervalle $I$.
     \par
     Alors la fonction $ f :  x\mapsto e^{u\left(x\right)}$ est dérivable sur $I$ et :
     \begin{center}$f^{\prime}\left(x\right)=u^{\prime}\left(x\right) e^{u\left(x\right)}$\end{center}
}
\bloc{orange}{Exemple}{% id="e80"
     Soit $f$ définie sur $\mathbb{R}$ par $f\left(x\right)=e^{-x}$
     \par
     $f$ est dérivable sur $\mathbb{R}$ et $f^{\prime}\left(x\right)=-e^{-x}$ (on pose $u\left(x\right)=-x$ donc $u^{\prime}\left(x\right)=-1$)
}

\end{document}
µ
\documentclass[a4paper]{article}

%================================================================================================================================
%
% Packages
%
%================================================================================================================================

\usepackage[T1]{fontenc} 	% pour caractères accentués
\usepackage[utf8]{inputenc}  % encodage utf8
\usepackage[french]{babel}	% langue : français
\usepackage{fourier}			% caractères plus lisibles
\usepackage[dvipsnames]{xcolor} % couleurs
\usepackage{fancyhdr}		% réglage header footer
\usepackage{needspace}		% empêcher sauts de page mal placés
\usepackage{graphicx}		% pour inclure des graphiques
\usepackage{enumitem,cprotect}		% personnalise les listes d'items (nécessaire pour ol, al ...)
\usepackage{hyperref}		% Liens hypertexte
\usepackage{pstricks,pst-all,pst-node,pstricks-add,pst-math,pst-plot,pst-tree,pst-eucl} % pstricks
\usepackage[a4paper,includeheadfoot,top=2cm,left=3cm, bottom=2cm,right=3cm]{geometry} % marges etc.
\usepackage{comment}			% commentaires multilignes
\usepackage{amsmath,environ} % maths (matrices, etc.)
\usepackage{amssymb,makeidx}
\usepackage{bm}				% bold maths
\usepackage{tabularx}		% tableaux
\usepackage{colortbl}		% tableaux en couleur
\usepackage{fontawesome}		% Fontawesome
\usepackage{environ}			% environment with command
\usepackage{fp}				% calculs pour ps-tricks
\usepackage{multido}			% pour ps tricks
\usepackage[np]{numprint}	% formattage nombre
\usepackage{tikz,tkz-tab} 			% package principal TikZ
\usepackage{pgfplots}   % axes
\usepackage{mathrsfs}    % cursives
\usepackage{calc}			% calcul taille boites
\usepackage[scaled=0.875]{helvet} % font sans serif
\usepackage{svg} % svg
\usepackage{scrextend} % local margin
\usepackage{scratch} %scratch
\usepackage{multicol} % colonnes
%\usepackage{infix-RPN,pst-func} % formule en notation polanaise inversée
\usepackage{listings}

%================================================================================================================================
%
% Réglages de base
%
%================================================================================================================================

\lstset{
language=Python,   % R code
literate=
{á}{{\'a}}1
{à}{{\`a}}1
{ã}{{\~a}}1
{é}{{\'e}}1
{è}{{\`e}}1
{ê}{{\^e}}1
{í}{{\'i}}1
{ó}{{\'o}}1
{õ}{{\~o}}1
{ú}{{\'u}}1
{ü}{{\"u}}1
{ç}{{\c{c}}}1
{~}{{ }}1
}


\definecolor{codegreen}{rgb}{0,0.6,0}
\definecolor{codegray}{rgb}{0.5,0.5,0.5}
\definecolor{codepurple}{rgb}{0.58,0,0.82}
\definecolor{backcolour}{rgb}{0.95,0.95,0.92}

\lstdefinestyle{mystyle}{
    backgroundcolor=\color{backcolour},   
    commentstyle=\color{codegreen},
    keywordstyle=\color{magenta},
    numberstyle=\tiny\color{codegray},
    stringstyle=\color{codepurple},
    basicstyle=\ttfamily\footnotesize,
    breakatwhitespace=false,         
    breaklines=true,                 
    captionpos=b,                    
    keepspaces=true,                 
    numbers=left,                    
xleftmargin=2em,
framexleftmargin=2em,            
    showspaces=false,                
    showstringspaces=false,
    showtabs=false,                  
    tabsize=2,
    upquote=true
}

\lstset{style=mystyle}


\lstset{style=mystyle}
\newcommand{\imgdir}{C:/laragon/www/newmc/assets/imgsvg/}
\newcommand{\imgsvgdir}{C:/laragon/www/newmc/assets/imgsvg/}

\definecolor{mcgris}{RGB}{220, 220, 220}% ancien~; pour compatibilité
\definecolor{mcbleu}{RGB}{52, 152, 219}
\definecolor{mcvert}{RGB}{125, 194, 70}
\definecolor{mcmauve}{RGB}{154, 0, 215}
\definecolor{mcorange}{RGB}{255, 96, 0}
\definecolor{mcturquoise}{RGB}{0, 153, 153}
\definecolor{mcrouge}{RGB}{255, 0, 0}
\definecolor{mclightvert}{RGB}{205, 234, 190}

\definecolor{gris}{RGB}{220, 220, 220}
\definecolor{bleu}{RGB}{52, 152, 219}
\definecolor{vert}{RGB}{125, 194, 70}
\definecolor{mauve}{RGB}{154, 0, 215}
\definecolor{orange}{RGB}{255, 96, 0}
\definecolor{turquoise}{RGB}{0, 153, 153}
\definecolor{rouge}{RGB}{255, 0, 0}
\definecolor{lightvert}{RGB}{205, 234, 190}
\setitemize[0]{label=\color{lightvert}  $\bullet$}

\pagestyle{fancy}
\renewcommand{\headrulewidth}{0.2pt}
\fancyhead[L]{maths-cours.fr}
\fancyhead[R]{\thepage}
\renewcommand{\footrulewidth}{0.2pt}
\fancyfoot[C]{}

\newcolumntype{C}{>{\centering\arraybackslash}X}
\newcolumntype{s}{>{\hsize=.35\hsize\arraybackslash}X}

\setlength{\parindent}{0pt}		 
\setlength{\parskip}{3mm}
\setlength{\headheight}{1cm}

\def\ebook{ebook}
\def\book{book}
\def\web{web}
\def\type{web}

\newcommand{\vect}[1]{\overrightarrow{\,\mathstrut#1\,}}

\def\Oij{$\left(\text{O}~;~\vect{\imath},~\vect{\jmath}\right)$}
\def\Oijk{$\left(\text{O}~;~\vect{\imath},~\vect{\jmath},~\vect{k}\right)$}
\def\Ouv{$\left(\text{O}~;~\vect{u},~\vect{v}\right)$}

\hypersetup{breaklinks=true, colorlinks = true, linkcolor = OliveGreen, urlcolor = OliveGreen, citecolor = OliveGreen, pdfauthor={Didier BONNEL - https://www.maths-cours.fr} } % supprime les bordures autour des liens

\renewcommand{\arg}[0]{\text{arg}}

\everymath{\displaystyle}

%================================================================================================================================
%
% Macros - Commandes
%
%================================================================================================================================

\newcommand\meta[2]{    			% Utilisé pour créer le post HTML.
	\def\titre{titre}
	\def\url{url}
	\def\arg{#1}
	\ifx\titre\arg
		\newcommand\maintitle{#2}
		\fancyhead[L]{#2}
		{\Large\sffamily \MakeUppercase{#2}}
		\vspace{1mm}\textcolor{mcvert}{\hrule}
	\fi 
	\ifx\url\arg
		\fancyfoot[L]{\href{https://www.maths-cours.fr#2}{\black \footnotesize{https://www.maths-cours.fr#2}}}
	\fi 
}


\newcommand\TitreC[1]{    		% Titre centré
     \needspace{3\baselineskip}
     \begin{center}\textbf{#1}\end{center}
}

\newcommand\newpar{    		% paragraphe
     \par
}

\newcommand\nosp {    		% commande vide (pas d'espace)
}
\newcommand{\id}[1]{} %ignore

\newcommand\boite[2]{				% Boite simple sans titre
	\vspace{5mm}
	\setlength{\fboxrule}{0.2mm}
	\setlength{\fboxsep}{5mm}	
	\fcolorbox{#1}{#1!3}{\makebox[\linewidth-2\fboxrule-2\fboxsep]{
  		\begin{minipage}[t]{\linewidth-2\fboxrule-4\fboxsep}\setlength{\parskip}{3mm}
  			 #2
  		\end{minipage}
	}}
	\vspace{5mm}
}

\newcommand\CBox[4]{				% Boites
	\vspace{5mm}
	\setlength{\fboxrule}{0.2mm}
	\setlength{\fboxsep}{5mm}
	
	\fcolorbox{#1}{#1!3}{\makebox[\linewidth-2\fboxrule-2\fboxsep]{
		\begin{minipage}[t]{1cm}\setlength{\parskip}{3mm}
	  		\textcolor{#1}{\LARGE{#2}}    
 	 	\end{minipage}  
  		\begin{minipage}[t]{\linewidth-2\fboxrule-4\fboxsep}\setlength{\parskip}{3mm}
			\raisebox{1.2mm}{\normalsize\sffamily{\textcolor{#1}{#3}}}						
  			 #4
  		\end{minipage}
	}}
	\vspace{5mm}
}

\newcommand\cadre[3]{				% Boites convertible html
	\par
	\vspace{2mm}
	\setlength{\fboxrule}{0.1mm}
	\setlength{\fboxsep}{5mm}
	\fcolorbox{#1}{white}{\makebox[\linewidth-2\fboxrule-2\fboxsep]{
  		\begin{minipage}[t]{\linewidth-2\fboxrule-4\fboxsep}\setlength{\parskip}{3mm}
			\raisebox{-2.5mm}{\sffamily \small{\textcolor{#1}{\MakeUppercase{#2}}}}		
			\par		
  			 #3
 	 		\end{minipage}
	}}
		\vspace{2mm}
	\par
}

\newcommand\bloc[3]{				% Boites convertible html sans bordure
     \needspace{2\baselineskip}
     {\sffamily \small{\textcolor{#1}{\MakeUppercase{#2}}}}    
		\par		
  			 #3
		\par
}

\newcommand\CHelp[1]{
     \CBox{Plum}{\faInfoCircle}{À RETENIR}{#1}
}

\newcommand\CUp[1]{
     \CBox{NavyBlue}{\faThumbsOUp}{EN PRATIQUE}{#1}
}

\newcommand\CInfo[1]{
     \CBox{Sepia}{\faArrowCircleRight}{REMARQUE}{#1}
}

\newcommand\CRedac[1]{
     \CBox{PineGreen}{\faEdit}{BIEN R\'EDIGER}{#1}
}

\newcommand\CError[1]{
     \CBox{Red}{\faExclamationTriangle}{ATTENTION}{#1}
}

\newcommand\TitreExo[2]{
\needspace{4\baselineskip}
 {\sffamily\large EXERCICE #1\ (\emph{#2 points})}
\vspace{5mm}
}

\newcommand\img[2]{
          \includegraphics[width=#2\paperwidth]{\imgdir#1}
}

\newcommand\imgsvg[2]{
       \begin{center}   \includegraphics[width=#2\paperwidth]{\imgsvgdir#1} \end{center}
}


\newcommand\Lien[2]{
     \href{#1}{#2 \tiny \faExternalLink}
}
\newcommand\mcLien[2]{
     \href{https~://www.maths-cours.fr/#1}{#2 \tiny \faExternalLink}
}

\newcommand{\euro}{\eurologo{}}

%================================================================================================================================
%
% Macros - Environement
%
%================================================================================================================================

\newenvironment{tex}{ %
}
{%
}

\newenvironment{indente}{ %
	\setlength\parindent{10mm}
}

{
	\setlength\parindent{0mm}
}

\newenvironment{corrige}{%
     \needspace{3\baselineskip}
     \medskip
     \textbf{\textsc{Corrigé}}
     \medskip
}
{
}

\newenvironment{extern}{%
     \begin{center}
     }
     {
     \end{center}
}

\NewEnviron{code}{%
	\par
     \boite{gray}{\texttt{%
     \BODY
     }}
     \par
}

\newenvironment{vbloc}{% boite sans cadre empeche saut de page
     \begin{minipage}[t]{\linewidth}
     }
     {
     \end{minipage}
}
\NewEnviron{h2}{%
    \needspace{3\baselineskip}
    \vspace{0.6cm}
	\noindent \MakeUppercase{\sffamily \large \BODY}
	\vspace{1mm}\textcolor{mcgris}{\hrule}\vspace{0.4cm}
	\par
}{}

\NewEnviron{h3}{%
    \needspace{3\baselineskip}
	\vspace{5mm}
	\textsc{\BODY}
	\par
}

\NewEnviron{margeneg}{ %
\begin{addmargin}[-1cm]{0cm}
\BODY
\end{addmargin}
}

\NewEnviron{html}{%
}

\begin{document}
\meta{url}{/cours/fonction-logarithme-neperien/}
\meta{pid}{436}
\meta{titre}{Fonction logarithme népérien en Terminale ES/L}
\meta{type}{cours}
\begin{h2}1. Définition de la fonction logarithme népérien\end{h2}
\cadre{bleu}{Théorème et définition}{%id="t10"
     Pour tout réel $x>0$, l'équation $e^{y}=x$, d'inconnue $y$, admet une \textbf{unique} solution.
     \par
     La fonction \textbf{logarithme népérien}, notée $\ln$, est la fonction définie sur $\left]0;+\infty \right[$ qui à $x > 0$, associe le réel $y$ solution de l'équation $e^{y}=x$.
}
\bloc{cyan}{Remarques}{%id="r10"
     \begin{itemize}
          \item Pour $x\leqslant 0$, par contre, l'équation $e^{y}=x$ n'a \textbf{pas de solution}.
     \end{itemize}
}
\cadre{vert}{Propriétés}{%id="p20"
     \begin{itemize}
          \item Pour tout réel $x > 0$ et tout $y \in \mathbb{R}$ : $ e^{y}=x  \Leftrightarrow y=\ln\left(x\right)$.
          \item Pour tout réel $x > 0$ : $e^{\ln\left(x\right)}=x$.
          \item Pour tout réel $x$ : $\ln\left(e^{x}\right)=x$.
     \end{itemize}
}
\bloc{cyan}{Remarques}{%id="r0"
     \begin{itemize}
          \item Ces propriétés se déduisent immédiatement de la définition.
          \item On dit que les fonctions «logarithme népérien» et «exponentielle» sont \textit{réciproques}.
          \item On en déduit immédiatement : $\ln\left(1\right)=0$ et $\ln\left(e\right)=1$.
     \end{itemize}
}
\begin{h2}2. Etude de la fonction logarithme népérien\end{h2}
\cadre{rouge}{Théorème}{%id="t30"
     La fonction logarithme népérien est dérivable sur $\left]0 ;+\infty \right[$ et sa dérivée est définie par :
     \begin{center}$\ln^{\prime}\left(x\right)=\frac{1}{x}.$\end{center}
}
\cadre{vert}{Propriété}{%id="p40"
     La fonction logarithme népérien est \textbf{strictement croissante} sur $\left]0;+\infty \right[$.
}
\bloc{cyan}{Démonstration}{%id="r40"
     Sa dérivée $\ln^{\prime}\left(x\right)=\frac{1}{x}$ est strictement positive sur $\left]0;+\infty \right[$.
}
\bloc{cyan}{Remarques}{%id="r40"
     \begin{itemize}
          \item Ces résultats permettent de tracer le tableau de variation et la courbe représentative de la fonction logarithme népérien :
     \end{itemize}
}
\begin{center}
     \begin{extern}%width="600" alt="Tableau de variation de la fonction logarithme népérien"
          \tikzset{double style/.style = {double,double distance=2pt}}
          \begin{tikzpicture}
               \tkzTabInit[lgt=3,espcl=10] {$x$ /1, $f'(x)=\dfrac{1}{x}$ /1.5,%
               $f(x)=\ln x$/2} {$0$ , $+\infty$}%
               \tkzTabLine{d,+,}%
               \tkzTabVar{ D- / $-\infty$, + / $+\infty$ }
               \tkzTabVal[draw]{1}{2}{0.33}{1}{0}
               \tkzTabVal[draw]{1}{2}{0.66}{e}{1}
          \end{tikzpicture}
     \end{extern}
\end{center}
\begin{center}
     \textit{Tableau de variation de la fonction logarithme népérien }
\end{center}
\par
\begin{center}
     \begin{extern} %width="400" alt="Représentation graphique de la fonction logarithme népérien"
          \resizebox{8cm}{!}{%
               % -+-+-+ variables modifiables
               \def\fonction{x ln }
               \def\xmin{-0.5}
               \def\xmax{5.5}
               \def\ymin{-3.5}
               \def\ymax{3}
               \def\xunit{2}  % unités en cm
               \def\yunit{2}
               \psset{xunit=\xunit,yunit=\yunit}
               \fontsize{15pt}{15pt}\selectfont
               \begin{pspicture*}[linewidth=1pt](\xmin,\ymin)(\xmax,\ymax)
                    %      \psgrid[gridcolor=mcgris, subgriddiv=5, gridlabels=0pt](\xmin,\ymin)(\xmax,\ymax)
                    \psaxes[linewidth=0.75pt]{->}(0,0)(\xmin,\ymin)(\xmax,\ymax)
                    \psplot[plotpoints=2000,linecolor=blue]{0.01}{\xmax}{\fonction}
                    \rput[tr](-0.1,-0.1){$O$}
                    \rput[tl](4.8,2){$\color{blue} \mathcal{C}_{\ln}$}
                    \psline(0,1)(2.7183,1)(2.7183,0)
                    \rput[t](2.7183,-0.25){$\text{e}$}
               \end{pspicture*}
          }
     \end{extern}
\end{center}
\begin{center}
     \textit{Représentation graphique de la fonction logarithme népérien }
\end{center}
\cadre{vert}{Propriété}{%id="p45"
     Soit $u$ une fonction dérivable et \textbf{strictement positive} sur un intervalle $I$.
     \par
     Alors la fonction $ f : x\mapsto \ln\left(u\left(x\right)\right)$ est dérivable sur $I$ et :
     \begin{center}$f^{\prime}=\frac{u^{\prime}}{u}$.\end{center}
}
\bloc{orange}{Exemple}{%id="e45"
     Soit $f$ définie sur $\mathbb{R}$ par $f\left(x\right)=\ln\left(x^{2}+1\right)$.
     \par
     $f$ est dérivable sur $\mathbb{R}$ et $f^{\prime}\left(x\right)=\frac{2x}{x^{2}+1}$.
}
\cadre{rouge}{Théorème}{%id="t70"
     Si $a$ et $b$ sont 2 réels strictement positifs :
     \begin{itemize}
          \item $\ln a=\ln b$ si et seulement si $a=b$.
          \item $\ln a < \ln b$ si et seulement si $a < b$.
     \end{itemize}
}
\bloc{cyan}{Remarques}{%id="r70"
     \begin{itemize}
          \item Le théorème précédent résulte de la stricte croissance de la fonction logarithme népérien.
          \item En particulier, comme $\ln\left(1\right)=0$ : $\ln x < 0 \Leftrightarrow x < 1$. N'oubliez donc pas que \textbf{$\ln\left(x\right)$ peut être négatif} (si $0 < x < 1$); c'est une cause d'erreurs fréquente dans les exercices notamment avec des inéquations !
     \end{itemize}
}
\begin{h2}3. Propriétés algébriques de la fonction logarithme népérien\end{h2}
\cadre{rouge}{Théorème}{%id="80"
     Si $a$ et $b$ sont 2 réels strictement positifs et si $n \in \mathbb{Z}$ :
     \begin{itemize}
          \item $\ln\left(ab\right)=\ln a+\ln b$.
          \item $\ln\left(\frac{1}{a}\right)=-\ln a$.
          \item $\ln\left(\frac{a}{b}\right)=\ln a-\ln b$.
          \item $\ln\left(a^{n}\right)=n \ln a $.
          \item $\ln\left(\sqrt{a}\right)=\frac{1}{2} \ln a $.
     \end{itemize}
}
\bloc{orange}{Exemples}{%id="e80"
     \begin{itemize}
          \item $\ln\left(4\right)=\ln\left(2^{2}\right)=2\ln\left(2\right)$.
          \item Pour $x > 1$ : $\ln\left(\frac{x+1}{x-1}\right)= \ln\left(x+1\right)-\ln\left(x-1\right)$.
          \par
          Cette égalité peut être intéressante (pour calculer la dérivée par exemple) mais il faut que $x > 1$.
          \par
          Si $x < -1$, l'expression $\ln\left(\frac{x+1}{x-1}\right)$ est définie mais pas $\ln\left(x+1\right)-\ln\left(x-1\right)$.
     \end{itemize}
}

\end{document}
µ
\documentclass[a4paper]{article}

%================================================================================================================================
%
% Packages
%
%================================================================================================================================

\usepackage[T1]{fontenc} 	% pour caractères accentués
\usepackage[utf8]{inputenc}  % encodage utf8
\usepackage[french]{babel}	% langue : français
\usepackage{fourier}			% caractères plus lisibles
\usepackage[dvipsnames]{xcolor} % couleurs
\usepackage{fancyhdr}		% réglage header footer
\usepackage{needspace}		% empêcher sauts de page mal placés
\usepackage{graphicx}		% pour inclure des graphiques
\usepackage{enumitem,cprotect}		% personnalise les listes d'items (nécessaire pour ol, al ...)
\usepackage{hyperref}		% Liens hypertexte
\usepackage{pstricks,pst-all,pst-node,pstricks-add,pst-math,pst-plot,pst-tree,pst-eucl} % pstricks
\usepackage[a4paper,includeheadfoot,top=2cm,left=3cm, bottom=2cm,right=3cm]{geometry} % marges etc.
\usepackage{comment}			% commentaires multilignes
\usepackage{amsmath,environ} % maths (matrices, etc.)
\usepackage{amssymb,makeidx}
\usepackage{bm}				% bold maths
\usepackage{tabularx}		% tableaux
\usepackage{colortbl}		% tableaux en couleur
\usepackage{fontawesome}		% Fontawesome
\usepackage{environ}			% environment with command
\usepackage{fp}				% calculs pour ps-tricks
\usepackage{multido}			% pour ps tricks
\usepackage[np]{numprint}	% formattage nombre
\usepackage{tikz,tkz-tab} 			% package principal TikZ
\usepackage{pgfplots}   % axes
\usepackage{mathrsfs}    % cursives
\usepackage{calc}			% calcul taille boites
\usepackage[scaled=0.875]{helvet} % font sans serif
\usepackage{svg} % svg
\usepackage{scrextend} % local margin
\usepackage{scratch} %scratch
\usepackage{multicol} % colonnes
%\usepackage{infix-RPN,pst-func} % formule en notation polanaise inversée
\usepackage{listings}

%================================================================================================================================
%
% Réglages de base
%
%================================================================================================================================

\lstset{
language=Python,   % R code
literate=
{á}{{\'a}}1
{à}{{\`a}}1
{ã}{{\~a}}1
{é}{{\'e}}1
{è}{{\`e}}1
{ê}{{\^e}}1
{í}{{\'i}}1
{ó}{{\'o}}1
{õ}{{\~o}}1
{ú}{{\'u}}1
{ü}{{\"u}}1
{ç}{{\c{c}}}1
{~}{{ }}1
}


\definecolor{codegreen}{rgb}{0,0.6,0}
\definecolor{codegray}{rgb}{0.5,0.5,0.5}
\definecolor{codepurple}{rgb}{0.58,0,0.82}
\definecolor{backcolour}{rgb}{0.95,0.95,0.92}

\lstdefinestyle{mystyle}{
    backgroundcolor=\color{backcolour},   
    commentstyle=\color{codegreen},
    keywordstyle=\color{magenta},
    numberstyle=\tiny\color{codegray},
    stringstyle=\color{codepurple},
    basicstyle=\ttfamily\footnotesize,
    breakatwhitespace=false,         
    breaklines=true,                 
    captionpos=b,                    
    keepspaces=true,                 
    numbers=left,                    
xleftmargin=2em,
framexleftmargin=2em,            
    showspaces=false,                
    showstringspaces=false,
    showtabs=false,                  
    tabsize=2,
    upquote=true
}

\lstset{style=mystyle}


\lstset{style=mystyle}
\newcommand{\imgdir}{C:/laragon/www/newmc/assets/imgsvg/}
\newcommand{\imgsvgdir}{C:/laragon/www/newmc/assets/imgsvg/}

\definecolor{mcgris}{RGB}{220, 220, 220}% ancien~; pour compatibilité
\definecolor{mcbleu}{RGB}{52, 152, 219}
\definecolor{mcvert}{RGB}{125, 194, 70}
\definecolor{mcmauve}{RGB}{154, 0, 215}
\definecolor{mcorange}{RGB}{255, 96, 0}
\definecolor{mcturquoise}{RGB}{0, 153, 153}
\definecolor{mcrouge}{RGB}{255, 0, 0}
\definecolor{mclightvert}{RGB}{205, 234, 190}

\definecolor{gris}{RGB}{220, 220, 220}
\definecolor{bleu}{RGB}{52, 152, 219}
\definecolor{vert}{RGB}{125, 194, 70}
\definecolor{mauve}{RGB}{154, 0, 215}
\definecolor{orange}{RGB}{255, 96, 0}
\definecolor{turquoise}{RGB}{0, 153, 153}
\definecolor{rouge}{RGB}{255, 0, 0}
\definecolor{lightvert}{RGB}{205, 234, 190}
\setitemize[0]{label=\color{lightvert}  $\bullet$}

\pagestyle{fancy}
\renewcommand{\headrulewidth}{0.2pt}
\fancyhead[L]{maths-cours.fr}
\fancyhead[R]{\thepage}
\renewcommand{\footrulewidth}{0.2pt}
\fancyfoot[C]{}

\newcolumntype{C}{>{\centering\arraybackslash}X}
\newcolumntype{s}{>{\hsize=.35\hsize\arraybackslash}X}

\setlength{\parindent}{0pt}		 
\setlength{\parskip}{3mm}
\setlength{\headheight}{1cm}

\def\ebook{ebook}
\def\book{book}
\def\web{web}
\def\type{web}

\newcommand{\vect}[1]{\overrightarrow{\,\mathstrut#1\,}}

\def\Oij{$\left(\text{O}~;~\vect{\imath},~\vect{\jmath}\right)$}
\def\Oijk{$\left(\text{O}~;~\vect{\imath},~\vect{\jmath},~\vect{k}\right)$}
\def\Ouv{$\left(\text{O}~;~\vect{u},~\vect{v}\right)$}

\hypersetup{breaklinks=true, colorlinks = true, linkcolor = OliveGreen, urlcolor = OliveGreen, citecolor = OliveGreen, pdfauthor={Didier BONNEL - https://www.maths-cours.fr} } % supprime les bordures autour des liens

\renewcommand{\arg}[0]{\text{arg}}

\everymath{\displaystyle}

%================================================================================================================================
%
% Macros - Commandes
%
%================================================================================================================================

\newcommand\meta[2]{    			% Utilisé pour créer le post HTML.
	\def\titre{titre}
	\def\url{url}
	\def\arg{#1}
	\ifx\titre\arg
		\newcommand\maintitle{#2}
		\fancyhead[L]{#2}
		{\Large\sffamily \MakeUppercase{#2}}
		\vspace{1mm}\textcolor{mcvert}{\hrule}
	\fi 
	\ifx\url\arg
		\fancyfoot[L]{\href{https://www.maths-cours.fr#2}{\black \footnotesize{https://www.maths-cours.fr#2}}}
	\fi 
}


\newcommand\TitreC[1]{    		% Titre centré
     \needspace{3\baselineskip}
     \begin{center}\textbf{#1}\end{center}
}

\newcommand\newpar{    		% paragraphe
     \par
}

\newcommand\nosp {    		% commande vide (pas d'espace)
}
\newcommand{\id}[1]{} %ignore

\newcommand\boite[2]{				% Boite simple sans titre
	\vspace{5mm}
	\setlength{\fboxrule}{0.2mm}
	\setlength{\fboxsep}{5mm}	
	\fcolorbox{#1}{#1!3}{\makebox[\linewidth-2\fboxrule-2\fboxsep]{
  		\begin{minipage}[t]{\linewidth-2\fboxrule-4\fboxsep}\setlength{\parskip}{3mm}
  			 #2
  		\end{minipage}
	}}
	\vspace{5mm}
}

\newcommand\CBox[4]{				% Boites
	\vspace{5mm}
	\setlength{\fboxrule}{0.2mm}
	\setlength{\fboxsep}{5mm}
	
	\fcolorbox{#1}{#1!3}{\makebox[\linewidth-2\fboxrule-2\fboxsep]{
		\begin{minipage}[t]{1cm}\setlength{\parskip}{3mm}
	  		\textcolor{#1}{\LARGE{#2}}    
 	 	\end{minipage}  
  		\begin{minipage}[t]{\linewidth-2\fboxrule-4\fboxsep}\setlength{\parskip}{3mm}
			\raisebox{1.2mm}{\normalsize\sffamily{\textcolor{#1}{#3}}}						
  			 #4
  		\end{minipage}
	}}
	\vspace{5mm}
}

\newcommand\cadre[3]{				% Boites convertible html
	\par
	\vspace{2mm}
	\setlength{\fboxrule}{0.1mm}
	\setlength{\fboxsep}{5mm}
	\fcolorbox{#1}{white}{\makebox[\linewidth-2\fboxrule-2\fboxsep]{
  		\begin{minipage}[t]{\linewidth-2\fboxrule-4\fboxsep}\setlength{\parskip}{3mm}
			\raisebox{-2.5mm}{\sffamily \small{\textcolor{#1}{\MakeUppercase{#2}}}}		
			\par		
  			 #3
 	 		\end{minipage}
	}}
		\vspace{2mm}
	\par
}

\newcommand\bloc[3]{				% Boites convertible html sans bordure
     \needspace{2\baselineskip}
     {\sffamily \small{\textcolor{#1}{\MakeUppercase{#2}}}}    
		\par		
  			 #3
		\par
}

\newcommand\CHelp[1]{
     \CBox{Plum}{\faInfoCircle}{À RETENIR}{#1}
}

\newcommand\CUp[1]{
     \CBox{NavyBlue}{\faThumbsOUp}{EN PRATIQUE}{#1}
}

\newcommand\CInfo[1]{
     \CBox{Sepia}{\faArrowCircleRight}{REMARQUE}{#1}
}

\newcommand\CRedac[1]{
     \CBox{PineGreen}{\faEdit}{BIEN R\'EDIGER}{#1}
}

\newcommand\CError[1]{
     \CBox{Red}{\faExclamationTriangle}{ATTENTION}{#1}
}

\newcommand\TitreExo[2]{
\needspace{4\baselineskip}
 {\sffamily\large EXERCICE #1\ (\emph{#2 points})}
\vspace{5mm}
}

\newcommand\img[2]{
          \includegraphics[width=#2\paperwidth]{\imgdir#1}
}

\newcommand\imgsvg[2]{
       \begin{center}   \includegraphics[width=#2\paperwidth]{\imgsvgdir#1} \end{center}
}


\newcommand\Lien[2]{
     \href{#1}{#2 \tiny \faExternalLink}
}
\newcommand\mcLien[2]{
     \href{https~://www.maths-cours.fr/#1}{#2 \tiny \faExternalLink}
}

\newcommand{\euro}{\eurologo{}}

%================================================================================================================================
%
% Macros - Environement
%
%================================================================================================================================

\newenvironment{tex}{ %
}
{%
}

\newenvironment{indente}{ %
	\setlength\parindent{10mm}
}

{
	\setlength\parindent{0mm}
}

\newenvironment{corrige}{%
     \needspace{3\baselineskip}
     \medskip
     \textbf{\textsc{Corrigé}}
     \medskip
}
{
}

\newenvironment{extern}{%
     \begin{center}
     }
     {
     \end{center}
}

\NewEnviron{code}{%
	\par
     \boite{gray}{\texttt{%
     \BODY
     }}
     \par
}

\newenvironment{vbloc}{% boite sans cadre empeche saut de page
     \begin{minipage}[t]{\linewidth}
     }
     {
     \end{minipage}
}
\NewEnviron{h2}{%
    \needspace{3\baselineskip}
    \vspace{0.6cm}
	\noindent \MakeUppercase{\sffamily \large \BODY}
	\vspace{1mm}\textcolor{mcgris}{\hrule}\vspace{0.4cm}
	\par
}{}

\NewEnviron{h3}{%
    \needspace{3\baselineskip}
	\vspace{5mm}
	\textsc{\BODY}
	\par
}

\NewEnviron{margeneg}{ %
\begin{addmargin}[-1cm]{0cm}
\BODY
\end{addmargin}
}

\NewEnviron{html}{%
}

\begin{document}
\meta{url}{/cours/convexite/}
\meta{pid}{441}
\meta{titre}{Convexité}
\meta{type}{cours}
\begin{h2}I. Fonction convexe - Fonction concave\end{h2}
\cadre{bleu}{Définition}{% id="d10"
     Soient $f$ une fonction dérivable sur un intervalle $I$ et $\mathscr C_{f}$ sa courbe représentative.
     \begin{itemize}\item On dit que $f$ est \textbf{convexe} sur $I$ si la courbe $\mathscr C_{f}$ est \textbf{au-dessus} de toutes ses tangentes sur l'intervalle $I$.
          \item On dit que $f$ est \textbf{concave} sur $I$ si la courbe $\mathscr C_{f}$ est \textbf{au-dessous} de toutes ses tangentes sur l'intervalle $I$.
     \end{itemize}
}
\bloc{orange}{Exemples}{% id="e10"
     \begin{center}
          \begin{extern}%width="460" alt="fonction convexe"
               % -+-+-+ variables modifiables
               \resizebox{10cm}{!}{%
                    \def\xmin{-3.5}
                    \def\xmax{3.5}
                    \def\ymin{-2.8}
                    \def\ymax{3.5}
                    \def\xunit{2}  % unités en cm
                    \def\yunit{2}
                    \psset{xunit=\xunit,yunit=\yunit,algebraic=true}
                    \fontsize{15pt}{15pt}\selectfont
                    \begin{pspicture*}[linewidth=1pt](\xmin,\ymin)(\xmax,\ymax)
                         \psgrid[gridcolor=mcgris,subgriddiv=0](-4,-3)(4,4)
                         \psaxes[linewidth=0.75pt]{->}(0,0)(\xmin,\ymin)(\xmax,\ymax)
                         \rput[tr](-0.2,-0.2){$O$}
                         \rput[tl](2.1,3){$\color{blue} \mathscr{C}_f$}
                         \psplot[plotpoints=1000,linewidth=0.8pt,linecolor=blue]{\xmin}{\xmax}{2.71828^x-x-2}
                         \psplot[plotpoints=10,linewidth=0.8pt,linecolor=vert]{\xmin}{\xmax}{-1}
                         \psplot[plotpoints=10,linewidth=0.8pt,linecolor=vert]{\xmin}{\xmax}{1.71828*x-2}
                         \psplot[plotpoints=10,linewidth=0.8pt,linecolor=vert]{\xmin}{\xmax}{(1/2.71828-1)*(x+2)}
                    \end{pspicture*}
               }
          \end{extern}
     \end{center}
     \begin{center}
          \textit{Fonction convexe (et quelques tangentes...)}
     \end{center}
     \begin{center}
          \begin{extern}%width="460" alt="fonction concave"
               % -+-+-+ variables modifiables
               \resizebox{10cm}{!}{%
                    \def\xmin{-3.5}
                    \def\xmax{3.5}
                    \def\ymin{-2.8}
                    \def\ymax{3.5}
                    \def\xunit{2}  % unités en cm
                    \def\yunit{2}
                    \psset{xunit=\xunit,yunit=\yunit,algebraic=true}
                    \fontsize{15pt}{15pt}\selectfont
                    \begin{pspicture*}[linewidth=1pt](\xmin,\ymin)(\xmax,\ymax)
                         \psgrid[gridcolor=mcgris,subgriddiv=0](-4,-3)(4,4)
                         \psaxes[linewidth=0.75pt]{->}(0,0)(\xmin,\ymin)(\xmax,\ymax)
                         \rput[tr](-0.2,-0.2){$O$}
                         \rput[tl](2.3,-2){$\color{blue} \mathscr{C}_f$}
                         \psplot[plotpoints=1000,linewidth=0.8pt,linecolor=blue]{\xmin}{\xmax}{-2.71828^x+x+4}
                         \psplot[plotpoints=10,linewidth=0.8pt,linecolor=vert]{\xmin}{\xmax}{3}
                         \psplot[plotpoints=10,linewidth=0.8pt,linecolor=vert]{\xmin}{\xmax}{-1.71828*x+4}
                         \psplot[plotpoints=10,linewidth=0.8pt,linecolor=vert]{\xmin}{\xmax}{(-1/2.71828+1)*(x+2)+2}
                    \end{pspicture*}
               }
          \end{extern}
     \end{center}
     \begin{center}
          \textit{Fonction concave (et quelques tangentes...)}
     \end{center}
}
\cadre{rouge}{Théorème}{% id="t20"
     Si $f$ est dérivable sur $I$ :
     \begin{itemize}\item  $f$ est convexe sur $I$ si et seulement si \textbf{$f^{\prime}$} est \textbf{croissante} sur $I$
          \item  $f$ est concave sur $I$ si et seulement si \textbf{$f^{\prime}$} est \textbf{décroissante} sur $I$
     \end{itemize}
}
\bloc{cyan}{Remarque}{% id="r20"
     L'étude de la convexité se ramène donc à l'étude des variations de $f^{\prime}$. Si $f^{\prime}$ est dérivable, on donc est amené a étudier le signe la dérivée de $f^{\prime}$. Cette dérivée s'appelle la \textbf{dérivée seconde} de $f$ et se note $f^{\prime\prime}$.
}
\cadre{rouge}{Théorème}{% id="t30"
     Si $f$ est dérivable sur $I$ et si $f^{\prime}$ est dérivable sur $I$ (on dit aussi que $f$ est 2 fois dérivable
     \par
     sur $I$) :
     \begin{itemize}\item  $f$ est convexe sur $I$ si et seulement si \textbf{$f^{\prime\prime}$} est \textbf{positive ou nulle} sur $I$
          \item  $f$ est concave sur $I$ si et seulement si \textbf{$f^{\prime\prime}$} est \textbf{négative ou nulle} sur $I$
     \end{itemize}
}
\bloc{orange}{Exemples}{% id="e30"
     \begin{itemize}
          \item  La fonction $f :   x \mapsto  x^{2}$ est deux fois dérivable sur $\mathbb{R}$.
          \par
          $f^{\prime}\left(x\right)=2x$ et $f^{\prime\prime}\left(x\right)=2$.
          \par
          Comme $f^{\prime\prime}$ est positive sur $\mathbb{R}$, $f$ est convexe sur $\mathbb{R}$.
          \item  La fonction $f :   x \mapsto  x^{3}$ est deux fois dérivable sur $\mathbb{R}$.
          \par
          $f^{\prime}\left(x\right)=3x^{2}$ et $f^{\prime\prime}\left(x\right)=6x$.
          \par
          $f^{\prime\prime}\geqslant 0$  sur $\left[0; +\infty \right[$, donc $f$ est convexe sur $\left[0; +\infty \right[$.
          \par
          $f^{\prime\prime}\leqslant 0$  sur $\left]-\infty ; 0\right]$, donc $f$ est concave sur $\left]-\infty ; 0\right]$.
     \end{itemize}
}
\begin{h2}II. Point d'inflexion\end{h2}
\cadre{bleu}{Définition}{% id="d50"
     Soient $f$ une fonction dérivable sur un intervalle $I$, $\mathscr C_{f}$ sa courbe représentative et $A\left(a;f\left(a\right)\right)$ un point de la courbe  $\mathscr C_{f}$ .
     \par
     On dit que $A$ est un \textbf{point d'inflexion} de la courbe  $\mathscr C_{f}$, si et seulement si la courbe  $\mathscr C_{f}$  traverse sa tangente en $A$.
}
\bloc{orange}{Exemple}{% id="e50"
     \begin{center}
          \begin{extern}%width="460" alt="point d'inflexion"
               % -+-+-+ variables modifiables
               \resizebox{10cm}{!}{%
                    \def\xmin{-3.5}
                    \def\xmax{3.5}
                    \def\ymin{-2.8}
                    \def\ymax{3.5}
                    \def\xunit{2}  % unités en cm
                    \def\yunit{2}
                    \psset{xunit=\xunit,yunit=\yunit,algebraic=true}
                    \fontsize{15pt}{15pt}\selectfont
                    \begin{pspicture*}[linewidth=1pt](\xmin,\ymin)(\xmax,\ymax)
                         \psgrid[gridcolor=mcgris,subgriddiv=0](-4,-3)(4,4)
                         \psaxes[linewidth=0.75pt]{->}(0,0)(\xmin,\ymin)(\xmax,\ymax)
                         \rput[tr](-0.2,-0.2){$O$}
                         \rput[tl](2.9,2.9){$\color{blue} \mathscr{C}_f$}
                         \rput[bl](1.1,0.433){$\red A$}
                         \psdots[linecolor=red](1,0.333)
                         \psplot[plotpoints=1000,linewidth=0.8pt,linecolor=blue]{\xmin}{\xmax}{1/3*x^3-x^2+1}
                         \psplot[plotpoints=10,linewidth=0.8pt,linecolor=vert]{\xmin}{\xmax}{-x+4/3}
                    \end{pspicture*}
               }
          \end{extern}
     \end{center}
     \begin{center}
          \textit{Point d'inflexion en A}
     \end{center}
}
\cadre{vert}{Propriété}{% id="p60"
     Si $A$ est un point d'inflexion d'abscisse $a$, $ f $ passe de concave à convexe ou de convexe à concave en $a$.
}
\cadre{rouge}{Théorème}{% id="t70"
     Soit $ f $ une fonction deux fois dérivable sur un intervalle $I$ de courbe représentative $\mathscr C_{f}$. Le point $A$ d'abscisse $a$ est un point d'inflexion de $\mathscr C_{f} $ si et seulement si \textbf{$f^{\prime\prime}$ s'annule et change de signe en $a$}.
}
\bloc{orange}{Exemple}{% id="e70"
     Le graphique de l'exemple précédent correspond à la fonction définie par :
     \par
     $f\left(x\right)=\frac{1}{3}x^{3}-x^{2}+1$
     \par
     On a $f^{\prime}\left(x\right)=x^{2}-2x$ et $f^{\prime\prime}\left(x\right)=2x-2$.
     \par
     On vérifie bien que $f^{\prime\prime}$ change de signe en $1$. Donc le point $A$ d'abscisse $1$ et d'ordonnée $f\left(1\right)=\frac{1}{3}$ est bien un point d'inflexion.
}

\end{document}
µ
\documentclass[a4paper]{article}

%================================================================================================================================
%
% Packages
%
%================================================================================================================================

\usepackage[T1]{fontenc} 	% pour caractères accentués
\usepackage[utf8]{inputenc}  % encodage utf8
\usepackage[french]{babel}	% langue : français
\usepackage{fourier}			% caractères plus lisibles
\usepackage[dvipsnames]{xcolor} % couleurs
\usepackage{fancyhdr}		% réglage header footer
\usepackage{needspace}		% empêcher sauts de page mal placés
\usepackage{graphicx}		% pour inclure des graphiques
\usepackage{enumitem,cprotect}		% personnalise les listes d'items (nécessaire pour ol, al ...)
\usepackage{hyperref}		% Liens hypertexte
\usepackage{pstricks,pst-all,pst-node,pstricks-add,pst-math,pst-plot,pst-tree,pst-eucl} % pstricks
\usepackage[a4paper,includeheadfoot,top=2cm,left=3cm, bottom=2cm,right=3cm]{geometry} % marges etc.
\usepackage{comment}			% commentaires multilignes
\usepackage{amsmath,environ} % maths (matrices, etc.)
\usepackage{amssymb,makeidx}
\usepackage{bm}				% bold maths
\usepackage{tabularx}		% tableaux
\usepackage{colortbl}		% tableaux en couleur
\usepackage{fontawesome}		% Fontawesome
\usepackage{environ}			% environment with command
\usepackage{fp}				% calculs pour ps-tricks
\usepackage{multido}			% pour ps tricks
\usepackage[np]{numprint}	% formattage nombre
\usepackage{tikz,tkz-tab} 			% package principal TikZ
\usepackage{pgfplots}   % axes
\usepackage{mathrsfs}    % cursives
\usepackage{calc}			% calcul taille boites
\usepackage[scaled=0.875]{helvet} % font sans serif
\usepackage{svg} % svg
\usepackage{scrextend} % local margin
\usepackage{scratch} %scratch
\usepackage{multicol} % colonnes
%\usepackage{infix-RPN,pst-func} % formule en notation polanaise inversée
\usepackage{listings}

%================================================================================================================================
%
% Réglages de base
%
%================================================================================================================================

\lstset{
language=Python,   % R code
literate=
{á}{{\'a}}1
{à}{{\`a}}1
{ã}{{\~a}}1
{é}{{\'e}}1
{è}{{\`e}}1
{ê}{{\^e}}1
{í}{{\'i}}1
{ó}{{\'o}}1
{õ}{{\~o}}1
{ú}{{\'u}}1
{ü}{{\"u}}1
{ç}{{\c{c}}}1
{~}{{ }}1
}


\definecolor{codegreen}{rgb}{0,0.6,0}
\definecolor{codegray}{rgb}{0.5,0.5,0.5}
\definecolor{codepurple}{rgb}{0.58,0,0.82}
\definecolor{backcolour}{rgb}{0.95,0.95,0.92}

\lstdefinestyle{mystyle}{
    backgroundcolor=\color{backcolour},   
    commentstyle=\color{codegreen},
    keywordstyle=\color{magenta},
    numberstyle=\tiny\color{codegray},
    stringstyle=\color{codepurple},
    basicstyle=\ttfamily\footnotesize,
    breakatwhitespace=false,         
    breaklines=true,                 
    captionpos=b,                    
    keepspaces=true,                 
    numbers=left,                    
xleftmargin=2em,
framexleftmargin=2em,            
    showspaces=false,                
    showstringspaces=false,
    showtabs=false,                  
    tabsize=2,
    upquote=true
}

\lstset{style=mystyle}


\lstset{style=mystyle}
\newcommand{\imgdir}{C:/laragon/www/newmc/assets/imgsvg/}
\newcommand{\imgsvgdir}{C:/laragon/www/newmc/assets/imgsvg/}

\definecolor{mcgris}{RGB}{220, 220, 220}% ancien~; pour compatibilité
\definecolor{mcbleu}{RGB}{52, 152, 219}
\definecolor{mcvert}{RGB}{125, 194, 70}
\definecolor{mcmauve}{RGB}{154, 0, 215}
\definecolor{mcorange}{RGB}{255, 96, 0}
\definecolor{mcturquoise}{RGB}{0, 153, 153}
\definecolor{mcrouge}{RGB}{255, 0, 0}
\definecolor{mclightvert}{RGB}{205, 234, 190}

\definecolor{gris}{RGB}{220, 220, 220}
\definecolor{bleu}{RGB}{52, 152, 219}
\definecolor{vert}{RGB}{125, 194, 70}
\definecolor{mauve}{RGB}{154, 0, 215}
\definecolor{orange}{RGB}{255, 96, 0}
\definecolor{turquoise}{RGB}{0, 153, 153}
\definecolor{rouge}{RGB}{255, 0, 0}
\definecolor{lightvert}{RGB}{205, 234, 190}
\setitemize[0]{label=\color{lightvert}  $\bullet$}

\pagestyle{fancy}
\renewcommand{\headrulewidth}{0.2pt}
\fancyhead[L]{maths-cours.fr}
\fancyhead[R]{\thepage}
\renewcommand{\footrulewidth}{0.2pt}
\fancyfoot[C]{}

\newcolumntype{C}{>{\centering\arraybackslash}X}
\newcolumntype{s}{>{\hsize=.35\hsize\arraybackslash}X}

\setlength{\parindent}{0pt}		 
\setlength{\parskip}{3mm}
\setlength{\headheight}{1cm}

\def\ebook{ebook}
\def\book{book}
\def\web{web}
\def\type{web}

\newcommand{\vect}[1]{\overrightarrow{\,\mathstrut#1\,}}

\def\Oij{$\left(\text{O}~;~\vect{\imath},~\vect{\jmath}\right)$}
\def\Oijk{$\left(\text{O}~;~\vect{\imath},~\vect{\jmath},~\vect{k}\right)$}
\def\Ouv{$\left(\text{O}~;~\vect{u},~\vect{v}\right)$}

\hypersetup{breaklinks=true, colorlinks = true, linkcolor = OliveGreen, urlcolor = OliveGreen, citecolor = OliveGreen, pdfauthor={Didier BONNEL - https://www.maths-cours.fr} } % supprime les bordures autour des liens

\renewcommand{\arg}[0]{\text{arg}}

\everymath{\displaystyle}

%================================================================================================================================
%
% Macros - Commandes
%
%================================================================================================================================

\newcommand\meta[2]{    			% Utilisé pour créer le post HTML.
	\def\titre{titre}
	\def\url{url}
	\def\arg{#1}
	\ifx\titre\arg
		\newcommand\maintitle{#2}
		\fancyhead[L]{#2}
		{\Large\sffamily \MakeUppercase{#2}}
		\vspace{1mm}\textcolor{mcvert}{\hrule}
	\fi 
	\ifx\url\arg
		\fancyfoot[L]{\href{https://www.maths-cours.fr#2}{\black \footnotesize{https://www.maths-cours.fr#2}}}
	\fi 
}


\newcommand\TitreC[1]{    		% Titre centré
     \needspace{3\baselineskip}
     \begin{center}\textbf{#1}\end{center}
}

\newcommand\newpar{    		% paragraphe
     \par
}

\newcommand\nosp {    		% commande vide (pas d'espace)
}
\newcommand{\id}[1]{} %ignore

\newcommand\boite[2]{				% Boite simple sans titre
	\vspace{5mm}
	\setlength{\fboxrule}{0.2mm}
	\setlength{\fboxsep}{5mm}	
	\fcolorbox{#1}{#1!3}{\makebox[\linewidth-2\fboxrule-2\fboxsep]{
  		\begin{minipage}[t]{\linewidth-2\fboxrule-4\fboxsep}\setlength{\parskip}{3mm}
  			 #2
  		\end{minipage}
	}}
	\vspace{5mm}
}

\newcommand\CBox[4]{				% Boites
	\vspace{5mm}
	\setlength{\fboxrule}{0.2mm}
	\setlength{\fboxsep}{5mm}
	
	\fcolorbox{#1}{#1!3}{\makebox[\linewidth-2\fboxrule-2\fboxsep]{
		\begin{minipage}[t]{1cm}\setlength{\parskip}{3mm}
	  		\textcolor{#1}{\LARGE{#2}}    
 	 	\end{minipage}  
  		\begin{minipage}[t]{\linewidth-2\fboxrule-4\fboxsep}\setlength{\parskip}{3mm}
			\raisebox{1.2mm}{\normalsize\sffamily{\textcolor{#1}{#3}}}						
  			 #4
  		\end{minipage}
	}}
	\vspace{5mm}
}

\newcommand\cadre[3]{				% Boites convertible html
	\par
	\vspace{2mm}
	\setlength{\fboxrule}{0.1mm}
	\setlength{\fboxsep}{5mm}
	\fcolorbox{#1}{white}{\makebox[\linewidth-2\fboxrule-2\fboxsep]{
  		\begin{minipage}[t]{\linewidth-2\fboxrule-4\fboxsep}\setlength{\parskip}{3mm}
			\raisebox{-2.5mm}{\sffamily \small{\textcolor{#1}{\MakeUppercase{#2}}}}		
			\par		
  			 #3
 	 		\end{minipage}
	}}
		\vspace{2mm}
	\par
}

\newcommand\bloc[3]{				% Boites convertible html sans bordure
     \needspace{2\baselineskip}
     {\sffamily \small{\textcolor{#1}{\MakeUppercase{#2}}}}    
		\par		
  			 #3
		\par
}

\newcommand\CHelp[1]{
     \CBox{Plum}{\faInfoCircle}{À RETENIR}{#1}
}

\newcommand\CUp[1]{
     \CBox{NavyBlue}{\faThumbsOUp}{EN PRATIQUE}{#1}
}

\newcommand\CInfo[1]{
     \CBox{Sepia}{\faArrowCircleRight}{REMARQUE}{#1}
}

\newcommand\CRedac[1]{
     \CBox{PineGreen}{\faEdit}{BIEN R\'EDIGER}{#1}
}

\newcommand\CError[1]{
     \CBox{Red}{\faExclamationTriangle}{ATTENTION}{#1}
}

\newcommand\TitreExo[2]{
\needspace{4\baselineskip}
 {\sffamily\large EXERCICE #1\ (\emph{#2 points})}
\vspace{5mm}
}

\newcommand\img[2]{
          \includegraphics[width=#2\paperwidth]{\imgdir#1}
}

\newcommand\imgsvg[2]{
       \begin{center}   \includegraphics[width=#2\paperwidth]{\imgsvgdir#1} \end{center}
}


\newcommand\Lien[2]{
     \href{#1}{#2 \tiny \faExternalLink}
}
\newcommand\mcLien[2]{
     \href{https~://www.maths-cours.fr/#1}{#2 \tiny \faExternalLink}
}

\newcommand{\euro}{\eurologo{}}

%================================================================================================================================
%
% Macros - Environement
%
%================================================================================================================================

\newenvironment{tex}{ %
}
{%
}

\newenvironment{indente}{ %
	\setlength\parindent{10mm}
}

{
	\setlength\parindent{0mm}
}

\newenvironment{corrige}{%
     \needspace{3\baselineskip}
     \medskip
     \textbf{\textsc{Corrigé}}
     \medskip
}
{
}

\newenvironment{extern}{%
     \begin{center}
     }
     {
     \end{center}
}

\NewEnviron{code}{%
	\par
     \boite{gray}{\texttt{%
     \BODY
     }}
     \par
}

\newenvironment{vbloc}{% boite sans cadre empeche saut de page
     \begin{minipage}[t]{\linewidth}
     }
     {
     \end{minipage}
}
\NewEnviron{h2}{%
    \needspace{3\baselineskip}
    \vspace{0.6cm}
	\noindent \MakeUppercase{\sffamily \large \BODY}
	\vspace{1mm}\textcolor{mcgris}{\hrule}\vspace{0.4cm}
	\par
}{}

\NewEnviron{h3}{%
    \needspace{3\baselineskip}
	\vspace{5mm}
	\textsc{\BODY}
	\par
}

\NewEnviron{margeneg}{ %
\begin{addmargin}[-1cm]{0cm}
\BODY
\end{addmargin}
}

\NewEnviron{html}{%
}

\begin{document}
\meta{url}{/cours/aires-integrales/}
\meta{pid}{445}
\meta{titre}{Primitives et intégrales}
\meta{type}{cours}
\begin{h2}1. Primitives d'une fonction\end{h2}
\cadre{bleu}{Définition}{% id="d10"
     Soit $f$ une fonction définie sur $I$.
     \par
     On dit que $F$  est une primitive de  $f$  sur l'intervalle $I$, si et seulement si $F$ est dérivable sur $I$ et pour tout $x$ de $I$, $F^{\prime}\left(x\right)=f\left(x\right)$.
}
\bloc{orange}{Exemple}{% id="e10"
     La fonction $F :  x\mapsto x^{2}$ est une primitive de la fonction $f :  x\mapsto 2x$ sur $\mathbb{R}$.
     \par
     La fonction $G :  x\mapsto x^{2}+1$ est aussi une primitive de cette même fonction $f$.
}
\cadre{vert}{Propriété}{% id="p20"
     Si $F$ est une primitive de $f$ sur $I$, alors les autres primitives de $f$ sur $I$ sont les fonctions de la forme $F+k$ où $k\in \mathbb{R}$.
}
\bloc{vert}{Remarque}{% id="r20"
     Une fonction continue ayant une infinité de primitives, il ne faut pas dire \textbf{la} primitive de $f$ mais \textbf{une} primitive de $f$.
}
\bloc{orange}{Exemple}{% id="e20"
     Les primitives de la fonction $f : x\mapsto 2x$ sont les fonctions $F :  x\mapsto x^{2}+k$ où $k \in  \mathbb{R}$.
}
\cadre{vert}{Propriété}{% id="p30"
     Toute fonction continue sur un intervalle $I$ admet des primitives sur $I$.
}
\cadre{vert}{Propriétés (Primitives des fonctions usuelles) }{% id="p40"
     \begin{tabularx}{0.8\linewidth}{|*{3}{>{\centering \arraybackslash }X|}}%class="compact" width="600"
          \hline
          \textbf{Fonction $f$} & \textbf{Primitives $F$} & \textbf{Ensemble de validité}
          \\ \hline
          $0$ & $k$ & $\mathbb{R}$
          \\ \hline
          $a$ & $ax+k$ & $\mathbb{R}$
          \\ \hline
          $x^{n} ~ \left(n\in \mathbb{N}\right)$ & $\frac{x^{n+1}}{n+1}+k$ & $\mathbb{R}$
          \\ \hline
          $\frac{1}{x}$ & $\ln x+k$ & $\left]0;+\infty \right[$
          \\ \hline
          $e^{x}$ & $e^{x}+k$ & $\mathbb{R}$
          \\ \hline
     \end{tabularx}
}
\cadre{vert}{Propriétés}{% id="p50"
     Si $f$ et $g$ sont deux fonctions définies sur $I$ et admettant respectivement $F$ et $G$ comme primitives sur $I$ et $k$ un réel quelconque.
     \begin{itemize}
          \item $F+G$ est une primitive de la fonction $f+g$ sur $I$.
          \item $k F$ est une primitive de la fonction $k f$ sur $I$.
     \end{itemize}
}
\cadre{vert}{Propriétés}{% id="p60"
     Soit $u$ une fonction définie et dérivable sur un intervalle $I$.
     \par
     Les primitives de la fonction $x \mapsto  u^{\prime}\left(x\right)e^{u\left(x\right)}$ sont les fonctions $x \mapsto  e^{u\left(x\right)}+k$ (où $k \in  \mathbb{R}$)
}
\bloc{orange}{Exemple}{% id="e60"
     La fonction $x\mapsto 2xe^{\left(x^{2}\right)}$ est de la forme $u^{\prime}e^{u}$ avec $u\left(x\right)=x^{2}$.
     \par
     Ses primitives sont donc les fonctions $x\mapsto e^{\left(x^{2}\right)}+k   \left(k \in  \mathbb{R}\right)$
}
\begin{h2}2. Intégrales\end{h2}
\cadre{bleu}{Définition}{% id="d90"
     Soit $f$ une fonction continue sur un intervalle $\left[a ; b\right]$ et $F$ une primitive de $f$ sur $\left[a;b\right]$.
     \textbf{L'intégrale de $a$ à $b$ de $f$} est le nombre réel noté $\int_{a}^{b}f\left(x\right)dx$ défini par:
     \par
     $\int_{a}^{b}f\left(x\right)dx=F\left(b\right)-F\left(a\right)$
}
\bloc{vert}{Remarque}{% id="e90"
     L'intégrale ne dépend pas de la primitive de $f$ choisie.
     \par
     En effet si $G$ est une autre primitive de $f$, on a $G=F+k$ donc :
     \par
     $G\left(b\right)-G\left(a\right)=F\left(b\right)+k-\left(F\left(a\right)+k\right)=F\left(b\right)-F\left(a\right)$
}
\bloc{vert}{Notations}{% id="n90"
     On note souvent : $F\left(b\right)-F\left(a\right)=\left[F\left(x\right)\right]_{a}^{b}$
     \par
     On obtient avec cette notation :
     \par
     $\int_{a}^{b}f\left(x\right)dx=\left[F\left(x\right)\right]_{a}^{b}$
}
\bloc{orange}{Exemple}{% id="e90"
     La fonction $F$ définie par $F\left(x\right)=\frac{x^{3}}{3}$ est une primitive de la fonction carré.
     \par
     On a donc :
     \par
     $\int_{0}^{1}x^{2}dx=\left[\frac{x^{3}}{3}\right]_{0}^{1}=\frac{1}{3}-\frac{0}{3}=\frac{1}{3}$
}
\begin{h2}3. Propriétés de l'intégrale\end{h2}
\cadre{vert}{Propriété}{% id="p110"
     \textbf{Relation de Chasles}
     Soit $f$ une fonction continue sur $\left[a;b\right]$ et $c\in \left[a;b\right]$.
     \par
     $\int_{a}^{b}f\left(x\right)dx=\int_{a}^{c}f\left(x\right)dx+\int_{c}^{b}f\left(x\right)dx$
}
\cadre{vert}{Propriété}{% id="p120"
     \textbf{Linéarité de l'intégrale}
     Soit $f$ et $g$ deux fonctions continues sur $\left[a;b\right]$ et $\lambda \in \mathbb{R}$.
     \begin{itemize}
          \item $\int_{a}^{b}f\left(x\right)+g\left(x\right)dx=\int_{a}^{b}f\left(x\right)dx+\int_{a}^{b}g\left(x\right)dx$
          \item $\int_{a}^{b} \lambda  f\left(x\right)dx=\lambda  \int_{a}^{b}f\left(x\right)dx$
     \end{itemize}
}
\cadre{vert}{Propriété}{% id="p130"
     \textbf{Comparaison d'intégrales}
     Soit $f$ et $g$ deux fonctions continues sur $\left[a;b\right]$ telles que $f\geqslant g$ sur $\left[a;b\right]$.
     \par
     $\int_{a}^{b}f\left(x\right)dx\geqslant \int_{a}^{b}g\left(x\right)dx$
}
\bloc{vert}{Remarque}{% id="r130"
     En particulier, en prenant pour $g$ la fonction nulle on obtient si $f\left(x\right)\geqslant 0$ sur $\left[a;b\right]$:
     \par
     $\int_{a}^{b}f\left(x\right)dx\geqslant 0$
}
\begin{h2}4. Interprétation graphique\end{h2}
\cadre{bleu}{Définition}{% id="d150"
     Le plan $P$ est rapporté à un repère orthogonal $\left(O,\vec{i},\vec{j}\right)$.
     \par
     On appelle \textbf{unité d'aire (u.a.)} l'aire d'un rectangle dont les côtés mesurent $||\vec{i}||$ et $||\vec{j}||$.
}
\begin{center}
     \begin{extern} %width="400" alt="unité d'aire"
          \resizebox{6cm}{!}{%
               % -+-+-+ variables modifiables
               \def\xmin{-1.2}
               \def\xmax{4.2}
               \def\ymin{-0.9}
               \def\ymax{1.8}
               \def\xunit{2}  % unités en cm
               \def\yunit{2}
               \psset{xunit=\xunit,yunit=\yunit,algebraic=true}
               \fontsize{15pt}{15pt}\selectfont
               \begin{pspicture*}[linewidth=1pt](\xmin,\ymin)(\xmax,\ymax)
                    \psgrid[gridcolor=mcgris, subgriddiv=1, gridlabels=0pt](-2,-1)(5,2)
                    \psaxes[linewidth=0.75pt]{->}(0,0)(\xmin,\ymin)(\xmax,\ymax)
                    \pscustom[fillstyle=solid,fillcolor=vert,linecolor=vert,linewidth=0.75pt,opacity=0.2]{%
                         \psline(0,0)(1,0)(1,1)(0,1)
                         \closepath
                    }
                    \rput[tr](-0.1,-0.1){$O$}
                    \psline[linewidth=1.25pt]{->}(0,0)(0,1)
                    \psline[linewidth=1.25pt]{->}(0,0)(1,0)
                    \rput[t](0.5,-0.03){$\vect{i}$}
                    \rput[r](-0.03,0.5){$\vect{j}$}
                    %                        \rput[l](1.1){ unité d'aire}
               \end{pspicture*}
          }
     \end{extern}
\end{center}
\begin{center}
     \textit{Unité d'aire dans le cas d'un repère orthonormé}
\end{center}
\cadre{vert}{Propriété}{% id="p160"
     Si $f$ est une fonction continue et \textbf{positive} sur $\left[a;b\right]$, alors l'intégrale $\int_{a}^{b}f\left(x\right)dx$ est l'aire, en unités d'aire, de la surface délimitée par :
     \begin{itemize}
          \item la courbe $C_{f}$
          \item l'axe des abscisses
          \item les droites (verticales) d'équations $x=a$ et $x=b$
     \end{itemize}
}
\bloc{orange}{Exemple}{% id="e160"
     \begin{center}
          \begin{extern} %width="600" alt="aire et intégrale"
               \resizebox{8cm}{!}{%
                    % -+-+-+ variables modifiables
                    \def\fonction{ln(1+x*x) }
                    \def\xmin{-2.2}
                    \def\xmax{6.2}
                    \def\ymin{-1.8}
                    \def\ymax{3.8}
                    \def\xunit{2}  % unités en cm
                    \def\yunit{2}
                    \psset{xunit=\xunit,yunit=\yunit,algebraic=true}
                    \fontsize{15pt}{15pt}\selectfont
                    \begin{pspicture*}[linewidth=1pt](\xmin,\ymin)(\xmax,\ymax)
                         \psgrid[gridcolor=mcgris, subgriddiv=1, gridlabels=0pt](-3,-1.8)(7,4)
                         \psaxes[linewidth=0.75pt]{->}(0,0)(\xmin,\ymin)(\xmax,\ymax)
                         \pscustom[fillstyle=solid,fillcolor=vert,linecolor=vert,linewidth=0.75pt,opacity=0.2]{%
                              \psplot{1}{3}{\fonction}
                              \psline(3,0)(1,0)
                              \closepath
                         }
                         \psplot[plotpoints=2000,linecolor=blue]{\xmin}{\xmax}{\fonction}
                         \rput[tr](-0.1,-0.1){$O$}
                         \rput[l](5.5,3.2){$\color{blue} \mathcal{C}_f$}
                    \end{pspicture*}
               }
          \end{extern}
     \end{center}
     L'aire colorée ci-dessus est égale (en unités d'aire) à $\int_{1}^{3}f\left(x\right)dx$
}
\bloc{vert}{Remarques}{% id="r160"
     \begin{itemize}
          \item Si $f$ est négative sur $\left[a;b\right]$, la propriété précédente appliquée à la fonction $-f$ montre que
          \par
          $\int_{a}^{b}f\left(x\right)dx$ est égale à l'\textbf{opposé} de l'aire délimitée par la courbe $C_{f}$, l'axe des abscisses, les droites d'équations $x=a$ et $x=b$
          \item Si le signe de $f$ varie sur $\left[a;b\right]$, on découpe $\left[a;b\right]$ en sous-intervalles sur lesquels $f$ garde un signe constant.
     \end{itemize}
}
\cadre{vert}{Propriété}{% id="p170"
     Si $f$ et $g$ sont des fonctions continues et telles que $f\leqslant g$ sur $\left[a;b\right]$, alors l'aire de la surface délimitée par :
     \begin{itemize}
          \item la courbe $C_{f}$
          \item la courbe $C_{g}$
          \item les droites (verticales) d'équations $x=a$ et $x=b$
     \end{itemize}
     est égale (en unités d'aire) à :
     \par
     $A=\int_{a}^{b}g\left(x\right)-f\left(x\right)dx$
}
\bloc{orange}{Exemple}{% id="e170"
     $f$ et $g$ définies par $f\left(x\right)=x^{2}-x$ et $g\left(x\right)=3x-x^{2}$ sont représentées par les paraboles ci-dessous :
     \begin{center}
          \begin{extern} %width="600" alt="aire entre deux courbes"
               \resizebox{8cm}{!}{%
                    % -+-+-+ variables modifiables
                    \def\fonction{x*x-x }
                    \def\g{3*x-x*x }
                    \def\xmin{-2.2}
                    \def\xmax{6.2}
                    \def\ymin{-1.8}
                    \def\ymax{3.8}
                    \def\xunit{2}  % unités en cm
                    \def\yunit{2}
                    \psset{xunit=\xunit,yunit=\yunit,algebraic=true}
                    \fontsize{15pt}{15pt}\selectfont
                    \begin{pspicture*}[linewidth=1pt](\xmin,\ymin)(\xmax,\ymax)
                         \psgrid[gridcolor=mcgris, subgriddiv=1, gridlabels=0pt](-3,-3)(7,4)
                         \psaxes[linewidth=0.75pt]{->}(0,0)(\xmin,\ymin)(\xmax,\ymax)
                         \pscustom[fillstyle=solid,fillcolor=vert,linestyle=solid,linewidth=0.2pt,opacity=0.2]{%
                              \psplot{0}{2}{\fonction}
                              \psplot{2}{0}{\g}
                         }
                         \psplot[plotpoints=2000,linecolor=blue]{\xmin}{\xmax}{\fonction}
                         \psplot[plotpoints=2000,linecolor=red]{\xmin}{\xmax}{\g}
                         \rput[tr](-0.1,-0.1){$O$}
                         \rput(-1.6,3){$\color{blue} \mathcal{C}_f$}
                         \rput(3.6,-1){$\color{red} \mathcal{C}_g$}
                    \end{pspicture*}
               }
          \end{extern}
     \end{center}
     L'aire colorée est égale (en unités d'aire) à :
     \par
     $A=\int_{0}^{2}g\left(x\right)-f\left(x\right)dx=\int_{0}^{2} 4x-2x^{2}=\left[2x^{2}-\frac{2}{3}x^{3}\right]_{0}^{2}=\frac{8}{3} \text{u.a.}$
}

\end{document}
µ
\documentclass[a4paper]{article}

%================================================================================================================================
%
% Packages
%
%================================================================================================================================

\usepackage[T1]{fontenc} 	% pour caractères accentués
\usepackage[utf8]{inputenc}  % encodage utf8
\usepackage[french]{babel}	% langue : français
\usepackage{fourier}			% caractères plus lisibles
\usepackage[dvipsnames]{xcolor} % couleurs
\usepackage{fancyhdr}		% réglage header footer
\usepackage{needspace}		% empêcher sauts de page mal placés
\usepackage{graphicx}		% pour inclure des graphiques
\usepackage{enumitem,cprotect}		% personnalise les listes d'items (nécessaire pour ol, al ...)
\usepackage{hyperref}		% Liens hypertexte
\usepackage{pstricks,pst-all,pst-node,pstricks-add,pst-math,pst-plot,pst-tree,pst-eucl} % pstricks
\usepackage[a4paper,includeheadfoot,top=2cm,left=3cm, bottom=2cm,right=3cm]{geometry} % marges etc.
\usepackage{comment}			% commentaires multilignes
\usepackage{amsmath,environ} % maths (matrices, etc.)
\usepackage{amssymb,makeidx}
\usepackage{bm}				% bold maths
\usepackage{tabularx}		% tableaux
\usepackage{colortbl}		% tableaux en couleur
\usepackage{fontawesome}		% Fontawesome
\usepackage{environ}			% environment with command
\usepackage{fp}				% calculs pour ps-tricks
\usepackage{multido}			% pour ps tricks
\usepackage[np]{numprint}	% formattage nombre
\usepackage{tikz,tkz-tab} 			% package principal TikZ
\usepackage{pgfplots}   % axes
\usepackage{mathrsfs}    % cursives
\usepackage{calc}			% calcul taille boites
\usepackage[scaled=0.875]{helvet} % font sans serif
\usepackage{svg} % svg
\usepackage{scrextend} % local margin
\usepackage{scratch} %scratch
\usepackage{multicol} % colonnes
%\usepackage{infix-RPN,pst-func} % formule en notation polanaise inversée
\usepackage{listings}

%================================================================================================================================
%
% Réglages de base
%
%================================================================================================================================

\lstset{
language=Python,   % R code
literate=
{á}{{\'a}}1
{à}{{\`a}}1
{ã}{{\~a}}1
{é}{{\'e}}1
{è}{{\`e}}1
{ê}{{\^e}}1
{í}{{\'i}}1
{ó}{{\'o}}1
{õ}{{\~o}}1
{ú}{{\'u}}1
{ü}{{\"u}}1
{ç}{{\c{c}}}1
{~}{{ }}1
}


\definecolor{codegreen}{rgb}{0,0.6,0}
\definecolor{codegray}{rgb}{0.5,0.5,0.5}
\definecolor{codepurple}{rgb}{0.58,0,0.82}
\definecolor{backcolour}{rgb}{0.95,0.95,0.92}

\lstdefinestyle{mystyle}{
    backgroundcolor=\color{backcolour},   
    commentstyle=\color{codegreen},
    keywordstyle=\color{magenta},
    numberstyle=\tiny\color{codegray},
    stringstyle=\color{codepurple},
    basicstyle=\ttfamily\footnotesize,
    breakatwhitespace=false,         
    breaklines=true,                 
    captionpos=b,                    
    keepspaces=true,                 
    numbers=left,                    
xleftmargin=2em,
framexleftmargin=2em,            
    showspaces=false,                
    showstringspaces=false,
    showtabs=false,                  
    tabsize=2,
    upquote=true
}

\lstset{style=mystyle}


\lstset{style=mystyle}
\newcommand{\imgdir}{C:/laragon/www/newmc/assets/imgsvg/}
\newcommand{\imgsvgdir}{C:/laragon/www/newmc/assets/imgsvg/}

\definecolor{mcgris}{RGB}{220, 220, 220}% ancien~; pour compatibilité
\definecolor{mcbleu}{RGB}{52, 152, 219}
\definecolor{mcvert}{RGB}{125, 194, 70}
\definecolor{mcmauve}{RGB}{154, 0, 215}
\definecolor{mcorange}{RGB}{255, 96, 0}
\definecolor{mcturquoise}{RGB}{0, 153, 153}
\definecolor{mcrouge}{RGB}{255, 0, 0}
\definecolor{mclightvert}{RGB}{205, 234, 190}

\definecolor{gris}{RGB}{220, 220, 220}
\definecolor{bleu}{RGB}{52, 152, 219}
\definecolor{vert}{RGB}{125, 194, 70}
\definecolor{mauve}{RGB}{154, 0, 215}
\definecolor{orange}{RGB}{255, 96, 0}
\definecolor{turquoise}{RGB}{0, 153, 153}
\definecolor{rouge}{RGB}{255, 0, 0}
\definecolor{lightvert}{RGB}{205, 234, 190}
\setitemize[0]{label=\color{lightvert}  $\bullet$}

\pagestyle{fancy}
\renewcommand{\headrulewidth}{0.2pt}
\fancyhead[L]{maths-cours.fr}
\fancyhead[R]{\thepage}
\renewcommand{\footrulewidth}{0.2pt}
\fancyfoot[C]{}

\newcolumntype{C}{>{\centering\arraybackslash}X}
\newcolumntype{s}{>{\hsize=.35\hsize\arraybackslash}X}

\setlength{\parindent}{0pt}		 
\setlength{\parskip}{3mm}
\setlength{\headheight}{1cm}

\def\ebook{ebook}
\def\book{book}
\def\web{web}
\def\type{web}

\newcommand{\vect}[1]{\overrightarrow{\,\mathstrut#1\,}}

\def\Oij{$\left(\text{O}~;~\vect{\imath},~\vect{\jmath}\right)$}
\def\Oijk{$\left(\text{O}~;~\vect{\imath},~\vect{\jmath},~\vect{k}\right)$}
\def\Ouv{$\left(\text{O}~;~\vect{u},~\vect{v}\right)$}

\hypersetup{breaklinks=true, colorlinks = true, linkcolor = OliveGreen, urlcolor = OliveGreen, citecolor = OliveGreen, pdfauthor={Didier BONNEL - https://www.maths-cours.fr} } % supprime les bordures autour des liens

\renewcommand{\arg}[0]{\text{arg}}

\everymath{\displaystyle}

%================================================================================================================================
%
% Macros - Commandes
%
%================================================================================================================================

\newcommand\meta[2]{    			% Utilisé pour créer le post HTML.
	\def\titre{titre}
	\def\url{url}
	\def\arg{#1}
	\ifx\titre\arg
		\newcommand\maintitle{#2}
		\fancyhead[L]{#2}
		{\Large\sffamily \MakeUppercase{#2}}
		\vspace{1mm}\textcolor{mcvert}{\hrule}
	\fi 
	\ifx\url\arg
		\fancyfoot[L]{\href{https://www.maths-cours.fr#2}{\black \footnotesize{https://www.maths-cours.fr#2}}}
	\fi 
}


\newcommand\TitreC[1]{    		% Titre centré
     \needspace{3\baselineskip}
     \begin{center}\textbf{#1}\end{center}
}

\newcommand\newpar{    		% paragraphe
     \par
}

\newcommand\nosp {    		% commande vide (pas d'espace)
}
\newcommand{\id}[1]{} %ignore

\newcommand\boite[2]{				% Boite simple sans titre
	\vspace{5mm}
	\setlength{\fboxrule}{0.2mm}
	\setlength{\fboxsep}{5mm}	
	\fcolorbox{#1}{#1!3}{\makebox[\linewidth-2\fboxrule-2\fboxsep]{
  		\begin{minipage}[t]{\linewidth-2\fboxrule-4\fboxsep}\setlength{\parskip}{3mm}
  			 #2
  		\end{minipage}
	}}
	\vspace{5mm}
}

\newcommand\CBox[4]{				% Boites
	\vspace{5mm}
	\setlength{\fboxrule}{0.2mm}
	\setlength{\fboxsep}{5mm}
	
	\fcolorbox{#1}{#1!3}{\makebox[\linewidth-2\fboxrule-2\fboxsep]{
		\begin{minipage}[t]{1cm}\setlength{\parskip}{3mm}
	  		\textcolor{#1}{\LARGE{#2}}    
 	 	\end{minipage}  
  		\begin{minipage}[t]{\linewidth-2\fboxrule-4\fboxsep}\setlength{\parskip}{3mm}
			\raisebox{1.2mm}{\normalsize\sffamily{\textcolor{#1}{#3}}}						
  			 #4
  		\end{minipage}
	}}
	\vspace{5mm}
}

\newcommand\cadre[3]{				% Boites convertible html
	\par
	\vspace{2mm}
	\setlength{\fboxrule}{0.1mm}
	\setlength{\fboxsep}{5mm}
	\fcolorbox{#1}{white}{\makebox[\linewidth-2\fboxrule-2\fboxsep]{
  		\begin{minipage}[t]{\linewidth-2\fboxrule-4\fboxsep}\setlength{\parskip}{3mm}
			\raisebox{-2.5mm}{\sffamily \small{\textcolor{#1}{\MakeUppercase{#2}}}}		
			\par		
  			 #3
 	 		\end{minipage}
	}}
		\vspace{2mm}
	\par
}

\newcommand\bloc[3]{				% Boites convertible html sans bordure
     \needspace{2\baselineskip}
     {\sffamily \small{\textcolor{#1}{\MakeUppercase{#2}}}}    
		\par		
  			 #3
		\par
}

\newcommand\CHelp[1]{
     \CBox{Plum}{\faInfoCircle}{À RETENIR}{#1}
}

\newcommand\CUp[1]{
     \CBox{NavyBlue}{\faThumbsOUp}{EN PRATIQUE}{#1}
}

\newcommand\CInfo[1]{
     \CBox{Sepia}{\faArrowCircleRight}{REMARQUE}{#1}
}

\newcommand\CRedac[1]{
     \CBox{PineGreen}{\faEdit}{BIEN R\'EDIGER}{#1}
}

\newcommand\CError[1]{
     \CBox{Red}{\faExclamationTriangle}{ATTENTION}{#1}
}

\newcommand\TitreExo[2]{
\needspace{4\baselineskip}
 {\sffamily\large EXERCICE #1\ (\emph{#2 points})}
\vspace{5mm}
}

\newcommand\img[2]{
          \includegraphics[width=#2\paperwidth]{\imgdir#1}
}

\newcommand\imgsvg[2]{
       \begin{center}   \includegraphics[width=#2\paperwidth]{\imgsvgdir#1} \end{center}
}


\newcommand\Lien[2]{
     \href{#1}{#2 \tiny \faExternalLink}
}
\newcommand\mcLien[2]{
     \href{https~://www.maths-cours.fr/#1}{#2 \tiny \faExternalLink}
}

\newcommand{\euro}{\eurologo{}}

%================================================================================================================================
%
% Macros - Environement
%
%================================================================================================================================

\newenvironment{tex}{ %
}
{%
}

\newenvironment{indente}{ %
	\setlength\parindent{10mm}
}

{
	\setlength\parindent{0mm}
}

\newenvironment{corrige}{%
     \needspace{3\baselineskip}
     \medskip
     \textbf{\textsc{Corrigé}}
     \medskip
}
{
}

\newenvironment{extern}{%
     \begin{center}
     }
     {
     \end{center}
}

\NewEnviron{code}{%
	\par
     \boite{gray}{\texttt{%
     \BODY
     }}
     \par
}

\newenvironment{vbloc}{% boite sans cadre empeche saut de page
     \begin{minipage}[t]{\linewidth}
     }
     {
     \end{minipage}
}
\NewEnviron{h2}{%
    \needspace{3\baselineskip}
    \vspace{0.6cm}
	\noindent \MakeUppercase{\sffamily \large \BODY}
	\vspace{1mm}\textcolor{mcgris}{\hrule}\vspace{0.4cm}
	\par
}{}

\NewEnviron{h3}{%
    \needspace{3\baselineskip}
	\vspace{5mm}
	\textsc{\BODY}
	\par
}

\NewEnviron{margeneg}{ %
\begin{addmargin}[-1cm]{0cm}
\BODY
\end{addmargin}
}

\NewEnviron{html}{%
}

\begin{document}
\meta{url}{/cours/les-suites-geometriques/}
\meta{pid}{447}
\meta{titre}{Les suites géométriques}
\meta{type}{cours}
\begin{h2}1 - Caractéristiques d'une suite géométrique\end{h2}
\cadre{bleu}{Définition}{% id="d10"
     On dit qu'une suite $\left(u_{n}\right)_{n\in \mathbb{N}}$ est une \textbf{suite géométrique} s'il existe un nombre réel $q$ tel que :
     \par
     pour tout $n\in \mathbb{N}$,  $u_{n+1}=q \times  u_{n}$
     \par
     Le réel $q$ s'appelle la \textbf{raison} de la suite géométrique $\left(u_{n}\right)$.
}
\bloc{cyan}{Remarque}{% id="r10"
     Pour démontrer qu'une suite $\left(u_{n}\right)_{n\in \mathbb{N}}$ dont les termes sont non nuls est une suite géométrique, on pourra calculer le rapport $\frac{u_{n+1}}{u_{n}}$.
     \par
     Si ce rapport est une constante $q$, on pourra affirmer que la suite est une suite géométrique de raison $q$.
}
\bloc{orange}{Exemple}{% id="e10"
     Soit la suite $\left(u_{n}\right)_{n\in \mathbb{N}}$ définie par $u_{n}=\frac{3}{2^{n}}$.
     \par
     Les termes de la suite sont tous strictement positifs et
     \par
     $\frac{u_{n+1}}{u_{n}}=$$\frac{3}{2^{n+1}}\times \frac{2^{n}}{3}=\frac{2^{n}}{2^{n+1}}=$$\frac{2^{n}}{2\times 2^{n}}=\frac{1}{2}$
     \par
     La suite $\left(u_{n}\right)$ est une suite géométrique de raison $\frac{1}{2}$
}
\cadre{vert}{Propriété}{% id="p20"
     Pour $n$ et $k$ quelconques entiers naturels, si la suite $\left(u_{n}\right)$ est géométrique de raison $q$ :$u_{n}=u_{k}\times q^{n-k}$.
     \par
     En particulier $u_{n}=u_{0}\times q^{n}$.
}
\cadre{vert}{Propriété}{% id="p30"
     Réciproquement, soient $a$ et $b$ deux nombres réels. La suite $\left(u_{n}\right)$ définie par $u_{n}=a\times b^{n}$ suite est une suite géométrique de raison $q=b$ et de premier terme $u_{0}=a$.
}
\bloc{cyan}{Démonstration}{% id="m30"
     $u_{n+1}=a\times b^{n+1}=a\times b^{n}\times b=u_{n}\times b$
     \par
     et
     \par
     $u_{0}=a\times b^{0}=a\times 1=a$
}
\cadre{rouge}{Théorème}{% id="t40"
     Soit $\left(u_{n}\right) $une suite géométrique de raison $q  > 0$ et de premier terme strictement positif :
     \begin{itemize}
          \item Si q >1, la suite $\left(u_{n}\right) $est strictement croissante
          \item Si 0 < q <1, la suite $\left(u_{n}\right) $est strictement décroissante
          \item Si q=1, la suite $\left(u_{n}\right) $est constante
     \end{itemize}
}
\cadre{rouge}{Théorème}{% id="t50"
     Si $\left(u_{n}\right)$ et $\left(v_{n}\right)$ sont deux suites géométriques de raison respectives $q$ et $q^{\prime}$ alors le produit $\left(w_{n}\right)$ de ces deux suites défini par :
     \par
     $w_{n}=u_{n}\times v_{n}$
     \par
     est une suite géométrique de raison $q^{\prime\prime}=q\times q^{\prime}$
}
\begin{h2}2 - Somme des puissances successives d'un nombre\end{h2}
\cadre{rouge}{Théorème}{% id="t70"
     Soit $q$ un nombre réel \textbf{différent de 1}:
     \begin{center}
          $1+q+q^{2}+ . . . +q^{n} = \frac{1-q^{n+1}}{1-q}$
     \end{center}
}
\bloc{cyan}{Remarque}{% id="r70"
     Cette formule n'est pas valable pour $q=1$. Mais dans ce cas le calcul est immédiat car tous les termes sont égaux à 1.
}
\bloc{orange}{Exemple}{% id="e70"
     Soit à calculer la somme $S=1+2+4+8+16 + . . .+2^{n}$
     \par
     Donc:
     \par
     $S=\frac{1-2^{n+1}}{1-2}=\frac{1-2^{n+1}}{-1}=2^{n+1}-1$
}
\begin{h2}3 - Limite de la suite $\left(q^{n}\right)$ où $q\geqslant 0$\end{h2}
\cadre{rouge}{Théorème}{% id="t90"
     Soit $q$ un nombre réel positif.
     \begin{itemize}
          \item \textbf{Si $q > 1$ :} alors $q^{n}$ est aussi grand que l'on veut dès que $n$ est suffisamment grand. On dit que la suite $\left(q^{n}\right)$ tend vers $+\infty $ et on écrit :
          \begin{center}$\lim\limits_{n\rightarrow +\infty }  q^{n} = +\infty  $ ( ou $\lim\limits_{n\rightarrow +\infty }\left(q^{n}\right) = +\infty $)\end{center}
          \item \textbf{Si $ 0 \leqslant  q < 1$ :} alors $q^{n}$ est aussi proche de zéro que l'on veut dès que $n$ est suffisamment grand. On dit que la suite $\left(q^{n}\right)$ tend vers $0$ et on écrit :
          \begin{center}$\lim\limits_{n\rightarrow +\infty }  q^{n} = 0 $ ( ou $\lim\limits_{n\rightarrow +\infty }\left(q^{n}\right) = 0$)\end{center}
     \end{itemize}
}
\bloc{cyan}{Remarque}{% id="r90"
     Pour $q=1$ $q^{n}=1^{n}=1$; la suite est constante, égale à $1$, et tend donc vers $1$;
}
\begin{h2}4 - Suites arithmético-géométriques\end{h2}
\cadre{bleu}{Définition}{% id="d110"
     Une suite arithmético-géométrique $u_{n}$ est définie par son premier terme $u_{0}$ et une relation de  récurrence du type :
     \begin{center}$u_{n+1} = a\times u_{n}+b$ pour tout entier $n$\end{center}
     où $a$ et $b$ sont deux nombres réels.
}
\bloc{cyan}{Remarque}{% id="r110"
     \textbf{Attention} : Ces suites ne sont \textbf{ni arithmétiques} (sauf si $a=1$) \textbf{ni géométriques} (sauf si $b=0$).
}
\cadre{vert}{Propriété}{% id="p130"
     Il existe un nombre réel $k$ tel que la suite $v_{n}$ définie, pour tout entier $n$, par $v_{n}=u_{n}+k$ soit une suite géométrique de raison $a$.
}
\bloc{cyan}{Remarques}{% id="r130"
     \begin{itemize}
          \item En général, dans les exercices, le nombre $k$ vous sera donné (et si ce n'est pas le cas on vous indiquera une démarche pour le trouver). On vous demandera de prouver que $v_{n}$ est une suite géométrique de raison $a$.
          \item Puisque  $v_{n}=u_{n}+k$, pour tout entier $n$, on a en particulier $v_{0}=u_{0}+k$ ce qui permet de connaître le premier terme de la suite $v_{n}$.
          \item $v_{n}=u_{n}+k$ signifie aussi que $u_{n}=v_{n}-k$.
          \par
          Donc une fois que l'on connaît $v_{n}$ on peut trouver $u_{n}$ (voir exemple ci-dessous)
     \end{itemize}
}
\bloc{orange}{Exemple détaillé}{% id="e130"
     Soit la suite $\left(u_{n}\right)$ définie par $u_{0}=5$ et $u_{n+1}=0,6u_{n}+4$.
     \begin{enumerate}\item Montrer que la suite $\left(v_{n}\right)$ définie par $v_{n}=u_{n}-10$ est une suite géométrique.
          \item En déduire l'expression de $u_{n}$ en fonction de $n$.
     \end{enumerate}
     \begin{enumerate}\item \textbf{Montrons que la suite $\left(v_{n}\right)$ est une suite géométrique}
          Pour montrer que la suite $\left(v_{n}\right)$ est géométrique on va calculer $v_{n+1}$ en fonction de $v_{n}$.
          \par
          $v_{n}=u_{n}-10$ pour tout entier $n$ donc :
          \par
          $v_{n+1}=u_{n+1}-10$
          \par
          or on sait que
          \par
          $u_{n+1}=0,6u_{n}+4$
          \par
          donc
          \par
          $v_{n+1}=0,6u_{n}+4-10 = 0,6u_{n}-6$
          \par
          Ici, une petite astuce consiste à mettre $0,6$ en facteur (on peut également dire que $u_{n}=v_{n}+10$ et remplacer $u_{n}$ par $v_{n}+10$)
          \par
          $v_{n+1}=0,6u_{n}-0,6\times 10=0,6\left(u_{n}-10\right)=0,6v_{n}$
          \par
          On a bien une relation du type $v_{n+1}=q\times v_{n}$ avec $q=0,6$ ce qui montre que \textbf{la suite $\left(v_{n}\right)$ est une suite géométrique de raison $0,6$}.
          \item \textbf{Expression de $u_{n}$ en fonction de $n$}
          Par ailleurs, $v_{0}=u_{0}-10=5-10=-5$
          \par
          $\left(v_{n}\right)$ est une suite géométrique de premier terme $v_{0}=5$ et de raison $q=0,6$ donc pour tout entier $n$:
          \par
          $v_{n}=v_{0}\times q^{n}=-5\times 0,6^{n}$
          \par
          Comme $u_{n}=v_{n}+10$, on obtient finalement :
          \begin{center}$u_{n}=-5\times 0,6^{n}+10$\end{center}
     \end{enumerate}
}

\end{document}
µ
\documentclass[a4paper]{article}

%================================================================================================================================
%
% Packages
%
%================================================================================================================================

\usepackage[T1]{fontenc} 	% pour caractères accentués
\usepackage[utf8]{inputenc}  % encodage utf8
\usepackage[french]{babel}	% langue : français
\usepackage{fourier}			% caractères plus lisibles
\usepackage[dvipsnames]{xcolor} % couleurs
\usepackage{fancyhdr}		% réglage header footer
\usepackage{needspace}		% empêcher sauts de page mal placés
\usepackage{graphicx}		% pour inclure des graphiques
\usepackage{enumitem,cprotect}		% personnalise les listes d'items (nécessaire pour ol, al ...)
\usepackage{hyperref}		% Liens hypertexte
\usepackage{pstricks,pst-all,pst-node,pstricks-add,pst-math,pst-plot,pst-tree,pst-eucl} % pstricks
\usepackage[a4paper,includeheadfoot,top=2cm,left=3cm, bottom=2cm,right=3cm]{geometry} % marges etc.
\usepackage{comment}			% commentaires multilignes
\usepackage{amsmath,environ} % maths (matrices, etc.)
\usepackage{amssymb,makeidx}
\usepackage{bm}				% bold maths
\usepackage{tabularx}		% tableaux
\usepackage{colortbl}		% tableaux en couleur
\usepackage{fontawesome}		% Fontawesome
\usepackage{environ}			% environment with command
\usepackage{fp}				% calculs pour ps-tricks
\usepackage{multido}			% pour ps tricks
\usepackage[np]{numprint}	% formattage nombre
\usepackage{tikz,tkz-tab} 			% package principal TikZ
\usepackage{pgfplots}   % axes
\usepackage{mathrsfs}    % cursives
\usepackage{calc}			% calcul taille boites
\usepackage[scaled=0.875]{helvet} % font sans serif
\usepackage{svg} % svg
\usepackage{scrextend} % local margin
\usepackage{scratch} %scratch
\usepackage{multicol} % colonnes
%\usepackage{infix-RPN,pst-func} % formule en notation polanaise inversée
\usepackage{listings}

%================================================================================================================================
%
% Réglages de base
%
%================================================================================================================================

\lstset{
language=Python,   % R code
literate=
{á}{{\'a}}1
{à}{{\`a}}1
{ã}{{\~a}}1
{é}{{\'e}}1
{è}{{\`e}}1
{ê}{{\^e}}1
{í}{{\'i}}1
{ó}{{\'o}}1
{õ}{{\~o}}1
{ú}{{\'u}}1
{ü}{{\"u}}1
{ç}{{\c{c}}}1
{~}{{ }}1
}


\definecolor{codegreen}{rgb}{0,0.6,0}
\definecolor{codegray}{rgb}{0.5,0.5,0.5}
\definecolor{codepurple}{rgb}{0.58,0,0.82}
\definecolor{backcolour}{rgb}{0.95,0.95,0.92}

\lstdefinestyle{mystyle}{
    backgroundcolor=\color{backcolour},   
    commentstyle=\color{codegreen},
    keywordstyle=\color{magenta},
    numberstyle=\tiny\color{codegray},
    stringstyle=\color{codepurple},
    basicstyle=\ttfamily\footnotesize,
    breakatwhitespace=false,         
    breaklines=true,                 
    captionpos=b,                    
    keepspaces=true,                 
    numbers=left,                    
xleftmargin=2em,
framexleftmargin=2em,            
    showspaces=false,                
    showstringspaces=false,
    showtabs=false,                  
    tabsize=2,
    upquote=true
}

\lstset{style=mystyle}


\lstset{style=mystyle}
\newcommand{\imgdir}{C:/laragon/www/newmc/assets/imgsvg/}
\newcommand{\imgsvgdir}{C:/laragon/www/newmc/assets/imgsvg/}

\definecolor{mcgris}{RGB}{220, 220, 220}% ancien~; pour compatibilité
\definecolor{mcbleu}{RGB}{52, 152, 219}
\definecolor{mcvert}{RGB}{125, 194, 70}
\definecolor{mcmauve}{RGB}{154, 0, 215}
\definecolor{mcorange}{RGB}{255, 96, 0}
\definecolor{mcturquoise}{RGB}{0, 153, 153}
\definecolor{mcrouge}{RGB}{255, 0, 0}
\definecolor{mclightvert}{RGB}{205, 234, 190}

\definecolor{gris}{RGB}{220, 220, 220}
\definecolor{bleu}{RGB}{52, 152, 219}
\definecolor{vert}{RGB}{125, 194, 70}
\definecolor{mauve}{RGB}{154, 0, 215}
\definecolor{orange}{RGB}{255, 96, 0}
\definecolor{turquoise}{RGB}{0, 153, 153}
\definecolor{rouge}{RGB}{255, 0, 0}
\definecolor{lightvert}{RGB}{205, 234, 190}
\setitemize[0]{label=\color{lightvert}  $\bullet$}

\pagestyle{fancy}
\renewcommand{\headrulewidth}{0.2pt}
\fancyhead[L]{maths-cours.fr}
\fancyhead[R]{\thepage}
\renewcommand{\footrulewidth}{0.2pt}
\fancyfoot[C]{}

\newcolumntype{C}{>{\centering\arraybackslash}X}
\newcolumntype{s}{>{\hsize=.35\hsize\arraybackslash}X}

\setlength{\parindent}{0pt}		 
\setlength{\parskip}{3mm}
\setlength{\headheight}{1cm}

\def\ebook{ebook}
\def\book{book}
\def\web{web}
\def\type{web}

\newcommand{\vect}[1]{\overrightarrow{\,\mathstrut#1\,}}

\def\Oij{$\left(\text{O}~;~\vect{\imath},~\vect{\jmath}\right)$}
\def\Oijk{$\left(\text{O}~;~\vect{\imath},~\vect{\jmath},~\vect{k}\right)$}
\def\Ouv{$\left(\text{O}~;~\vect{u},~\vect{v}\right)$}

\hypersetup{breaklinks=true, colorlinks = true, linkcolor = OliveGreen, urlcolor = OliveGreen, citecolor = OliveGreen, pdfauthor={Didier BONNEL - https://www.maths-cours.fr} } % supprime les bordures autour des liens

\renewcommand{\arg}[0]{\text{arg}}

\everymath{\displaystyle}

%================================================================================================================================
%
% Macros - Commandes
%
%================================================================================================================================

\newcommand\meta[2]{    			% Utilisé pour créer le post HTML.
	\def\titre{titre}
	\def\url{url}
	\def\arg{#1}
	\ifx\titre\arg
		\newcommand\maintitle{#2}
		\fancyhead[L]{#2}
		{\Large\sffamily \MakeUppercase{#2}}
		\vspace{1mm}\textcolor{mcvert}{\hrule}
	\fi 
	\ifx\url\arg
		\fancyfoot[L]{\href{https://www.maths-cours.fr#2}{\black \footnotesize{https://www.maths-cours.fr#2}}}
	\fi 
}


\newcommand\TitreC[1]{    		% Titre centré
     \needspace{3\baselineskip}
     \begin{center}\textbf{#1}\end{center}
}

\newcommand\newpar{    		% paragraphe
     \par
}

\newcommand\nosp {    		% commande vide (pas d'espace)
}
\newcommand{\id}[1]{} %ignore

\newcommand\boite[2]{				% Boite simple sans titre
	\vspace{5mm}
	\setlength{\fboxrule}{0.2mm}
	\setlength{\fboxsep}{5mm}	
	\fcolorbox{#1}{#1!3}{\makebox[\linewidth-2\fboxrule-2\fboxsep]{
  		\begin{minipage}[t]{\linewidth-2\fboxrule-4\fboxsep}\setlength{\parskip}{3mm}
  			 #2
  		\end{minipage}
	}}
	\vspace{5mm}
}

\newcommand\CBox[4]{				% Boites
	\vspace{5mm}
	\setlength{\fboxrule}{0.2mm}
	\setlength{\fboxsep}{5mm}
	
	\fcolorbox{#1}{#1!3}{\makebox[\linewidth-2\fboxrule-2\fboxsep]{
		\begin{minipage}[t]{1cm}\setlength{\parskip}{3mm}
	  		\textcolor{#1}{\LARGE{#2}}    
 	 	\end{minipage}  
  		\begin{minipage}[t]{\linewidth-2\fboxrule-4\fboxsep}\setlength{\parskip}{3mm}
			\raisebox{1.2mm}{\normalsize\sffamily{\textcolor{#1}{#3}}}						
  			 #4
  		\end{minipage}
	}}
	\vspace{5mm}
}

\newcommand\cadre[3]{				% Boites convertible html
	\par
	\vspace{2mm}
	\setlength{\fboxrule}{0.1mm}
	\setlength{\fboxsep}{5mm}
	\fcolorbox{#1}{white}{\makebox[\linewidth-2\fboxrule-2\fboxsep]{
  		\begin{minipage}[t]{\linewidth-2\fboxrule-4\fboxsep}\setlength{\parskip}{3mm}
			\raisebox{-2.5mm}{\sffamily \small{\textcolor{#1}{\MakeUppercase{#2}}}}		
			\par		
  			 #3
 	 		\end{minipage}
	}}
		\vspace{2mm}
	\par
}

\newcommand\bloc[3]{				% Boites convertible html sans bordure
     \needspace{2\baselineskip}
     {\sffamily \small{\textcolor{#1}{\MakeUppercase{#2}}}}    
		\par		
  			 #3
		\par
}

\newcommand\CHelp[1]{
     \CBox{Plum}{\faInfoCircle}{À RETENIR}{#1}
}

\newcommand\CUp[1]{
     \CBox{NavyBlue}{\faThumbsOUp}{EN PRATIQUE}{#1}
}

\newcommand\CInfo[1]{
     \CBox{Sepia}{\faArrowCircleRight}{REMARQUE}{#1}
}

\newcommand\CRedac[1]{
     \CBox{PineGreen}{\faEdit}{BIEN R\'EDIGER}{#1}
}

\newcommand\CError[1]{
     \CBox{Red}{\faExclamationTriangle}{ATTENTION}{#1}
}

\newcommand\TitreExo[2]{
\needspace{4\baselineskip}
 {\sffamily\large EXERCICE #1\ (\emph{#2 points})}
\vspace{5mm}
}

\newcommand\img[2]{
          \includegraphics[width=#2\paperwidth]{\imgdir#1}
}

\newcommand\imgsvg[2]{
       \begin{center}   \includegraphics[width=#2\paperwidth]{\imgsvgdir#1} \end{center}
}


\newcommand\Lien[2]{
     \href{#1}{#2 \tiny \faExternalLink}
}
\newcommand\mcLien[2]{
     \href{https~://www.maths-cours.fr/#1}{#2 \tiny \faExternalLink}
}

\newcommand{\euro}{\eurologo{}}

%================================================================================================================================
%
% Macros - Environement
%
%================================================================================================================================

\newenvironment{tex}{ %
}
{%
}

\newenvironment{indente}{ %
	\setlength\parindent{10mm}
}

{
	\setlength\parindent{0mm}
}

\newenvironment{corrige}{%
     \needspace{3\baselineskip}
     \medskip
     \textbf{\textsc{Corrigé}}
     \medskip
}
{
}

\newenvironment{extern}{%
     \begin{center}
     }
     {
     \end{center}
}

\NewEnviron{code}{%
	\par
     \boite{gray}{\texttt{%
     \BODY
     }}
     \par
}

\newenvironment{vbloc}{% boite sans cadre empeche saut de page
     \begin{minipage}[t]{\linewidth}
     }
     {
     \end{minipage}
}
\NewEnviron{h2}{%
    \needspace{3\baselineskip}
    \vspace{0.6cm}
	\noindent \MakeUppercase{\sffamily \large \BODY}
	\vspace{1mm}\textcolor{mcgris}{\hrule}\vspace{0.4cm}
	\par
}{}

\NewEnviron{h3}{%
    \needspace{3\baselineskip}
	\vspace{5mm}
	\textsc{\BODY}
	\par
}

\NewEnviron{margeneg}{ %
\begin{addmargin}[-1cm]{0cm}
\BODY
\end{addmargin}
}

\NewEnviron{html}{%
}

\begin{document}
\meta{url}{/cours/matrices/}
\meta{pid}{450}
\meta{titre}{Matrices [spé]}
\meta{type}{cours}
\begin{h2}1. Définitions\end{h2}
\cadre{bleu}{Définition}{%id="d10"
     Une \textbf{matrice} de dimension (ou d'\textit{ordre} or de \textit{taille}) $n\times p$ est un tableau de nombres réels (appelés coefficients ou termes) comportant $n$ lignes et $p$ colonnes.
     \par
     Si on désigne par $a_{ij}$ le coefficient situé à la $i$-ième ligne et la $j$-ième colonne la matrice s'écrira :
     \begin{center}$ A=\begin{pmatrix}  a_{11} & a_{12} & \ldots & a_{1p}\\ a_{21} & a_{22} & \ldots & a_{2p}  \\ \vdots & \vdots & \ddots & \vdots\\ a_{n1} & a_{n2} & \ldots & a_{np} \end{pmatrix}$\end{center}
}
\bloc{orange}{Exemple}{%id="e10"
     La matrice $A=\begin{pmatrix} 1 &amp; 2 &amp; 3 \\ 4 &amp; 5 &amp; 6 \end{pmatrix}$ est une matrice de dimension $2\times 3$
}
\bloc{cyan}{Notations}{%id="r10"
     On notera, en abrégé, $A=\left(a_{ij}\right)$ la matrice dont le coefficient situé à la $i$-ème ligne et la $j$-ième colonne est $a_{ij}$.
}
\cadre{bleu}{Définitions}{%id="d20"
     \begin{itemize}
          \item Une matrice \textbf{carrée} est une matrice dont le nombre de lignes est égal au nombre de colonnes.
          \item Une matrice \textbf{ligne} est une matrice dont le nombre de lignes est égal à $1$.
          \item Une matrice \textbf{colonne} est une matrice dont le nombre de colonnes est égal à $1$.
     \end{itemize}
}
\bloc{orange}{Exemples}{%id="e20"
     \begin{itemize}
          \item La matrice $A=\begin{pmatrix} 1 &amp; 2 \\ 1 &amp; 2 \end{pmatrix}$ est une matrice carrée (de dimension $2\times 2$ - ou on peut dire, plus simplement, de dimension 2)
          \item La matrice $B=\begin{pmatrix}1 &amp; 2 &amp; 0,5 \end{pmatrix}$ est une matrice ligne (de dimension $1\times 3$)
          \item La matrice $C=\begin{pmatrix} 1 \\ 2 \\ 0 \\ 4 \end{pmatrix}$ est une matrice colonne (de dimension $4\times 1$)
     \end{itemize}
}
\bloc{cyan}{Remarque}{%id="r20"
     Pour une matrice carrée, on appelle \textbf{diagonale principale}, la diagonale qui relie le coin situé en haut à gauche au coin situé en bas à droite. Sur l'exemple ci-dessous, les coefficients de la diagonale principale sont marqués en rouge :
     \begin{center}$A=\begin{pmatrix} \color{red}{1} &amp; 2 &amp; 3 &amp; 4 \\ 2 &amp; \color{red}{3} &amp; 4 &amp; 5 \\ 3 &amp; 4 &amp; \color{red}{5} &amp; 6 \\ 4 &amp; 5 &amp; 6 &amp; \color{red}{7} \end{pmatrix}$\end{center}
}
\cadre{bleu}{Définitions}{%id="d30"
     \begin{itemize}
          \item La matrice \textbf{nulle} de dimension $n\times p$ est la matrice de dimension $n\times p$ dont tous les coefficients sont nuls
          \item Une matrice \textbf{diagonale} est une matrice carrée dont tout les coefficients situés en dehors de la diagonale principale sont nuls.
          \item La matrice \textbf{unité} de dimension $n$ est la matrice carrée de dimension $n$ qui contient des $1$ sur la diagonale principale et des $0$ ailleurs :
          \begin{center}$ A=\begin{pmatrix}  1 & 0 & \ldots & 0\\ 0 & 1 & \ldots & 0\\ \vdots & \vdots & \ddots & \vdots\\ 0 & 0 & \ldots & 1 \end{pmatrix}$\end{center}
     \end{itemize}
}
\bloc{orange}{Exemples}{%id="e30"
     \begin{itemize}
          \item La matrice $A=\begin{pmatrix} 1 &amp; 0 &amp; 0 &amp; 0 \\ 0 &amp; 2 &amp; 0 &amp; 0 \\ 0 &amp; 0 &amp; 0 &amp; 0 \\ 0 &amp; 0 &amp; 0 &amp; 1 \end{pmatrix}$ est une matrice diagonale d'ordre 4.
          \item La matrice unité d'ordre 2 est $I_{2}=\begin{pmatrix} 1 &amp; 0 \\ 0 &amp; 1 \end{pmatrix}$.
     \end{itemize}
}
\begin{h2}2. Opérations sur les matrices\end{h2}
\cadre{bleu}{Définition (Somme de matrices)}{%id="d50"
     Soient $A$ et $B$ deux matrices de même dimension.
     \par
     La somme $A+B$ des matrices $A$ et $B$ s'obtient en ajoutant les coefficients de $A$ aux coefficients de $B$ situés \textbf{à la même position}.
}
\bloc{orange}{Exemple}{%id="e50"
     Soient $A=\begin{pmatrix} 2 &amp; -2 &amp; 1 \\ -1 &amp; 1 &amp; 0 \end{pmatrix}$ et $B=\begin{pmatrix} -1 &amp; 1 &amp; 1 \\ -2 &amp; 2 &amp; 0 \end{pmatrix}$.
     \par
     Alors :
     \par
     $A+B=\begin{pmatrix}2-1&amp;-2+1&amp;1+1\\-1-2&amp;1+2&amp;0+0\end{pmatrix}=\begin{pmatrix}1&amp;-1&amp;2\\-3&amp;3&amp;0\end{pmatrix}$
}
\bloc{cyan}{Remarques}{%id="r50"
     \begin{itemize}
          \item On ne peut additionner deux matrices que si elles ont les même dimensions, c'est à dire le même nombre de lignes et le même nombre de colonnes.
          \item On définit de manière analogue la différence de deux matrices.
     \end{itemize}
}
\cadre{bleu}{Définition (Produit d'une matrice par un nombre réel)}{%id="d60"
     Soient $A$ une matrice et $k$ un nombre réel..
     \par
     Le produit $kA$ est la matrice obtenue en multipliant chacun des coefficients de $A$ par $k$.
}
\bloc{orange}{Exemple}{%id="e60"
     Si $A=\begin{pmatrix} 1 &amp; 1 &amp; 0 \\ 2 &amp; 0 &amp; 0 \end{pmatrix}$ alors :
     \begin{itemize}
          \item $2A=\begin{pmatrix} 2\times 1 &amp; 2\times 1 &amp; 2\times 0 \\ 2\times 2 &amp; 2\times 0 &amp; 2\times 0\end{pmatrix}=\begin{pmatrix}2 &amp; 2 &amp; 0 \\ 4 &amp; 0 &amp; 0\end{pmatrix}$
          \item $-A=-1\times A=\begin{pmatrix} -1 &amp; -1 &amp; 0 \\ -2 &amp; 0 &amp; 0 \end{pmatrix}$
     \end{itemize}
}
\cadre{vert}{Propriétés}{%id="p65"
     Soient $A$, $B$ et $C$ trois matrices de mêmes dimensions et $k$ et $k^{\prime}$ deux réels.
     \begin{itemize}
          \item $A+B = B+A $ (commutativité de l'addition)
          \item $\left(A+B\right)+C = A+\left(B+C\right)$ (associativité de l'addition)
          \item $k\left(A+B\right) = kA+kB$
          \item $\left(k+k^{\prime}\right)A = kA+k^{\prime}A$
          \item $k\left(k^{\prime}A\right) = \left(kk^{\prime}\right)A$
     \end{itemize}
}
\cadre{bleu}{Définition (Produit d'une matrice ligne par une matrice colonne)}{%id="d70"
     Soient $A=\left(a_{1} a_{2} \cdots a_{n}\right)$ une matrice ligne $1\times n$ et $B=\begin{pmatrix} b_{1} \\ b_{2} \\ \cdots \\ b_{n} \end{pmatrix}$ une matrice colonne $n\times 1$. Le produit de $A$ par $B$ est le nombre réel :
     \par
     $A\times B = \left(a_{1} a_{2} \cdots a_{n}\right)\times \begin{pmatrix} b_{1} \\ b_{2} \\ \cdots \\ b_{n} \end{pmatrix} = a_{1}b_{1} + a_{2}b_{2} + \cdots + a_{n}b_{n}$
}
\bloc{cyan}{Remarque}{%id="r70"
     \begin{itemize}
          \item Les deux matrices $A$ et $B$ doivent avoir le même nombre $n$ de coefficients.
          \item Pour cette formule, la matrice ligne doit être impérativement en premier !
     \end{itemize}
}
\bloc{orange}{Exemple}{%id="e70"
     Si $ A=\left(1 2 3 4\right) $ et $ B=\begin{pmatrix} 5 \\ 6 \\ 7 \\ 8 \end{pmatrix}$
     \par
     $A\times B = 1\times 5 + 2\times 6 + 3\times 7 + 4\times 8 = 5 + 12 + 21 + 32 = 70$
}
\cadre{bleu}{Définition (Produit de deux matrices)}{%id="d80"
     Soient $A=\left(a_{ij}\right)$ une matrice $n\times p$ et $B=\left(b_{ij}\right)$ une matrice $p\times q$. Le produit de $A$ par $B$ est la matrice $C=\left(c_{ij}\right)$ à $n$ lignes et $q$ colonnes dont le coefficient situé à la $i$-ième ligne et la $j$-ième colonne est obtenu en multipliant la $i$-ième ligne de A par la $j$-ième colonne de B.
     \par
     C'est à dire que pour tout $1 \leqslant i \leqslant n$ et tout $1 \leqslant j \leqslant q$ :
     \begin{center}$c_{ij} = a_{i1}b_{1j} + a_{i2}b_{2j} + \cdots + a_{ip}b_{pj}$\end{center}
}
\bloc{cyan}{Remarque}{%id="r80"
     Faites bien attention aux dimensions des matrices : Le nombre de colonnes de la première matrice doit être égal au nombre de lignes de la seconde pour que le calcul soit possible.
     \par
     Par exemple, le produit d'une matrice $2\times \color{red}{3}$ par une matrice $\color{red}{3}\times 4$ est possible et donnera une matrice $2\times 4$.
     \par
     Par contre, le produit d'une matrice $2\times \color{red}{3}$ par une matrice $\color{red}{2}\times 3$ n'est pas possible.
}
\bloc{orange}{Exemple}{%id="e80"
     Calculons le produit $C=A\times B$ avec :
     \par
     $A=\begin{pmatrix} 2 &amp; 4 \\ 1 &amp; 0 \end{pmatrix} $ et $ B=\begin{pmatrix} -1 &amp; 0 &amp; 2 \\ -2 &amp; 1 &amp; 0 \end{pmatrix}$
     \par
     Ce calcul est possible car le nombre de colonnes de $A$ est égal au nombre de lignes de $B$. Le résultat $C$ sera une matrice $2\times 3$ ($\color{red}{2}\times 2 $par$ 2\times \color{red}{3} \rightarrow \color{red}{2}\times \color{red}{3}$)
     \par
     Notons $C=\begin{pmatrix} c_{11} &amp; c_{12} &amp; c_{13} \\ c_{21} &amp; c_{22} &amp; c_{23} \end{pmatrix}$
     \par
     Pour calculer $c_{11}$ on multiplie la première ligne de $A$ et la première colonne de $B$ :
     \par
     $C=\begin{pmatrix} \color{red}{2} &amp; \color{red}{4} \\ 1 &amp; 0\end{pmatrix}\times \begin{pmatrix} \color{red}{-1} &amp; 0 &amp; 2 \\ \color{red}{-2} &amp; 1 &amp; 0\end{pmatrix}$
     \par
     on a donc $c_{11}=2\times \left(-1\right)+4\times \left(-2\right)=-2-8=-10$
     \par
     $C=\begin{pmatrix} \color{red}{2} &amp; \color{red}{4} \\ 1 &amp; 0 \end{pmatrix}\times \begin{pmatrix}\color{red}{-1} &amp; 0 &amp; 2 \\ \color{red}{-2} &amp; 1 &amp; 0 \end{pmatrix}=\begin{pmatrix}\color{red}{-10} &amp; \cdots &amp; \cdots \\ \cdots &amp; \cdots &amp; \cdots \end{pmatrix}$
     \par
     Pour calculer $c_{12}$ on multiplie la première ligne de $A$ et la seconde colonne de $B$ :
     \par
     $C=\begin{pmatrix} \color{red}{2} &amp; \color{red}{4} \\ 1 &amp; 0\end{pmatrix}\times\begin{pmatrix}-1 &amp; \color{red}{0} &amp; 2 \\ -2 &amp; \color{red}{1} &amp; 0\end{pmatrix}$
     \par
     on a donc $c_{12}=2\times 0+4\times 1=0+4=4$
     \par
     $C=\begin{pmatrix} \color{red}{2} &amp; \color{red}{4} \\ 1 &amp; 0\end{pmatrix}\times \begin{pmatrix} -1 &amp; \color{red}{0} &amp; 2 \\ -2 &amp; \color{red}{1} &amp; 0\end{pmatrix}=\begin{pmatrix}-10 &amp; \color{red}{4} &amp; \cdots \\ \cdots &amp; \cdots &amp; \cdots \end{pmatrix}$
     \par
     Et ainsi de suite...
     \par
     Au final on trouve :
     \par
     $C=\begin{pmatrix} 2 &amp; 4 \\ 1 &amp; 0\end{pmatrix}\times \begin{pmatrix}-1 &amp; 0 &amp; 2 \\ -2 &amp; 1 &amp; 0\end{pmatrix}=\begin{pmatrix}-10 &amp; 4 &amp; 4 \\ -1 &amp; 0 &amp; 2 \end{pmatrix}$
}
Dans ce qui suit, on s'intéressera principalement à des matrices \textbf{carrées}.
\cadre{vert}{Propriété}{%id="p90"
     Soit $A, B$ et $C$, trois matrices carrées de même dimension.
     \begin{itemize}
          \item $A\times \left(B+C\right) = A\times B + A\times C$ (distributivité à gauche)
          \item $\left(A+B\right)\times C = A\times C + B\times C$ (distributivité à droite)
          \item $A\times \left(B\times C\right) = \left(A\times B\right)\times C$ (associativité de la multiplication)
     \end{itemize}
     Par contre en général : $A\times B\neq B\times A$ : la multiplication n'est \textbf{pas} commutative.
}
\bloc{orange}{Exemple}{%id="e90"
     Soit $A=\begin{pmatrix} 0 &amp; 2 \\ 0 &amp; 0 \end{pmatrix}$ et $B=\begin{pmatrix} 0 &amp; 2 \\ 1 &amp; 0 \end{pmatrix}$
     \par
     $A \times B=\begin{pmatrix} 2 &amp; 0 \\ 0 &amp; 0 \end{pmatrix}$
     \par
     tandis que :
     \par
     $B \times A=\begin{pmatrix} 0 &amp; 0 \\ 0 &amp; 2 \end{pmatrix}$
     \par
     Par conséquent $A\times B \neq B\times A$
}
\cadre{bleu}{Définition (Puissance d'une matrice)}{%id="d100"
     Soit $A$ une matrice carrée et $n$ un entier naturel.
     \par
     On note $A^{n}$ la matrice :
     \begin{center}$A^{n}=A\times A\times \cdots.\times A$ ($n$ facteurs)\end{center}
}
\bloc{cyan}{Remarque}{%id="r100"
     Par convention, on considèrera que $A^{0}$ est la matrice unité de même taille que $A$
}
\cadre{bleu}{Définition (Matrice inversible)}{%id="d110"
     Une matrice carrée A de dimension $n$ est \textbf{inversible} si et seulement si il existe une
     \par
     matrice $B$ telle que
     \begin{center}$A\times B = B\times A = I_{n}$\end{center}
     où $I_{n}$ est la matrice unité de dimension $n$
     \par
     La matrice $B$ est appelée \textbf{matrice inverse} de $A$ et notée $A^{-1}$.
}
\begin{h2}3. Résolution de systèmes d'équations\end{h2}
Soit le système :
\par
$\left(S\right) \left\{ \begin{matrix} ax+by=s \\ cx+dy=t \end{matrix}\right.$
\par
d'inconnues $x$ et $y$.
\par
Si l'on pose $A=\begin{pmatrix} a &amp; b \\ c &amp; d \end{pmatrix}$, $X=\begin{pmatrix} x \\ y \end{pmatrix}$ et $B=\begin{pmatrix} s \\ t \end{pmatrix}$, le système $\left(S\right)$ peut s'écrire :
\par
$A\times X=B$. Le théorème ci-dessous permet alors de résoudre ce système.
\cadre{rouge}{Théorème}{%id="t150"
     Soit $A$ une matrice carrée.
     \par
     Si $A$ est inversible, le système $A\times X=B$ admet une solution unique donnée par :
     \begin{center}$X=A^{-1}\times B$\end{center}
}
\bloc{orange}{Exemple}{%id="e150"
     On cherche à résoudre le système :
     \par
     $\left(S\right) \left\{ \begin{matrix} 3x+4y=1 \\ 5x+7y=2 \end{matrix}\right.$
     \par
     Pour cela on pose : $A=\begin{pmatrix} 3 &amp; 4 \\ 5 &amp; 7 \end{pmatrix}$, $X=\begin{pmatrix} x \\ y \end{pmatrix}$ et $B=\begin{pmatrix} 1 \\ 2 \end{pmatrix}$.
     \par
     L'écriture matricielle est alors $A\times X=B$
     \par
     A la calculatrice, on trouve que $A$ est inversible d'inverse $A^{-1}=\begin{pmatrix} 7 &amp; -4 \\ -5 &amp; 3 \end{pmatrix}$.
     \par
     La solution du système est donné par :
     \par
     $X=A^{-1}\times B=\begin{pmatrix} 7 &amp; -4 \\ -5 &amp; 3\end{pmatrix}\times \begin{pmatrix}1 \\ 2\end{pmatrix}=\begin{pmatrix}-1 \\ 1 \end{pmatrix}$
     \par
     C'est à dire $x=-1$ et $y=1$.
}

\end{document}
µ
\documentclass[a4paper]{article}

%================================================================================================================================
%
% Packages
%
%================================================================================================================================

\usepackage[T1]{fontenc} 	% pour caractères accentués
\usepackage[utf8]{inputenc}  % encodage utf8
\usepackage[french]{babel}	% langue : français
\usepackage{fourier}			% caractères plus lisibles
\usepackage[dvipsnames]{xcolor} % couleurs
\usepackage{fancyhdr}		% réglage header footer
\usepackage{needspace}		% empêcher sauts de page mal placés
\usepackage{graphicx}		% pour inclure des graphiques
\usepackage{enumitem,cprotect}		% personnalise les listes d'items (nécessaire pour ol, al ...)
\usepackage{hyperref}		% Liens hypertexte
\usepackage{pstricks,pst-all,pst-node,pstricks-add,pst-math,pst-plot,pst-tree,pst-eucl} % pstricks
\usepackage[a4paper,includeheadfoot,top=2cm,left=3cm, bottom=2cm,right=3cm]{geometry} % marges etc.
\usepackage{comment}			% commentaires multilignes
\usepackage{amsmath,environ} % maths (matrices, etc.)
\usepackage{amssymb,makeidx}
\usepackage{bm}				% bold maths
\usepackage{tabularx}		% tableaux
\usepackage{colortbl}		% tableaux en couleur
\usepackage{fontawesome}		% Fontawesome
\usepackage{environ}			% environment with command
\usepackage{fp}				% calculs pour ps-tricks
\usepackage{multido}			% pour ps tricks
\usepackage[np]{numprint}	% formattage nombre
\usepackage{tikz,tkz-tab} 			% package principal TikZ
\usepackage{pgfplots}   % axes
\usepackage{mathrsfs}    % cursives
\usepackage{calc}			% calcul taille boites
\usepackage[scaled=0.875]{helvet} % font sans serif
\usepackage{svg} % svg
\usepackage{scrextend} % local margin
\usepackage{scratch} %scratch
\usepackage{multicol} % colonnes
%\usepackage{infix-RPN,pst-func} % formule en notation polanaise inversée
\usepackage{listings}

%================================================================================================================================
%
% Réglages de base
%
%================================================================================================================================

\lstset{
language=Python,   % R code
literate=
{á}{{\'a}}1
{à}{{\`a}}1
{ã}{{\~a}}1
{é}{{\'e}}1
{è}{{\`e}}1
{ê}{{\^e}}1
{í}{{\'i}}1
{ó}{{\'o}}1
{õ}{{\~o}}1
{ú}{{\'u}}1
{ü}{{\"u}}1
{ç}{{\c{c}}}1
{~}{{ }}1
}


\definecolor{codegreen}{rgb}{0,0.6,0}
\definecolor{codegray}{rgb}{0.5,0.5,0.5}
\definecolor{codepurple}{rgb}{0.58,0,0.82}
\definecolor{backcolour}{rgb}{0.95,0.95,0.92}

\lstdefinestyle{mystyle}{
    backgroundcolor=\color{backcolour},   
    commentstyle=\color{codegreen},
    keywordstyle=\color{magenta},
    numberstyle=\tiny\color{codegray},
    stringstyle=\color{codepurple},
    basicstyle=\ttfamily\footnotesize,
    breakatwhitespace=false,         
    breaklines=true,                 
    captionpos=b,                    
    keepspaces=true,                 
    numbers=left,                    
xleftmargin=2em,
framexleftmargin=2em,            
    showspaces=false,                
    showstringspaces=false,
    showtabs=false,                  
    tabsize=2,
    upquote=true
}

\lstset{style=mystyle}


\lstset{style=mystyle}
\newcommand{\imgdir}{C:/laragon/www/newmc/assets/imgsvg/}
\newcommand{\imgsvgdir}{C:/laragon/www/newmc/assets/imgsvg/}

\definecolor{mcgris}{RGB}{220, 220, 220}% ancien~; pour compatibilité
\definecolor{mcbleu}{RGB}{52, 152, 219}
\definecolor{mcvert}{RGB}{125, 194, 70}
\definecolor{mcmauve}{RGB}{154, 0, 215}
\definecolor{mcorange}{RGB}{255, 96, 0}
\definecolor{mcturquoise}{RGB}{0, 153, 153}
\definecolor{mcrouge}{RGB}{255, 0, 0}
\definecolor{mclightvert}{RGB}{205, 234, 190}

\definecolor{gris}{RGB}{220, 220, 220}
\definecolor{bleu}{RGB}{52, 152, 219}
\definecolor{vert}{RGB}{125, 194, 70}
\definecolor{mauve}{RGB}{154, 0, 215}
\definecolor{orange}{RGB}{255, 96, 0}
\definecolor{turquoise}{RGB}{0, 153, 153}
\definecolor{rouge}{RGB}{255, 0, 0}
\definecolor{lightvert}{RGB}{205, 234, 190}
\setitemize[0]{label=\color{lightvert}  $\bullet$}

\pagestyle{fancy}
\renewcommand{\headrulewidth}{0.2pt}
\fancyhead[L]{maths-cours.fr}
\fancyhead[R]{\thepage}
\renewcommand{\footrulewidth}{0.2pt}
\fancyfoot[C]{}

\newcolumntype{C}{>{\centering\arraybackslash}X}
\newcolumntype{s}{>{\hsize=.35\hsize\arraybackslash}X}

\setlength{\parindent}{0pt}		 
\setlength{\parskip}{3mm}
\setlength{\headheight}{1cm}

\def\ebook{ebook}
\def\book{book}
\def\web{web}
\def\type{web}

\newcommand{\vect}[1]{\overrightarrow{\,\mathstrut#1\,}}

\def\Oij{$\left(\text{O}~;~\vect{\imath},~\vect{\jmath}\right)$}
\def\Oijk{$\left(\text{O}~;~\vect{\imath},~\vect{\jmath},~\vect{k}\right)$}
\def\Ouv{$\left(\text{O}~;~\vect{u},~\vect{v}\right)$}

\hypersetup{breaklinks=true, colorlinks = true, linkcolor = OliveGreen, urlcolor = OliveGreen, citecolor = OliveGreen, pdfauthor={Didier BONNEL - https://www.maths-cours.fr} } % supprime les bordures autour des liens

\renewcommand{\arg}[0]{\text{arg}}

\everymath{\displaystyle}

%================================================================================================================================
%
% Macros - Commandes
%
%================================================================================================================================

\newcommand\meta[2]{    			% Utilisé pour créer le post HTML.
	\def\titre{titre}
	\def\url{url}
	\def\arg{#1}
	\ifx\titre\arg
		\newcommand\maintitle{#2}
		\fancyhead[L]{#2}
		{\Large\sffamily \MakeUppercase{#2}}
		\vspace{1mm}\textcolor{mcvert}{\hrule}
	\fi 
	\ifx\url\arg
		\fancyfoot[L]{\href{https://www.maths-cours.fr#2}{\black \footnotesize{https://www.maths-cours.fr#2}}}
	\fi 
}


\newcommand\TitreC[1]{    		% Titre centré
     \needspace{3\baselineskip}
     \begin{center}\textbf{#1}\end{center}
}

\newcommand\newpar{    		% paragraphe
     \par
}

\newcommand\nosp {    		% commande vide (pas d'espace)
}
\newcommand{\id}[1]{} %ignore

\newcommand\boite[2]{				% Boite simple sans titre
	\vspace{5mm}
	\setlength{\fboxrule}{0.2mm}
	\setlength{\fboxsep}{5mm}	
	\fcolorbox{#1}{#1!3}{\makebox[\linewidth-2\fboxrule-2\fboxsep]{
  		\begin{minipage}[t]{\linewidth-2\fboxrule-4\fboxsep}\setlength{\parskip}{3mm}
  			 #2
  		\end{minipage}
	}}
	\vspace{5mm}
}

\newcommand\CBox[4]{				% Boites
	\vspace{5mm}
	\setlength{\fboxrule}{0.2mm}
	\setlength{\fboxsep}{5mm}
	
	\fcolorbox{#1}{#1!3}{\makebox[\linewidth-2\fboxrule-2\fboxsep]{
		\begin{minipage}[t]{1cm}\setlength{\parskip}{3mm}
	  		\textcolor{#1}{\LARGE{#2}}    
 	 	\end{minipage}  
  		\begin{minipage}[t]{\linewidth-2\fboxrule-4\fboxsep}\setlength{\parskip}{3mm}
			\raisebox{1.2mm}{\normalsize\sffamily{\textcolor{#1}{#3}}}						
  			 #4
  		\end{minipage}
	}}
	\vspace{5mm}
}

\newcommand\cadre[3]{				% Boites convertible html
	\par
	\vspace{2mm}
	\setlength{\fboxrule}{0.1mm}
	\setlength{\fboxsep}{5mm}
	\fcolorbox{#1}{white}{\makebox[\linewidth-2\fboxrule-2\fboxsep]{
  		\begin{minipage}[t]{\linewidth-2\fboxrule-4\fboxsep}\setlength{\parskip}{3mm}
			\raisebox{-2.5mm}{\sffamily \small{\textcolor{#1}{\MakeUppercase{#2}}}}		
			\par		
  			 #3
 	 		\end{minipage}
	}}
		\vspace{2mm}
	\par
}

\newcommand\bloc[3]{				% Boites convertible html sans bordure
     \needspace{2\baselineskip}
     {\sffamily \small{\textcolor{#1}{\MakeUppercase{#2}}}}    
		\par		
  			 #3
		\par
}

\newcommand\CHelp[1]{
     \CBox{Plum}{\faInfoCircle}{À RETENIR}{#1}
}

\newcommand\CUp[1]{
     \CBox{NavyBlue}{\faThumbsOUp}{EN PRATIQUE}{#1}
}

\newcommand\CInfo[1]{
     \CBox{Sepia}{\faArrowCircleRight}{REMARQUE}{#1}
}

\newcommand\CRedac[1]{
     \CBox{PineGreen}{\faEdit}{BIEN R\'EDIGER}{#1}
}

\newcommand\CError[1]{
     \CBox{Red}{\faExclamationTriangle}{ATTENTION}{#1}
}

\newcommand\TitreExo[2]{
\needspace{4\baselineskip}
 {\sffamily\large EXERCICE #1\ (\emph{#2 points})}
\vspace{5mm}
}

\newcommand\img[2]{
          \includegraphics[width=#2\paperwidth]{\imgdir#1}
}

\newcommand\imgsvg[2]{
       \begin{center}   \includegraphics[width=#2\paperwidth]{\imgsvgdir#1} \end{center}
}


\newcommand\Lien[2]{
     \href{#1}{#2 \tiny \faExternalLink}
}
\newcommand\mcLien[2]{
     \href{https~://www.maths-cours.fr/#1}{#2 \tiny \faExternalLink}
}

\newcommand{\euro}{\eurologo{}}

%================================================================================================================================
%
% Macros - Environement
%
%================================================================================================================================

\newenvironment{tex}{ %
}
{%
}

\newenvironment{indente}{ %
	\setlength\parindent{10mm}
}

{
	\setlength\parindent{0mm}
}

\newenvironment{corrige}{%
     \needspace{3\baselineskip}
     \medskip
     \textbf{\textsc{Corrigé}}
     \medskip
}
{
}

\newenvironment{extern}{%
     \begin{center}
     }
     {
     \end{center}
}

\NewEnviron{code}{%
	\par
     \boite{gray}{\texttt{%
     \BODY
     }}
     \par
}

\newenvironment{vbloc}{% boite sans cadre empeche saut de page
     \begin{minipage}[t]{\linewidth}
     }
     {
     \end{minipage}
}
\NewEnviron{h2}{%
    \needspace{3\baselineskip}
    \vspace{0.6cm}
	\noindent \MakeUppercase{\sffamily \large \BODY}
	\vspace{1mm}\textcolor{mcgris}{\hrule}\vspace{0.4cm}
	\par
}{}

\NewEnviron{h3}{%
    \needspace{3\baselineskip}
	\vspace{5mm}
	\textsc{\BODY}
	\par
}

\NewEnviron{margeneg}{ %
\begin{addmargin}[-1cm]{0cm}
\BODY
\end{addmargin}
}

\NewEnviron{html}{%
}

\begin{document}
\meta{url}{/cours/graphes/}
\meta{pid}{474}
\meta{titre}{Graphes : Généralités [spé]}
\meta{type}{cours}
\begin{h2}1. Vocabulaire\end{h2}
\cadre{bleu}{Définition}{%
     Un \textbf{graphe} est composé de \textbf{sommets} et d'\textbf{arêtes} (ou \textbf{arcs}) reliant certains de ces sommets.
}
\bloc{orange}{Exemple}{%
     Le diagramme ci-dessous représente un graphe comportant 4 sommets et 5 arêtes.
     \begin{center}
          \begin{extern}%width="350"
               \psset{unit=1.5cm}
               \begin{pspicture}(-1,0)(6,3)
                    \rput(0,2){\circlenode{A}{A}}
                    \rput(2,2){\circlenode{B}{B}}
                    \rput(5,2){\circlenode{C}{C}}
                    \rput(3.5,0.5){\circlenode{D}{D}}
                    \ncarc[arcangle=-20]{A}{B}
                    \ncarc[arcangle=-20]{B}{C}
                    \ncarc[arcangle=-20]{B}{D}
                    \ncarc[arcangle=20]{C}{D}
                    \nccircle[angleA=-50]{C}{.5cm}
               \end{pspicture}
          \end{extern}
     \end{center}
}
\cadre{bleu}{Définitions}{%
     \begin{itemize}
          \item L'\textbf{ordre} d'un graphe est le nombre de sommets de ce graphe.
          \item Le \textbf{degré} d'un sommet est le nombre d'arêtes dont ce sommet est une extrémité.
          \item Deux sommets reliés par une arête sont \textbf{adjacents}.
     \end{itemize}
}
\bloc{orange}{Exemple}{%
     \begin{itemize}
          \item Le graphe représenté ci-dessus est d'ordre 4.
          \item Le degré du sommet B est 3. Celui de C est 4 (la boucle compte 2 fois).
          \item A et B sont adjacents. A et D ne le sont pas.
     \end{itemize}
}
\cadre{bleu}{Définitions}{%
     Une \textbf{chaîne} (ou un \textbf{chemin}) est une suite de sommets telle que chaque sommet est relié au suivant par une arête.
     \par
     La \textbf{longueur} d'une chaîne est le nombre d'arêtes composant cette chaîne.
}
\bloc{orange}{Exemple}{%
     \begin{center}
          \begin{extern}%width="350"
               \psset{unit=1.5cm}
               \begin{pspicture}(-1,0)(6,3)
                    \rput(0,2){\circlenode{A}{A}}
                    \rput(2,2){\circlenode{B}{B}}
                    \rput(5,2){\circlenode{C}{C}}
                    \rput(3.5,0.5){\circlenode{D}{D}}
                    \ncarc[linecolor=red,arcangle=-20]{A}{B}
                    \ncarc[linecolor=red,arcangle=-20]{B}{C}
                    \ncarc[arcangle=-20]{B}{D}
                    \ncarc[linecolor=red,arcangle=20]{C}{D}
                    \nccircle[angleA=-50]{C}{.5cm}
               \end{pspicture}
          \end{extern}
     \end{center}
     (A; B; C; D) est une chaîne de longueur 3.
}
\cadre{bleu}{Définition}{%
     Un \textbf{cycle} est une chaîne \textbf{fermée} (c'est à dire dont l'origine et l'extrémité sont identiques) dont toutes les arêtes sont distinctes.
}
\bloc{orange}{Exemple}{%
     \begin{center}
          \begin{extern}%width="350"
               \psset{unit=1.5cm}
               \begin{pspicture}(-1,0)(6,3)
                    \rput(0,2){\circlenode{A}{A}}
                    \rput(2,2){\circlenode{B}{B}}
                    \rput(5,2){\circlenode{C}{C}}
                    \rput(3.5,0.5){\circlenode{D}{D}}
                    \ncarc[arcangle=-20]{A}{B}
                    \ncarc[linecolor=red,arcangle=-20]{B}{C}
                    \ncarc[linecolor=red,arcangle=-20]{B}{D}
                    \ncarc[linecolor=red,arcangle=20]{C}{D}
                    \nccircle[linecolor=red,angleA=-50]{C}{.5cm}
               \end{pspicture}
          \end{extern}
     \end{center}     (B; C; C; D; B) est un cycle.
}
\cadre{bleu}{Définition}{%
     On dit qu'un graphe est \textbf{connexe} si deux sommets quelconques peuvent être reliés par une chaîne.
}
\bloc{cyan}{Remarque}{%
     Intuitivement, cela signifie que le graphe comporte un seul "morceau"
}
\bloc{orange}{Exemple}{%
     \begin{vbloc}
          \begin{center}
               \begin{extern}%width="300"
                    \psset{unit=1.5cm}
                    \begin{pspicture}(-1,0)(6,3)
                         \rput(0,2){\circlenode{A}{A}}
                         \rput(2,2){\circlenode{B}{B}}
                         \rput(5,2){\circlenode{C}{C}}
                         \rput(3.5,0.5){\circlenode{D}{D}}
                         \ncarc[arcangle=-20]{A}{B}
                         \ncarc[arcangle=-20]{B}{C}
                         \ncarc[arcangle=-20]{B}{D}
                         \ncarc[arcangle=20]{C}{D}
                         \nccircle[angleA=-50]{C}{.5cm}
                    \end{pspicture}
               \end{extern}
          \end{center}
          \begin{center}
               Graphe connexe
          \end{center}
     \end{vbloc}
     \begin{vbloc}
          \begin{center}
               \begin{extern}%width="300"
                    \psset{unit=1.5cm}
                    \begin{pspicture}(-1,0)(6,3)
                         \rput(0,2){\circlenode{A}{A}}
                         \rput(2,2){\circlenode{B}{B}}
                         \rput(5,2){\circlenode{C}{C}}
                         \rput(3.5,0.5){\circlenode{D}{D}}
                         \ncarc[arcangle=-20]{A}{B}
                         \ncarc[arcangle=20]{C}{D}
                         \nccircle[angleA=-50]{C}{.5cm}
                    \end{pspicture}
               \end{extern}
          \end{center}
          \begin{center}
               Graphe non connexe
          \end{center}
     \end{vbloc}
}
\begin{h2}2. Chaînes et cycles eulériens\end{h2}
\cadre{bleu}{Définition}{%
     Une \textbf{chaîne eulérienne} est une chaîne qui contient une fois et une seule chacune des arêtes du graphe.
     \par
     Si cette chaîne est un cycle, on parle de \textbf{cycle eulérien}.
}
\bloc{orange}{Exemple}{%
     \begin{center}
          \begin{extern}%width="350"
               \psset{unit=1.5cm}
               \begin{pspicture}(-1,0)(6,3)
                    \rput(0,2){\circlenode{A}{A}}
                    \rput(2,2){\circlenode{B}{B}}
                    \rput(5,2){\circlenode{C}{C}}
                    \rput(3.5,0.5){\circlenode{D}{D}}
                    \ncarc[arcangle=-20]{A}{B}
                    \ncarc[arcangle=-20]{B}{C}
                    \ncarc[arcangle=-20]{B}{D}
                    \ncarc[arcangle=20]{C}{D}
                    \nccircle[angleA=-50]{C}{.5cm}
               \end{pspicture}
          \end{extern}
     \end{center}
     (A; B; C; C; D; B) est une chaîne eulérienne.
     \par
     Ce graphe ne contient aucun cycle eulérien.
}
\bloc{cyan}{Remarque}{%
     \begin{itemize}
          \item Un graphe connexe contient une chaîne eulérienne si et seulement si on peut le tracer "\textit{sans lever le crayon}". Le théorème d'Euler (ci-dessous) permet de déterminer facilement ce type de graphe.
          \item On ne peut jamais tracer un graphe \textbf{non connexe} sans lever le crayon !
     \end{itemize}
}
\cadre{rouge}{Théorème}{%
     \textbf{Théorème d'Euler.}
     Un graphe connexe contient une chaîne eulérienne si et seulement si il possède 0 ou 2 sommets de degré impair.
     \par
     Un graphe connexe contient un cycle eulérien si et seulement si il ne possède aucun sommet de degré impair (autrement dit tous ses sommets sont de degré pair)
}
\bloc{orange}{Exemples}{%
     %     <img src="/wp-content/uploads/th_euler.png" alt="" class="aligncenter size-full  img-bpc" />
     \begin{vbloc}
          \begin{center}
               \begin{extern}%width="350"
                    \psset{unit=1.5cm}
                    \begin{pspicture}(-1,0)(6,3)
                         \rput(0,2){\circlenode{A}{A}}
                         \rput(2,2){\circlenode{B}{B}}
                         \rput(5,2){\circlenode{C}{C}}
                         \rput(3.5,0.5){\circlenode{D}{D}}
                         \ncarc[arcangle=-20]{A}{B}
                         \ncarc[arcangle=-20]{B}{C}
                         \ncarc[arcangle=-20]{B}{D}
                         \ncarc[arcangle=20]{C}{D}
                         \nccircle[angleA=-50]{C}{.5cm}
                    \end{pspicture}
               \end{extern}
          \end{center}
          \begin{center}
               \textit{Exemple 1}
          \end{center}
     \end{vbloc}
     Dans l'\textit{exemple 1}, il y a deux sommets de degré impair (A:1 et B:3). Le graphe contient une chaîne eulérienne, par exemple (A; B; C; C; D; B) mais pas de cycle eulérien.
     \par
     \begin{vbloc}
          \begin{center}
               \begin{extern}%width="280"
                    \psset{unit=1.5cm}
                    \begin{pspicture}(-1,0)(6,3)
                         \rput(0,0){\circlenode{A}{A}}
                         \rput(0,2){\circlenode{B}{B}}
                         \rput(2,3){\circlenode{C}{C}}
                         \rput(4,2){\circlenode{D}{D}}
                         \rput(4,0){\circlenode{E}{E}}
                         \rput(2,1){\circlenode{F}{F}}
                         \ncline{A}{B} \ncline{B}{C}\ncline{C}{D}\ncline{D}{E}\ncline{E}{A}\ncline{B}{D}
                         \ncline{A}{F} \ncline{B}{F}\ncline{F}{D}\ncline{F}{E}
                    \end{pspicture}
               \end{extern}
          \end{center}
          \begin{center}
               \textit{Exemple 2}
          \end{center}
     \end{vbloc}
     Dans l'\textit{exemple 2}, il y a deux sommets de degré impair (A:3 et E:3). Le graphe contient une chaîne eulérienne, par exemple (A; F; D; B; F; E; D; C; B; A; E) mais pas de cycle eulérien.
     \par
     \begin{vbloc}
          \begin{center}
               \begin{extern}%width="280"
                    \psset{unit=1.5cm}
                    \begin{pspicture}(-1,0)(6,3)
                         \rput(0,0){\circlenode{B}{B}}
                         \rput(2,3.4){\circlenode{A}{A}}
                         \rput(4,0){\circlenode{C}{C}}
                         \rput(3,1.7){\circlenode{D}{D}}
                         \rput(2,0){\circlenode{E}{E}}
                         \rput(2,1.1){\circlenode{F}{F}}
                         \ncline{A}{B} \ncline{B}{E}\ncline{D}{A}\ncline{D}{C}\ncline{E}{C}
                         \ncline{A}{F} \ncline{B}{F}\ncline{F}{D}\ncline{F}{E}
                    \end{pspicture}
               \end{extern}
          \end{center}
          \begin{center}
               \textit{Exemple 3}
          \end{center}
          Dans l'\textit{exemple 3}, il y a 4 sommets de degré impair (A:3, B:3, D:3 et E:3). Le graphe ne contient pas de chaîne eulérienne.
     \end{vbloc}
     \par
     \begin{vbloc}
          \begin{center}
               \begin{extern}%width="350"
                    \psset{unit=1.5cm}
                    \begin{pspicture}(-1,-2)(7,5)
                         \rput(3,4){\circlenode{A}{A}}
                         \rput(0,1){\circlenode{B}{B}}
                         \rput(6,1){\circlenode{C}{C}}
                         \rput(3,-0){\circlenode{D}{D}}
                         \rput(0,3){\circlenode{E}{E}}
                         \rput(6,3){\circlenode{F}{F}}
                         \rput(2,3){\circlenode{G}{G}}
                         \rput(4,3){\circlenode{H}{H}}
                         \rput(5,2){\circlenode{I}{I}}
                         \rput(4,1){\circlenode{J}{J}}
                         \rput(2,1){\circlenode{K}{K}}
                         \rput(1,2){\circlenode{L}{L}}
                         \ncline{D}{K} \ncline{K}{J}\ncline{J}{D}
                         \ncline{B}{K}\ncline{K}{L}\ncline{L}{B}
                         \ncline{E}{G} \ncline{G}{L}\ncline{L}{E}
                         \ncline{A}{G}\ncline{G}{H}\ncline{H}{A}
                         \ncline{H}{F} \ncline{F}{I}\ncline{I}{H}
                         \ncline{I}{C}\ncline{C}{J}\ncline{J}{I}
                    \end{pspicture}
               \end{extern}
          \end{center}
          \begin{center}
               \textit{Exemple 4}
          \end{center}
     \end{vbloc}
     Dans l'\textit{exemple 4}, tous les sommets sont de degré pair . Le graphe contient un cycle eulérien, par exemple: (G; A; H; F; I; C; J; D; K; B; L; E; G; H; I; J; K; L; G).
}
\begin{h2}3. Coloration d'un graphe\end{h2}
\cadre{bleu}{Définition}{%
     \textbf{Colorier un graphe} c'est associer à tout sommet une couleur telle que deux sommets adjacents n'aient pas la même couleur.
     \par
     Le plus petit nombre de couleurs nécessaire pour colorier un graphe s'appelle le \textbf{nombre chromatique} du graphe.
}
\bloc{orange}{Exemple}{%
     \begin{vbloc}
          \begin{center}
               \begin{extern}%width="280"
                    \psset{unit=1.5cm}
                    \begin{pspicture}(-1,0)(6,3)
                         \rput(0,0){\circlenode[fillstyle=solid,fillcolor=blue]{A}{}}
                         \rput(0,2){\circlenode[fillstyle=solid,fillcolor=green]{B}{}}
                         \rput(2,3){\circlenode[fillstyle=solid,fillcolor=red]{C}{}}
                         \rput(4,2){\circlenode[fillstyle=solid,fillcolor=blue]{D}{}}
                         \rput(4,0){\circlenode[fillstyle=solid,fillcolor=green]{E}{}}
                         \rput(2,1){\circlenode[fillstyle=solid,fillcolor=red]{F}{}}
                         \ncline{A}{B} \ncline{B}{C}\ncline{C}{D}\ncline{D}{E}\ncline{E}{A}\ncline{B}{D}
                         \ncline{A}{F} \ncline{B}{F}\ncline{F}{D}\ncline{F}{E}
                    \end{pspicture}
               \end{extern}
          \end{center}
          \begin{center}
               \textit{Exemple 2}
          \end{center}
     \end{vbloc}
     Le graphe ci-dessus a été colorié a l'aide de 3 couleurs différentes. Il n'est pas possible de le colorier avec seulement 2 couleurs. Le nombre chromatique du graphe est donc 3.
}
\cadre{rouge}{Théorème}{%
     Le nombre chromatique d'un graphe est inférieur ou égal à $d_{max}+1$ où $d_{max}$ est le plus grand degré des sommets.
}
\bloc{orange}{Exemple}{%
     Dans l'exemple précédent le plus grand degré est 4. Le nombre chromatique du graphe est donc inférieur ou égal à 5 (On a vu que c'était 3).
}
\begin{h2}4. Algorithme de Dijkstra\end{h2}
L'algorithme de Dijkstra (\textit{prononcer approximativement \og Dextra \fg{}}) permet de trouver \textbf{le plus court chemin entre deux sommets d'un graphe} (orienté ou non orienté).
\par
Le fonctionnement de l'algorithme de Dijkstra est généralement présenté sous forme d'un tableau dans lequel chaque ligne représente une étape.
\par
La construction d'un tel tableau est détaillée dans la fiche méthode~:  \mcLien{/methode/algorithme-de-dijkstra-etape-par-etape/}{Algorithme de Dijkstra - Étape par étape}.

\end{document}
µ
\documentclass[a4paper]{article}

%================================================================================================================================
%
% Packages
%
%================================================================================================================================

\usepackage[T1]{fontenc} 	% pour caractères accentués
\usepackage[utf8]{inputenc}  % encodage utf8
\usepackage[french]{babel}	% langue : français
\usepackage{fourier}			% caractères plus lisibles
\usepackage[dvipsnames]{xcolor} % couleurs
\usepackage{fancyhdr}		% réglage header footer
\usepackage{needspace}		% empêcher sauts de page mal placés
\usepackage{graphicx}		% pour inclure des graphiques
\usepackage{enumitem,cprotect}		% personnalise les listes d'items (nécessaire pour ol, al ...)
\usepackage{hyperref}		% Liens hypertexte
\usepackage{pstricks,pst-all,pst-node,pstricks-add,pst-math,pst-plot,pst-tree,pst-eucl} % pstricks
\usepackage[a4paper,includeheadfoot,top=2cm,left=3cm, bottom=2cm,right=3cm]{geometry} % marges etc.
\usepackage{comment}			% commentaires multilignes
\usepackage{amsmath,environ} % maths (matrices, etc.)
\usepackage{amssymb,makeidx}
\usepackage{bm}				% bold maths
\usepackage{tabularx}		% tableaux
\usepackage{colortbl}		% tableaux en couleur
\usepackage{fontawesome}		% Fontawesome
\usepackage{environ}			% environment with command
\usepackage{fp}				% calculs pour ps-tricks
\usepackage{multido}			% pour ps tricks
\usepackage[np]{numprint}	% formattage nombre
\usepackage{tikz,tkz-tab} 			% package principal TikZ
\usepackage{pgfplots}   % axes
\usepackage{mathrsfs}    % cursives
\usepackage{calc}			% calcul taille boites
\usepackage[scaled=0.875]{helvet} % font sans serif
\usepackage{svg} % svg
\usepackage{scrextend} % local margin
\usepackage{scratch} %scratch
\usepackage{multicol} % colonnes
%\usepackage{infix-RPN,pst-func} % formule en notation polanaise inversée
\usepackage{listings}

%================================================================================================================================
%
% Réglages de base
%
%================================================================================================================================

\lstset{
language=Python,   % R code
literate=
{á}{{\'a}}1
{à}{{\`a}}1
{ã}{{\~a}}1
{é}{{\'e}}1
{è}{{\`e}}1
{ê}{{\^e}}1
{í}{{\'i}}1
{ó}{{\'o}}1
{õ}{{\~o}}1
{ú}{{\'u}}1
{ü}{{\"u}}1
{ç}{{\c{c}}}1
{~}{{ }}1
}


\definecolor{codegreen}{rgb}{0,0.6,0}
\definecolor{codegray}{rgb}{0.5,0.5,0.5}
\definecolor{codepurple}{rgb}{0.58,0,0.82}
\definecolor{backcolour}{rgb}{0.95,0.95,0.92}

\lstdefinestyle{mystyle}{
    backgroundcolor=\color{backcolour},   
    commentstyle=\color{codegreen},
    keywordstyle=\color{magenta},
    numberstyle=\tiny\color{codegray},
    stringstyle=\color{codepurple},
    basicstyle=\ttfamily\footnotesize,
    breakatwhitespace=false,         
    breaklines=true,                 
    captionpos=b,                    
    keepspaces=true,                 
    numbers=left,                    
xleftmargin=2em,
framexleftmargin=2em,            
    showspaces=false,                
    showstringspaces=false,
    showtabs=false,                  
    tabsize=2,
    upquote=true
}

\lstset{style=mystyle}


\lstset{style=mystyle}
\newcommand{\imgdir}{C:/laragon/www/newmc/assets/imgsvg/}
\newcommand{\imgsvgdir}{C:/laragon/www/newmc/assets/imgsvg/}

\definecolor{mcgris}{RGB}{220, 220, 220}% ancien~; pour compatibilité
\definecolor{mcbleu}{RGB}{52, 152, 219}
\definecolor{mcvert}{RGB}{125, 194, 70}
\definecolor{mcmauve}{RGB}{154, 0, 215}
\definecolor{mcorange}{RGB}{255, 96, 0}
\definecolor{mcturquoise}{RGB}{0, 153, 153}
\definecolor{mcrouge}{RGB}{255, 0, 0}
\definecolor{mclightvert}{RGB}{205, 234, 190}

\definecolor{gris}{RGB}{220, 220, 220}
\definecolor{bleu}{RGB}{52, 152, 219}
\definecolor{vert}{RGB}{125, 194, 70}
\definecolor{mauve}{RGB}{154, 0, 215}
\definecolor{orange}{RGB}{255, 96, 0}
\definecolor{turquoise}{RGB}{0, 153, 153}
\definecolor{rouge}{RGB}{255, 0, 0}
\definecolor{lightvert}{RGB}{205, 234, 190}
\setitemize[0]{label=\color{lightvert}  $\bullet$}

\pagestyle{fancy}
\renewcommand{\headrulewidth}{0.2pt}
\fancyhead[L]{maths-cours.fr}
\fancyhead[R]{\thepage}
\renewcommand{\footrulewidth}{0.2pt}
\fancyfoot[C]{}

\newcolumntype{C}{>{\centering\arraybackslash}X}
\newcolumntype{s}{>{\hsize=.35\hsize\arraybackslash}X}

\setlength{\parindent}{0pt}		 
\setlength{\parskip}{3mm}
\setlength{\headheight}{1cm}

\def\ebook{ebook}
\def\book{book}
\def\web{web}
\def\type{web}

\newcommand{\vect}[1]{\overrightarrow{\,\mathstrut#1\,}}

\def\Oij{$\left(\text{O}~;~\vect{\imath},~\vect{\jmath}\right)$}
\def\Oijk{$\left(\text{O}~;~\vect{\imath},~\vect{\jmath},~\vect{k}\right)$}
\def\Ouv{$\left(\text{O}~;~\vect{u},~\vect{v}\right)$}

\hypersetup{breaklinks=true, colorlinks = true, linkcolor = OliveGreen, urlcolor = OliveGreen, citecolor = OliveGreen, pdfauthor={Didier BONNEL - https://www.maths-cours.fr} } % supprime les bordures autour des liens

\renewcommand{\arg}[0]{\text{arg}}

\everymath{\displaystyle}

%================================================================================================================================
%
% Macros - Commandes
%
%================================================================================================================================

\newcommand\meta[2]{    			% Utilisé pour créer le post HTML.
	\def\titre{titre}
	\def\url{url}
	\def\arg{#1}
	\ifx\titre\arg
		\newcommand\maintitle{#2}
		\fancyhead[L]{#2}
		{\Large\sffamily \MakeUppercase{#2}}
		\vspace{1mm}\textcolor{mcvert}{\hrule}
	\fi 
	\ifx\url\arg
		\fancyfoot[L]{\href{https://www.maths-cours.fr#2}{\black \footnotesize{https://www.maths-cours.fr#2}}}
	\fi 
}


\newcommand\TitreC[1]{    		% Titre centré
     \needspace{3\baselineskip}
     \begin{center}\textbf{#1}\end{center}
}

\newcommand\newpar{    		% paragraphe
     \par
}

\newcommand\nosp {    		% commande vide (pas d'espace)
}
\newcommand{\id}[1]{} %ignore

\newcommand\boite[2]{				% Boite simple sans titre
	\vspace{5mm}
	\setlength{\fboxrule}{0.2mm}
	\setlength{\fboxsep}{5mm}	
	\fcolorbox{#1}{#1!3}{\makebox[\linewidth-2\fboxrule-2\fboxsep]{
  		\begin{minipage}[t]{\linewidth-2\fboxrule-4\fboxsep}\setlength{\parskip}{3mm}
  			 #2
  		\end{minipage}
	}}
	\vspace{5mm}
}

\newcommand\CBox[4]{				% Boites
	\vspace{5mm}
	\setlength{\fboxrule}{0.2mm}
	\setlength{\fboxsep}{5mm}
	
	\fcolorbox{#1}{#1!3}{\makebox[\linewidth-2\fboxrule-2\fboxsep]{
		\begin{minipage}[t]{1cm}\setlength{\parskip}{3mm}
	  		\textcolor{#1}{\LARGE{#2}}    
 	 	\end{minipage}  
  		\begin{minipage}[t]{\linewidth-2\fboxrule-4\fboxsep}\setlength{\parskip}{3mm}
			\raisebox{1.2mm}{\normalsize\sffamily{\textcolor{#1}{#3}}}						
  			 #4
  		\end{minipage}
	}}
	\vspace{5mm}
}

\newcommand\cadre[3]{				% Boites convertible html
	\par
	\vspace{2mm}
	\setlength{\fboxrule}{0.1mm}
	\setlength{\fboxsep}{5mm}
	\fcolorbox{#1}{white}{\makebox[\linewidth-2\fboxrule-2\fboxsep]{
  		\begin{minipage}[t]{\linewidth-2\fboxrule-4\fboxsep}\setlength{\parskip}{3mm}
			\raisebox{-2.5mm}{\sffamily \small{\textcolor{#1}{\MakeUppercase{#2}}}}		
			\par		
  			 #3
 	 		\end{minipage}
	}}
		\vspace{2mm}
	\par
}

\newcommand\bloc[3]{				% Boites convertible html sans bordure
     \needspace{2\baselineskip}
     {\sffamily \small{\textcolor{#1}{\MakeUppercase{#2}}}}    
		\par		
  			 #3
		\par
}

\newcommand\CHelp[1]{
     \CBox{Plum}{\faInfoCircle}{À RETENIR}{#1}
}

\newcommand\CUp[1]{
     \CBox{NavyBlue}{\faThumbsOUp}{EN PRATIQUE}{#1}
}

\newcommand\CInfo[1]{
     \CBox{Sepia}{\faArrowCircleRight}{REMARQUE}{#1}
}

\newcommand\CRedac[1]{
     \CBox{PineGreen}{\faEdit}{BIEN R\'EDIGER}{#1}
}

\newcommand\CError[1]{
     \CBox{Red}{\faExclamationTriangle}{ATTENTION}{#1}
}

\newcommand\TitreExo[2]{
\needspace{4\baselineskip}
 {\sffamily\large EXERCICE #1\ (\emph{#2 points})}
\vspace{5mm}
}

\newcommand\img[2]{
          \includegraphics[width=#2\paperwidth]{\imgdir#1}
}

\newcommand\imgsvg[2]{
       \begin{center}   \includegraphics[width=#2\paperwidth]{\imgsvgdir#1} \end{center}
}


\newcommand\Lien[2]{
     \href{#1}{#2 \tiny \faExternalLink}
}
\newcommand\mcLien[2]{
     \href{https~://www.maths-cours.fr/#1}{#2 \tiny \faExternalLink}
}

\newcommand{\euro}{\eurologo{}}

%================================================================================================================================
%
% Macros - Environement
%
%================================================================================================================================

\newenvironment{tex}{ %
}
{%
}

\newenvironment{indente}{ %
	\setlength\parindent{10mm}
}

{
	\setlength\parindent{0mm}
}

\newenvironment{corrige}{%
     \needspace{3\baselineskip}
     \medskip
     \textbf{\textsc{Corrigé}}
     \medskip
}
{
}

\newenvironment{extern}{%
     \begin{center}
     }
     {
     \end{center}
}

\NewEnviron{code}{%
	\par
     \boite{gray}{\texttt{%
     \BODY
     }}
     \par
}

\newenvironment{vbloc}{% boite sans cadre empeche saut de page
     \begin{minipage}[t]{\linewidth}
     }
     {
     \end{minipage}
}
\NewEnviron{h2}{%
    \needspace{3\baselineskip}
    \vspace{0.6cm}
	\noindent \MakeUppercase{\sffamily \large \BODY}
	\vspace{1mm}\textcolor{mcgris}{\hrule}\vspace{0.4cm}
	\par
}{}

\NewEnviron{h3}{%
    \needspace{3\baselineskip}
	\vspace{5mm}
	\textsc{\BODY}
	\par
}

\NewEnviron{margeneg}{ %
\begin{addmargin}[-1cm]{0cm}
\BODY
\end{addmargin}
}

\NewEnviron{html}{%
}

\begin{document}
\meta{url}{/cours/evenements-probabilites/}
\meta{pid}{485}
\meta{titre}{Probabilités conditionnelles - Indépendance}
\meta{type}{cours}
%
\begin{h2}1.Rappels\end{h2}
%
\cadre{bleu}{Rappels de définitions}{
     \begin{itemize}
          \item Une expérience \textbf{aléatoire} est une expérience dont le résultat dépend du hasard.
          \item Chacun des résultats possibles s'appelle une \textbf{éventualité} (ou une \textbf{issue}).
          \item L'ensemble $\Omega $ de tous les résultats possibles d'une expérience aléatoire s'appelle l'\textbf{univers} de l'expérience.
          \item On définit une \textbf{loi de probabilité} sur $\Omega $ en associant, à chaque éventualité $x_{i}$, un réel $p_{i}$ compris entre $0$ et $1$ tel que la somme de tous les $p_{i}$ soit égale à $1$.
          \item Un événement  est un sous-ensemble de $\Omega $.
     \end{itemize}
}
\bloc{orange}{Exemples}{
     Le lancer d'un dé à six faces est une expérience aléatoire d'univers comportant 6 \textbf{éventualités}:
     $\Omega =\left\{1; 2; 3; 4; 5; 6\right\}$
     \begin{itemize}
          \item L'ensemble $E_{1}=\left\{2; 4; 6\right\}$ est un \textbf{événement}. En français, cet événement peut se traduire par la phrase : \og\textit{le résultat du dé est un nombre pair}\fg{}
          \item L'ensemble $E_{2}=\left\{1; 2; 3\right\}$ est un autre événement. Ce second événement peut se traduire par la phrase : \og\textit{le résultat du dé est strictement inférieur à 4}\fg{}.
     \end{itemize}
     Ces événements peuvent être représentés par un diagramme de Venn :
     \img{Venn}{0.33}{Diagramme de Venn}
}
\cadre{bleu}{Définitions}{
     \begin{itemize}
          \item l'événement contraire de $A$ noté $\bar{A}$ est l'ensemble des éventualités de $\Omega $ qui n'appartiennent pas à $A$.
          \item l'événement $A \cup  B$ (lire \og A union B \fg{} ou \og A ou B \fg{} est constitué des éventualités qui appartiennent soit à A, soit à B, soit aux deux ensembles.
          \item l'événement $A \cap  B$ (lire \og A inter B \fg{} ou \og A et B \fg{} est constitué des éventualités qui appartiennent à la fois à A et à B.
     \end{itemize}
}
\bloc{orange}{Exemple}{
     On reprend l'exemple précédent :
     $E_{1}=\left\{2; 4; 6\right\}$
     $E_{2}=\left\{1; 2; 3\right\}$
     \begin{itemize}
          \item $\overline{E}_{1}=\left\{1; 3; 5\right\}$ : cet événement peut se traduire par \og le résultat est un nombre impair \fg{}
          \img{Venn-complementaire}{0.3}{Diagramme de Venn - Complémentaire}
          \item $E_{1} \cup  E_{2}=\left\{1; 2; 3; 4; 6\right\}$ : cet événement peut se traduire par \og le résultat est pair \textbf{ou} strictement inférieur à 4 \fg{}.
          \img{Venn-union}{0.3}{Diagramme de Venn - Union}
          \item $E_{1} \cap  E_{2}=\left\{2\right\}$ : cet événement peut se traduire par \og le résultat est pair \textbf{et} strictement inférieur à 4 \fg{}.
          \img{Venn-inter}{0.3}{Diagramme de Venn - Intersection}
     \end{itemize}
}
\cadre{bleu}{Définition}{
     On dit que A et B sont \textbf{incompatibles} si et seulement si $A \cap  B=\varnothing$
}
\bloc{mauve}{Remarque}{
     Deux événements contraires sont incompatibles mais deux événements peuvent être incompatibles sans être contraires.
}
\bloc{orange}{Exemple}{
     \og Obtenir un chiffre inférieur à 2 \fg{} et \og obtenir un chiffre supérieur à 4 \fg{} sont deux événements incompatibles.
}
\cadre{vert}{Propriétés}{
     \begin{itemize}
          \item $p\left(\varnothing\right)=0$
          \item $p\left(\Omega \right)=1$
          \item $p\left(\overline{A}\right)=1-p\left(A\right)$
          \item $p\left(A \cup  B\right)=p\left(A\right)+p\left(B\right)-p\left(A \cap  B\right)$.
     \end{itemize}
     Si A et B sont incompatibles, la dernière égalité devient :
     \begin{itemize}
          \item $p\left(A \cup  B\right)=p\left(A\right)+p\left(B\right)$.
     \end{itemize}
}
\begin{h2}2. Arbre\end{h2}
Lorsqu'une expérience aléatoire comporte plusieurs étapes, on utilise souvent un arbre pondéré pour la représenter.
\bloc{orange}{Exemple}{
     Dans une classe de Terminale, 52\% de garçons et 48\% de filles étaient candidats au baccalauréat.
     \par
     80\% des garçons et 85\% des filles ont obtenu leur diplôme.
     \par
     On choisit un élève au hasard et on note :
     \begin{itemize}
          \item $G$ : l'événement \og l'élève choisi est un garçon \fg{};
          \item $F$ : l'événement \og l'élève choisie est une fille \fg{};
          \item $B$ : l'événement \og l'élève choisi(e) a obtenu son baccalauréat \fg{}.
     \end{itemize}
     \medskip
     On peut représenter la situation à l'aide de l'arbre pondéré ci-dessous :
}
%:-+-+-+- Engendré par : http://math.et.info.free.fr/TikZ/Arbre/
\begin{extern} %width="350" alt="arbre pondéré" class="aligncenter"
     % Racine à Gauche, développement vers la droite
     \begin{tikzpicture}[xscale=1,yscale=1]
          % Styles (MODIFIABLES)
          \tikzstyle{fleche}=[-,>=latex,thick]
          \tikzstyle{noeud}=[fill=white,circle,draw]
          \tikzstyle{feuille}=[fill=white,circle,draw]
          \tikzstyle{etiquette}=[midway,fill=white]
          % Dimensions (MODIFIABLES)
          \def\DistanceInterNiveaux{3}
          \def\DistanceInterFeuilles{2}
          % Dimensions calculées (NON MODIFIABLES)
          \def\NiveauA{(0)*\DistanceInterNiveaux}
          \def\NiveauB{(1.5)*\DistanceInterNiveaux}
          \def\NiveauC{(2.5)*\DistanceInterNiveaux}
          \def\InterFeuilles{(-1)*\DistanceInterFeuilles}
          % Noeuds (MODIFIABLES : Styles et Coefficients d'InterFeuilles)
          \node[noeud] (R) at ({\NiveauA},{(1.5)*\InterFeuilles}) {$\ $};
          \node[noeud] (Ra) at ({\NiveauB},{(0.5)*\InterFeuilles}) {$G$};
          \node[feuille] (Raa) at ({\NiveauC},{(0)*\InterFeuilles}) {$B$};
          \node[feuille] (Rab) at ({\NiveauC},{(1)*\InterFeuilles}) {$\overline{B}$};
          \node[noeud] (Rb) at ({\NiveauB},{(2.5)*\InterFeuilles}) {$F$};
          \node[feuille] (Rba) at ({\NiveauC},{(2)*\InterFeuilles}) {$B$};
          \node[feuille] (Rbb) at ({\NiveauC},{(3)*\InterFeuilles}) {$\overline{B}$};
          % Arcs (MODIFIABLES : Styles)
          \draw[fleche] (R)--(Ra) node[etiquette] {$0,52$};
          \draw[fleche] (Ra)--(Raa) node[etiquette] {$0,8$};
          \draw[fleche] (Ra)--(Rab) node[etiquette] {$0,2$};
          \draw[fleche] (R)--(Rb) node[etiquette] {$0,48$};
          \draw[fleche] (Rb)--(Rba) node[etiquette] {$0,85$};
          \draw[fleche] (Rb)--(Rbb) node[etiquette] {$0,15$};
     \end{tikzpicture}
\end{extern}
%:-+-+-+-+- Fin
Le premier niveau indique le genre de l'élève ($G$ ou $F$) et le second indique l'obtention du diplôme ($B$ ou $\overline{B}$).
\par
On inscrit les probabilités sur chacune des branches.
\par
La \textbf{somme} des probabilités inscrites sur les branches partant d’un même nœud \textbf{est toujours égale à 1}.
%
\begin{h2}3. Probabilités conditionnelles\end{h2}
\cadre{bleu}{Définition}{
     Soit A et B deux événements tels que $p\left(A\right)\neq 0$, \textbf{la probabilité de B sachant A} est le nombre :
     \[  p_{A}\left(B\right)=\frac{p\left(A \cap  B\right)}{p\left(A\right)}. \]
     On peut aussi noter cette probabilité $p\left(B/A\right)$.
}
\bloc{orange}{Exemple}{
     On reprend l'exemple du lancer d'un dé.
     La probabilité d'obtenir un chiffre pair sachant que le chiffre obtenu est strictement inférieur à 4 est (en cas d'équiprobabilité) :
     \[ p_{E_{2}}\left(E_{1}\right)=\frac{p\left(E_{1} \cap  E_{2}\right)}{p\left(E_{2}\right)}=\frac{1}{3}. \]
}
\bloc{mauve}{Remarques}{
     \begin{itemize}
          \item L'égalité précédente s'emploie souvent sous la forme :
          \[p\left(A \cap  B\right)=p\left(A\right)\times p_{A}\left(B\right)\]
          pour calculer la probabilité de $A \cap  B$.
          \item \textbf{Attention} à ne pas confondre $p_{A}\left(B\right)$ et $p\left(A \cap  B\right)$ dans les exercices.\\
          On doit calculer  $p_{A}\left(B\right)$ lorsque l'\textbf{on sait que $A$ est réalisé.}
          \item Avec un arbre pondéré, les probabilités conditionnelles figurent sur les branches du second niveau et des niveaux supérieurs (s'il y en a).\\
          La probabilité inscrite sur la branche reliant $A$ à $B$ est $p_A(B)$.\\
          Typiquement, un arbre binaire à deux niveaux se présentera ainsi :
          %:-+-+-+- Engendré par : http://math.et.info.free.fr/TikZ/Arbre/
          \begin{extern} %width="350" alt="arbre pondéré" class="aligncenter"
               % Racine à Gauche, développement vers la droite
               \begin{tikzpicture}[xscale=1,yscale=1]
                    % Styles (MODIFIABLES)
                    \tikzstyle{fleche}=[-,>=latex,thick]
                    \tikzstyle{noeud}=[fill=white,circle,draw]
                    \tikzstyle{feuille}=[fill=white,circle,draw]
                    \tikzstyle{etiquette}=[midway,fill=white]
                    % Dimensions (MODIFIABLES)
                    \def\DistanceInterNiveaux{3}
                    \def\DistanceInterFeuilles{2}
                    % Dimensions calculées (NON MODIFIABLES)
                    \def\NiveauA{(0)*\DistanceInterNiveaux}
                    \def\NiveauB{(1.5)*\DistanceInterNiveaux}
                    \def\NiveauC{(2.5)*\DistanceInterNiveaux}
                    \def\InterFeuilles{(-1)*\DistanceInterFeuilles}
                    % Noeuds (MODIFIABLES : Styles et Coefficients d'InterFeuilles)
                    \node[noeud] (R) at ({\NiveauA},{(1.5)*\InterFeuilles}) {$\ $};
                    \node[noeud] (Ra) at ({\NiveauB},{(0.5)*\InterFeuilles}) {$A$};
                    \node[feuille] (Raa) at ({\NiveauC},{(0)*\InterFeuilles}) {$B$};
                    \node[feuille] (Rab) at ({\NiveauC},{(1)*\InterFeuilles}) {$\overline{B}$};
                    \node[noeud] (Rb) at ({\NiveauB},{(2.5)*\InterFeuilles}) {$\overline{A}$};
                    \node[feuille] (Rba) at ({\NiveauC},{(2)*\InterFeuilles}) {$B$};
                    \node[feuille] (Rbb) at ({\NiveauC},{(3)*\InterFeuilles}) {$\overline{B}$};
                    % Arcs (MODIFIABLES : Styles)
                    \draw[fleche] (R)--(Ra) node[etiquette] {$p(A)$};
                    \draw[fleche] (Ra)--(Raa) node[etiquette] {$p_A(B)$};
                    \draw[fleche] (Ra)--(Rab) node[etiquette] {$p_A(\overline{B})$};
                    \draw[fleche] (R)--(Rb) node[etiquette] {$p(\overline{A})$};
                    \draw[fleche] (Rb)--(Rba) node[etiquette] {$p_{\overline{A}}(B)$};
                    \draw[fleche] (Rb)--(Rbb) node[etiquette] {$p_{\overline{A}}(\overline{B})$};
               \end{tikzpicture}
          \end{extern}
          %:-+-+-+-+- Fin
          \item La formule $p\left(A \cap  B\right)=p\left(A\right)\times p_{A}\left(B\right)$ s'interprète alors de la façon suivante : \\
          \og La probabilité de l'événement  $A \cap  B$ s'obtient en faisant \textbf{le produit} des probabilités inscrites sur le chemin passant par $A$ et $B$\fg{}.
          %
          %:-+-+-+- Engendré par : http://math.et.info.free.fr/TikZ/Arbre/
          \begin{extern} %width="350" alt="arbre pondéré" class="aligncenter"
               % Racine à Gauche, développement vers la droite
               \begin{tikzpicture}[xscale=1,yscale=1]
                    % Styles (MODIFIABLES)
                    \tikzstyle{fleche}=[-,>=latex,thick]
                    \tikzstyle{rfleche}=[-,>=latex,color=red,thick]
                    \tikzstyle{noeud}=[fill=white,circle,draw]
                    \tikzstyle{rnoeud}=[fill=white,circle,draw=red]
                    \tikzstyle{feuille}=[fill=white,circle,draw]
                    \tikzstyle{rfeuille}=[fill=white,circle,draw=red]
                    \tikzstyle{etiquette}=[midway,fill=white]
                    % Dimensions (MODIFIABLES)
                    \def\DistanceInterNiveaux{3}
                    \def\DistanceInterFeuilles{2}
                    % Dimensions calculées (NON MODIFIABLES)
                    \def\NiveauA{(0)*\DistanceInterNiveaux}
                    \def\NiveauB{(1.5)*\DistanceInterNiveaux}
                    \def\NiveauC{(2.5)*\DistanceInterNiveaux}
                    \def\InterFeuilles{(-1)*\DistanceInterFeuilles}
                    % Noeuds (MODIFIABLES : Styles et Coefficients d'InterFeuilles)
                    \node[rnoeud] (R) at ({\NiveauA},{(1.5)*\InterFeuilles}) {$\ $};
                    \node[rnoeud] (Ra) at ({\NiveauB},{(0.5)*\InterFeuilles}) {$\red A$};
                    \node[rfeuille] (Raa) at ({\NiveauC},{(0)*\InterFeuilles}) {$\red B$};
                    \node[feuille] (Rab) at ({\NiveauC},{(1)*\InterFeuilles}) {$\overline{B}$};
                    \node[noeud] (Rb) at ({\NiveauB},{(2.5)*\InterFeuilles}) {$\overline{A}$};
                    \node[feuille] (Rba) at ({\NiveauC},{(2)*\InterFeuilles}) {$B$};
                    \node[feuille] (Rbb) at ({\NiveauC},{(3)*\InterFeuilles}) {$\overline{B}$};
                    % Arcs (MODIFIABLES : Styles)
                    \draw[rfleche] (R)--(Ra) node[etiquette] {\red $p(A)$};
                    \draw[rfleche] (Ra)--(Raa) node[etiquette] {\red $p_A(B)$};
                    \draw[fleche] (Ra)--(Rab) node[etiquette] {$p_A(\overline{B})$};
                    \draw[fleche] (R)--(Rb) node[etiquette] {$p(\overline{A})$};
                    \draw[fleche] (Rb)--(Rba) node[etiquette] {$p_{\overline{A}}(B)$};
                    \draw[fleche] (Rb)--(Rbb) node[etiquette] {$p_{\overline{A}}(\overline{B})$};
               \end{tikzpicture}
          \end{extern}
          %:-+-+-+-+- Fin
     \end{itemize}
}
\begin{h2}4. Événements indépendants\end{h2}
\cadre{bleu}{Définition}{
     Deux événements A et B sont \textbf{indépendants} si et seulement si :
     \[ p\left(A \cap  B\right)=p\left(A\right)\times p\left(B\right). \]
}
\cadre{vert}{Propriété}{
     $A$ et $B$ sont indépendants si et seulement si :
     \[ p_{A}\left(B\right)=p\left(B\right). \]
}
\bloc{orange}{Démonstration}{
     Elle résulte directement du fait que pour deux événements quelconques :
     \[ p\left(A \cap  B\right)=p\left(A\right)\times p_{A}\left(B\right). \]
}
\bloc{mauve}{Remarque}{
     Comme $A \cap  B=B \cap  A$, $A$ et $B$ sont interchangeables dans cette formule et on a également :
     \begin{center}
          $A$ et $B$ sont indépendants $ \Leftrightarrow  $ $p_{B}\left(A\right)=p\left(A\right)$.
     \end{center}
}
\begin{h2}5. Formule des probabilités totales\end{h2}
\cadre{bleu}{Définition}{
     $A_{1}$, $A_{2}$, ... , $A_{n}$ forment une partition de $\Omega $ si et seulement si  $A_{1} \cup  A_{2} . . . \cup  A_{n}=\Omega $ et $A_{i} \cap  A_{j}=\varnothing$ pour $i\neq j$.
}
\bloc{mauve}{Cas particulier fréquent}{
     Pour toute partie $A\subset\Omega $, $A$ et $\overline{A}$ forment une partition de $\Omega$.
}
\cadre{vert}{Propriété (Formule des probabilités totales)}{
     Si $A_{1}$, $A_{2}$,... $A_{n}$ forment une partition de $\Omega $, pour tout événement $B$, on a :
     \begin{center}
          $ p\left(B\right)=p\left(A_{1} \cap  B\right)+p\left(A_{2} \cap  B\right)+ \cdots $\nosp$ +p\left(A_{n} \cap  B\right). $
     \end{center}
     \par
     Cette formule peut également s'écrire à l'aide de probabilités conditionnelles :
     \begin{center}
          $p\left(B\right)=p\left(A_{1} \right)\times p_{A_{1} }\left(B\right)$\nosp$+p\left(A_{2} \right)\times p_{A_{2}}\left(B\right)+\cdots$\nosp$+p\left(A_{n}\right)\times p_{A_{n}}\left(B\right)$.
     \end{center}
}
\bloc{mauve}{Cas particulier fréquent}{
     En utilisant la partition  $\left\{A, \overline{A}\right\}$, quels que soient les événements $A$ et $B$ :
     \begin{center}
          $p\left(B\right)=p\left(A \cap  B\right)+p\left(\overline{A} \cap  B\right)$\\
          $p\left(B\right)=p\left(A\right)\times p_{A}\left(B\right)+p\left(\overline{A}\right)\times p_{\overline{A}}\left(B\right)$.
\end{center}}
\bloc{mauve}{Remarque}{
     \`A l'aide d'un arbre pondéré, ce résultat s'interprète de la façon suivante :\\
     \og La probabilité de l'événement $B$ est égale à la somme des probabilités des trajets menant à $B$ \fg{}.
     %:-+-+-+- Engendré par : http://math.et.info.free.fr/TikZ/Arbre/
     \begin{extern} %width="350" alt="arbre pondéré" class="aligncenter"
          % Racine à Gauche, développement vers la droite
          \begin{tikzpicture}[xscale=1,yscale=1]
               % Styles (MODIFIABLES)
               \tikzstyle{fleche}=[-,>=latex,thick]
               \tikzstyle{rfleche}=[-,>=latex,color=red,thick]
               \tikzstyle{noeud}=[fill=white,circle,draw]
               \tikzstyle{rnoeud}=[fill=white,circle,draw=red]
               \tikzstyle{feuille}=[fill=white,circle,draw]
               \tikzstyle{rfeuille}=[fill=white,circle,draw=red]
               \tikzstyle{etiquette}=[midway,fill=white]
               % Dimensions (MODIFIABLES)
               \def\DistanceInterNiveaux{3}
               \def\DistanceInterFeuilles{2}
               % Dimensions calculées (NON MODIFIABLES)
               \def\NiveauA{(0)*\DistanceInterNiveaux}
               \def\NiveauB{(1.5)*\DistanceInterNiveaux}
               \def\NiveauC{(2.5)*\DistanceInterNiveaux}
               \def\InterFeuilles{(-1)*\DistanceInterFeuilles}
               % Noeuds (MODIFIABLES : Styles et Coefficients d'InterFeuilles)
               \node[rnoeud] (R) at ({\NiveauA},{(1.5)*\InterFeuilles}) {$\ $};
               \node[rnoeud] (Ra) at ({\NiveauB},{(0.5)*\InterFeuilles}) {$\red A$};
               \node[rfeuille] (Raa) at ({\NiveauC},{(0)*\InterFeuilles}) {$\red B$};
               \node[feuille] (Rab) at ({\NiveauC},{(1)*\InterFeuilles}) {$\overline{B}$};
               \node[rnoeud] (Rb) at ({\NiveauB},{(2.5)*\InterFeuilles}) {$\red \overline{A}$};
               \node[rfeuille] (Rba) at ({\NiveauC},{(2)*\InterFeuilles}) {$\red B$};
               \node[feuille] (Rbb) at ({\NiveauC},{(3)*\InterFeuilles}) {$\overline{B}$};
               % Arcs (MODIFIABLES : Styles)
               \draw[rfleche] (R)--(Ra) node[etiquette] {\red $p(A)$};
               \draw[rfleche] (Ra)--(Raa) node[etiquette] {\red $p_A(B)$};
               \draw[fleche] (Ra)--(Rab) node[etiquette] {$p_A(\overline{B})$};
               \draw[rfleche] (R)--(Rb) node[etiquette] {$p(\overline{A})$};
               \draw[rfleche] (Rb)--(Rba) node[etiquette] {$p_{\overline{A}}(B)$};
               \draw[fleche] (Rb)--(Rbb) node[etiquette] {$p_{\overline{A}}(\overline{B})$};
          \end{tikzpicture}
     \end{extern}
     %:-+-+-+-+- Fin
}

\end{document}
µ
\documentclass[a4paper]{article}

%================================================================================================================================
%
% Packages
%
%================================================================================================================================

\usepackage[T1]{fontenc} 	% pour caractères accentués
\usepackage[utf8]{inputenc}  % encodage utf8
\usepackage[french]{babel}	% langue : français
\usepackage{fourier}			% caractères plus lisibles
\usepackage[dvipsnames]{xcolor} % couleurs
\usepackage{fancyhdr}		% réglage header footer
\usepackage{needspace}		% empêcher sauts de page mal placés
\usepackage{graphicx}		% pour inclure des graphiques
\usepackage{enumitem,cprotect}		% personnalise les listes d'items (nécessaire pour ol, al ...)
\usepackage{hyperref}		% Liens hypertexte
\usepackage{pstricks,pst-all,pst-node,pstricks-add,pst-math,pst-plot,pst-tree,pst-eucl} % pstricks
\usepackage[a4paper,includeheadfoot,top=2cm,left=3cm, bottom=2cm,right=3cm]{geometry} % marges etc.
\usepackage{comment}			% commentaires multilignes
\usepackage{amsmath,environ} % maths (matrices, etc.)
\usepackage{amssymb,makeidx}
\usepackage{bm}				% bold maths
\usepackage{tabularx}		% tableaux
\usepackage{colortbl}		% tableaux en couleur
\usepackage{fontawesome}		% Fontawesome
\usepackage{environ}			% environment with command
\usepackage{fp}				% calculs pour ps-tricks
\usepackage{multido}			% pour ps tricks
\usepackage[np]{numprint}	% formattage nombre
\usepackage{tikz,tkz-tab} 			% package principal TikZ
\usepackage{pgfplots}   % axes
\usepackage{mathrsfs}    % cursives
\usepackage{calc}			% calcul taille boites
\usepackage[scaled=0.875]{helvet} % font sans serif
\usepackage{svg} % svg
\usepackage{scrextend} % local margin
\usepackage{scratch} %scratch
\usepackage{multicol} % colonnes
%\usepackage{infix-RPN,pst-func} % formule en notation polanaise inversée
\usepackage{listings}

%================================================================================================================================
%
% Réglages de base
%
%================================================================================================================================

\lstset{
language=Python,   % R code
literate=
{á}{{\'a}}1
{à}{{\`a}}1
{ã}{{\~a}}1
{é}{{\'e}}1
{è}{{\`e}}1
{ê}{{\^e}}1
{í}{{\'i}}1
{ó}{{\'o}}1
{õ}{{\~o}}1
{ú}{{\'u}}1
{ü}{{\"u}}1
{ç}{{\c{c}}}1
{~}{{ }}1
}


\definecolor{codegreen}{rgb}{0,0.6,0}
\definecolor{codegray}{rgb}{0.5,0.5,0.5}
\definecolor{codepurple}{rgb}{0.58,0,0.82}
\definecolor{backcolour}{rgb}{0.95,0.95,0.92}

\lstdefinestyle{mystyle}{
    backgroundcolor=\color{backcolour},   
    commentstyle=\color{codegreen},
    keywordstyle=\color{magenta},
    numberstyle=\tiny\color{codegray},
    stringstyle=\color{codepurple},
    basicstyle=\ttfamily\footnotesize,
    breakatwhitespace=false,         
    breaklines=true,                 
    captionpos=b,                    
    keepspaces=true,                 
    numbers=left,                    
xleftmargin=2em,
framexleftmargin=2em,            
    showspaces=false,                
    showstringspaces=false,
    showtabs=false,                  
    tabsize=2,
    upquote=true
}

\lstset{style=mystyle}


\lstset{style=mystyle}
\newcommand{\imgdir}{C:/laragon/www/newmc/assets/imgsvg/}
\newcommand{\imgsvgdir}{C:/laragon/www/newmc/assets/imgsvg/}

\definecolor{mcgris}{RGB}{220, 220, 220}% ancien~; pour compatibilité
\definecolor{mcbleu}{RGB}{52, 152, 219}
\definecolor{mcvert}{RGB}{125, 194, 70}
\definecolor{mcmauve}{RGB}{154, 0, 215}
\definecolor{mcorange}{RGB}{255, 96, 0}
\definecolor{mcturquoise}{RGB}{0, 153, 153}
\definecolor{mcrouge}{RGB}{255, 0, 0}
\definecolor{mclightvert}{RGB}{205, 234, 190}

\definecolor{gris}{RGB}{220, 220, 220}
\definecolor{bleu}{RGB}{52, 152, 219}
\definecolor{vert}{RGB}{125, 194, 70}
\definecolor{mauve}{RGB}{154, 0, 215}
\definecolor{orange}{RGB}{255, 96, 0}
\definecolor{turquoise}{RGB}{0, 153, 153}
\definecolor{rouge}{RGB}{255, 0, 0}
\definecolor{lightvert}{RGB}{205, 234, 190}
\setitemize[0]{label=\color{lightvert}  $\bullet$}

\pagestyle{fancy}
\renewcommand{\headrulewidth}{0.2pt}
\fancyhead[L]{maths-cours.fr}
\fancyhead[R]{\thepage}
\renewcommand{\footrulewidth}{0.2pt}
\fancyfoot[C]{}

\newcolumntype{C}{>{\centering\arraybackslash}X}
\newcolumntype{s}{>{\hsize=.35\hsize\arraybackslash}X}

\setlength{\parindent}{0pt}		 
\setlength{\parskip}{3mm}
\setlength{\headheight}{1cm}

\def\ebook{ebook}
\def\book{book}
\def\web{web}
\def\type{web}

\newcommand{\vect}[1]{\overrightarrow{\,\mathstrut#1\,}}

\def\Oij{$\left(\text{O}~;~\vect{\imath},~\vect{\jmath}\right)$}
\def\Oijk{$\left(\text{O}~;~\vect{\imath},~\vect{\jmath},~\vect{k}\right)$}
\def\Ouv{$\left(\text{O}~;~\vect{u},~\vect{v}\right)$}

\hypersetup{breaklinks=true, colorlinks = true, linkcolor = OliveGreen, urlcolor = OliveGreen, citecolor = OliveGreen, pdfauthor={Didier BONNEL - https://www.maths-cours.fr} } % supprime les bordures autour des liens

\renewcommand{\arg}[0]{\text{arg}}

\everymath{\displaystyle}

%================================================================================================================================
%
% Macros - Commandes
%
%================================================================================================================================

\newcommand\meta[2]{    			% Utilisé pour créer le post HTML.
	\def\titre{titre}
	\def\url{url}
	\def\arg{#1}
	\ifx\titre\arg
		\newcommand\maintitle{#2}
		\fancyhead[L]{#2}
		{\Large\sffamily \MakeUppercase{#2}}
		\vspace{1mm}\textcolor{mcvert}{\hrule}
	\fi 
	\ifx\url\arg
		\fancyfoot[L]{\href{https://www.maths-cours.fr#2}{\black \footnotesize{https://www.maths-cours.fr#2}}}
	\fi 
}


\newcommand\TitreC[1]{    		% Titre centré
     \needspace{3\baselineskip}
     \begin{center}\textbf{#1}\end{center}
}

\newcommand\newpar{    		% paragraphe
     \par
}

\newcommand\nosp {    		% commande vide (pas d'espace)
}
\newcommand{\id}[1]{} %ignore

\newcommand\boite[2]{				% Boite simple sans titre
	\vspace{5mm}
	\setlength{\fboxrule}{0.2mm}
	\setlength{\fboxsep}{5mm}	
	\fcolorbox{#1}{#1!3}{\makebox[\linewidth-2\fboxrule-2\fboxsep]{
  		\begin{minipage}[t]{\linewidth-2\fboxrule-4\fboxsep}\setlength{\parskip}{3mm}
  			 #2
  		\end{minipage}
	}}
	\vspace{5mm}
}

\newcommand\CBox[4]{				% Boites
	\vspace{5mm}
	\setlength{\fboxrule}{0.2mm}
	\setlength{\fboxsep}{5mm}
	
	\fcolorbox{#1}{#1!3}{\makebox[\linewidth-2\fboxrule-2\fboxsep]{
		\begin{minipage}[t]{1cm}\setlength{\parskip}{3mm}
	  		\textcolor{#1}{\LARGE{#2}}    
 	 	\end{minipage}  
  		\begin{minipage}[t]{\linewidth-2\fboxrule-4\fboxsep}\setlength{\parskip}{3mm}
			\raisebox{1.2mm}{\normalsize\sffamily{\textcolor{#1}{#3}}}						
  			 #4
  		\end{minipage}
	}}
	\vspace{5mm}
}

\newcommand\cadre[3]{				% Boites convertible html
	\par
	\vspace{2mm}
	\setlength{\fboxrule}{0.1mm}
	\setlength{\fboxsep}{5mm}
	\fcolorbox{#1}{white}{\makebox[\linewidth-2\fboxrule-2\fboxsep]{
  		\begin{minipage}[t]{\linewidth-2\fboxrule-4\fboxsep}\setlength{\parskip}{3mm}
			\raisebox{-2.5mm}{\sffamily \small{\textcolor{#1}{\MakeUppercase{#2}}}}		
			\par		
  			 #3
 	 		\end{minipage}
	}}
		\vspace{2mm}
	\par
}

\newcommand\bloc[3]{				% Boites convertible html sans bordure
     \needspace{2\baselineskip}
     {\sffamily \small{\textcolor{#1}{\MakeUppercase{#2}}}}    
		\par		
  			 #3
		\par
}

\newcommand\CHelp[1]{
     \CBox{Plum}{\faInfoCircle}{À RETENIR}{#1}
}

\newcommand\CUp[1]{
     \CBox{NavyBlue}{\faThumbsOUp}{EN PRATIQUE}{#1}
}

\newcommand\CInfo[1]{
     \CBox{Sepia}{\faArrowCircleRight}{REMARQUE}{#1}
}

\newcommand\CRedac[1]{
     \CBox{PineGreen}{\faEdit}{BIEN R\'EDIGER}{#1}
}

\newcommand\CError[1]{
     \CBox{Red}{\faExclamationTriangle}{ATTENTION}{#1}
}

\newcommand\TitreExo[2]{
\needspace{4\baselineskip}
 {\sffamily\large EXERCICE #1\ (\emph{#2 points})}
\vspace{5mm}
}

\newcommand\img[2]{
          \includegraphics[width=#2\paperwidth]{\imgdir#1}
}

\newcommand\imgsvg[2]{
       \begin{center}   \includegraphics[width=#2\paperwidth]{\imgsvgdir#1} \end{center}
}


\newcommand\Lien[2]{
     \href{#1}{#2 \tiny \faExternalLink}
}
\newcommand\mcLien[2]{
     \href{https~://www.maths-cours.fr/#1}{#2 \tiny \faExternalLink}
}

\newcommand{\euro}{\eurologo{}}

%================================================================================================================================
%
% Macros - Environement
%
%================================================================================================================================

\newenvironment{tex}{ %
}
{%
}

\newenvironment{indente}{ %
	\setlength\parindent{10mm}
}

{
	\setlength\parindent{0mm}
}

\newenvironment{corrige}{%
     \needspace{3\baselineskip}
     \medskip
     \textbf{\textsc{Corrigé}}
     \medskip
}
{
}

\newenvironment{extern}{%
     \begin{center}
     }
     {
     \end{center}
}

\NewEnviron{code}{%
	\par
     \boite{gray}{\texttt{%
     \BODY
     }}
     \par
}

\newenvironment{vbloc}{% boite sans cadre empeche saut de page
     \begin{minipage}[t]{\linewidth}
     }
     {
     \end{minipage}
}
\NewEnviron{h2}{%
    \needspace{3\baselineskip}
    \vspace{0.6cm}
	\noindent \MakeUppercase{\sffamily \large \BODY}
	\vspace{1mm}\textcolor{mcgris}{\hrule}\vspace{0.4cm}
	\par
}{}

\NewEnviron{h3}{%
    \needspace{3\baselineskip}
	\vspace{5mm}
	\textsc{\BODY}
	\par
}

\NewEnviron{margeneg}{ %
\begin{addmargin}[-1cm]{0cm}
\BODY
\end{addmargin}
}

\NewEnviron{html}{%
}

\begin{document}
\meta{url}{/cours/variables-aleatoires-continues/}
\meta{pid}{488}
\meta{titre}{Variables aléatoires continues}
\meta{type}{cours}
\cadre{vert}{Introduction}{%id="i10"
     Il arrive qu'une variable aléatoire puisse prendre n'importe quelle valeur sur $\mathbb{R}$ ou sur un intervalle $I$ de $\mathbb{R}$. On parle alors de \textbf{variable aléatoire continue}.
     \par
     Pour une telle variable, les événements qui vont nous intéresser ne sont plus $(X=5)$, $(X=20)$, etc... , mais $(X \leqslant 5)$, $(5 \leqslant X \leqslant 20)$, etc...
}
\begin{h2}1. Généralités\end{h2}
\cadre{bleu}{Définition}{%id="d10"
     Soit $f$ une fonction \textbf{continue} et \textbf{positive} sur un intervalle $I=\left[a;b\right]$ telle que
     \[ \int_{a}^{b}f\left(x\right)dx=1. \]
     On dit que $X$ est une \textbf{variable aléatoire réelle continue de densité} $f$ si et seulement si pour tout $x_{1} \in I$ et tout $x_{2} \in I $ ($x_{1}\leqslant x_{2}$) :
     \begin{center}$p\left(x_{1}\leqslant X\leqslant x_{2}\right)=\int_{x_{1}}^{x_{2}}f\left(x\right)dx$\end{center}
}
\bloc{orange}{Exemple}{%id="e10"
     La fonction $f$ définie sur $I=\left[0;2\right]$ par $f\left(x\right)=\frac{x}{2}$ est une fonction continue et positive sur $I$.
     \par
     La fonction $F : x \longmapsto \dfrac{x^2}{4}$ est une primitive de $f$ sur $I$, par conséquent :
     \par
     $\int_{0}^{2}f\left(x\right)dx=\left[\frac{x^{2}}{4}\right]_{0}^{2}=1$.
     \par
     $f$ est donc une \textbf{densité de probabilité}.
     \par
     Soit X une variable aléatoire réelle à valeurs dans $I$ de densité $f$, on a alors, par exemple :
     \par
     $P\left(1\leqslant X\leqslant 1,5\right)=\int_{1}^{1,5}f\left(x\right)dx$.
     \par
     $P\left(1\leqslant X\leqslant 1,5\right)$ est donc l'aire (en u.a.) colorée ci-dessous :
     \begin{center}
          \begin{extern} %width="250" alt="densité linéaire"
               \begin{pspicture}(0,-0.5)(5,3)
                    \psset{xunit=2 cm, yunit=2 cm, algebraic=true}
                    %\psgrid[gridcolor=mcgris, subgriddiv=0, gridlabels=0pt](0,0)(2,1)
                    \psaxes{->}(0,0)(0,0)(2.5,1.5)
                    \psplot[linecolor=red,linewidth=0.75pt]{0}{2}{x/2}
                    \pscustom[linecolor=mcvert,linewidth=0.75pt,fillstyle=solid,fillcolor=mcvert,opacity=0.1]{
                         \psplot{1}{1.5}{x/2}
                         \psline(1.5,0)(1,0)(1,0.5)
                    }
               \end{pspicture}
          \end{extern}
     \end{center}
     Un calcul simple montre que $P\left(1\leqslant X\leqslant 1,5\right)=\left[\frac{x^{2}}{4}\right]_{1}^{1,5}=0,3125$.
}
\bloc{mauve}{Remarques}{%id="r10"
     \begin{itemize}
          \item On peut étendre cette définition aux cas où l'une ou les deux bornes $a$ et $b$ sont infinies.
          \par
          Dans ce cas, on remplace la condition $\int_{a}^{ b}f\left(x\right)dx=1$ par une condition portant sur une limite; par exemple si $b$ vaut $+\infty $, la condition $\int_{a}^{ b}f\left(x\right)dx=1$ deviendra $\lim\limits_{y\rightarrow +\infty }\int_{a}^{ y}f\left(x\right)dx=1$
          \item Comme indiqué en introduction, les événements du type $\left(X=k\right)$ ne sont pas intéressants car pour tout $k$ appartenant à $I$, $p\left(X=k\right)=\int_{k}^{ k}f\left(x\right)dx=0$.
          \item On peut employer indifféremment des inégalités larges ou strictes :
          \begin{center}$p\left(x_{1} < X < x_{2}\right)=p\left(x_{1}\leqslant X\leqslant x_{2}\right)$.\end{center}
     \end{itemize}
}
\cadre{bleu}{Définition}{%id="d20"
     L'espérance mathématique d'une variable aléatoire $X$ qui suit une loi de densité $f$ sur $\left[a;b\right]$ est le réel noté $E\left(X\right)$ défini par :
     \begin{center}$E\left(X\right)=\int_{a}^{b}xf\left(x\right)dx$.\end{center}
}
\bloc{orange}{Exemple}{%id="e20"
     Si l'on reprend l'exemple de la fonction $f$ définie sur $I=\left[0;2\right]$ par $f\left(x\right)=\frac{x}{2}$, l'espérance mathématique est :
     \par
     $E\left(X\right)=\int_{0}^{2}xf\left(x\right)dx$\nosp$=\int_{0}^{2}\frac{x^{2}}{2}dx$\nosp$=\left[\frac{x^{3}}{6}\right]_{0}^{2}$\nosp$=\frac{8}{6}=\frac{4}{3}$.
}
\begin{h2}2. Loi uniforme sur un intervalle\end{h2}
\cadre{bleu}{Définition}{%id="d40"
     On dit qu'une variable aléatoire $X$ suit la \textbf{loi uniforme} sur l'intervalle $\left[a~;~b\right]$ si sa densité de probabilité $f$ est constante sur $\left[a~;~b\right]$.
     \par
     Cette densité vaut alors, pour tout réel $x \in [a~;~b]$ :
     \[ f\left(x\right)=\frac{1}{b-a}. \]
}
\bloc{orange}{Exemple}{%id="e40"
     La densité de la loi uniforme sur l'intervalle $\left[0, 2\right]$ est représentée ci-dessous~:
     \begin{center}
          \begin{extern} %width="250" alt="loi uniforme"
               \begin{pspicture}(0,-0.5)(5,3)
                    \psset{xunit=2 cm, yunit=2 cm, algebraic=true}
                    %\psgrid[gridcolor=mcgris, subgriddiv=0, gridlabels=0pt](0,0)(2,1)
                    \psaxes{->}(0,0)(0,0)(2.5,1.5)
                    \psplot[linecolor=red,linewidth=0.75pt]{0}{2}{1/2}
               \end{pspicture}
          \end{extern}
     \end{center}
     \begin{center}\textit{Densité de la loi uniforme sur l'intervalle $\left[0, 2\right]$}\end{center}
}
\bloc{mauve}{Remarque}{%id="r40"
     Une primitive de la fonction $x \longmapsto \dfrac{1}{b-a}$ sur $[a~;~b]$ est  $x \longmapsto \dfrac{x}{b-a}$.
     \par
     On vérifie alors que :
     $\int_{a}^{b} \frac{1}{b-a} dx=\left[\frac{x}{b-a}\right]_{a}^{b}=1$.
}
\cadre{vert}{Propriété}{%id="p50"
     Si $X$ suit une \textbf{loi uniforme} sur $\left[a;b\right]$, alors pour tous réels $c$ et $d$ compris entre $a$ et $b$ avec $c < d$ :
     \begin{center}$p\left(c\leqslant X\leqslant d\right) = \frac{d-c}{b-a}$.\end{center}
}
\bloc{orange}{Démonstration}{%id="r50"
     En effet, si $a\leqslant c < d \leqslant b$ alors :
     \par
     $p\left(c \leqslant X\leqslant d \right)=\int_{c}^{d}\frac{1}{b-a}dx$\nosp$=\frac{d-c}{b-a}$
}
\cadre{rouge}{Théorème}{%id="t60"
     L'espérance mathématique d'une variable aléatoire $X$ qui suit une \textbf{loi uniforme} sur $\left[a;b\right]$ est :
     \begin{center}$E\left(X\right)=\frac{a+b}{2}$.\end{center}
}
\bloc{orange}{Démonstration}{%"m65"
     La fonction $x \longmapsto \dfrac{x^2}{2(b-a)}$ est une primitive de la fonction $x \longmapsto \dfrac{x}{b-a}$ sur $[a~;~b]$ ; par conséquent :
     \par
     $E\left(X\right) =\int_{a}^{ b}\frac{x}{b-a}dx$\\
     $\phantom{E\left(X\right)} =\left[\frac{x^{2}}{2\left(b-a\right)}\right]_{a}^{b}$\\
     $\phantom{E\left(X\right)}=\frac{b^{2}-a^{2}}{2\left(b-a\right)}$\\
     $\phantom{E\left(X\right)}=\frac{\left(b-a\right)\left(b+a\right)}{2\left(b-a\right)}$\\
     $\phantom{E\left(X\right)}=\frac{a+b}{2}$.
}
\begin{h2}3. Loi exponentielle de paramètre lambda\end{h2}
\cadre{bleu}{Définition}{%id="d70"
     On dit qu'une variable aléatoire X suit une \textbf{loi exponentielle de paramètre $\lambda  > 0$} sur $\left[0;+\infty \right[$ si sa densité de probabilité $f$ est définie sur $\left[0;+\infty \right[$ par :
     \begin{center}
          $f\left(x\right)=\lambda \text{e}^{-\lambda x}$.
     \end{center}
}
\bloc{orange}{Exemple}{%id="e70"
     La densité de la loi exponentielle de paramètre $\lambda =1,5$ est la fonction $f$ définie sur $\left[0;+\infty \right[$ par $f\left(x\right)=1,5 \text{e}^{-1,5 x}$.
     \par
     Cette fonction est représentée ci-dessous :
     \begin{center}
          \begin{extern} %width="350" alt="loi exponentielle"
               % -+-+-+ variables modifiables
               \def\e{2.7182818}
               \def\fonction{1.5*\e^(-1.5*x)  }
               \def\xmin{0}
               \def\xmax{3.5}
               \def\ymin{0}
               \def\ymax{2}
               \def\xunit{2}  % unités en cm
               \def\yunit{2}
               \begin{pspicture}(0,-0.5)(7,4.5)
                    \psset{xunit=\xunit, yunit=\yunit, algebraic=true}
                    %\psgrid[gridcolor=mcgris, subgriddiv=0, gridlabels=0pt](0,0)(2,1)
                    \psaxes{->}(0,0)(\xmin,\ymin)(\xmax,\ymax)
                    \psclip{
                         \psframe[linestyle=none](\xmin,\ymin)(\xmax,\ymax)
                    }
                    \psplot[linecolor=red,linewidth=0.75pt]{\xmin}{\xmax}{\fonction}
                    \endpsclip
                    \uput[l](0,1.5){$\red \lambda$}
               \end{pspicture}
          \end{extern}
     \end{center}
}
\bloc{mauve}{Remarque}{%id="r70"
     La fonction $x \longmapsto -\text{e}^{-\lambda x}$ est une primitive de la fonction $x \longmapsto \lambda \text{e}^{-\lambda x}$.
     \par
     On vérifie alors que :
     \par
     $\int_{0}^{+\infty } \lambda \text{e}^{-\lambda x} dx=\lim\limits_{t\rightarrow +\infty }\int_{0}^{t} \lambda \text{e}^{-\lambda x} dx$\\
     $\phantom{\int_{0}^{+\infty } \lambda \text{e}^{-\lambda x} dx}=\lim\limits_{t\rightarrow +\infty }\left[-\text{e}^{-\lambda x}\right]_{0}^{t}$\\
     $\phantom{\int_{0}^{+\infty } \lambda \text{e}^{-\lambda x} dx}=\lim\limits_{t\rightarrow +\infty }-\text{e}^{-\lambda t}+1=1$.
}
\cadre{vert}{Propriété}{%id="p80"
     Si $X$ suit une exponentielle de paramètre $\lambda $ sur $\left[0;+\infty \right[$, alors pour tous réels positifs $x_{1}$ et $x_{2}$ :
     \begin{itemize}
          \item %
          $p\left(x_{1}\leqslant X\leqslant x_{2}\right) = \text{e}^{-\lambda x_{1}}-\text{e}^{-\lambda x_{2}}$
          \item %
          $p\left(X\geqslant x_{1}\right) = \text{e}^{-\lambda x_{1}}$.
     \end{itemize}
}
\bloc{orange}{Démonstration}{%id="d80"
     $p\left(x_{1}\leqslant X\leqslant x_{2}\right)=\int_{x_{1}}^{x_{2}}\lambda \text{e}^{-\lambda x} dx$\\
     $\phantom{p\left(x_{1}\leqslant X\leqslant x_{2}\right)}=\left[-\text{e}^{-\lambda x}\right]_{x_{1}}^{x_{2}}$\\
     $\phantom{p\left(x_{1}\leqslant X\leqslant x_{2}\right)}=\text{e}^{-\lambda x_{1}}- \text{e}^{-\lambda x_{2}}$
     \par
     La seconde égalité s'obtient alors en faisant tendre $x_{2}$ vers $+\infty $.
}
\cadre{rouge}{Théorème}{%id="t90"
     L'espérance mathématique d'une variable aléatoire $X$ qui suit une \textbf{loi exponentielle} de paramètre $\lambda $ est :
     \begin{center}$E\left(X\right)=\frac{1}{\lambda}$\end{center}
}
\bloc{orange}{Démonstration}{%id="r95"
     Voir exercice : \Lien{https://www.maths-cours.fr/exercices/roc-esperance-mathematique-dune-loi-exponentielle/}{[ROC] Espérance mathématique d'une loi exponentielle}.
}
\cadre{vert}{Propriété}{%id="p100"
     Soient $X$ une variable aléatoire qui suit une exponentielle de paramètre $\lambda $ et $x$ et $x_{0}$ deux réels, alors :
     \begin{center}$p\left(X > x\right) = p_{(X > x_{0})}\left(X > x+x_{0}\right)$\end{center}
     On dit qu'une loi exponentielle est \og \textit{sans vieillissement} \fg{}.
}
\bloc{mauve}{Commentaire}{%id="r110"
     Tout d'abord, rappelons que la notation $p_{(X > x_{0})}\left(X > x+x_{0}\right)$ indique la probabilité (conditionnelle) de l'événement $\left(X > x+x_{0}\right)$ \textbf{sachant que} l'événement $(X > x_{0})$ est réalisé.
     \par
     Supposons que $X$ modélise la durée de vie d'une machine.
     \par
     \begin{itemize}
          \item %
          $p\left(X > x\right)$ correspond à la probabilité qu'une machine \og neuve \fg{} fonctionne pendant une durée supérieure ou égale à $x$ ;
          \item %
          $p_{(X > x_{0})}\left(X > x+x_{0}\right)$ est la probabilité qu'une machine, qui a déjà fonctionné pendant une durée $x_0$, fonctionne encore pendant une durée supérieure ou égale à $x$.
     \end{itemize}
     Dans le cadre d'une loi exponentielle, ces probabilités sont égales ce qui explique l'expression \og \textit{sans vieillissement} \fg{}.
}
\bloc{orange}{Démonstration}{%id="m120"
     Voir exercice : \Lien{https://www.maths-cours.fr/exercices/loi-exponentielle-bac-s-metropole-2008/}{Loi exponentielle - Bac S Métropole 2008}.
}

\end{document}
µ
\documentclass[a4paper]{article}

%================================================================================================================================
%
% Packages
%
%================================================================================================================================

\usepackage[T1]{fontenc} 	% pour caractères accentués
\usepackage[utf8]{inputenc}  % encodage utf8
\usepackage[french]{babel}	% langue : français
\usepackage{fourier}			% caractères plus lisibles
\usepackage[dvipsnames]{xcolor} % couleurs
\usepackage{fancyhdr}		% réglage header footer
\usepackage{needspace}		% empêcher sauts de page mal placés
\usepackage{graphicx}		% pour inclure des graphiques
\usepackage{enumitem,cprotect}		% personnalise les listes d'items (nécessaire pour ol, al ...)
\usepackage{hyperref}		% Liens hypertexte
\usepackage{pstricks,pst-all,pst-node,pstricks-add,pst-math,pst-plot,pst-tree,pst-eucl} % pstricks
\usepackage[a4paper,includeheadfoot,top=2cm,left=3cm, bottom=2cm,right=3cm]{geometry} % marges etc.
\usepackage{comment}			% commentaires multilignes
\usepackage{amsmath,environ} % maths (matrices, etc.)
\usepackage{amssymb,makeidx}
\usepackage{bm}				% bold maths
\usepackage{tabularx}		% tableaux
\usepackage{colortbl}		% tableaux en couleur
\usepackage{fontawesome}		% Fontawesome
\usepackage{environ}			% environment with command
\usepackage{fp}				% calculs pour ps-tricks
\usepackage{multido}			% pour ps tricks
\usepackage[np]{numprint}	% formattage nombre
\usepackage{tikz,tkz-tab} 			% package principal TikZ
\usepackage{pgfplots}   % axes
\usepackage{mathrsfs}    % cursives
\usepackage{calc}			% calcul taille boites
\usepackage[scaled=0.875]{helvet} % font sans serif
\usepackage{svg} % svg
\usepackage{scrextend} % local margin
\usepackage{scratch} %scratch
\usepackage{multicol} % colonnes
%\usepackage{infix-RPN,pst-func} % formule en notation polanaise inversée
\usepackage{listings}

%================================================================================================================================
%
% Réglages de base
%
%================================================================================================================================

\lstset{
language=Python,   % R code
literate=
{á}{{\'a}}1
{à}{{\`a}}1
{ã}{{\~a}}1
{é}{{\'e}}1
{è}{{\`e}}1
{ê}{{\^e}}1
{í}{{\'i}}1
{ó}{{\'o}}1
{õ}{{\~o}}1
{ú}{{\'u}}1
{ü}{{\"u}}1
{ç}{{\c{c}}}1
{~}{{ }}1
}


\definecolor{codegreen}{rgb}{0,0.6,0}
\definecolor{codegray}{rgb}{0.5,0.5,0.5}
\definecolor{codepurple}{rgb}{0.58,0,0.82}
\definecolor{backcolour}{rgb}{0.95,0.95,0.92}

\lstdefinestyle{mystyle}{
    backgroundcolor=\color{backcolour},   
    commentstyle=\color{codegreen},
    keywordstyle=\color{magenta},
    numberstyle=\tiny\color{codegray},
    stringstyle=\color{codepurple},
    basicstyle=\ttfamily\footnotesize,
    breakatwhitespace=false,         
    breaklines=true,                 
    captionpos=b,                    
    keepspaces=true,                 
    numbers=left,                    
xleftmargin=2em,
framexleftmargin=2em,            
    showspaces=false,                
    showstringspaces=false,
    showtabs=false,                  
    tabsize=2,
    upquote=true
}

\lstset{style=mystyle}


\lstset{style=mystyle}
\newcommand{\imgdir}{C:/laragon/www/newmc/assets/imgsvg/}
\newcommand{\imgsvgdir}{C:/laragon/www/newmc/assets/imgsvg/}

\definecolor{mcgris}{RGB}{220, 220, 220}% ancien~; pour compatibilité
\definecolor{mcbleu}{RGB}{52, 152, 219}
\definecolor{mcvert}{RGB}{125, 194, 70}
\definecolor{mcmauve}{RGB}{154, 0, 215}
\definecolor{mcorange}{RGB}{255, 96, 0}
\definecolor{mcturquoise}{RGB}{0, 153, 153}
\definecolor{mcrouge}{RGB}{255, 0, 0}
\definecolor{mclightvert}{RGB}{205, 234, 190}

\definecolor{gris}{RGB}{220, 220, 220}
\definecolor{bleu}{RGB}{52, 152, 219}
\definecolor{vert}{RGB}{125, 194, 70}
\definecolor{mauve}{RGB}{154, 0, 215}
\definecolor{orange}{RGB}{255, 96, 0}
\definecolor{turquoise}{RGB}{0, 153, 153}
\definecolor{rouge}{RGB}{255, 0, 0}
\definecolor{lightvert}{RGB}{205, 234, 190}
\setitemize[0]{label=\color{lightvert}  $\bullet$}

\pagestyle{fancy}
\renewcommand{\headrulewidth}{0.2pt}
\fancyhead[L]{maths-cours.fr}
\fancyhead[R]{\thepage}
\renewcommand{\footrulewidth}{0.2pt}
\fancyfoot[C]{}

\newcolumntype{C}{>{\centering\arraybackslash}X}
\newcolumntype{s}{>{\hsize=.35\hsize\arraybackslash}X}

\setlength{\parindent}{0pt}		 
\setlength{\parskip}{3mm}
\setlength{\headheight}{1cm}

\def\ebook{ebook}
\def\book{book}
\def\web{web}
\def\type{web}

\newcommand{\vect}[1]{\overrightarrow{\,\mathstrut#1\,}}

\def\Oij{$\left(\text{O}~;~\vect{\imath},~\vect{\jmath}\right)$}
\def\Oijk{$\left(\text{O}~;~\vect{\imath},~\vect{\jmath},~\vect{k}\right)$}
\def\Ouv{$\left(\text{O}~;~\vect{u},~\vect{v}\right)$}

\hypersetup{breaklinks=true, colorlinks = true, linkcolor = OliveGreen, urlcolor = OliveGreen, citecolor = OliveGreen, pdfauthor={Didier BONNEL - https://www.maths-cours.fr} } % supprime les bordures autour des liens

\renewcommand{\arg}[0]{\text{arg}}

\everymath{\displaystyle}

%================================================================================================================================
%
% Macros - Commandes
%
%================================================================================================================================

\newcommand\meta[2]{    			% Utilisé pour créer le post HTML.
	\def\titre{titre}
	\def\url{url}
	\def\arg{#1}
	\ifx\titre\arg
		\newcommand\maintitle{#2}
		\fancyhead[L]{#2}
		{\Large\sffamily \MakeUppercase{#2}}
		\vspace{1mm}\textcolor{mcvert}{\hrule}
	\fi 
	\ifx\url\arg
		\fancyfoot[L]{\href{https://www.maths-cours.fr#2}{\black \footnotesize{https://www.maths-cours.fr#2}}}
	\fi 
}


\newcommand\TitreC[1]{    		% Titre centré
     \needspace{3\baselineskip}
     \begin{center}\textbf{#1}\end{center}
}

\newcommand\newpar{    		% paragraphe
     \par
}

\newcommand\nosp {    		% commande vide (pas d'espace)
}
\newcommand{\id}[1]{} %ignore

\newcommand\boite[2]{				% Boite simple sans titre
	\vspace{5mm}
	\setlength{\fboxrule}{0.2mm}
	\setlength{\fboxsep}{5mm}	
	\fcolorbox{#1}{#1!3}{\makebox[\linewidth-2\fboxrule-2\fboxsep]{
  		\begin{minipage}[t]{\linewidth-2\fboxrule-4\fboxsep}\setlength{\parskip}{3mm}
  			 #2
  		\end{minipage}
	}}
	\vspace{5mm}
}

\newcommand\CBox[4]{				% Boites
	\vspace{5mm}
	\setlength{\fboxrule}{0.2mm}
	\setlength{\fboxsep}{5mm}
	
	\fcolorbox{#1}{#1!3}{\makebox[\linewidth-2\fboxrule-2\fboxsep]{
		\begin{minipage}[t]{1cm}\setlength{\parskip}{3mm}
	  		\textcolor{#1}{\LARGE{#2}}    
 	 	\end{minipage}  
  		\begin{minipage}[t]{\linewidth-2\fboxrule-4\fboxsep}\setlength{\parskip}{3mm}
			\raisebox{1.2mm}{\normalsize\sffamily{\textcolor{#1}{#3}}}						
  			 #4
  		\end{minipage}
	}}
	\vspace{5mm}
}

\newcommand\cadre[3]{				% Boites convertible html
	\par
	\vspace{2mm}
	\setlength{\fboxrule}{0.1mm}
	\setlength{\fboxsep}{5mm}
	\fcolorbox{#1}{white}{\makebox[\linewidth-2\fboxrule-2\fboxsep]{
  		\begin{minipage}[t]{\linewidth-2\fboxrule-4\fboxsep}\setlength{\parskip}{3mm}
			\raisebox{-2.5mm}{\sffamily \small{\textcolor{#1}{\MakeUppercase{#2}}}}		
			\par		
  			 #3
 	 		\end{minipage}
	}}
		\vspace{2mm}
	\par
}

\newcommand\bloc[3]{				% Boites convertible html sans bordure
     \needspace{2\baselineskip}
     {\sffamily \small{\textcolor{#1}{\MakeUppercase{#2}}}}    
		\par		
  			 #3
		\par
}

\newcommand\CHelp[1]{
     \CBox{Plum}{\faInfoCircle}{À RETENIR}{#1}
}

\newcommand\CUp[1]{
     \CBox{NavyBlue}{\faThumbsOUp}{EN PRATIQUE}{#1}
}

\newcommand\CInfo[1]{
     \CBox{Sepia}{\faArrowCircleRight}{REMARQUE}{#1}
}

\newcommand\CRedac[1]{
     \CBox{PineGreen}{\faEdit}{BIEN R\'EDIGER}{#1}
}

\newcommand\CError[1]{
     \CBox{Red}{\faExclamationTriangle}{ATTENTION}{#1}
}

\newcommand\TitreExo[2]{
\needspace{4\baselineskip}
 {\sffamily\large EXERCICE #1\ (\emph{#2 points})}
\vspace{5mm}
}

\newcommand\img[2]{
          \includegraphics[width=#2\paperwidth]{\imgdir#1}
}

\newcommand\imgsvg[2]{
       \begin{center}   \includegraphics[width=#2\paperwidth]{\imgsvgdir#1} \end{center}
}


\newcommand\Lien[2]{
     \href{#1}{#2 \tiny \faExternalLink}
}
\newcommand\mcLien[2]{
     \href{https~://www.maths-cours.fr/#1}{#2 \tiny \faExternalLink}
}

\newcommand{\euro}{\eurologo{}}

%================================================================================================================================
%
% Macros - Environement
%
%================================================================================================================================

\newenvironment{tex}{ %
}
{%
}

\newenvironment{indente}{ %
	\setlength\parindent{10mm}
}

{
	\setlength\parindent{0mm}
}

\newenvironment{corrige}{%
     \needspace{3\baselineskip}
     \medskip
     \textbf{\textsc{Corrigé}}
     \medskip
}
{
}

\newenvironment{extern}{%
     \begin{center}
     }
     {
     \end{center}
}

\NewEnviron{code}{%
	\par
     \boite{gray}{\texttt{%
     \BODY
     }}
     \par
}

\newenvironment{vbloc}{% boite sans cadre empeche saut de page
     \begin{minipage}[t]{\linewidth}
     }
     {
     \end{minipage}
}
\NewEnviron{h2}{%
    \needspace{3\baselineskip}
    \vspace{0.6cm}
	\noindent \MakeUppercase{\sffamily \large \BODY}
	\vspace{1mm}\textcolor{mcgris}{\hrule}\vspace{0.4cm}
	\par
}{}

\NewEnviron{h3}{%
    \needspace{3\baselineskip}
	\vspace{5mm}
	\textsc{\BODY}
	\par
}

\NewEnviron{margeneg}{ %
\begin{addmargin}[-1cm]{0cm}
\BODY
\end{addmargin}
}

\NewEnviron{html}{%
}

\begin{document}
\meta{url}{/cours/loi-normale/}
\meta{pid}{492}
\meta{titre}{Lois normales}
\meta{type}{cours}
\begin{h2}1. Loi normale centrée réduite\end{h2}
\cadre{bleu}{Définition}{%id="d10"
     On dit qu'une variable aléatoire $X$ suit la \textbf{loi normale centrée réduite} sur $\mathbb{R}$ (notée $\mathscr N \left(0;1\right)$) si sa densité de probabilité $f$ est définie par :
     \begin{center}$f\left(x\right)=\frac{1}{\sqrt{2\pi }}e^{^{-\frac{x^{2}}{2}}}$\end{center}
     Cela signifie que, pour tous réels $a$ et $b$ tels que $a\leqslant b$:
     \begin{center}$p\left(a\leqslant X\leqslant b\right)=\int_{a}^{ b}\frac{1}{\sqrt{2\pi }}e^{^{-\frac{t^{2}}{2}}}dt$\end{center}
}
\bloc{cyan}{Remarques}{%id="r10"
     \begin{itemize}
          \item On admet que $f$ définit bien une densité, c'est à dire que l'aire comprise entre l'axe des abscisses et la courbe représentative de $f$ est égale à 1
          \item On a également :
          \par
          $p\left(X\geqslant a\right) =\lim\limits_{x\rightarrow +\infty }\int_{a}^{ x}\frac{1}{\sqrt{2\pi }}e^{^{-\frac{t^{2}}{2}}}dt $ (limite que l'on peut noter : $\int_{a}^{ +\infty }\frac{1}{\sqrt{2\pi }}e^{^{-\frac{t^{2}}{2}}}dt $)
          \par
          $p\left(X\leqslant b\right) =\lim\limits_{x\rightarrow -\infty }\int_{x}^{b}\frac{1}{\sqrt{2\pi }}e^{^{-\frac{t^{2}}{2}}}dt $ (limite que l'on peut noter : $\int_{-\infty }^{ b}\frac{1}{\sqrt{2\pi }}e^{^{-\frac{t^{2}}{2}}}dt $)
          \item La fonction $f : x\mapsto \frac{1}{\sqrt{2\pi }}e^{^{-\frac{x^{2}}{2}}}$ est dérivable sur $\mathbb{R}$, paire, positive, son tableau de variation est :
          %:-+-+-+-+- Engendré par : http://math.et.info.free.fr/TikZ/TableauxVariations/
          \begin{center}
               \begin{extern}%width="350"
                    \begin{tikzpicture}[scale=0.875]
                         % Styles
                         \tikzstyle{cadre}=[thin]
                         \tikzstyle{fleche}=[->,>=latex,thin]
                         \tikzstyle{nondefini}=[lightgray]
                         % Dimensions Modifiables
                         \def\Lrg{1.5}
                         \def\HtX{1}
                         \def\HtY{0.5}
                         % Dimensions Calculées
                         \def\lignex{-0.5*\HtX}
                         \def\lignef{-1.5*\HtX}
                         \def\separateur{-0.5*\Lrg}
                         % Largeur du tableau
                         \def\gauche{-1.5*\Lrg}
                         \def\droite{4.5*\Lrg}
                         % Hauteur du tableau
                         \def\haut{0.5*\HtX}
                         \def\bas{-2.5*\HtX-2*\HtY}
                         % Ligne de l'abscisse : x
                         \node at (-1*\Lrg,0) {$x$};
                         \node at (0*\Lrg,0) {$-\infty$};
                         \node at (2*\Lrg,0) {$0$};
                         \node at (4*\Lrg,0) {$+\infty$};
                         % Ligne de la dérivée : f'(x)
                         \node at (-1*\Lrg,-1*\HtX) {$f'(x)$};
                         \node at (0*\Lrg,-1*\HtX) {$$};
                         \node at (1*\Lrg,-1*\HtX) {$+$};
                         \node at (2*\Lrg,-1*\HtX) {$0$};
                         \node at (3*\Lrg,-1*\HtX) {$-$};
                         \node at (4*\Lrg,-1*\HtX) {$$};
                         % Ligne de la fonction : f(x)
                         \node  at (-1*\Lrg,{-2*\HtX+(-1)*\HtY}) {$f(x)$};
                         \node (f1) at (0*\Lrg,{-2*\HtX+(-2)*\HtY}) {$0$};
                         \node (f2) at (2*\Lrg,{-2*\HtX+(-0.5)*\HtY}) {$\dfrac{1}{\sqrt{2\pi}}$};
                         \node (f3) at (4*\Lrg,{-2*\HtX+(-2)*\HtY}) {$0$};
                         % Flèches
                         \draw[fleche] (f1) -- (f2);
                         \draw[fleche] (f2) -- (f3);
                         % Encadrement
                         \draw[cadre] (\separateur,\haut) -- (\separateur,\bas);
                         \draw[cadre] (\gauche,\haut) rectangle  (\droite,\bas);
                         \draw[cadre] (\gauche,\lignex) -- (\droite,\lignex);
                         \draw[cadre] (\gauche,\lignef) -- (\droite,\lignef);
                    \end{tikzpicture}
               \end{extern}
          \end{center}
          et sa courbe représentative :
          \begin{center}
               \begin{extern} %width="550" alt="loi normale centrée réduite"
                    % -+-+-+ variables modifiables
                    \def\e{2.7182818}
                    \def\pi{3.1415926}
                    \def\fonction{1/sqrt(2*\pi)*\e^(-x*x/2)  }
                    \def\xmin{-3.5}
                    \def\xmax{3.5}
                    \def\ymin{0}
                    \def\ymax{0.5}
                    \def\xunit{2}  % unités en cm
                    \def\yunit{12}
                    \psset{xunit=\xunit,yunit=\yunit,algebraic=true}
                    \begin{pspicture*}(\xmin,-0.08)(\xmax,\ymax)
                         %      \psgrid[gridcolor=mcgris, subgriddiv=5, gridlabels=0pt](-5,-0.3)(5,1)
                         \psaxes[Dy=0.1,linewidth=0.75pt]{->}(0,0)(\xmin,-0.08)(\xmax,\ymax)
                         \psplot[plotpoints=2000,linecolor=red,linewidth=0.75pt]{\xmin}{\xmax}{\fonction}
                         \rput[br](-0.1,-0.045){$O$}
                    \end{pspicture*}
               \end{extern}
          \end{center}
          \item $p\left(a\leqslant X\leqslant b\right)$ est l'aire du domaine coloré ci-dessous :
          \begin{center}
               \begin{extern} %width="550" alt="loi normale probabilité"
                    % -+-+-+ variables modifiables
                    \def\e{2.7182818}
                    \def\pi{3.1415926}
                    \def\fonction{1/sqrt(2*\pi)*\e^(-x*x/2)  }
                    \def\xmin{-3.5}
                    \def\xmax{3.5}
                    \def\ymin{0}
                    \def\ymax{0.5}
                    \def\xunit{2}  % unités en cm
                    \def\yunit{12}
                    \psset{xunit=\xunit,yunit=\yunit,algebraic=true}
                    \begin{pspicture*}(\xmin,-0.08)(\xmax,\ymax)
                         %      \psgrid[gridcolor=mcgris, subgriddiv=5, gridlabels=0pt](-5,-0.3)(5,1)
                         \psaxes[Dy=0.1,linewidth=0.75pt]{->}(0,0)(\xmin,-0.08)(\xmax,\ymax)
                         \psplot[plotpoints=2000,linecolor=red,linewidth=0.75pt]{\xmin}{\xmax}{\fonction}
                         \pscustom[fillcolor=blue,fillstyle=solid,opacity=0.2,linecolor=blue]{
                              \moveto(1.14,0)
                              \psplot[plotpoints=3000,linewidth=0.75pt]{1.14}{2.24}{\fonction}
                              \psline(2.24,0)(1.14,0)
                         }
                         \rput[b](1.14,-0.045){$\color{blue} a$}
                         \rput[b](2.24,-0.045){$\color{blue} b$}
                         \rput[br](-0.1,-0.045){$O$}
                    \end{pspicture*}
               \end{extern}
          \end{center}
          \item $p\left(X\leqslant a\right)$ est l'aire du domaine coloré ci-dessous :
          \begin{center}
               \begin{extern} %width="550" alt="loi normale probabilité"
                    % -+-+-+ variables modifiables
                    \def\e{2.7182818}
                    \def\pi{3.1415926}
                    \def\fonction{1/sqrt(2*\pi)*\e^(-x*x/2)  }
                    \def\xmin{-3.5}
                    \def\xmax{3.5}
                    \def\ymin{0}
                    \def\ymax{0.5}
                    \def\xunit{2}  % unités en cm
                    \def\yunit{12}
                    \psset{xunit=\xunit,yunit=\yunit,algebraic=true}
                    \begin{pspicture*}(\xmin,-0.08)(\xmax,\ymax)
                         %      \psgrid[gridcolor=mcgris, subgriddiv=5, gridlabels=0pt](-5,-0.3)(5,1)
                         \psaxes[Dy=0.1,linewidth=0.75pt]{->}(0,0)(\xmin,-0.08)(\xmax,\ymax)
                         \psplot[plotpoints=2000,linecolor=red,linewidth=0.75pt]{\xmin}{\xmax}{\fonction}
                         \pscustom[fillcolor=blue,fillstyle=solid,opacity=0.2,linecolor=blue]{
                              \psplot[plotpoints=3000,linewidth=0.75pt]{-4}{1.14}{\fonction}
                              \psline(1.14,0)(-4,0)
                         }
                         \rput[b](1.14,-0.045){$\color{blue} a$}
                         \rput[br](-0.1,-0.045){$O$}
                    \end{pspicture*}
               \end{extern}
          \end{center}
     \end{itemize}
}
\cadre{vert}{Propriétés}{%id="p20"
     Soit $X$ une variable aléatoire qui suit la loi normale centrée réduite :
     \begin{itemize}
          \item L'espérance mathématique de $X$ est $E\left(X\right)=0$ (loi \textit{centrée})~;
          \item La variance de $X$ est $\sigma \left(X\right)=1$ (loi \textit{réduite}).
     \end{itemize}
}
\cadre{vert}{Propriétés}{%id="p30"
     Soit $X$ une variable aléatoire qui suit la loi normale centrée réduite et $ a $ un réel quelconque :
     \begin{itemize}
          \item $p\left(X\leqslant 0\right)=p\left(X\geqslant 0\right)=0,5$
          \item $p\left(X\leqslant -a\right)=p\left(X\geqslant a\right)$
          \item $p\left(-a \leqslant X\leqslant a\right)=1-2\times p\left(X\geqslant a\right)$\nosp$=2\times p\left(X\leqslant a\right)-1$
     \end{itemize}
}
\bloc{cyan}{Remarque}{%id="r30"
     Ces propriétés résultent du fait que :
     \begin{itemize}
          \item  la courbe de la fonction $x\mapsto \frac{1}{\sqrt{2\pi }}e^{^{-\frac{x^{2}}{2}}}$ est symétrique par rapport à l'axe des ordonnées
          \item l'aire comprise entre l'axe des abscisses et la courbe est égale à 1.
     \end{itemize}
     On retrouve facilement ces propriétés à l'aide d'une figure par exemple pour la seconde formule :
     \begin{center}
          \begin{extern} %width="550" alt="loi normale symétrie"
               % -+-+-+ variables modifiables
               \def\e{2.7182818}
               \def\pi{3.1415926}
               \def\fonction{1/sqrt(2*\pi)*\e^(-x*x/2)  }
               \def\xmin{-3.5}
               \def\xmax{3.5}
               \def\ymin{0}
               \def\ymax{0.5}
               \def\xunit{2}  % unités en cm
               \def\yunit{12}
               \psset{xunit=\xunit,yunit=\yunit,algebraic=true}
               \begin{pspicture*}(\xmin,-0.08)(\xmax,\ymax)
                    %      \psgrid[gridcolor=mcgris, subgriddiv=5, gridlabels=0pt](-5,-0.3)(5,1)
                    \psaxes[Dy=0.1,linewidth=0.75pt]{->}(0,0)(\xmin,-0.08)(\xmax,\ymax)
                    \psplot[plotpoints=2000,linecolor=red,linewidth=0.75pt]{\xmin}{\xmax}{\fonction}
                    \pscustom[fillcolor=vert,fillstyle=solid,opacity=0.2,linecolor=vert]{
                         \psplot[plotpoints=3000,linewidth=0.75pt]{-4}{-1.14}{\fonction}
                         \psline(-1.14,0)(-4,0)
                    }
                    \pscustom[fillcolor=blue,fillstyle=solid,opacity=0.2,linecolor=blue]{
                         \moveto(1.14,0)
                         \psplot[plotpoints=3000,linewidth=0.75pt]{1.14}{4}{\fonction}
                         \psline(4,0)(1.14,0)
                    }
                    \rput[br](-1.14,-0.045){$\color{vert} -a$}
                    \rput[b](1.14,-0.045){$\color{blue} a$}
                    \rput[br](-0.1,-0.045){$O$}
               \end{pspicture*}
          \end{extern}
     \end{center}
     \begin{center}$p\left(X\leqslant -a\right)=p\left(X\geqslant a\right)$\end{center}
}
\cadre{vert}{Propriété («Loi normale inverse»)}{%id="p50"
     Soient $X$ une variable aléatoire qui suit la loi normale centrée réduite et un réel $k \in \left]0;1\right[ $.
     \par
     Il existe un \textbf{unique} réel $m_{k}$ tel que $p\left(X\leqslant m_{k}\right)=k$.
}
\bloc{cyan}{Remarque}{%id="r50"
     On peut calculer les valeurs de $m_{k}$ à la calculatrice.
}
\cadre{rouge}{Théorème}{%id="t60"
     Soient $X$ une variable aléatoire qui suit la loi normale centrée réduite et un réel $\alpha \in \left]0;1\right[ $.
     \par
     Il existe un \textbf{unique} réel $u_\alpha $ tel que :
     \begin{center}
          $p\left(-u_\alpha \leqslant X\leqslant u_\alpha \right)=1-\alpha$.
     \end{center}
}
\begin{center}
     \begin{extern} %width="550" alt="loi normale seuil"
          % -+-+-+ variables modifiables
          \def\e{2.7182818}
          \def\pi{3.1415926}
          \def\fonction{1/sqrt(2*\pi)*\e^(-x*x/2)  }
          \def\xmin{-3.5}
          \def\xmax{3.5}
          \def\ymin{0}
          \def\ymax{0.5}
          \def\xunit{2}  % unités en cm
          \def\yunit{12}
          \psset{xunit=\xunit,yunit=\yunit,algebraic=true}
          \begin{pspicture*}(\xmin,-0.08)(\xmax,\ymax)
               %      \psgrid[gridcolor=mcgris, subgriddiv=5, gridlabels=0pt](-5,-0.3)(5,1)
               \psaxes[Dy=0.1,linewidth=0.75pt]{->}(0,0)(\xmin,-0.08)(\xmax,\ymax)
               \psplot[plotpoints=2000,linecolor=red,linewidth=0.75pt]{\xmin}{\xmax}{\fonction}
               \pscustom[fillcolor=blue,fillstyle=solid,opacity=0.1,linecolor=blue]{
                    \moveto(-1.6,0)
                    \psplot[plotpoints=3000,linewidth=0.75pt]{-1.6}{1.6}{\fonction}
                    \psline(1.6,0)(-1.6,0)
               }
               \rput[b](-1.6,-0.045){$\color{blue} -u_{\alpha}$}
               \rput[b](1.6,-0.045){$\color{blue} u_{\alpha}$}
               \rput[b](0.6,0.1){$\color{blue} 1-\alpha$}
               \rput[b](1.9,0.02){$\color{gray} \alpha/2$}
               \rput[b](-1.9,0.02){$\color{gray} \alpha/2$}
               \rput[br](-0.1,-0.045){$O$}
          \end{pspicture*}
     \end{extern}
\end{center}
\begin{center}
     $p\left(-u_\alpha \leqslant X\leqslant u_\alpha \right)=1-\alpha $
\end{center}
\bloc{cyan}{Remarques}{%id="r60"
     \begin{itemize}
          \item En utilisant la formule $p\left(-\alpha \leqslant X\leqslant a\right)=2\times p\left(X\leqslant a\right)-1$ et la \textit{«loi normale inverse»} on peut calculer les valeurs de $u_\alpha $ à la calculatrice.
          \item Deux valeurs à retenir :
          \par
          $u_{0,05}=1,96 $ c'est à dire que $p\left(-1.96\leqslant X\leqslant 1.96\right)=0,95$
          \par
          $u_{0,01}=2,58 $ c'est à dire que $p\left(-2,58\leqslant X\leqslant 2,58\right)=0,99$
     \end{itemize}
}
\begin{h2}2. Loi normale d'espérance $\mu $ et d'écart-type $\sigma $ \end{h2}
\cadre{bleu}{Définition et théorème}{%id="d70"
     Soient deux réels $\mu $ et $\sigma  > 0$.
     \par
     On dit qu'une variable aléatoire $X$ suit une \textbf{loi normale de paramètres $\mu $ et $\sigma ^{2}$} (notée $\mathscr N \left(\mu ; \sigma ^{2}\right)$) si la variable aléatoire $Y=\frac{X-\mu }{\sigma} $ suit la loi normale centrée réduite.
     \par
     L'espérance mathématique de $X$ est $\mu $ et son écart-type $\sigma $ (et donc sa variance $\sigma ^{2}$).
}
\bloc{cyan}{Remarque}{%id="r70"
     La courbe représentative de la distribution d'une loi $\mathscr N \left(\mu ; \sigma ^{2}\right)$ est une courbe «~en cloche~» qui admet la droite d'équation $x=\mu $ comme axe de symétrie. Elle est plus ou moins «~étirée~» selon les valeurs de $\sigma $
     \begin{center}
          \begin{extern} %width="500" alt="loi normale différents écarts-types"
               % -+-+-+ variables modifiables
               \def\e{2.7182818}
               \def\pi{3.1415926}
               \def\f{2/sqrt(2*\pi)*\e^(-(x-3)*(x-3)/0.5)  }
               \def\g{1/sqrt(2*\pi)*\e^(-(x-3)*(x-3)/2)  }
               \def\h{0.5/sqrt(2*\pi)*\e^(-(x-3)*(x-3)/8)  }
               \def\xmin{-2.3}
               \def\xmax{8.3}
               \def\ymin{0}
               \def\ymax{0.85}
               \def\xunit{1}  % unités en cm
               \def\yunit{10}
               \psset{xunit=\xunit,yunit=\yunit,algebraic=true}
               \begin{pspicture*}(\xmin,-0.08)(\xmax,\ymax)
                    %      \psgrid[gridcolor=mcgris, subgriddiv=5, gridlabels=0pt](-5,-0.3)(5,1)
                    \psaxes[Dy=0.1,linewidth=0.75pt]{->}(0,0)(\xmin,-0.08)(\xmax,\ymax)
                    \psplot[plotpoints=2000,linecolor=blue,linewidth=0.75pt]{\xmin}{\xmax}{\f}
                    \psplot[plotpoints=2000,linecolor=red,linewidth=0.75pt]{\xmin}{\xmax}{\g}
                    \psplot[plotpoints=2000,linecolor=vert,linewidth=0.75pt]{\xmin}{\xmax}{\h}
                    \psline[linecolor=mauve,linewidth=0.5pt](3,10)(3,-1)
                    \rput[br](2.9,0.02){$\color{mauve} \mu$}
                    \rput[br](5,0.5){$\color{blue} \sigma=0,5$}
                    \rput[br](5.2,0.3){$\color{red} \sigma=1$}
                    \rput[br](6,0.15){$\color{vert} \sigma=2$}
                    \rput[br](-0.1,-0.045){$O$}
               \end{pspicture*}
          \end{extern}
     \end{center}
     \begin{center}$\mu =3$ et $ \sigma =0,5$ ; $1$ ; $2$\end{center}
}
\cadre{vert}{Propriété (Règle des trois sigmas)}{%id="p80"
     Si $X$ suit une loi normale $\mathscr N \left(\mu ; \sigma ^{2}\right)$ alors :
     \begin{itemize}
          \item
          $p\left(\mu -\sigma \leqslant X\leqslant \mu + \sigma \right)\approx 0,68$ (à $10^{-2}$ près)
          \item
          $p\left(\mu -2\sigma \leqslant X\leqslant \mu + 2\sigma \right)\approx 0,95$ (à $10^{-2}$ près)
          \item
          $p\left(\mu -3\sigma \leqslant X\leqslant \mu + 3\sigma \right)\approx 0,997$ (à $10^{-3}$ près)
     \end{itemize}
}
\bloc{orange}{Exemple}{%id="e80"
     Si $X$ suit une loi normale $\mathscr N \left(11 ; 3^{2}\right)$ alors :
     \par
     $p\left(5\leqslant X\leqslant 17\right)\approx 0,95$
}
\begin{h2}3. Théorème de Moivre-Laplace\end{h2}
\cadre{rouge}{Théorème (Moivre-Laplace)}{%id="t90"
     Soit $X_{n}$ une variable aléatoire qui suit une loi \textbf{binomiale} $\mathscr B \left(n;p\right)$.
     \par
     On pose $Z_{n}=\frac{X_{n}-E\left(X_{n}\right)}{\sigma \left(X_{n}\right)}$.
     \par
     Alors pour tous réels $a$ et $b$ :
     \begin{center}$\lim\limits_{n\rightarrow +\infty }p\left(a\leqslant Z_{n}\leqslant b\right)=\int_{a}^{ b}\frac{1}{\sqrt{2\pi }}e^{^{-\frac{t^{2}}{2}}}dt$\end{center}
}
\bloc{cyan}{Remarques}{%id="r90"
     \begin{itemize}
          \item On rappelle que pour une loi binomiale $X$ de paramètres $n$ et $p$ :$E\left(X\right)=np$ et $\sigma \left(X\right)^{2}=np\left(1-p\right)$. $Z_{n}$ peut donc aussi s'écrire : $Z_{n}=\frac{X_{n}-np}{\sqrt{np\left(1-p\right)}}$
          \item Ce théorème signifie que pour $n$ élevé, la loi de $Z_{n}$ est proche de la loi normale centrée réduite :
          \begin{center}
               \begin{extern} %width="550" alt="comparaison loi normale loi binomiale"
                    % -+-+-+ variables modifiables
                    \def\e{2.7182818}
                    \def\pi{3.1415926}
                    \def\fonction{1/sqrt(2*\pi)*\e^(-x*x/2)  }
                    \def\xmin{-3.5}
                    \def\xmax{3.5}
                    \def\ymin{0}
                    \def\ymax{0.5}
                    \def\xunit{2}  % unités en cm
                    \def\yunit{12}
                    \psset{xunit=\xunit,yunit=\yunit,algebraic=true,dimen=middle,}
                    \begin{pspicture*}(\xmin,-0.08)(\xmax,\ymax)
                         %      \psgrid[gridcolor=mcgris, subgriddiv=5, gridlabels=0pt](-5,-0.3)(5,1)
                         \psframe[linewidth=0.8pt,linecolor=blue,fillcolor=blue,fillstyle=solid,opacity=0.05](-0.612,0)(-1.021,0.2863)
                         \psframe[linewidth=0.8pt,linecolor=blue,fillcolor=blue,fillstyle=solid,opacity=0.05](-1.021,0)(-1.429,0.1909)
                         \psframe[linewidth=0.8pt,linecolor=blue,fillcolor=blue,fillstyle=solid,opacity=0.05](-1.429,0)(-1.837,0.1073)
                         \psframe[linewidth=0.8pt,linecolor=blue,fillcolor=blue,fillstyle=solid,opacity=0.05](-1.837,0)(-2.245,0.0505)
                         \psframe[linewidth=0.8pt,linecolor=blue,fillcolor=blue,fillstyle=solid,opacity=0.05](-2.245,0)(-2.654,0.0196)
                         \psframe[linewidth=0.8pt,linecolor=blue,fillcolor=blue,fillstyle=solid,opacity=0.05](-2.654,0)(-3.062,0.0062)
                         \psframe[linewidth=0.8pt,linecolor=blue,fillcolor=blue,fillstyle=solid,opacity=0.05](-3.062,0)(-3.47,0.0015)
                         \psframe[linewidth=0.8pt,linecolor=blue,fillcolor=blue,fillstyle=solid,opacity=0.05](-0.204,0)(-0.612,0.3644)
                         \psframe[linewidth=0.8pt,linecolor=blue,fillcolor=blue,fillstyle=solid,opacity=0.05](-0.204,0)(0.204,0.3948)
                         \psframe[linewidth=0.8pt,linecolor=blue,fillcolor=blue,fillstyle=solid,opacity=0.05](0.204,0)(0.612,0.3644)
                         \psframe[linewidth=0.8pt,linecolor=blue,fillcolor=blue,fillstyle=solid,opacity=0.05](0.612,0)(1.021,0.2863)
                         \psframe[linewidth=0.8pt,linecolor=blue,fillcolor=blue,fillstyle=solid,opacity=0.05](1.021,0)(1.429,0.1909)
                         \psframe[linewidth=0.8pt,linecolor=blue,fillcolor=blue,fillstyle=solid,opacity=0.05](1.429,0)(1.837,0.1073)
                         \psframe[linewidth=0.8pt,linecolor=blue,fillcolor=blue,fillstyle=solid,opacity=0.05](1.837,0)(2.245,0.0505)
                         \psframe[linewidth=0.8pt,linecolor=blue,fillcolor=blue,fillstyle=solid,opacity=0.05](2.245,0)(2.654,0.0196)
                         \psframe[linewidth=0.8pt,linecolor=blue,fillcolor=blue,fillstyle=solid,opacity=0.05](2.654,0)(3.062,0.0062)
                         \psframe[linewidth=0.8pt,linecolor=blue,fillcolor=blue,fillstyle=solid,opacity=0.05](3.062,0)(3.47,0.0015)
                         \psplot[plotpoints=2000,linecolor=red,linewidth=0.75pt]{\xmin}{\xmax}{\fonction}
                         \psaxes[Dy=0.1,linewidth=0.75pt]{->}(0,0)(\xmin,-0.08)(\xmax,\ymax)
                    \end{pspicture*}
               \end{extern}
               \par
               \textit{Histogramme de $Z_{n}$ pour $n=24$ et $p=0,5$ et loi $\mathscr N \left(0;1\right)$}
          \end{center}
          \item En pratique, on considèrera que «$n$ est suffisamment élevé» si $n\geqslant 30$ ; $np\geqslant 5$ ; $n\left(1-p\right)\geqslant 5$.
          \par
          La loi binomiale $X$ pourra alors être approximée par la loi normale $\mathscr N \left(E\left(X\right);\sigma \left(X\right)^{2}\right)$
     \end{itemize}
}
\bloc{orange}{Exemple}{%id="e90"
     $X$ suit une loi binomiale $\mathscr B \left(30 ; 0,4\right)$.
     \par
     On cherche à calculer $p\left(7 < X \leqslant 17\right)$.
     \par
     Posons $Z=\frac{X-30\times 0.4}{\sqrt{30\times 0.4\times 0.6}}=\frac{X-12}{\sqrt{7,2}}$.
     \bigskip
     Alors :
     \par
     $7 < X \leqslant 17 \Leftrightarrow -5 < X-12\leqslant 5 $
     \smallskip
     $\phantom{7 < X \leqslant 17} \Leftrightarrow -\frac{5}{\sqrt{7,2}} < \frac{X-12}{\sqrt{7,2}}\leqslant \frac{5}{\sqrt{7,2}}$
     \smallskip
     $\phantom{7 < X \leqslant 17}\Leftrightarrow -1,86 < Z\leqslant 1,86$
     \bigskip
     On a bien $n\geqslant 30$ ; $np\geqslant 5$ ; $n\left(1-p\right)\geqslant 5$. On peut donc approximer $Z$ par une loi normale centrée réduite.
     \par
     A la calculatrice on trouve alors :
     \par
     $p\left(-1,86 < Z \leqslant 1,86\right)\approx 0,937 $(un calcul direct avec la loi binomiale donne $0,935$)
}

\end{document}
µ
\documentclass[a4paper]{article}

%================================================================================================================================
%
% Packages
%
%================================================================================================================================

\usepackage[T1]{fontenc} 	% pour caractères accentués
\usepackage[utf8]{inputenc}  % encodage utf8
\usepackage[french]{babel}	% langue : français
\usepackage{fourier}			% caractères plus lisibles
\usepackage[dvipsnames]{xcolor} % couleurs
\usepackage{fancyhdr}		% réglage header footer
\usepackage{needspace}		% empêcher sauts de page mal placés
\usepackage{graphicx}		% pour inclure des graphiques
\usepackage{enumitem,cprotect}		% personnalise les listes d'items (nécessaire pour ol, al ...)
\usepackage{hyperref}		% Liens hypertexte
\usepackage{pstricks,pst-all,pst-node,pstricks-add,pst-math,pst-plot,pst-tree,pst-eucl} % pstricks
\usepackage[a4paper,includeheadfoot,top=2cm,left=3cm, bottom=2cm,right=3cm]{geometry} % marges etc.
\usepackage{comment}			% commentaires multilignes
\usepackage{amsmath,environ} % maths (matrices, etc.)
\usepackage{amssymb,makeidx}
\usepackage{bm}				% bold maths
\usepackage{tabularx}		% tableaux
\usepackage{colortbl}		% tableaux en couleur
\usepackage{fontawesome}		% Fontawesome
\usepackage{environ}			% environment with command
\usepackage{fp}				% calculs pour ps-tricks
\usepackage{multido}			% pour ps tricks
\usepackage[np]{numprint}	% formattage nombre
\usepackage{tikz,tkz-tab} 			% package principal TikZ
\usepackage{pgfplots}   % axes
\usepackage{mathrsfs}    % cursives
\usepackage{calc}			% calcul taille boites
\usepackage[scaled=0.875]{helvet} % font sans serif
\usepackage{svg} % svg
\usepackage{scrextend} % local margin
\usepackage{scratch} %scratch
\usepackage{multicol} % colonnes
%\usepackage{infix-RPN,pst-func} % formule en notation polanaise inversée
\usepackage{listings}

%================================================================================================================================
%
% Réglages de base
%
%================================================================================================================================

\lstset{
language=Python,   % R code
literate=
{á}{{\'a}}1
{à}{{\`a}}1
{ã}{{\~a}}1
{é}{{\'e}}1
{è}{{\`e}}1
{ê}{{\^e}}1
{í}{{\'i}}1
{ó}{{\'o}}1
{õ}{{\~o}}1
{ú}{{\'u}}1
{ü}{{\"u}}1
{ç}{{\c{c}}}1
{~}{{ }}1
}


\definecolor{codegreen}{rgb}{0,0.6,0}
\definecolor{codegray}{rgb}{0.5,0.5,0.5}
\definecolor{codepurple}{rgb}{0.58,0,0.82}
\definecolor{backcolour}{rgb}{0.95,0.95,0.92}

\lstdefinestyle{mystyle}{
    backgroundcolor=\color{backcolour},   
    commentstyle=\color{codegreen},
    keywordstyle=\color{magenta},
    numberstyle=\tiny\color{codegray},
    stringstyle=\color{codepurple},
    basicstyle=\ttfamily\footnotesize,
    breakatwhitespace=false,         
    breaklines=true,                 
    captionpos=b,                    
    keepspaces=true,                 
    numbers=left,                    
xleftmargin=2em,
framexleftmargin=2em,            
    showspaces=false,                
    showstringspaces=false,
    showtabs=false,                  
    tabsize=2,
    upquote=true
}

\lstset{style=mystyle}


\lstset{style=mystyle}
\newcommand{\imgdir}{C:/laragon/www/newmc/assets/imgsvg/}
\newcommand{\imgsvgdir}{C:/laragon/www/newmc/assets/imgsvg/}

\definecolor{mcgris}{RGB}{220, 220, 220}% ancien~; pour compatibilité
\definecolor{mcbleu}{RGB}{52, 152, 219}
\definecolor{mcvert}{RGB}{125, 194, 70}
\definecolor{mcmauve}{RGB}{154, 0, 215}
\definecolor{mcorange}{RGB}{255, 96, 0}
\definecolor{mcturquoise}{RGB}{0, 153, 153}
\definecolor{mcrouge}{RGB}{255, 0, 0}
\definecolor{mclightvert}{RGB}{205, 234, 190}

\definecolor{gris}{RGB}{220, 220, 220}
\definecolor{bleu}{RGB}{52, 152, 219}
\definecolor{vert}{RGB}{125, 194, 70}
\definecolor{mauve}{RGB}{154, 0, 215}
\definecolor{orange}{RGB}{255, 96, 0}
\definecolor{turquoise}{RGB}{0, 153, 153}
\definecolor{rouge}{RGB}{255, 0, 0}
\definecolor{lightvert}{RGB}{205, 234, 190}
\setitemize[0]{label=\color{lightvert}  $\bullet$}

\pagestyle{fancy}
\renewcommand{\headrulewidth}{0.2pt}
\fancyhead[L]{maths-cours.fr}
\fancyhead[R]{\thepage}
\renewcommand{\footrulewidth}{0.2pt}
\fancyfoot[C]{}

\newcolumntype{C}{>{\centering\arraybackslash}X}
\newcolumntype{s}{>{\hsize=.35\hsize\arraybackslash}X}

\setlength{\parindent}{0pt}		 
\setlength{\parskip}{3mm}
\setlength{\headheight}{1cm}

\def\ebook{ebook}
\def\book{book}
\def\web{web}
\def\type{web}

\newcommand{\vect}[1]{\overrightarrow{\,\mathstrut#1\,}}

\def\Oij{$\left(\text{O}~;~\vect{\imath},~\vect{\jmath}\right)$}
\def\Oijk{$\left(\text{O}~;~\vect{\imath},~\vect{\jmath},~\vect{k}\right)$}
\def\Ouv{$\left(\text{O}~;~\vect{u},~\vect{v}\right)$}

\hypersetup{breaklinks=true, colorlinks = true, linkcolor = OliveGreen, urlcolor = OliveGreen, citecolor = OliveGreen, pdfauthor={Didier BONNEL - https://www.maths-cours.fr} } % supprime les bordures autour des liens

\renewcommand{\arg}[0]{\text{arg}}

\everymath{\displaystyle}

%================================================================================================================================
%
% Macros - Commandes
%
%================================================================================================================================

\newcommand\meta[2]{    			% Utilisé pour créer le post HTML.
	\def\titre{titre}
	\def\url{url}
	\def\arg{#1}
	\ifx\titre\arg
		\newcommand\maintitle{#2}
		\fancyhead[L]{#2}
		{\Large\sffamily \MakeUppercase{#2}}
		\vspace{1mm}\textcolor{mcvert}{\hrule}
	\fi 
	\ifx\url\arg
		\fancyfoot[L]{\href{https://www.maths-cours.fr#2}{\black \footnotesize{https://www.maths-cours.fr#2}}}
	\fi 
}


\newcommand\TitreC[1]{    		% Titre centré
     \needspace{3\baselineskip}
     \begin{center}\textbf{#1}\end{center}
}

\newcommand\newpar{    		% paragraphe
     \par
}

\newcommand\nosp {    		% commande vide (pas d'espace)
}
\newcommand{\id}[1]{} %ignore

\newcommand\boite[2]{				% Boite simple sans titre
	\vspace{5mm}
	\setlength{\fboxrule}{0.2mm}
	\setlength{\fboxsep}{5mm}	
	\fcolorbox{#1}{#1!3}{\makebox[\linewidth-2\fboxrule-2\fboxsep]{
  		\begin{minipage}[t]{\linewidth-2\fboxrule-4\fboxsep}\setlength{\parskip}{3mm}
  			 #2
  		\end{minipage}
	}}
	\vspace{5mm}
}

\newcommand\CBox[4]{				% Boites
	\vspace{5mm}
	\setlength{\fboxrule}{0.2mm}
	\setlength{\fboxsep}{5mm}
	
	\fcolorbox{#1}{#1!3}{\makebox[\linewidth-2\fboxrule-2\fboxsep]{
		\begin{minipage}[t]{1cm}\setlength{\parskip}{3mm}
	  		\textcolor{#1}{\LARGE{#2}}    
 	 	\end{minipage}  
  		\begin{minipage}[t]{\linewidth-2\fboxrule-4\fboxsep}\setlength{\parskip}{3mm}
			\raisebox{1.2mm}{\normalsize\sffamily{\textcolor{#1}{#3}}}						
  			 #4
  		\end{minipage}
	}}
	\vspace{5mm}
}

\newcommand\cadre[3]{				% Boites convertible html
	\par
	\vspace{2mm}
	\setlength{\fboxrule}{0.1mm}
	\setlength{\fboxsep}{5mm}
	\fcolorbox{#1}{white}{\makebox[\linewidth-2\fboxrule-2\fboxsep]{
  		\begin{minipage}[t]{\linewidth-2\fboxrule-4\fboxsep}\setlength{\parskip}{3mm}
			\raisebox{-2.5mm}{\sffamily \small{\textcolor{#1}{\MakeUppercase{#2}}}}		
			\par		
  			 #3
 	 		\end{minipage}
	}}
		\vspace{2mm}
	\par
}

\newcommand\bloc[3]{				% Boites convertible html sans bordure
     \needspace{2\baselineskip}
     {\sffamily \small{\textcolor{#1}{\MakeUppercase{#2}}}}    
		\par		
  			 #3
		\par
}

\newcommand\CHelp[1]{
     \CBox{Plum}{\faInfoCircle}{À RETENIR}{#1}
}

\newcommand\CUp[1]{
     \CBox{NavyBlue}{\faThumbsOUp}{EN PRATIQUE}{#1}
}

\newcommand\CInfo[1]{
     \CBox{Sepia}{\faArrowCircleRight}{REMARQUE}{#1}
}

\newcommand\CRedac[1]{
     \CBox{PineGreen}{\faEdit}{BIEN R\'EDIGER}{#1}
}

\newcommand\CError[1]{
     \CBox{Red}{\faExclamationTriangle}{ATTENTION}{#1}
}

\newcommand\TitreExo[2]{
\needspace{4\baselineskip}
 {\sffamily\large EXERCICE #1\ (\emph{#2 points})}
\vspace{5mm}
}

\newcommand\img[2]{
          \includegraphics[width=#2\paperwidth]{\imgdir#1}
}

\newcommand\imgsvg[2]{
       \begin{center}   \includegraphics[width=#2\paperwidth]{\imgsvgdir#1} \end{center}
}


\newcommand\Lien[2]{
     \href{#1}{#2 \tiny \faExternalLink}
}
\newcommand\mcLien[2]{
     \href{https~://www.maths-cours.fr/#1}{#2 \tiny \faExternalLink}
}

\newcommand{\euro}{\eurologo{}}

%================================================================================================================================
%
% Macros - Environement
%
%================================================================================================================================

\newenvironment{tex}{ %
}
{%
}

\newenvironment{indente}{ %
	\setlength\parindent{10mm}
}

{
	\setlength\parindent{0mm}
}

\newenvironment{corrige}{%
     \needspace{3\baselineskip}
     \medskip
     \textbf{\textsc{Corrigé}}
     \medskip
}
{
}

\newenvironment{extern}{%
     \begin{center}
     }
     {
     \end{center}
}

\NewEnviron{code}{%
	\par
     \boite{gray}{\texttt{%
     \BODY
     }}
     \par
}

\newenvironment{vbloc}{% boite sans cadre empeche saut de page
     \begin{minipage}[t]{\linewidth}
     }
     {
     \end{minipage}
}
\NewEnviron{h2}{%
    \needspace{3\baselineskip}
    \vspace{0.6cm}
	\noindent \MakeUppercase{\sffamily \large \BODY}
	\vspace{1mm}\textcolor{mcgris}{\hrule}\vspace{0.4cm}
	\par
}{}

\NewEnviron{h3}{%
    \needspace{3\baselineskip}
	\vspace{5mm}
	\textsc{\BODY}
	\par
}

\NewEnviron{margeneg}{ %
\begin{addmargin}[-1cm]{0cm}
\BODY
\end{addmargin}
}

\NewEnviron{html}{%
}

\begin{document}
\meta{url}{/cours/fluctuation-estimation/}
\meta{pid}{500}
\meta{titre}{Estimation en Terminale S}
\meta{type}{cours}
\begin{h2}I - Intervalle de fluctuation asymptotique\end{h2}
Pour étudier un caractère présent dans une population, on prélève de façon aléatoire un échantillon dans cette population.
\par
On suppose connues :
\begin{itemize}
     \item la proportion $p$ du caractère \textbf{dans la population}
     \item la taille $n$ de l'échantillon
\end{itemize}
On cherche à évaluer :
\begin{itemize}
     \item la fréquence $f$ du caractère \textbf{dans l'échantillon}
\end{itemize}
\bloc{orange}{Exemple}{%id="e10"
     On sait que 48\% des élèves d'un lycée sont des garçons (et donc 52\% sont des filles...).
     \par
     Si l'on sélectionne au hasard 100 élèves dans l'établissement, on devrait obtenir \textbf{\textit{environ}} 52 filles et 48 garçons mais il n'est pas du tout certain que l'on obtienne \textbf{exactement} ces chiffres.
     \par
     Par contre, on pourra rechercher un intervalle dans lequel se situera \textit{"probablement"} la proportion de garçons dans cet échantillon.
}
Si $n$ est élevé, on peut assimiler la sélection de l'échantillon à un tirage avec remise. Le nombre d'individus présentant le caractère étudié suit alors une loi binomiale $\mathscr B\left(n,p\right)$. Pour $n$ élevé, on peut approximer cette loi binomiale par une loi normale. On obtient alors le résultat suivant :
\cadre{bleu}{Définition et propriété}{%id="d20"
     Soit un réel $\alpha \in \left]0~;~1\right[$.
     \par
     Un \textbf{intervalle de fluctuation asymptotique au seuil de $1-\alpha $} est~:
\begin{center}$I=\left[ p-u_\alpha \times \frac{\sqrt{p\left(1-p\right)}}{\sqrt{n}} ~; \right.$\nosp$ \left. p+u_\alpha \times \frac{\sqrt{p\left(1-p\right)}}{\sqrt{n}} \right]$\end{center}
Si on note $f_{n}$ la fréquence du caractère étudié dans l'échantillon de taille $n$ :
\begin{center}$\lim\limits_{n\rightarrow +\infty }p\left(f _n\in I\right) = 1-\alpha $.\end{center}
}
\bloc{cyan}{Remarques}{%id="r20"
     \begin{itemize}
          \item \textbf{Rappel : }La définition de $u_\alpha $ est donné dans le chapitre \mcLien{/cours/terminale-s/loi-normale\#t60}{«~loi normale~»}.
          \par
          On a en particulier $u_{0,05}=1,96$ et $u_{0,01}=2,58$
          \item Cette propriété est démontrée en exercice
          \item On considèrera que, lorsque $n$ est suffisamment élevé ($n\geqslant 30$, $np\geqslant 5$ et $n\left(1-p\right)\geqslant 5$), $p\left(f \in I\right) \approx 1-\alpha $
     \end{itemize}
}
\cadre{bleu}{Cas particulier (Intervalle de fluctuation au seuil de $95\%$)}{%id="p30"
     L'\textbf{intervalle de fluctuation asymptotique au seuil de $95\%$} est l'intervalle :
\begin{center}$I=\left[ p-1,96\times \frac{\sqrt{p\left(1-p\right)}}{\sqrt{n}} ~;\right.$\nosp$ \left.  p+1,96\times \frac{\sqrt{p\left(1-p\right)}}{\sqrt{n}} \right]$\end{center}
}
\bloc{orange}{Exemple}{%id="e30"
     Si l'on reprend l'exemple donné en introduction, on a $n=100$ et $p=\frac{48}{100}$.
     \par
     On trouve $I= \left[0,38 ~;~ 0,58\right]$.
     \par
     La proportion de garçons dans l'échantillon devrait être comprise entre 38\% et 58\% (avec une probabilité de $0,95$)
}
\bloc{cyan}{Remarques}{%id="r30"
     \begin{itemize}
          \item L' intervalle de fluctuation peut être utilisé pour valider ou rejeter une hypothèse. On procède de la façon suivante :
          \begin{itemize}[label=---]
               \item %
               On suppose que la proportion du caractère étudié est $p$.
               \item %
               On prélève un échantillon de taille $n$
               \item %
               On regarde si la fréquence $f$ du caractère dans l'échantillon appartient à $I$. Si oui,
               l'hypothèse est validée, si non, elle est rejetée.
          \end{itemize}
          \item Une étude de fonction montre que pour tout $p \in \left[0 ~;~ 1\right]$, $1,96\times \sqrt{p\left(1-p\right)} < 1$. On en déduit que :
     \begin{center}$\left[ p-1,96\times \frac{\sqrt{p\left(1-p\right)}}{\sqrt{n}} ~;\right.$\nosp$ \left.  p+1,96\times \frac{\sqrt{p\left(1-p\right)}}{\sqrt{n}} \right] $ est inclus dans $ \left[ p-\frac{1}{\sqrt{n}} ~;~ p+\frac{1}{\sqrt{n}} \right]$\end{center}
     qui est l'intervalle vu en Seconde.
\end{itemize}
}
\begin{h2}II - Intervalle de confiance\end{h2}
Dans cette partie (contrairement à la première partie), on suppose que l'\textbf{on connait la fréquence $f$ du caractère dans l'échantillon} mais que l' \textbf{on ne connait pas la proportion $p$ du caractère dans la population}.
\par
On cherche alors à évaluer $p$.
\cadre{bleu}{Définition et propriété} l'intervalle :
     \begin{center}$I=\left[ f-\frac{1}{\sqrt{n}} ~;~ f+\frac{1}{\sqrt{n}} \right]$\end{center}
     Pour $n$ élevé, la proportion $p$ du caractère dans la population appartiendra à $I$ dans 95\% des cas.
}
\bloc{orange}{Exemple}{%id="e50"
     On recherche le pourcentage de truites femelles dans un élevage de truites.
     \par
     Pour cela, on a prélevé un échantillon de 50 truites et on a comptabilisé 28 femelles dans cet échantillon.
     \par
     Le pourcentage de truites femelles dans l'ensemble de l'élevage appartient donc à l'intervalle :
     \begin{center}$I=\left[ \frac{28}{50}-\frac{1}{\sqrt{50}} ~;~ \frac{28}{50}+\frac{1}{\sqrt{50}} \right]$\nosp$  \approx \left[0,42 ~;~ 0,70\right]$\end{center}
     avec un risque d'erreur inférieur à 5\%.
}
\bloc{cyan}{Remarque}{%id="r50"
     La longeur de l'intervalle $I$ est $\frac{2}{\sqrt{n}}$.
     \par
     Si l'on souhaite obtenir un intervalle d'amplitude maximale $a$, il faut choisir $n$ tel que
     \par
     $\frac{2}{\sqrt{n}}\leqslant a$ c'est à dire $n\geqslant \frac{4}{a^{2}}$
}

\end{document}
µ
\documentclass[a4paper]{article}

%================================================================================================================================
%
% Packages
%
%================================================================================================================================

\usepackage[T1]{fontenc} 	% pour caractères accentués
\usepackage[utf8]{inputenc}  % encodage utf8
\usepackage[french]{babel}	% langue : français
\usepackage{fourier}			% caractères plus lisibles
\usepackage[dvipsnames]{xcolor} % couleurs
\usepackage{fancyhdr}		% réglage header footer
\usepackage{needspace}		% empêcher sauts de page mal placés
\usepackage{graphicx}		% pour inclure des graphiques
\usepackage{enumitem,cprotect}		% personnalise les listes d'items (nécessaire pour ol, al ...)
\usepackage{hyperref}		% Liens hypertexte
\usepackage{pstricks,pst-all,pst-node,pstricks-add,pst-math,pst-plot,pst-tree,pst-eucl} % pstricks
\usepackage[a4paper,includeheadfoot,top=2cm,left=3cm, bottom=2cm,right=3cm]{geometry} % marges etc.
\usepackage{comment}			% commentaires multilignes
\usepackage{amsmath,environ} % maths (matrices, etc.)
\usepackage{amssymb,makeidx}
\usepackage{bm}				% bold maths
\usepackage{tabularx}		% tableaux
\usepackage{colortbl}		% tableaux en couleur
\usepackage{fontawesome}		% Fontawesome
\usepackage{environ}			% environment with command
\usepackage{fp}				% calculs pour ps-tricks
\usepackage{multido}			% pour ps tricks
\usepackage[np]{numprint}	% formattage nombre
\usepackage{tikz,tkz-tab} 			% package principal TikZ
\usepackage{pgfplots}   % axes
\usepackage{mathrsfs}    % cursives
\usepackage{calc}			% calcul taille boites
\usepackage[scaled=0.875]{helvet} % font sans serif
\usepackage{svg} % svg
\usepackage{scrextend} % local margin
\usepackage{scratch} %scratch
\usepackage{multicol} % colonnes
%\usepackage{infix-RPN,pst-func} % formule en notation polanaise inversée
\usepackage{listings}

%================================================================================================================================
%
% Réglages de base
%
%================================================================================================================================

\lstset{
language=Python,   % R code
literate=
{á}{{\'a}}1
{à}{{\`a}}1
{ã}{{\~a}}1
{é}{{\'e}}1
{è}{{\`e}}1
{ê}{{\^e}}1
{í}{{\'i}}1
{ó}{{\'o}}1
{õ}{{\~o}}1
{ú}{{\'u}}1
{ü}{{\"u}}1
{ç}{{\c{c}}}1
{~}{{ }}1
}


\definecolor{codegreen}{rgb}{0,0.6,0}
\definecolor{codegray}{rgb}{0.5,0.5,0.5}
\definecolor{codepurple}{rgb}{0.58,0,0.82}
\definecolor{backcolour}{rgb}{0.95,0.95,0.92}

\lstdefinestyle{mystyle}{
    backgroundcolor=\color{backcolour},   
    commentstyle=\color{codegreen},
    keywordstyle=\color{magenta},
    numberstyle=\tiny\color{codegray},
    stringstyle=\color{codepurple},
    basicstyle=\ttfamily\footnotesize,
    breakatwhitespace=false,         
    breaklines=true,                 
    captionpos=b,                    
    keepspaces=true,                 
    numbers=left,                    
xleftmargin=2em,
framexleftmargin=2em,            
    showspaces=false,                
    showstringspaces=false,
    showtabs=false,                  
    tabsize=2,
    upquote=true
}

\lstset{style=mystyle}


\lstset{style=mystyle}
\newcommand{\imgdir}{C:/laragon/www/newmc/assets/imgsvg/}
\newcommand{\imgsvgdir}{C:/laragon/www/newmc/assets/imgsvg/}

\definecolor{mcgris}{RGB}{220, 220, 220}% ancien~; pour compatibilité
\definecolor{mcbleu}{RGB}{52, 152, 219}
\definecolor{mcvert}{RGB}{125, 194, 70}
\definecolor{mcmauve}{RGB}{154, 0, 215}
\definecolor{mcorange}{RGB}{255, 96, 0}
\definecolor{mcturquoise}{RGB}{0, 153, 153}
\definecolor{mcrouge}{RGB}{255, 0, 0}
\definecolor{mclightvert}{RGB}{205, 234, 190}

\definecolor{gris}{RGB}{220, 220, 220}
\definecolor{bleu}{RGB}{52, 152, 219}
\definecolor{vert}{RGB}{125, 194, 70}
\definecolor{mauve}{RGB}{154, 0, 215}
\definecolor{orange}{RGB}{255, 96, 0}
\definecolor{turquoise}{RGB}{0, 153, 153}
\definecolor{rouge}{RGB}{255, 0, 0}
\definecolor{lightvert}{RGB}{205, 234, 190}
\setitemize[0]{label=\color{lightvert}  $\bullet$}

\pagestyle{fancy}
\renewcommand{\headrulewidth}{0.2pt}
\fancyhead[L]{maths-cours.fr}
\fancyhead[R]{\thepage}
\renewcommand{\footrulewidth}{0.2pt}
\fancyfoot[C]{}

\newcolumntype{C}{>{\centering\arraybackslash}X}
\newcolumntype{s}{>{\hsize=.35\hsize\arraybackslash}X}

\setlength{\parindent}{0pt}		 
\setlength{\parskip}{3mm}
\setlength{\headheight}{1cm}

\def\ebook{ebook}
\def\book{book}
\def\web{web}
\def\type{web}

\newcommand{\vect}[1]{\overrightarrow{\,\mathstrut#1\,}}

\def\Oij{$\left(\text{O}~;~\vect{\imath},~\vect{\jmath}\right)$}
\def\Oijk{$\left(\text{O}~;~\vect{\imath},~\vect{\jmath},~\vect{k}\right)$}
\def\Ouv{$\left(\text{O}~;~\vect{u},~\vect{v}\right)$}

\hypersetup{breaklinks=true, colorlinks = true, linkcolor = OliveGreen, urlcolor = OliveGreen, citecolor = OliveGreen, pdfauthor={Didier BONNEL - https://www.maths-cours.fr} } % supprime les bordures autour des liens

\renewcommand{\arg}[0]{\text{arg}}

\everymath{\displaystyle}

%================================================================================================================================
%
% Macros - Commandes
%
%================================================================================================================================

\newcommand\meta[2]{    			% Utilisé pour créer le post HTML.
	\def\titre{titre}
	\def\url{url}
	\def\arg{#1}
	\ifx\titre\arg
		\newcommand\maintitle{#2}
		\fancyhead[L]{#2}
		{\Large\sffamily \MakeUppercase{#2}}
		\vspace{1mm}\textcolor{mcvert}{\hrule}
	\fi 
	\ifx\url\arg
		\fancyfoot[L]{\href{https://www.maths-cours.fr#2}{\black \footnotesize{https://www.maths-cours.fr#2}}}
	\fi 
}


\newcommand\TitreC[1]{    		% Titre centré
     \needspace{3\baselineskip}
     \begin{center}\textbf{#1}\end{center}
}

\newcommand\newpar{    		% paragraphe
     \par
}

\newcommand\nosp {    		% commande vide (pas d'espace)
}
\newcommand{\id}[1]{} %ignore

\newcommand\boite[2]{				% Boite simple sans titre
	\vspace{5mm}
	\setlength{\fboxrule}{0.2mm}
	\setlength{\fboxsep}{5mm}	
	\fcolorbox{#1}{#1!3}{\makebox[\linewidth-2\fboxrule-2\fboxsep]{
  		\begin{minipage}[t]{\linewidth-2\fboxrule-4\fboxsep}\setlength{\parskip}{3mm}
  			 #2
  		\end{minipage}
	}}
	\vspace{5mm}
}

\newcommand\CBox[4]{				% Boites
	\vspace{5mm}
	\setlength{\fboxrule}{0.2mm}
	\setlength{\fboxsep}{5mm}
	
	\fcolorbox{#1}{#1!3}{\makebox[\linewidth-2\fboxrule-2\fboxsep]{
		\begin{minipage}[t]{1cm}\setlength{\parskip}{3mm}
	  		\textcolor{#1}{\LARGE{#2}}    
 	 	\end{minipage}  
  		\begin{minipage}[t]{\linewidth-2\fboxrule-4\fboxsep}\setlength{\parskip}{3mm}
			\raisebox{1.2mm}{\normalsize\sffamily{\textcolor{#1}{#3}}}						
  			 #4
  		\end{minipage}
	}}
	\vspace{5mm}
}

\newcommand\cadre[3]{				% Boites convertible html
	\par
	\vspace{2mm}
	\setlength{\fboxrule}{0.1mm}
	\setlength{\fboxsep}{5mm}
	\fcolorbox{#1}{white}{\makebox[\linewidth-2\fboxrule-2\fboxsep]{
  		\begin{minipage}[t]{\linewidth-2\fboxrule-4\fboxsep}\setlength{\parskip}{3mm}
			\raisebox{-2.5mm}{\sffamily \small{\textcolor{#1}{\MakeUppercase{#2}}}}		
			\par		
  			 #3
 	 		\end{minipage}
	}}
		\vspace{2mm}
	\par
}

\newcommand\bloc[3]{				% Boites convertible html sans bordure
     \needspace{2\baselineskip}
     {\sffamily \small{\textcolor{#1}{\MakeUppercase{#2}}}}    
		\par		
  			 #3
		\par
}

\newcommand\CHelp[1]{
     \CBox{Plum}{\faInfoCircle}{À RETENIR}{#1}
}

\newcommand\CUp[1]{
     \CBox{NavyBlue}{\faThumbsOUp}{EN PRATIQUE}{#1}
}

\newcommand\CInfo[1]{
     \CBox{Sepia}{\faArrowCircleRight}{REMARQUE}{#1}
}

\newcommand\CRedac[1]{
     \CBox{PineGreen}{\faEdit}{BIEN R\'EDIGER}{#1}
}

\newcommand\CError[1]{
     \CBox{Red}{\faExclamationTriangle}{ATTENTION}{#1}
}

\newcommand\TitreExo[2]{
\needspace{4\baselineskip}
 {\sffamily\large EXERCICE #1\ (\emph{#2 points})}
\vspace{5mm}
}

\newcommand\img[2]{
          \includegraphics[width=#2\paperwidth]{\imgdir#1}
}

\newcommand\imgsvg[2]{
       \begin{center}   \includegraphics[width=#2\paperwidth]{\imgsvgdir#1} \end{center}
}


\newcommand\Lien[2]{
     \href{#1}{#2 \tiny \faExternalLink}
}
\newcommand\mcLien[2]{
     \href{https~://www.maths-cours.fr/#1}{#2 \tiny \faExternalLink}
}

\newcommand{\euro}{\eurologo{}}

%================================================================================================================================
%
% Macros - Environement
%
%================================================================================================================================

\newenvironment{tex}{ %
}
{%
}

\newenvironment{indente}{ %
	\setlength\parindent{10mm}
}

{
	\setlength\parindent{0mm}
}

\newenvironment{corrige}{%
     \needspace{3\baselineskip}
     \medskip
     \textbf{\textsc{Corrigé}}
     \medskip
}
{
}

\newenvironment{extern}{%
     \begin{center}
     }
     {
     \end{center}
}

\NewEnviron{code}{%
	\par
     \boite{gray}{\texttt{%
     \BODY
     }}
     \par
}

\newenvironment{vbloc}{% boite sans cadre empeche saut de page
     \begin{minipage}[t]{\linewidth}
     }
     {
     \end{minipage}
}
\NewEnviron{h2}{%
    \needspace{3\baselineskip}
    \vspace{0.6cm}
	\noindent \MakeUppercase{\sffamily \large \BODY}
	\vspace{1mm}\textcolor{mcgris}{\hrule}\vspace{0.4cm}
	\par
}{}

\NewEnviron{h3}{%
    \needspace{3\baselineskip}
	\vspace{5mm}
	\textsc{\BODY}
	\par
}

\NewEnviron{margeneg}{ %
\begin{addmargin}[-1cm]{0cm}
\BODY
\end{addmargin}
}

\NewEnviron{html}{%
}

\begin{document}
\meta{url}{/cours/limites-fonctions/}
\meta{pid}{509}
\meta{titre}{Limites d'une fonction}
\meta{type}{cours}
\begin{h2}1. Définitions\end{h2}
\cadre{bleu}{Définition}{%id="d10"
     \textbf{Limite infinie quand $x$ tend vers l'infini.}
     \par
     Soit $f$ une fonction définie sur un intervalle $\left[a; +\infty \right[$.
     \par
     On dit que que $f\left(x\right)$ tend vers $+\infty $ quand $x$ tend vers $+\infty $ lorsque pour $x$ suffisamment grand, $f\left(x\right)$ est aussi grand que l'on veut. On écrit alors que $\lim\limits_{x\rightarrow +\infty } f\left(x\right)=+\infty $.
}
%<img src="/wp-content/uploads/lim_inf.svg" alt="" class="aligncenter" style="width:350px;"/>
\begin{center}
     \begin{extern} %width="400" alt="limite infinie"
          % -+-+-+ variables modifiables
          \resizebox{8cm}{!}{%
               \def\fonction{ 0.05*x*x + 0.5*x - 1 }
               \def\xmin{-1}
               \def\xmax{13}
               \def\ymin{-2}
               \def\ymax{9}
               \def\xunit{1}  % unités en cm
               \def\yunit{1}
               \psset{xunit=\xunit,yunit=\yunit,algebraic=true}
               \fontsize{15pt}{15pt}\selectfont
               \begin{pspicture*}[linewidth=1pt](\xmin,\ymin)(\xmax,\ymax)
                    %      \psgrid[gridcolor=mcgris, subgriddiv=5, gridlabels=0pt](-5,-0.3)(5,1)
                    \psaxes[Dx=100,Dy=100,linewidth=0.75pt]{->}(0,0)(\xmin,\ymin)(\xmax,\ymax)
                    \psplot[plotpoints=2000,linecolor=blue]{\xmin}{\xmax}{\fonction}
                    \rput[tr](-0.1,-0.1){$O$}
                    \rput[tl](9.5,8){$\color{blue} \mathcal{C}_f$}
               \end{pspicture*}
          }
     \end{extern}
\end{center}
\begin{center}$\lim\limits_{x\rightarrow +\infty } f\left(x\right)=+\infty $\end{center}
\bloc{cyan}{Remarque}{%id="r10"
     On définit de façon similaire les limites :
     \par
     $\lim\limits_{x\rightarrow +\infty } f\left(x\right)=-\infty $ ; $\lim\limits_{x\rightarrow -\infty } f\left(x\right)=+\infty $ ; $\lim\limits_{x\rightarrow -\infty } f\left(x\right)=-\infty $.
}
\cadre{bleu}{Définition}{%id="d20"
     \textbf{Limite finie quand $x$ tend vers l'infini.}
     \par
     Soit $f$ une fonction définie sur un intervalle $\left[a ; +\infty \right[$.
     \par
     On dit que que $f\left(x\right)$ tend vers $l$ quand $x$ tend vers $+\infty $ lorsque pour $x$ suffisamment grand, $f\left(x\right)$ est aussi proche de $l$ que l'on veut. On écrit alors que $\lim\limits_{x\rightarrow +\infty } f\left(x\right)=l$.
}
\begin{center}
     \begin{extern} %width="400" alt="limite nulle"
          \resizebox{8cm}{!}{%
               % -+-+-+ variables modifiables
               \def\fonction{15/(x*x+3*x+4) }
               \def\xmin{-1}
               \def\xmax{13}
               \def\ymin{-2}
               \def\ymax{9}
               \def\xunit{1}  % unités en cm
               \def\yunit{1}
               \psset{xunit=\xunit,yunit=\yunit,algebraic=true}
               \fontsize{15pt}{15pt}\selectfont
               \begin{pspicture*}[linewidth=1pt](\xmin,\ymin)(\xmax,\ymax)
                    %      \psgrid[gridcolor=mcgris, subgriddiv=5, gridlabels=0pt](-5,-0.3)(5,1)
                    \psaxes[Dx=100,Dy=100,linewidth=0.75pt]{->}(0,0)(\xmin,\ymin)(\xmax,\ymax)
                    \psplot[plotpoints=2000,linecolor=blue]{\xmin}{\xmax}{\fonction}
                    \rput[tr](-0.1,-0.1){$O$}
                    \rput[tl](-1,4.5){$\color{blue} \mathcal{C}_f$}
               \end{pspicture*}
          }
     \end{extern}
\end{center}
\begin{center}$\lim\limits_{x\rightarrow +\infty } f\left(x\right)=0$\end{center}
\bloc{cyan}{Remarque}{%id="r20"
     On définit de façon similaire la limite $\lim\limits_{x\rightarrow -\infty } f\left(x\right)=l$.
}
\cadre{bleu}{Définition}{%id="d30"
     Si $\lim\limits_{x\rightarrow -\infty }f\left(x\right)=l$ ou $\lim\limits_{x\rightarrow +\infty }f\left(x\right)=l$, on dit que la droite d'équation $y=l$ est \textbf{asymptote horizontale} à la courbe représentative de la fonction $f$.
}
\bloc{orange}{Exemple}{%id="e30"
     Sur la courbe ci-dessus, la droite d'équation $y=0$ est \textbf{asymptote horizontale} à la courbe représentative de $f$.
}
\cadre{bleu}{Définition}{%id="d40"
     \textbf{Limite infinie quand $x$ tend vers un réel.}
     \par
     Soit $f$ une fonction définie sur un intervalle $\left]a; b\right[$ (avec $a < b$).
     \par
     On dit que que $f\left(x\right)$ tend vers $+\infty $ quand $x$ tend vers $a$ par valeurs supérieures lorsque $f\left(x\right)$ est aussi grand que l'on veut quand $x$ se rapproche de $a$ en restant supérieur à $a$. On écrit alors $\lim\limits_{x\rightarrow a^+} f\left(x\right)=+\infty $ ou $\lim\limits_{\scriptstyle x\rightarrow a \atop\scriptstyle x > a} f\left(x\right)=+\infty $.
     \par
     De même, on dit que que $f\left(x\right)$ tend vers $+\infty $ quand $x$ tend vers $b$ par valeurs inférieures lorsque $f\left(x\right)$ est aussi grand que l'on veut quand $x$ se rapproche de $b$ en restant inférieur à $b$. On écrit alors $\lim\limits_{x\rightarrow b^-} f\left(x\right)=+\infty $ ou $\lim\limits_{\scriptstyle x\rightarrow b \atop\scriptstyle  x < b} f\left(x\right)=+\infty $.
     \par
     Enfin, si $c\in \left]a;b\right[$ , on dit que que $f\left(x\right)$ tend vers $+\infty $ quand $x$ tend vers $c$ si $f\left(x\right)$ tend vers $+\infty $ quand $x$ tend vers $c$ par valeurs supérieures et par valeurs inférieures. On écrit alors $\lim\limits_{x\rightarrow c} f\left(x\right)=+\infty $.
}
\bloc{cyan}{Remarque}{%id="r40"
     On définit de façon symétrique $\lim\limits_{x\rightarrow a^-} f\left(x\right)=-\infty $, $\lim\limits_{x\rightarrow a^+} f\left(x\right)=-\infty $ et $\lim\limits_{x\rightarrow a} f\left(x\right)=-\infty $ en remplaçant «\textit{ $f\left(x\right)$ est aussi grand que l'on veut} » par « \textit{$f\left(x\right)$ est aussi petit que l'on veut} » dans la définition.
}
\cadre{bleu}{Définition}{%id="d50"
     Si $\lim\limits_{x\rightarrow c^-}f\left(x\right)=\pm \infty $ ou $\lim\limits_{x\rightarrow c^+}f\left(x\right)=\pm \infty $ ou $\lim\limits_{x\rightarrow c}f\left(x\right)=\pm \infty $, on dit que la droite d'équation $x=c$ est \textbf{asymptote verticale} à la courbe représentative de la fonction $f$.
}
\bloc{orange}{Exemple}{%id="e50"
     Sur les trois courbes de la figure ci-dessous, la droite d'équation $x=0$ est \textbf{asymptote verticale} à la courbe représentative de $f$.
}
\begin{center}
     \begin{extern} %width="600" alt="limite à droite et à gauche"
          % -+-+-+ variables modifiables
          \def\f{4/abs(x)-abs(x) }
          \def\g{4/(x+9)-(x+9) }
          \def\h{-4/(x+14)+(x+14) }
          \def\xmin{-18}
          \def\xmax{3}
          \def\ymin{-4.8}
          \def\ymax{9}
          \def\xunit{0.5}  % unités en cm
          \def\yunit{0.5}
          \psset{xunit=\xunit,yunit=\yunit,algebraic=true}
          \fontsize{9pt}{9pt}\selectfont
          \begin{pspicture*}[linewidth=1pt](\xmin,\ymin)(\xmax,\ymax)
               %      \psgrid[gridcolor=mcgris, subgriddiv=5, gridlabels=0pt](-5,-0.3)(5,1)
               %   \psaxes[Dx=100,Dy=100,linewidth=0.75pt]{->}(0,0)(\xmin,-2)(\xmax,\ymax)
               \psplot[plotpoints=2000,linecolor=blue,linewidth=0.75pt]{-3}{3}{\f}
               \psplot[plotpoints=2000,linecolor=rouge,linewidth=0.75pt]{-8.9}{-6}{\g}
               \psplot[plotpoints=2000,linecolor=vert,linewidth=0.75pt]{-17}{-14.01}{\h}
               \rput[tr](-0.1,-0.1){$O$}
               \rput[tr](-9.1,-0.1){$O$}
               \rput[tr](-14.1,-0.1){$O$}
               \rput[tl](0.7,7.5){$\color{blue} \mathcal{C}_f$}
               \rput[tl](-8.3,7.5){$\color{rouge} \mathcal{C}_f$}
               \rput[tl](-16,7.5){$\color{vert} \mathcal{C}_f$}
               \psline[linewidth=0.75pt]{->}(0,-2)(0,9)
               \psline[linewidth=0.75pt]{->}(-9,-2)(-9,9)
               \psline[linewidth=0.75pt]{->}(-14,-2)(-14,9)
               \psline[linewidth=0.75pt]{->}(-3,0)(3,0)
               \psline[linewidth=0.75pt]{->}(-10,0)(-5,0)
               \psline[linewidth=0.75pt]{->}(-17,0)(-12,0)
               \rput[t](-14.5,-3){$\lim\limits_{x\rightarrow 0^{-}} f\left(x\right)=+\infty$}
               \rput[t](-7.5,-3){$\lim\limits_{x\rightarrow 0^{+}} f\left(x\right)=+\infty$}
               \rput[t](0,-3){$\lim\limits_{x\rightarrow 0} f\left(x\right)=+\infty$}
          \end{pspicture*}
     \end{extern}
\end{center}
\cadre{bleu}{Définition}{%id="d60"
     \textbf{Limite finie quand x tend vers un réel}.
     \par
     Soit $f$ une fonction définie sur un intervalle $\left]a;b\right[$ (avec $a < b$).
     \par
     On dit que que $f\left(x\right)$ tend vers $l$ quand $x$ tend vers $a$ par valeurs supérieures lorsque $f\left(x\right)$ se rapproche de $l$ quand x se rapproche de $a$ en restant supérieur à $a$.
     \par
     On écrit alors $\lim\limits_{x\rightarrow a^+} f\left(x\right)=l$ ou $\lim\limits_{\begin{matrix}x\rightarrow a \\ x > a\end{matrix}} f\left(x\right)=l$.
     \par
     De même, on dit que que $f\left(x\right)$ tend vers $l$ quand $x$ tend vers $b$ par valeurs inférieures lorsque $f\left(x\right)$ se rapproche de $l$ quand x se rapproche de $b$ en restant inférieur à $b$.
     \par
     On écrit alors $\lim\limits_{x\rightarrow b^-} f\left(x\right)=l$ ou $\lim\limits_{\begin{matrix}x\rightarrow b \\ x < b\end{matrix}} f\left(x\right)=l $.
     \par
     Enfin, si $c\in \left]a; b\right[$ , on dit que que $f\left(x\right)$ tend vers $l$ quand $x$ tend vers $c$ si $f\left(x\right)$ tend vers $l$ quand $x$ tend vers $c$ par valeurs supérieures et par valeurs inférieures.
     \par
     On écrit alors $\lim\limits_{x\rightarrow c} f\left(x\right)=l$.
}
\begin{h2}2. Limites usuelles\end{h2}
\cadre{vert}{Propriétés}{%id="p70"
     Pour tout entier $n > 1$ :
     \begin{itemize}
          \item $\lim\limits_{x\rightarrow -\infty }x^{n}=\left\{ \begin{matrix} -\infty \text{ si n est impair} \\ +\infty \text{ si n est pair} \end{matrix}\right. $
          \item $\lim\limits_{x\rightarrow +\infty }x^{n}=+\infty $
          \item $\lim\limits_{x\rightarrow -\infty }\frac{1}{x^{n}}=0$
          \item $\lim\limits_{x\rightarrow +\infty }\frac{1}{x^{n}}=0$
          \item $\lim\limits_{x\rightarrow 0^-}\frac{1}{x}=-\infty $
          \item $\lim\limits_{x\rightarrow 0^+}\frac{1}{x}=+\infty $
          \item $\lim\limits_{x\rightarrow +\infty }\sqrt{x}=+\infty $.
     \end{itemize}
}
\begin{h2}3. Opérations sur les limites\end{h2}
\cadre{vert}{Propriétés}{%id="p70"
     \textbf{Limite d'une somme.}
     \par
     $a$ désigne un réel ou $+\infty $ ou $-\infty $.
     \begin{center}
          \def\arraystretch{2}%
          \begin{tabularx}{0.8\linewidth}{|*{3}{>{\centering \arraybackslash }X|}}%class="compact" width="600"
               \hline
               $\lim\limits_{x\rightarrow a}f\left(x\right)$ & $\lim\limits_{x\rightarrow a}g\left(x\right)$ & $\lim\limits_{x\rightarrow a}f\left(x\right)+g\left(x\right)$          \\ \hline
               $l$ & $l^{\prime}$ & $l+l^{\prime}$          \\ \hline
               $l$ & $+\infty $ & $+\infty $          \\ \hline
               $l$ & $-\infty $ & $-\infty $          \\ \hline
               $+\infty $ & $+\infty $ & $+\infty $          \\ \hline
               $-\infty $ & $-\infty $ & $-\infty $          \\ \hline
               $+\infty $ & $-\infty $ & $F.I.$           \\ \hline
          \end{tabularx}
     \end{center}     $F.I.$ signifie forme indéterminée.
}
\bloc{cyan}{Remarque}{%id="r70"
     \textit{« Forme indéterminée »} ne signifie \textbf{pas} que la limite n'existe pas mais que les formules d'opérations sur les limites ne permettent pas de trouver directement limite. Pour la calculer, il faut alors \textit{« lever l'indétermination »} par exemple en simplifiant une fraction (cf. \textit{fiches méthodes}).
}
\cadre{vert}{Propriétés}{%id="p80"
     \textbf{Limite d'un produit.}
     \par
     $a$ désigne un réel ou $+\infty $ ou $-\infty $.
     \begin{center}
          \def\arraystretch{2}%
          \begin{tabularx}{0.8\linewidth}{|*{3}{>{\centering \arraybackslash }X|}}%class="compact" width="600"
               \hline
               $\lim\limits_{x\rightarrow a}f\left(x\right)$ & $\lim\limits_{x\rightarrow a}g\left(x\right)$ & $\lim\limits_{x\rightarrow a}f\left(x\right)\times g\left(x\right)$          \\ \hline
               $l$ & $l^{\prime}$ & $l\times l^{\prime}$          \\ \hline
               $l\neq 0$ & $\pm \infty $ & $\left(signe\right)\infty $          \\ \hline
               $\pm \infty $ & $\pm \infty $ & $\left(signe\right)\infty $          \\ \hline
               $0$ & $\pm \infty $ & $F.I.$     \\ \hline
          \end{tabularx}
     \end{center}
     \begin{itemize}
          \item $F.I.$ signifie forme indéterminée.
          \item $\pm \infty $ signifie que la formule s'applique pour $+\infty $ et pour $-\infty $.
          \item $\left(signe\right)\infty $ signifie que l'on utilise la règle des signes usuelle :
          \par
          $+\times +=+$
          \par
          $+\times -=-$
          \par
          $-\times -=+$
          \par
          pour déterminer si la limite vaut $+\infty $ ou $-\infty $.
     \end{itemize}
}
\cadre{vert}{Propriétés}{%id="p90"
     \textbf{Limite d'un quotient.}
     \par
     $a$ désigne un réel ou $+\infty $ ou $-\infty $.
     \begin{center}
          \def\arraystretch{3}%
          \begin{tabularx}{0.8\linewidth}{|*{3}{>{\centering \arraybackslash }X|}}%class="compact" width="600"
               \hline
               $\lim\limits_{x\rightarrow a}f\left(x\right)$ & $\lim\limits_{x\rightarrow a}g\left(x\right)$ & $\lim\limits_{x\rightarrow a}\frac{f\left(x\right)}{g\left(x\right)}$          \\ \hline
               $l$ & $l^{\prime}\neq 0$ & $\frac{l}{l^{\prime}}$          \\ \hline
               $l\neq 0$ & $0$ & $\left(signe\right)\infty $          \\ \hline
               $0$ & $0$ & $F.I.$          \\ \hline
               $l$ & $\pm \infty $ & $0$          \\ \hline
               $\pm \infty $ & $l$ & $\left(signe\right)\infty $          \\ \hline
               $\pm \infty $ & $\pm \infty $ & $F.I.$         \\ \hline
          \end{tabularx}
     \end{center}
}
\cadre{vert}{Propriété}{%id="p100"
     \textbf{Limite d'une fonction composée.}
     \par
     $a$, $b$ et $c$ désignent des réels ou $+\infty $ ou $-\infty $.
     \par
     Si $\lim\limits_{x\rightarrow a}f\left(x\right)=\color{red}{b}$ et $\lim\limits_{x\rightarrow \color{red}{b}}g\left(x\right)=c$ alors :
     \par
     $\lim\limits_{x\rightarrow a}g\left(f\left(x\right)\right)=c$.
}
\bloc{cyan}{Remarque}{%id="r100"
     On pose souvent $X=f\left(x\right)$ («changement de variable») et on écrit alors :
     \par
     $\lim\limits_{x\rightarrow a}X=\lim\limits_{x\rightarrow a}f\left(x\right)=b$
     \par
     $\lim\limits_{x\rightarrow a}g\left(f\left(x\right)\right)=\lim\limits_{X\rightarrow b}g\left(X\right)=c$.
}
\bloc{orange}{Exemple}{%id="e100"
     On cherche à calculer :
     \par
     $\lim\limits_{x\rightarrow -\infty }\sqrt{1+x^{2}}$.
     \par
     On pose $X=1+x^{2}$. Alors :
     \par
     $\lim\limits_{x\rightarrow -\infty }X=\lim\limits_{x\rightarrow -\infty }1+x^{2}=+\infty $
     \par
     et
     \par
     $\lim\limits_{x\rightarrow -\infty }\sqrt{1+x^{2}}=\lim\limits_{X\rightarrow +\infty }\sqrt{X}=+\infty $.
}
\begin{h2}4. Théorèmes de comparaison\end{h2}
\cadre{rouge}{Théorèmes}{%id="t110"
     \begin{itemize}
          \item Si $f\left(x\right)\geqslant g\left(x\right)$ sur un intervalle de la forme $\left[a;+\infty \right[$ et si $\lim\limits_{x\rightarrow +\infty }g\left(x\right)=+\infty $ alors :
          \par
          $\lim\limits_{x\rightarrow +\infty }f\left(x\right)=+\infty $.
          \item Si $f\left(x\right)\leqslant g\left(x\right)$ sur un intervalle de la forme $\left[a;+\infty \right[$ et si $\lim\limits_{x\rightarrow +\infty }g\left(x\right)=-\infty $ alors :
          \par
          $\lim\limits_{x\rightarrow +\infty }f\left(x\right)=-\infty $.
     \end{itemize}
}
\cadre{rouge}{Théorème}{%id="t120"
     \textbf{Théorème des "gendarmes".}
     \par
     Si $g\left(x\right)\leqslant f\left(x\right)\leqslant h\left(x\right)$ sur un intervalle de la forme $\left[a;+\infty \right[$ et si $\lim\limits_{x\rightarrow +\infty }g\left(x\right)=\lim\limits_{x\rightarrow +\infty }h\left(x\right)=l$ alors :
     \par
     $\lim\limits_{x\rightarrow +\infty }f\left(x\right)=l.$
}
\begin{center}
     \begin{extern} %width="400" alt="Théorème des gendarmes"
          \resizebox{8cm}{!}{%
               % -+-+-+ variables modifiables
               \def\fonction{10* sin(2*x)/(x+3)^2+3 }
               \def\g{25/(x+3)^2 +3}
               \def\h{-25/(x+3)^2 +3}
               \def\xmin{-1}
               \def\xmax{13}
               \def\ymin{-2}
               \def\ymax{9}
               \def\xunit{1}  % unités en cm
               \def\yunit{1}
               \psset{xunit=\xunit,yunit=\yunit,algebraic=true}
               \fontsize{15pt}{15pt}\selectfont
               \begin{pspicture*}[linewidth=1pt](\xmin,\ymin)(\xmax,\ymax)
                    %      \psgrid[gridcolor=mcgris, subgriddiv=5, gridlabels=0pt](-5,-0.3)(5,1)
                    \psaxes[Dx=100,Dy=100,linewidth=0.75pt]{->}(0,0)(\xmin,\ymin)(\xmax,\ymax)
                    \psline[linecolor=gray](-3,3)(14,3)
                    \psplot[plotpoints=2000,linecolor=rouge]{\xmin}{\xmax}{\fonction}
                    \psplot[plotpoints=2000,linecolor=blue]{\xmin}{\xmax}{\g}
                    \psplot[plotpoints=2000,linecolor=vert]{\xmin}{\xmax}{\h}
                    \rput[tr](-0.1,-0.1){$O$}
                    \rput[tl](-1,6.5){$\color{blue} \mathcal{C}_h$}
                    \rput[tl](-1,3.5){$\color{rouge} \mathcal{C}_f$}
                    \rput[tl](-1,0.6){$\color{vert} \mathcal{C}_g$}
               \end{pspicture*}
          }
     \end{extern}
\end{center}
\begin{center}Théorème des gendarmes \end{center}
\bloc{cyan}{Remarque}{%id="r120"
     On a des théorèmes similaires lorsque $x \rightarrow -\infty $.
}

\end{document}
µ
\documentclass[a4paper]{article}

%================================================================================================================================
%
% Packages
%
%================================================================================================================================

\usepackage[T1]{fontenc} 	% pour caractères accentués
\usepackage[utf8]{inputenc}  % encodage utf8
\usepackage[french]{babel}	% langue : français
\usepackage{fourier}			% caractères plus lisibles
\usepackage[dvipsnames]{xcolor} % couleurs
\usepackage{fancyhdr}		% réglage header footer
\usepackage{needspace}		% empêcher sauts de page mal placés
\usepackage{graphicx}		% pour inclure des graphiques
\usepackage{enumitem,cprotect}		% personnalise les listes d'items (nécessaire pour ol, al ...)
\usepackage{hyperref}		% Liens hypertexte
\usepackage{pstricks,pst-all,pst-node,pstricks-add,pst-math,pst-plot,pst-tree,pst-eucl} % pstricks
\usepackage[a4paper,includeheadfoot,top=2cm,left=3cm, bottom=2cm,right=3cm]{geometry} % marges etc.
\usepackage{comment}			% commentaires multilignes
\usepackage{amsmath,environ} % maths (matrices, etc.)
\usepackage{amssymb,makeidx}
\usepackage{bm}				% bold maths
\usepackage{tabularx}		% tableaux
\usepackage{colortbl}		% tableaux en couleur
\usepackage{fontawesome}		% Fontawesome
\usepackage{environ}			% environment with command
\usepackage{fp}				% calculs pour ps-tricks
\usepackage{multido}			% pour ps tricks
\usepackage[np]{numprint}	% formattage nombre
\usepackage{tikz,tkz-tab} 			% package principal TikZ
\usepackage{pgfplots}   % axes
\usepackage{mathrsfs}    % cursives
\usepackage{calc}			% calcul taille boites
\usepackage[scaled=0.875]{helvet} % font sans serif
\usepackage{svg} % svg
\usepackage{scrextend} % local margin
\usepackage{scratch} %scratch
\usepackage{multicol} % colonnes
%\usepackage{infix-RPN,pst-func} % formule en notation polanaise inversée
\usepackage{listings}

%================================================================================================================================
%
% Réglages de base
%
%================================================================================================================================

\lstset{
language=Python,   % R code
literate=
{á}{{\'a}}1
{à}{{\`a}}1
{ã}{{\~a}}1
{é}{{\'e}}1
{è}{{\`e}}1
{ê}{{\^e}}1
{í}{{\'i}}1
{ó}{{\'o}}1
{õ}{{\~o}}1
{ú}{{\'u}}1
{ü}{{\"u}}1
{ç}{{\c{c}}}1
{~}{{ }}1
}


\definecolor{codegreen}{rgb}{0,0.6,0}
\definecolor{codegray}{rgb}{0.5,0.5,0.5}
\definecolor{codepurple}{rgb}{0.58,0,0.82}
\definecolor{backcolour}{rgb}{0.95,0.95,0.92}

\lstdefinestyle{mystyle}{
    backgroundcolor=\color{backcolour},   
    commentstyle=\color{codegreen},
    keywordstyle=\color{magenta},
    numberstyle=\tiny\color{codegray},
    stringstyle=\color{codepurple},
    basicstyle=\ttfamily\footnotesize,
    breakatwhitespace=false,         
    breaklines=true,                 
    captionpos=b,                    
    keepspaces=true,                 
    numbers=left,                    
xleftmargin=2em,
framexleftmargin=2em,            
    showspaces=false,                
    showstringspaces=false,
    showtabs=false,                  
    tabsize=2,
    upquote=true
}

\lstset{style=mystyle}


\lstset{style=mystyle}
\newcommand{\imgdir}{C:/laragon/www/newmc/assets/imgsvg/}
\newcommand{\imgsvgdir}{C:/laragon/www/newmc/assets/imgsvg/}

\definecolor{mcgris}{RGB}{220, 220, 220}% ancien~; pour compatibilité
\definecolor{mcbleu}{RGB}{52, 152, 219}
\definecolor{mcvert}{RGB}{125, 194, 70}
\definecolor{mcmauve}{RGB}{154, 0, 215}
\definecolor{mcorange}{RGB}{255, 96, 0}
\definecolor{mcturquoise}{RGB}{0, 153, 153}
\definecolor{mcrouge}{RGB}{255, 0, 0}
\definecolor{mclightvert}{RGB}{205, 234, 190}

\definecolor{gris}{RGB}{220, 220, 220}
\definecolor{bleu}{RGB}{52, 152, 219}
\definecolor{vert}{RGB}{125, 194, 70}
\definecolor{mauve}{RGB}{154, 0, 215}
\definecolor{orange}{RGB}{255, 96, 0}
\definecolor{turquoise}{RGB}{0, 153, 153}
\definecolor{rouge}{RGB}{255, 0, 0}
\definecolor{lightvert}{RGB}{205, 234, 190}
\setitemize[0]{label=\color{lightvert}  $\bullet$}

\pagestyle{fancy}
\renewcommand{\headrulewidth}{0.2pt}
\fancyhead[L]{maths-cours.fr}
\fancyhead[R]{\thepage}
\renewcommand{\footrulewidth}{0.2pt}
\fancyfoot[C]{}

\newcolumntype{C}{>{\centering\arraybackslash}X}
\newcolumntype{s}{>{\hsize=.35\hsize\arraybackslash}X}

\setlength{\parindent}{0pt}		 
\setlength{\parskip}{3mm}
\setlength{\headheight}{1cm}

\def\ebook{ebook}
\def\book{book}
\def\web{web}
\def\type{web}

\newcommand{\vect}[1]{\overrightarrow{\,\mathstrut#1\,}}

\def\Oij{$\left(\text{O}~;~\vect{\imath},~\vect{\jmath}\right)$}
\def\Oijk{$\left(\text{O}~;~\vect{\imath},~\vect{\jmath},~\vect{k}\right)$}
\def\Ouv{$\left(\text{O}~;~\vect{u},~\vect{v}\right)$}

\hypersetup{breaklinks=true, colorlinks = true, linkcolor = OliveGreen, urlcolor = OliveGreen, citecolor = OliveGreen, pdfauthor={Didier BONNEL - https://www.maths-cours.fr} } % supprime les bordures autour des liens

\renewcommand{\arg}[0]{\text{arg}}

\everymath{\displaystyle}

%================================================================================================================================
%
% Macros - Commandes
%
%================================================================================================================================

\newcommand\meta[2]{    			% Utilisé pour créer le post HTML.
	\def\titre{titre}
	\def\url{url}
	\def\arg{#1}
	\ifx\titre\arg
		\newcommand\maintitle{#2}
		\fancyhead[L]{#2}
		{\Large\sffamily \MakeUppercase{#2}}
		\vspace{1mm}\textcolor{mcvert}{\hrule}
	\fi 
	\ifx\url\arg
		\fancyfoot[L]{\href{https://www.maths-cours.fr#2}{\black \footnotesize{https://www.maths-cours.fr#2}}}
	\fi 
}


\newcommand\TitreC[1]{    		% Titre centré
     \needspace{3\baselineskip}
     \begin{center}\textbf{#1}\end{center}
}

\newcommand\newpar{    		% paragraphe
     \par
}

\newcommand\nosp {    		% commande vide (pas d'espace)
}
\newcommand{\id}[1]{} %ignore

\newcommand\boite[2]{				% Boite simple sans titre
	\vspace{5mm}
	\setlength{\fboxrule}{0.2mm}
	\setlength{\fboxsep}{5mm}	
	\fcolorbox{#1}{#1!3}{\makebox[\linewidth-2\fboxrule-2\fboxsep]{
  		\begin{minipage}[t]{\linewidth-2\fboxrule-4\fboxsep}\setlength{\parskip}{3mm}
  			 #2
  		\end{minipage}
	}}
	\vspace{5mm}
}

\newcommand\CBox[4]{				% Boites
	\vspace{5mm}
	\setlength{\fboxrule}{0.2mm}
	\setlength{\fboxsep}{5mm}
	
	\fcolorbox{#1}{#1!3}{\makebox[\linewidth-2\fboxrule-2\fboxsep]{
		\begin{minipage}[t]{1cm}\setlength{\parskip}{3mm}
	  		\textcolor{#1}{\LARGE{#2}}    
 	 	\end{minipage}  
  		\begin{minipage}[t]{\linewidth-2\fboxrule-4\fboxsep}\setlength{\parskip}{3mm}
			\raisebox{1.2mm}{\normalsize\sffamily{\textcolor{#1}{#3}}}						
  			 #4
  		\end{minipage}
	}}
	\vspace{5mm}
}

\newcommand\cadre[3]{				% Boites convertible html
	\par
	\vspace{2mm}
	\setlength{\fboxrule}{0.1mm}
	\setlength{\fboxsep}{5mm}
	\fcolorbox{#1}{white}{\makebox[\linewidth-2\fboxrule-2\fboxsep]{
  		\begin{minipage}[t]{\linewidth-2\fboxrule-4\fboxsep}\setlength{\parskip}{3mm}
			\raisebox{-2.5mm}{\sffamily \small{\textcolor{#1}{\MakeUppercase{#2}}}}		
			\par		
  			 #3
 	 		\end{minipage}
	}}
		\vspace{2mm}
	\par
}

\newcommand\bloc[3]{				% Boites convertible html sans bordure
     \needspace{2\baselineskip}
     {\sffamily \small{\textcolor{#1}{\MakeUppercase{#2}}}}    
		\par		
  			 #3
		\par
}

\newcommand\CHelp[1]{
     \CBox{Plum}{\faInfoCircle}{À RETENIR}{#1}
}

\newcommand\CUp[1]{
     \CBox{NavyBlue}{\faThumbsOUp}{EN PRATIQUE}{#1}
}

\newcommand\CInfo[1]{
     \CBox{Sepia}{\faArrowCircleRight}{REMARQUE}{#1}
}

\newcommand\CRedac[1]{
     \CBox{PineGreen}{\faEdit}{BIEN R\'EDIGER}{#1}
}

\newcommand\CError[1]{
     \CBox{Red}{\faExclamationTriangle}{ATTENTION}{#1}
}

\newcommand\TitreExo[2]{
\needspace{4\baselineskip}
 {\sffamily\large EXERCICE #1\ (\emph{#2 points})}
\vspace{5mm}
}

\newcommand\img[2]{
          \includegraphics[width=#2\paperwidth]{\imgdir#1}
}

\newcommand\imgsvg[2]{
       \begin{center}   \includegraphics[width=#2\paperwidth]{\imgsvgdir#1} \end{center}
}


\newcommand\Lien[2]{
     \href{#1}{#2 \tiny \faExternalLink}
}
\newcommand\mcLien[2]{
     \href{https~://www.maths-cours.fr/#1}{#2 \tiny \faExternalLink}
}

\newcommand{\euro}{\eurologo{}}

%================================================================================================================================
%
% Macros - Environement
%
%================================================================================================================================

\newenvironment{tex}{ %
}
{%
}

\newenvironment{indente}{ %
	\setlength\parindent{10mm}
}

{
	\setlength\parindent{0mm}
}

\newenvironment{corrige}{%
     \needspace{3\baselineskip}
     \medskip
     \textbf{\textsc{Corrigé}}
     \medskip
}
{
}

\newenvironment{extern}{%
     \begin{center}
     }
     {
     \end{center}
}

\NewEnviron{code}{%
	\par
     \boite{gray}{\texttt{%
     \BODY
     }}
     \par
}

\newenvironment{vbloc}{% boite sans cadre empeche saut de page
     \begin{minipage}[t]{\linewidth}
     }
     {
     \end{minipage}
}
\NewEnviron{h2}{%
    \needspace{3\baselineskip}
    \vspace{0.6cm}
	\noindent \MakeUppercase{\sffamily \large \BODY}
	\vspace{1mm}\textcolor{mcgris}{\hrule}\vspace{0.4cm}
	\par
}{}

\NewEnviron{h3}{%
    \needspace{3\baselineskip}
	\vspace{5mm}
	\textsc{\BODY}
	\par
}

\NewEnviron{margeneg}{ %
\begin{addmargin}[-1cm]{0cm}
\BODY
\end{addmargin}
}

\NewEnviron{html}{%
}

\begin{document}
\meta{url}{/cours/fonctions-continues-2/}
\meta{pid}{519}
\meta{titre}{Continuité et calcul de dérivées}
\meta{type}{cours}
\begin{h2}1. Fonctions continues\end{h2}
\cadre{bleu}{Définition}{% id="d10"
     Une fonction définie sur un intervalle $I$ est \textit{\textbf{continue}} sur $I$ si l'on peut tracer sa courbe représentative \textit{sans lever le crayon}
}
\bloc{orange}{Exemples}{% id="e10"
     \begin{itemize}
          \item Les fonctions polynômes sont continues sur $\mathbb{R}$.
          \item Les fonctions rationnelles sont continues sur chaque intervalle contenu dans leur ensemble de définition.
          \item La fonction \textit{racine carrée} est continue sur $\mathbb{R}^+$.
          \item Les fonctions \textit{sinus} et \textit{cosinus} sont continues sur $\mathbb{R}$.
     \end{itemize}
}
\cadre{rouge}{Théorème}{% id="t20"
     Si $f$ et $g$ sont continues sur $I$, les fonctions $f+g$, $kf$ ( $k\in \mathbb{R}$ ) et $f\times g$ sont continues sur $I$.
     \par
     Si, de plus, $g$ ne s'annule pas sur $I$, la fonction $\frac{f}{g}$, est continue sur $I$.
}
\cadre{rouge}{Théorème (lien entre continuité et dérivabilité)}{% id="t30"
     Toute fonction \textbf{dérivable} sur un intervalle $I$ est \textbf{continue} sur $I$.
}
\bloc{cyan}{Remarque}{% id="r30"
     \textbf{Attention !} La réciproque est fausse.
     \par
     Par exemple, la fonction valeur absolue ($x\mapsto |x|$) est continue sur $\mathbb{R}$ tout entier mais n'est pas dérivable en 0.
}
\cadre{vert}{Propriété (lien entre continuité et limite)}{% id="p40"
     Si $f$ est une fonction continue sur un intervalle $\left[a ; b\right]$, alors pour tout $\alpha  \in  \left[a ; b\right]$ :
     \par
     $\lim\limits_{x\rightarrow \alpha }f\left(x\right)=\lim\limits_{x\rightarrow \alpha ^-}f\left(x\right)=\lim\limits_{x\rightarrow \alpha ^+}f\left(x\right)=f\left(\alpha \right)$.
}
\bloc{orange}{Exemple}{% id="e40"
     Montrons à l'aide de cette propriété que la fonction «partie entière» (notée $x\mapsto E\left(x\right)$), qui à tout réel $x$ associe le plus grand entier inférieur ou égal à $x$, n'est pas continue en $1$.
     \par
     Si $x$ est un réel positif et strictement inférieur à $1$, sa partie entière vaut $0$.
     \par
     Donc $\lim\limits_{x\rightarrow 1^-}E\left(x\right)=0$.
     \par
     Par ailleurs, la partie entière de $1$ vaut $1$ c'est à dire $E\left(1\right)=1$.
     \par
     Donc  $\lim\limits_{x\rightarrow 1^-}E\left(x\right)\neq E\left(1\right)$.
     \par
     La fonction «~partie entière~» n'est donc pas continue en $1$ (en fait, elle est discontinue en tout point d'abscisse entière).
     \begin{center}
          \begin{extern} %width="400" alt="limite nulle"
               \resizebox{8cm}{!}{%
                    % -+-+-+ variables modifiables
                    %          \def\fonction{int(x) }
                    \def\xmin{-2.8}
                    \def\xmax{2.8}
                    \def\ymin{-2.8}
                    \def\ymax{2.8}
                    \def\xunit{2}  % unités en cm
                    \def\yunit{2}
                    \psset{xunit=\xunit,yunit=\yunit,algebraic=true}
                    \fontsize{15pt}{15pt}\selectfont
                    \begin{pspicture*}[linewidth=1pt](\xmin,\ymin)(\xmax,\ymax)
                         %      \psgrid[gridcolor=mcgris, subgriddiv=5, gridlabels=0pt](\xmin,\ymin)(\xmax,\ymax)
                         \psaxes[linewidth=0.75pt]{->}(0,0)(\xmin,\ymin)(\xmax,\ymax)
                         %    \psplot[plotpoints=2000,linecolor=blue]{\xmin}{\xmax}{\fonction}
                         \psline[linecolor=blue](-1,-1)(0,-1)
                         \psline[linecolor=blue](-2,-2)(-1,-2)
                         \psline[linecolor=blue](0,0)(1,0)
                         \psline[linecolor=blue](1,1)(2,1)
                         \psline[linecolor=blue](2,2)(3,2)
                         \psdots[linecolor=blue,dotstyle=*,dotsize=3pt](0,0)(1,1)(2,2)(-1,-1)(-2,-2)
                         \rput[tr](-0.1,-0.1){$O$}
                         \rput[tl](-2,-2.2){$\color{blue} \mathcal{C}_E$}
                    \end{pspicture*}
               }
          \end{extern}
     \end{center}
     \begin{center}Fonction «~partie entière~»\end{center}
}
\begin{h2}2. Théorème des valeurs intermédiaires\end{h2}
\cadre{rouge}{Théorème des valeurs intermédiaires}{% id="t60"
     Si $f$ est une fonction \textbf{continue} sur un intervalle $\left[a;b\right]$ et si $y_{0}$ est compris entre $f\left(a\right)$ et $f\left(b\right)$, alors l'équation $f\left(x\right)=y_{0}$ admet \textbf{au moins une} solution sur l'intervalle $\left[a ; b\right]$.
}
\bloc{cyan}{Remarques}{% id="r60"
     \begin{itemize}
          \item Ce théorème dit que l'équation $f\left(x\right)=y_{0}$ admet \textbf{une ou plusieurs solutions} mais ne permet pas de déterminer le nombre de ces solutions. Dans les exercices où l'on recherche le nombre de solutions, il faut utiliser le corollaire ci-dessous.
          \item \textbf{Cas particulier fréquent : } Si $f$ est continue et si $f\left(a\right)$ et $f\left(b\right)$ sont de signes contraires, l'équation $f\left(x\right)=0$ admet au moins une solution sur l'intervalle $\left[a ; b\right]$ (en effet,  si $f\left(a\right)$ et $f\left(b\right)$ sont de signes contraires, $0$ est compris entre $f\left(a\right)$ et $f\left(b\right)$).
     \end{itemize}
}
\cadre{rouge}{Corollaire (du théorème des valeurs intermédiaires)}{% id="t70"
     Si $f$ est une fonction \textbf{continue} et \textbf{strictement monotone} sur un intervalle $\left[a ; b\right]$ et si $y_{0}$ est compris entre $f\left(a\right)$ et $f\left(b\right)$, l'équation $f\left(x\right)=y_{0}$ admet une \textbf{unique} solution sur l'intervalle $\left[a ; b\right]$.
}
\bloc{cyan}{Remarques}{% id="r70"
     \begin{itemize}
          \item Ce dernier théorème est aussi parfois appelé \textbf{"Théorème de la bijection"}
          \item
          Il faut vérifier \textbf{3 conditions} pour pouvoir appliquer ce corollaire:
          \begin{itemize}[label=---]
               \item
               $f$ est continue sur $\left[a ; b\right]$~;
               \item
               $f$ est strictement croissante ou strictement décroissante sur $\left[a ; b\right]$~;
               \item
               $y_{0}$ est compris entre $f\left(a\right)$ et $f\left(b\right)$.
          \end{itemize}
          \item Les deux théorèmes précédents se généralisent à un intervalle ouvert $\left]a ; b\right[$ où $a$ et $b$ sont éventuellement infinis. Il faut alors remplacer $f\left(a\right)$ et $f\left(b\right)$ (qui ne sont alors généralement pas définis) par $\lim\limits_{x\rightarrow a}f\left(x\right)$ et $\lim\limits_{x\rightarrow b}f\left(x\right)$
     \end{itemize}
}
\bloc{orange}{Exemple}{% id="e70"
     Soit une fonction $f$ définie sur $\left]0 ; +\infty \right[$ dont le tableau de variation est fourni ci-dessous :
     %:-+-+-+-+- Engendré par : http://math.et.info.free.fr/TikZ/TableauxVariations/
     \begin{center}
          \begin{extern}%width="230"
               %:-+-+-+-+- Engendré par : http://math.et.info.free.fr/TikZ/TableauxVariations/
               \begin{center}
                    \begin{tikzpicture}[scale=0.875]
                         % Styles
                         \tikzstyle{cadre}=[thin]
                         \tikzstyle{fleche}=[->,>=latex,thin]
                         \tikzstyle{nondefini}=[lightgray]
                         % Dimensions Modifiables
                         \def\Lrg{1.5}
                         \def\HtX{1}
                         \def\HtY{0.5}
                         % Dimensions Calculées
                         \def\lignex{-0.5*\HtX}
                         \def\lignef{-1.5*\HtX}
                         \def\separateur{-0.5*\Lrg}
                         % Largeur du tableau
                         \def\gauche{-1.5*\Lrg}
                         \def\droite{2.5*\Lrg}
                         % Hauteur du tableau
                         \def\haut{0.5*\HtX}
                         \def\bas{-2.5*\HtX-2*\HtY}
                         % Ligne de l'abscisse : x
                         \node at (-1*\Lrg,0) {$x$};
                         \node at (0*\Lrg,0) {$0$};
                         \node at (2*\Lrg,0) {$+\infty$};
                         % Ligne de la dérivée : f'(x)
                         \node at (-1*\Lrg,-1*\HtX) {$f'(x)$};
                         \node at (1*\Lrg,-1*\HtX) {$+$};
                         \node at (2*\Lrg,-1*\HtX) {$$};
                         % Ligne de la fonction : f(x)
                         \node  at (-1*\Lrg,{-2*\HtX+(-1)*\HtY}) {$f(x)$};
                         \node[left] (f1) at (0*\Lrg,{-2*\HtX+(0)*\HtY}) {$$};
                         \node[right] (f2) at (0*\Lrg,{-2*\HtX+(-2)*\HtY}) {$-\infty$};
                         \node (f3) at (2*\Lrg,{-2*\HtX+(0)*\HtY}) {$1$};
                         % Flèches
                         \draw[fleche] (f2) -- (f3);
                         % Doubles barres
                         \draw[double distance=2pt] (0*\Lrg,\lignex-0.1*\HtX) -- (0*\Lrg,\lignef+0.1*\HtX);
                         \draw[double distance=2pt] (0*\Lrg,\lignef-0.1*\HtX) -- (0*\Lrg,\bas+0.1*\HtX);
                         % Encadrement
                         \draw[cadre] (\separateur,\haut) -- (\separateur,\bas);
                         \draw[cadre] (\gauche,\haut) rectangle  (\droite,\bas);
                         \draw[cadre] (\gauche,\lignex) -- (\droite,\lignex);
                         \draw[cadre] (\gauche,\lignef) -- (\droite,\lignef);
                    \end{tikzpicture}
               \end{center}
               %:-+-+-+-+- Fin
               %:>>>>> code du tableau à ré-injecter
               %[
               %	["x", "f'(x)", "f(x)"],
               %	["0", ":", "+", ":-\\infty"],
               %	["+\\infty", "", "?", "1"]
               %]
          \end{extern}
     \end{center}
     %:-+-+-+-+- Fin
     %:>>>>> code du tableau à ré-injecter
     %[
     %	["x", "f'(x)", "f(x)"],
     %	["0", "", "+", "||-\\infty"],
     %	["+\\infty", "", "?", "1"]
     %]
     On cherche à déterminer le nombre de solutions de l'équation $f\left(x\right)=-1$.
     \par
     L'unique flèche oblique montre que la fonction $f$ est \textbf{continue} et \textbf{strictement croissante} sur $\left]0;+\infty \right[$.
     \par
     $-1$ est compris entre $\lim\limits_{x\rightarrow 0}f\left(x\right)=-\infty $ et  $\lim\limits_{x\rightarrow +\infty }f\left(x\right)=1$.
     \par
     Par conséquent, l'équation $f\left(x\right)=-1$ admet une \textbf{unique} solution sur l'intervalle  $\left]0 ; +\infty \right[$.
}
\begin{h2}3. Calcul de dérivées\end{h2}
Le tableau ci-dessous recense les dérivées usuelles à connaitre en Terminale S. Pour faciliter les révisions, toutes les formules du programme ont été recensées ; certaines seront étudiées dans les chapitres ultérieurs.
\cadre{vert}{Dérivée des fonctions usuelles }{% id="p80"
     \begin{tabularx}{0.8\linewidth}{|*{3}{>{\centering \arraybackslash }X|}}%class="compact" width="600"
          \hline
          \textbf{Fonction} & \textbf{Dérivée} & \textbf{Ensemble de dérivabilité}
          \\ \hline
          $k$  $\left(k\in \mathbb{R}\right)$  &  $0$  &  $\mathbb{R}$
          \\ \hline
          $x$ &  $1$  &  $\mathbb{R}$
          \\ \hline
          $x^{n}$ $\left(n\in \mathbb{N}\right)$  &  $nx^{n-1}$  &  $\mathbb{R}$
          \\ \hline
          $\frac{1}{x^{n}}$ $\left(n\in \mathbb{N}\right)$ &  $-\frac{n}{x^{n+1}}$  &  $\mathbb{R}-\left\{0\right\}$
          \\ \hline
          $\sqrt{x}$ &  $\frac{1}{2\sqrt{x}}$  &  $\left]0;+\infty \right[$
          \\ \hline
          $\sin\left(x\right)$ &  $\cos\left(x\right)$  &  $\mathbb{R}$
          \\ \hline
          $\cos\left(x\right)$ &  $-\sin\left(x\right)$  &  $\mathbb{R}$
          \\ \hline
          $e^{x}$ &  $e^{x}$  &  $\mathbb{R}$
          \\ \hline
          $\ln\left(x\right)$ &  $\frac{1}{x}$  &  $\left]0;-\infty \right[$
     \end{tabularx}
}
\cadre{vert}{Propriété}{% id="p90"
     Soient une fonction $f$ définie et dérivable sur un certain intervalle et $a$ et $b$ deux réels.
     \par
     Alors la fonction $g  : x\mapsto f\left(ax+b\right)$ est dérivable là où elle est définie et :
     \begin{center}$g^{\prime}\left(x\right)=af^{\prime}\left(ax+b\right)$.\end{center}
}
\bloc{orange}{Exemples}{% id="e90"
     \begin{itemize}
          \item La fonction $f  :  x\mapsto \left(5x+2\right)^{3}$ est définie et dérivable sur $\mathbb{R}$ et :
          \par
          $f^{\prime}\left(x\right)=5\times 3\left(5x+2\right)^{2}=15\left(5x+2\right)^{2}$.
          \item En particulier, si $g\left(x\right)=f\left(-x\right)$ on a $g^{\prime}\left(x\right)=-f^{\prime}\left(-x\right)$.
          \par
          Par exemple la dérivée de la fonction $x\mapsto e^{-x}$ est la fonction $x\mapsto -e^{-x}$.
     \end{itemize}
}
\bloc{cyan}{Remarque}{% id="r90"
     Le résultat précédent se généralise à l'aide du théorème suivant :
}
\cadre{rouge}{Théorème (dérivées des fonctions composées)}{% id="t100"
     Soit $u$ une fonction dérivable sur un intervalle $I$ et  prenant ses valeurs dans un intervalle $J$ et soit  $f$ une fonction dérivable sur $J$.
     \par
     Alors la fonction $g  :  x\mapsto f\left(u\left(x\right)\right)$ est dérivable sur $I$ et :
     \begin{center}$g^{\prime}\left(x\right)=u^{\prime}\left(x\right)\times f^{\prime}\left(u\left(x\right)\right).$\end{center}
}
\bloc{orange}{Exemples}{% id="e100"
     Soit $u$ une fonction dérivable sur intervalle $I$ :
     \begin{itemize}
          \item la fonction $u^{n}$ est dérivable sur $I$ et sa dérivée est $u^{\prime}\times nu^{n-1}$~;
          \item la fonction $\frac{1}{u}$ est dérivable sur la partie de $I$ où $u\neq 0$ et sa dérivée est $-\frac{u^{\prime}}{u^{2}}$~;
          \item la fonction $\sqrt{u}$ est dérivable sur la partie de $I$ où $u > 0$ et sa dérivée est $\frac{u^{\prime}}{2\sqrt{u}}$~;
          \item la fonction $\sin\left(u\right)$ est dérivable sur $I$ et sa dérivée est $u^{\prime}\times \cos\left(u\right)$~;
          \item la fonction $\cos\left(u\right)$ est dérivable sur $I$ et sa dérivée est $-u^{\prime}\times \sin\left(u\right)$~;
          \item la fonction $e^{u}$ est dérivable sur $I$ et sa dérivée est $u^{\prime}\times e^{u}$~;
          \item la fonction $\ln\left(u\right)$ est dérivable sur la partie de $I$ où $u > 0$ et sa dérivée est $\frac{u^{\prime}}{u}$.
     \end{itemize}
}

\end{document}
µ
\documentclass[a4paper]{article}

%================================================================================================================================
%
% Packages
%
%================================================================================================================================

\usepackage[T1]{fontenc} 	% pour caractères accentués
\usepackage[utf8]{inputenc}  % encodage utf8
\usepackage[french]{babel}	% langue : français
\usepackage{fourier}			% caractères plus lisibles
\usepackage[dvipsnames]{xcolor} % couleurs
\usepackage{fancyhdr}		% réglage header footer
\usepackage{needspace}		% empêcher sauts de page mal placés
\usepackage{graphicx}		% pour inclure des graphiques
\usepackage{enumitem,cprotect}		% personnalise les listes d'items (nécessaire pour ol, al ...)
\usepackage{hyperref}		% Liens hypertexte
\usepackage{pstricks,pst-all,pst-node,pstricks-add,pst-math,pst-plot,pst-tree,pst-eucl} % pstricks
\usepackage[a4paper,includeheadfoot,top=2cm,left=3cm, bottom=2cm,right=3cm]{geometry} % marges etc.
\usepackage{comment}			% commentaires multilignes
\usepackage{amsmath,environ} % maths (matrices, etc.)
\usepackage{amssymb,makeidx}
\usepackage{bm}				% bold maths
\usepackage{tabularx}		% tableaux
\usepackage{colortbl}		% tableaux en couleur
\usepackage{fontawesome}		% Fontawesome
\usepackage{environ}			% environment with command
\usepackage{fp}				% calculs pour ps-tricks
\usepackage{multido}			% pour ps tricks
\usepackage[np]{numprint}	% formattage nombre
\usepackage{tikz,tkz-tab} 			% package principal TikZ
\usepackage{pgfplots}   % axes
\usepackage{mathrsfs}    % cursives
\usepackage{calc}			% calcul taille boites
\usepackage[scaled=0.875]{helvet} % font sans serif
\usepackage{svg} % svg
\usepackage{scrextend} % local margin
\usepackage{scratch} %scratch
\usepackage{multicol} % colonnes
%\usepackage{infix-RPN,pst-func} % formule en notation polanaise inversée
\usepackage{listings}

%================================================================================================================================
%
% Réglages de base
%
%================================================================================================================================

\lstset{
language=Python,   % R code
literate=
{á}{{\'a}}1
{à}{{\`a}}1
{ã}{{\~a}}1
{é}{{\'e}}1
{è}{{\`e}}1
{ê}{{\^e}}1
{í}{{\'i}}1
{ó}{{\'o}}1
{õ}{{\~o}}1
{ú}{{\'u}}1
{ü}{{\"u}}1
{ç}{{\c{c}}}1
{~}{{ }}1
}


\definecolor{codegreen}{rgb}{0,0.6,0}
\definecolor{codegray}{rgb}{0.5,0.5,0.5}
\definecolor{codepurple}{rgb}{0.58,0,0.82}
\definecolor{backcolour}{rgb}{0.95,0.95,0.92}

\lstdefinestyle{mystyle}{
    backgroundcolor=\color{backcolour},   
    commentstyle=\color{codegreen},
    keywordstyle=\color{magenta},
    numberstyle=\tiny\color{codegray},
    stringstyle=\color{codepurple},
    basicstyle=\ttfamily\footnotesize,
    breakatwhitespace=false,         
    breaklines=true,                 
    captionpos=b,                    
    keepspaces=true,                 
    numbers=left,                    
xleftmargin=2em,
framexleftmargin=2em,            
    showspaces=false,                
    showstringspaces=false,
    showtabs=false,                  
    tabsize=2,
    upquote=true
}

\lstset{style=mystyle}


\lstset{style=mystyle}
\newcommand{\imgdir}{C:/laragon/www/newmc/assets/imgsvg/}
\newcommand{\imgsvgdir}{C:/laragon/www/newmc/assets/imgsvg/}

\definecolor{mcgris}{RGB}{220, 220, 220}% ancien~; pour compatibilité
\definecolor{mcbleu}{RGB}{52, 152, 219}
\definecolor{mcvert}{RGB}{125, 194, 70}
\definecolor{mcmauve}{RGB}{154, 0, 215}
\definecolor{mcorange}{RGB}{255, 96, 0}
\definecolor{mcturquoise}{RGB}{0, 153, 153}
\definecolor{mcrouge}{RGB}{255, 0, 0}
\definecolor{mclightvert}{RGB}{205, 234, 190}

\definecolor{gris}{RGB}{220, 220, 220}
\definecolor{bleu}{RGB}{52, 152, 219}
\definecolor{vert}{RGB}{125, 194, 70}
\definecolor{mauve}{RGB}{154, 0, 215}
\definecolor{orange}{RGB}{255, 96, 0}
\definecolor{turquoise}{RGB}{0, 153, 153}
\definecolor{rouge}{RGB}{255, 0, 0}
\definecolor{lightvert}{RGB}{205, 234, 190}
\setitemize[0]{label=\color{lightvert}  $\bullet$}

\pagestyle{fancy}
\renewcommand{\headrulewidth}{0.2pt}
\fancyhead[L]{maths-cours.fr}
\fancyhead[R]{\thepage}
\renewcommand{\footrulewidth}{0.2pt}
\fancyfoot[C]{}

\newcolumntype{C}{>{\centering\arraybackslash}X}
\newcolumntype{s}{>{\hsize=.35\hsize\arraybackslash}X}

\setlength{\parindent}{0pt}		 
\setlength{\parskip}{3mm}
\setlength{\headheight}{1cm}

\def\ebook{ebook}
\def\book{book}
\def\web{web}
\def\type{web}

\newcommand{\vect}[1]{\overrightarrow{\,\mathstrut#1\,}}

\def\Oij{$\left(\text{O}~;~\vect{\imath},~\vect{\jmath}\right)$}
\def\Oijk{$\left(\text{O}~;~\vect{\imath},~\vect{\jmath},~\vect{k}\right)$}
\def\Ouv{$\left(\text{O}~;~\vect{u},~\vect{v}\right)$}

\hypersetup{breaklinks=true, colorlinks = true, linkcolor = OliveGreen, urlcolor = OliveGreen, citecolor = OliveGreen, pdfauthor={Didier BONNEL - https://www.maths-cours.fr} } % supprime les bordures autour des liens

\renewcommand{\arg}[0]{\text{arg}}

\everymath{\displaystyle}

%================================================================================================================================
%
% Macros - Commandes
%
%================================================================================================================================

\newcommand\meta[2]{    			% Utilisé pour créer le post HTML.
	\def\titre{titre}
	\def\url{url}
	\def\arg{#1}
	\ifx\titre\arg
		\newcommand\maintitle{#2}
		\fancyhead[L]{#2}
		{\Large\sffamily \MakeUppercase{#2}}
		\vspace{1mm}\textcolor{mcvert}{\hrule}
	\fi 
	\ifx\url\arg
		\fancyfoot[L]{\href{https://www.maths-cours.fr#2}{\black \footnotesize{https://www.maths-cours.fr#2}}}
	\fi 
}


\newcommand\TitreC[1]{    		% Titre centré
     \needspace{3\baselineskip}
     \begin{center}\textbf{#1}\end{center}
}

\newcommand\newpar{    		% paragraphe
     \par
}

\newcommand\nosp {    		% commande vide (pas d'espace)
}
\newcommand{\id}[1]{} %ignore

\newcommand\boite[2]{				% Boite simple sans titre
	\vspace{5mm}
	\setlength{\fboxrule}{0.2mm}
	\setlength{\fboxsep}{5mm}	
	\fcolorbox{#1}{#1!3}{\makebox[\linewidth-2\fboxrule-2\fboxsep]{
  		\begin{minipage}[t]{\linewidth-2\fboxrule-4\fboxsep}\setlength{\parskip}{3mm}
  			 #2
  		\end{minipage}
	}}
	\vspace{5mm}
}

\newcommand\CBox[4]{				% Boites
	\vspace{5mm}
	\setlength{\fboxrule}{0.2mm}
	\setlength{\fboxsep}{5mm}
	
	\fcolorbox{#1}{#1!3}{\makebox[\linewidth-2\fboxrule-2\fboxsep]{
		\begin{minipage}[t]{1cm}\setlength{\parskip}{3mm}
	  		\textcolor{#1}{\LARGE{#2}}    
 	 	\end{minipage}  
  		\begin{minipage}[t]{\linewidth-2\fboxrule-4\fboxsep}\setlength{\parskip}{3mm}
			\raisebox{1.2mm}{\normalsize\sffamily{\textcolor{#1}{#3}}}						
  			 #4
  		\end{minipage}
	}}
	\vspace{5mm}
}

\newcommand\cadre[3]{				% Boites convertible html
	\par
	\vspace{2mm}
	\setlength{\fboxrule}{0.1mm}
	\setlength{\fboxsep}{5mm}
	\fcolorbox{#1}{white}{\makebox[\linewidth-2\fboxrule-2\fboxsep]{
  		\begin{minipage}[t]{\linewidth-2\fboxrule-4\fboxsep}\setlength{\parskip}{3mm}
			\raisebox{-2.5mm}{\sffamily \small{\textcolor{#1}{\MakeUppercase{#2}}}}		
			\par		
  			 #3
 	 		\end{minipage}
	}}
		\vspace{2mm}
	\par
}

\newcommand\bloc[3]{				% Boites convertible html sans bordure
     \needspace{2\baselineskip}
     {\sffamily \small{\textcolor{#1}{\MakeUppercase{#2}}}}    
		\par		
  			 #3
		\par
}

\newcommand\CHelp[1]{
     \CBox{Plum}{\faInfoCircle}{À RETENIR}{#1}
}

\newcommand\CUp[1]{
     \CBox{NavyBlue}{\faThumbsOUp}{EN PRATIQUE}{#1}
}

\newcommand\CInfo[1]{
     \CBox{Sepia}{\faArrowCircleRight}{REMARQUE}{#1}
}

\newcommand\CRedac[1]{
     \CBox{PineGreen}{\faEdit}{BIEN R\'EDIGER}{#1}
}

\newcommand\CError[1]{
     \CBox{Red}{\faExclamationTriangle}{ATTENTION}{#1}
}

\newcommand\TitreExo[2]{
\needspace{4\baselineskip}
 {\sffamily\large EXERCICE #1\ (\emph{#2 points})}
\vspace{5mm}
}

\newcommand\img[2]{
          \includegraphics[width=#2\paperwidth]{\imgdir#1}
}

\newcommand\imgsvg[2]{
       \begin{center}   \includegraphics[width=#2\paperwidth]{\imgsvgdir#1} \end{center}
}


\newcommand\Lien[2]{
     \href{#1}{#2 \tiny \faExternalLink}
}
\newcommand\mcLien[2]{
     \href{https~://www.maths-cours.fr/#1}{#2 \tiny \faExternalLink}
}

\newcommand{\euro}{\eurologo{}}

%================================================================================================================================
%
% Macros - Environement
%
%================================================================================================================================

\newenvironment{tex}{ %
}
{%
}

\newenvironment{indente}{ %
	\setlength\parindent{10mm}
}

{
	\setlength\parindent{0mm}
}

\newenvironment{corrige}{%
     \needspace{3\baselineskip}
     \medskip
     \textbf{\textsc{Corrigé}}
     \medskip
}
{
}

\newenvironment{extern}{%
     \begin{center}
     }
     {
     \end{center}
}

\NewEnviron{code}{%
	\par
     \boite{gray}{\texttt{%
     \BODY
     }}
     \par
}

\newenvironment{vbloc}{% boite sans cadre empeche saut de page
     \begin{minipage}[t]{\linewidth}
     }
     {
     \end{minipage}
}
\NewEnviron{h2}{%
    \needspace{3\baselineskip}
    \vspace{0.6cm}
	\noindent \MakeUppercase{\sffamily \large \BODY}
	\vspace{1mm}\textcolor{mcgris}{\hrule}\vspace{0.4cm}
	\par
}{}

\NewEnviron{h3}{%
    \needspace{3\baselineskip}
	\vspace{5mm}
	\textsc{\BODY}
	\par
}

\NewEnviron{margeneg}{ %
\begin{addmargin}[-1cm]{0cm}
\BODY
\end{addmargin}
}

\NewEnviron{html}{%
}

\begin{document}
\meta{url}{/cours/fonction-exponentielle/}
\meta{pid}{521}
\meta{titre}{Fonction exponentielle en Terminale S}
\meta{type}{cours}
\begin{h2}1. Définition de la fonction exponentielle\end{h2}
\cadre{rouge}{Théorème et Définition}{% id="t10"
     Il existe une unique fonction $f$ dérivable sur $\mathbb{R}$ telle que $f^{\prime}=f$ et $f\left(0\right)=1$
     \par
     Cette fonction est appelée \textbf{fonction exponentielle} (de base e) et notée $\text{exp}$.
}
\bloc{cyan}{Remarque}{% id="r10"
     L'existence d'une telle fonction est admise.
     \par
     Son unicité est démontrée dans l'exercice : \mcLien{/exercices/logarithme-exponentielle/roc-proprietes-fondamentales-exponentielle}{[ROC] Propriétés fondamentales de la fonction exponentielle}
}
\cadre{bleu}{Notation}{% id="n20"
     On note $\text{e}=\text{exp}\left(1\right)$.
     \par
     On démontre que pour tout entier relatif $n \in  \mathbb{Z}$ : $\text{exp}\left(n\right)=\text{e}^{n}$
     \par
     Cette propriété conduit à noter $\text{e}^{x}$ l'exponentielle de $x$ pour tout $x \in  \mathbb{R}$
}
\bloc{cyan}{Remarque}{% id="r20"
     On démontre (mais c'est hors programme) que $\text{e} \left(\approx 2,71828 . . . \right)$ est un nombre \textbf{irrationnel}, c'est à dire qu'il ne peut s'écrire sous forme de fraction.
}
\begin{h2}2. Etude de la fonction exponentielle\end{h2}
\cadre{vert}{Propriété}{% id="p30"
     La fonction exponentielle est \textbf{strictement positive} et \textbf{strictement croissante} sur $\mathbb{R}$.
}
\bloc{cyan}{Remarque}{% id="r30"
     Cette propriété très importante est démontrée dans l'exercice : \mcLien{/exercices/logarithme-exponentielle/roc-proprietes-fondamentales-exponentielle}{[ROC] Propriétés fondamentales de la fonction exponentielle}
}
\cadre{vert}{Propriété}{% id="p35"
     Soit $u$ une fonction dérivable sur un intervalle $I$.
     \par
     Alors la fonction $ f :  x\mapsto \text{e}^{u\left(x\right)}$ est dérivable sur $I$ et :
     \begin{center}$f^{\prime}=u^{\prime}\text{e}^{u}$\end{center}
}
\bloc{cyan}{Démonstration}{% id="r35"
     On utilise le  \mcLien{/cours/terminale-s/fonctions-continues-2\#t100}{théorème de dérivation de fonctions composées}.
}
\bloc{orange}{Exemple}{% id="e35"
     Soit $f$ définie sur $\mathbb{R}$ par $f\left(x\right)=\text{e}^{-x}$
     \par
     $f$ est dérivable sur $\mathbb{R}$ et $f^{\prime}\left(x\right)=-\text{e}^{-x}$
}
\cadre{vert}{Limites}{% id="t40"
     \begin{itemize}
          \item $\lim\limits_{x\rightarrow -\infty }\text{e}^{x}=0$
          \item $\lim\limits_{x\rightarrow +\infty }\text{e}^{x}=+\infty $
     \end{itemize}
}
\bloc{cyan}{Remarques}{% id="r40"
     \begin{itemize}
          \item Ces résultats sont démontrés dans l'exercice : \mcLien{/exercices/logarithme-exponentielle/roc-limites-exponentielle}{[ROC] Limites de la fonction exponentielle}
          \item On déduit des résultats précédents le tableau de variation et l'allure de la courbe de la fonction exponentielle:
     \end{itemize}
}
\begin{center}
     \begin{extern}%width="600" alt="Fonction exponentielle : tableau de variation"
          \begin{tikzpicture}
               \tkzTabInit[lgt=3,espcl=10] {$x$ /1, $f'(x)=\text{e}^{x}$ /1.5,%
               $f(x)=\text{e}^{x}$/2} {$-\infty$ , $+\infty$}%
               \tkzTabLine{,+,}%
               \tkzTabVar{ - / $0$, + / $+\infty$ }
               \tkzTabVal[draw]{1}{2}{0.33}{0}{1}
               \tkzTabVal[draw]{1}{2}{0.66}{1}{e}
          \end{tikzpicture}
     \end{extern}
\end{center}
\begin{center}
     \textit{Tableau de variation de la fonction exponentielle }
\end{center}
\par
\begin{center}
     \begin{extern} %width="400" alt="Fonction exponentielle : graphique"
          \resizebox{8cm}{!}{%
               % -+-+-+ variables modifiables
               \def\fonction{EXP(x) }
               \def\xmin{-3.5}
               \def\xmax{2.5}
               \def\ymin{-1.5}
               \def\ymax{5}
               \def\xunit{2}  % unités en cm
               \def\yunit{2}
               \psset{xunit=\xunit,yunit=\yunit,algebraic=true}
               \fontsize{15pt}{15pt}\selectfont
               \begin{pspicture*}[linewidth=1pt](\xmin,\ymin)(\xmax,\ymax)
                    %      \psgrid[gridcolor=mcgris, subgriddiv=5, gridlabels=0pt](\xmin,\ymin)(\xmax,\ymax)
                    \psaxes[linewidth=0.75pt]{->}(0,0)(\xmin,\ymin)(\xmax,\ymax)
                    \psplot[plotpoints=2000,linecolor=blue]{\xmin}{\xmax}{\fonction}
                    \rput[tr](-0.1,-0.1){$O$}
                    \rput[tl](1.5,4){$\color{blue} \mathcal{C}_{\text{exp}}$}
                    \psline(1,0)(1,2.7183)(0,2.7183)
                    \rput[r](-0.1,2.7183){$\text{e}$}
               \end{pspicture*}
          }
     \end{extern}
\end{center}
\begin{center}
     \textit{Graphique de la fonction exponentielle }
\end{center}
\cadre{rouge}{Théorème ( «Croissance comparée»)}{% id="t50"
     \begin{itemize}
          \item $\lim\limits_{x\rightarrow -\infty }x\text{e}^{x}=0$
          \item $\lim\limits_{x\rightarrow +\infty }\frac{\text{e}^{x}}{x}=+\infty $
          \item $\lim\limits_{x\rightarrow 0}\frac{\text{e}^{x}-1}{x}=1$
     \end{itemize}
}
\bloc{cyan}{Remarques}{% id="r50"
     \begin{itemize}
          \item Voir, à nouveau, l'exercice : \mcLien{/exercices/logarithme-exponentielle/roc-limites-exponentielle}{[ROC] Limites de la fonction exponentielle} pour la démonstration des deux premières formules.
          \item Les deux premières formules peuvent se généraliser de la façon suivante :
          \par
          Pour tout entier $n > 0$ :
          \par
          $    \lim\limits_{x\rightarrow -\infty }x^{n}\text{e}^{x}=0$
          \par
          $    \lim\limits_{x\rightarrow +\infty }\frac{\text{e}^{x}}{x^{n}}=+\infty $
          \item La troisième formule s'obtient en utilisant la définition du nombre dérivé pour x=0 : (voir \mcLien{/methodes/limites/calcul-limite-nombre-derive}{Calculer une limite à l'aide du nombre dérivé}).
          \par
          $\lim\limits_{x\rightarrow 0}\frac{\text{e}^{x}-1}{x}=\text{exp}^{\prime}\left(0\right)=\text{exp}\left(0\right)=1$
     \end{itemize}
}
\cadre{rouge}{Théorème}{% id="t60"
     La fonction exponentielle étant strictement \textbf{croissante}, si $a$ et $b$ sont deux réels :
     \begin{itemize}
          \item $\text{e}^{a}=\text{e}^{b}$ si et seulement si $a=b$
          \item $\text{e}^{a} <  \text{e}^{b}$ si et seulement si $ a < b $
     \end{itemize}
}
\bloc{cyan}{Remarque}{% id="r60"
     Ces résultats sont extrêmement utiles pour résoudre équations et inéquations.
}
\begin{h2}3. Propriétés algébriques de la fonction exponentielle\end{h2}
\cadre{vert}{Propriétés}{% id="p70"
     Pour tout réels $a$ et $b$ et tout entier $n \in  \mathbb{Z}$ :
     \begin{itemize}
          \item $\text{e}^{a+b}=\text{e}^{a} \times  \text{e}^{b}$
          \item $\text{e}^{-a}=\frac{1}{\text{e}^{a}}$
          \item $\text{e}^{a-b}=\frac{\text{e}^{a}}{\text{e}^{b}}$
          \item $\left(\text{e}^{a}\right)^{n}=\text{e}^{na}$
     \end{itemize}
}
\bloc{cyan}{Remarques}{% id="r70"
     \begin{itemize}
          \item Ces propriétés  sont démontrées dans l'exercice : \mcLien{/exercices/logarithme-exponentielle/roc-proprietes-algebriques-exponentielle}{[ROC] Propriétés algébriques de la fonction exponentielle}
          Elles sont similaires aux propriétés des puissances vues au collège (et justifient la notation $\text{e}^{x}$)
          \item Si l'on pose $a=\frac{1}{2}$ et $n=2$ dans la formule $\left(\text{e}^{a}\right)^{n}=\text{e}^{na}$ on obtient $\left(\text{e}^{^{\frac{1}{2}}}\right)^{2}=\text{e}^{1}=\text{e}$ donc comme  $\text{e}^{^{\frac{1}{2}}} > 0$ : $\text{e}^{^{\frac{1}{2}}}=\sqrt{\text{e}}$
     \end{itemize}
}

\end{document}
µ
\documentclass[a4paper]{article}

%================================================================================================================================
%
% Packages
%
%================================================================================================================================

\usepackage[T1]{fontenc} 	% pour caractères accentués
\usepackage[utf8]{inputenc}  % encodage utf8
\usepackage[french]{babel}	% langue : français
\usepackage{fourier}			% caractères plus lisibles
\usepackage[dvipsnames]{xcolor} % couleurs
\usepackage{fancyhdr}		% réglage header footer
\usepackage{needspace}		% empêcher sauts de page mal placés
\usepackage{graphicx}		% pour inclure des graphiques
\usepackage{enumitem,cprotect}		% personnalise les listes d'items (nécessaire pour ol, al ...)
\usepackage{hyperref}		% Liens hypertexte
\usepackage{pstricks,pst-all,pst-node,pstricks-add,pst-math,pst-plot,pst-tree,pst-eucl} % pstricks
\usepackage[a4paper,includeheadfoot,top=2cm,left=3cm, bottom=2cm,right=3cm]{geometry} % marges etc.
\usepackage{comment}			% commentaires multilignes
\usepackage{amsmath,environ} % maths (matrices, etc.)
\usepackage{amssymb,makeidx}
\usepackage{bm}				% bold maths
\usepackage{tabularx}		% tableaux
\usepackage{colortbl}		% tableaux en couleur
\usepackage{fontawesome}		% Fontawesome
\usepackage{environ}			% environment with command
\usepackage{fp}				% calculs pour ps-tricks
\usepackage{multido}			% pour ps tricks
\usepackage[np]{numprint}	% formattage nombre
\usepackage{tikz,tkz-tab} 			% package principal TikZ
\usepackage{pgfplots}   % axes
\usepackage{mathrsfs}    % cursives
\usepackage{calc}			% calcul taille boites
\usepackage[scaled=0.875]{helvet} % font sans serif
\usepackage{svg} % svg
\usepackage{scrextend} % local margin
\usepackage{scratch} %scratch
\usepackage{multicol} % colonnes
%\usepackage{infix-RPN,pst-func} % formule en notation polanaise inversée
\usepackage{listings}

%================================================================================================================================
%
% Réglages de base
%
%================================================================================================================================

\lstset{
language=Python,   % R code
literate=
{á}{{\'a}}1
{à}{{\`a}}1
{ã}{{\~a}}1
{é}{{\'e}}1
{è}{{\`e}}1
{ê}{{\^e}}1
{í}{{\'i}}1
{ó}{{\'o}}1
{õ}{{\~o}}1
{ú}{{\'u}}1
{ü}{{\"u}}1
{ç}{{\c{c}}}1
{~}{{ }}1
}


\definecolor{codegreen}{rgb}{0,0.6,0}
\definecolor{codegray}{rgb}{0.5,0.5,0.5}
\definecolor{codepurple}{rgb}{0.58,0,0.82}
\definecolor{backcolour}{rgb}{0.95,0.95,0.92}

\lstdefinestyle{mystyle}{
    backgroundcolor=\color{backcolour},   
    commentstyle=\color{codegreen},
    keywordstyle=\color{magenta},
    numberstyle=\tiny\color{codegray},
    stringstyle=\color{codepurple},
    basicstyle=\ttfamily\footnotesize,
    breakatwhitespace=false,         
    breaklines=true,                 
    captionpos=b,                    
    keepspaces=true,                 
    numbers=left,                    
xleftmargin=2em,
framexleftmargin=2em,            
    showspaces=false,                
    showstringspaces=false,
    showtabs=false,                  
    tabsize=2,
    upquote=true
}

\lstset{style=mystyle}


\lstset{style=mystyle}
\newcommand{\imgdir}{C:/laragon/www/newmc/assets/imgsvg/}
\newcommand{\imgsvgdir}{C:/laragon/www/newmc/assets/imgsvg/}

\definecolor{mcgris}{RGB}{220, 220, 220}% ancien~; pour compatibilité
\definecolor{mcbleu}{RGB}{52, 152, 219}
\definecolor{mcvert}{RGB}{125, 194, 70}
\definecolor{mcmauve}{RGB}{154, 0, 215}
\definecolor{mcorange}{RGB}{255, 96, 0}
\definecolor{mcturquoise}{RGB}{0, 153, 153}
\definecolor{mcrouge}{RGB}{255, 0, 0}
\definecolor{mclightvert}{RGB}{205, 234, 190}

\definecolor{gris}{RGB}{220, 220, 220}
\definecolor{bleu}{RGB}{52, 152, 219}
\definecolor{vert}{RGB}{125, 194, 70}
\definecolor{mauve}{RGB}{154, 0, 215}
\definecolor{orange}{RGB}{255, 96, 0}
\definecolor{turquoise}{RGB}{0, 153, 153}
\definecolor{rouge}{RGB}{255, 0, 0}
\definecolor{lightvert}{RGB}{205, 234, 190}
\setitemize[0]{label=\color{lightvert}  $\bullet$}

\pagestyle{fancy}
\renewcommand{\headrulewidth}{0.2pt}
\fancyhead[L]{maths-cours.fr}
\fancyhead[R]{\thepage}
\renewcommand{\footrulewidth}{0.2pt}
\fancyfoot[C]{}

\newcolumntype{C}{>{\centering\arraybackslash}X}
\newcolumntype{s}{>{\hsize=.35\hsize\arraybackslash}X}

\setlength{\parindent}{0pt}		 
\setlength{\parskip}{3mm}
\setlength{\headheight}{1cm}

\def\ebook{ebook}
\def\book{book}
\def\web{web}
\def\type{web}

\newcommand{\vect}[1]{\overrightarrow{\,\mathstrut#1\,}}

\def\Oij{$\left(\text{O}~;~\vect{\imath},~\vect{\jmath}\right)$}
\def\Oijk{$\left(\text{O}~;~\vect{\imath},~\vect{\jmath},~\vect{k}\right)$}
\def\Ouv{$\left(\text{O}~;~\vect{u},~\vect{v}\right)$}

\hypersetup{breaklinks=true, colorlinks = true, linkcolor = OliveGreen, urlcolor = OliveGreen, citecolor = OliveGreen, pdfauthor={Didier BONNEL - https://www.maths-cours.fr} } % supprime les bordures autour des liens

\renewcommand{\arg}[0]{\text{arg}}

\everymath{\displaystyle}

%================================================================================================================================
%
% Macros - Commandes
%
%================================================================================================================================

\newcommand\meta[2]{    			% Utilisé pour créer le post HTML.
	\def\titre{titre}
	\def\url{url}
	\def\arg{#1}
	\ifx\titre\arg
		\newcommand\maintitle{#2}
		\fancyhead[L]{#2}
		{\Large\sffamily \MakeUppercase{#2}}
		\vspace{1mm}\textcolor{mcvert}{\hrule}
	\fi 
	\ifx\url\arg
		\fancyfoot[L]{\href{https://www.maths-cours.fr#2}{\black \footnotesize{https://www.maths-cours.fr#2}}}
	\fi 
}


\newcommand\TitreC[1]{    		% Titre centré
     \needspace{3\baselineskip}
     \begin{center}\textbf{#1}\end{center}
}

\newcommand\newpar{    		% paragraphe
     \par
}

\newcommand\nosp {    		% commande vide (pas d'espace)
}
\newcommand{\id}[1]{} %ignore

\newcommand\boite[2]{				% Boite simple sans titre
	\vspace{5mm}
	\setlength{\fboxrule}{0.2mm}
	\setlength{\fboxsep}{5mm}	
	\fcolorbox{#1}{#1!3}{\makebox[\linewidth-2\fboxrule-2\fboxsep]{
  		\begin{minipage}[t]{\linewidth-2\fboxrule-4\fboxsep}\setlength{\parskip}{3mm}
  			 #2
  		\end{minipage}
	}}
	\vspace{5mm}
}

\newcommand\CBox[4]{				% Boites
	\vspace{5mm}
	\setlength{\fboxrule}{0.2mm}
	\setlength{\fboxsep}{5mm}
	
	\fcolorbox{#1}{#1!3}{\makebox[\linewidth-2\fboxrule-2\fboxsep]{
		\begin{minipage}[t]{1cm}\setlength{\parskip}{3mm}
	  		\textcolor{#1}{\LARGE{#2}}    
 	 	\end{minipage}  
  		\begin{minipage}[t]{\linewidth-2\fboxrule-4\fboxsep}\setlength{\parskip}{3mm}
			\raisebox{1.2mm}{\normalsize\sffamily{\textcolor{#1}{#3}}}						
  			 #4
  		\end{minipage}
	}}
	\vspace{5mm}
}

\newcommand\cadre[3]{				% Boites convertible html
	\par
	\vspace{2mm}
	\setlength{\fboxrule}{0.1mm}
	\setlength{\fboxsep}{5mm}
	\fcolorbox{#1}{white}{\makebox[\linewidth-2\fboxrule-2\fboxsep]{
  		\begin{minipage}[t]{\linewidth-2\fboxrule-4\fboxsep}\setlength{\parskip}{3mm}
			\raisebox{-2.5mm}{\sffamily \small{\textcolor{#1}{\MakeUppercase{#2}}}}		
			\par		
  			 #3
 	 		\end{minipage}
	}}
		\vspace{2mm}
	\par
}

\newcommand\bloc[3]{				% Boites convertible html sans bordure
     \needspace{2\baselineskip}
     {\sffamily \small{\textcolor{#1}{\MakeUppercase{#2}}}}    
		\par		
  			 #3
		\par
}

\newcommand\CHelp[1]{
     \CBox{Plum}{\faInfoCircle}{À RETENIR}{#1}
}

\newcommand\CUp[1]{
     \CBox{NavyBlue}{\faThumbsOUp}{EN PRATIQUE}{#1}
}

\newcommand\CInfo[1]{
     \CBox{Sepia}{\faArrowCircleRight}{REMARQUE}{#1}
}

\newcommand\CRedac[1]{
     \CBox{PineGreen}{\faEdit}{BIEN R\'EDIGER}{#1}
}

\newcommand\CError[1]{
     \CBox{Red}{\faExclamationTriangle}{ATTENTION}{#1}
}

\newcommand\TitreExo[2]{
\needspace{4\baselineskip}
 {\sffamily\large EXERCICE #1\ (\emph{#2 points})}
\vspace{5mm}
}

\newcommand\img[2]{
          \includegraphics[width=#2\paperwidth]{\imgdir#1}
}

\newcommand\imgsvg[2]{
       \begin{center}   \includegraphics[width=#2\paperwidth]{\imgsvgdir#1} \end{center}
}


\newcommand\Lien[2]{
     \href{#1}{#2 \tiny \faExternalLink}
}
\newcommand\mcLien[2]{
     \href{https~://www.maths-cours.fr/#1}{#2 \tiny \faExternalLink}
}

\newcommand{\euro}{\eurologo{}}

%================================================================================================================================
%
% Macros - Environement
%
%================================================================================================================================

\newenvironment{tex}{ %
}
{%
}

\newenvironment{indente}{ %
	\setlength\parindent{10mm}
}

{
	\setlength\parindent{0mm}
}

\newenvironment{corrige}{%
     \needspace{3\baselineskip}
     \medskip
     \textbf{\textsc{Corrigé}}
     \medskip
}
{
}

\newenvironment{extern}{%
     \begin{center}
     }
     {
     \end{center}
}

\NewEnviron{code}{%
	\par
     \boite{gray}{\texttt{%
     \BODY
     }}
     \par
}

\newenvironment{vbloc}{% boite sans cadre empeche saut de page
     \begin{minipage}[t]{\linewidth}
     }
     {
     \end{minipage}
}
\NewEnviron{h2}{%
    \needspace{3\baselineskip}
    \vspace{0.6cm}
	\noindent \MakeUppercase{\sffamily \large \BODY}
	\vspace{1mm}\textcolor{mcgris}{\hrule}\vspace{0.4cm}
	\par
}{}

\NewEnviron{h3}{%
    \needspace{3\baselineskip}
	\vspace{5mm}
	\textsc{\BODY}
	\par
}

\NewEnviron{margeneg}{ %
\begin{addmargin}[-1cm]{0cm}
\BODY
\end{addmargin}
}

\NewEnviron{html}{%
}

\begin{document}
\meta{url}{/cours/logarithme-neperien/}
\meta{pid}{523}
\meta{titre}{Fonction logarithme népérien en Terminale S}
\meta{type}{cours}
\begin{h2}1. Définition de la fonction logarithme népérien\end{h2}
\cadre{bleu}{Théorème et définition}{% id="t10"
     Pour tout réel $x > 0$, l'équation $e^{y}=x$, d'inconnue $y$, admet une \textbf{unique} solution.
     \par
     La fonction \textbf{logarithme népérien}, notée $\ln$, est la fonction définie sur $\left]0;+\infty \right[$ qui à $x > 0$, associe le réel $y$ solution de l'équation $e^{y}=x$.
}
\bloc{cyan}{Remarque}{% id="r10"
     Pour $x\leqslant 0$, par contre, l'équation $e^{y}=x$ n'a \textbf{pas de solution}
}
\cadre{vert}{Propriétés}{% id="p20"
     \begin{itemize}
          \item Pour tout réel $x > 0$ et tout $y \in  \mathbb{R}$ : $ e^{y}=x  \Leftrightarrow  y=\ln\left(x\right)$
          \item Pour tout réel $x > 0$ : $e^{\ln\left(x\right)}=x$
          \item  Pour tout réel $x$ : $\ln\left(e^{x}\right)=x$
     \end{itemize}
}
\bloc{cyan}{Remarques}{% id="r0"
     \begin{itemize}
          \item Ces propriétés se déduisent immédiatement de la définition
          \item On dit que les fonctions «logarithme népérien» et «exponentielle» sont \textit{réciproques}
          \item On en déduit immédiatement : $\ln\left(1\right)=0$ et $\ln\left(e\right)=1$
     \end{itemize}
}
\begin{h2}2. Etude de la fonction logarithme népérien\end{h2}
\cadre{rouge}{Théorème}{% id="t30"
     La fonction logarithme népérien est dérivable sur $\left]0 ;+\infty \right[$ et sa dérivée est définie par :
     \begin{center}$\ln^{\prime}\left(x\right)=\frac{1}{x}$\end{center}
}
\bloc{cyan}{Démonstration}{% id="r30"
     On dérive l'égalité $e^{\ln\left(x\right)}=x$ membre à membre.
     \par
     D'après le \mcLien{/cours/terminale-s/fonctions-continues\#t100}{théorème de dérivation des fonctions composées} on obtient :
     \par
     $\ln^{\prime}\left(x\right)\times e^{\ln\left(x\right)}=1$
     \par
     C'est à dire :
     \par
     $\ln^{\prime}\left(x\right)\times x=1$
     \par
     $\ln^{\prime}\left(x\right)=\frac{1}{x}$
}
\cadre{vert}{Propriété}{% id="p40"
     La fonction logarithme népérien est \textbf{strictement croissante} sur $\left]0;+\infty \right[$.
}
\bloc{cyan}{Démonstration}{% id="r30"
     Sa dérivée $\ln^{\prime}\left(x\right)=\frac{1}{x}$ est strictement positive sur $\left]0;+\infty \right[$
}
\cadre{vert}{Propriété}{% id="p45"
     Soit $u$ une fonction dérivable et \textbf{strictement positive} sur un intervalle $I$.
     \par
     Alors la fonction $ f :  x\mapsto  \ln\left(u\left(x\right)\right)$ est dérivable sur $I$ et :
     \begin{center}$f^{\prime}=\frac{u^{\prime}}{u}$\end{center}
}
\bloc{cyan}{Démonstration}{% id="r45"
     On utilise le \mcLien{/cours/terminale-s/fonctions-continues\#t100}{théorème de dérivation de fonctions composées} .
}
\bloc{orange}{Exemple}{% id="e45"
     Soit $f$ définie sur $\mathbb{R}$ par $f\left(x\right)=\ln\left(x^{2}+1\right)$
     \par
     $f$ est dérivable sur $\mathbb{R}$ et $f^{\prime}\left(x\right)=\frac{2x}{x^{2}+1}$
}
\cadre{vert}{Limites}{% id="t50"
     \begin{itemize}
          \item $\lim\limits_{x\rightarrow 0}\ln\left(x\right)=-\infty $
          \item $\lim\limits_{x\rightarrow +\infty }\ln\left(x\right)=+\infty $
     \end{itemize}
}
\bloc{cyan}{Remarques}{% id="r50"
     \begin{itemize}
          \item Ces résultats permettent de tracer le tableau de variation et la courbe représentative de la fonction  logarithme népérien :
     \end{itemize}
}
\begin{center}
     \begin{extern}%width="600" alt="Fonction logarithme népérien : tableau de variation"
          \tikzset{double style/.style = {double,double distance=2pt}}
          \begin{tikzpicture}
               \tkzTabInit[lgt=3,espcl=10] {$x$ /1, $f'(x)=\dfrac{1}{x}$ /1.5,%
               $f(x)=\ln x$/2} {$0$ , $+\infty$}%
               \tkzTabLine{d,+,}%
               \tkzTabVar{ D- / $-\infty$, + / $+\infty$ }
               \tkzTabVal[draw]{1}{2}{0.33}{1}{0}
               \tkzTabVal[draw]{1}{2}{0.66}{e}{1}
          \end{tikzpicture}
     \end{extern}
\end{center}
\begin{center}
     \textit{Tableau de variation de la fonction logarithme népérien }
\end{center}
\par
\begin{center}
     \begin{extern} %width="400" alt="Fonction logarithme népérien : graphique"
          \resizebox{8cm}{!}{%
               % -+-+-+ variables modifiables
               \def\fonction{x ln }
               \def\xmin{-0.5}
               \def\xmax{5.5}
               \def\ymin{-3.5}
               \def\ymax{3}
               \def\xunit{2}  % unités en cm
               \def\yunit{2}
               \psset{xunit=\xunit,yunit=\yunit}
               \fontsize{15pt}{15pt}\selectfont
               \begin{pspicture*}[linewidth=1pt](\xmin,\ymin)(\xmax,\ymax)
                    %      \psgrid[gridcolor=mcgris, subgriddiv=5, gridlabels=0pt](\xmin,\ymin)(\xmax,\ymax)
                    \psaxes[linewidth=0.75pt]{->}(0,0)(\xmin,\ymin)(\xmax,\ymax)
                    \psplot[plotpoints=2000,linecolor=blue]{0.01}{\xmax}{\fonction}
                    \rput[tr](-0.1,-0.1){$O$}
                    \rput[tl](4.8,2){$\color{blue} \mathcal{C}_{\ln}$}
                    \psline(0,1)(2.7183,1)(2.7183,0)
                    \rput[t](2.7183,-0.25){$\text{e}$}
               \end{pspicture*}
          }
     \end{extern}
\end{center}
\begin{center}
     \textit{Graphique de la fonction logarithme népérien }
\end{center}
\cadre{rouge}{Théorème ( «Croissance comparée»)}{% id="t60"
     \begin{itemize}
          \item $\lim\limits_{x\rightarrow 0}x \ln x=0$
          \item $\lim\limits_{x\rightarrow +\infty }\frac{\ln x}{x}=0$
          \item $\lim\limits_{x\rightarrow 0}\frac{\ln\left(1+x\right)}{x}=1$
     \end{itemize}
}
\bloc{cyan}{Remarque}{% id="r0"
     Comme dans le cas de la fonction exponentielle, on peut généraliser les deux premières formules :
     \par
     Pour tout entier $n > 1$:
     \begin{itemize}
          \item $\lim\limits_{x\rightarrow 0}x^{n} \ln x=0$
          \item $\lim\limits_{x\rightarrow +\infty }\frac{\ln x}{x^{n}}=0$
     \end{itemize}
}
\cadre{rouge}{Théorème}{% id="t70"
     Si $a$ et $b$ sont 2 réels strictement positifs :
     \begin{itemize}
          \item $\ln a=\ln b$ si et seulement si $a=b$
          \item $\ln a < \ln b$ si et seulement si $a < b$
     \end{itemize}
}
\bloc{cyan}{Remarques}{% id="r70"
     \begin{itemize}
          \item Le théorème précédent résulte de la stricte croissance de la fonction logarithme népérien.
          \item En particulier, comme $\ln\left(1\right)=0$ : $\ln x < 0  \Leftrightarrow  x < 1$. N'oubliez donc pas que \textbf{$\ln\left(x\right)$ peut être négatif} (si $0 < x < 1$); c'est une cause d'erreurs fréquente dans les exercices notamment avec des inéquations !
     \end{itemize}
}
\begin{h2}3. Propriétés algébriques de la fonction logarithme népérien\end{h2}
\cadre{rouge}{Théorème}{% id="80"
     Si $a$ et $b$ sont 2 réels strictement positifs et si $n \in  \mathbb{Z}$ :
     \begin{itemize}
          \item $\ln\left(ab\right)=\ln a+\ln b$
          \item $\ln\left(\frac{1}{a}\right)=-\ln a$
          \item $\ln\left(\frac{a}{b}\right)=\ln a-\ln b$
          \item $\ln\left(a^{n}\right)=n \ln a $
          \item $\ln\left(\sqrt{a}\right)=\frac{1}{2} \ln a $
     \end{itemize}
}
\bloc{orange}{Exemples}{% id="e80"
     \begin{itemize}
          \item $\ln\left(4\right)=\ln\left(2^{2}\right)=2\ln\left(2\right)$
          \item Pour $x > 1$ : $\ln\left(\frac{x+1}{x-1}\right)= \ln\left(x+1\right)-\ln\left(x-1\right)$
          \par
          Cette égalité peut être intéressante (pour calculer la dérivée par exemple) mais il faut que $x > 1$.
          \par
          Si $x < -1$, l'expression $\ln\left(\frac{x+1}{x-1}\right)$ est définie mais pas $\ln\left(x+1\right)-\ln\left(x-1\right)$.
     \end{itemize}
}

\end{document}
µ
\documentclass[a4paper]{article}

%================================================================================================================================
%
% Packages
%
%================================================================================================================================

\usepackage[T1]{fontenc} 	% pour caractères accentués
\usepackage[utf8]{inputenc}  % encodage utf8
\usepackage[french]{babel}	% langue : français
\usepackage{fourier}			% caractères plus lisibles
\usepackage[dvipsnames]{xcolor} % couleurs
\usepackage{fancyhdr}		% réglage header footer
\usepackage{needspace}		% empêcher sauts de page mal placés
\usepackage{graphicx}		% pour inclure des graphiques
\usepackage{enumitem,cprotect}		% personnalise les listes d'items (nécessaire pour ol, al ...)
\usepackage{hyperref}		% Liens hypertexte
\usepackage{pstricks,pst-all,pst-node,pstricks-add,pst-math,pst-plot,pst-tree,pst-eucl} % pstricks
\usepackage[a4paper,includeheadfoot,top=2cm,left=3cm, bottom=2cm,right=3cm]{geometry} % marges etc.
\usepackage{comment}			% commentaires multilignes
\usepackage{amsmath,environ} % maths (matrices, etc.)
\usepackage{amssymb,makeidx}
\usepackage{bm}				% bold maths
\usepackage{tabularx}		% tableaux
\usepackage{colortbl}		% tableaux en couleur
\usepackage{fontawesome}		% Fontawesome
\usepackage{environ}			% environment with command
\usepackage{fp}				% calculs pour ps-tricks
\usepackage{multido}			% pour ps tricks
\usepackage[np]{numprint}	% formattage nombre
\usepackage{tikz,tkz-tab} 			% package principal TikZ
\usepackage{pgfplots}   % axes
\usepackage{mathrsfs}    % cursives
\usepackage{calc}			% calcul taille boites
\usepackage[scaled=0.875]{helvet} % font sans serif
\usepackage{svg} % svg
\usepackage{scrextend} % local margin
\usepackage{scratch} %scratch
\usepackage{multicol} % colonnes
%\usepackage{infix-RPN,pst-func} % formule en notation polanaise inversée
\usepackage{listings}

%================================================================================================================================
%
% Réglages de base
%
%================================================================================================================================

\lstset{
language=Python,   % R code
literate=
{á}{{\'a}}1
{à}{{\`a}}1
{ã}{{\~a}}1
{é}{{\'e}}1
{è}{{\`e}}1
{ê}{{\^e}}1
{í}{{\'i}}1
{ó}{{\'o}}1
{õ}{{\~o}}1
{ú}{{\'u}}1
{ü}{{\"u}}1
{ç}{{\c{c}}}1
{~}{{ }}1
}


\definecolor{codegreen}{rgb}{0,0.6,0}
\definecolor{codegray}{rgb}{0.5,0.5,0.5}
\definecolor{codepurple}{rgb}{0.58,0,0.82}
\definecolor{backcolour}{rgb}{0.95,0.95,0.92}

\lstdefinestyle{mystyle}{
    backgroundcolor=\color{backcolour},   
    commentstyle=\color{codegreen},
    keywordstyle=\color{magenta},
    numberstyle=\tiny\color{codegray},
    stringstyle=\color{codepurple},
    basicstyle=\ttfamily\footnotesize,
    breakatwhitespace=false,         
    breaklines=true,                 
    captionpos=b,                    
    keepspaces=true,                 
    numbers=left,                    
xleftmargin=2em,
framexleftmargin=2em,            
    showspaces=false,                
    showstringspaces=false,
    showtabs=false,                  
    tabsize=2,
    upquote=true
}

\lstset{style=mystyle}


\lstset{style=mystyle}
\newcommand{\imgdir}{C:/laragon/www/newmc/assets/imgsvg/}
\newcommand{\imgsvgdir}{C:/laragon/www/newmc/assets/imgsvg/}

\definecolor{mcgris}{RGB}{220, 220, 220}% ancien~; pour compatibilité
\definecolor{mcbleu}{RGB}{52, 152, 219}
\definecolor{mcvert}{RGB}{125, 194, 70}
\definecolor{mcmauve}{RGB}{154, 0, 215}
\definecolor{mcorange}{RGB}{255, 96, 0}
\definecolor{mcturquoise}{RGB}{0, 153, 153}
\definecolor{mcrouge}{RGB}{255, 0, 0}
\definecolor{mclightvert}{RGB}{205, 234, 190}

\definecolor{gris}{RGB}{220, 220, 220}
\definecolor{bleu}{RGB}{52, 152, 219}
\definecolor{vert}{RGB}{125, 194, 70}
\definecolor{mauve}{RGB}{154, 0, 215}
\definecolor{orange}{RGB}{255, 96, 0}
\definecolor{turquoise}{RGB}{0, 153, 153}
\definecolor{rouge}{RGB}{255, 0, 0}
\definecolor{lightvert}{RGB}{205, 234, 190}
\setitemize[0]{label=\color{lightvert}  $\bullet$}

\pagestyle{fancy}
\renewcommand{\headrulewidth}{0.2pt}
\fancyhead[L]{maths-cours.fr}
\fancyhead[R]{\thepage}
\renewcommand{\footrulewidth}{0.2pt}
\fancyfoot[C]{}

\newcolumntype{C}{>{\centering\arraybackslash}X}
\newcolumntype{s}{>{\hsize=.35\hsize\arraybackslash}X}

\setlength{\parindent}{0pt}		 
\setlength{\parskip}{3mm}
\setlength{\headheight}{1cm}

\def\ebook{ebook}
\def\book{book}
\def\web{web}
\def\type{web}

\newcommand{\vect}[1]{\overrightarrow{\,\mathstrut#1\,}}

\def\Oij{$\left(\text{O}~;~\vect{\imath},~\vect{\jmath}\right)$}
\def\Oijk{$\left(\text{O}~;~\vect{\imath},~\vect{\jmath},~\vect{k}\right)$}
\def\Ouv{$\left(\text{O}~;~\vect{u},~\vect{v}\right)$}

\hypersetup{breaklinks=true, colorlinks = true, linkcolor = OliveGreen, urlcolor = OliveGreen, citecolor = OliveGreen, pdfauthor={Didier BONNEL - https://www.maths-cours.fr} } % supprime les bordures autour des liens

\renewcommand{\arg}[0]{\text{arg}}

\everymath{\displaystyle}

%================================================================================================================================
%
% Macros - Commandes
%
%================================================================================================================================

\newcommand\meta[2]{    			% Utilisé pour créer le post HTML.
	\def\titre{titre}
	\def\url{url}
	\def\arg{#1}
	\ifx\titre\arg
		\newcommand\maintitle{#2}
		\fancyhead[L]{#2}
		{\Large\sffamily \MakeUppercase{#2}}
		\vspace{1mm}\textcolor{mcvert}{\hrule}
	\fi 
	\ifx\url\arg
		\fancyfoot[L]{\href{https://www.maths-cours.fr#2}{\black \footnotesize{https://www.maths-cours.fr#2}}}
	\fi 
}


\newcommand\TitreC[1]{    		% Titre centré
     \needspace{3\baselineskip}
     \begin{center}\textbf{#1}\end{center}
}

\newcommand\newpar{    		% paragraphe
     \par
}

\newcommand\nosp {    		% commande vide (pas d'espace)
}
\newcommand{\id}[1]{} %ignore

\newcommand\boite[2]{				% Boite simple sans titre
	\vspace{5mm}
	\setlength{\fboxrule}{0.2mm}
	\setlength{\fboxsep}{5mm}	
	\fcolorbox{#1}{#1!3}{\makebox[\linewidth-2\fboxrule-2\fboxsep]{
  		\begin{minipage}[t]{\linewidth-2\fboxrule-4\fboxsep}\setlength{\parskip}{3mm}
  			 #2
  		\end{minipage}
	}}
	\vspace{5mm}
}

\newcommand\CBox[4]{				% Boites
	\vspace{5mm}
	\setlength{\fboxrule}{0.2mm}
	\setlength{\fboxsep}{5mm}
	
	\fcolorbox{#1}{#1!3}{\makebox[\linewidth-2\fboxrule-2\fboxsep]{
		\begin{minipage}[t]{1cm}\setlength{\parskip}{3mm}
	  		\textcolor{#1}{\LARGE{#2}}    
 	 	\end{minipage}  
  		\begin{minipage}[t]{\linewidth-2\fboxrule-4\fboxsep}\setlength{\parskip}{3mm}
			\raisebox{1.2mm}{\normalsize\sffamily{\textcolor{#1}{#3}}}						
  			 #4
  		\end{minipage}
	}}
	\vspace{5mm}
}

\newcommand\cadre[3]{				% Boites convertible html
	\par
	\vspace{2mm}
	\setlength{\fboxrule}{0.1mm}
	\setlength{\fboxsep}{5mm}
	\fcolorbox{#1}{white}{\makebox[\linewidth-2\fboxrule-2\fboxsep]{
  		\begin{minipage}[t]{\linewidth-2\fboxrule-4\fboxsep}\setlength{\parskip}{3mm}
			\raisebox{-2.5mm}{\sffamily \small{\textcolor{#1}{\MakeUppercase{#2}}}}		
			\par		
  			 #3
 	 		\end{minipage}
	}}
		\vspace{2mm}
	\par
}

\newcommand\bloc[3]{				% Boites convertible html sans bordure
     \needspace{2\baselineskip}
     {\sffamily \small{\textcolor{#1}{\MakeUppercase{#2}}}}    
		\par		
  			 #3
		\par
}

\newcommand\CHelp[1]{
     \CBox{Plum}{\faInfoCircle}{À RETENIR}{#1}
}

\newcommand\CUp[1]{
     \CBox{NavyBlue}{\faThumbsOUp}{EN PRATIQUE}{#1}
}

\newcommand\CInfo[1]{
     \CBox{Sepia}{\faArrowCircleRight}{REMARQUE}{#1}
}

\newcommand\CRedac[1]{
     \CBox{PineGreen}{\faEdit}{BIEN R\'EDIGER}{#1}
}

\newcommand\CError[1]{
     \CBox{Red}{\faExclamationTriangle}{ATTENTION}{#1}
}

\newcommand\TitreExo[2]{
\needspace{4\baselineskip}
 {\sffamily\large EXERCICE #1\ (\emph{#2 points})}
\vspace{5mm}
}

\newcommand\img[2]{
          \includegraphics[width=#2\paperwidth]{\imgdir#1}
}

\newcommand\imgsvg[2]{
       \begin{center}   \includegraphics[width=#2\paperwidth]{\imgsvgdir#1} \end{center}
}


\newcommand\Lien[2]{
     \href{#1}{#2 \tiny \faExternalLink}
}
\newcommand\mcLien[2]{
     \href{https~://www.maths-cours.fr/#1}{#2 \tiny \faExternalLink}
}

\newcommand{\euro}{\eurologo{}}

%================================================================================================================================
%
% Macros - Environement
%
%================================================================================================================================

\newenvironment{tex}{ %
}
{%
}

\newenvironment{indente}{ %
	\setlength\parindent{10mm}
}

{
	\setlength\parindent{0mm}
}

\newenvironment{corrige}{%
     \needspace{3\baselineskip}
     \medskip
     \textbf{\textsc{Corrigé}}
     \medskip
}
{
}

\newenvironment{extern}{%
     \begin{center}
     }
     {
     \end{center}
}

\NewEnviron{code}{%
	\par
     \boite{gray}{\texttt{%
     \BODY
     }}
     \par
}

\newenvironment{vbloc}{% boite sans cadre empeche saut de page
     \begin{minipage}[t]{\linewidth}
     }
     {
     \end{minipage}
}
\NewEnviron{h2}{%
    \needspace{3\baselineskip}
    \vspace{0.6cm}
	\noindent \MakeUppercase{\sffamily \large \BODY}
	\vspace{1mm}\textcolor{mcgris}{\hrule}\vspace{0.4cm}
	\par
}{}

\NewEnviron{h3}{%
    \needspace{3\baselineskip}
	\vspace{5mm}
	\textsc{\BODY}
	\par
}

\NewEnviron{margeneg}{ %
\begin{addmargin}[-1cm]{0cm}
\BODY
\end{addmargin}
}

\NewEnviron{html}{%
}

\begin{document}
\meta{url}{/cours/fonctions-trigonometriques/}
\meta{pid}{525}
\meta{titre}{Fonctions trigonométriques}
\meta{type}{cours}
\begin{h2}1. Rappels\end{h2}
Dans toute la suite, le plan est muni d'un repère orthonormé $\left(O ; \overrightarrow{OI} ,\overrightarrow{OJ}\right)$.\\
On oriente le \textbf{cercle trigonométrique} (cercle de centre $O$ et de rayon 1) dans le \textbf{sens direct} (sens inverse des aiguilles d'une montre).
\begin{center}
     \begin{extern}%width="480"
          \newrgbcolor{dblue}{0. 0. 0.7}
          \newrgbcolor{dvert}{0. 0.4 0.}
          \newrgbcolor{dmauve}{0.5 0. 0.5}
          \psset{xunit=5.0cm,yunit=5.0cm,algebraic=true,dimen=middle,dotstyle=o,dotsize=5pt 0,linewidth=0.8pt,arrowsize=3pt 2,arrowinset=0.25}
          \begin{pspicture*}(-1.2,-1.2)(1.2,1.2)
               \psaxes[linewidth=0.75pt,labelFontSize=\scriptstyle,xAxis=true,yAxis=true,Dx=10.,Dy=10.,ticksize=-2pt 0,subticks=1]{->}(0,0)(-1.2,-1.2)(1.2,1.2)
               \pscircle[linewidth=0.8pt](0.,0.){5.} %cercle trigo
               \parametricplot[linewidth=1.2pt,linecolor=red]{0.0}{0.698}{cos(t)|sin(t)}%arc angle
               \parametricplot[linewidth=0.8pt,arrows=->]{0.8}{1.3}{1.15*cos(t)|1.15*sin(t)}% sens trigo
               \rput[tl](0.58,1.07){+}
               \pscustom[linewidth=0.8pt,linecolor=dmauve,fillcolor=dmauve,fillstyle=solid,opacity=0.1]{ % color angle
                    \parametricplot{0.0}{0.698}{0.15*cos(t)|0.15*sin(t)}
               \lineto(0.,0.)\closepath}
               \psellipticarc[linewidth=0.8pt,linecolor=dmauve,arrows=->](0.,0.)(0.15,0.15){0.}{40} % fleche angle
               \psline[linewidth=0.8pt,linecolor=dmauve](0.,0.)(0.766,0.643)%rayon
               \psline[linewidth=0.8pt]{->}(0.,0.)(1.,0.) %vecteurs unités
               \psline[linewidth=0.8pt]{->}(0.,0.)(0,1)
               %\rput[tl](0.4,0.1){$\vec{i}$}
               %\rput[tl](-0.06,0.5){$\vec{j}$}
               \psdots[dotsize=2pt 0,dotstyle=*](0.,0.)
               \rput[bl](-0.09,-0.09){$O$}
               \psdots[dotsize=2pt 0,dotstyle=*,linecolor=dblue](1.,0.)
               \rput[bl](1.02,0.02){\dblue{$I$}}
               \psdots[dotsize=2pt 0,dotstyle=*,linecolor=dblue](0.766,0.643)
               \rput[bl](0.78,0.66){\dblue{$N$}}
               \psdots[dotsize=2pt 0,dotstyle=*,linecolor=dblue](0,1)
               \rput[bl](0.02,1.03){\dblue{$J$}}
               \rput[bl](0.19,0.05){\dmauve{$x$}}
               \psdots[dotsize=2pt 0,dotstyle=*,linecolor=dvert](0.766,0)
               \psdots[dotsize=2pt 0,dotstyle=*,linecolor=dvert](0,0.643)
               \psline[linewidth=1pt,linecolor=dvert](0.,0.)(0.766,0)
               \psline[linewidth=1pt,linecolor=dvert](0.,0.)(0,0.643)
               \psline[linewidth=0.4pt,linecolor=dvert](0.,0.643)(0.766,0.643)
               \psline[linewidth=0.4pt,linecolor=dvert](0.766,0.)(0.766,0.643)
               \rput(0.766,-0.05){\dvert{$\cos x$}}
               \rput(-0.10,0.643){\dvert{$\sin x$}}
          \end{pspicture*}
     \end{extern}
     \end{center}\cadre{bleu}{Définition}{%id="d10"
     Soit $N$ un point du cercle trigonométrique et $x$ une mesure en radians de l'angle $\left(\overrightarrow{OI},\overrightarrow{ON}\right)$.
     \par
     On appelle \textbf{cosinus} de $x$, noté\textbf{ $\cos x$} l'abscisse du point $N$.
     \par
     On appelle \textbf{sinus} de $x$, noté\textbf{ $\sin x$} l'ordonnée du point $N$.
}
\bloc{cyan}{Remarque}{%id="r10"
     Pour tout réel $x$ :
     \begin{itemize}
          \item $-1 \leqslant \cos x \leqslant 1$
          \item $-1 \leqslant \sin x \leqslant 1$
          \item $\left(\cos x\right)^{2} + \left(\sin x\right)^{2} = 1$ (d'après le théorème de Pythagore).
     \end{itemize}
}
\cadre{vert}{Quelques valeurs de sinus et de cosinus}{%id="p20"
     \begin{tabularx}{0.8\linewidth}{|*{7}{>{\centering \arraybackslash }X|}}%class="compact mw500"
          \hline
          \textbf{$x$} & $0$ & $\frac{\pi }{6}$ & $\frac{\pi }{4}$ & $\frac{\pi }{3}$ & $\frac{\pi }{2}$ & $\pi $\\ \hline
          \textbf{$\cos x$} & $1$ & $\frac{\sqrt{3}}{2}$ & $\frac{\sqrt{2}}{2}$ & $\frac{1}{2}$ & $0$ & $-1$\\ \hline
          \textbf{$\sin x$} & $0$ & $\frac{1}{2}$ & $\frac{\sqrt{2}}{2}$ & $\frac{\sqrt{3}}{2}$ & $1$ & $0$\\ \hline
     \end{tabularx}
}
\begin{center}
     \begin{extern}%width="600"
          \newrgbcolor{dblue}{0. 0. 0.7}
          \newrgbcolor{dvert}{0. 0.4 0.}
          \newrgbcolor{dmauve}{0.5 0. 0.5}
          \psset{xunit=5.0cm,yunit=5.0cm,algebraic=true,dimen=middle,dotstyle=o,dotsize=5pt 0,linewidth=0.8pt,arrowsize=3pt 2,arrowinset=0.25}
          \begin{pspicture*}(-1.2,-1.2)(1.2,1.2)
               \psaxes[linewidth=0.75pt,labelFontSize=\scriptstyle,xAxis=true,yAxis=true,Dx=10.,Dy=10.,ticksize=-2pt 0,subticks=1]{->}(0,0)(-1.2,-1.2)(1.2,1.2)
               \pscircle[linewidth=0.8pt](0.,0.){5.} %cercle trigo
               \psline[linewidth=0.8pt]{->}(0.,0.)(1.,0.) %vecteurs unités
               \psline[linewidth=0.8pt]{->}(0.,0.)(0,1)
               %\rput[tl](0.4,0.1){$\vec{i}$}
               %\rput[tl](-0.06,0.5){$\vec{j}$}
               \psdots[dotsize=2pt 0,dotstyle=*](0.,0.)
               %\rput[bl](-0.09,-0.09){$O$}
               \psframe[linewidth=0.4pt,linecolor=dvert](-0.707,-0.707)(0.707,0.707)
               \psline[linewidth=0.8pt,linecolor=dvert](-0.707,-0.707)(0.707,0.707)
               \psline[linewidth=0.8pt,linecolor=dvert](-0.707,0.707)(0.707,-0.707)
               \psframe[linewidth=0.4pt,linecolor=red](-0.866,-0.5)(0.866,0.5)
               \psline[linewidth=0.8pt,linecolor=red](-0.866,-0.5)(0.866,0.5)
               \psline[linewidth=0.8pt,linecolor=red](0.866,-0.5)(-0.866,0.5)
               \rput(0.943,0.55){$\red{\dfrac{\pi}{6}}$}
               \rput(-0.943,0.55){$\red{\dfrac{5\pi}{6}}$}
               \rput(-0.973,-0.55){$\red{-\dfrac{5\pi}{6}}$}
               \rput(0.943,-0.55){$\red{-\dfrac{\pi}{6}}$}
               \rput(0.05,-0.554){\fontsize{7 pt}{7 pt}\selectfont $\red{ -\dfrac{1}{2}}$}
               \rput(0.05,0.554){\fontsize{7 pt}{7 pt}\selectfont $\red{ \dfrac{1}{2}}$}
               \rput(0.93,0.07){\fontsize{7 pt}{7 pt}\selectfont $\red{ \dfrac{\sqrt{3}}{2}}$}
               \rput(-0.93,0.07){\fontsize{7 pt}{7 pt}\selectfont $\red{-\dfrac{\sqrt{3}}{2}}$}
               %
               \rput(0.777,0.777){$\dvert{\dfrac{\pi}{4}}$}
               \rput(-0.777,0.777){$\dvert{\dfrac{3\pi}{4}}$}
               \rput(-0.807,-0.777){$\dvert{-\dfrac{3\pi}{4}}$}
               \rput(0.777,-0.777){$\dvert{-\dfrac{\pi}{4}}$}
               \rput(0.06,-0.77){\fontsize{7 pt}{7 pt}\selectfont $\dvert{ -\dfrac{\sqrt{2}}{2}}$}
               \rput(0.06,0.77){\fontsize{7 pt}{7 pt}\selectfont $\dvert{ \dfrac{\sqrt{2}}{2}}$}
               \rput(0.764,0.07){\fontsize{7 pt}{7 pt}\selectfont $\dvert{ \dfrac{\sqrt{2}}{2}}$}
               \rput(-0.764,0.07){\fontsize{7 pt}{7 pt}\selectfont $\dvert{-\dfrac{\sqrt{2}}{2}}$}
               %
               \psframe[linewidth=0.4pt,linecolor=dblue](-0.5,-0.866)(0.5,0.866)
               \psline[linewidth=0.8pt,linecolor=dblue](-0.5,-0.866)(0.5,0.866)
               \psline[linewidth=0.8pt,linecolor=dblue](-0.5,0.866)(0.5,-0.866)
               \rput(0.55,0.943){$\dblue{\dfrac{\pi}{3}}$}
               \rput(-0.55,0.943){$\dblue{\dfrac{2\pi}{3}}$}
               \rput(-0.58,-0.943){$\dblue{-\dfrac{2\pi}{3}}$}
               \rput(0.55,-0.943){$\dblue{-\dfrac{\pi}{3}}$}
               \rput(0.06,-0.933){\fontsize{7 pt}{7 pt}\selectfont $\dblue{ -\dfrac{\sqrt{3}}{2}}$}
               \rput(0.06,0.933){\fontsize{7 pt}{7 pt}\selectfont $\dblue{ \dfrac{\sqrt{3}}{2}}$}
               \rput(0.538,0.07){\fontsize{7 pt}{7 pt}\selectfont $\dblue{ \dfrac{1}{2}}$}
               \rput(-0.538,0.07){\fontsize{7 pt}{7 pt}\selectfont $\dblue{-\dfrac{1}{2}}$}
               %
               \rput(1.06,0.06){$0$}
               \rput(0.06,1.1){$\dfrac{\pi}{2}$}
               \rput(0.06,-1.1){$-\dfrac{\pi}{2}$}
               \rput(-1.06,0.06){$\pi$}
          \end{pspicture*}
     \end{extern}
\end{center}
\cadre{rouge}{Théorème}{%id="t30"
     Soit $a$ un réel fixé.
     \par
     Les solutions de l'équation $\cos\left(x\right)=\cos\left(a\right)$ sont les réels de la forme :
     \begin{center}$a+2k\pi $ ou $ -a+2k\pi $ où $k$ décrit $\mathbb{Z}$\end{center}
     Les solutions de l'équation $\sin\left(x\right)=\sin\left(a\right)$ sont les réels de la forme :
     \begin{center}$a+2k\pi $ ou $ \pi -a+2k\pi $ où $k$ décrit $\mathbb{Z}$\end{center}
}
\bloc{orange}{Exemple}{%id="e30"
     Soit l'équation $\sin\left(x\right)=\frac{1}{2}$.
     \par
     Comme $\sin\frac{\pi }{6}=\frac{1}{2}$, l'équation peut s'écrire $\sin\left(x\right)=\sin\frac{\pi }{6}$.
     \par
     D'après le théorème précédent, l'ensemble des solutions est :
     \par
     $S=\left\{ \frac{\pi }{6}+2k\pi , \frac{5\pi }{6}+2k\pi | k\in \mathbb{Z} \right\}$.
}
\begin{h2}2. Fonctions sinus et cosinus\end{h2}
\cadre{bleu}{Définition}{%id="d40"
     La fonction, définie sur $\mathbb{R}$, qui à tout réel $x$ associe son cosinus : $x\mapsto \cos\left(x\right)$ est appelée \textbf{fonction cosinus}.
     \par
     La fonction, définie sur $\mathbb{R}$, qui à tout réel $x$ associe son sinus : $x\mapsto \sin\left(x\right)$ est appelée \textbf{fonction sinus}.
}
\cadre{rouge}{Formules de base}{%id="t50"
     Pour tout réel $x$ :
     \begin{itemize}
          \item $\cos\left(x+2\pi \right)=\cos\left(x\right)$
          \item $\sin\left(x+2\pi \right)=\sin\left(x\right)$.
     \end{itemize}
     On dit que les fonctions sinus et cosinus sont \textbf{périodiques} de période $2\pi $.
     \par
     \begin{itemize}
          \item $\cos\left(-x\right)=\cos\left(x\right) $ (la fonction cosinus est paire)
          \item $\sin\left(-x\right)=-\sin\left(x\right) $ (la fonction sinus est impaire)
     \end{itemize}
     \par
     \begin{itemize}
          \item $\cos\left(x+\pi \right)=-\cos\left(x\right)$
          \item $\sin\left(x+\pi \right)=-\sin\left(x\right)$
     \end{itemize}
     \par
     \begin{itemize}
          \item $\cos\left(x+\frac{\pi }{2}\right)=-\sin\left(x\right) $
          \item $\sin\left(x+\frac{\pi }{2}\right)=\cos\left(x\right) $.
     \end{itemize}
}
\bloc{cyan}{Remarque}{%id="r50"
     A partir des formules de base on peut montrer d'autres formules; par exemple :
     \par
     $\cos\left(\frac{\pi }{2}-x\right)=\cos\left(-x+\frac{\pi }{2}\right)=-\sin\left(-x\right)=\sin\left(x\right)$
     \par
     $\sin\left(\frac{\pi }{2}-x\right)=\sin\left(-x+\frac{\pi }{2}\right)=\cos\left(-x\right)=\cos\left(x\right)$
     \par
     etc.
}
\cadre{rouge}{Formules d'addition}{%id="t60"
     Pour tous réels $a$ et $b$ :
     \begin{itemize}
          \item $\cos\left(a+b\right)=\cos\left(a\right) \cos\left(b\right)-\sin\left(a\right) \sin\left(b\right)$
          \item $\sin\left(a+b\right)=\sin\left(a\right) \cos\left(b\right)+\cos\left(a\right) \sin\left(b\right)$
     \end{itemize}
}
\bloc{cyan}{Remarque}{%id="r60"
     En remplaçant $b$ par $-b$ et en utilisant la parité des fonctions sinus et cosinus on obtient les formules de soustraction:
     \begin{itemize}
          \item $\cos\left(a-b\right)=\cos\left(a\right) \cos\left(b\right)+\sin\left(a\right) \sin\left(b\right)$
          \item $\sin\left(a-b\right)=\sin\left(a\right) \cos\left(b\right)-\cos\left(a\right) \sin\left(b\right)$
     \end{itemize}
}
\cadre{vert}{Propriété (formules de duplication)}{%id="p70"
     Pour tout réel $a$ :
     \begin{itemize}
          \item $\cos\left(2a\right) = \cos^{2}\left(a\right)-\sin^{2}\left(a\right) = 2\cos^{2}\left(a\right)-1 = 1-2\sin^{2}\left(a\right)$
          \item $\sin\left(2a\right) = 2\sin\left(a\right) \cos\left(a\right)$.
     \end{itemize}
}
\bloc{cyan}{Remarques}{%id="r70"
     \begin{itemize}
          \item On démontre ces formules en posant $b=a$ dans les formules d'addition et en utilisant $\sin^{2}\left(a\right)+\cos^{2}\left(a\right)=1$.
          \item Rappel : $\sin^{2}\left(a\right)$ et $\cos^{2}\left(a\right)$ sont des écritures simplifiées pour $\left(\sin\left(a\right)\right)^{2}$ et $\left(\cos\left(a\right)\right)^{2}$.
     \end{itemize}
}
\begin{h2}3. Etude des fonctions sinus et cosinus\end{h2}
\cadre{rouge}{Théorème}{%id="t80"
     Les fonctions sinus et cosinus sont \textbf{dérivables} sur $\mathbb{R}$ et leurs dérivées sont :
     \begin{center}$\sin^{\prime}=\cos$\end{center}
     \begin{center}$\cos^{\prime}=-\sin$\end{center}
}
\cadre{vert}{Propriétés}{%id="p90"
     Soient $a$ et $b$ deux réels quelconques. Les fonctions $f$ et $g$ définies sur $\mathbb{R}$ par :
     \begin{itemize}
          \item $ f : x\mapsto \sin\left(ax+b\right)$
          \item $ g : x\mapsto \cos\left(ax+b\right)$
     \end{itemize}
     sont dérivables sur $\mathbb{R}$ et :
     \begin{itemize}
          \item $ f^{\prime}\left(x\right)=a \cos\left(ax+b\right)$
          \item $ g^{\prime}\left(x\right)=-a \sin\left(ax+b\right)$
     \end{itemize}
     Plus généralement, si $u$ est une fonction dérivable sur un intervalle $I$ et $f$ et $g$ définies sur $I$ par :
     \begin{itemize}
          \item $ f : x\mapsto \sin\left(u\left(x\right)\right)$
          \item $ g : x\mapsto \cos\left(u\left(x\right)\right)$
     \end{itemize}
     alors $f$ et $g$ sont dérivables sur $I$ et :
     \begin{itemize}
          \item $ f^{\prime}\left(x\right)=u^{\prime}\left(x\right)\times \cos\left(u\left(x\right)\right)$
          \item $ g^{\prime}\left(x\right)=-u^{\prime}\left(x\right)\times \sin\left(u\left(x\right)\right)$
     \end{itemize}
}
\bloc{cyan}{Remarque}{%id="r90"
     C'est un cas particulier du \mcLien{/cours/terminale-s/fonctions-continues\#t100}{ théorème de dérivation de fonctions composées}.
}
\cadre{vert}{Limites}{%id="p95"
     Les fonctions sinus et cosinus \textbf{ne possèdent pas de limite quand $x\rightarrow \pm\infty $}
     Par contre on démontre le résultat suivant :
     \par
     $\lim\limits_{x\rightarrow 0}\frac{\sin\left(x\right)}{x}=1$
}
\bloc{cyan}{Remarque}{%id="r95"
     Cette dernière limite peut s'obtenir en utilisant la définition du nombre dérivé de la fonction sinus pour $x=0$ (voir fiche méthode \mcLien{/methodes/limites/calcul-limite-nombre-derive}{Calculer une limite à l'aide du nombre dérivé}).
}
Les fonctions sinus et cosinus étant périodiques, il suffit de les étudier sur un intervalle d'amplitude $2\pi $, par exemple $\left[-\pi ; \pi \right]$.
\par
Pour obtenir la courbe complète, on effectue ensuite des translations de vecteurs $\pm2\pi \vec{i}$.
\bigskip
\begin{center}
     \begin{h3}Fonction sinus\end{h3}
\end{center}
\begin{extern}
     \begin{center}
          \begin{tikzpicture}[scale=0.875]
               % Styles
               \tikzstyle{cadre}=[thin]
               \tikzstyle{fleche}=[->,>=latex,thin]
               \tikzstyle{nondefini}=[lightgray]
               % Dimensions Modifiables
               \def\Lrg{1.5}
               \def\HtX{1.2}
               \def\HtY{0.5}
               % Dimensions Calculées
               \def\lignex{-0.5*\HtX}
               \def\lignef{-1.5*\HtX}
               \def\separateur{-0.5*\Lrg}
               % Largeur du tableau
               \def\gauche{-3.5*\Lrg}
               \def\droite{6.5*\Lrg}
               % Hauteur du tableau
               \def\haut{0.5*\HtX}
               \def\bas{-2.5*\HtX-2*\HtY}
               % Ligne de l'abscisse : x
               \node at (-2*\Lrg,0) {$x$};
               \node at (0*\Lrg,0) {$-\pi$};
               \node at (2*\Lrg,0) {$-\dfrac{\pi}{2}$};
               \node at (4*\Lrg,0) {$\dfrac{\pi}{2}$};
               \node at (6*\Lrg,0) {$\pi$};
               % Ligne de la dérivée : f'(x)
               \node at (-2*\Lrg,-1*\HtX) {$f'(x)=\cos(x)$};
               \node at (0*\Lrg,-1*\HtX) {$ $};
               \node at (1*\Lrg,-1*\HtX) {$-$};
               \node at (2*\Lrg,-1*\HtX) {$0$};
               \node at (3*\Lrg,-1*\HtX) {$+$};
               \node at (4*\Lrg,-1*\HtX) {$0$};
               \node at (5*\Lrg,-1*\HtX) {$-$};
               \node at (6*\Lrg,-1*\HtX) {$ $};
               % Ligne de la fonction : f(x)
               \node  at (-2*\Lrg,{-2*\HtX+(-1)*\HtY}) {$f(x)=\sin(x)$};
               \node (f1) at (0*\Lrg,{-2*\HtX+(0)*\HtY}) {$0$};
               \node (f2) at (2*\Lrg,{-2*\HtX+(-2)*\HtY}) {$-1$};
               \node (f3) at (4*\Lrg,{-2*\HtX+(0)*\HtY}) {$1$};
               \node (f4) at (6*\Lrg,{-2*\HtX+(-2)*\HtY}) {$0$};
               % Flèches
               \draw[fleche] (f1) -- (f2);
               \draw[fleche] (f2) -- (f3);
               \draw[fleche] (f3) -- (f4);
               % Encadrement
               \draw[cadre] (\separateur,\haut) -- (\separateur,\bas);
               \draw[cadre] (\gauche,\haut) rectangle  (\droite,\bas);
               \draw[cadre] (\gauche,\lignex) -- (\droite,\lignex);
               \draw[cadre] (\gauche,\lignef) -- (\droite,\lignef);
          \end{tikzpicture}
     \end{center}
\end{extern}
\begin{center}\textit{Tableau de variation de la fonction sinus}\end{center}
\begin{center}
     \begin{extern} %width="450" alt="fonction sinus"
          \resizebox{8cm}{!}{%
               % -+-+-+ variables modifiables
               \def\fonction{SIN(x) }
               \def\xmin{-6.5}
               \def\xmax{6.5}
               \def\ymin{-1.5}
               \def\ymax{1.5}
               \def\xunit{1}  % unités en cm
               \def\yunit{1.5}
               \psset{xunit=\xunit,yunit=\yunit,algebraic=true}
               \fontsize{15pt}{15pt}\selectfont
               \begin{pspicture*}[linewidth=1pt](\xmin,\ymin)(\xmax,\ymax)
                    %      \psgrid[gridcolor=mcgris, subgriddiv=5, gridlabels=0pt](\xmin,\ymin)(\xmax,\ymax)
                    \psaxes[linewidth=0.75pt]{->}(0,0)(\xmin,\ymin)(\xmax,\ymax)
                    \psplot[linewidth=0.75pt,plotpoints=2000,linecolor=black]{\xmin}{\xmax}{\fonction}
                    \psplot[linewidth=1.2pt,plotpoints=2000,linecolor=red]{-3.14159}{3.14159}{\fonction}
                    \rput[tr](-0.1,-0.2){$O$}
                    \psline[linewidth=1.25pt]{->}(0,0)(0,1)
                    \psline[linewidth=1.25pt]{->}(0,0)(1,0)
                    \rput[t](0.5,-0.03){$\vect{i}$}
                    \rput[r](-0.03,0.5){$\vect{j}$}
               \end{pspicture*}
          }
     \end{extern}
\end{center}
\begin{center}\textit{Représentation graphique de la fonction sinus}\end{center}
\bigskip
\begin{center}
     \begin{h3}Fonction cosinus\end{h3}
\end{center}
\begin{center}
     \begin{extern}%width="420"
          \begin{tikzpicture}[scale=0.875]
               % Styles
               \tikzstyle{cadre}=[thin]
               \tikzstyle{fleche}=[->,>=latex,thin]
               \tikzstyle{nondefini}=[lightgray]
               % Dimensions Modifiables
               \def\Lrg{1.5}
               \def\HtX{1}
               \def\HtY{0.5}
               % Dimensions Calculées
               \def\lignex{-0.5*\HtX}
               \def\lignef{-1.5*\HtX}
               \def\separateur{-0.5*\Lrg}
               % Largeur du tableau
               \def\gauche{-3.5*\Lrg}
               \def\droite{4.5*\Lrg}
               % Hauteur du tableau
               \def\haut{0.5*\HtX}
               \def\bas{-2.5*\HtX-2*\HtY}
               % Ligne de l'abscisse : x
               \node at (-2*\Lrg,0) {$x$};
               \node at (0*\Lrg,0) {$-\pi$};
               \node at (2*\Lrg,0) {$0$};
               \node at (4*\Lrg,0) {$\pi$};
               % Ligne de la dérivée : f'(x)
               \node at (-2*\Lrg,-1*\HtX) {$f'(x)=-\sin x$};
               \node at (0*\Lrg,-1*\HtX) {$ $};
               \node at (1*\Lrg,-1*\HtX) {$+$};
               \node at (2*\Lrg,-1*\HtX) {$0$};
               \node at (3*\Lrg,-1*\HtX) {$-$};
               \node at (4*\Lrg,-1*\HtX) {$0$};
               % Ligne de la fonction : f(x)
               \node  at (-2*\Lrg,{-2*\HtX+(-1)*\HtY}) {$f(x)=\cos x$};
               \node (f1) at (0*\Lrg,{-2*\HtX+(-2)*\HtY}) {$-1$};
               \node (f2) at (2*\Lrg,{-2*\HtX+(0)*\HtY}) {$1$};
               \node (f3) at (4*\Lrg,{-2*\HtX+(-2)*\HtY}) {$-1$};
               % Flèches
               \draw[fleche] (f1) -- (f2);
               \draw[fleche] (f2) -- (f3);
               % Encadrement
               \draw[cadre] (\separateur,\haut) -- (\separateur,\bas);
               \draw[cadre] (\gauche,\haut) rectangle  (\droite,\bas);
               \draw[cadre] (\gauche,\lignex) -- (\droite,\lignex);
               \draw[cadre] (\gauche,\lignef) -- (\droite,\lignef);
          \end{tikzpicture}
     \end{extern}
\end{center}
\begin{center}\textit{Tableau de variation de la fonction cosinus}\end{center}
\begin{center}
     \begin{extern} %width="450" alt="fonction cosinus"
          \resizebox{8cm}{!}{%
               % -+-+-+ variables modifiables
               \def\fonction{COS(x) }
               \def\xmin{-6.5}
               \def\xmax{6.5}
               \def\ymin{-1.5}
               \def\ymax{1.5}
               \def\xunit{1}  % unités en cm
               \def\yunit{1.5}
               \psset{xunit=\xunit,yunit=\yunit,algebraic=true}
               \fontsize{15pt}{15pt}\selectfont
               \begin{pspicture*}[linewidth=1pt](\xmin,\ymin)(\xmax,\ymax)
                    %      \psgrid[gridcolor=mcgris, subgriddiv=5, gridlabels=0pt](\xmin,\ymin)(\xmax,\ymax)
                    \psaxes[linewidth=0.75pt]{->}(0,0)(\xmin,\ymin)(\xmax,\ymax)
                    \psplot[linewidth=0.75pt,plotpoints=2000,linecolor=black]{\xmin}{\xmax}{\fonction}
                    \psplot[linewidth=1.2pt,plotpoints=2000,linecolor=blue]{-3.14159}{3.14159}{\fonction}
                    \rput[tr](-0.1,-0.2){$O$}
                    \psline[linewidth=1.25pt]{->}(0,0)(0,1)
                    \psline[linewidth=1.25pt]{->}(0,0)(1,0)
                    \rput[t](0.5,-0.03){$\vect{i}$}
                    \rput[r](-0.03,0.5){$\vect{j}$}
               \end{pspicture*}
          }
     \end{extern}
\end{center}\begin{center}\textit{Représentation graphique de la fonction cosinus}\end{center}
\bloc{cyan}{Remarque}{%id="r100"
     La relation $\sin\left(x+\frac{\pi }{2}\right)=\cos\left(x\right) $ montre que la courbe de la fonction sinus se déduit de la courbe de la fonction cosinus par une translation de vecteur $\frac{\pi }{2}\vec{i}$.
     \begin{center}
          \begin{extern} %width="450" alt="fonction cosinus"
               \resizebox{8cm}{!}{%
                    % -+-+-+ variables modifiables
                    \def\fonction{COS(x) }
                    \def\g{SIN(x) }
                    \def\xmin{-6.5}
                    \def\xmax{6.5}
                    \def\ymin{-1.5}
                    \def\ymax{1.5}
                    \def\xunit{1}  % unités en cm
                    \def\yunit{1.5}
                    \psset{xunit=\xunit,yunit=\yunit,algebraic=true}
                    \fontsize{15pt}{15pt}\selectfont
                    \begin{pspicture*}[linewidth=1pt](\xmin,\ymin)(\xmax,\ymax)
                         %      \psgrid[gridcolor=mcgris, subgriddiv=5, gridlabels=0pt](\xmin,\ymin)(\xmax,\ymax)
                         \psaxes[linewidth=0.75pt]{->}(0,0)(\xmin,\ymin)(\xmax,\ymax)
                         \psplot[linewidth=0.75pt,plotpoints=2000,linecolor=blue]{\xmin}{\xmax}{\fonction}
                         \psplot[linewidth=0.75pt,plotpoints=2000,linecolor=red]{\xmin}{\xmax}{\g}
                         \rput[tr](-0.1,-0.2){$O$}
                         \rput(4.5,0.7){$\blue \cos$}
                         \rput(-3,0.7){$\red \sin$}
                         \psline[linecolor=vert]{->}(-5,0.284)(-3.429,0.284)
                         \psline[linewidth=1.25pt]{->}(0,0)(0,1)
                         \psline[linewidth=1.25pt]{->}(0,0)(1,0)
                         \rput[t](0.5,-0.03){$\vect{i}$}
                         \rput[r](-0.03,0.5){$\vect{j}$}
                    \end{pspicture*}
               }
          \end{extern}
     \end{center}
     \begin{center}\textit{Position relative des deux courbes}\end{center}
}

\end{document}
µ
\documentclass[a4paper]{article}

%================================================================================================================================
%
% Packages
%
%================================================================================================================================

\usepackage[T1]{fontenc} 	% pour caractères accentués
\usepackage[utf8]{inputenc}  % encodage utf8
\usepackage[french]{babel}	% langue : français
\usepackage{fourier}			% caractères plus lisibles
\usepackage[dvipsnames]{xcolor} % couleurs
\usepackage{fancyhdr}		% réglage header footer
\usepackage{needspace}		% empêcher sauts de page mal placés
\usepackage{graphicx}		% pour inclure des graphiques
\usepackage{enumitem,cprotect}		% personnalise les listes d'items (nécessaire pour ol, al ...)
\usepackage{hyperref}		% Liens hypertexte
\usepackage{pstricks,pst-all,pst-node,pstricks-add,pst-math,pst-plot,pst-tree,pst-eucl} % pstricks
\usepackage[a4paper,includeheadfoot,top=2cm,left=3cm, bottom=2cm,right=3cm]{geometry} % marges etc.
\usepackage{comment}			% commentaires multilignes
\usepackage{amsmath,environ} % maths (matrices, etc.)
\usepackage{amssymb,makeidx}
\usepackage{bm}				% bold maths
\usepackage{tabularx}		% tableaux
\usepackage{colortbl}		% tableaux en couleur
\usepackage{fontawesome}		% Fontawesome
\usepackage{environ}			% environment with command
\usepackage{fp}				% calculs pour ps-tricks
\usepackage{multido}			% pour ps tricks
\usepackage[np]{numprint}	% formattage nombre
\usepackage{tikz,tkz-tab} 			% package principal TikZ
\usepackage{pgfplots}   % axes
\usepackage{mathrsfs}    % cursives
\usepackage{calc}			% calcul taille boites
\usepackage[scaled=0.875]{helvet} % font sans serif
\usepackage{svg} % svg
\usepackage{scrextend} % local margin
\usepackage{scratch} %scratch
\usepackage{multicol} % colonnes
%\usepackage{infix-RPN,pst-func} % formule en notation polanaise inversée
\usepackage{listings}

%================================================================================================================================
%
% Réglages de base
%
%================================================================================================================================

\lstset{
language=Python,   % R code
literate=
{á}{{\'a}}1
{à}{{\`a}}1
{ã}{{\~a}}1
{é}{{\'e}}1
{è}{{\`e}}1
{ê}{{\^e}}1
{í}{{\'i}}1
{ó}{{\'o}}1
{õ}{{\~o}}1
{ú}{{\'u}}1
{ü}{{\"u}}1
{ç}{{\c{c}}}1
{~}{{ }}1
}


\definecolor{codegreen}{rgb}{0,0.6,0}
\definecolor{codegray}{rgb}{0.5,0.5,0.5}
\definecolor{codepurple}{rgb}{0.58,0,0.82}
\definecolor{backcolour}{rgb}{0.95,0.95,0.92}

\lstdefinestyle{mystyle}{
    backgroundcolor=\color{backcolour},   
    commentstyle=\color{codegreen},
    keywordstyle=\color{magenta},
    numberstyle=\tiny\color{codegray},
    stringstyle=\color{codepurple},
    basicstyle=\ttfamily\footnotesize,
    breakatwhitespace=false,         
    breaklines=true,                 
    captionpos=b,                    
    keepspaces=true,                 
    numbers=left,                    
xleftmargin=2em,
framexleftmargin=2em,            
    showspaces=false,                
    showstringspaces=false,
    showtabs=false,                  
    tabsize=2,
    upquote=true
}

\lstset{style=mystyle}


\lstset{style=mystyle}
\newcommand{\imgdir}{C:/laragon/www/newmc/assets/imgsvg/}
\newcommand{\imgsvgdir}{C:/laragon/www/newmc/assets/imgsvg/}

\definecolor{mcgris}{RGB}{220, 220, 220}% ancien~; pour compatibilité
\definecolor{mcbleu}{RGB}{52, 152, 219}
\definecolor{mcvert}{RGB}{125, 194, 70}
\definecolor{mcmauve}{RGB}{154, 0, 215}
\definecolor{mcorange}{RGB}{255, 96, 0}
\definecolor{mcturquoise}{RGB}{0, 153, 153}
\definecolor{mcrouge}{RGB}{255, 0, 0}
\definecolor{mclightvert}{RGB}{205, 234, 190}

\definecolor{gris}{RGB}{220, 220, 220}
\definecolor{bleu}{RGB}{52, 152, 219}
\definecolor{vert}{RGB}{125, 194, 70}
\definecolor{mauve}{RGB}{154, 0, 215}
\definecolor{orange}{RGB}{255, 96, 0}
\definecolor{turquoise}{RGB}{0, 153, 153}
\definecolor{rouge}{RGB}{255, 0, 0}
\definecolor{lightvert}{RGB}{205, 234, 190}
\setitemize[0]{label=\color{lightvert}  $\bullet$}

\pagestyle{fancy}
\renewcommand{\headrulewidth}{0.2pt}
\fancyhead[L]{maths-cours.fr}
\fancyhead[R]{\thepage}
\renewcommand{\footrulewidth}{0.2pt}
\fancyfoot[C]{}

\newcolumntype{C}{>{\centering\arraybackslash}X}
\newcolumntype{s}{>{\hsize=.35\hsize\arraybackslash}X}

\setlength{\parindent}{0pt}		 
\setlength{\parskip}{3mm}
\setlength{\headheight}{1cm}

\def\ebook{ebook}
\def\book{book}
\def\web{web}
\def\type{web}

\newcommand{\vect}[1]{\overrightarrow{\,\mathstrut#1\,}}

\def\Oij{$\left(\text{O}~;~\vect{\imath},~\vect{\jmath}\right)$}
\def\Oijk{$\left(\text{O}~;~\vect{\imath},~\vect{\jmath},~\vect{k}\right)$}
\def\Ouv{$\left(\text{O}~;~\vect{u},~\vect{v}\right)$}

\hypersetup{breaklinks=true, colorlinks = true, linkcolor = OliveGreen, urlcolor = OliveGreen, citecolor = OliveGreen, pdfauthor={Didier BONNEL - https://www.maths-cours.fr} } % supprime les bordures autour des liens

\renewcommand{\arg}[0]{\text{arg}}

\everymath{\displaystyle}

%================================================================================================================================
%
% Macros - Commandes
%
%================================================================================================================================

\newcommand\meta[2]{    			% Utilisé pour créer le post HTML.
	\def\titre{titre}
	\def\url{url}
	\def\arg{#1}
	\ifx\titre\arg
		\newcommand\maintitle{#2}
		\fancyhead[L]{#2}
		{\Large\sffamily \MakeUppercase{#2}}
		\vspace{1mm}\textcolor{mcvert}{\hrule}
	\fi 
	\ifx\url\arg
		\fancyfoot[L]{\href{https://www.maths-cours.fr#2}{\black \footnotesize{https://www.maths-cours.fr#2}}}
	\fi 
}


\newcommand\TitreC[1]{    		% Titre centré
     \needspace{3\baselineskip}
     \begin{center}\textbf{#1}\end{center}
}

\newcommand\newpar{    		% paragraphe
     \par
}

\newcommand\nosp {    		% commande vide (pas d'espace)
}
\newcommand{\id}[1]{} %ignore

\newcommand\boite[2]{				% Boite simple sans titre
	\vspace{5mm}
	\setlength{\fboxrule}{0.2mm}
	\setlength{\fboxsep}{5mm}	
	\fcolorbox{#1}{#1!3}{\makebox[\linewidth-2\fboxrule-2\fboxsep]{
  		\begin{minipage}[t]{\linewidth-2\fboxrule-4\fboxsep}\setlength{\parskip}{3mm}
  			 #2
  		\end{minipage}
	}}
	\vspace{5mm}
}

\newcommand\CBox[4]{				% Boites
	\vspace{5mm}
	\setlength{\fboxrule}{0.2mm}
	\setlength{\fboxsep}{5mm}
	
	\fcolorbox{#1}{#1!3}{\makebox[\linewidth-2\fboxrule-2\fboxsep]{
		\begin{minipage}[t]{1cm}\setlength{\parskip}{3mm}
	  		\textcolor{#1}{\LARGE{#2}}    
 	 	\end{minipage}  
  		\begin{minipage}[t]{\linewidth-2\fboxrule-4\fboxsep}\setlength{\parskip}{3mm}
			\raisebox{1.2mm}{\normalsize\sffamily{\textcolor{#1}{#3}}}						
  			 #4
  		\end{minipage}
	}}
	\vspace{5mm}
}

\newcommand\cadre[3]{				% Boites convertible html
	\par
	\vspace{2mm}
	\setlength{\fboxrule}{0.1mm}
	\setlength{\fboxsep}{5mm}
	\fcolorbox{#1}{white}{\makebox[\linewidth-2\fboxrule-2\fboxsep]{
  		\begin{minipage}[t]{\linewidth-2\fboxrule-4\fboxsep}\setlength{\parskip}{3mm}
			\raisebox{-2.5mm}{\sffamily \small{\textcolor{#1}{\MakeUppercase{#2}}}}		
			\par		
  			 #3
 	 		\end{minipage}
	}}
		\vspace{2mm}
	\par
}

\newcommand\bloc[3]{				% Boites convertible html sans bordure
     \needspace{2\baselineskip}
     {\sffamily \small{\textcolor{#1}{\MakeUppercase{#2}}}}    
		\par		
  			 #3
		\par
}

\newcommand\CHelp[1]{
     \CBox{Plum}{\faInfoCircle}{À RETENIR}{#1}
}

\newcommand\CUp[1]{
     \CBox{NavyBlue}{\faThumbsOUp}{EN PRATIQUE}{#1}
}

\newcommand\CInfo[1]{
     \CBox{Sepia}{\faArrowCircleRight}{REMARQUE}{#1}
}

\newcommand\CRedac[1]{
     \CBox{PineGreen}{\faEdit}{BIEN R\'EDIGER}{#1}
}

\newcommand\CError[1]{
     \CBox{Red}{\faExclamationTriangle}{ATTENTION}{#1}
}

\newcommand\TitreExo[2]{
\needspace{4\baselineskip}
 {\sffamily\large EXERCICE #1\ (\emph{#2 points})}
\vspace{5mm}
}

\newcommand\img[2]{
          \includegraphics[width=#2\paperwidth]{\imgdir#1}
}

\newcommand\imgsvg[2]{
       \begin{center}   \includegraphics[width=#2\paperwidth]{\imgsvgdir#1} \end{center}
}


\newcommand\Lien[2]{
     \href{#1}{#2 \tiny \faExternalLink}
}
\newcommand\mcLien[2]{
     \href{https~://www.maths-cours.fr/#1}{#2 \tiny \faExternalLink}
}

\newcommand{\euro}{\eurologo{}}

%================================================================================================================================
%
% Macros - Environement
%
%================================================================================================================================

\newenvironment{tex}{ %
}
{%
}

\newenvironment{indente}{ %
	\setlength\parindent{10mm}
}

{
	\setlength\parindent{0mm}
}

\newenvironment{corrige}{%
     \needspace{3\baselineskip}
     \medskip
     \textbf{\textsc{Corrigé}}
     \medskip
}
{
}

\newenvironment{extern}{%
     \begin{center}
     }
     {
     \end{center}
}

\NewEnviron{code}{%
	\par
     \boite{gray}{\texttt{%
     \BODY
     }}
     \par
}

\newenvironment{vbloc}{% boite sans cadre empeche saut de page
     \begin{minipage}[t]{\linewidth}
     }
     {
     \end{minipage}
}
\NewEnviron{h2}{%
    \needspace{3\baselineskip}
    \vspace{0.6cm}
	\noindent \MakeUppercase{\sffamily \large \BODY}
	\vspace{1mm}\textcolor{mcgris}{\hrule}\vspace{0.4cm}
	\par
}{}

\NewEnviron{h3}{%
    \needspace{3\baselineskip}
	\vspace{5mm}
	\textsc{\BODY}
	\par
}

\NewEnviron{margeneg}{ %
\begin{addmargin}[-1cm]{0cm}
\BODY
\end{addmargin}
}

\NewEnviron{html}{%
}

\begin{document}
\meta{url}{/cours/primitives-integrales/}
\meta{pid}{529}
\meta{titre}{Primitives et intégrales}
\meta{type}{cours}
\begin{h2}1. Primitives d'une fonction\end{h2}
\cadre{bleu}{Définition}{% id="d10"
     Soit $f$ une fonction définie sur $I$.
     \par
     On dit que $F$  est une primitive de  $f$  sur l'intervalle $I$, si et seulement si $F$ est dérivable sur $I$ et pour tout $x$ de $I$, $F^{\prime}\left(x\right)=f\left(x\right)$.
}
\bloc{orange}{Exemple}{% id="e10"
     La fonction $F: ~x\mapsto x^{2}$ est une primitive de la fonction $f:~x\mapsto 2x$ sur $\mathbb{R}$.
     \par
     La fonction $G: ~x\mapsto x^{2}+1$ est aussi une primitive de cette même fonction $f$.
}
\cadre{vert}{Propriété}{% id="e20"
     Si $F$ est une primitive de $f$ sur $I$, alors les autres primitives de $f$ sur $I$ sont les fonctions de la forme $F+k$ où $k\in \mathbb{R}.$
}
\bloc{vert}{Remarque}{% id="r20"
     Une fonction continue ayant une infinité de primitives, il ne faut pas dire \textbf{la} primitive de $f$ mais \textbf{une} primitive de $f$.
}
\bloc{orange}{Exemple}{% id="e20"
     Les primitives de la fonction $f:~x\mapsto 2x$ sont les fonctions $F:~ x\mapsto x^{2}+k$ où $k\in \mathbb{R}.$
}
\cadre{vert}{Propriété}{% id="p30"
     Toute fonction \textbf{continue} sur un intervalle $I$ admet des primitives sur $I$.
}
\cadre{vert}{Propriétés}{% id="p40"
     \textbf{Primitives des fonctions usuelles :}
     \begin{tabularx}{0.8\linewidth}{|*{3}{>{\centering \arraybackslash }X|}}%class="compact" width="600"
          \hline
          \textbf{Fonction $f$} & \textbf{Primitives $F$} & \textbf{Ensemble de validité}
          \\ \hline
          $0$ & $k$ & $\mathbb{R}$
          \\ \hline
          $a$ & $ax+k$ & $\mathbb{R}$
          \\ \hline
          $x^{n} ~ \left(n\in \mathbb{N}\right)$ & $\frac{x^{n+1}}{n+1}+k$ & $\mathbb{R}$
          \\ \hline
          $\frac{1}{x^{n}} ~ \left(n\in \mathbb{N};~n>1\right)$ & $-\frac{1}{\left(n-1\right)x^{n-1}}+k$ & $\mathbb{R}-\left\{0\right\}$
          \\ \hline
          $\frac{1}{x}$ & $\ln x+k$ & $\left]0;+\infty \right[$
          \\ \hline
          $e^{x}$ & $e^{x}+k$ & $\mathbb{R}$
          \\ \hline
     \end{tabularx}
}
\cadre{vert}{Propriétés}{% id="p50"
     Si $f$ et $g$ sont deux fonctions définies sur $I$ et admettant respectivement $F$ et $G$ comme primitives sur $I$ et $k$ un réel quelconque.
     \begin{itemize}
          \item $F+G$ est une primitive de la fonction $f+g$ sur $I$.
          \item $kF$ est une primitive de la fonction $kf$ sur $I$.
     \end{itemize}
}
\cadre{vert}{Propriétés}{% id="p60"
     \textbf{Primitives et fonctions composées}
     \par
     Soit $u$ une fonction définie et dérivable sur un intervalle $I$.
     \begin{tabularx}{0.8\linewidth}{|*{3}{>{\centering \arraybackslash }X|}}%class="compact" width="600"
          \hline
          \textbf{Fonction $f$ }  &  \textbf{Primitives $F$}  &  \textbf{Condition}
          \\ \hline
          $u^{\prime}u^{n} ~ \left(n\in \mathbb{N}\right)$  &  $\frac{u^{n+1}}{n+1}+k$  &
          \\ \hline
          $\frac{u^{\prime}}{u}$  &  $\ln u+k$  &  si $u\left(x\right)>0$
          \\ \hline
          $\frac{u^{\prime}}{u^{n}} ~ \left(n\in \mathbb{N};~n>1\right)$  &  $-\frac{1}{\left(n-1\right)u^{n-1}}+k$  &  si $u\left(x\right)\neq 0$
          \\ \hline
          $\frac{u^{\prime}}{\sqrt{u}}$  &  $2\sqrt{u}+k$  &  si $u\left(x\right)>0$
          \\ \hline
          $u^{\prime}e^{u}$  &  $e^{u}+k$  &
          \\  \hline
     \end{tabularx}
}
\bloc{orange}{Exemple}{% id="e60"
     La fonction $x\mapsto \frac{2x}{x^{2}+1}$ admet comme primitives les fonctions de la forme $x\mapsto \ln\left(x^{2}+1\right)+k$ sur tout intervalle de $\mathbb{R}$ (forme $\frac{u^{\prime}}{u}$).
}
\begin{h2}2. Intégrales\end{h2}
\cadre{bleu}{Définition}{% id="d80"
     Soit $f$ une fonction continue sur un intervalle $\left[a;b\right]$ et $F$ une primitive de $f$ sur $\left[a;b\right]$.
     \textbf{L'intégrale de $a$ à $b$ de $f$} est le nombre réel noté $\int_{a}^{b}f\left(x\right)\text{d}x$ défini par:
     \par
     $\int_{a}^{b}f\left(x\right)\text{d}x=F\left(b\right)-F\left(a\right)$.
}
\bloc{vert}{Remarques}{% id="r80"
     \begin{itemize}
          \item L'intégrale ne dépend pas de la primitive de $f$ choisie.
          \par
          En effet si $G$ est une autre primitive de $f$, on a $G=F+k$ donc :
          \par
          $G\left(b\right)-G\left(a\right)=F\left(b\right)+k-\left(F\left(a\right)+k\right)=F\left(b\right)-F\left(a\right)$
          \item Dans l'expression  $\int_{a}^{b}f\left(x\right)\text{d}x$, $x$ est une variable \og~muette~\fg{}. C'est à dire que l'on ne change pas l'expression si on remplace $x$ par une autre lettre. En pratique, on emploie souvent la lettre $t$ notamment lorsque la lettre $x$ est employée par ailleurs.
     \end{itemize}
}
\bloc{vert}{Notations}{% id="n80"
     On note souvent : $F\left(b\right)-F\left(a\right)=\left[F\left(x\right)\right]_{a}^{b}$.
     \par
     On a avec cette notation :
     \par
     $\int_{a}^{b}f\left(x\right)\text{d}x=\left[F\left(x\right)\right]_{a}^{b}$.
}
\bloc{orange}{Exemple}{% id="e80"
     La fonction $F$ définie par $F\left(x\right)=\frac{x^{3}}{3}$ est une primitive de la fonction carré.
     \par
     On a donc :
     \par
     $\int_{0}^{1}x^{2}\text{d}x=\left[\frac{x^{3}}{3}\right]_{0}^{1}=\frac{1}{3}-\frac{0}{3}=\frac{1}{3}$.
}
\cadre{rouge}{Théorème (intégrale fonction de sa borne supérieure)}{% id="t90"
     Soit $f$ une fonction continue sur un intervalle $I$ et $a \in  I$; la fonction définie sur $I$ par~:
     \begin{center}$x\mapsto \int_{a}^{ x}f\left(t\right)\text{d}t$\end{center}
     est la primitive de $f$ qui s'annule pour $x=a$.
}
\bloc{orange}{Démonstration}{% id="m90"
     Soit $F$ une primitive (quelconque) de $f$. Posons $\Phi \left(x\right)=\int_{a}^{ x}f\left(t\right)\text{d}t$
     \par
     $\Phi \left(x\right)=\int_{a}^{ x}f\left(t\right)\text{d}t=F\left(x\right)-F\left(a\right)$
     \par
     donc:
     \par
     $\Phi ^{\prime}\left(x\right)=F^{\prime}\left(x\right)=f\left(x\right)$.
     \par
     Ce qui prouve que $\Phi $ est aussi une primitive de $f$.
     \par
     De plus $\Phi \left(a\right)=F\left(a\right)-F\left(a\right)=0$.
}
\bloc{cyan}{Remarque}{% id="r90"
     Notez bien la position du $x$ en borne supérieure de l'intégrale.
}
\bloc{orange}{Exemple}{% id="e90"
     La fonction définie sur $\left[0 ; +\infty \right[$ $x\mapsto \int_{1}^{ x}\frac{1}{t}\text{d}t$ (on peut aussi écrire $\int_{1}^{ x}\frac{\text{d}t}{t}$) est la primitive de la fonction inverse qui s'annule pour $x=1$. C'est donc la fonction logarithme népérien:
     \par
     $\ln\left(x\right)= \int_{1}^{ x}\frac{\text{d}t}{t}.$
}
\cadre{vert}{Propriété}{% id="p100"
     \textbf{Relation de Chasles}
     \par
     Soit $f$ une fonction continue sur $\left[a;b\right]$ et $c\in \left[a;b\right]$.
     \par
     $\int_{a}^{b}f\left(x\right)\text{d}x=\int_{a}^{c}f\left(x\right)\text{d}x+\int_{c}^{b}f\left(x\right)\text{d}x$.
}
\cadre{vert}{Propriété}{% id="p110"
     \textbf{Linéarité de l'intégrale}
     \par
     Soit $f$ et $g$ deux fonctions continues sur $\left[a;b\right]$ et $\lambda \in \mathbb{R}$.
     \begin{itemize}
          \item $\int_{a}^{b}f\left(x\right)+g\left(x\right)\text{d}x=\int_{a}^{b}f\left(x\right)\text{d}x+\int_{a}^{b}g\left(x\right)\text{d}x$
          \item $\int_{a}^{b} \lambda  f\left(x\right)\text{d}x=\lambda  \int_{a}^{b}f\left(x\right)\text{d}x$.
     \end{itemize}
}
\cadre{vert}{Propriété}{% id="p120"
     \textbf{Comparaison d'intégrales}
     \par
     Soit $f$ et $g$ deux fonctions continues sur $\left[a;b\right]$ telles que $f\geqslant g$ sur $\left[a;b\right]$.
     \par
     $\int_{a}^{b}f\left(x\right)\text{d}x\geqslant \int_{a}^{b}g\left(x\right)\text{d}x$.
}
\bloc{vert}{Remarque}{% id="r120"
     En particulier, en prenant pour $g$ la fonction nulle on obtient si $f\left(x\right)\geqslant 0$ sur $\left[a;b\right]$:
     \begin{center}
          $\int_{a}^{b}f\left(x\right)\text{d}x\geqslant 0$.
     \end{center}
}
\begin{h2}3. Interprétation graphique\end{h2}
\cadre{bleu}{Définition}{% id="d130"
     Le plan $P$ est rapporté à un repère orthogonal $\left(O,\vec{i},\vec{j}\right)$.
     \par
     On appelle \textbf{unité d'aire (u.a.)} l'aire d'un rectangle (qui est un carré si le repère est orthonormé) dont les côtés mesurent $||\vec{i}||$ et $||\vec{j}||$.
}
\begin{center}
     \begin{extern} %width="400" alt="unité d'aire"
          \resizebox{6cm}{!}{%
               % -+-+-+ variables modifiables
               \def\xmin{-1.2}
               \def\xmax{4.2}
               \def\ymin{-0.9}
               \def\ymax{1.8}
               \def\xunit{2}  % unités en cm
               \def\yunit{2}
               \psset{xunit=\xunit,yunit=\yunit,algebraic=true}
               \fontsize{15pt}{15pt}\selectfont
               \begin{pspicture*}[linewidth=1pt](\xmin,\ymin)(\xmax,\ymax)
                    \psgrid[gridcolor=mcgris, subgriddiv=1, gridlabels=0pt](-2,-1)(5,2)
                    \psaxes[linewidth=0.75pt]{->}(0,0)(\xmin,\ymin)(\xmax,\ymax)
                    \pscustom[fillstyle=solid,fillcolor=vert,linecolor=vert,linewidth=0.75pt,opacity=0.2]{%
                         \psline(0,0)(1,0)(1,1)(0,1)
                         \closepath
                    }
                    \rput[tr](-0.1,-0.1){$O$}
                    \psline[linewidth=1.25pt]{->}(0,0)(0,1)
                    \psline[linewidth=1.25pt]{->}(0,0)(1,0)
                    \rput[t](0.5,-0.03){$\vect{i}$}
                    \rput[r](-0.03,0.5){$\vect{j}$}
                    %                        \rput[l](1.1){ unité d'aire}
               \end{pspicture*}
          }
     \end{extern}
\end{center}
\begin{center}
     \textit{Unité d'aire dans le cas d'un repère orthonormé}
\end{center}
\cadre{vert}{Propriété}{% id="p130"
     Si $f$ est une fonction continue et \textbf{positive} sur $\left[a;b\right]$, alors l'intégrale $\int_{a}^{b}f\left(x\right)\text{d}x$ est l'aire, en unités d'aire, de la surface délimitée par :
     \begin{itemize}
          \item la courbe $C_{f}$,
          \item l'axe des abscisses,
          \item les droites (verticales) d'équations $x=a$ et $x=b$.
     \end{itemize}
}
\bloc{orange}{Exemple}{% id="e130"
     \begin{center}
          \begin{extern} %width="600" alt="aire et intégrale"
               \resizebox{8cm}{!}{%
                    % -+-+-+ variables modifiables
                    \def\fonction{ln(1+x*x) }
                    \def\xmin{-2.2}
                    \def\xmax{6.2}
                    \def\ymin{-1.8}
                    \def\ymax{3.8}
                    \def\xunit{2}  % unités en cm
                    \def\yunit{2}
                    \psset{xunit=\xunit,yunit=\yunit,algebraic=true}
                    \fontsize{15pt}{15pt}\selectfont
                    \begin{pspicture*}[linewidth=1pt](\xmin,\ymin)(\xmax,\ymax)
                         \psgrid[gridcolor=mcgris, subgriddiv=1, gridlabels=0pt](-3,-1.8)(7,4)
                         \psaxes[linewidth=0.75pt]{->}(0,0)(\xmin,\ymin)(\xmax,\ymax)
                         \pscustom[fillstyle=solid,fillcolor=vert,linecolor=vert,linewidth=0.75pt,opacity=0.2]{%
                              \psplot{1}{3}{\fonction}
                              \psline(3,0)(1,0)
                              \closepath
                         }
                         \psplot[plotpoints=2000,linecolor=blue]{\xmin}{\xmax}{\fonction}
                         \rput[tr](-0.1,-0.1){$O$}
                         \rput[l](5.5,3.2){$\color{blue} \mathcal{C}_f$}
                    \end{pspicture*}
               }
          \end{extern}
     \end{center}
     L'aire colorée ci-dessus est égale (en unités d'aire) à $\int_{1}^{3}f\left(x\right)\text{d}x$.
}
\bloc{vert}{Remarques}{% id="r130"
     \begin{itemize}
          \item Si $f$ est négative sur $\left[a;b\right]$, la propriété précédente appliquée à la fonction $-f$ montre que $\int_{a}^{b}f\left(x\right)\text{d}x$ est égale à l'\textbf{opposé} de l'aire délimitée par la courbe $C_{f}$, l'axe des abscisses, les droites d'équations $x=a$ et $x=b$.
          \item Si le signe de $f$ varie sur $\left[a;b\right]$, on découpe $\left[a;b\right]$ en sous-intervalles sur lesquels $f$ garde un signe constant.
     \end{itemize}
}
\cadre{vert}{Propriété}{% id="p140"
     Si $f$ et $g$ sont des fonctions continues et telles que $f\leqslant g$ sur $\left[a;b\right]$, alors l'aire de la surface délimitée par :
     \begin{itemize}
          \item la courbe $C_{f}$,
          \item la courbe $C_{g}$,
          \item les droites (verticales) d'équations $x=a$ et $x=b$.
     \end{itemize}
     est égale (en unités d'aire) à :
     \par
     $A=\int_{a}^{b}\left(g\left(x\right)-f\left(x\right)\right)\text{d}x$.
}
\bloc{orange}{Exemple}{% id="e140"
     $f$ et $g$ définies par $f\left(x\right)=x^{2}-x$ et $g\left(x\right)=3x-x^{2}$ sont représentées par les paraboles ci-dessous :
     \begin{center}
          \begin{extern} %width="600" alt="aire entre deux courbes"
               \resizebox{8cm}{!}{%
                    % -+-+-+ variables modifiables
                    \def\fonction{x*x-x }
                    \def\g{3*x-x*x }
                    \def\xmin{-2.2}
                    \def\xmax{6.2}
                    \def\ymin{-1.8}
                    \def\ymax{3.8}
                    \def\xunit{2}  % unités en cm
                    \def\yunit{2}
                    \psset{xunit=\xunit,yunit=\yunit,algebraic=true}
                    \fontsize{15pt}{15pt}\selectfont
                    \begin{pspicture*}[linewidth=1pt](\xmin,\ymin)(\xmax,\ymax)
                         \psgrid[gridcolor=mcgris, subgriddiv=1, gridlabels=0pt](-3,-3)(7,4)
                         \psaxes[linewidth=0.75pt]{->}(0,0)(\xmin,\ymin)(\xmax,\ymax)
                         \pscustom[fillstyle=solid,fillcolor=vert,linestyle=solid,linewidth=0.2pt,opacity=0.2]{%
                              \psplot{0}{2}{\fonction}
                              \psplot{2}{0}{\g}
                         }
                         \psplot[plotpoints=2000,linecolor=blue]{\xmin}{\xmax}{\fonction}
                         \psplot[plotpoints=2000,linecolor=red]{\xmin}{\xmax}{\g}
                         \rput[tr](-0.1,-0.1){$O$}
                         \rput(-1.6,3){$\color{blue} \mathcal{C}_f$}
                         \rput(3.6,-1){$\color{red} \mathcal{C}_g$}
                    \end{pspicture*}
               }
          \end{extern}
     \end{center}
     L'aire colorée est égale (en unités d'aire) à :
     \par
     $A=\int_{0}^{2}\left(g\left(x\right)-f\left(x\right)\right)\text{d}x=\int_{0}^{2} \left(4x-2x^{2}\right)\text{d}x=\left[2x^{2}-\frac{2}{3}x^{3}\right]_{0}^{2}=\frac{8}{3} \text{u.a.}$
}

\end{document}
µ
\documentclass[a4paper]{article}

%================================================================================================================================
%
% Packages
%
%================================================================================================================================

\usepackage[T1]{fontenc} 	% pour caractères accentués
\usepackage[utf8]{inputenc}  % encodage utf8
\usepackage[french]{babel}	% langue : français
\usepackage{fourier}			% caractères plus lisibles
\usepackage[dvipsnames]{xcolor} % couleurs
\usepackage{fancyhdr}		% réglage header footer
\usepackage{needspace}		% empêcher sauts de page mal placés
\usepackage{graphicx}		% pour inclure des graphiques
\usepackage{enumitem,cprotect}		% personnalise les listes d'items (nécessaire pour ol, al ...)
\usepackage{hyperref}		% Liens hypertexte
\usepackage{pstricks,pst-all,pst-node,pstricks-add,pst-math,pst-plot,pst-tree,pst-eucl} % pstricks
\usepackage[a4paper,includeheadfoot,top=2cm,left=3cm, bottom=2cm,right=3cm]{geometry} % marges etc.
\usepackage{comment}			% commentaires multilignes
\usepackage{amsmath,environ} % maths (matrices, etc.)
\usepackage{amssymb,makeidx}
\usepackage{bm}				% bold maths
\usepackage{tabularx}		% tableaux
\usepackage{colortbl}		% tableaux en couleur
\usepackage{fontawesome}		% Fontawesome
\usepackage{environ}			% environment with command
\usepackage{fp}				% calculs pour ps-tricks
\usepackage{multido}			% pour ps tricks
\usepackage[np]{numprint}	% formattage nombre
\usepackage{tikz,tkz-tab} 			% package principal TikZ
\usepackage{pgfplots}   % axes
\usepackage{mathrsfs}    % cursives
\usepackage{calc}			% calcul taille boites
\usepackage[scaled=0.875]{helvet} % font sans serif
\usepackage{svg} % svg
\usepackage{scrextend} % local margin
\usepackage{scratch} %scratch
\usepackage{multicol} % colonnes
%\usepackage{infix-RPN,pst-func} % formule en notation polanaise inversée
\usepackage{listings}

%================================================================================================================================
%
% Réglages de base
%
%================================================================================================================================

\lstset{
language=Python,   % R code
literate=
{á}{{\'a}}1
{à}{{\`a}}1
{ã}{{\~a}}1
{é}{{\'e}}1
{è}{{\`e}}1
{ê}{{\^e}}1
{í}{{\'i}}1
{ó}{{\'o}}1
{õ}{{\~o}}1
{ú}{{\'u}}1
{ü}{{\"u}}1
{ç}{{\c{c}}}1
{~}{{ }}1
}


\definecolor{codegreen}{rgb}{0,0.6,0}
\definecolor{codegray}{rgb}{0.5,0.5,0.5}
\definecolor{codepurple}{rgb}{0.58,0,0.82}
\definecolor{backcolour}{rgb}{0.95,0.95,0.92}

\lstdefinestyle{mystyle}{
    backgroundcolor=\color{backcolour},   
    commentstyle=\color{codegreen},
    keywordstyle=\color{magenta},
    numberstyle=\tiny\color{codegray},
    stringstyle=\color{codepurple},
    basicstyle=\ttfamily\footnotesize,
    breakatwhitespace=false,         
    breaklines=true,                 
    captionpos=b,                    
    keepspaces=true,                 
    numbers=left,                    
xleftmargin=2em,
framexleftmargin=2em,            
    showspaces=false,                
    showstringspaces=false,
    showtabs=false,                  
    tabsize=2,
    upquote=true
}

\lstset{style=mystyle}


\lstset{style=mystyle}
\newcommand{\imgdir}{C:/laragon/www/newmc/assets/imgsvg/}
\newcommand{\imgsvgdir}{C:/laragon/www/newmc/assets/imgsvg/}

\definecolor{mcgris}{RGB}{220, 220, 220}% ancien~; pour compatibilité
\definecolor{mcbleu}{RGB}{52, 152, 219}
\definecolor{mcvert}{RGB}{125, 194, 70}
\definecolor{mcmauve}{RGB}{154, 0, 215}
\definecolor{mcorange}{RGB}{255, 96, 0}
\definecolor{mcturquoise}{RGB}{0, 153, 153}
\definecolor{mcrouge}{RGB}{255, 0, 0}
\definecolor{mclightvert}{RGB}{205, 234, 190}

\definecolor{gris}{RGB}{220, 220, 220}
\definecolor{bleu}{RGB}{52, 152, 219}
\definecolor{vert}{RGB}{125, 194, 70}
\definecolor{mauve}{RGB}{154, 0, 215}
\definecolor{orange}{RGB}{255, 96, 0}
\definecolor{turquoise}{RGB}{0, 153, 153}
\definecolor{rouge}{RGB}{255, 0, 0}
\definecolor{lightvert}{RGB}{205, 234, 190}
\setitemize[0]{label=\color{lightvert}  $\bullet$}

\pagestyle{fancy}
\renewcommand{\headrulewidth}{0.2pt}
\fancyhead[L]{maths-cours.fr}
\fancyhead[R]{\thepage}
\renewcommand{\footrulewidth}{0.2pt}
\fancyfoot[C]{}

\newcolumntype{C}{>{\centering\arraybackslash}X}
\newcolumntype{s}{>{\hsize=.35\hsize\arraybackslash}X}

\setlength{\parindent}{0pt}		 
\setlength{\parskip}{3mm}
\setlength{\headheight}{1cm}

\def\ebook{ebook}
\def\book{book}
\def\web{web}
\def\type{web}

\newcommand{\vect}[1]{\overrightarrow{\,\mathstrut#1\,}}

\def\Oij{$\left(\text{O}~;~\vect{\imath},~\vect{\jmath}\right)$}
\def\Oijk{$\left(\text{O}~;~\vect{\imath},~\vect{\jmath},~\vect{k}\right)$}
\def\Ouv{$\left(\text{O}~;~\vect{u},~\vect{v}\right)$}

\hypersetup{breaklinks=true, colorlinks = true, linkcolor = OliveGreen, urlcolor = OliveGreen, citecolor = OliveGreen, pdfauthor={Didier BONNEL - https://www.maths-cours.fr} } % supprime les bordures autour des liens

\renewcommand{\arg}[0]{\text{arg}}

\everymath{\displaystyle}

%================================================================================================================================
%
% Macros - Commandes
%
%================================================================================================================================

\newcommand\meta[2]{    			% Utilisé pour créer le post HTML.
	\def\titre{titre}
	\def\url{url}
	\def\arg{#1}
	\ifx\titre\arg
		\newcommand\maintitle{#2}
		\fancyhead[L]{#2}
		{\Large\sffamily \MakeUppercase{#2}}
		\vspace{1mm}\textcolor{mcvert}{\hrule}
	\fi 
	\ifx\url\arg
		\fancyfoot[L]{\href{https://www.maths-cours.fr#2}{\black \footnotesize{https://www.maths-cours.fr#2}}}
	\fi 
}


\newcommand\TitreC[1]{    		% Titre centré
     \needspace{3\baselineskip}
     \begin{center}\textbf{#1}\end{center}
}

\newcommand\newpar{    		% paragraphe
     \par
}

\newcommand\nosp {    		% commande vide (pas d'espace)
}
\newcommand{\id}[1]{} %ignore

\newcommand\boite[2]{				% Boite simple sans titre
	\vspace{5mm}
	\setlength{\fboxrule}{0.2mm}
	\setlength{\fboxsep}{5mm}	
	\fcolorbox{#1}{#1!3}{\makebox[\linewidth-2\fboxrule-2\fboxsep]{
  		\begin{minipage}[t]{\linewidth-2\fboxrule-4\fboxsep}\setlength{\parskip}{3mm}
  			 #2
  		\end{minipage}
	}}
	\vspace{5mm}
}

\newcommand\CBox[4]{				% Boites
	\vspace{5mm}
	\setlength{\fboxrule}{0.2mm}
	\setlength{\fboxsep}{5mm}
	
	\fcolorbox{#1}{#1!3}{\makebox[\linewidth-2\fboxrule-2\fboxsep]{
		\begin{minipage}[t]{1cm}\setlength{\parskip}{3mm}
	  		\textcolor{#1}{\LARGE{#2}}    
 	 	\end{minipage}  
  		\begin{minipage}[t]{\linewidth-2\fboxrule-4\fboxsep}\setlength{\parskip}{3mm}
			\raisebox{1.2mm}{\normalsize\sffamily{\textcolor{#1}{#3}}}						
  			 #4
  		\end{minipage}
	}}
	\vspace{5mm}
}

\newcommand\cadre[3]{				% Boites convertible html
	\par
	\vspace{2mm}
	\setlength{\fboxrule}{0.1mm}
	\setlength{\fboxsep}{5mm}
	\fcolorbox{#1}{white}{\makebox[\linewidth-2\fboxrule-2\fboxsep]{
  		\begin{minipage}[t]{\linewidth-2\fboxrule-4\fboxsep}\setlength{\parskip}{3mm}
			\raisebox{-2.5mm}{\sffamily \small{\textcolor{#1}{\MakeUppercase{#2}}}}		
			\par		
  			 #3
 	 		\end{minipage}
	}}
		\vspace{2mm}
	\par
}

\newcommand\bloc[3]{				% Boites convertible html sans bordure
     \needspace{2\baselineskip}
     {\sffamily \small{\textcolor{#1}{\MakeUppercase{#2}}}}    
		\par		
  			 #3
		\par
}

\newcommand\CHelp[1]{
     \CBox{Plum}{\faInfoCircle}{À RETENIR}{#1}
}

\newcommand\CUp[1]{
     \CBox{NavyBlue}{\faThumbsOUp}{EN PRATIQUE}{#1}
}

\newcommand\CInfo[1]{
     \CBox{Sepia}{\faArrowCircleRight}{REMARQUE}{#1}
}

\newcommand\CRedac[1]{
     \CBox{PineGreen}{\faEdit}{BIEN R\'EDIGER}{#1}
}

\newcommand\CError[1]{
     \CBox{Red}{\faExclamationTriangle}{ATTENTION}{#1}
}

\newcommand\TitreExo[2]{
\needspace{4\baselineskip}
 {\sffamily\large EXERCICE #1\ (\emph{#2 points})}
\vspace{5mm}
}

\newcommand\img[2]{
          \includegraphics[width=#2\paperwidth]{\imgdir#1}
}

\newcommand\imgsvg[2]{
       \begin{center}   \includegraphics[width=#2\paperwidth]{\imgsvgdir#1} \end{center}
}


\newcommand\Lien[2]{
     \href{#1}{#2 \tiny \faExternalLink}
}
\newcommand\mcLien[2]{
     \href{https~://www.maths-cours.fr/#1}{#2 \tiny \faExternalLink}
}

\newcommand{\euro}{\eurologo{}}

%================================================================================================================================
%
% Macros - Environement
%
%================================================================================================================================

\newenvironment{tex}{ %
}
{%
}

\newenvironment{indente}{ %
	\setlength\parindent{10mm}
}

{
	\setlength\parindent{0mm}
}

\newenvironment{corrige}{%
     \needspace{3\baselineskip}
     \medskip
     \textbf{\textsc{Corrigé}}
     \medskip
}
{
}

\newenvironment{extern}{%
     \begin{center}
     }
     {
     \end{center}
}

\NewEnviron{code}{%
	\par
     \boite{gray}{\texttt{%
     \BODY
     }}
     \par
}

\newenvironment{vbloc}{% boite sans cadre empeche saut de page
     \begin{minipage}[t]{\linewidth}
     }
     {
     \end{minipage}
}
\NewEnviron{h2}{%
    \needspace{3\baselineskip}
    \vspace{0.6cm}
	\noindent \MakeUppercase{\sffamily \large \BODY}
	\vspace{1mm}\textcolor{mcgris}{\hrule}\vspace{0.4cm}
	\par
}{}

\NewEnviron{h3}{%
    \needspace{3\baselineskip}
	\vspace{5mm}
	\textsc{\BODY}
	\par
}

\NewEnviron{margeneg}{ %
\begin{addmargin}[-1cm]{0cm}
\BODY
\end{addmargin}
}

\NewEnviron{html}{%
}

\begin{document}
\meta{url}{/cours/nombres-complexes-geometrie/}
\meta{pid}{533}
\meta{titre}{Nombres complexes}
\meta{type}{cours}
\begin{h2}1. Ensemble des nombres complexes\end{h2}
\cadre{rouge}{Théorème et Définition}{% id="t10"
     On admet qu'il existe un ensemble de nombres (appelés \textbf{nombres complexes}), noté $\mathbb{C}$ tel que:
     \begin{itemize}
          \item $\mathbb{C}$ contient $\mathbb{R}$
          \item $\mathbb{C}$ est muni d'une addition et d'une multiplication qui suivent des règles de calcul analogues à celles de $\mathbb{R}$
          \item $\mathbb{C}$ contient un nombre noté $i$ tel que $i^{2}=-1$
          \item Chaque élément $z$ de $\mathbb{C}$ s'écrit \textbf{de manière unique} sous la forme $z=a+ib$ où $a$ et $b$ sont deux réels.
     \end{itemize}
}
\bloc{orange}{Exemple}{% id="e10"
     $\sqrt{5}+\frac{1}{2}i$ , $ 3i $ et $ \sqrt{2}$ sont des nombres complexes ($\sqrt{2}$ est un nombre réel mais comme $\mathbb{R}\subset\mathbb{C}$ c'est aussi un nombre complexe !)
}
\bloc{cyan}{Remarque}{% id="r10"
     \textbf{Attention~: }On définit une addition et une multiplication sur $\mathbb{C}$ mais on ne définit \textbf{pas} de relation d'ordre (comme $\leqslant $). En effet il n'est pas possible de définir une telle relation qui soit compatible avec celle définie sur $\mathbb{R}$ et possède les même propriétés que dans $\mathbb{R}$.
     \par
     Dans les exercices, attention donc à ne pas écrire de choses comme $z < z^{\prime}$, si $z$ et $z^{\prime}$ sont des nombres complexes non réels !
}
\cadre{bleu}{Définitions}{% id="d20"
     \begin{itemize}
          \item L'écriture $ z = a+ib$ est appelée la \textbf{forme algébrique} du nombre complexe $z$.
          \item Le nombre réel $a$ s'appelle la \textbf{partie réelle} du nombre complexe $z$.
          \item Le nombre réel $b$ s'appelle la \textbf{partie imaginaire} du nombre complexe $z$.
          \item Si la partie réelle de $z$ est nulle (c'est à dire $a=0$ et $z=bi$), on dit que $z$ est un \textbf{imaginaire pur} .
     \end{itemize}
}
\cadre{vert}{Propriété}{% id="p30"
     Deux nombres complexes sont égaux si et seulement si ils ont la même partie réelle et la même partie imaginaire.
}
\bloc{cyan}{Remarques}{% id="r30"
     \begin{itemize}
          \item Cela résulte immédiatement du fait que chaque élément de $\mathbb{C}$ s'écrit \textbf{de manière unique} sous la forme $z=a+ib$.
          \item En particulier, un nombre complexe est nul si et seulement si ses parties réelles et imaginaires sont nulles.
     \end{itemize}
}
\begin{h2}2. Conjugué\end{h2}
\cadre{bleu}{Définition}{% id="d40"
     Soit $z$ le nombre complexe $z=a+ib$.On appelle \textbf{conjugué} de $z$, le nombre complexe
     \par
     \begin{center}
          $\overline{z}=a-ib$.
     \end{center}
}
\bloc{orange}{Exemple}{% id="e40"
     Soit $z=3+4i$
     \par
     Le conjugué de $z$ est $\overline{z}=3-4i$.
}
\cadre{vert}{Propriétés des conjugués}{% id="p50"
     Pour tous nombres complexes $z$ et $z^{\prime}$ et tout entier naturel $n$~:
     \begin{itemize}
          \item $\overline{z+z^{\prime}} = \overline{z}+\overline{z}^{\prime}$
          \item $\overline{zz^{\prime}} = \overline{z}\times \overline{z}^{\prime}$
          \item $\overline{\left(\frac{z}{z^{\prime}}\right)} = \frac{\overline{z}}{\overline{z^{\prime}}}  $ pour $z^{\prime}\neq 0$
          \item $\overline{\left(z^{n}\right)} = \left(\overline{z}\right)^{n}$.
     \end{itemize}
}
\bloc{cyan}{Remarques}{% id="r50"
     \begin{itemize}
          \item Par contre, en général, $|z+z^{\prime}|$ n'est pas égal à $|z|+|z^{\prime}|$. On peut juste montrer que  $|z+z^{\prime}| \leqslant  |z|+|z^{\prime}|$ (inégalité triangulaire)~;
          \item \textbf{ROC~:} La démonstration de certaines de ces propriétés a été demandée au \mcLien{/exercices/nombres-complexes/nombres-complexes-bac-s-metropole-2014}{Bac 2014}.
     \end{itemize}
}
\begin{h2}3. Équation du second degré à coefficients réels\end{h2}
\cadre{vert}{Propriété}{% id="p60"
     Soient $a$, $b$, $c $ trois réels avec $a\neq 0$.
     \par
     Dans $\mathbb{C}$, l'équation $az^{2}+bz+c=0$ admet toujours au moins une solution.
     \par
     Plus précisément, si on note $\Delta $ son discriminant ($\Delta =b^{2}-4ac$)~:
     \par
     \begin{itemize}
          \item Si $\Delta  > 0$, l'équation possède \textbf{deux solutions réelles}~:
          \par
          $z_{1}=\frac{-b-\sqrt{\Delta }}{2a}  $ et $  z_{2}=\frac{-b+\sqrt{\Delta }}{2a}$
          \par
          \item Si $\Delta  = 0$, l'équation possède \textbf{une solution réelle}~:
          \par
          $z=\frac{-b}{2a}  $
          \par
          \item Si $\Delta  < 0$, l'équation possède \textbf{deux solutions complexes} conjuguées l'une de l'autre~:
          \par
          $z_{1}=\frac{-b-i\sqrt{-\Delta }}{2a}  $ et $  z_{2}=\frac{-b+i\sqrt{-\Delta }}{2a}$.
     \end{itemize}
}
\bloc{orange}{Exemple}{% id="e60"
     Soit à résoudre l'équation $z^{2}+2z+2=0$ dans $\mathbb{C}$
     \par
     $\Delta =4-8=-4$
     \par
     $\Delta  < 0$ donc l'équation admet 2 racines complexes conjuguées~:
     \par
     $z_{1}=\frac{-2-i\sqrt{4}}{2}=-1-i  $ et $  z_{2}=\frac{-2+i\sqrt{4}}{2}=-1+i$.
}
\begin{h2}4. Représentation géométrique\end{h2}
Le plan $\left(P\right)$ est muni d'un repère orthonormé $\left(O; \vec{u}, \vec{v}\right)$
\cadre{bleu}{Définitions}{% id="d70"
     A tout nombre complexe $z=a+ib$, on associe le point $M$ de coordonnées $\left(a ; b\right)$
     \par
     On dit que $M$ est l'\textbf{image} de $z$ et que $z$ est l'\textbf{affixe} du point $M$.
     \par
     A tout vecteur $\vec{k}$ de coordonnées $\left(a ; b\right)$  on associe le nombre complexe $z=a+ib$.
     \par
     On dit que $z$ est l'\textbf{affixe} du vecteur $\vec{k}$.
}
\begin{center}
     \begin{extern} %width="450" alt="représentation graphique des nombres complexes"
          \newrgbcolor{dblue}{0. 0. 0.7}
          \newrgbcolor{dvert}{0. 0.4 0.}
          \newrgbcolor{dmauve}{0.5 0. 0.5}
          \resizebox{8cm}{!}{%
               % -+-+-+ variables modifiables
               \def\xmin{-1.2}
               \def\xmax{6.5}
               \def\ymin{-0.8}
               \def\ymax{5.5}
               \def\xunit{1.5}  % unités en cm
               \def\yunit{1.5}
               \psset{xunit=\xunit,yunit=\yunit,algebraic=true,arrowsize=3pt 2,arrowinset=0.25}
               \fontsize{15pt}{15pt}\selectfont
               \begin{pspicture*}[linewidth=1pt](\xmin,\ymin)(\xmax,\ymax)
                    %      \psgrid[gridcolor=mcgris, subgriddiv=5, gridlabels=0pt](\xmin,\ymin)(\xmax,\ymax)
                    %           \psaxes[labels=none,linewidth=0.75pt]{->}(0,0)(\xmin,\ymin)(\xmax,\ymax)
                    \psline[linewidth=0.75pt,linecolor=dmauve]{->}(0,0)(\xmin,0)(\xmax,0)
                    \psline[linewidth=0.75pt,linecolor=dvert]{->}(0,0)(0,\ymin)(0,\ymax)
                    \rput[tr](-0.1,-0.1){$O$}
                    \psdots[dotsize=2pt 0,dotstyle=*,linecolor=dblue](3,2)
                    \rput[bl](3,2){$\color{dblue} M(z=a+\text{i}b)$}
                    \psline[linewidth=0.75pt,linecolor=dblue]{->}(0,0)(3,2)
                    \psline[linestyle=dotted,linewidth=0.75pt,linecolor=dblue](3,0)(3,2)(0,2)
                    \rput[t](3,-0.1){\color{dblue} $a$}
                    \rput[r](-0.1,2){\color{dblue} $b$}
                    \rput[tr](6.4,-0.1){\color{dmauve} axe des réels}
                    \rput[tr]{90}(-0.8,5.3){\color{dvert} axe des}
                    \rput[tr]{90}(-0.45,5.3){\color{dvert} imaginaires purs}
                    \psline[linewidth=1.25pt,linecolor=dvert]{->}(0,0)(0,1)
                    \psline[linewidth=1.25pt,linecolor=dmauve]{->}(0,0)(1,0)
                    \rput[t](0.5,-0.03){\color{dmauve} $\vect{u}$}
                    \rput[r](-0.03,0.5){\color{dvert} $\vect{v}$}
               \end{pspicture*}
          }
     \end{extern}
\end{center}
\cadre{vert}{Propriétés}{% id="p80"
     \begin{itemize}
          \item $M$ appartient à l'axe des abscisses si et seulement si son affixe $z$ est un nombre réel
          \item $M$ appartient à l'axe des ordonnées si et seulement si son affixe $z$ est un nombre imaginaire pur
          \item Deux nombres complexes conjugués ont des affixes symétriques par rapport à l'axe des abscisses
     \end{itemize}
}
\begin{center}
     \begin{extern} %width="450" alt="nombres complexes conjugués"
          \newrgbcolor{dblue}{0. 0. 0.7}
          \newrgbcolor{dvert}{0. 0.4 0.}
          \newrgbcolor{dmauve}{0.5 0. 0.5}
          \resizebox{8cm}{!}{%
               % -+-+-+ variables modifiables
               \def\xmin{-1.2}
               \def\xmax{6.5}
               \def\ymin{-2.5}
               \def\ymax{3.2}
               \def\xunit{1.5}  % unités en cm
               \def\yunit{1.5}
               \psset{xunit=\xunit,yunit=\yunit,algebraic=true,arrowsize=3pt 2,arrowinset=0.25}
               \fontsize{15pt}{15pt}\selectfont
               \begin{pspicture*}[linewidth=1pt](\xmin,\ymin)(\xmax,\ymax)
                    %      \psgrid[gridcolor=mcgris, subgriddiv=5, gridlabels=0pt](\xmin,\ymin)(\xmax,\ymax)
                    \psaxes[ticks=none,labels=none,linewidth=0.75pt]{->}(0,0)(\xmin,\ymin)(\xmax,\ymax)
                    \rput[tr](-0.1,-0.1){$O$}
                    \psdots[dotsize=2pt 0,dotstyle=*,linecolor=dblue](3,2)
                    \rput[bl](3,2){$\color{dblue} M(z)$}
                    \psline[linewidth=0.75pt,linecolor=dvert]{->}(0,0)(3,-2)
                    \rput[bl](3,-2){$\color{dvert} M'(\bar{z})$}
                    \psline[linewidth=0.75pt,linecolor=dblue]{->}(0,0)(3,2)
                    \psline[linestyle=dotted,linewidth=0.75pt,linecolor=dblue](3,0)(3,2)(0,2)
                    \psline[linestyle=dotted,linewidth=0.75pt,linecolor=dvert](3,0)(3,-2)(0,-2)
                    \rput[t](3,-0.1){\color{dblue} $a$}
                    \rput[r](-0.1,2){\color{dblue} $b$}
                    \rput[r](-0.1,-2){\color{dvert} $-b$}
                    \psline[linewidth=1.25pt]{->}(0,0)(0,1)
                    \psline[linewidth=1.25pt]{->}(0,0)(1,0)
                    \rput[t](0.5,-0.03){$\vect{u}$}
                    \rput[r](-0.03,0.5){ $\vect{v}$}
               \end{pspicture*}
          }
     \end{extern}
\end{center}
\cadre{vert}{Propriétés}{% id="p90"
     Soient $A$ et $B$ deux points d'affixes respectives $z_{A}$ et $z_{B}$.
     \begin{itemize}
          \item  l'affixe du vecteur $\overrightarrow{AB}$ est égale à~:
          \begin{center}$z_{\overrightarrow{AB}}= z_{B}-z_{A}$\end{center}
          \item  l'affixe du milieu $M$ du segment $\left[AB\right]$ est égale à~:
          \begin{center}$z_{M}= \frac{z_{A}+z_{B}}{2}$\end{center}
     \end{itemize}
}
\cadre{vert}{Propriétés}{% id="p100"
     Soient $\vec{w}\left(z\right)$ et $\overrightarrow{w^{\prime}}\left(z^{\prime}\right)$ deux vecteurs du plan et $k$ un nombre réel.
     \begin{itemize}
          \item Le vecteur $\vec{w}+\overrightarrow{w^{\prime}}$ a pour affixe $z+z^{\prime}$~;
          \item Le vecteur $k\vec{w}$ a pour affixe $kz$.
     \end{itemize}
}
\begin{h2}5. Forme trigonométrique\end{h2}
\cadre{bleu}{Définition}{% id="d110"
     Soit $z$ un nombre complexe \textbf{non nul} d'image $M$ dans le repère $\left(O; \vec{u}, \vec{v}\right)$.
     \par
     On appelle module de $z$, et on note $|z|$ le nombre \textbf{réel} positif ou nul $|z|=\sqrt{a^{2}+b^{2}}$.
     \par
     On appelle argument de $z$ et on note $\text{arg}\left(z\right)$ une mesure, exprimée en radians, de l'angle
     \par
     $\left(\vec{u}; \overrightarrow{OM}\right)$.
}
\begin{center}
     \begin{extern} %width="450" alt="forme trigonométrique des nombres complexes"
          \newrgbcolor{dblue}{0. 0. 0.7}
          \newrgbcolor{dvert}{0. 0.4 0.}
          \newrgbcolor{dmauve}{0.5 0. 0.5}
          \resizebox{8cm}{!}{%
               % -+-+-+ variables modifiables
               \def\xmin{-1.2}
               \def\xmax{6.5}
               \def\ymin{-0.8}
               \def\ymax{3.2}
               \def\xunit{1.5}  % unités en cm
               \def\yunit{1.5}
               \psset{xunit=\xunit,yunit=\yunit,algebraic=true,arrowsize=3pt 2,arrowinset=0.25}
               \fontsize{15pt}{15pt}\selectfont
               \begin{pspicture*}[linewidth=1pt](\xmin,\ymin)(\xmax,\ymax)
                    %      \psgrid[gridcolor=mcgris, subgriddiv=5, gridlabels=0pt](\xmin,\ymin)(\xmax,\ymax)
                    \psaxes[ticks=none,labels=none,linewidth=0.75pt]{->}(0,0)(\xmin,\ymin)(\xmax,\ymax)
                    \rput[tr](-0.1,-0.1){$O$}
                    \psdots[dotsize=2pt 0,dotstyle=*,linecolor=dblue](3,2)
                    \rput[bl](3,2){$\color{dblue} M(z)$}
                    \psline[linewidth=0.75pt,linecolor=dvert](0,0)(3,2)
                    \rput[t]{33.7}(1.3,1.3){$\color{dvert} |z|=OM$}
                    \psline[linestyle=dotted,linewidth=0.75pt,linecolor=dblue](3,0)(3,2)(0,2)
                    \rput[t](3,-0.1){\color{dblue} $a$}
                    \rput[r](-0.1,2){\color{dblue} $b$}
                    \pscustom[linewidth=0.8pt,linecolor=dmauve,fillcolor=dmauve,fillstyle=solid,opacity=0.1]{ % color angle
                         \parametricplot{0.0}{0.588}{0.6*cos(t)|0.6*sin(t)}
                         \lineto(0.,0.)
                    \closepath}
                    \psellipticarc[linewidth=0.8pt,linecolor=dmauve,arrows=->](0.,0.)(0.6,0.6){0.}{33.7} % fleche angle
                    \rput[t](1.4,0.38){$\color{dmauve} \theta=\text{arg}(z)$}
                    \psline[linewidth=1.25pt]{->}(0,0)(0,1)
                    \psline[linewidth=1.25pt]{->}(0,0)(1,0)
                    \rput[t](0.5,-0.03){$\vect{u}$}
                    \rput[r](-0.03,0.5){$\vect{v}$}
               \end{pspicture*}
          }
     \end{extern}
\end{center}
\cadre{vert}{Propriétés des modules}{% id="p130"
     Pour tous nombres complexes $z$ et $z^{\prime}$~:
     \begin{itemize}
          \item $|z|^{2} = z\times \overline{z}$
          \item $|zz^{\prime}| = |z|\times |z^{\prime}|$
          \item $|\frac{z}{z^{\prime}}| = \frac{|z|}{|z^{\prime}|}  $ pour $z^{\prime}\neq 0$
     \end{itemize}
}
\cadre{vert}{Propriétés des arguments}{% id="p140"
     Pour tous nombres complexes $z$ et $z^{\prime}$ \textbf{non nuls} et tout entier $n\in \mathbb{Z}$~:
     \begin{itemize}
          \item $\text{arg}\left(\overline{z}\right)=-\text{arg}\left(z\right)$
          \item $\text{arg}\left(zz^{\prime}\right)=\text{arg}\left(z\right)+\text{arg}\left(z^{\prime}\right)$
          \item $\text{arg}\left(z^{n}\right)=n\times \text{arg}\left(z\right)$
          \item $\text{arg}\left(\frac{z}{z^{\prime}}\right)=\text{arg}\left(z\right)-\text{arg}\left(z^{\prime}\right)$
     \end{itemize}
}
\bloc{cyan}{Remarque}{% id="r140"
     En particulier~:
     \begin{itemize}
          \item $\text{arg}\left(-z\right)=\text{arg}\left(z\right)+\text{arg}\left(-1\right) = \text{arg}\left(z\right)+\pi $
          \item $\text{arg}\left(\frac{1}{z}\right)=\text{arg}\left(1\right)-\text{arg}\left(z\right) = -\text{arg}\left(z\right)$.
     \end{itemize}
}
\cadre{rouge}{Théorème et définition}{% id="t170"
     Soit $z$ un nombre complexe non nul de module $r$ et d'argument $\theta $~:
     \begin{center}$z=r\left(\cos\theta  + i \sin\theta \right)$\end{center}
     Cette écriture s'appelle \textbf{forme trigonométrique} du nombre $z$.
}
\cadre{vert}{Passage de la forme algébrique à la forme trigonométrique}{% id="p180"
     Soit $z=a+ib$ un nombre complexe non nul.
     \begin{itemize}
          \item $r=|z|=\sqrt{a^{2}+b^{2}}$
          \item $\theta =\text{arg}\left(z\right)$ est défini par~:
          \par
          $\cos \theta  = \frac{a}{\sqrt{a^{2}+b^{2}}}$ et $ \sin \theta  = \frac{b}{\sqrt{a^{2}+b^{2}}}$.
     \end{itemize}
}
\bloc{orange}{Exemple}{% id="e180"
     Soit $z=\sqrt{3}+i$.
     \par
     $|z|=\sqrt{3+1}=2$
     \par
     Si $\theta $ est un argument de $z$~:
     \par
     $\cos \theta =\frac{\sqrt{3}}{2} $ et $ \sin \theta =\frac{1}{2}$ donc $\theta =\frac{\pi }{6}   \left(\text{mod. } 2\pi \right)$
     \par
     La forme trigonométrique de $z$ est donc~:
     \par
     $z=2\left(\cos \frac{\pi }{6} + i \sin \frac{\pi }{6}\right) $.
     \begin{center}
          \begin{extern} %width="380" alt="représentation graphique du nombre complexe racine de 3 + i"
               \newrgbcolor{dblue}{0. 0. 0.7}
               \newrgbcolor{dvert}{0. 0.4 0.}
               \newrgbcolor{dmauve}{0.5 0. 0.5}
               \resizebox{7cm}{!}{%
                    % -+-+-+ variables modifiables
                    \def\xmin{-1.2}
                    \def\xmax{3}
                    \def\ymin{-0.8}
                    \def\ymax{2}
                    \def\xunit{2.5}  % unités en cm
                    \def\yunit{2.5}
                    \psset{xunit=\xunit,yunit=\yunit,algebraic=true,arrowsize=3pt 2,arrowinset=0.25}
                    \fontsize{15pt}{15pt}\selectfont
                    \begin{pspicture*}[linewidth=1pt](\xmin,\ymin)(\xmax,\ymax)
                         %      \psgrid[gridcolor=mcgris, subgriddiv=5, gridlabels=0pt](\xmin,\ymin)(\xmax,\ymax)
                         \psaxes[ticks=none,labels=none,linewidth=0.75pt]{->}(0,0)(\xmin,\ymin)(\xmax,\ymax)
                         \rput[tr](-0.1,-0.1){$O$}
                         \psdots[dotsize=2pt 0,dotstyle=*,linecolor=dblue](1.732,1)
                         \rput[bl](1.732,1){$\color{dblue} M(\sqrt{3}+\text{i})$}
                         \psline[linewidth=0.75pt,linecolor=dvert](0,0)(1.732,1)
                         \rput[t]{30}(0.8,0.7){$\color{dvert} |z|=2$}
                         \psline[linestyle=dotted,linewidth=0.75pt,linecolor=dblue](1.732,0)(1.732,1)(0,1)
                         \rput[t](1.732,-0.1){\color{dblue} $\sqrt{3}$}
                         \rput[r](-0.1,1){\color{dblue} $1$}
                         \pscustom[linewidth=0.8pt,linecolor=dmauve,fillcolor=dmauve,fillstyle=solid,opacity=0.1]{ % color angle
                              \parametricplot{0.0}{0.524}{0.5*cos(t)|0.5*sin(t)}
                              \lineto(0.,0.)
                         \closepath}
                         \psellipticarc[linewidth=0.8pt,linecolor=dmauve,arrows=->](0.,0.)(0.5,0.5){0.}{30} % fleche angle
                         \rput[t](0.9,0.42){$\color{dmauve} \theta=\dfrac{\pi}{6}$}
                         \psline[linewidth=1.25pt]{->}(0,0)(0,1)
                         \psline[linewidth=1.25pt]{->}(0,0)(1,0)
                         \rput[t](0.5,-0.03){$\vect{u}$}
                         \rput[r](-0.03,0.5){$\vect{v}$}
                    \end{pspicture*}
               }
          \end{extern}
     \end{center}
}
\cadre{vert}{Angle de vecteurs et arguments}{% id="p190"
     Soit $A, B$ et $C$ trois points du plan d'afixes respectives $z_{A}$,$z_{B}$, $z_{C}$ avec $A\neq B$ et $A\neq C$~:
     \par
     $\left(\overrightarrow{AB};\overrightarrow{AC}\right)= \text{arg}\left(\frac{z_{C}-z_{A}}{z_{B}-z_{A}}\right)$.
}
\begin{center}
     \begin{extern} %width="450" alt="forme trigonométrique des nombres complexes"
          \newrgbcolor{dblue}{0. 0. 0.7}
          \newrgbcolor{dvert}{0. 0.4 0.}
          \newrgbcolor{dmauve}{0.5 0. 0.5}
          \resizebox{8cm}{!}{%
               % -+-+-+ variables modifiables
               \def\xmin{-1.2}
               \def\xmax{6.5}
               \def\ymin{-0.8}
               \def\ymax{4.2}
               \def\xunit{1.5}  % unités en cm
               \def\yunit{1.5}
               \psset{xunit=\xunit,yunit=\yunit,algebraic=true,arrowsize=3pt 2,arrowinset=0.25}
               \fontsize{15pt}{15pt}\selectfont
               \begin{pspicture*}[linewidth=1pt](\xmin,\ymin)(\xmax,\ymax)
                    %      \psgrid[gridcolor=mcgris, subgriddiv=5, gridlabels=0pt](\xmin,\ymin)(\xmax,\ymax)
                    \psaxes[ticks=none,labels=none,linewidth=0.75pt]{->}(0,0)(\xmin,\ymin)(\xmax,\ymax)
                    \rput[tr](-0.1,-0.1){$O$}
                    \psdots[dotsize=2pt 0,dotstyle=*,linecolor=dblue](1,1)
                    \rput[tr](0.9,0.9){$\color{dblue} A$}
                    \psdots[dotsize=2pt 0,dotstyle=*,linecolor=dblue](5,2)
                    \rput[bl](5,2.1){$\color{dblue} B$}
                    \psdots[dotsize=2pt 0,dotstyle=*,linecolor=dblue](2,3)
                    \rput[bl](2,3.1){$\color{dblue} C$}
                    \psline[linewidth=0.75pt,linecolor=dblue]{->}(1,1)(5,2)
                    \psline[linewidth=0.75pt,linecolor=dblue]{->}(1,1)(2,3)
                    \pscustom[linewidth=0.8pt,linecolor=dmauve,fillcolor=dmauve,fillstyle=solid,opacity=0.1]{ % color angle
                         \parametricplot{0.25}{1.11}{0.6*cos(t)+1|0.6*sin(t)+1}
                         \lineto(1,1)
                    \closepath}
                    \psellipticarc[linewidth=0.8pt,linecolor=dmauve,arrows=->](1.,1.)(0.6,0.6){14.}{63.4} % fleche angle
                    \rput[t](2.9,2.6){$\color{dmauve} \theta=\text{arg}\left(\frac{z_{C}-z_{A}}{z_{B}-z_{A}}\right)$}
                    \psline[linewidth=1.25pt]{->}(0,0)(0,1)
                    \psline[linewidth=1.25pt]{->}(0,0)(1,0)
                    \rput[t](0.5,-0.03){$\vect{u}$}
                    \rput[r](-0.03,0.5){$\vect{v}$}
               \end{pspicture*}
          }
     \end{extern}
\end{center}
\bloc{cyan}{Remarques}{% id="r190"
     \begin{itemize}
          \item Notez bien l'\textbf{ordre des affixes} (inverse de l'ordre des points dans l'écriture de l'angle).
          \item \textbf{Premier cas particulier important~:}
          \par
          $A, B$ et $C$ sont alignés \\
          $\phantom{A, B} \Leftrightarrow  \text{arg}\left(\frac{z_{C}-z_{A}}{z_{B}-z_{A}}\right) = 0~\text{ou}~\pi~\left[\text{mod. } 2\pi \right] $ \\
          $\phantom{A, B} \Leftrightarrow  \frac{z_{C}-z_{A}}{z_{B}-z_{A}} \in  \mathbb{R}$.
          \item \textbf{Second cas particulier important~:}
          \par
          $\widehat{BAC}$ est un angle droit \\
          $\phantom{A, B} \Leftrightarrow  \text{arg}\left(\frac{z_{C}-z_{A}}{z_{B}-z_{A}}\right) = \pm \frac{\pi }{2} ~  \left[\text{mod. } 2\pi \right] $\\
          $\phantom{A, B} \Leftrightarrow  \frac{z_{C}-z_{A}}{z_{B}-z_{A}}$ est un \textbf{imaginaire pur}.
     \end{itemize}
}
\begin{h2}6. Forme exponentielle\end{h2}
\cadre{bleu}{Notation}{% id="d200"
     Si $z$ est un nombre complexe de module $r$ et d'argument $\theta $, la notation exponentielle du nombre $z$ est~:
     \begin{center}$z=re^{i\theta }$\end{center}
}
\bloc{cyan}{Remarque}{% id="r200"
     Ce sont les propriétés des arguments~:
     \begin{itemize}
          \item $\text{arg}\left(zz^{\prime}\right)=\text{arg}\left(z\right)+\text{arg}\left(z^{\prime}\right)$
          \item $\text{arg}\left(z^{n}\right)=n\times \text{arg}\left(z\right)$
          \item $\text{arg}\left(\frac{z}{z^{\prime}}\right)=\text{arg}\left(z\right)-\text{arg}\left(z^{\prime}\right)$
     \end{itemize}
     similaires aux propriétés de l'exponentielle qui justifient cette notation.
}
\bloc{orange}{Exemple}{% id="e200"
     Le nombre $-1$ a pour module $1$ et pour argument $\pi   \left(\text{mod. } 2\pi \right)$. On peut donc écrire~:
     \par
     $-1=e^{i\pi }$ ou encore $e^{i\pi }+1=0$.
     \par
     C'est la célèbre \Lien{http://fr.wikipedia.org/wiki/Identit\%C3\%A9_d\%27Euler}{identité d'Euler} qui relie $0$, $1$, $e$, $i$ et $\pi $.
}
Les propriétés des arguments vues précédemment s'écrivent alors~:
\cadre{vert}{Propriétés}{% id="p210"
     Pour tous réels $\theta $ et $\theta ^{\prime}$~:
     \begin{itemize}
          \item $e^{i\theta }\times e^{i\theta ^{\prime}}=e^{i\left(\theta +\theta ^{\prime}\right)}$
          \item $\left(e^{i\theta }\right)^{n}=e^{in\theta }$
          \item $\frac{e^{i\theta }}{e^{i\theta ^{\prime}}}=e^{i\left(\theta -\theta ^{\prime}\right)}$.
     \end{itemize}
}

\end{document}
µ
\documentclass[a4paper]{article}

%================================================================================================================================
%
% Packages
%
%================================================================================================================================

\usepackage[T1]{fontenc} 	% pour caractères accentués
\usepackage[utf8]{inputenc}  % encodage utf8
\usepackage[french]{babel}	% langue : français
\usepackage{fourier}			% caractères plus lisibles
\usepackage[dvipsnames]{xcolor} % couleurs
\usepackage{fancyhdr}		% réglage header footer
\usepackage{needspace}		% empêcher sauts de page mal placés
\usepackage{graphicx}		% pour inclure des graphiques
\usepackage{enumitem,cprotect}		% personnalise les listes d'items (nécessaire pour ol, al ...)
\usepackage{hyperref}		% Liens hypertexte
\usepackage{pstricks,pst-all,pst-node,pstricks-add,pst-math,pst-plot,pst-tree,pst-eucl} % pstricks
\usepackage[a4paper,includeheadfoot,top=2cm,left=3cm, bottom=2cm,right=3cm]{geometry} % marges etc.
\usepackage{comment}			% commentaires multilignes
\usepackage{amsmath,environ} % maths (matrices, etc.)
\usepackage{amssymb,makeidx}
\usepackage{bm}				% bold maths
\usepackage{tabularx}		% tableaux
\usepackage{colortbl}		% tableaux en couleur
\usepackage{fontawesome}		% Fontawesome
\usepackage{environ}			% environment with command
\usepackage{fp}				% calculs pour ps-tricks
\usepackage{multido}			% pour ps tricks
\usepackage[np]{numprint}	% formattage nombre
\usepackage{tikz,tkz-tab} 			% package principal TikZ
\usepackage{pgfplots}   % axes
\usepackage{mathrsfs}    % cursives
\usepackage{calc}			% calcul taille boites
\usepackage[scaled=0.875]{helvet} % font sans serif
\usepackage{svg} % svg
\usepackage{scrextend} % local margin
\usepackage{scratch} %scratch
\usepackage{multicol} % colonnes
%\usepackage{infix-RPN,pst-func} % formule en notation polanaise inversée
\usepackage{listings}

%================================================================================================================================
%
% Réglages de base
%
%================================================================================================================================

\lstset{
language=Python,   % R code
literate=
{á}{{\'a}}1
{à}{{\`a}}1
{ã}{{\~a}}1
{é}{{\'e}}1
{è}{{\`e}}1
{ê}{{\^e}}1
{í}{{\'i}}1
{ó}{{\'o}}1
{õ}{{\~o}}1
{ú}{{\'u}}1
{ü}{{\"u}}1
{ç}{{\c{c}}}1
{~}{{ }}1
}


\definecolor{codegreen}{rgb}{0,0.6,0}
\definecolor{codegray}{rgb}{0.5,0.5,0.5}
\definecolor{codepurple}{rgb}{0.58,0,0.82}
\definecolor{backcolour}{rgb}{0.95,0.95,0.92}

\lstdefinestyle{mystyle}{
    backgroundcolor=\color{backcolour},   
    commentstyle=\color{codegreen},
    keywordstyle=\color{magenta},
    numberstyle=\tiny\color{codegray},
    stringstyle=\color{codepurple},
    basicstyle=\ttfamily\footnotesize,
    breakatwhitespace=false,         
    breaklines=true,                 
    captionpos=b,                    
    keepspaces=true,                 
    numbers=left,                    
xleftmargin=2em,
framexleftmargin=2em,            
    showspaces=false,                
    showstringspaces=false,
    showtabs=false,                  
    tabsize=2,
    upquote=true
}

\lstset{style=mystyle}


\lstset{style=mystyle}
\newcommand{\imgdir}{C:/laragon/www/newmc/assets/imgsvg/}
\newcommand{\imgsvgdir}{C:/laragon/www/newmc/assets/imgsvg/}

\definecolor{mcgris}{RGB}{220, 220, 220}% ancien~; pour compatibilité
\definecolor{mcbleu}{RGB}{52, 152, 219}
\definecolor{mcvert}{RGB}{125, 194, 70}
\definecolor{mcmauve}{RGB}{154, 0, 215}
\definecolor{mcorange}{RGB}{255, 96, 0}
\definecolor{mcturquoise}{RGB}{0, 153, 153}
\definecolor{mcrouge}{RGB}{255, 0, 0}
\definecolor{mclightvert}{RGB}{205, 234, 190}

\definecolor{gris}{RGB}{220, 220, 220}
\definecolor{bleu}{RGB}{52, 152, 219}
\definecolor{vert}{RGB}{125, 194, 70}
\definecolor{mauve}{RGB}{154, 0, 215}
\definecolor{orange}{RGB}{255, 96, 0}
\definecolor{turquoise}{RGB}{0, 153, 153}
\definecolor{rouge}{RGB}{255, 0, 0}
\definecolor{lightvert}{RGB}{205, 234, 190}
\setitemize[0]{label=\color{lightvert}  $\bullet$}

\pagestyle{fancy}
\renewcommand{\headrulewidth}{0.2pt}
\fancyhead[L]{maths-cours.fr}
\fancyhead[R]{\thepage}
\renewcommand{\footrulewidth}{0.2pt}
\fancyfoot[C]{}

\newcolumntype{C}{>{\centering\arraybackslash}X}
\newcolumntype{s}{>{\hsize=.35\hsize\arraybackslash}X}

\setlength{\parindent}{0pt}		 
\setlength{\parskip}{3mm}
\setlength{\headheight}{1cm}

\def\ebook{ebook}
\def\book{book}
\def\web{web}
\def\type{web}

\newcommand{\vect}[1]{\overrightarrow{\,\mathstrut#1\,}}

\def\Oij{$\left(\text{O}~;~\vect{\imath},~\vect{\jmath}\right)$}
\def\Oijk{$\left(\text{O}~;~\vect{\imath},~\vect{\jmath},~\vect{k}\right)$}
\def\Ouv{$\left(\text{O}~;~\vect{u},~\vect{v}\right)$}

\hypersetup{breaklinks=true, colorlinks = true, linkcolor = OliveGreen, urlcolor = OliveGreen, citecolor = OliveGreen, pdfauthor={Didier BONNEL - https://www.maths-cours.fr} } % supprime les bordures autour des liens

\renewcommand{\arg}[0]{\text{arg}}

\everymath{\displaystyle}

%================================================================================================================================
%
% Macros - Commandes
%
%================================================================================================================================

\newcommand\meta[2]{    			% Utilisé pour créer le post HTML.
	\def\titre{titre}
	\def\url{url}
	\def\arg{#1}
	\ifx\titre\arg
		\newcommand\maintitle{#2}
		\fancyhead[L]{#2}
		{\Large\sffamily \MakeUppercase{#2}}
		\vspace{1mm}\textcolor{mcvert}{\hrule}
	\fi 
	\ifx\url\arg
		\fancyfoot[L]{\href{https://www.maths-cours.fr#2}{\black \footnotesize{https://www.maths-cours.fr#2}}}
	\fi 
}


\newcommand\TitreC[1]{    		% Titre centré
     \needspace{3\baselineskip}
     \begin{center}\textbf{#1}\end{center}
}

\newcommand\newpar{    		% paragraphe
     \par
}

\newcommand\nosp {    		% commande vide (pas d'espace)
}
\newcommand{\id}[1]{} %ignore

\newcommand\boite[2]{				% Boite simple sans titre
	\vspace{5mm}
	\setlength{\fboxrule}{0.2mm}
	\setlength{\fboxsep}{5mm}	
	\fcolorbox{#1}{#1!3}{\makebox[\linewidth-2\fboxrule-2\fboxsep]{
  		\begin{minipage}[t]{\linewidth-2\fboxrule-4\fboxsep}\setlength{\parskip}{3mm}
  			 #2
  		\end{minipage}
	}}
	\vspace{5mm}
}

\newcommand\CBox[4]{				% Boites
	\vspace{5mm}
	\setlength{\fboxrule}{0.2mm}
	\setlength{\fboxsep}{5mm}
	
	\fcolorbox{#1}{#1!3}{\makebox[\linewidth-2\fboxrule-2\fboxsep]{
		\begin{minipage}[t]{1cm}\setlength{\parskip}{3mm}
	  		\textcolor{#1}{\LARGE{#2}}    
 	 	\end{minipage}  
  		\begin{minipage}[t]{\linewidth-2\fboxrule-4\fboxsep}\setlength{\parskip}{3mm}
			\raisebox{1.2mm}{\normalsize\sffamily{\textcolor{#1}{#3}}}						
  			 #4
  		\end{minipage}
	}}
	\vspace{5mm}
}

\newcommand\cadre[3]{				% Boites convertible html
	\par
	\vspace{2mm}
	\setlength{\fboxrule}{0.1mm}
	\setlength{\fboxsep}{5mm}
	\fcolorbox{#1}{white}{\makebox[\linewidth-2\fboxrule-2\fboxsep]{
  		\begin{minipage}[t]{\linewidth-2\fboxrule-4\fboxsep}\setlength{\parskip}{3mm}
			\raisebox{-2.5mm}{\sffamily \small{\textcolor{#1}{\MakeUppercase{#2}}}}		
			\par		
  			 #3
 	 		\end{minipage}
	}}
		\vspace{2mm}
	\par
}

\newcommand\bloc[3]{				% Boites convertible html sans bordure
     \needspace{2\baselineskip}
     {\sffamily \small{\textcolor{#1}{\MakeUppercase{#2}}}}    
		\par		
  			 #3
		\par
}

\newcommand\CHelp[1]{
     \CBox{Plum}{\faInfoCircle}{À RETENIR}{#1}
}

\newcommand\CUp[1]{
     \CBox{NavyBlue}{\faThumbsOUp}{EN PRATIQUE}{#1}
}

\newcommand\CInfo[1]{
     \CBox{Sepia}{\faArrowCircleRight}{REMARQUE}{#1}
}

\newcommand\CRedac[1]{
     \CBox{PineGreen}{\faEdit}{BIEN R\'EDIGER}{#1}
}

\newcommand\CError[1]{
     \CBox{Red}{\faExclamationTriangle}{ATTENTION}{#1}
}

\newcommand\TitreExo[2]{
\needspace{4\baselineskip}
 {\sffamily\large EXERCICE #1\ (\emph{#2 points})}
\vspace{5mm}
}

\newcommand\img[2]{
          \includegraphics[width=#2\paperwidth]{\imgdir#1}
}

\newcommand\imgsvg[2]{
       \begin{center}   \includegraphics[width=#2\paperwidth]{\imgsvgdir#1} \end{center}
}


\newcommand\Lien[2]{
     \href{#1}{#2 \tiny \faExternalLink}
}
\newcommand\mcLien[2]{
     \href{https~://www.maths-cours.fr/#1}{#2 \tiny \faExternalLink}
}

\newcommand{\euro}{\eurologo{}}

%================================================================================================================================
%
% Macros - Environement
%
%================================================================================================================================

\newenvironment{tex}{ %
}
{%
}

\newenvironment{indente}{ %
	\setlength\parindent{10mm}
}

{
	\setlength\parindent{0mm}
}

\newenvironment{corrige}{%
     \needspace{3\baselineskip}
     \medskip
     \textbf{\textsc{Corrigé}}
     \medskip
}
{
}

\newenvironment{extern}{%
     \begin{center}
     }
     {
     \end{center}
}

\NewEnviron{code}{%
	\par
     \boite{gray}{\texttt{%
     \BODY
     }}
     \par
}

\newenvironment{vbloc}{% boite sans cadre empeche saut de page
     \begin{minipage}[t]{\linewidth}
     }
     {
     \end{minipage}
}
\NewEnviron{h2}{%
    \needspace{3\baselineskip}
    \vspace{0.6cm}
	\noindent \MakeUppercase{\sffamily \large \BODY}
	\vspace{1mm}\textcolor{mcgris}{\hrule}\vspace{0.4cm}
	\par
}{}

\NewEnviron{h3}{%
    \needspace{3\baselineskip}
	\vspace{5mm}
	\textsc{\BODY}
	\par
}

\NewEnviron{margeneg}{ %
\begin{addmargin}[-1cm]{0cm}
\BODY
\end{addmargin}
}

\NewEnviron{html}{%
}

\begin{document}
\meta{url}{/cours/variations-convergence-suite/}
\meta{pid}{543}
\meta{titre}{Suites et récurrence}
\meta{type}{cours}
\begin{h2}I - Démonstration par récurrence\end{h2}
\cadre{rouge}{Théorème}{%id="t10"
     Soit $P\left(n\right)$ une proposition qui dépend d'un entier naturel $n$.
     \begin{itemize}
          \item Si $P\left(n_{0}\right)$ est vraie \textbf{(initialisation)}
          \item Et si $P\left(n\right)$ vraie entraîne $P\left(n+1\right)$ vraie \textbf{(hérédité)}
     \end{itemize}
     alors la propriété $P\left(n\right)$ est vraie pour tout entier $n\geqslant n_{0}$
}
\bloc{vert}{Remarques}{%id="r10"
     \begin{itemize}
          \item La démonstration par récurrence s'apparente au "principe des dominos" :
          \begin{center}
               \img{dominos}{0.4}%width="450" alt="dominos et récurrence"
          \end{center}
          \item L'étape d'initialisation est souvent facile à démontrer~; toutefois, faites attention à \textbf{ne pas l'oublier}~!
          \item Pour prouver l'hérédité, on suppose que la propriété est vraie \textbf{pour un certain entier $n$} (cette supposition est appelée \textbf{hypothèse de récurrence}) et on démontre qu'elle est alors vraie pour l'entier $n+1$. Pour cela, il est conseillé d'écrire ce que signifie $P\left(n+1\right)$ (que l'on souhaite démontrer), en remplaçant $n$ par $n+$1 dans la propriété $P\left(n\right)$
     \end{itemize}
}
\bloc{orange}{Exemple}{%id="e10"
     Montrons que pour tout entier n strictement positif $1+2+. . .+n=\frac{n\left(n+1\right)}{2}$.
     \bigbreak
     \textbf{Initialisation}
     \medbreak
     On commence à $n_{0}=1$ car l'énoncé précise "strictement positif".
     \par
     La proposition devient :
     \par
     $1=\frac{1\times 2}{2}$
     \par
     ce qui est vrai.
     \bigbreak
     \textbf{Hérédité}
     \medbreak
     On suppose que pour un certain entier $n$:
     \par
     $1+2+. . .+n=\frac{n\left(n+1\right)}{2}$ (\textbf{Hypothèse de récurrence})
     \par
     et on va montrer qu'alors :
     \par
     $1+2+. . .+n+1=\frac{\left(n+1\right)\left(n+2\right)}{2}$ (on a remplacé $n$ par $n+1$ dans la formule que l'on souhaite prouver).
     \par
     Isolons le dernier terme de notre somme
     \par
     $1+2+. . .+n+1=\left(1+2+. . . +n\right) + n+1$
     \par
     On applique maintenant notre hypothèse de récurrence à $1+2+. . . +n$:
     \par
     $1+2+. . .+n+1=\frac{n\left(n+1\right)}{2}+n+1=\frac{n\left(n+1\right)}{2}+\frac{2\left(n+1\right)}{2}=\frac{n\left(n+1\right)+2\left(n+1\right)}{2}$
     \par
     $1+2+. . .+n+1=\frac{\left(n+1\right)\left(n+2\right)}{2}$
     \par
     ce qui correspond bien à ce que nous voulions montrer.
     \medbreak
     En conclusion nous avons bien prouvé que pour pour tout entier n strictement positif :
     \par
     $1+2+. . .+n=\frac{n\left(n+1\right)}{2}$.
}
\begin{h2}II - Sens de variation - Suites majorées, minorées\end{h2}
\cadre{bleu}{Définitions (rappel)}{%id="d20"
     \begin{itemize}
          \item On dit que la suite $\left(u_{n}\right)$ est \textbf{croissante} si pour tout entier naturel $n$ : $u_{n+1} \geqslant u_{n}$
          \item On dit que la suite $\left(u_{n}\right)$ est \textbf{strictement croissante} si pour tout entier naturel $n$ : $u_{n+1} > u_{n}$
          \item On dit que la suite $\left(u_{n}\right)$ est \textbf{décroissante} si pour tout entier naturel $n$ : $u_{n+1} \leqslant u_{n}$
          \item On dit que la suite $\left(u_{n}\right)$ est \textbf{strictement décroissante} si pour tout entier naturel $n$ : $u_{n+1} < u_{n}$
          \item On dit que la suite $\left(u_{n}\right)$ est \textbf{constante} si pour tout entier naturel $n$ : $u_{n+1} = u_{n}$
     \end{itemize}
}
\cadre{bleu}{Définitions}{%id="d30"
     \begin{itemize}
          \item On dit que la suite $\left(u_{n}\right)$ est \textbf{majorée} par le réel $M$ si tout entier naturel $n$ : $u_{n} \leqslant M$.
          \par
          $M$ s'appelle alors un \textbf{majorant} de la suite $\left(u_{n}\right)$
          \item On dit que la suite $\left(u_{n}\right)$ est \textbf{minorée} par le réel $m$ si pour tout entier naturel $n$ : $u_{n} \geqslant m$.
          \par
          $m$ s'appelle un \textbf{minorant} de la suite $\left(u_{n}\right)$
     \end{itemize}
}
\bloc{cyan}{Remarque}{%id="r30"
     Si la suite $\left(u_{n}\right)$ est majorée (ou minorée), les majorants (ou minorants) \textbf{ne sont pas uniques}. Bien au contraire, si $M$ est un majorant de la suite $\left(u_{n}\right)$, tout réel supérieur à $M$ est aussi un majorant de la suite $\left(u_{n}\right)$
}
\bloc{orange}{Exemple}{%id="e30"
     Soit la suite $\left(u_{n}\right)$ définie par :
     \par
     $\left\{ \begin{matrix} u_{0}=1 \\ u_{n+1} =u_{n}^{2}+1 \end{matrix}\right.\text{pour tout} n \in \mathbb{N}$
     \par
     On vérifie aisément que pour tout $n \in \mathbb{N}$, $u_{n}$ est supérieur ou égal à $1$ donc la suite $\left(u_{n}\right)$ est minorée par $1$. Par contre cette suite n'est pas majorée (on peut, par exemple, démonter par récurrence que pour tout $n \in \mathbb{N}$ $u_{n} > n$.
}
\begin{h2}III - Convergence - Limite\end{h2}
\cadre{bleu}{Définition}{%id="d40"
     On dit que la suite $(u_{n})$ \textbf{converge} vers le nombre réel $l$ (ou \textbf{admet pour limite} le nombre réel $l$) si tout intervalle ouvert contenant $l$ contient tous les termes de la suite à partir d'un certain rang.
     \par
     On note alors $\lim\limits_{n\rightarrow +\infty }u_{n}=l$
}
\bloc{orange}{Exemple}{%id="e40"
     \begin{center}
          \begin{extern}%width="600" alt="suite convergente"
               % -+-+-+ variables modifiables
               \resizebox{8cm}{!}{%
                    \def\xmin{-1}
                    \def\xmax{13.5}
                    \def\ymin{-2}
                    \def\ymax{3.8}
                    \def\xunit{1}  % unités en cm
                    \def\yunit{1}
                    \psset{xunit=\xunit,yunit=\yunit,algebraic=true}
                    \fontsize{15pt}{15pt}\selectfont
                    \begin{pspicture*}[linewidth=1pt](\xmin,\ymin)(\xmax,\ymax)
                         \psaxes[Dx=100,Dy=100,linewidth=0.75pt]{->}(0,0)(\xmin,\ymin)(\xmax,\ymax)
                         \rput[tr](-0.1,-0.1){$O$}
                         \multido{\n=0.0+1}{15}{
                              \FPeval{\suite}{3*cos(\n*3.14159)/(\n+1)+2}
                              \psdots[linecolor=blue](\n,\suite)
                         }
                         \psline[linecolor=red,linewidth=0.5pt](-3,2)(16,2)
                         \rput[br](-0.1,2.1){$\red l$}
                         \psdots[linecolor=red](0,2)
                         \psframe[fillstyle=solid,fillcolor=vert,linecolor=vert,linewidth=0.75pt,opacity=0.2](-3,2.6)(16,1.4)
                    \end{pspicture*}
               }
          \end{extern}
     \end{center}
     \begin{center}
          \textit{Suite convergeant vers $l$}
     \end{center}
}
\bloc{cyan}{Remarques}{%id="r40"
     \begin{itemize}
          \item Une suite qui n'est pas convergente (c'est à dire qui n'a pas de limite ou qui a une limite infinie - voir ci-dessous) est dite \textbf{divergente}.
          \item La limite, si elle existe, est \textbf{unique}.
     \end{itemize}
}
\bloc{orange}{Exemple}{%id="e40"
     Les suites définies pour $n > 0$ par $u_{n}=\frac{1}{n^{k}}$ où $k$ est un entier strictement positif, \textbf{convergent vers zéro}
}
\cadre{bleu}{Définition}{%id="d50"
     On dit que la suite $u_{n}$ admet pour limite $+\infty $ si tout intervalle de la forme $\left]A;+\infty \right[$ contient tous les termes de la suite à partir d'un certain rang.
}
\bloc{orange}{Exemple}{%id="e50"
     Les suites définies pour $n > 0$ par $u_{n}=n^{k}$ où $k$ est un entier strictement positif, divergent vers $+\infty $
}
\cadre{rouge}{Théorème (des gendarmes)}{%id="t60"
     Si les suites $\left(v_{n}\right)$ et $\left(w_{n}\right)$ convergent vers \textbf{la même limite} $l$ et si $v_{n}\leqslant u_{n}\leqslant w_{n}$ pour tout entier $n$ à partir d'un certain rang, alors la suite $\left(u_{n}\right)$ converge vers $l$.
}
\bloc{orange}{Exemple}{%id="e60"
     Soit la suite définie pour $n > 0$ par $u_{n}=\frac{\sin\left(n\right)}{n}$.
     \par
     On sait que pour tout $n$, $-1\leqslant \sin\left(n\right)\leqslant 1$ donc $-\frac{1}{n}\leqslant \frac{\sin\left(n\right)}{n}\leqslant \frac{1}{n}$.
     \par
     Or les suites $\left(v_{n}\right)$ et $\left(w_{n}\right)$ définie sur $\mathbb{N}^*$ par $v_{n}=-\frac{1}{n}$ et $w_{n}=\frac{1}{n}$ convergent vers zéro donc, d'après le théorème des gendarmes \textbf{$\left(u_{n}\right)$ converge vers zéro}.
}
\cadre{rouge}{Théorème}{%id="t70"
     Soient deux suites $\left(u_{n}\right)$ et $\left(v_{n}\right)$ telles que pour tout $n \in \mathbb{N}$, $u_{n}\geqslant v_{n}$.
     \par
     Si $\lim\limits_{n\rightarrow +\infty }v_{n}=+\infty $, alors $\lim\limits_{n\rightarrow +\infty }u_{n}=+\infty $
}
\cadre{rouge}{Théorème}{%id="t80"
     Une suite \textbf{croissante et majorée} est convergente.
     \par
     Une suite \textbf{décroissante et minorée} est convergente.
}
\bloc{cyan}{Remarques}{%id="r80"
     \begin{itemize}
          \item Ce théorème est fréquemment utilisé dans les exercices
          \item Ce théorème permet de montrer qu'une suite est convergente mais, à lui seul, il ne permet pas de trouver la valeur de la limite $l$
     \end{itemize}
}
\bloc{orange}{Exemple}{%id="e80"
     Un cas particulier assez fréquent est celui d'une suite \textbf{décroissante et positive}. Puisqu'elle est positive, elle est minorée par zéro, donc d'après le théorème précédent, elle est convergente.
}
\cadre{rouge}{Théorème (limite d'une suite géométrique)}{%id="t90"
     Soit $\left(u_{n}\right)$ une suite géométrique de raison $q$.
     \begin{itemize}
          \item Si $-1 < q < 1$ la suite $\left(u_{n}\right)$ \textbf{converge vers 0}
          \item Si $q > 1$ la suite $\left(u_{n}\right)$ \textbf{tend vers $+\infty $}
          \item Si $q\leqslant -1$ la suite $\left(u_{n}\right)$ \textbf{n'a pas de limite.}
     \end{itemize}
}
\bloc{cyan}{Remarque}{%id="r90"
     Si $q=1$ la suite $\left(u_{n}\right)$ est constante (donc convergente)
}
\bloc{orange}{Exemple}{%id="e90"
     $\lim\limits_{n\rightarrow +\infty }\left(\frac{2}{3}\right)^{n}=0$ (suite géométrique de raison $q=\frac{2}{3} < 1$)
     \par
     $\lim\limits_{n\rightarrow +\infty }\left(\frac{4}{3}\right)^{n}=+\infty $ (suite géométrique de raison $q=\frac{4}{3} > 1$)
}

\end{document}
µ
\documentclass[a4paper]{article}

%================================================================================================================================
%
% Packages
%
%================================================================================================================================

\usepackage[T1]{fontenc} 	% pour caractères accentués
\usepackage[utf8]{inputenc}  % encodage utf8
\usepackage[french]{babel}	% langue : français
\usepackage{fourier}			% caractères plus lisibles
\usepackage[dvipsnames]{xcolor} % couleurs
\usepackage{fancyhdr}		% réglage header footer
\usepackage{needspace}		% empêcher sauts de page mal placés
\usepackage{graphicx}		% pour inclure des graphiques
\usepackage{enumitem,cprotect}		% personnalise les listes d'items (nécessaire pour ol, al ...)
\usepackage{hyperref}		% Liens hypertexte
\usepackage{pstricks,pst-all,pst-node,pstricks-add,pst-math,pst-plot,pst-tree,pst-eucl} % pstricks
\usepackage[a4paper,includeheadfoot,top=2cm,left=3cm, bottom=2cm,right=3cm]{geometry} % marges etc.
\usepackage{comment}			% commentaires multilignes
\usepackage{amsmath,environ} % maths (matrices, etc.)
\usepackage{amssymb,makeidx}
\usepackage{bm}				% bold maths
\usepackage{tabularx}		% tableaux
\usepackage{colortbl}		% tableaux en couleur
\usepackage{fontawesome}		% Fontawesome
\usepackage{environ}			% environment with command
\usepackage{fp}				% calculs pour ps-tricks
\usepackage{multido}			% pour ps tricks
\usepackage[np]{numprint}	% formattage nombre
\usepackage{tikz,tkz-tab} 			% package principal TikZ
\usepackage{pgfplots}   % axes
\usepackage{mathrsfs}    % cursives
\usepackage{calc}			% calcul taille boites
\usepackage[scaled=0.875]{helvet} % font sans serif
\usepackage{svg} % svg
\usepackage{scrextend} % local margin
\usepackage{scratch} %scratch
\usepackage{multicol} % colonnes
%\usepackage{infix-RPN,pst-func} % formule en notation polanaise inversée
\usepackage{listings}

%================================================================================================================================
%
% Réglages de base
%
%================================================================================================================================

\lstset{
language=Python,   % R code
literate=
{á}{{\'a}}1
{à}{{\`a}}1
{ã}{{\~a}}1
{é}{{\'e}}1
{è}{{\`e}}1
{ê}{{\^e}}1
{í}{{\'i}}1
{ó}{{\'o}}1
{õ}{{\~o}}1
{ú}{{\'u}}1
{ü}{{\"u}}1
{ç}{{\c{c}}}1
{~}{{ }}1
}


\definecolor{codegreen}{rgb}{0,0.6,0}
\definecolor{codegray}{rgb}{0.5,0.5,0.5}
\definecolor{codepurple}{rgb}{0.58,0,0.82}
\definecolor{backcolour}{rgb}{0.95,0.95,0.92}

\lstdefinestyle{mystyle}{
    backgroundcolor=\color{backcolour},   
    commentstyle=\color{codegreen},
    keywordstyle=\color{magenta},
    numberstyle=\tiny\color{codegray},
    stringstyle=\color{codepurple},
    basicstyle=\ttfamily\footnotesize,
    breakatwhitespace=false,         
    breaklines=true,                 
    captionpos=b,                    
    keepspaces=true,                 
    numbers=left,                    
xleftmargin=2em,
framexleftmargin=2em,            
    showspaces=false,                
    showstringspaces=false,
    showtabs=false,                  
    tabsize=2,
    upquote=true
}

\lstset{style=mystyle}


\lstset{style=mystyle}
\newcommand{\imgdir}{C:/laragon/www/newmc/assets/imgsvg/}
\newcommand{\imgsvgdir}{C:/laragon/www/newmc/assets/imgsvg/}

\definecolor{mcgris}{RGB}{220, 220, 220}% ancien~; pour compatibilité
\definecolor{mcbleu}{RGB}{52, 152, 219}
\definecolor{mcvert}{RGB}{125, 194, 70}
\definecolor{mcmauve}{RGB}{154, 0, 215}
\definecolor{mcorange}{RGB}{255, 96, 0}
\definecolor{mcturquoise}{RGB}{0, 153, 153}
\definecolor{mcrouge}{RGB}{255, 0, 0}
\definecolor{mclightvert}{RGB}{205, 234, 190}

\definecolor{gris}{RGB}{220, 220, 220}
\definecolor{bleu}{RGB}{52, 152, 219}
\definecolor{vert}{RGB}{125, 194, 70}
\definecolor{mauve}{RGB}{154, 0, 215}
\definecolor{orange}{RGB}{255, 96, 0}
\definecolor{turquoise}{RGB}{0, 153, 153}
\definecolor{rouge}{RGB}{255, 0, 0}
\definecolor{lightvert}{RGB}{205, 234, 190}
\setitemize[0]{label=\color{lightvert}  $\bullet$}

\pagestyle{fancy}
\renewcommand{\headrulewidth}{0.2pt}
\fancyhead[L]{maths-cours.fr}
\fancyhead[R]{\thepage}
\renewcommand{\footrulewidth}{0.2pt}
\fancyfoot[C]{}

\newcolumntype{C}{>{\centering\arraybackslash}X}
\newcolumntype{s}{>{\hsize=.35\hsize\arraybackslash}X}

\setlength{\parindent}{0pt}		 
\setlength{\parskip}{3mm}
\setlength{\headheight}{1cm}

\def\ebook{ebook}
\def\book{book}
\def\web{web}
\def\type{web}

\newcommand{\vect}[1]{\overrightarrow{\,\mathstrut#1\,}}

\def\Oij{$\left(\text{O}~;~\vect{\imath},~\vect{\jmath}\right)$}
\def\Oijk{$\left(\text{O}~;~\vect{\imath},~\vect{\jmath},~\vect{k}\right)$}
\def\Ouv{$\left(\text{O}~;~\vect{u},~\vect{v}\right)$}

\hypersetup{breaklinks=true, colorlinks = true, linkcolor = OliveGreen, urlcolor = OliveGreen, citecolor = OliveGreen, pdfauthor={Didier BONNEL - https://www.maths-cours.fr} } % supprime les bordures autour des liens

\renewcommand{\arg}[0]{\text{arg}}

\everymath{\displaystyle}

%================================================================================================================================
%
% Macros - Commandes
%
%================================================================================================================================

\newcommand\meta[2]{    			% Utilisé pour créer le post HTML.
	\def\titre{titre}
	\def\url{url}
	\def\arg{#1}
	\ifx\titre\arg
		\newcommand\maintitle{#2}
		\fancyhead[L]{#2}
		{\Large\sffamily \MakeUppercase{#2}}
		\vspace{1mm}\textcolor{mcvert}{\hrule}
	\fi 
	\ifx\url\arg
		\fancyfoot[L]{\href{https://www.maths-cours.fr#2}{\black \footnotesize{https://www.maths-cours.fr#2}}}
	\fi 
}


\newcommand\TitreC[1]{    		% Titre centré
     \needspace{3\baselineskip}
     \begin{center}\textbf{#1}\end{center}
}

\newcommand\newpar{    		% paragraphe
     \par
}

\newcommand\nosp {    		% commande vide (pas d'espace)
}
\newcommand{\id}[1]{} %ignore

\newcommand\boite[2]{				% Boite simple sans titre
	\vspace{5mm}
	\setlength{\fboxrule}{0.2mm}
	\setlength{\fboxsep}{5mm}	
	\fcolorbox{#1}{#1!3}{\makebox[\linewidth-2\fboxrule-2\fboxsep]{
  		\begin{minipage}[t]{\linewidth-2\fboxrule-4\fboxsep}\setlength{\parskip}{3mm}
  			 #2
  		\end{minipage}
	}}
	\vspace{5mm}
}

\newcommand\CBox[4]{				% Boites
	\vspace{5mm}
	\setlength{\fboxrule}{0.2mm}
	\setlength{\fboxsep}{5mm}
	
	\fcolorbox{#1}{#1!3}{\makebox[\linewidth-2\fboxrule-2\fboxsep]{
		\begin{minipage}[t]{1cm}\setlength{\parskip}{3mm}
	  		\textcolor{#1}{\LARGE{#2}}    
 	 	\end{minipage}  
  		\begin{minipage}[t]{\linewidth-2\fboxrule-4\fboxsep}\setlength{\parskip}{3mm}
			\raisebox{1.2mm}{\normalsize\sffamily{\textcolor{#1}{#3}}}						
  			 #4
  		\end{minipage}
	}}
	\vspace{5mm}
}

\newcommand\cadre[3]{				% Boites convertible html
	\par
	\vspace{2mm}
	\setlength{\fboxrule}{0.1mm}
	\setlength{\fboxsep}{5mm}
	\fcolorbox{#1}{white}{\makebox[\linewidth-2\fboxrule-2\fboxsep]{
  		\begin{minipage}[t]{\linewidth-2\fboxrule-4\fboxsep}\setlength{\parskip}{3mm}
			\raisebox{-2.5mm}{\sffamily \small{\textcolor{#1}{\MakeUppercase{#2}}}}		
			\par		
  			 #3
 	 		\end{minipage}
	}}
		\vspace{2mm}
	\par
}

\newcommand\bloc[3]{				% Boites convertible html sans bordure
     \needspace{2\baselineskip}
     {\sffamily \small{\textcolor{#1}{\MakeUppercase{#2}}}}    
		\par		
  			 #3
		\par
}

\newcommand\CHelp[1]{
     \CBox{Plum}{\faInfoCircle}{À RETENIR}{#1}
}

\newcommand\CUp[1]{
     \CBox{NavyBlue}{\faThumbsOUp}{EN PRATIQUE}{#1}
}

\newcommand\CInfo[1]{
     \CBox{Sepia}{\faArrowCircleRight}{REMARQUE}{#1}
}

\newcommand\CRedac[1]{
     \CBox{PineGreen}{\faEdit}{BIEN R\'EDIGER}{#1}
}

\newcommand\CError[1]{
     \CBox{Red}{\faExclamationTriangle}{ATTENTION}{#1}
}

\newcommand\TitreExo[2]{
\needspace{4\baselineskip}
 {\sffamily\large EXERCICE #1\ (\emph{#2 points})}
\vspace{5mm}
}

\newcommand\img[2]{
          \includegraphics[width=#2\paperwidth]{\imgdir#1}
}

\newcommand\imgsvg[2]{
       \begin{center}   \includegraphics[width=#2\paperwidth]{\imgsvgdir#1} \end{center}
}


\newcommand\Lien[2]{
     \href{#1}{#2 \tiny \faExternalLink}
}
\newcommand\mcLien[2]{
     \href{https~://www.maths-cours.fr/#1}{#2 \tiny \faExternalLink}
}

\newcommand{\euro}{\eurologo{}}

%================================================================================================================================
%
% Macros - Environement
%
%================================================================================================================================

\newenvironment{tex}{ %
}
{%
}

\newenvironment{indente}{ %
	\setlength\parindent{10mm}
}

{
	\setlength\parindent{0mm}
}

\newenvironment{corrige}{%
     \needspace{3\baselineskip}
     \medskip
     \textbf{\textsc{Corrigé}}
     \medskip
}
{
}

\newenvironment{extern}{%
     \begin{center}
     }
     {
     \end{center}
}

\NewEnviron{code}{%
	\par
     \boite{gray}{\texttt{%
     \BODY
     }}
     \par
}

\newenvironment{vbloc}{% boite sans cadre empeche saut de page
     \begin{minipage}[t]{\linewidth}
     }
     {
     \end{minipage}
}
\NewEnviron{h2}{%
    \needspace{3\baselineskip}
    \vspace{0.6cm}
	\noindent \MakeUppercase{\sffamily \large \BODY}
	\vspace{1mm}\textcolor{mcgris}{\hrule}\vspace{0.4cm}
	\par
}{}

\NewEnviron{h3}{%
    \needspace{3\baselineskip}
	\vspace{5mm}
	\textsc{\BODY}
	\par
}

\NewEnviron{margeneg}{ %
\begin{addmargin}[-1cm]{0cm}
\BODY
\end{addmargin}
}

\NewEnviron{html}{%
}

\begin{document}
\meta{url}{/cours/droites-plans-espace/}
\meta{pid}{548}
\meta{titre}{Droites et plans dans l'espace}
\meta{type}{cours}
\begin{h2}1. Rappels sur les droites et plans\end{h2}
\cadre{vert}{Propriété}{% id="p10"
     Par deux points distincts de l'espace, il passe une et une seule droite.
}
\bloc{cyan}{Remarque}{% id="r10"
     Dans les exercices où l'on cherche à déterminer une droite (par exemple, pour tracer l'intersection de deux plans), il suffira donc de trouver deux points distincts qui appartiennent à cette droite.
}
\cadre{vert}{Propriété}{% id="p10"
     Par trois points distincts et \textbf{non alignés} de l'espace, il passe un et un seul plan.
}
\cadre{bleu}{Positions relatives de deux plans}{% id="t30"
     Deux plans distincts de l'espace peuvent être~:
     \begin{itemize}
          \item \textbf{strictement parallèles}~: dans ce cas, ils n'ont aucun point commun
          \item \textbf{sécants}~: dans ce cas, leur intersection est une droite
     \end{itemize}
}
\begin{vbloc}
     \begin{center}
          \begin{extern}%width="400" alt="Plans parallèles"
               \newrgbcolor{ttzzqq}{0.2 0.6 0.}
               \newrgbcolor{qqzzff}{0. 0.6 1.}
               \newrgbcolor{qqwwtt}{0. 0.4 0.2}
               \psset{xunit=1.0cm,yunit=1.0cm,algebraic=true,dimen=middle,dotstyle=o,dotsize=5pt 0,linewidth=1.6pt,arrowsize=3pt 2,arrowinset=0.25}
               \begin{pspicture*}(1.,2.8)(12,6.5)
                    \pspolygon[linewidth=0.8pt,linecolor=ttzzqq,fillcolor=ttzzqq,fillstyle=solid,opacity=0.1](4.,6.)(2.,5.)(9.,5.)(11.,6.)
                    \pspolygon[linewidth=0.8pt,linecolor=qqzzff,fillcolor=qqzzff,fillstyle=solid,opacity=0.1](4.,4.)(2.,3.)(9.,3.)(11.,4.)
                    \psline[linewidth=0.8pt,linecolor=ttzzqq](4.,6.)(2.,5.)
                    \psline[linewidth=0.8pt,linecolor=ttzzqq](2.,5.)(9.,5.)
                    \psline[linewidth=0.8pt,linecolor=ttzzqq](9.,5.)(11.,6.)
                    \psline[linewidth=0.8pt,linecolor=ttzzqq](11.,6.)(4.,6.)
                    \psline[linewidth=0.8pt,linecolor=qqzzff](4.,4.)(2.,3.)
                    \psline[linewidth=0.8pt,linecolor=qqzzff](2.,3.)(9.,3.)
                    \psline[linewidth=0.8pt,linecolor=qqzzff](9.,3.)(11.,4.)
                    \psline[linewidth=0.8pt,linecolor=qqzzff](11.,4.)(4.,4.)
                    \rput[tl](3,5.4){$\qqwwtt{\mathscr{P}'}$}
                    \rput[tl](3,3.4){$\qqzzff{\mathscr{P}}$}
               \end{pspicture*}
          \end{extern}
     \end{center}
     \begin{center}
          \textit{Plans parallèles}
     \end{center}
\end{vbloc}
\begin{center}
     \begin{extern}%width="400" alt=""
          \newrgbcolor{ttzzqq}{0.2 0.6 0.}
          \newrgbcolor{qqzzcc}{0. 0.6 0.8}
          \newrgbcolor{qqwwtt}{0. 0.4 0.2}
          \psset{xunit=1.0cm,yunit=1.0cm,algebraic=true,dimen=middle,dotstyle=o,dotsize=5pt 0,linewidth=1.6pt,arrowsize=3pt 2,arrowinset=0.25}
          \begin{pspicture*}(0.,2.5)(12,8.5)
               \pspolygon[linewidth=0.8pt,linecolor=ttzzqq,fillcolor=ttzzqq,fillstyle=solid,opacity=0.1](4.,6.)(2.,5.)(9.,5.)(11.,6.)
               \pspolygon[linewidth=0.8pt,linecolor=qqzzcc,fillcolor=qqzzcc,fillstyle=solid,opacity=0.1](4.,4.)(2.,3.)(8.,7.)(10.,8.)
               \psline[linewidth=0.8pt,linecolor=ttzzqq](4.,6.)(2.,5.)
               \psline[linewidth=0.8pt,linecolor=ttzzqq](2.,5.)(9.,5.)
               \psline[linewidth=0.8pt,linecolor=ttzzqq](9.,5.)(11.,6.)
               \psline[linewidth=0.8pt,linecolor=ttzzqq](11.,6.)(4.,6.)
               \psline[linewidth=0.8pt,linecolor=qqzzcc](4.,4.)(2.,3.)
               \psline[linewidth=0.8pt,linecolor=qqzzcc](2.,3.)(8.,7.)
               \psline[linewidth=0.8pt,linecolor=qqzzcc](8.,7.)(10.,8.)
               \psline[linewidth=0.8pt,linecolor=qqzzcc](10.,8.)(4.,4.)
               \rput[tl](3.,5.4){$\qqwwtt{\mathscr{P}}$}
               \rput[tl](2.6,3.2){$\qqzzcc{\mathscr{P}\ '}$}
               \psline[linewidth=0.8pt,linecolor=red](5.,5.)(7.,6.)
               \rput[tl](7.,5.9){$\red{\mathscr{D}}$}
          \end{pspicture*}
     \end{extern}
\end{center}
\begin{center}
     \textit{Plans sécants}
\end{center}
\cadre{bleu}{Positions relatives d'une droite et d'un plan}{% id="t40"
     Soient $\mathscr D$ une droite et $\mathscr P$ un plan de l'espace.
     \par
     La droite $\mathscr D$ peut être~:
     \begin{itemize}
          \item \textbf{strictement parallèle} au plan  $\mathscr P$~: dans ce cas, $\mathscr D$  et $\mathscr P$ n'ont aucun point commun
          \item \textbf{sécante} avec le plan  $\mathscr P$~: dans ce cas, $\mathscr D$  et $\mathscr P$ ont un unique point commun
          \item \textbf{contenue} dans le plan  $\mathscr P$
     \end{itemize}
}
\begin{center}
     \begin{extern}%width="400" alt=""
          \newrgbcolor{ttzzqq}{0.2 0.6 0.}
          \newrgbcolor{qqwwtt}{0. 0.4 0.2}
          \psset{xunit=1.0cm,yunit=1.0cm,algebraic=true,dimen=middle,dotstyle=o,dotsize=5pt 0,linewidth=1.6pt,arrowsize=3pt 2,arrowinset=0.25}
          \begin{pspicture*}(1.5,5)(11.5,8.3)
               \pspolygon[linewidth=0.8pt,linecolor=ttzzqq,fillcolor=ttzzqq,fillstyle=solid,opacity=0.1](4.,6.)(2.,5.)(9.,5.)(11.,6.)
               \psline[linewidth=0.8pt,linecolor=ttzzqq](4.,6.)(2.,5.)
               \psline[linewidth=0.8pt,linecolor=ttzzqq](2.,5.)(9.,5.)
               \psline[linewidth=0.8pt,linecolor=ttzzqq](9.,5.)(11.,6.)
               \psline[linewidth=0.8pt,linecolor=ttzzqq](11.,6.)(4.,6.)
               \rput[tl](3.0,5.4){$\qqwwtt{\mathscr{P}}$}
               \rput[tl](2.25,7.5){$\red{\mathscr{D}}$}
               \psplot[linewidth=0.8pt,linecolor=red]{1.75}{11.2}{(--56.-0.*x)/8.}
          \end{pspicture*}
     \end{extern}
\end{center}
\begin{center}
     \textit{Droite strictement parallèle à un plan}
\end{center}
\begin{vbloc}
     \begin{center}
          \begin{extern}%width="400" alt=""
               \newrgbcolor{ttzzqq}{0.2 0.6 0.}
               \newrgbcolor{qqwwtt}{0. 0.4 0.2}
               \newrgbcolor{ttttff}{0.2 0.2 1.}
               \newrgbcolor{ffewdf}{1. 0.9 0.9}
               \newrgbcolor{qqqqcc}{0. 0. 0.8}
               \psset{xunit=1.0cm,yunit=1.0cm,algebraic=true,dimen=middle,dotstyle=o,dotsize=5pt 0,linewidth=1.6pt,arrowsize=3pt 2,arrowinset=0.25}
               \begin{pspicture*}(1.5,3.3)(11.5,8.5)
                    \pspolygon[linewidth=0.8pt,linecolor=ttzzqq,fillcolor=ttzzqq,fillstyle=solid,opacity=0.1](4.,6.)(2.,5.)(9.,5.)(11.,6.)
                    \psline[linewidth=0.8pt,linecolor=ttzzqq](4.,6.)(2.,5.)
                    \psline[linewidth=0.8pt,linecolor=ttzzqq](2.,5.)(9.,5.)
                    \psline[linewidth=0.8pt,linecolor=ttzzqq](9.,5.)(11.,6.)
                    \psline[linewidth=0.8pt,linecolor=ttzzqq](11.,6.)(4.,6.)
                    \rput[tl](3,5.4){$\qqwwtt{\mathscr{P}}$}
                    \rput[tl](7.5,8){$\red{\mathscr{D}}$}
                    \psplot[linewidth=0.8pt,linecolor=red]{1.5}{11.2}{(-7.2--2.17*x)/1.1}
                    \psline[linewidth=0.8pt,linecolor=ffewdf](6.15,5.555)(5.87,5.)
                    \rput[tl](5.9,5.9){$\qqqqcc{I}$}
                    \begin{scriptsize}
                         \psdots[dotsize=1pt 0,dotstyle=*,linecolor=ttttff](6.15,5.55)
                    \end{scriptsize}
               \end{pspicture*}
          \end{extern}
     \end{center}
     \begin{center}
          \textit{Droite sécante à un plan}
     \end{center}
\end{vbloc}
\begin{center}
     \begin{extern}%width="400" alt=""
          \newrgbcolor{ttzzqq}{0.2 0.6 0.}
          \newrgbcolor{qqwwtt}{0. 0.4 0.2}
          \psset{xunit=1.0cm,yunit=1.0cm,algebraic=true,dimen=middle,dotstyle=o,dotsize=5pt 0,linewidth=1.6pt,arrowsize=3pt 2,arrowinset=0.25}
          \begin{pspicture*}(2.,5)(12.,7.)
               \pspolygon[linewidth=0.8pt,linecolor=ttzzqq,fillcolor=ttzzqq,fillstyle=solid,opacity=0.1](4.,6.)(2.,5.)(9.,5.)(11.,6.)
               \psline[linewidth=0.8pt,linecolor=ttzzqq](4.,6.)(2.,5.)
               \psline[linewidth=0.8pt,linecolor=ttzzqq](2.,5.)(9.,5.)
               \psline[linewidth=0.8pt,linecolor=ttzzqq](9.,5.)(11.,6.)
               \psline[linewidth=0.8pt,linecolor=ttzzqq](11.,6.)(4.,6.)
               \rput[tl](3.,5.4){$\qqwwtt{\mathscr{P}}$}
               \rput[tl](6.405,5.4){$\red{\mathscr{D}}$}
               \psline[linewidth=0.8pt,linecolor=red](4.77,5.)(8.4,6.)
          \end{pspicture*}
     \end{extern}
\end{center}
\begin{center}
     \textit{Droite contenue (incluse) dans un plan}
\end{center}
\cadre{bleu}{Positions relatives de deux droites}{% id="t50"
     Soient $\mathscr D$ et $\mathscr D^{\prime}$ deux droites distinctes de l'espace.
     \par
     Ces droites peuvent être~:
     \begin{itemize}
          \item \textbf{non coplanaires}~: dans ce cas, elles n'ont aucun point commun
          \item \textbf{coplanaires},  c'est à dire contenues dans un même plan~; elles peuvent alors être~:
          \begin{itemize}[label=---]
               \item \textbf{strictement parallèles}~: dans ce cas, elles n'ont aucun point commun
               \item \textbf{sécantes}~: dans ce cas, leur intersection est un point
          \end{itemize}
     \end{itemize}
}
\begin{center}
     \begin{extern}%width="400" alt=""
          \newrgbcolor{ttzzqq}{0.2 0.6 0.}
          \newrgbcolor{qqzzff}{0. 0.6 1.}
          \newrgbcolor{xfqqff}{0.5 0. 1.}
          \newrgbcolor{wwqqzz}{0.4 0. 0.6}
          \psset{xunit=1.0cm,yunit=1.0cm,algebraic=true,dimen=middle,dotstyle=o,dotsize=5pt 0,linewidth=1.6pt,arrowsize=3pt 2,arrowinset=0.25}
          \begin{pspicture*}(1.5,3)(12.,6.5)
               \pspolygon[linewidth=0.8pt,linecolor=ttzzqq,fillcolor=ttzzqq,fillstyle=solid,opacity=0.1](4.,6.)(2.,5.)(9.,5.)(11.,6.)
               \pspolygon[linewidth=0.8pt,linecolor=qqzzff,fillcolor=qqzzff,fillstyle=solid,opacity=0.1](4.,4.)(2.,3.)(9.,3.)(11.,4.)
               \psline[linewidth=0.8pt,linecolor=ttzzqq](4.,6.)(2.,5.)
               \psline[linewidth=0.8pt,linecolor=ttzzqq](2.,5.)(9.,5.)
               \psline[linewidth=0.8pt,linecolor=ttzzqq](9.,5.)(11.,6.)
               \psline[linewidth=0.8pt,linecolor=ttzzqq](11.,6.)(4.,6.)
               \psline[linewidth=0.8pt,linecolor=qqzzff](4.,4.)(2.,3.)
               \psline[linewidth=0.8pt,linecolor=qqzzff](2.,3.)(9.,3.)
               \psline[linewidth=0.8pt,linecolor=qqzzff](9.,3.)(11.,4.)
               \psline[linewidth=0.8pt,linecolor=qqzzff](11.,4.)(4.,4.)
               \psline[linewidth=0.8pt,linecolor=red](3.8,5.)(7.5,6.)
               \psline[linewidth=0.8pt,linecolor=xfqqff](6.0,4.)(7.2,3.)
               \rput[tl](5,5.8){$\red{\mathscr{D}}$}
               \rput[tl](6.9,3.8){$\wwqqzz{\mathscr{D}\ '}$}
          \end{pspicture*}
     \end{extern}
\end{center}
\begin{center}
     \textit{Droites non coplanaires}
\end{center}
\begin{center}
     \begin{extern}%width="400" alt=""
          \newrgbcolor{ttzzqq}{0.2 0.6 0.}
          \newrgbcolor{wwqqzz}{0.4 0. 0.6}
          \psset{xunit=1.0cm,yunit=1.0cm,algebraic=true,dimen=middle,dotstyle=o,dotsize=5pt 0,linewidth=1.6pt,arrowsize=3pt 2,arrowinset=0.25}
          \begin{pspicture*}(1.3,4.9)(12.,6.6)
               \pspolygon[linewidth=0.8pt,linecolor=ttzzqq,fillcolor=ttzzqq,fillstyle=solid,opacity=0.1](4.,6.)(2.,5.)(9.,5.)(11.,6.)
               \psline[linewidth=0.8pt,linecolor=ttzzqq](4.,6.)(2.,5.)
               \psline[linewidth=0.8pt,linecolor=ttzzqq](2.,5.)(9.,5.)
               \psline[linewidth=0.8pt,linecolor=ttzzqq](9.,5.)(11.,6.)
               \psline[linewidth=0.8pt,linecolor=ttzzqq](11.,6.)(4.,6.)
               \psline[linewidth=0.8pt,linecolor=red](3.8,5.)(7.5,6.)
               \rput[tl](4.8,5.7){$\red{\mathscr{D}}$}
               \rput[tl](8.0,5.5){$\wwqqzz{\mathscr{D}\ '}$}
               \psline[linewidth=0.8pt,linecolor=wwqqzz](6.0,5.)(9.75,6.)
          \end{pspicture*}
     \end{extern}
\end{center}
\begin{center}
     \textit{Droites strictement parallèles}
\end{center}
\begin{vbloc}
     \begin{center}
          \begin{extern}%width="400" alt=""
               \newrgbcolor{ttzzqq}{0.2 0.6 0.}
               \newrgbcolor{wwqqzz}{0.4 0. 0.6}
               \newrgbcolor{qqttcc}{0. 0.2 0.8}
               \psset{xunit=1.0cm,yunit=1.0cm,algebraic=true,dimen=middle,dotstyle=o,dotsize=5pt 0,linewidth=1.6pt,arrowsize=3pt 2,arrowinset=0.25}
               \begin{pspicture*}(1.3,4.7)(12.,6.6)
                    \pspolygon[linewidth=0.8pt,linecolor=ttzzqq,fillcolor=ttzzqq,fillstyle=solid,opacity=0.1](4.,6.)(2.,5.)(9.,5.)(11.,6.)
                    \psline[linewidth=0.8pt,linecolor=ttzzqq](4.,6.)(2.,5.)
                    \psline[linewidth=0.8pt,linecolor=ttzzqq](2.,5.)(9.,5.)
                    \psline[linewidth=0.8pt,linecolor=ttzzqq](9.,5.)(11.,6.)
                    \psline[linewidth=0.8pt,linecolor=ttzzqq](11.,6.)(4.,6.)
                    \psline[linewidth=0.8pt,linecolor=red](3.80,5.)(7.527,6.)
                    \rput[tl](4.846,5.7){$\red{\mathscr{D}}$}
                    \rput[tl](6.66,5.47){$\wwqqzz{\mathscr{D}\ '}$}
                    \psline[linewidth=0.8pt,linecolor=wwqqzz](5.36,6.)(6.758,5.)
                    \rput[tl](5.96,5.9){$\qqttcc{I}$}
                    \begin{scriptsize}
                    \end{scriptsize}
               \end{pspicture*}
          \end{extern}
     \end{center}
     \begin{center}
          \textit{Droites sécantes}
     \end{center}
\end{vbloc}
\begin{h2}2. Parallélisme\end{h2}
\cadre{vert}{Propriété}{% id="p60"
     Si un plan $\mathscr P_{1}$ contient deux droites sécantes $\mathscr D$ et $\mathscr D^{\prime}$ parallèles à un plan $\mathscr P_{2}$, alors le plan $\mathscr P_{1}$ est parallèle au plan $\mathscr P_{2}$. .
}
\begin{center}
     \begin{extern}%width="400" alt=""
          \newrgbcolor{qqzzff}{0. 0.6 1.}
          \newrgbcolor{ttzzqq}{0.2 0.6 0.}
          \newrgbcolor{qqwuqq}{0. 0.4 0.}
          \psset{xunit=1.0cm,yunit=1.0cm,algebraic=true,dimen=middle,dotstyle=o,dotsize=5pt 0,linewidth=1.6pt,arrowsize=3pt 2,arrowinset=0.25}
          \begin{pspicture*}(-1.7,-2.5)(7.5,1.6)
               \pspolygon[linewidth=0.8pt,linecolor=qqzzff,fillcolor=qqzzff,fillstyle=solid,opacity=0.1](1.,-1.)(-1.,-2.)(5.,-2.)(7.,-1.)
               \pspolygon[linewidth=0.8pt,linecolor=ttzzqq,fillcolor=ttzzqq,fillstyle=solid,opacity=0.1](1.,1.)(-1.,0.)(5.,0.)(7.,1.)
               \psline[linewidth=0.8pt,linecolor=qqzzff](1.,-1.)(-1.,-2.)
               \psline[linewidth=0.8pt,linecolor=qqzzff](-1.,-2.)(5.,-2.)
               \psline[linewidth=0.8pt,linecolor=qqzzff](5.,-2.)(7.,-1.)
               \psline[linewidth=0.8pt,linecolor=qqzzff](7.,-1.)(1.,-1.)
               \psline[linewidth=0.8pt,linecolor=ttzzqq](1.,1.)(-1.,0.)
               \psline[linewidth=0.8pt,linecolor=ttzzqq](-1.,0.)(5.,0.)
               \psline[linewidth=0.8pt,linecolor=ttzzqq](5.,0.)(7.,1.)
               \psline[linewidth=0.8pt,linecolor=ttzzqq](7.,1.)(1.,1.)
               \rput[tl](4.5,0.38){$\qqwuqq{\mathscr{P}_1}$}
               \rput[tl](4.5,-1.6){$\qqzzff{\mathscr{P}_2}$}
               \rput[tl](1.6685443902439065,1.5){$\red{\mathscr{D'}}$}
               \psline[linewidth=0.8pt,linecolor=red](0.18756878048780434,0.)(6.,1.)
               \psline[linewidth=0.8pt,linecolor=red](3.,0.)(2.,1.)
               \rput[tl](-0.18,0.38){$\red{\mathscr{D}}$}
          \end{pspicture*}
     \end{extern}
\end{center}
\cadre{vert}{Propriété}{% id="p65"
     Si deux plans $\mathscr P_{1}$ et $\mathscr P_{2}$ sont parallèles, alors tout plan $\mathscr P$ sécant à $\mathscr P_{1}$  est sécant à $\mathscr P_{2}$ et leurs intersections sont deux droites parallèles.
}
\begin{center}
     \begin{extern}%width="400" alt=""
          \newrgbcolor{ttzzqq}{0.2 0.6 0.}
          \newrgbcolor{qqzzcc}{0. 0.6 0.8}
          \newrgbcolor{qqwwtt}{0. 0.4 0.2}
          \begin{pspicture*}(1.,2.7)(13.4,10.3)
               \pspolygon[linewidth=0.8pt,linecolor=ttzzqq,fillcolor=ttzzqq,fillstyle=solid,opacity=0.1](4.,6.)(2.,5.)(9.,5.)(11.,6.)
               \pspolygon[linewidth=0.8pt,linecolor=qqzzcc,fillcolor=qqzzcc,fillstyle=solid,opacity=0.1](4.,4.)(2.,3.)(8.,9.)(10.,10.)
               \pspolygon[linewidth=0.8pt,linecolor=ttzzqq,fillcolor=ttzzqq,fillstyle=solid,opacity=0.1](4.,8.)(2.,7.)(9.,7.)(11.,8.)
               \psline[linewidth=0.8pt,linecolor=ttzzqq](4.,6.)(2.,5.)
               \psline[linewidth=0.8pt,linecolor=ttzzqq](2.,5.)(9.,5.)
               \psline[linewidth=0.8pt,linecolor=ttzzqq](9.,5.)(11.,6.)
               \psline[linewidth=0.8pt,linecolor=ttzzqq](11.,6.)(4.,6.)
               \psline[linewidth=0.8pt,linecolor=qqzzcc](4.,4.)(2.,3.)
               \psline[linewidth=0.8pt,linecolor=qqzzcc](2.,3.)(8.,9.)
               \psline[linewidth=0.8pt,linecolor=qqzzcc](8.,9.)(10.,10.)
               \psline[linewidth=0.8pt,linecolor=qqzzcc](10.,10.)(4.,4.)
               \rput[tl](2.8,5.4){$\qqwwtt{\mathscr{P}_2}$}
               \rput[tl](2.8,3.4){$\qqzzcc{\mathscr{P}}$}
               \psline[linewidth=0.8pt,linecolor=ttzzqq](4.,8.)(2.,7.)
               \psline[linewidth=0.8pt,linecolor=ttzzqq](2.,7.)(9.,7.)
               \psline[linewidth=0.8pt,linecolor=ttzzqq](9.,7.)(11.,8.)
               \psline[linewidth=0.8pt,linecolor=ttzzqq](11.,8.)(4.,8.)
               \psline[linewidth=0.8pt,linecolor=red](4.,5.)(6.,6.)
               \psline[linewidth=0.8pt,linecolor=red](6.,7.)(8.,8.)
               \rput[tl](2.8,7.4){$\qqwwtt{\mathscr{P}_1}$}
          \end{pspicture*}
     \end{extern}
\end{center}
\cadre{vert}{Propriété (Théorème du toit)}{% id="p70"
     Si $\mathscr P_{1}$ et $\mathscr P_{2}$ sont deux plans sécants et si une droite $\mathscr D_{1}$ incluse dans $\mathscr P_{1}$ est parallèle à une droite $\mathscr D_{2}$ incluse dans $\mathscr P_{2}$ alors la droite $\mathscr D$ intersection de $\mathscr P_{1}$ et $\mathscr P_{2}$ est parallèle à $\mathscr D_{1}$ et $\mathscr D_{2}$.
}
\begin{center}
     \begin{extern}%width="400" alt=""
          \newrgbcolor{wwccff}{0.4 0.8 1.}
          \newrgbcolor{ttzzqq}{0.2 0.6 0.}
          \newrgbcolor{ffzzcc}{0. 0.4 0.2}
          \newrgbcolor{wwqqzz}{0.4 0. 0.6}
          \newrgbcolor{ffwwzz}{1. 0.4 0.6}
          \psset{xunit=1.0cm,yunit=1.0cm,algebraic=true,dimen=middle,dotstyle=o,dotsize=5pt 0,linewidth=1.6pt,arrowsize=3pt 2,arrowinset=0.25}
          \begin{pspicture*}(-3.3,-1.1)(7.18,2.46)
               \pspolygon[linewidth=0.8pt,linecolor=wwccff,fillcolor=wwccff,fillstyle=solid,opacity=0.1](0.,2.)(4.,2.)(1.,0.)(-3.,0.)
               \pspolygon[linewidth=0.8pt,linecolor=ttzzqq,fillcolor=ttzzqq,fillstyle=solid,opacity=0.1](4.,2.)(7.,-1.)(3.,-1.)(0.,2.)
               \psline[linewidth=0.8pt,linecolor=wwccff](4.,2.)(1.,0.)
               \psline[linewidth=0.8pt,linecolor=wwccff](1.,0.)(-3.,0.)
               \psline[linewidth=0.8pt,linecolor=wwccff](-3.,0.)(0.,2.)
               \psline[linewidth=0.8pt,linecolor=ttzzqq](4.,2.)(7.,-1.)
               \psline[linewidth=0.8pt,linecolor=ttzzqq](7.,-1.)(3.,-1.)
               \psline[linewidth=0.8pt,linecolor=ttzzqq](3.,-1.)(0.,2.)
               \psline[linewidth=0.8pt,linecolor=ffzzcc](0.,2.)(4.,2.)
               \psline[linewidth=0.8pt,linecolor=red](-1.48,1.0)(2.52,1.)
               \psline[linewidth=0.8pt,linecolor=wwqqzz](2.,0.)(6.,0.)
               \rput[tl](-0.68,1.4){$\red{\mathscr{D}_1}$}
               \rput[tl](3.32,-0.5){$\ttzzqq{\mathscr{P}_2}$}
               \rput[tl](4.9,0.4){$\wwqqzz{\mathscr{D}_2}$}
               \rput[tl](-0.2,0.4){$\wwccff{\mathscr{P}_1}$}
               \rput[tl](2.32,2.4){$\ffzzcc{\mathscr{D}}$}
               \psline[linewidth=1.2pt,linecolor=ttzzqq](0.,2.)(4.,2.)
          \end{pspicture*}
     \end{extern}
\end{center}
\begin{h2}3. Vecteurs de l'espace\end{h2}
\cadre{bleu}{Définition (vecteurs colinéaires)}{% id="d80"
     Soient $\vec{u}$ et $\vec{v}$ deux vecteurs non nuls de l'espace. On dit que les vecteurs $\vec{u}$ et $\vec{v}$ sont \textbf{colinéaires} si et seulement si il existe un réel $k$ tel que $\vec{u}=k\vec{v}$
}
\bloc{cyan}{Remarques}{% id="r80"
     \begin{itemize}
          \item Par  convention, on considèrera que le vecteur nul $\overrightarrow{0}$ est colinéaire a n'importe quel vecteur de l'espace
          \item Intuitivement, deux vecteurs sont colinéaires s'ils ont la même «direction» (mais pas nécessairement le même «sens»). La notion de vecteurs colinéaires est à rapprocher de la notion de droites parallèles (voir théorème ci-dessous).
     \end{itemize}
}
\cadre{vert}{Propriété}{% id="t85"
     Soient quatre points distincts $A$, $B$, $C$ et $D$.
     \par
     Les droites $\left(AB\right)$ et $\left(CD\right)$ sont parallèles si et seulement si les vecteurs $\overrightarrow{AB}$ et $\overrightarrow{CD}$ sont colinéaires.
}
\cadre{vert}{Propriété}{% id="t90"
     Soient deux points distincts $A$ et $B$.
     \par
     Un point $M$ appartient à la droite $\left(AB\right)$ si et seulement si les vecteurs $\overrightarrow{AM}$ et $\overrightarrow{AB}$ sont colinéaires.
}
\begin{center}
     \begin{extern}%width="600" alt=""
          \newrgbcolor{qqqqcc}{0. 0. 0.8}
          \psset{xunit=1.0cm,yunit=1.0cm,algebraic=true,dimen=middle,dotstyle=o,dotsize=5pt 0,linewidth=1.6pt,arrowsize=3pt 2,arrowinset=0.25}
          \begin{pspicture*}(-3.12,-1.28)(7.98,2.3)
               \psplot[linewidth=0.8pt]{-3.12}{7.98}{(-1.344--0.96*x)/2.88}
               \psline[linewidth=0.8pt,linecolor=qqqqcc]{->}(-0.58,-0.66)(2.3,0.3)
               \psline[linewidth=0.8pt,linecolor=red]{->}(-0.58,-0.66)(5.22,1.27)
               \begin{scriptsize}
                    \psdots[dotsize=1pt 0,dotstyle=*,linecolor=blue](-0.58,-0.66)
                    \rput[bl](-0.7,-0.54){\blue{$A$}}
                    \psdots[dotsize=1pt 0,dotstyle=*,linecolor=blue](2.3,0.3)
                    \rput[bl](2.1,0.4){\blue{$B$}}
                    \psdots[dotsize=1pt 0,dotstyle=*,linecolor=red](5.22,1.27)
                    \rput[bl](5.04,1.44){\red{$M$}}
               \end{scriptsize}
          \end{pspicture*}
     \end{extern}
\end{center}
\bloc{cyan}{Remarque}{% id="r90"
     Le vecteur $\overrightarrow{AB}$ où $A$ et $B$ sont deux points distincts de la droite $\mathscr D$ est appelé \textbf{vecteur directeur} de $\mathscr D$.
}
\cadre{bleu}{Définition (vecteurs coplanaires)}{% id="d100"
     Soient $\vec{u}$ et $\vec{v}$ deux vecteurs non nuls et non colinéaires. On dit que le vecteur $\vec{w}$ est coplanaires à $\vec{u}$ et $\vec{v}$ si et seulement si il existe deux réels $k$ et $k^{\prime}$ tels que~:
     \begin{center}$\vec{w}=k\vec{u}+k^{\prime}\vec{v}$\end{center}
}
\bloc{cyan}{Remarques}{% id="r100"
     \begin{itemize}
          \item Intuitivement, le fait que $\vec{u}$, $\vec{v}$ et $\vec{w}$ soient coplanaires signifie que si on choisit quatre points $O, A, B, C$ tels que $\vec{u}=\overrightarrow{OA}, \vec{v}=\overrightarrow{OB}$ et $\vec{w}=\overrightarrow{OC}$ alors les points $O, A, B$ et $C$ appartiennent à un même plan.
          \begin{center}
               \begin{extern}%width="500" alt=""
                    \newrgbcolor{aqaqaq}{0.6 0.6 0.6}
                    \newrgbcolor{qqqqcc}{0. 0. 0.8}
                    \newrgbcolor{qqccqq}{0. 0.8 0.}
                    \newrgbcolor{qqttcc}{0. 0.2 0.8}
                    \psset{xunit=1.0cm,yunit=1.0cm,algebraic=true,dimen=middle,dotstyle=o,dotsize=5pt 0,linewidth=1.6pt,arrowsize=3pt 2,arrowinset=0.25}
                    \begin{pspicture*}(-4.5,-2.1)(8.7,4.7)
                         \pspolygon[linewidth=0.4pt,linecolor=aqaqaq,fillcolor=aqaqaq,fillstyle=solid,opacity=0.1](0.,1.)(-4.,-2.)(4.,-2.)(8.,1.)
                         \psline[linewidth=0.4pt,linecolor=aqaqaq](0.,1.)(-4.,-2.)
                         \psline[linewidth=0.4pt,linecolor=aqaqaq](-4.,-2.)(4.,-2.)
                         \psline[linewidth=0.4pt,linecolor=aqaqaq](4.,-2.)(8.,1.)
                         \psline[linewidth=0.4pt,linecolor=aqaqaq](8.,1.)(0.,1.)
                         \psline[linewidth=0.8pt,linecolor=qqqqcc]{->}(1.97,1.55)(5.97,1.55)
                         \psline[linewidth=0.8pt,linecolor=qqqqcc]{->}(-1.,-1.)(3.,-1.)
                         \psline[linewidth=0.8pt,linecolor=qqccqq]{->}(2.,2.)(3.,3.)
                         \psline[linewidth=0.8pt,linecolor=qqccqq]{->}(-1.,-1.)(0.,0.)
                         \psline[linewidth=0.8pt,linecolor=red]{->}(-1.,-1.)(2.,0.)
                         \psline[linewidth=0.8pt,linecolor=red]{->}(2.027,3.228)(5.02662,4.22802)
                         \rput[tl](3.88,1.94){$\qqttcc{\vec{u}}$}
                         \rput[tl](2.19,2.76){$\qqccqq{\vec{v}}$}
                         \rput[tl](3.26,4.15){$\red{\vec{w}}$}
                         \psline[linewidth=0.4pt,linestyle=dashed,dash=4pt 4pt,linecolor=gray](0.,0.)(2.,0.)
                         \rput[tl](0.45,-0.1){$\red{\vec{w}}$}
                         \rput[tl](-0.766,-0.21){$\qqccqq{\vec{v}}$}
                         \rput[tl](1.04,-0.66){$\qqttcc{\vec{u}}$}
                         \psline[linewidth=0.4pt,linestyle=dashed,dash=4pt 4pt,linecolor=gray](2.,0.)(1.,-1.)
                         \rput[tl](3.15,-0.85){$\gray{A}$}
                         \rput[tl](0,0.5){$\gray{B}$}
                         \rput[tl](2.03,0.4){$\gray{C}$}
                         \rput[tl](-1.3,-0.84){$\gray{O}$}
                         \begin{scriptsize}
                              \psdots[dotsize=1pt 0,dotstyle=*,linecolor=aqaqaq](-1.,-1.)
                              \psdots[dotsize=1pt 0,dotstyle=*,linecolor=aqaqaq](3.,-1.)
                              \psdots[dotsize=1pt 0,dotstyle=*,linecolor=aqaqaq](0.,0.)
                              \psdots[dotsize=1pt 0,dotstyle=*,linecolor=aqaqaq](2.,0.)
                         \end{scriptsize}
                    \end{pspicture*}
               \end{extern}
          \end{center}
          \begin{center}\textit{(sur la figure ci-dessus }$\vec{w}=\frac{1}{2}\vec{u}+\vec{v}$\textit{)}\end{center}
          \item La définition précédente peut se généraliser à plus de trois vecteurs. Pour \textbf{deux} vecteurs, par contre, elle n'a guère d'intérêt car deux vecteurs sont toujours coplanaires
          \item Pour des vecteurs ou des points coplanaires, on peut se placer dans le plan contenant ces points ou ces vecteurs et appliquer les résultats classiques de géométrie plane (théorème des milieux, théorème de Thalès, relation de Chasles, etc.)
     \end{itemize}
}
\cadre{vert}{Propriété}{% id="p105"
     Soient $A, B, C$ trois points non alignés.
     \par
     Le point $M$ appartient au plan $\left(ABC\right)$ si et seulement si les vecteurs $\overrightarrow{AB}, \overrightarrow{AC}$ et $\overrightarrow{AM}$ sont coplanaires, c'est à dire si et seulement il existe deux réels $k$ et $k^{\prime}$ tels que~:
     \begin{center}$\overrightarrow{AM}=k\overrightarrow{AB}+k^{\prime}\overrightarrow{AC}$\end{center}
}
\begin{center}
     \begin{extern}%width="400" alt=""
          \newrgbcolor{aqaqaq}{0.62 0.62 0.62}
          \psset{xunit=1.0cm,yunit=1.0cm,algebraic=true,dimen=middle,dotstyle=o,dotsize=5pt 0,linewidth=1.6pt,arrowsize=3pt 2,arrowinset=0.25}
          \begin{pspicture*}(-2.5,0.6)(7.3,3.5)
               \pspolygon[linewidth=0.8pt,linecolor=aqaqaq,fillcolor=aqaqaq,fillstyle=solid,opacity=0.1](0.575,3)(-1.42,1.)(4.63,1)(6.63,3)
               \psline[linewidth=0.8pt,linecolor=aqaqaq](0.57,3)(-1.42,1)
               \psline[linewidth=0.8pt,linecolor=aqaqaq](-1.42,1)(4.63,1)
               \psline[linewidth=0.8pt,linecolor=aqaqaq](4.63,1)(6.63,3)
               \psline[linewidth=0.8pt,linecolor=aqaqaq](6.6,3)(0.57,3)
               \psline[linewidth=0.8pt]{->}(1.,2.)(3.43,1.36)
               \psline[linewidth=0.8pt]{->}(1.,2.)(2.95,2.56)
               \psline[linewidth=0.8pt,linecolor=red]{->}(1.,2.)(4.25,1.95)
               \psline[linewidth=0.8pt,linestyle=dashed,dash=3pt 3pt,linecolor=aqaqaq](4.25,1.95)(2.47,2.42)
               \psline[linewidth=0.8pt,linestyle=dashed,dash=3pt 3pt,linecolor=gray](4.25,1.95)(2.82,1.52)
               \begin{scriptsize}
                    \psdots[dotsize=1pt 0,dotstyle=*,linecolor=gray](1.,2.)
                    \rput[bl](0.65,2){\gray{$A$}}
                    \psdots[dotsize=1pt 0,dotstyle=*,linecolor=gray](3.43,1.36)
                    \rput[bl](3.51,1.39){\gray{$B$}}
                    \psdots[dotsize=1pt 0,dotstyle=*,linecolor=gray](2.95,2.56)
                    \rput[bl](3.03,2.59){\gray{$C$}}
                    \psdots[dotsize=1pt 0,dotstyle=*,linecolor=red](4.25,1.95)
                    \rput[bl](4.34,2){\red{$M$}}
               \end{scriptsize}
          \end{pspicture*}
     \end{extern}
\end{center}
\bloc{cyan}{Remarque}{% id="r105"
     \textbf{Attention ! } Notez que dans la propriété ci-dessus, tous les vecteurs ont la même origine $A$.
     \par
     Le fait que des vecteurs $\overrightarrow{AB}$ et $\overrightarrow{CD}$ soient coplanaires ne signifie pas que les points $A, B, C, D$ soient coplanaires.
     \par
     Par exemple, si on considère le cube ci-dessous, $\overrightarrow{AB}$ et $\overrightarrow{FG}$ sont coplanaires (parce que $\overrightarrow{FG}=\overrightarrow{AD}$ ou tout simplement parce que \textbf{deux} vecteurs sont toujours coplanaires !) mais les points $A, B, F$ et $G$ ne le sont pas.
     \begin{center}
          \begin{extern}%width="380" alt=""
               \newrgbcolor{wqwqwq}{0.37 0.37 0.37}
               \newrgbcolor{qqccqq}{0. 0.8 0.}
               \psset{xunit=1.0cm,yunit=1.0cm,algebraic=true,dimen=middle,dotstyle=o,dotsize=5pt 0,linewidth=1.6pt,arrowsize=3pt 2,arrowinset=0.25}
               \begin{pspicture*}(1.,1.)(8.,7.)
                    \psline[linewidth=0.8pt](2.,1.)(6.,1.)
                    \psline[linewidth=0.8pt,linecolor=wqwqwq](6.,1.)(7.58,2.3)
                    \psline[linewidth=0.8pt](7.58,2.3)(7.58,6.3)
                    \psline[linewidth=0.8pt,linecolor=wqwqwq](7.58,6.3)(6.,5.)
                    \psline[linewidth=0.8pt](6.,5.)(2.,5.)
                    \psline[linewidth=0.8pt,linecolor=gray](2.,5.)(3.58,6.3)
                    \psline[linewidth=0.8pt](3.58,6.3)(7.58,6.3)
                    \psline[linewidth=0.8pt](2.,5.)(2.,1.)
                    \psline[linewidth=0.8pt](6.,5.)(6.,1.)
                    \psline[linewidth=0.8pt,linestyle=dashed,dash=3pt 3pt](2.,1.)(3.58,2.3)
                    \psline[linewidth=0.8pt,linestyle=dashed,dash=3pt 3pt](3.58,2.3)(3.58,6.3)
                    \psline[linewidth=0.8pt,linestyle=dashed,dash=3pt 3pt](3.58,2.3)(7.58,2.3)
                    \psline[linewidth=0.8pt,linecolor=red]{->}(2.,1.)(6.,1.)
                    \psline[linewidth=0.8pt,linecolor=qqccqq]{->}(6.,5.)(7.58,6.3)
                    \begin{scriptsize}
                         \psdots[dotsize=1pt 0,dotstyle=*,linecolor=gray](2.,1.)
                         \rput[bl](1.74,0.67){\gray{$A$}}
                         \psdots[dotsize=1pt 0,dotstyle=*,linecolor=gray](6.,1.)
                         \rput[bl](5.91,0.67){\gray{$B$}}
                         \psdots[dotsize=1pt 0,dotstyle=*,linecolor=gray](7.58,2.3)
                         \rput[bl](7.69,2.21){\gray{$C$}}
                         \psdots[dotsize=1pt 0,dotstyle=*,linecolor=gray](3.58,2.3)
                         \rput[bl](3.25,2.27){\gray{$D$}}
                         \psdots[dotsize=1pt 0,dotstyle=*,linecolor=gray](2.,5.)
                         \rput[bl](1.63,4.80){\gray{$E$}}
                         \psdots[dotsize=1pt 0,dotstyle=*,linecolor=gray](6.,5.)
                         \rput[bl](5.63,4.66){\gray{$F$}}
                         \psdots[dotsize=1pt 0,dotstyle=*,linecolor=gray](7.58,6.3)
                         \rput[bl](7.69,6.36){\gray{$G$}}
                         \psdots[dotsize=1pt 0,dotstyle=*,linecolor=gray](3.58,6.3)
                         \rput[bl](3.27,6.44){\gray{$H$}}
                    \end{scriptsize}
               \end{pspicture*}
          \end{extern}
     \end{center}
     \begin{center}
     \textit{Vecteurs coplanaires - Points non coplanaires}\end{center}
}
\begin{h2}4. Repérage - Représentations paramétriques\end{h2}
\cadre{bleu}{Définition}{% id="d110"
     Un repère de l'espace est un quadruplet $\left(O,\vec{i},\vec{j},\vec{k}\right)$ où $O$ est un point et $\vec{i}, \vec{j}, \vec{k}$ trois vecteurs \textbf{non coplanaires}.
}
\cadre{vert}{Définition}{% id="p120"
     Pour tout point $M$ de l'espace, il existe trois réels $x, y$ et $z$ tels que~:
     \begin{center}$\overrightarrow{OM}=x\vec{i}+y\vec{j}+z\vec{k}$\end{center}
     $\left(x~; y~; z\right)$ s'appellent les \textbf{coordonnées} de $M$ dans le repère  $\left(O,\vec{i},\vec{j},\vec{k}\right)$
}
\begin{center}
     \begin{extern}%width="380" alt=""
          \newrgbcolor{ttttff}{0.2 0.2 1.}
          \newrgbcolor{qqqqcc}{0. 0. 0.8}
          \psset{xunit=1.5cm,yunit=1.5cm,algebraic=true,dimen=middle,dotstyle=o,dotsize=5pt 0,linewidth=1.6pt,arrowsize=3pt 2,arrowinset=0.25}
          \begin{pspicture*}(-2.,-1.5)(5.4,4.6)
               \psline[linewidth=0.8pt](1.,0.)(1.,4.6)
               \psline[linewidth=0.8pt](1.,0.)(5.3,0)
               \psline[linewidth=0.8pt](1.,0.)(-0.95,-1.26)
               \psplot[linewidth=0.8pt]{1.}{1.}{(--2.-2.*x)/-3.}
               \psline[linewidth=0.8pt,linecolor=ttttff]{->}(1.,0.)(2.,0.)
               \psline[linewidth=0.8pt,linecolor=ttttff]{->}(1.,0.)(0.35,-0.42)
               \psline[linewidth=0.8pt,linecolor=ttttff]{->}(1.,0.)(1.,1.)
               \psline[linewidth=0.8pt,linestyle=dashed,dash=2pt 2pt](4.,3.)(1.,4.)
               \psline[linewidth=0.8pt,linestyle=dashed,dash=2pt 2pt](4.,3.)(4.,-1.)
               \psline[linewidth=0.8pt,linestyle=dashed,dash=2pt 2pt](4.,-1.)(5.,0.)
               \psline[linewidth=0.8pt,linestyle=dashed,dash=2pt 2pt](4.,-1.)(-0.51,-1.01)
               \psline[linewidth=0.8pt,linestyle=dashed,dash=2pt 2pt](1.,0.)(4.,-1.)
               \rput[tl](0.56,0.25){$\ttttff{\vec{i}}$}
               \rput[tl](1.43,0.52){$\ttttff{\vec{j}}$}
               \rput[tl](0.75,0.61){$\ttttff{\vec{k}}$}
               \rput[tl](-0.92,-0.87){$\red{x}$}
               \rput[tl](4.85,0.38){$\red{y}$}
               \rput[tl](0.67,4.1){$\red{z}$}
               \rput[tl](4.00,3.35){$\red{M}$}
               \rput[tl](1.1,0.41){$\qqqqcc{O}$}
               \begin{scriptsize}
                    \psdots[dotsize=1pt 0,dotstyle=*,linecolor=blue](1.,0.)
                    \psdots[dotsize=1pt 0,dotstyle=*,linecolor=red](4.,3.)
               \end{scriptsize}
          \end{pspicture*}
     \end{extern}
\end{center}
\bloc{cyan}{Remarques}{% id="r120"
     \begin{itemize}
          \item $x, y$ et $z$ s'appellent respectivement l'\textbf{abscisse}, l'\textbf{ordonnée} et la \textbf{cote} du point $M$
          \item Comme dans le plan, on définit également les coordonnées d'un vecteur de la façon suivante~:
          \par
          les coordonnées du vecteur $\vec{u}$ dans le repère  $\left(O,\vec{i},\vec{j},\vec{k}\right)$ sont les coordonnées du point $M$ tel que $\overrightarrow{OM}=\vec{u}$
     \end{itemize}
}
\cadre{vert}{Propriétés}{% id="p130"
     Pour tous points $A \left(x_{A}~; y_{A}~; z_{A}\right)$ et $B\left(x_{B}~; y_{B}~; z_{B}\right)$~:
     \begin{itemize}
          \item le vecteur $\overrightarrow{AB}$ a pour coordonnées $\left(x_{B}-x_{A}~; y_{B}-y_{A}~; z_{B}-z_{A}\right)$
          \item le point $M$ milieu de $\left[AB\right]$ a pour coordonnées $\left(\frac{x_{A}+x_{B}}{2}~; \frac{y_{A}+y_{B}}{2}~; \frac{z_{A}+z_{B}}{2} \right)$
     \end{itemize}
}
\bloc{cyan}{Remarque}{% id="r130"
     Les notions relatives aux repères orthonormés, distances, etc. sont abordées dans le chapitre «Orthogonalité et produit scalaire»
}
\cadre{rouge}{Théorème et définition (Représentation paramétrique d'une droite)}{% id="t140"
     Un point $M\left(x~; y~; z\right)$ appartient à la droite $\mathscr D$ passant par $A\left(x_{A}~; y_{A}~; z_{A}\right)$ et de vecteur directeur $\vec{u}\left(a~; b~; c\right)$ si et seulement si il existe un réel $k$ tel que~:
     \begin{center}$\left\{ \begin{matrix}  x=x_{A}+ak  \\  y=y_{A}+bk  \\  z=z_{A}+ck    \end{matrix}\right.   $    avec $k \in  \mathbb{R}$\end{center}
     Ce système est appelé \textbf{représentation paramétrique de la droite} $\mathscr D$
}
\bloc{cyan}{Démonstration}{% id="m140"
     Elle est immédiate en utilisant~:
     \par
     $M\left(x~; y~; z\right) \in  \mathscr D  \Leftrightarrow   \overrightarrow{AM}$ et $\vec{u}$ sont colinéaires$   \Leftrightarrow    \overrightarrow{AM}=k\vec{u}$
}
\bloc{cyan}{Remarque}{% id="r140"
     Une droite admet une infinité de représentations paramétriques
}
\bloc{orange}{Exemple}{% id="e140"
     La droite passant par l'origine et de vecteur directeur $\vec{u}\left(1~; 1~; 1\right)$ a pour représentation paramétrique~:
     \begin{center}$\left\{ \begin{matrix}  x=k  \\  y=k  \\  z=k   \end{matrix}\right.    $    avec $k \in  \mathbb{R}$\end{center}
}
\cadre{rouge}{Théorème et définition (Représentation paramétrique d'un plan)}{% id="t150"
     Soient  $\mathscr P$ un plan passant par $A\left(x_{A}~; y_{A}~; z_{A}\right)$ et $\vec{u}\left(a~; b~; c\right)$ et $\vec{u}^{\prime}\left(a^{\prime}~; b^{\prime}~; c^{\prime}\right)$ deux vecteurs non colinéaires de ce plan.
     \par
     Un point $M\left(x~; y~; z\right)$ appartient au plan $\mathscr P$ si et seulement si il existe deux réels $k$ et $k^{\prime}$ tels que~:
     \begin{center}$\left\{ \begin{matrix}  x=x_{A}+ak+a^{\prime}k^{\prime}  \\  y=y_{A}+bk+b^{\prime}k^{\prime}  \\  z=z_{A}+ck+c^{\prime}k^{\prime}   \end{matrix}\right.    $    avec $k \in  \mathbb{R}$ et  $k^{\prime} \in  \mathbb{R}$\end{center}
     Ce système est appelé \textbf{représentation paramétrique du plan} $\mathscr P$
}
\bloc{cyan}{Démonstration}{% id="m150"
     On utilise le fait que~:
     \par
     $M\left(x~; y~; z\right) \in  \mathscr P  \Leftrightarrow   \overrightarrow{AM}$ est coplanaire à $\vec{u}$ et $\vec{u}^{\prime}$ $  \Leftrightarrow    \overrightarrow{AM}=k\vec{u}+k^{\prime}\vec{u}^{\prime}$
}
\bloc{cyan}{Remarque}{% id="r130"
     Là encore, un plan admet une infinité de représentations paramétriques
}
\bloc{orange}{Exemple}{% id="e150"
     Le plan $\left(xOy\right)$ passe par l'origine et $\vec{i}\left(1~; 0~; 0\right)$ et $\vec{j}\left(0~; 1~; 0\right)$ sont deux vecteurs non colinéaires de ce plan. Une représentation paramétrique du plan $\left(xOy\right)$ est donc~:
     \begin{center}$\left\{ \begin{matrix}  x=k  \\  y=k^{\prime}  \\  z=0  \end{matrix}\right.     $    avec $\left(k,k^{\prime}\right) \in  \mathbb{R}^{2}$\end{center}
}

\end{document}
µ
\documentclass[a4paper]{article}

%================================================================================================================================
%
% Packages
%
%================================================================================================================================

\usepackage[T1]{fontenc} 	% pour caractères accentués
\usepackage[utf8]{inputenc}  % encodage utf8
\usepackage[french]{babel}	% langue : français
\usepackage{fourier}			% caractères plus lisibles
\usepackage[dvipsnames]{xcolor} % couleurs
\usepackage{fancyhdr}		% réglage header footer
\usepackage{needspace}		% empêcher sauts de page mal placés
\usepackage{graphicx}		% pour inclure des graphiques
\usepackage{enumitem,cprotect}		% personnalise les listes d'items (nécessaire pour ol, al ...)
\usepackage{hyperref}		% Liens hypertexte
\usepackage{pstricks,pst-all,pst-node,pstricks-add,pst-math,pst-plot,pst-tree,pst-eucl} % pstricks
\usepackage[a4paper,includeheadfoot,top=2cm,left=3cm, bottom=2cm,right=3cm]{geometry} % marges etc.
\usepackage{comment}			% commentaires multilignes
\usepackage{amsmath,environ} % maths (matrices, etc.)
\usepackage{amssymb,makeidx}
\usepackage{bm}				% bold maths
\usepackage{tabularx}		% tableaux
\usepackage{colortbl}		% tableaux en couleur
\usepackage{fontawesome}		% Fontawesome
\usepackage{environ}			% environment with command
\usepackage{fp}				% calculs pour ps-tricks
\usepackage{multido}			% pour ps tricks
\usepackage[np]{numprint}	% formattage nombre
\usepackage{tikz,tkz-tab} 			% package principal TikZ
\usepackage{pgfplots}   % axes
\usepackage{mathrsfs}    % cursives
\usepackage{calc}			% calcul taille boites
\usepackage[scaled=0.875]{helvet} % font sans serif
\usepackage{svg} % svg
\usepackage{scrextend} % local margin
\usepackage{scratch} %scratch
\usepackage{multicol} % colonnes
%\usepackage{infix-RPN,pst-func} % formule en notation polanaise inversée
\usepackage{listings}

%================================================================================================================================
%
% Réglages de base
%
%================================================================================================================================

\lstset{
language=Python,   % R code
literate=
{á}{{\'a}}1
{à}{{\`a}}1
{ã}{{\~a}}1
{é}{{\'e}}1
{è}{{\`e}}1
{ê}{{\^e}}1
{í}{{\'i}}1
{ó}{{\'o}}1
{õ}{{\~o}}1
{ú}{{\'u}}1
{ü}{{\"u}}1
{ç}{{\c{c}}}1
{~}{{ }}1
}


\definecolor{codegreen}{rgb}{0,0.6,0}
\definecolor{codegray}{rgb}{0.5,0.5,0.5}
\definecolor{codepurple}{rgb}{0.58,0,0.82}
\definecolor{backcolour}{rgb}{0.95,0.95,0.92}

\lstdefinestyle{mystyle}{
    backgroundcolor=\color{backcolour},   
    commentstyle=\color{codegreen},
    keywordstyle=\color{magenta},
    numberstyle=\tiny\color{codegray},
    stringstyle=\color{codepurple},
    basicstyle=\ttfamily\footnotesize,
    breakatwhitespace=false,         
    breaklines=true,                 
    captionpos=b,                    
    keepspaces=true,                 
    numbers=left,                    
xleftmargin=2em,
framexleftmargin=2em,            
    showspaces=false,                
    showstringspaces=false,
    showtabs=false,                  
    tabsize=2,
    upquote=true
}

\lstset{style=mystyle}


\lstset{style=mystyle}
\newcommand{\imgdir}{C:/laragon/www/newmc/assets/imgsvg/}
\newcommand{\imgsvgdir}{C:/laragon/www/newmc/assets/imgsvg/}

\definecolor{mcgris}{RGB}{220, 220, 220}% ancien~; pour compatibilité
\definecolor{mcbleu}{RGB}{52, 152, 219}
\definecolor{mcvert}{RGB}{125, 194, 70}
\definecolor{mcmauve}{RGB}{154, 0, 215}
\definecolor{mcorange}{RGB}{255, 96, 0}
\definecolor{mcturquoise}{RGB}{0, 153, 153}
\definecolor{mcrouge}{RGB}{255, 0, 0}
\definecolor{mclightvert}{RGB}{205, 234, 190}

\definecolor{gris}{RGB}{220, 220, 220}
\definecolor{bleu}{RGB}{52, 152, 219}
\definecolor{vert}{RGB}{125, 194, 70}
\definecolor{mauve}{RGB}{154, 0, 215}
\definecolor{orange}{RGB}{255, 96, 0}
\definecolor{turquoise}{RGB}{0, 153, 153}
\definecolor{rouge}{RGB}{255, 0, 0}
\definecolor{lightvert}{RGB}{205, 234, 190}
\setitemize[0]{label=\color{lightvert}  $\bullet$}

\pagestyle{fancy}
\renewcommand{\headrulewidth}{0.2pt}
\fancyhead[L]{maths-cours.fr}
\fancyhead[R]{\thepage}
\renewcommand{\footrulewidth}{0.2pt}
\fancyfoot[C]{}

\newcolumntype{C}{>{\centering\arraybackslash}X}
\newcolumntype{s}{>{\hsize=.35\hsize\arraybackslash}X}

\setlength{\parindent}{0pt}		 
\setlength{\parskip}{3mm}
\setlength{\headheight}{1cm}

\def\ebook{ebook}
\def\book{book}
\def\web{web}
\def\type{web}

\newcommand{\vect}[1]{\overrightarrow{\,\mathstrut#1\,}}

\def\Oij{$\left(\text{O}~;~\vect{\imath},~\vect{\jmath}\right)$}
\def\Oijk{$\left(\text{O}~;~\vect{\imath},~\vect{\jmath},~\vect{k}\right)$}
\def\Ouv{$\left(\text{O}~;~\vect{u},~\vect{v}\right)$}

\hypersetup{breaklinks=true, colorlinks = true, linkcolor = OliveGreen, urlcolor = OliveGreen, citecolor = OliveGreen, pdfauthor={Didier BONNEL - https://www.maths-cours.fr} } % supprime les bordures autour des liens

\renewcommand{\arg}[0]{\text{arg}}

\everymath{\displaystyle}

%================================================================================================================================
%
% Macros - Commandes
%
%================================================================================================================================

\newcommand\meta[2]{    			% Utilisé pour créer le post HTML.
	\def\titre{titre}
	\def\url{url}
	\def\arg{#1}
	\ifx\titre\arg
		\newcommand\maintitle{#2}
		\fancyhead[L]{#2}
		{\Large\sffamily \MakeUppercase{#2}}
		\vspace{1mm}\textcolor{mcvert}{\hrule}
	\fi 
	\ifx\url\arg
		\fancyfoot[L]{\href{https://www.maths-cours.fr#2}{\black \footnotesize{https://www.maths-cours.fr#2}}}
	\fi 
}


\newcommand\TitreC[1]{    		% Titre centré
     \needspace{3\baselineskip}
     \begin{center}\textbf{#1}\end{center}
}

\newcommand\newpar{    		% paragraphe
     \par
}

\newcommand\nosp {    		% commande vide (pas d'espace)
}
\newcommand{\id}[1]{} %ignore

\newcommand\boite[2]{				% Boite simple sans titre
	\vspace{5mm}
	\setlength{\fboxrule}{0.2mm}
	\setlength{\fboxsep}{5mm}	
	\fcolorbox{#1}{#1!3}{\makebox[\linewidth-2\fboxrule-2\fboxsep]{
  		\begin{minipage}[t]{\linewidth-2\fboxrule-4\fboxsep}\setlength{\parskip}{3mm}
  			 #2
  		\end{minipage}
	}}
	\vspace{5mm}
}

\newcommand\CBox[4]{				% Boites
	\vspace{5mm}
	\setlength{\fboxrule}{0.2mm}
	\setlength{\fboxsep}{5mm}
	
	\fcolorbox{#1}{#1!3}{\makebox[\linewidth-2\fboxrule-2\fboxsep]{
		\begin{minipage}[t]{1cm}\setlength{\parskip}{3mm}
	  		\textcolor{#1}{\LARGE{#2}}    
 	 	\end{minipage}  
  		\begin{minipage}[t]{\linewidth-2\fboxrule-4\fboxsep}\setlength{\parskip}{3mm}
			\raisebox{1.2mm}{\normalsize\sffamily{\textcolor{#1}{#3}}}						
  			 #4
  		\end{minipage}
	}}
	\vspace{5mm}
}

\newcommand\cadre[3]{				% Boites convertible html
	\par
	\vspace{2mm}
	\setlength{\fboxrule}{0.1mm}
	\setlength{\fboxsep}{5mm}
	\fcolorbox{#1}{white}{\makebox[\linewidth-2\fboxrule-2\fboxsep]{
  		\begin{minipage}[t]{\linewidth-2\fboxrule-4\fboxsep}\setlength{\parskip}{3mm}
			\raisebox{-2.5mm}{\sffamily \small{\textcolor{#1}{\MakeUppercase{#2}}}}		
			\par		
  			 #3
 	 		\end{minipage}
	}}
		\vspace{2mm}
	\par
}

\newcommand\bloc[3]{				% Boites convertible html sans bordure
     \needspace{2\baselineskip}
     {\sffamily \small{\textcolor{#1}{\MakeUppercase{#2}}}}    
		\par		
  			 #3
		\par
}

\newcommand\CHelp[1]{
     \CBox{Plum}{\faInfoCircle}{À RETENIR}{#1}
}

\newcommand\CUp[1]{
     \CBox{NavyBlue}{\faThumbsOUp}{EN PRATIQUE}{#1}
}

\newcommand\CInfo[1]{
     \CBox{Sepia}{\faArrowCircleRight}{REMARQUE}{#1}
}

\newcommand\CRedac[1]{
     \CBox{PineGreen}{\faEdit}{BIEN R\'EDIGER}{#1}
}

\newcommand\CError[1]{
     \CBox{Red}{\faExclamationTriangle}{ATTENTION}{#1}
}

\newcommand\TitreExo[2]{
\needspace{4\baselineskip}
 {\sffamily\large EXERCICE #1\ (\emph{#2 points})}
\vspace{5mm}
}

\newcommand\img[2]{
          \includegraphics[width=#2\paperwidth]{\imgdir#1}
}

\newcommand\imgsvg[2]{
       \begin{center}   \includegraphics[width=#2\paperwidth]{\imgsvgdir#1} \end{center}
}


\newcommand\Lien[2]{
     \href{#1}{#2 \tiny \faExternalLink}
}
\newcommand\mcLien[2]{
     \href{https~://www.maths-cours.fr/#1}{#2 \tiny \faExternalLink}
}

\newcommand{\euro}{\eurologo{}}

%================================================================================================================================
%
% Macros - Environement
%
%================================================================================================================================

\newenvironment{tex}{ %
}
{%
}

\newenvironment{indente}{ %
	\setlength\parindent{10mm}
}

{
	\setlength\parindent{0mm}
}

\newenvironment{corrige}{%
     \needspace{3\baselineskip}
     \medskip
     \textbf{\textsc{Corrigé}}
     \medskip
}
{
}

\newenvironment{extern}{%
     \begin{center}
     }
     {
     \end{center}
}

\NewEnviron{code}{%
	\par
     \boite{gray}{\texttt{%
     \BODY
     }}
     \par
}

\newenvironment{vbloc}{% boite sans cadre empeche saut de page
     \begin{minipage}[t]{\linewidth}
     }
     {
     \end{minipage}
}
\NewEnviron{h2}{%
    \needspace{3\baselineskip}
    \vspace{0.6cm}
	\noindent \MakeUppercase{\sffamily \large \BODY}
	\vspace{1mm}\textcolor{mcgris}{\hrule}\vspace{0.4cm}
	\par
}{}

\NewEnviron{h3}{%
    \needspace{3\baselineskip}
	\vspace{5mm}
	\textsc{\BODY}
	\par
}

\NewEnviron{margeneg}{ %
\begin{addmargin}[-1cm]{0cm}
\BODY
\end{addmargin}
}

\NewEnviron{html}{%
}

\begin{document}
\meta{url}{/cours/produit-scalaire-espace/}
\meta{pid}{555}
\meta{titre}{Orthogonalité et produit scalaire dans l'espace}
\meta{type}{cours}
\begin{h2}1. Produit scalaire\end{h2}
Deux vecteurs de l'espace sont toujours coplanaires (voir chapitre précédent). On peut alors définir le produit scalaire dans l'espace à l'aide de la définition donnée en Première pour deux vecteurs d'un plan.
\par
La plupart des propriétés vues en Première seront donc encore valables pour le produit scalaire dans l'espace, en particulier pour tous vecteurs $\vec{u}$ et $\vec{v}$ :
\begin{itemize}
     \item $\vec{u}.\vec{v}=||\vec{u}||\times ||\vec{v}||\times  \cos\left(\vec{u}, \vec{v}\right)$
     \item $\vec{u}.\vec{v}=\frac{1}{2} \left(||\vec{u}+\vec{v}||^{2}-||\vec{u}||^{2}-||\vec{v}||^{2}\right)$
     \item $\vec{u}^{2} = ||\vec{u}||^{2}$
\end{itemize}
La notion d'\textbf{orthogonalité de vecteurs} vue en Première est encore valable dans l'espace. Pour tous vecteurs $\vec{u}$ et $\vec{v}$ : $\vec{u}$ et $\vec{v}$ sont orthogonaux  $ \Leftrightarrow   \vec{u}.\vec{v}=0$.
\par
Les principales distinctions concernent les formules faisant intervenir les coordonnées puisque, dans l'espace, chaque vecteur possède trois coordonnées.
\cadre{vert}{Propriété}{% id="p10"
     L'espace est rapporté à un repère orthonormé $\left(O; \vec{i}, \vec{j}, \vec{k}\right)$
     \par
     Soient $\vec{u}$ et $\vec{v}$ deux vecteurs de coordonnées respectives $\left(x ; y ; z\right)$ et $\left(x^{\prime} ; y^{\prime} ; z^{\prime}\right)$ dans ce repère. Alors:
     \begin{center}$\vec{u}.\vec{v} =xx^{\prime}+yy^{\prime}+zz^{\prime}$\end{center}
}
\bloc{cyan}{Conséquences}{% id="r10"
     \begin{itemize}
          \item $||\vec{u}|| = \sqrt{x^{2}+y^{2}+z^{2}}$
          \item $AB=||\overrightarrow{AB}|| = \sqrt{\left(x_{B}-x_{A}\right)^{2}+\left(y_{B}-y_{A}\right)^{2}+\left(z_{B}-z_{A}\right)^{2}}$
     \end{itemize}
}
\begin{h2}2. Orthogonalité dans l'espace\end{h2}
\cadre{bleu}{Définition}{% id="d50"
     Deux droites $d_{1}$ et $d_{2}$ sont \textbf{orthogonales} si et seulement si il existe une droite qui est à la fois parallèle à $d_{1}$ et perpendiculaire à $d_{2}$
}
\begin{center}
     \begin{extern}%width="400" alt=""
          \newrgbcolor{afeeee}{0.7 0.9 0.9}
          \newrgbcolor{qqwuqq}{0. 0.4 0.}
          \newrgbcolor{wwzzff}{0.4 0.6 1.}
          \newrgbcolor{qqwwtt}{0. 0.4 0.2}
          \psset{xunit=1.0cm,yunit=1.0cm,algebraic=true,dimen=middle,dotstyle=o,dotsize=5pt 0,linewidth=1.6pt,arrowsize=3pt 2,arrowinset=0.25}
          \begin{pspicture*}(-5.6,-2.8)(5.6,3.5)
               \pspolygon[linewidth=0.8pt,linecolor=afeeee,fillcolor=afeeee,fillstyle=solid,opacity=0.1](-3.,0.)(-5.,-2.)(3.,-2.)(5.,0.)
               \psline[linewidth=0.8pt,linecolor=afeeee](-3.,0.)(-5.,-2.)
               \psline[linewidth=0.8pt,linecolor=afeeee](-5.,-2.)(3.,-2.)
               \psline[linewidth=0.8pt,linecolor=afeeee](3.,-2.)(5.,0.)
               \psline[linewidth=0.8pt,linecolor=afeeee](5.,0.)(-3.,0.)
               \rput[tl](-0.6097241218054483,2.8){$\qqwuqq{d_1}$}
               \rput[tl](2.672271188641368,-1.6){$\wwzzff{\mathscr{P}}$}
               \psline[linewidth=0.8pt,linecolor=qqwwtt](-4.69749899688947,0.6225150605167632)(0.39015188716309995,2.589170024100124)
               \psline[linewidth=0.8pt,linecolor=qqwuqq](-4.017190664344944,-2.)(0.8786019214951395,0.)
               \psline[linewidth=0.8pt,linecolor=red](-2.340323340435117,0.)(1.0884968352036755,-2)
               \psline[linewidth=0.4pt](-1.1428090352869673,-0.6557784132517245)(-0.9566973185809743,-0.5797491642822004)
               \psline[linewidth=0.4pt](-0.9566973185809743,-0.5797491642822004)(-0.7786761090029106,-0.6772364893524531)
               \rput[tl](0.812473846054839,-0.1){$\qqwuqq{d'_1}$}
               \rput[tl](1.2,-1.6){$\red{d_2}$}
          \end{pspicture*}
     \end{extern}
\end{center}
\begin{center}\textit{$d_{1}$ et $d_{2}$ sont orthogonales}\end{center}
\bloc{cyan}{Remarque}{% id="r50"
     Attention à ne pas confondre \textit{orthogonales} et \textit{perpendiculaires}. Le terme \textit{perpendiculaires} s'emploie uniquement pour des droites \textbf{sécantes} (donc \textbf{coplanaires}).
}
\cadre{vert}{Propriétés}{% id="t60"
     Soient deux droites $d_{1}$ et $d_{2}$, $\overrightarrow{u_{1}}$ un vecteur directeur de $d_{1}$ et $\overrightarrow{u_{2}}$ un vecteur directeur de $d_{2}$.
     \par
     $d_{1}$ et $d_{2}$ sont orthogonales si et seulement si les vecteurs $\overrightarrow{u_{1}}$ et $\overrightarrow{u_{2}}$ sont orthogonaux, c'est à dire si et seulement si $\overrightarrow{u_{1}}.\overrightarrow{u_{2}}=0$
}
\cadre{bleu}{Définition (Droite perpendiculaire à un plan)}{% id="d65"
     Une droite $d$ est \textbf{perpendiculaire} (ou \textbf{orthogonale}) à un plan $\mathscr P$ si et seulement si elle est orthogonale à toutes les droites incluses dans ce plan.
}
\begin{center}
     \begin{extern}%width="400" alt=""
          \newrgbcolor{afeeee}{0.7 0.9 0.9}
          \newrgbcolor{wwzzff}{0.4 0.6 1.}
          \newrgbcolor{qqwuqq}{0. 0.4 0.}
          \psset{xunit=1.0cm,yunit=1.0cm,algebraic=true,dimen=middle,dotstyle=o,dotsize=5pt 0,linewidth=1.6pt,arrowsize=3pt 2,arrowinset=0.25}
          \begin{pspicture*}(-5.5,-4.6)(6.4,2.6)
               \pspolygon[linewidth=0.8pt,linecolor=afeeee,fillcolor=afeeee,fillstyle=solid,opacity=0.1](-3.,0.)(-5.,-2.)(3.,-2.)(5.,0.)
               \psline[linewidth=0.8pt,linecolor=afeeee](-3.,0.)(-5.,-2.)
               \psline[linewidth=0.8pt,linecolor=afeeee](-5.,-2.)(3.,-2.)
               \psline[linewidth=0.8pt,linecolor=afeeee](3.,-2.)(5.,0.)
               \psline[linewidth=0.8pt,linecolor=afeeee](5.,0.)(-3.,0.)
               \rput[tl](0.20176035252591304,2.59630929745675){$\red{d}$}
               \rput[tl](2.6,-1.5){$\wwzzff{\mathscr{P}}$}
               \psline[linewidth=0.8pt,linecolor=red](0.,3.)(0.,-1.)
               \psline[linewidth=0.8pt,linecolor=red](0.,-2.)(0.,-7.)
               \begin{scriptsize}
                    \psdots[dotsize=1pt 0,dotstyle=*,linecolor=blue](0.,-1.)
               \end{scriptsize}
          \end{pspicture*}
     \end{extern}
\end{center}
\begin{center}\textbf{\textit{Droite perpendiculaire à un plan}}\end{center}
\bloc{cyan}{Remarque}{% id="r65"
     Une droite orthogonale à un plan coupe nécessairement ce plan en un point. Il n'y a donc plus lieu ici de distinguer orthogonalité et perpendicularité.
}
\cadre{vert}{Propriété}{% id="p70"
     La droite $d$ est perpendiculaire au plan $\mathscr P$ si et seulement si elle est orthogonale à \textbf{deux droites sécantes} incluses dans ce plan.
}
\begin{center}
     \begin{extern}%width="400" alt=""
          \newrgbcolor{afeeee}{0.7 0.9 0.9}
          \newrgbcolor{wwzzff}{0.4 0.6 1.}
          \newrgbcolor{qqwuqq}{0. 0.4 0.}
          \psset{xunit=1.0cm,yunit=1.0cm,algebraic=true,dimen=middle,dotstyle=o,dotsize=5pt 0,linewidth=1.6pt,arrowsize=3pt 2,arrowinset=0.25}
          \begin{pspicture*}(-5.5,-4.6)(6.4,2.6)
               \pspolygon[linewidth=0.8pt,linecolor=afeeee,fillcolor=afeeee,fillstyle=solid,opacity=0.1](-3.,0.)(-5.,-2.)(3.,-2.)(5.,0.)
               \psline[linewidth=0.8pt,linecolor=afeeee](-3.,0.)(-5.,-2.)
               \psline[linewidth=0.8pt,linecolor=afeeee](-5.,-2.)(3.,-2.)
               \psline[linewidth=0.8pt,linecolor=afeeee](3.,-2.)(5.,0.)
               \psline[linewidth=0.8pt,linecolor=afeeee](5.,0.)(-3.,0.)
               \rput[tl](0.20176035252591304,2.59630929745675){$\red{d}$}
               \rput[tl](2.6,-1.5){$\wwzzff{\mathscr{P}}$}
               \psline[linewidth=0.8pt,linecolor=red](0.,3.)(0.,-1.)
               \psline[linewidth=0.8pt,linecolor=red](0.,-2.)(0.,-7.)
               \psline[linewidth=0.8pt,linecolor=qqwuqq](-3.9857227292164437,-2.)(-0.5037395145067665,0.)
               \psline[linewidth=0.8pt,linecolor=qqwuqq](-3.332609727222885,-0.3326097272228852)(0.7024376775168468,-2.)
               \begin{scriptsize}
                    \psdots[dotsize=1pt 0,dotstyle=*,linecolor=blue](0.,-1.)
               \end{scriptsize}
          \end{pspicture*}
     \end{extern}
\end{center}
\cadre{bleu}{Définition (Plans perpendiculaires)}{% id="d70"
     Deux plans $\mathscr P_{1}$ et  $\mathscr P_{1}$ sont \textbf{perpendiculaires} (ou \textbf{orthogonaux}) si et seulement si $\mathscr P_{1}$ contient une droite $d$ perpendiculaire à $\mathscr P_{2}$.
}
\begin{center}
     \begin{extern}%width="350" alt=""
          \newrgbcolor{afeeee}{0.7 0.9 0.9}
          \newrgbcolor{zzffzz}{0.6 1. 0.6}
          \newrgbcolor{ccccff}{0.8 0.8 1.}
          \newrgbcolor{ffzzff}{1. 0.6 1.}
          \newrgbcolor{qqccqq}{0. 0.4 0.}
          \newrgbcolor{qqzzff}{0. 0.6 1.}
          \psset{xunit=1.0cm,yunit=1.0cm,algebraic=true,dimen=middle,dotstyle=o,dotsize=5pt 0,linewidth=1.6pt,arrowsize=3pt 2,arrowinset=0.25}
          \begin{pspicture*}(-2.3,-1.3)(6.9,7.4)
               \pspolygon[linewidth=0.8pt,linecolor=afeeee,fillcolor=afeeee,fillstyle=solid,opacity=0.1](0.061284443758347074,4.)(-1.938715556241653,2.)(4.061284443758347,2.)(6.061284443758347,4.)
               \pspolygon[linewidth=0.4pt,linecolor=qqccqq,fillcolor=qqccqq,fillstyle=solid,opacity=0.03](1.0080587194837574,-0.8588531512120481)(1.,5.)(3.,7.)(3.,1.)
               \psline[linewidth=0.8pt,linecolor=ccccff](1.0041264318847574,2.)(3.,4.)
               \psline[linewidth=0.8pt,linecolor=red](1.9956031708383426,2.0000048330325084)(1.9934714932952935,0.060720963628051905)
               \psline[linewidth=0.8pt,linecolor=red](1.9966964627665804,2.994622151160741)(2.,6.)
               \rput[tl](3.1583617599248424,7.0096617252930775){$\qqccqq{\mathscr{P}_1}$}
               \rput[tl](5.5,4.4){$\qqzzff{\mathscr{P}_2}$}
               \rput[tl](2.1,5.9){$\red{d}$}
          \end{pspicture*}
     \end{extern}
\end{center}
\bloc{cyan}{Remarque}{% id="r70"
     \textbf{Attention, } cela ne signifie \textbf{pas} que toutes les droites de $\mathscr P_{1}$ sont orthogonales à toutes les droites de  $\mathscr P_{2}$
}
\cadre{bleu}{Définition (Vecteur normal à un plan)}{% id="d80"
     On dit qu'un vecteur $\vec{n}$ non nul est un vecteur \textbf{normal} au plan $\mathscr P$ si et seulement si la droite dirigée par $\vec{n}$ est perpendiculaire au plan $\mathscr P$.
}
\begin{center}
     \begin{extern}%width="400" alt=""
          \newrgbcolor{afeeee}{0.7 0.9 0.9}
          \newrgbcolor{ttzzqq}{0.2 0.6 0.}
          \newrgbcolor{wwzzff}{0.4 0.6 1.}
          \psset{xunit=1.0cm,yunit=1.0cm,algebraic=true,dimen=middle,dotstyle=o,dotsize=5pt 0,linewidth=1.6pt,arrowsize=3pt 2,arrowinset=0.25}
          \begin{pspicture*}(-5.3,-3.8)(5.5,2.5)
               \pspolygon[linewidth=0.8pt,linecolor=afeeee,fillcolor=afeeee,fillstyle=solid,opacity=0.1](-3.,0.)(-5.,-2.)(3.,-2.)(5.,0.)
               \psline[linewidth=0.8pt,linecolor=afeeee](-3.,0.)(-5.,-2.)
               \psline[linewidth=0.8pt,linecolor=afeeee](-5.,-2.)(3.,-2.)
               \psline[linewidth=0.8pt,linecolor=afeeee](3.,-2.)(5.,0.)
               \psline[linewidth=0.8pt,linecolor=afeeee](5.,0.)(-3.,0.)
               \rput[tl](0.2,2.2){$\ttzzqq{d}$}
               \rput[tl](2.66,-1.5){$\wwzzff{\mathscr{P}}$}
               \psline[linewidth=0.8pt,linecolor=ttzzqq](0.,3.)(0.,-1.)
               \psline[linewidth=0.8pt,linecolor=ttzzqq](0.,-2.)(0.,-7.)
               \psline[linewidth=0.8pt,linecolor=red]{->}(0.,-1.)(0.,0.9122128406690555)
               \rput[tl](0.2,0.5){$\red{\vec{n}}$}
          \end{pspicture*}
     \end{extern}
\end{center}
\cadre{rouge}{Théorème}{% id="t90"
     Soit $\mathscr P$ un plan de vecteur normal $\vec{n}$ et soit $A$ un point de $\mathscr P$.
     \begin{center}$M \in  \mathscr P   \Leftrightarrow  \overrightarrow{AM}.\vec{n} = 0$.\end{center}
}
\cadre{rouge}{Théorème}{% id="t100"
     L'espace est rapporté à un repère orthonormé $\left(O; \vec{i}, \vec{j}, \vec{k}\right)$
     \par
     Le plan $\mathscr P$ de vecteur normal $\vec{n} \left(a ; b; c\right)$ admet une équation cartésienne de la forme :
     \begin{center}$ax+by+cz+d=0$\end{center}
     où $a$, $b$, $c$ sont les coordonnées de $\vec{n}$ et $d$ un nombre réel.
     \par
     \textbf{Réciproquement}, l'ensemble des points $M\left(x ; y ; z\right)$ tels que $ax+by+cz+d=0$ ($a, b, c, d$ étant des réels avec $a\neq 0$ ou $b\neq 0$ ou $c\neq 0$)  est un plan dont un vecteur normal est  $\vec{n}\left(a ; b ; c\right)$.
}
\bloc{cyan}{Démonstration}{% id="m100"
     Soit $A\left(x_{A} ; y_{A} ; z_{A}\right)$ un point de $\mathscr P$ :
     \par
     $M \in  \mathscr P   \Leftrightarrow   \overrightarrow{AM}.\vec{n} = 0   \Leftrightarrow   a\left(x-x_{A}\right)+b\left(y-y_{A}\right)+c\left(z-z_{A}\right)= 0$
     <span style="visibility:hidden">$M \in  \mathscr P   \Leftrightarrow   \overrightarrow{AM}.\vec{n} = 0 $</span>$ \Leftrightarrow   ax+by+cz-ax_{A}-by_{A}-cz_{A}= 0$
     \par
     et il suffit de poser $d=-ax_{A}-by_{A}-cz_{A}$.
     \par
     \textbf{Réciproquement}, supposons par exemple $a\neq 0$.
     \par
     Soit $A$ le point de coordonnées $\left(-\frac{d}{a} ; 0 ;0\right)$ Les coordonnées de $A$ vérifient :
     \par
     $ax_{A}+by_{A}+cz_{A}+d= 0$.
     \par
     On a alors $d = -ax_{A}-by_{A}-cz_{A}$ donc :
     \par
     $ax+by+cz+d=0  \Leftrightarrow   a\left(x-x_{A}\right)+b\left(y-y_{A}\right)+c\left(z-z_{A}\right)= 0  \Leftrightarrow   \overrightarrow{AM}.\vec{n} = 0$
     \par
     donc $M\left(x ; y ; z\right)$ appartient au plan passant par $A$ et dont un vecteur normal est $\vec{n}\left(a ; b ; c\right)$
}
\bloc{orange}{Exemple}{% id="e100"
     On cherche une équation cartésienne du plan passant par $A\left(1 ; 3 ; -2\right)$ et de vecteur normal $\vec{n}\left(1 ; 1 ; 1\right)$.
     \par
     Ce plan admet une équation cartésienne de la forme :
     \par
     $\left(E\right)      x+y+z+d=0$
     \par
     Le point $A\left(1 ; 3 ; -2\right)$ appartient à ce plan, donc les coordonnées de $A$ vérifient l'équation $\left(E\right)$ :
     \par
     $1+3-2+d=0$ soit $d=-2$
     \par
     Une équation cartésienne du plan est donc :
     \par
     $\left(E\right)      x+y+z-2=0$
}
\cadre{vert}{Propriétés}{% id="p110"
     \begin{itemize}
          \item Une droite $d$ est parallèle à un plan $\mathscr P$ si et seulement si un vecteur directeur de $d$ est orthogonal à un vecteur normal de $\mathscr P$.
          \item Une droite $d$ est perpendiculaire à un plan $\mathscr P$ si et seulement si un vecteur directeur de $d$ est colinéaire à un vecteur normal de $\mathscr P$.
          \item Deux plans sont parallèles si et seulement si leurs vecteurs normaux sont colinéaires.
          \item Deux plans sont perpendiculaires si et seulement si leurs vecteurs normaux sont  orthogonaux.
     \end{itemize}
}

\end{document}
µ
\documentclass[a4paper]{article}

%================================================================================================================================
%
% Packages
%
%================================================================================================================================

\usepackage[T1]{fontenc} 	% pour caractères accentués
\usepackage[utf8]{inputenc}  % encodage utf8
\usepackage[french]{babel}	% langue : français
\usepackage{fourier}			% caractères plus lisibles
\usepackage[dvipsnames]{xcolor} % couleurs
\usepackage{fancyhdr}		% réglage header footer
\usepackage{needspace}		% empêcher sauts de page mal placés
\usepackage{graphicx}		% pour inclure des graphiques
\usepackage{enumitem,cprotect}		% personnalise les listes d'items (nécessaire pour ol, al ...)
\usepackage{hyperref}		% Liens hypertexte
\usepackage{pstricks,pst-all,pst-node,pstricks-add,pst-math,pst-plot,pst-tree,pst-eucl} % pstricks
\usepackage[a4paper,includeheadfoot,top=2cm,left=3cm, bottom=2cm,right=3cm]{geometry} % marges etc.
\usepackage{comment}			% commentaires multilignes
\usepackage{amsmath,environ} % maths (matrices, etc.)
\usepackage{amssymb,makeidx}
\usepackage{bm}				% bold maths
\usepackage{tabularx}		% tableaux
\usepackage{colortbl}		% tableaux en couleur
\usepackage{fontawesome}		% Fontawesome
\usepackage{environ}			% environment with command
\usepackage{fp}				% calculs pour ps-tricks
\usepackage{multido}			% pour ps tricks
\usepackage[np]{numprint}	% formattage nombre
\usepackage{tikz,tkz-tab} 			% package principal TikZ
\usepackage{pgfplots}   % axes
\usepackage{mathrsfs}    % cursives
\usepackage{calc}			% calcul taille boites
\usepackage[scaled=0.875]{helvet} % font sans serif
\usepackage{svg} % svg
\usepackage{scrextend} % local margin
\usepackage{scratch} %scratch
\usepackage{multicol} % colonnes
%\usepackage{infix-RPN,pst-func} % formule en notation polanaise inversée
\usepackage{listings}

%================================================================================================================================
%
% Réglages de base
%
%================================================================================================================================

\lstset{
language=Python,   % R code
literate=
{á}{{\'a}}1
{à}{{\`a}}1
{ã}{{\~a}}1
{é}{{\'e}}1
{è}{{\`e}}1
{ê}{{\^e}}1
{í}{{\'i}}1
{ó}{{\'o}}1
{õ}{{\~o}}1
{ú}{{\'u}}1
{ü}{{\"u}}1
{ç}{{\c{c}}}1
{~}{{ }}1
}


\definecolor{codegreen}{rgb}{0,0.6,0}
\definecolor{codegray}{rgb}{0.5,0.5,0.5}
\definecolor{codepurple}{rgb}{0.58,0,0.82}
\definecolor{backcolour}{rgb}{0.95,0.95,0.92}

\lstdefinestyle{mystyle}{
    backgroundcolor=\color{backcolour},   
    commentstyle=\color{codegreen},
    keywordstyle=\color{magenta},
    numberstyle=\tiny\color{codegray},
    stringstyle=\color{codepurple},
    basicstyle=\ttfamily\footnotesize,
    breakatwhitespace=false,         
    breaklines=true,                 
    captionpos=b,                    
    keepspaces=true,                 
    numbers=left,                    
xleftmargin=2em,
framexleftmargin=2em,            
    showspaces=false,                
    showstringspaces=false,
    showtabs=false,                  
    tabsize=2,
    upquote=true
}

\lstset{style=mystyle}


\lstset{style=mystyle}
\newcommand{\imgdir}{C:/laragon/www/newmc/assets/imgsvg/}
\newcommand{\imgsvgdir}{C:/laragon/www/newmc/assets/imgsvg/}

\definecolor{mcgris}{RGB}{220, 220, 220}% ancien~; pour compatibilité
\definecolor{mcbleu}{RGB}{52, 152, 219}
\definecolor{mcvert}{RGB}{125, 194, 70}
\definecolor{mcmauve}{RGB}{154, 0, 215}
\definecolor{mcorange}{RGB}{255, 96, 0}
\definecolor{mcturquoise}{RGB}{0, 153, 153}
\definecolor{mcrouge}{RGB}{255, 0, 0}
\definecolor{mclightvert}{RGB}{205, 234, 190}

\definecolor{gris}{RGB}{220, 220, 220}
\definecolor{bleu}{RGB}{52, 152, 219}
\definecolor{vert}{RGB}{125, 194, 70}
\definecolor{mauve}{RGB}{154, 0, 215}
\definecolor{orange}{RGB}{255, 96, 0}
\definecolor{turquoise}{RGB}{0, 153, 153}
\definecolor{rouge}{RGB}{255, 0, 0}
\definecolor{lightvert}{RGB}{205, 234, 190}
\setitemize[0]{label=\color{lightvert}  $\bullet$}

\pagestyle{fancy}
\renewcommand{\headrulewidth}{0.2pt}
\fancyhead[L]{maths-cours.fr}
\fancyhead[R]{\thepage}
\renewcommand{\footrulewidth}{0.2pt}
\fancyfoot[C]{}

\newcolumntype{C}{>{\centering\arraybackslash}X}
\newcolumntype{s}{>{\hsize=.35\hsize\arraybackslash}X}

\setlength{\parindent}{0pt}		 
\setlength{\parskip}{3mm}
\setlength{\headheight}{1cm}

\def\ebook{ebook}
\def\book{book}
\def\web{web}
\def\type{web}

\newcommand{\vect}[1]{\overrightarrow{\,\mathstrut#1\,}}

\def\Oij{$\left(\text{O}~;~\vect{\imath},~\vect{\jmath}\right)$}
\def\Oijk{$\left(\text{O}~;~\vect{\imath},~\vect{\jmath},~\vect{k}\right)$}
\def\Ouv{$\left(\text{O}~;~\vect{u},~\vect{v}\right)$}

\hypersetup{breaklinks=true, colorlinks = true, linkcolor = OliveGreen, urlcolor = OliveGreen, citecolor = OliveGreen, pdfauthor={Didier BONNEL - https://www.maths-cours.fr} } % supprime les bordures autour des liens

\renewcommand{\arg}[0]{\text{arg}}

\everymath{\displaystyle}

%================================================================================================================================
%
% Macros - Commandes
%
%================================================================================================================================

\newcommand\meta[2]{    			% Utilisé pour créer le post HTML.
	\def\titre{titre}
	\def\url{url}
	\def\arg{#1}
	\ifx\titre\arg
		\newcommand\maintitle{#2}
		\fancyhead[L]{#2}
		{\Large\sffamily \MakeUppercase{#2}}
		\vspace{1mm}\textcolor{mcvert}{\hrule}
	\fi 
	\ifx\url\arg
		\fancyfoot[L]{\href{https://www.maths-cours.fr#2}{\black \footnotesize{https://www.maths-cours.fr#2}}}
	\fi 
}


\newcommand\TitreC[1]{    		% Titre centré
     \needspace{3\baselineskip}
     \begin{center}\textbf{#1}\end{center}
}

\newcommand\newpar{    		% paragraphe
     \par
}

\newcommand\nosp {    		% commande vide (pas d'espace)
}
\newcommand{\id}[1]{} %ignore

\newcommand\boite[2]{				% Boite simple sans titre
	\vspace{5mm}
	\setlength{\fboxrule}{0.2mm}
	\setlength{\fboxsep}{5mm}	
	\fcolorbox{#1}{#1!3}{\makebox[\linewidth-2\fboxrule-2\fboxsep]{
  		\begin{minipage}[t]{\linewidth-2\fboxrule-4\fboxsep}\setlength{\parskip}{3mm}
  			 #2
  		\end{minipage}
	}}
	\vspace{5mm}
}

\newcommand\CBox[4]{				% Boites
	\vspace{5mm}
	\setlength{\fboxrule}{0.2mm}
	\setlength{\fboxsep}{5mm}
	
	\fcolorbox{#1}{#1!3}{\makebox[\linewidth-2\fboxrule-2\fboxsep]{
		\begin{minipage}[t]{1cm}\setlength{\parskip}{3mm}
	  		\textcolor{#1}{\LARGE{#2}}    
 	 	\end{minipage}  
  		\begin{minipage}[t]{\linewidth-2\fboxrule-4\fboxsep}\setlength{\parskip}{3mm}
			\raisebox{1.2mm}{\normalsize\sffamily{\textcolor{#1}{#3}}}						
  			 #4
  		\end{minipage}
	}}
	\vspace{5mm}
}

\newcommand\cadre[3]{				% Boites convertible html
	\par
	\vspace{2mm}
	\setlength{\fboxrule}{0.1mm}
	\setlength{\fboxsep}{5mm}
	\fcolorbox{#1}{white}{\makebox[\linewidth-2\fboxrule-2\fboxsep]{
  		\begin{minipage}[t]{\linewidth-2\fboxrule-4\fboxsep}\setlength{\parskip}{3mm}
			\raisebox{-2.5mm}{\sffamily \small{\textcolor{#1}{\MakeUppercase{#2}}}}		
			\par		
  			 #3
 	 		\end{minipage}
	}}
		\vspace{2mm}
	\par
}

\newcommand\bloc[3]{				% Boites convertible html sans bordure
     \needspace{2\baselineskip}
     {\sffamily \small{\textcolor{#1}{\MakeUppercase{#2}}}}    
		\par		
  			 #3
		\par
}

\newcommand\CHelp[1]{
     \CBox{Plum}{\faInfoCircle}{À RETENIR}{#1}
}

\newcommand\CUp[1]{
     \CBox{NavyBlue}{\faThumbsOUp}{EN PRATIQUE}{#1}
}

\newcommand\CInfo[1]{
     \CBox{Sepia}{\faArrowCircleRight}{REMARQUE}{#1}
}

\newcommand\CRedac[1]{
     \CBox{PineGreen}{\faEdit}{BIEN R\'EDIGER}{#1}
}

\newcommand\CError[1]{
     \CBox{Red}{\faExclamationTriangle}{ATTENTION}{#1}
}

\newcommand\TitreExo[2]{
\needspace{4\baselineskip}
 {\sffamily\large EXERCICE #1\ (\emph{#2 points})}
\vspace{5mm}
}

\newcommand\img[2]{
          \includegraphics[width=#2\paperwidth]{\imgdir#1}
}

\newcommand\imgsvg[2]{
       \begin{center}   \includegraphics[width=#2\paperwidth]{\imgsvgdir#1} \end{center}
}


\newcommand\Lien[2]{
     \href{#1}{#2 \tiny \faExternalLink}
}
\newcommand\mcLien[2]{
     \href{https~://www.maths-cours.fr/#1}{#2 \tiny \faExternalLink}
}

\newcommand{\euro}{\eurologo{}}

%================================================================================================================================
%
% Macros - Environement
%
%================================================================================================================================

\newenvironment{tex}{ %
}
{%
}

\newenvironment{indente}{ %
	\setlength\parindent{10mm}
}

{
	\setlength\parindent{0mm}
}

\newenvironment{corrige}{%
     \needspace{3\baselineskip}
     \medskip
     \textbf{\textsc{Corrigé}}
     \medskip
}
{
}

\newenvironment{extern}{%
     \begin{center}
     }
     {
     \end{center}
}

\NewEnviron{code}{%
	\par
     \boite{gray}{\texttt{%
     \BODY
     }}
     \par
}

\newenvironment{vbloc}{% boite sans cadre empeche saut de page
     \begin{minipage}[t]{\linewidth}
     }
     {
     \end{minipage}
}
\NewEnviron{h2}{%
    \needspace{3\baselineskip}
    \vspace{0.6cm}
	\noindent \MakeUppercase{\sffamily \large \BODY}
	\vspace{1mm}\textcolor{mcgris}{\hrule}\vspace{0.4cm}
	\par
}{}

\NewEnviron{h3}{%
    \needspace{3\baselineskip}
	\vspace{5mm}
	\textsc{\BODY}
	\par
}

\NewEnviron{margeneg}{ %
\begin{addmargin}[-1cm]{0cm}
\BODY
\end{addmargin}
}

\NewEnviron{html}{%
}

\begin{document}
\meta{url}{/cours/divisibilite-congruences/}
\meta{pid}{561}
\meta{titre}{Divisibilité et congruences (Spécialité)}
\meta{type}{cours}
\begin{h2}1. Division euclidienne\end{h2}
\cadre{bleu}{Définition}{% id="d10"
     Soient $a$ et  $b$ deux entiers relatifs tels qu'il existe un entier relatif $k$ tel que $a=bk$.
     \par
     On dit alors que :
     \begin{itemize}
          \item $b$ \textbf{divise} $a$ ;
          \item $b$ est un \textbf{diviseur} de $a$ ;
          \item $a$ est un \textbf{multiple} de $b$.
     \end{itemize}
     Ceci se note $b|a$
}
\bloc{orange}{Exemple}{% id="e10"
     $15=3\times 5$ donc :
     \begin{itemize}
          \item 3 divise 15.
          \item 3 est un diviseur de 15.
          \item 15 est un multiple de 3.
     \end{itemize}
}
\bloc{vert}{Remarques}{% id="r10"
     \begin{itemize}
          \item 0 est un multiple de tout entier relatif.
          \item 1 et -1 sont des diviseurs de tout entier relatif.
          \item $a$ et $-a$ ont les mêmes diviseurs.
     \end{itemize}
}
\cadre{vert}{Propriétés}{% id="p20"
     \begin{itemize}
          \item Si $a$ divise $b$ et $b$ divise $a$, alors $a$ et $b$ sont égaux ou opposés.
          \item Si $a$ divise $b$ et $b$  divise $c$, alors $a$ divise $c$.
          \item Si $c$ divise $a$ et $c$ divise $b$, alors $c$ divise toute combinaison linéaire de $a$ et $b$ (c'est-à-dire tout nombre de la forme $au+bv ; u\in \mathbb{Z}, v\in \mathbb{Z}$).
     \end{itemize}
}
\cadre{rouge}{Théorème et définitions}{% id="d30"
     \textbf{Division euclidienne dans $\mathbb{Z}$}
     \par
     Soient $a$ et $b$ deux entiers relatifs avec $b\neq 0$.
     \par
     Il existe un et un seul couple d'entiers relatifs $\left(q,r\right)$ tels que :
     \par     \begin{center}
          $a=bq+r$ et $0 \leqslant  r < |b|$.
     \end{center}
     $q$ et $r$ s'appelle respectivement le \textbf{quotient} et le \textbf{reste} de la \textbf{division euclidienne} de $a$ par $b$.
}
\bloc{orange}{Exemple}{% id="e30"
     -14=3$\times $(-5)+1 et 0$\leqslant $1$ < $3
     \par
     La division euclidienne de -14 par 3 donne un quotient de -5 est un reste de 1.
}
\bloc{vert}{Remarques}{% id="r30"
     \begin{itemize}
          \item \textbf{Attention !} Ne pas oublier la condition $0 \leqslant  r <  |b|$. La seule égalité $a=bq+r$  ne suffit pas à prouver que $q$ et $r$ sont les quotient et reste dans la division euclidienne de $a$ par $b$.
          \item $a$ est divisible par $b$ si et seulement si le reste de la division de $a$ par $b$ est égal à zéro.
     \end{itemize}
}
\begin{h2}2. Congruences\end{h2}
\cadre{bleu}{Définition}{% id="d40"
     On dit que deux entiers relatifs $a$ et $b$ son congrus modulo $n$ ( $n\in \mathbb{N}^*$ ) et l'on écrit $a\equiv b  \left[n\right]$ si et seulement si $a$ et $b$ ont le même reste dans la division par $n$.
}
\bloc{orange}{Exemple}{% id="e40"
     $18\equiv 23  \left[5\right]$ car 18 et 23 ont tous les deux 3 comme reste dans la division par 5.
}
\cadre{vert}{Propriétés}{% id="p50"
     \begin{itemize}
          \item $a\equiv b  \left[n\right]$ si et seulement si $n$ divise $a-b$ en particulier $a\equiv 0  \left[n\right]$ si et seulement si $n$ divise $a$.
          \item Si $a\equiv b  \left[n\right]$ et $b\equiv c  \left[n\right]$, alors $a\equiv c  \left[n\right]$.
     \end{itemize}
}
\cadre{vert}{Propriétés (Congruences et opérations)}{% id="p60"
     Soient quatre entiers relatifs $a, b, c, d$ tels que $a\equiv b  \left[n\right]$ et $c\equiv d  \left[n\right]$. Alors :
     \begin{itemize}
          \item $a+c\equiv b+d  \left[n\right]$ et $a-c\equiv b-d  \left[n\right]$.
          \item $ac\equiv bd  \left[n\right]$.
          \item $ka\equiv kb  \left[n\right]$ pour tout entier relatif $k$.
          \item $a^{m}\equiv b^{m}  \left[n\right]$ pour tout entier naturel $m$.
     \end{itemize}
}
\cadre{vert}{Propriété}{% id="p70"
     $r$ est le reste de la division euclidienne de $a$ par $b$ si et seulement si :
     \par
     \begin{center}
          $\left\{ \begin{matrix} r\equiv a  \left[b\right] \\ r < |b| \end{matrix}\right.$
     \end{center}
}
\bloc{orange}{Exemple}{% id="e70"
     On cherche à déterminer le reste de la division euclidienne de $2009^{2009}$ par 5.
     \par
     $2009\equiv -1  \left[5\right]$ car 2009-(-1)=2010 est divisible par 5.
     \par
     Donc :
     \par
     $2009^{2009}\equiv \left(-1\right)^{2009}  \left[5\right]$ c'est-à-dire $2009^{2009}\equiv -1  \left[5\right]$
     \par
     Or $-1\equiv 4  \left[5\right]$ donc $2009^{2009}\equiv 4  \left[5\right]$
     \par
     Comme $0\leqslant 4 < 5$, le reste de la division euclidienne de $2009^{2009}$ par 5 est 4.
}

\end{document}
µ
\documentclass[a4paper]{article}

%================================================================================================================================
%
% Packages
%
%================================================================================================================================

\usepackage[T1]{fontenc} 	% pour caractères accentués
\usepackage[utf8]{inputenc}  % encodage utf8
\usepackage[french]{babel}	% langue : français
\usepackage{fourier}			% caractères plus lisibles
\usepackage[dvipsnames]{xcolor} % couleurs
\usepackage{fancyhdr}		% réglage header footer
\usepackage{needspace}		% empêcher sauts de page mal placés
\usepackage{graphicx}		% pour inclure des graphiques
\usepackage{enumitem,cprotect}		% personnalise les listes d'items (nécessaire pour ol, al ...)
\usepackage{hyperref}		% Liens hypertexte
\usepackage{pstricks,pst-all,pst-node,pstricks-add,pst-math,pst-plot,pst-tree,pst-eucl} % pstricks
\usepackage[a4paper,includeheadfoot,top=2cm,left=3cm, bottom=2cm,right=3cm]{geometry} % marges etc.
\usepackage{comment}			% commentaires multilignes
\usepackage{amsmath,environ} % maths (matrices, etc.)
\usepackage{amssymb,makeidx}
\usepackage{bm}				% bold maths
\usepackage{tabularx}		% tableaux
\usepackage{colortbl}		% tableaux en couleur
\usepackage{fontawesome}		% Fontawesome
\usepackage{environ}			% environment with command
\usepackage{fp}				% calculs pour ps-tricks
\usepackage{multido}			% pour ps tricks
\usepackage[np]{numprint}	% formattage nombre
\usepackage{tikz,tkz-tab} 			% package principal TikZ
\usepackage{pgfplots}   % axes
\usepackage{mathrsfs}    % cursives
\usepackage{calc}			% calcul taille boites
\usepackage[scaled=0.875]{helvet} % font sans serif
\usepackage{svg} % svg
\usepackage{scrextend} % local margin
\usepackage{scratch} %scratch
\usepackage{multicol} % colonnes
%\usepackage{infix-RPN,pst-func} % formule en notation polanaise inversée
\usepackage{listings}

%================================================================================================================================
%
% Réglages de base
%
%================================================================================================================================

\lstset{
language=Python,   % R code
literate=
{á}{{\'a}}1
{à}{{\`a}}1
{ã}{{\~a}}1
{é}{{\'e}}1
{è}{{\`e}}1
{ê}{{\^e}}1
{í}{{\'i}}1
{ó}{{\'o}}1
{õ}{{\~o}}1
{ú}{{\'u}}1
{ü}{{\"u}}1
{ç}{{\c{c}}}1
{~}{{ }}1
}


\definecolor{codegreen}{rgb}{0,0.6,0}
\definecolor{codegray}{rgb}{0.5,0.5,0.5}
\definecolor{codepurple}{rgb}{0.58,0,0.82}
\definecolor{backcolour}{rgb}{0.95,0.95,0.92}

\lstdefinestyle{mystyle}{
    backgroundcolor=\color{backcolour},   
    commentstyle=\color{codegreen},
    keywordstyle=\color{magenta},
    numberstyle=\tiny\color{codegray},
    stringstyle=\color{codepurple},
    basicstyle=\ttfamily\footnotesize,
    breakatwhitespace=false,         
    breaklines=true,                 
    captionpos=b,                    
    keepspaces=true,                 
    numbers=left,                    
xleftmargin=2em,
framexleftmargin=2em,            
    showspaces=false,                
    showstringspaces=false,
    showtabs=false,                  
    tabsize=2,
    upquote=true
}

\lstset{style=mystyle}


\lstset{style=mystyle}
\newcommand{\imgdir}{C:/laragon/www/newmc/assets/imgsvg/}
\newcommand{\imgsvgdir}{C:/laragon/www/newmc/assets/imgsvg/}

\definecolor{mcgris}{RGB}{220, 220, 220}% ancien~; pour compatibilité
\definecolor{mcbleu}{RGB}{52, 152, 219}
\definecolor{mcvert}{RGB}{125, 194, 70}
\definecolor{mcmauve}{RGB}{154, 0, 215}
\definecolor{mcorange}{RGB}{255, 96, 0}
\definecolor{mcturquoise}{RGB}{0, 153, 153}
\definecolor{mcrouge}{RGB}{255, 0, 0}
\definecolor{mclightvert}{RGB}{205, 234, 190}

\definecolor{gris}{RGB}{220, 220, 220}
\definecolor{bleu}{RGB}{52, 152, 219}
\definecolor{vert}{RGB}{125, 194, 70}
\definecolor{mauve}{RGB}{154, 0, 215}
\definecolor{orange}{RGB}{255, 96, 0}
\definecolor{turquoise}{RGB}{0, 153, 153}
\definecolor{rouge}{RGB}{255, 0, 0}
\definecolor{lightvert}{RGB}{205, 234, 190}
\setitemize[0]{label=\color{lightvert}  $\bullet$}

\pagestyle{fancy}
\renewcommand{\headrulewidth}{0.2pt}
\fancyhead[L]{maths-cours.fr}
\fancyhead[R]{\thepage}
\renewcommand{\footrulewidth}{0.2pt}
\fancyfoot[C]{}

\newcolumntype{C}{>{\centering\arraybackslash}X}
\newcolumntype{s}{>{\hsize=.35\hsize\arraybackslash}X}

\setlength{\parindent}{0pt}		 
\setlength{\parskip}{3mm}
\setlength{\headheight}{1cm}

\def\ebook{ebook}
\def\book{book}
\def\web{web}
\def\type{web}

\newcommand{\vect}[1]{\overrightarrow{\,\mathstrut#1\,}}

\def\Oij{$\left(\text{O}~;~\vect{\imath},~\vect{\jmath}\right)$}
\def\Oijk{$\left(\text{O}~;~\vect{\imath},~\vect{\jmath},~\vect{k}\right)$}
\def\Ouv{$\left(\text{O}~;~\vect{u},~\vect{v}\right)$}

\hypersetup{breaklinks=true, colorlinks = true, linkcolor = OliveGreen, urlcolor = OliveGreen, citecolor = OliveGreen, pdfauthor={Didier BONNEL - https://www.maths-cours.fr} } % supprime les bordures autour des liens

\renewcommand{\arg}[0]{\text{arg}}

\everymath{\displaystyle}

%================================================================================================================================
%
% Macros - Commandes
%
%================================================================================================================================

\newcommand\meta[2]{    			% Utilisé pour créer le post HTML.
	\def\titre{titre}
	\def\url{url}
	\def\arg{#1}
	\ifx\titre\arg
		\newcommand\maintitle{#2}
		\fancyhead[L]{#2}
		{\Large\sffamily \MakeUppercase{#2}}
		\vspace{1mm}\textcolor{mcvert}{\hrule}
	\fi 
	\ifx\url\arg
		\fancyfoot[L]{\href{https://www.maths-cours.fr#2}{\black \footnotesize{https://www.maths-cours.fr#2}}}
	\fi 
}


\newcommand\TitreC[1]{    		% Titre centré
     \needspace{3\baselineskip}
     \begin{center}\textbf{#1}\end{center}
}

\newcommand\newpar{    		% paragraphe
     \par
}

\newcommand\nosp {    		% commande vide (pas d'espace)
}
\newcommand{\id}[1]{} %ignore

\newcommand\boite[2]{				% Boite simple sans titre
	\vspace{5mm}
	\setlength{\fboxrule}{0.2mm}
	\setlength{\fboxsep}{5mm}	
	\fcolorbox{#1}{#1!3}{\makebox[\linewidth-2\fboxrule-2\fboxsep]{
  		\begin{minipage}[t]{\linewidth-2\fboxrule-4\fboxsep}\setlength{\parskip}{3mm}
  			 #2
  		\end{minipage}
	}}
	\vspace{5mm}
}

\newcommand\CBox[4]{				% Boites
	\vspace{5mm}
	\setlength{\fboxrule}{0.2mm}
	\setlength{\fboxsep}{5mm}
	
	\fcolorbox{#1}{#1!3}{\makebox[\linewidth-2\fboxrule-2\fboxsep]{
		\begin{minipage}[t]{1cm}\setlength{\parskip}{3mm}
	  		\textcolor{#1}{\LARGE{#2}}    
 	 	\end{minipage}  
  		\begin{minipage}[t]{\linewidth-2\fboxrule-4\fboxsep}\setlength{\parskip}{3mm}
			\raisebox{1.2mm}{\normalsize\sffamily{\textcolor{#1}{#3}}}						
  			 #4
  		\end{minipage}
	}}
	\vspace{5mm}
}

\newcommand\cadre[3]{				% Boites convertible html
	\par
	\vspace{2mm}
	\setlength{\fboxrule}{0.1mm}
	\setlength{\fboxsep}{5mm}
	\fcolorbox{#1}{white}{\makebox[\linewidth-2\fboxrule-2\fboxsep]{
  		\begin{minipage}[t]{\linewidth-2\fboxrule-4\fboxsep}\setlength{\parskip}{3mm}
			\raisebox{-2.5mm}{\sffamily \small{\textcolor{#1}{\MakeUppercase{#2}}}}		
			\par		
  			 #3
 	 		\end{minipage}
	}}
		\vspace{2mm}
	\par
}

\newcommand\bloc[3]{				% Boites convertible html sans bordure
     \needspace{2\baselineskip}
     {\sffamily \small{\textcolor{#1}{\MakeUppercase{#2}}}}    
		\par		
  			 #3
		\par
}

\newcommand\CHelp[1]{
     \CBox{Plum}{\faInfoCircle}{À RETENIR}{#1}
}

\newcommand\CUp[1]{
     \CBox{NavyBlue}{\faThumbsOUp}{EN PRATIQUE}{#1}
}

\newcommand\CInfo[1]{
     \CBox{Sepia}{\faArrowCircleRight}{REMARQUE}{#1}
}

\newcommand\CRedac[1]{
     \CBox{PineGreen}{\faEdit}{BIEN R\'EDIGER}{#1}
}

\newcommand\CError[1]{
     \CBox{Red}{\faExclamationTriangle}{ATTENTION}{#1}
}

\newcommand\TitreExo[2]{
\needspace{4\baselineskip}
 {\sffamily\large EXERCICE #1\ (\emph{#2 points})}
\vspace{5mm}
}

\newcommand\img[2]{
          \includegraphics[width=#2\paperwidth]{\imgdir#1}
}

\newcommand\imgsvg[2]{
       \begin{center}   \includegraphics[width=#2\paperwidth]{\imgsvgdir#1} \end{center}
}


\newcommand\Lien[2]{
     \href{#1}{#2 \tiny \faExternalLink}
}
\newcommand\mcLien[2]{
     \href{https~://www.maths-cours.fr/#1}{#2 \tiny \faExternalLink}
}

\newcommand{\euro}{\eurologo{}}

%================================================================================================================================
%
% Macros - Environement
%
%================================================================================================================================

\newenvironment{tex}{ %
}
{%
}

\newenvironment{indente}{ %
	\setlength\parindent{10mm}
}

{
	\setlength\parindent{0mm}
}

\newenvironment{corrige}{%
     \needspace{3\baselineskip}
     \medskip
     \textbf{\textsc{Corrigé}}
     \medskip
}
{
}

\newenvironment{extern}{%
     \begin{center}
     }
     {
     \end{center}
}

\NewEnviron{code}{%
	\par
     \boite{gray}{\texttt{%
     \BODY
     }}
     \par
}

\newenvironment{vbloc}{% boite sans cadre empeche saut de page
     \begin{minipage}[t]{\linewidth}
     }
     {
     \end{minipage}
}
\NewEnviron{h2}{%
    \needspace{3\baselineskip}
    \vspace{0.6cm}
	\noindent \MakeUppercase{\sffamily \large \BODY}
	\vspace{1mm}\textcolor{mcgris}{\hrule}\vspace{0.4cm}
	\par
}{}

\NewEnviron{h3}{%
    \needspace{3\baselineskip}
	\vspace{5mm}
	\textsc{\BODY}
	\par
}

\NewEnviron{margeneg}{ %
\begin{addmargin}[-1cm]{0cm}
\BODY
\end{addmargin}
}

\NewEnviron{html}{%
}

\begin{document}
\meta{url}{/cours/pgcd-nombres-premiers/}
\meta{pid}{567}
\meta{titre}{PGCD et nombres premiers (Spécialité)}
\meta{type}{cours}
\begin{h2}1. PGCD\end{h2}
\cadre{bleu}{Définition}{% id="d10"
     Soient $a$ et $b$ deux entiers naturels tels que $a\neq 0$ ou $b\neq 0$.
     \par
     Le PGCD de $a$ et de $b$ est le plus grand diviseur commun à $a$ et à $b$.
}
\bloc{orange}{Exemple}{% id="e10"
     On cherche le PGCD de 60 et de 45.
     \par
     Les diviseurs de 60 sont : 1; 2; 3; 4; 5; 6; 10; 12; 15; 20; 30 et 60.
     \par
     Les diviseurs de 45 sont : 1; 3;  5; 9; 15 et 45.
     \par
     Les diviseurs communs à 60 et 45 sont : 1; 3; 5 et 15.
     \par
     Donc le PGCD de 60 et 45 est 15.
}
\bloc{cyan}{Remarques}{% id="r10"
     \begin{itemize}
          \item Si $b$ divise $a$, PGCD($a ; b$) $= b$. En effet, $b$ divise alors $a$ et $b$, et $b$ est le plus grand diviseur de $b$.
          \par
          En particulier, PGCD($a ; 1$) $= 1$ et PGCD(0 ; $b$) $= b$
          \item On prolonge la notion de PGCD à des entiers \textbf{relatifs} $a$ et $b$ par PGCD($a ; b$)=PGCD($|a| ; |b|$).
     \end{itemize}
}
\cadre{rouge}{Théorème}{% id="t20"
     Soient $a$ et  $b$ deux entiers naturels non nuls et $r$ le reste de la division euclidienne de $a$ par $b$.
     \par
     Alors : PGCD($a ; b$) = PGCD($b ; r$).
}
\bloc{orange}{Exemple}{% id="e20"
     Le reste de la division euclidienne de 60 par 45 est 15. donc PGCD(60 ; 45) = PGCD(45 ; 15).
     \par
     Si l'on réitère le processus, le reste de la division euclidienne de 45 par 15 est 0 donc PGCD(45 ; 15) = PGCD(15 ; 0) = 15.
}
\cadre{rouge}{Algorithme d'Euclide}{% id="t30"
     \begin{itemize}
          \item On effectue la division euclidienne de $a$ par $b$ et on note $r_{1}$ le reste de cette division.
          \item Puis si $r_{1}\neq 0$, on effectue la division euclidienne de $b$ par $r_{1}$ et on note $r_{2}$ le reste de cette division.
          \item Puis si $r_{2}\neq 0$, on effectue la division euclidienne de $r_{1}$ par $r_{2}$, et ainsi de suite...
     \end{itemize}
     La suite $r_{0}=b$, $r_{1}$, $r_{2}$, ... est strictement décroissante, et pour un certain rang $n$ on aura $r_{n}=0$.
     \par
     Par conséquent :
     \par
     PGCD($a$ ; $b$) = PGCD($b$ ; $r_{0}$) = PGCD($r_{0}$ ; $r_{1}$) = ...
     \par
     $                   $= PGCD($r_{n-1}$ ; $r_{n}$) = PGCD($r_{n-1}$ ; 0) = $r_{n-1}$
     \par
     Le PGCD de $a$ et $b$ est donc \textbf{le dernier reste non nul} dans cette suite.
}
\bloc{orange}{Exemple}{% id="e30"
     On cherche à déterminer le PGCD de 2691 et de 1404.
     \begin{itemize}
          \item le reste de la division euclidienne de 2691 par 1404 est 1287,
          \item le reste de la division euclidienne de 1404 par 1287 est 117,
          \item le reste de la division euclidienne de 1287 par 117 est 0.
     \end{itemize}
     Par conséquent PGCD(2691 ; 1404) = 117.
}
\cadre{vert}{Propriété}{% id="p40"
     Soient $a$ et $b$ deux entiers naturels non nuls.
     \par
     L'ensemble des diviseurs communs à $a$ et à $b$ est l'ensemble des diviseurs de leur PGCD.
}
\cadre{bleu}{Définition}{% id="d50"
     On dit que deux entiers naturels non nuls sont \textbf{premiers entre eux} si leur PGCD est égal à 1.
}
\bloc{cyan}{Remarque}{% id="r50"
     On peut généraliser cette notion à plus de deux entiers de deux façons différentes.
     \par
     Si $a$, $b$ et $c$ sont trois entiers non nuls :
     \begin{itemize}
          \item on dit que $a$, $b$ et $c$ sont premiers entre eux \textbf{dans leur ensemble} lorsque le seul diviseur commun à $a$, $b$ et $c$ est 1 ;
          \item on dit que $a$, $b$ et $c$ sont premiers entre eux \textbf{deux à deux} lorsque PGCD($a$ ; $b$) = 1, PGCD($b$ ; $c$) = 1 et PGCD($a$ ; $c$) = 1.
     \end{itemize}
     Par exemple 4 , 6 et 9 sont premiers entre eux dans leur ensemble (pas de diviseur commun à ces trois nombres autre que 1)  mais ne sont pas premiers entre eux deux à deux puisque PGCD(4 ; 6) = 2 et PGCD(6 ; 9) = 3.
}
\cadre{vert}{Propriété}{% id="p55"
     Soient $a$ et $b$ deux entiers naturels non nuls.
     \par
     $d$ est le PGCD de $a$ et de $b$ si et seulement si il esiste deux entiers $a^{\prime}$ et $b^{\prime}$ \textbf{premiers entre eux} tels que $a=a^{\prime}d$ et $b=b^{\prime}d$.
}
\bloc{orange}{Exemple}{% id="e55"
     Le  PGCD de 60 et de 45 est 15. On a :
     \par
     60 = 4×15 et 45 = 3×15 et 4 et 3 sont premiers entre eux.
}
\cadre{rouge}{Théorème (de Bézout)}{% id="t60"
     Deux entiers naturels $a$ et $b$ non nuls sont premiers entre eux si et seulement si il existe deux entiers relatifs $u$ et $v$ tels que :
     \begin{center}$au+bv = 1$.\end{center}
}
\bloc{cyan}{Remarque}{% id="r60"
     Les valeurs de $u$ et de $v$ peuvent être obtenues à l'aide de l'algorithme d'Euclide (fiche méthode à venir...)
}
\bloc{orange}{Exemple}{% id="e60"
     Pour tout entier naturel $n$, $2n+1$ et $n$ sont premiers entre eux.
     \par
     En effet $1\times \left(2n+1\right)-2\times n=1$. Donc d'après le théorème de Bézout (avec $u=1$ et $v=-2$), $n$ et $2n+1$ sont premiers entre eux.
}
\cadre{vert}{Propriété}{% id="p70"
     Soient $a$ et $b$ deux entiers naturels non nuls et $d$ leur PGCD.
     \par
     Alors, il existe deux entiers relatifs $u$ et $v$ tels que :
     \begin{center}$au+bv = d$.\end{center}
}
\bloc{cyan}{Remarque}{% id="r70"
     Attention, la réciproque est fausse.
     \par
     Si $au+bv = d$ on peut seulement en déduire que le PGCD de $a$ et de $b$ divise $d$ (d'après une \mcLien{/cours/terminale-s/divisibilite-congruences\#p20}{propriété du chapitre précédent}). Par exemple $3\times 4+2\times \left(-5\right)=2$ mais le PGCD de 3 et de 2 est 1 (ils sont premiers entre eux) et non 2.
}
\cadre{rouge}{Théorème (de Gauss)}{% id="t90"
     Soient $a$, $b$ et $c$ trois entiers naturels non nuls.
     \begin{itemize}
          \item Si $a$ divise le produit $bc$
          \item et si $a$ est premier avec $b$,
     \end{itemize}
     alors, $a$ divise $c$.
}
\bloc{orange}{Exemple}{% id="e90"
     On cherche tous les couples d'entiers naturels $\left(m ; n\right)$ tels que $5m=3n$.
     \par
     L'égalité $5m=3n$ signifie que $5$ divise $3n$. Comme $5$ et $3$ sont premiers entre eux, d'après le théorème de Gauss $5$ divise $n$. Donc il existe un entier naturel $k$ tel que $n=5k$. On a alors $5m=3n=15k$ soit $m=3k$.
     \par
     Réciproquement, on vérifie aisément que tout couple de la forme $\left(3k ; 5k\right)$ (où $k \in  \mathbb{N}$) est solution de l'équation proposée.
}
\cadre{vert}{Propriété}{% id="p100"
     Si $a$ et $b$ divisent $c$ et sont premiers entre eux, alors le produit $ab$ divise $c$.
}
\bloc{orange}{Exemples}{% id="e100"
     D'après cette propriété :
     \begin{itemize}
          \item $n$ est divisible par 6 si et seulement si il est divisible par 2 et par 3 (car 2 et 3 sont premiers entre eux).
          \item $n$ est divisible par 15 si et seulement si il est divisible par 3 et par 5 (car 3 et 5 sont premiers entre eux).
     \end{itemize}
}
\bloc{cyan}{Remarque}{% id="r100"
     L'hypothèse « $a$ et $b$ sont premiers entre eux » est essentielle. Par exemple 90 est divisible par 6 et par 10 mais n'est pas divisible par 6×10 = 60.
}
\begin{h2}2. Nombres premiers\end{h2}
\cadre{bleu}{Définition}{% id="d150"
     Un entier naturel est premier s'il admet exactement deux diviseurs (dans $\mathbb{N}$) : 1 et lui-même.
}
\bloc{cyan}{Remarque}{% id="r150"
     1 n'est pas un nombre premier (il possède un seul diviseur).
}
\cadre{vert}{Propriétés}{% id="p160"
     \begin{itemize}
          \item Tout entier naturel $n > 1$ admet au moins un diviseur premier.
          \item Tout entier naturel $n > 1$ \textbf{non premier} admet au moins un diviseur premier inférieur ou égal à $\sqrt{n}$.
\end{itemize}}
\bloc{cyan}{Remarque}{% id="r160"
     La seconde propriété est souvent utilisée pour démontrer (par l'absurde) qu'un entier naturel $n$ est premier. Il suffit, en effet, de montrer que $n$ n'est divisible par aucun nombre premier $p$ inférieur ou égal à $\sqrt{n}$. On peut donc arrêter la recherche de diviseurs premiers $p$ dès que $p^{2} > n$.
}
\bloc{orange}{Exemple}{% id="e160"
     41 est-il premier ?
     \begin{itemize}
          \item 41 n'est pas divisible par 2 (dernier chiffre impair),
          \item 41 n'est pas divisible par 3 (somme des chiffres 4+1=5),
          \item 41 n'est pas divisible par 5 (dernier chiffre différent de 0 et de 5),
          \item 7²=49 > 41 (donc $7 > \sqrt{41}$) : inutile de chercher plus loin...
     \end{itemize}
     Conclusion : 41 est un nombre premier.
}
\cadre{vert}{Propriété}{% id="p170"
     Il existe une infinité de nombres premiers.
}
\bloc{orange}{Démonstration}{% id="d170"
     On raisonne par l'absurde en supposant que l'ensemble des nombres premiers n'est pas infini. Il existe alors un plus grand nombre premier $p$.
     \par
     On pose $N=2\times 3\times 5\times \cdots \times p$ (produit de tous les nombres premiers).
     \par
     Comme tout entier naturel supérieur à 1 admet au moins un diviseur premier, $N+1$ admet un diviseur premier $d$.
     \par
     Or $d$ divise aussi le nombre $N$ puisque $N$ est le produit de \textbf{tous} les nombres premiers.
     \par
     $d$ divise $N+1$ et $N$, donc il divise leur différence 1, ce qui est impossible.
}
\cadre{vert}{Propriété}{% id="p180"
     Si $p$ est un nombre premier et $a$ un entier naturel non nul non divisible par $p$, alors $p$ et $a$ sont premier entre eux.
}
\cadre{vert}{Propriété}{% id="p185"
     Soient $a$ et $b$ deux entiers naturels non nuls.
     \par
     Si un nombre premier $p$ divise le produit $ab$, alors $p$ divise $a$ ou $b$.
}
\bloc{cyan}{Remarque}{% id="r185"
     Cette propriété résulte immédiatement de la propriété précédente et du théorème de Gauss.
}
\cadre{rouge}{Théorème (théorème fondamental de l'arithmétique)}{% id="t190"
     Tout entier naturel $n > 1$ se décompose en produit de nombres premiers.
     \par
     Cette décomposition peut s'écrire :
     \begin{center}$n=p_{1}^{a_{1}}p_{2}^{a_{2}}\cdots p_{k}^{a_{k}}$\end{center}
     où les $p_{i}$ sont des nombres premiers distincts et les $a_{i}$ des entiers naturels non nuls.
     \par
     Cette décomposition est unique à l'ordre près des facteurs.
}
\bloc{orange}{Exemple}{% id="e190"
     Cherchons la décomposition de 60 en facteurs premiers.
     \begin{itemize}
          \item 60 est divisible par 2 et le quotient de cette division est 30.
          \item 30 est divisible par 2 et le quotient est 15.
          \item 15 est divisible par 3 et le quotient est 5.
          \item Enfin, 5 est premier.
     \end{itemize}
     Donc $60=2^{2}\times 3\times 5$.
}
\cadre{vert}{Propriété}{% id="p200"
     Soit $n$ un entier naturel supérieur à 1 dont la décomposition en facteurs premiers s'écrit $n=p_{1}^{a_{1}}p_{2}^{a_{2}}\cdots p_{k}^{a_{k}}$.
     \par
     Alors, les diviseurs de $n$ sont les entiers de la forme :
     \begin{center}$n=p_{1}^{b_{1}}p_{2}^{b_{2}}\cdots p_{k}^{b_{k}}$\end{center}
     avec $0 \leqslant  b_{i} \leqslant  a_{i}$ pour tout $0 \leqslant  i \leqslant  k$.
}
\bloc{orange}{Exemple}{% id="e200"
     $60=2^{2}\times 3\times 5$ admet comme diviseurs les nombres de la forme $2^{b_{1}}\times 3^{b_{2}}\times 5^{b_{3}}$ avec $0 \leqslant  b_{1} \leqslant  2$, $0 \leqslant  b_{2} \leqslant  1$ et $0 \leqslant  b_{3} \leqslant  1$.
     \par
     Il y a trois valeurs possibles pour $b_{1}$ et deux valeurs possibles pour $b_{2}$ et pour $b_{3}$. Au total, $60$ possède donc $3\times 2\times 2=12$ diviseurs (en comptant $1$ et lui-même).
}

\end{document}
µ
\documentclass[a4paper]{article}

%================================================================================================================================
%
% Packages
%
%================================================================================================================================

\usepackage[T1]{fontenc} 	% pour caractères accentués
\usepackage[utf8]{inputenc}  % encodage utf8
\usepackage[french]{babel}	% langue : français
\usepackage{fourier}			% caractères plus lisibles
\usepackage[dvipsnames]{xcolor} % couleurs
\usepackage{fancyhdr}		% réglage header footer
\usepackage{needspace}		% empêcher sauts de page mal placés
\usepackage{graphicx}		% pour inclure des graphiques
\usepackage{enumitem,cprotect}		% personnalise les listes d'items (nécessaire pour ol, al ...)
\usepackage{hyperref}		% Liens hypertexte
\usepackage{pstricks,pst-all,pst-node,pstricks-add,pst-math,pst-plot,pst-tree,pst-eucl} % pstricks
\usepackage[a4paper,includeheadfoot,top=2cm,left=3cm, bottom=2cm,right=3cm]{geometry} % marges etc.
\usepackage{comment}			% commentaires multilignes
\usepackage{amsmath,environ} % maths (matrices, etc.)
\usepackage{amssymb,makeidx}
\usepackage{bm}				% bold maths
\usepackage{tabularx}		% tableaux
\usepackage{colortbl}		% tableaux en couleur
\usepackage{fontawesome}		% Fontawesome
\usepackage{environ}			% environment with command
\usepackage{fp}				% calculs pour ps-tricks
\usepackage{multido}			% pour ps tricks
\usepackage[np]{numprint}	% formattage nombre
\usepackage{tikz,tkz-tab} 			% package principal TikZ
\usepackage{pgfplots}   % axes
\usepackage{mathrsfs}    % cursives
\usepackage{calc}			% calcul taille boites
\usepackage[scaled=0.875]{helvet} % font sans serif
\usepackage{svg} % svg
\usepackage{scrextend} % local margin
\usepackage{scratch} %scratch
\usepackage{multicol} % colonnes
%\usepackage{infix-RPN,pst-func} % formule en notation polanaise inversée
\usepackage{listings}

%================================================================================================================================
%
% Réglages de base
%
%================================================================================================================================

\lstset{
language=Python,   % R code
literate=
{á}{{\'a}}1
{à}{{\`a}}1
{ã}{{\~a}}1
{é}{{\'e}}1
{è}{{\`e}}1
{ê}{{\^e}}1
{í}{{\'i}}1
{ó}{{\'o}}1
{õ}{{\~o}}1
{ú}{{\'u}}1
{ü}{{\"u}}1
{ç}{{\c{c}}}1
{~}{{ }}1
}


\definecolor{codegreen}{rgb}{0,0.6,0}
\definecolor{codegray}{rgb}{0.5,0.5,0.5}
\definecolor{codepurple}{rgb}{0.58,0,0.82}
\definecolor{backcolour}{rgb}{0.95,0.95,0.92}

\lstdefinestyle{mystyle}{
    backgroundcolor=\color{backcolour},   
    commentstyle=\color{codegreen},
    keywordstyle=\color{magenta},
    numberstyle=\tiny\color{codegray},
    stringstyle=\color{codepurple},
    basicstyle=\ttfamily\footnotesize,
    breakatwhitespace=false,         
    breaklines=true,                 
    captionpos=b,                    
    keepspaces=true,                 
    numbers=left,                    
xleftmargin=2em,
framexleftmargin=2em,            
    showspaces=false,                
    showstringspaces=false,
    showtabs=false,                  
    tabsize=2,
    upquote=true
}

\lstset{style=mystyle}


\lstset{style=mystyle}
\newcommand{\imgdir}{C:/laragon/www/newmc/assets/imgsvg/}
\newcommand{\imgsvgdir}{C:/laragon/www/newmc/assets/imgsvg/}

\definecolor{mcgris}{RGB}{220, 220, 220}% ancien~; pour compatibilité
\definecolor{mcbleu}{RGB}{52, 152, 219}
\definecolor{mcvert}{RGB}{125, 194, 70}
\definecolor{mcmauve}{RGB}{154, 0, 215}
\definecolor{mcorange}{RGB}{255, 96, 0}
\definecolor{mcturquoise}{RGB}{0, 153, 153}
\definecolor{mcrouge}{RGB}{255, 0, 0}
\definecolor{mclightvert}{RGB}{205, 234, 190}

\definecolor{gris}{RGB}{220, 220, 220}
\definecolor{bleu}{RGB}{52, 152, 219}
\definecolor{vert}{RGB}{125, 194, 70}
\definecolor{mauve}{RGB}{154, 0, 215}
\definecolor{orange}{RGB}{255, 96, 0}
\definecolor{turquoise}{RGB}{0, 153, 153}
\definecolor{rouge}{RGB}{255, 0, 0}
\definecolor{lightvert}{RGB}{205, 234, 190}
\setitemize[0]{label=\color{lightvert}  $\bullet$}

\pagestyle{fancy}
\renewcommand{\headrulewidth}{0.2pt}
\fancyhead[L]{maths-cours.fr}
\fancyhead[R]{\thepage}
\renewcommand{\footrulewidth}{0.2pt}
\fancyfoot[C]{}

\newcolumntype{C}{>{\centering\arraybackslash}X}
\newcolumntype{s}{>{\hsize=.35\hsize\arraybackslash}X}

\setlength{\parindent}{0pt}		 
\setlength{\parskip}{3mm}
\setlength{\headheight}{1cm}

\def\ebook{ebook}
\def\book{book}
\def\web{web}
\def\type{web}

\newcommand{\vect}[1]{\overrightarrow{\,\mathstrut#1\,}}

\def\Oij{$\left(\text{O}~;~\vect{\imath},~\vect{\jmath}\right)$}
\def\Oijk{$\left(\text{O}~;~\vect{\imath},~\vect{\jmath},~\vect{k}\right)$}
\def\Ouv{$\left(\text{O}~;~\vect{u},~\vect{v}\right)$}

\hypersetup{breaklinks=true, colorlinks = true, linkcolor = OliveGreen, urlcolor = OliveGreen, citecolor = OliveGreen, pdfauthor={Didier BONNEL - https://www.maths-cours.fr} } % supprime les bordures autour des liens

\renewcommand{\arg}[0]{\text{arg}}

\everymath{\displaystyle}

%================================================================================================================================
%
% Macros - Commandes
%
%================================================================================================================================

\newcommand\meta[2]{    			% Utilisé pour créer le post HTML.
	\def\titre{titre}
	\def\url{url}
	\def\arg{#1}
	\ifx\titre\arg
		\newcommand\maintitle{#2}
		\fancyhead[L]{#2}
		{\Large\sffamily \MakeUppercase{#2}}
		\vspace{1mm}\textcolor{mcvert}{\hrule}
	\fi 
	\ifx\url\arg
		\fancyfoot[L]{\href{https://www.maths-cours.fr#2}{\black \footnotesize{https://www.maths-cours.fr#2}}}
	\fi 
}


\newcommand\TitreC[1]{    		% Titre centré
     \needspace{3\baselineskip}
     \begin{center}\textbf{#1}\end{center}
}

\newcommand\newpar{    		% paragraphe
     \par
}

\newcommand\nosp {    		% commande vide (pas d'espace)
}
\newcommand{\id}[1]{} %ignore

\newcommand\boite[2]{				% Boite simple sans titre
	\vspace{5mm}
	\setlength{\fboxrule}{0.2mm}
	\setlength{\fboxsep}{5mm}	
	\fcolorbox{#1}{#1!3}{\makebox[\linewidth-2\fboxrule-2\fboxsep]{
  		\begin{minipage}[t]{\linewidth-2\fboxrule-4\fboxsep}\setlength{\parskip}{3mm}
  			 #2
  		\end{minipage}
	}}
	\vspace{5mm}
}

\newcommand\CBox[4]{				% Boites
	\vspace{5mm}
	\setlength{\fboxrule}{0.2mm}
	\setlength{\fboxsep}{5mm}
	
	\fcolorbox{#1}{#1!3}{\makebox[\linewidth-2\fboxrule-2\fboxsep]{
		\begin{minipage}[t]{1cm}\setlength{\parskip}{3mm}
	  		\textcolor{#1}{\LARGE{#2}}    
 	 	\end{minipage}  
  		\begin{minipage}[t]{\linewidth-2\fboxrule-4\fboxsep}\setlength{\parskip}{3mm}
			\raisebox{1.2mm}{\normalsize\sffamily{\textcolor{#1}{#3}}}						
  			 #4
  		\end{minipage}
	}}
	\vspace{5mm}
}

\newcommand\cadre[3]{				% Boites convertible html
	\par
	\vspace{2mm}
	\setlength{\fboxrule}{0.1mm}
	\setlength{\fboxsep}{5mm}
	\fcolorbox{#1}{white}{\makebox[\linewidth-2\fboxrule-2\fboxsep]{
  		\begin{minipage}[t]{\linewidth-2\fboxrule-4\fboxsep}\setlength{\parskip}{3mm}
			\raisebox{-2.5mm}{\sffamily \small{\textcolor{#1}{\MakeUppercase{#2}}}}		
			\par		
  			 #3
 	 		\end{minipage}
	}}
		\vspace{2mm}
	\par
}

\newcommand\bloc[3]{				% Boites convertible html sans bordure
     \needspace{2\baselineskip}
     {\sffamily \small{\textcolor{#1}{\MakeUppercase{#2}}}}    
		\par		
  			 #3
		\par
}

\newcommand\CHelp[1]{
     \CBox{Plum}{\faInfoCircle}{À RETENIR}{#1}
}

\newcommand\CUp[1]{
     \CBox{NavyBlue}{\faThumbsOUp}{EN PRATIQUE}{#1}
}

\newcommand\CInfo[1]{
     \CBox{Sepia}{\faArrowCircleRight}{REMARQUE}{#1}
}

\newcommand\CRedac[1]{
     \CBox{PineGreen}{\faEdit}{BIEN R\'EDIGER}{#1}
}

\newcommand\CError[1]{
     \CBox{Red}{\faExclamationTriangle}{ATTENTION}{#1}
}

\newcommand\TitreExo[2]{
\needspace{4\baselineskip}
 {\sffamily\large EXERCICE #1\ (\emph{#2 points})}
\vspace{5mm}
}

\newcommand\img[2]{
          \includegraphics[width=#2\paperwidth]{\imgdir#1}
}

\newcommand\imgsvg[2]{
       \begin{center}   \includegraphics[width=#2\paperwidth]{\imgsvgdir#1} \end{center}
}


\newcommand\Lien[2]{
     \href{#1}{#2 \tiny \faExternalLink}
}
\newcommand\mcLien[2]{
     \href{https~://www.maths-cours.fr/#1}{#2 \tiny \faExternalLink}
}

\newcommand{\euro}{\eurologo{}}

%================================================================================================================================
%
% Macros - Environement
%
%================================================================================================================================

\newenvironment{tex}{ %
}
{%
}

\newenvironment{indente}{ %
	\setlength\parindent{10mm}
}

{
	\setlength\parindent{0mm}
}

\newenvironment{corrige}{%
     \needspace{3\baselineskip}
     \medskip
     \textbf{\textsc{Corrigé}}
     \medskip
}
{
}

\newenvironment{extern}{%
     \begin{center}
     }
     {
     \end{center}
}

\NewEnviron{code}{%
	\par
     \boite{gray}{\texttt{%
     \BODY
     }}
     \par
}

\newenvironment{vbloc}{% boite sans cadre empeche saut de page
     \begin{minipage}[t]{\linewidth}
     }
     {
     \end{minipage}
}
\NewEnviron{h2}{%
    \needspace{3\baselineskip}
    \vspace{0.6cm}
	\noindent \MakeUppercase{\sffamily \large \BODY}
	\vspace{1mm}\textcolor{mcgris}{\hrule}\vspace{0.4cm}
	\par
}{}

\NewEnviron{h3}{%
    \needspace{3\baselineskip}
	\vspace{5mm}
	\textsc{\BODY}
	\par
}

\NewEnviron{margeneg}{ %
\begin{addmargin}[-1cm]{0cm}
\BODY
\end{addmargin}
}

\NewEnviron{html}{%
}

\begin{document}
\meta{url}{/exercices/signe-quotient-140918/}
\meta{pid}{599}
\meta{pi_}{599}
\meta{titre}{Signe d'un quotient - Inéquation}
\meta{type}{exercices}
\begin{enumerate}
     \item Étudier le signe du quotient $\frac{5-2x}{2x-1}.$
     \item En déduire l'ensemble des solutions de l'inéquation $\frac{5-2x}{2x-1} \leqslant 0$
\end{enumerate}
\begin{corrige}
     \begin{enumerate}
          \item Le quotient est défini lorsque $2x-1\neq 0$ c'est à dire $x\neq \frac{1}{2}.$
          \par
          On étudie le signe du numérateur $5-2x$ et du dénominateur $2x-1$ en utilisant \mcLien{/cours/equations-et-inequations\#r50}{la règle qui donne le signe de $ax+b$}
          \par
          On obtient le tableau de signes suivant :
          \begin{center}
               \begin{extern}%width="500" alt="Exercice tableau de signes d'un quotient"
                    \resizebox{11cm}{!}{
                         \begin{tikzpicture}[scale=0.875]
                              % Styles
                              \tikzstyle{cadre}=[thin]
                              \tikzstyle{fleche}=[->,>=latex,thin]
                              \tikzstyle{nondefini}=[lightgray]
                              % Dimensions Modifiables
                              \def\Lrg{1.5}
                              \def\HtX{1.2}
                              \def\HtY{0.5}
                              % Dimensions Calculées
                              \def\lignex{-0.5*\HtX}
                              \def\lignea{-1.5*\HtX}
                              \def\ligneb{-2.5*\HtX}
                              \def\lignec{-3.5*\HtX}
                              \def\separateur{-0.5*\Lrg}
                              % Largeur du tableau
                              \def\gauche{-3.1*\Lrg}
                              \def\droite{6.5*\Lrg}
                              % Hauteur du tableau
                              \def\haut{0.5*\HtX}
                              \def\bas{-2.5*\HtX-2*\HtY}
                              % Pointillés
                              \draw[gray] (2*\Lrg,\lignex) -- (2*\Lrg,\ligneb);
                              \draw[gray] (4*\Lrg,\lignex) -- (4*\Lrg,\lignec);
                              \draw[double distance=2pt] (2*\Lrg,\ligneb) -- (2*\Lrg,\lignec);
                              % Ligne de l'abscisse : x
                              \node at (-1.8*\Lrg,0) {$x$};
                              \node at (0*\Lrg,0) {$-\infty$};
                              \node at (2*\Lrg,0) {$\dfrac{1}{2}$};
                              \node at (4*\Lrg,0) {$\dfrac{5}{2}$};
                              \node at (6*\Lrg,0) {$+\infty$};
                              % Ligne a
                              \node at (-1.8*\Lrg,-1*\HtX) {$5-2x$};
                              \node at (0*\Lrg,-1*\HtX) {$ $};
                              \node at (1*\Lrg,-1*\HtX) {$+$};
                              \node at (2*\Lrg,-1*\HtX) {$ $};
                              \node at (3*\Lrg,-1*\HtX) {$+$};
                              \node at (4*\Lrg,-1*\HtX) {$0$};
                              \node at (5*\Lrg,-1*\HtX) {$-$};
                              \node at (6*\Lrg,-1*\HtX) {$ $};
                              % Ligne b
                              \node at (-1.8*\Lrg,-2*\HtX) {$2x-1$};
                              \node at (2*\Lrg,-2*\HtX) {$ $};
                              \node at (1*\Lrg,-2*\HtX) {$-$};
                              \node at (2*\Lrg,-2*\HtX) {$0$};
                              \node at (3*\Lrg,-2*\HtX) {$+$};
                              \node at (4*\Lrg,-2*\HtX) {$ $};
                              \node at (5*\Lrg,-2*\HtX) {$+$};
                              \node at (6*\Lrg,-2*\HtX) {$ $};
                              % Ligne c
                              \node at (-1.8*\Lrg,-3*\HtX) {$\dfrac{5-2x}{2x-1}$};
                              \node at (0*\Lrg,-3*\HtX) {$ $};
                              \node at (1*\Lrg,-3*\HtX) {$-$};
                              \node at (2*\Lrg,-3*\HtX) {$ $};
                              \node at (3*\Lrg,-3*\HtX) {$+$};
                              \node at (4*\Lrg,-3*\HtX) {$0$};
                              \node at (5*\Lrg,-3*\HtX) {$-$};
                              \node at (6*\Lrg,-3*\HtX) {$ $};
                              % Encadrement
                              \draw[cadre] (\separateur,\haut) -- (\separateur, \lignec);
                              \draw[cadre] (\gauche,\haut) rectangle  (\droite, \lignec);
                              \draw[cadre] (\gauche,\lignex) -- (\droite,\lignex);
                              \draw[cadre] (\gauche,\lignea) -- (\droite,\lignea);
                              \draw[cadre] (\gauche,\ligneb) -- (\droite,\ligneb);
                         \end{tikzpicture}
                    }
               \end{extern}
          \end{center}
          \item A partir du tableau, on obtient l'ensemble des solutions de l'inéquation $\frac{5-2x}{2x-1} \leqslant 0 ~:$
          \begin{center}
               $S=\left]-\infty ; \frac{1}{2}\right[ \cup \left[\frac{5}{2} ; +\infty \right[$
          \end{center}
          Le crochet est ouvert en $\frac{1}{2}$ car $\frac{1}{2}$ est une « valeur interdite » et fermé en $\frac{5}{2}$ car $\frac{5-2x}{2x-1}$ s'annule en $\frac{5}{2}$ (et l'inéquation est donc alors vérifiée puisque $0 \leqslant 0 $).
     \end{enumerate}
\end{corrige}

\end{document}
µ
\documentclass[a4paper]{article}

%================================================================================================================================
%
% Packages
%
%================================================================================================================================

\usepackage[T1]{fontenc} 	% pour caractères accentués
\usepackage[utf8]{inputenc}  % encodage utf8
\usepackage[french]{babel}	% langue : français
\usepackage{fourier}			% caractères plus lisibles
\usepackage[dvipsnames]{xcolor} % couleurs
\usepackage{fancyhdr}		% réglage header footer
\usepackage{needspace}		% empêcher sauts de page mal placés
\usepackage{graphicx}		% pour inclure des graphiques
\usepackage{enumitem,cprotect}		% personnalise les listes d'items (nécessaire pour ol, al ...)
\usepackage{hyperref}		% Liens hypertexte
\usepackage{pstricks,pst-all,pst-node,pstricks-add,pst-math,pst-plot,pst-tree,pst-eucl} % pstricks
\usepackage[a4paper,includeheadfoot,top=2cm,left=3cm, bottom=2cm,right=3cm]{geometry} % marges etc.
\usepackage{comment}			% commentaires multilignes
\usepackage{amsmath,environ} % maths (matrices, etc.)
\usepackage{amssymb,makeidx}
\usepackage{bm}				% bold maths
\usepackage{tabularx}		% tableaux
\usepackage{colortbl}		% tableaux en couleur
\usepackage{fontawesome}		% Fontawesome
\usepackage{environ}			% environment with command
\usepackage{fp}				% calculs pour ps-tricks
\usepackage{multido}			% pour ps tricks
\usepackage[np]{numprint}	% formattage nombre
\usepackage{tikz,tkz-tab} 			% package principal TikZ
\usepackage{pgfplots}   % axes
\usepackage{mathrsfs}    % cursives
\usepackage{calc}			% calcul taille boites
\usepackage[scaled=0.875]{helvet} % font sans serif
\usepackage{svg} % svg
\usepackage{scrextend} % local margin
\usepackage{scratch} %scratch
\usepackage{multicol} % colonnes
%\usepackage{infix-RPN,pst-func} % formule en notation polanaise inversée
\usepackage{listings}

%================================================================================================================================
%
% Réglages de base
%
%================================================================================================================================

\lstset{
language=Python,   % R code
literate=
{á}{{\'a}}1
{à}{{\`a}}1
{ã}{{\~a}}1
{é}{{\'e}}1
{è}{{\`e}}1
{ê}{{\^e}}1
{í}{{\'i}}1
{ó}{{\'o}}1
{õ}{{\~o}}1
{ú}{{\'u}}1
{ü}{{\"u}}1
{ç}{{\c{c}}}1
{~}{{ }}1
}


\definecolor{codegreen}{rgb}{0,0.6,0}
\definecolor{codegray}{rgb}{0.5,0.5,0.5}
\definecolor{codepurple}{rgb}{0.58,0,0.82}
\definecolor{backcolour}{rgb}{0.95,0.95,0.92}

\lstdefinestyle{mystyle}{
    backgroundcolor=\color{backcolour},   
    commentstyle=\color{codegreen},
    keywordstyle=\color{magenta},
    numberstyle=\tiny\color{codegray},
    stringstyle=\color{codepurple},
    basicstyle=\ttfamily\footnotesize,
    breakatwhitespace=false,         
    breaklines=true,                 
    captionpos=b,                    
    keepspaces=true,                 
    numbers=left,                    
xleftmargin=2em,
framexleftmargin=2em,            
    showspaces=false,                
    showstringspaces=false,
    showtabs=false,                  
    tabsize=2,
    upquote=true
}

\lstset{style=mystyle}


\lstset{style=mystyle}
\newcommand{\imgdir}{C:/laragon/www/newmc/assets/imgsvg/}
\newcommand{\imgsvgdir}{C:/laragon/www/newmc/assets/imgsvg/}

\definecolor{mcgris}{RGB}{220, 220, 220}% ancien~; pour compatibilité
\definecolor{mcbleu}{RGB}{52, 152, 219}
\definecolor{mcvert}{RGB}{125, 194, 70}
\definecolor{mcmauve}{RGB}{154, 0, 215}
\definecolor{mcorange}{RGB}{255, 96, 0}
\definecolor{mcturquoise}{RGB}{0, 153, 153}
\definecolor{mcrouge}{RGB}{255, 0, 0}
\definecolor{mclightvert}{RGB}{205, 234, 190}

\definecolor{gris}{RGB}{220, 220, 220}
\definecolor{bleu}{RGB}{52, 152, 219}
\definecolor{vert}{RGB}{125, 194, 70}
\definecolor{mauve}{RGB}{154, 0, 215}
\definecolor{orange}{RGB}{255, 96, 0}
\definecolor{turquoise}{RGB}{0, 153, 153}
\definecolor{rouge}{RGB}{255, 0, 0}
\definecolor{lightvert}{RGB}{205, 234, 190}
\setitemize[0]{label=\color{lightvert}  $\bullet$}

\pagestyle{fancy}
\renewcommand{\headrulewidth}{0.2pt}
\fancyhead[L]{maths-cours.fr}
\fancyhead[R]{\thepage}
\renewcommand{\footrulewidth}{0.2pt}
\fancyfoot[C]{}

\newcolumntype{C}{>{\centering\arraybackslash}X}
\newcolumntype{s}{>{\hsize=.35\hsize\arraybackslash}X}

\setlength{\parindent}{0pt}		 
\setlength{\parskip}{3mm}
\setlength{\headheight}{1cm}

\def\ebook{ebook}
\def\book{book}
\def\web{web}
\def\type{web}

\newcommand{\vect}[1]{\overrightarrow{\,\mathstrut#1\,}}

\def\Oij{$\left(\text{O}~;~\vect{\imath},~\vect{\jmath}\right)$}
\def\Oijk{$\left(\text{O}~;~\vect{\imath},~\vect{\jmath},~\vect{k}\right)$}
\def\Ouv{$\left(\text{O}~;~\vect{u},~\vect{v}\right)$}

\hypersetup{breaklinks=true, colorlinks = true, linkcolor = OliveGreen, urlcolor = OliveGreen, citecolor = OliveGreen, pdfauthor={Didier BONNEL - https://www.maths-cours.fr} } % supprime les bordures autour des liens

\renewcommand{\arg}[0]{\text{arg}}

\everymath{\displaystyle}

%================================================================================================================================
%
% Macros - Commandes
%
%================================================================================================================================

\newcommand\meta[2]{    			% Utilisé pour créer le post HTML.
	\def\titre{titre}
	\def\url{url}
	\def\arg{#1}
	\ifx\titre\arg
		\newcommand\maintitle{#2}
		\fancyhead[L]{#2}
		{\Large\sffamily \MakeUppercase{#2}}
		\vspace{1mm}\textcolor{mcvert}{\hrule}
	\fi 
	\ifx\url\arg
		\fancyfoot[L]{\href{https://www.maths-cours.fr#2}{\black \footnotesize{https://www.maths-cours.fr#2}}}
	\fi 
}


\newcommand\TitreC[1]{    		% Titre centré
     \needspace{3\baselineskip}
     \begin{center}\textbf{#1}\end{center}
}

\newcommand\newpar{    		% paragraphe
     \par
}

\newcommand\nosp {    		% commande vide (pas d'espace)
}
\newcommand{\id}[1]{} %ignore

\newcommand\boite[2]{				% Boite simple sans titre
	\vspace{5mm}
	\setlength{\fboxrule}{0.2mm}
	\setlength{\fboxsep}{5mm}	
	\fcolorbox{#1}{#1!3}{\makebox[\linewidth-2\fboxrule-2\fboxsep]{
  		\begin{minipage}[t]{\linewidth-2\fboxrule-4\fboxsep}\setlength{\parskip}{3mm}
  			 #2
  		\end{minipage}
	}}
	\vspace{5mm}
}

\newcommand\CBox[4]{				% Boites
	\vspace{5mm}
	\setlength{\fboxrule}{0.2mm}
	\setlength{\fboxsep}{5mm}
	
	\fcolorbox{#1}{#1!3}{\makebox[\linewidth-2\fboxrule-2\fboxsep]{
		\begin{minipage}[t]{1cm}\setlength{\parskip}{3mm}
	  		\textcolor{#1}{\LARGE{#2}}    
 	 	\end{minipage}  
  		\begin{minipage}[t]{\linewidth-2\fboxrule-4\fboxsep}\setlength{\parskip}{3mm}
			\raisebox{1.2mm}{\normalsize\sffamily{\textcolor{#1}{#3}}}						
  			 #4
  		\end{minipage}
	}}
	\vspace{5mm}
}

\newcommand\cadre[3]{				% Boites convertible html
	\par
	\vspace{2mm}
	\setlength{\fboxrule}{0.1mm}
	\setlength{\fboxsep}{5mm}
	\fcolorbox{#1}{white}{\makebox[\linewidth-2\fboxrule-2\fboxsep]{
  		\begin{minipage}[t]{\linewidth-2\fboxrule-4\fboxsep}\setlength{\parskip}{3mm}
			\raisebox{-2.5mm}{\sffamily \small{\textcolor{#1}{\MakeUppercase{#2}}}}		
			\par		
  			 #3
 	 		\end{minipage}
	}}
		\vspace{2mm}
	\par
}

\newcommand\bloc[3]{				% Boites convertible html sans bordure
     \needspace{2\baselineskip}
     {\sffamily \small{\textcolor{#1}{\MakeUppercase{#2}}}}    
		\par		
  			 #3
		\par
}

\newcommand\CHelp[1]{
     \CBox{Plum}{\faInfoCircle}{À RETENIR}{#1}
}

\newcommand\CUp[1]{
     \CBox{NavyBlue}{\faThumbsOUp}{EN PRATIQUE}{#1}
}

\newcommand\CInfo[1]{
     \CBox{Sepia}{\faArrowCircleRight}{REMARQUE}{#1}
}

\newcommand\CRedac[1]{
     \CBox{PineGreen}{\faEdit}{BIEN R\'EDIGER}{#1}
}

\newcommand\CError[1]{
     \CBox{Red}{\faExclamationTriangle}{ATTENTION}{#1}
}

\newcommand\TitreExo[2]{
\needspace{4\baselineskip}
 {\sffamily\large EXERCICE #1\ (\emph{#2 points})}
\vspace{5mm}
}

\newcommand\img[2]{
          \includegraphics[width=#2\paperwidth]{\imgdir#1}
}

\newcommand\imgsvg[2]{
       \begin{center}   \includegraphics[width=#2\paperwidth]{\imgsvgdir#1} \end{center}
}


\newcommand\Lien[2]{
     \href{#1}{#2 \tiny \faExternalLink}
}
\newcommand\mcLien[2]{
     \href{https~://www.maths-cours.fr/#1}{#2 \tiny \faExternalLink}
}

\newcommand{\euro}{\eurologo{}}

%================================================================================================================================
%
% Macros - Environement
%
%================================================================================================================================

\newenvironment{tex}{ %
}
{%
}

\newenvironment{indente}{ %
	\setlength\parindent{10mm}
}

{
	\setlength\parindent{0mm}
}

\newenvironment{corrige}{%
     \needspace{3\baselineskip}
     \medskip
     \textbf{\textsc{Corrigé}}
     \medskip
}
{
}

\newenvironment{extern}{%
     \begin{center}
     }
     {
     \end{center}
}

\NewEnviron{code}{%
	\par
     \boite{gray}{\texttt{%
     \BODY
     }}
     \par
}

\newenvironment{vbloc}{% boite sans cadre empeche saut de page
     \begin{minipage}[t]{\linewidth}
     }
     {
     \end{minipage}
}
\NewEnviron{h2}{%
    \needspace{3\baselineskip}
    \vspace{0.6cm}
	\noindent \MakeUppercase{\sffamily \large \BODY}
	\vspace{1mm}\textcolor{mcgris}{\hrule}\vspace{0.4cm}
	\par
}{}

\NewEnviron{h3}{%
    \needspace{3\baselineskip}
	\vspace{5mm}
	\textsc{\BODY}
	\par
}

\NewEnviron{margeneg}{ %
\begin{addmargin}[-1cm]{0cm}
\BODY
\end{addmargin}
}

\NewEnviron{html}{%
}

\begin{document}
\meta{url}{/methode/valeurs-absolues-graphiquement/}
\meta{pid}{949}
\meta{titre}{Résoudre graphiquement une équation avec des valeurs absolues}
\meta{type}{methode}
%
\cadre{vert}{Méthode}{%
     Pour résoudre graphiquement des équations du type $\left|x-a\right|=b$, on utilise la propriété du cours qui dit que $\left|x-a\right|$ représente la \textbf{distance} entre $x$ et $a$ (c'est à dire entre les points d'abscisses $x$ et $a$).
}
\bloc{orange}{Exemple}{%
     Par exemple, soit l'équation $\left|x-2\right|=3$.
     \par
     On interprète ceci comme "\textit{la distance entre x et 2 est égale à 3}". 
\par
On dessine alors le graphique suivant :
\begin{center}
\begin{extern} % alt="Equation valeurs absolues" 
\newrgbcolor{tttttt}{0.2 0.2 0.2}
\psset{xunit=1.0cm,yunit=1.0cm,algebraic=true,dimen=middle,dotstyle=*,dotsize=4pt 0,linewidth=.5pt,arrowsize=3pt 2,arrowinset=0.25}
\begin{pspicture*}(-2.5,-0.75)(7.5,1.25)
\psaxes[labelFontSize=\small,xAxis=true,yAxis=false,Dx=1.,Dy=1.,ticksize=-2pt 0,subticks=2]{->}(0,0)(-2.5,-0.75)(7.5,1.25) 
\rput[tl](-0.1,-0.25){$\small 0$} 
\parametricplot[linewidth=0.4pt,linecolor=tttttt]{1.22}{1.92}{4.45*cos(t)+0.5|4.45*sin(t)-4}
\parametricplot[linewidth=0.4pt,linecolor=tttttt]{1.22}{1.92}{4.45*cos(t)+3.5|4.45*sin(t)-4} 
\begin{scriptsize}
\rput[bl](0.5,0.6){\tttttt\normalsize{$3$}} 
\rput[bl](3.5,0.6){\tttttt\normalsize{$3$}}
\end{scriptsize}
 \psdots[linecolor=red](2,0)
 \psdots[linecolor=green](-1,0)
 \psdots[linecolor=green](5.,0)
\end{pspicture*}        
 \end{extern}
\end{center}     
Sur le graphique on voit qu'il y a deux nombres situés à 3 unités du nombre 2; ce sont -1 et 5. 
\par
Donc:
     \par
     $S=\left\{-1; 5\right\}$
}
\cadre{vert}{Variante 1}{%
     Pour une équation du type $\left|x+a\right|=b$ on utilise le fait que $x+a=x-\left(-a\right)$
}
\bloc{orange}{Exemple}{%
     Par exemple l'équation $\left|x+2\right|=3$ est identique à $\left|x-\left(-2\right)\right|=3$. 
\par
On applique alors la même méthode : \textit{la distance entre x et \textbf{-2} est égale à 3} etc. (faites le graphique!) et on trouve :
     \par
     $S=\left\{-5; 1\right\}$
}
\cadre{vert}{Variante 2}{%
     Pour une équation du type $\left|mx+a\right|=b$ on met m en facteur puis on se ramène au cas précédent en divisant chaque membre par $\left|m\right|$.
}
\bloc{orange}{Exemple}{%
     Par exemple l'équation $\left|2x-1\right|=3$ donne:
     \par
     $\left|2\left(x-\frac{1}{2}\right)\right|=3$
     \par
     $\left|2\right|\times \left|x-\frac{1}{2}\right|=3$ car $\left|ab\right|=\left|a\right|\times \left|b\right|$
     \par
     $2\times \left|x-\frac{1}{2}\right|=3$
     \par
     $\left|x-\frac{1}{2}\right|=\frac{3}{2}$ en divisant chaque membre par 2.
     \par
     On est revenu au cas précédent et on obtient~:
     \par
     $S=\left\{-1; 2\right\}$
}

\end{document}


µ
\documentclass[a4paper]{article}

%================================================================================================================================
%
% Packages
%
%================================================================================================================================

\usepackage[T1]{fontenc} 	% pour caractères accentués
\usepackage[utf8]{inputenc}  % encodage utf8
\usepackage[french]{babel}	% langue : français
\usepackage{fourier}			% caractères plus lisibles
\usepackage[dvipsnames]{xcolor} % couleurs
\usepackage{fancyhdr}		% réglage header footer
\usepackage{needspace}		% empêcher sauts de page mal placés
\usepackage{graphicx}		% pour inclure des graphiques
\usepackage{enumitem,cprotect}		% personnalise les listes d'items (nécessaire pour ol, al ...)
\usepackage{hyperref}		% Liens hypertexte
\usepackage{pstricks,pst-all,pst-node,pstricks-add,pst-math,pst-plot,pst-tree,pst-eucl} % pstricks
\usepackage[a4paper,includeheadfoot,top=2cm,left=3cm, bottom=2cm,right=3cm]{geometry} % marges etc.
\usepackage{comment}			% commentaires multilignes
\usepackage{amsmath,environ} % maths (matrices, etc.)
\usepackage{amssymb,makeidx}
\usepackage{bm}				% bold maths
\usepackage{tabularx}		% tableaux
\usepackage{colortbl}		% tableaux en couleur
\usepackage{fontawesome}		% Fontawesome
\usepackage{environ}			% environment with command
\usepackage{fp}				% calculs pour ps-tricks
\usepackage{multido}			% pour ps tricks
\usepackage[np]{numprint}	% formattage nombre
\usepackage{tikz,tkz-tab} 			% package principal TikZ
\usepackage{pgfplots}   % axes
\usepackage{mathrsfs}    % cursives
\usepackage{calc}			% calcul taille boites
\usepackage[scaled=0.875]{helvet} % font sans serif
\usepackage{svg} % svg
\usepackage{scrextend} % local margin
\usepackage{scratch} %scratch
\usepackage{multicol} % colonnes
%\usepackage{infix-RPN,pst-func} % formule en notation polanaise inversée
\usepackage{listings}

%================================================================================================================================
%
% Réglages de base
%
%================================================================================================================================

\lstset{
language=Python,   % R code
literate=
{á}{{\'a}}1
{à}{{\`a}}1
{ã}{{\~a}}1
{é}{{\'e}}1
{è}{{\`e}}1
{ê}{{\^e}}1
{í}{{\'i}}1
{ó}{{\'o}}1
{õ}{{\~o}}1
{ú}{{\'u}}1
{ü}{{\"u}}1
{ç}{{\c{c}}}1
{~}{{ }}1
}


\definecolor{codegreen}{rgb}{0,0.6,0}
\definecolor{codegray}{rgb}{0.5,0.5,0.5}
\definecolor{codepurple}{rgb}{0.58,0,0.82}
\definecolor{backcolour}{rgb}{0.95,0.95,0.92}

\lstdefinestyle{mystyle}{
    backgroundcolor=\color{backcolour},   
    commentstyle=\color{codegreen},
    keywordstyle=\color{magenta},
    numberstyle=\tiny\color{codegray},
    stringstyle=\color{codepurple},
    basicstyle=\ttfamily\footnotesize,
    breakatwhitespace=false,         
    breaklines=true,                 
    captionpos=b,                    
    keepspaces=true,                 
    numbers=left,                    
xleftmargin=2em,
framexleftmargin=2em,            
    showspaces=false,                
    showstringspaces=false,
    showtabs=false,                  
    tabsize=2,
    upquote=true
}

\lstset{style=mystyle}


\lstset{style=mystyle}
\newcommand{\imgdir}{C:/laragon/www/newmc/assets/imgsvg/}
\newcommand{\imgsvgdir}{C:/laragon/www/newmc/assets/imgsvg/}

\definecolor{mcgris}{RGB}{220, 220, 220}% ancien~; pour compatibilité
\definecolor{mcbleu}{RGB}{52, 152, 219}
\definecolor{mcvert}{RGB}{125, 194, 70}
\definecolor{mcmauve}{RGB}{154, 0, 215}
\definecolor{mcorange}{RGB}{255, 96, 0}
\definecolor{mcturquoise}{RGB}{0, 153, 153}
\definecolor{mcrouge}{RGB}{255, 0, 0}
\definecolor{mclightvert}{RGB}{205, 234, 190}

\definecolor{gris}{RGB}{220, 220, 220}
\definecolor{bleu}{RGB}{52, 152, 219}
\definecolor{vert}{RGB}{125, 194, 70}
\definecolor{mauve}{RGB}{154, 0, 215}
\definecolor{orange}{RGB}{255, 96, 0}
\definecolor{turquoise}{RGB}{0, 153, 153}
\definecolor{rouge}{RGB}{255, 0, 0}
\definecolor{lightvert}{RGB}{205, 234, 190}
\setitemize[0]{label=\color{lightvert}  $\bullet$}

\pagestyle{fancy}
\renewcommand{\headrulewidth}{0.2pt}
\fancyhead[L]{maths-cours.fr}
\fancyhead[R]{\thepage}
\renewcommand{\footrulewidth}{0.2pt}
\fancyfoot[C]{}

\newcolumntype{C}{>{\centering\arraybackslash}X}
\newcolumntype{s}{>{\hsize=.35\hsize\arraybackslash}X}

\setlength{\parindent}{0pt}		 
\setlength{\parskip}{3mm}
\setlength{\headheight}{1cm}

\def\ebook{ebook}
\def\book{book}
\def\web{web}
\def\type{web}

\newcommand{\vect}[1]{\overrightarrow{\,\mathstrut#1\,}}

\def\Oij{$\left(\text{O}~;~\vect{\imath},~\vect{\jmath}\right)$}
\def\Oijk{$\left(\text{O}~;~\vect{\imath},~\vect{\jmath},~\vect{k}\right)$}
\def\Ouv{$\left(\text{O}~;~\vect{u},~\vect{v}\right)$}

\hypersetup{breaklinks=true, colorlinks = true, linkcolor = OliveGreen, urlcolor = OliveGreen, citecolor = OliveGreen, pdfauthor={Didier BONNEL - https://www.maths-cours.fr} } % supprime les bordures autour des liens

\renewcommand{\arg}[0]{\text{arg}}

\everymath{\displaystyle}

%================================================================================================================================
%
% Macros - Commandes
%
%================================================================================================================================

\newcommand\meta[2]{    			% Utilisé pour créer le post HTML.
	\def\titre{titre}
	\def\url{url}
	\def\arg{#1}
	\ifx\titre\arg
		\newcommand\maintitle{#2}
		\fancyhead[L]{#2}
		{\Large\sffamily \MakeUppercase{#2}}
		\vspace{1mm}\textcolor{mcvert}{\hrule}
	\fi 
	\ifx\url\arg
		\fancyfoot[L]{\href{https://www.maths-cours.fr#2}{\black \footnotesize{https://www.maths-cours.fr#2}}}
	\fi 
}


\newcommand\TitreC[1]{    		% Titre centré
     \needspace{3\baselineskip}
     \begin{center}\textbf{#1}\end{center}
}

\newcommand\newpar{    		% paragraphe
     \par
}

\newcommand\nosp {    		% commande vide (pas d'espace)
}
\newcommand{\id}[1]{} %ignore

\newcommand\boite[2]{				% Boite simple sans titre
	\vspace{5mm}
	\setlength{\fboxrule}{0.2mm}
	\setlength{\fboxsep}{5mm}	
	\fcolorbox{#1}{#1!3}{\makebox[\linewidth-2\fboxrule-2\fboxsep]{
  		\begin{minipage}[t]{\linewidth-2\fboxrule-4\fboxsep}\setlength{\parskip}{3mm}
  			 #2
  		\end{minipage}
	}}
	\vspace{5mm}
}

\newcommand\CBox[4]{				% Boites
	\vspace{5mm}
	\setlength{\fboxrule}{0.2mm}
	\setlength{\fboxsep}{5mm}
	
	\fcolorbox{#1}{#1!3}{\makebox[\linewidth-2\fboxrule-2\fboxsep]{
		\begin{minipage}[t]{1cm}\setlength{\parskip}{3mm}
	  		\textcolor{#1}{\LARGE{#2}}    
 	 	\end{minipage}  
  		\begin{minipage}[t]{\linewidth-2\fboxrule-4\fboxsep}\setlength{\parskip}{3mm}
			\raisebox{1.2mm}{\normalsize\sffamily{\textcolor{#1}{#3}}}						
  			 #4
  		\end{minipage}
	}}
	\vspace{5mm}
}

\newcommand\cadre[3]{				% Boites convertible html
	\par
	\vspace{2mm}
	\setlength{\fboxrule}{0.1mm}
	\setlength{\fboxsep}{5mm}
	\fcolorbox{#1}{white}{\makebox[\linewidth-2\fboxrule-2\fboxsep]{
  		\begin{minipage}[t]{\linewidth-2\fboxrule-4\fboxsep}\setlength{\parskip}{3mm}
			\raisebox{-2.5mm}{\sffamily \small{\textcolor{#1}{\MakeUppercase{#2}}}}		
			\par		
  			 #3
 	 		\end{minipage}
	}}
		\vspace{2mm}
	\par
}

\newcommand\bloc[3]{				% Boites convertible html sans bordure
     \needspace{2\baselineskip}
     {\sffamily \small{\textcolor{#1}{\MakeUppercase{#2}}}}    
		\par		
  			 #3
		\par
}

\newcommand\CHelp[1]{
     \CBox{Plum}{\faInfoCircle}{À RETENIR}{#1}
}

\newcommand\CUp[1]{
     \CBox{NavyBlue}{\faThumbsOUp}{EN PRATIQUE}{#1}
}

\newcommand\CInfo[1]{
     \CBox{Sepia}{\faArrowCircleRight}{REMARQUE}{#1}
}

\newcommand\CRedac[1]{
     \CBox{PineGreen}{\faEdit}{BIEN R\'EDIGER}{#1}
}

\newcommand\CError[1]{
     \CBox{Red}{\faExclamationTriangle}{ATTENTION}{#1}
}

\newcommand\TitreExo[2]{
\needspace{4\baselineskip}
 {\sffamily\large EXERCICE #1\ (\emph{#2 points})}
\vspace{5mm}
}

\newcommand\img[2]{
          \includegraphics[width=#2\paperwidth]{\imgdir#1}
}

\newcommand\imgsvg[2]{
       \begin{center}   \includegraphics[width=#2\paperwidth]{\imgsvgdir#1} \end{center}
}


\newcommand\Lien[2]{
     \href{#1}{#2 \tiny \faExternalLink}
}
\newcommand\mcLien[2]{
     \href{https~://www.maths-cours.fr/#1}{#2 \tiny \faExternalLink}
}

\newcommand{\euro}{\eurologo{}}

%================================================================================================================================
%
% Macros - Environement
%
%================================================================================================================================

\newenvironment{tex}{ %
}
{%
}

\newenvironment{indente}{ %
	\setlength\parindent{10mm}
}

{
	\setlength\parindent{0mm}
}

\newenvironment{corrige}{%
     \needspace{3\baselineskip}
     \medskip
     \textbf{\textsc{Corrigé}}
     \medskip
}
{
}

\newenvironment{extern}{%
     \begin{center}
     }
     {
     \end{center}
}

\NewEnviron{code}{%
	\par
     \boite{gray}{\texttt{%
     \BODY
     }}
     \par
}

\newenvironment{vbloc}{% boite sans cadre empeche saut de page
     \begin{minipage}[t]{\linewidth}
     }
     {
     \end{minipage}
}
\NewEnviron{h2}{%
    \needspace{3\baselineskip}
    \vspace{0.6cm}
	\noindent \MakeUppercase{\sffamily \large \BODY}
	\vspace{1mm}\textcolor{mcgris}{\hrule}\vspace{0.4cm}
	\par
}{}

\NewEnviron{h3}{%
    \needspace{3\baselineskip}
	\vspace{5mm}
	\textsc{\BODY}
	\par
}

\NewEnviron{margeneg}{ %
\begin{addmargin}[-1cm]{0cm}
\BODY
\end{addmargin}
}

\NewEnviron{html}{%
}

\begin{document}
\meta{url}{/methode/inequation-avec-valeurs-absolues/}
\meta{pid}{950}
\meta{titre}{Résoudre graphiquement une inéquation avec valeurs absolues}
\meta{type}{methode}
%
\cadre{vert}{Méthode}{%
     Pour résoudre graphiquement des inéquations du type $\left|x-a\right| < b$ ou $\left|x-a\right| \leqslant b$ ou $\left|x-a\right| > b$ ou $\left|x-a\right| \geqslant b $, on utilise la propriété du cours qui dit que $\left|x-a\right|$ représente la \textbf{distance} entre $x$ et $a$ (plus précisément entre les points d'abscisses $x$ et $a$).
}
\bloc{orange}{Exemple}{%
     Par exemple, soit l'inéquation $\left|x-2\right| < 3$.
     \par
     On interprète ceci comme  \og \textit{la distance entre x et 2 est strictement inférieure à 3} \fg{}. 
\\
On trace donc le graphique suivant :

\begin{center}
\begin{extern}%alt="Inéquation valeurs absolues" 
\newrgbcolor{tttttt}{0.2 0.2 0.2}
\psset{xunit=1.0cm,yunit=1.0cm,algebraic=true,dimen=middle,dotstyle=*,dotsize=4pt 0,linewidth=.5pt,arrowsize=3pt 2,arrowinset=0.25}
\begin{pspicture*}(-2.5,-0.75)(7.5,1.25)
\psaxes[labelFontSize=\small,xAxis=true,yAxis=false,Dx=1.,Dy=1.,ticksize=-2pt 0,subticks=2]{->}(0,0)(-2.5,-0.75)(7.5,1.25) 
\rput[tl](-0.1,-0.25){$\small 0$} 
\parametricplot[linewidth=0.4pt,linecolor=tttttt]{1.22}{1.92}{4.45*cos(t)+0.5|4.45*sin(t)-4}
\parametricplot[linewidth=0.4pt,linecolor=tttttt]{1.22}{1.92}{4.45*cos(t)+3.5|4.45*sin(t)-4} 
\begin{scriptsize}
\rput[bl](0.5,0.6){\tttttt\normalsize{$3$}} 
\rput[bl](3.5,0.6){\tttttt\normalsize{$3$}}
\end{scriptsize}
 \psdots[linecolor=red](2,0)
\psline[linewidth=1.2pt,linecolor=red](-1.,0.)(5.,0.)
\rput[tl](-1.07,0.2){$\red\Large\textbf{]}$}
\rput[tl](4.93,0.2){$\red\Large\textbf{[}$} 
\end{pspicture*}        
  \end{extern}
\end{center}
          
     Sur le graphique on voit que les nombres situés à moins de 3 unités du nombre 2 sont les nombres de l'intervalle $\left]-1; 5\right[$. Donc:
     \par
     $S=\left]-1; 5\right[$
     \par
     Si l'inéquation avait été $\left|x-2\right| \leqslant 3$, il fallait prendre les extrémités de l'intervalle. L'ensemble des solutions était alors l'intervalle \textit{fermé}:
     \par
     $S=\left[-1; 5\right]$
}
\cadre{vert}{Variante 1}{%
     Pour une inéquation du type $\left|x-a\right| > b$ l'ensemble des solutions est la réunion de deux intervalles.
}
\bloc{orange}{Exemple}{%
     Par exemple pour l'inéquation $\left|x-2\right| > 3$, les solutions sont les nombres situés à plus de 3 unités du nombre 2. 
\par
On trouve donc :
     \par
     $S=\left]-\infty ; -1\right[ \cup  \left]5; \infty \right[$
}
\cadre{vert}{Variante 2}{%
     Pour une inéquation du type $\left|x+a\right| < b$ on utilise le fait que $x+a=x-\left(-a\right)$.
}
\bloc{orange}{Exemple}{%
     Par exemple l'inéquation $\left|x+2\right| < 3$ est identique à $\left|x-\left(-2\right)\right| < 3$. 
\par
On applique alors la même méthode : \textit{la distance entre x et \textbf{-2} est strictement à 3} etc. (faites le graphique!) et on trouve :
     \par
     $S=\left]-5; 1\right[$
}
\cadre{vert}{Variante 3}{%
     Pour une inéquation du type $\left|mx+a\right| < b$ on met $m$ en facteur puis on se ramène au cas précédent en divisant chaque membre par $\left|m\right|$.
}
\bloc{orange}{Exemple}{%
     Par exemple l'inéquation $\left|2x-1\right| < 3$ donne:
     \par
     $\left|2\left(x-\frac{1}{2}\right)\right| < 3$
     \par
     $\left|2\right|\times \left|x-\frac{1}{2}\right| < 3$ car $\left|ab\right|=\left|a\right|\times \left|b\right|$
     \par
     $2\times \left|x-\frac{1}{2}\right| < 3$
     \par
     $\left|x-\frac{1}{2}\right| < \frac{3}{2}$ en divisant chaque membre par 2.
     \par
     On est revenu au cas précédent et on trouve:
     \par
     $S=\left]-1; 2\right[$
}

\end{document}

µ
\documentclass[a4paper]{article}

%================================================================================================================================
%
% Packages
%
%================================================================================================================================

\usepackage[T1]{fontenc} 	% pour caractères accentués
\usepackage[utf8]{inputenc}  % encodage utf8
\usepackage[french]{babel}	% langue : français
\usepackage{fourier}			% caractères plus lisibles
\usepackage[dvipsnames]{xcolor} % couleurs
\usepackage{fancyhdr}		% réglage header footer
\usepackage{needspace}		% empêcher sauts de page mal placés
\usepackage{graphicx}		% pour inclure des graphiques
\usepackage{enumitem,cprotect}		% personnalise les listes d'items (nécessaire pour ol, al ...)
\usepackage{hyperref}		% Liens hypertexte
\usepackage{pstricks,pst-all,pst-node,pstricks-add,pst-math,pst-plot,pst-tree,pst-eucl} % pstricks
\usepackage[a4paper,includeheadfoot,top=2cm,left=3cm, bottom=2cm,right=3cm]{geometry} % marges etc.
\usepackage{comment}			% commentaires multilignes
\usepackage{amsmath,environ} % maths (matrices, etc.)
\usepackage{amssymb,makeidx}
\usepackage{bm}				% bold maths
\usepackage{tabularx}		% tableaux
\usepackage{colortbl}		% tableaux en couleur
\usepackage{fontawesome}		% Fontawesome
\usepackage{environ}			% environment with command
\usepackage{fp}				% calculs pour ps-tricks
\usepackage{multido}			% pour ps tricks
\usepackage[np]{numprint}	% formattage nombre
\usepackage{tikz,tkz-tab} 			% package principal TikZ
\usepackage{pgfplots}   % axes
\usepackage{mathrsfs}    % cursives
\usepackage{calc}			% calcul taille boites
\usepackage[scaled=0.875]{helvet} % font sans serif
\usepackage{svg} % svg
\usepackage{scrextend} % local margin
\usepackage{scratch} %scratch
\usepackage{multicol} % colonnes
%\usepackage{infix-RPN,pst-func} % formule en notation polanaise inversée
\usepackage{listings}

%================================================================================================================================
%
% Réglages de base
%
%================================================================================================================================

\lstset{
language=Python,   % R code
literate=
{á}{{\'a}}1
{à}{{\`a}}1
{ã}{{\~a}}1
{é}{{\'e}}1
{è}{{\`e}}1
{ê}{{\^e}}1
{í}{{\'i}}1
{ó}{{\'o}}1
{õ}{{\~o}}1
{ú}{{\'u}}1
{ü}{{\"u}}1
{ç}{{\c{c}}}1
{~}{{ }}1
}


\definecolor{codegreen}{rgb}{0,0.6,0}
\definecolor{codegray}{rgb}{0.5,0.5,0.5}
\definecolor{codepurple}{rgb}{0.58,0,0.82}
\definecolor{backcolour}{rgb}{0.95,0.95,0.92}

\lstdefinestyle{mystyle}{
    backgroundcolor=\color{backcolour},   
    commentstyle=\color{codegreen},
    keywordstyle=\color{magenta},
    numberstyle=\tiny\color{codegray},
    stringstyle=\color{codepurple},
    basicstyle=\ttfamily\footnotesize,
    breakatwhitespace=false,         
    breaklines=true,                 
    captionpos=b,                    
    keepspaces=true,                 
    numbers=left,                    
xleftmargin=2em,
framexleftmargin=2em,            
    showspaces=false,                
    showstringspaces=false,
    showtabs=false,                  
    tabsize=2,
    upquote=true
}

\lstset{style=mystyle}


\lstset{style=mystyle}
\newcommand{\imgdir}{C:/laragon/www/newmc/assets/imgsvg/}
\newcommand{\imgsvgdir}{C:/laragon/www/newmc/assets/imgsvg/}

\definecolor{mcgris}{RGB}{220, 220, 220}% ancien~; pour compatibilité
\definecolor{mcbleu}{RGB}{52, 152, 219}
\definecolor{mcvert}{RGB}{125, 194, 70}
\definecolor{mcmauve}{RGB}{154, 0, 215}
\definecolor{mcorange}{RGB}{255, 96, 0}
\definecolor{mcturquoise}{RGB}{0, 153, 153}
\definecolor{mcrouge}{RGB}{255, 0, 0}
\definecolor{mclightvert}{RGB}{205, 234, 190}

\definecolor{gris}{RGB}{220, 220, 220}
\definecolor{bleu}{RGB}{52, 152, 219}
\definecolor{vert}{RGB}{125, 194, 70}
\definecolor{mauve}{RGB}{154, 0, 215}
\definecolor{orange}{RGB}{255, 96, 0}
\definecolor{turquoise}{RGB}{0, 153, 153}
\definecolor{rouge}{RGB}{255, 0, 0}
\definecolor{lightvert}{RGB}{205, 234, 190}
\setitemize[0]{label=\color{lightvert}  $\bullet$}

\pagestyle{fancy}
\renewcommand{\headrulewidth}{0.2pt}
\fancyhead[L]{maths-cours.fr}
\fancyhead[R]{\thepage}
\renewcommand{\footrulewidth}{0.2pt}
\fancyfoot[C]{}

\newcolumntype{C}{>{\centering\arraybackslash}X}
\newcolumntype{s}{>{\hsize=.35\hsize\arraybackslash}X}

\setlength{\parindent}{0pt}		 
\setlength{\parskip}{3mm}
\setlength{\headheight}{1cm}

\def\ebook{ebook}
\def\book{book}
\def\web{web}
\def\type{web}

\newcommand{\vect}[1]{\overrightarrow{\,\mathstrut#1\,}}

\def\Oij{$\left(\text{O}~;~\vect{\imath},~\vect{\jmath}\right)$}
\def\Oijk{$\left(\text{O}~;~\vect{\imath},~\vect{\jmath},~\vect{k}\right)$}
\def\Ouv{$\left(\text{O}~;~\vect{u},~\vect{v}\right)$}

\hypersetup{breaklinks=true, colorlinks = true, linkcolor = OliveGreen, urlcolor = OliveGreen, citecolor = OliveGreen, pdfauthor={Didier BONNEL - https://www.maths-cours.fr} } % supprime les bordures autour des liens

\renewcommand{\arg}[0]{\text{arg}}

\everymath{\displaystyle}

%================================================================================================================================
%
% Macros - Commandes
%
%================================================================================================================================

\newcommand\meta[2]{    			% Utilisé pour créer le post HTML.
	\def\titre{titre}
	\def\url{url}
	\def\arg{#1}
	\ifx\titre\arg
		\newcommand\maintitle{#2}
		\fancyhead[L]{#2}
		{\Large\sffamily \MakeUppercase{#2}}
		\vspace{1mm}\textcolor{mcvert}{\hrule}
	\fi 
	\ifx\url\arg
		\fancyfoot[L]{\href{https://www.maths-cours.fr#2}{\black \footnotesize{https://www.maths-cours.fr#2}}}
	\fi 
}


\newcommand\TitreC[1]{    		% Titre centré
     \needspace{3\baselineskip}
     \begin{center}\textbf{#1}\end{center}
}

\newcommand\newpar{    		% paragraphe
     \par
}

\newcommand\nosp {    		% commande vide (pas d'espace)
}
\newcommand{\id}[1]{} %ignore

\newcommand\boite[2]{				% Boite simple sans titre
	\vspace{5mm}
	\setlength{\fboxrule}{0.2mm}
	\setlength{\fboxsep}{5mm}	
	\fcolorbox{#1}{#1!3}{\makebox[\linewidth-2\fboxrule-2\fboxsep]{
  		\begin{minipage}[t]{\linewidth-2\fboxrule-4\fboxsep}\setlength{\parskip}{3mm}
  			 #2
  		\end{minipage}
	}}
	\vspace{5mm}
}

\newcommand\CBox[4]{				% Boites
	\vspace{5mm}
	\setlength{\fboxrule}{0.2mm}
	\setlength{\fboxsep}{5mm}
	
	\fcolorbox{#1}{#1!3}{\makebox[\linewidth-2\fboxrule-2\fboxsep]{
		\begin{minipage}[t]{1cm}\setlength{\parskip}{3mm}
	  		\textcolor{#1}{\LARGE{#2}}    
 	 	\end{minipage}  
  		\begin{minipage}[t]{\linewidth-2\fboxrule-4\fboxsep}\setlength{\parskip}{3mm}
			\raisebox{1.2mm}{\normalsize\sffamily{\textcolor{#1}{#3}}}						
  			 #4
  		\end{minipage}
	}}
	\vspace{5mm}
}

\newcommand\cadre[3]{				% Boites convertible html
	\par
	\vspace{2mm}
	\setlength{\fboxrule}{0.1mm}
	\setlength{\fboxsep}{5mm}
	\fcolorbox{#1}{white}{\makebox[\linewidth-2\fboxrule-2\fboxsep]{
  		\begin{minipage}[t]{\linewidth-2\fboxrule-4\fboxsep}\setlength{\parskip}{3mm}
			\raisebox{-2.5mm}{\sffamily \small{\textcolor{#1}{\MakeUppercase{#2}}}}		
			\par		
  			 #3
 	 		\end{minipage}
	}}
		\vspace{2mm}
	\par
}

\newcommand\bloc[3]{				% Boites convertible html sans bordure
     \needspace{2\baselineskip}
     {\sffamily \small{\textcolor{#1}{\MakeUppercase{#2}}}}    
		\par		
  			 #3
		\par
}

\newcommand\CHelp[1]{
     \CBox{Plum}{\faInfoCircle}{À RETENIR}{#1}
}

\newcommand\CUp[1]{
     \CBox{NavyBlue}{\faThumbsOUp}{EN PRATIQUE}{#1}
}

\newcommand\CInfo[1]{
     \CBox{Sepia}{\faArrowCircleRight}{REMARQUE}{#1}
}

\newcommand\CRedac[1]{
     \CBox{PineGreen}{\faEdit}{BIEN R\'EDIGER}{#1}
}

\newcommand\CError[1]{
     \CBox{Red}{\faExclamationTriangle}{ATTENTION}{#1}
}

\newcommand\TitreExo[2]{
\needspace{4\baselineskip}
 {\sffamily\large EXERCICE #1\ (\emph{#2 points})}
\vspace{5mm}
}

\newcommand\img[2]{
          \includegraphics[width=#2\paperwidth]{\imgdir#1}
}

\newcommand\imgsvg[2]{
       \begin{center}   \includegraphics[width=#2\paperwidth]{\imgsvgdir#1} \end{center}
}


\newcommand\Lien[2]{
     \href{#1}{#2 \tiny \faExternalLink}
}
\newcommand\mcLien[2]{
     \href{https~://www.maths-cours.fr/#1}{#2 \tiny \faExternalLink}
}

\newcommand{\euro}{\eurologo{}}

%================================================================================================================================
%
% Macros - Environement
%
%================================================================================================================================

\newenvironment{tex}{ %
}
{%
}

\newenvironment{indente}{ %
	\setlength\parindent{10mm}
}

{
	\setlength\parindent{0mm}
}

\newenvironment{corrige}{%
     \needspace{3\baselineskip}
     \medskip
     \textbf{\textsc{Corrigé}}
     \medskip
}
{
}

\newenvironment{extern}{%
     \begin{center}
     }
     {
     \end{center}
}

\NewEnviron{code}{%
	\par
     \boite{gray}{\texttt{%
     \BODY
     }}
     \par
}

\newenvironment{vbloc}{% boite sans cadre empeche saut de page
     \begin{minipage}[t]{\linewidth}
     }
     {
     \end{minipage}
}
\NewEnviron{h2}{%
    \needspace{3\baselineskip}
    \vspace{0.6cm}
	\noindent \MakeUppercase{\sffamily \large \BODY}
	\vspace{1mm}\textcolor{mcgris}{\hrule}\vspace{0.4cm}
	\par
}{}

\NewEnviron{h3}{%
    \needspace{3\baselineskip}
	\vspace{5mm}
	\textsc{\BODY}
	\par
}

\NewEnviron{margeneg}{ %
\begin{addmargin}[-1cm]{0cm}
\BODY
\end{addmargin}
}

\NewEnviron{html}{%
}

\begin{document}
\meta{url}{/exercices/droites-equations-vecteur-directeur-140818/}
\meta{pid}{967}
\meta{pi_}{967}
\meta{titre}{Équation cartésienne d'une droite -Vecteur directeur}
\meta{type}{exercices}
Pour chacune des droites dont une équation est donnée ci-dessous, déterminer :
\begin{itemize}
     \item un vecteur directeur
     \item l'équation réduite
     \item le coefficient directeur
\end{itemize}
\begin{enumerate}\item $x-y+1=0$
     \item $x+2y=0$
     \item $4x-2y+5=0$
     \item $3y+1=0$
     \item $-x+1=0$
\end{enumerate}
\begin{corrige}
     \begin{enumerate}
          \item   $\vec{u}\left(1 ; 1\right)  $  (voir \mcLien{/premiere-s/vecteurs-droites}{cours})
          \par
          $y = x+1$
          \par
          $a=1$
          \item   $\vec{u}\left(-2 ; 1\right)$
          \par
          $y = -\frac{x}{2}$
          \par
          $a=-\frac{1}{2}$
          \item   $\vec{u}\left(2 ; 4\right)  $ (ou $\vec{u}\left(1 ; 2\right)$, ou tout autre vecteur colinéaire à celui-ci...)
          \par
          $y = 2x+\frac{5}{2}$
          \par
          $a=2$
          \item   $\vec{u}\left(3 ; 0\right)  $ (ou $\vec{u}\left(1 ; 0\right)$, ...)
          \par
          $y = -\frac{1}{3}$
          \par
          $a=0$
          \item   $\vec{u}\left(0 ; 1\right)$ (ou $\vec{u}\left( 0 ; -1 \right)$, ...)
          \par
          $x = 1$
          \par
          Il n'y a pas de coefficient directeur (droite parallèle à l'axe des ordonnées)
     \end{enumerate}
     \textit{Remarque : pour le vecteur directeur, il y a, à chaque question,  une infinité de réponses possibles...}
\end{corrige}

\end{document}
µ
\documentclass[a4paper]{article}

%================================================================================================================================
%
% Packages
%
%================================================================================================================================

\usepackage[T1]{fontenc} 	% pour caractères accentués
\usepackage[utf8]{inputenc}  % encodage utf8
\usepackage[french]{babel}	% langue : français
\usepackage{fourier}			% caractères plus lisibles
\usepackage[dvipsnames]{xcolor} % couleurs
\usepackage{fancyhdr}		% réglage header footer
\usepackage{needspace}		% empêcher sauts de page mal placés
\usepackage{graphicx}		% pour inclure des graphiques
\usepackage{enumitem,cprotect}		% personnalise les listes d'items (nécessaire pour ol, al ...)
\usepackage{hyperref}		% Liens hypertexte
\usepackage{pstricks,pst-all,pst-node,pstricks-add,pst-math,pst-plot,pst-tree,pst-eucl} % pstricks
\usepackage[a4paper,includeheadfoot,top=2cm,left=3cm, bottom=2cm,right=3cm]{geometry} % marges etc.
\usepackage{comment}			% commentaires multilignes
\usepackage{amsmath,environ} % maths (matrices, etc.)
\usepackage{amssymb,makeidx}
\usepackage{bm}				% bold maths
\usepackage{tabularx}		% tableaux
\usepackage{colortbl}		% tableaux en couleur
\usepackage{fontawesome}		% Fontawesome
\usepackage{environ}			% environment with command
\usepackage{fp}				% calculs pour ps-tricks
\usepackage{multido}			% pour ps tricks
\usepackage[np]{numprint}	% formattage nombre
\usepackage{tikz,tkz-tab} 			% package principal TikZ
\usepackage{pgfplots}   % axes
\usepackage{mathrsfs}    % cursives
\usepackage{calc}			% calcul taille boites
\usepackage[scaled=0.875]{helvet} % font sans serif
\usepackage{svg} % svg
\usepackage{scrextend} % local margin
\usepackage{scratch} %scratch
\usepackage{multicol} % colonnes
%\usepackage{infix-RPN,pst-func} % formule en notation polanaise inversée
\usepackage{listings}

%================================================================================================================================
%
% Réglages de base
%
%================================================================================================================================

\lstset{
language=Python,   % R code
literate=
{á}{{\'a}}1
{à}{{\`a}}1
{ã}{{\~a}}1
{é}{{\'e}}1
{è}{{\`e}}1
{ê}{{\^e}}1
{í}{{\'i}}1
{ó}{{\'o}}1
{õ}{{\~o}}1
{ú}{{\'u}}1
{ü}{{\"u}}1
{ç}{{\c{c}}}1
{~}{{ }}1
}


\definecolor{codegreen}{rgb}{0,0.6,0}
\definecolor{codegray}{rgb}{0.5,0.5,0.5}
\definecolor{codepurple}{rgb}{0.58,0,0.82}
\definecolor{backcolour}{rgb}{0.95,0.95,0.92}

\lstdefinestyle{mystyle}{
    backgroundcolor=\color{backcolour},   
    commentstyle=\color{codegreen},
    keywordstyle=\color{magenta},
    numberstyle=\tiny\color{codegray},
    stringstyle=\color{codepurple},
    basicstyle=\ttfamily\footnotesize,
    breakatwhitespace=false,         
    breaklines=true,                 
    captionpos=b,                    
    keepspaces=true,                 
    numbers=left,                    
xleftmargin=2em,
framexleftmargin=2em,            
    showspaces=false,                
    showstringspaces=false,
    showtabs=false,                  
    tabsize=2,
    upquote=true
}

\lstset{style=mystyle}


\lstset{style=mystyle}
\newcommand{\imgdir}{C:/laragon/www/newmc/assets/imgsvg/}
\newcommand{\imgsvgdir}{C:/laragon/www/newmc/assets/imgsvg/}

\definecolor{mcgris}{RGB}{220, 220, 220}% ancien~; pour compatibilité
\definecolor{mcbleu}{RGB}{52, 152, 219}
\definecolor{mcvert}{RGB}{125, 194, 70}
\definecolor{mcmauve}{RGB}{154, 0, 215}
\definecolor{mcorange}{RGB}{255, 96, 0}
\definecolor{mcturquoise}{RGB}{0, 153, 153}
\definecolor{mcrouge}{RGB}{255, 0, 0}
\definecolor{mclightvert}{RGB}{205, 234, 190}

\definecolor{gris}{RGB}{220, 220, 220}
\definecolor{bleu}{RGB}{52, 152, 219}
\definecolor{vert}{RGB}{125, 194, 70}
\definecolor{mauve}{RGB}{154, 0, 215}
\definecolor{orange}{RGB}{255, 96, 0}
\definecolor{turquoise}{RGB}{0, 153, 153}
\definecolor{rouge}{RGB}{255, 0, 0}
\definecolor{lightvert}{RGB}{205, 234, 190}
\setitemize[0]{label=\color{lightvert}  $\bullet$}

\pagestyle{fancy}
\renewcommand{\headrulewidth}{0.2pt}
\fancyhead[L]{maths-cours.fr}
\fancyhead[R]{\thepage}
\renewcommand{\footrulewidth}{0.2pt}
\fancyfoot[C]{}

\newcolumntype{C}{>{\centering\arraybackslash}X}
\newcolumntype{s}{>{\hsize=.35\hsize\arraybackslash}X}

\setlength{\parindent}{0pt}		 
\setlength{\parskip}{3mm}
\setlength{\headheight}{1cm}

\def\ebook{ebook}
\def\book{book}
\def\web{web}
\def\type{web}

\newcommand{\vect}[1]{\overrightarrow{\,\mathstrut#1\,}}

\def\Oij{$\left(\text{O}~;~\vect{\imath},~\vect{\jmath}\right)$}
\def\Oijk{$\left(\text{O}~;~\vect{\imath},~\vect{\jmath},~\vect{k}\right)$}
\def\Ouv{$\left(\text{O}~;~\vect{u},~\vect{v}\right)$}

\hypersetup{breaklinks=true, colorlinks = true, linkcolor = OliveGreen, urlcolor = OliveGreen, citecolor = OliveGreen, pdfauthor={Didier BONNEL - https://www.maths-cours.fr} } % supprime les bordures autour des liens

\renewcommand{\arg}[0]{\text{arg}}

\everymath{\displaystyle}

%================================================================================================================================
%
% Macros - Commandes
%
%================================================================================================================================

\newcommand\meta[2]{    			% Utilisé pour créer le post HTML.
	\def\titre{titre}
	\def\url{url}
	\def\arg{#1}
	\ifx\titre\arg
		\newcommand\maintitle{#2}
		\fancyhead[L]{#2}
		{\Large\sffamily \MakeUppercase{#2}}
		\vspace{1mm}\textcolor{mcvert}{\hrule}
	\fi 
	\ifx\url\arg
		\fancyfoot[L]{\href{https://www.maths-cours.fr#2}{\black \footnotesize{https://www.maths-cours.fr#2}}}
	\fi 
}


\newcommand\TitreC[1]{    		% Titre centré
     \needspace{3\baselineskip}
     \begin{center}\textbf{#1}\end{center}
}

\newcommand\newpar{    		% paragraphe
     \par
}

\newcommand\nosp {    		% commande vide (pas d'espace)
}
\newcommand{\id}[1]{} %ignore

\newcommand\boite[2]{				% Boite simple sans titre
	\vspace{5mm}
	\setlength{\fboxrule}{0.2mm}
	\setlength{\fboxsep}{5mm}	
	\fcolorbox{#1}{#1!3}{\makebox[\linewidth-2\fboxrule-2\fboxsep]{
  		\begin{minipage}[t]{\linewidth-2\fboxrule-4\fboxsep}\setlength{\parskip}{3mm}
  			 #2
  		\end{minipage}
	}}
	\vspace{5mm}
}

\newcommand\CBox[4]{				% Boites
	\vspace{5mm}
	\setlength{\fboxrule}{0.2mm}
	\setlength{\fboxsep}{5mm}
	
	\fcolorbox{#1}{#1!3}{\makebox[\linewidth-2\fboxrule-2\fboxsep]{
		\begin{minipage}[t]{1cm}\setlength{\parskip}{3mm}
	  		\textcolor{#1}{\LARGE{#2}}    
 	 	\end{minipage}  
  		\begin{minipage}[t]{\linewidth-2\fboxrule-4\fboxsep}\setlength{\parskip}{3mm}
			\raisebox{1.2mm}{\normalsize\sffamily{\textcolor{#1}{#3}}}						
  			 #4
  		\end{minipage}
	}}
	\vspace{5mm}
}

\newcommand\cadre[3]{				% Boites convertible html
	\par
	\vspace{2mm}
	\setlength{\fboxrule}{0.1mm}
	\setlength{\fboxsep}{5mm}
	\fcolorbox{#1}{white}{\makebox[\linewidth-2\fboxrule-2\fboxsep]{
  		\begin{minipage}[t]{\linewidth-2\fboxrule-4\fboxsep}\setlength{\parskip}{3mm}
			\raisebox{-2.5mm}{\sffamily \small{\textcolor{#1}{\MakeUppercase{#2}}}}		
			\par		
  			 #3
 	 		\end{minipage}
	}}
		\vspace{2mm}
	\par
}

\newcommand\bloc[3]{				% Boites convertible html sans bordure
     \needspace{2\baselineskip}
     {\sffamily \small{\textcolor{#1}{\MakeUppercase{#2}}}}    
		\par		
  			 #3
		\par
}

\newcommand\CHelp[1]{
     \CBox{Plum}{\faInfoCircle}{À RETENIR}{#1}
}

\newcommand\CUp[1]{
     \CBox{NavyBlue}{\faThumbsOUp}{EN PRATIQUE}{#1}
}

\newcommand\CInfo[1]{
     \CBox{Sepia}{\faArrowCircleRight}{REMARQUE}{#1}
}

\newcommand\CRedac[1]{
     \CBox{PineGreen}{\faEdit}{BIEN R\'EDIGER}{#1}
}

\newcommand\CError[1]{
     \CBox{Red}{\faExclamationTriangle}{ATTENTION}{#1}
}

\newcommand\TitreExo[2]{
\needspace{4\baselineskip}
 {\sffamily\large EXERCICE #1\ (\emph{#2 points})}
\vspace{5mm}
}

\newcommand\img[2]{
          \includegraphics[width=#2\paperwidth]{\imgdir#1}
}

\newcommand\imgsvg[2]{
       \begin{center}   \includegraphics[width=#2\paperwidth]{\imgsvgdir#1} \end{center}
}


\newcommand\Lien[2]{
     \href{#1}{#2 \tiny \faExternalLink}
}
\newcommand\mcLien[2]{
     \href{https~://www.maths-cours.fr/#1}{#2 \tiny \faExternalLink}
}

\newcommand{\euro}{\eurologo{}}

%================================================================================================================================
%
% Macros - Environement
%
%================================================================================================================================

\newenvironment{tex}{ %
}
{%
}

\newenvironment{indente}{ %
	\setlength\parindent{10mm}
}

{
	\setlength\parindent{0mm}
}

\newenvironment{corrige}{%
     \needspace{3\baselineskip}
     \medskip
     \textbf{\textsc{Corrigé}}
     \medskip
}
{
}

\newenvironment{extern}{%
     \begin{center}
     }
     {
     \end{center}
}

\NewEnviron{code}{%
	\par
     \boite{gray}{\texttt{%
     \BODY
     }}
     \par
}

\newenvironment{vbloc}{% boite sans cadre empeche saut de page
     \begin{minipage}[t]{\linewidth}
     }
     {
     \end{minipage}
}
\NewEnviron{h2}{%
    \needspace{3\baselineskip}
    \vspace{0.6cm}
	\noindent \MakeUppercase{\sffamily \large \BODY}
	\vspace{1mm}\textcolor{mcgris}{\hrule}\vspace{0.4cm}
	\par
}{}

\NewEnviron{h3}{%
    \needspace{3\baselineskip}
	\vspace{5mm}
	\textsc{\BODY}
	\par
}

\NewEnviron{margeneg}{ %
\begin{addmargin}[-1cm]{0cm}
\BODY
\end{addmargin}
}

\NewEnviron{html}{%
}

\begin{document}
\meta{url}{/cours/graphe-probabiliste-spe/}
\meta{pid}{1163}
\meta{titre}{Graphe probabiliste [spé]}
\meta{type}{cours}
\begin{h2}I. Étude d'un exemple\end{h2}
On étudie la propagation d'une maladie dans une population.
\par
On choisit au hasard une personne dans cette population.
\par
On note :
\begin{itemize}
     \item $M_n$ l'événement \og la personne est malade le $n$ième jour de l'étude \fg{} ;
     \item $\overline{M_n}$ l'événement \og la personne est saine le $n$ième jour de l'étude \fg{} ;
     \item $p_n$ la probabilité de l'événement $M_n$ ;
     \item $q_n$ la probabilité de l'événement $\overline{M_n}$.
\end{itemize}
On suppose que :
\begin{itemize}
     \item la probabilité qu'une personne malade soit guérie le lendemain est $0,3$ ;
     \item la probabilité qu'une personne saine tombe malade le lendemain est $0,2$.
\end{itemize}
Au début de l'étude, la maladie touche 5 \% de la population. On a donc $p_0=0,05$ et $q_0=0,95$.
\bloc{orange}{a. Utilisation d'un arbre}{% id="e010"
     On peut représenter la situation au jour 0 et au jour 1 par l'arbre ci-dessous :
     \begin{center}
          \begin{extern} %width="350" alt="arbre pondéré" class="aligncenter"
               % Racine à Gauche, développement vers la droite
               \begin{tikzpicture}[xscale=1,yscale=1]
                    % Styles (MODIFIABLES)
                    \tikzstyle{fleche}=[-,>=latex,thick]
                    \tikzstyle{noeud}=[fill=white,circle,draw]
                    \tikzstyle{feuille}=[fill=white,circle,draw]
                    \tikzstyle{etiquette}=[midway,fill=white]
                    % Dimensions (MODIFIABLES)
                    \def\DistanceInterNiveaux{3}
                    \def\DistanceInterFeuilles{2}
                    % Dimensions calculées (NON MODIFIABLES)
                    \def\NiveauA{(0)*\DistanceInterNiveaux}
                    \def\NiveauB{(1.5)*\DistanceInterNiveaux}
                    \def\NiveauC{(2.5)*\DistanceInterNiveaux}
                    \def\InterFeuilles{(-1)*\DistanceInterFeuilles}
                    % Noeuds (MODIFIABLES : Styles et Coefficients d'InterFeuilles)
                    \node[noeud] (R) at ({\NiveauA},{(1.5)*\InterFeuilles}) {$\ $};
                    \node[noeud] (Ra) at ({\NiveauB},{(0.5)*\InterFeuilles}) {$M_0$};
                    \node[feuille] (Raa) at ({\NiveauC},{(0)*\InterFeuilles}) {$M_1$};
                    \node[feuille] (Rab) at ({\NiveauC},{(1)*\InterFeuilles}) {$\overline{M_1}$};
                    \node[noeud] (Rb) at ({\NiveauB},{(2.5)*\InterFeuilles}) {$\overline{M_0}$};
                    \node[feuille] (Rba) at ({\NiveauC},{(2)*\InterFeuilles}) {$M_1$};
                    \node[feuille] (Rbb) at ({\NiveauC},{(3)*\InterFeuilles}) {$\overline{M_1}$};
                    % Arcs (MODIFIABLES : Styles)
                    \draw[fleche] (R)--(Ra) node[etiquette] {$0,05$};
                    \draw[fleche] (Ra)--(Raa) node[etiquette] {$0,7$};
                    \draw[fleche] (Ra)--(Rab) node[etiquette] {$0,3$};
                    \draw[fleche] (R)--(Rb) node[etiquette] {$0,95$};
                    \draw[fleche] (Rb)--(Rba) node[etiquette] {$0,2$};
                    \draw[fleche] (Rb)--(Rbb) node[etiquette] {$0,8$};
               \end{tikzpicture}
          \end{extern}
     \end{center}
     La formule des probabilités totales permet de calculer les probabilités $p_1$ et $q_1$ :
     \par
     $p_1=0,05\times 0,7 + 0,95\times 0,2 = 0,225$
     \par
     $q_1 =0,05\times 0,3 + 0,95\times 0,8  = 0,775$
     \par
     De même, on peut représenter l'évolution du jour $n$ au jour $n+1$ grâce à l'arbre ci-dessous :
     \begin{center}
          \begin{extern} %width="350" alt="arbre pondéré" class="aligncenter"
               % Racine à Gauche, développement vers la droite
               \begin{tikzpicture}[xscale=1,yscale=1]
                    % Styles (MODIFIABLES)
                    \tikzstyle{fleche}=[-,>=latex,thick]
                    \tikzstyle{noeud}=[fill=white,circle,draw]
                    \tikzstyle{feuille}=[fill=white,circle,draw]
                    \tikzstyle{etiquette}=[midway,fill=white]
                    % Dimensions (MODIFIABLES)
                    \def\DistanceInterNiveaux{3}
                    \def\DistanceInterFeuilles{2}
                    % Dimensions calculées (NON MODIFIABLES)
                    \def\NiveauA{(0)*\DistanceInterNiveaux}
                    \def\NiveauB{(1.5)*\DistanceInterNiveaux}
                    \def\NiveauC{(2.5)*\DistanceInterNiveaux}
                    \def\InterFeuilles{(-1)*\DistanceInterFeuilles}
                    % Noeuds (MODIFIABLES : Styles et Coefficients d'InterFeuilles)
                    \node[noeud] (R) at ({\NiveauA},{(1.5)*\InterFeuilles}) {$\ $};
                    \node[noeud] (Ra) at ({\NiveauB},{(0.5)*\InterFeuilles}) {$M_n$};
                    \node[feuille] (Raa) at ({\NiveauC},{(0)*\InterFeuilles}) {$M_{n+1}$};
                    \node[feuille] (Rab) at ({\NiveauC},{(1)*\InterFeuilles}) {$\overline{M_{n+1}}$};
                    \node[noeud] (Rb) at ({\NiveauB},{(2.5)*\InterFeuilles}) {$\overline{M_n}$};
                    \node[feuille] (Rba) at ({\NiveauC},{(2)*\InterFeuilles}) {$M_{n+1}$};
                    \node[feuille] (Rbb) at ({\NiveauC},{(3)*\InterFeuilles}) {$\overline{M_{n+1}}$};
                    % Arcs (MODIFIABLES : Styles)
                    \draw[fleche] (R)--(Ra) node[etiquette] {$p_n$};
                    \draw[fleche] (Ra)--(Raa) node[etiquette] {$0,7$};
                    \draw[fleche] (Ra)--(Rab) node[etiquette] {$0,3$};
                    \draw[fleche] (R)--(Rb) node[etiquette] {$q_n$};
                    \draw[fleche] (Rb)--(Rba) node[etiquette] {$0,2$};
                    \draw[fleche] (Rb)--(Rbb) node[etiquette] {$0,8$};
               \end{tikzpicture}
          \end{extern}
     \end{center}
     On obtient, en utilisant à nouveau la formule des probabilités totales :
     \par
     $p_{n+1}=0,7 p_n + 0,2 q_n$
     \par
     $q_{n+1}=0,3 p_n + 0,8 q_n$.
}
\bloc{orange}{b. Graphe probabiliste}{% id="e020"
     À une date donnée, un individu se trouve dans l'un ou l'autre des deux états suivants :
     \begin{itemize}
          \item la personne est malade (état noté M)
          \item la personne est saine (état noté S)
     \end{itemize}
     Un \textbf{graphe probabiliste} représente ces deux \textbf{états} sous la forme de \textbf{sommets} et les \textbf{probabilités} de passer d'un état à l'autre sous la forme d'\textbf{arcs orientés}.
     \par
     Dans notre exemple, on obtient le graphe suivant :
     \begin{center}
          \begin{extern}%width="450" alt="graphe probabiliste d'ordre 2"
               \psset{unit=1cm}
               \begin{pspicture}(-2,-1)(8,1)
                    %\psgrid[subgriddiv=2,gridlabels=0,gridcolor=gray]
                    \psset{nodesep=3pt,arcangle=-15,arrowsize=3pt 3}
                    %%% Données à entrer
                    %%% G pour Gauche, d pour Droit
                    \def\nomG{$M$} \def\nomD{$S$}
                    \def\valGG{$0,7$} \def\valGD{$0,3$}
                    \def\valDG{$0,2$} \def\valDD{$0,8$}
                    %%% couleurs des sommets
                    \newrgbcolor{colorG}{0 0 1} \newrgbcolor{colorD}{1 0 0}
                    %%%
                    %%% Ne plus rien modifier à partir de cette ligne %%%
                    %%%
                    \psnode(0,0){G}{\colorG  \nomG}
                    \psnode(6,0){D}{\colorD  \nomD}
                    \psset{ArrowInside=->,ArrowInsideNo=1,arrowscale=1}
                    %%%
                    \nccircle[angleA=90,linecolor=colorG]{->}{G}{.5cm}  \Bput{\colorG \valGG}
                    \ncarc[linecolor=colorG]{G}{D} \Bput{\colorG \valGD}
                    %%%
                    \ncarc[linecolor=colorD]{D}{G} \Bput{\colorD \valDG}
                    \nccircle[angleA=-90,linecolor=colorD]{->}{D}{.5cm} \Bput{\colorD  \valDD}
               \end{pspicture}
          \end{extern}
     \end{center}
     On remarque que la somme des probabilités issues d'un même sommet (en bleu pour M et en rouge pour S) est toujours égale à 1.
}
\bloc{orange}{c. Matrice de transition}{% id="e030"
     Les relations trouvées grâce à l'arbre :
     \par
     $p_{n+1}=0,7 p_n + 0,2 q_n$
     \par
     $q_{n+1}=0,3 p_n + 0,8 q_n$
     \par
     peuvent s'écrire sous forme matricielle :
     \par
     $A=\begin{pmatrix}  p_{n+1} & q_{n+1} \end{pmatrix} = \begin{pmatrix}  p_{n} & q_{n} \end{pmatrix} \times \begin{pmatrix}  0,7 & 0,3 \\ 0,2 & 0,8 \end{pmatrix}$
     \par
     N.B. Vérifier le en effectuant le calcul :  $\begin{pmatrix}  p_{n} & q_{n} \end{pmatrix} \times \begin{pmatrix}  0,7 & 0,3 \\ 0,2 & 0,8 \end{pmatrix}$.
     \par
     En notant $T = \begin{pmatrix}  0,7 & 0,3 \\ 0,2 & 0,8 \end{pmatrix}$ et pour tout entier naturel $n$, $P_n = \begin{pmatrix}  p_{n} & q_{n} \end{pmatrix}$, la relation précédente s'écrit :
     \par
     $ p_{n+1}=P_n \times T  $
     \par
     La matrice $T$ s'appelle \textbf{matrice de transition} et les matrices lignes $P_n$, \textbf{états probabilistes}.
     \par
     On a alors :
     \par
     $ p_{1}=P_0 \times T  $ où $P_0$ est l'état initial$P_0=\begin{pmatrix}  p_{0} & q_{0} \end{pmatrix}=\begin{pmatrix}  0,05 & 0,95 \end{pmatrix}$
     \par
     $ p_{2}=P_1 \times T=P_0 \times T  \times T = P_0 \times T^2 $
     \par
     $ p_{3}=P_2 \times T=P_0 \times T^2  \times T = P_0 \times T^3 $
     \par
     et ainsi de suite...
     \par
     $ p_{n}= P_0 \times T^n $
     \par
     Par exemple, l'état probabiliste au cinquième jour sera :
     \par
     $ p_{5}= P_0 \times T^5 $
     \par
     À la calculatrice, on trouve $ p_{5}=\begin{pmatrix}  0,389 & 0,611 \end{pmatrix}$ (au millième près).
     \par
     Le cinquième jour, 38,9\% de la population sera malade.
}
\begin{h2}II. Graphe probabiliste - Matrice de transition\end{h2}
Dans toute cette partie, on considère un système possédant $n$ états possibles notés $A_1$, $A_2$, ... $A_n$. Ce système peut changer d'état au cours du temps et on suppose que la probabilité de passer de l'état $A_i$ à l'état $A_j$ reste constante.
\cadre{bleu}{Définition}{% id="d030"
     Un \textbf{graphe probabiliste} est un graphe orienté et pondéré dans lequel :
     \begin{itemize}
          \item Les sommets du graphe représentent les différents états possibles d'un système
          \item Les poids des arcs indiquent les probabilités de passage d'un état à l'autre
     \end{itemize}
     Dans un graphe orienté, la somme des poids des arcs issus d'un même sommet est égale à 1.
}
\bloc{orange}{Exemples}{% id="e033"
     \textbf{a. Graphe probabiliste d'ordre 2}
     \begin{center}
          \begin{extern}%width="450" alt="graphe probabiliste d'ordre 2"
               \psset{unit=1cm}
               \begin{pspicture}(-2,-1)(8,1)
                    %\psgrid[subgriddiv=2,gridlabels=0,gridcolor=gray]
                    \psset{nodesep=3pt,arcangle=-15,arrowsize=3pt 3}
                    %%% Données à entrer
                    %%% G pour Gauche, d pour Droit
                    \def\nomG{$A_1$} \def\nomD{$A_2$}
                    \def\valGG{$0,7$} \def\valGD{$0,3$}
                    \def\valDG{$0,2$} \def\valDD{$0,8$}
                    %%% couleurs des sommets G et H
                    \newrgbcolor{colorG}{0 0 0} \newrgbcolor{colorD}{0 0 0}
                    %%%
                    %%% Ne plus rien modifier à partir de cette ligne %%%
                    %%%
                    \psnode(0,0){G}{\colorG  \nomG}
                    \psnode(6,0){D}{\colorD  \nomD}
                    \psset{ArrowInside=->,ArrowInsideNo=1,arrowscale=1}
                    %%%
                    \nccircle[angleA=90,linecolor=colorG]{->}{G}{.5cm}  \Bput{\colorG \valGG}
                    \ncarc[linecolor=colorG]{G}{D} \Bput{\colorG \valGD}
                    %%%
                    \ncarc[linecolor=colorD]{D}{G} \Bput{\colorD \valDG}
                    \nccircle[angleA=-90,linecolor=colorD]{->}{D}{.5cm} \Bput{\colorD  \valDD}
               \end{pspicture}
          \end{extern}
     \end{center}
     \begin{center}Graphe probabiliste à 2 états $A_1$ et $A_2$\end{center}
     \textbf{b. Graphe probabiliste d'ordre 3}
     \begin{center}
          \begin{extern}%width="450" alt="graphe probabiliste d'ordre 3"
               \psset{unit=1cm}
               \begin{pspicture}(-2,-1)(8,6.5)
                    %\psgrid[subgriddiv=2,gridlabels=0,gridcolor=gray]
                    \psset{nodesep=3pt,arcangle=-15,arrowsize=3pt 3}
                    %%% Données à entrer
                    %%% G pour Gauche, d pour Droit et H pour Haut
                    \def\nomG{$A_1$} \def\nomD{$A_2$} \def\nomH{$A_3$}
                    \def\valGG{$0,5$} \def\valGD{$0,2$} \def\valGH{$0,3$}
                    \def\valDG{$0,5$} \def\valDD{$0,1$} \def\valDH{$0,4$}
                    \def\valHG{$0,8$} \def\valHD{$0,1$} \def\valHH{$0,1$}
                    %%% couleurs des sommets G et H
                    \newrgbcolor{colorG}{0 0 0} \newrgbcolor{colorD}{0 0 0} \newrgbcolor{colorH}{0 0 0}
                    %%%
                    %%% Ne plus rien modifier à partir de cette ligne %%%
                    %%%
                    \psnode(0,0){G}{\colorG  \nomG}
                    \psnode(6,0){D}{\colorD  \nomD}
                    \psnode(6;60){H}{\colorH \nomH}
                    \psset{ArrowInside=->,ArrowInsideNo=1,arrowscale=1}
                    %%%
                    \nccircle[angleA=90,linecolor=colorG]{->}{G}{.5cm}  \Bput{\colorG \valGG}
                    \ncarc[linecolor=colorG]{G}{D} \Bput{\colorG \valGD}
                    \ncarc[linecolor=colorG]{G}{H} \Bput{\colorG \valGH}
                    %%%
                    \ncarc[linecolor=colorD]{D}{G} \Bput{\colorD \valDG}
                    \nccircle[angleA=-90,linecolor=colorD]{->}{D}{.5cm} \Bput{\colorD  \valDD}
                    \ncarc[linecolor=colorD]{D}{H} \Bput{\colorD \valDH}
                    %%%
                    \ncarc[linecolor=colorH]{H}{G} \Bput{\colorH \valHG}
                    \ncarc[linecolor=colorH]{H}{D} \Bput{\colorH \valHD}
                    \nccircle[angleA=0,linecolor=colorH]{->}{H}{.5cm} \Bput{\colorH \valHH}
               \end{pspicture}
          \end{extern}
     \end{center}
     \begin{center}Graphe probabiliste à 3 états $A_1$, $A_2$ et $A_3$\end{center}
}
\cadre{bleu}{Définition}{% id="d035"
     Soit un graphe probabiliste d'ordre $n$.
     \par
     Les \textbf{états probabilistes} $P_k$ sont des matrices à une ligne à $n$ colonnes qui indiquent pour chacun des $n$ sommets, la probabilité de se trouver dans cet état à l'étape $k$.
     \par
     La somme des coefficients d'un état probabiliste est égale à 1.
}
\bloc{orange}{Remarques}{% id="e037"
     \begin{itemize}
          \item L'état probabiliste $P_0$ s'appelle l'état probabiliste \textbf{initial}.
          \item Pour un système à deux états A et B, les états probabilistes seront de la forme $P_k = \begin{pmatrix}  p_k & q_k \end{pmatrix} $ où $p_k$ et $q_k$ sont les probabilités de se trouver  respectivement dans les états A et B à l'étape $k$.
          \par
          Le système se trouvant, à chaque étape, soit dans l'état A soit dans l'état B, on aura, pour tout entier naturel $k$, $p_k + q_k = 1$
          \item Pour un système à trois états, les états probabilistes seront de la forme $P_k = \begin{pmatrix}  p_k & q_k & r_k\end{pmatrix} $ avec, pour tout entier naturel $k$, $p_k + q_k + r_k = 1  $.
     \end{itemize}
}
\cadre{bleu}{Définition}{% id="d040"
     Soit un graphe probabiliste d'ordre $n$ dont les états sont notés $A_1$, $A_2$, ... $A_n$.
     \par
     La \textbf{matrice de transition} associée un graphe probabiliste d'ordre $n$ est une matrice carrée $n \times n$ dont le terme $p_{i,j}$ situé à l'intersection de la $i$-ème ligne et de la $j$-ème colonne représente la probabilité de passer de l'état $A_i$ à l'état $A_j$.
     \par
     Dans une matrice de transition, la somme des coefficients situés sur une même ligne est égale à 1.
}
\bloc{orange}{Exemples}{% id="e033"
     \textbf{a. Graphe probabiliste d'ordre 2}
     \par
     Considérons le graphe probabiliste suivant~:
     \begin{center}
          \begin{extern}%width="450" alt="graphe probabiliste d'ordre 2"
               \psset{unit=1cm}
               \begin{pspicture}(-2,-1)(8,1)
                    %\psgrid[subgriddiv=2,gridlabels=0,gridcolor=gray]
                    \psset{nodesep=3pt,arcangle=-15,arrowsize=3pt 3}
                    %%% Données à entrer
                    %%% G pour Gauche, d pour Droit
                    \def\nomG{$A_1$} \def\nomD{$A_2$}
                    \def\valGG{$0,1$} \def\valGD{$0,9$}
                    \def\valDG{$0,4$} \def\valDD{$0,6$}
                    %%% couleurs des sommets G et H
                    \newrgbcolor{colorG}{0 0 0} \newrgbcolor{colorD}{0 0 0}
                    %%%
                    %%% Ne plus rien modifier à partir de cette ligne %%%
                    %%%
                    \psnode(0,0){G}{\colorG  \nomG}
                    \psnode(6,0){D}{\colorD  \nomD}
                    \psset{ArrowInside=->,ArrowInsideNo=1,arrowscale=1}
                    %%%
                    \nccircle[angleA=90,linecolor=colorG]{->}{G}{.5cm}  \Bput{\colorG \valGG}
                    \ncarc[linecolor=colorG]{G}{D} \Bput{\colorG \valGD}
                    %%%
                    \ncarc[linecolor=colorD]{D}{G} \Bput{\colorD \valDG}
                    \nccircle[angleA=-90,linecolor=colorD]{->}{D}{.5cm} \Bput{\colorD  \valDD}
               \end{pspicture}
          \end{extern}
     \end{center}
     La matrice de transition associée à ce graphe est :
     \begin{center}
          \img{matrice-transition-2x2-ltx}{0.4}%width="280" alt="matrice de transition 2x2"
     \end{center}
     \textbf{b. Graphe probabiliste d'ordre 3}
     \par
     Dans l'exemple du graphe d'ordre 3 :
     \begin{center}
          \begin{extern}%width="450" alt="graphe probabiliste d'ordre 3"
               \psset{unit=1cm}
               \begin{pspicture}(-2,-1)(8,6.5)
                    %\psgrid[subgriddiv=2,gridlabels=0,gridcolor=gray]
                    \psset{nodesep=3pt,arcangle=-15,arrowsize=3pt 3}
                    %%% Données à entrer
                    %%% G pour Gauche, d pour Droit et H pour Haut
                    \def\nomG{$A_1$} \def\nomD{$A_2$} \def\nomH{$A_3$}
                    \def\valGG{$0,5$} \def\valGD{$0,2$} \def\valGH{$0,3$}
                    \def\valDG{$0,5$} \def\valDD{$0,1$} \def\valDH{$0,4$}
                    \def\valHG{$0,8$} \def\valHD{$0,1$} \def\valHH{$0,1$}
                    %%% couleurs des sommets G et H
                    \newrgbcolor{colorG}{0 0 0} \newrgbcolor{colorD}{0 0 0}\newrgbcolor{colorH}{0 0 0}
                    %%%
                    %%% Ne plus rien modifier à partir de cette ligne %%%
                    %%%
                    \psnode(0,0){G}{\colorG  \nomG}
                    \psnode(6,0){D}{\colorD  \nomD}
                    \psnode(6;60){H}{\colorH \nomH}
                    \psset{ArrowInside=->,ArrowInsideNo=1,arrowscale=1}
                    %%%
                    \nccircle[angleA=90,linecolor=colorG]{->}{G}{.5cm}  \Bput{\colorG \valGG}
                    \ncarc[linecolor=colorG]{G}{D} \Bput{\colorG \valGD}
                    \ncarc[linecolor=colorG]{G}{H} \Bput{\colorG \valGH}
                    %%%
                    \ncarc[linecolor=colorD]{D}{G} \Bput{\colorD \valDG}
                    \nccircle[angleA=-90,linecolor=colorD]{->}{D}{.5cm} \Bput{\colorD  \valDD}
                    \ncarc[linecolor=colorD]{D}{H} \Bput{\colorD \valDH}
                    %%%
                    \ncarc[linecolor=colorH]{H}{G} \Bput{\colorH \valHG}
                    \ncarc[linecolor=colorH]{H}{D} \Bput{\colorH \valHD}
                    \nccircle[angleA=0,linecolor=colorH]{->}{H}{.5cm} \Bput{\colorH \valHH}
               \end{pspicture}
          \end{extern}
     \end{center}
     La matrice de transition est :
     \begin{center}
          \img{matrice-transition-3x3-ltx}{0.4}%width="320" alt="matrice de transition 3x3"
     \end{center}
}
\cadre{rouge}{Théorème}{% id="t080"
     Soit un système dont la matrice de transition est notée $T$, d'état initial $P_0$ et d'état probabiliste $P_k$ à l'étape $k$.
     \par
     Alors, pour tout entier naturel $k$ :
     \begin{itemize}
          \item $P_{k+1}=P_k \times T$
          \item $P_{k}=P_0 \times T^k$
     \end{itemize}
}
\bloc{cyan}{Démonstration}{% id="r090"
     Ce résultat se démontre en utilisant la formule des probabilités totales (voir partie {I. Étude d'un exemple } )
}
\bloc{cyan}{Remarque}{% id="r100"
     On peut rapprocher ces formules de celles obtenues pour les suites géométriques : $u_{n+1}=u_n \times q$ et $u_n=u_0 \times q^n$.
     \par
     Mais attention à l'ordre des matrices : le produit de matrices n'est pas commutatif !
}
\begin{h2}III. États stables\end{h2}
\cadre{bleu}{Définition}{% id="d130"
     Soit un graphe probabiliste de matrice de transition $T$.
     \par
     Un \textbf{état stable} est un état probabiliste $P$ qui vérifie $P=P \times T$.
}
\bloc{cyan}{Remarque}{% id="r140"
     En pratique pour trouver un état stable :
     \begin{itemize}
          \item On pose $P= \begin{pmatrix} x & y \end{pmatrix} $ si le graphe possède 2 états ( ou $P= \begin{pmatrix} x & y & z \end{pmatrix} $ si le graphe possède 3 états)
          \item On écrit le système correspondant à l'égalité matricielle $P=P \times T$
          \item On ajoute au système l'équation $x+y=1$ ( ou $x+y+z=1$ si le graphe possède 3 états ) qui caractérise les états probabilistes ( la somme des coefficients est égale à 1)
          \item On résout le système obtenu...
     \end{itemize}
}
\bloc{orange}{Exemple}{% id="e145"
     Reprenons le graphe :
     \begin{center}
          \begin{extern}%width="450" alt="graphe probabiliste d'ordre 2"
               \psset{unit=1cm}
               \begin{pspicture}(-2,-1)(8,1)
                    %\psgrid[subgriddiv=2,gridlabels=0,gridcolor=gray]
                    \psset{nodesep=3pt,arcangle=-15,arrowsize=3pt 3}
                    %%% Données à entrer
                    %%% G pour Gauche, d pour Droit
                    \def\nomG{$A_1$} \def\nomD{$A_2$}
                    \def\valGG{$0,1$} \def\valGD{$0,9$}
                    \def\valDG{$0,4$} \def\valDD{$0,6$}
                    %%% couleurs des sommets G et H
                    \newrgbcolor{colorG}{0 0 0} \newrgbcolor{colorD}{0 0 0}
                    %%%
                    %%% Ne plus rien modifier à partir de cette ligne %%%
                    %%%
                    \psnode(0,0){G}{\colorG  \nomG}
                    \psnode(6,0){D}{\colorD  \nomD}
                    \psset{ArrowInside=->,ArrowInsideNo=1,arrowscale=1}
                    %%%
                    \nccircle[angleA=90,linecolor=colorG]{->}{G}{.5cm}  \Bput{\colorG \valGG}
                    \ncarc[linecolor=colorG]{G}{D} \Bput{\colorG \valGD}
                    %%%
                    \ncarc[linecolor=colorD]{D}{G} \Bput{\colorD \valDG}
                    \nccircle[angleA=-90,linecolor=colorD]{->}{D}{.5cm} \Bput{\colorD  \valDD}
               \end{pspicture}
          \end{extern}
     \end{center}
     La matrice de transition associée est $T= \begin{pmatrix} 0,1 & 0,9  \\  0,4 & 0,6 \end{pmatrix}$.
     \par
     En posant $P= \begin{pmatrix} x & y  \end{pmatrix}$ on obtient le système
     \par
     $ \begin{cases} x = 0,1x+0,4y \\ y = 0,9x+0,6y \\ x+y=1\end{cases} $
     \par
     qui est équivalent à :
     \par
     $ \begin{cases} 0,9x-0,4y=0 \\ 0,9x-0,4y=0 \\ y=1-x \end{cases} $
     \par
     Les deux premières équations sont identiques et en remplaçant $y$ par $1-x$ dans la première équation on obtient :
     \par
     $ \begin{cases} 0,9x-0,4(1-x)=0 \\  y=1-x \end{cases} $
     \par
     $ \begin{cases} x=\dfrac{4}{13} \\  \\ y=\dfrac{9}{13}  \end{cases} $
     \par
     Il y a donc un état stable $P = \begin{pmatrix} \dfrac{4}{13} & \dfrac{9}{13}  \end{pmatrix}$
}
\cadre{vert}{Propriété}{% id="p160"
     Si la matrice de transition d'un graphe probabiliste d'ordre 2 ou 3 ne comporte pas de coefficient nul, alors :
     \begin{itemize}
          \item le graphe possède un état stable $P$
          \item les états probabilistes $P_n$ tendent vers l'état stable $P$ quand $n$ devient grand.
     \end{itemize}
}
\bloc{cyan}{Remarques}{% id="r165"
     \begin{itemize}
          \item Cette propriété donne une condition suffisante mais non nécessaire. Il est possible qu'une matrice de transition comporte des coefficients nuls mais que le système converge malgré tout vers un état stable.
          \item Le résultat précédent est également valable si le degré du graphe est supérieur à 3 (mais n'est pas au programme dans ce cas). Il s'agit d'un cas particulier du \Lien{http://www.bibmath.net/dico/index.php?action=affiche&quoi=./p/perron-frobenius.html}{théorème de Perron-Frobenius}
     \end{itemize}
}

\end{document}
µ
\documentclass[a4paper]{article}

%================================================================================================================================
%
% Packages
%
%================================================================================================================================

\usepackage[T1]{fontenc} 	% pour caractères accentués
\usepackage[utf8]{inputenc}  % encodage utf8
\usepackage[french]{babel}	% langue : français
\usepackage{fourier}			% caractères plus lisibles
\usepackage[dvipsnames]{xcolor} % couleurs
\usepackage{fancyhdr}		% réglage header footer
\usepackage{needspace}		% empêcher sauts de page mal placés
\usepackage{graphicx}		% pour inclure des graphiques
\usepackage{enumitem,cprotect}		% personnalise les listes d'items (nécessaire pour ol, al ...)
\usepackage{hyperref}		% Liens hypertexte
\usepackage{pstricks,pst-all,pst-node,pstricks-add,pst-math,pst-plot,pst-tree,pst-eucl} % pstricks
\usepackage[a4paper,includeheadfoot,top=2cm,left=3cm, bottom=2cm,right=3cm]{geometry} % marges etc.
\usepackage{comment}			% commentaires multilignes
\usepackage{amsmath,environ} % maths (matrices, etc.)
\usepackage{amssymb,makeidx}
\usepackage{bm}				% bold maths
\usepackage{tabularx}		% tableaux
\usepackage{colortbl}		% tableaux en couleur
\usepackage{fontawesome}		% Fontawesome
\usepackage{environ}			% environment with command
\usepackage{fp}				% calculs pour ps-tricks
\usepackage{multido}			% pour ps tricks
\usepackage[np]{numprint}	% formattage nombre
\usepackage{tikz,tkz-tab} 			% package principal TikZ
\usepackage{pgfplots}   % axes
\usepackage{mathrsfs}    % cursives
\usepackage{calc}			% calcul taille boites
\usepackage[scaled=0.875]{helvet} % font sans serif
\usepackage{svg} % svg
\usepackage{scrextend} % local margin
\usepackage{scratch} %scratch
\usepackage{multicol} % colonnes
%\usepackage{infix-RPN,pst-func} % formule en notation polanaise inversée
\usepackage{listings}

%================================================================================================================================
%
% Réglages de base
%
%================================================================================================================================

\lstset{
language=Python,   % R code
literate=
{á}{{\'a}}1
{à}{{\`a}}1
{ã}{{\~a}}1
{é}{{\'e}}1
{è}{{\`e}}1
{ê}{{\^e}}1
{í}{{\'i}}1
{ó}{{\'o}}1
{õ}{{\~o}}1
{ú}{{\'u}}1
{ü}{{\"u}}1
{ç}{{\c{c}}}1
{~}{{ }}1
}


\definecolor{codegreen}{rgb}{0,0.6,0}
\definecolor{codegray}{rgb}{0.5,0.5,0.5}
\definecolor{codepurple}{rgb}{0.58,0,0.82}
\definecolor{backcolour}{rgb}{0.95,0.95,0.92}

\lstdefinestyle{mystyle}{
    backgroundcolor=\color{backcolour},   
    commentstyle=\color{codegreen},
    keywordstyle=\color{magenta},
    numberstyle=\tiny\color{codegray},
    stringstyle=\color{codepurple},
    basicstyle=\ttfamily\footnotesize,
    breakatwhitespace=false,         
    breaklines=true,                 
    captionpos=b,                    
    keepspaces=true,                 
    numbers=left,                    
xleftmargin=2em,
framexleftmargin=2em,            
    showspaces=false,                
    showstringspaces=false,
    showtabs=false,                  
    tabsize=2,
    upquote=true
}

\lstset{style=mystyle}


\lstset{style=mystyle}
\newcommand{\imgdir}{C:/laragon/www/newmc/assets/imgsvg/}
\newcommand{\imgsvgdir}{C:/laragon/www/newmc/assets/imgsvg/}

\definecolor{mcgris}{RGB}{220, 220, 220}% ancien~; pour compatibilité
\definecolor{mcbleu}{RGB}{52, 152, 219}
\definecolor{mcvert}{RGB}{125, 194, 70}
\definecolor{mcmauve}{RGB}{154, 0, 215}
\definecolor{mcorange}{RGB}{255, 96, 0}
\definecolor{mcturquoise}{RGB}{0, 153, 153}
\definecolor{mcrouge}{RGB}{255, 0, 0}
\definecolor{mclightvert}{RGB}{205, 234, 190}

\definecolor{gris}{RGB}{220, 220, 220}
\definecolor{bleu}{RGB}{52, 152, 219}
\definecolor{vert}{RGB}{125, 194, 70}
\definecolor{mauve}{RGB}{154, 0, 215}
\definecolor{orange}{RGB}{255, 96, 0}
\definecolor{turquoise}{RGB}{0, 153, 153}
\definecolor{rouge}{RGB}{255, 0, 0}
\definecolor{lightvert}{RGB}{205, 234, 190}
\setitemize[0]{label=\color{lightvert}  $\bullet$}

\pagestyle{fancy}
\renewcommand{\headrulewidth}{0.2pt}
\fancyhead[L]{maths-cours.fr}
\fancyhead[R]{\thepage}
\renewcommand{\footrulewidth}{0.2pt}
\fancyfoot[C]{}

\newcolumntype{C}{>{\centering\arraybackslash}X}
\newcolumntype{s}{>{\hsize=.35\hsize\arraybackslash}X}

\setlength{\parindent}{0pt}		 
\setlength{\parskip}{3mm}
\setlength{\headheight}{1cm}

\def\ebook{ebook}
\def\book{book}
\def\web{web}
\def\type{web}

\newcommand{\vect}[1]{\overrightarrow{\,\mathstrut#1\,}}

\def\Oij{$\left(\text{O}~;~\vect{\imath},~\vect{\jmath}\right)$}
\def\Oijk{$\left(\text{O}~;~\vect{\imath},~\vect{\jmath},~\vect{k}\right)$}
\def\Ouv{$\left(\text{O}~;~\vect{u},~\vect{v}\right)$}

\hypersetup{breaklinks=true, colorlinks = true, linkcolor = OliveGreen, urlcolor = OliveGreen, citecolor = OliveGreen, pdfauthor={Didier BONNEL - https://www.maths-cours.fr} } % supprime les bordures autour des liens

\renewcommand{\arg}[0]{\text{arg}}

\everymath{\displaystyle}

%================================================================================================================================
%
% Macros - Commandes
%
%================================================================================================================================

\newcommand\meta[2]{    			% Utilisé pour créer le post HTML.
	\def\titre{titre}
	\def\url{url}
	\def\arg{#1}
	\ifx\titre\arg
		\newcommand\maintitle{#2}
		\fancyhead[L]{#2}
		{\Large\sffamily \MakeUppercase{#2}}
		\vspace{1mm}\textcolor{mcvert}{\hrule}
	\fi 
	\ifx\url\arg
		\fancyfoot[L]{\href{https://www.maths-cours.fr#2}{\black \footnotesize{https://www.maths-cours.fr#2}}}
	\fi 
}


\newcommand\TitreC[1]{    		% Titre centré
     \needspace{3\baselineskip}
     \begin{center}\textbf{#1}\end{center}
}

\newcommand\newpar{    		% paragraphe
     \par
}

\newcommand\nosp {    		% commande vide (pas d'espace)
}
\newcommand{\id}[1]{} %ignore

\newcommand\boite[2]{				% Boite simple sans titre
	\vspace{5mm}
	\setlength{\fboxrule}{0.2mm}
	\setlength{\fboxsep}{5mm}	
	\fcolorbox{#1}{#1!3}{\makebox[\linewidth-2\fboxrule-2\fboxsep]{
  		\begin{minipage}[t]{\linewidth-2\fboxrule-4\fboxsep}\setlength{\parskip}{3mm}
  			 #2
  		\end{minipage}
	}}
	\vspace{5mm}
}

\newcommand\CBox[4]{				% Boites
	\vspace{5mm}
	\setlength{\fboxrule}{0.2mm}
	\setlength{\fboxsep}{5mm}
	
	\fcolorbox{#1}{#1!3}{\makebox[\linewidth-2\fboxrule-2\fboxsep]{
		\begin{minipage}[t]{1cm}\setlength{\parskip}{3mm}
	  		\textcolor{#1}{\LARGE{#2}}    
 	 	\end{minipage}  
  		\begin{minipage}[t]{\linewidth-2\fboxrule-4\fboxsep}\setlength{\parskip}{3mm}
			\raisebox{1.2mm}{\normalsize\sffamily{\textcolor{#1}{#3}}}						
  			 #4
  		\end{minipage}
	}}
	\vspace{5mm}
}

\newcommand\cadre[3]{				% Boites convertible html
	\par
	\vspace{2mm}
	\setlength{\fboxrule}{0.1mm}
	\setlength{\fboxsep}{5mm}
	\fcolorbox{#1}{white}{\makebox[\linewidth-2\fboxrule-2\fboxsep]{
  		\begin{minipage}[t]{\linewidth-2\fboxrule-4\fboxsep}\setlength{\parskip}{3mm}
			\raisebox{-2.5mm}{\sffamily \small{\textcolor{#1}{\MakeUppercase{#2}}}}		
			\par		
  			 #3
 	 		\end{minipage}
	}}
		\vspace{2mm}
	\par
}

\newcommand\bloc[3]{				% Boites convertible html sans bordure
     \needspace{2\baselineskip}
     {\sffamily \small{\textcolor{#1}{\MakeUppercase{#2}}}}    
		\par		
  			 #3
		\par
}

\newcommand\CHelp[1]{
     \CBox{Plum}{\faInfoCircle}{À RETENIR}{#1}
}

\newcommand\CUp[1]{
     \CBox{NavyBlue}{\faThumbsOUp}{EN PRATIQUE}{#1}
}

\newcommand\CInfo[1]{
     \CBox{Sepia}{\faArrowCircleRight}{REMARQUE}{#1}
}

\newcommand\CRedac[1]{
     \CBox{PineGreen}{\faEdit}{BIEN R\'EDIGER}{#1}
}

\newcommand\CError[1]{
     \CBox{Red}{\faExclamationTriangle}{ATTENTION}{#1}
}

\newcommand\TitreExo[2]{
\needspace{4\baselineskip}
 {\sffamily\large EXERCICE #1\ (\emph{#2 points})}
\vspace{5mm}
}

\newcommand\img[2]{
          \includegraphics[width=#2\paperwidth]{\imgdir#1}
}

\newcommand\imgsvg[2]{
       \begin{center}   \includegraphics[width=#2\paperwidth]{\imgsvgdir#1} \end{center}
}


\newcommand\Lien[2]{
     \href{#1}{#2 \tiny \faExternalLink}
}
\newcommand\mcLien[2]{
     \href{https~://www.maths-cours.fr/#1}{#2 \tiny \faExternalLink}
}

\newcommand{\euro}{\eurologo{}}

%================================================================================================================================
%
% Macros - Environement
%
%================================================================================================================================

\newenvironment{tex}{ %
}
{%
}

\newenvironment{indente}{ %
	\setlength\parindent{10mm}
}

{
	\setlength\parindent{0mm}
}

\newenvironment{corrige}{%
     \needspace{3\baselineskip}
     \medskip
     \textbf{\textsc{Corrigé}}
     \medskip
}
{
}

\newenvironment{extern}{%
     \begin{center}
     }
     {
     \end{center}
}

\NewEnviron{code}{%
	\par
     \boite{gray}{\texttt{%
     \BODY
     }}
     \par
}

\newenvironment{vbloc}{% boite sans cadre empeche saut de page
     \begin{minipage}[t]{\linewidth}
     }
     {
     \end{minipage}
}
\NewEnviron{h2}{%
    \needspace{3\baselineskip}
    \vspace{0.6cm}
	\noindent \MakeUppercase{\sffamily \large \BODY}
	\vspace{1mm}\textcolor{mcgris}{\hrule}\vspace{0.4cm}
	\par
}{}

\NewEnviron{h3}{%
    \needspace{3\baselineskip}
	\vspace{5mm}
	\textsc{\BODY}
	\par
}

\NewEnviron{margeneg}{ %
\begin{addmargin}[-1cm]{0cm}
\BODY
\end{addmargin}
}

\NewEnviron{html}{%
}

\begin{document}
\meta{url}{/methode/limites-du-type-k0/}
\meta{pid}{1302}
\meta{pi_}{1302}
\meta{titre}{Limites du type "k/0"}
\meta{type}{methode}
\cadre{vert}{Situation}{%id=s100
     On cherche à calculer la limite d'une fraction rationnelle lorsque$x$ tend vers une valeur $a$ qui annule le dénominateur; par exemple $\lim\limits_{x\rightarrow 1} \frac{x+2}{x^{2}-1}. $
}
\cadre{rouge}{Méthode}{%id=m100
     \begin{itemize}
          \item Si on a affaire à une limite du type \og $\frac{0}{0}$ \fg{} (forme indéterminée), on lève l'indétermination en factorisant le numérateur et le dénominateur puis en simplifiant la fraction
          \item Si on a affaire à une limite du type \og  $\frac{k}{0}$ \fg{} avec $k \neq 0$:\begin{itemize}
               \item on distingue les limites à gauche et à droite :
               \newpar
               $\lim\limits_{x\rightarrow  a^-} f\left(x\right)$ et $\lim\limits_{x\rightarrow  a^+} f\left(x\right)$
               \item les limites seront égales à $+\infty $ ou $-\infty $
               \item pour déterminer le signe de la limite on étudie le signe du quotient. On peut toutefois se limiter à l'étude de signe au voisinage de $a$ (voir exemple 3)
          \end{itemize}
     \end{itemize}
}
\bloc{orange}{Exemple 1}{%id=e100
     Calculer $\lim\limits_{x\rightarrow 2} \frac{x^{2}-3x+2}{x^{2}-4}$
     \newpar
     En remplaçant $x$ par 2 dans la fraction rationnelle on obtient \og $\frac{0}{0}$ \fg{}.
     \newpar
     On lève l'indétermination en simplifiant la fraction.
     \newpar
     2 est racine de $x^{2}-3x+2$ comme on vient de le voir. Le produit des racines vaut $\frac{c}{a}=2$ donc l'autre racine est 1 (on peut, si l'on préfère, calculer le discriminant puis les racines, mais c'est plus long…).
     \newpar
     $x^{2}-3x+2$ peut donc se factoriser sous la forme $\left(x-1\right)\left(x-2\right)$.
     \newpar
     $x^{2}-4=\left(x-2\right)\left(x+2\right)$ (identité remarquable)
     \newpar
     Donc :
     \newpar
     $\lim\limits_{x\rightarrow 2} \frac{x^{2}-3x+2}{x^{2}-4} = \lim\limits_{x\rightarrow 2} \frac{\left(x-1\right)\left(x-2\right)}{\left(x-2\right)\left(x+2\right)}=\lim\limits_{x\rightarrow 2} \frac{x-1}{x+2} = \frac{1}{4}$
}
\bloc{orange}{Exemple 2}{%id=e200
     Calculer $\lim\limits_{x\rightarrow -1} \frac{2}{1+x}$
     \newpar
     En remplaçant $x$ par -1 dans la fraction rationnelle on obtient \og $\frac{2}{0}$  \fg{}.
     \newpar
     La limite est donc infinie.
     \newpar
     Pour \textbf{l'étude du signe} on distingue les limites à gauche et à droite.
     \newpar
     Le numérateur est toujours positif.
     \begin{itemize}
          \item si $x < -1$, $1+x$ est strictement négatif
          \item si $x > -1$, $1+x$ est strictement positif donc :
     \end{itemize}
     $\lim\limits_{x\rightarrow -1^-} \frac{2}{1+x}=-\infty $
     \newpar
     $\lim\limits_{x\rightarrow -1^+} \frac{2}{1+x}=+\infty $
}
\bloc{orange}{Exemple 3}{%id=e300
     Calculer $\lim\limits_{x\rightarrow 0} \frac{x^{3}+x-3}{x^{2}-x}$
     \newpar
     En «remplaçant $x$ par 0» dans la fraction rationnelle on obtient «$-\frac{3}{0}$».
     \newpar
     La limite sera donc infinie. On distingue les limites à gauche et à droite.
     \newpar
     Il n'est pas facile de factoriser le numérateur qui est du troisième degré. Heureusement, cela ne sera pas nécessaire ici !
     \newpar
     On ne va pas construire le tableau de signes sur $\mathbb{R}$ tout entier mais \textbf{seulement au voisinage de zéro}.
     \newpar
     Si $x$ est proche de zéro le numérateur sera proche de $-3$ donc négatif.
     \newpar
     Le dénominateur se factorise $x^{2}-x=x\left(x-1\right)$ et $x-1$ est proche de $-1$ (donc négatif) lorsque $x$ est proche de 0.
     \newpar
     On obtient alors le tableau de signe au voisinage de $0$ :
     \begin{center}
          \begin{extern}%width="500" alt="Exemple tableau de signes d'un quotient"
               \resizebox{11cm}{!}{
                    \begin{tikzpicture}[scale=0.875]
                         % Styles
                         \tikzstyle{cadre}=[thin]
                         \tikzstyle{fleche}=[->,>=latex,thin]
                         \tikzstyle{nondefini}=[lightgray]
                         % Dimensions Modifiables
                         \def\Lrg{1.5}
                         \def\HtX{1.2}
                         \def\HtY{0.5}
                         % Dimensions Calculées
                         \def\lignex{-0.5*\HtX}
                         \def\lignea{-1.5*\HtX}
                         \def\ligneb{-2.5*\HtX}
                         \def\lignec{-3.5*\HtX}
                         \def\ligned{-4.5*\HtX}
                         \def\separateur{-0.5*\Lrg}
                         % Largeur du tableau
                         \def\gauche{-3.1*\Lrg}
                         \def\droite{4.5*\Lrg}
                         % Hauteur du tableau
                         \def\haut{0.5*\HtX}
                         \def\bas{-2.5*\HtX-2*\HtY}
                         % Pointillés
                         \draw[gray] (2*\Lrg,\lignex) -- (2*\Lrg,\lignec); les tensions
                         \draw[double distance=2pt] (2*\Lrg,\lignec) -- (2*\Lrg,\ligned);
                         % Ligne de l'abscisse : x
                         \node at (-1.8*\Lrg,0) {$x$};
                         \node at (0*\Lrg,0) {$\cdots$};
                         \node at (2*\Lrg,0) {$0$};
                         \node at (4*\Lrg,0) {$\cdots$};
                         % Ligne a
                         \node at (-1.8*\Lrg,-1*\HtX) {$x^{3}+x-3$};
                         \node at (0*\Lrg,-1*\HtX) {$ $};
                         \node at (1*\Lrg,-1*\HtX) {$-$};
                         \node at (2*\Lrg,-1*\HtX) {$ $};
                         \node at (3*\Lrg,-1*\HtX) {$-$};
                         % Ligne b
                         \node at (-1.8*\Lrg,-2*\HtX) {$x$};
                         \node at (2*\Lrg,-2*\HtX) {$ $};
                         \node at (1*\Lrg,-2*\HtX) {$-$};
                         \node at (2*\Lrg,-2*\HtX) {$0$};
                         \node at (3*\Lrg,-2*\HtX) {$+$};
                         % Ligne c
                         \node at (-1.8*\Lrg,-3*\HtX) {$x-1$};
                         \node at (0*\Lrg,-3*\HtX) {$ $};
                         \node at (1*\Lrg,-3*\HtX) {$-$};
                         \node at (2*\Lrg,-3*\HtX) {$ $};
                         \node at (3*\Lrg,-3*\HtX) {$-$};
                         % Ligne d
                         \node at (-1.8*\Lrg,-4*\HtX) {$\frac{x^{3}+x-3}{x^{2}-x}$};
                         \node at (0*\Lrg,-4*\HtX) {$ $};
                         \node at (1*\Lrg,-4*\HtX) {$-$};
                         \node at (2*\Lrg,-4*\HtX) {$ $};
                         \node at (3*\Lrg,-4*\HtX) {$+$};
                         % Encadrement
                         \draw[cadre] (\separateur,\haut) -- (\separateur, \ligned);
                         \draw[cadre] (\gauche,\haut) rectangle  (\droite, \ligned);
                         \draw[cadre] (\gauche,\lignex) -- (\droite,\lignex);
                         \draw[cadre] (\gauche,\lignea) -- (\droite,\lignea);
                         \draw[cadre] (\gauche,\ligneb) -- (\droite,\ligneb);
                         \draw[cadre] (\gauche,\lignec) -- (\droite,\lignec);
                    \end{tikzpicture}
               }
          \end{extern}
     \end{center}
     Donc~:
     \newpar
     $\lim\limits_{x\rightarrow 0^-}\frac{x^{3}+x-3}{x^{2}-x}=-\infty $
     \newpar
     $\lim\limits_{x\rightarrow 0^+}\frac{x^{3}+x-3}{x^{2}-x}=+\infty $
}
\bloc{cyan}{Remarque}{%id=r400
     Une petite astuce pour vérifier votre résultat \textbf{à la calculatrice}.
     \newpar
     Pour avoir une idée de la valeur de $\lim\limits_{x\rightarrow  a}f\left(x\right)$, donnez à $x$ des valeurs proches de $a$ et calculer $f\left(x\right)$
     \newpar
     Par exemple, pour l'exemple 3, on saisit la fonction $x\mapsto \frac{x^{3}+x-3}{x^{2}-x}$ et on calcule :
     \newpar
     $f\left(-0,0000000001\right)\approx -3\times 10^{10}$
     \newpar
     $f\left(0,0000000001\right)\approx 3\times 10^{10}$
     \newpar
     ce qui confirme les valeurs ( et surtout les signes ! ) que nous avons trouvées ($-\infty $ et $+\infty $).
}

\end{document}
µ
\documentclass[a4paper]{article}

%================================================================================================================================
%
% Packages
%
%================================================================================================================================

\usepackage[T1]{fontenc} 	% pour caractères accentués
\usepackage[utf8]{inputenc}  % encodage utf8
\usepackage[french]{babel}	% langue : français
\usepackage{fourier}			% caractères plus lisibles
\usepackage[dvipsnames]{xcolor} % couleurs
\usepackage{fancyhdr}		% réglage header footer
\usepackage{needspace}		% empêcher sauts de page mal placés
\usepackage{graphicx}		% pour inclure des graphiques
\usepackage{enumitem,cprotect}		% personnalise les listes d'items (nécessaire pour ol, al ...)
\usepackage{hyperref}		% Liens hypertexte
\usepackage{pstricks,pst-all,pst-node,pstricks-add,pst-math,pst-plot,pst-tree,pst-eucl} % pstricks
\usepackage[a4paper,includeheadfoot,top=2cm,left=3cm, bottom=2cm,right=3cm]{geometry} % marges etc.
\usepackage{comment}			% commentaires multilignes
\usepackage{amsmath,environ} % maths (matrices, etc.)
\usepackage{amssymb,makeidx}
\usepackage{bm}				% bold maths
\usepackage{tabularx}		% tableaux
\usepackage{colortbl}		% tableaux en couleur
\usepackage{fontawesome}		% Fontawesome
\usepackage{environ}			% environment with command
\usepackage{fp}				% calculs pour ps-tricks
\usepackage{multido}			% pour ps tricks
\usepackage[np]{numprint}	% formattage nombre
\usepackage{tikz,tkz-tab} 			% package principal TikZ
\usepackage{pgfplots}   % axes
\usepackage{mathrsfs}    % cursives
\usepackage{calc}			% calcul taille boites
\usepackage[scaled=0.875]{helvet} % font sans serif
\usepackage{svg} % svg
\usepackage{scrextend} % local margin
\usepackage{scratch} %scratch
\usepackage{multicol} % colonnes
%\usepackage{infix-RPN,pst-func} % formule en notation polanaise inversée
\usepackage{listings}

%================================================================================================================================
%
% Réglages de base
%
%================================================================================================================================

\lstset{
language=Python,   % R code
literate=
{á}{{\'a}}1
{à}{{\`a}}1
{ã}{{\~a}}1
{é}{{\'e}}1
{è}{{\`e}}1
{ê}{{\^e}}1
{í}{{\'i}}1
{ó}{{\'o}}1
{õ}{{\~o}}1
{ú}{{\'u}}1
{ü}{{\"u}}1
{ç}{{\c{c}}}1
{~}{{ }}1
}


\definecolor{codegreen}{rgb}{0,0.6,0}
\definecolor{codegray}{rgb}{0.5,0.5,0.5}
\definecolor{codepurple}{rgb}{0.58,0,0.82}
\definecolor{backcolour}{rgb}{0.95,0.95,0.92}

\lstdefinestyle{mystyle}{
    backgroundcolor=\color{backcolour},   
    commentstyle=\color{codegreen},
    keywordstyle=\color{magenta},
    numberstyle=\tiny\color{codegray},
    stringstyle=\color{codepurple},
    basicstyle=\ttfamily\footnotesize,
    breakatwhitespace=false,         
    breaklines=true,                 
    captionpos=b,                    
    keepspaces=true,                 
    numbers=left,                    
xleftmargin=2em,
framexleftmargin=2em,            
    showspaces=false,                
    showstringspaces=false,
    showtabs=false,                  
    tabsize=2,
    upquote=true
}

\lstset{style=mystyle}


\lstset{style=mystyle}
\newcommand{\imgdir}{C:/laragon/www/newmc/assets/imgsvg/}
\newcommand{\imgsvgdir}{C:/laragon/www/newmc/assets/imgsvg/}

\definecolor{mcgris}{RGB}{220, 220, 220}% ancien~; pour compatibilité
\definecolor{mcbleu}{RGB}{52, 152, 219}
\definecolor{mcvert}{RGB}{125, 194, 70}
\definecolor{mcmauve}{RGB}{154, 0, 215}
\definecolor{mcorange}{RGB}{255, 96, 0}
\definecolor{mcturquoise}{RGB}{0, 153, 153}
\definecolor{mcrouge}{RGB}{255, 0, 0}
\definecolor{mclightvert}{RGB}{205, 234, 190}

\definecolor{gris}{RGB}{220, 220, 220}
\definecolor{bleu}{RGB}{52, 152, 219}
\definecolor{vert}{RGB}{125, 194, 70}
\definecolor{mauve}{RGB}{154, 0, 215}
\definecolor{orange}{RGB}{255, 96, 0}
\definecolor{turquoise}{RGB}{0, 153, 153}
\definecolor{rouge}{RGB}{255, 0, 0}
\definecolor{lightvert}{RGB}{205, 234, 190}
\setitemize[0]{label=\color{lightvert}  $\bullet$}

\pagestyle{fancy}
\renewcommand{\headrulewidth}{0.2pt}
\fancyhead[L]{maths-cours.fr}
\fancyhead[R]{\thepage}
\renewcommand{\footrulewidth}{0.2pt}
\fancyfoot[C]{}

\newcolumntype{C}{>{\centering\arraybackslash}X}
\newcolumntype{s}{>{\hsize=.35\hsize\arraybackslash}X}

\setlength{\parindent}{0pt}		 
\setlength{\parskip}{3mm}
\setlength{\headheight}{1cm}

\def\ebook{ebook}
\def\book{book}
\def\web{web}
\def\type{web}

\newcommand{\vect}[1]{\overrightarrow{\,\mathstrut#1\,}}

\def\Oij{$\left(\text{O}~;~\vect{\imath},~\vect{\jmath}\right)$}
\def\Oijk{$\left(\text{O}~;~\vect{\imath},~\vect{\jmath},~\vect{k}\right)$}
\def\Ouv{$\left(\text{O}~;~\vect{u},~\vect{v}\right)$}

\hypersetup{breaklinks=true, colorlinks = true, linkcolor = OliveGreen, urlcolor = OliveGreen, citecolor = OliveGreen, pdfauthor={Didier BONNEL - https://www.maths-cours.fr} } % supprime les bordures autour des liens

\renewcommand{\arg}[0]{\text{arg}}

\everymath{\displaystyle}

%================================================================================================================================
%
% Macros - Commandes
%
%================================================================================================================================

\newcommand\meta[2]{    			% Utilisé pour créer le post HTML.
	\def\titre{titre}
	\def\url{url}
	\def\arg{#1}
	\ifx\titre\arg
		\newcommand\maintitle{#2}
		\fancyhead[L]{#2}
		{\Large\sffamily \MakeUppercase{#2}}
		\vspace{1mm}\textcolor{mcvert}{\hrule}
	\fi 
	\ifx\url\arg
		\fancyfoot[L]{\href{https://www.maths-cours.fr#2}{\black \footnotesize{https://www.maths-cours.fr#2}}}
	\fi 
}


\newcommand\TitreC[1]{    		% Titre centré
     \needspace{3\baselineskip}
     \begin{center}\textbf{#1}\end{center}
}

\newcommand\newpar{    		% paragraphe
     \par
}

\newcommand\nosp {    		% commande vide (pas d'espace)
}
\newcommand{\id}[1]{} %ignore

\newcommand\boite[2]{				% Boite simple sans titre
	\vspace{5mm}
	\setlength{\fboxrule}{0.2mm}
	\setlength{\fboxsep}{5mm}	
	\fcolorbox{#1}{#1!3}{\makebox[\linewidth-2\fboxrule-2\fboxsep]{
  		\begin{minipage}[t]{\linewidth-2\fboxrule-4\fboxsep}\setlength{\parskip}{3mm}
  			 #2
  		\end{minipage}
	}}
	\vspace{5mm}
}

\newcommand\CBox[4]{				% Boites
	\vspace{5mm}
	\setlength{\fboxrule}{0.2mm}
	\setlength{\fboxsep}{5mm}
	
	\fcolorbox{#1}{#1!3}{\makebox[\linewidth-2\fboxrule-2\fboxsep]{
		\begin{minipage}[t]{1cm}\setlength{\parskip}{3mm}
	  		\textcolor{#1}{\LARGE{#2}}    
 	 	\end{minipage}  
  		\begin{minipage}[t]{\linewidth-2\fboxrule-4\fboxsep}\setlength{\parskip}{3mm}
			\raisebox{1.2mm}{\normalsize\sffamily{\textcolor{#1}{#3}}}						
  			 #4
  		\end{minipage}
	}}
	\vspace{5mm}
}

\newcommand\cadre[3]{				% Boites convertible html
	\par
	\vspace{2mm}
	\setlength{\fboxrule}{0.1mm}
	\setlength{\fboxsep}{5mm}
	\fcolorbox{#1}{white}{\makebox[\linewidth-2\fboxrule-2\fboxsep]{
  		\begin{minipage}[t]{\linewidth-2\fboxrule-4\fboxsep}\setlength{\parskip}{3mm}
			\raisebox{-2.5mm}{\sffamily \small{\textcolor{#1}{\MakeUppercase{#2}}}}		
			\par		
  			 #3
 	 		\end{minipage}
	}}
		\vspace{2mm}
	\par
}

\newcommand\bloc[3]{				% Boites convertible html sans bordure
     \needspace{2\baselineskip}
     {\sffamily \small{\textcolor{#1}{\MakeUppercase{#2}}}}    
		\par		
  			 #3
		\par
}

\newcommand\CHelp[1]{
     \CBox{Plum}{\faInfoCircle}{À RETENIR}{#1}
}

\newcommand\CUp[1]{
     \CBox{NavyBlue}{\faThumbsOUp}{EN PRATIQUE}{#1}
}

\newcommand\CInfo[1]{
     \CBox{Sepia}{\faArrowCircleRight}{REMARQUE}{#1}
}

\newcommand\CRedac[1]{
     \CBox{PineGreen}{\faEdit}{BIEN R\'EDIGER}{#1}
}

\newcommand\CError[1]{
     \CBox{Red}{\faExclamationTriangle}{ATTENTION}{#1}
}

\newcommand\TitreExo[2]{
\needspace{4\baselineskip}
 {\sffamily\large EXERCICE #1\ (\emph{#2 points})}
\vspace{5mm}
}

\newcommand\img[2]{
          \includegraphics[width=#2\paperwidth]{\imgdir#1}
}

\newcommand\imgsvg[2]{
       \begin{center}   \includegraphics[width=#2\paperwidth]{\imgsvgdir#1} \end{center}
}


\newcommand\Lien[2]{
     \href{#1}{#2 \tiny \faExternalLink}
}
\newcommand\mcLien[2]{
     \href{https~://www.maths-cours.fr/#1}{#2 \tiny \faExternalLink}
}

\newcommand{\euro}{\eurologo{}}

%================================================================================================================================
%
% Macros - Environement
%
%================================================================================================================================

\newenvironment{tex}{ %
}
{%
}

\newenvironment{indente}{ %
	\setlength\parindent{10mm}
}

{
	\setlength\parindent{0mm}
}

\newenvironment{corrige}{%
     \needspace{3\baselineskip}
     \medskip
     \textbf{\textsc{Corrigé}}
     \medskip
}
{
}

\newenvironment{extern}{%
     \begin{center}
     }
     {
     \end{center}
}

\NewEnviron{code}{%
	\par
     \boite{gray}{\texttt{%
     \BODY
     }}
     \par
}

\newenvironment{vbloc}{% boite sans cadre empeche saut de page
     \begin{minipage}[t]{\linewidth}
     }
     {
     \end{minipage}
}
\NewEnviron{h2}{%
    \needspace{3\baselineskip}
    \vspace{0.6cm}
	\noindent \MakeUppercase{\sffamily \large \BODY}
	\vspace{1mm}\textcolor{mcgris}{\hrule}\vspace{0.4cm}
	\par
}{}

\NewEnviron{h3}{%
    \needspace{3\baselineskip}
	\vspace{5mm}
	\textsc{\BODY}
	\par
}

\NewEnviron{margeneg}{ %
\begin{addmargin}[-1cm]{0cm}
\BODY
\end{addmargin}
}

\NewEnviron{html}{%
}

\begin{document}
\meta{url}{/cours/division-euclidienne-pgcd/}
\meta{pid}{1552}
\meta{titre}{Division euclidienne - Nombres premiers - PGCD}
\meta{type}{cours}
\begin{h2}1 - Division euclidienne\end{h2}
\cadre{bleu}{Définition}{% id="d10"
     Soient $a$ et $b$, deux nombres entiers naturels (c'est à dire positifs) avec $b\neq 0$.
     \par
     Effectuer la \textbf{division euclidienne} de $a$ par $b$, c'est trouver deux entiers naturels $q$ et $r$ tels que~:
     \begin{center}$a = b\times q+r $ et $ r < b$\end{center}
     $q$ s'appelle le \textbf{quotient} et $r$ le \textbf{reste}.
}
\bloc{orange}{Exemple}{% id="e10"
     \begin{center}
          \img{division}{0.2}%width="180" alt="division euclidienne"
     \end{center}
     Écriture en ligne~:
     \par
     $6894 = 23\times 299 + 17$
     \par
     $299$ est le \textbf{quotient} et $17$ le \textbf{reste}.
}
\bloc{cyan}{Remarque}{% id="r10"
     Sur la plupart des calculatrices de collège la touche qui permet d'effectuer la division euclidienne est notée~: \img{touche-divise}{0.008}%width="8" alt="touche division euclidienne"
     .
     \par
     Par exemple, la suite de touches à entrer pour obtenir la division euclidienne de $6894$ par $23$ sur une TI-Collège est~:
     \begin{center}
          \img{touches}{0.25}%width="300" alt="suite de touches division euclidienne"
     \end{center}
     et voici le résultat obtenu à l'écran~:
     \begin{center}
          \img{result}{0.15}%width="160" alt="resultat division euclidienne"
     \end{center}
}
\cadre{bleu}{Définition}{% id="d20"
     On dit que $a$ est \textbf{divisible} par $b$ si le reste de la division euclidienne de $a$ par $b$ est nul.
     \par
     Cela revient à dire qu'il existe un entier naturel $q$ tel que $a = b\times q$.
     \par
     Les expressions suivantes sont synonymes~:
     \begin{itemize}
          \item $a$ est divisible par $b$
          \item $a$ est un multiple de $b$
          \item $b$ est un diviseur de $a$
          \item $b$ divise $a$ (que l'on écrit parfois $b | a$)
     \end{itemize}
}
\bloc{orange}{Exemple}{% id="e20"
     La division euclidienne de $630$ par $15$ donne un quotient de $42$ et un reste nul.
     \par
     On a donc $630 = 15\times 42$.
     \par
     On peut dire que~:
     \begin{itemize}
          \item $630$ est divisible par $15$
          \item $630$ est un multiple de $15$
          \item $15$ est un diviseur de $630$
          \item $15$ divise $630$
     \end{itemize}
     (On peut aussi dire que $630$ est divisible par $42$, etc.)
}
\cadre{rouge}{Critères de divisibilité}{% id="t30"
     \begin{itemize}
          \item Un entier naturel est divisible par 2 si son \textbf{chiffre des unités} est 0, 2, 4, 6 ou 8.
          \item Un entier naturel est divisible par 3 si la \textbf{somme de ses chiffres} est divisible par 3.
          \item Un entier naturel est divisible par 4 si le nombre formé par ses \textbf{deux derniers chiffres} est divisible par 4.
          \item Un entier naturel est divisible par 5 si son \textbf{chiffre des unités} est 0 ou 5.
          \item Un entier naturel est divisible par 9 si la \textbf{somme de ses chiffres} est divisible par 9.
          \item Un entier naturel est divisible par 10 si son\textbf{ chiffre des unités} est 0.
     \end{itemize}
}
\bloc{cyan}{Remarques}{% id="r30"
     \begin{itemize}
          \item \textbf{Attention~: }Pour les critères de divisibilité par 3 et par 9, il faut effectuer \textbf{la somme des chiffres} (et non regarder le chiffre des unités)
          \item Il n'existe pas de critère de divisibilité par 7 qui soit très simple. Le plus rapide est en général d'effectuer la division~!
     \end{itemize}
}
\bloc{orange}{Exemple}{% id="e30"
     \begin{itemize}
          \item $1314$ est divisible par $2$ (chiffre des unités~: 4)
          \item $1314$ est divisible par $3$ (somme des chiffres~: 9)
          \item $1314$ n'est pas divisible par $4$ (deux derniers chiffres~: 14)
          \item $1314$ n'est pas divisible par $5$ (chiffre des unités~: 4)
          \item $1314$ est divisible par $9$ (somme des chiffres~: 9)
          \item $1314$ n'est pas divisible par $10$ (chiffre des unités~: 4)
     \end{itemize}
}
\begin{h2}2 - Nombres premiers\end{h2}
\cadre{bleu}{Définition}{ % id="d45"
     On dit qu'un nombre entier naturel est \textbf{premier} s'il possède exactement deux diviseurs~: 1 et lui-même.
} % fin définition
\bloc{orange}{Exemples}{ % id=e47
     \begin{itemize}
          \item
          2; 3; 5 sont des nombres premiers~;
          \item
          0 \textbf{n'est pas} un nombre premier car il est divisible par tous les entiers supérieurs ou égal à 1.
          \item
          1 \textbf{n'est pas} un nombre premier car il n'admet qu' \textbf{un seul} diviseur (lui-même).
          \item
          À l'exception du nombre 2, tous les entiers pairs \textbf{ne sont pas} des nombres premiers (car ils sont divisibles par 2). Cela signifie qu'à l'exception du nombre 2, tous les nombres premiers sont impairs. Par contre, la réciproque est fausse~: tous les nombres impairs ne sont pas premiers~; par exemple 1 (voir ci-dessus) et 15 (divisible par 1; 3; 5 et 15) ne sont pas premiers.
     \end{itemize}
} % fin exemple
\bloc{cyan}{Remarque}{ % id=r47
     Il est utile de connaître par cœur la liste des nombres premiers inférieurs à 20 (ou plus ...):
     \begin{center}
          \textbf{2~; 3~; 5~; 7~; 11~; 13~; 17~; 19 }
     \end{center}
} % fin remarque
\cadre{rouge}{Théorème}{ % id=t48
     \textbf{Décomposition en produit de facteurs premiers}
     \par
     Tout nombre entier supérieur ou égal à 2 peut s'écrire sous la forme d'un produit de nombres premiers. Cette décomposition est \textbf{unique} (à l'ordre des facteurs près).
} % fin théorème
\bloc{cyan}{Remarque}{ % id=r48
     Ce résultat très important est également appelé \textbf{ \og Théorème fondamental de l'arithmétique \fg{} }
} % fin remarque
\bloc{orange}{Exemple}{ % id=e48
     \begin{itemize}
          \item
          $10 = 2 \times 5$
          \item
          $84 = 2 \times 2 \times 3 \times 7 = 2^2 \times 3 \times 7$
          \item
          $23 = 23$ (un seul facteur car 23 est premier~!)
     \end{itemize}
} % fin exemple
\cadre{vert}{Méthode}{ % id=p48
     Pour décomposer un nombre $ N $ en produit de facteurs premiers, on peut essayer de le diviser successivement par chaque nombre premier inférieur ou égal à $ \sqrt{ n } $ . Le méthode détaillée est décrite sur la fiche~: Décomposition en produit de facteurs premiers.
} % fin propriété@
\begin{h2}3 - PGCD\end{h2}
\cadre{bleu}{Définition}{% id="d50"
     Le \textbf{PGCD} de deux entiers naturels non nuls $a$ et $b$ est le plus grand diviseur commun à $a$ et à $b$, c'est à dire le plus grand entier naturel qui divise à la fois $a$ et $b$.
}
\bloc{orange}{Exemple}{% id="e50"
     Soit à déterminer le PGCD de $600$ et $315$.
     \par
     Les diviseurs de $600$ sont~:
     \par
     $1; 2; 3; 4; 5; 6; 8; 10; 12; 15; 20; 24; 25; 30; 40; 50; 60; 75; 100; 120; 150; 200; 300; 600$
     \par
     Les diviseurs de $315$ sont~:
     \par
     $1; 3; 5; 7; 9; 15; 21; 35; 45; 63; 105; 315$
     \par
     Le plus grand diviseur commun est donc $15$ (le plus grand nombre figurant à la fois dans les deux listes).
     \par
     $PGCD\left(600~; 315\right)=15$.
     \par
     Il existe plusieurs méthodes permettant de trouver le PGCD de deux nombres de façon plus rapide, sans avoir besoin de faire la liste de tous les diviseurs.
     \medskip
     En classe de Troisième, il faut connaître la méthode utilisant la décomposition en facteurs premiers (voir ci-dessous). D'autres méthodes sont proposées en compléments~: \mcLien{https://www.maths-cours.fr/supplement/deux-methodes-de-calcul-du-pgcd/}{Calcul du PGCD par soustractions successives et algorithme d'Euclide}.
     \par
     Par ailleurs, de nombreuses calculatrices (de niveau collège ou lycée) possède une touche permettant de calculer le PGCD de deux entiers naturels.
}
\bloc{orange}{Exemples}{ % id=e59
     \textbf{Calcul du PGCD à l'aide de décomposition en produit de facteurs premiers}
     \par
     \begin{itemize}
          \item
          Exemple 1~: Calcul du PGCD de 45 et de 150~:
          \par
          Les décompositions en facteurs premiers de 45 et de 150 sont~:
          \par
          $45 = \color{red}{3 }\color{black} \times 3 \times \color{red}{5} \color{black}= 3^2 \times 5$
          \par
          $ 150 = 2 \times \color{red}{3}\color{black} \times \color{red}{5}\color{black} \times 5 = 2 \times 3 \times 5^2 $
          \par
          $3$ et $5$ sont les facteurs premiers figurant dans les deux décompositions donc le PGCD de $45$ et de $150$ est $ 3 \times 5 = 15. $
          \item
          Exemple 2~: Calcul du PGCD de 108 et de 144~:
          \par
          Les décompositions en produit de facteurs premiers de 108 et de 144 sont~:
          \par
          $108 = \color{red}{2 \times 2}\color{black} \times \color{red}{ 3 \times 3}\color{black} \times 3 = 2^2 \times 3^3$
          \par
          $ 144 = \color{red}{2 \times 2}\color{black} \times 2 \times 2 \times \color{red}{3 \times 3}\color{black} = 2^4 \times 3^2 $
          \par
          Le facteur $2$ est présent (au moins) deux fois dans chacune des décompositions ainsi que le facteur $ 3 $~; donc le PGCD de $108$ et de $ 144 $ est $ 2 \times 2 \times 3 \times 3 = 36. $
     \end{itemize}
} % fin exemple
\cadre{bleu}{Définition}{% id="d90"
     Une fraction est \textbf{irréductible} si son numérateur et son dénominateur n'ont aucun diviseur commun mis à part $1$, c'est à dire si le PGCD du numérateur et du dénominateur est égal à 1.
}
\bloc{orange}{Exemples}{% id="e90"
     \begin{itemize}
          \item $\frac{5}{6}$ est une fraction irréductible car $PGCD\left(5~; 6\right)=1$.
          \item $\frac{121}{99}$ n'est pas une fraction irréductible car $PGCD\left(121~; 99\right)=11$.\\ La fraction se simplifie donc par $11$~:
          \par
          $\frac{121}{99}=\frac{11\times 11}{9\times 11}=\frac{11}{9}$
     \end{itemize}
}

\end{document}
µ
\documentclass[a4paper]{article}

%================================================================================================================================
%
% Packages
%
%================================================================================================================================

\usepackage[T1]{fontenc} 	% pour caractères accentués
\usepackage[utf8]{inputenc}  % encodage utf8
\usepackage[french]{babel}	% langue : français
\usepackage{fourier}			% caractères plus lisibles
\usepackage[dvipsnames]{xcolor} % couleurs
\usepackage{fancyhdr}		% réglage header footer
\usepackage{needspace}		% empêcher sauts de page mal placés
\usepackage{graphicx}		% pour inclure des graphiques
\usepackage{enumitem,cprotect}		% personnalise les listes d'items (nécessaire pour ol, al ...)
\usepackage{hyperref}		% Liens hypertexte
\usepackage{pstricks,pst-all,pst-node,pstricks-add,pst-math,pst-plot,pst-tree,pst-eucl} % pstricks
\usepackage[a4paper,includeheadfoot,top=2cm,left=3cm, bottom=2cm,right=3cm]{geometry} % marges etc.
\usepackage{comment}			% commentaires multilignes
\usepackage{amsmath,environ} % maths (matrices, etc.)
\usepackage{amssymb,makeidx}
\usepackage{bm}				% bold maths
\usepackage{tabularx}		% tableaux
\usepackage{colortbl}		% tableaux en couleur
\usepackage{fontawesome}		% Fontawesome
\usepackage{environ}			% environment with command
\usepackage{fp}				% calculs pour ps-tricks
\usepackage{multido}			% pour ps tricks
\usepackage[np]{numprint}	% formattage nombre
\usepackage{tikz,tkz-tab} 			% package principal TikZ
\usepackage{pgfplots}   % axes
\usepackage{mathrsfs}    % cursives
\usepackage{calc}			% calcul taille boites
\usepackage[scaled=0.875]{helvet} % font sans serif
\usepackage{svg} % svg
\usepackage{scrextend} % local margin
\usepackage{scratch} %scratch
\usepackage{multicol} % colonnes
%\usepackage{infix-RPN,pst-func} % formule en notation polanaise inversée
\usepackage{listings}

%================================================================================================================================
%
% Réglages de base
%
%================================================================================================================================

\lstset{
language=Python,   % R code
literate=
{á}{{\'a}}1
{à}{{\`a}}1
{ã}{{\~a}}1
{é}{{\'e}}1
{è}{{\`e}}1
{ê}{{\^e}}1
{í}{{\'i}}1
{ó}{{\'o}}1
{õ}{{\~o}}1
{ú}{{\'u}}1
{ü}{{\"u}}1
{ç}{{\c{c}}}1
{~}{{ }}1
}


\definecolor{codegreen}{rgb}{0,0.6,0}
\definecolor{codegray}{rgb}{0.5,0.5,0.5}
\definecolor{codepurple}{rgb}{0.58,0,0.82}
\definecolor{backcolour}{rgb}{0.95,0.95,0.92}

\lstdefinestyle{mystyle}{
    backgroundcolor=\color{backcolour},   
    commentstyle=\color{codegreen},
    keywordstyle=\color{magenta},
    numberstyle=\tiny\color{codegray},
    stringstyle=\color{codepurple},
    basicstyle=\ttfamily\footnotesize,
    breakatwhitespace=false,         
    breaklines=true,                 
    captionpos=b,                    
    keepspaces=true,                 
    numbers=left,                    
xleftmargin=2em,
framexleftmargin=2em,            
    showspaces=false,                
    showstringspaces=false,
    showtabs=false,                  
    tabsize=2,
    upquote=true
}

\lstset{style=mystyle}


\lstset{style=mystyle}
\newcommand{\imgdir}{C:/laragon/www/newmc/assets/imgsvg/}
\newcommand{\imgsvgdir}{C:/laragon/www/newmc/assets/imgsvg/}

\definecolor{mcgris}{RGB}{220, 220, 220}% ancien~; pour compatibilité
\definecolor{mcbleu}{RGB}{52, 152, 219}
\definecolor{mcvert}{RGB}{125, 194, 70}
\definecolor{mcmauve}{RGB}{154, 0, 215}
\definecolor{mcorange}{RGB}{255, 96, 0}
\definecolor{mcturquoise}{RGB}{0, 153, 153}
\definecolor{mcrouge}{RGB}{255, 0, 0}
\definecolor{mclightvert}{RGB}{205, 234, 190}

\definecolor{gris}{RGB}{220, 220, 220}
\definecolor{bleu}{RGB}{52, 152, 219}
\definecolor{vert}{RGB}{125, 194, 70}
\definecolor{mauve}{RGB}{154, 0, 215}
\definecolor{orange}{RGB}{255, 96, 0}
\definecolor{turquoise}{RGB}{0, 153, 153}
\definecolor{rouge}{RGB}{255, 0, 0}
\definecolor{lightvert}{RGB}{205, 234, 190}
\setitemize[0]{label=\color{lightvert}  $\bullet$}

\pagestyle{fancy}
\renewcommand{\headrulewidth}{0.2pt}
\fancyhead[L]{maths-cours.fr}
\fancyhead[R]{\thepage}
\renewcommand{\footrulewidth}{0.2pt}
\fancyfoot[C]{}

\newcolumntype{C}{>{\centering\arraybackslash}X}
\newcolumntype{s}{>{\hsize=.35\hsize\arraybackslash}X}

\setlength{\parindent}{0pt}		 
\setlength{\parskip}{3mm}
\setlength{\headheight}{1cm}

\def\ebook{ebook}
\def\book{book}
\def\web{web}
\def\type{web}

\newcommand{\vect}[1]{\overrightarrow{\,\mathstrut#1\,}}

\def\Oij{$\left(\text{O}~;~\vect{\imath},~\vect{\jmath}\right)$}
\def\Oijk{$\left(\text{O}~;~\vect{\imath},~\vect{\jmath},~\vect{k}\right)$}
\def\Ouv{$\left(\text{O}~;~\vect{u},~\vect{v}\right)$}

\hypersetup{breaklinks=true, colorlinks = true, linkcolor = OliveGreen, urlcolor = OliveGreen, citecolor = OliveGreen, pdfauthor={Didier BONNEL - https://www.maths-cours.fr} } % supprime les bordures autour des liens

\renewcommand{\arg}[0]{\text{arg}}

\everymath{\displaystyle}

%================================================================================================================================
%
% Macros - Commandes
%
%================================================================================================================================

\newcommand\meta[2]{    			% Utilisé pour créer le post HTML.
	\def\titre{titre}
	\def\url{url}
	\def\arg{#1}
	\ifx\titre\arg
		\newcommand\maintitle{#2}
		\fancyhead[L]{#2}
		{\Large\sffamily \MakeUppercase{#2}}
		\vspace{1mm}\textcolor{mcvert}{\hrule}
	\fi 
	\ifx\url\arg
		\fancyfoot[L]{\href{https://www.maths-cours.fr#2}{\black \footnotesize{https://www.maths-cours.fr#2}}}
	\fi 
}


\newcommand\TitreC[1]{    		% Titre centré
     \needspace{3\baselineskip}
     \begin{center}\textbf{#1}\end{center}
}

\newcommand\newpar{    		% paragraphe
     \par
}

\newcommand\nosp {    		% commande vide (pas d'espace)
}
\newcommand{\id}[1]{} %ignore

\newcommand\boite[2]{				% Boite simple sans titre
	\vspace{5mm}
	\setlength{\fboxrule}{0.2mm}
	\setlength{\fboxsep}{5mm}	
	\fcolorbox{#1}{#1!3}{\makebox[\linewidth-2\fboxrule-2\fboxsep]{
  		\begin{minipage}[t]{\linewidth-2\fboxrule-4\fboxsep}\setlength{\parskip}{3mm}
  			 #2
  		\end{minipage}
	}}
	\vspace{5mm}
}

\newcommand\CBox[4]{				% Boites
	\vspace{5mm}
	\setlength{\fboxrule}{0.2mm}
	\setlength{\fboxsep}{5mm}
	
	\fcolorbox{#1}{#1!3}{\makebox[\linewidth-2\fboxrule-2\fboxsep]{
		\begin{minipage}[t]{1cm}\setlength{\parskip}{3mm}
	  		\textcolor{#1}{\LARGE{#2}}    
 	 	\end{minipage}  
  		\begin{minipage}[t]{\linewidth-2\fboxrule-4\fboxsep}\setlength{\parskip}{3mm}
			\raisebox{1.2mm}{\normalsize\sffamily{\textcolor{#1}{#3}}}						
  			 #4
  		\end{minipage}
	}}
	\vspace{5mm}
}

\newcommand\cadre[3]{				% Boites convertible html
	\par
	\vspace{2mm}
	\setlength{\fboxrule}{0.1mm}
	\setlength{\fboxsep}{5mm}
	\fcolorbox{#1}{white}{\makebox[\linewidth-2\fboxrule-2\fboxsep]{
  		\begin{minipage}[t]{\linewidth-2\fboxrule-4\fboxsep}\setlength{\parskip}{3mm}
			\raisebox{-2.5mm}{\sffamily \small{\textcolor{#1}{\MakeUppercase{#2}}}}		
			\par		
  			 #3
 	 		\end{minipage}
	}}
		\vspace{2mm}
	\par
}

\newcommand\bloc[3]{				% Boites convertible html sans bordure
     \needspace{2\baselineskip}
     {\sffamily \small{\textcolor{#1}{\MakeUppercase{#2}}}}    
		\par		
  			 #3
		\par
}

\newcommand\CHelp[1]{
     \CBox{Plum}{\faInfoCircle}{À RETENIR}{#1}
}

\newcommand\CUp[1]{
     \CBox{NavyBlue}{\faThumbsOUp}{EN PRATIQUE}{#1}
}

\newcommand\CInfo[1]{
     \CBox{Sepia}{\faArrowCircleRight}{REMARQUE}{#1}
}

\newcommand\CRedac[1]{
     \CBox{PineGreen}{\faEdit}{BIEN R\'EDIGER}{#1}
}

\newcommand\CError[1]{
     \CBox{Red}{\faExclamationTriangle}{ATTENTION}{#1}
}

\newcommand\TitreExo[2]{
\needspace{4\baselineskip}
 {\sffamily\large EXERCICE #1\ (\emph{#2 points})}
\vspace{5mm}
}

\newcommand\img[2]{
          \includegraphics[width=#2\paperwidth]{\imgdir#1}
}

\newcommand\imgsvg[2]{
       \begin{center}   \includegraphics[width=#2\paperwidth]{\imgsvgdir#1} \end{center}
}


\newcommand\Lien[2]{
     \href{#1}{#2 \tiny \faExternalLink}
}
\newcommand\mcLien[2]{
     \href{https~://www.maths-cours.fr/#1}{#2 \tiny \faExternalLink}
}

\newcommand{\euro}{\eurologo{}}

%================================================================================================================================
%
% Macros - Environement
%
%================================================================================================================================

\newenvironment{tex}{ %
}
{%
}

\newenvironment{indente}{ %
	\setlength\parindent{10mm}
}

{
	\setlength\parindent{0mm}
}

\newenvironment{corrige}{%
     \needspace{3\baselineskip}
     \medskip
     \textbf{\textsc{Corrigé}}
     \medskip
}
{
}

\newenvironment{extern}{%
     \begin{center}
     }
     {
     \end{center}
}

\NewEnviron{code}{%
	\par
     \boite{gray}{\texttt{%
     \BODY
     }}
     \par
}

\newenvironment{vbloc}{% boite sans cadre empeche saut de page
     \begin{minipage}[t]{\linewidth}
     }
     {
     \end{minipage}
}
\NewEnviron{h2}{%
    \needspace{3\baselineskip}
    \vspace{0.6cm}
	\noindent \MakeUppercase{\sffamily \large \BODY}
	\vspace{1mm}\textcolor{mcgris}{\hrule}\vspace{0.4cm}
	\par
}{}

\NewEnviron{h3}{%
    \needspace{3\baselineskip}
	\vspace{5mm}
	\textsc{\BODY}
	\par
}

\NewEnviron{margeneg}{ %
\begin{addmargin}[-1cm]{0cm}
\BODY
\end{addmargin}
}

\NewEnviron{html}{%
}

\begin{document}
\meta{url}{/cours/regles-de-calculs-fractions-puissances/}
\meta{pid}{1559}
\meta{titre}{Les règles de calculs, fractions, puissances}
\meta{type}{cours}
\begin{h2}1 - Vocabulaire\end{h2}
\cadre{bleu}{Définitions}{% id="d010"
     \begin{itemize}
          \item La \textbf{somme} de deux \textbf{termes} est le résultat de l'\textbf{addition} de ces nombres.
          \item La \textbf{différence} de deux \textbf{termes} est le résultat de la \textbf{soustraction} de ces nombres.
          \item Le \textbf{produit} de deux \textbf{facteurs} est le résultat de la \textbf{multiplication} de ces nombres.
     \end{itemize}
}

\bloc{orange}{Exemples}{% id="e020"
     \begin{itemize}
          \item $5 = 3+2$ : $\quad 5$ est la \textbf{somme} des termes $3$ et $2$.
          \item $1 = 3-2$ : $\quad1$ est la \textbf{différence} des termes $3$ et $2$.
          \item $6 = 3\times 2$ : $\quad6$ est le \textbf{produit} des facteurs $3$ et $2$.
     \end{itemize}
}
\bloc{cyan}{Remarques}{% id="r040"
     On regroupe souvent \textbf{somme} et \textbf{différence} sous le même terme : \textbf{somme algébrique}. En effet, une soustraction d'un nombre positif correspond à une addition d'un nombre négatif.
     \par
     Lorsqu'une expression contient plusieurs opérations, il s'agit :
     \begin{itemize}
          \item d'une somme algébrique si la \textbf{dernière} opération effectuée (la \textbf{moins} prioritaire) est une addition ou une soustraction. Par exemple : $2x-3y$;
          \item d'un produit si la \textbf{dernière} opération effectuée (la \textbf{moins} prioritaire) est une multiplication.  Par exemple : $3x\left(y-3\right)$.
     \end{itemize}
}
\begin{h2}2 - Priorités de calculs\end{h2}
\cadre{vert}{Propriétés}{% id="p040"
     \begin{itemize}
          \item On effectue d'abord les calculs des expressions entre \textbf{parenthèses}, en commençant par les parenthèses les plus intérieures.
          \item Puis on effectue les \textbf{puissances} avant les multiplications, les divisions, les additions et les soustractions.
          \item Puis on effectue d'abord les \textbf{multiplications} et les \textbf{divisions} avant les additions et les soustractions.
          \item Enfin, on effectue les calculs de la gauche vers la droite.
          \par
          Dans une somme algébrique, on peut également regrouper ensemble les termes de même signe.
     \end{itemize}
}
\bloc{orange}{Exemples}{% id="e040"
     \begin{itemize}
          \item $A=5-3\times 7+2\times \left(4-1\right)$
          \par
          On effectue d'abord les parenthèses :
          \par
          $A=5-3\times 7+2\times 3$
          \par
          Puis les multiplications :
          \par
          $A=5-21+6$
          \par
          Puis les opérations restantes (en regroupant les termes positifs par exemple) :
          \par
          $A=5-21+6=11-21=-10$
          \item \textbf{Attention} à bien tenir compte de la priorité des opération même si l'expression contient des \textit{lettres}. Par exemple :
          \par
          $B=5+\left(7-4\right)\times x$
          \par
          On peut effectuer le calcul dans la parenthèse :
          \par
          $B=5+3\times x=5+3x$
          \par
          On ne peut pas effectuer l'addition $5+3$ car la multiplication $3\times x$ est prioritaire. On ne peut donc pas aller plus loin.
     \end{itemize}
}
\begin{h2}3 - Fractions\end{h2}
\cadre{vert}{Propriétés}{%  id="p060"
     \begin{itemize}
          \item Pour additionner (ou soustraire) des fractions, on ajoute (ou on soustrait) leurs numérateurs, après les avoir mises au \textbf{même dénominateur}.
          \item Pour multiplier des fractions, on multiplie les numérateurs entre eux et les dénominateurs entre eux, \textbf{en simplifiant} au maximum.
          \item Pour diviser par une fraction, on multiplie par son inverse.
     \end{itemize}
}
\bloc{orange}{Exemples}{% id="e060"
     \begin{itemize}
          \item %
          $A=\frac{3}{4}-\frac{2}{5}$
          \par
          $A=\frac{3\times 5}{4\times 5}-\frac{2\times 4}{5\times 4}$
          \par
          $A=\frac{15}{20}-\frac{8}{20}$
          \par
          $A=\frac{7}{20}$
          \item %
          $B=\frac{3}{4}\times \frac{2}{5}$
          \par
          $B=\frac{3\times 2}{4 \times 5}$
          \par
          $B=\frac{3\times 2}{2\times 2\times 5}$
          \par
          $B=\frac{3}{10}$
          \item %
          $C=\frac{3}{4} \div \frac{2}{5}$
          \par
          $C=\frac{3}{4}\times \frac{5}{2}$
          \par
          $C=\frac{3\times 5}{4\times 2}$
          \par
          $C=\frac{15}{8}$
     \end{itemize}
}
\begin{h2}4 - Puissances\end{h2}
\cadre{vert}{Propriétés}{% id="p080"
     \begin{itemize}
          \item Produit : $a^{n}\times a^{m}=a^{n+m}$
          \item Inverse : $\frac{1}{a^{m}}=a^{-m}$
          \item Quotient :$\frac{a^{n}}{a^{m}}=a^{n-m}$
          \item Puissance de puissance :$\left(a^{n}\right)^{m}=a^{n\times m}$
          \item Exposants identiques :$a^{n}\times b^{n}=\left(ab\right)^{n}$
     \end{itemize}
}
\bloc{orange}{Exemples}{% id="e080"
     \begin{itemize}
          \item $A=3^{2}\times 3^{3}=3^{2+3}=3^{5}$
          \item $B=\frac{2^{3}}{2^{-4}}=2^{3-\left(-4\right)}=2^{7}$
          \item $C=\left(10^{2}\right)^{-3}=10^{-6}$
     \end{itemize}
}
\bloc{cyan}{Remarques}{%  id="r090"
     \begin{itemize}
          \item Ces formules peuvent, bien sûr, être utilisées \textit{dans les deux sens}. Par exemple, pour passer de $\frac{1}{a^{m}}$ à $a^{-m}$ ou pour passer de $a^{-m}$ à $\frac{1}{a^{m}}$
          \item Cas particulier de la dernière formule :
          \par
          $\left(-a\right)^{n}=\left(-1\times a\right)^{n}=\left(-1\right)^{n}\times a^{n}$
          \par
          Donc pour $n$ \textbf{impair} : $\left(-a\right)^{n}=-a^{n}$ car alors $\left(-1\right)^{n}=-1$
          \par
          Pour $n$ \textbf{pair} : $\left(-a\right)^{n}=a^{n}$ car alors $\left(-1\right)^{n}=1$
     \end{itemize}
}
\cadre{bleu}{Définition}{% id="d100"
     On appelle \textbf{écriture scientifique} d'un  nombre positif, la notation $a\times 10^{n}$ avec $n$ entier relatif et $1 \leqslant  a < 10$.
}
\bloc{cyan}{Remarque}{% id="r100"
     L'encadrement $1 \leqslant  a < 10$ signifie que l'écriture décimale de $a$ comporte \textbf{un et un seul chiffre non nul avant la virgule}.
}
\bloc{orange}{Exemple}{% id="e100"
     $ D=\frac{5\times 10^{5}\times 10^{-2}\times 7}{2\times 10^{7}}$
     \par
     Donner l'écriture scientifique de $D$, puis son écriture décimale.
     \par
     On regroupe les puissances de 10 d'un coté et les nombres restants de l'autre:
     \par
     $ D=\frac{5\times 7}{2}\times \frac{10^{5}\times 10^{-2}}{10^{7}}$
     \par
     On simplifie :
     \par
     $ D=\frac{35}{2}\times \frac{10^{3}}{10^{7}}$
     \par
     $ D=17,5\times 10^{-4}$
     \par
     L'écriture scientifique de $D$ est :
     \par
     $ D=1,75\times 10^{-3}$
     \par
     L'écriture décimale de $D$ est :
     \par
     $ D=0,00175 $
}

\end{document}

µ
\documentclass[a4paper]{article}

%================================================================================================================================
%
% Packages
%
%================================================================================================================================

\usepackage[T1]{fontenc} 	% pour caractères accentués
\usepackage[utf8]{inputenc}  % encodage utf8
\usepackage[french]{babel}	% langue : français
\usepackage{fourier}			% caractères plus lisibles
\usepackage[dvipsnames]{xcolor} % couleurs
\usepackage{fancyhdr}		% réglage header footer
\usepackage{needspace}		% empêcher sauts de page mal placés
\usepackage{graphicx}		% pour inclure des graphiques
\usepackage{enumitem,cprotect}		% personnalise les listes d'items (nécessaire pour ol, al ...)
\usepackage{hyperref}		% Liens hypertexte
\usepackage{pstricks,pst-all,pst-node,pstricks-add,pst-math,pst-plot,pst-tree,pst-eucl} % pstricks
\usepackage[a4paper,includeheadfoot,top=2cm,left=3cm, bottom=2cm,right=3cm]{geometry} % marges etc.
\usepackage{comment}			% commentaires multilignes
\usepackage{amsmath,environ} % maths (matrices, etc.)
\usepackage{amssymb,makeidx}
\usepackage{bm}				% bold maths
\usepackage{tabularx}		% tableaux
\usepackage{colortbl}		% tableaux en couleur
\usepackage{fontawesome}		% Fontawesome
\usepackage{environ}			% environment with command
\usepackage{fp}				% calculs pour ps-tricks
\usepackage{multido}			% pour ps tricks
\usepackage[np]{numprint}	% formattage nombre
\usepackage{tikz,tkz-tab} 			% package principal TikZ
\usepackage{pgfplots}   % axes
\usepackage{mathrsfs}    % cursives
\usepackage{calc}			% calcul taille boites
\usepackage[scaled=0.875]{helvet} % font sans serif
\usepackage{svg} % svg
\usepackage{scrextend} % local margin
\usepackage{scratch} %scratch
\usepackage{multicol} % colonnes
%\usepackage{infix-RPN,pst-func} % formule en notation polanaise inversée
\usepackage{listings}

%================================================================================================================================
%
% Réglages de base
%
%================================================================================================================================

\lstset{
language=Python,   % R code
literate=
{á}{{\'a}}1
{à}{{\`a}}1
{ã}{{\~a}}1
{é}{{\'e}}1
{è}{{\`e}}1
{ê}{{\^e}}1
{í}{{\'i}}1
{ó}{{\'o}}1
{õ}{{\~o}}1
{ú}{{\'u}}1
{ü}{{\"u}}1
{ç}{{\c{c}}}1
{~}{{ }}1
}


\definecolor{codegreen}{rgb}{0,0.6,0}
\definecolor{codegray}{rgb}{0.5,0.5,0.5}
\definecolor{codepurple}{rgb}{0.58,0,0.82}
\definecolor{backcolour}{rgb}{0.95,0.95,0.92}

\lstdefinestyle{mystyle}{
    backgroundcolor=\color{backcolour},   
    commentstyle=\color{codegreen},
    keywordstyle=\color{magenta},
    numberstyle=\tiny\color{codegray},
    stringstyle=\color{codepurple},
    basicstyle=\ttfamily\footnotesize,
    breakatwhitespace=false,         
    breaklines=true,                 
    captionpos=b,                    
    keepspaces=true,                 
    numbers=left,                    
xleftmargin=2em,
framexleftmargin=2em,            
    showspaces=false,                
    showstringspaces=false,
    showtabs=false,                  
    tabsize=2,
    upquote=true
}

\lstset{style=mystyle}


\lstset{style=mystyle}
\newcommand{\imgdir}{C:/laragon/www/newmc/assets/imgsvg/}
\newcommand{\imgsvgdir}{C:/laragon/www/newmc/assets/imgsvg/}

\definecolor{mcgris}{RGB}{220, 220, 220}% ancien~; pour compatibilité
\definecolor{mcbleu}{RGB}{52, 152, 219}
\definecolor{mcvert}{RGB}{125, 194, 70}
\definecolor{mcmauve}{RGB}{154, 0, 215}
\definecolor{mcorange}{RGB}{255, 96, 0}
\definecolor{mcturquoise}{RGB}{0, 153, 153}
\definecolor{mcrouge}{RGB}{255, 0, 0}
\definecolor{mclightvert}{RGB}{205, 234, 190}

\definecolor{gris}{RGB}{220, 220, 220}
\definecolor{bleu}{RGB}{52, 152, 219}
\definecolor{vert}{RGB}{125, 194, 70}
\definecolor{mauve}{RGB}{154, 0, 215}
\definecolor{orange}{RGB}{255, 96, 0}
\definecolor{turquoise}{RGB}{0, 153, 153}
\definecolor{rouge}{RGB}{255, 0, 0}
\definecolor{lightvert}{RGB}{205, 234, 190}
\setitemize[0]{label=\color{lightvert}  $\bullet$}

\pagestyle{fancy}
\renewcommand{\headrulewidth}{0.2pt}
\fancyhead[L]{maths-cours.fr}
\fancyhead[R]{\thepage}
\renewcommand{\footrulewidth}{0.2pt}
\fancyfoot[C]{}

\newcolumntype{C}{>{\centering\arraybackslash}X}
\newcolumntype{s}{>{\hsize=.35\hsize\arraybackslash}X}

\setlength{\parindent}{0pt}		 
\setlength{\parskip}{3mm}
\setlength{\headheight}{1cm}

\def\ebook{ebook}
\def\book{book}
\def\web{web}
\def\type{web}

\newcommand{\vect}[1]{\overrightarrow{\,\mathstrut#1\,}}

\def\Oij{$\left(\text{O}~;~\vect{\imath},~\vect{\jmath}\right)$}
\def\Oijk{$\left(\text{O}~;~\vect{\imath},~\vect{\jmath},~\vect{k}\right)$}
\def\Ouv{$\left(\text{O}~;~\vect{u},~\vect{v}\right)$}

\hypersetup{breaklinks=true, colorlinks = true, linkcolor = OliveGreen, urlcolor = OliveGreen, citecolor = OliveGreen, pdfauthor={Didier BONNEL - https://www.maths-cours.fr} } % supprime les bordures autour des liens

\renewcommand{\arg}[0]{\text{arg}}

\everymath{\displaystyle}

%================================================================================================================================
%
% Macros - Commandes
%
%================================================================================================================================

\newcommand\meta[2]{    			% Utilisé pour créer le post HTML.
	\def\titre{titre}
	\def\url{url}
	\def\arg{#1}
	\ifx\titre\arg
		\newcommand\maintitle{#2}
		\fancyhead[L]{#2}
		{\Large\sffamily \MakeUppercase{#2}}
		\vspace{1mm}\textcolor{mcvert}{\hrule}
	\fi 
	\ifx\url\arg
		\fancyfoot[L]{\href{https://www.maths-cours.fr#2}{\black \footnotesize{https://www.maths-cours.fr#2}}}
	\fi 
}


\newcommand\TitreC[1]{    		% Titre centré
     \needspace{3\baselineskip}
     \begin{center}\textbf{#1}\end{center}
}

\newcommand\newpar{    		% paragraphe
     \par
}

\newcommand\nosp {    		% commande vide (pas d'espace)
}
\newcommand{\id}[1]{} %ignore

\newcommand\boite[2]{				% Boite simple sans titre
	\vspace{5mm}
	\setlength{\fboxrule}{0.2mm}
	\setlength{\fboxsep}{5mm}	
	\fcolorbox{#1}{#1!3}{\makebox[\linewidth-2\fboxrule-2\fboxsep]{
  		\begin{minipage}[t]{\linewidth-2\fboxrule-4\fboxsep}\setlength{\parskip}{3mm}
  			 #2
  		\end{minipage}
	}}
	\vspace{5mm}
}

\newcommand\CBox[4]{				% Boites
	\vspace{5mm}
	\setlength{\fboxrule}{0.2mm}
	\setlength{\fboxsep}{5mm}
	
	\fcolorbox{#1}{#1!3}{\makebox[\linewidth-2\fboxrule-2\fboxsep]{
		\begin{minipage}[t]{1cm}\setlength{\parskip}{3mm}
	  		\textcolor{#1}{\LARGE{#2}}    
 	 	\end{minipage}  
  		\begin{minipage}[t]{\linewidth-2\fboxrule-4\fboxsep}\setlength{\parskip}{3mm}
			\raisebox{1.2mm}{\normalsize\sffamily{\textcolor{#1}{#3}}}						
  			 #4
  		\end{minipage}
	}}
	\vspace{5mm}
}

\newcommand\cadre[3]{				% Boites convertible html
	\par
	\vspace{2mm}
	\setlength{\fboxrule}{0.1mm}
	\setlength{\fboxsep}{5mm}
	\fcolorbox{#1}{white}{\makebox[\linewidth-2\fboxrule-2\fboxsep]{
  		\begin{minipage}[t]{\linewidth-2\fboxrule-4\fboxsep}\setlength{\parskip}{3mm}
			\raisebox{-2.5mm}{\sffamily \small{\textcolor{#1}{\MakeUppercase{#2}}}}		
			\par		
  			 #3
 	 		\end{minipage}
	}}
		\vspace{2mm}
	\par
}

\newcommand\bloc[3]{				% Boites convertible html sans bordure
     \needspace{2\baselineskip}
     {\sffamily \small{\textcolor{#1}{\MakeUppercase{#2}}}}    
		\par		
  			 #3
		\par
}

\newcommand\CHelp[1]{
     \CBox{Plum}{\faInfoCircle}{À RETENIR}{#1}
}

\newcommand\CUp[1]{
     \CBox{NavyBlue}{\faThumbsOUp}{EN PRATIQUE}{#1}
}

\newcommand\CInfo[1]{
     \CBox{Sepia}{\faArrowCircleRight}{REMARQUE}{#1}
}

\newcommand\CRedac[1]{
     \CBox{PineGreen}{\faEdit}{BIEN R\'EDIGER}{#1}
}

\newcommand\CError[1]{
     \CBox{Red}{\faExclamationTriangle}{ATTENTION}{#1}
}

\newcommand\TitreExo[2]{
\needspace{4\baselineskip}
 {\sffamily\large EXERCICE #1\ (\emph{#2 points})}
\vspace{5mm}
}

\newcommand\img[2]{
          \includegraphics[width=#2\paperwidth]{\imgdir#1}
}

\newcommand\imgsvg[2]{
       \begin{center}   \includegraphics[width=#2\paperwidth]{\imgsvgdir#1} \end{center}
}


\newcommand\Lien[2]{
     \href{#1}{#2 \tiny \faExternalLink}
}
\newcommand\mcLien[2]{
     \href{https~://www.maths-cours.fr/#1}{#2 \tiny \faExternalLink}
}

\newcommand{\euro}{\eurologo{}}

%================================================================================================================================
%
% Macros - Environement
%
%================================================================================================================================

\newenvironment{tex}{ %
}
{%
}

\newenvironment{indente}{ %
	\setlength\parindent{10mm}
}

{
	\setlength\parindent{0mm}
}

\newenvironment{corrige}{%
     \needspace{3\baselineskip}
     \medskip
     \textbf{\textsc{Corrigé}}
     \medskip
}
{
}

\newenvironment{extern}{%
     \begin{center}
     }
     {
     \end{center}
}

\NewEnviron{code}{%
	\par
     \boite{gray}{\texttt{%
     \BODY
     }}
     \par
}

\newenvironment{vbloc}{% boite sans cadre empeche saut de page
     \begin{minipage}[t]{\linewidth}
     }
     {
     \end{minipage}
}
\NewEnviron{h2}{%
    \needspace{3\baselineskip}
    \vspace{0.6cm}
	\noindent \MakeUppercase{\sffamily \large \BODY}
	\vspace{1mm}\textcolor{mcgris}{\hrule}\vspace{0.4cm}
	\par
}{}

\NewEnviron{h3}{%
    \needspace{3\baselineskip}
	\vspace{5mm}
	\textsc{\BODY}
	\par
}

\NewEnviron{margeneg}{ %
\begin{addmargin}[-1cm]{0cm}
\BODY
\end{addmargin}
}

\NewEnviron{html}{%
}

\begin{document}
\meta{url}{/cours/calcul-litteral/}
\meta{pid}{1561}
\meta{titre}{Calcul littéral}
\meta{type}{cours}
\begin{h2}1 - Développer\end{h2}
\cadre{bleu}{Définition}{% id="d10"
     \textbf{Développer} un produit, c'est l'écrire sous la forme d'une somme (ou d'une différence).
}
\bloc{cyan}{Rappel}{% id="r10"
     \begin{itemize}
          \item Une expression est une somme (algébrique) si la \textbf{dernière} opération effectuée (celle qui donne le résultat final) est une addition ou une soustraction.
          \item Une expression est un produit si la \textbf{dernière} opération effectuée (celle qui donne le résultat final) est une multiplication.
     \end{itemize}
     Par exemple :
     \begin{itemize}
          \item $3\times 5-2\times 45$ et $2x+8y$ sont des sommes algébriques
          \item $5\times \left(3+8\right)$ et $\left(x+1\right)\left(y-5\right)$ sont des produits.
     \end{itemize}
}
\cadre{vert}{Propriétés (Distributivité)}{% id="p20"
     \begin{itemize}
          \item $k\left(a+b\right)=ka+kb$
          \item $k\left(a-b\right)=ka-kb$
          \item $\left(a+b\right)\left(c+d\right)=ac+ad+bc+bd$
     \end{itemize}
}
\bloc{orange}{Exemples}{% id="e20"
     Développer les expressions suivantes:
     \begin{itemize}
          \item $A=3\left(x-2\right)$
          \par
          $A=3x-6$
          \item $B=\left(x+3\right)\left(2x-5\right)$
          \par
          $B=2x^{2}-5x+6x-15$
          \par
          $B=2x^{2}+x-15$
     \end{itemize}
}
\cadre{vert}{Propriétés (Identités remarquables - Développement)}{% id="p30"
     \begin{itemize}
          \item $\left(a+b\right)^{2}=a^{2}+2ab+b^{2}$
          \item $\left(a-b\right)^{2}=a^{2}-2ab+b^{2}$
          \item $\left(a+b\right)\left(a-b\right)=a^{2}-b^{2}$
     \end{itemize}
}
\bloc{orange}{Exemples}{% id="e30"
     Développer les expressions suivantes:
     \begin{itemize}
          \item $C=\left(x+1\right)^{2}$
          \par
          $C=x^{2}+2\times x\times 1+1^{2} $  \textit{(première identité remarquable avec $a=x$ et $b=1$)}
          $C=x^{2}+2x+1 $
          \item $D=\left(2x-1\right)^{2}$
          \par
          $D=4x^{2}-2\times 2x\times 1+1^{2} $  \textit{(seconde identité remarquable avec $a=2x$ et $b=1$)}
          $D=4x^{2}-4x+1 $
          \item $E=\left(x+2\right)\left(x-2\right)$
          \par
          $E=x^{2}-2^{2} $  \textit{(troisième identité remarquable avec $a=x$ et $b=2$)}
          $E=x^{2}-4 $
     \end{itemize}
}
\begin{h2}2 - Factoriser\end{h2}
\cadre{bleu}{Définition}{% id="d40"
     \textbf{Factoriser} une somme (ou une différence), c'est l'écrire sous la forme d'un produit.
}
\cadre{vert}{Propriétés}{% id="p50"
     \begin{itemize}
          \item $ka+kb=k\left(a+b\right)$
          \item $ka-kb=k\left(a-b\right)$
     \end{itemize}
     k est le \textbf{facteur commun}
}
\bloc{orange}{Exemples}{% id="e50"
     Factoriser les expressions suivantes:
     \begin{itemize}
          \item $A=\left(x+3\right)\left(x+2\right)-7\left(x+2\right)$
          \par
          Le facteur commun est $\left(x+2\right)$
          \par
          $A=\left(x+2\right)\left[\left(x+3\right)-7\right]$
          \par
          $A=\left(x+2\right)\left(x-4\right)$
          \item $B=\left(2x+1\right)^{2}-\left(2x+1\right)\left(x+3\right)$
          \par
          $B=\left(2x+1\right)\left(2x+1\right)-\left(2x+1\right)\left(x+3\right)$
          \par
          Le facteur commun est $\left(2x+1\right)$
          \par
          $B=\left(2x+1\right)\left[\left(2x+1\right)-\left(x+3\right)\right]$
          \par
          $B=\left(2x+1\right)\left(2x+1-x-3\right)$
          \par
          $B=\left(2x+1\right)\left(x-2\right)$
     \end{itemize}
}
\bloc{cyan}{Remarques}{% id="r50"
     \begin{itemize}
          \item \textbf{Avec des carrés :}
\par
          Pour factoriser $\left(x+1\right)^{2}+\left(x+1\right)\left(x+2\right)$, on utilise le fait que $\left(x+1\right)^{2}=\left(x+1\right)\left(x+1\right)$ ce qui fait apparaître le facteur commun $\left(x+1\right)$ :
          \par
          $\left(x+1\right)^{2}+\left(x+1\right)\left(x+2\right)={\color{red} {\left(x+1\right)}}\left(x+1\right)+{\color{red} {\left(x+1\right)}}\left(x+2\right)$
          \par
          ~ ~ ~ ~ $=\left(x+1\right)\left[\left(x+1\right)+\left(x+2\right)\right]$
          \par
          ~ ~ ~ ~ $=\left(x+1\right)\left(2x+3\right)$
          \item \textbf{Attention à ne pas oublier le 1 !}
\par
          Pour factoriser $x^{2}-x$ on écrit que $x^{2}=x\times x$ et $x=x\times 1$;
          \par
          $x$ est alors facteur commun :
          \par
          $x^{2}-x = {\color{red} x}\times x-{\color{red} x}\times 1 = x \left(x-1\right)$
     \end{itemize}
}
\cadre{vert}{Propriétés (Identités remarquables - Factorisation)}{% id="p60"
     \begin{itemize}
          \item $a^{2}+2ab+b^{2}=\left(a+b\right)^{2}$
          \item $a^{2}-2ab+b^{2}=\left(a-b\right)^{2}$
          \item $a^{2}-b^{2}=\left(a+b\right)\left(a-b\right)$
     \end{itemize}
}
\bloc{orange}{Exemples}{% id="e60"
     Factoriser les expressions suivantes:
     \begin{itemize}
          \item $C=x^{2}-6x+9$
          \par
          $C=x^{2}-2\times x\times 3+3^{2}$
          \par
          $C=\left(x-3\right)^{2} $ \textit{(seconde identité remarquable avec $a=x$ et $b=3$)}
          \item $D=25x^{2}-4$
          \par
          $D=\left(5x\right)^{2}-2^{2}$
          \par
          $D=\left(5x+2\right)\left(5x-2\right)$ \textit{(troisième identité remarquable avec $a=5x$ et $b=2$)}
     \end{itemize}
}

\end{document}

µ
\documentclass[a4paper]{article}

%================================================================================================================================
%
% Packages
%
%================================================================================================================================

\usepackage[T1]{fontenc} 	% pour caractères accentués
\usepackage[utf8]{inputenc}  % encodage utf8
\usepackage[french]{babel}	% langue : français
\usepackage{fourier}			% caractères plus lisibles
\usepackage[dvipsnames]{xcolor} % couleurs
\usepackage{fancyhdr}		% réglage header footer
\usepackage{needspace}		% empêcher sauts de page mal placés
\usepackage{graphicx}		% pour inclure des graphiques
\usepackage{enumitem,cprotect}		% personnalise les listes d'items (nécessaire pour ol, al ...)
\usepackage{hyperref}		% Liens hypertexte
\usepackage{pstricks,pst-all,pst-node,pstricks-add,pst-math,pst-plot,pst-tree,pst-eucl} % pstricks
\usepackage[a4paper,includeheadfoot,top=2cm,left=3cm, bottom=2cm,right=3cm]{geometry} % marges etc.
\usepackage{comment}			% commentaires multilignes
\usepackage{amsmath,environ} % maths (matrices, etc.)
\usepackage{amssymb,makeidx}
\usepackage{bm}				% bold maths
\usepackage{tabularx}		% tableaux
\usepackage{colortbl}		% tableaux en couleur
\usepackage{fontawesome}		% Fontawesome
\usepackage{environ}			% environment with command
\usepackage{fp}				% calculs pour ps-tricks
\usepackage{multido}			% pour ps tricks
\usepackage[np]{numprint}	% formattage nombre
\usepackage{tikz,tkz-tab} 			% package principal TikZ
\usepackage{pgfplots}   % axes
\usepackage{mathrsfs}    % cursives
\usepackage{calc}			% calcul taille boites
\usepackage[scaled=0.875]{helvet} % font sans serif
\usepackage{svg} % svg
\usepackage{scrextend} % local margin
\usepackage{scratch} %scratch
\usepackage{multicol} % colonnes
%\usepackage{infix-RPN,pst-func} % formule en notation polanaise inversée
\usepackage{listings}

%================================================================================================================================
%
% Réglages de base
%
%================================================================================================================================

\lstset{
language=Python,   % R code
literate=
{á}{{\'a}}1
{à}{{\`a}}1
{ã}{{\~a}}1
{é}{{\'e}}1
{è}{{\`e}}1
{ê}{{\^e}}1
{í}{{\'i}}1
{ó}{{\'o}}1
{õ}{{\~o}}1
{ú}{{\'u}}1
{ü}{{\"u}}1
{ç}{{\c{c}}}1
{~}{{ }}1
}


\definecolor{codegreen}{rgb}{0,0.6,0}
\definecolor{codegray}{rgb}{0.5,0.5,0.5}
\definecolor{codepurple}{rgb}{0.58,0,0.82}
\definecolor{backcolour}{rgb}{0.95,0.95,0.92}

\lstdefinestyle{mystyle}{
    backgroundcolor=\color{backcolour},   
    commentstyle=\color{codegreen},
    keywordstyle=\color{magenta},
    numberstyle=\tiny\color{codegray},
    stringstyle=\color{codepurple},
    basicstyle=\ttfamily\footnotesize,
    breakatwhitespace=false,         
    breaklines=true,                 
    captionpos=b,                    
    keepspaces=true,                 
    numbers=left,                    
xleftmargin=2em,
framexleftmargin=2em,            
    showspaces=false,                
    showstringspaces=false,
    showtabs=false,                  
    tabsize=2,
    upquote=true
}

\lstset{style=mystyle}


\lstset{style=mystyle}
\newcommand{\imgdir}{C:/laragon/www/newmc/assets/imgsvg/}
\newcommand{\imgsvgdir}{C:/laragon/www/newmc/assets/imgsvg/}

\definecolor{mcgris}{RGB}{220, 220, 220}% ancien~; pour compatibilité
\definecolor{mcbleu}{RGB}{52, 152, 219}
\definecolor{mcvert}{RGB}{125, 194, 70}
\definecolor{mcmauve}{RGB}{154, 0, 215}
\definecolor{mcorange}{RGB}{255, 96, 0}
\definecolor{mcturquoise}{RGB}{0, 153, 153}
\definecolor{mcrouge}{RGB}{255, 0, 0}
\definecolor{mclightvert}{RGB}{205, 234, 190}

\definecolor{gris}{RGB}{220, 220, 220}
\definecolor{bleu}{RGB}{52, 152, 219}
\definecolor{vert}{RGB}{125, 194, 70}
\definecolor{mauve}{RGB}{154, 0, 215}
\definecolor{orange}{RGB}{255, 96, 0}
\definecolor{turquoise}{RGB}{0, 153, 153}
\definecolor{rouge}{RGB}{255, 0, 0}
\definecolor{lightvert}{RGB}{205, 234, 190}
\setitemize[0]{label=\color{lightvert}  $\bullet$}

\pagestyle{fancy}
\renewcommand{\headrulewidth}{0.2pt}
\fancyhead[L]{maths-cours.fr}
\fancyhead[R]{\thepage}
\renewcommand{\footrulewidth}{0.2pt}
\fancyfoot[C]{}

\newcolumntype{C}{>{\centering\arraybackslash}X}
\newcolumntype{s}{>{\hsize=.35\hsize\arraybackslash}X}

\setlength{\parindent}{0pt}		 
\setlength{\parskip}{3mm}
\setlength{\headheight}{1cm}

\def\ebook{ebook}
\def\book{book}
\def\web{web}
\def\type{web}

\newcommand{\vect}[1]{\overrightarrow{\,\mathstrut#1\,}}

\def\Oij{$\left(\text{O}~;~\vect{\imath},~\vect{\jmath}\right)$}
\def\Oijk{$\left(\text{O}~;~\vect{\imath},~\vect{\jmath},~\vect{k}\right)$}
\def\Ouv{$\left(\text{O}~;~\vect{u},~\vect{v}\right)$}

\hypersetup{breaklinks=true, colorlinks = true, linkcolor = OliveGreen, urlcolor = OliveGreen, citecolor = OliveGreen, pdfauthor={Didier BONNEL - https://www.maths-cours.fr} } % supprime les bordures autour des liens

\renewcommand{\arg}[0]{\text{arg}}

\everymath{\displaystyle}

%================================================================================================================================
%
% Macros - Commandes
%
%================================================================================================================================

\newcommand\meta[2]{    			% Utilisé pour créer le post HTML.
	\def\titre{titre}
	\def\url{url}
	\def\arg{#1}
	\ifx\titre\arg
		\newcommand\maintitle{#2}
		\fancyhead[L]{#2}
		{\Large\sffamily \MakeUppercase{#2}}
		\vspace{1mm}\textcolor{mcvert}{\hrule}
	\fi 
	\ifx\url\arg
		\fancyfoot[L]{\href{https://www.maths-cours.fr#2}{\black \footnotesize{https://www.maths-cours.fr#2}}}
	\fi 
}


\newcommand\TitreC[1]{    		% Titre centré
     \needspace{3\baselineskip}
     \begin{center}\textbf{#1}\end{center}
}

\newcommand\newpar{    		% paragraphe
     \par
}

\newcommand\nosp {    		% commande vide (pas d'espace)
}
\newcommand{\id}[1]{} %ignore

\newcommand\boite[2]{				% Boite simple sans titre
	\vspace{5mm}
	\setlength{\fboxrule}{0.2mm}
	\setlength{\fboxsep}{5mm}	
	\fcolorbox{#1}{#1!3}{\makebox[\linewidth-2\fboxrule-2\fboxsep]{
  		\begin{minipage}[t]{\linewidth-2\fboxrule-4\fboxsep}\setlength{\parskip}{3mm}
  			 #2
  		\end{minipage}
	}}
	\vspace{5mm}
}

\newcommand\CBox[4]{				% Boites
	\vspace{5mm}
	\setlength{\fboxrule}{0.2mm}
	\setlength{\fboxsep}{5mm}
	
	\fcolorbox{#1}{#1!3}{\makebox[\linewidth-2\fboxrule-2\fboxsep]{
		\begin{minipage}[t]{1cm}\setlength{\parskip}{3mm}
	  		\textcolor{#1}{\LARGE{#2}}    
 	 	\end{minipage}  
  		\begin{minipage}[t]{\linewidth-2\fboxrule-4\fboxsep}\setlength{\parskip}{3mm}
			\raisebox{1.2mm}{\normalsize\sffamily{\textcolor{#1}{#3}}}						
  			 #4
  		\end{minipage}
	}}
	\vspace{5mm}
}

\newcommand\cadre[3]{				% Boites convertible html
	\par
	\vspace{2mm}
	\setlength{\fboxrule}{0.1mm}
	\setlength{\fboxsep}{5mm}
	\fcolorbox{#1}{white}{\makebox[\linewidth-2\fboxrule-2\fboxsep]{
  		\begin{minipage}[t]{\linewidth-2\fboxrule-4\fboxsep}\setlength{\parskip}{3mm}
			\raisebox{-2.5mm}{\sffamily \small{\textcolor{#1}{\MakeUppercase{#2}}}}		
			\par		
  			 #3
 	 		\end{minipage}
	}}
		\vspace{2mm}
	\par
}

\newcommand\bloc[3]{				% Boites convertible html sans bordure
     \needspace{2\baselineskip}
     {\sffamily \small{\textcolor{#1}{\MakeUppercase{#2}}}}    
		\par		
  			 #3
		\par
}

\newcommand\CHelp[1]{
     \CBox{Plum}{\faInfoCircle}{À RETENIR}{#1}
}

\newcommand\CUp[1]{
     \CBox{NavyBlue}{\faThumbsOUp}{EN PRATIQUE}{#1}
}

\newcommand\CInfo[1]{
     \CBox{Sepia}{\faArrowCircleRight}{REMARQUE}{#1}
}

\newcommand\CRedac[1]{
     \CBox{PineGreen}{\faEdit}{BIEN R\'EDIGER}{#1}
}

\newcommand\CError[1]{
     \CBox{Red}{\faExclamationTriangle}{ATTENTION}{#1}
}

\newcommand\TitreExo[2]{
\needspace{4\baselineskip}
 {\sffamily\large EXERCICE #1\ (\emph{#2 points})}
\vspace{5mm}
}

\newcommand\img[2]{
          \includegraphics[width=#2\paperwidth]{\imgdir#1}
}

\newcommand\imgsvg[2]{
       \begin{center}   \includegraphics[width=#2\paperwidth]{\imgsvgdir#1} \end{center}
}


\newcommand\Lien[2]{
     \href{#1}{#2 \tiny \faExternalLink}
}
\newcommand\mcLien[2]{
     \href{https~://www.maths-cours.fr/#1}{#2 \tiny \faExternalLink}
}

\newcommand{\euro}{\eurologo{}}

%================================================================================================================================
%
% Macros - Environement
%
%================================================================================================================================

\newenvironment{tex}{ %
}
{%
}

\newenvironment{indente}{ %
	\setlength\parindent{10mm}
}

{
	\setlength\parindent{0mm}
}

\newenvironment{corrige}{%
     \needspace{3\baselineskip}
     \medskip
     \textbf{\textsc{Corrigé}}
     \medskip
}
{
}

\newenvironment{extern}{%
     \begin{center}
     }
     {
     \end{center}
}

\NewEnviron{code}{%
	\par
     \boite{gray}{\texttt{%
     \BODY
     }}
     \par
}

\newenvironment{vbloc}{% boite sans cadre empeche saut de page
     \begin{minipage}[t]{\linewidth}
     }
     {
     \end{minipage}
}
\NewEnviron{h2}{%
    \needspace{3\baselineskip}
    \vspace{0.6cm}
	\noindent \MakeUppercase{\sffamily \large \BODY}
	\vspace{1mm}\textcolor{mcgris}{\hrule}\vspace{0.4cm}
	\par
}{}

\NewEnviron{h3}{%
    \needspace{3\baselineskip}
	\vspace{5mm}
	\textsc{\BODY}
	\par
}

\NewEnviron{margeneg}{ %
\begin{addmargin}[-1cm]{0cm}
\BODY
\end{addmargin}
}

\NewEnviron{html}{%
}

\begin{document}
\meta{url}{/cours/fonction/}
\meta{pid}{1566}
\meta{titre}{Notion de fonction}
\meta{type}{cours}
\begin{h2}1 - Généralités\end{h2}
\cadre{bleu}{Définition}{% id="d10"
     Une \textbf{fonction} $f$ est un procédé qui à tout nombre réel $x$ associe \textbf{ un seul }nombre réel $y$.
     \begin{itemize}
          \item $x$ s'appelle la \textbf{variable}.
          \item $y$ s'appelle l'\textbf{image} de $x$ par la fonction $f$ et se note $f\left(x\right)$
          \item $f$ est la \textbf{fonction} et se note: $f : x\mapsto y$.
          \item On note aussi $y=f\left(x\right)$.
     \end{itemize}
}
\bloc{cyan}{Remarque}{% id="r10"
     Les procédés permettant d'associer un nombre à un autre nombre peuvent être :
     \begin{itemize}
          \item Des formules mathématiques (par exemple : $f\left(x\right)=2x+5$)
          \item Une courbe (par exemple : la courbe donnant le cours d'une action en Bourse en fonction du temps)
          \item Un instrument de mesure ou de conversion (par exemple : le compteur d'un taxi qui donne le prix à payer en fonction du trajet parcouru)
          \item Un tableau de valeurs, chaque élément de la seconde ligne étant associé à un élément de la première ligne
          \item Une touche de calculatrice (par exemple: \textit{sin, cos, ln, log}, etc.) qui affiche un résultat dépendant du nombre saisi auparavant
          \item Etc...
     \end{itemize}
}
\cadre{rouge}{Méthode}{% id="t20"
     Pour calculer l'image d'un nombre par une fonction $f$, on remplace $x$ par ce nombre dans la formule donnant $f\left(x\right)$.
}
\cadre{rouge}{Attention !}{%
     N'oubliez pas les parenthèses quand vous remplacez $x$ par un nombre négatif ou par une expression composée (comme $1+\sqrt{2}$ par exemple).
}
\bloc{orange}{Exemple}{% id="e20"
     Soit $f\left(x\right)=x^{2}+1$
     \par
     L'image de $-1$ par $f$ s'obtient en remplaçant $x$ par $\left(-1\right)$ dans la formule ci-dessus :
     \par
     $f\left(-1\right) =\left(-1\right)^{2}+1=1+1=2$.
}
\cadre{bleu}{Définition}{% id="d30"
     Soit $y$ un nombre réel. Déterminer les \textbf{antécédents} de $y$ par $f$, c'est trouver les valeurs de $x$ telles que $f\left(x\right)=y$.
}
\bloc{cyan}{Remarque}{% id="r30"
     Un nombre peut avoir \textbf{aucun, un ou plusieurs} antécédent(s).
}
\cadre{rouge}{Méthode}{% id="t40"
     Soit $\alpha $ un nombre réel.
     \par
     Pour trouver les antécédents de $\alpha $ par la fonction $f$, on résout l'équation $f\left(x\right)=\alpha $ d'inconnue $x$.
}
\bloc{orange}{Exemple}{% id="e40"
     Soit la fonction $f$ définie par $f\left(x\right)=2x-3$.
     \par
     Pour trouver le(s) antécédent(s) du nombre $1$ on résout l'équation $f\left(x\right)=1$ c'est à dire :
     \par
     $2x-3=1$
     \par
     $2x=4$
     \par
     $x=2$
     \par
     Donc $1$ a un seul antécédent qui est le nombre$2$.
}
\begin{h2}2 - Représentation graphique\end{h2}
\cadre{bleu}{Définitions}{% id="d50"
     Un \textbf{repère} du plan est un triplet de points non alignés $\left(O,I,J\right)$.
     \par
     Le point $O$ est appelé \textbf{l'origine du repère}, la droite $\left(OI\right)$, \textbf{l'axe des abscisses} et la droite $\left(OJ\right)$, \textbf{l'axe des ordonnées}.
     \par
     Un repère est  \textbf{orthonormé} (ou \textbf{orthonormal}) si les points $O, I, J$ forment un triangle rectangle isocèle en $O$.
}
\bloc{cyan}{Remarque}{% id="r50"
     On note généralement $\left(Ox\right)$ l'axe des abscisses et $\left(Oy\right)$ l'axe des ordonnées.
}
\cadre{bleu}{Rappel vocabulaire}{% id="d60"
     Le plan est muni d'un repère $\left(O ; I, J\right)$. On désigne par $M$ un point du plan.
     \par
     $M$ a pour \textbf{coordonnées} $\left(x; y\right)$, le nombre $x$ est \textbf{l'abscisse} du point $M$ et le nombre $y$ est son \textbf{ordonnée}.
}
\begin{center}
     \begin{extern}%width="500" alt="repère orthonormé"
          \resizebox{10cm}{!}{%
               % -+-+-+ variables modifiables
               \def\xmin{-4.5}
               \def\xmax{4.5}
               \def\ymin{-3.5}
               \def\ymax{5.5}
               \def\xunit{1.5}  % unités en cm
               \def\yunit{1.5}
               \psset{xunit=\xunit,yunit=\yunit,algebraic=true}
               \fontsize{15pt}{15pt}\selectfont
               \begin{pspicture*}[linewidth=1pt](\xmin,\ymin)(\xmax,\ymax)
                    \psgrid[gridcolor=mcgris, subgriddiv=1, gridlabels=0pt](-5,-4)(5,6)
                    %     \psaxes[linewidth=0.pt,Dx=1,Dy=1]{->}(0,0)(\xmin,\ymin)(\xmax,\ymax)
                    \psline[linewidth=0.75pt,linecolor=mcmauve,arrowsize=6pt]{->}(0,\ymin)(0,\ymax)
                    \psline[linewidth=0.75pt,linecolor=mcvert,arrowsize=6pt]{->}(\xmin,0)(\xmax,0)
                    \psline[linewidth=0.75pt,linecolor=gray,linestyle=dashed]{->}(3,2)(0,2)
                    \psline[linewidth=0.75pt,linecolor=gray,linestyle=dashed]{->}(3,2)(3,0)
                    \rput[tr](-0.1,-0.1){$\red O$}
                    \rput[tr](4.4,-0.1){\color{mcvert} axe des abscisses }
                    \rput{90}(-0.25,3.9){\color{mcmauve} axe des ordonnées }
                    \rput[t](1,-0.1){$\red I$}
                    \rput[r](-0.1,1){$\red J$}
                    \psdot[linecolor=blue,dotsize=3pt,dotstyle=+](3,2)
                    \psdots[linecolor=red,dotsize=3pt,dotstyle=+](0,0)(0,1)(1,0)
                    \rput[lb](3.1,2.1){$\blue M(3~;~2)$}
               \end{pspicture*}
          }
     \end{extern}
\end{center}
\bloc{orange}{Exemple}{% id="e60"
     \begin{itemize}
          \item Les coordonnées du point $O$ sont $(0~;~0)$.
          \item Les coordonnées du point $I$ sont $(1~;~0)$.
          \item Les coordonnées du point $J$ sont $(0~;~1)$.
          \item Les coordonnées du point $M$ sont $(3~;~2)$.
     \end{itemize}
}
\cadre{bleu}{Définition}{% id="d70"
     La courbe représentative de la fonction $f$ dans un repère $\left(O; I, J\right)$ est l'ensemble des points $M$ de coordonnées $\left(x ; f\left(x\right)\right)$
}
\bloc{cyan}{Remarque}{% id="r70"
     La définition précédente donne un critère permettant de déterminer si un point $A\left(\alpha  ; \beta \right)$ appartient à la courbe représentative d'une fonction $f$ : on calcule $f\left(\alpha \right)$ et on regarde si $f\left(\alpha \right)=\beta $
}
\bloc{orange}{Exemple}{% id="e70"
     $f\left(x\right)=1+x^{2}$. Les points $A\left(1 ; 3\right)$ et $B\left(2 ; 5\right)$ appartiennent-ils à la courbe représentative $\mathscr C_{f} $ de la fonction $f$ ?
     \par
     Pour $A$ : $f\left(1\right)=1+1^{2}=2$ n'est pas l'ordonnée de $A$. Donc $A$ n'est pas situé sur la courbe $\mathscr C_{f} $.
     \par
     Pour $B$ : $f\left(2\right)=1+2^{2}=1+4=5$ est l'ordonnée de $B$.  Donc $B$ est situé sur la courbe $\mathscr C_{f} $.
}
\cadre{rouge}{Méthode}{% id="t80"
     Une méthode simple mais approximative pour tracer la courbe représentative d'une fonction $f$ consiste :
     \begin{itemize}
          \item à calculer $f\left(x\right)$ pour plusieurs valeurs de $x$ ;
          \item puis à placer les points de coordonnées $\left(x ; f\left(x\right)\right)$ correspondant aux valeurs obtenues ;
          \item et enfin à relier ces différents points.
     \end{itemize}
}
\bloc{orange}{Exemple}{% id="e80"
     Pour tracer la courbe représentative de la fonction $f~ : ~ x \mapsto  x^{2}-1$ on calcule quelques images :
     \begin{center}
          \begin{tabularx}{0.6\linewidth}{|*{5}{>{\centering \arraybackslash }X|}}%class="compact" width="600"
               \hline
               $x$ & -1  &  0 &  1 &  2
               \\ \hline
               $f\left(x\right)$ & 0 & -1 & 0 & 3
               \\ \hline
          \end{tabularx}
     \end{center}
     On place les points correspondants puis on les relie pour obtenir la courbe :
     \begin{center}
          \begin{extern}%width="500" alt="repère orthonormé"
               \resizebox{10cm}{!}{%
                    % -+-+-+ variables modifiables
                    \def\fonction{x*x-1}
                    \def\xmin{-4.5}
                    \def\xmax{4.5}
                    \def\ymin{-2.5}
                    \def\ymax{3.5}
                    \def\xunit{1.5}  % unités en cm
                    \def\yunit{1.5}
                    \psset{xunit=\xunit,yunit=\yunit,algebraic=true}
                    \fontsize{15pt}{15pt}\selectfont
                    \begin{pspicture*}[linewidth=1pt,arrowsize=6pt](\xmin,\ymin)(\xmax,\ymax)
                         \psgrid[gridcolor=mcgris, subgriddiv=1, gridlabels=0pt](-5,-3)(5,4)
                         \psaxes[linewidth=0.pt,Dx=1,Dy=1]{->}(0,0)(\xmin,\ymin)(\xmax,\ymax)
                         \rput[tr](-0.2,-0.2){$O$}
                         \psdots[linecolor=red,dotsize=4pt,dotstyle=*](2,3)(0,-1)(-1,0)
                         \psdots[linecolor=red,dotsize=4pt,dotstyle=*](1,0)
                         \psplot[plotpoints=2000,linecolor=blue]{-1.5}{2.2}{\fonction}
                    \end{pspicture*}
               }
          \end{extern}
     \end{center}
}

\end{document}
µ
\documentclass[a4paper]{article}

%================================================================================================================================
%
% Packages
%
%================================================================================================================================

\usepackage[T1]{fontenc} 	% pour caractères accentués
\usepackage[utf8]{inputenc}  % encodage utf8
\usepackage[french]{babel}	% langue : français
\usepackage{fourier}			% caractères plus lisibles
\usepackage[dvipsnames]{xcolor} % couleurs
\usepackage{fancyhdr}		% réglage header footer
\usepackage{needspace}		% empêcher sauts de page mal placés
\usepackage{graphicx}		% pour inclure des graphiques
\usepackage{enumitem,cprotect}		% personnalise les listes d'items (nécessaire pour ol, al ...)
\usepackage{hyperref}		% Liens hypertexte
\usepackage{pstricks,pst-all,pst-node,pstricks-add,pst-math,pst-plot,pst-tree,pst-eucl} % pstricks
\usepackage[a4paper,includeheadfoot,top=2cm,left=3cm, bottom=2cm,right=3cm]{geometry} % marges etc.
\usepackage{comment}			% commentaires multilignes
\usepackage{amsmath,environ} % maths (matrices, etc.)
\usepackage{amssymb,makeidx}
\usepackage{bm}				% bold maths
\usepackage{tabularx}		% tableaux
\usepackage{colortbl}		% tableaux en couleur
\usepackage{fontawesome}		% Fontawesome
\usepackage{environ}			% environment with command
\usepackage{fp}				% calculs pour ps-tricks
\usepackage{multido}			% pour ps tricks
\usepackage[np]{numprint}	% formattage nombre
\usepackage{tikz,tkz-tab} 			% package principal TikZ
\usepackage{pgfplots}   % axes
\usepackage{mathrsfs}    % cursives
\usepackage{calc}			% calcul taille boites
\usepackage[scaled=0.875]{helvet} % font sans serif
\usepackage{svg} % svg
\usepackage{scrextend} % local margin
\usepackage{scratch} %scratch
\usepackage{multicol} % colonnes
%\usepackage{infix-RPN,pst-func} % formule en notation polanaise inversée
\usepackage{listings}

%================================================================================================================================
%
% Réglages de base
%
%================================================================================================================================

\lstset{
language=Python,   % R code
literate=
{á}{{\'a}}1
{à}{{\`a}}1
{ã}{{\~a}}1
{é}{{\'e}}1
{è}{{\`e}}1
{ê}{{\^e}}1
{í}{{\'i}}1
{ó}{{\'o}}1
{õ}{{\~o}}1
{ú}{{\'u}}1
{ü}{{\"u}}1
{ç}{{\c{c}}}1
{~}{{ }}1
}


\definecolor{codegreen}{rgb}{0,0.6,0}
\definecolor{codegray}{rgb}{0.5,0.5,0.5}
\definecolor{codepurple}{rgb}{0.58,0,0.82}
\definecolor{backcolour}{rgb}{0.95,0.95,0.92}

\lstdefinestyle{mystyle}{
    backgroundcolor=\color{backcolour},   
    commentstyle=\color{codegreen},
    keywordstyle=\color{magenta},
    numberstyle=\tiny\color{codegray},
    stringstyle=\color{codepurple},
    basicstyle=\ttfamily\footnotesize,
    breakatwhitespace=false,         
    breaklines=true,                 
    captionpos=b,                    
    keepspaces=true,                 
    numbers=left,                    
xleftmargin=2em,
framexleftmargin=2em,            
    showspaces=false,                
    showstringspaces=false,
    showtabs=false,                  
    tabsize=2,
    upquote=true
}

\lstset{style=mystyle}


\lstset{style=mystyle}
\newcommand{\imgdir}{C:/laragon/www/newmc/assets/imgsvg/}
\newcommand{\imgsvgdir}{C:/laragon/www/newmc/assets/imgsvg/}

\definecolor{mcgris}{RGB}{220, 220, 220}% ancien~; pour compatibilité
\definecolor{mcbleu}{RGB}{52, 152, 219}
\definecolor{mcvert}{RGB}{125, 194, 70}
\definecolor{mcmauve}{RGB}{154, 0, 215}
\definecolor{mcorange}{RGB}{255, 96, 0}
\definecolor{mcturquoise}{RGB}{0, 153, 153}
\definecolor{mcrouge}{RGB}{255, 0, 0}
\definecolor{mclightvert}{RGB}{205, 234, 190}

\definecolor{gris}{RGB}{220, 220, 220}
\definecolor{bleu}{RGB}{52, 152, 219}
\definecolor{vert}{RGB}{125, 194, 70}
\definecolor{mauve}{RGB}{154, 0, 215}
\definecolor{orange}{RGB}{255, 96, 0}
\definecolor{turquoise}{RGB}{0, 153, 153}
\definecolor{rouge}{RGB}{255, 0, 0}
\definecolor{lightvert}{RGB}{205, 234, 190}
\setitemize[0]{label=\color{lightvert}  $\bullet$}

\pagestyle{fancy}
\renewcommand{\headrulewidth}{0.2pt}
\fancyhead[L]{maths-cours.fr}
\fancyhead[R]{\thepage}
\renewcommand{\footrulewidth}{0.2pt}
\fancyfoot[C]{}

\newcolumntype{C}{>{\centering\arraybackslash}X}
\newcolumntype{s}{>{\hsize=.35\hsize\arraybackslash}X}

\setlength{\parindent}{0pt}		 
\setlength{\parskip}{3mm}
\setlength{\headheight}{1cm}

\def\ebook{ebook}
\def\book{book}
\def\web{web}
\def\type{web}

\newcommand{\vect}[1]{\overrightarrow{\,\mathstrut#1\,}}

\def\Oij{$\left(\text{O}~;~\vect{\imath},~\vect{\jmath}\right)$}
\def\Oijk{$\left(\text{O}~;~\vect{\imath},~\vect{\jmath},~\vect{k}\right)$}
\def\Ouv{$\left(\text{O}~;~\vect{u},~\vect{v}\right)$}

\hypersetup{breaklinks=true, colorlinks = true, linkcolor = OliveGreen, urlcolor = OliveGreen, citecolor = OliveGreen, pdfauthor={Didier BONNEL - https://www.maths-cours.fr} } % supprime les bordures autour des liens

\renewcommand{\arg}[0]{\text{arg}}

\everymath{\displaystyle}

%================================================================================================================================
%
% Macros - Commandes
%
%================================================================================================================================

\newcommand\meta[2]{    			% Utilisé pour créer le post HTML.
	\def\titre{titre}
	\def\url{url}
	\def\arg{#1}
	\ifx\titre\arg
		\newcommand\maintitle{#2}
		\fancyhead[L]{#2}
		{\Large\sffamily \MakeUppercase{#2}}
		\vspace{1mm}\textcolor{mcvert}{\hrule}
	\fi 
	\ifx\url\arg
		\fancyfoot[L]{\href{https://www.maths-cours.fr#2}{\black \footnotesize{https://www.maths-cours.fr#2}}}
	\fi 
}


\newcommand\TitreC[1]{    		% Titre centré
     \needspace{3\baselineskip}
     \begin{center}\textbf{#1}\end{center}
}

\newcommand\newpar{    		% paragraphe
     \par
}

\newcommand\nosp {    		% commande vide (pas d'espace)
}
\newcommand{\id}[1]{} %ignore

\newcommand\boite[2]{				% Boite simple sans titre
	\vspace{5mm}
	\setlength{\fboxrule}{0.2mm}
	\setlength{\fboxsep}{5mm}	
	\fcolorbox{#1}{#1!3}{\makebox[\linewidth-2\fboxrule-2\fboxsep]{
  		\begin{minipage}[t]{\linewidth-2\fboxrule-4\fboxsep}\setlength{\parskip}{3mm}
  			 #2
  		\end{minipage}
	}}
	\vspace{5mm}
}

\newcommand\CBox[4]{				% Boites
	\vspace{5mm}
	\setlength{\fboxrule}{0.2mm}
	\setlength{\fboxsep}{5mm}
	
	\fcolorbox{#1}{#1!3}{\makebox[\linewidth-2\fboxrule-2\fboxsep]{
		\begin{minipage}[t]{1cm}\setlength{\parskip}{3mm}
	  		\textcolor{#1}{\LARGE{#2}}    
 	 	\end{minipage}  
  		\begin{minipage}[t]{\linewidth-2\fboxrule-4\fboxsep}\setlength{\parskip}{3mm}
			\raisebox{1.2mm}{\normalsize\sffamily{\textcolor{#1}{#3}}}						
  			 #4
  		\end{minipage}
	}}
	\vspace{5mm}
}

\newcommand\cadre[3]{				% Boites convertible html
	\par
	\vspace{2mm}
	\setlength{\fboxrule}{0.1mm}
	\setlength{\fboxsep}{5mm}
	\fcolorbox{#1}{white}{\makebox[\linewidth-2\fboxrule-2\fboxsep]{
  		\begin{minipage}[t]{\linewidth-2\fboxrule-4\fboxsep}\setlength{\parskip}{3mm}
			\raisebox{-2.5mm}{\sffamily \small{\textcolor{#1}{\MakeUppercase{#2}}}}		
			\par		
  			 #3
 	 		\end{minipage}
	}}
		\vspace{2mm}
	\par
}

\newcommand\bloc[3]{				% Boites convertible html sans bordure
     \needspace{2\baselineskip}
     {\sffamily \small{\textcolor{#1}{\MakeUppercase{#2}}}}    
		\par		
  			 #3
		\par
}

\newcommand\CHelp[1]{
     \CBox{Plum}{\faInfoCircle}{À RETENIR}{#1}
}

\newcommand\CUp[1]{
     \CBox{NavyBlue}{\faThumbsOUp}{EN PRATIQUE}{#1}
}

\newcommand\CInfo[1]{
     \CBox{Sepia}{\faArrowCircleRight}{REMARQUE}{#1}
}

\newcommand\CRedac[1]{
     \CBox{PineGreen}{\faEdit}{BIEN R\'EDIGER}{#1}
}

\newcommand\CError[1]{
     \CBox{Red}{\faExclamationTriangle}{ATTENTION}{#1}
}

\newcommand\TitreExo[2]{
\needspace{4\baselineskip}
 {\sffamily\large EXERCICE #1\ (\emph{#2 points})}
\vspace{5mm}
}

\newcommand\img[2]{
          \includegraphics[width=#2\paperwidth]{\imgdir#1}
}

\newcommand\imgsvg[2]{
       \begin{center}   \includegraphics[width=#2\paperwidth]{\imgsvgdir#1} \end{center}
}


\newcommand\Lien[2]{
     \href{#1}{#2 \tiny \faExternalLink}
}
\newcommand\mcLien[2]{
     \href{https~://www.maths-cours.fr/#1}{#2 \tiny \faExternalLink}
}

\newcommand{\euro}{\eurologo{}}

%================================================================================================================================
%
% Macros - Environement
%
%================================================================================================================================

\newenvironment{tex}{ %
}
{%
}

\newenvironment{indente}{ %
	\setlength\parindent{10mm}
}

{
	\setlength\parindent{0mm}
}

\newenvironment{corrige}{%
     \needspace{3\baselineskip}
     \medskip
     \textbf{\textsc{Corrigé}}
     \medskip
}
{
}

\newenvironment{extern}{%
     \begin{center}
     }
     {
     \end{center}
}

\NewEnviron{code}{%
	\par
     \boite{gray}{\texttt{%
     \BODY
     }}
     \par
}

\newenvironment{vbloc}{% boite sans cadre empeche saut de page
     \begin{minipage}[t]{\linewidth}
     }
     {
     \end{minipage}
}
\NewEnviron{h2}{%
    \needspace{3\baselineskip}
    \vspace{0.6cm}
	\noindent \MakeUppercase{\sffamily \large \BODY}
	\vspace{1mm}\textcolor{mcgris}{\hrule}\vspace{0.4cm}
	\par
}{}

\NewEnviron{h3}{%
    \needspace{3\baselineskip}
	\vspace{5mm}
	\textsc{\BODY}
	\par
}

\NewEnviron{margeneg}{ %
\begin{addmargin}[-1cm]{0cm}
\BODY
\end{addmargin}
}

\NewEnviron{html}{%
}

\begin{document}
\meta{url}{/cours/fonctions-lineaires-et-fonctions-affines/}
\meta{pid}{1568}
\meta{titre}{Fonctions linéaires et fonctions affines}
\meta{type}{cours}
\begin{h2}1. Fonctions linéaires\end{h2}
\cadre{bleu}{Définition}{% id="d10"
     Une fonction \textbf{linéaire} est une fonction définie par une formule du type : $x\mapsto ax$ .
     \par
     $a$ s'appelle le \textbf{coefficient directeur}.
}
\bloc{orange}{Exemple}{% id="e10"
     La fonction qui à tout nombre réel associe son double est une fonction linéaire de coefficient directeur $2$.
     \par
     On la note : $f :  x\mapsto 2x$.
}
\cadre{vert}{Propriété}{% id="p20"
     Pour une fonction linéaire $f$, les valeurs de $f\left(x\right)$ sont \textbf{proportionnelles} aux valeurs de $x$
}
\bloc{orange}{Exemple}{% id="e20"
     Voici un tableau de valeur de la fonction $f :  x\mapsto 2x$ :
     \begin{center}
          \begin{tabularx}{0.6\linewidth}{|*{8}{>{\centering \arraybackslash }X|}}%class="compact" width="600"
               \hline
               $x$ & -3 & -2 & -1 & 0 & 1 & 2 & 3
               \\ \hline
               $f\left(x\right)$ & -6 & -4 & -2 & 0 & 2 & 4 & 6
               \\ \hline
          \end{tabularx}
     \end{center}
     Ce tableau est un tableau de proportionnalité.
}
\cadre{vert}{Propriété}{% id="p30"
     La représentation graphique d'une fonction linéaire est une droite qui passe par le point $O$ origine du repère.
}
\begin{center}
     \begin{extern}%width="380" alt="fonction linéaire"
          \resizebox{7cm}{!}{%
               % -+-+-+ variables modifiables
               \def\fonction{2*x}
               \def\xmin{-3.6}
               \def\xmax{4.6}
               \def\ymin{-4.6}
               \def\ymax{7.6}
               \def\xunit{1.5}  % unités en cm
               \def\yunit{1.5}
               \psset{xunit=\xunit,yunit=\yunit,algebraic=true}
               \fontsize{15pt}{15pt}\selectfont
               \begin{pspicture*}[linewidth=1pt,arrowsize=6pt](\xmin,\ymin)(\xmax,\ymax)
                    \psgrid[gridcolor=mcgris, subgriddiv=1, gridlabels=0pt](\xmin,\ymin)(\xmax,\ymax)
                    \psaxes[linewidth=0.75pt,Dx=1,Dy=1]{->}(0,0)(\xmin,\ymin)(\xmax,\ymax)
                    \psplot[plotpoints=2000,linecolor=blue]{\xmin}{\xmax}{\fonction}
                    \rput[tr](-0.2,-0.2){$O$}
               \end{pspicture*}
          }
     \end{extern}
\end{center}
\begin{center}\textit{Représentation graphique de la fonction linéaire $x\mapsto 2x$}\end{center}
\begin{h2}2. Fonctions affines\end{h2}
\cadre{bleu}{Définition}{% id="d50"
     Une fonction \textbf{affine} est une fonction définie par une formule du type : $x\mapsto ax+b$.
     \par
     $a$ s'appelle le \textbf{coefficient directeur }et $b$ s'appelle \textbf{l'ordonnée à l'origine}.
}
\bloc{cyan}{Remarques}{% id="r50"
     Si $b=0$, la fonction est linéaire. Les fonctions linéaires sont donc des cas particuliers des fonctions affines.
}
\bloc{orange}{Exemple}{% id="e50"
     La fonction $f : x\mapsto -2x+1$ est une fonction affine avec $a=-2$ et $b=1$
}
\cadre{rouge}{Théorème}{% id="t60"
     Soit une fonction affine $f : x\mapsto ax+b$.
     \par
     Pour tous nombres réels distincts $x_{1}$ et $x_{2}$, le coefficient directeur $a$ est égal à :
     \begin{center}
          $a=\frac{f\left(x_{2}\right)-f\left(x_{1}\right)}{x_{2}-x_{1}}$
     \end{center}
}
\bloc{orange}{Exercice corrigé}{% id="e60"
     \textbf{Déterminer la fonction affine $f$ telle $f\left(2\right)=1$ et $f\left(4\right)=5$.}
     \par
     $f$ étant une fonction affine, la formule donnant $f\left(x\right)$ est de la forme $f\left(x\right)=ax+b$.
     \par
     D'après le théorème précédent, le coefficient directeur $a$ est égal à :
     \par
     $a=\frac{f\left(4\right)-f\left(2\right)}{4-2}=\frac{5-1}{4-2}=\frac{4}{2}=2$
     \par
     On a donc $f\left(x\right)=2x+b$
     \par
     Pour trouver la valeur de $b$, on utilise le fait que $f\left(2\right)=1$ donc $2\times 2+b=1$.
     \par
     $4+b=1$
     \par
     $b=1-4$
     \par
     $b=-3$
     \par
     Par conséquent $f$ est définie par $f\left(x\right)=2x-3$
}
\cadre{vert}{Propriété}{% id="p70"
     La représentation graphique d'une fonction affine est une droite.
}
\bloc{cyan}{Remarque}{% id="r70"
     Pour tracer une droite, il suffit de connaître deux points de cette droite. Il suffit donc de calculer les images de deux nombres pour tracer la représentation graphique d'une fonction affine.
}
\bloc{orange}{Exemple}{% id="e70"
     On veut tracer la représentation graphique de la fonction $f : x\mapsto -2x+1$.
     \par
     Cette représentation graphique est une droite.
     \begin{itemize}
          \item comme $f\left(0\right)=-2\times 0+1=1$, cette droite passe par le point $A\left(0;1\right)$
          \item comme $f\left(1\right)=-2\times 1+1=-1$, cette droite passe par le point $B\left(1;-1\right)$
     \end{itemize}
     On en déduit la représentation ci-dessous :
     \begin{center}
          \begin{extern}%width="380" alt="fonction affine"
               \resizebox{7cm}{!}{%
                    % -+-+-+ variables modifiables
                    \def\fonction{1-2*x}
                    \def\xmin{-3.6}
                    \def\xmax{4.6}
                    \def\ymin{-4.6}
                    \def\ymax{7.6}
                    \def\xunit{1.5}  % unités en cm
                    \def\yunit{1.5}
                    \psset{xunit=\xunit,yunit=\yunit,algebraic=true}
                    \fontsize{15pt}{15pt}\selectfont
                    \begin{pspicture*}[linewidth=1pt,arrowsize=6pt](\xmin,\ymin)(\xmax,\ymax)
                         \psgrid[gridcolor=mcgris, subgriddiv=1, gridlabels=0pt](\xmin,\ymin)(\xmax,\ymax)
                         \psaxes[linewidth=0.75pt,Dx=1,Dy=1]{->}(0,0)(\xmin,\ymin)(\xmax,\ymax)
                         \psplot[plotpoints=2000,linecolor=blue]{\xmin}{\xmax}{\fonction}
                         \psdots[linecolor=red,dotsize=4pt,dotstyle=*](0,1)(1,-1)
                         \rput[tr](-0.2,-0.2){$O$}
                         \rput[bl](0.1,1.1){$\red A$}
                         \rput[bl](1.1,-0.9){$\red B$}
                    \end{pspicture*}
               }
          \end{extern}
     \end{center}
     \begin{center}\textit{Représentation graphique de la fonction $x\mapsto -2x+1$}\end{center}
}

\end{document}
µ
\documentclass[a4paper]{article}

%================================================================================================================================
%
% Packages
%
%================================================================================================================================

\usepackage[T1]{fontenc} 	% pour caractères accentués
\usepackage[utf8]{inputenc}  % encodage utf8
\usepackage[french]{babel}	% langue : français
\usepackage{fourier}			% caractères plus lisibles
\usepackage[dvipsnames]{xcolor} % couleurs
\usepackage{fancyhdr}		% réglage header footer
\usepackage{needspace}		% empêcher sauts de page mal placés
\usepackage{graphicx}		% pour inclure des graphiques
\usepackage{enumitem,cprotect}		% personnalise les listes d'items (nécessaire pour ol, al ...)
\usepackage{hyperref}		% Liens hypertexte
\usepackage{pstricks,pst-all,pst-node,pstricks-add,pst-math,pst-plot,pst-tree,pst-eucl} % pstricks
\usepackage[a4paper,includeheadfoot,top=2cm,left=3cm, bottom=2cm,right=3cm]{geometry} % marges etc.
\usepackage{comment}			% commentaires multilignes
\usepackage{amsmath,environ} % maths (matrices, etc.)
\usepackage{amssymb,makeidx}
\usepackage{bm}				% bold maths
\usepackage{tabularx}		% tableaux
\usepackage{colortbl}		% tableaux en couleur
\usepackage{fontawesome}		% Fontawesome
\usepackage{environ}			% environment with command
\usepackage{fp}				% calculs pour ps-tricks
\usepackage{multido}			% pour ps tricks
\usepackage[np]{numprint}	% formattage nombre
\usepackage{tikz,tkz-tab} 			% package principal TikZ
\usepackage{pgfplots}   % axes
\usepackage{mathrsfs}    % cursives
\usepackage{calc}			% calcul taille boites
\usepackage[scaled=0.875]{helvet} % font sans serif
\usepackage{svg} % svg
\usepackage{scrextend} % local margin
\usepackage{scratch} %scratch
\usepackage{multicol} % colonnes
%\usepackage{infix-RPN,pst-func} % formule en notation polanaise inversée
\usepackage{listings}

%================================================================================================================================
%
% Réglages de base
%
%================================================================================================================================

\lstset{
language=Python,   % R code
literate=
{á}{{\'a}}1
{à}{{\`a}}1
{ã}{{\~a}}1
{é}{{\'e}}1
{è}{{\`e}}1
{ê}{{\^e}}1
{í}{{\'i}}1
{ó}{{\'o}}1
{õ}{{\~o}}1
{ú}{{\'u}}1
{ü}{{\"u}}1
{ç}{{\c{c}}}1
{~}{{ }}1
}


\definecolor{codegreen}{rgb}{0,0.6,0}
\definecolor{codegray}{rgb}{0.5,0.5,0.5}
\definecolor{codepurple}{rgb}{0.58,0,0.82}
\definecolor{backcolour}{rgb}{0.95,0.95,0.92}

\lstdefinestyle{mystyle}{
    backgroundcolor=\color{backcolour},   
    commentstyle=\color{codegreen},
    keywordstyle=\color{magenta},
    numberstyle=\tiny\color{codegray},
    stringstyle=\color{codepurple},
    basicstyle=\ttfamily\footnotesize,
    breakatwhitespace=false,         
    breaklines=true,                 
    captionpos=b,                    
    keepspaces=true,                 
    numbers=left,                    
xleftmargin=2em,
framexleftmargin=2em,            
    showspaces=false,                
    showstringspaces=false,
    showtabs=false,                  
    tabsize=2,
    upquote=true
}

\lstset{style=mystyle}


\lstset{style=mystyle}
\newcommand{\imgdir}{C:/laragon/www/newmc/assets/imgsvg/}
\newcommand{\imgsvgdir}{C:/laragon/www/newmc/assets/imgsvg/}

\definecolor{mcgris}{RGB}{220, 220, 220}% ancien~; pour compatibilité
\definecolor{mcbleu}{RGB}{52, 152, 219}
\definecolor{mcvert}{RGB}{125, 194, 70}
\definecolor{mcmauve}{RGB}{154, 0, 215}
\definecolor{mcorange}{RGB}{255, 96, 0}
\definecolor{mcturquoise}{RGB}{0, 153, 153}
\definecolor{mcrouge}{RGB}{255, 0, 0}
\definecolor{mclightvert}{RGB}{205, 234, 190}

\definecolor{gris}{RGB}{220, 220, 220}
\definecolor{bleu}{RGB}{52, 152, 219}
\definecolor{vert}{RGB}{125, 194, 70}
\definecolor{mauve}{RGB}{154, 0, 215}
\definecolor{orange}{RGB}{255, 96, 0}
\definecolor{turquoise}{RGB}{0, 153, 153}
\definecolor{rouge}{RGB}{255, 0, 0}
\definecolor{lightvert}{RGB}{205, 234, 190}
\setitemize[0]{label=\color{lightvert}  $\bullet$}

\pagestyle{fancy}
\renewcommand{\headrulewidth}{0.2pt}
\fancyhead[L]{maths-cours.fr}
\fancyhead[R]{\thepage}
\renewcommand{\footrulewidth}{0.2pt}
\fancyfoot[C]{}

\newcolumntype{C}{>{\centering\arraybackslash}X}
\newcolumntype{s}{>{\hsize=.35\hsize\arraybackslash}X}

\setlength{\parindent}{0pt}		 
\setlength{\parskip}{3mm}
\setlength{\headheight}{1cm}

\def\ebook{ebook}
\def\book{book}
\def\web{web}
\def\type{web}

\newcommand{\vect}[1]{\overrightarrow{\,\mathstrut#1\,}}

\def\Oij{$\left(\text{O}~;~\vect{\imath},~\vect{\jmath}\right)$}
\def\Oijk{$\left(\text{O}~;~\vect{\imath},~\vect{\jmath},~\vect{k}\right)$}
\def\Ouv{$\left(\text{O}~;~\vect{u},~\vect{v}\right)$}

\hypersetup{breaklinks=true, colorlinks = true, linkcolor = OliveGreen, urlcolor = OliveGreen, citecolor = OliveGreen, pdfauthor={Didier BONNEL - https://www.maths-cours.fr} } % supprime les bordures autour des liens

\renewcommand{\arg}[0]{\text{arg}}

\everymath{\displaystyle}

%================================================================================================================================
%
% Macros - Commandes
%
%================================================================================================================================

\newcommand\meta[2]{    			% Utilisé pour créer le post HTML.
	\def\titre{titre}
	\def\url{url}
	\def\arg{#1}
	\ifx\titre\arg
		\newcommand\maintitle{#2}
		\fancyhead[L]{#2}
		{\Large\sffamily \MakeUppercase{#2}}
		\vspace{1mm}\textcolor{mcvert}{\hrule}
	\fi 
	\ifx\url\arg
		\fancyfoot[L]{\href{https://www.maths-cours.fr#2}{\black \footnotesize{https://www.maths-cours.fr#2}}}
	\fi 
}


\newcommand\TitreC[1]{    		% Titre centré
     \needspace{3\baselineskip}
     \begin{center}\textbf{#1}\end{center}
}

\newcommand\newpar{    		% paragraphe
     \par
}

\newcommand\nosp {    		% commande vide (pas d'espace)
}
\newcommand{\id}[1]{} %ignore

\newcommand\boite[2]{				% Boite simple sans titre
	\vspace{5mm}
	\setlength{\fboxrule}{0.2mm}
	\setlength{\fboxsep}{5mm}	
	\fcolorbox{#1}{#1!3}{\makebox[\linewidth-2\fboxrule-2\fboxsep]{
  		\begin{minipage}[t]{\linewidth-2\fboxrule-4\fboxsep}\setlength{\parskip}{3mm}
  			 #2
  		\end{minipage}
	}}
	\vspace{5mm}
}

\newcommand\CBox[4]{				% Boites
	\vspace{5mm}
	\setlength{\fboxrule}{0.2mm}
	\setlength{\fboxsep}{5mm}
	
	\fcolorbox{#1}{#1!3}{\makebox[\linewidth-2\fboxrule-2\fboxsep]{
		\begin{minipage}[t]{1cm}\setlength{\parskip}{3mm}
	  		\textcolor{#1}{\LARGE{#2}}    
 	 	\end{minipage}  
  		\begin{minipage}[t]{\linewidth-2\fboxrule-4\fboxsep}\setlength{\parskip}{3mm}
			\raisebox{1.2mm}{\normalsize\sffamily{\textcolor{#1}{#3}}}						
  			 #4
  		\end{minipage}
	}}
	\vspace{5mm}
}

\newcommand\cadre[3]{				% Boites convertible html
	\par
	\vspace{2mm}
	\setlength{\fboxrule}{0.1mm}
	\setlength{\fboxsep}{5mm}
	\fcolorbox{#1}{white}{\makebox[\linewidth-2\fboxrule-2\fboxsep]{
  		\begin{minipage}[t]{\linewidth-2\fboxrule-4\fboxsep}\setlength{\parskip}{3mm}
			\raisebox{-2.5mm}{\sffamily \small{\textcolor{#1}{\MakeUppercase{#2}}}}		
			\par		
  			 #3
 	 		\end{minipage}
	}}
		\vspace{2mm}
	\par
}

\newcommand\bloc[3]{				% Boites convertible html sans bordure
     \needspace{2\baselineskip}
     {\sffamily \small{\textcolor{#1}{\MakeUppercase{#2}}}}    
		\par		
  			 #3
		\par
}

\newcommand\CHelp[1]{
     \CBox{Plum}{\faInfoCircle}{À RETENIR}{#1}
}

\newcommand\CUp[1]{
     \CBox{NavyBlue}{\faThumbsOUp}{EN PRATIQUE}{#1}
}

\newcommand\CInfo[1]{
     \CBox{Sepia}{\faArrowCircleRight}{REMARQUE}{#1}
}

\newcommand\CRedac[1]{
     \CBox{PineGreen}{\faEdit}{BIEN R\'EDIGER}{#1}
}

\newcommand\CError[1]{
     \CBox{Red}{\faExclamationTriangle}{ATTENTION}{#1}
}

\newcommand\TitreExo[2]{
\needspace{4\baselineskip}
 {\sffamily\large EXERCICE #1\ (\emph{#2 points})}
\vspace{5mm}
}

\newcommand\img[2]{
          \includegraphics[width=#2\paperwidth]{\imgdir#1}
}

\newcommand\imgsvg[2]{
       \begin{center}   \includegraphics[width=#2\paperwidth]{\imgsvgdir#1} \end{center}
}


\newcommand\Lien[2]{
     \href{#1}{#2 \tiny \faExternalLink}
}
\newcommand\mcLien[2]{
     \href{https~://www.maths-cours.fr/#1}{#2 \tiny \faExternalLink}
}

\newcommand{\euro}{\eurologo{}}

%================================================================================================================================
%
% Macros - Environement
%
%================================================================================================================================

\newenvironment{tex}{ %
}
{%
}

\newenvironment{indente}{ %
	\setlength\parindent{10mm}
}

{
	\setlength\parindent{0mm}
}

\newenvironment{corrige}{%
     \needspace{3\baselineskip}
     \medskip
     \textbf{\textsc{Corrigé}}
     \medskip
}
{
}

\newenvironment{extern}{%
     \begin{center}
     }
     {
     \end{center}
}

\NewEnviron{code}{%
	\par
     \boite{gray}{\texttt{%
     \BODY
     }}
     \par
}

\newenvironment{vbloc}{% boite sans cadre empeche saut de page
     \begin{minipage}[t]{\linewidth}
     }
     {
     \end{minipage}
}
\NewEnviron{h2}{%
    \needspace{3\baselineskip}
    \vspace{0.6cm}
	\noindent \MakeUppercase{\sffamily \large \BODY}
	\vspace{1mm}\textcolor{mcgris}{\hrule}\vspace{0.4cm}
	\par
}{}

\NewEnviron{h3}{%
    \needspace{3\baselineskip}
	\vspace{5mm}
	\textsc{\BODY}
	\par
}

\NewEnviron{margeneg}{ %
\begin{addmargin}[-1cm]{0cm}
\BODY
\end{addmargin}
}

\NewEnviron{html}{%
}

\begin{document}
\meta{url}{/cours/theoreme-pythagore/}
\meta{pid}{1571}
\meta{titre}{Théorème de Pythagore -Trigonométrie}
\meta{type}{cours}
\begin{h2}1. Théorème de Pythagore (rappels de 4ème)\end{h2}
\cadre{rouge}{Théorème de Pythagore}{% id="d10"
     Si un triangle est rectangle alors le carré de la longueur de l'hypoténuse est égal à la somme des carrés des longueurs des côtés de l'angle droit.
}
\bloc{cyan}{Remarque}{% id="r10"
     \begin{itemize}
          \item On rappelle que l'hypoténuse est le côté opposé à l'angle droit et le côté ayant la plus grande longueur.
          \item Ce théorème sert à calculer la longueur d'un côté connaissant les longueurs des deux autres lorsque l'on \textbf{sait} que le triangle est rectangle
     \end{itemize}
}
\bloc{orange}{Exemple}{% id="e10"
     Soit $ABC$ un triangle rectangle en $A$ tel que $AB=4$cm et $AC=3$cm
     \begin{center}
          \begin{extern}%width="200" alt="sinus cosinus tangente"
               \psset{xunit=1.0cm,yunit=1.0cm,algebraic=true,dimen=middle,dotstyle=*,dotsize=5pt 0,linewidth=1.pt,arrowsize=3pt 2,arrowinset=0.25}
               \begin{pspicture*}(0.,0.5)(6.,5)
                    \psframe[linewidth=0.6pt](1,1)(1.2,1.2)
                    \psline(1.,1.)(5.,1.)
                    \psline(1.,4.)(5.,1.)
                    \psline(1.,4.)(1.,1.)
                    \rput[t](3,0.9){$4$}
                    \rput[r](0.9,2.5){$3$}
                    \rput[bl](0.6,0.8){$A$}
                    \rput[bl](5.1,0.84){$B$}
                    \rput[bl](0.6,4.07){$C$}
               \end{pspicture*}
          \end{extern}
     \end{center}
     D'après le théorème de Pythagore :
     \par
     $ BC^{2}=AB^{2}+AC^{2}=4^{2}+3^{2}=16+9=25$
     \par
     Donc $BC=\sqrt{25}=5$cm.
}
\cadre{rouge}{Théorème (Réciproque du théorème de Pythagore)}{% id="t20"
     Un triangle est rectangle si et seulement si le carré de la longueur du plus grand coté est égal à la somme des carrés des longueurs des deux autres côtés.
}
\bloc{cyan}{Remarques}{% id="r20"
     Ce théorème sert à \textbf{démontrer} qu'un triangle est un triangle rectangle lorsqu'on connait les longueurs de ses trois côtés.
}
\bloc{orange}{Exemple}{% id="e20"
     Soit $ABC$ un triangle tel que $AB=12$cm,  $AC=5$cm et $BC=13$cm.
     \par
     $ABC$ est-il rectangle ?
     \par
     On calcule séparément $BC^{2}$ (carré de la longueur du plus grand coté) et $AB^{2}+AC^{2}$ (somme des carrés des longueurs des deux autres cotés) :
     \par
     $BC^{2}=13^{2}=169$
     \par
     $AB^{2}+AC^{2}=12^{2}+5^{2}=144+25=169$
     \par
     $BC^{2} = AB^{2}+AC^{2}$ donc le triangle $ABC$ est rectangle en $A$ d'après la réciproque du théorème de Pythagore.
}
\begin{h2}2. Trigonométrie\end{h2}
\cadre{bleu}{Définitions}{% id="d50"
     Soit $ABC$ un triangle rectangle en $A$ :
     \begin{itemize}
          \item le \textbf{sinus} de $\widehat{ABC}$ est le nombre :
          \par
          $\sin\left(\widehat{ABC}\right)=$\nosp$\frac{\text{longueur\ du\ côté\ opposé\ à\ B}}{\text{longueur\ de\ l'hypoténuse}}$
          \item le \textbf{cosinus} de $\widehat{ABC}$ est le nombre :
          \par
          $\cos\left(\widehat{ABC}\right)=$\nosp$\frac{\text{longueur\ du\ côté\ adjacent\ à\ B}}{\text{longueur\ de\ l'hypoténuse}}$
          \item la \textbf{tangente} de $\widehat{ABC}$ est le nombre :
          \par
          $\tan\left(\widehat{ABC}\right)=$\nosp$\frac{\text{longueur\ du\ côté\ opposé\ à\ B}}{\text{longueur\ du\ côté\ adjacent\ à\ B}}$
     \end{itemize}
}
\bloc{orange}{Exemple}{% id="e50"
     \begin{center}
          \begin{extern}%width="200" alt="sinus cosinus tangente"
               \psset{xunit=1.0cm,yunit=1.0cm,algebraic=true,dimen=middle,dotstyle=*,dotsize=5pt 0,linewidth=1.pt,arrowsize=3pt 2,arrowinset=0.25}
               \begin{pspicture*}(0.,0.5)(6.,5)
                    \psframe[linewidth=0.6pt](1,1)(1.2,1.2)
                    \pscustom[linewidth=0.4pt,linecolor=mcvert,fillcolor=mcvert,fillstyle=solid,opacity=0.1]{
                         \parametricplot{2.5}{3.1416}{0.66*cos(t)+5.|0.66*sin(t)+1.}
                    \lineto(5.,1.)\closepath}
                    \psline(1.,1.)(5.,1.)
                    \psline(1.,4.)(5.,1.)
                    \psline(1.,4.)(1.,1.)
                    \rput[t](3,0.9){$4$}
                    \rput[l](3.2,2.6){$5$}
                    \rput[r](0.9,2.5){$3$}
                    \rput[bl](0.6,0.8){$A$}
                    \rput[bl](5.1,0.84){$B$}
                    \rput[bl](0.6,4.07){$C$}
               \end{pspicture*}
          \end{extern}
     \end{center}
     Dans le triangle rectangle $ABC$ ci-dessus :
     \begin{itemize}
          \item $\sin\left(\widehat{ABC}\right)=\frac{AC}{BC}=\frac{3}{5}=0,6$
          \item $\cos\left(\widehat{ABC}\right)=\frac{AB}{BC}=\frac{4}{5}=0,8$
          \item $\tan\left(\widehat{ABC}\right)=\frac{AC}{AB}=\frac{3}{4}=0,75$
     \end{itemize}
}
\bloc{cyan}{Remarques}{% id="r50"
     \begin{itemize}
          \item Les sinus, cosinus et tangente n'ont pas d'unité !
          \item Les sinus et cosinus d'un angle aigu sont compris entre 0 et 1. Par contre, la tangente peut être supérieure à 1.
          \item Connaissant le sinus, il est possible de calculer la mesure de l'angle en degré à la calculatrice à l'aide de la touche \textbf{$\sin^{-1}$} (ou \textbf{Arcsin} ou \textbf{asin} suivant le modèle de la calculatrice). Vérifiez bien que la calculatrice est en mode \textbf{degré} !
     \end{itemize}
}
\cadre{vert}{Propriétés}{% id="p60"
     Pour tout angle aigu $\widehat{a}$ d'un triangle rectangle :
     \begin{center}$\left(\cos \widehat{a}\right)^{2}+\left(\sin \widehat{a}\right)^{2}=1$\end{center}
     \begin{center}$\tan \widehat{a}=\frac{\sin \widehat{a}}{\cos \widehat{a}}$\end{center}
}
\bloc{cyan}{Remarque}{% id="r60"
     Pour simplifier les notations, on écrit en général $\cos^{2} \widehat{a}$ pour $\left(\cos \widehat{a}\right)^{2}$. La première formule s'écrit alors :
     \par
     $\cos^{2} \widehat{a}+\sin^{2} \widehat{a}=1$
}
\bloc{cyan}{Démonstrations}{% id="m60"
     \begin{center}
          \begin{extern}%width="200" alt=""
               \psset{xunit=1.0cm,yunit=1.0cm,algebraic=true,dimen=middle,dotstyle=*,dotsize=5pt 0,linewidth=1.pt,arrowsize=3pt 2,arrowinset=0.25}
               \begin{pspicture*}(0.,0.5)(6.,5)
                    \psframe[linewidth=0.6pt](1,1)(1.2,1.2)
                    \pscustom[linewidth=0.4pt,linecolor=mcvert,fillcolor=mcvert,fillstyle=solid,opacity=0.1]{
                         \parametricplot{2.5}{3.1416}{0.66*cos(t)+5.|0.66*sin(t)+1.}
                    \lineto(5.,1.)\closepath}
                    \rput[tl](4.05,1.45){\color{mcvert}{$\hat a$}}
                    \psline(1.,1.)(5.,1.)
                    \psline(1.,4.)(5.,1.)
                    \psline(1.,4.)(1.,1.)
                    \rput[bl](0.6,0.8){$A$}
                    \rput[bl](5.1,0.84){$B$}
                    \rput[bl](0.6,4.07){$C$}
               \end{pspicture*}
          \end{extern}
     \end{center}
     \begin{itemize}
          \item $\cos \widehat{a}=\frac{AB}{BC}$ donc $\left(\cos \widehat{a}\right)^{2}=\frac{AB^{2}}{BC^{2}}$
          \par
          $\sin \widehat{a}=\frac{AC}{BC}$ donc $\left(\sin \widehat{a}\right)^{2}=\frac{AC^{2}}{BC^{2}}$
          \par
          Par conséquent :
          \par
          $\left(\cos \widehat{a}\right)^{2}+\left(\sin \widehat{a}\right)^{2}=\frac{AB^{2}}{BC^{2}}+\frac{AC^{2}}{BC^{2}}=\frac{AB^{2}+AC^{2}}{BC^{2}}$
          \par
          Or d'après le théorème de Pythagore $AB^{2}+AC^{2}=BC^{2}$ donc :
          \par
          $\left(\cos \widehat{a}\right)^{2}+\left(\sin \widehat{a}\right)^{2}=\frac{BC^{2}}{BC^{2}}=1$ après simplification par $BC^{2}$
          \par
          \item $\frac{\sin \widehat{a}}{\cos \widehat{a}}=\frac{\frac{AC}{BC}}{\frac{AB}{BC}}=\frac{AC}{BC}\times \frac{BC}{AB}=\frac{AC}{AB}$ après simplification par $BC$.
          \par
          Or, $\frac{AC}{AB}=\tan \widehat{a}$, par conséquent :
          \par
          $\tan \widehat{a}=\frac{\sin \widehat{a}}{\cos \widehat{a}}.$
     \end{itemize}
}
\bloc{orange}{Exemple}{% id="e60"
     On sait que le cosinus d'un angle $\widehat{a}$ vaut $0,5$. Calculer une valeur approchée à $10^{-2}$ du sinus puis de la tangente de cet angle.
     \par
     $\cos^{2} \widehat{a}+\sin^{2} \widehat{a}=1$
     \par
     $\sin^{2} \widehat{a}=1-\cos^{2}\widehat{a}=1-0,5^{2}=0,75$
     \par
     $\sin \widehat{a}=\sqrt{0,75}\approx 0,87$ à $10^{-2}$ près
     \par
     $\tan \widehat{a}=\frac{\sin \widehat{a}}{\cos \widehat{a}}=\frac{\sqrt{0,75}}{0,5}\approx 1,73$ à $10^{-2}$ près.
}

\end{document}
µ
\documentclass[a4paper]{article}

%================================================================================================================================
%
% Packages
%
%================================================================================================================================

\usepackage[T1]{fontenc} 	% pour caractères accentués
\usepackage[utf8]{inputenc}  % encodage utf8
\usepackage[french]{babel}	% langue : français
\usepackage{fourier}			% caractères plus lisibles
\usepackage[dvipsnames]{xcolor} % couleurs
\usepackage{fancyhdr}		% réglage header footer
\usepackage{needspace}		% empêcher sauts de page mal placés
\usepackage{graphicx}		% pour inclure des graphiques
\usepackage{enumitem,cprotect}		% personnalise les listes d'items (nécessaire pour ol, al ...)
\usepackage{hyperref}		% Liens hypertexte
\usepackage{pstricks,pst-all,pst-node,pstricks-add,pst-math,pst-plot,pst-tree,pst-eucl} % pstricks
\usepackage[a4paper,includeheadfoot,top=2cm,left=3cm, bottom=2cm,right=3cm]{geometry} % marges etc.
\usepackage{comment}			% commentaires multilignes
\usepackage{amsmath,environ} % maths (matrices, etc.)
\usepackage{amssymb,makeidx}
\usepackage{bm}				% bold maths
\usepackage{tabularx}		% tableaux
\usepackage{colortbl}		% tableaux en couleur
\usepackage{fontawesome}		% Fontawesome
\usepackage{environ}			% environment with command
\usepackage{fp}				% calculs pour ps-tricks
\usepackage{multido}			% pour ps tricks
\usepackage[np]{numprint}	% formattage nombre
\usepackage{tikz,tkz-tab} 			% package principal TikZ
\usepackage{pgfplots}   % axes
\usepackage{mathrsfs}    % cursives
\usepackage{calc}			% calcul taille boites
\usepackage[scaled=0.875]{helvet} % font sans serif
\usepackage{svg} % svg
\usepackage{scrextend} % local margin
\usepackage{scratch} %scratch
\usepackage{multicol} % colonnes
%\usepackage{infix-RPN,pst-func} % formule en notation polanaise inversée
\usepackage{listings}

%================================================================================================================================
%
% Réglages de base
%
%================================================================================================================================

\lstset{
language=Python,   % R code
literate=
{á}{{\'a}}1
{à}{{\`a}}1
{ã}{{\~a}}1
{é}{{\'e}}1
{è}{{\`e}}1
{ê}{{\^e}}1
{í}{{\'i}}1
{ó}{{\'o}}1
{õ}{{\~o}}1
{ú}{{\'u}}1
{ü}{{\"u}}1
{ç}{{\c{c}}}1
{~}{{ }}1
}


\definecolor{codegreen}{rgb}{0,0.6,0}
\definecolor{codegray}{rgb}{0.5,0.5,0.5}
\definecolor{codepurple}{rgb}{0.58,0,0.82}
\definecolor{backcolour}{rgb}{0.95,0.95,0.92}

\lstdefinestyle{mystyle}{
    backgroundcolor=\color{backcolour},   
    commentstyle=\color{codegreen},
    keywordstyle=\color{magenta},
    numberstyle=\tiny\color{codegray},
    stringstyle=\color{codepurple},
    basicstyle=\ttfamily\footnotesize,
    breakatwhitespace=false,         
    breaklines=true,                 
    captionpos=b,                    
    keepspaces=true,                 
    numbers=left,                    
xleftmargin=2em,
framexleftmargin=2em,            
    showspaces=false,                
    showstringspaces=false,
    showtabs=false,                  
    tabsize=2,
    upquote=true
}

\lstset{style=mystyle}


\lstset{style=mystyle}
\newcommand{\imgdir}{C:/laragon/www/newmc/assets/imgsvg/}
\newcommand{\imgsvgdir}{C:/laragon/www/newmc/assets/imgsvg/}

\definecolor{mcgris}{RGB}{220, 220, 220}% ancien~; pour compatibilité
\definecolor{mcbleu}{RGB}{52, 152, 219}
\definecolor{mcvert}{RGB}{125, 194, 70}
\definecolor{mcmauve}{RGB}{154, 0, 215}
\definecolor{mcorange}{RGB}{255, 96, 0}
\definecolor{mcturquoise}{RGB}{0, 153, 153}
\definecolor{mcrouge}{RGB}{255, 0, 0}
\definecolor{mclightvert}{RGB}{205, 234, 190}

\definecolor{gris}{RGB}{220, 220, 220}
\definecolor{bleu}{RGB}{52, 152, 219}
\definecolor{vert}{RGB}{125, 194, 70}
\definecolor{mauve}{RGB}{154, 0, 215}
\definecolor{orange}{RGB}{255, 96, 0}
\definecolor{turquoise}{RGB}{0, 153, 153}
\definecolor{rouge}{RGB}{255, 0, 0}
\definecolor{lightvert}{RGB}{205, 234, 190}
\setitemize[0]{label=\color{lightvert}  $\bullet$}

\pagestyle{fancy}
\renewcommand{\headrulewidth}{0.2pt}
\fancyhead[L]{maths-cours.fr}
\fancyhead[R]{\thepage}
\renewcommand{\footrulewidth}{0.2pt}
\fancyfoot[C]{}

\newcolumntype{C}{>{\centering\arraybackslash}X}
\newcolumntype{s}{>{\hsize=.35\hsize\arraybackslash}X}

\setlength{\parindent}{0pt}		 
\setlength{\parskip}{3mm}
\setlength{\headheight}{1cm}

\def\ebook{ebook}
\def\book{book}
\def\web{web}
\def\type{web}

\newcommand{\vect}[1]{\overrightarrow{\,\mathstrut#1\,}}

\def\Oij{$\left(\text{O}~;~\vect{\imath},~\vect{\jmath}\right)$}
\def\Oijk{$\left(\text{O}~;~\vect{\imath},~\vect{\jmath},~\vect{k}\right)$}
\def\Ouv{$\left(\text{O}~;~\vect{u},~\vect{v}\right)$}

\hypersetup{breaklinks=true, colorlinks = true, linkcolor = OliveGreen, urlcolor = OliveGreen, citecolor = OliveGreen, pdfauthor={Didier BONNEL - https://www.maths-cours.fr} } % supprime les bordures autour des liens

\renewcommand{\arg}[0]{\text{arg}}

\everymath{\displaystyle}

%================================================================================================================================
%
% Macros - Commandes
%
%================================================================================================================================

\newcommand\meta[2]{    			% Utilisé pour créer le post HTML.
	\def\titre{titre}
	\def\url{url}
	\def\arg{#1}
	\ifx\titre\arg
		\newcommand\maintitle{#2}
		\fancyhead[L]{#2}
		{\Large\sffamily \MakeUppercase{#2}}
		\vspace{1mm}\textcolor{mcvert}{\hrule}
	\fi 
	\ifx\url\arg
		\fancyfoot[L]{\href{https://www.maths-cours.fr#2}{\black \footnotesize{https://www.maths-cours.fr#2}}}
	\fi 
}


\newcommand\TitreC[1]{    		% Titre centré
     \needspace{3\baselineskip}
     \begin{center}\textbf{#1}\end{center}
}

\newcommand\newpar{    		% paragraphe
     \par
}

\newcommand\nosp {    		% commande vide (pas d'espace)
}
\newcommand{\id}[1]{} %ignore

\newcommand\boite[2]{				% Boite simple sans titre
	\vspace{5mm}
	\setlength{\fboxrule}{0.2mm}
	\setlength{\fboxsep}{5mm}	
	\fcolorbox{#1}{#1!3}{\makebox[\linewidth-2\fboxrule-2\fboxsep]{
  		\begin{minipage}[t]{\linewidth-2\fboxrule-4\fboxsep}\setlength{\parskip}{3mm}
  			 #2
  		\end{minipage}
	}}
	\vspace{5mm}
}

\newcommand\CBox[4]{				% Boites
	\vspace{5mm}
	\setlength{\fboxrule}{0.2mm}
	\setlength{\fboxsep}{5mm}
	
	\fcolorbox{#1}{#1!3}{\makebox[\linewidth-2\fboxrule-2\fboxsep]{
		\begin{minipage}[t]{1cm}\setlength{\parskip}{3mm}
	  		\textcolor{#1}{\LARGE{#2}}    
 	 	\end{minipage}  
  		\begin{minipage}[t]{\linewidth-2\fboxrule-4\fboxsep}\setlength{\parskip}{3mm}
			\raisebox{1.2mm}{\normalsize\sffamily{\textcolor{#1}{#3}}}						
  			 #4
  		\end{minipage}
	}}
	\vspace{5mm}
}

\newcommand\cadre[3]{				% Boites convertible html
	\par
	\vspace{2mm}
	\setlength{\fboxrule}{0.1mm}
	\setlength{\fboxsep}{5mm}
	\fcolorbox{#1}{white}{\makebox[\linewidth-2\fboxrule-2\fboxsep]{
  		\begin{minipage}[t]{\linewidth-2\fboxrule-4\fboxsep}\setlength{\parskip}{3mm}
			\raisebox{-2.5mm}{\sffamily \small{\textcolor{#1}{\MakeUppercase{#2}}}}		
			\par		
  			 #3
 	 		\end{minipage}
	}}
		\vspace{2mm}
	\par
}

\newcommand\bloc[3]{				% Boites convertible html sans bordure
     \needspace{2\baselineskip}
     {\sffamily \small{\textcolor{#1}{\MakeUppercase{#2}}}}    
		\par		
  			 #3
		\par
}

\newcommand\CHelp[1]{
     \CBox{Plum}{\faInfoCircle}{À RETENIR}{#1}
}

\newcommand\CUp[1]{
     \CBox{NavyBlue}{\faThumbsOUp}{EN PRATIQUE}{#1}
}

\newcommand\CInfo[1]{
     \CBox{Sepia}{\faArrowCircleRight}{REMARQUE}{#1}
}

\newcommand\CRedac[1]{
     \CBox{PineGreen}{\faEdit}{BIEN R\'EDIGER}{#1}
}

\newcommand\CError[1]{
     \CBox{Red}{\faExclamationTriangle}{ATTENTION}{#1}
}

\newcommand\TitreExo[2]{
\needspace{4\baselineskip}
 {\sffamily\large EXERCICE #1\ (\emph{#2 points})}
\vspace{5mm}
}

\newcommand\img[2]{
          \includegraphics[width=#2\paperwidth]{\imgdir#1}
}

\newcommand\imgsvg[2]{
       \begin{center}   \includegraphics[width=#2\paperwidth]{\imgsvgdir#1} \end{center}
}


\newcommand\Lien[2]{
     \href{#1}{#2 \tiny \faExternalLink}
}
\newcommand\mcLien[2]{
     \href{https~://www.maths-cours.fr/#1}{#2 \tiny \faExternalLink}
}

\newcommand{\euro}{\eurologo{}}

%================================================================================================================================
%
% Macros - Environement
%
%================================================================================================================================

\newenvironment{tex}{ %
}
{%
}

\newenvironment{indente}{ %
	\setlength\parindent{10mm}
}

{
	\setlength\parindent{0mm}
}

\newenvironment{corrige}{%
     \needspace{3\baselineskip}
     \medskip
     \textbf{\textsc{Corrigé}}
     \medskip
}
{
}

\newenvironment{extern}{%
     \begin{center}
     }
     {
     \end{center}
}

\NewEnviron{code}{%
	\par
     \boite{gray}{\texttt{%
     \BODY
     }}
     \par
}

\newenvironment{vbloc}{% boite sans cadre empeche saut de page
     \begin{minipage}[t]{\linewidth}
     }
     {
     \end{minipage}
}
\NewEnviron{h2}{%
    \needspace{3\baselineskip}
    \vspace{0.6cm}
	\noindent \MakeUppercase{\sffamily \large \BODY}
	\vspace{1mm}\textcolor{mcgris}{\hrule}\vspace{0.4cm}
	\par
}{}

\NewEnviron{h3}{%
    \needspace{3\baselineskip}
	\vspace{5mm}
	\textsc{\BODY}
	\par
}

\NewEnviron{margeneg}{ %
\begin{addmargin}[-1cm]{0cm}
\BODY
\end{addmargin}
}

\NewEnviron{html}{%
}

\begin{document}
\meta{url}{/cours/theoreme-thales/}
\meta{pid}{1579}
\meta{pi_}{1579}
\meta{titre}{Théorème de Thalès}
\meta{type}{cours}
\cadre{bleu}{}{
     Le théorème de Thalès doit son nom au philosophe, astronome et mathématicien grec Thalès de Milet (env. 600 ans avant J.C.). S'il n'est pas l'\og inventeur \fg de ce théorème qui était déjà connu des babyloniens, Thalès l'aurait utilisé pour mesurer la hauteur de la grande pyramide de Kheops.
     \par
     Le \textbf{théorème de Thalès} permet de \textbf{calculer des distances} dans une configuration géométrique comportant des droites parallèles.
     \par
     La \textbf{réciproque} du théorème de Thalès sert à démontrer que \textbf{deux droites sont parallèles} en calculant des rapports de distances.
}
\begin{h2}1. Théorème de Thalès\end{h2}
\cadre{rouge}{Théorème de Thalès}{% id="d10"
     Si $A, B, C, D, E$ sont cinq points tels que~:
     \begin{itemize}
          \item les points $A, B, D$ et les points $A, C, E$ sont alignés
          \item les droites $\left(BC\right)$ et $\left(DE\right)$ sont parallèles
     \end{itemize}
     alors~:
     \begin{center}$\frac{AB}{AD}=\frac{AC}{AE}=\frac{BC}{DE}$\end{center}
}
\bloc{cyan}{Remarques}{% id="r05"
     Deux configurations différentes peuvent se présenter selon l'ordre des points $A, B, D$ et $A, C, E$. Il faut être capable de repérer chacune de ces configurations dans les exercices de géométrie.
}
\begin{center}
     \begin{extern}%width="550" alt="Théorème de Thalès"
          \begin{tabular}{ccc}
               \resizebox{5cm}{!}{
                    \psset{xunit=1.0cm,yunit=1.0cm,algebraic=true,dimen=middle,dotstyle=o,dotsize=5pt 0,linewidth=1pt,arrowsize=3pt 2,arrowinset=0.25}
                    \begin{pspicture*}(-2.,-2.)(8.,5.5)
                         \psplot{-3.}{6.}{(-26.+6.*x)/1.}
                         \psplot{-3.}{6.}{(12.+4.*x)/8.}
                         \psplot[linecolor=red]{-0.5}{6.}{(7.97-0.96*x)/3.51}
                         \psplot[linecolor=red]{-2.}{6.}{(2.55-0.96*x)/3.51}
                         \begin{Large}
                              \rput[bl](4.5,4.1){\blue{$A$}}
                              \rput[bl](4.75,1.1){\blue{$B$}}
                              \rput[bl](0.8,2.2){\blue{$C$}}
                              \rput[bl](-1.08,1.32){\blue{$E$}}
                              \rput[bl](4.41,-0.32){\blue{$D$}}
                         \end{Large}
                    \end{pspicture*}
               }
               &
               \resizebox{5cm}{!}{
                    \psset{xunit=1.0cm,yunit=1.0cm,algebraic=true,dimen=middle,dotstyle=o,dotsize=5pt 0,linewidth=1pt,arrowsize=3pt 2,arrowinset=0.25}
                    \begin{pspicture*}(0.,0.5)(10.,8.)
                         \psplot{0.}{10.}{(-26.+6.*x)/1.}
                         \psplot{0.}{10.}{(12.+4.*x)/8.}
                         \psplot[linecolor=red]{0.}{7.}{(7.97-0.96*x)/3.51}
                         \psplot[linecolor=red]{4.5}{10.}{(30-0.96*x)/3.51}
                         \begin{Large}
                              \rput[bl](4.5,4.1){\blue{$A$}}
                              \rput[bl](4.75,1.1){\blue{$B$}}
                              \rput[bl](0.8,2.2){\blue{$C$}}
                              \rput[bl](9.,6.3){\blue{$E$}}
                              \rput[bl](5.7,7.2){\blue{$D$}}
                         \end{Large}
                    \end{pspicture*}
               }
               \\
               \scriptsize Première configuration &\scriptsize Deuxième configuration
          \end{tabular}
     \end{extern}
\end{center}
\begin{center}\textit{Théorème de Thalès}\end{center}
\bloc{cyan}{Remarques}{% id="r10"
     \begin{itemize}
          \item Il est important de bien faire attention à l'ordre des points. On pourra s'aider en notant la correspondance entre les points. Dans les deux figures ci-dessus~:
          \par
          $A \rightarrow A$
          \par
          $B \rightarrow D$
          \par
          $C \rightarrow E$
          \par
          Par conséquent~:
          \par
          $AB \rightarrow AD$
          \par
          $AC \rightarrow AE$
          \par
          $BC \rightarrow DE$
     \end{itemize}
}
\bloc{orange}{Exemple}{% id="e10"
     \begin{center}
          \begin{extern}%width="350" alt="Exemple théorème de Thalès"
               \resizebox{6cm}{!}{
                    \psset{xunit=1.0cm,yunit=1.0cm,algebraic=true,dimen=middle,dotstyle=o,dotsize=5pt 0,linewidth=1.pt,arrowsize=3pt 2,arrowinset=0.25}
                    \begin{pspicture*}(0.,0.)(12.,4.5)
                         \psplot{0.}{12.}{(--3.--1.*x)/4.}
                         \psplot{0.}{12.}{(--11.-1.*x)/3.}
                         \psplot[linecolor=red]{0.}{12.}{(--1.-2.*x)/-1.}
                         \psplot[linecolor=red]{0.}{12.}{(--18.5-2.*x)/-1.}
                         \begin{Large}
                              \rput[bl](0.8,1.13){\blue{$I$}}
                              \rput[bl](4.9,2.14){\blue{$O$}}
                              \rput[bl](10.7,3.71){\blue{$L$}}
                              \rput[bl](1.8,3.20){\blue{$J$}}
                              \rput[bl](9.25,0.75){\blue{$K$}}
                              \rput[bl](7.8,2.9){$6$}
                              \rput[bl](2.7,1.6){$4$}
                              \rput[bl](1.1,2.){$2$}
                         \end{Large}
                    \end{pspicture*}
               }
          \end{extern}
     \end{center}
     Sur la figure ci-dessus, on sait que $OL=6$cm,$ OI=4$cm et $IJ=2$cm et que les droites $\left(IJ\right)$ et $\left(KL\right)$ sont parallèles.
     \par
     Quelle est la longueur du segment $\left[KL\right]$~?
     \begin{itemize}
          \item les points $O, J, K$ et les points $O, I, L$ sont alignés
          \item les droites $\left(IJ\right)$ et $\left(KL\right)$ sont parallèles
     \end{itemize}
     Par conséquent, d'après le théorème de Thalès~:
     \begin{center}$\frac{OJ}{OK}=\frac{OI}{OL}=\frac{IJ}{KL}$\end{center}
     On remplace les longueurs dont ont connait les mesures~:
     \begin{center}$\frac{OJ}{OK}=\frac{\color{red}{4}}{\color{red}{6}}=\frac{\color{red}{2}}{KL}$\end{center}
     L'égalité $\frac{4}{6}=\frac{2}{KL}$ nous permet de trouver $KL$ («quatrième proportionnelle»)~:
     \par
     $KL=\frac{2\times 6}{4}=3$cm.
}
\begin{h2}2. Réciproque du théorème de Thalès\end{h2}
\cadre{rouge}{Théorème (Réciproque du théorème de Thalès)}{% id="t50"
     Si $A, B, C, D, E$ sont cinq points tels que les points $A, B, D$ et les points $A, C, E$ sont alignés dans le même ordre.
     \par
     Si $\frac{AB}{AD}=\frac{AC}{AE}$ alors, les droites $\left(BC\right)$ et $\left(DE\right)$ sont parallèles.
}
\begin{center}
     \begin{extern}%width="240" alt="Réciproque du théorème de Thalès"
          \resizebox{6cm}{!}{
               \psset{xunit=1.0cm,yunit=1.0cm,algebraic=true,dimen=middle,dotstyle=o,dotsize=5pt 0,linewidth=1pt,arrowsize=3pt 2,arrowinset=0.25}
               \begin{pspicture*}(-2.,-2.)(8.,5.5)
                    \psplot{-3.}{6.}{(-26.+6.*x)/1.}
                    \psplot{-3.}{6.}{(12.+4.*x)/8.}
                    \psplot[linecolor=red]{-0.5}{6.}{(7.97-0.96*x)/3.51}
                    \psplot[linecolor=red]{-2.}{6.}{(2.55-0.96*x)/3.51}
                    \begin{Large}
                         \rput[bl](4.5,4.1){\blue{$A$}}
                         \rput[bl](4.75,1.1){\blue{$B$}}
                         \rput[bl](0.8,2.2){\blue{$C$}}
                         \rput[bl](-1.08,1.32){\blue{$E$}}
                         \rput[bl](4.41,-0.32){\blue{$D$}}
                    \end{Large}
               \end{pspicture*}
          }
     \end{extern}
\end{center}
\bloc{cyan}{Remarques}{% id="r20"
     \begin{itemize}
          \item
          Ce théorème sert à \textbf{démontrer que deux droites sont parallèles}.
          \item
          Si $\frac{AB}{AD}\neq \frac{AC}{AE}$ alors, les droites $\left(BC\right)$ et $\left(DE\right)$ ne sont pas parallèles~; cela ne résulte toutefois pas de la réciproque du théorème de Thalès mais c'est une conséquence du théorème de Thalès lui-même (en effet d'après le théorème de Thalès si les droites $\left(BC\right)$ et $\left(DE\right)$ étaient parallèles on aurait $\frac{AB}{AD}= \frac{AC}{AE}$ - voir la fiche méthode \og \mcLien{https://www.maths-cours.fr/methode/determiner-si-deux-droites-sont-paralleles-thales}{Déterminer si deux droites sont parallèles} \fg{} ).
     \end{itemize}
}
\bloc{orange}{Exemple}{% id="e20"
     \begin{center}
          \begin{extern}%width="500" alt="Réciproque du théorème de Thalès~: exemple"
               \resizebox{10cm}{!}{
                    \psset{xunit=1.0cm,yunit=1.0cm,algebraic=true,dimen=middle,dotstyle=o,dotsize=5pt 0,linewidth=1.pt,arrowsize=3pt 2,arrowinset=0.25}
                    \begin{pspicture*}(-3.,-7.)(17.5,8.)
                         \psplot{-3.}{17.5}{(-16.9--5*x)/4.4}
                         \psplot{-3.}{17.5}{(--6.--2.*x)/8.}
                         \psplot[linecolor=red]{-3.}{17.5}{(--12.4--5.6*x)/10}
                         \psplot[linecolor=red]{-3.}{17.5}{(-47.4--5.8*x)/10}
                         \begin{Large}
                              \rput[bl](4.6,2.2){\blue{$O$}}
                              \rput[bl](8.6,6.6){\blue{$J$}}
                              \rput[bl](-1.8,0.6){\blue{$I$}}
                              \rput[bl](16.4,5.1){\blue{$L$}}
                              \rput[bl](-2,-5.5){\blue{$K$}}
                              \rput[bl](7.4,4){$5,6$}
                              \rput[bl](1.4,1.4){$6,2$}
                              \rput[bl](10.2,3.6){$7,2$}
                              \rput[bl](1.,-1.8){$6.8$}
                         \end{Large}
                    \end{pspicture*}
               }
          \end{extern}
     \end{center}
     Dans la figure ci-dessus, on sait que $OI=6,2$cm,$ OJ=5,6$cm, $OK=6,8$cm et $OL=7,2$cm.
     \par
     Les droites $\left(IJ\right)$ et $\left(KL\right)$ sont-elles parallèles~?
     \par
     \textbf{Méthode}~: On calcule séparément $\frac{OI}{OL}$ et $\frac{OJ}{OK}$
     \par
     $\frac{OI}{OL}=\frac{6,2}{7,2}=\frac{62}{72}=\frac{31}{36}$
     \par
     $\frac{OJ}{OK}=\frac{5,6}{6,8}=\frac{56}{68}=\frac{14}{17}$
     \par
     $\frac{OI}{OL} \neq \frac{OJ}{OK}$ (si vous n'êtes pas sûr, vérifiez à la calculatrice~!) donc les droites $\left(IJ\right)$ et $\left(KL\right)$ ne sont pas parallèles.
}

\end{document}
µ
\documentclass[a4paper]{article}

%================================================================================================================================
%
% Packages
%
%================================================================================================================================

\usepackage[T1]{fontenc} 	% pour caractères accentués
\usepackage[utf8]{inputenc}  % encodage utf8
\usepackage[french]{babel}	% langue : français
\usepackage{fourier}			% caractères plus lisibles
\usepackage[dvipsnames]{xcolor} % couleurs
\usepackage{fancyhdr}		% réglage header footer
\usepackage{needspace}		% empêcher sauts de page mal placés
\usepackage{graphicx}		% pour inclure des graphiques
\usepackage{enumitem,cprotect}		% personnalise les listes d'items (nécessaire pour ol, al ...)
\usepackage{hyperref}		% Liens hypertexte
\usepackage{pstricks,pst-all,pst-node,pstricks-add,pst-math,pst-plot,pst-tree,pst-eucl} % pstricks
\usepackage[a4paper,includeheadfoot,top=2cm,left=3cm, bottom=2cm,right=3cm]{geometry} % marges etc.
\usepackage{comment}			% commentaires multilignes
\usepackage{amsmath,environ} % maths (matrices, etc.)
\usepackage{amssymb,makeidx}
\usepackage{bm}				% bold maths
\usepackage{tabularx}		% tableaux
\usepackage{colortbl}		% tableaux en couleur
\usepackage{fontawesome}		% Fontawesome
\usepackage{environ}			% environment with command
\usepackage{fp}				% calculs pour ps-tricks
\usepackage{multido}			% pour ps tricks
\usepackage[np]{numprint}	% formattage nombre
\usepackage{tikz,tkz-tab} 			% package principal TikZ
\usepackage{pgfplots}   % axes
\usepackage{mathrsfs}    % cursives
\usepackage{calc}			% calcul taille boites
\usepackage[scaled=0.875]{helvet} % font sans serif
\usepackage{svg} % svg
\usepackage{scrextend} % local margin
\usepackage{scratch} %scratch
\usepackage{multicol} % colonnes
%\usepackage{infix-RPN,pst-func} % formule en notation polanaise inversée
\usepackage{listings}

%================================================================================================================================
%
% Réglages de base
%
%================================================================================================================================

\lstset{
language=Python,   % R code
literate=
{á}{{\'a}}1
{à}{{\`a}}1
{ã}{{\~a}}1
{é}{{\'e}}1
{è}{{\`e}}1
{ê}{{\^e}}1
{í}{{\'i}}1
{ó}{{\'o}}1
{õ}{{\~o}}1
{ú}{{\'u}}1
{ü}{{\"u}}1
{ç}{{\c{c}}}1
{~}{{ }}1
}


\definecolor{codegreen}{rgb}{0,0.6,0}
\definecolor{codegray}{rgb}{0.5,0.5,0.5}
\definecolor{codepurple}{rgb}{0.58,0,0.82}
\definecolor{backcolour}{rgb}{0.95,0.95,0.92}

\lstdefinestyle{mystyle}{
    backgroundcolor=\color{backcolour},   
    commentstyle=\color{codegreen},
    keywordstyle=\color{magenta},
    numberstyle=\tiny\color{codegray},
    stringstyle=\color{codepurple},
    basicstyle=\ttfamily\footnotesize,
    breakatwhitespace=false,         
    breaklines=true,                 
    captionpos=b,                    
    keepspaces=true,                 
    numbers=left,                    
xleftmargin=2em,
framexleftmargin=2em,            
    showspaces=false,                
    showstringspaces=false,
    showtabs=false,                  
    tabsize=2,
    upquote=true
}

\lstset{style=mystyle}


\lstset{style=mystyle}
\newcommand{\imgdir}{C:/laragon/www/newmc/assets/imgsvg/}
\newcommand{\imgsvgdir}{C:/laragon/www/newmc/assets/imgsvg/}

\definecolor{mcgris}{RGB}{220, 220, 220}% ancien~; pour compatibilité
\definecolor{mcbleu}{RGB}{52, 152, 219}
\definecolor{mcvert}{RGB}{125, 194, 70}
\definecolor{mcmauve}{RGB}{154, 0, 215}
\definecolor{mcorange}{RGB}{255, 96, 0}
\definecolor{mcturquoise}{RGB}{0, 153, 153}
\definecolor{mcrouge}{RGB}{255, 0, 0}
\definecolor{mclightvert}{RGB}{205, 234, 190}

\definecolor{gris}{RGB}{220, 220, 220}
\definecolor{bleu}{RGB}{52, 152, 219}
\definecolor{vert}{RGB}{125, 194, 70}
\definecolor{mauve}{RGB}{154, 0, 215}
\definecolor{orange}{RGB}{255, 96, 0}
\definecolor{turquoise}{RGB}{0, 153, 153}
\definecolor{rouge}{RGB}{255, 0, 0}
\definecolor{lightvert}{RGB}{205, 234, 190}
\setitemize[0]{label=\color{lightvert}  $\bullet$}

\pagestyle{fancy}
\renewcommand{\headrulewidth}{0.2pt}
\fancyhead[L]{maths-cours.fr}
\fancyhead[R]{\thepage}
\renewcommand{\footrulewidth}{0.2pt}
\fancyfoot[C]{}

\newcolumntype{C}{>{\centering\arraybackslash}X}
\newcolumntype{s}{>{\hsize=.35\hsize\arraybackslash}X}

\setlength{\parindent}{0pt}		 
\setlength{\parskip}{3mm}
\setlength{\headheight}{1cm}

\def\ebook{ebook}
\def\book{book}
\def\web{web}
\def\type{web}

\newcommand{\vect}[1]{\overrightarrow{\,\mathstrut#1\,}}

\def\Oij{$\left(\text{O}~;~\vect{\imath},~\vect{\jmath}\right)$}
\def\Oijk{$\left(\text{O}~;~\vect{\imath},~\vect{\jmath},~\vect{k}\right)$}
\def\Ouv{$\left(\text{O}~;~\vect{u},~\vect{v}\right)$}

\hypersetup{breaklinks=true, colorlinks = true, linkcolor = OliveGreen, urlcolor = OliveGreen, citecolor = OliveGreen, pdfauthor={Didier BONNEL - https://www.maths-cours.fr} } % supprime les bordures autour des liens

\renewcommand{\arg}[0]{\text{arg}}

\everymath{\displaystyle}

%================================================================================================================================
%
% Macros - Commandes
%
%================================================================================================================================

\newcommand\meta[2]{    			% Utilisé pour créer le post HTML.
	\def\titre{titre}
	\def\url{url}
	\def\arg{#1}
	\ifx\titre\arg
		\newcommand\maintitle{#2}
		\fancyhead[L]{#2}
		{\Large\sffamily \MakeUppercase{#2}}
		\vspace{1mm}\textcolor{mcvert}{\hrule}
	\fi 
	\ifx\url\arg
		\fancyfoot[L]{\href{https://www.maths-cours.fr#2}{\black \footnotesize{https://www.maths-cours.fr#2}}}
	\fi 
}


\newcommand\TitreC[1]{    		% Titre centré
     \needspace{3\baselineskip}
     \begin{center}\textbf{#1}\end{center}
}

\newcommand\newpar{    		% paragraphe
     \par
}

\newcommand\nosp {    		% commande vide (pas d'espace)
}
\newcommand{\id}[1]{} %ignore

\newcommand\boite[2]{				% Boite simple sans titre
	\vspace{5mm}
	\setlength{\fboxrule}{0.2mm}
	\setlength{\fboxsep}{5mm}	
	\fcolorbox{#1}{#1!3}{\makebox[\linewidth-2\fboxrule-2\fboxsep]{
  		\begin{minipage}[t]{\linewidth-2\fboxrule-4\fboxsep}\setlength{\parskip}{3mm}
  			 #2
  		\end{minipage}
	}}
	\vspace{5mm}
}

\newcommand\CBox[4]{				% Boites
	\vspace{5mm}
	\setlength{\fboxrule}{0.2mm}
	\setlength{\fboxsep}{5mm}
	
	\fcolorbox{#1}{#1!3}{\makebox[\linewidth-2\fboxrule-2\fboxsep]{
		\begin{minipage}[t]{1cm}\setlength{\parskip}{3mm}
	  		\textcolor{#1}{\LARGE{#2}}    
 	 	\end{minipage}  
  		\begin{minipage}[t]{\linewidth-2\fboxrule-4\fboxsep}\setlength{\parskip}{3mm}
			\raisebox{1.2mm}{\normalsize\sffamily{\textcolor{#1}{#3}}}						
  			 #4
  		\end{minipage}
	}}
	\vspace{5mm}
}

\newcommand\cadre[3]{				% Boites convertible html
	\par
	\vspace{2mm}
	\setlength{\fboxrule}{0.1mm}
	\setlength{\fboxsep}{5mm}
	\fcolorbox{#1}{white}{\makebox[\linewidth-2\fboxrule-2\fboxsep]{
  		\begin{minipage}[t]{\linewidth-2\fboxrule-4\fboxsep}\setlength{\parskip}{3mm}
			\raisebox{-2.5mm}{\sffamily \small{\textcolor{#1}{\MakeUppercase{#2}}}}		
			\par		
  			 #3
 	 		\end{minipage}
	}}
		\vspace{2mm}
	\par
}

\newcommand\bloc[3]{				% Boites convertible html sans bordure
     \needspace{2\baselineskip}
     {\sffamily \small{\textcolor{#1}{\MakeUppercase{#2}}}}    
		\par		
  			 #3
		\par
}

\newcommand\CHelp[1]{
     \CBox{Plum}{\faInfoCircle}{À RETENIR}{#1}
}

\newcommand\CUp[1]{
     \CBox{NavyBlue}{\faThumbsOUp}{EN PRATIQUE}{#1}
}

\newcommand\CInfo[1]{
     \CBox{Sepia}{\faArrowCircleRight}{REMARQUE}{#1}
}

\newcommand\CRedac[1]{
     \CBox{PineGreen}{\faEdit}{BIEN R\'EDIGER}{#1}
}

\newcommand\CError[1]{
     \CBox{Red}{\faExclamationTriangle}{ATTENTION}{#1}
}

\newcommand\TitreExo[2]{
\needspace{4\baselineskip}
 {\sffamily\large EXERCICE #1\ (\emph{#2 points})}
\vspace{5mm}
}

\newcommand\img[2]{
          \includegraphics[width=#2\paperwidth]{\imgdir#1}
}

\newcommand\imgsvg[2]{
       \begin{center}   \includegraphics[width=#2\paperwidth]{\imgsvgdir#1} \end{center}
}


\newcommand\Lien[2]{
     \href{#1}{#2 \tiny \faExternalLink}
}
\newcommand\mcLien[2]{
     \href{https~://www.maths-cours.fr/#1}{#2 \tiny \faExternalLink}
}

\newcommand{\euro}{\eurologo{}}

%================================================================================================================================
%
% Macros - Environement
%
%================================================================================================================================

\newenvironment{tex}{ %
}
{%
}

\newenvironment{indente}{ %
	\setlength\parindent{10mm}
}

{
	\setlength\parindent{0mm}
}

\newenvironment{corrige}{%
     \needspace{3\baselineskip}
     \medskip
     \textbf{\textsc{Corrigé}}
     \medskip
}
{
}

\newenvironment{extern}{%
     \begin{center}
     }
     {
     \end{center}
}

\NewEnviron{code}{%
	\par
     \boite{gray}{\texttt{%
     \BODY
     }}
     \par
}

\newenvironment{vbloc}{% boite sans cadre empeche saut de page
     \begin{minipage}[t]{\linewidth}
     }
     {
     \end{minipage}
}
\NewEnviron{h2}{%
    \needspace{3\baselineskip}
    \vspace{0.6cm}
	\noindent \MakeUppercase{\sffamily \large \BODY}
	\vspace{1mm}\textcolor{mcgris}{\hrule}\vspace{0.4cm}
	\par
}{}

\NewEnviron{h3}{%
    \needspace{3\baselineskip}
	\vspace{5mm}
	\textsc{\BODY}
	\par
}

\NewEnviron{margeneg}{ %
\begin{addmargin}[-1cm]{0cm}
\BODY
\end{addmargin}
}

\NewEnviron{html}{%
}

\begin{document}
\meta{url}{/exercices/criteres-de-divisibilite-141103/}
\meta{pid}{1658}
\meta{titre}{Critères de divisibilité}
\meta{type}{exercices}
Parmi les nombres ci-dessous, indiquer ceux qui sont divisibles par 2, 3, 5, 9 ou 10.
\begin{enumerate}
     \item 1 544
     \item 3 600
     \item 1 325
     \item 1 001
\end{enumerate}
\begin{corrige}
Pour cet exercice, on utilise les \mcLien{https://www.maths-cours.fr/cours/division-euclidienne-pgcd/\#t30}{critères de divisibilité}.
     \begin{enumerate}
          \item 1 544 \textbf{est divisible} par 2
          \par
          1 544 \textbf{n'est pas divisible} par 3 (1+5+4+4=14)
          \par
          1 544 \textbf{n'est pas divisible} par 5
          \par
          1 544 \textbf{n'est pas divisible} par 9
          \par
          1 544 \textbf{n'est pas divisible} par 10
          \item 3 600\textbf{ est divisible} par 2
          \par
          3 600 \textbf{est divisible} par 3 (3+6+0+0=9)
          \par
          3 600 \textbf{est divisible} par 5
          \par
          3 600 \textbf{est divisible} par 9
          \par
          3 600 \textbf{est divisible} par 10
          \item 1 325\textbf{ n'est pas divisible} par 2
          \par
          1 325 \textbf{n'est pas divisible} par 3 (1+3+2+5=11)
          \par
          1 325 \textbf{est divisible} par 5
          \par
          1 325 \textbf{n'est pas divisible} par 9
          \par
          1 325 \textbf{n'est pas divisible} par 10
          \item 1 001\textbf{ n'est pas divisible} par 2
          \par
          1 001 \textbf{n'est pas divisible} par 3 (1+0+0+1=2)
          \par
          1 001 \textbf{n'est pas divisible} par 5
          \par
          1 001 \textbf{n'est pas divisible} par 9
          \par
          1 001 \textbf{n'est pas divisible} par 10
     \end{enumerate}
\end{corrige}

\end{document}

µ
\documentclass[a4paper]{article}

%================================================================================================================================
%
% Packages
%
%================================================================================================================================

\usepackage[T1]{fontenc} 	% pour caractères accentués
\usepackage[utf8]{inputenc}  % encodage utf8
\usepackage[french]{babel}	% langue : français
\usepackage{fourier}			% caractères plus lisibles
\usepackage[dvipsnames]{xcolor} % couleurs
\usepackage{fancyhdr}		% réglage header footer
\usepackage{needspace}		% empêcher sauts de page mal placés
\usepackage{graphicx}		% pour inclure des graphiques
\usepackage{enumitem,cprotect}		% personnalise les listes d'items (nécessaire pour ol, al ...)
\usepackage{hyperref}		% Liens hypertexte
\usepackage{pstricks,pst-all,pst-node,pstricks-add,pst-math,pst-plot,pst-tree,pst-eucl} % pstricks
\usepackage[a4paper,includeheadfoot,top=2cm,left=3cm, bottom=2cm,right=3cm]{geometry} % marges etc.
\usepackage{comment}			% commentaires multilignes
\usepackage{amsmath,environ} % maths (matrices, etc.)
\usepackage{amssymb,makeidx}
\usepackage{bm}				% bold maths
\usepackage{tabularx}		% tableaux
\usepackage{colortbl}		% tableaux en couleur
\usepackage{fontawesome}		% Fontawesome
\usepackage{environ}			% environment with command
\usepackage{fp}				% calculs pour ps-tricks
\usepackage{multido}			% pour ps tricks
\usepackage[np]{numprint}	% formattage nombre
\usepackage{tikz,tkz-tab} 			% package principal TikZ
\usepackage{pgfplots}   % axes
\usepackage{mathrsfs}    % cursives
\usepackage{calc}			% calcul taille boites
\usepackage[scaled=0.875]{helvet} % font sans serif
\usepackage{svg} % svg
\usepackage{scrextend} % local margin
\usepackage{scratch} %scratch
\usepackage{multicol} % colonnes
%\usepackage{infix-RPN,pst-func} % formule en notation polanaise inversée
\usepackage{listings}

%================================================================================================================================
%
% Réglages de base
%
%================================================================================================================================

\lstset{
language=Python,   % R code
literate=
{á}{{\'a}}1
{à}{{\`a}}1
{ã}{{\~a}}1
{é}{{\'e}}1
{è}{{\`e}}1
{ê}{{\^e}}1
{í}{{\'i}}1
{ó}{{\'o}}1
{õ}{{\~o}}1
{ú}{{\'u}}1
{ü}{{\"u}}1
{ç}{{\c{c}}}1
{~}{{ }}1
}


\definecolor{codegreen}{rgb}{0,0.6,0}
\definecolor{codegray}{rgb}{0.5,0.5,0.5}
\definecolor{codepurple}{rgb}{0.58,0,0.82}
\definecolor{backcolour}{rgb}{0.95,0.95,0.92}

\lstdefinestyle{mystyle}{
    backgroundcolor=\color{backcolour},   
    commentstyle=\color{codegreen},
    keywordstyle=\color{magenta},
    numberstyle=\tiny\color{codegray},
    stringstyle=\color{codepurple},
    basicstyle=\ttfamily\footnotesize,
    breakatwhitespace=false,         
    breaklines=true,                 
    captionpos=b,                    
    keepspaces=true,                 
    numbers=left,                    
xleftmargin=2em,
framexleftmargin=2em,            
    showspaces=false,                
    showstringspaces=false,
    showtabs=false,                  
    tabsize=2,
    upquote=true
}

\lstset{style=mystyle}


\lstset{style=mystyle}
\newcommand{\imgdir}{C:/laragon/www/newmc/assets/imgsvg/}
\newcommand{\imgsvgdir}{C:/laragon/www/newmc/assets/imgsvg/}

\definecolor{mcgris}{RGB}{220, 220, 220}% ancien~; pour compatibilité
\definecolor{mcbleu}{RGB}{52, 152, 219}
\definecolor{mcvert}{RGB}{125, 194, 70}
\definecolor{mcmauve}{RGB}{154, 0, 215}
\definecolor{mcorange}{RGB}{255, 96, 0}
\definecolor{mcturquoise}{RGB}{0, 153, 153}
\definecolor{mcrouge}{RGB}{255, 0, 0}
\definecolor{mclightvert}{RGB}{205, 234, 190}

\definecolor{gris}{RGB}{220, 220, 220}
\definecolor{bleu}{RGB}{52, 152, 219}
\definecolor{vert}{RGB}{125, 194, 70}
\definecolor{mauve}{RGB}{154, 0, 215}
\definecolor{orange}{RGB}{255, 96, 0}
\definecolor{turquoise}{RGB}{0, 153, 153}
\definecolor{rouge}{RGB}{255, 0, 0}
\definecolor{lightvert}{RGB}{205, 234, 190}
\setitemize[0]{label=\color{lightvert}  $\bullet$}

\pagestyle{fancy}
\renewcommand{\headrulewidth}{0.2pt}
\fancyhead[L]{maths-cours.fr}
\fancyhead[R]{\thepage}
\renewcommand{\footrulewidth}{0.2pt}
\fancyfoot[C]{}

\newcolumntype{C}{>{\centering\arraybackslash}X}
\newcolumntype{s}{>{\hsize=.35\hsize\arraybackslash}X}

\setlength{\parindent}{0pt}		 
\setlength{\parskip}{3mm}
\setlength{\headheight}{1cm}

\def\ebook{ebook}
\def\book{book}
\def\web{web}
\def\type{web}

\newcommand{\vect}[1]{\overrightarrow{\,\mathstrut#1\,}}

\def\Oij{$\left(\text{O}~;~\vect{\imath},~\vect{\jmath}\right)$}
\def\Oijk{$\left(\text{O}~;~\vect{\imath},~\vect{\jmath},~\vect{k}\right)$}
\def\Ouv{$\left(\text{O}~;~\vect{u},~\vect{v}\right)$}

\hypersetup{breaklinks=true, colorlinks = true, linkcolor = OliveGreen, urlcolor = OliveGreen, citecolor = OliveGreen, pdfauthor={Didier BONNEL - https://www.maths-cours.fr} } % supprime les bordures autour des liens

\renewcommand{\arg}[0]{\text{arg}}

\everymath{\displaystyle}

%================================================================================================================================
%
% Macros - Commandes
%
%================================================================================================================================

\newcommand\meta[2]{    			% Utilisé pour créer le post HTML.
	\def\titre{titre}
	\def\url{url}
	\def\arg{#1}
	\ifx\titre\arg
		\newcommand\maintitle{#2}
		\fancyhead[L]{#2}
		{\Large\sffamily \MakeUppercase{#2}}
		\vspace{1mm}\textcolor{mcvert}{\hrule}
	\fi 
	\ifx\url\arg
		\fancyfoot[L]{\href{https://www.maths-cours.fr#2}{\black \footnotesize{https://www.maths-cours.fr#2}}}
	\fi 
}


\newcommand\TitreC[1]{    		% Titre centré
     \needspace{3\baselineskip}
     \begin{center}\textbf{#1}\end{center}
}

\newcommand\newpar{    		% paragraphe
     \par
}

\newcommand\nosp {    		% commande vide (pas d'espace)
}
\newcommand{\id}[1]{} %ignore

\newcommand\boite[2]{				% Boite simple sans titre
	\vspace{5mm}
	\setlength{\fboxrule}{0.2mm}
	\setlength{\fboxsep}{5mm}	
	\fcolorbox{#1}{#1!3}{\makebox[\linewidth-2\fboxrule-2\fboxsep]{
  		\begin{minipage}[t]{\linewidth-2\fboxrule-4\fboxsep}\setlength{\parskip}{3mm}
  			 #2
  		\end{minipage}
	}}
	\vspace{5mm}
}

\newcommand\CBox[4]{				% Boites
	\vspace{5mm}
	\setlength{\fboxrule}{0.2mm}
	\setlength{\fboxsep}{5mm}
	
	\fcolorbox{#1}{#1!3}{\makebox[\linewidth-2\fboxrule-2\fboxsep]{
		\begin{minipage}[t]{1cm}\setlength{\parskip}{3mm}
	  		\textcolor{#1}{\LARGE{#2}}    
 	 	\end{minipage}  
  		\begin{minipage}[t]{\linewidth-2\fboxrule-4\fboxsep}\setlength{\parskip}{3mm}
			\raisebox{1.2mm}{\normalsize\sffamily{\textcolor{#1}{#3}}}						
  			 #4
  		\end{minipage}
	}}
	\vspace{5mm}
}

\newcommand\cadre[3]{				% Boites convertible html
	\par
	\vspace{2mm}
	\setlength{\fboxrule}{0.1mm}
	\setlength{\fboxsep}{5mm}
	\fcolorbox{#1}{white}{\makebox[\linewidth-2\fboxrule-2\fboxsep]{
  		\begin{minipage}[t]{\linewidth-2\fboxrule-4\fboxsep}\setlength{\parskip}{3mm}
			\raisebox{-2.5mm}{\sffamily \small{\textcolor{#1}{\MakeUppercase{#2}}}}		
			\par		
  			 #3
 	 		\end{minipage}
	}}
		\vspace{2mm}
	\par
}

\newcommand\bloc[3]{				% Boites convertible html sans bordure
     \needspace{2\baselineskip}
     {\sffamily \small{\textcolor{#1}{\MakeUppercase{#2}}}}    
		\par		
  			 #3
		\par
}

\newcommand\CHelp[1]{
     \CBox{Plum}{\faInfoCircle}{À RETENIR}{#1}
}

\newcommand\CUp[1]{
     \CBox{NavyBlue}{\faThumbsOUp}{EN PRATIQUE}{#1}
}

\newcommand\CInfo[1]{
     \CBox{Sepia}{\faArrowCircleRight}{REMARQUE}{#1}
}

\newcommand\CRedac[1]{
     \CBox{PineGreen}{\faEdit}{BIEN R\'EDIGER}{#1}
}

\newcommand\CError[1]{
     \CBox{Red}{\faExclamationTriangle}{ATTENTION}{#1}
}

\newcommand\TitreExo[2]{
\needspace{4\baselineskip}
 {\sffamily\large EXERCICE #1\ (\emph{#2 points})}
\vspace{5mm}
}

\newcommand\img[2]{
          \includegraphics[width=#2\paperwidth]{\imgdir#1}
}

\newcommand\imgsvg[2]{
       \begin{center}   \includegraphics[width=#2\paperwidth]{\imgsvgdir#1} \end{center}
}


\newcommand\Lien[2]{
     \href{#1}{#2 \tiny \faExternalLink}
}
\newcommand\mcLien[2]{
     \href{https~://www.maths-cours.fr/#1}{#2 \tiny \faExternalLink}
}

\newcommand{\euro}{\eurologo{}}

%================================================================================================================================
%
% Macros - Environement
%
%================================================================================================================================

\newenvironment{tex}{ %
}
{%
}

\newenvironment{indente}{ %
	\setlength\parindent{10mm}
}

{
	\setlength\parindent{0mm}
}

\newenvironment{corrige}{%
     \needspace{3\baselineskip}
     \medskip
     \textbf{\textsc{Corrigé}}
     \medskip
}
{
}

\newenvironment{extern}{%
     \begin{center}
     }
     {
     \end{center}
}

\NewEnviron{code}{%
	\par
     \boite{gray}{\texttt{%
     \BODY
     }}
     \par
}

\newenvironment{vbloc}{% boite sans cadre empeche saut de page
     \begin{minipage}[t]{\linewidth}
     }
     {
     \end{minipage}
}
\NewEnviron{h2}{%
    \needspace{3\baselineskip}
    \vspace{0.6cm}
	\noindent \MakeUppercase{\sffamily \large \BODY}
	\vspace{1mm}\textcolor{mcgris}{\hrule}\vspace{0.4cm}
	\par
}{}

\NewEnviron{h3}{%
    \needspace{3\baselineskip}
	\vspace{5mm}
	\textsc{\BODY}
	\par
}

\NewEnviron{margeneg}{ %
\begin{addmargin}[-1cm]{0cm}
\BODY
\end{addmargin}
}

\NewEnviron{html}{%
}

\begin{document}
\meta{url}{/exercices/vocabulaire-divisibilite-141103/}
\meta{pid}{1661}
\meta{titre}{Divisibilité: Vocabulaire}
\meta{type}{exercices}
Dans chacun des exemples ci-dessous, indiquer si les propositions sont justes ou fausses~:
\begin{enumerate}
     \item
     \begin{itemize}
          \item
          $12$ est divisible par $4$
          \item
          $12$ est un multiple de $4$
          \item
          $12$ est un diviseur de $4$
          \item
          $12$ divise $4$
     \end{itemize}
     \item
     \begin{itemize}
          \item
          $7$ est divisible par $14$
          \item
          $7$ est un multiple de $14$
          \item
          $7$ est un diviseur de $14$
          \item
          $7$ divise $14$
     \end{itemize}
\end{enumerate}
\begin{corrige}
     \begin{enumerate}
          \item
          $12=3\times 4$ donc~:
          \begin{itemize}
               \item
               $12$ est divisible par $4$~: \textbf{VRAI}
               \item
               $12$ est un multiple de $4$~: \textbf{VRAI}
               \item
               $12$ est un diviseur de $4$~: \textbf{FAUX}
               \item
               $12$ divise $4$~: \textbf{FAUX}
          \end{itemize}
          \item
          \begin{itemize}
               \item
               $7$ est divisible par $14$~: \textbf{FAUX}
               \item
               $7$ est un multiple de $14$~: \textbf{FAUX}
               \item
               $7$ est un diviseur de $14$~: \textbf{VRAI}
               \item
               $7$ divise $14$~: \textbf{VRAI}
          \end{itemize}
     \end{enumerate}
\end{corrige}

\end{document}
µ
\documentclass[a4paper]{article}

%================================================================================================================================
%
% Packages
%
%================================================================================================================================

\usepackage[T1]{fontenc} 	% pour caractères accentués
\usepackage[utf8]{inputenc}  % encodage utf8
\usepackage[french]{babel}	% langue : français
\usepackage{fourier}			% caractères plus lisibles
\usepackage[dvipsnames]{xcolor} % couleurs
\usepackage{fancyhdr}		% réglage header footer
\usepackage{needspace}		% empêcher sauts de page mal placés
\usepackage{graphicx}		% pour inclure des graphiques
\usepackage{enumitem,cprotect}		% personnalise les listes d'items (nécessaire pour ol, al ...)
\usepackage{hyperref}		% Liens hypertexte
\usepackage{pstricks,pst-all,pst-node,pstricks-add,pst-math,pst-plot,pst-tree,pst-eucl} % pstricks
\usepackage[a4paper,includeheadfoot,top=2cm,left=3cm, bottom=2cm,right=3cm]{geometry} % marges etc.
\usepackage{comment}			% commentaires multilignes
\usepackage{amsmath,environ} % maths (matrices, etc.)
\usepackage{amssymb,makeidx}
\usepackage{bm}				% bold maths
\usepackage{tabularx}		% tableaux
\usepackage{colortbl}		% tableaux en couleur
\usepackage{fontawesome}		% Fontawesome
\usepackage{environ}			% environment with command
\usepackage{fp}				% calculs pour ps-tricks
\usepackage{multido}			% pour ps tricks
\usepackage[np]{numprint}	% formattage nombre
\usepackage{tikz,tkz-tab} 			% package principal TikZ
\usepackage{pgfplots}   % axes
\usepackage{mathrsfs}    % cursives
\usepackage{calc}			% calcul taille boites
\usepackage[scaled=0.875]{helvet} % font sans serif
\usepackage{svg} % svg
\usepackage{scrextend} % local margin
\usepackage{scratch} %scratch
\usepackage{multicol} % colonnes
%\usepackage{infix-RPN,pst-func} % formule en notation polanaise inversée
\usepackage{listings}

%================================================================================================================================
%
% Réglages de base
%
%================================================================================================================================

\lstset{
language=Python,   % R code
literate=
{á}{{\'a}}1
{à}{{\`a}}1
{ã}{{\~a}}1
{é}{{\'e}}1
{è}{{\`e}}1
{ê}{{\^e}}1
{í}{{\'i}}1
{ó}{{\'o}}1
{õ}{{\~o}}1
{ú}{{\'u}}1
{ü}{{\"u}}1
{ç}{{\c{c}}}1
{~}{{ }}1
}


\definecolor{codegreen}{rgb}{0,0.6,0}
\definecolor{codegray}{rgb}{0.5,0.5,0.5}
\definecolor{codepurple}{rgb}{0.58,0,0.82}
\definecolor{backcolour}{rgb}{0.95,0.95,0.92}

\lstdefinestyle{mystyle}{
    backgroundcolor=\color{backcolour},   
    commentstyle=\color{codegreen},
    keywordstyle=\color{magenta},
    numberstyle=\tiny\color{codegray},
    stringstyle=\color{codepurple},
    basicstyle=\ttfamily\footnotesize,
    breakatwhitespace=false,         
    breaklines=true,                 
    captionpos=b,                    
    keepspaces=true,                 
    numbers=left,                    
xleftmargin=2em,
framexleftmargin=2em,            
    showspaces=false,                
    showstringspaces=false,
    showtabs=false,                  
    tabsize=2,
    upquote=true
}

\lstset{style=mystyle}


\lstset{style=mystyle}
\newcommand{\imgdir}{C:/laragon/www/newmc/assets/imgsvg/}
\newcommand{\imgsvgdir}{C:/laragon/www/newmc/assets/imgsvg/}

\definecolor{mcgris}{RGB}{220, 220, 220}% ancien~; pour compatibilité
\definecolor{mcbleu}{RGB}{52, 152, 219}
\definecolor{mcvert}{RGB}{125, 194, 70}
\definecolor{mcmauve}{RGB}{154, 0, 215}
\definecolor{mcorange}{RGB}{255, 96, 0}
\definecolor{mcturquoise}{RGB}{0, 153, 153}
\definecolor{mcrouge}{RGB}{255, 0, 0}
\definecolor{mclightvert}{RGB}{205, 234, 190}

\definecolor{gris}{RGB}{220, 220, 220}
\definecolor{bleu}{RGB}{52, 152, 219}
\definecolor{vert}{RGB}{125, 194, 70}
\definecolor{mauve}{RGB}{154, 0, 215}
\definecolor{orange}{RGB}{255, 96, 0}
\definecolor{turquoise}{RGB}{0, 153, 153}
\definecolor{rouge}{RGB}{255, 0, 0}
\definecolor{lightvert}{RGB}{205, 234, 190}
\setitemize[0]{label=\color{lightvert}  $\bullet$}

\pagestyle{fancy}
\renewcommand{\headrulewidth}{0.2pt}
\fancyhead[L]{maths-cours.fr}
\fancyhead[R]{\thepage}
\renewcommand{\footrulewidth}{0.2pt}
\fancyfoot[C]{}

\newcolumntype{C}{>{\centering\arraybackslash}X}
\newcolumntype{s}{>{\hsize=.35\hsize\arraybackslash}X}

\setlength{\parindent}{0pt}		 
\setlength{\parskip}{3mm}
\setlength{\headheight}{1cm}

\def\ebook{ebook}
\def\book{book}
\def\web{web}
\def\type{web}

\newcommand{\vect}[1]{\overrightarrow{\,\mathstrut#1\,}}

\def\Oij{$\left(\text{O}~;~\vect{\imath},~\vect{\jmath}\right)$}
\def\Oijk{$\left(\text{O}~;~\vect{\imath},~\vect{\jmath},~\vect{k}\right)$}
\def\Ouv{$\left(\text{O}~;~\vect{u},~\vect{v}\right)$}

\hypersetup{breaklinks=true, colorlinks = true, linkcolor = OliveGreen, urlcolor = OliveGreen, citecolor = OliveGreen, pdfauthor={Didier BONNEL - https://www.maths-cours.fr} } % supprime les bordures autour des liens

\renewcommand{\arg}[0]{\text{arg}}

\everymath{\displaystyle}

%================================================================================================================================
%
% Macros - Commandes
%
%================================================================================================================================

\newcommand\meta[2]{    			% Utilisé pour créer le post HTML.
	\def\titre{titre}
	\def\url{url}
	\def\arg{#1}
	\ifx\titre\arg
		\newcommand\maintitle{#2}
		\fancyhead[L]{#2}
		{\Large\sffamily \MakeUppercase{#2}}
		\vspace{1mm}\textcolor{mcvert}{\hrule}
	\fi 
	\ifx\url\arg
		\fancyfoot[L]{\href{https://www.maths-cours.fr#2}{\black \footnotesize{https://www.maths-cours.fr#2}}}
	\fi 
}


\newcommand\TitreC[1]{    		% Titre centré
     \needspace{3\baselineskip}
     \begin{center}\textbf{#1}\end{center}
}

\newcommand\newpar{    		% paragraphe
     \par
}

\newcommand\nosp {    		% commande vide (pas d'espace)
}
\newcommand{\id}[1]{} %ignore

\newcommand\boite[2]{				% Boite simple sans titre
	\vspace{5mm}
	\setlength{\fboxrule}{0.2mm}
	\setlength{\fboxsep}{5mm}	
	\fcolorbox{#1}{#1!3}{\makebox[\linewidth-2\fboxrule-2\fboxsep]{
  		\begin{minipage}[t]{\linewidth-2\fboxrule-4\fboxsep}\setlength{\parskip}{3mm}
  			 #2
  		\end{minipage}
	}}
	\vspace{5mm}
}

\newcommand\CBox[4]{				% Boites
	\vspace{5mm}
	\setlength{\fboxrule}{0.2mm}
	\setlength{\fboxsep}{5mm}
	
	\fcolorbox{#1}{#1!3}{\makebox[\linewidth-2\fboxrule-2\fboxsep]{
		\begin{minipage}[t]{1cm}\setlength{\parskip}{3mm}
	  		\textcolor{#1}{\LARGE{#2}}    
 	 	\end{minipage}  
  		\begin{minipage}[t]{\linewidth-2\fboxrule-4\fboxsep}\setlength{\parskip}{3mm}
			\raisebox{1.2mm}{\normalsize\sffamily{\textcolor{#1}{#3}}}						
  			 #4
  		\end{minipage}
	}}
	\vspace{5mm}
}

\newcommand\cadre[3]{				% Boites convertible html
	\par
	\vspace{2mm}
	\setlength{\fboxrule}{0.1mm}
	\setlength{\fboxsep}{5mm}
	\fcolorbox{#1}{white}{\makebox[\linewidth-2\fboxrule-2\fboxsep]{
  		\begin{minipage}[t]{\linewidth-2\fboxrule-4\fboxsep}\setlength{\parskip}{3mm}
			\raisebox{-2.5mm}{\sffamily \small{\textcolor{#1}{\MakeUppercase{#2}}}}		
			\par		
  			 #3
 	 		\end{minipage}
	}}
		\vspace{2mm}
	\par
}

\newcommand\bloc[3]{				% Boites convertible html sans bordure
     \needspace{2\baselineskip}
     {\sffamily \small{\textcolor{#1}{\MakeUppercase{#2}}}}    
		\par		
  			 #3
		\par
}

\newcommand\CHelp[1]{
     \CBox{Plum}{\faInfoCircle}{À RETENIR}{#1}
}

\newcommand\CUp[1]{
     \CBox{NavyBlue}{\faThumbsOUp}{EN PRATIQUE}{#1}
}

\newcommand\CInfo[1]{
     \CBox{Sepia}{\faArrowCircleRight}{REMARQUE}{#1}
}

\newcommand\CRedac[1]{
     \CBox{PineGreen}{\faEdit}{BIEN R\'EDIGER}{#1}
}

\newcommand\CError[1]{
     \CBox{Red}{\faExclamationTriangle}{ATTENTION}{#1}
}

\newcommand\TitreExo[2]{
\needspace{4\baselineskip}
 {\sffamily\large EXERCICE #1\ (\emph{#2 points})}
\vspace{5mm}
}

\newcommand\img[2]{
          \includegraphics[width=#2\paperwidth]{\imgdir#1}
}

\newcommand\imgsvg[2]{
       \begin{center}   \includegraphics[width=#2\paperwidth]{\imgsvgdir#1} \end{center}
}


\newcommand\Lien[2]{
     \href{#1}{#2 \tiny \faExternalLink}
}
\newcommand\mcLien[2]{
     \href{https~://www.maths-cours.fr/#1}{#2 \tiny \faExternalLink}
}

\newcommand{\euro}{\eurologo{}}

%================================================================================================================================
%
% Macros - Environement
%
%================================================================================================================================

\newenvironment{tex}{ %
}
{%
}

\newenvironment{indente}{ %
	\setlength\parindent{10mm}
}

{
	\setlength\parindent{0mm}
}

\newenvironment{corrige}{%
     \needspace{3\baselineskip}
     \medskip
     \textbf{\textsc{Corrigé}}
     \medskip
}
{
}

\newenvironment{extern}{%
     \begin{center}
     }
     {
     \end{center}
}

\NewEnviron{code}{%
	\par
     \boite{gray}{\texttt{%
     \BODY
     }}
     \par
}

\newenvironment{vbloc}{% boite sans cadre empeche saut de page
     \begin{minipage}[t]{\linewidth}
     }
     {
     \end{minipage}
}
\NewEnviron{h2}{%
    \needspace{3\baselineskip}
    \vspace{0.6cm}
	\noindent \MakeUppercase{\sffamily \large \BODY}
	\vspace{1mm}\textcolor{mcgris}{\hrule}\vspace{0.4cm}
	\par
}{}

\NewEnviron{h3}{%
    \needspace{3\baselineskip}
	\vspace{5mm}
	\textsc{\BODY}
	\par
}

\NewEnviron{margeneg}{ %
\begin{addmargin}[-1cm]{0cm}
\BODY
\end{addmargin}
}

\NewEnviron{html}{%
}

\begin{document}
\meta{url}{/exercices/algorithme-deuclide-simplifications/}
\meta{pid}{1664}
\meta{titre}{PGCD~: Simplification d'une fraction}
\meta{type}{exercices}
\begin{enumerate}
     \item
     Décomposer les entiers 180 et 252 en produits de facteurs premiers.
     \item
     En déduire le PGCD de 180 et 252.
     \item
     Simplifier la fraction $A=\frac{180}{252}$
\end{enumerate}
\begin{corrige}
     \begin{enumerate}
          \item
          $180 = 2 \times 2 \times 3 \times 3 \times 5 = 2^2 \times 3^2 \times 5 $
          \\
          $252 = 2 \times 2 \times 3 \times 3 \times 7 = 2^2 \times 3^2 \times 7$
          \item
          Le PGCD s'obtient en sélectionnant les facteurs communs aux deux décompositions (avec les plus petits exposants)~; par conséquent~:
          $ PGCD(180, 252) = 2^2 \times 3^2 = 4 \times 9 = 36 $
          \item
          $A$ se simplifie donc par $36$~:
          \par
          $A=\frac{180}{252}=\frac{36\times 5}{36\times 7}=\frac{5}{7}$
     \end{enumerate}
\end{corrige}

\end{document}
µ
\documentclass[a4paper]{article}

%================================================================================================================================
%
% Packages
%
%================================================================================================================================

\usepackage[T1]{fontenc} 	% pour caractères accentués
\usepackage[utf8]{inputenc}  % encodage utf8
\usepackage[french]{babel}	% langue : français
\usepackage{fourier}			% caractères plus lisibles
\usepackage[dvipsnames]{xcolor} % couleurs
\usepackage{fancyhdr}		% réglage header footer
\usepackage{needspace}		% empêcher sauts de page mal placés
\usepackage{graphicx}		% pour inclure des graphiques
\usepackage{enumitem,cprotect}		% personnalise les listes d'items (nécessaire pour ol, al ...)
\usepackage{hyperref}		% Liens hypertexte
\usepackage{pstricks,pst-all,pst-node,pstricks-add,pst-math,pst-plot,pst-tree,pst-eucl} % pstricks
\usepackage[a4paper,includeheadfoot,top=2cm,left=3cm, bottom=2cm,right=3cm]{geometry} % marges etc.
\usepackage{comment}			% commentaires multilignes
\usepackage{amsmath,environ} % maths (matrices, etc.)
\usepackage{amssymb,makeidx}
\usepackage{bm}				% bold maths
\usepackage{tabularx}		% tableaux
\usepackage{colortbl}		% tableaux en couleur
\usepackage{fontawesome}		% Fontawesome
\usepackage{environ}			% environment with command
\usepackage{fp}				% calculs pour ps-tricks
\usepackage{multido}			% pour ps tricks
\usepackage[np]{numprint}	% formattage nombre
\usepackage{tikz,tkz-tab} 			% package principal TikZ
\usepackage{pgfplots}   % axes
\usepackage{mathrsfs}    % cursives
\usepackage{calc}			% calcul taille boites
\usepackage[scaled=0.875]{helvet} % font sans serif
\usepackage{svg} % svg
\usepackage{scrextend} % local margin
\usepackage{scratch} %scratch
\usepackage{multicol} % colonnes
%\usepackage{infix-RPN,pst-func} % formule en notation polanaise inversée
\usepackage{listings}

%================================================================================================================================
%
% Réglages de base
%
%================================================================================================================================

\lstset{
language=Python,   % R code
literate=
{á}{{\'a}}1
{à}{{\`a}}1
{ã}{{\~a}}1
{é}{{\'e}}1
{è}{{\`e}}1
{ê}{{\^e}}1
{í}{{\'i}}1
{ó}{{\'o}}1
{õ}{{\~o}}1
{ú}{{\'u}}1
{ü}{{\"u}}1
{ç}{{\c{c}}}1
{~}{{ }}1
}


\definecolor{codegreen}{rgb}{0,0.6,0}
\definecolor{codegray}{rgb}{0.5,0.5,0.5}
\definecolor{codepurple}{rgb}{0.58,0,0.82}
\definecolor{backcolour}{rgb}{0.95,0.95,0.92}

\lstdefinestyle{mystyle}{
    backgroundcolor=\color{backcolour},   
    commentstyle=\color{codegreen},
    keywordstyle=\color{magenta},
    numberstyle=\tiny\color{codegray},
    stringstyle=\color{codepurple},
    basicstyle=\ttfamily\footnotesize,
    breakatwhitespace=false,         
    breaklines=true,                 
    captionpos=b,                    
    keepspaces=true,                 
    numbers=left,                    
xleftmargin=2em,
framexleftmargin=2em,            
    showspaces=false,                
    showstringspaces=false,
    showtabs=false,                  
    tabsize=2,
    upquote=true
}

\lstset{style=mystyle}


\lstset{style=mystyle}
\newcommand{\imgdir}{C:/laragon/www/newmc/assets/imgsvg/}
\newcommand{\imgsvgdir}{C:/laragon/www/newmc/assets/imgsvg/}

\definecolor{mcgris}{RGB}{220, 220, 220}% ancien~; pour compatibilité
\definecolor{mcbleu}{RGB}{52, 152, 219}
\definecolor{mcvert}{RGB}{125, 194, 70}
\definecolor{mcmauve}{RGB}{154, 0, 215}
\definecolor{mcorange}{RGB}{255, 96, 0}
\definecolor{mcturquoise}{RGB}{0, 153, 153}
\definecolor{mcrouge}{RGB}{255, 0, 0}
\definecolor{mclightvert}{RGB}{205, 234, 190}

\definecolor{gris}{RGB}{220, 220, 220}
\definecolor{bleu}{RGB}{52, 152, 219}
\definecolor{vert}{RGB}{125, 194, 70}
\definecolor{mauve}{RGB}{154, 0, 215}
\definecolor{orange}{RGB}{255, 96, 0}
\definecolor{turquoise}{RGB}{0, 153, 153}
\definecolor{rouge}{RGB}{255, 0, 0}
\definecolor{lightvert}{RGB}{205, 234, 190}
\setitemize[0]{label=\color{lightvert}  $\bullet$}

\pagestyle{fancy}
\renewcommand{\headrulewidth}{0.2pt}
\fancyhead[L]{maths-cours.fr}
\fancyhead[R]{\thepage}
\renewcommand{\footrulewidth}{0.2pt}
\fancyfoot[C]{}

\newcolumntype{C}{>{\centering\arraybackslash}X}
\newcolumntype{s}{>{\hsize=.35\hsize\arraybackslash}X}

\setlength{\parindent}{0pt}		 
\setlength{\parskip}{3mm}
\setlength{\headheight}{1cm}

\def\ebook{ebook}
\def\book{book}
\def\web{web}
\def\type{web}

\newcommand{\vect}[1]{\overrightarrow{\,\mathstrut#1\,}}

\def\Oij{$\left(\text{O}~;~\vect{\imath},~\vect{\jmath}\right)$}
\def\Oijk{$\left(\text{O}~;~\vect{\imath},~\vect{\jmath},~\vect{k}\right)$}
\def\Ouv{$\left(\text{O}~;~\vect{u},~\vect{v}\right)$}

\hypersetup{breaklinks=true, colorlinks = true, linkcolor = OliveGreen, urlcolor = OliveGreen, citecolor = OliveGreen, pdfauthor={Didier BONNEL - https://www.maths-cours.fr} } % supprime les bordures autour des liens

\renewcommand{\arg}[0]{\text{arg}}

\everymath{\displaystyle}

%================================================================================================================================
%
% Macros - Commandes
%
%================================================================================================================================

\newcommand\meta[2]{    			% Utilisé pour créer le post HTML.
	\def\titre{titre}
	\def\url{url}
	\def\arg{#1}
	\ifx\titre\arg
		\newcommand\maintitle{#2}
		\fancyhead[L]{#2}
		{\Large\sffamily \MakeUppercase{#2}}
		\vspace{1mm}\textcolor{mcvert}{\hrule}
	\fi 
	\ifx\url\arg
		\fancyfoot[L]{\href{https://www.maths-cours.fr#2}{\black \footnotesize{https://www.maths-cours.fr#2}}}
	\fi 
}


\newcommand\TitreC[1]{    		% Titre centré
     \needspace{3\baselineskip}
     \begin{center}\textbf{#1}\end{center}
}

\newcommand\newpar{    		% paragraphe
     \par
}

\newcommand\nosp {    		% commande vide (pas d'espace)
}
\newcommand{\id}[1]{} %ignore

\newcommand\boite[2]{				% Boite simple sans titre
	\vspace{5mm}
	\setlength{\fboxrule}{0.2mm}
	\setlength{\fboxsep}{5mm}	
	\fcolorbox{#1}{#1!3}{\makebox[\linewidth-2\fboxrule-2\fboxsep]{
  		\begin{minipage}[t]{\linewidth-2\fboxrule-4\fboxsep}\setlength{\parskip}{3mm}
  			 #2
  		\end{minipage}
	}}
	\vspace{5mm}
}

\newcommand\CBox[4]{				% Boites
	\vspace{5mm}
	\setlength{\fboxrule}{0.2mm}
	\setlength{\fboxsep}{5mm}
	
	\fcolorbox{#1}{#1!3}{\makebox[\linewidth-2\fboxrule-2\fboxsep]{
		\begin{minipage}[t]{1cm}\setlength{\parskip}{3mm}
	  		\textcolor{#1}{\LARGE{#2}}    
 	 	\end{minipage}  
  		\begin{minipage}[t]{\linewidth-2\fboxrule-4\fboxsep}\setlength{\parskip}{3mm}
			\raisebox{1.2mm}{\normalsize\sffamily{\textcolor{#1}{#3}}}						
  			 #4
  		\end{minipage}
	}}
	\vspace{5mm}
}

\newcommand\cadre[3]{				% Boites convertible html
	\par
	\vspace{2mm}
	\setlength{\fboxrule}{0.1mm}
	\setlength{\fboxsep}{5mm}
	\fcolorbox{#1}{white}{\makebox[\linewidth-2\fboxrule-2\fboxsep]{
  		\begin{minipage}[t]{\linewidth-2\fboxrule-4\fboxsep}\setlength{\parskip}{3mm}
			\raisebox{-2.5mm}{\sffamily \small{\textcolor{#1}{\MakeUppercase{#2}}}}		
			\par		
  			 #3
 	 		\end{minipage}
	}}
		\vspace{2mm}
	\par
}

\newcommand\bloc[3]{				% Boites convertible html sans bordure
     \needspace{2\baselineskip}
     {\sffamily \small{\textcolor{#1}{\MakeUppercase{#2}}}}    
		\par		
  			 #3
		\par
}

\newcommand\CHelp[1]{
     \CBox{Plum}{\faInfoCircle}{À RETENIR}{#1}
}

\newcommand\CUp[1]{
     \CBox{NavyBlue}{\faThumbsOUp}{EN PRATIQUE}{#1}
}

\newcommand\CInfo[1]{
     \CBox{Sepia}{\faArrowCircleRight}{REMARQUE}{#1}
}

\newcommand\CRedac[1]{
     \CBox{PineGreen}{\faEdit}{BIEN R\'EDIGER}{#1}
}

\newcommand\CError[1]{
     \CBox{Red}{\faExclamationTriangle}{ATTENTION}{#1}
}

\newcommand\TitreExo[2]{
\needspace{4\baselineskip}
 {\sffamily\large EXERCICE #1\ (\emph{#2 points})}
\vspace{5mm}
}

\newcommand\img[2]{
          \includegraphics[width=#2\paperwidth]{\imgdir#1}
}

\newcommand\imgsvg[2]{
       \begin{center}   \includegraphics[width=#2\paperwidth]{\imgsvgdir#1} \end{center}
}


\newcommand\Lien[2]{
     \href{#1}{#2 \tiny \faExternalLink}
}
\newcommand\mcLien[2]{
     \href{https~://www.maths-cours.fr/#1}{#2 \tiny \faExternalLink}
}

\newcommand{\euro}{\eurologo{}}

%================================================================================================================================
%
% Macros - Environement
%
%================================================================================================================================

\newenvironment{tex}{ %
}
{%
}

\newenvironment{indente}{ %
	\setlength\parindent{10mm}
}

{
	\setlength\parindent{0mm}
}

\newenvironment{corrige}{%
     \needspace{3\baselineskip}
     \medskip
     \textbf{\textsc{Corrigé}}
     \medskip
}
{
}

\newenvironment{extern}{%
     \begin{center}
     }
     {
     \end{center}
}

\NewEnviron{code}{%
	\par
     \boite{gray}{\texttt{%
     \BODY
     }}
     \par
}

\newenvironment{vbloc}{% boite sans cadre empeche saut de page
     \begin{minipage}[t]{\linewidth}
     }
     {
     \end{minipage}
}
\NewEnviron{h2}{%
    \needspace{3\baselineskip}
    \vspace{0.6cm}
	\noindent \MakeUppercase{\sffamily \large \BODY}
	\vspace{1mm}\textcolor{mcgris}{\hrule}\vspace{0.4cm}
	\par
}{}

\NewEnviron{h3}{%
    \needspace{3\baselineskip}
	\vspace{5mm}
	\textsc{\BODY}
	\par
}

\NewEnviron{margeneg}{ %
\begin{addmargin}[-1cm]{0cm}
\BODY
\end{addmargin}
}

\NewEnviron{html}{%
}

\begin{document}
\meta{url}{/exercices/pgcd-brevet-2006-141104/}
\meta{pid}{1667}
\meta{titre}{Problème de partage (Brevet Nord 2006)}
\meta{type}{exercices}
\textit{(Brevet Groupement Nord 2006)}
\par
Pierre a gagné 84 sucettes et 147 bonbons à un jeu. Etant très généreux, et ayant surtout très
peur du dentiste, il décide de les partager avec des amis.
\par
Pour ne pas faire de jaloux, chacun doit avoir le même nombre de sucettes et le même nombre de bonbons.
\begin{enumerate}
     \item
     Combien de personnes au maximum pourront bénéficier de ces friandises (Pierre étant inclus dans ces personnes)~?
     \\Expliquer votre raisonnement.
     \item
     Combien de sucettes et de bonbons aura alors chaque personne~?
\end{enumerate}
\begin{corrige}
     \begin{enumerate}
          \item
          Pour qu'un partage équitable soit possible, il faut que le nombre de personnes divise le nombre de sucettes et le nombre de bonbons.
          \par
          Au maximum, ce nombre sera donc égal au PGCD de 84 et 147.
          \par
          La décomposition de 84 en produit de facteurs premiers est~:
          \\
          $ 84 = 2 \times 2 \times 3 \times 7 $
          \par
          La décomposition de 147 est~:
          \\
          $ 147 = 3 \times 7 \times 7 $
          \par
          Le PGCD de 147 et 84 est donc $3 \times 7 = 21$ .
          \par
          21 personnes au maximum pourront donc bénéficier de ces friandises.
          \item
          84 ÷ 21 = 4
          \par
          147 ÷ 21 = 7
          \par
          Chacune des 21 personnes aura alors 4 sucettes et 7 bonbons.
     \end{enumerate}
\end{corrige}

\end{document}
µ
\documentclass[a4paper]{article}

%================================================================================================================================
%
% Packages
%
%================================================================================================================================

\usepackage[T1]{fontenc} 	% pour caractères accentués
\usepackage[utf8]{inputenc}  % encodage utf8
\usepackage[french]{babel}	% langue : français
\usepackage{fourier}			% caractères plus lisibles
\usepackage[dvipsnames]{xcolor} % couleurs
\usepackage{fancyhdr}		% réglage header footer
\usepackage{needspace}		% empêcher sauts de page mal placés
\usepackage{graphicx}		% pour inclure des graphiques
\usepackage{enumitem,cprotect}		% personnalise les listes d'items (nécessaire pour ol, al ...)
\usepackage{hyperref}		% Liens hypertexte
\usepackage{pstricks,pst-all,pst-node,pstricks-add,pst-math,pst-plot,pst-tree,pst-eucl} % pstricks
\usepackage[a4paper,includeheadfoot,top=2cm,left=3cm, bottom=2cm,right=3cm]{geometry} % marges etc.
\usepackage{comment}			% commentaires multilignes
\usepackage{amsmath,environ} % maths (matrices, etc.)
\usepackage{amssymb,makeidx}
\usepackage{bm}				% bold maths
\usepackage{tabularx}		% tableaux
\usepackage{colortbl}		% tableaux en couleur
\usepackage{fontawesome}		% Fontawesome
\usepackage{environ}			% environment with command
\usepackage{fp}				% calculs pour ps-tricks
\usepackage{multido}			% pour ps tricks
\usepackage[np]{numprint}	% formattage nombre
\usepackage{tikz,tkz-tab} 			% package principal TikZ
\usepackage{pgfplots}   % axes
\usepackage{mathrsfs}    % cursives
\usepackage{calc}			% calcul taille boites
\usepackage[scaled=0.875]{helvet} % font sans serif
\usepackage{svg} % svg
\usepackage{scrextend} % local margin
\usepackage{scratch} %scratch
\usepackage{multicol} % colonnes
%\usepackage{infix-RPN,pst-func} % formule en notation polanaise inversée
\usepackage{listings}

%================================================================================================================================
%
% Réglages de base
%
%================================================================================================================================

\lstset{
language=Python,   % R code
literate=
{á}{{\'a}}1
{à}{{\`a}}1
{ã}{{\~a}}1
{é}{{\'e}}1
{è}{{\`e}}1
{ê}{{\^e}}1
{í}{{\'i}}1
{ó}{{\'o}}1
{õ}{{\~o}}1
{ú}{{\'u}}1
{ü}{{\"u}}1
{ç}{{\c{c}}}1
{~}{{ }}1
}


\definecolor{codegreen}{rgb}{0,0.6,0}
\definecolor{codegray}{rgb}{0.5,0.5,0.5}
\definecolor{codepurple}{rgb}{0.58,0,0.82}
\definecolor{backcolour}{rgb}{0.95,0.95,0.92}

\lstdefinestyle{mystyle}{
    backgroundcolor=\color{backcolour},   
    commentstyle=\color{codegreen},
    keywordstyle=\color{magenta},
    numberstyle=\tiny\color{codegray},
    stringstyle=\color{codepurple},
    basicstyle=\ttfamily\footnotesize,
    breakatwhitespace=false,         
    breaklines=true,                 
    captionpos=b,                    
    keepspaces=true,                 
    numbers=left,                    
xleftmargin=2em,
framexleftmargin=2em,            
    showspaces=false,                
    showstringspaces=false,
    showtabs=false,                  
    tabsize=2,
    upquote=true
}

\lstset{style=mystyle}


\lstset{style=mystyle}
\newcommand{\imgdir}{C:/laragon/www/newmc/assets/imgsvg/}
\newcommand{\imgsvgdir}{C:/laragon/www/newmc/assets/imgsvg/}

\definecolor{mcgris}{RGB}{220, 220, 220}% ancien~; pour compatibilité
\definecolor{mcbleu}{RGB}{52, 152, 219}
\definecolor{mcvert}{RGB}{125, 194, 70}
\definecolor{mcmauve}{RGB}{154, 0, 215}
\definecolor{mcorange}{RGB}{255, 96, 0}
\definecolor{mcturquoise}{RGB}{0, 153, 153}
\definecolor{mcrouge}{RGB}{255, 0, 0}
\definecolor{mclightvert}{RGB}{205, 234, 190}

\definecolor{gris}{RGB}{220, 220, 220}
\definecolor{bleu}{RGB}{52, 152, 219}
\definecolor{vert}{RGB}{125, 194, 70}
\definecolor{mauve}{RGB}{154, 0, 215}
\definecolor{orange}{RGB}{255, 96, 0}
\definecolor{turquoise}{RGB}{0, 153, 153}
\definecolor{rouge}{RGB}{255, 0, 0}
\definecolor{lightvert}{RGB}{205, 234, 190}
\setitemize[0]{label=\color{lightvert}  $\bullet$}

\pagestyle{fancy}
\renewcommand{\headrulewidth}{0.2pt}
\fancyhead[L]{maths-cours.fr}
\fancyhead[R]{\thepage}
\renewcommand{\footrulewidth}{0.2pt}
\fancyfoot[C]{}

\newcolumntype{C}{>{\centering\arraybackslash}X}
\newcolumntype{s}{>{\hsize=.35\hsize\arraybackslash}X}

\setlength{\parindent}{0pt}		 
\setlength{\parskip}{3mm}
\setlength{\headheight}{1cm}

\def\ebook{ebook}
\def\book{book}
\def\web{web}
\def\type{web}

\newcommand{\vect}[1]{\overrightarrow{\,\mathstrut#1\,}}

\def\Oij{$\left(\text{O}~;~\vect{\imath},~\vect{\jmath}\right)$}
\def\Oijk{$\left(\text{O}~;~\vect{\imath},~\vect{\jmath},~\vect{k}\right)$}
\def\Ouv{$\left(\text{O}~;~\vect{u},~\vect{v}\right)$}

\hypersetup{breaklinks=true, colorlinks = true, linkcolor = OliveGreen, urlcolor = OliveGreen, citecolor = OliveGreen, pdfauthor={Didier BONNEL - https://www.maths-cours.fr} } % supprime les bordures autour des liens

\renewcommand{\arg}[0]{\text{arg}}

\everymath{\displaystyle}

%================================================================================================================================
%
% Macros - Commandes
%
%================================================================================================================================

\newcommand\meta[2]{    			% Utilisé pour créer le post HTML.
	\def\titre{titre}
	\def\url{url}
	\def\arg{#1}
	\ifx\titre\arg
		\newcommand\maintitle{#2}
		\fancyhead[L]{#2}
		{\Large\sffamily \MakeUppercase{#2}}
		\vspace{1mm}\textcolor{mcvert}{\hrule}
	\fi 
	\ifx\url\arg
		\fancyfoot[L]{\href{https://www.maths-cours.fr#2}{\black \footnotesize{https://www.maths-cours.fr#2}}}
	\fi 
}


\newcommand\TitreC[1]{    		% Titre centré
     \needspace{3\baselineskip}
     \begin{center}\textbf{#1}\end{center}
}

\newcommand\newpar{    		% paragraphe
     \par
}

\newcommand\nosp {    		% commande vide (pas d'espace)
}
\newcommand{\id}[1]{} %ignore

\newcommand\boite[2]{				% Boite simple sans titre
	\vspace{5mm}
	\setlength{\fboxrule}{0.2mm}
	\setlength{\fboxsep}{5mm}	
	\fcolorbox{#1}{#1!3}{\makebox[\linewidth-2\fboxrule-2\fboxsep]{
  		\begin{minipage}[t]{\linewidth-2\fboxrule-4\fboxsep}\setlength{\parskip}{3mm}
  			 #2
  		\end{minipage}
	}}
	\vspace{5mm}
}

\newcommand\CBox[4]{				% Boites
	\vspace{5mm}
	\setlength{\fboxrule}{0.2mm}
	\setlength{\fboxsep}{5mm}
	
	\fcolorbox{#1}{#1!3}{\makebox[\linewidth-2\fboxrule-2\fboxsep]{
		\begin{minipage}[t]{1cm}\setlength{\parskip}{3mm}
	  		\textcolor{#1}{\LARGE{#2}}    
 	 	\end{minipage}  
  		\begin{minipage}[t]{\linewidth-2\fboxrule-4\fboxsep}\setlength{\parskip}{3mm}
			\raisebox{1.2mm}{\normalsize\sffamily{\textcolor{#1}{#3}}}						
  			 #4
  		\end{minipage}
	}}
	\vspace{5mm}
}

\newcommand\cadre[3]{				% Boites convertible html
	\par
	\vspace{2mm}
	\setlength{\fboxrule}{0.1mm}
	\setlength{\fboxsep}{5mm}
	\fcolorbox{#1}{white}{\makebox[\linewidth-2\fboxrule-2\fboxsep]{
  		\begin{minipage}[t]{\linewidth-2\fboxrule-4\fboxsep}\setlength{\parskip}{3mm}
			\raisebox{-2.5mm}{\sffamily \small{\textcolor{#1}{\MakeUppercase{#2}}}}		
			\par		
  			 #3
 	 		\end{minipage}
	}}
		\vspace{2mm}
	\par
}

\newcommand\bloc[3]{				% Boites convertible html sans bordure
     \needspace{2\baselineskip}
     {\sffamily \small{\textcolor{#1}{\MakeUppercase{#2}}}}    
		\par		
  			 #3
		\par
}

\newcommand\CHelp[1]{
     \CBox{Plum}{\faInfoCircle}{À RETENIR}{#1}
}

\newcommand\CUp[1]{
     \CBox{NavyBlue}{\faThumbsOUp}{EN PRATIQUE}{#1}
}

\newcommand\CInfo[1]{
     \CBox{Sepia}{\faArrowCircleRight}{REMARQUE}{#1}
}

\newcommand\CRedac[1]{
     \CBox{PineGreen}{\faEdit}{BIEN R\'EDIGER}{#1}
}

\newcommand\CError[1]{
     \CBox{Red}{\faExclamationTriangle}{ATTENTION}{#1}
}

\newcommand\TitreExo[2]{
\needspace{4\baselineskip}
 {\sffamily\large EXERCICE #1\ (\emph{#2 points})}
\vspace{5mm}
}

\newcommand\img[2]{
          \includegraphics[width=#2\paperwidth]{\imgdir#1}
}

\newcommand\imgsvg[2]{
       \begin{center}   \includegraphics[width=#2\paperwidth]{\imgsvgdir#1} \end{center}
}


\newcommand\Lien[2]{
     \href{#1}{#2 \tiny \faExternalLink}
}
\newcommand\mcLien[2]{
     \href{https~://www.maths-cours.fr/#1}{#2 \tiny \faExternalLink}
}

\newcommand{\euro}{\eurologo{}}

%================================================================================================================================
%
% Macros - Environement
%
%================================================================================================================================

\newenvironment{tex}{ %
}
{%
}

\newenvironment{indente}{ %
	\setlength\parindent{10mm}
}

{
	\setlength\parindent{0mm}
}

\newenvironment{corrige}{%
     \needspace{3\baselineskip}
     \medskip
     \textbf{\textsc{Corrigé}}
     \medskip
}
{
}

\newenvironment{extern}{%
     \begin{center}
     }
     {
     \end{center}
}

\NewEnviron{code}{%
	\par
     \boite{gray}{\texttt{%
     \BODY
     }}
     \par
}

\newenvironment{vbloc}{% boite sans cadre empeche saut de page
     \begin{minipage}[t]{\linewidth}
     }
     {
     \end{minipage}
}
\NewEnviron{h2}{%
    \needspace{3\baselineskip}
    \vspace{0.6cm}
	\noindent \MakeUppercase{\sffamily \large \BODY}
	\vspace{1mm}\textcolor{mcgris}{\hrule}\vspace{0.4cm}
	\par
}{}

\NewEnviron{h3}{%
    \needspace{3\baselineskip}
	\vspace{5mm}
	\textsc{\BODY}
	\par
}

\NewEnviron{margeneg}{ %
\begin{addmargin}[-1cm]{0cm}
\BODY
\end{addmargin}
}

\NewEnviron{html}{%
}

\begin{document}
\meta{url}{/exercices/pgcd-brevet-2005-141112/}
\meta{pid}{1669}
\meta{titre}{PGCD - Décompte de carreaux (Brevet Besançon 2005)}
\meta{type}{exercices}
\textit{(Brevet Besançon 2005)}
\begin{enumerate}
     \item
     Calculer le PGCD des nombres 135 et 210.
     \item
     Dans une salle de bains, on veut recouvrir le mur situé au dessus de la baignoire avec un
     nombre entier de carreaux de faïence de forme carrée dont le côté est un nombre entier de centimètres le plus grand possible.
     \begin{enumerate}[label=\alph*.]
          \item
          Déterminer la longueur, en cm, du côté d'un carreau, sachant que le mur mesure 210 cm de hauteur et 135 cm de largeur.
          \item
          Combien faudra-t-il alors de carreaux~?
     \end{enumerate}
\end{enumerate}
\begin{corrige}
     Solution rédigée par _Orion_ (modifiée pour correspondre aux programmes actuels)
     \begin{enumerate}
          \item
          Les décompositions en produit de facteurs premiers de 135 et 210 sont~:
          \par
          $ 210 = 2 \times 3 \times 5 \times 7 $
          \\
          $ 135 = 3 \times 3 \times 3 \times 5 $
          \par
          En recherchant les facteurs communs on trouve~:
          \\
          $ PGCD(210,135) = 3 \times 5 =15 $
          \item
          \begin{enumerate}[label=\alph*.]
               \item
               Comme on souhaite mettre un nombre entier de carreaux, on cherche à ce que la taille du côté de celui-ci soit un diviseur de la largeur mais aussi la hauteur du mur. Comme on cherche également des carreaux de plus grand côté possible cela revient à chercher le PGCD des dimensions du mur qui est 15 cm comme nous avons pu le calculer dans la première question.
               \item
               Il faudra alors (210/15) $ \times $ (135/15) = 14 $ \times $ 9 = 126 carreaux.
          \end{enumerate}
     \end{corrige}
     
\end{document}
µ
\documentclass[a4paper]{article}

%================================================================================================================================
%
% Packages
%
%================================================================================================================================

\usepackage[T1]{fontenc} 	% pour caractères accentués
\usepackage[utf8]{inputenc}  % encodage utf8
\usepackage[french]{babel}	% langue : français
\usepackage{fourier}			% caractères plus lisibles
\usepackage[dvipsnames]{xcolor} % couleurs
\usepackage{fancyhdr}		% réglage header footer
\usepackage{needspace}		% empêcher sauts de page mal placés
\usepackage{graphicx}		% pour inclure des graphiques
\usepackage{enumitem,cprotect}		% personnalise les listes d'items (nécessaire pour ol, al ...)
\usepackage{hyperref}		% Liens hypertexte
\usepackage{pstricks,pst-all,pst-node,pstricks-add,pst-math,pst-plot,pst-tree,pst-eucl} % pstricks
\usepackage[a4paper,includeheadfoot,top=2cm,left=3cm, bottom=2cm,right=3cm]{geometry} % marges etc.
\usepackage{comment}			% commentaires multilignes
\usepackage{amsmath,environ} % maths (matrices, etc.)
\usepackage{amssymb,makeidx}
\usepackage{bm}				% bold maths
\usepackage{tabularx}		% tableaux
\usepackage{colortbl}		% tableaux en couleur
\usepackage{fontawesome}		% Fontawesome
\usepackage{environ}			% environment with command
\usepackage{fp}				% calculs pour ps-tricks
\usepackage{multido}			% pour ps tricks
\usepackage[np]{numprint}	% formattage nombre
\usepackage{tikz,tkz-tab} 			% package principal TikZ
\usepackage{pgfplots}   % axes
\usepackage{mathrsfs}    % cursives
\usepackage{calc}			% calcul taille boites
\usepackage[scaled=0.875]{helvet} % font sans serif
\usepackage{svg} % svg
\usepackage{scrextend} % local margin
\usepackage{scratch} %scratch
\usepackage{multicol} % colonnes
%\usepackage{infix-RPN,pst-func} % formule en notation polanaise inversée
\usepackage{listings}

%================================================================================================================================
%
% Réglages de base
%
%================================================================================================================================

\lstset{
language=Python,   % R code
literate=
{á}{{\'a}}1
{à}{{\`a}}1
{ã}{{\~a}}1
{é}{{\'e}}1
{è}{{\`e}}1
{ê}{{\^e}}1
{í}{{\'i}}1
{ó}{{\'o}}1
{õ}{{\~o}}1
{ú}{{\'u}}1
{ü}{{\"u}}1
{ç}{{\c{c}}}1
{~}{{ }}1
}


\definecolor{codegreen}{rgb}{0,0.6,0}
\definecolor{codegray}{rgb}{0.5,0.5,0.5}
\definecolor{codepurple}{rgb}{0.58,0,0.82}
\definecolor{backcolour}{rgb}{0.95,0.95,0.92}

\lstdefinestyle{mystyle}{
    backgroundcolor=\color{backcolour},   
    commentstyle=\color{codegreen},
    keywordstyle=\color{magenta},
    numberstyle=\tiny\color{codegray},
    stringstyle=\color{codepurple},
    basicstyle=\ttfamily\footnotesize,
    breakatwhitespace=false,         
    breaklines=true,                 
    captionpos=b,                    
    keepspaces=true,                 
    numbers=left,                    
xleftmargin=2em,
framexleftmargin=2em,            
    showspaces=false,                
    showstringspaces=false,
    showtabs=false,                  
    tabsize=2,
    upquote=true
}

\lstset{style=mystyle}


\lstset{style=mystyle}
\newcommand{\imgdir}{C:/laragon/www/newmc/assets/imgsvg/}
\newcommand{\imgsvgdir}{C:/laragon/www/newmc/assets/imgsvg/}

\definecolor{mcgris}{RGB}{220, 220, 220}% ancien~; pour compatibilité
\definecolor{mcbleu}{RGB}{52, 152, 219}
\definecolor{mcvert}{RGB}{125, 194, 70}
\definecolor{mcmauve}{RGB}{154, 0, 215}
\definecolor{mcorange}{RGB}{255, 96, 0}
\definecolor{mcturquoise}{RGB}{0, 153, 153}
\definecolor{mcrouge}{RGB}{255, 0, 0}
\definecolor{mclightvert}{RGB}{205, 234, 190}

\definecolor{gris}{RGB}{220, 220, 220}
\definecolor{bleu}{RGB}{52, 152, 219}
\definecolor{vert}{RGB}{125, 194, 70}
\definecolor{mauve}{RGB}{154, 0, 215}
\definecolor{orange}{RGB}{255, 96, 0}
\definecolor{turquoise}{RGB}{0, 153, 153}
\definecolor{rouge}{RGB}{255, 0, 0}
\definecolor{lightvert}{RGB}{205, 234, 190}
\setitemize[0]{label=\color{lightvert}  $\bullet$}

\pagestyle{fancy}
\renewcommand{\headrulewidth}{0.2pt}
\fancyhead[L]{maths-cours.fr}
\fancyhead[R]{\thepage}
\renewcommand{\footrulewidth}{0.2pt}
\fancyfoot[C]{}

\newcolumntype{C}{>{\centering\arraybackslash}X}
\newcolumntype{s}{>{\hsize=.35\hsize\arraybackslash}X}

\setlength{\parindent}{0pt}		 
\setlength{\parskip}{3mm}
\setlength{\headheight}{1cm}

\def\ebook{ebook}
\def\book{book}
\def\web{web}
\def\type{web}

\newcommand{\vect}[1]{\overrightarrow{\,\mathstrut#1\,}}

\def\Oij{$\left(\text{O}~;~\vect{\imath},~\vect{\jmath}\right)$}
\def\Oijk{$\left(\text{O}~;~\vect{\imath},~\vect{\jmath},~\vect{k}\right)$}
\def\Ouv{$\left(\text{O}~;~\vect{u},~\vect{v}\right)$}

\hypersetup{breaklinks=true, colorlinks = true, linkcolor = OliveGreen, urlcolor = OliveGreen, citecolor = OliveGreen, pdfauthor={Didier BONNEL - https://www.maths-cours.fr} } % supprime les bordures autour des liens

\renewcommand{\arg}[0]{\text{arg}}

\everymath{\displaystyle}

%================================================================================================================================
%
% Macros - Commandes
%
%================================================================================================================================

\newcommand\meta[2]{    			% Utilisé pour créer le post HTML.
	\def\titre{titre}
	\def\url{url}
	\def\arg{#1}
	\ifx\titre\arg
		\newcommand\maintitle{#2}
		\fancyhead[L]{#2}
		{\Large\sffamily \MakeUppercase{#2}}
		\vspace{1mm}\textcolor{mcvert}{\hrule}
	\fi 
	\ifx\url\arg
		\fancyfoot[L]{\href{https://www.maths-cours.fr#2}{\black \footnotesize{https://www.maths-cours.fr#2}}}
	\fi 
}


\newcommand\TitreC[1]{    		% Titre centré
     \needspace{3\baselineskip}
     \begin{center}\textbf{#1}\end{center}
}

\newcommand\newpar{    		% paragraphe
     \par
}

\newcommand\nosp {    		% commande vide (pas d'espace)
}
\newcommand{\id}[1]{} %ignore

\newcommand\boite[2]{				% Boite simple sans titre
	\vspace{5mm}
	\setlength{\fboxrule}{0.2mm}
	\setlength{\fboxsep}{5mm}	
	\fcolorbox{#1}{#1!3}{\makebox[\linewidth-2\fboxrule-2\fboxsep]{
  		\begin{minipage}[t]{\linewidth-2\fboxrule-4\fboxsep}\setlength{\parskip}{3mm}
  			 #2
  		\end{minipage}
	}}
	\vspace{5mm}
}

\newcommand\CBox[4]{				% Boites
	\vspace{5mm}
	\setlength{\fboxrule}{0.2mm}
	\setlength{\fboxsep}{5mm}
	
	\fcolorbox{#1}{#1!3}{\makebox[\linewidth-2\fboxrule-2\fboxsep]{
		\begin{minipage}[t]{1cm}\setlength{\parskip}{3mm}
	  		\textcolor{#1}{\LARGE{#2}}    
 	 	\end{minipage}  
  		\begin{minipage}[t]{\linewidth-2\fboxrule-4\fboxsep}\setlength{\parskip}{3mm}
			\raisebox{1.2mm}{\normalsize\sffamily{\textcolor{#1}{#3}}}						
  			 #4
  		\end{minipage}
	}}
	\vspace{5mm}
}

\newcommand\cadre[3]{				% Boites convertible html
	\par
	\vspace{2mm}
	\setlength{\fboxrule}{0.1mm}
	\setlength{\fboxsep}{5mm}
	\fcolorbox{#1}{white}{\makebox[\linewidth-2\fboxrule-2\fboxsep]{
  		\begin{minipage}[t]{\linewidth-2\fboxrule-4\fboxsep}\setlength{\parskip}{3mm}
			\raisebox{-2.5mm}{\sffamily \small{\textcolor{#1}{\MakeUppercase{#2}}}}		
			\par		
  			 #3
 	 		\end{minipage}
	}}
		\vspace{2mm}
	\par
}

\newcommand\bloc[3]{				% Boites convertible html sans bordure
     \needspace{2\baselineskip}
     {\sffamily \small{\textcolor{#1}{\MakeUppercase{#2}}}}    
		\par		
  			 #3
		\par
}

\newcommand\CHelp[1]{
     \CBox{Plum}{\faInfoCircle}{À RETENIR}{#1}
}

\newcommand\CUp[1]{
     \CBox{NavyBlue}{\faThumbsOUp}{EN PRATIQUE}{#1}
}

\newcommand\CInfo[1]{
     \CBox{Sepia}{\faArrowCircleRight}{REMARQUE}{#1}
}

\newcommand\CRedac[1]{
     \CBox{PineGreen}{\faEdit}{BIEN R\'EDIGER}{#1}
}

\newcommand\CError[1]{
     \CBox{Red}{\faExclamationTriangle}{ATTENTION}{#1}
}

\newcommand\TitreExo[2]{
\needspace{4\baselineskip}
 {\sffamily\large EXERCICE #1\ (\emph{#2 points})}
\vspace{5mm}
}

\newcommand\img[2]{
          \includegraphics[width=#2\paperwidth]{\imgdir#1}
}

\newcommand\imgsvg[2]{
       \begin{center}   \includegraphics[width=#2\paperwidth]{\imgsvgdir#1} \end{center}
}


\newcommand\Lien[2]{
     \href{#1}{#2 \tiny \faExternalLink}
}
\newcommand\mcLien[2]{
     \href{https~://www.maths-cours.fr/#1}{#2 \tiny \faExternalLink}
}

\newcommand{\euro}{\eurologo{}}

%================================================================================================================================
%
% Macros - Environement
%
%================================================================================================================================

\newenvironment{tex}{ %
}
{%
}

\newenvironment{indente}{ %
	\setlength\parindent{10mm}
}

{
	\setlength\parindent{0mm}
}

\newenvironment{corrige}{%
     \needspace{3\baselineskip}
     \medskip
     \textbf{\textsc{Corrigé}}
     \medskip
}
{
}

\newenvironment{extern}{%
     \begin{center}
     }
     {
     \end{center}
}

\NewEnviron{code}{%
	\par
     \boite{gray}{\texttt{%
     \BODY
     }}
     \par
}

\newenvironment{vbloc}{% boite sans cadre empeche saut de page
     \begin{minipage}[t]{\linewidth}
     }
     {
     \end{minipage}
}
\NewEnviron{h2}{%
    \needspace{3\baselineskip}
    \vspace{0.6cm}
	\noindent \MakeUppercase{\sffamily \large \BODY}
	\vspace{1mm}\textcolor{mcgris}{\hrule}\vspace{0.4cm}
	\par
}{}

\NewEnviron{h3}{%
    \needspace{3\baselineskip}
	\vspace{5mm}
	\textsc{\BODY}
	\par
}

\NewEnviron{margeneg}{ %
\begin{addmargin}[-1cm]{0cm}
\BODY
\end{addmargin}
}

\NewEnviron{html}{%
}

\begin{document}
\meta{url}{/exercices/somme-ou-produit-141028/}
\meta{pid}{1674}
\meta{titre}{Somme ou produit ?}
\meta{type}{exercices}
%
Pour chacune des expressions suivantes, indiquer s'il s'agit d'une somme algébrique ou d'un produit.
\begin{enumerate}
     \item
     $7x+2$
     \item
     $3\left(5-x\right)$
     \item
     $4-2x$
     \item
     $\left(x+5\right)\left(x-2\right)$
     \item
     $4\left(x+2\right)+2\left(3x-1\right)$
     \item
     $\left(3x-2\right)^{2}$
\end{enumerate}
\begin{corrige}
     L'opération principale (la moins prioritaire donc celle que l'on exécute en dernier et qui donne le résultat final)  est représentée en rouge ci-dessous :
     \begin{enumerate}
          \item
          $7x+2$
          \par
          $7x\color{red}{+}2$ est une somme algébrique
          \item
          $3\left(5-x\right)$
          \par
          $3\color{red}{\times }\left(5-x\right)$ est un produit
          \item
          $4-2x$
          \par
          $4\color{red}{-}2x$ est une somme algébrique
          \item
          $\left(x+5\right)\left(x-2\right)$
          \par
          $\left(x+5\right)\color{red}{\times }\left(x-2\right)$ est un produit
          \item
          $4\left(x+2\right)+2\left(3x-1\right)$
          \par
          $4\left(x+2\right)\color{red}{+}2\left(3x-1\right)$ est une somme algébrique
          \item
          $\left(3x-2\right)^{2}$
          \par
          $\left(3x-2\right)^{2}=\left(3x-2\right)\color{red}{\times }\left(3x-2\right)$ est un produit
     \end{enumerate}
\end{corrige}

\end{document}

µ
\documentclass[a4paper]{article}

%================================================================================================================================
%
% Packages
%
%================================================================================================================================

\usepackage[T1]{fontenc} 	% pour caractères accentués
\usepackage[utf8]{inputenc}  % encodage utf8
\usepackage[french]{babel}	% langue : français
\usepackage{fourier}			% caractères plus lisibles
\usepackage[dvipsnames]{xcolor} % couleurs
\usepackage{fancyhdr}		% réglage header footer
\usepackage{needspace}		% empêcher sauts de page mal placés
\usepackage{graphicx}		% pour inclure des graphiques
\usepackage{enumitem,cprotect}		% personnalise les listes d'items (nécessaire pour ol, al ...)
\usepackage{hyperref}		% Liens hypertexte
\usepackage{pstricks,pst-all,pst-node,pstricks-add,pst-math,pst-plot,pst-tree,pst-eucl} % pstricks
\usepackage[a4paper,includeheadfoot,top=2cm,left=3cm, bottom=2cm,right=3cm]{geometry} % marges etc.
\usepackage{comment}			% commentaires multilignes
\usepackage{amsmath,environ} % maths (matrices, etc.)
\usepackage{amssymb,makeidx}
\usepackage{bm}				% bold maths
\usepackage{tabularx}		% tableaux
\usepackage{colortbl}		% tableaux en couleur
\usepackage{fontawesome}		% Fontawesome
\usepackage{environ}			% environment with command
\usepackage{fp}				% calculs pour ps-tricks
\usepackage{multido}			% pour ps tricks
\usepackage[np]{numprint}	% formattage nombre
\usepackage{tikz,tkz-tab} 			% package principal TikZ
\usepackage{pgfplots}   % axes
\usepackage{mathrsfs}    % cursives
\usepackage{calc}			% calcul taille boites
\usepackage[scaled=0.875]{helvet} % font sans serif
\usepackage{svg} % svg
\usepackage{scrextend} % local margin
\usepackage{scratch} %scratch
\usepackage{multicol} % colonnes
%\usepackage{infix-RPN,pst-func} % formule en notation polanaise inversée
\usepackage{listings}

%================================================================================================================================
%
% Réglages de base
%
%================================================================================================================================

\lstset{
language=Python,   % R code
literate=
{á}{{\'a}}1
{à}{{\`a}}1
{ã}{{\~a}}1
{é}{{\'e}}1
{è}{{\`e}}1
{ê}{{\^e}}1
{í}{{\'i}}1
{ó}{{\'o}}1
{õ}{{\~o}}1
{ú}{{\'u}}1
{ü}{{\"u}}1
{ç}{{\c{c}}}1
{~}{{ }}1
}


\definecolor{codegreen}{rgb}{0,0.6,0}
\definecolor{codegray}{rgb}{0.5,0.5,0.5}
\definecolor{codepurple}{rgb}{0.58,0,0.82}
\definecolor{backcolour}{rgb}{0.95,0.95,0.92}

\lstdefinestyle{mystyle}{
    backgroundcolor=\color{backcolour},   
    commentstyle=\color{codegreen},
    keywordstyle=\color{magenta},
    numberstyle=\tiny\color{codegray},
    stringstyle=\color{codepurple},
    basicstyle=\ttfamily\footnotesize,
    breakatwhitespace=false,         
    breaklines=true,                 
    captionpos=b,                    
    keepspaces=true,                 
    numbers=left,                    
xleftmargin=2em,
framexleftmargin=2em,            
    showspaces=false,                
    showstringspaces=false,
    showtabs=false,                  
    tabsize=2,
    upquote=true
}

\lstset{style=mystyle}


\lstset{style=mystyle}
\newcommand{\imgdir}{C:/laragon/www/newmc/assets/imgsvg/}
\newcommand{\imgsvgdir}{C:/laragon/www/newmc/assets/imgsvg/}

\definecolor{mcgris}{RGB}{220, 220, 220}% ancien~; pour compatibilité
\definecolor{mcbleu}{RGB}{52, 152, 219}
\definecolor{mcvert}{RGB}{125, 194, 70}
\definecolor{mcmauve}{RGB}{154, 0, 215}
\definecolor{mcorange}{RGB}{255, 96, 0}
\definecolor{mcturquoise}{RGB}{0, 153, 153}
\definecolor{mcrouge}{RGB}{255, 0, 0}
\definecolor{mclightvert}{RGB}{205, 234, 190}

\definecolor{gris}{RGB}{220, 220, 220}
\definecolor{bleu}{RGB}{52, 152, 219}
\definecolor{vert}{RGB}{125, 194, 70}
\definecolor{mauve}{RGB}{154, 0, 215}
\definecolor{orange}{RGB}{255, 96, 0}
\definecolor{turquoise}{RGB}{0, 153, 153}
\definecolor{rouge}{RGB}{255, 0, 0}
\definecolor{lightvert}{RGB}{205, 234, 190}
\setitemize[0]{label=\color{lightvert}  $\bullet$}

\pagestyle{fancy}
\renewcommand{\headrulewidth}{0.2pt}
\fancyhead[L]{maths-cours.fr}
\fancyhead[R]{\thepage}
\renewcommand{\footrulewidth}{0.2pt}
\fancyfoot[C]{}

\newcolumntype{C}{>{\centering\arraybackslash}X}
\newcolumntype{s}{>{\hsize=.35\hsize\arraybackslash}X}

\setlength{\parindent}{0pt}		 
\setlength{\parskip}{3mm}
\setlength{\headheight}{1cm}

\def\ebook{ebook}
\def\book{book}
\def\web{web}
\def\type{web}

\newcommand{\vect}[1]{\overrightarrow{\,\mathstrut#1\,}}

\def\Oij{$\left(\text{O}~;~\vect{\imath},~\vect{\jmath}\right)$}
\def\Oijk{$\left(\text{O}~;~\vect{\imath},~\vect{\jmath},~\vect{k}\right)$}
\def\Ouv{$\left(\text{O}~;~\vect{u},~\vect{v}\right)$}

\hypersetup{breaklinks=true, colorlinks = true, linkcolor = OliveGreen, urlcolor = OliveGreen, citecolor = OliveGreen, pdfauthor={Didier BONNEL - https://www.maths-cours.fr} } % supprime les bordures autour des liens

\renewcommand{\arg}[0]{\text{arg}}

\everymath{\displaystyle}

%================================================================================================================================
%
% Macros - Commandes
%
%================================================================================================================================

\newcommand\meta[2]{    			% Utilisé pour créer le post HTML.
	\def\titre{titre}
	\def\url{url}
	\def\arg{#1}
	\ifx\titre\arg
		\newcommand\maintitle{#2}
		\fancyhead[L]{#2}
		{\Large\sffamily \MakeUppercase{#2}}
		\vspace{1mm}\textcolor{mcvert}{\hrule}
	\fi 
	\ifx\url\arg
		\fancyfoot[L]{\href{https://www.maths-cours.fr#2}{\black \footnotesize{https://www.maths-cours.fr#2}}}
	\fi 
}


\newcommand\TitreC[1]{    		% Titre centré
     \needspace{3\baselineskip}
     \begin{center}\textbf{#1}\end{center}
}

\newcommand\newpar{    		% paragraphe
     \par
}

\newcommand\nosp {    		% commande vide (pas d'espace)
}
\newcommand{\id}[1]{} %ignore

\newcommand\boite[2]{				% Boite simple sans titre
	\vspace{5mm}
	\setlength{\fboxrule}{0.2mm}
	\setlength{\fboxsep}{5mm}	
	\fcolorbox{#1}{#1!3}{\makebox[\linewidth-2\fboxrule-2\fboxsep]{
  		\begin{minipage}[t]{\linewidth-2\fboxrule-4\fboxsep}\setlength{\parskip}{3mm}
  			 #2
  		\end{minipage}
	}}
	\vspace{5mm}
}

\newcommand\CBox[4]{				% Boites
	\vspace{5mm}
	\setlength{\fboxrule}{0.2mm}
	\setlength{\fboxsep}{5mm}
	
	\fcolorbox{#1}{#1!3}{\makebox[\linewidth-2\fboxrule-2\fboxsep]{
		\begin{minipage}[t]{1cm}\setlength{\parskip}{3mm}
	  		\textcolor{#1}{\LARGE{#2}}    
 	 	\end{minipage}  
  		\begin{minipage}[t]{\linewidth-2\fboxrule-4\fboxsep}\setlength{\parskip}{3mm}
			\raisebox{1.2mm}{\normalsize\sffamily{\textcolor{#1}{#3}}}						
  			 #4
  		\end{minipage}
	}}
	\vspace{5mm}
}

\newcommand\cadre[3]{				% Boites convertible html
	\par
	\vspace{2mm}
	\setlength{\fboxrule}{0.1mm}
	\setlength{\fboxsep}{5mm}
	\fcolorbox{#1}{white}{\makebox[\linewidth-2\fboxrule-2\fboxsep]{
  		\begin{minipage}[t]{\linewidth-2\fboxrule-4\fboxsep}\setlength{\parskip}{3mm}
			\raisebox{-2.5mm}{\sffamily \small{\textcolor{#1}{\MakeUppercase{#2}}}}		
			\par		
  			 #3
 	 		\end{minipage}
	}}
		\vspace{2mm}
	\par
}

\newcommand\bloc[3]{				% Boites convertible html sans bordure
     \needspace{2\baselineskip}
     {\sffamily \small{\textcolor{#1}{\MakeUppercase{#2}}}}    
		\par		
  			 #3
		\par
}

\newcommand\CHelp[1]{
     \CBox{Plum}{\faInfoCircle}{À RETENIR}{#1}
}

\newcommand\CUp[1]{
     \CBox{NavyBlue}{\faThumbsOUp}{EN PRATIQUE}{#1}
}

\newcommand\CInfo[1]{
     \CBox{Sepia}{\faArrowCircleRight}{REMARQUE}{#1}
}

\newcommand\CRedac[1]{
     \CBox{PineGreen}{\faEdit}{BIEN R\'EDIGER}{#1}
}

\newcommand\CError[1]{
     \CBox{Red}{\faExclamationTriangle}{ATTENTION}{#1}
}

\newcommand\TitreExo[2]{
\needspace{4\baselineskip}
 {\sffamily\large EXERCICE #1\ (\emph{#2 points})}
\vspace{5mm}
}

\newcommand\img[2]{
          \includegraphics[width=#2\paperwidth]{\imgdir#1}
}

\newcommand\imgsvg[2]{
       \begin{center}   \includegraphics[width=#2\paperwidth]{\imgsvgdir#1} \end{center}
}


\newcommand\Lien[2]{
     \href{#1}{#2 \tiny \faExternalLink}
}
\newcommand\mcLien[2]{
     \href{https~://www.maths-cours.fr/#1}{#2 \tiny \faExternalLink}
}

\newcommand{\euro}{\eurologo{}}

%================================================================================================================================
%
% Macros - Environement
%
%================================================================================================================================

\newenvironment{tex}{ %
}
{%
}

\newenvironment{indente}{ %
	\setlength\parindent{10mm}
}

{
	\setlength\parindent{0mm}
}

\newenvironment{corrige}{%
     \needspace{3\baselineskip}
     \medskip
     \textbf{\textsc{Corrigé}}
     \medskip
}
{
}

\newenvironment{extern}{%
     \begin{center}
     }
     {
     \end{center}
}

\NewEnviron{code}{%
	\par
     \boite{gray}{\texttt{%
     \BODY
     }}
     \par
}

\newenvironment{vbloc}{% boite sans cadre empeche saut de page
     \begin{minipage}[t]{\linewidth}
     }
     {
     \end{minipage}
}
\NewEnviron{h2}{%
    \needspace{3\baselineskip}
    \vspace{0.6cm}
	\noindent \MakeUppercase{\sffamily \large \BODY}
	\vspace{1mm}\textcolor{mcgris}{\hrule}\vspace{0.4cm}
	\par
}{}

\NewEnviron{h3}{%
    \needspace{3\baselineskip}
	\vspace{5mm}
	\textsc{\BODY}
	\par
}

\NewEnviron{margeneg}{ %
\begin{addmargin}[-1cm]{0cm}
\BODY
\end{addmargin}
}

\NewEnviron{html}{%
}

\begin{document}
\meta{url}{/exercices/fractions-racines-carrees-brevet-2010-141101/}
\meta{pid}{1676}
\meta{titre}{Fractions - Racines carrées (Brevet 2010)}
\meta{type}{exercices}
%
\textit{(Brevet Asie 2010)}
\par
On donne les nombres suivants :
\par
$A =\frac{3}{4}-\frac{2}{3}$÷$\frac{8}{15}$ , 
\\
$B =\frac{6\times 10^{-2} \times 5 \times 10^{2}}{1,5 \times 10^{-4}}\quad$  
\\ 
$C =\sqrt{12}-5\sqrt{3}+2\sqrt{48}$.
\par
\textit{Pour les trois questions suivantes, on écrira au moins une étape de calcul.}
\begin{enumerate}
     \item
     Calculer $A$ et donner le résultat sous la forme d'une fraction irréductible.
     \item
     Calculer $B$ et donner le résultat sous forme scientifique.
     \item
     Écrire $C$ sous la forme $a\sqrt{3}$ où $a$ est un nombre entier.
\end{enumerate}
\begin{corrige}
     \begin{enumerate}
          \item
          $A = \frac{3}{4}-\frac{2}{3}\times \frac{15}{8} = \frac{3}{4}-\frac{2\times 3\times 5}{3\times 2\times 4} $$= \frac{3}{4}-\frac{5}{4}=-\frac{2}{4} = -\frac{1}{2}$
\medskip
          \item
          $B = \frac{6 \times 5 \times 10^{-2} \times 10^{2}}{1,5 \times 10^{-4}} $$= \frac{30}{1,5}\times 10^{-2+2-\left(-4\right)} = 20\times 10^{4}$
          \par
          La forme scientifique de $B$ est :
          \par
          $B=2\times 10^{5}$
\medskip
          \item
          $\sqrt{12} = \sqrt{4\times 3}=2\sqrt{3} $  et  $ \sqrt{48} = \sqrt{16\times 3}=4\sqrt{3}$
          \par
          Par conséquent :
          \par
          $C = 2\sqrt{3}-5\sqrt{3}+2\times 4\sqrt{3}$$= 2\sqrt{3}-5\sqrt{3}+8\sqrt{3} =5\sqrt{3}$
     \end{enumerate}
\end{corrige}

\end{document}

µ
\documentclass[a4paper]{article}

%================================================================================================================================
%
% Packages
%
%================================================================================================================================

\usepackage[T1]{fontenc} 	% pour caractères accentués
\usepackage[utf8]{inputenc}  % encodage utf8
\usepackage[french]{babel}	% langue : français
\usepackage{fourier}			% caractères plus lisibles
\usepackage[dvipsnames]{xcolor} % couleurs
\usepackage{fancyhdr}		% réglage header footer
\usepackage{needspace}		% empêcher sauts de page mal placés
\usepackage{graphicx}		% pour inclure des graphiques
\usepackage{enumitem,cprotect}		% personnalise les listes d'items (nécessaire pour ol, al ...)
\usepackage{hyperref}		% Liens hypertexte
\usepackage{pstricks,pst-all,pst-node,pstricks-add,pst-math,pst-plot,pst-tree,pst-eucl} % pstricks
\usepackage[a4paper,includeheadfoot,top=2cm,left=3cm, bottom=2cm,right=3cm]{geometry} % marges etc.
\usepackage{comment}			% commentaires multilignes
\usepackage{amsmath,environ} % maths (matrices, etc.)
\usepackage{amssymb,makeidx}
\usepackage{bm}				% bold maths
\usepackage{tabularx}		% tableaux
\usepackage{colortbl}		% tableaux en couleur
\usepackage{fontawesome}		% Fontawesome
\usepackage{environ}			% environment with command
\usepackage{fp}				% calculs pour ps-tricks
\usepackage{multido}			% pour ps tricks
\usepackage[np]{numprint}	% formattage nombre
\usepackage{tikz,tkz-tab} 			% package principal TikZ
\usepackage{pgfplots}   % axes
\usepackage{mathrsfs}    % cursives
\usepackage{calc}			% calcul taille boites
\usepackage[scaled=0.875]{helvet} % font sans serif
\usepackage{svg} % svg
\usepackage{scrextend} % local margin
\usepackage{scratch} %scratch
\usepackage{multicol} % colonnes
%\usepackage{infix-RPN,pst-func} % formule en notation polanaise inversée
\usepackage{listings}

%================================================================================================================================
%
% Réglages de base
%
%================================================================================================================================

\lstset{
language=Python,   % R code
literate=
{á}{{\'a}}1
{à}{{\`a}}1
{ã}{{\~a}}1
{é}{{\'e}}1
{è}{{\`e}}1
{ê}{{\^e}}1
{í}{{\'i}}1
{ó}{{\'o}}1
{õ}{{\~o}}1
{ú}{{\'u}}1
{ü}{{\"u}}1
{ç}{{\c{c}}}1
{~}{{ }}1
}


\definecolor{codegreen}{rgb}{0,0.6,0}
\definecolor{codegray}{rgb}{0.5,0.5,0.5}
\definecolor{codepurple}{rgb}{0.58,0,0.82}
\definecolor{backcolour}{rgb}{0.95,0.95,0.92}

\lstdefinestyle{mystyle}{
    backgroundcolor=\color{backcolour},   
    commentstyle=\color{codegreen},
    keywordstyle=\color{magenta},
    numberstyle=\tiny\color{codegray},
    stringstyle=\color{codepurple},
    basicstyle=\ttfamily\footnotesize,
    breakatwhitespace=false,         
    breaklines=true,                 
    captionpos=b,                    
    keepspaces=true,                 
    numbers=left,                    
xleftmargin=2em,
framexleftmargin=2em,            
    showspaces=false,                
    showstringspaces=false,
    showtabs=false,                  
    tabsize=2,
    upquote=true
}

\lstset{style=mystyle}


\lstset{style=mystyle}
\newcommand{\imgdir}{C:/laragon/www/newmc/assets/imgsvg/}
\newcommand{\imgsvgdir}{C:/laragon/www/newmc/assets/imgsvg/}

\definecolor{mcgris}{RGB}{220, 220, 220}% ancien~; pour compatibilité
\definecolor{mcbleu}{RGB}{52, 152, 219}
\definecolor{mcvert}{RGB}{125, 194, 70}
\definecolor{mcmauve}{RGB}{154, 0, 215}
\definecolor{mcorange}{RGB}{255, 96, 0}
\definecolor{mcturquoise}{RGB}{0, 153, 153}
\definecolor{mcrouge}{RGB}{255, 0, 0}
\definecolor{mclightvert}{RGB}{205, 234, 190}

\definecolor{gris}{RGB}{220, 220, 220}
\definecolor{bleu}{RGB}{52, 152, 219}
\definecolor{vert}{RGB}{125, 194, 70}
\definecolor{mauve}{RGB}{154, 0, 215}
\definecolor{orange}{RGB}{255, 96, 0}
\definecolor{turquoise}{RGB}{0, 153, 153}
\definecolor{rouge}{RGB}{255, 0, 0}
\definecolor{lightvert}{RGB}{205, 234, 190}
\setitemize[0]{label=\color{lightvert}  $\bullet$}

\pagestyle{fancy}
\renewcommand{\headrulewidth}{0.2pt}
\fancyhead[L]{maths-cours.fr}
\fancyhead[R]{\thepage}
\renewcommand{\footrulewidth}{0.2pt}
\fancyfoot[C]{}

\newcolumntype{C}{>{\centering\arraybackslash}X}
\newcolumntype{s}{>{\hsize=.35\hsize\arraybackslash}X}

\setlength{\parindent}{0pt}		 
\setlength{\parskip}{3mm}
\setlength{\headheight}{1cm}

\def\ebook{ebook}
\def\book{book}
\def\web{web}
\def\type{web}

\newcommand{\vect}[1]{\overrightarrow{\,\mathstrut#1\,}}

\def\Oij{$\left(\text{O}~;~\vect{\imath},~\vect{\jmath}\right)$}
\def\Oijk{$\left(\text{O}~;~\vect{\imath},~\vect{\jmath},~\vect{k}\right)$}
\def\Ouv{$\left(\text{O}~;~\vect{u},~\vect{v}\right)$}

\hypersetup{breaklinks=true, colorlinks = true, linkcolor = OliveGreen, urlcolor = OliveGreen, citecolor = OliveGreen, pdfauthor={Didier BONNEL - https://www.maths-cours.fr} } % supprime les bordures autour des liens

\renewcommand{\arg}[0]{\text{arg}}

\everymath{\displaystyle}

%================================================================================================================================
%
% Macros - Commandes
%
%================================================================================================================================

\newcommand\meta[2]{    			% Utilisé pour créer le post HTML.
	\def\titre{titre}
	\def\url{url}
	\def\arg{#1}
	\ifx\titre\arg
		\newcommand\maintitle{#2}
		\fancyhead[L]{#2}
		{\Large\sffamily \MakeUppercase{#2}}
		\vspace{1mm}\textcolor{mcvert}{\hrule}
	\fi 
	\ifx\url\arg
		\fancyfoot[L]{\href{https://www.maths-cours.fr#2}{\black \footnotesize{https://www.maths-cours.fr#2}}}
	\fi 
}


\newcommand\TitreC[1]{    		% Titre centré
     \needspace{3\baselineskip}
     \begin{center}\textbf{#1}\end{center}
}

\newcommand\newpar{    		% paragraphe
     \par
}

\newcommand\nosp {    		% commande vide (pas d'espace)
}
\newcommand{\id}[1]{} %ignore

\newcommand\boite[2]{				% Boite simple sans titre
	\vspace{5mm}
	\setlength{\fboxrule}{0.2mm}
	\setlength{\fboxsep}{5mm}	
	\fcolorbox{#1}{#1!3}{\makebox[\linewidth-2\fboxrule-2\fboxsep]{
  		\begin{minipage}[t]{\linewidth-2\fboxrule-4\fboxsep}\setlength{\parskip}{3mm}
  			 #2
  		\end{minipage}
	}}
	\vspace{5mm}
}

\newcommand\CBox[4]{				% Boites
	\vspace{5mm}
	\setlength{\fboxrule}{0.2mm}
	\setlength{\fboxsep}{5mm}
	
	\fcolorbox{#1}{#1!3}{\makebox[\linewidth-2\fboxrule-2\fboxsep]{
		\begin{minipage}[t]{1cm}\setlength{\parskip}{3mm}
	  		\textcolor{#1}{\LARGE{#2}}    
 	 	\end{minipage}  
  		\begin{minipage}[t]{\linewidth-2\fboxrule-4\fboxsep}\setlength{\parskip}{3mm}
			\raisebox{1.2mm}{\normalsize\sffamily{\textcolor{#1}{#3}}}						
  			 #4
  		\end{minipage}
	}}
	\vspace{5mm}
}

\newcommand\cadre[3]{				% Boites convertible html
	\par
	\vspace{2mm}
	\setlength{\fboxrule}{0.1mm}
	\setlength{\fboxsep}{5mm}
	\fcolorbox{#1}{white}{\makebox[\linewidth-2\fboxrule-2\fboxsep]{
  		\begin{minipage}[t]{\linewidth-2\fboxrule-4\fboxsep}\setlength{\parskip}{3mm}
			\raisebox{-2.5mm}{\sffamily \small{\textcolor{#1}{\MakeUppercase{#2}}}}		
			\par		
  			 #3
 	 		\end{minipage}
	}}
		\vspace{2mm}
	\par
}

\newcommand\bloc[3]{				% Boites convertible html sans bordure
     \needspace{2\baselineskip}
     {\sffamily \small{\textcolor{#1}{\MakeUppercase{#2}}}}    
		\par		
  			 #3
		\par
}

\newcommand\CHelp[1]{
     \CBox{Plum}{\faInfoCircle}{À RETENIR}{#1}
}

\newcommand\CUp[1]{
     \CBox{NavyBlue}{\faThumbsOUp}{EN PRATIQUE}{#1}
}

\newcommand\CInfo[1]{
     \CBox{Sepia}{\faArrowCircleRight}{REMARQUE}{#1}
}

\newcommand\CRedac[1]{
     \CBox{PineGreen}{\faEdit}{BIEN R\'EDIGER}{#1}
}

\newcommand\CError[1]{
     \CBox{Red}{\faExclamationTriangle}{ATTENTION}{#1}
}

\newcommand\TitreExo[2]{
\needspace{4\baselineskip}
 {\sffamily\large EXERCICE #1\ (\emph{#2 points})}
\vspace{5mm}
}

\newcommand\img[2]{
          \includegraphics[width=#2\paperwidth]{\imgdir#1}
}

\newcommand\imgsvg[2]{
       \begin{center}   \includegraphics[width=#2\paperwidth]{\imgsvgdir#1} \end{center}
}


\newcommand\Lien[2]{
     \href{#1}{#2 \tiny \faExternalLink}
}
\newcommand\mcLien[2]{
     \href{https~://www.maths-cours.fr/#1}{#2 \tiny \faExternalLink}
}

\newcommand{\euro}{\eurologo{}}

%================================================================================================================================
%
% Macros - Environement
%
%================================================================================================================================

\newenvironment{tex}{ %
}
{%
}

\newenvironment{indente}{ %
	\setlength\parindent{10mm}
}

{
	\setlength\parindent{0mm}
}

\newenvironment{corrige}{%
     \needspace{3\baselineskip}
     \medskip
     \textbf{\textsc{Corrigé}}
     \medskip
}
{
}

\newenvironment{extern}{%
     \begin{center}
     }
     {
     \end{center}
}

\NewEnviron{code}{%
	\par
     \boite{gray}{\texttt{%
     \BODY
     }}
     \par
}

\newenvironment{vbloc}{% boite sans cadre empeche saut de page
     \begin{minipage}[t]{\linewidth}
     }
     {
     \end{minipage}
}
\NewEnviron{h2}{%
    \needspace{3\baselineskip}
    \vspace{0.6cm}
	\noindent \MakeUppercase{\sffamily \large \BODY}
	\vspace{1mm}\textcolor{mcgris}{\hrule}\vspace{0.4cm}
	\par
}{}

\NewEnviron{h3}{%
    \needspace{3\baselineskip}
	\vspace{5mm}
	\textsc{\BODY}
	\par
}

\NewEnviron{margeneg}{ %
\begin{addmargin}[-1cm]{0cm}
\BODY
\end{addmargin}
}

\NewEnviron{html}{%
}

\begin{document}
\meta{url}{/exercices/puissances-de-10-brevet-2012-141030/}
\meta{pid}{1683}
\meta{titre}{Puissances de dix (Brevet 2012)}
\meta{type}{exercices}
%
\textit{(Brevet Métropole 2012)}
\begin{enumerate}
     \item
     Quelle est l'écriture décimale du nombre $\frac{10^{5}+1}{10^{5}}$?
     \item
     Antoine utilise sa calculatrice pour calculer le nombre suivant : $\frac{10^{15} +1}{10^{15}}$.
     \par
     Le résultat affiché est $1$.
     \par
     Antoine pense que ce résultat n'est pas exact. A-t-il raison?
\end{enumerate}
\begin{corrige}
     \begin{enumerate}
          \item
          $\frac{10^{5}+1}{10^{5}}=\frac{10^{5}}{10^{5}}+\frac{1}{10^{5}}$.
          \par
          Or  $\frac{10^{5}}{10^{5}}=1$ (simplification par $10^{5}$) et $\frac{1}{10^{5}}=10^{-5}=0,00001$
          \par
          Par conséquent :  $\frac{10^{5}+1}{10^{5}}=1+0,00001=1,00001$
          \par
          (Ici une calculatrice donnerait le bon résultat.)
          \item
          De la même façon :
          \par
          $\frac{10^{15}+1}{10^{15}}=\frac{10^{15}}{10^{15}}+\frac{1}{10^{15}}=1+10^{-15}=1,000000000000001$
          \par
          Antoine a raison. La calculatrice (qui calcule avec un nombre limité de décimales) a arrondi le résultat.
     \end{enumerate}
\end{corrige}

\end{document}

µ
\documentclass[a4paper]{article}

%================================================================================================================================
%
% Packages
%
%================================================================================================================================

\usepackage[T1]{fontenc} 	% pour caractères accentués
\usepackage[utf8]{inputenc}  % encodage utf8
\usepackage[french]{babel}	% langue : français
\usepackage{fourier}			% caractères plus lisibles
\usepackage[dvipsnames]{xcolor} % couleurs
\usepackage{fancyhdr}		% réglage header footer
\usepackage{needspace}		% empêcher sauts de page mal placés
\usepackage{graphicx}		% pour inclure des graphiques
\usepackage{enumitem,cprotect}		% personnalise les listes d'items (nécessaire pour ol, al ...)
\usepackage{hyperref}		% Liens hypertexte
\usepackage{pstricks,pst-all,pst-node,pstricks-add,pst-math,pst-plot,pst-tree,pst-eucl} % pstricks
\usepackage[a4paper,includeheadfoot,top=2cm,left=3cm, bottom=2cm,right=3cm]{geometry} % marges etc.
\usepackage{comment}			% commentaires multilignes
\usepackage{amsmath,environ} % maths (matrices, etc.)
\usepackage{amssymb,makeidx}
\usepackage{bm}				% bold maths
\usepackage{tabularx}		% tableaux
\usepackage{colortbl}		% tableaux en couleur
\usepackage{fontawesome}		% Fontawesome
\usepackage{environ}			% environment with command
\usepackage{fp}				% calculs pour ps-tricks
\usepackage{multido}			% pour ps tricks
\usepackage[np]{numprint}	% formattage nombre
\usepackage{tikz,tkz-tab} 			% package principal TikZ
\usepackage{pgfplots}   % axes
\usepackage{mathrsfs}    % cursives
\usepackage{calc}			% calcul taille boites
\usepackage[scaled=0.875]{helvet} % font sans serif
\usepackage{svg} % svg
\usepackage{scrextend} % local margin
\usepackage{scratch} %scratch
\usepackage{multicol} % colonnes
%\usepackage{infix-RPN,pst-func} % formule en notation polanaise inversée
\usepackage{listings}

%================================================================================================================================
%
% Réglages de base
%
%================================================================================================================================

\lstset{
language=Python,   % R code
literate=
{á}{{\'a}}1
{à}{{\`a}}1
{ã}{{\~a}}1
{é}{{\'e}}1
{è}{{\`e}}1
{ê}{{\^e}}1
{í}{{\'i}}1
{ó}{{\'o}}1
{õ}{{\~o}}1
{ú}{{\'u}}1
{ü}{{\"u}}1
{ç}{{\c{c}}}1
{~}{{ }}1
}


\definecolor{codegreen}{rgb}{0,0.6,0}
\definecolor{codegray}{rgb}{0.5,0.5,0.5}
\definecolor{codepurple}{rgb}{0.58,0,0.82}
\definecolor{backcolour}{rgb}{0.95,0.95,0.92}

\lstdefinestyle{mystyle}{
    backgroundcolor=\color{backcolour},   
    commentstyle=\color{codegreen},
    keywordstyle=\color{magenta},
    numberstyle=\tiny\color{codegray},
    stringstyle=\color{codepurple},
    basicstyle=\ttfamily\footnotesize,
    breakatwhitespace=false,         
    breaklines=true,                 
    captionpos=b,                    
    keepspaces=true,                 
    numbers=left,                    
xleftmargin=2em,
framexleftmargin=2em,            
    showspaces=false,                
    showstringspaces=false,
    showtabs=false,                  
    tabsize=2,
    upquote=true
}

\lstset{style=mystyle}


\lstset{style=mystyle}
\newcommand{\imgdir}{C:/laragon/www/newmc/assets/imgsvg/}
\newcommand{\imgsvgdir}{C:/laragon/www/newmc/assets/imgsvg/}

\definecolor{mcgris}{RGB}{220, 220, 220}% ancien~; pour compatibilité
\definecolor{mcbleu}{RGB}{52, 152, 219}
\definecolor{mcvert}{RGB}{125, 194, 70}
\definecolor{mcmauve}{RGB}{154, 0, 215}
\definecolor{mcorange}{RGB}{255, 96, 0}
\definecolor{mcturquoise}{RGB}{0, 153, 153}
\definecolor{mcrouge}{RGB}{255, 0, 0}
\definecolor{mclightvert}{RGB}{205, 234, 190}

\definecolor{gris}{RGB}{220, 220, 220}
\definecolor{bleu}{RGB}{52, 152, 219}
\definecolor{vert}{RGB}{125, 194, 70}
\definecolor{mauve}{RGB}{154, 0, 215}
\definecolor{orange}{RGB}{255, 96, 0}
\definecolor{turquoise}{RGB}{0, 153, 153}
\definecolor{rouge}{RGB}{255, 0, 0}
\definecolor{lightvert}{RGB}{205, 234, 190}
\setitemize[0]{label=\color{lightvert}  $\bullet$}

\pagestyle{fancy}
\renewcommand{\headrulewidth}{0.2pt}
\fancyhead[L]{maths-cours.fr}
\fancyhead[R]{\thepage}
\renewcommand{\footrulewidth}{0.2pt}
\fancyfoot[C]{}

\newcolumntype{C}{>{\centering\arraybackslash}X}
\newcolumntype{s}{>{\hsize=.35\hsize\arraybackslash}X}

\setlength{\parindent}{0pt}		 
\setlength{\parskip}{3mm}
\setlength{\headheight}{1cm}

\def\ebook{ebook}
\def\book{book}
\def\web{web}
\def\type{web}

\newcommand{\vect}[1]{\overrightarrow{\,\mathstrut#1\,}}

\def\Oij{$\left(\text{O}~;~\vect{\imath},~\vect{\jmath}\right)$}
\def\Oijk{$\left(\text{O}~;~\vect{\imath},~\vect{\jmath},~\vect{k}\right)$}
\def\Ouv{$\left(\text{O}~;~\vect{u},~\vect{v}\right)$}

\hypersetup{breaklinks=true, colorlinks = true, linkcolor = OliveGreen, urlcolor = OliveGreen, citecolor = OliveGreen, pdfauthor={Didier BONNEL - https://www.maths-cours.fr} } % supprime les bordures autour des liens

\renewcommand{\arg}[0]{\text{arg}}

\everymath{\displaystyle}

%================================================================================================================================
%
% Macros - Commandes
%
%================================================================================================================================

\newcommand\meta[2]{    			% Utilisé pour créer le post HTML.
	\def\titre{titre}
	\def\url{url}
	\def\arg{#1}
	\ifx\titre\arg
		\newcommand\maintitle{#2}
		\fancyhead[L]{#2}
		{\Large\sffamily \MakeUppercase{#2}}
		\vspace{1mm}\textcolor{mcvert}{\hrule}
	\fi 
	\ifx\url\arg
		\fancyfoot[L]{\href{https://www.maths-cours.fr#2}{\black \footnotesize{https://www.maths-cours.fr#2}}}
	\fi 
}


\newcommand\TitreC[1]{    		% Titre centré
     \needspace{3\baselineskip}
     \begin{center}\textbf{#1}\end{center}
}

\newcommand\newpar{    		% paragraphe
     \par
}

\newcommand\nosp {    		% commande vide (pas d'espace)
}
\newcommand{\id}[1]{} %ignore

\newcommand\boite[2]{				% Boite simple sans titre
	\vspace{5mm}
	\setlength{\fboxrule}{0.2mm}
	\setlength{\fboxsep}{5mm}	
	\fcolorbox{#1}{#1!3}{\makebox[\linewidth-2\fboxrule-2\fboxsep]{
  		\begin{minipage}[t]{\linewidth-2\fboxrule-4\fboxsep}\setlength{\parskip}{3mm}
  			 #2
  		\end{minipage}
	}}
	\vspace{5mm}
}

\newcommand\CBox[4]{				% Boites
	\vspace{5mm}
	\setlength{\fboxrule}{0.2mm}
	\setlength{\fboxsep}{5mm}
	
	\fcolorbox{#1}{#1!3}{\makebox[\linewidth-2\fboxrule-2\fboxsep]{
		\begin{minipage}[t]{1cm}\setlength{\parskip}{3mm}
	  		\textcolor{#1}{\LARGE{#2}}    
 	 	\end{minipage}  
  		\begin{minipage}[t]{\linewidth-2\fboxrule-4\fboxsep}\setlength{\parskip}{3mm}
			\raisebox{1.2mm}{\normalsize\sffamily{\textcolor{#1}{#3}}}						
  			 #4
  		\end{minipage}
	}}
	\vspace{5mm}
}

\newcommand\cadre[3]{				% Boites convertible html
	\par
	\vspace{2mm}
	\setlength{\fboxrule}{0.1mm}
	\setlength{\fboxsep}{5mm}
	\fcolorbox{#1}{white}{\makebox[\linewidth-2\fboxrule-2\fboxsep]{
  		\begin{minipage}[t]{\linewidth-2\fboxrule-4\fboxsep}\setlength{\parskip}{3mm}
			\raisebox{-2.5mm}{\sffamily \small{\textcolor{#1}{\MakeUppercase{#2}}}}		
			\par		
  			 #3
 	 		\end{minipage}
	}}
		\vspace{2mm}
	\par
}

\newcommand\bloc[3]{				% Boites convertible html sans bordure
     \needspace{2\baselineskip}
     {\sffamily \small{\textcolor{#1}{\MakeUppercase{#2}}}}    
		\par		
  			 #3
		\par
}

\newcommand\CHelp[1]{
     \CBox{Plum}{\faInfoCircle}{À RETENIR}{#1}
}

\newcommand\CUp[1]{
     \CBox{NavyBlue}{\faThumbsOUp}{EN PRATIQUE}{#1}
}

\newcommand\CInfo[1]{
     \CBox{Sepia}{\faArrowCircleRight}{REMARQUE}{#1}
}

\newcommand\CRedac[1]{
     \CBox{PineGreen}{\faEdit}{BIEN R\'EDIGER}{#1}
}

\newcommand\CError[1]{
     \CBox{Red}{\faExclamationTriangle}{ATTENTION}{#1}
}

\newcommand\TitreExo[2]{
\needspace{4\baselineskip}
 {\sffamily\large EXERCICE #1\ (\emph{#2 points})}
\vspace{5mm}
}

\newcommand\img[2]{
          \includegraphics[width=#2\paperwidth]{\imgdir#1}
}

\newcommand\imgsvg[2]{
       \begin{center}   \includegraphics[width=#2\paperwidth]{\imgsvgdir#1} \end{center}
}


\newcommand\Lien[2]{
     \href{#1}{#2 \tiny \faExternalLink}
}
\newcommand\mcLien[2]{
     \href{https~://www.maths-cours.fr/#1}{#2 \tiny \faExternalLink}
}

\newcommand{\euro}{\eurologo{}}

%================================================================================================================================
%
% Macros - Environement
%
%================================================================================================================================

\newenvironment{tex}{ %
}
{%
}

\newenvironment{indente}{ %
	\setlength\parindent{10mm}
}

{
	\setlength\parindent{0mm}
}

\newenvironment{corrige}{%
     \needspace{3\baselineskip}
     \medskip
     \textbf{\textsc{Corrigé}}
     \medskip
}
{
}

\newenvironment{extern}{%
     \begin{center}
     }
     {
     \end{center}
}

\NewEnviron{code}{%
	\par
     \boite{gray}{\texttt{%
     \BODY
     }}
     \par
}

\newenvironment{vbloc}{% boite sans cadre empeche saut de page
     \begin{minipage}[t]{\linewidth}
     }
     {
     \end{minipage}
}
\NewEnviron{h2}{%
    \needspace{3\baselineskip}
    \vspace{0.6cm}
	\noindent \MakeUppercase{\sffamily \large \BODY}
	\vspace{1mm}\textcolor{mcgris}{\hrule}\vspace{0.4cm}
	\par
}{}

\NewEnviron{h3}{%
    \needspace{3\baselineskip}
	\vspace{5mm}
	\textsc{\BODY}
	\par
}

\NewEnviron{margeneg}{ %
\begin{addmargin}[-1cm]{0cm}
\BODY
\end{addmargin}
}

\NewEnviron{html}{%
}

\begin{document}
\meta{url}{/exercices/simplifications-brevet-2001-141030/}
\meta{pid}{1686}
\meta{titre}{Simplifications (Brevet 2001)}
\meta{type}{exercices}
%
\textit{(Brevet Paris 2001 - À faire sans calculatrice)}
\par
Soit~:
\par
 $A = \frac{2}{3}-\frac{7}{3}\times \frac{5}{14}$
\par
 $B = \frac{5\times 10^{2000}}{20\times 10^{2001}}$
\par
 $C = \frac{5,1 \times 10^{2}-270 \times 10^{-1}}{4,83 \times 10^{2}}$. 
\medskip
\begin{enumerate}
     \item
     Calculer $A$ et mettre le résultat sous la forme d'une fraction irréductible.
     \item
     Calculer $B$ et donner l'écriture scientifique du résultat.
     \item
     Démontrer que $C$ est un nombre entier.
\end{enumerate}
\begin{corrige}
     \begin{enumerate}
          \item
          On commence par effectuer le produit (qui est prioritaire) en simplifiant par $7$ :
          \par
          $A = \frac{2}{3}-\frac{7}{3}\times \frac{5}{14} $\nosp$=\frac{2}{3}-\frac{7\times 5}{3\times 14} $\nosp$=\frac{2}{3}-\frac{7\times 5}{3\times 2\times 7} $\nosp$=\frac{2}{3}-\frac{5}{6}$
          \par
          Puis on réduit au même dénominateur :
          \par
          $A = \frac{2}{3}-\frac{5}{6}=\frac{4}{6}-\frac{5}{6}=-\frac{1}{6}$
          \medskip
          \item
          $B = \frac{5\times 10^{2000}}{20\times 10^{2001}} = \frac{5}{20} \times \frac{10^{2000}}{10^{2001}}$$ = \frac{5}{4\times 5}\times 10^{2000-2001}=\frac{1}{4}\times 10^{-1}$
          \par
          Or :
          \par
          $\frac{1}{4}=0,25=2,5\times 10^{-1}$
          \par
          Donc la forme scientifique de $B$ est :
          \par
          $B=\frac{1}{4}\times 10^{-1} $\nosp$=2,5\times 10^{-1}\times 10^{-1} $\nosp$=2,5\times 10^{-2}$
          \medskip
          \item
          $C = \frac{5,1 \times 10^{2}-270 \times 10^{-1}}{4,83 \times 10^{2}}$
          \par
          Calculons chaque produit :
          \par
          $5,1 \times 10^{2}=510$
          \par
          $270 \times 10^{-1}=27$
          \par
          $4,83 \times 10^{2}=483$
          \medskip
          Par conséquent :
          \par
          $C = \frac{510-27}{483}=\frac{483}{483}=1$
     \end{enumerate}
\end{corrige}

\end{document}


µ
\documentclass[a4paper]{article}

%================================================================================================================================
%
% Packages
%
%================================================================================================================================

\usepackage[T1]{fontenc} 	% pour caractères accentués
\usepackage[utf8]{inputenc}  % encodage utf8
\usepackage[french]{babel}	% langue : français
\usepackage{fourier}			% caractères plus lisibles
\usepackage[dvipsnames]{xcolor} % couleurs
\usepackage{fancyhdr}		% réglage header footer
\usepackage{needspace}		% empêcher sauts de page mal placés
\usepackage{graphicx}		% pour inclure des graphiques
\usepackage{enumitem,cprotect}		% personnalise les listes d'items (nécessaire pour ol, al ...)
\usepackage{hyperref}		% Liens hypertexte
\usepackage{pstricks,pst-all,pst-node,pstricks-add,pst-math,pst-plot,pst-tree,pst-eucl} % pstricks
\usepackage[a4paper,includeheadfoot,top=2cm,left=3cm, bottom=2cm,right=3cm]{geometry} % marges etc.
\usepackage{comment}			% commentaires multilignes
\usepackage{amsmath,environ} % maths (matrices, etc.)
\usepackage{amssymb,makeidx}
\usepackage{bm}				% bold maths
\usepackage{tabularx}		% tableaux
\usepackage{colortbl}		% tableaux en couleur
\usepackage{fontawesome}		% Fontawesome
\usepackage{environ}			% environment with command
\usepackage{fp}				% calculs pour ps-tricks
\usepackage{multido}			% pour ps tricks
\usepackage[np]{numprint}	% formattage nombre
\usepackage{tikz,tkz-tab} 			% package principal TikZ
\usepackage{pgfplots}   % axes
\usepackage{mathrsfs}    % cursives
\usepackage{calc}			% calcul taille boites
\usepackage[scaled=0.875]{helvet} % font sans serif
\usepackage{svg} % svg
\usepackage{scrextend} % local margin
\usepackage{scratch} %scratch
\usepackage{multicol} % colonnes
%\usepackage{infix-RPN,pst-func} % formule en notation polanaise inversée
\usepackage{listings}

%================================================================================================================================
%
% Réglages de base
%
%================================================================================================================================

\lstset{
language=Python,   % R code
literate=
{á}{{\'a}}1
{à}{{\`a}}1
{ã}{{\~a}}1
{é}{{\'e}}1
{è}{{\`e}}1
{ê}{{\^e}}1
{í}{{\'i}}1
{ó}{{\'o}}1
{õ}{{\~o}}1
{ú}{{\'u}}1
{ü}{{\"u}}1
{ç}{{\c{c}}}1
{~}{{ }}1
}


\definecolor{codegreen}{rgb}{0,0.6,0}
\definecolor{codegray}{rgb}{0.5,0.5,0.5}
\definecolor{codepurple}{rgb}{0.58,0,0.82}
\definecolor{backcolour}{rgb}{0.95,0.95,0.92}

\lstdefinestyle{mystyle}{
    backgroundcolor=\color{backcolour},   
    commentstyle=\color{codegreen},
    keywordstyle=\color{magenta},
    numberstyle=\tiny\color{codegray},
    stringstyle=\color{codepurple},
    basicstyle=\ttfamily\footnotesize,
    breakatwhitespace=false,         
    breaklines=true,                 
    captionpos=b,                    
    keepspaces=true,                 
    numbers=left,                    
xleftmargin=2em,
framexleftmargin=2em,            
    showspaces=false,                
    showstringspaces=false,
    showtabs=false,                  
    tabsize=2,
    upquote=true
}

\lstset{style=mystyle}


\lstset{style=mystyle}
\newcommand{\imgdir}{C:/laragon/www/newmc/assets/imgsvg/}
\newcommand{\imgsvgdir}{C:/laragon/www/newmc/assets/imgsvg/}

\definecolor{mcgris}{RGB}{220, 220, 220}% ancien~; pour compatibilité
\definecolor{mcbleu}{RGB}{52, 152, 219}
\definecolor{mcvert}{RGB}{125, 194, 70}
\definecolor{mcmauve}{RGB}{154, 0, 215}
\definecolor{mcorange}{RGB}{255, 96, 0}
\definecolor{mcturquoise}{RGB}{0, 153, 153}
\definecolor{mcrouge}{RGB}{255, 0, 0}
\definecolor{mclightvert}{RGB}{205, 234, 190}

\definecolor{gris}{RGB}{220, 220, 220}
\definecolor{bleu}{RGB}{52, 152, 219}
\definecolor{vert}{RGB}{125, 194, 70}
\definecolor{mauve}{RGB}{154, 0, 215}
\definecolor{orange}{RGB}{255, 96, 0}
\definecolor{turquoise}{RGB}{0, 153, 153}
\definecolor{rouge}{RGB}{255, 0, 0}
\definecolor{lightvert}{RGB}{205, 234, 190}
\setitemize[0]{label=\color{lightvert}  $\bullet$}

\pagestyle{fancy}
\renewcommand{\headrulewidth}{0.2pt}
\fancyhead[L]{maths-cours.fr}
\fancyhead[R]{\thepage}
\renewcommand{\footrulewidth}{0.2pt}
\fancyfoot[C]{}

\newcolumntype{C}{>{\centering\arraybackslash}X}
\newcolumntype{s}{>{\hsize=.35\hsize\arraybackslash}X}

\setlength{\parindent}{0pt}		 
\setlength{\parskip}{3mm}
\setlength{\headheight}{1cm}

\def\ebook{ebook}
\def\book{book}
\def\web{web}
\def\type{web}

\newcommand{\vect}[1]{\overrightarrow{\,\mathstrut#1\,}}

\def\Oij{$\left(\text{O}~;~\vect{\imath},~\vect{\jmath}\right)$}
\def\Oijk{$\left(\text{O}~;~\vect{\imath},~\vect{\jmath},~\vect{k}\right)$}
\def\Ouv{$\left(\text{O}~;~\vect{u},~\vect{v}\right)$}

\hypersetup{breaklinks=true, colorlinks = true, linkcolor = OliveGreen, urlcolor = OliveGreen, citecolor = OliveGreen, pdfauthor={Didier BONNEL - https://www.maths-cours.fr} } % supprime les bordures autour des liens

\renewcommand{\arg}[0]{\text{arg}}

\everymath{\displaystyle}

%================================================================================================================================
%
% Macros - Commandes
%
%================================================================================================================================

\newcommand\meta[2]{    			% Utilisé pour créer le post HTML.
	\def\titre{titre}
	\def\url{url}
	\def\arg{#1}
	\ifx\titre\arg
		\newcommand\maintitle{#2}
		\fancyhead[L]{#2}
		{\Large\sffamily \MakeUppercase{#2}}
		\vspace{1mm}\textcolor{mcvert}{\hrule}
	\fi 
	\ifx\url\arg
		\fancyfoot[L]{\href{https://www.maths-cours.fr#2}{\black \footnotesize{https://www.maths-cours.fr#2}}}
	\fi 
}


\newcommand\TitreC[1]{    		% Titre centré
     \needspace{3\baselineskip}
     \begin{center}\textbf{#1}\end{center}
}

\newcommand\newpar{    		% paragraphe
     \par
}

\newcommand\nosp {    		% commande vide (pas d'espace)
}
\newcommand{\id}[1]{} %ignore

\newcommand\boite[2]{				% Boite simple sans titre
	\vspace{5mm}
	\setlength{\fboxrule}{0.2mm}
	\setlength{\fboxsep}{5mm}	
	\fcolorbox{#1}{#1!3}{\makebox[\linewidth-2\fboxrule-2\fboxsep]{
  		\begin{minipage}[t]{\linewidth-2\fboxrule-4\fboxsep}\setlength{\parskip}{3mm}
  			 #2
  		\end{minipage}
	}}
	\vspace{5mm}
}

\newcommand\CBox[4]{				% Boites
	\vspace{5mm}
	\setlength{\fboxrule}{0.2mm}
	\setlength{\fboxsep}{5mm}
	
	\fcolorbox{#1}{#1!3}{\makebox[\linewidth-2\fboxrule-2\fboxsep]{
		\begin{minipage}[t]{1cm}\setlength{\parskip}{3mm}
	  		\textcolor{#1}{\LARGE{#2}}    
 	 	\end{minipage}  
  		\begin{minipage}[t]{\linewidth-2\fboxrule-4\fboxsep}\setlength{\parskip}{3mm}
			\raisebox{1.2mm}{\normalsize\sffamily{\textcolor{#1}{#3}}}						
  			 #4
  		\end{minipage}
	}}
	\vspace{5mm}
}

\newcommand\cadre[3]{				% Boites convertible html
	\par
	\vspace{2mm}
	\setlength{\fboxrule}{0.1mm}
	\setlength{\fboxsep}{5mm}
	\fcolorbox{#1}{white}{\makebox[\linewidth-2\fboxrule-2\fboxsep]{
  		\begin{minipage}[t]{\linewidth-2\fboxrule-4\fboxsep}\setlength{\parskip}{3mm}
			\raisebox{-2.5mm}{\sffamily \small{\textcolor{#1}{\MakeUppercase{#2}}}}		
			\par		
  			 #3
 	 		\end{minipage}
	}}
		\vspace{2mm}
	\par
}

\newcommand\bloc[3]{				% Boites convertible html sans bordure
     \needspace{2\baselineskip}
     {\sffamily \small{\textcolor{#1}{\MakeUppercase{#2}}}}    
		\par		
  			 #3
		\par
}

\newcommand\CHelp[1]{
     \CBox{Plum}{\faInfoCircle}{À RETENIR}{#1}
}

\newcommand\CUp[1]{
     \CBox{NavyBlue}{\faThumbsOUp}{EN PRATIQUE}{#1}
}

\newcommand\CInfo[1]{
     \CBox{Sepia}{\faArrowCircleRight}{REMARQUE}{#1}
}

\newcommand\CRedac[1]{
     \CBox{PineGreen}{\faEdit}{BIEN R\'EDIGER}{#1}
}

\newcommand\CError[1]{
     \CBox{Red}{\faExclamationTriangle}{ATTENTION}{#1}
}

\newcommand\TitreExo[2]{
\needspace{4\baselineskip}
 {\sffamily\large EXERCICE #1\ (\emph{#2 points})}
\vspace{5mm}
}

\newcommand\img[2]{
          \includegraphics[width=#2\paperwidth]{\imgdir#1}
}

\newcommand\imgsvg[2]{
       \begin{center}   \includegraphics[width=#2\paperwidth]{\imgsvgdir#1} \end{center}
}


\newcommand\Lien[2]{
     \href{#1}{#2 \tiny \faExternalLink}
}
\newcommand\mcLien[2]{
     \href{https~://www.maths-cours.fr/#1}{#2 \tiny \faExternalLink}
}

\newcommand{\euro}{\eurologo{}}

%================================================================================================================================
%
% Macros - Environement
%
%================================================================================================================================

\newenvironment{tex}{ %
}
{%
}

\newenvironment{indente}{ %
	\setlength\parindent{10mm}
}

{
	\setlength\parindent{0mm}
}

\newenvironment{corrige}{%
     \needspace{3\baselineskip}
     \medskip
     \textbf{\textsc{Corrigé}}
     \medskip
}
{
}

\newenvironment{extern}{%
     \begin{center}
     }
     {
     \end{center}
}

\NewEnviron{code}{%
	\par
     \boite{gray}{\texttt{%
     \BODY
     }}
     \par
}

\newenvironment{vbloc}{% boite sans cadre empeche saut de page
     \begin{minipage}[t]{\linewidth}
     }
     {
     \end{minipage}
}
\NewEnviron{h2}{%
    \needspace{3\baselineskip}
    \vspace{0.6cm}
	\noindent \MakeUppercase{\sffamily \large \BODY}
	\vspace{1mm}\textcolor{mcgris}{\hrule}\vspace{0.4cm}
	\par
}{}

\NewEnviron{h3}{%
    \needspace{3\baselineskip}
	\vspace{5mm}
	\textsc{\BODY}
	\par
}

\NewEnviron{margeneg}{ %
\begin{addmargin}[-1cm]{0cm}
\BODY
\end{addmargin}
}

\NewEnviron{html}{%
}

\begin{document}
\meta{url}{/exercices/developper-reduire-141028/}
\meta{pid}{1690}
\meta{titre}{Développer avec les identités remarquables}
\meta{type}{exercices}
%
\documentclass[a4paper]{article}

%================================================================================================================================
%
% Packages
%
%================================================================================================================================

\usepackage[T1]{fontenc} 	% pour caractères accentués
\usepackage[utf8]{inputenc}  % encodage utf8
\usepackage[french]{babel}	% langue : français
\usepackage{fourier}			% caractères plus lisibles
\usepackage[dvipsnames]{xcolor} % couleurs
\usepackage{fancyhdr}		% réglage header footer
\usepackage{needspace}		% empêcher sauts de page mal placés
\usepackage{graphicx}		% pour inclure des graphiques
\usepackage{enumitem,cprotect}		% personnalise les listes d'items (nécessaire pour ol, al ...)
\usepackage{hyperref}		% Liens hypertexte
\usepackage{pstricks,pst-all,pst-node,pstricks-add,pst-math,pst-plot,pst-tree,pst-eucl} % pstricks
\usepackage[a4paper,includeheadfoot,top=2cm,left=3cm, bottom=2cm,right=3cm]{geometry} % marges etc.
\usepackage{comment}			% commentaires multilignes
\usepackage{amsmath,environ} % maths (matrices, etc.)
\usepackage{amssymb,makeidx}
\usepackage{bm}				% bold maths
\usepackage{tabularx}		% tableaux
\usepackage{colortbl}		% tableaux en couleur
\usepackage{fontawesome}		% Fontawesome
\usepackage{environ}			% environment with command
\usepackage{fp}				% calculs pour ps-tricks
\usepackage{multido}			% pour ps tricks
\usepackage[np]{numprint}	% formattage nombre
\usepackage{tikz,tkz-tab} 			% package principal TikZ
\usepackage{pgfplots}   % axes
\usepackage{mathrsfs}    % cursives
\usepackage{calc}			% calcul taille boites
\usepackage[scaled=0.875]{helvet} % font sans serif
\usepackage{svg} % svg
\usepackage{scrextend} % local margin
\usepackage{scratch} %scratch
\usepackage{multicol} % colonnes
%\usepackage{infix-RPN,pst-func} % formule en notation polanaise inversée
\usepackage{listings}

%================================================================================================================================
%
% Réglages de base
%
%================================================================================================================================

\lstset{
language=Python,   % R code
literate=
{á}{{\'a}}1
{à}{{\`a}}1
{ã}{{\~a}}1
{é}{{\'e}}1
{è}{{\`e}}1
{ê}{{\^e}}1
{í}{{\'i}}1
{ó}{{\'o}}1
{õ}{{\~o}}1
{ú}{{\'u}}1
{ü}{{\"u}}1
{ç}{{\c{c}}}1
{~}{{ }}1
}


\definecolor{codegreen}{rgb}{0,0.6,0}
\definecolor{codegray}{rgb}{0.5,0.5,0.5}
\definecolor{codepurple}{rgb}{0.58,0,0.82}
\definecolor{backcolour}{rgb}{0.95,0.95,0.92}

\lstdefinestyle{mystyle}{
    backgroundcolor=\color{backcolour},   
    commentstyle=\color{codegreen},
    keywordstyle=\color{magenta},
    numberstyle=\tiny\color{codegray},
    stringstyle=\color{codepurple},
    basicstyle=\ttfamily\footnotesize,
    breakatwhitespace=false,         
    breaklines=true,                 
    captionpos=b,                    
    keepspaces=true,                 
    numbers=left,                    
xleftmargin=2em,
framexleftmargin=2em,            
    showspaces=false,                
    showstringspaces=false,
    showtabs=false,                  
    tabsize=2,
    upquote=true
}

\lstset{style=mystyle}


\lstset{style=mystyle}
\newcommand{\imgdir}{C:/laragon/www/newmc/assets/imgsvg/}
\newcommand{\imgsvgdir}{C:/laragon/www/newmc/assets/imgsvg/}

\definecolor{mcgris}{RGB}{220, 220, 220}% ancien~; pour compatibilité
\definecolor{mcbleu}{RGB}{52, 152, 219}
\definecolor{mcvert}{RGB}{125, 194, 70}
\definecolor{mcmauve}{RGB}{154, 0, 215}
\definecolor{mcorange}{RGB}{255, 96, 0}
\definecolor{mcturquoise}{RGB}{0, 153, 153}
\definecolor{mcrouge}{RGB}{255, 0, 0}
\definecolor{mclightvert}{RGB}{205, 234, 190}

\definecolor{gris}{RGB}{220, 220, 220}
\definecolor{bleu}{RGB}{52, 152, 219}
\definecolor{vert}{RGB}{125, 194, 70}
\definecolor{mauve}{RGB}{154, 0, 215}
\definecolor{orange}{RGB}{255, 96, 0}
\definecolor{turquoise}{RGB}{0, 153, 153}
\definecolor{rouge}{RGB}{255, 0, 0}
\definecolor{lightvert}{RGB}{205, 234, 190}
\setitemize[0]{label=\color{lightvert}  $\bullet$}

\pagestyle{fancy}
\renewcommand{\headrulewidth}{0.2pt}
\fancyhead[L]{maths-cours.fr}
\fancyhead[R]{\thepage}
\renewcommand{\footrulewidth}{0.2pt}
\fancyfoot[C]{}

\newcolumntype{C}{>{\centering\arraybackslash}X}
\newcolumntype{s}{>{\hsize=.35\hsize\arraybackslash}X}

\setlength{\parindent}{0pt}		 
\setlength{\parskip}{3mm}
\setlength{\headheight}{1cm}

\def\ebook{ebook}
\def\book{book}
\def\web{web}
\def\type{web}

\newcommand{\vect}[1]{\overrightarrow{\,\mathstrut#1\,}}

\def\Oij{$\left(\text{O}~;~\vect{\imath},~\vect{\jmath}\right)$}
\def\Oijk{$\left(\text{O}~;~\vect{\imath},~\vect{\jmath},~\vect{k}\right)$}
\def\Ouv{$\left(\text{O}~;~\vect{u},~\vect{v}\right)$}

\hypersetup{breaklinks=true, colorlinks = true, linkcolor = OliveGreen, urlcolor = OliveGreen, citecolor = OliveGreen, pdfauthor={Didier BONNEL - https://www.maths-cours.fr} } % supprime les bordures autour des liens

\renewcommand{\arg}[0]{\text{arg}}

\everymath{\displaystyle}

%================================================================================================================================
%
% Macros - Commandes
%
%================================================================================================================================

\newcommand\meta[2]{    			% Utilisé pour créer le post HTML.
	\def\titre{titre}
	\def\url{url}
	\def\arg{#1}
	\ifx\titre\arg
		\newcommand\maintitle{#2}
		\fancyhead[L]{#2}
		{\Large\sffamily \MakeUppercase{#2}}
		\vspace{1mm}\textcolor{mcvert}{\hrule}
	\fi 
	\ifx\url\arg
		\fancyfoot[L]{\href{https://www.maths-cours.fr#2}{\black \footnotesize{https://www.maths-cours.fr#2}}}
	\fi 
}


\newcommand\TitreC[1]{    		% Titre centré
     \needspace{3\baselineskip}
     \begin{center}\textbf{#1}\end{center}
}

\newcommand\newpar{    		% paragraphe
     \par
}

\newcommand\nosp {    		% commande vide (pas d'espace)
}
\newcommand{\id}[1]{} %ignore

\newcommand\boite[2]{				% Boite simple sans titre
	\vspace{5mm}
	\setlength{\fboxrule}{0.2mm}
	\setlength{\fboxsep}{5mm}	
	\fcolorbox{#1}{#1!3}{\makebox[\linewidth-2\fboxrule-2\fboxsep]{
  		\begin{minipage}[t]{\linewidth-2\fboxrule-4\fboxsep}\setlength{\parskip}{3mm}
  			 #2
  		\end{minipage}
	}}
	\vspace{5mm}
}

\newcommand\CBox[4]{				% Boites
	\vspace{5mm}
	\setlength{\fboxrule}{0.2mm}
	\setlength{\fboxsep}{5mm}
	
	\fcolorbox{#1}{#1!3}{\makebox[\linewidth-2\fboxrule-2\fboxsep]{
		\begin{minipage}[t]{1cm}\setlength{\parskip}{3mm}
	  		\textcolor{#1}{\LARGE{#2}}    
 	 	\end{minipage}  
  		\begin{minipage}[t]{\linewidth-2\fboxrule-4\fboxsep}\setlength{\parskip}{3mm}
			\raisebox{1.2mm}{\normalsize\sffamily{\textcolor{#1}{#3}}}						
  			 #4
  		\end{minipage}
	}}
	\vspace{5mm}
}

\newcommand\cadre[3]{				% Boites convertible html
	\par
	\vspace{2mm}
	\setlength{\fboxrule}{0.1mm}
	\setlength{\fboxsep}{5mm}
	\fcolorbox{#1}{white}{\makebox[\linewidth-2\fboxrule-2\fboxsep]{
  		\begin{minipage}[t]{\linewidth-2\fboxrule-4\fboxsep}\setlength{\parskip}{3mm}
			\raisebox{-2.5mm}{\sffamily \small{\textcolor{#1}{\MakeUppercase{#2}}}}		
			\par		
  			 #3
 	 		\end{minipage}
	}}
		\vspace{2mm}
	\par
}

\newcommand\bloc[3]{				% Boites convertible html sans bordure
     \needspace{2\baselineskip}
     {\sffamily \small{\textcolor{#1}{\MakeUppercase{#2}}}}    
		\par		
  			 #3
		\par
}

\newcommand\CHelp[1]{
     \CBox{Plum}{\faInfoCircle}{À RETENIR}{#1}
}

\newcommand\CUp[1]{
     \CBox{NavyBlue}{\faThumbsOUp}{EN PRATIQUE}{#1}
}

\newcommand\CInfo[1]{
     \CBox{Sepia}{\faArrowCircleRight}{REMARQUE}{#1}
}

\newcommand\CRedac[1]{
     \CBox{PineGreen}{\faEdit}{BIEN R\'EDIGER}{#1}
}

\newcommand\CError[1]{
     \CBox{Red}{\faExclamationTriangle}{ATTENTION}{#1}
}

\newcommand\TitreExo[2]{
\needspace{4\baselineskip}
 {\sffamily\large EXERCICE #1\ (\emph{#2 points})}
\vspace{5mm}
}

\newcommand\img[2]{
          \includegraphics[width=#2\paperwidth]{\imgdir#1}
}

\newcommand\imgsvg[2]{
       \begin{center}   \includegraphics[width=#2\paperwidth]{\imgsvgdir#1} \end{center}
}


\newcommand\Lien[2]{
     \href{#1}{#2 \tiny \faExternalLink}
}
\newcommand\mcLien[2]{
     \href{https~://www.maths-cours.fr/#1}{#2 \tiny \faExternalLink}
}

\newcommand{\euro}{\eurologo{}}

%================================================================================================================================
%
% Macros - Environement
%
%================================================================================================================================

\newenvironment{tex}{ %
}
{%
}

\newenvironment{indente}{ %
	\setlength\parindent{10mm}
}

{
	\setlength\parindent{0mm}
}

\newenvironment{corrige}{%
     \needspace{3\baselineskip}
     \medskip
     \textbf{\textsc{Corrigé}}
     \medskip
}
{
}

\newenvironment{extern}{%
     \begin{center}
     }
     {
     \end{center}
}

\NewEnviron{code}{%
	\par
     \boite{gray}{\texttt{%
     \BODY
     }}
     \par
}

\newenvironment{vbloc}{% boite sans cadre empeche saut de page
     \begin{minipage}[t]{\linewidth}
     }
     {
     \end{minipage}
}
\NewEnviron{h2}{%
    \needspace{3\baselineskip}
    \vspace{0.6cm}
	\noindent \MakeUppercase{\sffamily \large \BODY}
	\vspace{1mm}\textcolor{mcgris}{\hrule}\vspace{0.4cm}
	\par
}{}

\NewEnviron{h3}{%
    \needspace{3\baselineskip}
	\vspace{5mm}
	\textsc{\BODY}
	\par
}

\NewEnviron{margeneg}{ %
\begin{addmargin}[-1cm]{0cm}
\BODY
\end{addmargin}
}

\NewEnviron{html}{%
}

\begin{document}
Développer et réduire les expressions suivantes :
\begin{enumerate}
     \item
     $A=\left(x+2\right)^{2}$
     \item
     $B=\left(5+x\right)\left(5-x\right)$
     \item
     $C=\left(2x-3\right)^{2}$
     \item
     $D=\left(x+2y\right)\left(x-2y\right)$
\end{enumerate}
\begin{corrige}
     \begin{enumerate}
          \item
          On utilise l'identité remarquable $\left(a+b\right)^{2}=a^{2}+2ab+b^{2}$ avec $a=x$ et $b=2$
          \par
          $A=\left(x+2\right)^{2}=x^{2}+2\times x\times 2+2^{2}=x^{2}+4x+4$
          \item
          On utilise l'identité remarquable $\left(a+b\right)\left(a-b\right)=a^{2}-b^{2}$ avec $a=5$ et $b=x$
          \par
          $B=\left(5+x\right)\left(5-x\right)=5^{2}-x^{2}=25-x^{2}$
          \item
          On utilise l'identité remarquable $\left(a-b\right)^{2}=a^{2}-2ab+b^{2}$ avec $a=2x$ et $b=3$
          \par
          C=$\left(2x-3\right)^{2}=\left(2x\right)^{2}-2\times 2x\times 3+3^{2}=4x^{2}-12x+9$
          \item
          On utilise l'identité remarquable $\left(a+b\right)\left(a-b\right)=a^{2}-b^{2}$ avec $a=x$ et $b=2y$
          \par
          $D=\left(x+2y\right)\left(x-2y\right)=x^{2}-\left(2y\right)^{2}=x^{2}-4y^{2}$
     \end{enumerate}
\end{corrige}

\end{document}
\end{document}


µ
\documentclass[a4paper]{article}

%================================================================================================================================
%
% Packages
%
%================================================================================================================================

\usepackage[T1]{fontenc} 	% pour caractères accentués
\usepackage[utf8]{inputenc}  % encodage utf8
\usepackage[french]{babel}	% langue : français
\usepackage{fourier}			% caractères plus lisibles
\usepackage[dvipsnames]{xcolor} % couleurs
\usepackage{fancyhdr}		% réglage header footer
\usepackage{needspace}		% empêcher sauts de page mal placés
\usepackage{graphicx}		% pour inclure des graphiques
\usepackage{enumitem,cprotect}		% personnalise les listes d'items (nécessaire pour ol, al ...)
\usepackage{hyperref}		% Liens hypertexte
\usepackage{pstricks,pst-all,pst-node,pstricks-add,pst-math,pst-plot,pst-tree,pst-eucl} % pstricks
\usepackage[a4paper,includeheadfoot,top=2cm,left=3cm, bottom=2cm,right=3cm]{geometry} % marges etc.
\usepackage{comment}			% commentaires multilignes
\usepackage{amsmath,environ} % maths (matrices, etc.)
\usepackage{amssymb,makeidx}
\usepackage{bm}				% bold maths
\usepackage{tabularx}		% tableaux
\usepackage{colortbl}		% tableaux en couleur
\usepackage{fontawesome}		% Fontawesome
\usepackage{environ}			% environment with command
\usepackage{fp}				% calculs pour ps-tricks
\usepackage{multido}			% pour ps tricks
\usepackage[np]{numprint}	% formattage nombre
\usepackage{tikz,tkz-tab} 			% package principal TikZ
\usepackage{pgfplots}   % axes
\usepackage{mathrsfs}    % cursives
\usepackage{calc}			% calcul taille boites
\usepackage[scaled=0.875]{helvet} % font sans serif
\usepackage{svg} % svg
\usepackage{scrextend} % local margin
\usepackage{scratch} %scratch
\usepackage{multicol} % colonnes
%\usepackage{infix-RPN,pst-func} % formule en notation polanaise inversée
\usepackage{listings}

%================================================================================================================================
%
% Réglages de base
%
%================================================================================================================================

\lstset{
language=Python,   % R code
literate=
{á}{{\'a}}1
{à}{{\`a}}1
{ã}{{\~a}}1
{é}{{\'e}}1
{è}{{\`e}}1
{ê}{{\^e}}1
{í}{{\'i}}1
{ó}{{\'o}}1
{õ}{{\~o}}1
{ú}{{\'u}}1
{ü}{{\"u}}1
{ç}{{\c{c}}}1
{~}{{ }}1
}


\definecolor{codegreen}{rgb}{0,0.6,0}
\definecolor{codegray}{rgb}{0.5,0.5,0.5}
\definecolor{codepurple}{rgb}{0.58,0,0.82}
\definecolor{backcolour}{rgb}{0.95,0.95,0.92}

\lstdefinestyle{mystyle}{
    backgroundcolor=\color{backcolour},   
    commentstyle=\color{codegreen},
    keywordstyle=\color{magenta},
    numberstyle=\tiny\color{codegray},
    stringstyle=\color{codepurple},
    basicstyle=\ttfamily\footnotesize,
    breakatwhitespace=false,         
    breaklines=true,                 
    captionpos=b,                    
    keepspaces=true,                 
    numbers=left,                    
xleftmargin=2em,
framexleftmargin=2em,            
    showspaces=false,                
    showstringspaces=false,
    showtabs=false,                  
    tabsize=2,
    upquote=true
}

\lstset{style=mystyle}


\lstset{style=mystyle}
\newcommand{\imgdir}{C:/laragon/www/newmc/assets/imgsvg/}
\newcommand{\imgsvgdir}{C:/laragon/www/newmc/assets/imgsvg/}

\definecolor{mcgris}{RGB}{220, 220, 220}% ancien~; pour compatibilité
\definecolor{mcbleu}{RGB}{52, 152, 219}
\definecolor{mcvert}{RGB}{125, 194, 70}
\definecolor{mcmauve}{RGB}{154, 0, 215}
\definecolor{mcorange}{RGB}{255, 96, 0}
\definecolor{mcturquoise}{RGB}{0, 153, 153}
\definecolor{mcrouge}{RGB}{255, 0, 0}
\definecolor{mclightvert}{RGB}{205, 234, 190}

\definecolor{gris}{RGB}{220, 220, 220}
\definecolor{bleu}{RGB}{52, 152, 219}
\definecolor{vert}{RGB}{125, 194, 70}
\definecolor{mauve}{RGB}{154, 0, 215}
\definecolor{orange}{RGB}{255, 96, 0}
\definecolor{turquoise}{RGB}{0, 153, 153}
\definecolor{rouge}{RGB}{255, 0, 0}
\definecolor{lightvert}{RGB}{205, 234, 190}
\setitemize[0]{label=\color{lightvert}  $\bullet$}

\pagestyle{fancy}
\renewcommand{\headrulewidth}{0.2pt}
\fancyhead[L]{maths-cours.fr}
\fancyhead[R]{\thepage}
\renewcommand{\footrulewidth}{0.2pt}
\fancyfoot[C]{}

\newcolumntype{C}{>{\centering\arraybackslash}X}
\newcolumntype{s}{>{\hsize=.35\hsize\arraybackslash}X}

\setlength{\parindent}{0pt}		 
\setlength{\parskip}{3mm}
\setlength{\headheight}{1cm}

\def\ebook{ebook}
\def\book{book}
\def\web{web}
\def\type{web}

\newcommand{\vect}[1]{\overrightarrow{\,\mathstrut#1\,}}

\def\Oij{$\left(\text{O}~;~\vect{\imath},~\vect{\jmath}\right)$}
\def\Oijk{$\left(\text{O}~;~\vect{\imath},~\vect{\jmath},~\vect{k}\right)$}
\def\Ouv{$\left(\text{O}~;~\vect{u},~\vect{v}\right)$}

\hypersetup{breaklinks=true, colorlinks = true, linkcolor = OliveGreen, urlcolor = OliveGreen, citecolor = OliveGreen, pdfauthor={Didier BONNEL - https://www.maths-cours.fr} } % supprime les bordures autour des liens

\renewcommand{\arg}[0]{\text{arg}}

\everymath{\displaystyle}

%================================================================================================================================
%
% Macros - Commandes
%
%================================================================================================================================

\newcommand\meta[2]{    			% Utilisé pour créer le post HTML.
	\def\titre{titre}
	\def\url{url}
	\def\arg{#1}
	\ifx\titre\arg
		\newcommand\maintitle{#2}
		\fancyhead[L]{#2}
		{\Large\sffamily \MakeUppercase{#2}}
		\vspace{1mm}\textcolor{mcvert}{\hrule}
	\fi 
	\ifx\url\arg
		\fancyfoot[L]{\href{https://www.maths-cours.fr#2}{\black \footnotesize{https://www.maths-cours.fr#2}}}
	\fi 
}


\newcommand\TitreC[1]{    		% Titre centré
     \needspace{3\baselineskip}
     \begin{center}\textbf{#1}\end{center}
}

\newcommand\newpar{    		% paragraphe
     \par
}

\newcommand\nosp {    		% commande vide (pas d'espace)
}
\newcommand{\id}[1]{} %ignore

\newcommand\boite[2]{				% Boite simple sans titre
	\vspace{5mm}
	\setlength{\fboxrule}{0.2mm}
	\setlength{\fboxsep}{5mm}	
	\fcolorbox{#1}{#1!3}{\makebox[\linewidth-2\fboxrule-2\fboxsep]{
  		\begin{minipage}[t]{\linewidth-2\fboxrule-4\fboxsep}\setlength{\parskip}{3mm}
  			 #2
  		\end{minipage}
	}}
	\vspace{5mm}
}

\newcommand\CBox[4]{				% Boites
	\vspace{5mm}
	\setlength{\fboxrule}{0.2mm}
	\setlength{\fboxsep}{5mm}
	
	\fcolorbox{#1}{#1!3}{\makebox[\linewidth-2\fboxrule-2\fboxsep]{
		\begin{minipage}[t]{1cm}\setlength{\parskip}{3mm}
	  		\textcolor{#1}{\LARGE{#2}}    
 	 	\end{minipage}  
  		\begin{minipage}[t]{\linewidth-2\fboxrule-4\fboxsep}\setlength{\parskip}{3mm}
			\raisebox{1.2mm}{\normalsize\sffamily{\textcolor{#1}{#3}}}						
  			 #4
  		\end{minipage}
	}}
	\vspace{5mm}
}

\newcommand\cadre[3]{				% Boites convertible html
	\par
	\vspace{2mm}
	\setlength{\fboxrule}{0.1mm}
	\setlength{\fboxsep}{5mm}
	\fcolorbox{#1}{white}{\makebox[\linewidth-2\fboxrule-2\fboxsep]{
  		\begin{minipage}[t]{\linewidth-2\fboxrule-4\fboxsep}\setlength{\parskip}{3mm}
			\raisebox{-2.5mm}{\sffamily \small{\textcolor{#1}{\MakeUppercase{#2}}}}		
			\par		
  			 #3
 	 		\end{minipage}
	}}
		\vspace{2mm}
	\par
}

\newcommand\bloc[3]{				% Boites convertible html sans bordure
     \needspace{2\baselineskip}
     {\sffamily \small{\textcolor{#1}{\MakeUppercase{#2}}}}    
		\par		
  			 #3
		\par
}

\newcommand\CHelp[1]{
     \CBox{Plum}{\faInfoCircle}{À RETENIR}{#1}
}

\newcommand\CUp[1]{
     \CBox{NavyBlue}{\faThumbsOUp}{EN PRATIQUE}{#1}
}

\newcommand\CInfo[1]{
     \CBox{Sepia}{\faArrowCircleRight}{REMARQUE}{#1}
}

\newcommand\CRedac[1]{
     \CBox{PineGreen}{\faEdit}{BIEN R\'EDIGER}{#1}
}

\newcommand\CError[1]{
     \CBox{Red}{\faExclamationTriangle}{ATTENTION}{#1}
}

\newcommand\TitreExo[2]{
\needspace{4\baselineskip}
 {\sffamily\large EXERCICE #1\ (\emph{#2 points})}
\vspace{5mm}
}

\newcommand\img[2]{
          \includegraphics[width=#2\paperwidth]{\imgdir#1}
}

\newcommand\imgsvg[2]{
       \begin{center}   \includegraphics[width=#2\paperwidth]{\imgsvgdir#1} \end{center}
}


\newcommand\Lien[2]{
     \href{#1}{#2 \tiny \faExternalLink}
}
\newcommand\mcLien[2]{
     \href{https~://www.maths-cours.fr/#1}{#2 \tiny \faExternalLink}
}

\newcommand{\euro}{\eurologo{}}

%================================================================================================================================
%
% Macros - Environement
%
%================================================================================================================================

\newenvironment{tex}{ %
}
{%
}

\newenvironment{indente}{ %
	\setlength\parindent{10mm}
}

{
	\setlength\parindent{0mm}
}

\newenvironment{corrige}{%
     \needspace{3\baselineskip}
     \medskip
     \textbf{\textsc{Corrigé}}
     \medskip
}
{
}

\newenvironment{extern}{%
     \begin{center}
     }
     {
     \end{center}
}

\NewEnviron{code}{%
	\par
     \boite{gray}{\texttt{%
     \BODY
     }}
     \par
}

\newenvironment{vbloc}{% boite sans cadre empeche saut de page
     \begin{minipage}[t]{\linewidth}
     }
     {
     \end{minipage}
}
\NewEnviron{h2}{%
    \needspace{3\baselineskip}
    \vspace{0.6cm}
	\noindent \MakeUppercase{\sffamily \large \BODY}
	\vspace{1mm}\textcolor{mcgris}{\hrule}\vspace{0.4cm}
	\par
}{}

\NewEnviron{h3}{%
    \needspace{3\baselineskip}
	\vspace{5mm}
	\textsc{\BODY}
	\par
}

\NewEnviron{margeneg}{ %
\begin{addmargin}[-1cm]{0cm}
\BODY
\end{addmargin}
}

\NewEnviron{html}{%
}

\begin{document}
\meta{url}{/exercices/factorisations-141028/}
\meta{pid}{1693}
\meta{titre}{Factorisations (avec identités remarquables)}
\meta{type}{exercices}
%
\documentclass[a4paper]{article}

%================================================================================================================================
%
% Packages
%
%================================================================================================================================

\usepackage[T1]{fontenc} 	% pour caractères accentués
\usepackage[utf8]{inputenc}  % encodage utf8
\usepackage[french]{babel}	% langue : français
\usepackage{fourier}			% caractères plus lisibles
\usepackage[dvipsnames]{xcolor} % couleurs
\usepackage{fancyhdr}		% réglage header footer
\usepackage{needspace}		% empêcher sauts de page mal placés
\usepackage{graphicx}		% pour inclure des graphiques
\usepackage{enumitem,cprotect}		% personnalise les listes d'items (nécessaire pour ol, al ...)
\usepackage{hyperref}		% Liens hypertexte
\usepackage{pstricks,pst-all,pst-node,pstricks-add,pst-math,pst-plot,pst-tree,pst-eucl} % pstricks
\usepackage[a4paper,includeheadfoot,top=2cm,left=3cm, bottom=2cm,right=3cm]{geometry} % marges etc.
\usepackage{comment}			% commentaires multilignes
\usepackage{amsmath,environ} % maths (matrices, etc.)
\usepackage{amssymb,makeidx}
\usepackage{bm}				% bold maths
\usepackage{tabularx}		% tableaux
\usepackage{colortbl}		% tableaux en couleur
\usepackage{fontawesome}		% Fontawesome
\usepackage{environ}			% environment with command
\usepackage{fp}				% calculs pour ps-tricks
\usepackage{multido}			% pour ps tricks
\usepackage[np]{numprint}	% formattage nombre
\usepackage{tikz,tkz-tab} 			% package principal TikZ
\usepackage{pgfplots}   % axes
\usepackage{mathrsfs}    % cursives
\usepackage{calc}			% calcul taille boites
\usepackage[scaled=0.875]{helvet} % font sans serif
\usepackage{svg} % svg
\usepackage{scrextend} % local margin
\usepackage{scratch} %scratch
\usepackage{multicol} % colonnes
%\usepackage{infix-RPN,pst-func} % formule en notation polanaise inversée
\usepackage{listings}

%================================================================================================================================
%
% Réglages de base
%
%================================================================================================================================

\lstset{
language=Python,   % R code
literate=
{á}{{\'a}}1
{à}{{\`a}}1
{ã}{{\~a}}1
{é}{{\'e}}1
{è}{{\`e}}1
{ê}{{\^e}}1
{í}{{\'i}}1
{ó}{{\'o}}1
{õ}{{\~o}}1
{ú}{{\'u}}1
{ü}{{\"u}}1
{ç}{{\c{c}}}1
{~}{{ }}1
}


\definecolor{codegreen}{rgb}{0,0.6,0}
\definecolor{codegray}{rgb}{0.5,0.5,0.5}
\definecolor{codepurple}{rgb}{0.58,0,0.82}
\definecolor{backcolour}{rgb}{0.95,0.95,0.92}

\lstdefinestyle{mystyle}{
    backgroundcolor=\color{backcolour},   
    commentstyle=\color{codegreen},
    keywordstyle=\color{magenta},
    numberstyle=\tiny\color{codegray},
    stringstyle=\color{codepurple},
    basicstyle=\ttfamily\footnotesize,
    breakatwhitespace=false,         
    breaklines=true,                 
    captionpos=b,                    
    keepspaces=true,                 
    numbers=left,                    
xleftmargin=2em,
framexleftmargin=2em,            
    showspaces=false,                
    showstringspaces=false,
    showtabs=false,                  
    tabsize=2,
    upquote=true
}

\lstset{style=mystyle}


\lstset{style=mystyle}
\newcommand{\imgdir}{C:/laragon/www/newmc/assets/imgsvg/}
\newcommand{\imgsvgdir}{C:/laragon/www/newmc/assets/imgsvg/}

\definecolor{mcgris}{RGB}{220, 220, 220}% ancien~; pour compatibilité
\definecolor{mcbleu}{RGB}{52, 152, 219}
\definecolor{mcvert}{RGB}{125, 194, 70}
\definecolor{mcmauve}{RGB}{154, 0, 215}
\definecolor{mcorange}{RGB}{255, 96, 0}
\definecolor{mcturquoise}{RGB}{0, 153, 153}
\definecolor{mcrouge}{RGB}{255, 0, 0}
\definecolor{mclightvert}{RGB}{205, 234, 190}

\definecolor{gris}{RGB}{220, 220, 220}
\definecolor{bleu}{RGB}{52, 152, 219}
\definecolor{vert}{RGB}{125, 194, 70}
\definecolor{mauve}{RGB}{154, 0, 215}
\definecolor{orange}{RGB}{255, 96, 0}
\definecolor{turquoise}{RGB}{0, 153, 153}
\definecolor{rouge}{RGB}{255, 0, 0}
\definecolor{lightvert}{RGB}{205, 234, 190}
\setitemize[0]{label=\color{lightvert}  $\bullet$}

\pagestyle{fancy}
\renewcommand{\headrulewidth}{0.2pt}
\fancyhead[L]{maths-cours.fr}
\fancyhead[R]{\thepage}
\renewcommand{\footrulewidth}{0.2pt}
\fancyfoot[C]{}

\newcolumntype{C}{>{\centering\arraybackslash}X}
\newcolumntype{s}{>{\hsize=.35\hsize\arraybackslash}X}

\setlength{\parindent}{0pt}		 
\setlength{\parskip}{3mm}
\setlength{\headheight}{1cm}

\def\ebook{ebook}
\def\book{book}
\def\web{web}
\def\type{web}

\newcommand{\vect}[1]{\overrightarrow{\,\mathstrut#1\,}}

\def\Oij{$\left(\text{O}~;~\vect{\imath},~\vect{\jmath}\right)$}
\def\Oijk{$\left(\text{O}~;~\vect{\imath},~\vect{\jmath},~\vect{k}\right)$}
\def\Ouv{$\left(\text{O}~;~\vect{u},~\vect{v}\right)$}

\hypersetup{breaklinks=true, colorlinks = true, linkcolor = OliveGreen, urlcolor = OliveGreen, citecolor = OliveGreen, pdfauthor={Didier BONNEL - https://www.maths-cours.fr} } % supprime les bordures autour des liens

\renewcommand{\arg}[0]{\text{arg}}

\everymath{\displaystyle}

%================================================================================================================================
%
% Macros - Commandes
%
%================================================================================================================================

\newcommand\meta[2]{    			% Utilisé pour créer le post HTML.
	\def\titre{titre}
	\def\url{url}
	\def\arg{#1}
	\ifx\titre\arg
		\newcommand\maintitle{#2}
		\fancyhead[L]{#2}
		{\Large\sffamily \MakeUppercase{#2}}
		\vspace{1mm}\textcolor{mcvert}{\hrule}
	\fi 
	\ifx\url\arg
		\fancyfoot[L]{\href{https://www.maths-cours.fr#2}{\black \footnotesize{https://www.maths-cours.fr#2}}}
	\fi 
}


\newcommand\TitreC[1]{    		% Titre centré
     \needspace{3\baselineskip}
     \begin{center}\textbf{#1}\end{center}
}

\newcommand\newpar{    		% paragraphe
     \par
}

\newcommand\nosp {    		% commande vide (pas d'espace)
}
\newcommand{\id}[1]{} %ignore

\newcommand\boite[2]{				% Boite simple sans titre
	\vspace{5mm}
	\setlength{\fboxrule}{0.2mm}
	\setlength{\fboxsep}{5mm}	
	\fcolorbox{#1}{#1!3}{\makebox[\linewidth-2\fboxrule-2\fboxsep]{
  		\begin{minipage}[t]{\linewidth-2\fboxrule-4\fboxsep}\setlength{\parskip}{3mm}
  			 #2
  		\end{minipage}
	}}
	\vspace{5mm}
}

\newcommand\CBox[4]{				% Boites
	\vspace{5mm}
	\setlength{\fboxrule}{0.2mm}
	\setlength{\fboxsep}{5mm}
	
	\fcolorbox{#1}{#1!3}{\makebox[\linewidth-2\fboxrule-2\fboxsep]{
		\begin{minipage}[t]{1cm}\setlength{\parskip}{3mm}
	  		\textcolor{#1}{\LARGE{#2}}    
 	 	\end{minipage}  
  		\begin{minipage}[t]{\linewidth-2\fboxrule-4\fboxsep}\setlength{\parskip}{3mm}
			\raisebox{1.2mm}{\normalsize\sffamily{\textcolor{#1}{#3}}}						
  			 #4
  		\end{minipage}
	}}
	\vspace{5mm}
}

\newcommand\cadre[3]{				% Boites convertible html
	\par
	\vspace{2mm}
	\setlength{\fboxrule}{0.1mm}
	\setlength{\fboxsep}{5mm}
	\fcolorbox{#1}{white}{\makebox[\linewidth-2\fboxrule-2\fboxsep]{
  		\begin{minipage}[t]{\linewidth-2\fboxrule-4\fboxsep}\setlength{\parskip}{3mm}
			\raisebox{-2.5mm}{\sffamily \small{\textcolor{#1}{\MakeUppercase{#2}}}}		
			\par		
  			 #3
 	 		\end{minipage}
	}}
		\vspace{2mm}
	\par
}

\newcommand\bloc[3]{				% Boites convertible html sans bordure
     \needspace{2\baselineskip}
     {\sffamily \small{\textcolor{#1}{\MakeUppercase{#2}}}}    
		\par		
  			 #3
		\par
}

\newcommand\CHelp[1]{
     \CBox{Plum}{\faInfoCircle}{À RETENIR}{#1}
}

\newcommand\CUp[1]{
     \CBox{NavyBlue}{\faThumbsOUp}{EN PRATIQUE}{#1}
}

\newcommand\CInfo[1]{
     \CBox{Sepia}{\faArrowCircleRight}{REMARQUE}{#1}
}

\newcommand\CRedac[1]{
     \CBox{PineGreen}{\faEdit}{BIEN R\'EDIGER}{#1}
}

\newcommand\CError[1]{
     \CBox{Red}{\faExclamationTriangle}{ATTENTION}{#1}
}

\newcommand\TitreExo[2]{
\needspace{4\baselineskip}
 {\sffamily\large EXERCICE #1\ (\emph{#2 points})}
\vspace{5mm}
}

\newcommand\img[2]{
          \includegraphics[width=#2\paperwidth]{\imgdir#1}
}

\newcommand\imgsvg[2]{
       \begin{center}   \includegraphics[width=#2\paperwidth]{\imgsvgdir#1} \end{center}
}


\newcommand\Lien[2]{
     \href{#1}{#2 \tiny \faExternalLink}
}
\newcommand\mcLien[2]{
     \href{https~://www.maths-cours.fr/#1}{#2 \tiny \faExternalLink}
}

\newcommand{\euro}{\eurologo{}}

%================================================================================================================================
%
% Macros - Environement
%
%================================================================================================================================

\newenvironment{tex}{ %
}
{%
}

\newenvironment{indente}{ %
	\setlength\parindent{10mm}
}

{
	\setlength\parindent{0mm}
}

\newenvironment{corrige}{%
     \needspace{3\baselineskip}
     \medskip
     \textbf{\textsc{Corrigé}}
     \medskip
}
{
}

\newenvironment{extern}{%
     \begin{center}
     }
     {
     \end{center}
}

\NewEnviron{code}{%
	\par
     \boite{gray}{\texttt{%
     \BODY
     }}
     \par
}

\newenvironment{vbloc}{% boite sans cadre empeche saut de page
     \begin{minipage}[t]{\linewidth}
     }
     {
     \end{minipage}
}
\NewEnviron{h2}{%
    \needspace{3\baselineskip}
    \vspace{0.6cm}
	\noindent \MakeUppercase{\sffamily \large \BODY}
	\vspace{1mm}\textcolor{mcgris}{\hrule}\vspace{0.4cm}
	\par
}{}

\NewEnviron{h3}{%
    \needspace{3\baselineskip}
	\vspace{5mm}
	\textsc{\BODY}
	\par
}

\NewEnviron{margeneg}{ %
\begin{addmargin}[-1cm]{0cm}
\BODY
\end{addmargin}
}

\NewEnviron{html}{%
}

\begin{document}
Factoriser les expressions suivantes :
\begin{enumerate}
     \item
     $A=x^{2}-36$
     \item
     $B=4x^{2}+4x+1$
     \item
     $C=7x^{2}-3x$
     \item
     $D=x^{2}-6xy+9y^{2}$
\end{enumerate}
\begin{corrige}
     \begin{enumerate}
          \item
          $A=x^{2}-36=x^{2}-6^{2}$
          \par
          On utilise l'identité remarquable $a^{2}-b^{2}=\left(a+b\right)\left(a-b\right)$ avec $a=x$ et $b=6$
          \par
          $A=\left(x+6\right)\left(x-6\right)$
          \item
          $B=4x^{2}+4x+1=\left(2x\right)^{2}+2\times 2x\times 1+1^{2}$
          \par
          On utilise l'identité remarquable $a^{2}+2ab+b^{2}=\left(a+b\right)^{2}$ avec $a=2x$ et $b=1$
          \par
          $B=\left(2x+1\right)^{2}$
          \item
          Ici, pas d'identité remarquable mais on peut mettre $x$ en facteur :
          \par
          $C=7x^{2}-3x=7x\times \color{red}{x}-3\color{red}{x}=x\left(7x-3\right)$
          \item
          $D=x^{2}-6xy+9y^{2}=x^{2}-2\times x\times 3y+\left(3y\right)^{2}$
          \par
          On utilise l'identité remarquable $a^{2}-2ab+b^{2}=\left(a-b\right)^{2}$ avec $a=x$ et $b=3y$
          \par
          $D=\left(x-3y\right)^{2}$
     \end{enumerate}
\end{corrige}

\end{document}
\end{document}


µ
\documentclass[a4paper]{article}

%================================================================================================================================
%
% Packages
%
%================================================================================================================================

\usepackage[T1]{fontenc} 	% pour caractères accentués
\usepackage[utf8]{inputenc}  % encodage utf8
\usepackage[french]{babel}	% langue : français
\usepackage{fourier}			% caractères plus lisibles
\usepackage[dvipsnames]{xcolor} % couleurs
\usepackage{fancyhdr}		% réglage header footer
\usepackage{needspace}		% empêcher sauts de page mal placés
\usepackage{graphicx}		% pour inclure des graphiques
\usepackage{enumitem,cprotect}		% personnalise les listes d'items (nécessaire pour ol, al ...)
\usepackage{hyperref}		% Liens hypertexte
\usepackage{pstricks,pst-all,pst-node,pstricks-add,pst-math,pst-plot,pst-tree,pst-eucl} % pstricks
\usepackage[a4paper,includeheadfoot,top=2cm,left=3cm, bottom=2cm,right=3cm]{geometry} % marges etc.
\usepackage{comment}			% commentaires multilignes
\usepackage{amsmath,environ} % maths (matrices, etc.)
\usepackage{amssymb,makeidx}
\usepackage{bm}				% bold maths
\usepackage{tabularx}		% tableaux
\usepackage{colortbl}		% tableaux en couleur
\usepackage{fontawesome}		% Fontawesome
\usepackage{environ}			% environment with command
\usepackage{fp}				% calculs pour ps-tricks
\usepackage{multido}			% pour ps tricks
\usepackage[np]{numprint}	% formattage nombre
\usepackage{tikz,tkz-tab} 			% package principal TikZ
\usepackage{pgfplots}   % axes
\usepackage{mathrsfs}    % cursives
\usepackage{calc}			% calcul taille boites
\usepackage[scaled=0.875]{helvet} % font sans serif
\usepackage{svg} % svg
\usepackage{scrextend} % local margin
\usepackage{scratch} %scratch
\usepackage{multicol} % colonnes
%\usepackage{infix-RPN,pst-func} % formule en notation polanaise inversée
\usepackage{listings}

%================================================================================================================================
%
% Réglages de base
%
%================================================================================================================================

\lstset{
language=Python,   % R code
literate=
{á}{{\'a}}1
{à}{{\`a}}1
{ã}{{\~a}}1
{é}{{\'e}}1
{è}{{\`e}}1
{ê}{{\^e}}1
{í}{{\'i}}1
{ó}{{\'o}}1
{õ}{{\~o}}1
{ú}{{\'u}}1
{ü}{{\"u}}1
{ç}{{\c{c}}}1
{~}{{ }}1
}


\definecolor{codegreen}{rgb}{0,0.6,0}
\definecolor{codegray}{rgb}{0.5,0.5,0.5}
\definecolor{codepurple}{rgb}{0.58,0,0.82}
\definecolor{backcolour}{rgb}{0.95,0.95,0.92}

\lstdefinestyle{mystyle}{
    backgroundcolor=\color{backcolour},   
    commentstyle=\color{codegreen},
    keywordstyle=\color{magenta},
    numberstyle=\tiny\color{codegray},
    stringstyle=\color{codepurple},
    basicstyle=\ttfamily\footnotesize,
    breakatwhitespace=false,         
    breaklines=true,                 
    captionpos=b,                    
    keepspaces=true,                 
    numbers=left,                    
xleftmargin=2em,
framexleftmargin=2em,            
    showspaces=false,                
    showstringspaces=false,
    showtabs=false,                  
    tabsize=2,
    upquote=true
}

\lstset{style=mystyle}


\lstset{style=mystyle}
\newcommand{\imgdir}{C:/laragon/www/newmc/assets/imgsvg/}
\newcommand{\imgsvgdir}{C:/laragon/www/newmc/assets/imgsvg/}

\definecolor{mcgris}{RGB}{220, 220, 220}% ancien~; pour compatibilité
\definecolor{mcbleu}{RGB}{52, 152, 219}
\definecolor{mcvert}{RGB}{125, 194, 70}
\definecolor{mcmauve}{RGB}{154, 0, 215}
\definecolor{mcorange}{RGB}{255, 96, 0}
\definecolor{mcturquoise}{RGB}{0, 153, 153}
\definecolor{mcrouge}{RGB}{255, 0, 0}
\definecolor{mclightvert}{RGB}{205, 234, 190}

\definecolor{gris}{RGB}{220, 220, 220}
\definecolor{bleu}{RGB}{52, 152, 219}
\definecolor{vert}{RGB}{125, 194, 70}
\definecolor{mauve}{RGB}{154, 0, 215}
\definecolor{orange}{RGB}{255, 96, 0}
\definecolor{turquoise}{RGB}{0, 153, 153}
\definecolor{rouge}{RGB}{255, 0, 0}
\definecolor{lightvert}{RGB}{205, 234, 190}
\setitemize[0]{label=\color{lightvert}  $\bullet$}

\pagestyle{fancy}
\renewcommand{\headrulewidth}{0.2pt}
\fancyhead[L]{maths-cours.fr}
\fancyhead[R]{\thepage}
\renewcommand{\footrulewidth}{0.2pt}
\fancyfoot[C]{}

\newcolumntype{C}{>{\centering\arraybackslash}X}
\newcolumntype{s}{>{\hsize=.35\hsize\arraybackslash}X}

\setlength{\parindent}{0pt}		 
\setlength{\parskip}{3mm}
\setlength{\headheight}{1cm}

\def\ebook{ebook}
\def\book{book}
\def\web{web}
\def\type{web}

\newcommand{\vect}[1]{\overrightarrow{\,\mathstrut#1\,}}

\def\Oij{$\left(\text{O}~;~\vect{\imath},~\vect{\jmath}\right)$}
\def\Oijk{$\left(\text{O}~;~\vect{\imath},~\vect{\jmath},~\vect{k}\right)$}
\def\Ouv{$\left(\text{O}~;~\vect{u},~\vect{v}\right)$}

\hypersetup{breaklinks=true, colorlinks = true, linkcolor = OliveGreen, urlcolor = OliveGreen, citecolor = OliveGreen, pdfauthor={Didier BONNEL - https://www.maths-cours.fr} } % supprime les bordures autour des liens

\renewcommand{\arg}[0]{\text{arg}}

\everymath{\displaystyle}

%================================================================================================================================
%
% Macros - Commandes
%
%================================================================================================================================

\newcommand\meta[2]{    			% Utilisé pour créer le post HTML.
	\def\titre{titre}
	\def\url{url}
	\def\arg{#1}
	\ifx\titre\arg
		\newcommand\maintitle{#2}
		\fancyhead[L]{#2}
		{\Large\sffamily \MakeUppercase{#2}}
		\vspace{1mm}\textcolor{mcvert}{\hrule}
	\fi 
	\ifx\url\arg
		\fancyfoot[L]{\href{https://www.maths-cours.fr#2}{\black \footnotesize{https://www.maths-cours.fr#2}}}
	\fi 
}


\newcommand\TitreC[1]{    		% Titre centré
     \needspace{3\baselineskip}
     \begin{center}\textbf{#1}\end{center}
}

\newcommand\newpar{    		% paragraphe
     \par
}

\newcommand\nosp {    		% commande vide (pas d'espace)
}
\newcommand{\id}[1]{} %ignore

\newcommand\boite[2]{				% Boite simple sans titre
	\vspace{5mm}
	\setlength{\fboxrule}{0.2mm}
	\setlength{\fboxsep}{5mm}	
	\fcolorbox{#1}{#1!3}{\makebox[\linewidth-2\fboxrule-2\fboxsep]{
  		\begin{minipage}[t]{\linewidth-2\fboxrule-4\fboxsep}\setlength{\parskip}{3mm}
  			 #2
  		\end{minipage}
	}}
	\vspace{5mm}
}

\newcommand\CBox[4]{				% Boites
	\vspace{5mm}
	\setlength{\fboxrule}{0.2mm}
	\setlength{\fboxsep}{5mm}
	
	\fcolorbox{#1}{#1!3}{\makebox[\linewidth-2\fboxrule-2\fboxsep]{
		\begin{minipage}[t]{1cm}\setlength{\parskip}{3mm}
	  		\textcolor{#1}{\LARGE{#2}}    
 	 	\end{minipage}  
  		\begin{minipage}[t]{\linewidth-2\fboxrule-4\fboxsep}\setlength{\parskip}{3mm}
			\raisebox{1.2mm}{\normalsize\sffamily{\textcolor{#1}{#3}}}						
  			 #4
  		\end{minipage}
	}}
	\vspace{5mm}
}

\newcommand\cadre[3]{				% Boites convertible html
	\par
	\vspace{2mm}
	\setlength{\fboxrule}{0.1mm}
	\setlength{\fboxsep}{5mm}
	\fcolorbox{#1}{white}{\makebox[\linewidth-2\fboxrule-2\fboxsep]{
  		\begin{minipage}[t]{\linewidth-2\fboxrule-4\fboxsep}\setlength{\parskip}{3mm}
			\raisebox{-2.5mm}{\sffamily \small{\textcolor{#1}{\MakeUppercase{#2}}}}		
			\par		
  			 #3
 	 		\end{minipage}
	}}
		\vspace{2mm}
	\par
}

\newcommand\bloc[3]{				% Boites convertible html sans bordure
     \needspace{2\baselineskip}
     {\sffamily \small{\textcolor{#1}{\MakeUppercase{#2}}}}    
		\par		
  			 #3
		\par
}

\newcommand\CHelp[1]{
     \CBox{Plum}{\faInfoCircle}{À RETENIR}{#1}
}

\newcommand\CUp[1]{
     \CBox{NavyBlue}{\faThumbsOUp}{EN PRATIQUE}{#1}
}

\newcommand\CInfo[1]{
     \CBox{Sepia}{\faArrowCircleRight}{REMARQUE}{#1}
}

\newcommand\CRedac[1]{
     \CBox{PineGreen}{\faEdit}{BIEN R\'EDIGER}{#1}
}

\newcommand\CError[1]{
     \CBox{Red}{\faExclamationTriangle}{ATTENTION}{#1}
}

\newcommand\TitreExo[2]{
\needspace{4\baselineskip}
 {\sffamily\large EXERCICE #1\ (\emph{#2 points})}
\vspace{5mm}
}

\newcommand\img[2]{
          \includegraphics[width=#2\paperwidth]{\imgdir#1}
}

\newcommand\imgsvg[2]{
       \begin{center}   \includegraphics[width=#2\paperwidth]{\imgsvgdir#1} \end{center}
}


\newcommand\Lien[2]{
     \href{#1}{#2 \tiny \faExternalLink}
}
\newcommand\mcLien[2]{
     \href{https~://www.maths-cours.fr/#1}{#2 \tiny \faExternalLink}
}

\newcommand{\euro}{\eurologo{}}

%================================================================================================================================
%
% Macros - Environement
%
%================================================================================================================================

\newenvironment{tex}{ %
}
{%
}

\newenvironment{indente}{ %
	\setlength\parindent{10mm}
}

{
	\setlength\parindent{0mm}
}

\newenvironment{corrige}{%
     \needspace{3\baselineskip}
     \medskip
     \textbf{\textsc{Corrigé}}
     \medskip
}
{
}

\newenvironment{extern}{%
     \begin{center}
     }
     {
     \end{center}
}

\NewEnviron{code}{%
	\par
     \boite{gray}{\texttt{%
     \BODY
     }}
     \par
}

\newenvironment{vbloc}{% boite sans cadre empeche saut de page
     \begin{minipage}[t]{\linewidth}
     }
     {
     \end{minipage}
}
\NewEnviron{h2}{%
    \needspace{3\baselineskip}
    \vspace{0.6cm}
	\noindent \MakeUppercase{\sffamily \large \BODY}
	\vspace{1mm}\textcolor{mcgris}{\hrule}\vspace{0.4cm}
	\par
}{}

\NewEnviron{h3}{%
    \needspace{3\baselineskip}
	\vspace{5mm}
	\textsc{\BODY}
	\par
}

\NewEnviron{margeneg}{ %
\begin{addmargin}[-1cm]{0cm}
\BODY
\end{addmargin}
}

\NewEnviron{html}{%
}

\begin{document}
\meta{url}{/exercices/identites-remarquables-brevet-2002-141030/}
\meta{pid}{1698}
\meta{titre}{Identités remarquables (Brevet 2002)}
\meta{type}{exercices}
%
\documentclass[a4paper]{article}

%================================================================================================================================
%
% Packages
%
%================================================================================================================================

\usepackage[T1]{fontenc} 	% pour caractères accentués
\usepackage[utf8]{inputenc}  % encodage utf8
\usepackage[french]{babel}	% langue : français
\usepackage{fourier}			% caractères plus lisibles
\usepackage[dvipsnames]{xcolor} % couleurs
\usepackage{fancyhdr}		% réglage header footer
\usepackage{needspace}		% empêcher sauts de page mal placés
\usepackage{graphicx}		% pour inclure des graphiques
\usepackage{enumitem,cprotect}		% personnalise les listes d'items (nécessaire pour ol, al ...)
\usepackage{hyperref}		% Liens hypertexte
\usepackage{pstricks,pst-all,pst-node,pstricks-add,pst-math,pst-plot,pst-tree,pst-eucl} % pstricks
\usepackage[a4paper,includeheadfoot,top=2cm,left=3cm, bottom=2cm,right=3cm]{geometry} % marges etc.
\usepackage{comment}			% commentaires multilignes
\usepackage{amsmath,environ} % maths (matrices, etc.)
\usepackage{amssymb,makeidx}
\usepackage{bm}				% bold maths
\usepackage{tabularx}		% tableaux
\usepackage{colortbl}		% tableaux en couleur
\usepackage{fontawesome}		% Fontawesome
\usepackage{environ}			% environment with command
\usepackage{fp}				% calculs pour ps-tricks
\usepackage{multido}			% pour ps tricks
\usepackage[np]{numprint}	% formattage nombre
\usepackage{tikz,tkz-tab} 			% package principal TikZ
\usepackage{pgfplots}   % axes
\usepackage{mathrsfs}    % cursives
\usepackage{calc}			% calcul taille boites
\usepackage[scaled=0.875]{helvet} % font sans serif
\usepackage{svg} % svg
\usepackage{scrextend} % local margin
\usepackage{scratch} %scratch
\usepackage{multicol} % colonnes
%\usepackage{infix-RPN,pst-func} % formule en notation polanaise inversée
\usepackage{listings}

%================================================================================================================================
%
% Réglages de base
%
%================================================================================================================================

\lstset{
language=Python,   % R code
literate=
{á}{{\'a}}1
{à}{{\`a}}1
{ã}{{\~a}}1
{é}{{\'e}}1
{è}{{\`e}}1
{ê}{{\^e}}1
{í}{{\'i}}1
{ó}{{\'o}}1
{õ}{{\~o}}1
{ú}{{\'u}}1
{ü}{{\"u}}1
{ç}{{\c{c}}}1
{~}{{ }}1
}


\definecolor{codegreen}{rgb}{0,0.6,0}
\definecolor{codegray}{rgb}{0.5,0.5,0.5}
\definecolor{codepurple}{rgb}{0.58,0,0.82}
\definecolor{backcolour}{rgb}{0.95,0.95,0.92}

\lstdefinestyle{mystyle}{
    backgroundcolor=\color{backcolour},   
    commentstyle=\color{codegreen},
    keywordstyle=\color{magenta},
    numberstyle=\tiny\color{codegray},
    stringstyle=\color{codepurple},
    basicstyle=\ttfamily\footnotesize,
    breakatwhitespace=false,         
    breaklines=true,                 
    captionpos=b,                    
    keepspaces=true,                 
    numbers=left,                    
xleftmargin=2em,
framexleftmargin=2em,            
    showspaces=false,                
    showstringspaces=false,
    showtabs=false,                  
    tabsize=2,
    upquote=true
}

\lstset{style=mystyle}


\lstset{style=mystyle}
\newcommand{\imgdir}{C:/laragon/www/newmc/assets/imgsvg/}
\newcommand{\imgsvgdir}{C:/laragon/www/newmc/assets/imgsvg/}

\definecolor{mcgris}{RGB}{220, 220, 220}% ancien~; pour compatibilité
\definecolor{mcbleu}{RGB}{52, 152, 219}
\definecolor{mcvert}{RGB}{125, 194, 70}
\definecolor{mcmauve}{RGB}{154, 0, 215}
\definecolor{mcorange}{RGB}{255, 96, 0}
\definecolor{mcturquoise}{RGB}{0, 153, 153}
\definecolor{mcrouge}{RGB}{255, 0, 0}
\definecolor{mclightvert}{RGB}{205, 234, 190}

\definecolor{gris}{RGB}{220, 220, 220}
\definecolor{bleu}{RGB}{52, 152, 219}
\definecolor{vert}{RGB}{125, 194, 70}
\definecolor{mauve}{RGB}{154, 0, 215}
\definecolor{orange}{RGB}{255, 96, 0}
\definecolor{turquoise}{RGB}{0, 153, 153}
\definecolor{rouge}{RGB}{255, 0, 0}
\definecolor{lightvert}{RGB}{205, 234, 190}
\setitemize[0]{label=\color{lightvert}  $\bullet$}

\pagestyle{fancy}
\renewcommand{\headrulewidth}{0.2pt}
\fancyhead[L]{maths-cours.fr}
\fancyhead[R]{\thepage}
\renewcommand{\footrulewidth}{0.2pt}
\fancyfoot[C]{}

\newcolumntype{C}{>{\centering\arraybackslash}X}
\newcolumntype{s}{>{\hsize=.35\hsize\arraybackslash}X}

\setlength{\parindent}{0pt}		 
\setlength{\parskip}{3mm}
\setlength{\headheight}{1cm}

\def\ebook{ebook}
\def\book{book}
\def\web{web}
\def\type{web}

\newcommand{\vect}[1]{\overrightarrow{\,\mathstrut#1\,}}

\def\Oij{$\left(\text{O}~;~\vect{\imath},~\vect{\jmath}\right)$}
\def\Oijk{$\left(\text{O}~;~\vect{\imath},~\vect{\jmath},~\vect{k}\right)$}
\def\Ouv{$\left(\text{O}~;~\vect{u},~\vect{v}\right)$}

\hypersetup{breaklinks=true, colorlinks = true, linkcolor = OliveGreen, urlcolor = OliveGreen, citecolor = OliveGreen, pdfauthor={Didier BONNEL - https://www.maths-cours.fr} } % supprime les bordures autour des liens

\renewcommand{\arg}[0]{\text{arg}}

\everymath{\displaystyle}

%================================================================================================================================
%
% Macros - Commandes
%
%================================================================================================================================

\newcommand\meta[2]{    			% Utilisé pour créer le post HTML.
	\def\titre{titre}
	\def\url{url}
	\def\arg{#1}
	\ifx\titre\arg
		\newcommand\maintitle{#2}
		\fancyhead[L]{#2}
		{\Large\sffamily \MakeUppercase{#2}}
		\vspace{1mm}\textcolor{mcvert}{\hrule}
	\fi 
	\ifx\url\arg
		\fancyfoot[L]{\href{https://www.maths-cours.fr#2}{\black \footnotesize{https://www.maths-cours.fr#2}}}
	\fi 
}


\newcommand\TitreC[1]{    		% Titre centré
     \needspace{3\baselineskip}
     \begin{center}\textbf{#1}\end{center}
}

\newcommand\newpar{    		% paragraphe
     \par
}

\newcommand\nosp {    		% commande vide (pas d'espace)
}
\newcommand{\id}[1]{} %ignore

\newcommand\boite[2]{				% Boite simple sans titre
	\vspace{5mm}
	\setlength{\fboxrule}{0.2mm}
	\setlength{\fboxsep}{5mm}	
	\fcolorbox{#1}{#1!3}{\makebox[\linewidth-2\fboxrule-2\fboxsep]{
  		\begin{minipage}[t]{\linewidth-2\fboxrule-4\fboxsep}\setlength{\parskip}{3mm}
  			 #2
  		\end{minipage}
	}}
	\vspace{5mm}
}

\newcommand\CBox[4]{				% Boites
	\vspace{5mm}
	\setlength{\fboxrule}{0.2mm}
	\setlength{\fboxsep}{5mm}
	
	\fcolorbox{#1}{#1!3}{\makebox[\linewidth-2\fboxrule-2\fboxsep]{
		\begin{minipage}[t]{1cm}\setlength{\parskip}{3mm}
	  		\textcolor{#1}{\LARGE{#2}}    
 	 	\end{minipage}  
  		\begin{minipage}[t]{\linewidth-2\fboxrule-4\fboxsep}\setlength{\parskip}{3mm}
			\raisebox{1.2mm}{\normalsize\sffamily{\textcolor{#1}{#3}}}						
  			 #4
  		\end{minipage}
	}}
	\vspace{5mm}
}

\newcommand\cadre[3]{				% Boites convertible html
	\par
	\vspace{2mm}
	\setlength{\fboxrule}{0.1mm}
	\setlength{\fboxsep}{5mm}
	\fcolorbox{#1}{white}{\makebox[\linewidth-2\fboxrule-2\fboxsep]{
  		\begin{minipage}[t]{\linewidth-2\fboxrule-4\fboxsep}\setlength{\parskip}{3mm}
			\raisebox{-2.5mm}{\sffamily \small{\textcolor{#1}{\MakeUppercase{#2}}}}		
			\par		
  			 #3
 	 		\end{minipage}
	}}
		\vspace{2mm}
	\par
}

\newcommand\bloc[3]{				% Boites convertible html sans bordure
     \needspace{2\baselineskip}
     {\sffamily \small{\textcolor{#1}{\MakeUppercase{#2}}}}    
		\par		
  			 #3
		\par
}

\newcommand\CHelp[1]{
     \CBox{Plum}{\faInfoCircle}{À RETENIR}{#1}
}

\newcommand\CUp[1]{
     \CBox{NavyBlue}{\faThumbsOUp}{EN PRATIQUE}{#1}
}

\newcommand\CInfo[1]{
     \CBox{Sepia}{\faArrowCircleRight}{REMARQUE}{#1}
}

\newcommand\CRedac[1]{
     \CBox{PineGreen}{\faEdit}{BIEN R\'EDIGER}{#1}
}

\newcommand\CError[1]{
     \CBox{Red}{\faExclamationTriangle}{ATTENTION}{#1}
}

\newcommand\TitreExo[2]{
\needspace{4\baselineskip}
 {\sffamily\large EXERCICE #1\ (\emph{#2 points})}
\vspace{5mm}
}

\newcommand\img[2]{
          \includegraphics[width=#2\paperwidth]{\imgdir#1}
}

\newcommand\imgsvg[2]{
       \begin{center}   \includegraphics[width=#2\paperwidth]{\imgsvgdir#1} \end{center}
}


\newcommand\Lien[2]{
     \href{#1}{#2 \tiny \faExternalLink}
}
\newcommand\mcLien[2]{
     \href{https~://www.maths-cours.fr/#1}{#2 \tiny \faExternalLink}
}

\newcommand{\euro}{\eurologo{}}

%================================================================================================================================
%
% Macros - Environement
%
%================================================================================================================================

\newenvironment{tex}{ %
}
{%
}

\newenvironment{indente}{ %
	\setlength\parindent{10mm}
}

{
	\setlength\parindent{0mm}
}

\newenvironment{corrige}{%
     \needspace{3\baselineskip}
     \medskip
     \textbf{\textsc{Corrigé}}
     \medskip
}
{
}

\newenvironment{extern}{%
     \begin{center}
     }
     {
     \end{center}
}

\NewEnviron{code}{%
	\par
     \boite{gray}{\texttt{%
     \BODY
     }}
     \par
}

\newenvironment{vbloc}{% boite sans cadre empeche saut de page
     \begin{minipage}[t]{\linewidth}
     }
     {
     \end{minipage}
}
\NewEnviron{h2}{%
    \needspace{3\baselineskip}
    \vspace{0.6cm}
	\noindent \MakeUppercase{\sffamily \large \BODY}
	\vspace{1mm}\textcolor{mcgris}{\hrule}\vspace{0.4cm}
	\par
}{}

\NewEnviron{h3}{%
    \needspace{3\baselineskip}
	\vspace{5mm}
	\textsc{\BODY}
	\par
}

\NewEnviron{margeneg}{ %
\begin{addmargin}[-1cm]{0cm}
\BODY
\end{addmargin}
}

\NewEnviron{html}{%
}

\begin{document}
\textit{(Brevet, Centres étrangers 2002)}
\par
Recopier et compléter pour que les égalités soient vraies pour toutes les valeurs de $x$ :
\begin{enumerate}
     \item
     $\left(x+\cdots\right)^{2} = \cdots + 6x + \cdots$
     \item
     $\left(\cdots-\cdots\right)^{2} = 4x^{2} \cdots + 25$
     \item
     $\cdots-64 = \left(7x-\cdots\right)\left( \cdots + \cdots\right)$
\end{enumerate}
\begin{corrige}
     \begin{enumerate}
          \item
          $\left(x+\color{red}{3}\right)^{2} = \color{red}{x^{2}} + 6x + \color{red}{9}   $identité remarquable $\left(a+b\right)^{2}=a^{2}+2ab+b^{2}$
          \item
          $\left(\color{red}{2x}-\color{red}{5}\right)^{2} = 4x^{2} \color{red}{-} \color{red}{20x} + 25    $ identité remarquable $\left(a-b\right)^{2}=a^{2}-2ab+b^{2}$
          \item
          $\color{red}{49x^{2}}-64 = \left(7x-\color{red}{8}\right)\left(\color{red}{7x} + \color{red}{8}\right)   $ identité remarquable $a^{2}-b^{2} = \left(a-b\right)\left(a+b\right)$
     \end{enumerate}
\end{corrige}

\end{document}
\end{document}


µ
\documentclass[a4paper]{article}

%================================================================================================================================
%
% Packages
%
%================================================================================================================================

\usepackage[T1]{fontenc} 	% pour caractères accentués
\usepackage[utf8]{inputenc}  % encodage utf8
\usepackage[french]{babel}	% langue : français
\usepackage{fourier}			% caractères plus lisibles
\usepackage[dvipsnames]{xcolor} % couleurs
\usepackage{fancyhdr}		% réglage header footer
\usepackage{needspace}		% empêcher sauts de page mal placés
\usepackage{graphicx}		% pour inclure des graphiques
\usepackage{enumitem,cprotect}		% personnalise les listes d'items (nécessaire pour ol, al ...)
\usepackage{hyperref}		% Liens hypertexte
\usepackage{pstricks,pst-all,pst-node,pstricks-add,pst-math,pst-plot,pst-tree,pst-eucl} % pstricks
\usepackage[a4paper,includeheadfoot,top=2cm,left=3cm, bottom=2cm,right=3cm]{geometry} % marges etc.
\usepackage{comment}			% commentaires multilignes
\usepackage{amsmath,environ} % maths (matrices, etc.)
\usepackage{amssymb,makeidx}
\usepackage{bm}				% bold maths
\usepackage{tabularx}		% tableaux
\usepackage{colortbl}		% tableaux en couleur
\usepackage{fontawesome}		% Fontawesome
\usepackage{environ}			% environment with command
\usepackage{fp}				% calculs pour ps-tricks
\usepackage{multido}			% pour ps tricks
\usepackage[np]{numprint}	% formattage nombre
\usepackage{tikz,tkz-tab} 			% package principal TikZ
\usepackage{pgfplots}   % axes
\usepackage{mathrsfs}    % cursives
\usepackage{calc}			% calcul taille boites
\usepackage[scaled=0.875]{helvet} % font sans serif
\usepackage{svg} % svg
\usepackage{scrextend} % local margin
\usepackage{scratch} %scratch
\usepackage{multicol} % colonnes
%\usepackage{infix-RPN,pst-func} % formule en notation polanaise inversée
\usepackage{listings}

%================================================================================================================================
%
% Réglages de base
%
%================================================================================================================================

\lstset{
language=Python,   % R code
literate=
{á}{{\'a}}1
{à}{{\`a}}1
{ã}{{\~a}}1
{é}{{\'e}}1
{è}{{\`e}}1
{ê}{{\^e}}1
{í}{{\'i}}1
{ó}{{\'o}}1
{õ}{{\~o}}1
{ú}{{\'u}}1
{ü}{{\"u}}1
{ç}{{\c{c}}}1
{~}{{ }}1
}


\definecolor{codegreen}{rgb}{0,0.6,0}
\definecolor{codegray}{rgb}{0.5,0.5,0.5}
\definecolor{codepurple}{rgb}{0.58,0,0.82}
\definecolor{backcolour}{rgb}{0.95,0.95,0.92}

\lstdefinestyle{mystyle}{
    backgroundcolor=\color{backcolour},   
    commentstyle=\color{codegreen},
    keywordstyle=\color{magenta},
    numberstyle=\tiny\color{codegray},
    stringstyle=\color{codepurple},
    basicstyle=\ttfamily\footnotesize,
    breakatwhitespace=false,         
    breaklines=true,                 
    captionpos=b,                    
    keepspaces=true,                 
    numbers=left,                    
xleftmargin=2em,
framexleftmargin=2em,            
    showspaces=false,                
    showstringspaces=false,
    showtabs=false,                  
    tabsize=2,
    upquote=true
}

\lstset{style=mystyle}


\lstset{style=mystyle}
\newcommand{\imgdir}{C:/laragon/www/newmc/assets/imgsvg/}
\newcommand{\imgsvgdir}{C:/laragon/www/newmc/assets/imgsvg/}

\definecolor{mcgris}{RGB}{220, 220, 220}% ancien~; pour compatibilité
\definecolor{mcbleu}{RGB}{52, 152, 219}
\definecolor{mcvert}{RGB}{125, 194, 70}
\definecolor{mcmauve}{RGB}{154, 0, 215}
\definecolor{mcorange}{RGB}{255, 96, 0}
\definecolor{mcturquoise}{RGB}{0, 153, 153}
\definecolor{mcrouge}{RGB}{255, 0, 0}
\definecolor{mclightvert}{RGB}{205, 234, 190}

\definecolor{gris}{RGB}{220, 220, 220}
\definecolor{bleu}{RGB}{52, 152, 219}
\definecolor{vert}{RGB}{125, 194, 70}
\definecolor{mauve}{RGB}{154, 0, 215}
\definecolor{orange}{RGB}{255, 96, 0}
\definecolor{turquoise}{RGB}{0, 153, 153}
\definecolor{rouge}{RGB}{255, 0, 0}
\definecolor{lightvert}{RGB}{205, 234, 190}
\setitemize[0]{label=\color{lightvert}  $\bullet$}

\pagestyle{fancy}
\renewcommand{\headrulewidth}{0.2pt}
\fancyhead[L]{maths-cours.fr}
\fancyhead[R]{\thepage}
\renewcommand{\footrulewidth}{0.2pt}
\fancyfoot[C]{}

\newcolumntype{C}{>{\centering\arraybackslash}X}
\newcolumntype{s}{>{\hsize=.35\hsize\arraybackslash}X}

\setlength{\parindent}{0pt}		 
\setlength{\parskip}{3mm}
\setlength{\headheight}{1cm}

\def\ebook{ebook}
\def\book{book}
\def\web{web}
\def\type{web}

\newcommand{\vect}[1]{\overrightarrow{\,\mathstrut#1\,}}

\def\Oij{$\left(\text{O}~;~\vect{\imath},~\vect{\jmath}\right)$}
\def\Oijk{$\left(\text{O}~;~\vect{\imath},~\vect{\jmath},~\vect{k}\right)$}
\def\Ouv{$\left(\text{O}~;~\vect{u},~\vect{v}\right)$}

\hypersetup{breaklinks=true, colorlinks = true, linkcolor = OliveGreen, urlcolor = OliveGreen, citecolor = OliveGreen, pdfauthor={Didier BONNEL - https://www.maths-cours.fr} } % supprime les bordures autour des liens

\renewcommand{\arg}[0]{\text{arg}}

\everymath{\displaystyle}

%================================================================================================================================
%
% Macros - Commandes
%
%================================================================================================================================

\newcommand\meta[2]{    			% Utilisé pour créer le post HTML.
	\def\titre{titre}
	\def\url{url}
	\def\arg{#1}
	\ifx\titre\arg
		\newcommand\maintitle{#2}
		\fancyhead[L]{#2}
		{\Large\sffamily \MakeUppercase{#2}}
		\vspace{1mm}\textcolor{mcvert}{\hrule}
	\fi 
	\ifx\url\arg
		\fancyfoot[L]{\href{https://www.maths-cours.fr#2}{\black \footnotesize{https://www.maths-cours.fr#2}}}
	\fi 
}


\newcommand\TitreC[1]{    		% Titre centré
     \needspace{3\baselineskip}
     \begin{center}\textbf{#1}\end{center}
}

\newcommand\newpar{    		% paragraphe
     \par
}

\newcommand\nosp {    		% commande vide (pas d'espace)
}
\newcommand{\id}[1]{} %ignore

\newcommand\boite[2]{				% Boite simple sans titre
	\vspace{5mm}
	\setlength{\fboxrule}{0.2mm}
	\setlength{\fboxsep}{5mm}	
	\fcolorbox{#1}{#1!3}{\makebox[\linewidth-2\fboxrule-2\fboxsep]{
  		\begin{minipage}[t]{\linewidth-2\fboxrule-4\fboxsep}\setlength{\parskip}{3mm}
  			 #2
  		\end{minipage}
	}}
	\vspace{5mm}
}

\newcommand\CBox[4]{				% Boites
	\vspace{5mm}
	\setlength{\fboxrule}{0.2mm}
	\setlength{\fboxsep}{5mm}
	
	\fcolorbox{#1}{#1!3}{\makebox[\linewidth-2\fboxrule-2\fboxsep]{
		\begin{minipage}[t]{1cm}\setlength{\parskip}{3mm}
	  		\textcolor{#1}{\LARGE{#2}}    
 	 	\end{minipage}  
  		\begin{minipage}[t]{\linewidth-2\fboxrule-4\fboxsep}\setlength{\parskip}{3mm}
			\raisebox{1.2mm}{\normalsize\sffamily{\textcolor{#1}{#3}}}						
  			 #4
  		\end{minipage}
	}}
	\vspace{5mm}
}

\newcommand\cadre[3]{				% Boites convertible html
	\par
	\vspace{2mm}
	\setlength{\fboxrule}{0.1mm}
	\setlength{\fboxsep}{5mm}
	\fcolorbox{#1}{white}{\makebox[\linewidth-2\fboxrule-2\fboxsep]{
  		\begin{minipage}[t]{\linewidth-2\fboxrule-4\fboxsep}\setlength{\parskip}{3mm}
			\raisebox{-2.5mm}{\sffamily \small{\textcolor{#1}{\MakeUppercase{#2}}}}		
			\par		
  			 #3
 	 		\end{minipage}
	}}
		\vspace{2mm}
	\par
}

\newcommand\bloc[3]{				% Boites convertible html sans bordure
     \needspace{2\baselineskip}
     {\sffamily \small{\textcolor{#1}{\MakeUppercase{#2}}}}    
		\par		
  			 #3
		\par
}

\newcommand\CHelp[1]{
     \CBox{Plum}{\faInfoCircle}{À RETENIR}{#1}
}

\newcommand\CUp[1]{
     \CBox{NavyBlue}{\faThumbsOUp}{EN PRATIQUE}{#1}
}

\newcommand\CInfo[1]{
     \CBox{Sepia}{\faArrowCircleRight}{REMARQUE}{#1}
}

\newcommand\CRedac[1]{
     \CBox{PineGreen}{\faEdit}{BIEN R\'EDIGER}{#1}
}

\newcommand\CError[1]{
     \CBox{Red}{\faExclamationTriangle}{ATTENTION}{#1}
}

\newcommand\TitreExo[2]{
\needspace{4\baselineskip}
 {\sffamily\large EXERCICE #1\ (\emph{#2 points})}
\vspace{5mm}
}

\newcommand\img[2]{
          \includegraphics[width=#2\paperwidth]{\imgdir#1}
}

\newcommand\imgsvg[2]{
       \begin{center}   \includegraphics[width=#2\paperwidth]{\imgsvgdir#1} \end{center}
}


\newcommand\Lien[2]{
     \href{#1}{#2 \tiny \faExternalLink}
}
\newcommand\mcLien[2]{
     \href{https~://www.maths-cours.fr/#1}{#2 \tiny \faExternalLink}
}

\newcommand{\euro}{\eurologo{}}

%================================================================================================================================
%
% Macros - Environement
%
%================================================================================================================================

\newenvironment{tex}{ %
}
{%
}

\newenvironment{indente}{ %
	\setlength\parindent{10mm}
}

{
	\setlength\parindent{0mm}
}

\newenvironment{corrige}{%
     \needspace{3\baselineskip}
     \medskip
     \textbf{\textsc{Corrigé}}
     \medskip
}
{
}

\newenvironment{extern}{%
     \begin{center}
     }
     {
     \end{center}
}

\NewEnviron{code}{%
	\par
     \boite{gray}{\texttt{%
     \BODY
     }}
     \par
}

\newenvironment{vbloc}{% boite sans cadre empeche saut de page
     \begin{minipage}[t]{\linewidth}
     }
     {
     \end{minipage}
}
\NewEnviron{h2}{%
    \needspace{3\baselineskip}
    \vspace{0.6cm}
	\noindent \MakeUppercase{\sffamily \large \BODY}
	\vspace{1mm}\textcolor{mcgris}{\hrule}\vspace{0.4cm}
	\par
}{}

\NewEnviron{h3}{%
    \needspace{3\baselineskip}
	\vspace{5mm}
	\textsc{\BODY}
	\par
}

\NewEnviron{margeneg}{ %
\begin{addmargin}[-1cm]{0cm}
\BODY
\end{addmargin}
}

\NewEnviron{html}{%
}

\begin{document}
\meta{url}{/exercices/identites-remarquables-calcul-mental-141028/}
\meta{pid}{1705}
\meta{titre}{Identités remarquables et calcul mental}
\meta{type}{exercices}
%
\documentclass[a4paper]{article}

%================================================================================================================================
%
% Packages
%
%================================================================================================================================

\usepackage[T1]{fontenc} 	% pour caractères accentués
\usepackage[utf8]{inputenc}  % encodage utf8
\usepackage[french]{babel}	% langue : français
\usepackage{fourier}			% caractères plus lisibles
\usepackage[dvipsnames]{xcolor} % couleurs
\usepackage{fancyhdr}		% réglage header footer
\usepackage{needspace}		% empêcher sauts de page mal placés
\usepackage{graphicx}		% pour inclure des graphiques
\usepackage{enumitem,cprotect}		% personnalise les listes d'items (nécessaire pour ol, al ...)
\usepackage{hyperref}		% Liens hypertexte
\usepackage{pstricks,pst-all,pst-node,pstricks-add,pst-math,pst-plot,pst-tree,pst-eucl} % pstricks
\usepackage[a4paper,includeheadfoot,top=2cm,left=3cm, bottom=2cm,right=3cm]{geometry} % marges etc.
\usepackage{comment}			% commentaires multilignes
\usepackage{amsmath,environ} % maths (matrices, etc.)
\usepackage{amssymb,makeidx}
\usepackage{bm}				% bold maths
\usepackage{tabularx}		% tableaux
\usepackage{colortbl}		% tableaux en couleur
\usepackage{fontawesome}		% Fontawesome
\usepackage{environ}			% environment with command
\usepackage{fp}				% calculs pour ps-tricks
\usepackage{multido}			% pour ps tricks
\usepackage[np]{numprint}	% formattage nombre
\usepackage{tikz,tkz-tab} 			% package principal TikZ
\usepackage{pgfplots}   % axes
\usepackage{mathrsfs}    % cursives
\usepackage{calc}			% calcul taille boites
\usepackage[scaled=0.875]{helvet} % font sans serif
\usepackage{svg} % svg
\usepackage{scrextend} % local margin
\usepackage{scratch} %scratch
\usepackage{multicol} % colonnes
%\usepackage{infix-RPN,pst-func} % formule en notation polanaise inversée
\usepackage{listings}

%================================================================================================================================
%
% Réglages de base
%
%================================================================================================================================

\lstset{
language=Python,   % R code
literate=
{á}{{\'a}}1
{à}{{\`a}}1
{ã}{{\~a}}1
{é}{{\'e}}1
{è}{{\`e}}1
{ê}{{\^e}}1
{í}{{\'i}}1
{ó}{{\'o}}1
{õ}{{\~o}}1
{ú}{{\'u}}1
{ü}{{\"u}}1
{ç}{{\c{c}}}1
{~}{{ }}1
}


\definecolor{codegreen}{rgb}{0,0.6,0}
\definecolor{codegray}{rgb}{0.5,0.5,0.5}
\definecolor{codepurple}{rgb}{0.58,0,0.82}
\definecolor{backcolour}{rgb}{0.95,0.95,0.92}

\lstdefinestyle{mystyle}{
    backgroundcolor=\color{backcolour},   
    commentstyle=\color{codegreen},
    keywordstyle=\color{magenta},
    numberstyle=\tiny\color{codegray},
    stringstyle=\color{codepurple},
    basicstyle=\ttfamily\footnotesize,
    breakatwhitespace=false,         
    breaklines=true,                 
    captionpos=b,                    
    keepspaces=true,                 
    numbers=left,                    
xleftmargin=2em,
framexleftmargin=2em,            
    showspaces=false,                
    showstringspaces=false,
    showtabs=false,                  
    tabsize=2,
    upquote=true
}

\lstset{style=mystyle}


\lstset{style=mystyle}
\newcommand{\imgdir}{C:/laragon/www/newmc/assets/imgsvg/}
\newcommand{\imgsvgdir}{C:/laragon/www/newmc/assets/imgsvg/}

\definecolor{mcgris}{RGB}{220, 220, 220}% ancien~; pour compatibilité
\definecolor{mcbleu}{RGB}{52, 152, 219}
\definecolor{mcvert}{RGB}{125, 194, 70}
\definecolor{mcmauve}{RGB}{154, 0, 215}
\definecolor{mcorange}{RGB}{255, 96, 0}
\definecolor{mcturquoise}{RGB}{0, 153, 153}
\definecolor{mcrouge}{RGB}{255, 0, 0}
\definecolor{mclightvert}{RGB}{205, 234, 190}

\definecolor{gris}{RGB}{220, 220, 220}
\definecolor{bleu}{RGB}{52, 152, 219}
\definecolor{vert}{RGB}{125, 194, 70}
\definecolor{mauve}{RGB}{154, 0, 215}
\definecolor{orange}{RGB}{255, 96, 0}
\definecolor{turquoise}{RGB}{0, 153, 153}
\definecolor{rouge}{RGB}{255, 0, 0}
\definecolor{lightvert}{RGB}{205, 234, 190}
\setitemize[0]{label=\color{lightvert}  $\bullet$}

\pagestyle{fancy}
\renewcommand{\headrulewidth}{0.2pt}
\fancyhead[L]{maths-cours.fr}
\fancyhead[R]{\thepage}
\renewcommand{\footrulewidth}{0.2pt}
\fancyfoot[C]{}

\newcolumntype{C}{>{\centering\arraybackslash}X}
\newcolumntype{s}{>{\hsize=.35\hsize\arraybackslash}X}

\setlength{\parindent}{0pt}		 
\setlength{\parskip}{3mm}
\setlength{\headheight}{1cm}

\def\ebook{ebook}
\def\book{book}
\def\web{web}
\def\type{web}

\newcommand{\vect}[1]{\overrightarrow{\,\mathstrut#1\,}}

\def\Oij{$\left(\text{O}~;~\vect{\imath},~\vect{\jmath}\right)$}
\def\Oijk{$\left(\text{O}~;~\vect{\imath},~\vect{\jmath},~\vect{k}\right)$}
\def\Ouv{$\left(\text{O}~;~\vect{u},~\vect{v}\right)$}

\hypersetup{breaklinks=true, colorlinks = true, linkcolor = OliveGreen, urlcolor = OliveGreen, citecolor = OliveGreen, pdfauthor={Didier BONNEL - https://www.maths-cours.fr} } % supprime les bordures autour des liens

\renewcommand{\arg}[0]{\text{arg}}

\everymath{\displaystyle}

%================================================================================================================================
%
% Macros - Commandes
%
%================================================================================================================================

\newcommand\meta[2]{    			% Utilisé pour créer le post HTML.
	\def\titre{titre}
	\def\url{url}
	\def\arg{#1}
	\ifx\titre\arg
		\newcommand\maintitle{#2}
		\fancyhead[L]{#2}
		{\Large\sffamily \MakeUppercase{#2}}
		\vspace{1mm}\textcolor{mcvert}{\hrule}
	\fi 
	\ifx\url\arg
		\fancyfoot[L]{\href{https://www.maths-cours.fr#2}{\black \footnotesize{https://www.maths-cours.fr#2}}}
	\fi 
}


\newcommand\TitreC[1]{    		% Titre centré
     \needspace{3\baselineskip}
     \begin{center}\textbf{#1}\end{center}
}

\newcommand\newpar{    		% paragraphe
     \par
}

\newcommand\nosp {    		% commande vide (pas d'espace)
}
\newcommand{\id}[1]{} %ignore

\newcommand\boite[2]{				% Boite simple sans titre
	\vspace{5mm}
	\setlength{\fboxrule}{0.2mm}
	\setlength{\fboxsep}{5mm}	
	\fcolorbox{#1}{#1!3}{\makebox[\linewidth-2\fboxrule-2\fboxsep]{
  		\begin{minipage}[t]{\linewidth-2\fboxrule-4\fboxsep}\setlength{\parskip}{3mm}
  			 #2
  		\end{minipage}
	}}
	\vspace{5mm}
}

\newcommand\CBox[4]{				% Boites
	\vspace{5mm}
	\setlength{\fboxrule}{0.2mm}
	\setlength{\fboxsep}{5mm}
	
	\fcolorbox{#1}{#1!3}{\makebox[\linewidth-2\fboxrule-2\fboxsep]{
		\begin{minipage}[t]{1cm}\setlength{\parskip}{3mm}
	  		\textcolor{#1}{\LARGE{#2}}    
 	 	\end{minipage}  
  		\begin{minipage}[t]{\linewidth-2\fboxrule-4\fboxsep}\setlength{\parskip}{3mm}
			\raisebox{1.2mm}{\normalsize\sffamily{\textcolor{#1}{#3}}}						
  			 #4
  		\end{minipage}
	}}
	\vspace{5mm}
}

\newcommand\cadre[3]{				% Boites convertible html
	\par
	\vspace{2mm}
	\setlength{\fboxrule}{0.1mm}
	\setlength{\fboxsep}{5mm}
	\fcolorbox{#1}{white}{\makebox[\linewidth-2\fboxrule-2\fboxsep]{
  		\begin{minipage}[t]{\linewidth-2\fboxrule-4\fboxsep}\setlength{\parskip}{3mm}
			\raisebox{-2.5mm}{\sffamily \small{\textcolor{#1}{\MakeUppercase{#2}}}}		
			\par		
  			 #3
 	 		\end{minipage}
	}}
		\vspace{2mm}
	\par
}

\newcommand\bloc[3]{				% Boites convertible html sans bordure
     \needspace{2\baselineskip}
     {\sffamily \small{\textcolor{#1}{\MakeUppercase{#2}}}}    
		\par		
  			 #3
		\par
}

\newcommand\CHelp[1]{
     \CBox{Plum}{\faInfoCircle}{À RETENIR}{#1}
}

\newcommand\CUp[1]{
     \CBox{NavyBlue}{\faThumbsOUp}{EN PRATIQUE}{#1}
}

\newcommand\CInfo[1]{
     \CBox{Sepia}{\faArrowCircleRight}{REMARQUE}{#1}
}

\newcommand\CRedac[1]{
     \CBox{PineGreen}{\faEdit}{BIEN R\'EDIGER}{#1}
}

\newcommand\CError[1]{
     \CBox{Red}{\faExclamationTriangle}{ATTENTION}{#1}
}

\newcommand\TitreExo[2]{
\needspace{4\baselineskip}
 {\sffamily\large EXERCICE #1\ (\emph{#2 points})}
\vspace{5mm}
}

\newcommand\img[2]{
          \includegraphics[width=#2\paperwidth]{\imgdir#1}
}

\newcommand\imgsvg[2]{
       \begin{center}   \includegraphics[width=#2\paperwidth]{\imgsvgdir#1} \end{center}
}


\newcommand\Lien[2]{
     \href{#1}{#2 \tiny \faExternalLink}
}
\newcommand\mcLien[2]{
     \href{https~://www.maths-cours.fr/#1}{#2 \tiny \faExternalLink}
}

\newcommand{\euro}{\eurologo{}}

%================================================================================================================================
%
% Macros - Environement
%
%================================================================================================================================

\newenvironment{tex}{ %
}
{%
}

\newenvironment{indente}{ %
	\setlength\parindent{10mm}
}

{
	\setlength\parindent{0mm}
}

\newenvironment{corrige}{%
     \needspace{3\baselineskip}
     \medskip
     \textbf{\textsc{Corrigé}}
     \medskip
}
{
}

\newenvironment{extern}{%
     \begin{center}
     }
     {
     \end{center}
}

\NewEnviron{code}{%
	\par
     \boite{gray}{\texttt{%
     \BODY
     }}
     \par
}

\newenvironment{vbloc}{% boite sans cadre empeche saut de page
     \begin{minipage}[t]{\linewidth}
     }
     {
     \end{minipage}
}
\NewEnviron{h2}{%
    \needspace{3\baselineskip}
    \vspace{0.6cm}
	\noindent \MakeUppercase{\sffamily \large \BODY}
	\vspace{1mm}\textcolor{mcgris}{\hrule}\vspace{0.4cm}
	\par
}{}

\NewEnviron{h3}{%
    \needspace{3\baselineskip}
	\vspace{5mm}
	\textsc{\BODY}
	\par
}

\NewEnviron{margeneg}{ %
\begin{addmargin}[-1cm]{0cm}
\BODY
\end{addmargin}
}

\NewEnviron{html}{%
}

\begin{document}
Calculer mentalement :
\begin{enumerate}
     \item
     $A=501\times 499$
     \item
     $B=423^{2}-422^{2}$
     \item
     $C=2015^{2}-2005^{2}$
     \item
     $D=1002^{2}$
\end{enumerate}
\begin{corrige}
     \begin{enumerate}
          \item
          $A=501\times 499=\left(500+1\right)\left(500-1\right) $\nosp$=500^{2}-1^{2}=250 000-1=249 999$
          \item
          $B=423^{2}-422^{2}=\left(423+422\right)\times \left(423-422\right) $\nosp$=845\times 1=845$
          \item
          $C=2015^{2}-2005^{2}=\left(2015+2005\right)\times \left(2015-2005\right)=4020\times 10 $\nosp$=40 200$
          \item
          $D=1002^{2}=\left(1000+2\right)^{2}=1000^{2}+2\times 1000\times 2+2^{2} $\nosp$=1000000+4000+4=1 004 004$
     \end{enumerate}
\end{corrige}

\end{document}
\end{document}


µ
\documentclass[a4paper]{article}

%================================================================================================================================
%
% Packages
%
%================================================================================================================================

\usepackage[T1]{fontenc} 	% pour caractères accentués
\usepackage[utf8]{inputenc}  % encodage utf8
\usepackage[french]{babel}	% langue : français
\usepackage{fourier}			% caractères plus lisibles
\usepackage[dvipsnames]{xcolor} % couleurs
\usepackage{fancyhdr}		% réglage header footer
\usepackage{needspace}		% empêcher sauts de page mal placés
\usepackage{graphicx}		% pour inclure des graphiques
\usepackage{enumitem,cprotect}		% personnalise les listes d'items (nécessaire pour ol, al ...)
\usepackage{hyperref}		% Liens hypertexte
\usepackage{pstricks,pst-all,pst-node,pstricks-add,pst-math,pst-plot,pst-tree,pst-eucl} % pstricks
\usepackage[a4paper,includeheadfoot,top=2cm,left=3cm, bottom=2cm,right=3cm]{geometry} % marges etc.
\usepackage{comment}			% commentaires multilignes
\usepackage{amsmath,environ} % maths (matrices, etc.)
\usepackage{amssymb,makeidx}
\usepackage{bm}				% bold maths
\usepackage{tabularx}		% tableaux
\usepackage{colortbl}		% tableaux en couleur
\usepackage{fontawesome}		% Fontawesome
\usepackage{environ}			% environment with command
\usepackage{fp}				% calculs pour ps-tricks
\usepackage{multido}			% pour ps tricks
\usepackage[np]{numprint}	% formattage nombre
\usepackage{tikz,tkz-tab} 			% package principal TikZ
\usepackage{pgfplots}   % axes
\usepackage{mathrsfs}    % cursives
\usepackage{calc}			% calcul taille boites
\usepackage[scaled=0.875]{helvet} % font sans serif
\usepackage{svg} % svg
\usepackage{scrextend} % local margin
\usepackage{scratch} %scratch
\usepackage{multicol} % colonnes
%\usepackage{infix-RPN,pst-func} % formule en notation polanaise inversée
\usepackage{listings}

%================================================================================================================================
%
% Réglages de base
%
%================================================================================================================================

\lstset{
language=Python,   % R code
literate=
{á}{{\'a}}1
{à}{{\`a}}1
{ã}{{\~a}}1
{é}{{\'e}}1
{è}{{\`e}}1
{ê}{{\^e}}1
{í}{{\'i}}1
{ó}{{\'o}}1
{õ}{{\~o}}1
{ú}{{\'u}}1
{ü}{{\"u}}1
{ç}{{\c{c}}}1
{~}{{ }}1
}


\definecolor{codegreen}{rgb}{0,0.6,0}
\definecolor{codegray}{rgb}{0.5,0.5,0.5}
\definecolor{codepurple}{rgb}{0.58,0,0.82}
\definecolor{backcolour}{rgb}{0.95,0.95,0.92}

\lstdefinestyle{mystyle}{
    backgroundcolor=\color{backcolour},   
    commentstyle=\color{codegreen},
    keywordstyle=\color{magenta},
    numberstyle=\tiny\color{codegray},
    stringstyle=\color{codepurple},
    basicstyle=\ttfamily\footnotesize,
    breakatwhitespace=false,         
    breaklines=true,                 
    captionpos=b,                    
    keepspaces=true,                 
    numbers=left,                    
xleftmargin=2em,
framexleftmargin=2em,            
    showspaces=false,                
    showstringspaces=false,
    showtabs=false,                  
    tabsize=2,
    upquote=true
}

\lstset{style=mystyle}


\lstset{style=mystyle}
\newcommand{\imgdir}{C:/laragon/www/newmc/assets/imgsvg/}
\newcommand{\imgsvgdir}{C:/laragon/www/newmc/assets/imgsvg/}

\definecolor{mcgris}{RGB}{220, 220, 220}% ancien~; pour compatibilité
\definecolor{mcbleu}{RGB}{52, 152, 219}
\definecolor{mcvert}{RGB}{125, 194, 70}
\definecolor{mcmauve}{RGB}{154, 0, 215}
\definecolor{mcorange}{RGB}{255, 96, 0}
\definecolor{mcturquoise}{RGB}{0, 153, 153}
\definecolor{mcrouge}{RGB}{255, 0, 0}
\definecolor{mclightvert}{RGB}{205, 234, 190}

\definecolor{gris}{RGB}{220, 220, 220}
\definecolor{bleu}{RGB}{52, 152, 219}
\definecolor{vert}{RGB}{125, 194, 70}
\definecolor{mauve}{RGB}{154, 0, 215}
\definecolor{orange}{RGB}{255, 96, 0}
\definecolor{turquoise}{RGB}{0, 153, 153}
\definecolor{rouge}{RGB}{255, 0, 0}
\definecolor{lightvert}{RGB}{205, 234, 190}
\setitemize[0]{label=\color{lightvert}  $\bullet$}

\pagestyle{fancy}
\renewcommand{\headrulewidth}{0.2pt}
\fancyhead[L]{maths-cours.fr}
\fancyhead[R]{\thepage}
\renewcommand{\footrulewidth}{0.2pt}
\fancyfoot[C]{}

\newcolumntype{C}{>{\centering\arraybackslash}X}
\newcolumntype{s}{>{\hsize=.35\hsize\arraybackslash}X}

\setlength{\parindent}{0pt}		 
\setlength{\parskip}{3mm}
\setlength{\headheight}{1cm}

\def\ebook{ebook}
\def\book{book}
\def\web{web}
\def\type{web}

\newcommand{\vect}[1]{\overrightarrow{\,\mathstrut#1\,}}

\def\Oij{$\left(\text{O}~;~\vect{\imath},~\vect{\jmath}\right)$}
\def\Oijk{$\left(\text{O}~;~\vect{\imath},~\vect{\jmath},~\vect{k}\right)$}
\def\Ouv{$\left(\text{O}~;~\vect{u},~\vect{v}\right)$}

\hypersetup{breaklinks=true, colorlinks = true, linkcolor = OliveGreen, urlcolor = OliveGreen, citecolor = OliveGreen, pdfauthor={Didier BONNEL - https://www.maths-cours.fr} } % supprime les bordures autour des liens

\renewcommand{\arg}[0]{\text{arg}}

\everymath{\displaystyle}

%================================================================================================================================
%
% Macros - Commandes
%
%================================================================================================================================

\newcommand\meta[2]{    			% Utilisé pour créer le post HTML.
	\def\titre{titre}
	\def\url{url}
	\def\arg{#1}
	\ifx\titre\arg
		\newcommand\maintitle{#2}
		\fancyhead[L]{#2}
		{\Large\sffamily \MakeUppercase{#2}}
		\vspace{1mm}\textcolor{mcvert}{\hrule}
	\fi 
	\ifx\url\arg
		\fancyfoot[L]{\href{https://www.maths-cours.fr#2}{\black \footnotesize{https://www.maths-cours.fr#2}}}
	\fi 
}


\newcommand\TitreC[1]{    		% Titre centré
     \needspace{3\baselineskip}
     \begin{center}\textbf{#1}\end{center}
}

\newcommand\newpar{    		% paragraphe
     \par
}

\newcommand\nosp {    		% commande vide (pas d'espace)
}
\newcommand{\id}[1]{} %ignore

\newcommand\boite[2]{				% Boite simple sans titre
	\vspace{5mm}
	\setlength{\fboxrule}{0.2mm}
	\setlength{\fboxsep}{5mm}	
	\fcolorbox{#1}{#1!3}{\makebox[\linewidth-2\fboxrule-2\fboxsep]{
  		\begin{minipage}[t]{\linewidth-2\fboxrule-4\fboxsep}\setlength{\parskip}{3mm}
  			 #2
  		\end{minipage}
	}}
	\vspace{5mm}
}

\newcommand\CBox[4]{				% Boites
	\vspace{5mm}
	\setlength{\fboxrule}{0.2mm}
	\setlength{\fboxsep}{5mm}
	
	\fcolorbox{#1}{#1!3}{\makebox[\linewidth-2\fboxrule-2\fboxsep]{
		\begin{minipage}[t]{1cm}\setlength{\parskip}{3mm}
	  		\textcolor{#1}{\LARGE{#2}}    
 	 	\end{minipage}  
  		\begin{minipage}[t]{\linewidth-2\fboxrule-4\fboxsep}\setlength{\parskip}{3mm}
			\raisebox{1.2mm}{\normalsize\sffamily{\textcolor{#1}{#3}}}						
  			 #4
  		\end{minipage}
	}}
	\vspace{5mm}
}

\newcommand\cadre[3]{				% Boites convertible html
	\par
	\vspace{2mm}
	\setlength{\fboxrule}{0.1mm}
	\setlength{\fboxsep}{5mm}
	\fcolorbox{#1}{white}{\makebox[\linewidth-2\fboxrule-2\fboxsep]{
  		\begin{minipage}[t]{\linewidth-2\fboxrule-4\fboxsep}\setlength{\parskip}{3mm}
			\raisebox{-2.5mm}{\sffamily \small{\textcolor{#1}{\MakeUppercase{#2}}}}		
			\par		
  			 #3
 	 		\end{minipage}
	}}
		\vspace{2mm}
	\par
}

\newcommand\bloc[3]{				% Boites convertible html sans bordure
     \needspace{2\baselineskip}
     {\sffamily \small{\textcolor{#1}{\MakeUppercase{#2}}}}    
		\par		
  			 #3
		\par
}

\newcommand\CHelp[1]{
     \CBox{Plum}{\faInfoCircle}{À RETENIR}{#1}
}

\newcommand\CUp[1]{
     \CBox{NavyBlue}{\faThumbsOUp}{EN PRATIQUE}{#1}
}

\newcommand\CInfo[1]{
     \CBox{Sepia}{\faArrowCircleRight}{REMARQUE}{#1}
}

\newcommand\CRedac[1]{
     \CBox{PineGreen}{\faEdit}{BIEN R\'EDIGER}{#1}
}

\newcommand\CError[1]{
     \CBox{Red}{\faExclamationTriangle}{ATTENTION}{#1}
}

\newcommand\TitreExo[2]{
\needspace{4\baselineskip}
 {\sffamily\large EXERCICE #1\ (\emph{#2 points})}
\vspace{5mm}
}

\newcommand\img[2]{
          \includegraphics[width=#2\paperwidth]{\imgdir#1}
}

\newcommand\imgsvg[2]{
       \begin{center}   \includegraphics[width=#2\paperwidth]{\imgsvgdir#1} \end{center}
}


\newcommand\Lien[2]{
     \href{#1}{#2 \tiny \faExternalLink}
}
\newcommand\mcLien[2]{
     \href{https~://www.maths-cours.fr/#1}{#2 \tiny \faExternalLink}
}

\newcommand{\euro}{\eurologo{}}

%================================================================================================================================
%
% Macros - Environement
%
%================================================================================================================================

\newenvironment{tex}{ %
}
{%
}

\newenvironment{indente}{ %
	\setlength\parindent{10mm}
}

{
	\setlength\parindent{0mm}
}

\newenvironment{corrige}{%
     \needspace{3\baselineskip}
     \medskip
     \textbf{\textsc{Corrigé}}
     \medskip
}
{
}

\newenvironment{extern}{%
     \begin{center}
     }
     {
     \end{center}
}

\NewEnviron{code}{%
	\par
     \boite{gray}{\texttt{%
     \BODY
     }}
     \par
}

\newenvironment{vbloc}{% boite sans cadre empeche saut de page
     \begin{minipage}[t]{\linewidth}
     }
     {
     \end{minipage}
}
\NewEnviron{h2}{%
    \needspace{3\baselineskip}
    \vspace{0.6cm}
	\noindent \MakeUppercase{\sffamily \large \BODY}
	\vspace{1mm}\textcolor{mcgris}{\hrule}\vspace{0.4cm}
	\par
}{}

\NewEnviron{h3}{%
    \needspace{3\baselineskip}
	\vspace{5mm}
	\textsc{\BODY}
	\par
}

\NewEnviron{margeneg}{ %
\begin{addmargin}[-1cm]{0cm}
\BODY
\end{addmargin}
}

\NewEnviron{html}{%
}

\begin{document}
\meta{url}{/exercices/mesure-arbre-141012/}
\meta{pid}{1743}
\meta{titre}{Mesure d'un arbre (Brevet 2013)}
\meta{type}{exercices}
\textit{(D'après Brevet Polynésie 2013)}
\medskip
Teiki se promène en montagne et aimerait connaître la hauteur d'un Pinus (ou Pin des Caraïbes) situé devant lui. Pour cela, il utilise un bâton et prend quelques mesures au sol.
\par
Il procède de la façon suivante~:
\begin{itemize}
     \item Il pique le bâton en terre, verticalement, à 12 mètres du Pinus.
     \item La partie visible (hors du sol) du bâton mesure 2 m.
     \item Teiki se place derrière le bâton, de façon à ce que son œil, situé à 1,60 m au dessus du sol, voit en alignement le sommet de l'arbre et l'extrémité du bâton.
     \item Teiki marque sa position au sol, puis mesure la distance entre sa position et le bâton. Il trouve alors 1,2 m.
\end{itemize}
On peut représenter cette situation à l'aide du schéma ci-dessous~:
\begin{center}
     \img{mc-0528}{0.1}%width="400" alt="exercice théorème de Thalès
     Brevet Métropole 2018"
\end{center}
Quelle est la hauteur du Pinus au-dessus du sol~?
%
\begin{corrige}
     \begin{center}
          \img{mc-0529}{0.1}%width="400" alt="corrigé théorème de Thalès Brevet Métropole 2018"
     \end{center}
     On peut modéliser la situation à l'aide de la figure ci-dessus où $\left[AE\right]$ représente l'arbre et $\left[FG\right]$ le bâton.
     \par
     D'après les données de l'énoncé on a~:
     \begin{itemize}
          \item $EF=BH=12$m
          \item $GF=2$m
          \item $HF=CD=1,6$m
          \item $FD=HC=1,2$m
     \end{itemize}
     On cherche à calculer la hauteur de l'arbre c'est à dire la longueur $AE$.
     \par
     Les points $F, H$ et $G$ étant alignés~:
     \par
     $GH=GF-HF=2-1,6=0,4$
     \par
     Les points $E, F$ et $D$ étant alignés~:
     \par
     $ED=EF+FD=12+1,2=13,2$ et par conséquent $BC=13,2$
     \par
     Les droites $\left(AE\right)$ et $\left(GF\right)$, étant toutes deux verticales, sont parallèles~; donc d'après le \mcLien{https://www.maths-cours.fr/cours/theoreme-thales/\#d10}{théorème de Thalès}~:
     \par
     $\frac{GC}{AC}=\frac{HC}{BC}=\frac{GH}{AB}$
     \par
     $\frac{GC}{AC}=\frac{1,2}{13,2}=\frac{0,4}{AB}$
     \par
     De l'égalité des rapports $\frac{1,2}{13,2}=\frac{0,4}{AB}$ on déduit~:
     \par
     $AB=\frac{0,4\times 13,2}{1,2}=4,4$
     \par
     La hauteur totale de l'arbre est donc~:
     \par
     $AE=AB+BE=4,4+1,6=6$m
     \par
     La hauteur du Pinus au-dessus du sol est $6$ mètres.
\end{corrige}

\end{document}
µ
\documentclass[a4paper]{article}

%================================================================================================================================
%
% Packages
%
%================================================================================================================================

\usepackage[T1]{fontenc} 	% pour caractères accentués
\usepackage[utf8]{inputenc}  % encodage utf8
\usepackage[french]{babel}	% langue : français
\usepackage{fourier}			% caractères plus lisibles
\usepackage[dvipsnames]{xcolor} % couleurs
\usepackage{fancyhdr}		% réglage header footer
\usepackage{needspace}		% empêcher sauts de page mal placés
\usepackage{graphicx}		% pour inclure des graphiques
\usepackage{enumitem,cprotect}		% personnalise les listes d'items (nécessaire pour ol, al ...)
\usepackage{hyperref}		% Liens hypertexte
\usepackage{pstricks,pst-all,pst-node,pstricks-add,pst-math,pst-plot,pst-tree,pst-eucl} % pstricks
\usepackage[a4paper,includeheadfoot,top=2cm,left=3cm, bottom=2cm,right=3cm]{geometry} % marges etc.
\usepackage{comment}			% commentaires multilignes
\usepackage{amsmath,environ} % maths (matrices, etc.)
\usepackage{amssymb,makeidx}
\usepackage{bm}				% bold maths
\usepackage{tabularx}		% tableaux
\usepackage{colortbl}		% tableaux en couleur
\usepackage{fontawesome}		% Fontawesome
\usepackage{environ}			% environment with command
\usepackage{fp}				% calculs pour ps-tricks
\usepackage{multido}			% pour ps tricks
\usepackage[np]{numprint}	% formattage nombre
\usepackage{tikz,tkz-tab} 			% package principal TikZ
\usepackage{pgfplots}   % axes
\usepackage{mathrsfs}    % cursives
\usepackage{calc}			% calcul taille boites
\usepackage[scaled=0.875]{helvet} % font sans serif
\usepackage{svg} % svg
\usepackage{scrextend} % local margin
\usepackage{scratch} %scratch
\usepackage{multicol} % colonnes
%\usepackage{infix-RPN,pst-func} % formule en notation polanaise inversée
\usepackage{listings}

%================================================================================================================================
%
% Réglages de base
%
%================================================================================================================================

\lstset{
language=Python,   % R code
literate=
{á}{{\'a}}1
{à}{{\`a}}1
{ã}{{\~a}}1
{é}{{\'e}}1
{è}{{\`e}}1
{ê}{{\^e}}1
{í}{{\'i}}1
{ó}{{\'o}}1
{õ}{{\~o}}1
{ú}{{\'u}}1
{ü}{{\"u}}1
{ç}{{\c{c}}}1
{~}{{ }}1
}


\definecolor{codegreen}{rgb}{0,0.6,0}
\definecolor{codegray}{rgb}{0.5,0.5,0.5}
\definecolor{codepurple}{rgb}{0.58,0,0.82}
\definecolor{backcolour}{rgb}{0.95,0.95,0.92}

\lstdefinestyle{mystyle}{
    backgroundcolor=\color{backcolour},   
    commentstyle=\color{codegreen},
    keywordstyle=\color{magenta},
    numberstyle=\tiny\color{codegray},
    stringstyle=\color{codepurple},
    basicstyle=\ttfamily\footnotesize,
    breakatwhitespace=false,         
    breaklines=true,                 
    captionpos=b,                    
    keepspaces=true,                 
    numbers=left,                    
xleftmargin=2em,
framexleftmargin=2em,            
    showspaces=false,                
    showstringspaces=false,
    showtabs=false,                  
    tabsize=2,
    upquote=true
}

\lstset{style=mystyle}


\lstset{style=mystyle}
\newcommand{\imgdir}{C:/laragon/www/newmc/assets/imgsvg/}
\newcommand{\imgsvgdir}{C:/laragon/www/newmc/assets/imgsvg/}

\definecolor{mcgris}{RGB}{220, 220, 220}% ancien~; pour compatibilité
\definecolor{mcbleu}{RGB}{52, 152, 219}
\definecolor{mcvert}{RGB}{125, 194, 70}
\definecolor{mcmauve}{RGB}{154, 0, 215}
\definecolor{mcorange}{RGB}{255, 96, 0}
\definecolor{mcturquoise}{RGB}{0, 153, 153}
\definecolor{mcrouge}{RGB}{255, 0, 0}
\definecolor{mclightvert}{RGB}{205, 234, 190}

\definecolor{gris}{RGB}{220, 220, 220}
\definecolor{bleu}{RGB}{52, 152, 219}
\definecolor{vert}{RGB}{125, 194, 70}
\definecolor{mauve}{RGB}{154, 0, 215}
\definecolor{orange}{RGB}{255, 96, 0}
\definecolor{turquoise}{RGB}{0, 153, 153}
\definecolor{rouge}{RGB}{255, 0, 0}
\definecolor{lightvert}{RGB}{205, 234, 190}
\setitemize[0]{label=\color{lightvert}  $\bullet$}

\pagestyle{fancy}
\renewcommand{\headrulewidth}{0.2pt}
\fancyhead[L]{maths-cours.fr}
\fancyhead[R]{\thepage}
\renewcommand{\footrulewidth}{0.2pt}
\fancyfoot[C]{}

\newcolumntype{C}{>{\centering\arraybackslash}X}
\newcolumntype{s}{>{\hsize=.35\hsize\arraybackslash}X}

\setlength{\parindent}{0pt}		 
\setlength{\parskip}{3mm}
\setlength{\headheight}{1cm}

\def\ebook{ebook}
\def\book{book}
\def\web{web}
\def\type{web}

\newcommand{\vect}[1]{\overrightarrow{\,\mathstrut#1\,}}

\def\Oij{$\left(\text{O}~;~\vect{\imath},~\vect{\jmath}\right)$}
\def\Oijk{$\left(\text{O}~;~\vect{\imath},~\vect{\jmath},~\vect{k}\right)$}
\def\Ouv{$\left(\text{O}~;~\vect{u},~\vect{v}\right)$}

\hypersetup{breaklinks=true, colorlinks = true, linkcolor = OliveGreen, urlcolor = OliveGreen, citecolor = OliveGreen, pdfauthor={Didier BONNEL - https://www.maths-cours.fr} } % supprime les bordures autour des liens

\renewcommand{\arg}[0]{\text{arg}}

\everymath{\displaystyle}

%================================================================================================================================
%
% Macros - Commandes
%
%================================================================================================================================

\newcommand\meta[2]{    			% Utilisé pour créer le post HTML.
	\def\titre{titre}
	\def\url{url}
	\def\arg{#1}
	\ifx\titre\arg
		\newcommand\maintitle{#2}
		\fancyhead[L]{#2}
		{\Large\sffamily \MakeUppercase{#2}}
		\vspace{1mm}\textcolor{mcvert}{\hrule}
	\fi 
	\ifx\url\arg
		\fancyfoot[L]{\href{https://www.maths-cours.fr#2}{\black \footnotesize{https://www.maths-cours.fr#2}}}
	\fi 
}


\newcommand\TitreC[1]{    		% Titre centré
     \needspace{3\baselineskip}
     \begin{center}\textbf{#1}\end{center}
}

\newcommand\newpar{    		% paragraphe
     \par
}

\newcommand\nosp {    		% commande vide (pas d'espace)
}
\newcommand{\id}[1]{} %ignore

\newcommand\boite[2]{				% Boite simple sans titre
	\vspace{5mm}
	\setlength{\fboxrule}{0.2mm}
	\setlength{\fboxsep}{5mm}	
	\fcolorbox{#1}{#1!3}{\makebox[\linewidth-2\fboxrule-2\fboxsep]{
  		\begin{minipage}[t]{\linewidth-2\fboxrule-4\fboxsep}\setlength{\parskip}{3mm}
  			 #2
  		\end{minipage}
	}}
	\vspace{5mm}
}

\newcommand\CBox[4]{				% Boites
	\vspace{5mm}
	\setlength{\fboxrule}{0.2mm}
	\setlength{\fboxsep}{5mm}
	
	\fcolorbox{#1}{#1!3}{\makebox[\linewidth-2\fboxrule-2\fboxsep]{
		\begin{minipage}[t]{1cm}\setlength{\parskip}{3mm}
	  		\textcolor{#1}{\LARGE{#2}}    
 	 	\end{minipage}  
  		\begin{minipage}[t]{\linewidth-2\fboxrule-4\fboxsep}\setlength{\parskip}{3mm}
			\raisebox{1.2mm}{\normalsize\sffamily{\textcolor{#1}{#3}}}						
  			 #4
  		\end{minipage}
	}}
	\vspace{5mm}
}

\newcommand\cadre[3]{				% Boites convertible html
	\par
	\vspace{2mm}
	\setlength{\fboxrule}{0.1mm}
	\setlength{\fboxsep}{5mm}
	\fcolorbox{#1}{white}{\makebox[\linewidth-2\fboxrule-2\fboxsep]{
  		\begin{minipage}[t]{\linewidth-2\fboxrule-4\fboxsep}\setlength{\parskip}{3mm}
			\raisebox{-2.5mm}{\sffamily \small{\textcolor{#1}{\MakeUppercase{#2}}}}		
			\par		
  			 #3
 	 		\end{minipage}
	}}
		\vspace{2mm}
	\par
}

\newcommand\bloc[3]{				% Boites convertible html sans bordure
     \needspace{2\baselineskip}
     {\sffamily \small{\textcolor{#1}{\MakeUppercase{#2}}}}    
		\par		
  			 #3
		\par
}

\newcommand\CHelp[1]{
     \CBox{Plum}{\faInfoCircle}{À RETENIR}{#1}
}

\newcommand\CUp[1]{
     \CBox{NavyBlue}{\faThumbsOUp}{EN PRATIQUE}{#1}
}

\newcommand\CInfo[1]{
     \CBox{Sepia}{\faArrowCircleRight}{REMARQUE}{#1}
}

\newcommand\CRedac[1]{
     \CBox{PineGreen}{\faEdit}{BIEN R\'EDIGER}{#1}
}

\newcommand\CError[1]{
     \CBox{Red}{\faExclamationTriangle}{ATTENTION}{#1}
}

\newcommand\TitreExo[2]{
\needspace{4\baselineskip}
 {\sffamily\large EXERCICE #1\ (\emph{#2 points})}
\vspace{5mm}
}

\newcommand\img[2]{
          \includegraphics[width=#2\paperwidth]{\imgdir#1}
}

\newcommand\imgsvg[2]{
       \begin{center}   \includegraphics[width=#2\paperwidth]{\imgsvgdir#1} \end{center}
}


\newcommand\Lien[2]{
     \href{#1}{#2 \tiny \faExternalLink}
}
\newcommand\mcLien[2]{
     \href{https~://www.maths-cours.fr/#1}{#2 \tiny \faExternalLink}
}

\newcommand{\euro}{\eurologo{}}

%================================================================================================================================
%
% Macros - Environement
%
%================================================================================================================================

\newenvironment{tex}{ %
}
{%
}

\newenvironment{indente}{ %
	\setlength\parindent{10mm}
}

{
	\setlength\parindent{0mm}
}

\newenvironment{corrige}{%
     \needspace{3\baselineskip}
     \medskip
     \textbf{\textsc{Corrigé}}
     \medskip
}
{
}

\newenvironment{extern}{%
     \begin{center}
     }
     {
     \end{center}
}

\NewEnviron{code}{%
	\par
     \boite{gray}{\texttt{%
     \BODY
     }}
     \par
}

\newenvironment{vbloc}{% boite sans cadre empeche saut de page
     \begin{minipage}[t]{\linewidth}
     }
     {
     \end{minipage}
}
\NewEnviron{h2}{%
    \needspace{3\baselineskip}
    \vspace{0.6cm}
	\noindent \MakeUppercase{\sffamily \large \BODY}
	\vspace{1mm}\textcolor{mcgris}{\hrule}\vspace{0.4cm}
	\par
}{}

\NewEnviron{h3}{%
    \needspace{3\baselineskip}
	\vspace{5mm}
	\textsc{\BODY}
	\par
}

\NewEnviron{margeneg}{ %
\begin{addmargin}[-1cm]{0cm}
\BODY
\end{addmargin}
}

\NewEnviron{html}{%
}

\begin{document}
\meta{url}{/exercices/thales-calcul-longueurs-141012/}
\meta{pid}{1745}
\meta{titre}{Th. de Thalès (Brevet 2013)}
\meta{type}{exercices}
\textit{(D'après Brevet Centres étrangers 2013)}
\par
\textit{Dans cet exercice, toute trace de recherche, même incomplète, sera prise en compte dans l'évaluation.}
\medskip
On considère la figure ci-dessous,  \textbf{ qui n'est pas en vraie grandeur.}
\begin{center}
     \begin{extern}%width="550" alt="Courbe représentative de f"
          \psset{xunit=1.0cm,yunit=1.0cm,algebraic=true,dimen=middle,dotstyle=o,dotsize=5pt 0,linewidth=1.6pt,arrowsize=3pt 2,arrowinset=0.25}
          \newrgbcolor{tttttt}{0.2 0.2 0.2}
          \psset{xunit=1.0cm,yunit=1.0cm,algebraic=true,dimen=middle,dotstyle=o,dotsize=5pt 0,linewidth=1.6pt,arrowsize=3pt 2,arrowinset=0.25}
          \begin{pspicture*}(0.,0.)(11.,8.)
               \psline[linewidth=0.4pt,linecolor=tttttt](1.,7.)(10.,7.)
               \psline[linewidth=0.4pt,linecolor=tttttt](10.,7.)(10.,1.)
               \psline[linewidth=0.4pt,linecolor=tttttt](10.,1.)(4.,1.)
               \psline[linewidth=0.4pt,linecolor=tttttt](4.,1.)(4.,7.)
               \psline[linewidth=0.4pt,linecolor=tttttt](1.,7.)(10.,3.)
               \begin{scriptsize}
                    \psdots[dotsize=2pt 0,dotstyle=*,linecolor=tttttt](1.,7.)
                    \rput[bl](0.7560766474115809,7.1187630144416305){\tttttt{$A$}}
                    \psdots[dotsize=2pt 0,dotstyle=*,linecolor=tttttt](4.,7.)
                    \rput[bl](4.004426603925136,7.1187630144416305){\tttttt{$B$}}
                    \psdots[dotsize=2pt 0,dotstyle=*,linecolor=tttttt](10.,7.)
                    \rput[bl](10.045865348804515,7.044936879066323){\tttttt{$C$}}
                    \psdots[dotsize=2pt 0,dotstyle=*,linecolor=tttttt](10.,1.)
                    \rput[bl](10.070474060596284,0.9542807106034092){\tttttt{$D$}}
                    \psdots[dotsize=2pt 0,dotstyle=*,linecolor=tttttt](4.,1.)
                    \rput[bl](3.684513350632135,1.0281068459787173){\tttttt{$E$}}
                    \psdots[dotsize=2pt 0,dotstyle=*,linecolor=tttttt](10.,3.)
                    \rput[bl](10.095082772388054,2.9475863657367265){\tttttt{$F$}}
                    \psdots[dotsize=2pt 0,dotstyle=*,linecolor=darkgray](4.,5.666666666666667)
                    \rput[bl](4.053644027508676,5.716066442310778){\darkgray{$M$}}
               \end{scriptsize}
          \end{pspicture*}
     \end{extern}
\end{center}
\par
$BCDE$ est un carré de $6$ cm de côté.
\par
Les points $A$, $B$ et $C$ sont alignés et $AB=3$cm.
\par
$F$ est un point du segment $\left[CD\right]$.
\par
La droite $\left(AF\right)$ coupe le segment $\left[BE\right]$ en $M$.
\par
Déterminer la longueur $CF$ pour que les longueurs $BM$ et $FD$ soient égales.
\begin{corrige}
     Notons $x=CF$.
     \par
     Comme les points $C, F$ et $D$ sont alignés :
     \par
     $FD=CD-CF=6-x     \qquad     $\textbf{(1)}
     \par
     Comme $BCDE$ est un carré, les droites $\left(BE\right)$ et $\left(CD\right)$ sont parallèles.
     \par
     D'après le théorème de Thalès :
     \par
     $\frac{AB}{AC}=\frac{AM}{AF}=\frac{BM}{CF}$
     \par
     $\frac{3}{9}=\frac{AM}{AF}=\frac{BM}{x}$
     \medskip
     De l'égalité $\frac{3}{9}=\frac{BM}{x}$, on déduit :
     \par
     $BM=\frac{3x}{9}=\frac{x}{3}  \qquad     $\textbf{(2)}
     \medskip
     En utilisant les égalités \textbf{(1)} et \textbf{(2)}, on peut dire que les longueurs $FD$ et $BM$ sont donc égales lorsque :
     \par
     $6-x=\frac{x}{3}$
     \par
     $6=\frac{x}{3}+x$
     \par
     $6=\frac{4}{3}x$
     \par
     $\frac{4}{3}x=6$
     \par
     $x=6\times \frac{3}{4}$
     \par
     $x=4,5$
     \medskip
     Les longueurs $BM$ et $FD$ sont donc égales lorsque  $CF$ vaut $4,5$cm.
\end{corrige}

\end{document}
µ
\documentclass[a4paper]{article}

%================================================================================================================================
%
% Packages
%
%================================================================================================================================

\usepackage[T1]{fontenc} 	% pour caractères accentués
\usepackage[utf8]{inputenc}  % encodage utf8
\usepackage[french]{babel}	% langue : français
\usepackage{fourier}			% caractères plus lisibles
\usepackage[dvipsnames]{xcolor} % couleurs
\usepackage{fancyhdr}		% réglage header footer
\usepackage{needspace}		% empêcher sauts de page mal placés
\usepackage{graphicx}		% pour inclure des graphiques
\usepackage{enumitem,cprotect}		% personnalise les listes d'items (nécessaire pour ol, al ...)
\usepackage{hyperref}		% Liens hypertexte
\usepackage{pstricks,pst-all,pst-node,pstricks-add,pst-math,pst-plot,pst-tree,pst-eucl} % pstricks
\usepackage[a4paper,includeheadfoot,top=2cm,left=3cm, bottom=2cm,right=3cm]{geometry} % marges etc.
\usepackage{comment}			% commentaires multilignes
\usepackage{amsmath,environ} % maths (matrices, etc.)
\usepackage{amssymb,makeidx}
\usepackage{bm}				% bold maths
\usepackage{tabularx}		% tableaux
\usepackage{colortbl}		% tableaux en couleur
\usepackage{fontawesome}		% Fontawesome
\usepackage{environ}			% environment with command
\usepackage{fp}				% calculs pour ps-tricks
\usepackage{multido}			% pour ps tricks
\usepackage[np]{numprint}	% formattage nombre
\usepackage{tikz,tkz-tab} 			% package principal TikZ
\usepackage{pgfplots}   % axes
\usepackage{mathrsfs}    % cursives
\usepackage{calc}			% calcul taille boites
\usepackage[scaled=0.875]{helvet} % font sans serif
\usepackage{svg} % svg
\usepackage{scrextend} % local margin
\usepackage{scratch} %scratch
\usepackage{multicol} % colonnes
%\usepackage{infix-RPN,pst-func} % formule en notation polanaise inversée
\usepackage{listings}

%================================================================================================================================
%
% Réglages de base
%
%================================================================================================================================

\lstset{
language=Python,   % R code
literate=
{á}{{\'a}}1
{à}{{\`a}}1
{ã}{{\~a}}1
{é}{{\'e}}1
{è}{{\`e}}1
{ê}{{\^e}}1
{í}{{\'i}}1
{ó}{{\'o}}1
{õ}{{\~o}}1
{ú}{{\'u}}1
{ü}{{\"u}}1
{ç}{{\c{c}}}1
{~}{{ }}1
}


\definecolor{codegreen}{rgb}{0,0.6,0}
\definecolor{codegray}{rgb}{0.5,0.5,0.5}
\definecolor{codepurple}{rgb}{0.58,0,0.82}
\definecolor{backcolour}{rgb}{0.95,0.95,0.92}

\lstdefinestyle{mystyle}{
    backgroundcolor=\color{backcolour},   
    commentstyle=\color{codegreen},
    keywordstyle=\color{magenta},
    numberstyle=\tiny\color{codegray},
    stringstyle=\color{codepurple},
    basicstyle=\ttfamily\footnotesize,
    breakatwhitespace=false,         
    breaklines=true,                 
    captionpos=b,                    
    keepspaces=true,                 
    numbers=left,                    
xleftmargin=2em,
framexleftmargin=2em,            
    showspaces=false,                
    showstringspaces=false,
    showtabs=false,                  
    tabsize=2,
    upquote=true
}

\lstset{style=mystyle}


\lstset{style=mystyle}
\newcommand{\imgdir}{C:/laragon/www/newmc/assets/imgsvg/}
\newcommand{\imgsvgdir}{C:/laragon/www/newmc/assets/imgsvg/}

\definecolor{mcgris}{RGB}{220, 220, 220}% ancien~; pour compatibilité
\definecolor{mcbleu}{RGB}{52, 152, 219}
\definecolor{mcvert}{RGB}{125, 194, 70}
\definecolor{mcmauve}{RGB}{154, 0, 215}
\definecolor{mcorange}{RGB}{255, 96, 0}
\definecolor{mcturquoise}{RGB}{0, 153, 153}
\definecolor{mcrouge}{RGB}{255, 0, 0}
\definecolor{mclightvert}{RGB}{205, 234, 190}

\definecolor{gris}{RGB}{220, 220, 220}
\definecolor{bleu}{RGB}{52, 152, 219}
\definecolor{vert}{RGB}{125, 194, 70}
\definecolor{mauve}{RGB}{154, 0, 215}
\definecolor{orange}{RGB}{255, 96, 0}
\definecolor{turquoise}{RGB}{0, 153, 153}
\definecolor{rouge}{RGB}{255, 0, 0}
\definecolor{lightvert}{RGB}{205, 234, 190}
\setitemize[0]{label=\color{lightvert}  $\bullet$}

\pagestyle{fancy}
\renewcommand{\headrulewidth}{0.2pt}
\fancyhead[L]{maths-cours.fr}
\fancyhead[R]{\thepage}
\renewcommand{\footrulewidth}{0.2pt}
\fancyfoot[C]{}

\newcolumntype{C}{>{\centering\arraybackslash}X}
\newcolumntype{s}{>{\hsize=.35\hsize\arraybackslash}X}

\setlength{\parindent}{0pt}		 
\setlength{\parskip}{3mm}
\setlength{\headheight}{1cm}

\def\ebook{ebook}
\def\book{book}
\def\web{web}
\def\type{web}

\newcommand{\vect}[1]{\overrightarrow{\,\mathstrut#1\,}}

\def\Oij{$\left(\text{O}~;~\vect{\imath},~\vect{\jmath}\right)$}
\def\Oijk{$\left(\text{O}~;~\vect{\imath},~\vect{\jmath},~\vect{k}\right)$}
\def\Ouv{$\left(\text{O}~;~\vect{u},~\vect{v}\right)$}

\hypersetup{breaklinks=true, colorlinks = true, linkcolor = OliveGreen, urlcolor = OliveGreen, citecolor = OliveGreen, pdfauthor={Didier BONNEL - https://www.maths-cours.fr} } % supprime les bordures autour des liens

\renewcommand{\arg}[0]{\text{arg}}

\everymath{\displaystyle}

%================================================================================================================================
%
% Macros - Commandes
%
%================================================================================================================================

\newcommand\meta[2]{    			% Utilisé pour créer le post HTML.
	\def\titre{titre}
	\def\url{url}
	\def\arg{#1}
	\ifx\titre\arg
		\newcommand\maintitle{#2}
		\fancyhead[L]{#2}
		{\Large\sffamily \MakeUppercase{#2}}
		\vspace{1mm}\textcolor{mcvert}{\hrule}
	\fi 
	\ifx\url\arg
		\fancyfoot[L]{\href{https://www.maths-cours.fr#2}{\black \footnotesize{https://www.maths-cours.fr#2}}}
	\fi 
}


\newcommand\TitreC[1]{    		% Titre centré
     \needspace{3\baselineskip}
     \begin{center}\textbf{#1}\end{center}
}

\newcommand\newpar{    		% paragraphe
     \par
}

\newcommand\nosp {    		% commande vide (pas d'espace)
}
\newcommand{\id}[1]{} %ignore

\newcommand\boite[2]{				% Boite simple sans titre
	\vspace{5mm}
	\setlength{\fboxrule}{0.2mm}
	\setlength{\fboxsep}{5mm}	
	\fcolorbox{#1}{#1!3}{\makebox[\linewidth-2\fboxrule-2\fboxsep]{
  		\begin{minipage}[t]{\linewidth-2\fboxrule-4\fboxsep}\setlength{\parskip}{3mm}
  			 #2
  		\end{minipage}
	}}
	\vspace{5mm}
}

\newcommand\CBox[4]{				% Boites
	\vspace{5mm}
	\setlength{\fboxrule}{0.2mm}
	\setlength{\fboxsep}{5mm}
	
	\fcolorbox{#1}{#1!3}{\makebox[\linewidth-2\fboxrule-2\fboxsep]{
		\begin{minipage}[t]{1cm}\setlength{\parskip}{3mm}
	  		\textcolor{#1}{\LARGE{#2}}    
 	 	\end{minipage}  
  		\begin{minipage}[t]{\linewidth-2\fboxrule-4\fboxsep}\setlength{\parskip}{3mm}
			\raisebox{1.2mm}{\normalsize\sffamily{\textcolor{#1}{#3}}}						
  			 #4
  		\end{minipage}
	}}
	\vspace{5mm}
}

\newcommand\cadre[3]{				% Boites convertible html
	\par
	\vspace{2mm}
	\setlength{\fboxrule}{0.1mm}
	\setlength{\fboxsep}{5mm}
	\fcolorbox{#1}{white}{\makebox[\linewidth-2\fboxrule-2\fboxsep]{
  		\begin{minipage}[t]{\linewidth-2\fboxrule-4\fboxsep}\setlength{\parskip}{3mm}
			\raisebox{-2.5mm}{\sffamily \small{\textcolor{#1}{\MakeUppercase{#2}}}}		
			\par		
  			 #3
 	 		\end{minipage}
	}}
		\vspace{2mm}
	\par
}

\newcommand\bloc[3]{				% Boites convertible html sans bordure
     \needspace{2\baselineskip}
     {\sffamily \small{\textcolor{#1}{\MakeUppercase{#2}}}}    
		\par		
  			 #3
		\par
}

\newcommand\CHelp[1]{
     \CBox{Plum}{\faInfoCircle}{À RETENIR}{#1}
}

\newcommand\CUp[1]{
     \CBox{NavyBlue}{\faThumbsOUp}{EN PRATIQUE}{#1}
}

\newcommand\CInfo[1]{
     \CBox{Sepia}{\faArrowCircleRight}{REMARQUE}{#1}
}

\newcommand\CRedac[1]{
     \CBox{PineGreen}{\faEdit}{BIEN R\'EDIGER}{#1}
}

\newcommand\CError[1]{
     \CBox{Red}{\faExclamationTriangle}{ATTENTION}{#1}
}

\newcommand\TitreExo[2]{
\needspace{4\baselineskip}
 {\sffamily\large EXERCICE #1\ (\emph{#2 points})}
\vspace{5mm}
}

\newcommand\img[2]{
          \includegraphics[width=#2\paperwidth]{\imgdir#1}
}

\newcommand\imgsvg[2]{
       \begin{center}   \includegraphics[width=#2\paperwidth]{\imgsvgdir#1} \end{center}
}


\newcommand\Lien[2]{
     \href{#1}{#2 \tiny \faExternalLink}
}
\newcommand\mcLien[2]{
     \href{https~://www.maths-cours.fr/#1}{#2 \tiny \faExternalLink}
}

\newcommand{\euro}{\eurologo{}}

%================================================================================================================================
%
% Macros - Environement
%
%================================================================================================================================

\newenvironment{tex}{ %
}
{%
}

\newenvironment{indente}{ %
	\setlength\parindent{10mm}
}

{
	\setlength\parindent{0mm}
}

\newenvironment{corrige}{%
     \needspace{3\baselineskip}
     \medskip
     \textbf{\textsc{Corrigé}}
     \medskip
}
{
}

\newenvironment{extern}{%
     \begin{center}
     }
     {
     \end{center}
}

\NewEnviron{code}{%
	\par
     \boite{gray}{\texttt{%
     \BODY
     }}
     \par
}

\newenvironment{vbloc}{% boite sans cadre empeche saut de page
     \begin{minipage}[t]{\linewidth}
     }
     {
     \end{minipage}
}
\NewEnviron{h2}{%
    \needspace{3\baselineskip}
    \vspace{0.6cm}
	\noindent \MakeUppercase{\sffamily \large \BODY}
	\vspace{1mm}\textcolor{mcgris}{\hrule}\vspace{0.4cm}
	\par
}{}

\NewEnviron{h3}{%
    \needspace{3\baselineskip}
	\vspace{5mm}
	\textsc{\BODY}
	\par
}

\NewEnviron{margeneg}{ %
\begin{addmargin}[-1cm]{0cm}
\BODY
\end{addmargin}
}

\NewEnviron{html}{%
}

\begin{document}
\meta{url}{/exercices/reciproque-thales-brevet-141010/}
\meta{pid}{1747}
\meta{titre}{Réciproque du théorème de Thalès (Brevet 2013)}
\meta{type}{exercices}
\textit{(D'après Brevet Pondichéry 2013)}
\par
On considère la figure ci-dessous~:
\begin{center}
     \begin{extern}
          \begin{pspicture*}(2.,-1.)(12.,6.)
               \psline[linewidth=0.5pt](4.,5.)(8.,5.)
               \psline[linewidth=0.5pt](11.,0.)(5.,0.)
               \psline[linewidth=0.5pt](8.,5.)(11.,0.)
               \psline[linewidth=0.5pt](5.,0.)(4.,5.)
               \psline[linewidth=0.5pt](4.,5.)(11.,0.)
               \psline[linewidth=0.5pt](5.,0.)(8.,5.)
               \begin{scriptsize}
                    \fontsize{13pt}{13pt}\selectfont
                    \psdots[dotsize=2pt 0,dotstyle=*](4.,5.)
                    \rput[bl](3.6,5.05){$A$}
                    \psdots[dotsize=2pt 0,dotstyle=*](8.,5.)
                    \rput[bl](8.04,5.05){$B$}
                    \psdots[dotsize=2pt 0,dotstyle=*](11.,0.)
                    \rput[bl](11.11,-0.08){$C$}
                    \psdots[dotsize=2pt 0,dotstyle=*](5.,0.)
                    \rput[bl](4.4,-0.10){$D$}
                    \psdots[dotsize=2pt 0,dotstyle=*](6.8,3.)
                    \rput[bl](6.98,2.92){$O$}
               \end{scriptsize}
          \end{pspicture*}
     \end{extern}
\end{center}
\begin{enumerate}
     \item %
     On donne~:
     \par
     $OA=2,8$cm\\
     $OB=2$cm\\
     $OC=5$cm\\
     $OD=3,5$cm.
     \par
     Les droites $\left(AB\right)$ et $\left(CD\right)$ sont-elles parallèles~?
     \item %
     On donne~:
     \par
     $OA=4$cm\\
     $OB=2,8$cm\\
     $OC=6$cm\\
     $OD=4,2$cm.
     \par
     Les droites $\left(AB\right)$ et $\left(CD\right)$ sont-elles parallèles~?
\end{enumerate}
\begin{corrige}
     \cadre{bleu}{Méthode}{
          Pour savoir si les droites $\left(AB\right)$ et $\left(CD\right)$ sont parallèles, on calcule séparément les rapports $\dfrac{OA}{OC}$ et $\dfrac{OB}{OD}$.
          \par
          Si ces deux rapports sont égaux, les droites $\left(AB\right)$ et $\left(CD\right)$ sont parallèles d'après \mcLien{/cours/theoreme-thales\#t50}{la réciproque du théorème de Thalès}. Sinon, les droites $\left(AB\right)$ et $\left(CD\right)$ ne sont pas parallèles.
     } % fin methode
     \begin{enumerate}
          \item %
          Pour la question \textbf{1.}~:
          \par
          $\dfrac{OA}{OC}=\dfrac{2,8}{5}=0,56$
          \par
          $\dfrac{OB}{OD}=\dfrac{2}{3,5}=\dfrac{4}{7} \approx 0,571$
          \par
          $\dfrac{OA}{OC} \neq \dfrac{OB}{OD}$ donc les droites $\left(AB\right)$ et $\left(CD\right)$ ne sont pas parallèles.
          \item %
          Pour la question \textbf{2.}~:
          \par
          $\dfrac{OA}{OC}=\dfrac{4}{6}=\dfrac{2}{3}$
          \par
          $\dfrac{OB}{OD}=\dfrac{2,8}{4,2}=\dfrac{28}{42}=\dfrac{2}{3}$
          \par
          $\dfrac{OA}{OC} = \dfrac{OB}{OD}$ donc les droites $\left(AB\right)$ et $\left(CD\right)$ sont parallèles d'après la réciproque du théorème de Thalès.
          \bloc{cyan}{Remarque}{ % id="r10"
               \textbf{Attention~: }Ne pas calculer de valeur approchée (par exemple $0,67$) pour cette question~! On veut montrer que les rapports sont \textbf{exactement} égaux (et pas seulement qu'ils sont à peu près égaux).
          } % fin rem
     \end{enumerate}
\end{corrige}

\end{document}
µ
\documentclass[a4paper]{article}

%================================================================================================================================
%
% Packages
%
%================================================================================================================================

\usepackage[T1]{fontenc} 	% pour caractères accentués
\usepackage[utf8]{inputenc}  % encodage utf8
\usepackage[french]{babel}	% langue : français
\usepackage{fourier}			% caractères plus lisibles
\usepackage[dvipsnames]{xcolor} % couleurs
\usepackage{fancyhdr}		% réglage header footer
\usepackage{needspace}		% empêcher sauts de page mal placés
\usepackage{graphicx}		% pour inclure des graphiques
\usepackage{enumitem,cprotect}		% personnalise les listes d'items (nécessaire pour ol, al ...)
\usepackage{hyperref}		% Liens hypertexte
\usepackage{pstricks,pst-all,pst-node,pstricks-add,pst-math,pst-plot,pst-tree,pst-eucl} % pstricks
\usepackage[a4paper,includeheadfoot,top=2cm,left=3cm, bottom=2cm,right=3cm]{geometry} % marges etc.
\usepackage{comment}			% commentaires multilignes
\usepackage{amsmath,environ} % maths (matrices, etc.)
\usepackage{amssymb,makeidx}
\usepackage{bm}				% bold maths
\usepackage{tabularx}		% tableaux
\usepackage{colortbl}		% tableaux en couleur
\usepackage{fontawesome}		% Fontawesome
\usepackage{environ}			% environment with command
\usepackage{fp}				% calculs pour ps-tricks
\usepackage{multido}			% pour ps tricks
\usepackage[np]{numprint}	% formattage nombre
\usepackage{tikz,tkz-tab} 			% package principal TikZ
\usepackage{pgfplots}   % axes
\usepackage{mathrsfs}    % cursives
\usepackage{calc}			% calcul taille boites
\usepackage[scaled=0.875]{helvet} % font sans serif
\usepackage{svg} % svg
\usepackage{scrextend} % local margin
\usepackage{scratch} %scratch
\usepackage{multicol} % colonnes
%\usepackage{infix-RPN,pst-func} % formule en notation polanaise inversée
\usepackage{listings}

%================================================================================================================================
%
% Réglages de base
%
%================================================================================================================================

\lstset{
language=Python,   % R code
literate=
{á}{{\'a}}1
{à}{{\`a}}1
{ã}{{\~a}}1
{é}{{\'e}}1
{è}{{\`e}}1
{ê}{{\^e}}1
{í}{{\'i}}1
{ó}{{\'o}}1
{õ}{{\~o}}1
{ú}{{\'u}}1
{ü}{{\"u}}1
{ç}{{\c{c}}}1
{~}{{ }}1
}


\definecolor{codegreen}{rgb}{0,0.6,0}
\definecolor{codegray}{rgb}{0.5,0.5,0.5}
\definecolor{codepurple}{rgb}{0.58,0,0.82}
\definecolor{backcolour}{rgb}{0.95,0.95,0.92}

\lstdefinestyle{mystyle}{
    backgroundcolor=\color{backcolour},   
    commentstyle=\color{codegreen},
    keywordstyle=\color{magenta},
    numberstyle=\tiny\color{codegray},
    stringstyle=\color{codepurple},
    basicstyle=\ttfamily\footnotesize,
    breakatwhitespace=false,         
    breaklines=true,                 
    captionpos=b,                    
    keepspaces=true,                 
    numbers=left,                    
xleftmargin=2em,
framexleftmargin=2em,            
    showspaces=false,                
    showstringspaces=false,
    showtabs=false,                  
    tabsize=2,
    upquote=true
}

\lstset{style=mystyle}


\lstset{style=mystyle}
\newcommand{\imgdir}{C:/laragon/www/newmc/assets/imgsvg/}
\newcommand{\imgsvgdir}{C:/laragon/www/newmc/assets/imgsvg/}

\definecolor{mcgris}{RGB}{220, 220, 220}% ancien~; pour compatibilité
\definecolor{mcbleu}{RGB}{52, 152, 219}
\definecolor{mcvert}{RGB}{125, 194, 70}
\definecolor{mcmauve}{RGB}{154, 0, 215}
\definecolor{mcorange}{RGB}{255, 96, 0}
\definecolor{mcturquoise}{RGB}{0, 153, 153}
\definecolor{mcrouge}{RGB}{255, 0, 0}
\definecolor{mclightvert}{RGB}{205, 234, 190}

\definecolor{gris}{RGB}{220, 220, 220}
\definecolor{bleu}{RGB}{52, 152, 219}
\definecolor{vert}{RGB}{125, 194, 70}
\definecolor{mauve}{RGB}{154, 0, 215}
\definecolor{orange}{RGB}{255, 96, 0}
\definecolor{turquoise}{RGB}{0, 153, 153}
\definecolor{rouge}{RGB}{255, 0, 0}
\definecolor{lightvert}{RGB}{205, 234, 190}
\setitemize[0]{label=\color{lightvert}  $\bullet$}

\pagestyle{fancy}
\renewcommand{\headrulewidth}{0.2pt}
\fancyhead[L]{maths-cours.fr}
\fancyhead[R]{\thepage}
\renewcommand{\footrulewidth}{0.2pt}
\fancyfoot[C]{}

\newcolumntype{C}{>{\centering\arraybackslash}X}
\newcolumntype{s}{>{\hsize=.35\hsize\arraybackslash}X}

\setlength{\parindent}{0pt}		 
\setlength{\parskip}{3mm}
\setlength{\headheight}{1cm}

\def\ebook{ebook}
\def\book{book}
\def\web{web}
\def\type{web}

\newcommand{\vect}[1]{\overrightarrow{\,\mathstrut#1\,}}

\def\Oij{$\left(\text{O}~;~\vect{\imath},~\vect{\jmath}\right)$}
\def\Oijk{$\left(\text{O}~;~\vect{\imath},~\vect{\jmath},~\vect{k}\right)$}
\def\Ouv{$\left(\text{O}~;~\vect{u},~\vect{v}\right)$}

\hypersetup{breaklinks=true, colorlinks = true, linkcolor = OliveGreen, urlcolor = OliveGreen, citecolor = OliveGreen, pdfauthor={Didier BONNEL - https://www.maths-cours.fr} } % supprime les bordures autour des liens

\renewcommand{\arg}[0]{\text{arg}}

\everymath{\displaystyle}

%================================================================================================================================
%
% Macros - Commandes
%
%================================================================================================================================

\newcommand\meta[2]{    			% Utilisé pour créer le post HTML.
	\def\titre{titre}
	\def\url{url}
	\def\arg{#1}
	\ifx\titre\arg
		\newcommand\maintitle{#2}
		\fancyhead[L]{#2}
		{\Large\sffamily \MakeUppercase{#2}}
		\vspace{1mm}\textcolor{mcvert}{\hrule}
	\fi 
	\ifx\url\arg
		\fancyfoot[L]{\href{https://www.maths-cours.fr#2}{\black \footnotesize{https://www.maths-cours.fr#2}}}
	\fi 
}


\newcommand\TitreC[1]{    		% Titre centré
     \needspace{3\baselineskip}
     \begin{center}\textbf{#1}\end{center}
}

\newcommand\newpar{    		% paragraphe
     \par
}

\newcommand\nosp {    		% commande vide (pas d'espace)
}
\newcommand{\id}[1]{} %ignore

\newcommand\boite[2]{				% Boite simple sans titre
	\vspace{5mm}
	\setlength{\fboxrule}{0.2mm}
	\setlength{\fboxsep}{5mm}	
	\fcolorbox{#1}{#1!3}{\makebox[\linewidth-2\fboxrule-2\fboxsep]{
  		\begin{minipage}[t]{\linewidth-2\fboxrule-4\fboxsep}\setlength{\parskip}{3mm}
  			 #2
  		\end{minipage}
	}}
	\vspace{5mm}
}

\newcommand\CBox[4]{				% Boites
	\vspace{5mm}
	\setlength{\fboxrule}{0.2mm}
	\setlength{\fboxsep}{5mm}
	
	\fcolorbox{#1}{#1!3}{\makebox[\linewidth-2\fboxrule-2\fboxsep]{
		\begin{minipage}[t]{1cm}\setlength{\parskip}{3mm}
	  		\textcolor{#1}{\LARGE{#2}}    
 	 	\end{minipage}  
  		\begin{minipage}[t]{\linewidth-2\fboxrule-4\fboxsep}\setlength{\parskip}{3mm}
			\raisebox{1.2mm}{\normalsize\sffamily{\textcolor{#1}{#3}}}						
  			 #4
  		\end{minipage}
	}}
	\vspace{5mm}
}

\newcommand\cadre[3]{				% Boites convertible html
	\par
	\vspace{2mm}
	\setlength{\fboxrule}{0.1mm}
	\setlength{\fboxsep}{5mm}
	\fcolorbox{#1}{white}{\makebox[\linewidth-2\fboxrule-2\fboxsep]{
  		\begin{minipage}[t]{\linewidth-2\fboxrule-4\fboxsep}\setlength{\parskip}{3mm}
			\raisebox{-2.5mm}{\sffamily \small{\textcolor{#1}{\MakeUppercase{#2}}}}		
			\par		
  			 #3
 	 		\end{minipage}
	}}
		\vspace{2mm}
	\par
}

\newcommand\bloc[3]{				% Boites convertible html sans bordure
     \needspace{2\baselineskip}
     {\sffamily \small{\textcolor{#1}{\MakeUppercase{#2}}}}    
		\par		
  			 #3
		\par
}

\newcommand\CHelp[1]{
     \CBox{Plum}{\faInfoCircle}{À RETENIR}{#1}
}

\newcommand\CUp[1]{
     \CBox{NavyBlue}{\faThumbsOUp}{EN PRATIQUE}{#1}
}

\newcommand\CInfo[1]{
     \CBox{Sepia}{\faArrowCircleRight}{REMARQUE}{#1}
}

\newcommand\CRedac[1]{
     \CBox{PineGreen}{\faEdit}{BIEN R\'EDIGER}{#1}
}

\newcommand\CError[1]{
     \CBox{Red}{\faExclamationTriangle}{ATTENTION}{#1}
}

\newcommand\TitreExo[2]{
\needspace{4\baselineskip}
 {\sffamily\large EXERCICE #1\ (\emph{#2 points})}
\vspace{5mm}
}

\newcommand\img[2]{
          \includegraphics[width=#2\paperwidth]{\imgdir#1}
}

\newcommand\imgsvg[2]{
       \begin{center}   \includegraphics[width=#2\paperwidth]{\imgsvgdir#1} \end{center}
}


\newcommand\Lien[2]{
     \href{#1}{#2 \tiny \faExternalLink}
}
\newcommand\mcLien[2]{
     \href{https~://www.maths-cours.fr/#1}{#2 \tiny \faExternalLink}
}

\newcommand{\euro}{\eurologo{}}

%================================================================================================================================
%
% Macros - Environement
%
%================================================================================================================================

\newenvironment{tex}{ %
}
{%
}

\newenvironment{indente}{ %
	\setlength\parindent{10mm}
}

{
	\setlength\parindent{0mm}
}

\newenvironment{corrige}{%
     \needspace{3\baselineskip}
     \medskip
     \textbf{\textsc{Corrigé}}
     \medskip
}
{
}

\newenvironment{extern}{%
     \begin{center}
     }
     {
     \end{center}
}

\NewEnviron{code}{%
	\par
     \boite{gray}{\texttt{%
     \BODY
     }}
     \par
}

\newenvironment{vbloc}{% boite sans cadre empeche saut de page
     \begin{minipage}[t]{\linewidth}
     }
     {
     \end{minipage}
}
\NewEnviron{h2}{%
    \needspace{3\baselineskip}
    \vspace{0.6cm}
	\noindent \MakeUppercase{\sffamily \large \BODY}
	\vspace{1mm}\textcolor{mcgris}{\hrule}\vspace{0.4cm}
	\par
}{}

\NewEnviron{h3}{%
    \needspace{3\baselineskip}
	\vspace{5mm}
	\textsc{\BODY}
	\par
}

\NewEnviron{margeneg}{ %
\begin{addmargin}[-1cm]{0cm}
\BODY
\end{addmargin}
}

\NewEnviron{html}{%
}

\begin{document}
\meta{url}{/cours/statistiques/}
\meta{pid}{1756}
\meta{titre}{Statistiques}
\meta{type}{cours}
%
\begin{h2}I. Exemple et vocabulaire\end{h2}
On interroge les 25 élèves d'un club sportif afin de connaître leurs âges.
\par
Voici leurs réponses, triées par ordre croissant~:
\begin{center}
     11~; 11~; 12~; 12~; 12~; 12~; 13~; 13~; 13~; 13~; 14~; 14~; 14~; \\14~; 14~; 14~; 15~; 15~; 16~; 16~; 16~; 17~; 17~; 17~; 17.
\end{center}
\begin{itemize}
     \item %
     L'ensemble de ces résultats forme une \textbf{série statistique}.
     \item %
     La \textbf{population} étudiée est l'ensemble des élèves du club sportif.
     \item %
     Le \textbf{caractère} étudié est l'âge des élèves.
     \item %
     Dans notre exemple, ce caractère peut prendre sept \textbf{valeurs} distinctes : 11~; 12~; 13~; 14~; 15~; 16~; 17.
     \item %
     Pour chacune de ces valeurs, l'\textbf{effectif} correspond au nombre de fois où la valeur a été obtenue.\\
     Par exemple, l'effectif de la valeur 11 est 2, l'effectif de la valeur 12 est 4, etc.
     \item %
     L'\textbf{effectif total} est le nombre d'élèves du club.\\
     Ici, l'effectif total est 25.
\end{itemize}
Pour présenter les résultats de manière plus pratique, on utilise fréquemment  un tableau des effectifs :
\begin{center}
     \begin{tabular}{|c|c|c|c|c|c|c|c|} %class="compact" width="600"
          \hline
          âges   &  11  &   12  &  13  &  14  &  15  &  16  &  17
          \\ \hline
          effectifs       & 2   &  4  &  4  & 6  & 2 &  3  &  4
          \\ \hline
     \end{tabular}
\end{center}
\begin{h2}II. Fréquences\end{h2}
\cadre{bleu}{Définition}{ % id="d010"
     La fréquence d'une valeur s'obtient en divisant l'effectif de cette valeur par l'effectif total~:
     \begin{center}
          $\text{fréquence}=$\nosp$\frac{\text{effectif}}{\text{effectif\ total}}$
     \end{center}
} % fin def
\bloc{cyan}{Remarque}{ % id="r020"
     Les fréquences peuvent être exprimées sous forme fractionnaire, sous forme décimale ou sous forme de pourcentage.
} % fin rem
\bloc{orange}{Exemple}{ % id="e030"
     Si l'on reprend l'exemple du paragraphe I., la fréquence des élèves âgés de 11 ans est :
     \begin{center}
          $f=\dfrac{2}{25}=0,08=8\%$
     \end{center}
     L'ensemble des fréquences de cet exemple peut être présenté dans un tableau~:
     \begin{center}
          \begin{tabular}{|c|c|c|c|c|c|c|c|} %class="compact" width="600"
               \hline
               âges   &  11  &   12  &  13  &  14  &  15  &  16  &  17
               \\ \hline
               fréquences       & 0,08   &  0,16  & 0,16  & 0,24  & 0,08 & 0,12  & 0,16
               \\ \hline
               fréquences en \%      & 8\%   & 16\%  & 16\%  & 24\%  & 8\% &  12\%  & 16\%
               \\ \hline
          \end{tabular}
     \end{center}
} % fin ex
\cadre{vert}{Propriété}{ % id="p040"
     La somme de toutes les fréquences est égale à 1 (c'est à dire 100\%).
} % fin pr
\begin{h2}III. Moyenne\end{h2}
\cadre{bleu}{Définition}{ % id="d050"
     La \textbf{moyenne} d'une série statistique s'obtient en divisant la somme de toutes les valeurs de la série par l'effectif total.
} % fin def
Si l'on note $x_1,  x_2,  x_3, \cdots, x_N$ les valeurs de la série et $N$ l'effectif total, la moyenne $M$ vaut~:
\begin{center}
     $ M=\dfrac{x_1 +  x_2 +  x_3 + \cdots + x_N}{N}$
\end{center}
\bloc{orange}{Exemple}{ % id="e070"
     Sonia a obtenu les notes suivantes à ses contrôles de mathématiques :
     \begin{center}
          14~; 9~; 12~; 13~; 12~; 15
     \end{center}
     L'effectif total est de 6 notes. La moyenne vaut~:
     \begin{center}
          $M=\dfrac{14+9+12+13+12+15}{6}$\nosp$=12,5$
     \end{center}
} % fin ex
\bloc{cyan}{Moyenne pondérée}{ % id="r080"
     Lorsque l'effectif est important, ce mode de calcul peut rapidement devenir fastidieux.
     \par
     On peut alors utiliser la méthode de la moyenne \og pondérée \fg{} .
     \par
     Par exemple, si l'on reprend la série statistique de la partie I., le tableau des effectifs montre que l'âge de 11 ans est présent 2 fois,  l'âge de 12 ans~: 4 fois, l'âge de 13 ans~: 4 fois, etc.\\
     La somme $11+11+12+12+12+12$\nosp$+13+13+13$\nosp$+13+\cdots$ peut donc être remplacée par $11 \times 2 + 12 \times 4 + 13 \times 4 + \cdots$.
     \par
     La moyenne vaut alors :
     \begin{overflow}
          $M=\dfrac{11 \times 2 + 12 \times 4 + 13 \times 4 + 14 \times 6 + 15 \times 2 + 16 \times 3 + 17 \times 4}{25}$\\
          $\phantom{M}=14,08$
     \end{overflow}
} % fin Moyenne pondérée
\begin{h2}IV. Médiane\end{h2}
\cadre{bleu}{Définition}{ % id="d100"
     On s'intéresse à une série statistique dont on a classé les valeurs par ordre croissant.\\
     La \textbf{médiane} est la valeur qui partage cette série en deux groupes de même effectif.
} % fin def
\bloc{orange}{Exemple 1 (effectif total impair)}{ % id="e110"
     Chiara a obtenu les cinq notes suivantes~:
     \begin{center}
          15~;~12~; 7~; 13~; 17.
     \end{center}
     On ordonne ces notes~:
     \begin{center}
          7~;~12~; \textcolor{red}{13}~; 15~; 17.
     \end{center}
     La note médiane est 13 (il y a deux notes inférieures et deux notes supérieures).
} % fin ex
\bloc{orange}{Exemple 2 (effectif total pair)}{ % id="e120"
     Luc a obtenu les six notes suivantes (déjà triées par ordre croissant)~:
     \begin{center}
          11~;~12~; 14~\textcolor{red}{;} 14~; 15~; 17.
     \end{center}
     La note médiane est 14 (le \og milieu \fg{} étant situé entre deux notes égales à 14 ).
} % fin ex
\bloc{orange}{Exemple 3 (effectif total pair)}{ % id="e130"
     Sacha a obtenu les six notes suivantes (triées par ordre croissant)~:
     \begin{center}
          7~;~12~; 12~\textcolor{red}{;} 13~; 15~; 16.
     \end{center}
     Sa note médiane est située entre 12 et 13.\\
     On choisira la moyenne de 12 et 13 soit 12,5 comme médiane.
} % fin ex
\begin{h2}V. Étendue d'une série statistique\end{h2}
\cadre{bleu}{Définition}{ % id="d140"
     L'\textbf{étendue} d'une série statistique est la différence entre la plus grande et la plus petite valeur de cette série.
} % fin def
\bloc{orange}{Exemple}{ % id="e150"
     Si l'on reprend l'exemple de la partie I.~:
     \begin{center}
          \begin{tabular}{|c|c|c|c|c|c|c|c|} %class="compact" width="600"
               \hline
               âges   &  11  &   12  &  13  &  14  &  15  &  16  &  17
               \\ \hline
               effectifs       & 2   &  4  &  4  & 6  & 2 &  3  &  4
               \\ \hline
          \end{tabular}
     \end{center}
     l'étendue est~:
     \begin{center}
          $17-11=6$
     \end{center}
     Cela correspond à la différence d'âge entre l'élève le plus âgé et l'élève le plus jeune.
} % fin ex
\bloc{cyan}{Remarque}{ % id="r160"
     L'étendue mesure la \textbf{dispersion} ou l'hétérogénéité d'une série statistique.
     \par
     Par exemple, si l'on considère les notes d'un élève, une étendue élevée signifiera que les résultats de l'élève sont irréguliers.
} % fin rem

\end{document}
µ
\documentclass[a4paper]{article}

%================================================================================================================================
%
% Packages
%
%================================================================================================================================

\usepackage[T1]{fontenc} 	% pour caractères accentués
\usepackage[utf8]{inputenc}  % encodage utf8
\usepackage[french]{babel}	% langue : français
\usepackage{fourier}			% caractères plus lisibles
\usepackage[dvipsnames]{xcolor} % couleurs
\usepackage{fancyhdr}		% réglage header footer
\usepackage{needspace}		% empêcher sauts de page mal placés
\usepackage{graphicx}		% pour inclure des graphiques
\usepackage{enumitem,cprotect}		% personnalise les listes d'items (nécessaire pour ol, al ...)
\usepackage{hyperref}		% Liens hypertexte
\usepackage{pstricks,pst-all,pst-node,pstricks-add,pst-math,pst-plot,pst-tree,pst-eucl} % pstricks
\usepackage[a4paper,includeheadfoot,top=2cm,left=3cm, bottom=2cm,right=3cm]{geometry} % marges etc.
\usepackage{comment}			% commentaires multilignes
\usepackage{amsmath,environ} % maths (matrices, etc.)
\usepackage{amssymb,makeidx}
\usepackage{bm}				% bold maths
\usepackage{tabularx}		% tableaux
\usepackage{colortbl}		% tableaux en couleur
\usepackage{fontawesome}		% Fontawesome
\usepackage{environ}			% environment with command
\usepackage{fp}				% calculs pour ps-tricks
\usepackage{multido}			% pour ps tricks
\usepackage[np]{numprint}	% formattage nombre
\usepackage{tikz,tkz-tab} 			% package principal TikZ
\usepackage{pgfplots}   % axes
\usepackage{mathrsfs}    % cursives
\usepackage{calc}			% calcul taille boites
\usepackage[scaled=0.875]{helvet} % font sans serif
\usepackage{svg} % svg
\usepackage{scrextend} % local margin
\usepackage{scratch} %scratch
\usepackage{multicol} % colonnes
%\usepackage{infix-RPN,pst-func} % formule en notation polanaise inversée
\usepackage{listings}

%================================================================================================================================
%
% Réglages de base
%
%================================================================================================================================

\lstset{
language=Python,   % R code
literate=
{á}{{\'a}}1
{à}{{\`a}}1
{ã}{{\~a}}1
{é}{{\'e}}1
{è}{{\`e}}1
{ê}{{\^e}}1
{í}{{\'i}}1
{ó}{{\'o}}1
{õ}{{\~o}}1
{ú}{{\'u}}1
{ü}{{\"u}}1
{ç}{{\c{c}}}1
{~}{{ }}1
}


\definecolor{codegreen}{rgb}{0,0.6,0}
\definecolor{codegray}{rgb}{0.5,0.5,0.5}
\definecolor{codepurple}{rgb}{0.58,0,0.82}
\definecolor{backcolour}{rgb}{0.95,0.95,0.92}

\lstdefinestyle{mystyle}{
    backgroundcolor=\color{backcolour},   
    commentstyle=\color{codegreen},
    keywordstyle=\color{magenta},
    numberstyle=\tiny\color{codegray},
    stringstyle=\color{codepurple},
    basicstyle=\ttfamily\footnotesize,
    breakatwhitespace=false,         
    breaklines=true,                 
    captionpos=b,                    
    keepspaces=true,                 
    numbers=left,                    
xleftmargin=2em,
framexleftmargin=2em,            
    showspaces=false,                
    showstringspaces=false,
    showtabs=false,                  
    tabsize=2,
    upquote=true
}

\lstset{style=mystyle}


\lstset{style=mystyle}
\newcommand{\imgdir}{C:/laragon/www/newmc/assets/imgsvg/}
\newcommand{\imgsvgdir}{C:/laragon/www/newmc/assets/imgsvg/}

\definecolor{mcgris}{RGB}{220, 220, 220}% ancien~; pour compatibilité
\definecolor{mcbleu}{RGB}{52, 152, 219}
\definecolor{mcvert}{RGB}{125, 194, 70}
\definecolor{mcmauve}{RGB}{154, 0, 215}
\definecolor{mcorange}{RGB}{255, 96, 0}
\definecolor{mcturquoise}{RGB}{0, 153, 153}
\definecolor{mcrouge}{RGB}{255, 0, 0}
\definecolor{mclightvert}{RGB}{205, 234, 190}

\definecolor{gris}{RGB}{220, 220, 220}
\definecolor{bleu}{RGB}{52, 152, 219}
\definecolor{vert}{RGB}{125, 194, 70}
\definecolor{mauve}{RGB}{154, 0, 215}
\definecolor{orange}{RGB}{255, 96, 0}
\definecolor{turquoise}{RGB}{0, 153, 153}
\definecolor{rouge}{RGB}{255, 0, 0}
\definecolor{lightvert}{RGB}{205, 234, 190}
\setitemize[0]{label=\color{lightvert}  $\bullet$}

\pagestyle{fancy}
\renewcommand{\headrulewidth}{0.2pt}
\fancyhead[L]{maths-cours.fr}
\fancyhead[R]{\thepage}
\renewcommand{\footrulewidth}{0.2pt}
\fancyfoot[C]{}

\newcolumntype{C}{>{\centering\arraybackslash}X}
\newcolumntype{s}{>{\hsize=.35\hsize\arraybackslash}X}

\setlength{\parindent}{0pt}		 
\setlength{\parskip}{3mm}
\setlength{\headheight}{1cm}

\def\ebook{ebook}
\def\book{book}
\def\web{web}
\def\type{web}

\newcommand{\vect}[1]{\overrightarrow{\,\mathstrut#1\,}}

\def\Oij{$\left(\text{O}~;~\vect{\imath},~\vect{\jmath}\right)$}
\def\Oijk{$\left(\text{O}~;~\vect{\imath},~\vect{\jmath},~\vect{k}\right)$}
\def\Ouv{$\left(\text{O}~;~\vect{u},~\vect{v}\right)$}

\hypersetup{breaklinks=true, colorlinks = true, linkcolor = OliveGreen, urlcolor = OliveGreen, citecolor = OliveGreen, pdfauthor={Didier BONNEL - https://www.maths-cours.fr} } % supprime les bordures autour des liens

\renewcommand{\arg}[0]{\text{arg}}

\everymath{\displaystyle}

%================================================================================================================================
%
% Macros - Commandes
%
%================================================================================================================================

\newcommand\meta[2]{    			% Utilisé pour créer le post HTML.
	\def\titre{titre}
	\def\url{url}
	\def\arg{#1}
	\ifx\titre\arg
		\newcommand\maintitle{#2}
		\fancyhead[L]{#2}
		{\Large\sffamily \MakeUppercase{#2}}
		\vspace{1mm}\textcolor{mcvert}{\hrule}
	\fi 
	\ifx\url\arg
		\fancyfoot[L]{\href{https://www.maths-cours.fr#2}{\black \footnotesize{https://www.maths-cours.fr#2}}}
	\fi 
}


\newcommand\TitreC[1]{    		% Titre centré
     \needspace{3\baselineskip}
     \begin{center}\textbf{#1}\end{center}
}

\newcommand\newpar{    		% paragraphe
     \par
}

\newcommand\nosp {    		% commande vide (pas d'espace)
}
\newcommand{\id}[1]{} %ignore

\newcommand\boite[2]{				% Boite simple sans titre
	\vspace{5mm}
	\setlength{\fboxrule}{0.2mm}
	\setlength{\fboxsep}{5mm}	
	\fcolorbox{#1}{#1!3}{\makebox[\linewidth-2\fboxrule-2\fboxsep]{
  		\begin{minipage}[t]{\linewidth-2\fboxrule-4\fboxsep}\setlength{\parskip}{3mm}
  			 #2
  		\end{minipage}
	}}
	\vspace{5mm}
}

\newcommand\CBox[4]{				% Boites
	\vspace{5mm}
	\setlength{\fboxrule}{0.2mm}
	\setlength{\fboxsep}{5mm}
	
	\fcolorbox{#1}{#1!3}{\makebox[\linewidth-2\fboxrule-2\fboxsep]{
		\begin{minipage}[t]{1cm}\setlength{\parskip}{3mm}
	  		\textcolor{#1}{\LARGE{#2}}    
 	 	\end{minipage}  
  		\begin{minipage}[t]{\linewidth-2\fboxrule-4\fboxsep}\setlength{\parskip}{3mm}
			\raisebox{1.2mm}{\normalsize\sffamily{\textcolor{#1}{#3}}}						
  			 #4
  		\end{minipage}
	}}
	\vspace{5mm}
}

\newcommand\cadre[3]{				% Boites convertible html
	\par
	\vspace{2mm}
	\setlength{\fboxrule}{0.1mm}
	\setlength{\fboxsep}{5mm}
	\fcolorbox{#1}{white}{\makebox[\linewidth-2\fboxrule-2\fboxsep]{
  		\begin{minipage}[t]{\linewidth-2\fboxrule-4\fboxsep}\setlength{\parskip}{3mm}
			\raisebox{-2.5mm}{\sffamily \small{\textcolor{#1}{\MakeUppercase{#2}}}}		
			\par		
  			 #3
 	 		\end{minipage}
	}}
		\vspace{2mm}
	\par
}

\newcommand\bloc[3]{				% Boites convertible html sans bordure
     \needspace{2\baselineskip}
     {\sffamily \small{\textcolor{#1}{\MakeUppercase{#2}}}}    
		\par		
  			 #3
		\par
}

\newcommand\CHelp[1]{
     \CBox{Plum}{\faInfoCircle}{À RETENIR}{#1}
}

\newcommand\CUp[1]{
     \CBox{NavyBlue}{\faThumbsOUp}{EN PRATIQUE}{#1}
}

\newcommand\CInfo[1]{
     \CBox{Sepia}{\faArrowCircleRight}{REMARQUE}{#1}
}

\newcommand\CRedac[1]{
     \CBox{PineGreen}{\faEdit}{BIEN R\'EDIGER}{#1}
}

\newcommand\CError[1]{
     \CBox{Red}{\faExclamationTriangle}{ATTENTION}{#1}
}

\newcommand\TitreExo[2]{
\needspace{4\baselineskip}
 {\sffamily\large EXERCICE #1\ (\emph{#2 points})}
\vspace{5mm}
}

\newcommand\img[2]{
          \includegraphics[width=#2\paperwidth]{\imgdir#1}
}

\newcommand\imgsvg[2]{
       \begin{center}   \includegraphics[width=#2\paperwidth]{\imgsvgdir#1} \end{center}
}


\newcommand\Lien[2]{
     \href{#1}{#2 \tiny \faExternalLink}
}
\newcommand\mcLien[2]{
     \href{https~://www.maths-cours.fr/#1}{#2 \tiny \faExternalLink}
}

\newcommand{\euro}{\eurologo{}}

%================================================================================================================================
%
% Macros - Environement
%
%================================================================================================================================

\newenvironment{tex}{ %
}
{%
}

\newenvironment{indente}{ %
	\setlength\parindent{10mm}
}

{
	\setlength\parindent{0mm}
}

\newenvironment{corrige}{%
     \needspace{3\baselineskip}
     \medskip
     \textbf{\textsc{Corrigé}}
     \medskip
}
{
}

\newenvironment{extern}{%
     \begin{center}
     }
     {
     \end{center}
}

\NewEnviron{code}{%
	\par
     \boite{gray}{\texttt{%
     \BODY
     }}
     \par
}

\newenvironment{vbloc}{% boite sans cadre empeche saut de page
     \begin{minipage}[t]{\linewidth}
     }
     {
     \end{minipage}
}
\NewEnviron{h2}{%
    \needspace{3\baselineskip}
    \vspace{0.6cm}
	\noindent \MakeUppercase{\sffamily \large \BODY}
	\vspace{1mm}\textcolor{mcgris}{\hrule}\vspace{0.4cm}
	\par
}{}

\NewEnviron{h3}{%
    \needspace{3\baselineskip}
	\vspace{5mm}
	\textsc{\BODY}
	\par
}

\NewEnviron{margeneg}{ %
\begin{addmargin}[-1cm]{0cm}
\BODY
\end{addmargin}
}

\NewEnviron{html}{%
}

\begin{document}
\meta{url}{/cours/probabilites-2/}
\meta{pid}{1760}
\meta{titre}{Probabilités}
\meta{type}{cours}
\begin{h2}1. Expérience aléatoire - Issues - \'Evénements \end{h2}
\cadre{bleu}{Définition}{% id="d10"
     Une \textbf{expérience aléatoire} est une expérience dont le résultat dépend du hasard.
} % fin def
\bloc{orange}{Exemples}{% id="e20"
     \begin{itemize}
          \item %
          Le lancer d'une pièce de monnaie à \og\textit{ Pile ou face} \fg{} est une expérience aléatoire dont les résultats possibles sont \og Pile \fg{} et \og Face \fg{}   .
          \item %
          Le lancer d'un dé à six faces est une expérience aléatoire dont les résultats possibles correspondent  aux entiers compris entre 1 et 6.
     \end{itemize}
} % fin ex
\cadre{bleu}{Définition}{ % id="d30"
     On appelle \textbf{issue} (ou \textbf{éventualité} ou \textbf{événement élémentaire}) un résultat possible d'une expérience aléatoire.
     \par
     On appelle \textbf{événement} un ensemble d'issues.
} % fin def
\bloc{orange}{Exemple}{ % id="e40"
     On lance un dé à six faces.
     \begin{itemize}
          \item %
          \og Obtenir le chiffre 6 \fg{} est une issue de cette expérience.
          \item %
          \og Obtenir un chiffre pair \fg{} est un événement composé des trois issues :  \og obtenir le chiffre 2 \fg{}, \og obtenir le chiffre 4 \fg{} et \og obtenir le chiffre 6 \fg{}.
     \end{itemize}
} % fin ex
%
\begin{h2}2. Probabilité d'un événement\end{h2}
\cadre{bleu}{Définitions}{ % id="d100"
     La \textbf{probabilité} d'un événement est un nombre \textbf{compris entre 0 et 1} qui mesure la \og chance \fg{} que cet événement se réalise.
     \par
     Un événement qui ne peut pas se réaliser s'appelle  événement \textbf{impossible}. Sa probabilité est égale à 0.
     \par
     Un événement qui  se réalisera obligatoirement s'appelle  événement \textbf{certain}. Sa probabilité est égale à 1.
} % fin def
\bloc{orange}{Exemples}{ % id="e110"
     \begin{itemize}
          \item %
          La probabilité d'obtenir un chiffre supérieur à 7 en lançant un dé à six faces est égale à 0 (événement impossible).
          \item %
          La probabilité d'obtenir un chiffre inférieur à 7 en lançant un dé à six faces est égale à 1 (événement certain).
     \end{itemize}
} % fin ex
\cadre{bleu}{Définition}{ % id="d120"
     On dit qu'il y a \textbf{équiprobabilité} si toutes les issues d'une expérience aléatoire ont la même probabilité de se réaliser.
} % fin def
\bloc{cyan}{Remarque}{ % id="r130"
     C'est en général l'énoncé d'un exercice ou la logique qui indiquera si l'on est - ou non - dans une situation d'équiprobabilité.
     \par
     Voici des exemples d'énoncés indiquant qu'il y a équiprobabilité :
     \begin{itemize}
          \item %
          On choisit \textit{au hasard} sous-entend que tous les choix sont équiprobables.
          \item %
          On lance un dé (ou une pièce) \textit{non truqué(e)} (ou \textit{bien équilibré(e)}) signifie que chacune des faces possède la même probabilité d'apparaître.
          \item %
          Une urne contient des boules \textit{indiscernables au toucher} signifie que toutes les boules ont la même probabilité d'être tirées.
     \end{itemize}
} % fin rem
\cadre{vert}{Propriété}{ % id="p140"
     Dans le cas d'une expérience aléatoire dans laquelle il y a \textbf{équiprobabilité}, la probabilité d'un événement est égale à :
     \begin{center}
          $p=\dfrac{\text{nombre d'issues favorables à l'événement}}{\text{nombre total d'issues possibles}}$
     \end{center}
} % fin pr
\bloc{orange}{Exercice corrigé}{ % id="e150"
     Une urne contient six boules indiscernables au toucher. Quatre sont blanches, une et rouge et la dernière est noire. \\
     On tire une boule au hasard.\\
     Quelle est la probabilité que cette boule soit blanche~?
     \par
     \textbf{Solution : }
     \par
     On est en situation d'équiprobabilité.\\
     Il y a six boules donc 6 issues possibles. \\
     Il y a quatre boules blanches donc 4 issues satisfaisant l'événement \og la boule tirée est blanche \fg{}.
     \par
     La probabilité demandée est donc :
     \begin{center}
          $p=\dfrac{4}{6}=\dfrac{2}{3}.$
     \end{center}
} % fin ex
\cadre{vert}{Propriété}{ % id="p160"
     La probabilité d'un événement est égale à la somme des probabilités des issues qui composent cet événement.
} % fin pr
\bloc{cyan}{Remarque}{ % id="r170"
     Cette propriété est valable même si l'on n'est pas en situation d'équiprobabilité.
} % fin rem
\bloc{orange}{Exercice corrigé}{ % id="e150"
     Un dé à six faces a été truqué de façon à obtenir le chiffre 6 une fois sur deux.
     \\
     On suppose qu'alors, les probabilités de chacune des issues sont les suivantes :
     \begin{center}
          \begin{tabular}{|c|c|c|c|c|c|c|} % class="compact"
               \hline
               Chiffre & 1 & 2 & 3 & 4 & 5 & 6 \\
               \hline
               Probabilité & 0,1 & 0,1  & 0,1  & 0,1  & 0,1  & 0,5  \\
               \hline
          \end{tabular}
     \end{center}
     Quelle est la probabilité d'obtenir un chiffre pair en lançant le dé une fois~?
     \par
     \textbf{Solution : }
     \par
     L'événement \og obtenir un chiffre pair \fg{} est constitué des issues : \og obtenir le chiffre 2 \fg{} (probabilité : 0,1),  \og obtenir le chiffre 4 \fg{} (probabilité : 0,1) et  \og obtenir le chiffre 6 \fg{}(probabilité : 0,5).
     \par
     La probabilité cherchée est la somme de ces trois probabilités~:
     \begin{center}
          $p=0,1+0,1+0,5=0,7.$
     \end{center}
} % fin ex

\end{document}
µ
\documentclass[a4paper]{article}

%================================================================================================================================
%
% Packages
%
%================================================================================================================================

\usepackage[T1]{fontenc} 	% pour caractères accentués
\usepackage[utf8]{inputenc}  % encodage utf8
\usepackage[french]{babel}	% langue : français
\usepackage{fourier}			% caractères plus lisibles
\usepackage[dvipsnames]{xcolor} % couleurs
\usepackage{fancyhdr}		% réglage header footer
\usepackage{needspace}		% empêcher sauts de page mal placés
\usepackage{graphicx}		% pour inclure des graphiques
\usepackage{enumitem,cprotect}		% personnalise les listes d'items (nécessaire pour ol, al ...)
\usepackage{hyperref}		% Liens hypertexte
\usepackage{pstricks,pst-all,pst-node,pstricks-add,pst-math,pst-plot,pst-tree,pst-eucl} % pstricks
\usepackage[a4paper,includeheadfoot,top=2cm,left=3cm, bottom=2cm,right=3cm]{geometry} % marges etc.
\usepackage{comment}			% commentaires multilignes
\usepackage{amsmath,environ} % maths (matrices, etc.)
\usepackage{amssymb,makeidx}
\usepackage{bm}				% bold maths
\usepackage{tabularx}		% tableaux
\usepackage{colortbl}		% tableaux en couleur
\usepackage{fontawesome}		% Fontawesome
\usepackage{environ}			% environment with command
\usepackage{fp}				% calculs pour ps-tricks
\usepackage{multido}			% pour ps tricks
\usepackage[np]{numprint}	% formattage nombre
\usepackage{tikz,tkz-tab} 			% package principal TikZ
\usepackage{pgfplots}   % axes
\usepackage{mathrsfs}    % cursives
\usepackage{calc}			% calcul taille boites
\usepackage[scaled=0.875]{helvet} % font sans serif
\usepackage{svg} % svg
\usepackage{scrextend} % local margin
\usepackage{scratch} %scratch
\usepackage{multicol} % colonnes
%\usepackage{infix-RPN,pst-func} % formule en notation polanaise inversée
\usepackage{listings}

%================================================================================================================================
%
% Réglages de base
%
%================================================================================================================================

\lstset{
language=Python,   % R code
literate=
{á}{{\'a}}1
{à}{{\`a}}1
{ã}{{\~a}}1
{é}{{\'e}}1
{è}{{\`e}}1
{ê}{{\^e}}1
{í}{{\'i}}1
{ó}{{\'o}}1
{õ}{{\~o}}1
{ú}{{\'u}}1
{ü}{{\"u}}1
{ç}{{\c{c}}}1
{~}{{ }}1
}


\definecolor{codegreen}{rgb}{0,0.6,0}
\definecolor{codegray}{rgb}{0.5,0.5,0.5}
\definecolor{codepurple}{rgb}{0.58,0,0.82}
\definecolor{backcolour}{rgb}{0.95,0.95,0.92}

\lstdefinestyle{mystyle}{
    backgroundcolor=\color{backcolour},   
    commentstyle=\color{codegreen},
    keywordstyle=\color{magenta},
    numberstyle=\tiny\color{codegray},
    stringstyle=\color{codepurple},
    basicstyle=\ttfamily\footnotesize,
    breakatwhitespace=false,         
    breaklines=true,                 
    captionpos=b,                    
    keepspaces=true,                 
    numbers=left,                    
xleftmargin=2em,
framexleftmargin=2em,            
    showspaces=false,                
    showstringspaces=false,
    showtabs=false,                  
    tabsize=2,
    upquote=true
}

\lstset{style=mystyle}


\lstset{style=mystyle}
\newcommand{\imgdir}{C:/laragon/www/newmc/assets/imgsvg/}
\newcommand{\imgsvgdir}{C:/laragon/www/newmc/assets/imgsvg/}

\definecolor{mcgris}{RGB}{220, 220, 220}% ancien~; pour compatibilité
\definecolor{mcbleu}{RGB}{52, 152, 219}
\definecolor{mcvert}{RGB}{125, 194, 70}
\definecolor{mcmauve}{RGB}{154, 0, 215}
\definecolor{mcorange}{RGB}{255, 96, 0}
\definecolor{mcturquoise}{RGB}{0, 153, 153}
\definecolor{mcrouge}{RGB}{255, 0, 0}
\definecolor{mclightvert}{RGB}{205, 234, 190}

\definecolor{gris}{RGB}{220, 220, 220}
\definecolor{bleu}{RGB}{52, 152, 219}
\definecolor{vert}{RGB}{125, 194, 70}
\definecolor{mauve}{RGB}{154, 0, 215}
\definecolor{orange}{RGB}{255, 96, 0}
\definecolor{turquoise}{RGB}{0, 153, 153}
\definecolor{rouge}{RGB}{255, 0, 0}
\definecolor{lightvert}{RGB}{205, 234, 190}
\setitemize[0]{label=\color{lightvert}  $\bullet$}

\pagestyle{fancy}
\renewcommand{\headrulewidth}{0.2pt}
\fancyhead[L]{maths-cours.fr}
\fancyhead[R]{\thepage}
\renewcommand{\footrulewidth}{0.2pt}
\fancyfoot[C]{}

\newcolumntype{C}{>{\centering\arraybackslash}X}
\newcolumntype{s}{>{\hsize=.35\hsize\arraybackslash}X}

\setlength{\parindent}{0pt}		 
\setlength{\parskip}{3mm}
\setlength{\headheight}{1cm}

\def\ebook{ebook}
\def\book{book}
\def\web{web}
\def\type{web}

\newcommand{\vect}[1]{\overrightarrow{\,\mathstrut#1\,}}

\def\Oij{$\left(\text{O}~;~\vect{\imath},~\vect{\jmath}\right)$}
\def\Oijk{$\left(\text{O}~;~\vect{\imath},~\vect{\jmath},~\vect{k}\right)$}
\def\Ouv{$\left(\text{O}~;~\vect{u},~\vect{v}\right)$}

\hypersetup{breaklinks=true, colorlinks = true, linkcolor = OliveGreen, urlcolor = OliveGreen, citecolor = OliveGreen, pdfauthor={Didier BONNEL - https://www.maths-cours.fr} } % supprime les bordures autour des liens

\renewcommand{\arg}[0]{\text{arg}}

\everymath{\displaystyle}

%================================================================================================================================
%
% Macros - Commandes
%
%================================================================================================================================

\newcommand\meta[2]{    			% Utilisé pour créer le post HTML.
	\def\titre{titre}
	\def\url{url}
	\def\arg{#1}
	\ifx\titre\arg
		\newcommand\maintitle{#2}
		\fancyhead[L]{#2}
		{\Large\sffamily \MakeUppercase{#2}}
		\vspace{1mm}\textcolor{mcvert}{\hrule}
	\fi 
	\ifx\url\arg
		\fancyfoot[L]{\href{https://www.maths-cours.fr#2}{\black \footnotesize{https://www.maths-cours.fr#2}}}
	\fi 
}


\newcommand\TitreC[1]{    		% Titre centré
     \needspace{3\baselineskip}
     \begin{center}\textbf{#1}\end{center}
}

\newcommand\newpar{    		% paragraphe
     \par
}

\newcommand\nosp {    		% commande vide (pas d'espace)
}
\newcommand{\id}[1]{} %ignore

\newcommand\boite[2]{				% Boite simple sans titre
	\vspace{5mm}
	\setlength{\fboxrule}{0.2mm}
	\setlength{\fboxsep}{5mm}	
	\fcolorbox{#1}{#1!3}{\makebox[\linewidth-2\fboxrule-2\fboxsep]{
  		\begin{minipage}[t]{\linewidth-2\fboxrule-4\fboxsep}\setlength{\parskip}{3mm}
  			 #2
  		\end{minipage}
	}}
	\vspace{5mm}
}

\newcommand\CBox[4]{				% Boites
	\vspace{5mm}
	\setlength{\fboxrule}{0.2mm}
	\setlength{\fboxsep}{5mm}
	
	\fcolorbox{#1}{#1!3}{\makebox[\linewidth-2\fboxrule-2\fboxsep]{
		\begin{minipage}[t]{1cm}\setlength{\parskip}{3mm}
	  		\textcolor{#1}{\LARGE{#2}}    
 	 	\end{minipage}  
  		\begin{minipage}[t]{\linewidth-2\fboxrule-4\fboxsep}\setlength{\parskip}{3mm}
			\raisebox{1.2mm}{\normalsize\sffamily{\textcolor{#1}{#3}}}						
  			 #4
  		\end{minipage}
	}}
	\vspace{5mm}
}

\newcommand\cadre[3]{				% Boites convertible html
	\par
	\vspace{2mm}
	\setlength{\fboxrule}{0.1mm}
	\setlength{\fboxsep}{5mm}
	\fcolorbox{#1}{white}{\makebox[\linewidth-2\fboxrule-2\fboxsep]{
  		\begin{minipage}[t]{\linewidth-2\fboxrule-4\fboxsep}\setlength{\parskip}{3mm}
			\raisebox{-2.5mm}{\sffamily \small{\textcolor{#1}{\MakeUppercase{#2}}}}		
			\par		
  			 #3
 	 		\end{minipage}
	}}
		\vspace{2mm}
	\par
}

\newcommand\bloc[3]{				% Boites convertible html sans bordure
     \needspace{2\baselineskip}
     {\sffamily \small{\textcolor{#1}{\MakeUppercase{#2}}}}    
		\par		
  			 #3
		\par
}

\newcommand\CHelp[1]{
     \CBox{Plum}{\faInfoCircle}{À RETENIR}{#1}
}

\newcommand\CUp[1]{
     \CBox{NavyBlue}{\faThumbsOUp}{EN PRATIQUE}{#1}
}

\newcommand\CInfo[1]{
     \CBox{Sepia}{\faArrowCircleRight}{REMARQUE}{#1}
}

\newcommand\CRedac[1]{
     \CBox{PineGreen}{\faEdit}{BIEN R\'EDIGER}{#1}
}

\newcommand\CError[1]{
     \CBox{Red}{\faExclamationTriangle}{ATTENTION}{#1}
}

\newcommand\TitreExo[2]{
\needspace{4\baselineskip}
 {\sffamily\large EXERCICE #1\ (\emph{#2 points})}
\vspace{5mm}
}

\newcommand\img[2]{
          \includegraphics[width=#2\paperwidth]{\imgdir#1}
}

\newcommand\imgsvg[2]{
       \begin{center}   \includegraphics[width=#2\paperwidth]{\imgsvgdir#1} \end{center}
}


\newcommand\Lien[2]{
     \href{#1}{#2 \tiny \faExternalLink}
}
\newcommand\mcLien[2]{
     \href{https~://www.maths-cours.fr/#1}{#2 \tiny \faExternalLink}
}

\newcommand{\euro}{\eurologo{}}

%================================================================================================================================
%
% Macros - Environement
%
%================================================================================================================================

\newenvironment{tex}{ %
}
{%
}

\newenvironment{indente}{ %
	\setlength\parindent{10mm}
}

{
	\setlength\parindent{0mm}
}

\newenvironment{corrige}{%
     \needspace{3\baselineskip}
     \medskip
     \textbf{\textsc{Corrigé}}
     \medskip
}
{
}

\newenvironment{extern}{%
     \begin{center}
     }
     {
     \end{center}
}

\NewEnviron{code}{%
	\par
     \boite{gray}{\texttt{%
     \BODY
     }}
     \par
}

\newenvironment{vbloc}{% boite sans cadre empeche saut de page
     \begin{minipage}[t]{\linewidth}
     }
     {
     \end{minipage}
}
\NewEnviron{h2}{%
    \needspace{3\baselineskip}
    \vspace{0.6cm}
	\noindent \MakeUppercase{\sffamily \large \BODY}
	\vspace{1mm}\textcolor{mcgris}{\hrule}\vspace{0.4cm}
	\par
}{}

\NewEnviron{h3}{%
    \needspace{3\baselineskip}
	\vspace{5mm}
	\textsc{\BODY}
	\par
}

\NewEnviron{margeneg}{ %
\begin{addmargin}[-1cm]{0cm}
\BODY
\end{addmargin}
}

\NewEnviron{html}{%
}

\begin{document}
\meta{url}{/exercices/loi-exponentielle-bac-s-metropole-2008/}
\meta{pid}{2218}
\meta{titre}{Loi exponentielle - Bac S Métropole 2008}
\meta{type}{exercices}
\begin{h2}Exercice 3 (5 points)\end{h2}
La durée de vie, exprimée en heures, d'un agenda électronique est une variable aléatoire $X$ qui suit une loi exponentielle de paramètre $\lambda $ où $\lambda $ est un réel strictement positif.
\par
On rappelle que pour tout $t \geqslant 0$ : $P\left(X \leqslant t\right)= \int_{0}^{t} \lambda  e^{-\lambda  x} \text{d}x$.
\par
La fonction $R$ définie sur l'intervalle $\left[0 ; +\infty \right[$ par $R\left(t\right)=P\left(X > t\right)$ est appelée fonction de fiabilité.
\begin{enumerate}
     \item \textbf{Restitution organisée de connaissances}
     \begin{enumerate}[label=\alph*.]
          \item Démontrer que pour tout $t \geqslant 0$, on a $R\left(t\right)=e^{-\lambda  t}$.
          \item Démontrer que la variable  $X$ suit une loi de durée de vie sans vieillissement, c'est-à-dire que pour tout réel $s \geqslant 0$, la probabilité conditionnelle $P_{X > t}\left(X > t+s\right)$ ne dépend pas du nombre $t \geqslant 0$.
     \end{enumerate}
     \item Dans cette question, on prendra $\lambda =0,00026$.
     \begin{enumerate}[label=\alph*.]
          \item Calculer $P\left(X \leqslant 1~000\right)$ et $P\left(X > 1~000\right)$.
          \item Sachant que l'événement $\left(X > 1~000\right)$ est réalisé, calculer la probabilité de l'événement $\left(X > 2~000\right)$.
          \item Sachant qu'un agenda a fonctionné plus de $2~000$ heures, quelle est la probabilité qu'il tombe en panne avant $3~000$ heures ? Pouvait-on prévoir ce résultat ?
     \end{enumerate}
\end{enumerate}
\begin{corrige}
     \begin{enumerate}
          \item
          \begin{enumerate}[label=\alph*.]
               \item  Pour tout $t\geqslant 0$, les événements $\left(X\leqslant t\right)$ et $\left(X > t\right)$ sont complémentaires donc :
               \par
               $R\left(t\right)=1-P\left(X\leqslant t\right)=1-\int_{0}^{t}\lambda  e^{-\lambda  x}dx$
               \par
               Une primitive de la fonction $x\mapsto \lambda  e^{-\lambda  x}$ est la fonction  $x\mapsto -e^{-\lambda  x}$ donc :
               \par
               $R\left(t\right)=1-\left[- e^{-\lambda  x}\right]_{0}^{t}=1+\left(e^{-\lambda  t}-1\right)=e^{-\lambda  t}$
               \item  $P_{X > t}\left(X > t+s\right)=\frac{P\left(\left(X > t\right) \cap  \left(X > t+s\right)\right)}{P\left(X > t\right)}$
               \par
               Or $\left(X > t\right) \cap  \left(X > t+s\right)=\left(X > t+s\right)$
               \par
               donc :
               \par
               $P_{X > t}\left(X > t+s\right)=\frac{P\left(X > t+s\right)}{P\left(X > t\right)}=\frac{R\left(t+s\right)}{R\left(t\right)}$
               \par
               et d'après le \textbf{a.} :
               \par
               $P_{X > t}\left(X > t+s\right)=\frac{e^{-\lambda  \left(t+s\right)}}{e^{-\lambda  t}}=e^{-\lambda  s}$
               \par
               ce qui est indépendant de $t$.
          \end{enumerate}
          \item
          \begin{enumerate}[label=\alph*.]
               \item
               $P\left(X > 1~000\right)=R\left(1~000\right)=e^{-1~000\lambda }=e^{-0,26}$
               \par
               $P\left(X\leqslant 1~000\right)=1-R\left(1~000\right)=1-e^{-0,26}$
               \item  D'après la question \textbf{1.b.} :
               \par
               $P_{X > 1~000}\left(X > 2~000\right)=P\left(X > 1~000\right)=e^{-0,26}$
               \item $P_{X > 2~000}\left(X\leqslant 3~000\right)=1-P_{X > 2~000}\left(X > 3~000\right) $
               \par
               Donc d'après la question \textbf{1.b.} :
               \par
               $P_{X > 2~000}\left(X\leqslant 3~000\right)=1-P\left(X > 1~\right)=1-e^{-0,26}$
               \par
               Ce résultat était prévisible, la loi de durée de vie étant sans vieillissement.
          \end{enumerate}
     \end{enumerate}
\end{corrige}

\end{document}
µ
\documentclass[a4paper]{article}

%================================================================================================================================
%
% Packages
%
%================================================================================================================================

\usepackage[T1]{fontenc} 	% pour caractères accentués
\usepackage[utf8]{inputenc}  % encodage utf8
\usepackage[french]{babel}	% langue : français
\usepackage{fourier}			% caractères plus lisibles
\usepackage[dvipsnames]{xcolor} % couleurs
\usepackage{fancyhdr}		% réglage header footer
\usepackage{needspace}		% empêcher sauts de page mal placés
\usepackage{graphicx}		% pour inclure des graphiques
\usepackage{enumitem,cprotect}		% personnalise les listes d'items (nécessaire pour ol, al ...)
\usepackage{hyperref}		% Liens hypertexte
\usepackage{pstricks,pst-all,pst-node,pstricks-add,pst-math,pst-plot,pst-tree,pst-eucl} % pstricks
\usepackage[a4paper,includeheadfoot,top=2cm,left=3cm, bottom=2cm,right=3cm]{geometry} % marges etc.
\usepackage{comment}			% commentaires multilignes
\usepackage{amsmath,environ} % maths (matrices, etc.)
\usepackage{amssymb,makeidx}
\usepackage{bm}				% bold maths
\usepackage{tabularx}		% tableaux
\usepackage{colortbl}		% tableaux en couleur
\usepackage{fontawesome}		% Fontawesome
\usepackage{environ}			% environment with command
\usepackage{fp}				% calculs pour ps-tricks
\usepackage{multido}			% pour ps tricks
\usepackage[np]{numprint}	% formattage nombre
\usepackage{tikz,tkz-tab} 			% package principal TikZ
\usepackage{pgfplots}   % axes
\usepackage{mathrsfs}    % cursives
\usepackage{calc}			% calcul taille boites
\usepackage[scaled=0.875]{helvet} % font sans serif
\usepackage{svg} % svg
\usepackage{scrextend} % local margin
\usepackage{scratch} %scratch
\usepackage{multicol} % colonnes
%\usepackage{infix-RPN,pst-func} % formule en notation polanaise inversée
\usepackage{listings}

%================================================================================================================================
%
% Réglages de base
%
%================================================================================================================================

\lstset{
language=Python,   % R code
literate=
{á}{{\'a}}1
{à}{{\`a}}1
{ã}{{\~a}}1
{é}{{\'e}}1
{è}{{\`e}}1
{ê}{{\^e}}1
{í}{{\'i}}1
{ó}{{\'o}}1
{õ}{{\~o}}1
{ú}{{\'u}}1
{ü}{{\"u}}1
{ç}{{\c{c}}}1
{~}{{ }}1
}


\definecolor{codegreen}{rgb}{0,0.6,0}
\definecolor{codegray}{rgb}{0.5,0.5,0.5}
\definecolor{codepurple}{rgb}{0.58,0,0.82}
\definecolor{backcolour}{rgb}{0.95,0.95,0.92}

\lstdefinestyle{mystyle}{
    backgroundcolor=\color{backcolour},   
    commentstyle=\color{codegreen},
    keywordstyle=\color{magenta},
    numberstyle=\tiny\color{codegray},
    stringstyle=\color{codepurple},
    basicstyle=\ttfamily\footnotesize,
    breakatwhitespace=false,         
    breaklines=true,                 
    captionpos=b,                    
    keepspaces=true,                 
    numbers=left,                    
xleftmargin=2em,
framexleftmargin=2em,            
    showspaces=false,                
    showstringspaces=false,
    showtabs=false,                  
    tabsize=2,
    upquote=true
}

\lstset{style=mystyle}


\lstset{style=mystyle}
\newcommand{\imgdir}{C:/laragon/www/newmc/assets/imgsvg/}
\newcommand{\imgsvgdir}{C:/laragon/www/newmc/assets/imgsvg/}

\definecolor{mcgris}{RGB}{220, 220, 220}% ancien~; pour compatibilité
\definecolor{mcbleu}{RGB}{52, 152, 219}
\definecolor{mcvert}{RGB}{125, 194, 70}
\definecolor{mcmauve}{RGB}{154, 0, 215}
\definecolor{mcorange}{RGB}{255, 96, 0}
\definecolor{mcturquoise}{RGB}{0, 153, 153}
\definecolor{mcrouge}{RGB}{255, 0, 0}
\definecolor{mclightvert}{RGB}{205, 234, 190}

\definecolor{gris}{RGB}{220, 220, 220}
\definecolor{bleu}{RGB}{52, 152, 219}
\definecolor{vert}{RGB}{125, 194, 70}
\definecolor{mauve}{RGB}{154, 0, 215}
\definecolor{orange}{RGB}{255, 96, 0}
\definecolor{turquoise}{RGB}{0, 153, 153}
\definecolor{rouge}{RGB}{255, 0, 0}
\definecolor{lightvert}{RGB}{205, 234, 190}
\setitemize[0]{label=\color{lightvert}  $\bullet$}

\pagestyle{fancy}
\renewcommand{\headrulewidth}{0.2pt}
\fancyhead[L]{maths-cours.fr}
\fancyhead[R]{\thepage}
\renewcommand{\footrulewidth}{0.2pt}
\fancyfoot[C]{}

\newcolumntype{C}{>{\centering\arraybackslash}X}
\newcolumntype{s}{>{\hsize=.35\hsize\arraybackslash}X}

\setlength{\parindent}{0pt}		 
\setlength{\parskip}{3mm}
\setlength{\headheight}{1cm}

\def\ebook{ebook}
\def\book{book}
\def\web{web}
\def\type{web}

\newcommand{\vect}[1]{\overrightarrow{\,\mathstrut#1\,}}

\def\Oij{$\left(\text{O}~;~\vect{\imath},~\vect{\jmath}\right)$}
\def\Oijk{$\left(\text{O}~;~\vect{\imath},~\vect{\jmath},~\vect{k}\right)$}
\def\Ouv{$\left(\text{O}~;~\vect{u},~\vect{v}\right)$}

\hypersetup{breaklinks=true, colorlinks = true, linkcolor = OliveGreen, urlcolor = OliveGreen, citecolor = OliveGreen, pdfauthor={Didier BONNEL - https://www.maths-cours.fr} } % supprime les bordures autour des liens

\renewcommand{\arg}[0]{\text{arg}}

\everymath{\displaystyle}

%================================================================================================================================
%
% Macros - Commandes
%
%================================================================================================================================

\newcommand\meta[2]{    			% Utilisé pour créer le post HTML.
	\def\titre{titre}
	\def\url{url}
	\def\arg{#1}
	\ifx\titre\arg
		\newcommand\maintitle{#2}
		\fancyhead[L]{#2}
		{\Large\sffamily \MakeUppercase{#2}}
		\vspace{1mm}\textcolor{mcvert}{\hrule}
	\fi 
	\ifx\url\arg
		\fancyfoot[L]{\href{https://www.maths-cours.fr#2}{\black \footnotesize{https://www.maths-cours.fr#2}}}
	\fi 
}


\newcommand\TitreC[1]{    		% Titre centré
     \needspace{3\baselineskip}
     \begin{center}\textbf{#1}\end{center}
}

\newcommand\newpar{    		% paragraphe
     \par
}

\newcommand\nosp {    		% commande vide (pas d'espace)
}
\newcommand{\id}[1]{} %ignore

\newcommand\boite[2]{				% Boite simple sans titre
	\vspace{5mm}
	\setlength{\fboxrule}{0.2mm}
	\setlength{\fboxsep}{5mm}	
	\fcolorbox{#1}{#1!3}{\makebox[\linewidth-2\fboxrule-2\fboxsep]{
  		\begin{minipage}[t]{\linewidth-2\fboxrule-4\fboxsep}\setlength{\parskip}{3mm}
  			 #2
  		\end{minipage}
	}}
	\vspace{5mm}
}

\newcommand\CBox[4]{				% Boites
	\vspace{5mm}
	\setlength{\fboxrule}{0.2mm}
	\setlength{\fboxsep}{5mm}
	
	\fcolorbox{#1}{#1!3}{\makebox[\linewidth-2\fboxrule-2\fboxsep]{
		\begin{minipage}[t]{1cm}\setlength{\parskip}{3mm}
	  		\textcolor{#1}{\LARGE{#2}}    
 	 	\end{minipage}  
  		\begin{minipage}[t]{\linewidth-2\fboxrule-4\fboxsep}\setlength{\parskip}{3mm}
			\raisebox{1.2mm}{\normalsize\sffamily{\textcolor{#1}{#3}}}						
  			 #4
  		\end{minipage}
	}}
	\vspace{5mm}
}

\newcommand\cadre[3]{				% Boites convertible html
	\par
	\vspace{2mm}
	\setlength{\fboxrule}{0.1mm}
	\setlength{\fboxsep}{5mm}
	\fcolorbox{#1}{white}{\makebox[\linewidth-2\fboxrule-2\fboxsep]{
  		\begin{minipage}[t]{\linewidth-2\fboxrule-4\fboxsep}\setlength{\parskip}{3mm}
			\raisebox{-2.5mm}{\sffamily \small{\textcolor{#1}{\MakeUppercase{#2}}}}		
			\par		
  			 #3
 	 		\end{minipage}
	}}
		\vspace{2mm}
	\par
}

\newcommand\bloc[3]{				% Boites convertible html sans bordure
     \needspace{2\baselineskip}
     {\sffamily \small{\textcolor{#1}{\MakeUppercase{#2}}}}    
		\par		
  			 #3
		\par
}

\newcommand\CHelp[1]{
     \CBox{Plum}{\faInfoCircle}{À RETENIR}{#1}
}

\newcommand\CUp[1]{
     \CBox{NavyBlue}{\faThumbsOUp}{EN PRATIQUE}{#1}
}

\newcommand\CInfo[1]{
     \CBox{Sepia}{\faArrowCircleRight}{REMARQUE}{#1}
}

\newcommand\CRedac[1]{
     \CBox{PineGreen}{\faEdit}{BIEN R\'EDIGER}{#1}
}

\newcommand\CError[1]{
     \CBox{Red}{\faExclamationTriangle}{ATTENTION}{#1}
}

\newcommand\TitreExo[2]{
\needspace{4\baselineskip}
 {\sffamily\large EXERCICE #1\ (\emph{#2 points})}
\vspace{5mm}
}

\newcommand\img[2]{
          \includegraphics[width=#2\paperwidth]{\imgdir#1}
}

\newcommand\imgsvg[2]{
       \begin{center}   \includegraphics[width=#2\paperwidth]{\imgsvgdir#1} \end{center}
}


\newcommand\Lien[2]{
     \href{#1}{#2 \tiny \faExternalLink}
}
\newcommand\mcLien[2]{
     \href{https~://www.maths-cours.fr/#1}{#2 \tiny \faExternalLink}
}

\newcommand{\euro}{\eurologo{}}

%================================================================================================================================
%
% Macros - Environement
%
%================================================================================================================================

\newenvironment{tex}{ %
}
{%
}

\newenvironment{indente}{ %
	\setlength\parindent{10mm}
}

{
	\setlength\parindent{0mm}
}

\newenvironment{corrige}{%
     \needspace{3\baselineskip}
     \medskip
     \textbf{\textsc{Corrigé}}
     \medskip
}
{
}

\newenvironment{extern}{%
     \begin{center}
     }
     {
     \end{center}
}

\NewEnviron{code}{%
	\par
     \boite{gray}{\texttt{%
     \BODY
     }}
     \par
}

\newenvironment{vbloc}{% boite sans cadre empeche saut de page
     \begin{minipage}[t]{\linewidth}
     }
     {
     \end{minipage}
}
\NewEnviron{h2}{%
    \needspace{3\baselineskip}
    \vspace{0.6cm}
	\noindent \MakeUppercase{\sffamily \large \BODY}
	\vspace{1mm}\textcolor{mcgris}{\hrule}\vspace{0.4cm}
	\par
}{}

\NewEnviron{h3}{%
    \needspace{3\baselineskip}
	\vspace{5mm}
	\textsc{\BODY}
	\par
}

\NewEnviron{margeneg}{ %
\begin{addmargin}[-1cm]{0cm}
\BODY
\end{addmargin}
}

\NewEnviron{html}{%
}

\begin{document}
\meta{url}{/supplement/fonctions-paires-impaires/}
\meta{pid}{3538}
\meta{titre}{Fonctions paires et impaires}
\meta{type}{supplement}
%
\cadre{bleu}{Définition}{% id="d10"
     Une fonction $f$ définie sur un ensemble $\mathscr D$ symétrique par rapport à 0 est \textbf{paire} si et seulement si pour tout $x \in \mathscr D$ :
     \begin{center}$f(-x)=f(x)$\end{center}
}
\cadre{vert}{Propriété}{% id="p10"
     Dans un repère orthogonal, la courbe représentative d'une fonction paire est symétrique par rapport à l'axe des ordonnées.
}
\cadre{bleu}{Définition}{% id="d20"
     Une fonction $f$ définie sur un ensemble $\mathscr D$ symétrique par rapport à 0 est \textbf{impaire} si et seulement si pour tout $x \in \mathscr D$ :
     \begin{center}$f(-x)=-f(x)$
     \end{center}
}
\cadre{vert}{Propriété}{% id="p10"
     La courbe représentative d'une fonction impaire est symétrique par rapport à l'origine du repère.
}
\cadre{vert}{Méthode}{%
     \textbf{Préalable :} On vérifie que l'ensemble de définition de la fonction est symétrique par rapport à 0.
     \par
     C'est le cas, en particulier, pour les ensembles $\mathbb{R}$, $\mathbb{R}\backslash\left\{0\right\}$ et les intervalles du type $\left[-a;a\right]$ et $\left]-a;a\right[$. Si l'ensemble de définition n'est \textbf{pas} symétrique par rapport à 0,\textbf{ la fonction n'est ni paire ni impaire}.
     \begin{enumerate}
          \item
          Pour montrer qu'une fonction $f$ \textbf{est paire}:
          \begin{itemize}
               \item
               On calcule $f\left(-x\right)$ en remplaçant $x$ par $\left(-x\right)$ dans l'expression de $f\left(x\right)$.
               \item
          On montre que $f\left(-x\right)=f\left(x\right)$
\end{itemize}
          \item
          Pour montrer qu'une fonction $f$ \textbf{est impaire} :\begin{itemize}
               \item
               On calcule $f\left(-x\right)$ en remplaçant $x$ par $\left(-x\right)$ dans l'expression de $f\left(x\right)$.
               \item
               On calcule $-f\left(x\right)$
               \item
          On montre que $f\left(-x\right)=-f\left(x\right)$ 
\end{itemize}
          \item
          Pour montrer qu'une fonction $f$ \textbf{n'est pas paire} :
          \par
          Il suffit d'un contre-exemple c'est à dire qu'il suffit de trouver un nombre $a$ tel que $f\left(-a\right)\neq f\left(a\right)$
          \item
          Pour montrer qu'une fonction $f$ \textbf{n'est pas impaire} :
          \par
          Il suffit d'un contre-exemple c'est à dire qu'il suffit de trouver un nombre $a$ tel que $f\left(-a\right)\neq -f\left(a\right)$
     \end{enumerate}
}
\bloc{cyan}{Remarques}{%
     \begin{enumerate}
          \item
          Si l'énoncé ne précise pas s'il faut montrer que $f$ est paire ou s'il faut montrer que $f$ est impaire, il  peut s'avérer utile de tracer la courbe représentative de $f$ à la calculatrice.
          \par
          si la courbe est \textbf{symétrique} par rapport à l'\textbf{axe des ordonnées}, la fonction est \textbf{paire}.
          \par
          si la courbe est \textbf{symétrique} par rapport à l'\textbf{origine}, la fonction est \textbf{impaire}.
          \item
          Une fonction peut n'être ni paire, ni impaire (c'est même le cas général ! )
          \item
          Seule la fonction nulle ($x\mapsto 0$) est à la fois paire et impaire.
     \end{enumerate}
}
\bloc{orange}{Exemple 1}{%
     Montrer que la fonction définie sur $\mathbb{R}\backslash\left\{0\right\}$ par $f : x\mapsto \frac{1+x^{2}}{x^{2}}$ est paire.
     \par
     Pour tout réel non nul $x$ :
     \par
     $f\left(-x\right)=\frac{1+\left(-x\right)^{2}}{\left(-x\right)^{2}}$
     \par
     Or $\left(-x\right)^{2}=x^{2}$ donc
     \par
     $f\left(-x\right)=\frac{1+x^{2}}{x^{2}}$
     \par
     Pour tout $x\in \mathbb{R}\backslash\left\{0\right\}$, $f\left(-x\right)=f\left(x\right)$ donc la fonction $f$ est \textbf{paire}.
}
\bloc{orange}{Exemple 2}{%
     Etudier la parité de la fonction définie sur $\mathbb{R}$ par $f : x\mapsto \frac{2x}{1+x^{2}}$
     \par
     La courbe de la fonction $f$ donnée par la calculatrice semble symétrique par rapport à l'origine du repère.
    \imgsvg{calculatrice-fonction-impaire}{0.3}%alt="calculatrice-fonction-impaire" style="width :30rem" class="aligncenter"
     On va donc montrer que $f$ est impaire.
     \par
     Pour tout réel $x$ :
     \par
     $f\left(-x\right)=\frac{2\times \left(-x\right)}{1+\left(-x\right)^{2}}$
     \par
     Or $\left(-x\right)^{2}=x^{2}$ donc
     \par
     $f\left(-x\right)=\frac{-2x}{1+x^{2}}$
     \par
     Par ailleurs :
     \par
     $-f\left(x\right)=-\frac{2x}{1+x^{2}}$
     \par
     Pour tout réel $x$, $f\left(-x\right)=-f\left(x\right)$ donc la fonction $f$ est \textbf{impaire}.
}
\bloc{orange}{Exemple 3}{%
     Etudier la parité de la fonction définie sur $\mathbb{R}$ par $f : x\mapsto \frac{1+ x}{1+x^{2}}$
     \par
     La courbe de la fonction $f$ donnée par la calculatrice ne présente aucune symétrie.
\imgsvg{ecran-calculatrice-fonction}{0.3} % alt="ecran-calculatrice-fonction" style="width :30rem" class="aligncenter"
     On va donc montrer que $f$ n'est ni paire ni impaire.
     \par
     Calculons par exemple $f\left(1\right)$ et $f\left(-1\right)$
     \par
     $f\left(1\right)=\frac{2}{2}=1$ et $f\left(-1\right)=\frac{0}{2}=0$
     \par
     On a donc $f\left(-1\right)\neq f\left(1\right)$ et  $f\left(-1\right)\neq -f\left(1\right)$
     \par
     Donc $f$ n'est \textbf{ni paire ni impaire}.
}
\end{document}

µ
\documentclass[a4paper]{article}

%================================================================================================================================
%
% Packages
%
%================================================================================================================================

\usepackage[T1]{fontenc} 	% pour caractères accentués
\usepackage[utf8]{inputenc}  % encodage utf8
\usepackage[french]{babel}	% langue : français
\usepackage{fourier}			% caractères plus lisibles
\usepackage[dvipsnames]{xcolor} % couleurs
\usepackage{fancyhdr}		% réglage header footer
\usepackage{needspace}		% empêcher sauts de page mal placés
\usepackage{graphicx}		% pour inclure des graphiques
\usepackage{enumitem,cprotect}		% personnalise les listes d'items (nécessaire pour ol, al ...)
\usepackage{hyperref}		% Liens hypertexte
\usepackage{pstricks,pst-all,pst-node,pstricks-add,pst-math,pst-plot,pst-tree,pst-eucl} % pstricks
\usepackage[a4paper,includeheadfoot,top=2cm,left=3cm, bottom=2cm,right=3cm]{geometry} % marges etc.
\usepackage{comment}			% commentaires multilignes
\usepackage{amsmath,environ} % maths (matrices, etc.)
\usepackage{amssymb,makeidx}
\usepackage{bm}				% bold maths
\usepackage{tabularx}		% tableaux
\usepackage{colortbl}		% tableaux en couleur
\usepackage{fontawesome}		% Fontawesome
\usepackage{environ}			% environment with command
\usepackage{fp}				% calculs pour ps-tricks
\usepackage{multido}			% pour ps tricks
\usepackage[np]{numprint}	% formattage nombre
\usepackage{tikz,tkz-tab} 			% package principal TikZ
\usepackage{pgfplots}   % axes
\usepackage{mathrsfs}    % cursives
\usepackage{calc}			% calcul taille boites
\usepackage[scaled=0.875]{helvet} % font sans serif
\usepackage{svg} % svg
\usepackage{scrextend} % local margin
\usepackage{scratch} %scratch
\usepackage{multicol} % colonnes
%\usepackage{infix-RPN,pst-func} % formule en notation polanaise inversée
\usepackage{listings}

%================================================================================================================================
%
% Réglages de base
%
%================================================================================================================================

\lstset{
language=Python,   % R code
literate=
{á}{{\'a}}1
{à}{{\`a}}1
{ã}{{\~a}}1
{é}{{\'e}}1
{è}{{\`e}}1
{ê}{{\^e}}1
{í}{{\'i}}1
{ó}{{\'o}}1
{õ}{{\~o}}1
{ú}{{\'u}}1
{ü}{{\"u}}1
{ç}{{\c{c}}}1
{~}{{ }}1
}


\definecolor{codegreen}{rgb}{0,0.6,0}
\definecolor{codegray}{rgb}{0.5,0.5,0.5}
\definecolor{codepurple}{rgb}{0.58,0,0.82}
\definecolor{backcolour}{rgb}{0.95,0.95,0.92}

\lstdefinestyle{mystyle}{
    backgroundcolor=\color{backcolour},   
    commentstyle=\color{codegreen},
    keywordstyle=\color{magenta},
    numberstyle=\tiny\color{codegray},
    stringstyle=\color{codepurple},
    basicstyle=\ttfamily\footnotesize,
    breakatwhitespace=false,         
    breaklines=true,                 
    captionpos=b,                    
    keepspaces=true,                 
    numbers=left,                    
xleftmargin=2em,
framexleftmargin=2em,            
    showspaces=false,                
    showstringspaces=false,
    showtabs=false,                  
    tabsize=2,
    upquote=true
}

\lstset{style=mystyle}


\lstset{style=mystyle}
\newcommand{\imgdir}{C:/laragon/www/newmc/assets/imgsvg/}
\newcommand{\imgsvgdir}{C:/laragon/www/newmc/assets/imgsvg/}

\definecolor{mcgris}{RGB}{220, 220, 220}% ancien~; pour compatibilité
\definecolor{mcbleu}{RGB}{52, 152, 219}
\definecolor{mcvert}{RGB}{125, 194, 70}
\definecolor{mcmauve}{RGB}{154, 0, 215}
\definecolor{mcorange}{RGB}{255, 96, 0}
\definecolor{mcturquoise}{RGB}{0, 153, 153}
\definecolor{mcrouge}{RGB}{255, 0, 0}
\definecolor{mclightvert}{RGB}{205, 234, 190}

\definecolor{gris}{RGB}{220, 220, 220}
\definecolor{bleu}{RGB}{52, 152, 219}
\definecolor{vert}{RGB}{125, 194, 70}
\definecolor{mauve}{RGB}{154, 0, 215}
\definecolor{orange}{RGB}{255, 96, 0}
\definecolor{turquoise}{RGB}{0, 153, 153}
\definecolor{rouge}{RGB}{255, 0, 0}
\definecolor{lightvert}{RGB}{205, 234, 190}
\setitemize[0]{label=\color{lightvert}  $\bullet$}

\pagestyle{fancy}
\renewcommand{\headrulewidth}{0.2pt}
\fancyhead[L]{maths-cours.fr}
\fancyhead[R]{\thepage}
\renewcommand{\footrulewidth}{0.2pt}
\fancyfoot[C]{}

\newcolumntype{C}{>{\centering\arraybackslash}X}
\newcolumntype{s}{>{\hsize=.35\hsize\arraybackslash}X}

\setlength{\parindent}{0pt}		 
\setlength{\parskip}{3mm}
\setlength{\headheight}{1cm}

\def\ebook{ebook}
\def\book{book}
\def\web{web}
\def\type{web}

\newcommand{\vect}[1]{\overrightarrow{\,\mathstrut#1\,}}

\def\Oij{$\left(\text{O}~;~\vect{\imath},~\vect{\jmath}\right)$}
\def\Oijk{$\left(\text{O}~;~\vect{\imath},~\vect{\jmath},~\vect{k}\right)$}
\def\Ouv{$\left(\text{O}~;~\vect{u},~\vect{v}\right)$}

\hypersetup{breaklinks=true, colorlinks = true, linkcolor = OliveGreen, urlcolor = OliveGreen, citecolor = OliveGreen, pdfauthor={Didier BONNEL - https://www.maths-cours.fr} } % supprime les bordures autour des liens

\renewcommand{\arg}[0]{\text{arg}}

\everymath{\displaystyle}

%================================================================================================================================
%
% Macros - Commandes
%
%================================================================================================================================

\newcommand\meta[2]{    			% Utilisé pour créer le post HTML.
	\def\titre{titre}
	\def\url{url}
	\def\arg{#1}
	\ifx\titre\arg
		\newcommand\maintitle{#2}
		\fancyhead[L]{#2}
		{\Large\sffamily \MakeUppercase{#2}}
		\vspace{1mm}\textcolor{mcvert}{\hrule}
	\fi 
	\ifx\url\arg
		\fancyfoot[L]{\href{https://www.maths-cours.fr#2}{\black \footnotesize{https://www.maths-cours.fr#2}}}
	\fi 
}


\newcommand\TitreC[1]{    		% Titre centré
     \needspace{3\baselineskip}
     \begin{center}\textbf{#1}\end{center}
}

\newcommand\newpar{    		% paragraphe
     \par
}

\newcommand\nosp {    		% commande vide (pas d'espace)
}
\newcommand{\id}[1]{} %ignore

\newcommand\boite[2]{				% Boite simple sans titre
	\vspace{5mm}
	\setlength{\fboxrule}{0.2mm}
	\setlength{\fboxsep}{5mm}	
	\fcolorbox{#1}{#1!3}{\makebox[\linewidth-2\fboxrule-2\fboxsep]{
  		\begin{minipage}[t]{\linewidth-2\fboxrule-4\fboxsep}\setlength{\parskip}{3mm}
  			 #2
  		\end{minipage}
	}}
	\vspace{5mm}
}

\newcommand\CBox[4]{				% Boites
	\vspace{5mm}
	\setlength{\fboxrule}{0.2mm}
	\setlength{\fboxsep}{5mm}
	
	\fcolorbox{#1}{#1!3}{\makebox[\linewidth-2\fboxrule-2\fboxsep]{
		\begin{minipage}[t]{1cm}\setlength{\parskip}{3mm}
	  		\textcolor{#1}{\LARGE{#2}}    
 	 	\end{minipage}  
  		\begin{minipage}[t]{\linewidth-2\fboxrule-4\fboxsep}\setlength{\parskip}{3mm}
			\raisebox{1.2mm}{\normalsize\sffamily{\textcolor{#1}{#3}}}						
  			 #4
  		\end{minipage}
	}}
	\vspace{5mm}
}

\newcommand\cadre[3]{				% Boites convertible html
	\par
	\vspace{2mm}
	\setlength{\fboxrule}{0.1mm}
	\setlength{\fboxsep}{5mm}
	\fcolorbox{#1}{white}{\makebox[\linewidth-2\fboxrule-2\fboxsep]{
  		\begin{minipage}[t]{\linewidth-2\fboxrule-4\fboxsep}\setlength{\parskip}{3mm}
			\raisebox{-2.5mm}{\sffamily \small{\textcolor{#1}{\MakeUppercase{#2}}}}		
			\par		
  			 #3
 	 		\end{minipage}
	}}
		\vspace{2mm}
	\par
}

\newcommand\bloc[3]{				% Boites convertible html sans bordure
     \needspace{2\baselineskip}
     {\sffamily \small{\textcolor{#1}{\MakeUppercase{#2}}}}    
		\par		
  			 #3
		\par
}

\newcommand\CHelp[1]{
     \CBox{Plum}{\faInfoCircle}{À RETENIR}{#1}
}

\newcommand\CUp[1]{
     \CBox{NavyBlue}{\faThumbsOUp}{EN PRATIQUE}{#1}
}

\newcommand\CInfo[1]{
     \CBox{Sepia}{\faArrowCircleRight}{REMARQUE}{#1}
}

\newcommand\CRedac[1]{
     \CBox{PineGreen}{\faEdit}{BIEN R\'EDIGER}{#1}
}

\newcommand\CError[1]{
     \CBox{Red}{\faExclamationTriangle}{ATTENTION}{#1}
}

\newcommand\TitreExo[2]{
\needspace{4\baselineskip}
 {\sffamily\large EXERCICE #1\ (\emph{#2 points})}
\vspace{5mm}
}

\newcommand\img[2]{
          \includegraphics[width=#2\paperwidth]{\imgdir#1}
}

\newcommand\imgsvg[2]{
       \begin{center}   \includegraphics[width=#2\paperwidth]{\imgsvgdir#1} \end{center}
}


\newcommand\Lien[2]{
     \href{#1}{#2 \tiny \faExternalLink}
}
\newcommand\mcLien[2]{
     \href{https~://www.maths-cours.fr/#1}{#2 \tiny \faExternalLink}
}

\newcommand{\euro}{\eurologo{}}

%================================================================================================================================
%
% Macros - Environement
%
%================================================================================================================================

\newenvironment{tex}{ %
}
{%
}

\newenvironment{indente}{ %
	\setlength\parindent{10mm}
}

{
	\setlength\parindent{0mm}
}

\newenvironment{corrige}{%
     \needspace{3\baselineskip}
     \medskip
     \textbf{\textsc{Corrigé}}
     \medskip
}
{
}

\newenvironment{extern}{%
     \begin{center}
     }
     {
     \end{center}
}

\NewEnviron{code}{%
	\par
     \boite{gray}{\texttt{%
     \BODY
     }}
     \par
}

\newenvironment{vbloc}{% boite sans cadre empeche saut de page
     \begin{minipage}[t]{\linewidth}
     }
     {
     \end{minipage}
}
\NewEnviron{h2}{%
    \needspace{3\baselineskip}
    \vspace{0.6cm}
	\noindent \MakeUppercase{\sffamily \large \BODY}
	\vspace{1mm}\textcolor{mcgris}{\hrule}\vspace{0.4cm}
	\par
}{}

\NewEnviron{h3}{%
    \needspace{3\baselineskip}
	\vspace{5mm}
	\textsc{\BODY}
	\par
}

\NewEnviron{margeneg}{ %
\begin{addmargin}[-1cm]{0cm}
\BODY
\end{addmargin}
}

\NewEnviron{html}{%
}

\begin{document}
\meta{url}{http://maths-cours.local/supplement/ecriture-fractionnaire/}
\meta{pid}{5527}
\meta{titre}{Écriture fractionnaire d'un nombre}
\meta{type}{supplement}
%
\documentclass[a4paper]{article}

%================================================================================================================================
%
% Packages
%
%================================================================================================================================

\usepackage[T1]{fontenc} 	% pour caractères accentués
\usepackage[utf8]{inputenc}  % encodage utf8
\usepackage[french]{babel}	% langue : français
\usepackage{fourier}			% caractères plus lisibles
\usepackage[dvipsnames]{xcolor} % couleurs
\usepackage{fancyhdr}		% réglage header footer
\usepackage{needspace}		% empêcher sauts de page mal placés
\usepackage{graphicx}		% pour inclure des graphiques
\usepackage{enumitem,cprotect}		% personnalise les listes d'items (nécessaire pour ol, al ...)
\usepackage{hyperref}		% Liens hypertexte
\usepackage{pstricks,pst-all,pst-node,pstricks-add,pst-math,pst-plot,pst-tree,pst-eucl} % pstricks
\usepackage[a4paper,includeheadfoot,top=2cm,left=3cm, bottom=2cm,right=3cm]{geometry} % marges etc.
\usepackage{comment}			% commentaires multilignes
\usepackage{amsmath,environ} % maths (matrices, etc.)
\usepackage{amssymb,makeidx}
\usepackage{bm}				% bold maths
\usepackage{tabularx}		% tableaux
\usepackage{colortbl}		% tableaux en couleur
\usepackage{fontawesome}		% Fontawesome
\usepackage{environ}			% environment with command
\usepackage{fp}				% calculs pour ps-tricks
\usepackage{multido}			% pour ps tricks
\usepackage[np]{numprint}	% formattage nombre
\usepackage{tikz,tkz-tab} 			% package principal TikZ
\usepackage{pgfplots}   % axes
\usepackage{mathrsfs}    % cursives
\usepackage{calc}			% calcul taille boites
\usepackage[scaled=0.875]{helvet} % font sans serif
\usepackage{svg} % svg
\usepackage{scrextend} % local margin
\usepackage{scratch} %scratch
\usepackage{multicol} % colonnes
%\usepackage{infix-RPN,pst-func} % formule en notation polanaise inversée
\usepackage{listings}

%================================================================================================================================
%
% Réglages de base
%
%================================================================================================================================

\lstset{
language=Python,   % R code
literate=
{á}{{\'a}}1
{à}{{\`a}}1
{ã}{{\~a}}1
{é}{{\'e}}1
{è}{{\`e}}1
{ê}{{\^e}}1
{í}{{\'i}}1
{ó}{{\'o}}1
{õ}{{\~o}}1
{ú}{{\'u}}1
{ü}{{\"u}}1
{ç}{{\c{c}}}1
{~}{{ }}1
}


\definecolor{codegreen}{rgb}{0,0.6,0}
\definecolor{codegray}{rgb}{0.5,0.5,0.5}
\definecolor{codepurple}{rgb}{0.58,0,0.82}
\definecolor{backcolour}{rgb}{0.95,0.95,0.92}

\lstdefinestyle{mystyle}{
    backgroundcolor=\color{backcolour},   
    commentstyle=\color{codegreen},
    keywordstyle=\color{magenta},
    numberstyle=\tiny\color{codegray},
    stringstyle=\color{codepurple},
    basicstyle=\ttfamily\footnotesize,
    breakatwhitespace=false,         
    breaklines=true,                 
    captionpos=b,                    
    keepspaces=true,                 
    numbers=left,                    
xleftmargin=2em,
framexleftmargin=2em,            
    showspaces=false,                
    showstringspaces=false,
    showtabs=false,                  
    tabsize=2,
    upquote=true
}

\lstset{style=mystyle}


\lstset{style=mystyle}
\newcommand{\imgdir}{C:/laragon/www/newmc/assets/imgsvg/}
\newcommand{\imgsvgdir}{C:/laragon/www/newmc/assets/imgsvg/}

\definecolor{mcgris}{RGB}{220, 220, 220}% ancien~; pour compatibilité
\definecolor{mcbleu}{RGB}{52, 152, 219}
\definecolor{mcvert}{RGB}{125, 194, 70}
\definecolor{mcmauve}{RGB}{154, 0, 215}
\definecolor{mcorange}{RGB}{255, 96, 0}
\definecolor{mcturquoise}{RGB}{0, 153, 153}
\definecolor{mcrouge}{RGB}{255, 0, 0}
\definecolor{mclightvert}{RGB}{205, 234, 190}

\definecolor{gris}{RGB}{220, 220, 220}
\definecolor{bleu}{RGB}{52, 152, 219}
\definecolor{vert}{RGB}{125, 194, 70}
\definecolor{mauve}{RGB}{154, 0, 215}
\definecolor{orange}{RGB}{255, 96, 0}
\definecolor{turquoise}{RGB}{0, 153, 153}
\definecolor{rouge}{RGB}{255, 0, 0}
\definecolor{lightvert}{RGB}{205, 234, 190}
\setitemize[0]{label=\color{lightvert}  $\bullet$}

\pagestyle{fancy}
\renewcommand{\headrulewidth}{0.2pt}
\fancyhead[L]{maths-cours.fr}
\fancyhead[R]{\thepage}
\renewcommand{\footrulewidth}{0.2pt}
\fancyfoot[C]{}

\newcolumntype{C}{>{\centering\arraybackslash}X}
\newcolumntype{s}{>{\hsize=.35\hsize\arraybackslash}X}

\setlength{\parindent}{0pt}		 
\setlength{\parskip}{3mm}
\setlength{\headheight}{1cm}

\def\ebook{ebook}
\def\book{book}
\def\web{web}
\def\type{web}

\newcommand{\vect}[1]{\overrightarrow{\,\mathstrut#1\,}}

\def\Oij{$\left(\text{O}~;~\vect{\imath},~\vect{\jmath}\right)$}
\def\Oijk{$\left(\text{O}~;~\vect{\imath},~\vect{\jmath},~\vect{k}\right)$}
\def\Ouv{$\left(\text{O}~;~\vect{u},~\vect{v}\right)$}

\hypersetup{breaklinks=true, colorlinks = true, linkcolor = OliveGreen, urlcolor = OliveGreen, citecolor = OliveGreen, pdfauthor={Didier BONNEL - https://www.maths-cours.fr} } % supprime les bordures autour des liens

\renewcommand{\arg}[0]{\text{arg}}

\everymath{\displaystyle}

%================================================================================================================================
%
% Macros - Commandes
%
%================================================================================================================================

\newcommand\meta[2]{    			% Utilisé pour créer le post HTML.
	\def\titre{titre}
	\def\url{url}
	\def\arg{#1}
	\ifx\titre\arg
		\newcommand\maintitle{#2}
		\fancyhead[L]{#2}
		{\Large\sffamily \MakeUppercase{#2}}
		\vspace{1mm}\textcolor{mcvert}{\hrule}
	\fi 
	\ifx\url\arg
		\fancyfoot[L]{\href{https://www.maths-cours.fr#2}{\black \footnotesize{https://www.maths-cours.fr#2}}}
	\fi 
}


\newcommand\TitreC[1]{    		% Titre centré
     \needspace{3\baselineskip}
     \begin{center}\textbf{#1}\end{center}
}

\newcommand\newpar{    		% paragraphe
     \par
}

\newcommand\nosp {    		% commande vide (pas d'espace)
}
\newcommand{\id}[1]{} %ignore

\newcommand\boite[2]{				% Boite simple sans titre
	\vspace{5mm}
	\setlength{\fboxrule}{0.2mm}
	\setlength{\fboxsep}{5mm}	
	\fcolorbox{#1}{#1!3}{\makebox[\linewidth-2\fboxrule-2\fboxsep]{
  		\begin{minipage}[t]{\linewidth-2\fboxrule-4\fboxsep}\setlength{\parskip}{3mm}
  			 #2
  		\end{minipage}
	}}
	\vspace{5mm}
}

\newcommand\CBox[4]{				% Boites
	\vspace{5mm}
	\setlength{\fboxrule}{0.2mm}
	\setlength{\fboxsep}{5mm}
	
	\fcolorbox{#1}{#1!3}{\makebox[\linewidth-2\fboxrule-2\fboxsep]{
		\begin{minipage}[t]{1cm}\setlength{\parskip}{3mm}
	  		\textcolor{#1}{\LARGE{#2}}    
 	 	\end{minipage}  
  		\begin{minipage}[t]{\linewidth-2\fboxrule-4\fboxsep}\setlength{\parskip}{3mm}
			\raisebox{1.2mm}{\normalsize\sffamily{\textcolor{#1}{#3}}}						
  			 #4
  		\end{minipage}
	}}
	\vspace{5mm}
}

\newcommand\cadre[3]{				% Boites convertible html
	\par
	\vspace{2mm}
	\setlength{\fboxrule}{0.1mm}
	\setlength{\fboxsep}{5mm}
	\fcolorbox{#1}{white}{\makebox[\linewidth-2\fboxrule-2\fboxsep]{
  		\begin{minipage}[t]{\linewidth-2\fboxrule-4\fboxsep}\setlength{\parskip}{3mm}
			\raisebox{-2.5mm}{\sffamily \small{\textcolor{#1}{\MakeUppercase{#2}}}}		
			\par		
  			 #3
 	 		\end{minipage}
	}}
		\vspace{2mm}
	\par
}

\newcommand\bloc[3]{				% Boites convertible html sans bordure
     \needspace{2\baselineskip}
     {\sffamily \small{\textcolor{#1}{\MakeUppercase{#2}}}}    
		\par		
  			 #3
		\par
}

\newcommand\CHelp[1]{
     \CBox{Plum}{\faInfoCircle}{À RETENIR}{#1}
}

\newcommand\CUp[1]{
     \CBox{NavyBlue}{\faThumbsOUp}{EN PRATIQUE}{#1}
}

\newcommand\CInfo[1]{
     \CBox{Sepia}{\faArrowCircleRight}{REMARQUE}{#1}
}

\newcommand\CRedac[1]{
     \CBox{PineGreen}{\faEdit}{BIEN R\'EDIGER}{#1}
}

\newcommand\CError[1]{
     \CBox{Red}{\faExclamationTriangle}{ATTENTION}{#1}
}

\newcommand\TitreExo[2]{
\needspace{4\baselineskip}
 {\sffamily\large EXERCICE #1\ (\emph{#2 points})}
\vspace{5mm}
}

\newcommand\img[2]{
          \includegraphics[width=#2\paperwidth]{\imgdir#1}
}

\newcommand\imgsvg[2]{
       \begin{center}   \includegraphics[width=#2\paperwidth]{\imgsvgdir#1} \end{center}
}


\newcommand\Lien[2]{
     \href{#1}{#2 \tiny \faExternalLink}
}
\newcommand\mcLien[2]{
     \href{https~://www.maths-cours.fr/#1}{#2 \tiny \faExternalLink}
}

\newcommand{\euro}{\eurologo{}}

%================================================================================================================================
%
% Macros - Environement
%
%================================================================================================================================

\newenvironment{tex}{ %
}
{%
}

\newenvironment{indente}{ %
	\setlength\parindent{10mm}
}

{
	\setlength\parindent{0mm}
}

\newenvironment{corrige}{%
     \needspace{3\baselineskip}
     \medskip
     \textbf{\textsc{Corrigé}}
     \medskip
}
{
}

\newenvironment{extern}{%
     \begin{center}
     }
     {
     \end{center}
}

\NewEnviron{code}{%
	\par
     \boite{gray}{\texttt{%
     \BODY
     }}
     \par
}

\newenvironment{vbloc}{% boite sans cadre empeche saut de page
     \begin{minipage}[t]{\linewidth}
     }
     {
     \end{minipage}
}
\NewEnviron{h2}{%
    \needspace{3\baselineskip}
    \vspace{0.6cm}
	\noindent \MakeUppercase{\sffamily \large \BODY}
	\vspace{1mm}\textcolor{mcgris}{\hrule}\vspace{0.4cm}
	\par
}{}

\NewEnviron{h3}{%
    \needspace{3\baselineskip}
	\vspace{5mm}
	\textsc{\BODY}
	\par
}

\NewEnviron{margeneg}{ %
\begin{addmargin}[-1cm]{0cm}
\BODY
\end{addmargin}
}

\NewEnviron{html}{%
}

\begin{document}
\cadre{jaune}{Situation}{%
     On cherche à écrire un nombre dont on connaît une écriture décimale illimitée (par exemple 1,7323232...) sous forme d'une fraction $\frac{p}{q}$ ( $p\in \mathbb{Z}, q\in \mathbb{N}^*$ )
}
\cadre{rouge}{Méthode}{%
     Un nombre est rationnel si et seulement si une séquence de ses décimales se répète indéfiniment à partir d'un certain rang.
     \par
     Par exemple $ 1,7323232... $ dans lequel la séquence $32$   se répète indéfiniment, est un nombre rationnel que l'on note parfois $1,7  \overline{  32} $ (le surlignage désignant la séquence qui se répète).
     \begin{enumerate}
          \item
          On part de l'égalité :
          \par
          $ x= $\textit{écriture décimale illimitée} $ \ \ $\textbf{[égalité 1]}
\par
          $ x  $ repésentant le nombre dont on cherche l'écriture fractionnaire.
\par
          \textbf{On multiplie chaque membre de l'égalité par $10^n$, où n est la longueur de la séquence qui se répète.}
\par
( \textbf{Rappel :}  
$10^n$ s'écrit 1 suivi de n zéros)
\par
Par exemple dans le cas de 1,7323232... on multipliera chaque membre par 100 ($10^2$).
     \par
     On obtient une égalité de la forme :
     \par
     $10^nx$=\textit{autre écriture décimale illimitée} $ \ \ $\textbf{[égalité 2]}
     \item
     \textbf{On soustrait membre à membre l'égalité 1 de l'égalité 2}.
\par
On remarque que les décimales s'éliminent à partir d'un certain rang !
     \item
     Il est alors facile d'écrire $ x $ sous forme fractionnaire qu'il suffit ensuite de simplifier.
\end{enumerate}
}
\bloc{orange}{Exemple 1}{%
     \textbf{Ecrire 0,666666... sous forme de fraction.}
     \begin{enumerate}
          \item
          On part de l'égalité :
          \par
          $  x=0,666666... \ \ $ \textbf{[égalité 1]}
\par
          La séquence qui se répète est constituée d'un seul chiffre. 
\\On multiplie donc chaque membre de l'égalité par $10^1=10$
          \par
          $  10x=6,66666.... \ \ $\textbf{[égalité 2]}
          \item
          On soustrait membre à membre [égalité 2]-[égalité 1] :
          \par
          $ 10x-x=6,66666...-0,666666... $
          \par
          $ 9x=6 $
          \item
          On divise chaque membre par $ 9 $ et on simplifie par $ 3 $
          \par
          $x=\frac{6}{9}=\frac{2}{3}$
     \end{enumerate}
}
\bloc{orange}{Exemple 2}{%
     \textbf{Quelle fraction vaut $ 5,153454545...  $?}
     \begin{enumerate}
          \item
          On part cette fois de l'égalité :
          \par
          $  x=5,153454545... \ \ $\textbf{[égalité 1]}
\par
          La séquence qui se répète est comporte 2 chiffres (45). 
\par
On va donc multiplier chaque membre par $10^2=100$.
          \par
           $  100x=515,3454545... \ \ $\textbf{[égalité 2]}
          \item
          On soustrait les deux égalités :
          \par
          $  100x-x=515,3454545...-5,153454545...$
          \par
          $ 99x=510,192 $
          \item
     On multiplie chaque membre par $ 1000  $pour supprimer la virgule :

\cadre{bleu}{Revoir}{ % id=r10
Revoir les règles de simplification de fractions et les critères de divisibilité
} % fin définition
$  99000 x=510192$
     \par
     on divise par $  99000 $ :
     \par
     $x=\frac{510192}{99000}$
     \par
     la fraction se simplifie par 8 puis par 9 (donc par 72) :
     \par
     $x=\frac{7086}{1375}$
\end{enumerate}
}
\bloc{orange}{Petit complément}{%
     Appliquer cette méthode à $ x=0,99999... $
     \par
     Le résultat vous surprend-il?
}

\end{document}
\end{document}

µ
\documentclass[a4paper]{article}

%================================================================================================================================
%
% Packages
%
%================================================================================================================================

\usepackage[T1]{fontenc} 	% pour caractères accentués
\usepackage[utf8]{inputenc}  % encodage utf8
\usepackage[french]{babel}	% langue : français
\usepackage{fourier}			% caractères plus lisibles
\usepackage[dvipsnames]{xcolor} % couleurs
\usepackage{fancyhdr}		% réglage header footer
\usepackage{needspace}		% empêcher sauts de page mal placés
\usepackage{graphicx}		% pour inclure des graphiques
\usepackage{enumitem,cprotect}		% personnalise les listes d'items (nécessaire pour ol, al ...)
\usepackage{hyperref}		% Liens hypertexte
\usepackage{pstricks,pst-all,pst-node,pstricks-add,pst-math,pst-plot,pst-tree,pst-eucl} % pstricks
\usepackage[a4paper,includeheadfoot,top=2cm,left=3cm, bottom=2cm,right=3cm]{geometry} % marges etc.
\usepackage{comment}			% commentaires multilignes
\usepackage{amsmath,environ} % maths (matrices, etc.)
\usepackage{amssymb,makeidx}
\usepackage{bm}				% bold maths
\usepackage{tabularx}		% tableaux
\usepackage{colortbl}		% tableaux en couleur
\usepackage{fontawesome}		% Fontawesome
\usepackage{environ}			% environment with command
\usepackage{fp}				% calculs pour ps-tricks
\usepackage{multido}			% pour ps tricks
\usepackage[np]{numprint}	% formattage nombre
\usepackage{tikz,tkz-tab} 			% package principal TikZ
\usepackage{pgfplots}   % axes
\usepackage{mathrsfs}    % cursives
\usepackage{calc}			% calcul taille boites
\usepackage[scaled=0.875]{helvet} % font sans serif
\usepackage{svg} % svg
\usepackage{scrextend} % local margin
\usepackage{scratch} %scratch
\usepackage{multicol} % colonnes
%\usepackage{infix-RPN,pst-func} % formule en notation polanaise inversée
\usepackage{listings}

%================================================================================================================================
%
% Réglages de base
%
%================================================================================================================================

\lstset{
language=Python,   % R code
literate=
{á}{{\'a}}1
{à}{{\`a}}1
{ã}{{\~a}}1
{é}{{\'e}}1
{è}{{\`e}}1
{ê}{{\^e}}1
{í}{{\'i}}1
{ó}{{\'o}}1
{õ}{{\~o}}1
{ú}{{\'u}}1
{ü}{{\"u}}1
{ç}{{\c{c}}}1
{~}{{ }}1
}


\definecolor{codegreen}{rgb}{0,0.6,0}
\definecolor{codegray}{rgb}{0.5,0.5,0.5}
\definecolor{codepurple}{rgb}{0.58,0,0.82}
\definecolor{backcolour}{rgb}{0.95,0.95,0.92}

\lstdefinestyle{mystyle}{
    backgroundcolor=\color{backcolour},   
    commentstyle=\color{codegreen},
    keywordstyle=\color{magenta},
    numberstyle=\tiny\color{codegray},
    stringstyle=\color{codepurple},
    basicstyle=\ttfamily\footnotesize,
    breakatwhitespace=false,         
    breaklines=true,                 
    captionpos=b,                    
    keepspaces=true,                 
    numbers=left,                    
xleftmargin=2em,
framexleftmargin=2em,            
    showspaces=false,                
    showstringspaces=false,
    showtabs=false,                  
    tabsize=2,
    upquote=true
}

\lstset{style=mystyle}


\lstset{style=mystyle}
\newcommand{\imgdir}{C:/laragon/www/newmc/assets/imgsvg/}
\newcommand{\imgsvgdir}{C:/laragon/www/newmc/assets/imgsvg/}

\definecolor{mcgris}{RGB}{220, 220, 220}% ancien~; pour compatibilité
\definecolor{mcbleu}{RGB}{52, 152, 219}
\definecolor{mcvert}{RGB}{125, 194, 70}
\definecolor{mcmauve}{RGB}{154, 0, 215}
\definecolor{mcorange}{RGB}{255, 96, 0}
\definecolor{mcturquoise}{RGB}{0, 153, 153}
\definecolor{mcrouge}{RGB}{255, 0, 0}
\definecolor{mclightvert}{RGB}{205, 234, 190}

\definecolor{gris}{RGB}{220, 220, 220}
\definecolor{bleu}{RGB}{52, 152, 219}
\definecolor{vert}{RGB}{125, 194, 70}
\definecolor{mauve}{RGB}{154, 0, 215}
\definecolor{orange}{RGB}{255, 96, 0}
\definecolor{turquoise}{RGB}{0, 153, 153}
\definecolor{rouge}{RGB}{255, 0, 0}
\definecolor{lightvert}{RGB}{205, 234, 190}
\setitemize[0]{label=\color{lightvert}  $\bullet$}

\pagestyle{fancy}
\renewcommand{\headrulewidth}{0.2pt}
\fancyhead[L]{maths-cours.fr}
\fancyhead[R]{\thepage}
\renewcommand{\footrulewidth}{0.2pt}
\fancyfoot[C]{}

\newcolumntype{C}{>{\centering\arraybackslash}X}
\newcolumntype{s}{>{\hsize=.35\hsize\arraybackslash}X}

\setlength{\parindent}{0pt}		 
\setlength{\parskip}{3mm}
\setlength{\headheight}{1cm}

\def\ebook{ebook}
\def\book{book}
\def\web{web}
\def\type{web}

\newcommand{\vect}[1]{\overrightarrow{\,\mathstrut#1\,}}

\def\Oij{$\left(\text{O}~;~\vect{\imath},~\vect{\jmath}\right)$}
\def\Oijk{$\left(\text{O}~;~\vect{\imath},~\vect{\jmath},~\vect{k}\right)$}
\def\Ouv{$\left(\text{O}~;~\vect{u},~\vect{v}\right)$}

\hypersetup{breaklinks=true, colorlinks = true, linkcolor = OliveGreen, urlcolor = OliveGreen, citecolor = OliveGreen, pdfauthor={Didier BONNEL - https://www.maths-cours.fr} } % supprime les bordures autour des liens

\renewcommand{\arg}[0]{\text{arg}}

\everymath{\displaystyle}

%================================================================================================================================
%
% Macros - Commandes
%
%================================================================================================================================

\newcommand\meta[2]{    			% Utilisé pour créer le post HTML.
	\def\titre{titre}
	\def\url{url}
	\def\arg{#1}
	\ifx\titre\arg
		\newcommand\maintitle{#2}
		\fancyhead[L]{#2}
		{\Large\sffamily \MakeUppercase{#2}}
		\vspace{1mm}\textcolor{mcvert}{\hrule}
	\fi 
	\ifx\url\arg
		\fancyfoot[L]{\href{https://www.maths-cours.fr#2}{\black \footnotesize{https://www.maths-cours.fr#2}}}
	\fi 
}


\newcommand\TitreC[1]{    		% Titre centré
     \needspace{3\baselineskip}
     \begin{center}\textbf{#1}\end{center}
}

\newcommand\newpar{    		% paragraphe
     \par
}

\newcommand\nosp {    		% commande vide (pas d'espace)
}
\newcommand{\id}[1]{} %ignore

\newcommand\boite[2]{				% Boite simple sans titre
	\vspace{5mm}
	\setlength{\fboxrule}{0.2mm}
	\setlength{\fboxsep}{5mm}	
	\fcolorbox{#1}{#1!3}{\makebox[\linewidth-2\fboxrule-2\fboxsep]{
  		\begin{minipage}[t]{\linewidth-2\fboxrule-4\fboxsep}\setlength{\parskip}{3mm}
  			 #2
  		\end{minipage}
	}}
	\vspace{5mm}
}

\newcommand\CBox[4]{				% Boites
	\vspace{5mm}
	\setlength{\fboxrule}{0.2mm}
	\setlength{\fboxsep}{5mm}
	
	\fcolorbox{#1}{#1!3}{\makebox[\linewidth-2\fboxrule-2\fboxsep]{
		\begin{minipage}[t]{1cm}\setlength{\parskip}{3mm}
	  		\textcolor{#1}{\LARGE{#2}}    
 	 	\end{minipage}  
  		\begin{minipage}[t]{\linewidth-2\fboxrule-4\fboxsep}\setlength{\parskip}{3mm}
			\raisebox{1.2mm}{\normalsize\sffamily{\textcolor{#1}{#3}}}						
  			 #4
  		\end{minipage}
	}}
	\vspace{5mm}
}

\newcommand\cadre[3]{				% Boites convertible html
	\par
	\vspace{2mm}
	\setlength{\fboxrule}{0.1mm}
	\setlength{\fboxsep}{5mm}
	\fcolorbox{#1}{white}{\makebox[\linewidth-2\fboxrule-2\fboxsep]{
  		\begin{minipage}[t]{\linewidth-2\fboxrule-4\fboxsep}\setlength{\parskip}{3mm}
			\raisebox{-2.5mm}{\sffamily \small{\textcolor{#1}{\MakeUppercase{#2}}}}		
			\par		
  			 #3
 	 		\end{minipage}
	}}
		\vspace{2mm}
	\par
}

\newcommand\bloc[3]{				% Boites convertible html sans bordure
     \needspace{2\baselineskip}
     {\sffamily \small{\textcolor{#1}{\MakeUppercase{#2}}}}    
		\par		
  			 #3
		\par
}

\newcommand\CHelp[1]{
     \CBox{Plum}{\faInfoCircle}{À RETENIR}{#1}
}

\newcommand\CUp[1]{
     \CBox{NavyBlue}{\faThumbsOUp}{EN PRATIQUE}{#1}
}

\newcommand\CInfo[1]{
     \CBox{Sepia}{\faArrowCircleRight}{REMARQUE}{#1}
}

\newcommand\CRedac[1]{
     \CBox{PineGreen}{\faEdit}{BIEN R\'EDIGER}{#1}
}

\newcommand\CError[1]{
     \CBox{Red}{\faExclamationTriangle}{ATTENTION}{#1}
}

\newcommand\TitreExo[2]{
\needspace{4\baselineskip}
 {\sffamily\large EXERCICE #1\ (\emph{#2 points})}
\vspace{5mm}
}

\newcommand\img[2]{
          \includegraphics[width=#2\paperwidth]{\imgdir#1}
}

\newcommand\imgsvg[2]{
       \begin{center}   \includegraphics[width=#2\paperwidth]{\imgsvgdir#1} \end{center}
}


\newcommand\Lien[2]{
     \href{#1}{#2 \tiny \faExternalLink}
}
\newcommand\mcLien[2]{
     \href{https~://www.maths-cours.fr/#1}{#2 \tiny \faExternalLink}
}

\newcommand{\euro}{\eurologo{}}

%================================================================================================================================
%
% Macros - Environement
%
%================================================================================================================================

\newenvironment{tex}{ %
}
{%
}

\newenvironment{indente}{ %
	\setlength\parindent{10mm}
}

{
	\setlength\parindent{0mm}
}

\newenvironment{corrige}{%
     \needspace{3\baselineskip}
     \medskip
     \textbf{\textsc{Corrigé}}
     \medskip
}
{
}

\newenvironment{extern}{%
     \begin{center}
     }
     {
     \end{center}
}

\NewEnviron{code}{%
	\par
     \boite{gray}{\texttt{%
     \BODY
     }}
     \par
}

\newenvironment{vbloc}{% boite sans cadre empeche saut de page
     \begin{minipage}[t]{\linewidth}
     }
     {
     \end{minipage}
}
\NewEnviron{h2}{%
    \needspace{3\baselineskip}
    \vspace{0.6cm}
	\noindent \MakeUppercase{\sffamily \large \BODY}
	\vspace{1mm}\textcolor{mcgris}{\hrule}\vspace{0.4cm}
	\par
}{}

\NewEnviron{h3}{%
    \needspace{3\baselineskip}
	\vspace{5mm}
	\textsc{\BODY}
	\par
}

\NewEnviron{margeneg}{ %
\begin{addmargin}[-1cm]{0cm}
\BODY
\end{addmargin}
}

\NewEnviron{html}{%
}

\begin{document}
\meta{url}{/methode/ensemble-de-points-dont-affixe-verifie-condition/}
\meta{pid}{5765}
\meta{pi_}{5765}
\meta{titre}{Ensemble de points dont l'affixe vérifie une condition}
\meta{type}{methode}
\cadre{bleu}{Situation}{%
     On vous demande de trouver l'ensemble des points $M$ du plan complexe dont l'affixe $z$ vérifie une certaine condition.
}
\bloc{orange}{Exemples}{% id="e010"
     \begin{enumerate}
          \item
          Déterminer l'ensemble des points $M$ d'affixe $z$ tels que $ \frac{z-1}{z+i} $ soit un imaginaire pur.
          \item
          Déterminer l'ensemble des points $M$ d'affixe $z$ tels que $\left| z-1+i\right| =1 $.
     \end{enumerate}
}
\begin{h2}1 - Méthode algébrique\end{h2}
\cadre{rouge}{Méthode}{% id="m020"
     On pose $z=x+iy$ (avec $x \in \mathbb{R},\ y \in\mathbb{R}$) dans la condition et l'on essaie de se ramener à une équation cartésienne.
}
\bloc{cyan}{Rappels}{% id="r030"
     <ol>
     \item
     Une équation cartésienne d'une droite dans le plan est de la forme~:
     \begin{center}$ax+by+c=0$\end{center}
     \item
     Une équation cartésienne du cercle de centre $ \Omega (x_0\~;\ y_0)$ et de rayon $r$ est~:
     \begin{center}$(x-x_0)^2+(y-y_0)^2=r^2$\end{center}
\end{enumerate}
}
\bloc{orange}{Exemple 1}{% id="e040"
<div class="note">Déterminer l'ensemble des points $M$ d'affixe $z$ tels que $ \frac{z-1}{z+i} $ soit un imaginaire pur.}
Tout d'abord notons que $ \frac{z-1}{z+i} $ n'est défini que si $z \neq -i$.
\par
Pour $z \neq -i$, on pose $z=x+iy$~:
\par
$ \frac{z-1}{z+i}=\frac{x+iy-1}{x+iy+i}=\frac{x+iy-1}{x+i(y+1)}$
\par
On multiplie le numérateur et le dénominateur par le conjugué du dénominateur~:
\par
${\frac{z-1}{z+i}}=\frac{(x+iy-1)(x-iy-i)}{(x+i(y+1))(x-i(y+1))} $
\par
$\phantom {\frac{z-1}{z+i}}=\frac{x^2-ixy-ix+ixy+y^2+y-x+iy+i}{x^2+(y+1)^2} $
\par
On réduit et on sépare partie réelle et partie imaginaire~:
\par
${\frac{z-1}{z+i}}=\frac{x^2+y^2-x+y}{x^2+(y+1)^2}+i\frac{-x+y+1}{x^2+(y+1)^2} $
\par
${\frac{z-1}{z+i}}$ est un imaginaire pur \textbf{si et seulement si sa partie réelle est nulle} donc si et seulement si $x^2+y^2-x+y = 0$.
\par
En remarquant que $x^2-x$ est le début de l'identité remarquable $\left(x- \frac{1}{2}\right)^2 $ on trouve $x^2-x=\left(x- \frac{1}{2}\right)^2 -\frac{1}{4}$.
\par
De même, $y^2+y=\left(y+ \frac{1}{2}\right)^2 -\frac{1}{4}$.
\par
L'équation $x^2+y^2-x+y = 0$ est donc équivalente à~:
\par
$\left(x- \frac{1}{2}\right)^2 -\frac{1}{4}$$+\left(y+ \frac{1}{2}\right)^2 -\frac{1}{4} = 0$
\par
$\left(x- \frac{1}{2}\right)^2 $$+\left(y+ \frac{1}{2}\right)^2 =\frac{1}{2}$
\par
D'après le rappel, l'ensemble cherché est donc le cercle de centre $ \Omega \left( \frac{1}{2}\~; \ -\frac{1}{2}\right) $ et de rayon $r=\frac{1}{ \sqrt{2} }=\frac{ \sqrt{2} }{2}$. Il faut toutefois retrancher à cet ensemble le point $A$ d'affixe $-i$ pour lequel le rapport ${\frac{z-1}{z+i}}$ n'est pas défini.
<img src="/wp-content/uploads/methode-lieu-geometrique.svg" alt="lieu géométrique" class="aligncenter" style="width:450px;"/>
On pourrait également traiter l'exemple 2 de manière algébrique. On va voir que la méthode géométrique est plus plus rapide et nécessite moins de calculs.
}
\begin{h2}2 - Méthode géométrique\end{h2}
\cadre{rouge}{Méthode}{% id="m020"
     Cette méthode est particulièrement adaptée au cas où la condition est exprimée à l'aide de module.
     \par
     On utilise le résultat suivant~:
     \par
     Si $M$ est un point d'affixe $z$ et $A$ un point d'affixe $a$ alors $|z-a|$ s'interprète géométriquement comme la distance $AM$.
     \par
     La condition imposée peut alors s'interpréter en terme de distance.
}
\bloc{cyan}{Rappels}{% id="r050"
     <ol>
     \item
     L'ensemble des points du plan tels que $AM=BM$ est la médiatrice du segment $[AB]$
     \item
     L'ensemble des points du plan tels que $AM=r$ est~:
     \begin{itemize}
          \item
          le cercle de centre $A$ et de rayon $r$ si $r > 0$
          \item
          le point $A$ si $r = 0$
          \item
          l'ensemble vide si $r < 0$
     \end{itemize}
\end{enumerate}
}
\bloc{orange}{Exemple 2}{% id="e040"
     Déterminer l'ensemble des points $M$ d'affixe $z$ tels que $\left| z-1+i\right| =1 $.
     \par
     On écrit $\left| z-1+i\right| $ $= \left| z-(1-i)\right| $ (qui est de la forme $|z-a|$ avec $a=1-i$).
     \par
     On place le point $A$ d'affixe $1-i$. On a alors $\left| z-(1-i)\right| = AM $.
     \par
     La condition $\left| z-1+i\right| =1 $ s'écrit alors $AM = 1$. L'ensemble cherché est donc le cercle de centre $A(1-i)$ et de rayon $1$.
     <img src="/wp-content/uploads/methode-lieu-geometrique2.svg" alt="méthode lieu géométrique" class="aligncenter" style="width:450px;"/>
}

\end{document}
µ
\documentclass[a4paper]{article}

%================================================================================================================================
%
% Packages
%
%================================================================================================================================

\usepackage[T1]{fontenc} 	% pour caractères accentués
\usepackage[utf8]{inputenc}  % encodage utf8
\usepackage[french]{babel}	% langue : français
\usepackage{fourier}			% caractères plus lisibles
\usepackage[dvipsnames]{xcolor} % couleurs
\usepackage{fancyhdr}		% réglage header footer
\usepackage{needspace}		% empêcher sauts de page mal placés
\usepackage{graphicx}		% pour inclure des graphiques
\usepackage{enumitem,cprotect}		% personnalise les listes d'items (nécessaire pour ol, al ...)
\usepackage{hyperref}		% Liens hypertexte
\usepackage{pstricks,pst-all,pst-node,pstricks-add,pst-math,pst-plot,pst-tree,pst-eucl} % pstricks
\usepackage[a4paper,includeheadfoot,top=2cm,left=3cm, bottom=2cm,right=3cm]{geometry} % marges etc.
\usepackage{comment}			% commentaires multilignes
\usepackage{amsmath,environ} % maths (matrices, etc.)
\usepackage{amssymb,makeidx}
\usepackage{bm}				% bold maths
\usepackage{tabularx}		% tableaux
\usepackage{colortbl}		% tableaux en couleur
\usepackage{fontawesome}		% Fontawesome
\usepackage{environ}			% environment with command
\usepackage{fp}				% calculs pour ps-tricks
\usepackage{multido}			% pour ps tricks
\usepackage[np]{numprint}	% formattage nombre
\usepackage{tikz,tkz-tab} 			% package principal TikZ
\usepackage{pgfplots}   % axes
\usepackage{mathrsfs}    % cursives
\usepackage{calc}			% calcul taille boites
\usepackage[scaled=0.875]{helvet} % font sans serif
\usepackage{svg} % svg
\usepackage{scrextend} % local margin
\usepackage{scratch} %scratch
\usepackage{multicol} % colonnes
%\usepackage{infix-RPN,pst-func} % formule en notation polanaise inversée
\usepackage{listings}

%================================================================================================================================
%
% Réglages de base
%
%================================================================================================================================

\lstset{
language=Python,   % R code
literate=
{á}{{\'a}}1
{à}{{\`a}}1
{ã}{{\~a}}1
{é}{{\'e}}1
{è}{{\`e}}1
{ê}{{\^e}}1
{í}{{\'i}}1
{ó}{{\'o}}1
{õ}{{\~o}}1
{ú}{{\'u}}1
{ü}{{\"u}}1
{ç}{{\c{c}}}1
{~}{{ }}1
}


\definecolor{codegreen}{rgb}{0,0.6,0}
\definecolor{codegray}{rgb}{0.5,0.5,0.5}
\definecolor{codepurple}{rgb}{0.58,0,0.82}
\definecolor{backcolour}{rgb}{0.95,0.95,0.92}

\lstdefinestyle{mystyle}{
    backgroundcolor=\color{backcolour},   
    commentstyle=\color{codegreen},
    keywordstyle=\color{magenta},
    numberstyle=\tiny\color{codegray},
    stringstyle=\color{codepurple},
    basicstyle=\ttfamily\footnotesize,
    breakatwhitespace=false,         
    breaklines=true,                 
    captionpos=b,                    
    keepspaces=true,                 
    numbers=left,                    
xleftmargin=2em,
framexleftmargin=2em,            
    showspaces=false,                
    showstringspaces=false,
    showtabs=false,                  
    tabsize=2,
    upquote=true
}

\lstset{style=mystyle}


\lstset{style=mystyle}
\newcommand{\imgdir}{C:/laragon/www/newmc/assets/imgsvg/}
\newcommand{\imgsvgdir}{C:/laragon/www/newmc/assets/imgsvg/}

\definecolor{mcgris}{RGB}{220, 220, 220}% ancien~; pour compatibilité
\definecolor{mcbleu}{RGB}{52, 152, 219}
\definecolor{mcvert}{RGB}{125, 194, 70}
\definecolor{mcmauve}{RGB}{154, 0, 215}
\definecolor{mcorange}{RGB}{255, 96, 0}
\definecolor{mcturquoise}{RGB}{0, 153, 153}
\definecolor{mcrouge}{RGB}{255, 0, 0}
\definecolor{mclightvert}{RGB}{205, 234, 190}

\definecolor{gris}{RGB}{220, 220, 220}
\definecolor{bleu}{RGB}{52, 152, 219}
\definecolor{vert}{RGB}{125, 194, 70}
\definecolor{mauve}{RGB}{154, 0, 215}
\definecolor{orange}{RGB}{255, 96, 0}
\definecolor{turquoise}{RGB}{0, 153, 153}
\definecolor{rouge}{RGB}{255, 0, 0}
\definecolor{lightvert}{RGB}{205, 234, 190}
\setitemize[0]{label=\color{lightvert}  $\bullet$}

\pagestyle{fancy}
\renewcommand{\headrulewidth}{0.2pt}
\fancyhead[L]{maths-cours.fr}
\fancyhead[R]{\thepage}
\renewcommand{\footrulewidth}{0.2pt}
\fancyfoot[C]{}

\newcolumntype{C}{>{\centering\arraybackslash}X}
\newcolumntype{s}{>{\hsize=.35\hsize\arraybackslash}X}

\setlength{\parindent}{0pt}		 
\setlength{\parskip}{3mm}
\setlength{\headheight}{1cm}

\def\ebook{ebook}
\def\book{book}
\def\web{web}
\def\type{web}

\newcommand{\vect}[1]{\overrightarrow{\,\mathstrut#1\,}}

\def\Oij{$\left(\text{O}~;~\vect{\imath},~\vect{\jmath}\right)$}
\def\Oijk{$\left(\text{O}~;~\vect{\imath},~\vect{\jmath},~\vect{k}\right)$}
\def\Ouv{$\left(\text{O}~;~\vect{u},~\vect{v}\right)$}

\hypersetup{breaklinks=true, colorlinks = true, linkcolor = OliveGreen, urlcolor = OliveGreen, citecolor = OliveGreen, pdfauthor={Didier BONNEL - https://www.maths-cours.fr} } % supprime les bordures autour des liens

\renewcommand{\arg}[0]{\text{arg}}

\everymath{\displaystyle}

%================================================================================================================================
%
% Macros - Commandes
%
%================================================================================================================================

\newcommand\meta[2]{    			% Utilisé pour créer le post HTML.
	\def\titre{titre}
	\def\url{url}
	\def\arg{#1}
	\ifx\titre\arg
		\newcommand\maintitle{#2}
		\fancyhead[L]{#2}
		{\Large\sffamily \MakeUppercase{#2}}
		\vspace{1mm}\textcolor{mcvert}{\hrule}
	\fi 
	\ifx\url\arg
		\fancyfoot[L]{\href{https://www.maths-cours.fr#2}{\black \footnotesize{https://www.maths-cours.fr#2}}}
	\fi 
}


\newcommand\TitreC[1]{    		% Titre centré
     \needspace{3\baselineskip}
     \begin{center}\textbf{#1}\end{center}
}

\newcommand\newpar{    		% paragraphe
     \par
}

\newcommand\nosp {    		% commande vide (pas d'espace)
}
\newcommand{\id}[1]{} %ignore

\newcommand\boite[2]{				% Boite simple sans titre
	\vspace{5mm}
	\setlength{\fboxrule}{0.2mm}
	\setlength{\fboxsep}{5mm}	
	\fcolorbox{#1}{#1!3}{\makebox[\linewidth-2\fboxrule-2\fboxsep]{
  		\begin{minipage}[t]{\linewidth-2\fboxrule-4\fboxsep}\setlength{\parskip}{3mm}
  			 #2
  		\end{minipage}
	}}
	\vspace{5mm}
}

\newcommand\CBox[4]{				% Boites
	\vspace{5mm}
	\setlength{\fboxrule}{0.2mm}
	\setlength{\fboxsep}{5mm}
	
	\fcolorbox{#1}{#1!3}{\makebox[\linewidth-2\fboxrule-2\fboxsep]{
		\begin{minipage}[t]{1cm}\setlength{\parskip}{3mm}
	  		\textcolor{#1}{\LARGE{#2}}    
 	 	\end{minipage}  
  		\begin{minipage}[t]{\linewidth-2\fboxrule-4\fboxsep}\setlength{\parskip}{3mm}
			\raisebox{1.2mm}{\normalsize\sffamily{\textcolor{#1}{#3}}}						
  			 #4
  		\end{minipage}
	}}
	\vspace{5mm}
}

\newcommand\cadre[3]{				% Boites convertible html
	\par
	\vspace{2mm}
	\setlength{\fboxrule}{0.1mm}
	\setlength{\fboxsep}{5mm}
	\fcolorbox{#1}{white}{\makebox[\linewidth-2\fboxrule-2\fboxsep]{
  		\begin{minipage}[t]{\linewidth-2\fboxrule-4\fboxsep}\setlength{\parskip}{3mm}
			\raisebox{-2.5mm}{\sffamily \small{\textcolor{#1}{\MakeUppercase{#2}}}}		
			\par		
  			 #3
 	 		\end{minipage}
	}}
		\vspace{2mm}
	\par
}

\newcommand\bloc[3]{				% Boites convertible html sans bordure
     \needspace{2\baselineskip}
     {\sffamily \small{\textcolor{#1}{\MakeUppercase{#2}}}}    
		\par		
  			 #3
		\par
}

\newcommand\CHelp[1]{
     \CBox{Plum}{\faInfoCircle}{À RETENIR}{#1}
}

\newcommand\CUp[1]{
     \CBox{NavyBlue}{\faThumbsOUp}{EN PRATIQUE}{#1}
}

\newcommand\CInfo[1]{
     \CBox{Sepia}{\faArrowCircleRight}{REMARQUE}{#1}
}

\newcommand\CRedac[1]{
     \CBox{PineGreen}{\faEdit}{BIEN R\'EDIGER}{#1}
}

\newcommand\CError[1]{
     \CBox{Red}{\faExclamationTriangle}{ATTENTION}{#1}
}

\newcommand\TitreExo[2]{
\needspace{4\baselineskip}
 {\sffamily\large EXERCICE #1\ (\emph{#2 points})}
\vspace{5mm}
}

\newcommand\img[2]{
          \includegraphics[width=#2\paperwidth]{\imgdir#1}
}

\newcommand\imgsvg[2]{
       \begin{center}   \includegraphics[width=#2\paperwidth]{\imgsvgdir#1} \end{center}
}


\newcommand\Lien[2]{
     \href{#1}{#2 \tiny \faExternalLink}
}
\newcommand\mcLien[2]{
     \href{https~://www.maths-cours.fr/#1}{#2 \tiny \faExternalLink}
}

\newcommand{\euro}{\eurologo{}}

%================================================================================================================================
%
% Macros - Environement
%
%================================================================================================================================

\newenvironment{tex}{ %
}
{%
}

\newenvironment{indente}{ %
	\setlength\parindent{10mm}
}

{
	\setlength\parindent{0mm}
}

\newenvironment{corrige}{%
     \needspace{3\baselineskip}
     \medskip
     \textbf{\textsc{Corrigé}}
     \medskip
}
{
}

\newenvironment{extern}{%
     \begin{center}
     }
     {
     \end{center}
}

\NewEnviron{code}{%
	\par
     \boite{gray}{\texttt{%
     \BODY
     }}
     \par
}

\newenvironment{vbloc}{% boite sans cadre empeche saut de page
     \begin{minipage}[t]{\linewidth}
     }
     {
     \end{minipage}
}
\NewEnviron{h2}{%
    \needspace{3\baselineskip}
    \vspace{0.6cm}
	\noindent \MakeUppercase{\sffamily \large \BODY}
	\vspace{1mm}\textcolor{mcgris}{\hrule}\vspace{0.4cm}
	\par
}{}

\NewEnviron{h3}{%
    \needspace{3\baselineskip}
	\vspace{5mm}
	\textsc{\BODY}
	\par
}

\NewEnviron{margeneg}{ %
\begin{addmargin}[-1cm]{0cm}
\BODY
\end{addmargin}
}

\NewEnviron{html}{%
}

\begin{document}
\meta{url}{/cours/introduction-aux-matrices-spe/}
\meta{pid}{5846}
\meta{titre}{Introduction aux matrices [spé]}
\meta{type}{cours}
\begin{h2}1. Définitions\end{h2}
\cadre{bleu}{Définition}{%id="d10"
     Une \textbf{matrice} de dimension (ou d'\textit{ordre} or de \textit{taille}) $n\times p$ est un tableau de nombres réels (appelés coefficients ou termes) comportant $n$ lignes et $p$ colonnes.
     \par
     Si on désigne par $a_{ij}$ le coefficient situé à la $i$-ième ligne et la $j$-ième colonne la matrice s'écrira :
     \begin{center}
          $ A=\begin{pmatrix}  a_{11} & a_{12} & \ldots & a_{1p}\\ a_{21} & a_{22} & \ldots & a_{2p}  \\ \vdots & \vdots & \ddots & \vdots\\ a_{n1} & a_{n2} & \ldots & a_{np} \end{pmatrix}.$
     \end{center}
}
\bloc{orange}{Exemple}{%id="e10"
     La matrice $A=\begin{pmatrix} 1 &amp; 2 & 3 \\ 4 & 5 &amp; 6 \end{pmatrix}$ est une matrice de dimension $2\times 3$.
}
\bloc{cyan}{Notations}{%id="r10"
     On notera, en abrégé, $A=\left(a_{ij}\right)$ la matrice dont le coefficient situé à la $i$-ème ligne et la $j$-ième colonne est $a_{ij}$.
}
\cadre{bleu}{Définitions}{%id="d20"
     \begin{itemize}
          \item Une matrice \textbf{carrée} est une matrice dont le nombre de lignes est égal au nombre de colonnes.
          \item Une matrice \textbf{ligne} est une matrice dont le nombre de lignes est égal à $1$.
          \item Une matrice \textbf{colonne} est une matrice dont le nombre de colonnes est égal à $1$.
     \end{itemize}
}
\bloc{orange}{Exemples}{%id="e20"
     \begin{itemize}
          \item La matrice $A=\begin{pmatrix} 1 & 2 \\ 1 & 2 \end{pmatrix}$ est une matrice carrée (de dimension $2\times 2$ - ou on peut dire, plus simplement, de dimension 2).
          \item La matrice $B=\begin{pmatrix}1 & 2 & 0,5 \end{pmatrix}$ est une matrice ligne (de dimension $1\times 3$).
          \item La matrice $C=\begin{pmatrix} 1 \\ 2 \\ 0 \\ 4 \end{pmatrix}$ est une matrice colonne (de dimension $4\times 1$).
     \end{itemize}
}
\bloc{cyan}{Remarque}{%id="r20"
     Pour une matrice carrée, on appelle \textbf{diagonale principale}, la diagonale qui relie le coin situé en haut à gauche au coin situé en bas à droite. Sur l'exemple ci-dessous, les coefficients de la diagonale principale sont marqués en rouge :
     \begin{center}$A=\begin{pmatrix} \color{red}{1} & 2 & 3 & 4 \\ 2 & \color{red}{3} & 4 & 5 \\ 3 & 4 & \color{red}{5} & 6 \\ 4 & 5 & 6 & \color{red}{7} \end{pmatrix}$.\end{center}
}
\cadre{bleu}{Définitions}{%id="d30"
     \begin{itemize}
          \item La matrice \textbf{nulle} de dimension $n\times p$ est la matrice de dimension $n\times p$ dont tous les coefficients sont nuls.
          \item Une matrice \textbf{diagonale} est une matrice carrée dont tout les coefficients situés en dehors de la diagonale principale sont nuls.
          \item La matrice \textbf{unité} de dimension $n$ est la matrice carrée de dimension $n$ qui contient des $1$ sur la diagonale principale et des $0$ ailleurs :
          \begin{center}
               $ A=\begin{pmatrix}  1 & 0 & \ldots & 0\\ 0 & 1 & \ldots & 0\\ \vdots & \vdots & \ddots & \vdots\\ 0 & 0 & \ldots & 1 \end{pmatrix}$.
          \end{center}
     \end{itemize}
}
\bloc{orange}{Exemples}{%id="e30"
     \begin{itemize}
          \item La matrice $A=\begin{pmatrix} 1 & 0 & 0 & 0 \\ 0 & 2 & 0 & 0 \\ 0 & 0 & 0 & 0 \\ 0 & 0 & 0 & 1 \end{pmatrix}$ est une matrice diagonale d'ordre 4.
          \item La matrice unité d'ordre 2 est $I_{2}=\begin{pmatrix} 1 & 0 \\ 0 & 1 \end{pmatrix}$.
     \end{itemize}
}
\begin{h2}2. Opérations sur les matrices\end{h2}
\cadre{bleu}{Définition (Somme de matrices)}{%id="d50"
     Soient $A$ et $B$ deux matrices de même dimension.
     \par
     La somme $A+B$ des matrices $A$ et $B$ s'obtient en ajoutant les coefficients de $A$ aux coefficients de $B$ situés \textbf{à la même position}.
}
\bloc{orange}{Exemple}{%id="e50"
     Soient $A=\begin{pmatrix} 2 & -2 & 1 \\ -1 & 1 & 0 \end{pmatrix}$ et $B=\begin{pmatrix} -1 & 1 & 1 \\ -2 & 2 & 0 \end{pmatrix}$.
     \par
     Alors :
     \par
     $A+B=\begin{pmatrix}2-1&-2+1&1+1\\-1-2&1+2&0+0\end{pmatrix}=\begin{pmatrix}1&-1&2\\-3&3&0\end{pmatrix}$.
}
\bloc{cyan}{Remarques}{%id="r50"
     \begin{itemize}
          \item On ne peut additionner deux matrices que si elles ont les même dimensions, c'est à dire le même nombre de lignes et le même nombre de colonnes.
          \item On définit de manière analogue la différence de deux matrices.
     \end{itemize}
}
\cadre{bleu}{Définition (Produit d'une matrice par un nombre réel)}{%id="d60"
     Soient $A$ une matrice et $k$ un nombre réel..
     \par
     Le produit $kA$ est la matrice obtenue en multipliant chacun des coefficients de $A$ par $k$.
}
\bloc{orange}{Exemple}{%id="e60"
     Si $A=\begin{pmatrix} 1 & 1 & 0 \\ 2 & 0 & 0 \end{pmatrix}$ alors :
     \begin{itemize}
          \item $2A=\begin{pmatrix} 2\times 1 & 2\times 1 & 2\times 0 \\ 2\times 2 & 2\times 0 & 2\times 0\end{pmatrix}=\begin{pmatrix}2 & 2 & 0 \\ 4 & 0 & 0\end{pmatrix}$.
          \item $-A=-1\times A=\begin{pmatrix} -1 & -1 & 0 \\ -2 & 0 & 0 \end{pmatrix}$.
     \end{itemize}
}
\cadre{vert}{Propriétés}{%id="p65"
     Soient $A$, $B$ et $C$ trois matrices de mêmes dimensions et $k$ et $k^{\prime}$ deux réels.
     \begin{itemize}
          \item $A+B = B+A $ (commutativité de l'addition) ;
          \item $\left(A+B\right)+C = A+\left(B+C\right)$ (associativité de l'addition) ;
          \item $k\left(A+B\right) = kA+kB$ ;
          \item $\left(k+k^{\prime}\right)A = kA+k^{\prime}A$ ;
          \item $k\left(k^{\prime}A\right) = \left(kk^{\prime}\right)A$.
     \end{itemize}
}
\cadre{bleu}{Définition (Produit d'une matrice ligne par une matrice colonne)}{%id="d70"
     Soient $A=\left(a_{1} a_{2} \cdots a_{n}\right)$ une matrice ligne $1\times n$ et $B=\begin{pmatrix} b_{1} \\ b_{2} \\ \cdots \\ b_{n} \end{pmatrix}$ une matrice colonne $n\times 1$. Le produit de $A$ par $B$ est le nombre réel :
     \par
     $A\times B = \left(a_{1} a_{2} \cdots a_{n}\right)\times \begin{pmatrix} b_{1} \\ b_{2} \\ \cdots \\ b_{n} \end{pmatrix} = a_{1}b_{1} + a_{2}b_{2} + \cdots + a_{n}b_{n}$.
}
\bloc{cyan}{Remarque}{%id="r70"
     \begin{itemize}
          \item Les deux matrices $A$ et $B$ doivent avoir le même nombre $n$ de coefficients.
          \item Pour cette formule, la matrice ligne doit être impérativement en premier !
     \end{itemize}
}
\bloc{orange}{Exemple}{%id="e70"
     Si $ A=\left(1 2 3 4\right) $ et $ B=\begin{pmatrix} 5 \\ 6 \\ 7 \\ 8 \end{pmatrix}$ :
     \par
     $A\times B = 1\times 5 + 2\times 6 + 3\times 7 + 4\times 8 = 5 + 12 + 21 + 32 = 70$.
}
\cadre{bleu}{Définition (Produit de deux matrices)}{%id="d80"
     Soient $A=\left(a_{ij}\right)$ une matrice $n\times p$ et $B=\left(b_{ij}\right)$ une matrice $p\times q$. Le produit de $A$ par $B$ est la matrice $C=\left(c_{ij}\right)$ à $n$ lignes et $q$ colonnes dont le coefficient situé à la $i$-ième ligne et la $j$-ième colonne est obtenu en multipliant la $i$-ième ligne de A par la $j$-ième colonne de B.
     \par
     C'est à dire que pour tout $1 \leqslant i \leqslant n$ et tout $1 \leqslant j \leqslant q$ :
     \begin{center}$c_{ij} = a_{i1}b_{1j} + a_{i2}b_{2j} + \cdots + a_{ip}b_{pj}$.\end{center}
}
\bloc{cyan}{Remarque}{%id="r80"
     Faites bien attention aux dimensions des matrices : Le nombre de colonnes de la première matrice doit être égal au nombre de lignes de la seconde pour que le calcul soit possible.
     \par
     Par exemple, le produit d'une matrice $2\times \color{red}{3}$ par une matrice $\color{red}{3}\times 4$ est possible et donnera une matrice $2\times 4$.
     \par
     Par contre, le produit d'une matrice $2\times \color{red}{3}$ par une matrice $\color{red}{2}\times 3$ n'est pas possible.
}
\bloc{orange}{Exemple}{%id="e80"
     Calculons le produit $C=A\times B$ avec :
     \par
     $A=\begin{pmatrix} 2 & 4 \\ 1 & 0 \end{pmatrix} $ et $ B=\begin{pmatrix} -1 & 0 & 2 \\ -2 & 1 & 0 \end{pmatrix}$.
     \par
     Ce calcul est possible car le nombre de colonnes de $A$ est égal au nombre de lignes de $B$. Le résultat $C$ sera une matrice $2\times 3$ ($\color{red}{2}\times 2 $par$ 2\times \color{red}{3} \rightarrow \color{red}{2}\times \color{red}{3}$).
     \par
     Notons $C=\begin{pmatrix} c_{11} & c_{12} & c_{13} \\ c_{21} & c_{22} & c_{23} \end{pmatrix}$.
     \par
     Pour calculer $c_{11}$ on multiplie la première ligne de $A$ et la première colonne de $B$ :
     \par
     $C=\begin{pmatrix} \color{red}{2} & \color{red}{4} \\ 1 & 0\end{pmatrix}\times \begin{pmatrix} \color{red}{-1} & 0 & 2 \\ \color{red}{-2} & 1 & 0\end{pmatrix}$ ;
     \par
     on a donc $c_{11}=2\times \left(-1\right)+4\times \left(-2\right)=-2-8=-10$.
     \par
     $C=\begin{pmatrix} \color{red}{2} & \color{red}{4} \\ 1 & 0 \end{pmatrix}\times \begin{pmatrix}\color{red}{-1} & 0 & 2 \\ \color{red}{-2} & 1 & 0 \end{pmatrix}=\begin{pmatrix}\color{red}{-10} & \cdots & \cdots \\ \cdots & \cdots & \cdots \end{pmatrix}$.
     \par
     Pour calculer $c_{12}$ on multiplie la première ligne de $A$ et la seconde colonne de $B$ :
     \par
     $C=\begin{pmatrix} \color{red}{2} & \color{red}{4} \\ 1 & 0\end{pmatrix}\times\begin{pmatrix}-1 & \color{red}{0} & 2 \\ -2 & \color{red}{1} & 0\end{pmatrix}$ ;
     \par
     on a donc $c_{12}=2\times 0+4\times 1=0+4=4$.
     \par
     $C=\begin{pmatrix} \color{red}{2} & \color{red}{4} \\ 1 & 0\end{pmatrix}\times \begin{pmatrix} -1 & \color{red}{0} & 2 \\ -2 & \color{red}{1} & 0\end{pmatrix}=\begin{pmatrix}-10 & \color{red}{4} & \cdots \\ \cdots & \cdots & \cdots \end{pmatrix}$.
     \par
     Et ainsi de suite...
     \par
     Au final on trouve :
     \par
     $C=\begin{pmatrix} 2 & 4 \\ 1 & 0\end{pmatrix}\times \begin{pmatrix}-1 & 0 & 2 \\ -2 & 1 & 0\end{pmatrix}=\begin{pmatrix}-10 & 4 & 4 \\ -1 & 0 & 2 \end{pmatrix}$.
}
Dans ce qui suit, on s'intéressera principalement à des matrices \textbf{carrées}.
\cadre{vert}{Propriété}{%id="p90"
     Soit $A, B$ et $C$, trois matrices carrées de même dimension.
     \begin{itemize}
          \item $A\times \left(B+C\right) = A\times B + A\times C$ (distributivité à gauche)
          \item $\left(A+B\right)\times C = A\times C + B\times C$ (distributivité à droite)
          \item $A\times \left(B\times C\right) = \left(A\times B\right)\times C$ (associativité de la multiplication)
     \end{itemize}
     Par contre en général : $A\times B\neq B\times A$ : la multiplication n'est \textbf{pas} commutative.
}
\bloc{orange}{Exemple}{%id="e90"
     Soit $A=\begin{pmatrix} 0 & 2 \\ 0 & 0 \end{pmatrix}$ et $B=\begin{pmatrix} 0 & 2 \\ 1 & 0 \end{pmatrix}$
     \par
     $A \times B=\begin{pmatrix} 2 & 0 \\ 0 & 0 \end{pmatrix}$
     \par
     tandis que :
     \par
     $B \times A=\begin{pmatrix} 0 & 0 \\ 0 & 2 \end{pmatrix}$
     \par
     Par conséquent $A\times B \neq B\times A$.
}
\cadre{bleu}{Définition (Puissance d'une matrice)}{%id="d100"
     Soit $A$ une matrice carrée et $n$ un entier naturel.
     \par
     On note $A^{n}$ la matrice :
     \begin{center}$A^{n}=A\times A\times \cdots.\times A$ ($n$ facteurs).\end{center}
}
\bloc{cyan}{Remarque}{%id="r100"
     Par convention, on considèrera que $A^{0}$ est la matrice unité de même taille que $A$.
}
\cadre{bleu}{Définition (Matrice inversible)}{%id="d110"
     Une matrice carrée A de dimension $n$ est \textbf{inversible} si et seulement si il existe une
     \par
     matrice $B$ telle que
     \begin{center}$A\times B = B\times A = I_{n}$\end{center}
     où $I_{n}$ est la matrice unité de dimension $n$.
     \par
     La matrice $B$ est appelée \textbf{matrice inverse} de $A$ et notée $A^{-1}$.
}
\begin{h2}3. Résolution de systèmes d'équations\end{h2}
Soit le système :
\par
$\left(S\right) \left\{ \begin{matrix} ax+by=s \\ cx+dy=t \end{matrix}\right.$
\par
d'inconnues $x$ et $y$.
\par
Si l'on pose $A=\begin{pmatrix} a & b \\ c & d \end{pmatrix}$, $X=\begin{pmatrix} x \\ y \end{pmatrix}$ et $B=\begin{pmatrix} s \\ t \end{pmatrix}$, le système $\left(S\right)$ peut s'écrire :
\par
$A\times X=B$. Le théorème ci-dessous permet alors de résoudre ce système.
\cadre{rouge}{Théorème}{%id="t150"
     Soit $A$ une matrice carrée.
     \par
     Si $A$ est inversible, le système $A\times X=B$ admet une solution unique donnée par :
     \begin{center}$X=A^{-1}\times B$.\end{center}
}
\bloc{orange}{Exemple}{%id="e150"
     On cherche à résoudre le système :
     \par
     $\left(S\right) \left\{ \begin{matrix} 3x+4y=1 \\ 5x+7y=2 \end{matrix}\right.$
     \par
     Pour cela on pose : $A=\begin{pmatrix} 3 & 4 \\ 5 & 7 \end{pmatrix}$, $X=\begin{pmatrix} x \\ y \end{pmatrix}$ et $B=\begin{pmatrix} 1 \\ 2 \end{pmatrix}$.
     \par
     L'écriture matricielle est alors $A\times X=B$.
     \par
     A la calculatrice, on trouve que $A$ est inversible d'inverse $A^{-1}=\begin{pmatrix} 7 & -4 \\ -5 & 3 \end{pmatrix}$.
     \par
     La solution du système est donné par :
     \par
     $X=A^{-1}\times B=\begin{pmatrix} 7 & -4 \\ -5 & 3\end{pmatrix}\times \begin{pmatrix}1 \\ 2\end{pmatrix}=\begin{pmatrix}-1 \\ 1 \end{pmatrix}$.
     \par
     C'est à dire $x=-1$ et $y=1$.
}

\end{document}
µ
%================================================================================================================================
%
% Bac blanc ES/L sujet 3 exercice 2
%
%================================================================================================================================
\par
\documentclass[a4paper]{article}

%================================================================================================================================
%
% Packages
%
%================================================================================================================================

\usepackage[T1]{fontenc} 	% pour caractères accentués
\usepackage[utf8]{inputenc}  % encodage utf8
\usepackage[french]{babel}	% langue : français
\usepackage{fourier}			% caractères plus lisibles
\usepackage[dvipsnames]{xcolor} % couleurs
\usepackage{fancyhdr}		% réglage header footer
\usepackage{needspace}		% empêcher sauts de page mal placés
\usepackage{graphicx}		% pour inclure des graphiques
\usepackage{enumitem,cprotect}		% personnalise les listes d'items (nécessaire pour ol, al ...)
\usepackage{hyperref}		% Liens hypertexte
\usepackage{pstricks,pst-all,pst-node,pstricks-add,pst-math,pst-plot,pst-tree,pst-eucl} % pstricks
\usepackage[a4paper,includeheadfoot,top=2cm,left=3cm, bottom=2cm,right=3cm]{geometry} % marges etc.
\usepackage{comment}			% commentaires multilignes
\usepackage{amsmath,environ} % maths (matrices, etc.)
\usepackage{amssymb,makeidx}
\usepackage{bm}				% bold maths
\usepackage{tabularx}		% tableaux
\usepackage{colortbl}		% tableaux en couleur
\usepackage{fontawesome}		% Fontawesome
\usepackage{environ}			% environment with command
\usepackage{fp}				% calculs pour ps-tricks
\usepackage{multido}			% pour ps tricks
\usepackage[np]{numprint}	% formattage nombre
\usepackage{tikz,tkz-tab} 			% package principal TikZ
\usepackage{pgfplots}   % axes
\usepackage{mathrsfs}    % cursives
\usepackage{calc}			% calcul taille boites
\usepackage[scaled=0.875]{helvet} % font sans serif
\usepackage{svg} % svg
\usepackage{scrextend} % local margin
\usepackage{scratch} %scratch
\usepackage{multicol} % colonnes
%\usepackage{infix-RPN,pst-func} % formule en notation polanaise inversée
\usepackage{listings}

%================================================================================================================================
%
% Réglages de base
%
%================================================================================================================================

\lstset{
language=Python,   % R code
literate=
{á}{{\'a}}1
{à}{{\`a}}1
{ã}{{\~a}}1
{é}{{\'e}}1
{è}{{\`e}}1
{ê}{{\^e}}1
{í}{{\'i}}1
{ó}{{\'o}}1
{õ}{{\~o}}1
{ú}{{\'u}}1
{ü}{{\"u}}1
{ç}{{\c{c}}}1
{~}{{ }}1
}


\definecolor{codegreen}{rgb}{0,0.6,0}
\definecolor{codegray}{rgb}{0.5,0.5,0.5}
\definecolor{codepurple}{rgb}{0.58,0,0.82}
\definecolor{backcolour}{rgb}{0.95,0.95,0.92}

\lstdefinestyle{mystyle}{
    backgroundcolor=\color{backcolour},   
    commentstyle=\color{codegreen},
    keywordstyle=\color{magenta},
    numberstyle=\tiny\color{codegray},
    stringstyle=\color{codepurple},
    basicstyle=\ttfamily\footnotesize,
    breakatwhitespace=false,         
    breaklines=true,                 
    captionpos=b,                    
    keepspaces=true,                 
    numbers=left,                    
xleftmargin=2em,
framexleftmargin=2em,            
    showspaces=false,                
    showstringspaces=false,
    showtabs=false,                  
    tabsize=2,
    upquote=true
}

\lstset{style=mystyle}


\lstset{style=mystyle}
\newcommand{\imgdir}{C:/laragon/www/newmc/assets/imgsvg/}
\newcommand{\imgsvgdir}{C:/laragon/www/newmc/assets/imgsvg/}

\definecolor{mcgris}{RGB}{220, 220, 220}% ancien~; pour compatibilité
\definecolor{mcbleu}{RGB}{52, 152, 219}
\definecolor{mcvert}{RGB}{125, 194, 70}
\definecolor{mcmauve}{RGB}{154, 0, 215}
\definecolor{mcorange}{RGB}{255, 96, 0}
\definecolor{mcturquoise}{RGB}{0, 153, 153}
\definecolor{mcrouge}{RGB}{255, 0, 0}
\definecolor{mclightvert}{RGB}{205, 234, 190}

\definecolor{gris}{RGB}{220, 220, 220}
\definecolor{bleu}{RGB}{52, 152, 219}
\definecolor{vert}{RGB}{125, 194, 70}
\definecolor{mauve}{RGB}{154, 0, 215}
\definecolor{orange}{RGB}{255, 96, 0}
\definecolor{turquoise}{RGB}{0, 153, 153}
\definecolor{rouge}{RGB}{255, 0, 0}
\definecolor{lightvert}{RGB}{205, 234, 190}
\setitemize[0]{label=\color{lightvert}  $\bullet$}

\pagestyle{fancy}
\renewcommand{\headrulewidth}{0.2pt}
\fancyhead[L]{maths-cours.fr}
\fancyhead[R]{\thepage}
\renewcommand{\footrulewidth}{0.2pt}
\fancyfoot[C]{}

\newcolumntype{C}{>{\centering\arraybackslash}X}
\newcolumntype{s}{>{\hsize=.35\hsize\arraybackslash}X}

\setlength{\parindent}{0pt}		 
\setlength{\parskip}{3mm}
\setlength{\headheight}{1cm}

\def\ebook{ebook}
\def\book{book}
\def\web{web}
\def\type{web}

\newcommand{\vect}[1]{\overrightarrow{\,\mathstrut#1\,}}

\def\Oij{$\left(\text{O}~;~\vect{\imath},~\vect{\jmath}\right)$}
\def\Oijk{$\left(\text{O}~;~\vect{\imath},~\vect{\jmath},~\vect{k}\right)$}
\def\Ouv{$\left(\text{O}~;~\vect{u},~\vect{v}\right)$}

\hypersetup{breaklinks=true, colorlinks = true, linkcolor = OliveGreen, urlcolor = OliveGreen, citecolor = OliveGreen, pdfauthor={Didier BONNEL - https://www.maths-cours.fr} } % supprime les bordures autour des liens

\renewcommand{\arg}[0]{\text{arg}}

\everymath{\displaystyle}

%================================================================================================================================
%
% Macros - Commandes
%
%================================================================================================================================

\newcommand\meta[2]{    			% Utilisé pour créer le post HTML.
	\def\titre{titre}
	\def\url{url}
	\def\arg{#1}
	\ifx\titre\arg
		\newcommand\maintitle{#2}
		\fancyhead[L]{#2}
		{\Large\sffamily \MakeUppercase{#2}}
		\vspace{1mm}\textcolor{mcvert}{\hrule}
	\fi 
	\ifx\url\arg
		\fancyfoot[L]{\href{https://www.maths-cours.fr#2}{\black \footnotesize{https://www.maths-cours.fr#2}}}
	\fi 
}


\newcommand\TitreC[1]{    		% Titre centré
     \needspace{3\baselineskip}
     \begin{center}\textbf{#1}\end{center}
}

\newcommand\newpar{    		% paragraphe
     \par
}

\newcommand\nosp {    		% commande vide (pas d'espace)
}
\newcommand{\id}[1]{} %ignore

\newcommand\boite[2]{				% Boite simple sans titre
	\vspace{5mm}
	\setlength{\fboxrule}{0.2mm}
	\setlength{\fboxsep}{5mm}	
	\fcolorbox{#1}{#1!3}{\makebox[\linewidth-2\fboxrule-2\fboxsep]{
  		\begin{minipage}[t]{\linewidth-2\fboxrule-4\fboxsep}\setlength{\parskip}{3mm}
  			 #2
  		\end{minipage}
	}}
	\vspace{5mm}
}

\newcommand\CBox[4]{				% Boites
	\vspace{5mm}
	\setlength{\fboxrule}{0.2mm}
	\setlength{\fboxsep}{5mm}
	
	\fcolorbox{#1}{#1!3}{\makebox[\linewidth-2\fboxrule-2\fboxsep]{
		\begin{minipage}[t]{1cm}\setlength{\parskip}{3mm}
	  		\textcolor{#1}{\LARGE{#2}}    
 	 	\end{minipage}  
  		\begin{minipage}[t]{\linewidth-2\fboxrule-4\fboxsep}\setlength{\parskip}{3mm}
			\raisebox{1.2mm}{\normalsize\sffamily{\textcolor{#1}{#3}}}						
  			 #4
  		\end{minipage}
	}}
	\vspace{5mm}
}

\newcommand\cadre[3]{				% Boites convertible html
	\par
	\vspace{2mm}
	\setlength{\fboxrule}{0.1mm}
	\setlength{\fboxsep}{5mm}
	\fcolorbox{#1}{white}{\makebox[\linewidth-2\fboxrule-2\fboxsep]{
  		\begin{minipage}[t]{\linewidth-2\fboxrule-4\fboxsep}\setlength{\parskip}{3mm}
			\raisebox{-2.5mm}{\sffamily \small{\textcolor{#1}{\MakeUppercase{#2}}}}		
			\par		
  			 #3
 	 		\end{minipage}
	}}
		\vspace{2mm}
	\par
}

\newcommand\bloc[3]{				% Boites convertible html sans bordure
     \needspace{2\baselineskip}
     {\sffamily \small{\textcolor{#1}{\MakeUppercase{#2}}}}    
		\par		
  			 #3
		\par
}

\newcommand\CHelp[1]{
     \CBox{Plum}{\faInfoCircle}{À RETENIR}{#1}
}

\newcommand\CUp[1]{
     \CBox{NavyBlue}{\faThumbsOUp}{EN PRATIQUE}{#1}
}

\newcommand\CInfo[1]{
     \CBox{Sepia}{\faArrowCircleRight}{REMARQUE}{#1}
}

\newcommand\CRedac[1]{
     \CBox{PineGreen}{\faEdit}{BIEN R\'EDIGER}{#1}
}

\newcommand\CError[1]{
     \CBox{Red}{\faExclamationTriangle}{ATTENTION}{#1}
}

\newcommand\TitreExo[2]{
\needspace{4\baselineskip}
 {\sffamily\large EXERCICE #1\ (\emph{#2 points})}
\vspace{5mm}
}

\newcommand\img[2]{
          \includegraphics[width=#2\paperwidth]{\imgdir#1}
}

\newcommand\imgsvg[2]{
       \begin{center}   \includegraphics[width=#2\paperwidth]{\imgsvgdir#1} \end{center}
}


\newcommand\Lien[2]{
     \href{#1}{#2 \tiny \faExternalLink}
}
\newcommand\mcLien[2]{
     \href{https~://www.maths-cours.fr/#1}{#2 \tiny \faExternalLink}
}

\newcommand{\euro}{\eurologo{}}

%================================================================================================================================
%
% Macros - Environement
%
%================================================================================================================================

\newenvironment{tex}{ %
}
{%
}

\newenvironment{indente}{ %
	\setlength\parindent{10mm}
}

{
	\setlength\parindent{0mm}
}

\newenvironment{corrige}{%
     \needspace{3\baselineskip}
     \medskip
     \textbf{\textsc{Corrigé}}
     \medskip
}
{
}

\newenvironment{extern}{%
     \begin{center}
     }
     {
     \end{center}
}

\NewEnviron{code}{%
	\par
     \boite{gray}{\texttt{%
     \BODY
     }}
     \par
}

\newenvironment{vbloc}{% boite sans cadre empeche saut de page
     \begin{minipage}[t]{\linewidth}
     }
     {
     \end{minipage}
}
\NewEnviron{h2}{%
    \needspace{3\baselineskip}
    \vspace{0.6cm}
	\noindent \MakeUppercase{\sffamily \large \BODY}
	\vspace{1mm}\textcolor{mcgris}{\hrule}\vspace{0.4cm}
	\par
}{}

\NewEnviron{h3}{%
    \needspace{3\baselineskip}
	\vspace{5mm}
	\textsc{\BODY}
	\par
}

\NewEnviron{margeneg}{ %
\begin{addmargin}[-1cm]{0cm}
\BODY
\end{addmargin}
}

\NewEnviron{html}{%
}

\begin{document}
\meta{url}{/exercices/etude-graphique-suite-arithmetico-geometrique/}
\meta{pid}{6782}
\meta{titre}{\'Etude graphique suite arithmético-géométrique}
\meta{type}{exercices}
%============================================================================================================================
L'objectif de ce problème est d'étudier la convergence de la suite $(u_n)$ définie par $u_0=2$ et pour tout entier naturel $n$~:
\[ u_{n+1} = 0,9u_n+2.\]
%============================================================================================================================
\begin{center}\begin{h3}Partie A\\ \'Etude graphique \end{h3}\end{center}
Sur le graphique fourni en Annexe, on a représenté les droites $D$ et $\Delta$ d'équations respectives $y=0,9x+2$ et $y=x$.
\par
Ces deux droites se coupent en un point $M$.
\begin{enumerate}
     \item
     Déterminer, par le calcul, les coordonnées exactes du point $M$.
     \item
     $A_0$ est le point de la droite $D$ d'abscisse $u_0=2$.
     \par
     Expliquer pourquoi l'ordonnée de $A_0$ est égale à $u_1$.
     \item
     $B_1$ est le point de la droite $\Delta$ tel que la droite $(A_0B_1)$ est parallèle à l'axe des abscisses.
     \par
     Exprimer, en fonction de $u_1$, les coordonnées de $B_1$.
     \item
     Compléter le graphique de l'annexe de manière à faire apparaître, sur l'axe des abscisses, les valeurs de $u_1,\ u_2,\ u_3,\ u_4,\ u_5$ et $u_6$.
     \item
     \`A l'aide du graphique, conjecturer la limite de la suite $(u_n)$.
\end{enumerate}
%============================================================================================================================
\begin{center}\begin{h3}Partie B\\ Utilisation d'une suite annexe \end{h3}\end{center}
Pour tout entier naturel $n$, on pose $v_n=u_n-20$.
\begin{enumerate}
     \item
     Montrer que la suite $(v_n)$ est une suite géométrique dont on précisera le premier terme et la raison.
     \item
     Exprimer $v_n$ en fonction de $n$.
     \item
     Montrer que pour tout entier naturel $n$~:
     \[ u_n=20-18 \times 0,9^n. \]
     \item
     En déduire la limite de la suite $(u_n)$.
\end{enumerate}
\newpage
%============================================================================================================================
%
%			Annexe
%
%============================================================================================================================
\begin{center}\begin{h3}ANNEXE\end{h3}\end{center}
\medskip
\includegraphics[width=0.9\textwidth]{images/BBESL-s3-2-1.eps}%width="650"
\par
\ifx\type\ebook
\par
\small{\emph{Ce graphique peut être téléchargé et imprimé à partir de cette page~:}
     \par
\href{https~://www.maths-cours.fr/bb-esl-2018-anx/}{https~://www.maths-cours.fr/bb-esl-2018-anx/}}
\par
\fi
\end{center}

\end{document}
µ
\documentclass[a4paper]{article}

%================================================================================================================================
%
% Packages
%
%================================================================================================================================

\usepackage[T1]{fontenc} 	% pour caractères accentués
\usepackage[utf8]{inputenc}  % encodage utf8
\usepackage[french]{babel}	% langue : français
\usepackage{fourier}			% caractères plus lisibles
\usepackage[dvipsnames]{xcolor} % couleurs
\usepackage{fancyhdr}		% réglage header footer
\usepackage{needspace}		% empêcher sauts de page mal placés
\usepackage{graphicx}		% pour inclure des graphiques
\usepackage{enumitem,cprotect}		% personnalise les listes d'items (nécessaire pour ol, al ...)
\usepackage{hyperref}		% Liens hypertexte
\usepackage{pstricks,pst-all,pst-node,pstricks-add,pst-math,pst-plot,pst-tree,pst-eucl} % pstricks
\usepackage[a4paper,includeheadfoot,top=2cm,left=3cm, bottom=2cm,right=3cm]{geometry} % marges etc.
\usepackage{comment}			% commentaires multilignes
\usepackage{amsmath,environ} % maths (matrices, etc.)
\usepackage{amssymb,makeidx}
\usepackage{bm}				% bold maths
\usepackage{tabularx}		% tableaux
\usepackage{colortbl}		% tableaux en couleur
\usepackage{fontawesome}		% Fontawesome
\usepackage{environ}			% environment with command
\usepackage{fp}				% calculs pour ps-tricks
\usepackage{multido}			% pour ps tricks
\usepackage[np]{numprint}	% formattage nombre
\usepackage{tikz,tkz-tab} 			% package principal TikZ
\usepackage{pgfplots}   % axes
\usepackage{mathrsfs}    % cursives
\usepackage{calc}			% calcul taille boites
\usepackage[scaled=0.875]{helvet} % font sans serif
\usepackage{svg} % svg
\usepackage{scrextend} % local margin
\usepackage{scratch} %scratch
\usepackage{multicol} % colonnes
%\usepackage{infix-RPN,pst-func} % formule en notation polanaise inversée
\usepackage{listings}

%================================================================================================================================
%
% Réglages de base
%
%================================================================================================================================

\lstset{
language=Python,   % R code
literate=
{á}{{\'a}}1
{à}{{\`a}}1
{ã}{{\~a}}1
{é}{{\'e}}1
{è}{{\`e}}1
{ê}{{\^e}}1
{í}{{\'i}}1
{ó}{{\'o}}1
{õ}{{\~o}}1
{ú}{{\'u}}1
{ü}{{\"u}}1
{ç}{{\c{c}}}1
{~}{{ }}1
}


\definecolor{codegreen}{rgb}{0,0.6,0}
\definecolor{codegray}{rgb}{0.5,0.5,0.5}
\definecolor{codepurple}{rgb}{0.58,0,0.82}
\definecolor{backcolour}{rgb}{0.95,0.95,0.92}

\lstdefinestyle{mystyle}{
    backgroundcolor=\color{backcolour},   
    commentstyle=\color{codegreen},
    keywordstyle=\color{magenta},
    numberstyle=\tiny\color{codegray},
    stringstyle=\color{codepurple},
    basicstyle=\ttfamily\footnotesize,
    breakatwhitespace=false,         
    breaklines=true,                 
    captionpos=b,                    
    keepspaces=true,                 
    numbers=left,                    
xleftmargin=2em,
framexleftmargin=2em,            
    showspaces=false,                
    showstringspaces=false,
    showtabs=false,                  
    tabsize=2,
    upquote=true
}

\lstset{style=mystyle}


\lstset{style=mystyle}
\newcommand{\imgdir}{C:/laragon/www/newmc/assets/imgsvg/}
\newcommand{\imgsvgdir}{C:/laragon/www/newmc/assets/imgsvg/}

\definecolor{mcgris}{RGB}{220, 220, 220}% ancien~; pour compatibilité
\definecolor{mcbleu}{RGB}{52, 152, 219}
\definecolor{mcvert}{RGB}{125, 194, 70}
\definecolor{mcmauve}{RGB}{154, 0, 215}
\definecolor{mcorange}{RGB}{255, 96, 0}
\definecolor{mcturquoise}{RGB}{0, 153, 153}
\definecolor{mcrouge}{RGB}{255, 0, 0}
\definecolor{mclightvert}{RGB}{205, 234, 190}

\definecolor{gris}{RGB}{220, 220, 220}
\definecolor{bleu}{RGB}{52, 152, 219}
\definecolor{vert}{RGB}{125, 194, 70}
\definecolor{mauve}{RGB}{154, 0, 215}
\definecolor{orange}{RGB}{255, 96, 0}
\definecolor{turquoise}{RGB}{0, 153, 153}
\definecolor{rouge}{RGB}{255, 0, 0}
\definecolor{lightvert}{RGB}{205, 234, 190}
\setitemize[0]{label=\color{lightvert}  $\bullet$}

\pagestyle{fancy}
\renewcommand{\headrulewidth}{0.2pt}
\fancyhead[L]{maths-cours.fr}
\fancyhead[R]{\thepage}
\renewcommand{\footrulewidth}{0.2pt}
\fancyfoot[C]{}

\newcolumntype{C}{>{\centering\arraybackslash}X}
\newcolumntype{s}{>{\hsize=.35\hsize\arraybackslash}X}

\setlength{\parindent}{0pt}		 
\setlength{\parskip}{3mm}
\setlength{\headheight}{1cm}

\def\ebook{ebook}
\def\book{book}
\def\web{web}
\def\type{web}

\newcommand{\vect}[1]{\overrightarrow{\,\mathstrut#1\,}}

\def\Oij{$\left(\text{O}~;~\vect{\imath},~\vect{\jmath}\right)$}
\def\Oijk{$\left(\text{O}~;~\vect{\imath},~\vect{\jmath},~\vect{k}\right)$}
\def\Ouv{$\left(\text{O}~;~\vect{u},~\vect{v}\right)$}

\hypersetup{breaklinks=true, colorlinks = true, linkcolor = OliveGreen, urlcolor = OliveGreen, citecolor = OliveGreen, pdfauthor={Didier BONNEL - https://www.maths-cours.fr} } % supprime les bordures autour des liens

\renewcommand{\arg}[0]{\text{arg}}

\everymath{\displaystyle}

%================================================================================================================================
%
% Macros - Commandes
%
%================================================================================================================================

\newcommand\meta[2]{    			% Utilisé pour créer le post HTML.
	\def\titre{titre}
	\def\url{url}
	\def\arg{#1}
	\ifx\titre\arg
		\newcommand\maintitle{#2}
		\fancyhead[L]{#2}
		{\Large\sffamily \MakeUppercase{#2}}
		\vspace{1mm}\textcolor{mcvert}{\hrule}
	\fi 
	\ifx\url\arg
		\fancyfoot[L]{\href{https://www.maths-cours.fr#2}{\black \footnotesize{https://www.maths-cours.fr#2}}}
	\fi 
}


\newcommand\TitreC[1]{    		% Titre centré
     \needspace{3\baselineskip}
     \begin{center}\textbf{#1}\end{center}
}

\newcommand\newpar{    		% paragraphe
     \par
}

\newcommand\nosp {    		% commande vide (pas d'espace)
}
\newcommand{\id}[1]{} %ignore

\newcommand\boite[2]{				% Boite simple sans titre
	\vspace{5mm}
	\setlength{\fboxrule}{0.2mm}
	\setlength{\fboxsep}{5mm}	
	\fcolorbox{#1}{#1!3}{\makebox[\linewidth-2\fboxrule-2\fboxsep]{
  		\begin{minipage}[t]{\linewidth-2\fboxrule-4\fboxsep}\setlength{\parskip}{3mm}
  			 #2
  		\end{minipage}
	}}
	\vspace{5mm}
}

\newcommand\CBox[4]{				% Boites
	\vspace{5mm}
	\setlength{\fboxrule}{0.2mm}
	\setlength{\fboxsep}{5mm}
	
	\fcolorbox{#1}{#1!3}{\makebox[\linewidth-2\fboxrule-2\fboxsep]{
		\begin{minipage}[t]{1cm}\setlength{\parskip}{3mm}
	  		\textcolor{#1}{\LARGE{#2}}    
 	 	\end{minipage}  
  		\begin{minipage}[t]{\linewidth-2\fboxrule-4\fboxsep}\setlength{\parskip}{3mm}
			\raisebox{1.2mm}{\normalsize\sffamily{\textcolor{#1}{#3}}}						
  			 #4
  		\end{minipage}
	}}
	\vspace{5mm}
}

\newcommand\cadre[3]{				% Boites convertible html
	\par
	\vspace{2mm}
	\setlength{\fboxrule}{0.1mm}
	\setlength{\fboxsep}{5mm}
	\fcolorbox{#1}{white}{\makebox[\linewidth-2\fboxrule-2\fboxsep]{
  		\begin{minipage}[t]{\linewidth-2\fboxrule-4\fboxsep}\setlength{\parskip}{3mm}
			\raisebox{-2.5mm}{\sffamily \small{\textcolor{#1}{\MakeUppercase{#2}}}}		
			\par		
  			 #3
 	 		\end{minipage}
	}}
		\vspace{2mm}
	\par
}

\newcommand\bloc[3]{				% Boites convertible html sans bordure
     \needspace{2\baselineskip}
     {\sffamily \small{\textcolor{#1}{\MakeUppercase{#2}}}}    
		\par		
  			 #3
		\par
}

\newcommand\CHelp[1]{
     \CBox{Plum}{\faInfoCircle}{À RETENIR}{#1}
}

\newcommand\CUp[1]{
     \CBox{NavyBlue}{\faThumbsOUp}{EN PRATIQUE}{#1}
}

\newcommand\CInfo[1]{
     \CBox{Sepia}{\faArrowCircleRight}{REMARQUE}{#1}
}

\newcommand\CRedac[1]{
     \CBox{PineGreen}{\faEdit}{BIEN R\'EDIGER}{#1}
}

\newcommand\CError[1]{
     \CBox{Red}{\faExclamationTriangle}{ATTENTION}{#1}
}

\newcommand\TitreExo[2]{
\needspace{4\baselineskip}
 {\sffamily\large EXERCICE #1\ (\emph{#2 points})}
\vspace{5mm}
}

\newcommand\img[2]{
          \includegraphics[width=#2\paperwidth]{\imgdir#1}
}

\newcommand\imgsvg[2]{
       \begin{center}   \includegraphics[width=#2\paperwidth]{\imgsvgdir#1} \end{center}
}


\newcommand\Lien[2]{
     \href{#1}{#2 \tiny \faExternalLink}
}
\newcommand\mcLien[2]{
     \href{https~://www.maths-cours.fr/#1}{#2 \tiny \faExternalLink}
}

\newcommand{\euro}{\eurologo{}}

%================================================================================================================================
%
% Macros - Environement
%
%================================================================================================================================

\newenvironment{tex}{ %
}
{%
}

\newenvironment{indente}{ %
	\setlength\parindent{10mm}
}

{
	\setlength\parindent{0mm}
}

\newenvironment{corrige}{%
     \needspace{3\baselineskip}
     \medskip
     \textbf{\textsc{Corrigé}}
     \medskip
}
{
}

\newenvironment{extern}{%
     \begin{center}
     }
     {
     \end{center}
}

\NewEnviron{code}{%
	\par
     \boite{gray}{\texttt{%
     \BODY
     }}
     \par
}

\newenvironment{vbloc}{% boite sans cadre empeche saut de page
     \begin{minipage}[t]{\linewidth}
     }
     {
     \end{minipage}
}
\NewEnviron{h2}{%
    \needspace{3\baselineskip}
    \vspace{0.6cm}
	\noindent \MakeUppercase{\sffamily \large \BODY}
	\vspace{1mm}\textcolor{mcgris}{\hrule}\vspace{0.4cm}
	\par
}{}

\NewEnviron{h3}{%
    \needspace{3\baselineskip}
	\vspace{5mm}
	\textsc{\BODY}
	\par
}

\NewEnviron{margeneg}{ %
\begin{addmargin}[-1cm]{0cm}
\BODY
\end{addmargin}
}

\NewEnviron{html}{%
}

\begin{document}
\meta{url}{/exercices/graphes-algorithme-de-dijksta/}
\meta{pid}{6795}
\meta{titre}{Graphes~: Algorithme de Dijksta}
\meta{type}{exercices}
\par
%============================================================================================================================
%
%			SPE 2
%
%============================================================================================================================
\par
\par
Une agence de tourisme propose la visite de certains monuments parisiens.
\par
Chacun de ces monuments est désigné par une lettre comme suit~:
\begin{itemize}
     \item %
     E~: Tour Eiffel
     \item %
     L~: Musée du Louvre
     \item %
     M~: Tour Montparnasse
     \item %
     N~: Cathédrale Notre-Dame de Paris
     \item %
     S~: Basilique du Sacré-Cœur de Montmartre
     \item %
     T~: Arc de triomphe
\end{itemize}
Cette agence fait appel à une société de transport par autocar qui propose les liaisons suivantes (chacune de ces liaisons pouvant s'effectuer dans les deux sens de circulation)~:
\begin{center}
     \begin{extern}%width="300" alt="graphe non pondéré"
          \psset{unit=0.7cm}
          \begin{pspicture}(8,18)(21,5)
               % <html :>
               \rput(10,10){\circlenode{E}{E}}
               \rput(16,11){\circlenode{L}{L}}
               \rput(16,6){\circlenode{M}{M}}
               \rput(19,9){\circlenode{N}{N}}
               \rput(17,17){\circlenode{S}{S}}
               \rput(10,14){\circlenode{T}{T}}
               \ncarc[arcangle=20]{E}{L}
               \ncarc[arcangle=-30]{M}{L}
               \ncarc[arcangle=-30]{M}{N}
               \ncarc[arcangle=20]{L}{N}
               \ncarc[arcangle=20]{E}{S}
               \ncarc[arcangle=20]{L}{S}
               \ncarc[arcangle=-30]{T}{E}
               \ncarc[arcangle=20]{T}{S}
               \ncarc[arcangle=-50]{E}{M}
               \ncarc[arcangle=20]{S}{N}
          \end{pspicture}
     \end{extern}
\end{center}
\begin{enumerate}
     \item %1
     Expliquer pourquoi il est possible de trouver un trajet empruntant une fois et une seule chacune des dix liaisons indiquées sur le graphe.\\
     Donner un exemple d'un tel trajet.
     \item %2
     \begin{enumerate}[label=\alph*.]
          \item %2a
          Donner la matrice d'adjacence $M$ associée à ce graphe en classant les sommets par ordre alphabétique.
          \item %2b
          On donne~:
          \par
          \[ M^2 = \begin{pmatrix}
               4 &2 &1 &3 &2 &1 \\
               2 &4 &2 &2 &2 &2 \\
               1 &2 &3 &1 &3 &1 \\
               3 &2 &1 &3 &1 &1 \\
               2 &2 &3 &1 &4 &1\\
          1 &2 &1 &1 &1 &2  \end{pmatrix}\]
          \[ M^3 = \begin{pmatrix}
               6 &10 &9 &5 &10 &6 \\
               10 &8 &8 &8 &10 &4 \\
               9 &8 &4 &8 &5 &4 \\
               5 &8 &8 &4 &9 &4 \\
               10 &10 &5 &9 &6 &6\\
          6 &4 &4 &4 &6 &2  \end{pmatrix}\]
          Combien y a-t-il de trajets permettant de relier la cathédrale Notre-Dame de Paris et la tour Eiffel en utilisant au maximum trois liaisons.\\
          Justifier votre réponse.
     \end{enumerate}
     \item %3
     \par
     On complète le graphe précédent en indiquant, sur chacune des branches, la durée du trajet, en minutes, entre deux monuments.
     \begin{center}
          \begin{extern}%width="300" alt="graphe pondéré"
               \psset{unit=0.7cm}
               \begin{pspicture}(8,18)(21,5)
                    % <html :width="300px">
                    \rput(10,10){\circlenode{E}{E}}
                    \rput(16,11){\circlenode{L}{L}}
                    \rput(16,6){\circlenode{M}{M}}
                    \rput(19,9){\circlenode{N}{N}}
                    \rput(17,17){\circlenode{S}{S}}
                    \rput(10,14){\circlenode{T}{T}}
                    \ncarc[arcangle=20]{E}{L}\ncput*[nrot= :U]{8}
                    \ncarc[arcangle=-30]{M}{L}\ncput*[nrot= :U]{7}
                    \ncarc[arcangle=-30]{M}{N}\ncput*[nrot= :U]{4}
                    \ncarc[arcangle=20]{L}{N}\ncput*[nrot= :U]{2}
                    \ncarc[arcangle=20]{E}{S}\ncput*[nrot= :U]{10}
                    \ncarc[arcangle=20]{L}{S}\ncput*[nrot= :U]{5}
                    \ncarc[arcangle=-30]{T}{E}\ncput*[nrot= :U]{4}
                    \ncarc[arcangle=20]{T}{S}\ncput*[nrot= :U]{8}
                    \ncarc[arcangle=-50]{E}{M}\ncput*[nrot= :U]{10}
                    \ncarc[arcangle=20]{S}{N}\ncput*[nrot= :U]{8}
               \end{pspicture}
          \end{extern}
     \end{center}
     On souhaite aller de la tour Montparnasse à la Basilique du Sacré-Cœur de Montmartre.
     \par
     En utilisant un algorithme, déterminer le trajet le plus rapide ainsi que la durée de ce trajet.
\end{enumerate}

\end{document}
µ
\documentclass[a4paper]{article}

%================================================================================================================================
%
% Packages
%
%================================================================================================================================

\usepackage[T1]{fontenc} 	% pour caractères accentués
\usepackage[utf8]{inputenc}  % encodage utf8
\usepackage[french]{babel}	% langue : français
\usepackage{fourier}			% caractères plus lisibles
\usepackage[dvipsnames]{xcolor} % couleurs
\usepackage{fancyhdr}		% réglage header footer
\usepackage{needspace}		% empêcher sauts de page mal placés
\usepackage{graphicx}		% pour inclure des graphiques
\usepackage{enumitem,cprotect}		% personnalise les listes d'items (nécessaire pour ol, al ...)
\usepackage{hyperref}		% Liens hypertexte
\usepackage{pstricks,pst-all,pst-node,pstricks-add,pst-math,pst-plot,pst-tree,pst-eucl} % pstricks
\usepackage[a4paper,includeheadfoot,top=2cm,left=3cm, bottom=2cm,right=3cm]{geometry} % marges etc.
\usepackage{comment}			% commentaires multilignes
\usepackage{amsmath,environ} % maths (matrices, etc.)
\usepackage{amssymb,makeidx}
\usepackage{bm}				% bold maths
\usepackage{tabularx}		% tableaux
\usepackage{colortbl}		% tableaux en couleur
\usepackage{fontawesome}		% Fontawesome
\usepackage{environ}			% environment with command
\usepackage{fp}				% calculs pour ps-tricks
\usepackage{multido}			% pour ps tricks
\usepackage[np]{numprint}	% formattage nombre
\usepackage{tikz,tkz-tab} 			% package principal TikZ
\usepackage{pgfplots}   % axes
\usepackage{mathrsfs}    % cursives
\usepackage{calc}			% calcul taille boites
\usepackage[scaled=0.875]{helvet} % font sans serif
\usepackage{svg} % svg
\usepackage{scrextend} % local margin
\usepackage{scratch} %scratch
\usepackage{multicol} % colonnes
%\usepackage{infix-RPN,pst-func} % formule en notation polanaise inversée
\usepackage{listings}

%================================================================================================================================
%
% Réglages de base
%
%================================================================================================================================

\lstset{
language=Python,   % R code
literate=
{á}{{\'a}}1
{à}{{\`a}}1
{ã}{{\~a}}1
{é}{{\'e}}1
{è}{{\`e}}1
{ê}{{\^e}}1
{í}{{\'i}}1
{ó}{{\'o}}1
{õ}{{\~o}}1
{ú}{{\'u}}1
{ü}{{\"u}}1
{ç}{{\c{c}}}1
{~}{{ }}1
}


\definecolor{codegreen}{rgb}{0,0.6,0}
\definecolor{codegray}{rgb}{0.5,0.5,0.5}
\definecolor{codepurple}{rgb}{0.58,0,0.82}
\definecolor{backcolour}{rgb}{0.95,0.95,0.92}

\lstdefinestyle{mystyle}{
    backgroundcolor=\color{backcolour},   
    commentstyle=\color{codegreen},
    keywordstyle=\color{magenta},
    numberstyle=\tiny\color{codegray},
    stringstyle=\color{codepurple},
    basicstyle=\ttfamily\footnotesize,
    breakatwhitespace=false,         
    breaklines=true,                 
    captionpos=b,                    
    keepspaces=true,                 
    numbers=left,                    
xleftmargin=2em,
framexleftmargin=2em,            
    showspaces=false,                
    showstringspaces=false,
    showtabs=false,                  
    tabsize=2,
    upquote=true
}

\lstset{style=mystyle}


\lstset{style=mystyle}
\newcommand{\imgdir}{C:/laragon/www/newmc/assets/imgsvg/}
\newcommand{\imgsvgdir}{C:/laragon/www/newmc/assets/imgsvg/}

\definecolor{mcgris}{RGB}{220, 220, 220}% ancien~; pour compatibilité
\definecolor{mcbleu}{RGB}{52, 152, 219}
\definecolor{mcvert}{RGB}{125, 194, 70}
\definecolor{mcmauve}{RGB}{154, 0, 215}
\definecolor{mcorange}{RGB}{255, 96, 0}
\definecolor{mcturquoise}{RGB}{0, 153, 153}
\definecolor{mcrouge}{RGB}{255, 0, 0}
\definecolor{mclightvert}{RGB}{205, 234, 190}

\definecolor{gris}{RGB}{220, 220, 220}
\definecolor{bleu}{RGB}{52, 152, 219}
\definecolor{vert}{RGB}{125, 194, 70}
\definecolor{mauve}{RGB}{154, 0, 215}
\definecolor{orange}{RGB}{255, 96, 0}
\definecolor{turquoise}{RGB}{0, 153, 153}
\definecolor{rouge}{RGB}{255, 0, 0}
\definecolor{lightvert}{RGB}{205, 234, 190}
\setitemize[0]{label=\color{lightvert}  $\bullet$}

\pagestyle{fancy}
\renewcommand{\headrulewidth}{0.2pt}
\fancyhead[L]{maths-cours.fr}
\fancyhead[R]{\thepage}
\renewcommand{\footrulewidth}{0.2pt}
\fancyfoot[C]{}

\newcolumntype{C}{>{\centering\arraybackslash}X}
\newcolumntype{s}{>{\hsize=.35\hsize\arraybackslash}X}

\setlength{\parindent}{0pt}		 
\setlength{\parskip}{3mm}
\setlength{\headheight}{1cm}

\def\ebook{ebook}
\def\book{book}
\def\web{web}
\def\type{web}

\newcommand{\vect}[1]{\overrightarrow{\,\mathstrut#1\,}}

\def\Oij{$\left(\text{O}~;~\vect{\imath},~\vect{\jmath}\right)$}
\def\Oijk{$\left(\text{O}~;~\vect{\imath},~\vect{\jmath},~\vect{k}\right)$}
\def\Ouv{$\left(\text{O}~;~\vect{u},~\vect{v}\right)$}

\hypersetup{breaklinks=true, colorlinks = true, linkcolor = OliveGreen, urlcolor = OliveGreen, citecolor = OliveGreen, pdfauthor={Didier BONNEL - https://www.maths-cours.fr} } % supprime les bordures autour des liens

\renewcommand{\arg}[0]{\text{arg}}

\everymath{\displaystyle}

%================================================================================================================================
%
% Macros - Commandes
%
%================================================================================================================================

\newcommand\meta[2]{    			% Utilisé pour créer le post HTML.
	\def\titre{titre}
	\def\url{url}
	\def\arg{#1}
	\ifx\titre\arg
		\newcommand\maintitle{#2}
		\fancyhead[L]{#2}
		{\Large\sffamily \MakeUppercase{#2}}
		\vspace{1mm}\textcolor{mcvert}{\hrule}
	\fi 
	\ifx\url\arg
		\fancyfoot[L]{\href{https://www.maths-cours.fr#2}{\black \footnotesize{https://www.maths-cours.fr#2}}}
	\fi 
}


\newcommand\TitreC[1]{    		% Titre centré
     \needspace{3\baselineskip}
     \begin{center}\textbf{#1}\end{center}
}

\newcommand\newpar{    		% paragraphe
     \par
}

\newcommand\nosp {    		% commande vide (pas d'espace)
}
\newcommand{\id}[1]{} %ignore

\newcommand\boite[2]{				% Boite simple sans titre
	\vspace{5mm}
	\setlength{\fboxrule}{0.2mm}
	\setlength{\fboxsep}{5mm}	
	\fcolorbox{#1}{#1!3}{\makebox[\linewidth-2\fboxrule-2\fboxsep]{
  		\begin{minipage}[t]{\linewidth-2\fboxrule-4\fboxsep}\setlength{\parskip}{3mm}
  			 #2
  		\end{minipage}
	}}
	\vspace{5mm}
}

\newcommand\CBox[4]{				% Boites
	\vspace{5mm}
	\setlength{\fboxrule}{0.2mm}
	\setlength{\fboxsep}{5mm}
	
	\fcolorbox{#1}{#1!3}{\makebox[\linewidth-2\fboxrule-2\fboxsep]{
		\begin{minipage}[t]{1cm}\setlength{\parskip}{3mm}
	  		\textcolor{#1}{\LARGE{#2}}    
 	 	\end{minipage}  
  		\begin{minipage}[t]{\linewidth-2\fboxrule-4\fboxsep}\setlength{\parskip}{3mm}
			\raisebox{1.2mm}{\normalsize\sffamily{\textcolor{#1}{#3}}}						
  			 #4
  		\end{minipage}
	}}
	\vspace{5mm}
}

\newcommand\cadre[3]{				% Boites convertible html
	\par
	\vspace{2mm}
	\setlength{\fboxrule}{0.1mm}
	\setlength{\fboxsep}{5mm}
	\fcolorbox{#1}{white}{\makebox[\linewidth-2\fboxrule-2\fboxsep]{
  		\begin{minipage}[t]{\linewidth-2\fboxrule-4\fboxsep}\setlength{\parskip}{3mm}
			\raisebox{-2.5mm}{\sffamily \small{\textcolor{#1}{\MakeUppercase{#2}}}}		
			\par		
  			 #3
 	 		\end{minipage}
	}}
		\vspace{2mm}
	\par
}

\newcommand\bloc[3]{				% Boites convertible html sans bordure
     \needspace{2\baselineskip}
     {\sffamily \small{\textcolor{#1}{\MakeUppercase{#2}}}}    
		\par		
  			 #3
		\par
}

\newcommand\CHelp[1]{
     \CBox{Plum}{\faInfoCircle}{À RETENIR}{#1}
}

\newcommand\CUp[1]{
     \CBox{NavyBlue}{\faThumbsOUp}{EN PRATIQUE}{#1}
}

\newcommand\CInfo[1]{
     \CBox{Sepia}{\faArrowCircleRight}{REMARQUE}{#1}
}

\newcommand\CRedac[1]{
     \CBox{PineGreen}{\faEdit}{BIEN R\'EDIGER}{#1}
}

\newcommand\CError[1]{
     \CBox{Red}{\faExclamationTriangle}{ATTENTION}{#1}
}

\newcommand\TitreExo[2]{
\needspace{4\baselineskip}
 {\sffamily\large EXERCICE #1\ (\emph{#2 points})}
\vspace{5mm}
}

\newcommand\img[2]{
          \includegraphics[width=#2\paperwidth]{\imgdir#1}
}

\newcommand\imgsvg[2]{
       \begin{center}   \includegraphics[width=#2\paperwidth]{\imgsvgdir#1} \end{center}
}


\newcommand\Lien[2]{
     \href{#1}{#2 \tiny \faExternalLink}
}
\newcommand\mcLien[2]{
     \href{https~://www.maths-cours.fr/#1}{#2 \tiny \faExternalLink}
}

\newcommand{\euro}{\eurologo{}}

%================================================================================================================================
%
% Macros - Environement
%
%================================================================================================================================

\newenvironment{tex}{ %
}
{%
}

\newenvironment{indente}{ %
	\setlength\parindent{10mm}
}

{
	\setlength\parindent{0mm}
}

\newenvironment{corrige}{%
     \needspace{3\baselineskip}
     \medskip
     \textbf{\textsc{Corrigé}}
     \medskip
}
{
}

\newenvironment{extern}{%
     \begin{center}
     }
     {
     \end{center}
}

\NewEnviron{code}{%
	\par
     \boite{gray}{\texttt{%
     \BODY
     }}
     \par
}

\newenvironment{vbloc}{% boite sans cadre empeche saut de page
     \begin{minipage}[t]{\linewidth}
     }
     {
     \end{minipage}
}
\NewEnviron{h2}{%
    \needspace{3\baselineskip}
    \vspace{0.6cm}
	\noindent \MakeUppercase{\sffamily \large \BODY}
	\vspace{1mm}\textcolor{mcgris}{\hrule}\vspace{0.4cm}
	\par
}{}

\NewEnviron{h3}{%
    \needspace{3\baselineskip}
	\vspace{5mm}
	\textsc{\BODY}
	\par
}

\NewEnviron{margeneg}{ %
\begin{addmargin}[-1cm]{0cm}
\BODY
\end{addmargin}
}

\NewEnviron{html}{%
}

\begin{document}
\meta{url}{/exercices/statistiques-regroupement-en-classes/}
\meta{pid}{6949}
\meta{titre}{Statistiques~: Regroupement en classes}
\meta{type}{exercices}
\par
%============================================================================================================================
\par
Le tableau ci-dessous (source INSEE) présente la répartition par âge de la population française métropolitaine au $1^\text{er}$ janvier 2018
\textit{(L'âge révolu est l'âge de la personne à son dernier anniversaire).}
\par
\`A cette date, la doyenne des français était âgée de 113 ans.
\begin{center}
     \begin{tabular}{|c|c|c|}\hline %class="compact"
          \textbf{Âge révolu}  &  \textbf{Effectifs}  &  \textbf{Effectifs cumulés croissants} \\ \hline
          0  &  691~165  &  691~165 \\ \hline
          1  &  710~534  &  1~401~699 \\ \hline
          2  &  728~579  &  2~130~278 \\ \hline
          3  &  749~270  &  2~879~548 \\ \hline
          4  &  763~228  &  3~642~776 \\ \hline
          5  &  782~484  &  4~425~260 \\ \hline
          6  &  792~558  &  5~217~818 \\ \hline
          7  &  813~001  &  6~030~819 \\ \hline
          8  &  808~393  &  6~839~212 \\ \hline
          9  &  813~680  &  7~652~892 \\ \hline
          10  &  807~548  &  8~460~440 \\ \hline
          11  &  822~302  &  9~282~742 \\ \hline
          12  &  802~674  &  10~085~416 \\ \hline
          13  &  800~480  &  10~885~896 \\ \hline
          14  &  796~320  &  11~682~216 \\ \hline
          15  &  800~560  &  12~482~776 \\ \hline
          16  &  816~021  &  13~298~797 \\ \hline
          17  &  828~193  &  14~126~990 \\ \hline
          18  &  785~471  &  14~912~461 \\ \hline
          19  &  775~524  &  15~687~985 \\ \hline
          20  &  750~885  &  16~438~870 \\ \hline
          21  &  751~084  &  17~189~954 \\ \hline
          22  &  734~838  &  17~924~792 \\ \hline
          23  &  705~808  &  18~630~600 \\ \hline
          24  &  698~780  &  19~329~380 \\ \hline
          25  &  732~693  &  20~062~073 \\ \hline
          26  &  742~199  &  20~804~272 \\ \hline
          27  &  758~458  &  21~562~730 \\ \hline
          28  &  763~258  &  22~325~988 \\ \hline
          29  &  774~435  &  23~100~423 \\ \hline
          30  &  778~738  &  23~879~161 \\ \hline
          31  &  793~507  &  24~672~668 \\ \hline
          32  &  794~025  &  25~466~693 \\ \hline
          33  &  789~604  &  26~256~297 \\ \hline
          34  &  781~093  &  27~037~390 \\ \hline
     \end{tabular}
\end{center}
\newpage
\begin{center}
     \begin{tabular}{|c|c|c|}\hline %class="compact"
          \textbf{Âge révolu}  &  \textbf{Effectifs}  &  \textbf{Effectifs cumulés croissants} \\ \hline
          35  &  829~365  &  27~866~755 \\ \hline
          36  &  837~426  &  28~704~181 \\ \hline
          37  &  849~108  &  29~553~289 \\ \hline
          38  &  804~184  &  30~357~473 \\ \hline
          39  &  788~264  &  31~145~737 \\ \hline
          40  &  794~640  &  31~940~377 \\ \hline
          41  &  773~344  &  32~713~721 \\ \hline
          42  &  793~019  &  33~506~740 \\ \hline
          43  &  836~502  &  34~343~242 \\ \hline
          44  &  885~498  &  35~228~740 \\ \hline
          45  &  903~921  &  36~132~661 \\ \hline
          46  &  897~885  &  37~030~546 \\ \hline
          47  &  881~680  &  37~912~226 \\ \hline
          48  &  868~783  &  38~781~009 \\ \hline
          49  &  860~664  &  39~641~673 \\ \hline
          50  &  856~564  &  40~498~237 \\ \hline
          51  &  873~805  &  41~372~042 \\ \hline
          52  &  875~149  &  42~247~191 \\ \hline
          53  &  884~018  &  43~131~209 \\ \hline
          54  &  874~390  &  44~005~599 \\ \hline
          55  &  842~410  &  44~848~009 \\ \hline
          56  &  844~453  &  45~692~462 \\ \hline
          57  &  840~074  &  46~532~536 \\ \hline
          58  &  833~430  &  47~365~966 \\ \hline
          59  &  813~824  &  48~179~790 \\ \hline
          60  &  809~540  &  48~989~330 \\ \hline
          61  &  801~271  &  49~790~601 \\ \hline
          62  &  790~864  &  50~581~465 \\ \hline
          63  &  788~149  &  51~369~614 \\ \hline
          64  &  769~877  &  52~139~491 \\ \hline
          65  &  780~203  &  52~919~694 \\ \hline
          66  &  760~764  &  53~680~458 \\ \hline
          67  &  788~609  &  54~469~067 \\ \hline
          68  &  771~267  &  55~240~334 \\ \hline
          69  &  763~386  &  56~003~720 \\ \hline
     \end{tabular}
\end{center}
\newpage
\begin{center}
     \begin{tabular}{|c|c|c|}\hline %class="compact"
          \textbf{Âge révolu}  &  \textbf{Effectifs}  &  \textbf{Effectifs cumulés croissants} \\ \hline
          70  &  746~205  &  56~749~925 \\ \hline
          71  &  703~078  &  57~453~003 \\ \hline
          72  &  526~166  &  57~979~169 \\ \hline
          73  &  510~477  &  58~489~646 \\ \hline
          74  &  493~523  &  58~983~169 \\ \hline
          75  &  452~195  &  59~435~364 \\ \hline
          76  &  399~640  &  59~835~004 \\ \hline
          77  &  410~887  &  60~245~891 \\ \hline
          78  &  424~148  &  60~670~039 \\ \hline
          79  &  410~734  &  61~080~773 \\ \hline
          80  &  394~185  &  61~474~958 \\ \hline
          81  &  385~845  &  61~860~803 \\ \hline
          82  &  365~057  &  62~225~860 \\ \hline
          83  &  356~869  &  62~582~729 \\ \hline
          84  &  327~225  &  62~909~954 \\ \hline
          85  &  319~458  &  63~229~412 \\ \hline
          86  &  290~749  &  63~520~161 \\ \hline
          87  &  268~489  &  63~788~650 \\ \hline
          88  &  227~255  &  64~015~905 \\ \hline
          89  &  201~758  &  64~217~663 \\ \hline
          90  &  171~893  &  64~389~556 \\ \hline
          91  &  147~011  &  64~536~567 \\ \hline
          92  &  123~524  &  64~660~091 \\ \hline
          93  &  98~697  &  64~758~788 \\ \hline
          94  &  78~283  &  64~837~071 \\ \hline
          95  &  61~359  &  64~898~430 \\ \hline
          96  &  46~186  &  64~944~616 \\ \hline
          97  &  34~225  &  64~978~841 \\ \hline
          98  &  14~599  &  64~993~440 \\ \hline
          99  &  8~401  &  65~001~841 \\ \hline
          100  &  5~174  &  65~007~015 \\ \hline
          101  &  3~135  &  65~010~150 \\ \hline
          102  &  2~297  &  65~012~447 \\ \hline
          103  &  2~281  &  65~014~728 \\ \hline
          104  &  1~208  &  65~015~936 \\ \hline
          105~ou~plus  &  2~160  &  65~018~096 \\ \hline
          \textbf{Total}  &  \textbf{65~018~096} &\\ \hline
     \end{tabular}
     \textbf{Tableau A}
\end{center}
%============================================================================================================================
%
\begin{center}\begin{h3}Partie A \end{h3}\end{center}
%
%============================================================================================================================
\begin{enumerate}
     \item %1
     Quel sont le mode et l'étendue de cette série statistique~?
     \item %2
     Calculer le pourcentage de personnes mineures (c'est à dire ayant un âge révolu inférieur ou égal à 17 ans) en métropole au $1^\text{er}$ janvier 2018.
     \item %3
     Déterminer la médiane ainsi que les premier et troisième quartiles de cette série.
     \item %4
     Représenter cette série par un diagramme en boîte.
\end{enumerate}
%============================================================================================================================
%
\begin{center}\begin{h3}Partie B \end{h3}\end{center}
%============================================================================================================================
\par
Le tableau \textbf{A} comporte trop de lignes pour être facilement exploitable.
\par
On décide donc de regrouper ces résultats en classes d'amplitude 10 ans (à l'exception de la dernière).
\par
On obtient alors le tableau suivant~:
\begin{center}
     \begin{tabular}{|c|c|c|c|c|c|c|}\hline %class="compact left"
          \textbf{Classes d'âge }& [0~;~10[ & [10~;~20[ & [20~;~30[ & [30~;~40[ & [40~;~50[ & [50~;~60[\\ \hline
          \textbf{Effectifs} & 7~652~892  &  8~035~093  &  7~412~438  &  8~045~314  &  8~495~936 &  8~538~117   \\ \hline
          \textbf{E.C.C} & 7~652~892  &  15~687~985  &  23~100~423  &  31~145~737  &  39~641~673  &  48~179~790  \\ \hline
          \textbf{Classes d'âge} &[60~;~70[ & [70~;~80[ & [80~;~90[ & [90~;~100[ & [100~;~114[ \\ \hline
          \textbf{Effectifs} & 7~823~930  &  5~077~053  &  3~136~890  &  784~178  &  16~255 \\ \hline
          \textbf{E.C.C} & 56~003~720  &  61~080~773  &  64~217~663  &  65~001~841  &  65~018~096 \\ \hline
     \end{tabular}
     \textbf{Tableau B}
\end{center}
\bigskip
\begin{enumerate}
     \item %1
     Expliquer comment, à partir du tableau \textbf{A}, on a obtenu le tableau \textbf{B}.
     \item %2
     Calculer la moyenne de cette série en utilisant le tableau \textbf{B}.
     \item %3
     Tracer le graphique des effectifs cumulés croissants en choisissant une échelle appropriée.
     \item %4
     \`A l'aide du graphique de la question précédente, déterminer des valeurs approchées à l'unité près de la médiane et des premier et troisième quartiles.\\
     Comparer ce résultat à celui de la question \textbf{3.} de la partie A.
\end{enumerate}

\end{document}
µ
\documentclass[a4paper]{article}

%================================================================================================================================
%
% Packages
%
%================================================================================================================================

\usepackage[T1]{fontenc} 	% pour caractères accentués
\usepackage[utf8]{inputenc}  % encodage utf8
\usepackage[french]{babel}	% langue : français
\usepackage{fourier}			% caractères plus lisibles
\usepackage[dvipsnames]{xcolor} % couleurs
\usepackage{fancyhdr}		% réglage header footer
\usepackage{needspace}		% empêcher sauts de page mal placés
\usepackage{graphicx}		% pour inclure des graphiques
\usepackage{enumitem,cprotect}		% personnalise les listes d'items (nécessaire pour ol, al ...)
\usepackage{hyperref}		% Liens hypertexte
\usepackage{pstricks,pst-all,pst-node,pstricks-add,pst-math,pst-plot,pst-tree,pst-eucl} % pstricks
\usepackage[a4paper,includeheadfoot,top=2cm,left=3cm, bottom=2cm,right=3cm]{geometry} % marges etc.
\usepackage{comment}			% commentaires multilignes
\usepackage{amsmath,environ} % maths (matrices, etc.)
\usepackage{amssymb,makeidx}
\usepackage{bm}				% bold maths
\usepackage{tabularx}		% tableaux
\usepackage{colortbl}		% tableaux en couleur
\usepackage{fontawesome}		% Fontawesome
\usepackage{environ}			% environment with command
\usepackage{fp}				% calculs pour ps-tricks
\usepackage{multido}			% pour ps tricks
\usepackage[np]{numprint}	% formattage nombre
\usepackage{tikz,tkz-tab} 			% package principal TikZ
\usepackage{pgfplots}   % axes
\usepackage{mathrsfs}    % cursives
\usepackage{calc}			% calcul taille boites
\usepackage[scaled=0.875]{helvet} % font sans serif
\usepackage{svg} % svg
\usepackage{scrextend} % local margin
\usepackage{scratch} %scratch
\usepackage{multicol} % colonnes
%\usepackage{infix-RPN,pst-func} % formule en notation polanaise inversée
\usepackage{listings}

%================================================================================================================================
%
% Réglages de base
%
%================================================================================================================================

\lstset{
language=Python,   % R code
literate=
{á}{{\'a}}1
{à}{{\`a}}1
{ã}{{\~a}}1
{é}{{\'e}}1
{è}{{\`e}}1
{ê}{{\^e}}1
{í}{{\'i}}1
{ó}{{\'o}}1
{õ}{{\~o}}1
{ú}{{\'u}}1
{ü}{{\"u}}1
{ç}{{\c{c}}}1
{~}{{ }}1
}


\definecolor{codegreen}{rgb}{0,0.6,0}
\definecolor{codegray}{rgb}{0.5,0.5,0.5}
\definecolor{codepurple}{rgb}{0.58,0,0.82}
\definecolor{backcolour}{rgb}{0.95,0.95,0.92}

\lstdefinestyle{mystyle}{
    backgroundcolor=\color{backcolour},   
    commentstyle=\color{codegreen},
    keywordstyle=\color{magenta},
    numberstyle=\tiny\color{codegray},
    stringstyle=\color{codepurple},
    basicstyle=\ttfamily\footnotesize,
    breakatwhitespace=false,         
    breaklines=true,                 
    captionpos=b,                    
    keepspaces=true,                 
    numbers=left,                    
xleftmargin=2em,
framexleftmargin=2em,            
    showspaces=false,                
    showstringspaces=false,
    showtabs=false,                  
    tabsize=2,
    upquote=true
}

\lstset{style=mystyle}


\lstset{style=mystyle}
\newcommand{\imgdir}{C:/laragon/www/newmc/assets/imgsvg/}
\newcommand{\imgsvgdir}{C:/laragon/www/newmc/assets/imgsvg/}

\definecolor{mcgris}{RGB}{220, 220, 220}% ancien~; pour compatibilité
\definecolor{mcbleu}{RGB}{52, 152, 219}
\definecolor{mcvert}{RGB}{125, 194, 70}
\definecolor{mcmauve}{RGB}{154, 0, 215}
\definecolor{mcorange}{RGB}{255, 96, 0}
\definecolor{mcturquoise}{RGB}{0, 153, 153}
\definecolor{mcrouge}{RGB}{255, 0, 0}
\definecolor{mclightvert}{RGB}{205, 234, 190}

\definecolor{gris}{RGB}{220, 220, 220}
\definecolor{bleu}{RGB}{52, 152, 219}
\definecolor{vert}{RGB}{125, 194, 70}
\definecolor{mauve}{RGB}{154, 0, 215}
\definecolor{orange}{RGB}{255, 96, 0}
\definecolor{turquoise}{RGB}{0, 153, 153}
\definecolor{rouge}{RGB}{255, 0, 0}
\definecolor{lightvert}{RGB}{205, 234, 190}
\setitemize[0]{label=\color{lightvert}  $\bullet$}

\pagestyle{fancy}
\renewcommand{\headrulewidth}{0.2pt}
\fancyhead[L]{maths-cours.fr}
\fancyhead[R]{\thepage}
\renewcommand{\footrulewidth}{0.2pt}
\fancyfoot[C]{}

\newcolumntype{C}{>{\centering\arraybackslash}X}
\newcolumntype{s}{>{\hsize=.35\hsize\arraybackslash}X}

\setlength{\parindent}{0pt}		 
\setlength{\parskip}{3mm}
\setlength{\headheight}{1cm}

\def\ebook{ebook}
\def\book{book}
\def\web{web}
\def\type{web}

\newcommand{\vect}[1]{\overrightarrow{\,\mathstrut#1\,}}

\def\Oij{$\left(\text{O}~;~\vect{\imath},~\vect{\jmath}\right)$}
\def\Oijk{$\left(\text{O}~;~\vect{\imath},~\vect{\jmath},~\vect{k}\right)$}
\def\Ouv{$\left(\text{O}~;~\vect{u},~\vect{v}\right)$}

\hypersetup{breaklinks=true, colorlinks = true, linkcolor = OliveGreen, urlcolor = OliveGreen, citecolor = OliveGreen, pdfauthor={Didier BONNEL - https://www.maths-cours.fr} } % supprime les bordures autour des liens

\renewcommand{\arg}[0]{\text{arg}}

\everymath{\displaystyle}

%================================================================================================================================
%
% Macros - Commandes
%
%================================================================================================================================

\newcommand\meta[2]{    			% Utilisé pour créer le post HTML.
	\def\titre{titre}
	\def\url{url}
	\def\arg{#1}
	\ifx\titre\arg
		\newcommand\maintitle{#2}
		\fancyhead[L]{#2}
		{\Large\sffamily \MakeUppercase{#2}}
		\vspace{1mm}\textcolor{mcvert}{\hrule}
	\fi 
	\ifx\url\arg
		\fancyfoot[L]{\href{https://www.maths-cours.fr#2}{\black \footnotesize{https://www.maths-cours.fr#2}}}
	\fi 
}


\newcommand\TitreC[1]{    		% Titre centré
     \needspace{3\baselineskip}
     \begin{center}\textbf{#1}\end{center}
}

\newcommand\newpar{    		% paragraphe
     \par
}

\newcommand\nosp {    		% commande vide (pas d'espace)
}
\newcommand{\id}[1]{} %ignore

\newcommand\boite[2]{				% Boite simple sans titre
	\vspace{5mm}
	\setlength{\fboxrule}{0.2mm}
	\setlength{\fboxsep}{5mm}	
	\fcolorbox{#1}{#1!3}{\makebox[\linewidth-2\fboxrule-2\fboxsep]{
  		\begin{minipage}[t]{\linewidth-2\fboxrule-4\fboxsep}\setlength{\parskip}{3mm}
  			 #2
  		\end{minipage}
	}}
	\vspace{5mm}
}

\newcommand\CBox[4]{				% Boites
	\vspace{5mm}
	\setlength{\fboxrule}{0.2mm}
	\setlength{\fboxsep}{5mm}
	
	\fcolorbox{#1}{#1!3}{\makebox[\linewidth-2\fboxrule-2\fboxsep]{
		\begin{minipage}[t]{1cm}\setlength{\parskip}{3mm}
	  		\textcolor{#1}{\LARGE{#2}}    
 	 	\end{minipage}  
  		\begin{minipage}[t]{\linewidth-2\fboxrule-4\fboxsep}\setlength{\parskip}{3mm}
			\raisebox{1.2mm}{\normalsize\sffamily{\textcolor{#1}{#3}}}						
  			 #4
  		\end{minipage}
	}}
	\vspace{5mm}
}

\newcommand\cadre[3]{				% Boites convertible html
	\par
	\vspace{2mm}
	\setlength{\fboxrule}{0.1mm}
	\setlength{\fboxsep}{5mm}
	\fcolorbox{#1}{white}{\makebox[\linewidth-2\fboxrule-2\fboxsep]{
  		\begin{minipage}[t]{\linewidth-2\fboxrule-4\fboxsep}\setlength{\parskip}{3mm}
			\raisebox{-2.5mm}{\sffamily \small{\textcolor{#1}{\MakeUppercase{#2}}}}		
			\par		
  			 #3
 	 		\end{minipage}
	}}
		\vspace{2mm}
	\par
}

\newcommand\bloc[3]{				% Boites convertible html sans bordure
     \needspace{2\baselineskip}
     {\sffamily \small{\textcolor{#1}{\MakeUppercase{#2}}}}    
		\par		
  			 #3
		\par
}

\newcommand\CHelp[1]{
     \CBox{Plum}{\faInfoCircle}{À RETENIR}{#1}
}

\newcommand\CUp[1]{
     \CBox{NavyBlue}{\faThumbsOUp}{EN PRATIQUE}{#1}
}

\newcommand\CInfo[1]{
     \CBox{Sepia}{\faArrowCircleRight}{REMARQUE}{#1}
}

\newcommand\CRedac[1]{
     \CBox{PineGreen}{\faEdit}{BIEN R\'EDIGER}{#1}
}

\newcommand\CError[1]{
     \CBox{Red}{\faExclamationTriangle}{ATTENTION}{#1}
}

\newcommand\TitreExo[2]{
\needspace{4\baselineskip}
 {\sffamily\large EXERCICE #1\ (\emph{#2 points})}
\vspace{5mm}
}

\newcommand\img[2]{
          \includegraphics[width=#2\paperwidth]{\imgdir#1}
}

\newcommand\imgsvg[2]{
       \begin{center}   \includegraphics[width=#2\paperwidth]{\imgsvgdir#1} \end{center}
}


\newcommand\Lien[2]{
     \href{#1}{#2 \tiny \faExternalLink}
}
\newcommand\mcLien[2]{
     \href{https~://www.maths-cours.fr/#1}{#2 \tiny \faExternalLink}
}

\newcommand{\euro}{\eurologo{}}

%================================================================================================================================
%
% Macros - Environement
%
%================================================================================================================================

\newenvironment{tex}{ %
}
{%
}

\newenvironment{indente}{ %
	\setlength\parindent{10mm}
}

{
	\setlength\parindent{0mm}
}

\newenvironment{corrige}{%
     \needspace{3\baselineskip}
     \medskip
     \textbf{\textsc{Corrigé}}
     \medskip
}
{
}

\newenvironment{extern}{%
     \begin{center}
     }
     {
     \end{center}
}

\NewEnviron{code}{%
	\par
     \boite{gray}{\texttt{%
     \BODY
     }}
     \par
}

\newenvironment{vbloc}{% boite sans cadre empeche saut de page
     \begin{minipage}[t]{\linewidth}
     }
     {
     \end{minipage}
}
\NewEnviron{h2}{%
    \needspace{3\baselineskip}
    \vspace{0.6cm}
	\noindent \MakeUppercase{\sffamily \large \BODY}
	\vspace{1mm}\textcolor{mcgris}{\hrule}\vspace{0.4cm}
	\par
}{}

\NewEnviron{h3}{%
    \needspace{3\baselineskip}
	\vspace{5mm}
	\textsc{\BODY}
	\par
}

\NewEnviron{margeneg}{ %
\begin{addmargin}[-1cm]{0cm}
\BODY
\end{addmargin}
}

\NewEnviron{html}{%
}

\begin{document}
\meta{url}{/exercices/esperance-mathematique-loi-binomiale/}
\par
\meta{pid}{6975}
\meta{titre}{Espérance mathématique - Loi binomiale}
\meta{type}{exercice}
\par
Un constructeur fabrique des tablettes informatiques. Le coût de production est 250~euros par unité.
\par
Les tablettes sont garanties contre un défaut de fonctionnement de l'écran ou du disque dur.
\par
Cette garantie permet à l'acheteur, en cas de panne, d'effectuer les réparations suivantes aux frais du constructeur~:
\begin{itemize}
     \item réparation de l'écran (coût pour le constructeur~: 50~euros)~;
     \item réparation du disque dur (coût pour le constructeur~: 30~euros).
\end{itemize}
Une étude statistique a montré que~:
\begin{itemize}
     \item 3\% des tablettes présentent un défaut de disque dur~;
     \item 4\% des tablettes présentent un défaut d'écran~;
     \item 95\% des tablettes ne présentent aucun des deux défauts.
\end{itemize}
%============================================================================================================================
%
\begin{center}\begin{h3}Partie A \end{h3}\end{center}
%
%============================================================================================================================
\begin{enumerate}
     \item %1
     Recopier et compléter le tableau ci-après à l'aide des données de l'énoncé.
     \begin{center}
          \begin{extern}%width="400" alt="tableau statistique"
               \renewcommand\arraystretch{1.5}
               \begin{tabular}{|c|p{2cm}|p{2cm}|c|}
                    \hline
                    $\ $ & \centering Disque dur OK & \centering Disque dur défectueux & Total \\
                    \hline
                    \'Ecran OK & \centering  $\cdots$ & \centering $\cdots$ & $\cdots$ \\
                    \hline
                    \'Ecran défectueux & \centering  $\cdots$ & \centering $\cdots$ & $\cdots$ \\
                    \hline
                    Total & \centering $\cdots$ & \centering 3\% & 100 \% \\
                    \hline
               \end{tabular}
          \end{extern}
     \end{center}
     \item %2
     Le prix de revient d'une tablette est égal à son coût de production augmenté du coût de réparation éventuel.
     On note $X$ la variable aléatoire correspondant au prix de revient d'une tablette.\\
     \'Etablir la loi de probabilité de $X$.
     \item %3
     Calculer l'espérance mathématique de $X$. Interpréter ce résultat dans le contexte de l'énoncé.
     \item %4
     L'entreprise vend chaque tablette 400~euros. Quel sera son bénéfice mensuel moyen si elle vend 750 tablettes par mois~?
\end{enumerate}
%============================================================================================================================
%
\begin{center}\begin{h3}Partie B \end{h3}\end{center}
%
%============================================================================================================================
\par
Un établissement scolaire achète 50 tablettes à ce constructeur.
\par
On suppose que l'on peut assimiler cet achat à un tirage aléatoire de 50 tablettes avec remise, les tirages étant supposés indépendants.
\par
On rappelle que 95\% des tablettes ne présentent aucun défaut couvert par la garantie constructeur.
\par
On note $Y$ la variable aléatoire égale au nombre de tablettes achetées par l'établissement présentant un défaut couvert par la garantie constructeur.
\begin{enumerate}
     \item %1
     Justifier que $Y$ suit une loi binomiale dont on précisera les paramètres.
     \item %2
     Quelle est la probabilité qu'aucune des tablettes achetées par l'établissement ne présente de défaut couvert par la garantie constructeur~?
     \item %3
     \par
     Quelle est l'espérance mathématique de $Y$~? Interpréter ce résultat.
\end{enumerate}

\end{document}
µ
\documentclass[a4paper]{article}

%================================================================================================================================
%
% Packages
%
%================================================================================================================================

\usepackage[T1]{fontenc} 	% pour caractères accentués
\usepackage[utf8]{inputenc}  % encodage utf8
\usepackage[french]{babel}	% langue : français
\usepackage{fourier}			% caractères plus lisibles
\usepackage[dvipsnames]{xcolor} % couleurs
\usepackage{fancyhdr}		% réglage header footer
\usepackage{needspace}		% empêcher sauts de page mal placés
\usepackage{graphicx}		% pour inclure des graphiques
\usepackage{enumitem,cprotect}		% personnalise les listes d'items (nécessaire pour ol, al ...)
\usepackage{hyperref}		% Liens hypertexte
\usepackage{pstricks,pst-all,pst-node,pstricks-add,pst-math,pst-plot,pst-tree,pst-eucl} % pstricks
\usepackage[a4paper,includeheadfoot,top=2cm,left=3cm, bottom=2cm,right=3cm]{geometry} % marges etc.
\usepackage{comment}			% commentaires multilignes
\usepackage{amsmath,environ} % maths (matrices, etc.)
\usepackage{amssymb,makeidx}
\usepackage{bm}				% bold maths
\usepackage{tabularx}		% tableaux
\usepackage{colortbl}		% tableaux en couleur
\usepackage{fontawesome}		% Fontawesome
\usepackage{environ}			% environment with command
\usepackage{fp}				% calculs pour ps-tricks
\usepackage{multido}			% pour ps tricks
\usepackage[np]{numprint}	% formattage nombre
\usepackage{tikz,tkz-tab} 			% package principal TikZ
\usepackage{pgfplots}   % axes
\usepackage{mathrsfs}    % cursives
\usepackage{calc}			% calcul taille boites
\usepackage[scaled=0.875]{helvet} % font sans serif
\usepackage{svg} % svg
\usepackage{scrextend} % local margin
\usepackage{scratch} %scratch
\usepackage{multicol} % colonnes
%\usepackage{infix-RPN,pst-func} % formule en notation polanaise inversée
\usepackage{listings}

%================================================================================================================================
%
% Réglages de base
%
%================================================================================================================================

\lstset{
language=Python,   % R code
literate=
{á}{{\'a}}1
{à}{{\`a}}1
{ã}{{\~a}}1
{é}{{\'e}}1
{è}{{\`e}}1
{ê}{{\^e}}1
{í}{{\'i}}1
{ó}{{\'o}}1
{õ}{{\~o}}1
{ú}{{\'u}}1
{ü}{{\"u}}1
{ç}{{\c{c}}}1
{~}{{ }}1
}


\definecolor{codegreen}{rgb}{0,0.6,0}
\definecolor{codegray}{rgb}{0.5,0.5,0.5}
\definecolor{codepurple}{rgb}{0.58,0,0.82}
\definecolor{backcolour}{rgb}{0.95,0.95,0.92}

\lstdefinestyle{mystyle}{
    backgroundcolor=\color{backcolour},   
    commentstyle=\color{codegreen},
    keywordstyle=\color{magenta},
    numberstyle=\tiny\color{codegray},
    stringstyle=\color{codepurple},
    basicstyle=\ttfamily\footnotesize,
    breakatwhitespace=false,         
    breaklines=true,                 
    captionpos=b,                    
    keepspaces=true,                 
    numbers=left,                    
xleftmargin=2em,
framexleftmargin=2em,            
    showspaces=false,                
    showstringspaces=false,
    showtabs=false,                  
    tabsize=2,
    upquote=true
}

\lstset{style=mystyle}


\lstset{style=mystyle}
\newcommand{\imgdir}{C:/laragon/www/newmc/assets/imgsvg/}
\newcommand{\imgsvgdir}{C:/laragon/www/newmc/assets/imgsvg/}

\definecolor{mcgris}{RGB}{220, 220, 220}% ancien~; pour compatibilité
\definecolor{mcbleu}{RGB}{52, 152, 219}
\definecolor{mcvert}{RGB}{125, 194, 70}
\definecolor{mcmauve}{RGB}{154, 0, 215}
\definecolor{mcorange}{RGB}{255, 96, 0}
\definecolor{mcturquoise}{RGB}{0, 153, 153}
\definecolor{mcrouge}{RGB}{255, 0, 0}
\definecolor{mclightvert}{RGB}{205, 234, 190}

\definecolor{gris}{RGB}{220, 220, 220}
\definecolor{bleu}{RGB}{52, 152, 219}
\definecolor{vert}{RGB}{125, 194, 70}
\definecolor{mauve}{RGB}{154, 0, 215}
\definecolor{orange}{RGB}{255, 96, 0}
\definecolor{turquoise}{RGB}{0, 153, 153}
\definecolor{rouge}{RGB}{255, 0, 0}
\definecolor{lightvert}{RGB}{205, 234, 190}
\setitemize[0]{label=\color{lightvert}  $\bullet$}

\pagestyle{fancy}
\renewcommand{\headrulewidth}{0.2pt}
\fancyhead[L]{maths-cours.fr}
\fancyhead[R]{\thepage}
\renewcommand{\footrulewidth}{0.2pt}
\fancyfoot[C]{}

\newcolumntype{C}{>{\centering\arraybackslash}X}
\newcolumntype{s}{>{\hsize=.35\hsize\arraybackslash}X}

\setlength{\parindent}{0pt}		 
\setlength{\parskip}{3mm}
\setlength{\headheight}{1cm}

\def\ebook{ebook}
\def\book{book}
\def\web{web}
\def\type{web}

\newcommand{\vect}[1]{\overrightarrow{\,\mathstrut#1\,}}

\def\Oij{$\left(\text{O}~;~\vect{\imath},~\vect{\jmath}\right)$}
\def\Oijk{$\left(\text{O}~;~\vect{\imath},~\vect{\jmath},~\vect{k}\right)$}
\def\Ouv{$\left(\text{O}~;~\vect{u},~\vect{v}\right)$}

\hypersetup{breaklinks=true, colorlinks = true, linkcolor = OliveGreen, urlcolor = OliveGreen, citecolor = OliveGreen, pdfauthor={Didier BONNEL - https://www.maths-cours.fr} } % supprime les bordures autour des liens

\renewcommand{\arg}[0]{\text{arg}}

\everymath{\displaystyle}

%================================================================================================================================
%
% Macros - Commandes
%
%================================================================================================================================

\newcommand\meta[2]{    			% Utilisé pour créer le post HTML.
	\def\titre{titre}
	\def\url{url}
	\def\arg{#1}
	\ifx\titre\arg
		\newcommand\maintitle{#2}
		\fancyhead[L]{#2}
		{\Large\sffamily \MakeUppercase{#2}}
		\vspace{1mm}\textcolor{mcvert}{\hrule}
	\fi 
	\ifx\url\arg
		\fancyfoot[L]{\href{https://www.maths-cours.fr#2}{\black \footnotesize{https://www.maths-cours.fr#2}}}
	\fi 
}


\newcommand\TitreC[1]{    		% Titre centré
     \needspace{3\baselineskip}
     \begin{center}\textbf{#1}\end{center}
}

\newcommand\newpar{    		% paragraphe
     \par
}

\newcommand\nosp {    		% commande vide (pas d'espace)
}
\newcommand{\id}[1]{} %ignore

\newcommand\boite[2]{				% Boite simple sans titre
	\vspace{5mm}
	\setlength{\fboxrule}{0.2mm}
	\setlength{\fboxsep}{5mm}	
	\fcolorbox{#1}{#1!3}{\makebox[\linewidth-2\fboxrule-2\fboxsep]{
  		\begin{minipage}[t]{\linewidth-2\fboxrule-4\fboxsep}\setlength{\parskip}{3mm}
  			 #2
  		\end{minipage}
	}}
	\vspace{5mm}
}

\newcommand\CBox[4]{				% Boites
	\vspace{5mm}
	\setlength{\fboxrule}{0.2mm}
	\setlength{\fboxsep}{5mm}
	
	\fcolorbox{#1}{#1!3}{\makebox[\linewidth-2\fboxrule-2\fboxsep]{
		\begin{minipage}[t]{1cm}\setlength{\parskip}{3mm}
	  		\textcolor{#1}{\LARGE{#2}}    
 	 	\end{minipage}  
  		\begin{minipage}[t]{\linewidth-2\fboxrule-4\fboxsep}\setlength{\parskip}{3mm}
			\raisebox{1.2mm}{\normalsize\sffamily{\textcolor{#1}{#3}}}						
  			 #4
  		\end{minipage}
	}}
	\vspace{5mm}
}

\newcommand\cadre[3]{				% Boites convertible html
	\par
	\vspace{2mm}
	\setlength{\fboxrule}{0.1mm}
	\setlength{\fboxsep}{5mm}
	\fcolorbox{#1}{white}{\makebox[\linewidth-2\fboxrule-2\fboxsep]{
  		\begin{minipage}[t]{\linewidth-2\fboxrule-4\fboxsep}\setlength{\parskip}{3mm}
			\raisebox{-2.5mm}{\sffamily \small{\textcolor{#1}{\MakeUppercase{#2}}}}		
			\par		
  			 #3
 	 		\end{minipage}
	}}
		\vspace{2mm}
	\par
}

\newcommand\bloc[3]{				% Boites convertible html sans bordure
     \needspace{2\baselineskip}
     {\sffamily \small{\textcolor{#1}{\MakeUppercase{#2}}}}    
		\par		
  			 #3
		\par
}

\newcommand\CHelp[1]{
     \CBox{Plum}{\faInfoCircle}{À RETENIR}{#1}
}

\newcommand\CUp[1]{
     \CBox{NavyBlue}{\faThumbsOUp}{EN PRATIQUE}{#1}
}

\newcommand\CInfo[1]{
     \CBox{Sepia}{\faArrowCircleRight}{REMARQUE}{#1}
}

\newcommand\CRedac[1]{
     \CBox{PineGreen}{\faEdit}{BIEN R\'EDIGER}{#1}
}

\newcommand\CError[1]{
     \CBox{Red}{\faExclamationTriangle}{ATTENTION}{#1}
}

\newcommand\TitreExo[2]{
\needspace{4\baselineskip}
 {\sffamily\large EXERCICE #1\ (\emph{#2 points})}
\vspace{5mm}
}

\newcommand\img[2]{
          \includegraphics[width=#2\paperwidth]{\imgdir#1}
}

\newcommand\imgsvg[2]{
       \begin{center}   \includegraphics[width=#2\paperwidth]{\imgsvgdir#1} \end{center}
}


\newcommand\Lien[2]{
     \href{#1}{#2 \tiny \faExternalLink}
}
\newcommand\mcLien[2]{
     \href{https~://www.maths-cours.fr/#1}{#2 \tiny \faExternalLink}
}

\newcommand{\euro}{\eurologo{}}

%================================================================================================================================
%
% Macros - Environement
%
%================================================================================================================================

\newenvironment{tex}{ %
}
{%
}

\newenvironment{indente}{ %
	\setlength\parindent{10mm}
}

{
	\setlength\parindent{0mm}
}

\newenvironment{corrige}{%
     \needspace{3\baselineskip}
     \medskip
     \textbf{\textsc{Corrigé}}
     \medskip
}
{
}

\newenvironment{extern}{%
     \begin{center}
     }
     {
     \end{center}
}

\NewEnviron{code}{%
	\par
     \boite{gray}{\texttt{%
     \BODY
     }}
     \par
}

\newenvironment{vbloc}{% boite sans cadre empeche saut de page
     \begin{minipage}[t]{\linewidth}
     }
     {
     \end{minipage}
}
\NewEnviron{h2}{%
    \needspace{3\baselineskip}
    \vspace{0.6cm}
	\noindent \MakeUppercase{\sffamily \large \BODY}
	\vspace{1mm}\textcolor{mcgris}{\hrule}\vspace{0.4cm}
	\par
}{}

\NewEnviron{h3}{%
    \needspace{3\baselineskip}
	\vspace{5mm}
	\textsc{\BODY}
	\par
}

\NewEnviron{margeneg}{ %
\begin{addmargin}[-1cm]{0cm}
\BODY
\end{addmargin}
}

\NewEnviron{html}{%
}

\begin{document}
\meta{url}{/methode/algorithme-de-dijkstra-etape-par-etape/}
\meta{pid}{7015}
\meta{titre}{Algorithme de Dijkstra - Etape par etape}
\meta{type}{methode}
L'algorithme de Dijkstra (\textit{prononcer approximativement « Dextra »}) permet de trouver \textbf{le plus court chemin entre deux sommets d'un graphe} (orienté ou non orienté).
Dans l'exemple du graphe ci-dessous, on va rechercher le chemin le plus court menant de M à S.
\begin{center}
     \begin{extern}%width="300"
          \psset{unit=0.7cm}
          \begin{pspicture}(8,18)(21,5)
               \rput(10,10){\circlenode{E}{E}}
               \rput(16,11){\circlenode{L}{L}}
               \rput(16,6){\circlenode{M}{M}}
               \rput(19,9){\circlenode{N}{N}}
               \rput(17,17){\circlenode{S}{S}}
               \rput(10,14){\circlenode{T}{T}}
               \ncarc[arcangle=20]{E}{L}\ncput*[nrot=:U]{8}
               \ncarc[arcangle=-30]{M}{L}\ncput*[nrot=:U]{7}
               \ncarc[arcangle=-30]{M}{N}\ncput*[nrot=:U]{4}
               \ncarc[arcangle=20]{L}{N}\ncput*[nrot=:U]{2}
               \ncarc[arcangle=20]{E}{S}\ncput*[nrot=:U]{10}
               \ncarc[arcangle=20]{L}{S}\ncput*[nrot=:U]{5}
               \ncarc[arcangle=-30]{T}{E}\ncput*[nrot=:U]{4}
               \ncarc[arcangle=20]{T}{S}\ncput*[nrot=:U]{8}
               \ncarc[arcangle=-50]{E}{M}\ncput*[nrot=:U]{10}
               \ncarc[arcangle=20]{S}{N}\ncput*[nrot=:U]{8}
          \end{pspicture}
     \end{extern}
\end{center}
\begin{h2}Initialisation :\end{h2}
On construit un tableau ayant pour colonnes chacun des sommets du graphe. On ajoute à gauche une colonne qui recensera les sommets choisis à chaque étape (cette colonne est facultative mais facilitera la compréhension de l'algorithme).
\par
Puisque l'on part du sommet M, on inscrit, sur la première ligne intitulée \og Départ \fg{}, $0_{\text{M}}$ \textbf{dans la colonne M et $\bm{\infty}$ dans les autres colonnes}.
\par
\textit{Cela signifie qu'à ce stade, on peut rejoindre M en 0 minute et on n'a rejoint aucun autre sommet puisque l'on n'a pas encore emprunté de chemin...}
\begin{center}
     \begin{extern}
          \begin{tabularx}{0.9\linewidth}{|c|C|C|C|C|C|C|}
               \hline
               \			&  E 						& L							& M							& N 							& S								& T  						\\ \hline
               Départ			&  $\infty$	 				& $\infty$					& $0_{\text{M}}$				& $\infty$					& $\infty$						& $\infty$	  				\\ \hline
               \ 				&  \ 						& \ 							& \							& \ 							& \ 								& \ 											\\
          \end{tabularx}
     \end{extern}
\end{center}
\begin{h2}\'Etape 1 :\end{h2}
On sélectionne \textbf{le plus petit résultat} de la dernière ligne. Ici, c'est \og $0_{\text{M}}$ \fg{} qui correspond au chemin menant au \textbf{sommet M} en 0 minute.
\begin{itemize}
     \item \textbf{On met en évidence cette sélection} (nous l'écrirons en rouge mais il est également possible de la souligner, de l'entourer, etc.).
     \item \textbf{On inscrit le sommet retenu et la durée correspondante dans la première colonne} (ici on écrit M(0)).
     \item \textbf{On désactive les cases situées en dessous de notre sélection} en les grisant par exemple. En effet, on a trouvé le trajet le plus court menant à M ; il sera inutile d'en chercher d'autres.
\end{itemize}
\begin{center}
     \begin{extern}
          \begin{tabularx}{0.9\linewidth}{|c|C|C|C|C|C|C|}
               \hline
               \			&  E 						& L							& M							& N 							& S								& T  						\\ \hline
               Départ			&  $\infty$	 				& $\infty$					& $\color{red}0_{\text{M}}$	& $\infty$					& $\infty$						& $\infty$	  				\\ \hline
               M (0) 			&  \ 						& \ 							& \cellcolor{black!20}		& \ 							& \ 								& \ 											\\ \hline
               \ 				&  \ 						& \ 							& \cellcolor{black!20}		& \ 							& \ 								& \ 											\\
          \end{tabularx}
     \end{extern}
\end{center}
\`A partir de M, on voit sur le graphe que l'on peut rejoindre E, L et N en respectivement 10, 7 et 4 minutes. Ces durées sont les durées les plus courtes ; elles sont inférieures au durées inscrites sur la ligne précédente qui étaient \og $\infty$ \fg{}.
\par
\textbf{On inscrit donc $\bm{10_{\text{M}}, 7_{\text{M}}}$ et $\bm{4_{\text{M}}}$ dans les colonnes E, L et N}. Le M situé en indice signifie que l'on vient du sommet M.
\par
Enfin on complète la ligne en recopiant dans les cellules vides les valeurs de la ligne précédente.
\begin{center}
     \begin{extern}
          \begin{tabularx}{0.9\linewidth}{|c|C|C|C|C|C|C|}
               \hline
               \			&  E 						& L							& M							& N 							& S								& T  						\\ \hline
               Départ			&  $\infty$	 				& $\infty$					&  $\color{red}0_{\text{M}}$	& $\infty$					& $\infty$						& $\infty$	  				\\ \hline
               M (0) 			&  $10_{\text{M}}$	 		& $7_{\text{M}}$	 			& \cellcolor{black!20}		& $4_{\text{M}}$	& $\infty$						& $\infty$ 					\\ \hline
               \ 				&  \ 						& \ 							& \cellcolor{black!20}		& \ 							& \ 								& \ 											\\
          \end{tabularx}
     \end{extern}
\end{center}
\begin{h2}\'Etape 2 :\end{h2}
On sélectionne \textbf{le plus petit résultat} de la dernière ligne. Ici, c'est \og $4_{\text{M}}$ \fg{} qui correspond au chemin menant au \textbf{sommet N} en 4 minutes.
\begin{itemize}
     \item \textbf{On met en évidence cette sélection}.
     \item \textbf{On inscrit le sommet retenu et la durée correspondante dans la première colonne} : N (4).
     \item \textbf{On désactive les cases situées en dessous de notre sélection}. On a trouvé le trajet le plus court menant à N ; il dure 4 minutes.
\end{itemize}
\begin{center}
     \begin{extern}
          \begin{tabularx}{0.9\linewidth}{|c|C|C|C|C|C|C|}
               \hline
               \			&  E 						& L							& M							& N 							& S								& T  						\\ \hline
               Départ			&  $\infty$	 				& $\infty$					& $\color{red}0_{\text{M}}$	& $\infty$					& $\infty$						& $\infty$	  				\\ \hline
               M (0) 			&  $10_{\text{M}}$	 		& $7_{\text{M}}$	 			& \cellcolor{black!20}		& $\color{red}4_{\text{M}}$	& $\infty$						& $\infty$ 					\\ \hline
               N (4)			&  \ 						& \ 							& \cellcolor{black!20}		& \ 				\cellcolor{black!20}			& \ 								& \ 											\\ \hline
               \ 				&  \ 						& \ 							& \cellcolor{black!20}		& \cellcolor{black!20}	 							& \
          \end{tabularx}
     \end{extern}
\end{center}
\`A partir de N, on peut rejoindre L et S (on ne se préoccupe plus de M qui a été \og désactivé \fg{}).
\begin{itemize}
     \item \textbf{Si l'on rejoint L :} On mettra 2 minutes pour aller de N à L et 4 minutes pour aller de M à N (ces 4 minutes sont inscrites dans la première colonne) soit au total 6 minutes. \textbf{Ce trajet est plus rapide que le précédent} qui durait 7 minutes. \textbf{On indique donc $\bm{6_{\text{N}}}$ dans la colonne L}. Le N situé en indice signifie que l'on vient du sommet N.
     \item \textbf{Si l'on rejoint S :} On mettra 8 minutes pour aller de N à S et 4 minutes pour aller de M à N soit au total 12 minutes. \textbf{Ce trajet est plus rapide que le précédent} qui était $\infty$. \textbf{On indique donc $\bm{12_{\text{N}}}$ dans la colonne S}.
\end{itemize}
Puis on complète la ligne en recopiant dans les cellules vides les valeurs de la ligne précédente.
\begin{center}
     \begin{extern}
          \begin{tabularx}{0.9\linewidth}{|c|C|C|C|C|C|C|}
               \hline
               \			&  E 						& L							& M							& N 							& S								& T  						\\ \hline
               Départ			&  $\infty$	 				& $\infty$					& $\color{red}0_{\text{M}}$	& $\infty$					& $\infty$						& $\infty$	  				\\ \hline
               M (0) 			&  $10_{\text{M}}$	 		& $7_{\text{M}}$	 			& \cellcolor{black!20}		& $\color{red}4_{\text{M}}$	& $\infty$						& $\infty$ 					\\ \hline
               N (4)			&  $10_{\text{M}}$	 		& $6_{\text{N}}$	& \cellcolor{black!20}		& \cellcolor{black!20}		& $12_{\text{N}}$				& $\infty$ 					\\ \hline
               \ 				&  \ 						& \ 							& \cellcolor{black!20}		& \cellcolor{black!20}	 							& \
          \end{tabularx}
     \end{extern}
\end{center}
\begin{h2}\'Etape 3 :\end{h2}
On sélectionne \textbf{le plus petit résultat} de la dernière ligne. Ici, c'est \og $6_{\text{N}}$ \fg{} qui correspond au chemin menant au \textbf{sommet L} en 6 minutes.
\begin{itemize}
     \item \textbf{On met en évidence cette sélection}.
     \item \textbf{On inscrit le sommet retenu et la durée correspondante dans la première colonne} : L (6).
     \item \textbf{On désactive les cases situées en dessous de notre sélection}. On a trouvé le trajet le plus court menant à L ; il dure 6 minutes.
\end{itemize}
\begin{center}
     \begin{extern}
          \begin{tabularx}{0.9\linewidth}{|c|C|C|C|C|C|C|}
               \hline
               \			&  E 						& L							& M							& N 							& S								& T  						\\ \hline
               Départ			&  $\infty$	 				& $\infty$					& $\color{red}0_{\text{M}}$	& $\infty$					& $\infty$						& $\infty$	  				\\ \hline
               M (0) 			&  $10_{\text{M}}$	 		& $7_{\text{M}}$	 			& \cellcolor{black!20}		& $\color{red}4_{\text{M}}$	& $\infty$						& $\infty$ 					\\ \hline
               N (4)			&  $10_{\text{M}}$	 		& $\color{red}6_{\text{N}}$	& \cellcolor{black!20}		& \cellcolor{black!20}		& $12_{\text{N}}$				& $\infty$ 					\\ \hline
               L (6)			&  \ 	& \cellcolor{black!20}		& \cellcolor{black!20}		& \cellcolor{black!20}		& \				& \ 					\\ \hline
               &  \ 	& \cellcolor{black!20}		& \cellcolor{black!20}		& \cellcolor{black!20}		& \				& \ 					\\
          \end{tabularx}
     \end{extern}
\end{center}
\`A partir de L, on peut rejoindre E et S (on ne se préoccupe plus de M ni de N qui ont été \og désactivés \fg{}).
\begin{itemize}
     \item \textbf{Si l'on rejoint E :} On mettra 8 minutes pour aller de L à E et 6 minutes pour aller de M à L soit, au total, 14 minutes. \textbf{Ce trajet N'EST PAS plus rapide que le précédent} qui durait 10 minutes.
     \par
     \textbf{On se contente donc de recopier le contenu précédent $\bm{10_{\text{M}}}$ dans la colonne E}.
     \item \textbf{Si l'on rejoint S :} On mettra 5 minutes pour aller de L à S et 6 minutes pour aller de M à L soit au total 11 minutes. \textbf{Ce trajet est plus rapide que le précédent} qui durait 12 minutes. \textbf{On indique donc $\bm{11_{\text{L}}}$ dans la colonne S}.
\end{itemize}
\cadre{vert}{Important !}{
     On inscrit la durée d'un trajet dans le tableau \textbf{uniquement si elle est inférieure} à la durée figurant sur la ligne précédente.
     Dans le cas contraire, on recopie la valeur précédente.
}
Puis on complète la ligne en recopiant dans les cellules vides les valeurs de la ligne précédente.
\begin{center}
     \begin{extern}
          \begin{tabularx}{0.9\linewidth}{|c|C|C|C|C|C|C|}
               \hline
               \			&  E 						& L							& M							& N 							& S								& T  						\\ \hline
               Départ			&  $\infty$	 				& $\infty$					& $\color{red}0_{\text{M}}$	& $\infty$					& $\infty$						& $\infty$	  				\\ \hline
               M (0) 			&  $10_{\text{M}}$	 		& $7_{\text{M}}$	 			& \cellcolor{black!20}		& $\color{red}4_{\text{M}}$	& $\infty$						& $\infty$ 					\\ \hline
               N (4)			&  $10_{\text{M}}$	 		& $\color{red}6_{\text{N}}$	& \cellcolor{black!20}		& \cellcolor{black!20}		& $12_{\text{N}}$				& $\infty$ 					\\ \hline
               L (6)			&  $10_{\text{M}}$	& \cellcolor{black!20}		& \cellcolor{black!20}		& \cellcolor{black!20}		& $11_{\text{L}}$				& $\infty$ 					\\ \hline
               &  \ 	& \cellcolor{black!20}		& \cellcolor{black!20}		& \cellcolor{black!20}		& \				& \ 					\\
          \end{tabularx}
     \end{extern}
\end{center}
\begin{h2}\'Etape 4 :\end{h2}
On sélectionne \textbf{le plus petit résultat}. C'est \og $10_{\text{M}}$ \fg{} qui correspond au chemin menant au \textbf{sommet E} en 10 minutes.
\begin{itemize}
     \item \textbf{On met en évidence cette sélection}.
     \item \textbf{On inscrit le sommet retenu et la durée correspondante dans la première colonne} : E (10).
     \item \textbf{On désactive les cases situées en dessous de notre sélection}. On a trouvé le trajet le plus court menant à E ; il dure 10 minutes.
\end{itemize}
\begin{center}
     \begin{extern}
          \begin{tabularx}{0.9\linewidth}{|c|C|C|C|C|C|C|}
               \hline
               \			&  E 						& L							& M							& N 							& S								& T  						\\ \hline
               Départ			&  $\infty$	 				& $\infty$					& $\color{red}0_{\text{M}}$	& $\infty$					& $\infty$						& $\infty$	  				\\ \hline
               M (0) 			&  $10_{\text{M}}$	 		& $7_{\text{M}}$	 			& \cellcolor{black!20}		& $\color{red}4_{\text{M}}$	& $\infty$						& $\infty$ 					\\ \hline
               N (4)			&  $10_{\text{M}}$	 		& $\color{red}6_{\text{N}}$	& \cellcolor{black!20}		& \cellcolor{black!20}		& $12_{\text{N}}$				& $\infty$ 					\\ \hline
               L (6)			&  $\color{red}10_{\text{M}}$	& \cellcolor{black!20}		& \cellcolor{black!20}		& \cellcolor{black!20}		& $11_{\text{L}}$				& $\infty$ 					\\ \hline
               E (10)			&  \cellcolor{black!20}		& \cellcolor{black!20}		& \cellcolor{black!20}		& \cellcolor{black!20}		& \		& \	 		\\ \hline
               &  \cellcolor{black!20}		& \cellcolor{black!20}		& \cellcolor{black!20}		& \cellcolor{black!20}		& \	& \	 		\\ \hline
          \end{tabularx}
     \end{extern}
\end{center}
\`A partir de E, on peut rejoindre S et T (on ne se préoccupe plus des autres sommets qui ont été \og désactivés \fg{}).
\begin{itemize}
     \item \textbf{Si l'on rejoint S :} On mettra 10 minutes pour aller de E à S et 10 minutes pour aller de M à E (ces 10 minutes sont inscrites dans la première colonne) soit au total 20 minutes.
     \par
     \textbf{Ce trajet N'EST PAS plus rapide que le précédent} qui durait 11 minutes. \textbf{On se contente donc de recopier le contenu précédent $\bm{11_{\text{L}}}$ dans la colonne S}.
     \item \textbf{Si l'on rejoint T :} On mettra 4 minutes pour aller de E à T et 10 minutes pour aller de M à E soit au total 14 minutes. \textbf{Ce trajet est plus rapide que le précédent} qui était $\infty$. \textbf{On indique donc $\bm{14_{\text{E}}}$ dans la colonne T}.
\end{itemize}
\begin{center}
     \begin{extern}
          \begin{tabularx}{0.9\linewidth}{|c|C|C|C|C|C|C|}
               \hline
               \			&  E 						& L							& M							& N 							& S								& T  						\\ \hline
               Départ			&  $\infty$	 				& $\infty$					& $\color{red}0_{\text{M}}$	& $\infty$					& $\infty$						& $\infty$	  				\\ \hline
               M (0) 			&  $10_{\text{M}}$	 		& $7_{\text{M}}$	 			& \cellcolor{black!20}		& $\color{red}4_{\text{M}}$	& $\infty$						& $\infty$ 					\\ \hline
               N (4)			&  $10_{\text{M}}$	 		& $\color{red}6_{\text{N}}$	& \cellcolor{black!20}		& \cellcolor{black!20}		& $12_{\text{N}}$				& $\infty$ 					\\ \hline
               L (6)			&  $\color{red}10_{\text{M}}$	& \cellcolor{black!20}		& \cellcolor{black!20}		& \cellcolor{black!20}		& $11_{\text{L}}$				& $\infty$ 					\\ \hline
               E (10)			&  \cellcolor{black!20}		& \cellcolor{black!20}		& \cellcolor{black!20}		& \cellcolor{black!20}		& $11_{\text{L}}$		& $14_{\text{E}}$	 		\\ \hline
               &  \cellcolor{black!20}		& \cellcolor{black!20}		& \cellcolor{black!20}		& \cellcolor{black!20}		& \	& \	 		\\ \hline
          \end{tabularx}
     \end{extern}
\end{center}
\begin{h2}\'Etape 5 :\end{h2}
On sélectionne \textbf{le plus petit résultat}. C'est \og $11_{\text{L}}$ \fg{} qui correspond au chemin menant au \textbf{sommet S} en 11 minutes.
\par
On a trouvé le trajet le plus court menant à S : il dure \textbf{11 minutes}. Comme c'est la question posée dans l'énoncé, il est inutile d'aller plus loin et le tableau est terminé !
\begin{center}
     \begin{extern}
          \begin{tabularx}{0.9\linewidth}{|c|C|C|C|C|C|C|}
               \hline
               \			&  E 						& L							& M							& N 							& S								& T  						\\ \hline
               Départ			&  $\infty$	 				& $\infty$					& $\color{red}0_{\text{M}}$	& $\infty$					& $\infty$						& $\infty$	  				\\ \hline
               M (0) 			&  $10_{\text{M}}$	 		& $7_{\text{M}}$	 			& \cellcolor{black!20}		& $\color{red}4_{\text{M}}$	& $\infty$						& $\infty$ 					\\ \hline
               N (4)			&  $10_{\text{M}}$	 		& $\color{red}6_{\text{N}}$	& \cellcolor{black!20}		& \cellcolor{black!20}		& $12_{\text{N}}$				& $\infty$ 					\\ \hline
               L (6)			&  $\color{red}10_{\text{M}}$	& \cellcolor{black!20}		& \cellcolor{black!20}		& \cellcolor{black!20}		& $11_{\text{L}}$				& $\infty$ 					\\ \hline
               E (10)			&  \cellcolor{black!20}		& \cellcolor{black!20}		& \cellcolor{black!20}		& \cellcolor{black!20}		& $\color{red}11_{\text{L}}$		& $14_{\text{E}}$	 		\\ \hline
          \end{tabularx}
     \end{extern}
\end{center}
Il reste toutefois à reconstituer le trajet qui correspond à cette durée de 11 minutes.
En pratique, il est plus facile de trouver le trajet en sens inverse en \og remontant \fg{} dans le tableau de la façon suivante :
\begin{itemize}
     \item On part de notre point d'arrivée : \textbf{S}
     \item On recherche la cellule marquée en rouge de la colonne \textbf{S} ; elle contient $\color{red}{11_{\text{L}}}$. On note la lettre écrite en indice : \textbf{L}.
     \item On recherche la cellule marquée en rouge de la colonne \textbf{L} ; elle contient $\color{red}{6_{\text{N}}}$. On note la lettre écrite en indice : \textbf{N}.
     \item On recherche la cellule marquée en rouge de la colonne \textbf{N} ; elle contient $\color{red}{4_{\text{M}}}$. On note la lettre écrite en indice : \textbf{M}.
\end{itemize}
On est arrivé à notre point de départ M après être passé par N et L et S (liste obtenue en listant les sommets en ordre inverse).
\par
Le trajet optimal est donc \textbf{M - N - L - S}.
\par
Enfin, on peut vérifier sur le graphe que ce trajet est correct et dure 11 minutes !

\end{document}
µ
\documentclass[a4paper]{article}

%================================================================================================================================
%
% Packages
%
%================================================================================================================================

\usepackage[T1]{fontenc} 	% pour caractères accentués
\usepackage[utf8]{inputenc}  % encodage utf8
\usepackage[french]{babel}	% langue : français
\usepackage{fourier}			% caractères plus lisibles
\usepackage[dvipsnames]{xcolor} % couleurs
\usepackage{fancyhdr}		% réglage header footer
\usepackage{needspace}		% empêcher sauts de page mal placés
\usepackage{graphicx}		% pour inclure des graphiques
\usepackage{enumitem,cprotect}		% personnalise les listes d'items (nécessaire pour ol, al ...)
\usepackage{hyperref}		% Liens hypertexte
\usepackage{pstricks,pst-all,pst-node,pstricks-add,pst-math,pst-plot,pst-tree,pst-eucl} % pstricks
\usepackage[a4paper,includeheadfoot,top=2cm,left=3cm, bottom=2cm,right=3cm]{geometry} % marges etc.
\usepackage{comment}			% commentaires multilignes
\usepackage{amsmath,environ} % maths (matrices, etc.)
\usepackage{amssymb,makeidx}
\usepackage{bm}				% bold maths
\usepackage{tabularx}		% tableaux
\usepackage{colortbl}		% tableaux en couleur
\usepackage{fontawesome}		% Fontawesome
\usepackage{environ}			% environment with command
\usepackage{fp}				% calculs pour ps-tricks
\usepackage{multido}			% pour ps tricks
\usepackage[np]{numprint}	% formattage nombre
\usepackage{tikz,tkz-tab} 			% package principal TikZ
\usepackage{pgfplots}   % axes
\usepackage{mathrsfs}    % cursives
\usepackage{calc}			% calcul taille boites
\usepackage[scaled=0.875]{helvet} % font sans serif
\usepackage{svg} % svg
\usepackage{scrextend} % local margin
\usepackage{scratch} %scratch
\usepackage{multicol} % colonnes
%\usepackage{infix-RPN,pst-func} % formule en notation polanaise inversée
\usepackage{listings}

%================================================================================================================================
%
% Réglages de base
%
%================================================================================================================================

\lstset{
language=Python,   % R code
literate=
{á}{{\'a}}1
{à}{{\`a}}1
{ã}{{\~a}}1
{é}{{\'e}}1
{è}{{\`e}}1
{ê}{{\^e}}1
{í}{{\'i}}1
{ó}{{\'o}}1
{õ}{{\~o}}1
{ú}{{\'u}}1
{ü}{{\"u}}1
{ç}{{\c{c}}}1
{~}{{ }}1
}


\definecolor{codegreen}{rgb}{0,0.6,0}
\definecolor{codegray}{rgb}{0.5,0.5,0.5}
\definecolor{codepurple}{rgb}{0.58,0,0.82}
\definecolor{backcolour}{rgb}{0.95,0.95,0.92}

\lstdefinestyle{mystyle}{
    backgroundcolor=\color{backcolour},   
    commentstyle=\color{codegreen},
    keywordstyle=\color{magenta},
    numberstyle=\tiny\color{codegray},
    stringstyle=\color{codepurple},
    basicstyle=\ttfamily\footnotesize,
    breakatwhitespace=false,         
    breaklines=true,                 
    captionpos=b,                    
    keepspaces=true,                 
    numbers=left,                    
xleftmargin=2em,
framexleftmargin=2em,            
    showspaces=false,                
    showstringspaces=false,
    showtabs=false,                  
    tabsize=2,
    upquote=true
}

\lstset{style=mystyle}


\lstset{style=mystyle}
\newcommand{\imgdir}{C:/laragon/www/newmc/assets/imgsvg/}
\newcommand{\imgsvgdir}{C:/laragon/www/newmc/assets/imgsvg/}

\definecolor{mcgris}{RGB}{220, 220, 220}% ancien~; pour compatibilité
\definecolor{mcbleu}{RGB}{52, 152, 219}
\definecolor{mcvert}{RGB}{125, 194, 70}
\definecolor{mcmauve}{RGB}{154, 0, 215}
\definecolor{mcorange}{RGB}{255, 96, 0}
\definecolor{mcturquoise}{RGB}{0, 153, 153}
\definecolor{mcrouge}{RGB}{255, 0, 0}
\definecolor{mclightvert}{RGB}{205, 234, 190}

\definecolor{gris}{RGB}{220, 220, 220}
\definecolor{bleu}{RGB}{52, 152, 219}
\definecolor{vert}{RGB}{125, 194, 70}
\definecolor{mauve}{RGB}{154, 0, 215}
\definecolor{orange}{RGB}{255, 96, 0}
\definecolor{turquoise}{RGB}{0, 153, 153}
\definecolor{rouge}{RGB}{255, 0, 0}
\definecolor{lightvert}{RGB}{205, 234, 190}
\setitemize[0]{label=\color{lightvert}  $\bullet$}

\pagestyle{fancy}
\renewcommand{\headrulewidth}{0.2pt}
\fancyhead[L]{maths-cours.fr}
\fancyhead[R]{\thepage}
\renewcommand{\footrulewidth}{0.2pt}
\fancyfoot[C]{}

\newcolumntype{C}{>{\centering\arraybackslash}X}
\newcolumntype{s}{>{\hsize=.35\hsize\arraybackslash}X}

\setlength{\parindent}{0pt}		 
\setlength{\parskip}{3mm}
\setlength{\headheight}{1cm}

\def\ebook{ebook}
\def\book{book}
\def\web{web}
\def\type{web}

\newcommand{\vect}[1]{\overrightarrow{\,\mathstrut#1\,}}

\def\Oij{$\left(\text{O}~;~\vect{\imath},~\vect{\jmath}\right)$}
\def\Oijk{$\left(\text{O}~;~\vect{\imath},~\vect{\jmath},~\vect{k}\right)$}
\def\Ouv{$\left(\text{O}~;~\vect{u},~\vect{v}\right)$}

\hypersetup{breaklinks=true, colorlinks = true, linkcolor = OliveGreen, urlcolor = OliveGreen, citecolor = OliveGreen, pdfauthor={Didier BONNEL - https://www.maths-cours.fr} } % supprime les bordures autour des liens

\renewcommand{\arg}[0]{\text{arg}}

\everymath{\displaystyle}

%================================================================================================================================
%
% Macros - Commandes
%
%================================================================================================================================

\newcommand\meta[2]{    			% Utilisé pour créer le post HTML.
	\def\titre{titre}
	\def\url{url}
	\def\arg{#1}
	\ifx\titre\arg
		\newcommand\maintitle{#2}
		\fancyhead[L]{#2}
		{\Large\sffamily \MakeUppercase{#2}}
		\vspace{1mm}\textcolor{mcvert}{\hrule}
	\fi 
	\ifx\url\arg
		\fancyfoot[L]{\href{https://www.maths-cours.fr#2}{\black \footnotesize{https://www.maths-cours.fr#2}}}
	\fi 
}


\newcommand\TitreC[1]{    		% Titre centré
     \needspace{3\baselineskip}
     \begin{center}\textbf{#1}\end{center}
}

\newcommand\newpar{    		% paragraphe
     \par
}

\newcommand\nosp {    		% commande vide (pas d'espace)
}
\newcommand{\id}[1]{} %ignore

\newcommand\boite[2]{				% Boite simple sans titre
	\vspace{5mm}
	\setlength{\fboxrule}{0.2mm}
	\setlength{\fboxsep}{5mm}	
	\fcolorbox{#1}{#1!3}{\makebox[\linewidth-2\fboxrule-2\fboxsep]{
  		\begin{minipage}[t]{\linewidth-2\fboxrule-4\fboxsep}\setlength{\parskip}{3mm}
  			 #2
  		\end{minipage}
	}}
	\vspace{5mm}
}

\newcommand\CBox[4]{				% Boites
	\vspace{5mm}
	\setlength{\fboxrule}{0.2mm}
	\setlength{\fboxsep}{5mm}
	
	\fcolorbox{#1}{#1!3}{\makebox[\linewidth-2\fboxrule-2\fboxsep]{
		\begin{minipage}[t]{1cm}\setlength{\parskip}{3mm}
	  		\textcolor{#1}{\LARGE{#2}}    
 	 	\end{minipage}  
  		\begin{minipage}[t]{\linewidth-2\fboxrule-4\fboxsep}\setlength{\parskip}{3mm}
			\raisebox{1.2mm}{\normalsize\sffamily{\textcolor{#1}{#3}}}						
  			 #4
  		\end{minipage}
	}}
	\vspace{5mm}
}

\newcommand\cadre[3]{				% Boites convertible html
	\par
	\vspace{2mm}
	\setlength{\fboxrule}{0.1mm}
	\setlength{\fboxsep}{5mm}
	\fcolorbox{#1}{white}{\makebox[\linewidth-2\fboxrule-2\fboxsep]{
  		\begin{minipage}[t]{\linewidth-2\fboxrule-4\fboxsep}\setlength{\parskip}{3mm}
			\raisebox{-2.5mm}{\sffamily \small{\textcolor{#1}{\MakeUppercase{#2}}}}		
			\par		
  			 #3
 	 		\end{minipage}
	}}
		\vspace{2mm}
	\par
}

\newcommand\bloc[3]{				% Boites convertible html sans bordure
     \needspace{2\baselineskip}
     {\sffamily \small{\textcolor{#1}{\MakeUppercase{#2}}}}    
		\par		
  			 #3
		\par
}

\newcommand\CHelp[1]{
     \CBox{Plum}{\faInfoCircle}{À RETENIR}{#1}
}

\newcommand\CUp[1]{
     \CBox{NavyBlue}{\faThumbsOUp}{EN PRATIQUE}{#1}
}

\newcommand\CInfo[1]{
     \CBox{Sepia}{\faArrowCircleRight}{REMARQUE}{#1}
}

\newcommand\CRedac[1]{
     \CBox{PineGreen}{\faEdit}{BIEN R\'EDIGER}{#1}
}

\newcommand\CError[1]{
     \CBox{Red}{\faExclamationTriangle}{ATTENTION}{#1}
}

\newcommand\TitreExo[2]{
\needspace{4\baselineskip}
 {\sffamily\large EXERCICE #1\ (\emph{#2 points})}
\vspace{5mm}
}

\newcommand\img[2]{
          \includegraphics[width=#2\paperwidth]{\imgdir#1}
}

\newcommand\imgsvg[2]{
       \begin{center}   \includegraphics[width=#2\paperwidth]{\imgsvgdir#1} \end{center}
}


\newcommand\Lien[2]{
     \href{#1}{#2 \tiny \faExternalLink}
}
\newcommand\mcLien[2]{
     \href{https~://www.maths-cours.fr/#1}{#2 \tiny \faExternalLink}
}

\newcommand{\euro}{\eurologo{}}

%================================================================================================================================
%
% Macros - Environement
%
%================================================================================================================================

\newenvironment{tex}{ %
}
{%
}

\newenvironment{indente}{ %
	\setlength\parindent{10mm}
}

{
	\setlength\parindent{0mm}
}

\newenvironment{corrige}{%
     \needspace{3\baselineskip}
     \medskip
     \textbf{\textsc{Corrigé}}
     \medskip
}
{
}

\newenvironment{extern}{%
     \begin{center}
     }
     {
     \end{center}
}

\NewEnviron{code}{%
	\par
     \boite{gray}{\texttt{%
     \BODY
     }}
     \par
}

\newenvironment{vbloc}{% boite sans cadre empeche saut de page
     \begin{minipage}[t]{\linewidth}
     }
     {
     \end{minipage}
}
\NewEnviron{h2}{%
    \needspace{3\baselineskip}
    \vspace{0.6cm}
	\noindent \MakeUppercase{\sffamily \large \BODY}
	\vspace{1mm}\textcolor{mcgris}{\hrule}\vspace{0.4cm}
	\par
}{}

\NewEnviron{h3}{%
    \needspace{3\baselineskip}
	\vspace{5mm}
	\textsc{\BODY}
	\par
}

\NewEnviron{margeneg}{ %
\begin{addmargin}[-1cm]{0cm}
\BODY
\end{addmargin}
}

\NewEnviron{html}{%
}

\begin{document}
\meta{url}{/exercices/qcm-bac-es-l-pondichery-2018/}
\meta{pid}{7028}
\meta{titre}{QCM – Bac ES/L Pondichéry 2018}
\meta{type}{exercice}
\begin{h2}Exercice 1 (5 points)\end{h2}
\textbf{Commun à tous les candidats}
\medskip
\emph{Cet exercice est un QCM (questionnaire à choix multiples). Pour chacune des questions posées, une seule des trois réponses est exacte. Recopier le numéro de la question et la réponse exacte. Aucune justification n'est demandée. Une réponse exacte rapporte 1~point, une réponse fausse ou l'absence de réponse ne rapporte ni n'enlève de point. Une réponse multiple ne rapporte aucun point.}
\medskip
On considère la fonction $f$ définie sur l'intervalle [0,5~;~5] par~:
\par
\[f(x) = \dfrac{5 + 5\ln x}{x}.\]
\par
Sa représentation graphique est la courbe $\mathcal{C}$ donnée ci-dessous dans un repère d'origine O.
\par
On admet que le point A placé sur le graphique est le seul point d'inflexion de la courbe $\mathcal{C}$ sur l'intervalle [0,5~;~5].
\par
On note B le point de cette courbe d'abscisse e.
\par
On admet que la fonction $f$ est deux fois dérivable sur cet intervalle.
\par
On rappelle que $f'$ désigne la fonction dérivée de la fonction $f$ et $f''$ sa fonction dérivée seconde.
\begin{center}
     \begin{extern}%alt="courbe représentative fonction"
          \definecolor{darkgreen}{rgb}{0.,0.4,0.}
          \psset{xunit=2cm,yunit=1cm,arrowsize=2pt 3}
          \begin{pspicture}(-0.5,-1)(6,7)
               \psgrid[unit=1cm,gridwidth=0.1pt,gridcolor=darkgray,subgriddiv=0,gridlabels=0](0,0)(-1,-1)(12,7)
               \psaxes[linewidth=1pt,ticksize=-2pt 2pt,Dx=1,Dy=1](0,0)(0,0)(5.5,6.5)[$x$,-45][$y$,45]
               \psplot[plotpoints=500,linewidth=1.25pt, linecolor=darkgreen]{0.5}{5}{5 5 x ln mul add x div}
               \uput[dl](-0.05,-0.15){O}
               \psdots[linecolor=black](1.649,4.549)(2.718,3.679)
               \psline[linecolor=black,linestyle=dashed](2.718,3.679)(2.718,0)
               \uput[dl](1.649,4.549){\black A} \uput[dl](2.718,3.679){\black B}
               \uput[d](2.718,0){\black e}
               \uput[u](4.5,2.9){\color{darkgreen} \large $\mathcal{C}$}
          \end{pspicture}
     \end{extern}
\end{center}
On admet que pour tout $x$ de l'intervalle [0,5~;~5] on a~:
\par
$f'(x) = \dfrac{- 5\ln x}{x^2}$ $\qquad\qquad$ $f''(x) = \dfrac{10\ln x - 5}{x^3}$.
\medskip
\begin{enumerate}
     \item La fonction $f'$ est~:
     \begin{enumerate}[label=\alph*.]
          \item positive ou nulle sur l'intervalle [0,5~;~5]
          \item négative ou nulle sur l'intervalle [1~;~5]
          \item négative ou nulle sur l'intervalle [0,5~;~1]
     \end{enumerate}
     \medskip
     \item  Le coefficient directeur de la tangente à la courbe $\mathcal{C}$ au point B est égal à~:
     \begin{tabularx}{\linewidth}{*{3}X}  %class="noborder"
          \textbf{a.~~}$- \dfrac{5}{\text{e}^2}$&\textbf{b.~~}$\dfrac{10}{\text{e}}$&\textbf{c.~~}$ \dfrac{5}{\text{e}^3}$
     \end{tabularx}
     \medskip
     \item  La fonction $f'$ est~:
     \begin{enumerate}[label=\alph*.]
          \item croissante sur l'intervalle [0,5~;~1]
          \item décroissante sur l'intervalle [1~;~5]
          \item croissante sur l'intervalle [2~;~5]
     \end{enumerate}
     \medskip
     \item  La valeur exacte de l'abscisse du point A de la courbe $\mathcal{C}$ est égale à~:
     \begin{tabularx}{\linewidth}{*{3}X} %class="noborder"
          \textbf{a.~~} 1,65 &\textbf{b.~~} 1,6 &\textbf{c.~~} $\text{e}^{0,5}$
     \end{tabularx}
     \medskip
     \item  On note $\mathcal{A}$ l'aire, mesurée en unités d'aire, du domaine plan délimité par la courbe $\mathcal{C}$, l'axe des abscisses et les droites d'équation $x = 1$ et $x = 4$. Cette aire vérifie~:
     \begin{enumerate}[label=\alph*.]
          \item $20 \leqslant \mathcal{A} \leqslant 30$
          \item $10\leqslant \mathcal{A} \leqslant 15$
          \item $5 \leqslant \mathcal{A} \leqslant 8$
     \end{enumerate}
\end{enumerate}
\begin{corrige}
     \begin{enumerate}
          \item
          \textbf{Réponse correcte~:\quad b.}
          \par
          On peut utiliser le graphique ou la formule pour répondre à cette question.
          \begin{itemize}
               \item
               \textbf{\`A l'aide du graphique~:}
               \par
               On voit que la fonction $f$ est décroissante sur l'intervalle $[1~;~5]$. Sa dérivée $f'$ est donc négative ou nulle sur cet intervalle.
               \item
               \textbf{\`A partir de la formule~:}
               \par
               $f'(x) = \dfrac{- 5\ln x}{x^2}$ $\qquad\qquad$
               \par
               Le dénominateur est strictement positif sur l'intervalle $[1~;~5]$.
               \par
               Pour $x \geqslant 1$, $\ln x \geqslant \ln 1 = 0$, donc le numérateur est négatif ou nul.
               \par
               $f'$ est donc négative ou nulle sur l'intervalle $[1~;~5]$.
          \end{itemize}
          \item
          \textbf{Réponse correcte~:\quad a.}
          \par
          Le coefficient directeur de la tangente en $B$ d'abscisse $\text{e}$ à la courbe $\mathscr{C}$ est égal à $f'(e)$.
          \par
          $f'(e)== \dfrac{- 5\ln \text{e}}{\text{e}^2} = - \dfrac{5}{\text{e}^2}$
          \item
          \textbf{Réponse correcte~:\quad\textbf{ c.}}
          \par
          Là encore, on peut utiliser le graphique ou la formule pour répondre à cette question.
          \begin{itemize}
               \item
               \textbf{\`A l'aide du graphique~:}
               \par
               La fonction $f'$ est croissante sur l'intervalle $[2~;~5]$ si et seulement si la courbe $\mathscr{C}$ est convexe sur cet intervalle.
               \par
               On vérifie sur le graphique que c'est bien le cas.
               \item
               \textbf{\`A partir de la formule~:}
               \par
               La fonction $f'$ est croissante sur l'intervalle $[2~;~5]$ si et seulement si sa fonction dérivée $f''$ est positive ou nulle sur cet intervalle.
               \par
               Or pour $x \geqslant 2$~:
               \par
               $x \geqslant 2 \Leftrightarrow \ln x \geqslant \ln 2$ (car la fonction $\ln$ est croissante sur $]0~;~+oo[$)\\
               $\phantom{x \geqslant 2} \Leftrightarrow 10\ln x \geqslant 10\ln 2$\\
               $\phantom{x \geqslant 2} \Leftrightarrow 10\ln x-5 \geqslant 10\ln 2-5 $\\
               \par
               Or,  $10\ln 2-5 ~=1,9$ est positif donc le numérateur de $f''$ est positif sur l'intervalle $[2~;~5]$. Comme son dénominateur est également strictement positif sur cet intervalle, $f''$ est positive sur $[2~;~5]$.
               \par
               La fonction $f'$ est donc croissante sur l'intervalle $[2~;~5]$.
          \end{itemize}
          \item
          \textbf{Réponse correcte~:\quad c.}
          \par
          $A$ est l'unique point d'inflexion de la courbe $\mathscr{C}$. Son abscisse est donc la solution de l'équation $f''(x)=0$.
          \par
          $f''(x)=0 \Leftrightarrow  \dfrac{10\ln x - 5}{x^3}=0$\\
          $\phantom{f''(x)=0} \Leftrightarrow 10\ln x - 5=0$\\
          $\phantom{f''(x)=0} \Leftrightarrow \ln x =\dfrac{5}{10}=0,5$\\
          $\phantom{f''(x)=0} \Leftrightarrow x =\text{e}^{0,5}$\\
          \item
          Réponse correcte~:\quad\textbf{ b.}
          \par
          On compte le nombre de carreaux de la surface colorée ci-dessous.
          \begin{center}
               \begin{extern}%alt="aire sous la courbe"
                    \definecolor{darkgreen}{rgb}{0.,0.4,0.}
                    \psset{xunit=2cm,yunit=1cm,arrowsize=2pt 3}
                    \begin{pspicture}(-0.5,-1)(6,7)
                         \psgrid[unit=1cm,gridwidth=0.1pt,gridcolor=darkgray,subgriddiv=0,gridlabels=0](0,0)(-1,-1)(12,7)
                         \psaxes[linewidth=1pt,ticksize=-2pt 2pt,Dx=1,Dy=1](0,0)(0,0)(5.5,6.5)[$x$,-45][$y$,45]
                         \pscustom[fillstyle=solid,fillcolor=darkgreen,opacity=0.1]{
                              \psplot[plotpoints=500,linewidth=1.25pt, linecolor=darkgreen]{1}{4}{5 5 x ln mul add x div}
                         \psline[linewidth=0.75pt,linecolor=darkgreen](4,0)(1,0)(1,5)}
                         \psplot[plotpoints=500,linewidth=1.25pt, linecolor=darkgreen]{0.5}{5}{5 5 x ln mul add x div}
                         \uput[dl](-0.05,-0.15){O}
                         \psdots[linecolor=black](1.649,4.549)(2.718,3.679)
                         \psline[linecolor=black,linestyle=dashed](2.718,3.679)(2.718,0)
                         \uput[dl](1.649,4.549){\black A} \uput[dl](2.718,3.679){\black B}
                         \uput[d](2.718,0){\black e}
                         \uput[u](4.5,2.9){\color{darkgreen} \large $\mathcal{C}$}
                    \end{pspicture}
               \end{extern}
          \end{center}
          On trouve approximativement 24 carreaux.
          \par
          Chaque carreau a une aire de 0,5 unité d'aire.
          \par
          L'aire cherchée est donc approximativement égale à 12.
     \end{enumerate}
\end{corrige}

\end{document}
µ
\documentclass[a4paper]{article}

%================================================================================================================================
%
% Packages
%
%================================================================================================================================

\usepackage[T1]{fontenc} 	% pour caractères accentués
\usepackage[utf8]{inputenc}  % encodage utf8
\usepackage[french]{babel}	% langue : français
\usepackage{fourier}			% caractères plus lisibles
\usepackage[dvipsnames]{xcolor} % couleurs
\usepackage{fancyhdr}		% réglage header footer
\usepackage{needspace}		% empêcher sauts de page mal placés
\usepackage{graphicx}		% pour inclure des graphiques
\usepackage{enumitem,cprotect}		% personnalise les listes d'items (nécessaire pour ol, al ...)
\usepackage{hyperref}		% Liens hypertexte
\usepackage{pstricks,pst-all,pst-node,pstricks-add,pst-math,pst-plot,pst-tree,pst-eucl} % pstricks
\usepackage[a4paper,includeheadfoot,top=2cm,left=3cm, bottom=2cm,right=3cm]{geometry} % marges etc.
\usepackage{comment}			% commentaires multilignes
\usepackage{amsmath,environ} % maths (matrices, etc.)
\usepackage{amssymb,makeidx}
\usepackage{bm}				% bold maths
\usepackage{tabularx}		% tableaux
\usepackage{colortbl}		% tableaux en couleur
\usepackage{fontawesome}		% Fontawesome
\usepackage{environ}			% environment with command
\usepackage{fp}				% calculs pour ps-tricks
\usepackage{multido}			% pour ps tricks
\usepackage[np]{numprint}	% formattage nombre
\usepackage{tikz,tkz-tab} 			% package principal TikZ
\usepackage{pgfplots}   % axes
\usepackage{mathrsfs}    % cursives
\usepackage{calc}			% calcul taille boites
\usepackage[scaled=0.875]{helvet} % font sans serif
\usepackage{svg} % svg
\usepackage{scrextend} % local margin
\usepackage{scratch} %scratch
\usepackage{multicol} % colonnes
%\usepackage{infix-RPN,pst-func} % formule en notation polanaise inversée
\usepackage{listings}

%================================================================================================================================
%
% Réglages de base
%
%================================================================================================================================

\lstset{
language=Python,   % R code
literate=
{á}{{\'a}}1
{à}{{\`a}}1
{ã}{{\~a}}1
{é}{{\'e}}1
{è}{{\`e}}1
{ê}{{\^e}}1
{í}{{\'i}}1
{ó}{{\'o}}1
{õ}{{\~o}}1
{ú}{{\'u}}1
{ü}{{\"u}}1
{ç}{{\c{c}}}1
{~}{{ }}1
}


\definecolor{codegreen}{rgb}{0,0.6,0}
\definecolor{codegray}{rgb}{0.5,0.5,0.5}
\definecolor{codepurple}{rgb}{0.58,0,0.82}
\definecolor{backcolour}{rgb}{0.95,0.95,0.92}

\lstdefinestyle{mystyle}{
    backgroundcolor=\color{backcolour},   
    commentstyle=\color{codegreen},
    keywordstyle=\color{magenta},
    numberstyle=\tiny\color{codegray},
    stringstyle=\color{codepurple},
    basicstyle=\ttfamily\footnotesize,
    breakatwhitespace=false,         
    breaklines=true,                 
    captionpos=b,                    
    keepspaces=true,                 
    numbers=left,                    
xleftmargin=2em,
framexleftmargin=2em,            
    showspaces=false,                
    showstringspaces=false,
    showtabs=false,                  
    tabsize=2,
    upquote=true
}

\lstset{style=mystyle}


\lstset{style=mystyle}
\newcommand{\imgdir}{C:/laragon/www/newmc/assets/imgsvg/}
\newcommand{\imgsvgdir}{C:/laragon/www/newmc/assets/imgsvg/}

\definecolor{mcgris}{RGB}{220, 220, 220}% ancien~; pour compatibilité
\definecolor{mcbleu}{RGB}{52, 152, 219}
\definecolor{mcvert}{RGB}{125, 194, 70}
\definecolor{mcmauve}{RGB}{154, 0, 215}
\definecolor{mcorange}{RGB}{255, 96, 0}
\definecolor{mcturquoise}{RGB}{0, 153, 153}
\definecolor{mcrouge}{RGB}{255, 0, 0}
\definecolor{mclightvert}{RGB}{205, 234, 190}

\definecolor{gris}{RGB}{220, 220, 220}
\definecolor{bleu}{RGB}{52, 152, 219}
\definecolor{vert}{RGB}{125, 194, 70}
\definecolor{mauve}{RGB}{154, 0, 215}
\definecolor{orange}{RGB}{255, 96, 0}
\definecolor{turquoise}{RGB}{0, 153, 153}
\definecolor{rouge}{RGB}{255, 0, 0}
\definecolor{lightvert}{RGB}{205, 234, 190}
\setitemize[0]{label=\color{lightvert}  $\bullet$}

\pagestyle{fancy}
\renewcommand{\headrulewidth}{0.2pt}
\fancyhead[L]{maths-cours.fr}
\fancyhead[R]{\thepage}
\renewcommand{\footrulewidth}{0.2pt}
\fancyfoot[C]{}

\newcolumntype{C}{>{\centering\arraybackslash}X}
\newcolumntype{s}{>{\hsize=.35\hsize\arraybackslash}X}

\setlength{\parindent}{0pt}		 
\setlength{\parskip}{3mm}
\setlength{\headheight}{1cm}

\def\ebook{ebook}
\def\book{book}
\def\web{web}
\def\type{web}

\newcommand{\vect}[1]{\overrightarrow{\,\mathstrut#1\,}}

\def\Oij{$\left(\text{O}~;~\vect{\imath},~\vect{\jmath}\right)$}
\def\Oijk{$\left(\text{O}~;~\vect{\imath},~\vect{\jmath},~\vect{k}\right)$}
\def\Ouv{$\left(\text{O}~;~\vect{u},~\vect{v}\right)$}

\hypersetup{breaklinks=true, colorlinks = true, linkcolor = OliveGreen, urlcolor = OliveGreen, citecolor = OliveGreen, pdfauthor={Didier BONNEL - https://www.maths-cours.fr} } % supprime les bordures autour des liens

\renewcommand{\arg}[0]{\text{arg}}

\everymath{\displaystyle}

%================================================================================================================================
%
% Macros - Commandes
%
%================================================================================================================================

\newcommand\meta[2]{    			% Utilisé pour créer le post HTML.
	\def\titre{titre}
	\def\url{url}
	\def\arg{#1}
	\ifx\titre\arg
		\newcommand\maintitle{#2}
		\fancyhead[L]{#2}
		{\Large\sffamily \MakeUppercase{#2}}
		\vspace{1mm}\textcolor{mcvert}{\hrule}
	\fi 
	\ifx\url\arg
		\fancyfoot[L]{\href{https://www.maths-cours.fr#2}{\black \footnotesize{https://www.maths-cours.fr#2}}}
	\fi 
}


\newcommand\TitreC[1]{    		% Titre centré
     \needspace{3\baselineskip}
     \begin{center}\textbf{#1}\end{center}
}

\newcommand\newpar{    		% paragraphe
     \par
}

\newcommand\nosp {    		% commande vide (pas d'espace)
}
\newcommand{\id}[1]{} %ignore

\newcommand\boite[2]{				% Boite simple sans titre
	\vspace{5mm}
	\setlength{\fboxrule}{0.2mm}
	\setlength{\fboxsep}{5mm}	
	\fcolorbox{#1}{#1!3}{\makebox[\linewidth-2\fboxrule-2\fboxsep]{
  		\begin{minipage}[t]{\linewidth-2\fboxrule-4\fboxsep}\setlength{\parskip}{3mm}
  			 #2
  		\end{minipage}
	}}
	\vspace{5mm}
}

\newcommand\CBox[4]{				% Boites
	\vspace{5mm}
	\setlength{\fboxrule}{0.2mm}
	\setlength{\fboxsep}{5mm}
	
	\fcolorbox{#1}{#1!3}{\makebox[\linewidth-2\fboxrule-2\fboxsep]{
		\begin{minipage}[t]{1cm}\setlength{\parskip}{3mm}
	  		\textcolor{#1}{\LARGE{#2}}    
 	 	\end{minipage}  
  		\begin{minipage}[t]{\linewidth-2\fboxrule-4\fboxsep}\setlength{\parskip}{3mm}
			\raisebox{1.2mm}{\normalsize\sffamily{\textcolor{#1}{#3}}}						
  			 #4
  		\end{minipage}
	}}
	\vspace{5mm}
}

\newcommand\cadre[3]{				% Boites convertible html
	\par
	\vspace{2mm}
	\setlength{\fboxrule}{0.1mm}
	\setlength{\fboxsep}{5mm}
	\fcolorbox{#1}{white}{\makebox[\linewidth-2\fboxrule-2\fboxsep]{
  		\begin{minipage}[t]{\linewidth-2\fboxrule-4\fboxsep}\setlength{\parskip}{3mm}
			\raisebox{-2.5mm}{\sffamily \small{\textcolor{#1}{\MakeUppercase{#2}}}}		
			\par		
  			 #3
 	 		\end{minipage}
	}}
		\vspace{2mm}
	\par
}

\newcommand\bloc[3]{				% Boites convertible html sans bordure
     \needspace{2\baselineskip}
     {\sffamily \small{\textcolor{#1}{\MakeUppercase{#2}}}}    
		\par		
  			 #3
		\par
}

\newcommand\CHelp[1]{
     \CBox{Plum}{\faInfoCircle}{À RETENIR}{#1}
}

\newcommand\CUp[1]{
     \CBox{NavyBlue}{\faThumbsOUp}{EN PRATIQUE}{#1}
}

\newcommand\CInfo[1]{
     \CBox{Sepia}{\faArrowCircleRight}{REMARQUE}{#1}
}

\newcommand\CRedac[1]{
     \CBox{PineGreen}{\faEdit}{BIEN R\'EDIGER}{#1}
}

\newcommand\CError[1]{
     \CBox{Red}{\faExclamationTriangle}{ATTENTION}{#1}
}

\newcommand\TitreExo[2]{
\needspace{4\baselineskip}
 {\sffamily\large EXERCICE #1\ (\emph{#2 points})}
\vspace{5mm}
}

\newcommand\img[2]{
          \includegraphics[width=#2\paperwidth]{\imgdir#1}
}

\newcommand\imgsvg[2]{
       \begin{center}   \includegraphics[width=#2\paperwidth]{\imgsvgdir#1} \end{center}
}


\newcommand\Lien[2]{
     \href{#1}{#2 \tiny \faExternalLink}
}
\newcommand\mcLien[2]{
     \href{https~://www.maths-cours.fr/#1}{#2 \tiny \faExternalLink}
}

\newcommand{\euro}{\eurologo{}}

%================================================================================================================================
%
% Macros - Environement
%
%================================================================================================================================

\newenvironment{tex}{ %
}
{%
}

\newenvironment{indente}{ %
	\setlength\parindent{10mm}
}

{
	\setlength\parindent{0mm}
}

\newenvironment{corrige}{%
     \needspace{3\baselineskip}
     \medskip
     \textbf{\textsc{Corrigé}}
     \medskip
}
{
}

\newenvironment{extern}{%
     \begin{center}
     }
     {
     \end{center}
}

\NewEnviron{code}{%
	\par
     \boite{gray}{\texttt{%
     \BODY
     }}
     \par
}

\newenvironment{vbloc}{% boite sans cadre empeche saut de page
     \begin{minipage}[t]{\linewidth}
     }
     {
     \end{minipage}
}
\NewEnviron{h2}{%
    \needspace{3\baselineskip}
    \vspace{0.6cm}
	\noindent \MakeUppercase{\sffamily \large \BODY}
	\vspace{1mm}\textcolor{mcgris}{\hrule}\vspace{0.4cm}
	\par
}{}

\NewEnviron{h3}{%
    \needspace{3\baselineskip}
	\vspace{5mm}
	\textsc{\BODY}
	\par
}

\NewEnviron{margeneg}{ %
\begin{addmargin}[-1cm]{0cm}
\BODY
\end{addmargin}
}

\NewEnviron{html}{%
}

\begin{document}
\meta{url}{/exercices/probabilites-bac-es-l-pondichery-2018/}
\meta{pid}{7070}
\meta{titre}{Probabilités – Bac ES/L Pondichéry 2018}
\meta{type}{exercice}
\par
\begin{h2}Exercice 2 (5 points)\end{h2}
\textbf{Commun à tous les candidats}
\medskip
\textit{Les différentes parties de cet exercice peuvent être traitées de façon indépendante.}\\
\textit{Les résultats numériques seront donnés, si nécessaire, sous forme approchée à 0,01~près.}
\begin{center}\begin{h3}Partie A \end{h3}\end{center}
Un commerçant dispose dans sa boutique d'un terminal qui permet à ses clients, s'ils souhaitent
régler leurs achats par carte bancaire, d'utiliser celle-ci en mode sans contact (quand le montant de
la transaction est inférieur ou égal à 30~\euro) ou bien en mode code secret (quel que soit le montant
de la transaction).
\smallskip
Il remarque que~:
\begin{itemize}
     \item 80\,\% de ses clients règlent des sommes inférieures ou égales à 30~\euro. Parmi eux~:
     \begin{itemize}[label=---]
          \item 40\,\% paient en espèces~;
          \item 40\,\% paient avec une carte bancaire en mode sans contact~;
          \item les autres paient avec une carte bancaire en mode code secret.
     \end{itemize}
     \item 20\,\% de ses clients règlent des sommes strictement supérieures à 30~\euro. Parmi eux~:
     \begin{itemize}[label=---]
          \item 70\,\% paient avec une carte bancaire en mode code secret~;
          \item les autres paient en espèces.
     \end{itemize}
\end{itemize}
\smallskip
On interroge au hasard un client qui vient de régler un achat dans la boutique.
\medskip
On considère les événements suivants~:
\begin{itemize}
     \item
     $V$~: \og pour son achat, le client a réglé un montant inférieur ou égal à 30~\euro \fg~;
     \item
     $E$~: \og pour son achat, le client a réglé en espèces \fg~;
     \item
     $C$~: \og pour son achat, le client a réglé avec sa carte bancaire en mode code secret \fg~;
     \item
     $S$~: \og pour son achat, le client a réglé avec sa carte bancaire en mode sans contact \fg.
\end{itemize}
\bigskip
\begin{enumerate}
     \item
     \begin{enumerate}[label=\alph*.]
          \item Donner la probabilité de l'événement $V$, notée $P(V)$, ainsi que la probabilité de $S$ sachant
          $V$ notée $P_V(S)$.
          \item Traduire la situation de l'énoncé à l'aide d'un arbre pondéré.
     \end{enumerate}
     \item
     \begin{enumerate}[label=\alph*.]
          \item Calculer la probabilité que pour son achat, le client ait réglé un montant inférieur ou égal à
          30~\euro{} et qu'il ait utilisé sa carte bancaire en mode sans contact.
          \item Montrer que la probabilité de l'événement~: \og pour son achat, le client a réglé avec sa carte
          bancaire en utilisant l'un des deux modes\fg{} est égale à $0,62$.
     \end{enumerate}
\end{enumerate}
\begin{center}\begin{h3}Partie B \end{h3}\end{center}
On note $X$ la variable aléatoire qui prend pour valeur la dépense en euros d'un client suite à un
achat chez ce commerçant.
\par
On admet que $X$ suit la loi normale de moyenne $27,5$ et d'écart-type $3$.
\par
On interroge au hasard un client qui vient d'effectuer un achat dans la boutique.
\medskip
\begin{enumerate}
     \item Calculer la probabilité que ce client ait dépensé moins de 30~\euro.
     \item Calculer la probabilité que ce client ait dépensé entre 24,50~\euro{} et 30,50~\euro.
\end{enumerate}
\begin{center}\begin{h3}Partie C \end{h3}\end{center}
Une enquête de satisfaction a été réalisée auprès d'un échantillon de $200$ clients de cette boutique.
\par
Parmi eux, 175 trouvent que le dispositif sans contact du terminal est pratique.
\par
Déterminer, avec un niveau de confiance de $0,95$, l'intervalle de confiance de la proportion $p$ de
clients qui trouvent que le dispositif sans contact est pratique.
\par

\end{document}
µ
\documentclass[a4paper]{article}

%================================================================================================================================
%
% Packages
%
%================================================================================================================================

\usepackage[T1]{fontenc} 	% pour caractères accentués
\usepackage[utf8]{inputenc}  % encodage utf8
\usepackage[french]{babel}	% langue : français
\usepackage{fourier}			% caractères plus lisibles
\usepackage[dvipsnames]{xcolor} % couleurs
\usepackage{fancyhdr}		% réglage header footer
\usepackage{needspace}		% empêcher sauts de page mal placés
\usepackage{graphicx}		% pour inclure des graphiques
\usepackage{enumitem,cprotect}		% personnalise les listes d'items (nécessaire pour ol, al ...)
\usepackage{hyperref}		% Liens hypertexte
\usepackage{pstricks,pst-all,pst-node,pstricks-add,pst-math,pst-plot,pst-tree,pst-eucl} % pstricks
\usepackage[a4paper,includeheadfoot,top=2cm,left=3cm, bottom=2cm,right=3cm]{geometry} % marges etc.
\usepackage{comment}			% commentaires multilignes
\usepackage{amsmath,environ} % maths (matrices, etc.)
\usepackage{amssymb,makeidx}
\usepackage{bm}				% bold maths
\usepackage{tabularx}		% tableaux
\usepackage{colortbl}		% tableaux en couleur
\usepackage{fontawesome}		% Fontawesome
\usepackage{environ}			% environment with command
\usepackage{fp}				% calculs pour ps-tricks
\usepackage{multido}			% pour ps tricks
\usepackage[np]{numprint}	% formattage nombre
\usepackage{tikz,tkz-tab} 			% package principal TikZ
\usepackage{pgfplots}   % axes
\usepackage{mathrsfs}    % cursives
\usepackage{calc}			% calcul taille boites
\usepackage[scaled=0.875]{helvet} % font sans serif
\usepackage{svg} % svg
\usepackage{scrextend} % local margin
\usepackage{scratch} %scratch
\usepackage{multicol} % colonnes
%\usepackage{infix-RPN,pst-func} % formule en notation polanaise inversée
\usepackage{listings}

%================================================================================================================================
%
% Réglages de base
%
%================================================================================================================================

\lstset{
language=Python,   % R code
literate=
{á}{{\'a}}1
{à}{{\`a}}1
{ã}{{\~a}}1
{é}{{\'e}}1
{è}{{\`e}}1
{ê}{{\^e}}1
{í}{{\'i}}1
{ó}{{\'o}}1
{õ}{{\~o}}1
{ú}{{\'u}}1
{ü}{{\"u}}1
{ç}{{\c{c}}}1
{~}{{ }}1
}


\definecolor{codegreen}{rgb}{0,0.6,0}
\definecolor{codegray}{rgb}{0.5,0.5,0.5}
\definecolor{codepurple}{rgb}{0.58,0,0.82}
\definecolor{backcolour}{rgb}{0.95,0.95,0.92}

\lstdefinestyle{mystyle}{
    backgroundcolor=\color{backcolour},   
    commentstyle=\color{codegreen},
    keywordstyle=\color{magenta},
    numberstyle=\tiny\color{codegray},
    stringstyle=\color{codepurple},
    basicstyle=\ttfamily\footnotesize,
    breakatwhitespace=false,         
    breaklines=true,                 
    captionpos=b,                    
    keepspaces=true,                 
    numbers=left,                    
xleftmargin=2em,
framexleftmargin=2em,            
    showspaces=false,                
    showstringspaces=false,
    showtabs=false,                  
    tabsize=2,
    upquote=true
}

\lstset{style=mystyle}


\lstset{style=mystyle}
\newcommand{\imgdir}{C:/laragon/www/newmc/assets/imgsvg/}
\newcommand{\imgsvgdir}{C:/laragon/www/newmc/assets/imgsvg/}

\definecolor{mcgris}{RGB}{220, 220, 220}% ancien~; pour compatibilité
\definecolor{mcbleu}{RGB}{52, 152, 219}
\definecolor{mcvert}{RGB}{125, 194, 70}
\definecolor{mcmauve}{RGB}{154, 0, 215}
\definecolor{mcorange}{RGB}{255, 96, 0}
\definecolor{mcturquoise}{RGB}{0, 153, 153}
\definecolor{mcrouge}{RGB}{255, 0, 0}
\definecolor{mclightvert}{RGB}{205, 234, 190}

\definecolor{gris}{RGB}{220, 220, 220}
\definecolor{bleu}{RGB}{52, 152, 219}
\definecolor{vert}{RGB}{125, 194, 70}
\definecolor{mauve}{RGB}{154, 0, 215}
\definecolor{orange}{RGB}{255, 96, 0}
\definecolor{turquoise}{RGB}{0, 153, 153}
\definecolor{rouge}{RGB}{255, 0, 0}
\definecolor{lightvert}{RGB}{205, 234, 190}
\setitemize[0]{label=\color{lightvert}  $\bullet$}

\pagestyle{fancy}
\renewcommand{\headrulewidth}{0.2pt}
\fancyhead[L]{maths-cours.fr}
\fancyhead[R]{\thepage}
\renewcommand{\footrulewidth}{0.2pt}
\fancyfoot[C]{}

\newcolumntype{C}{>{\centering\arraybackslash}X}
\newcolumntype{s}{>{\hsize=.35\hsize\arraybackslash}X}

\setlength{\parindent}{0pt}		 
\setlength{\parskip}{3mm}
\setlength{\headheight}{1cm}

\def\ebook{ebook}
\def\book{book}
\def\web{web}
\def\type{web}

\newcommand{\vect}[1]{\overrightarrow{\,\mathstrut#1\,}}

\def\Oij{$\left(\text{O}~;~\vect{\imath},~\vect{\jmath}\right)$}
\def\Oijk{$\left(\text{O}~;~\vect{\imath},~\vect{\jmath},~\vect{k}\right)$}
\def\Ouv{$\left(\text{O}~;~\vect{u},~\vect{v}\right)$}

\hypersetup{breaklinks=true, colorlinks = true, linkcolor = OliveGreen, urlcolor = OliveGreen, citecolor = OliveGreen, pdfauthor={Didier BONNEL - https://www.maths-cours.fr} } % supprime les bordures autour des liens

\renewcommand{\arg}[0]{\text{arg}}

\everymath{\displaystyle}

%================================================================================================================================
%
% Macros - Commandes
%
%================================================================================================================================

\newcommand\meta[2]{    			% Utilisé pour créer le post HTML.
	\def\titre{titre}
	\def\url{url}
	\def\arg{#1}
	\ifx\titre\arg
		\newcommand\maintitle{#2}
		\fancyhead[L]{#2}
		{\Large\sffamily \MakeUppercase{#2}}
		\vspace{1mm}\textcolor{mcvert}{\hrule}
	\fi 
	\ifx\url\arg
		\fancyfoot[L]{\href{https://www.maths-cours.fr#2}{\black \footnotesize{https://www.maths-cours.fr#2}}}
	\fi 
}


\newcommand\TitreC[1]{    		% Titre centré
     \needspace{3\baselineskip}
     \begin{center}\textbf{#1}\end{center}
}

\newcommand\newpar{    		% paragraphe
     \par
}

\newcommand\nosp {    		% commande vide (pas d'espace)
}
\newcommand{\id}[1]{} %ignore

\newcommand\boite[2]{				% Boite simple sans titre
	\vspace{5mm}
	\setlength{\fboxrule}{0.2mm}
	\setlength{\fboxsep}{5mm}	
	\fcolorbox{#1}{#1!3}{\makebox[\linewidth-2\fboxrule-2\fboxsep]{
  		\begin{minipage}[t]{\linewidth-2\fboxrule-4\fboxsep}\setlength{\parskip}{3mm}
  			 #2
  		\end{minipage}
	}}
	\vspace{5mm}
}

\newcommand\CBox[4]{				% Boites
	\vspace{5mm}
	\setlength{\fboxrule}{0.2mm}
	\setlength{\fboxsep}{5mm}
	
	\fcolorbox{#1}{#1!3}{\makebox[\linewidth-2\fboxrule-2\fboxsep]{
		\begin{minipage}[t]{1cm}\setlength{\parskip}{3mm}
	  		\textcolor{#1}{\LARGE{#2}}    
 	 	\end{minipage}  
  		\begin{minipage}[t]{\linewidth-2\fboxrule-4\fboxsep}\setlength{\parskip}{3mm}
			\raisebox{1.2mm}{\normalsize\sffamily{\textcolor{#1}{#3}}}						
  			 #4
  		\end{minipage}
	}}
	\vspace{5mm}
}

\newcommand\cadre[3]{				% Boites convertible html
	\par
	\vspace{2mm}
	\setlength{\fboxrule}{0.1mm}
	\setlength{\fboxsep}{5mm}
	\fcolorbox{#1}{white}{\makebox[\linewidth-2\fboxrule-2\fboxsep]{
  		\begin{minipage}[t]{\linewidth-2\fboxrule-4\fboxsep}\setlength{\parskip}{3mm}
			\raisebox{-2.5mm}{\sffamily \small{\textcolor{#1}{\MakeUppercase{#2}}}}		
			\par		
  			 #3
 	 		\end{minipage}
	}}
		\vspace{2mm}
	\par
}

\newcommand\bloc[3]{				% Boites convertible html sans bordure
     \needspace{2\baselineskip}
     {\sffamily \small{\textcolor{#1}{\MakeUppercase{#2}}}}    
		\par		
  			 #3
		\par
}

\newcommand\CHelp[1]{
     \CBox{Plum}{\faInfoCircle}{À RETENIR}{#1}
}

\newcommand\CUp[1]{
     \CBox{NavyBlue}{\faThumbsOUp}{EN PRATIQUE}{#1}
}

\newcommand\CInfo[1]{
     \CBox{Sepia}{\faArrowCircleRight}{REMARQUE}{#1}
}

\newcommand\CRedac[1]{
     \CBox{PineGreen}{\faEdit}{BIEN R\'EDIGER}{#1}
}

\newcommand\CError[1]{
     \CBox{Red}{\faExclamationTriangle}{ATTENTION}{#1}
}

\newcommand\TitreExo[2]{
\needspace{4\baselineskip}
 {\sffamily\large EXERCICE #1\ (\emph{#2 points})}
\vspace{5mm}
}

\newcommand\img[2]{
          \includegraphics[width=#2\paperwidth]{\imgdir#1}
}

\newcommand\imgsvg[2]{
       \begin{center}   \includegraphics[width=#2\paperwidth]{\imgsvgdir#1} \end{center}
}


\newcommand\Lien[2]{
     \href{#1}{#2 \tiny \faExternalLink}
}
\newcommand\mcLien[2]{
     \href{https~://www.maths-cours.fr/#1}{#2 \tiny \faExternalLink}
}

\newcommand{\euro}{\eurologo{}}

%================================================================================================================================
%
% Macros - Environement
%
%================================================================================================================================

\newenvironment{tex}{ %
}
{%
}

\newenvironment{indente}{ %
	\setlength\parindent{10mm}
}

{
	\setlength\parindent{0mm}
}

\newenvironment{corrige}{%
     \needspace{3\baselineskip}
     \medskip
     \textbf{\textsc{Corrigé}}
     \medskip
}
{
}

\newenvironment{extern}{%
     \begin{center}
     }
     {
     \end{center}
}

\NewEnviron{code}{%
	\par
     \boite{gray}{\texttt{%
     \BODY
     }}
     \par
}

\newenvironment{vbloc}{% boite sans cadre empeche saut de page
     \begin{minipage}[t]{\linewidth}
     }
     {
     \end{minipage}
}
\NewEnviron{h2}{%
    \needspace{3\baselineskip}
    \vspace{0.6cm}
	\noindent \MakeUppercase{\sffamily \large \BODY}
	\vspace{1mm}\textcolor{mcgris}{\hrule}\vspace{0.4cm}
	\par
}{}

\NewEnviron{h3}{%
    \needspace{3\baselineskip}
	\vspace{5mm}
	\textsc{\BODY}
	\par
}

\NewEnviron{margeneg}{ %
\begin{addmargin}[-1cm]{0cm}
\BODY
\end{addmargin}
}

\NewEnviron{html}{%
}

\begin{document}
\meta{url}{/exercices/suites-bac-es-l-pondichery-2018/}
\meta{pid}{7099}
\meta{titre}{Suites - Bac ES/L Pondichéry 2018}
\meta{type}{exercice}
\begin{h2}Exercice 3 (5 points)\end{h2}
\textbf{Candidats n'ayant pas suivi l'enseignement de spécialité}
\medskip
On considère la suite $\left(u_n\right)$ définie par $u_0 = 65$ et pour tout entier naturel $n$~:
\par
\[u_{n+1} = 0,8u_n + 18.\]
\medskip
\begin{enumerate}
     \item Calculer $u_1$ et $u_2$.
     \item Pour tout entier naturel $n$, on pose~: $v_n = u_n - 90$.
     \begin{enumerate}[label=\alph*.]
          \item Démontrer que la suite $\left(v_n\right)$ est géométrique de raison $0,8$.
          \par
          On précisera la valeur de $v_0$.
          \item Démontrer que, pour tout entier naturel $n$~:
          \[u_n = 90 - 25 \times  0,8^n.\]
     \end{enumerate}
     \item  On considère l'algorithme ci-dessous~:
     \begin{center}
          \begin{extern}%width="330px" alt="algorithme suite géométrique"
               \begin{tabularx}{0.5\linewidth}{|c|X|}\hline
                    ligne 1&$u \gets 65$\\
                    ligne 2&$n \gets 0$\\
                    ligne 3&Tant que .........\\
                    ligne 4&\hspace{1cm}$n \gets n+1$\\
                    ligne 5&\hspace{1cm}$u \gets 0,8 \times u + 18$\\
                    ligne 6&Fin Tant que\\ \hline
               \end{tabularx}
          \end{extern}
     \end{center}
     \medskip
     \begin{enumerate}[label=\alph*.]
          \item Recopier et compléter la ligne 3 de cet algorithme afin qu'il détermine le plus petit entier
          naturel $n$ tel que $u_n \geqslant 85$.
          \item Quelle est la valeur de la variable $n$ à la fin de l'exécution de l'algorithme~?
          \item Retrouver par le calcul le résultat de la question précédente en résolvant l'inéquation
          $u_n \geqslant 85$.
     \end{enumerate}
     \item  La société Biocagette propose la livraison hebdomadaire d'un panier bio qui contient des fruits
     et des légumes de saison issus de l'agriculture biologique. Les clients ont la possibilité de
     souscrire un abonnement de $52$~\euro par mois qui permet de recevoir chaque semaine ce panier
     bio.
     \par
     En juillet 2017, $65$ particuliers ont souscrit cet abonnement.
     \smallskip
     Les responsables de la société Biocagette font les hypothèses suivantes~:
     \begin{itemize}[label=---]
          \item d'un mois à l'autre, environ 20\,\% des abonnements sont résiliés~;
          \item chaque mois, $18$ particuliers supplémentaires souscrivent à l'abonnement.
     \end{itemize}
     \begin{enumerate}[label=\alph*.]
          \item Justifier que la suite $\left(u_n\right)$ permet de modéliser le nombre d'abonnés au panier bio le $n$-ième mois qui suit le mois de juillet 2017.
          \item Selon ce modèle, la recette mensuelle de la société Biocagette va-t-elle dépasser 4~420~\euro{} durant l'année 2018~? Justifier la réponse.
          \item Selon ce modèle, vers quelle valeur tend la recette mensuelle de la société Biocagette~?
          \par
          Argumenter la réponse.
     \end{enumerate}
\end{enumerate}
\begin{corrige}
     \begin{enumerate}
          \item
          Pour tout entier naturel $n$, $u_{n+1} = 0,8u_n + 18$, par conséquent~:
          $u_1 = 0,8 u_0 + 18 = 0,8\times 65 + 18 = 70$ \\
          $u_2 = 0,8 u_1 + 18 = 0,8\times 70 + 18 = 74$\\
          \item
          \begin{enumerate}[label=\alph*.]
               \item
               Pour tout entier naturel $n$~:
               \par
               $v_{n+1}=u_{n+1}-90 $\\
               $\phantom{v_{n+1}}=0,8 u_n + 18 -90$\\
               $\phantom{v_{n+1}}=0,8u_n-72$.
               \par
               Or $v_n = u_n - 90$~; donc $u_n=v_n+90$~; alors~:
               \par
               $v_{n+1}=0,8 \left ( v_n+90\right ) - 72$\\
               $\phantom{v_{n+1}}=0,8 v_n + 72-72$\\
               $\phantom{v_{n+1}}=0,8v_n$.
               \par
               De plus $v_0=u_0-90 = 65- 90 = -25$~; par conséquent, la suite $(v_n)$ est une suite géométrique de premier terme ${v_0=-25}$  et de raison ${q=0,8}$.
               \item
               On en déduit que~:
               \par
               $v_n=v_0q^n=-25 \times 0,8^n$.
               \par
               Et comme $u_n=v_n+90$, pour tout entier naturel $n$~:
               $u_n=90-25\times 0,8^{n}$.
          \end{enumerate}
          \item
          \begin{enumerate}[label=\alph*.]
               \item  On souhaite déterminer le plus petit entier naturel $n$ tel que $u_n \geqslant 85$.
               \par
               Pour cela, on doit rester dans la boucle \og Tant que \fg{} aussi longtemps que $u_n$ est strictement inférieur à 85.
               \par
               On doit donc compléter l'algorithme comme suit~:
               \begin{center}
                    \begin{extern}%width="330px" alt="algoritme bac correction"
                         \begin{tabularx}{0.5\linewidth}{|c|X|}\hline
                              ligne 1&$u \gets 65$\\
                              ligne 2&$n \gets 0$\\
                              ligne 3&Tant que  $\red{u<85}$\\
                              ligne 4&\hspace{1cm}$n \gets n+1$\\
                              ligne 5&\hspace{1cm}$u \gets 0,8 \times u + 18$\\
                              ligne 6&Fin Tant que\\ \hline
                         \end{tabularx}
                    \end{extern}
               \end{center}
               \medskip
               \item
               \par
               On peut programmer l'algorithme sur sa calculatrice ou plus simplement utiliser le menu \og Suite \fg{} de la calculatrice (ou même le menu \og Fonction \fg{} en entrant la fonction $Y_1= 90-25 \times 0,8\hat~X$ et un pas de 1) de façon à calculer les premiers termes de la suite.
               \par
               On trouve alors~:
               \par
               $u_ 7 ~=  84,8$\\
               \par
               et $u_8 ~= 85,8$\\
               \par
               \`A la fin de l'exécution de l'algorithme, la variable $n$ contiendra donc la valeur 8.
               \item
               $u_n \geqslant 85 \Leftrightarrow 90-25\times 0,8^{n} \geqslant 85$\\
               $\phantom{u_n \geqslant 85 } \Leftrightarrow -25\times 0,8^{n} \geqslant -5$\\
               $\phantom{u_n \geqslant 85 } \Leftrightarrow 25\times 0,8^{n} \leqslant 5$\\
               $\phantom{u_n \geqslant 85 }\Leftrightarrow   0,8^{n} \leqslant \dfrac{5}{25}$\\
               $\phantom{u_n \geqslant 85 }\Leftrightarrow   0,8^{n} \leqslant 0,2$\\
               \par
               Comme la fonction $\ln$ est strictement croissante sur l'intervalle $]0~;~+\infty[$, on peut appliquer, à chaque membre, la fonction $\ln$~:\\
               \par
               $u_n \geqslant 85 \Leftrightarrow   \ln \left ( 0,8^{n} \right )\leqslant \ln 0,2$\\
               $\phantom{u_n \geqslant 85}\Leftrightarrow n \times \ln 0,8 \leqslant \ln 0,2$\\
               \par
               On divise chaque membre par $ \ln 0,8 $ qui est strictement négatif~; il faut donc changer le sens de l'inégalité~:
               \par
               $u_n \geqslant 85 \Leftrightarrow n \geqslant \dfrac{\ln 0,2}{\ln 0,8} $
               \par
               Or $\dfrac{\ln 0,2}{\ln 0,8} \approx 7,21$~; le plus petit entier $n$ vérifiant l'inégalité est donc bien  $n=8$.
          \end{enumerate}
          \item
          \begin{enumerate}[label=\alph*.]
               \item
               \par
               Notons $a_n$ le nombre d'abonnés au panier bio le $n$-ième mois qui suit le mois de juillet 2017.
               \par
               En juillet 2017, 65 particuliers avaient souscrit l'abonnement, donc $a_0=65$.
               \par
               Une diminution de 20\% correspond à un coefficient multiplicateur de ${1-\dfrac{20}{100}=0,8}$~; on ajoute ensuite les 18 nouveaux abonnés.
               \par
               On a donc~:
               \[ a_{n+1}=0,8a_n+60. \]
               Les suites $(u_n)$ et $(a_n)$ sont définies par la même relation de récurrence et le même premier terme~; elles sont donc identiques.
               \par
               La suite $\left(u_n\right)$ permet donc bien de modéliser le nombre d'abonnés au panier bio le $n$-ième mois qui suit le mois de juillet 2017.
               \item
               \par
               Puisque le prix d'un abonnement est  $52$~\euro par mois, la recette mensuelle pour le mois $n$ est $52 u_n$ euros.
               \par
               On cherche donc à résoudre l'inéquation $52 u_n > 4~420$. Or~:
               \par
               $52 u_n > 4~420 \Leftrightarrow u_n > \dfrac{4~420}{52} \Leftrightarrow u_n > 85$.
               \par
               Et d'après la question \textbf{3} ceci produit pour $n=8$ soit 8 mois après le mois de juillet 2017 c'est à dire en mars 2018.
               \par
               La recette mensuelle dépassera donc 4~420~\euro{} durant l'année 2018.
               \item
               La suite $(v_n)$ est une géométrique de raison $q=0,8$. Comme $0<0,8<1$, la suite $(v_n)$ converge vers 0.
               \par
               Or, pour tout entier naturel $n$, $u_n=v_n+90$ donc  $\lim\limits_{n \rightarrow +\infty}u_n=90$.
               \par
               Au cours du temps le nombre d'abonnés se rapprochera de $90$.
               \par
               La recette mensuelle  de la société Biocagette tendra donc vers $52 \times 90= 4~680$~\euro.
          \end{enumerate}
     \end{enumerate}
\end{corrige}

\end{document}
µ
\documentclass[a4paper]{article}

%================================================================================================================================
%
% Packages
%
%================================================================================================================================

\usepackage[T1]{fontenc} 	% pour caractères accentués
\usepackage[utf8]{inputenc}  % encodage utf8
\usepackage[french]{babel}	% langue : français
\usepackage{fourier}			% caractères plus lisibles
\usepackage[dvipsnames]{xcolor} % couleurs
\usepackage{fancyhdr}		% réglage header footer
\usepackage{needspace}		% empêcher sauts de page mal placés
\usepackage{graphicx}		% pour inclure des graphiques
\usepackage{enumitem,cprotect}		% personnalise les listes d'items (nécessaire pour ol, al ...)
\usepackage{hyperref}		% Liens hypertexte
\usepackage{pstricks,pst-all,pst-node,pstricks-add,pst-math,pst-plot,pst-tree,pst-eucl} % pstricks
\usepackage[a4paper,includeheadfoot,top=2cm,left=3cm, bottom=2cm,right=3cm]{geometry} % marges etc.
\usepackage{comment}			% commentaires multilignes
\usepackage{amsmath,environ} % maths (matrices, etc.)
\usepackage{amssymb,makeidx}
\usepackage{bm}				% bold maths
\usepackage{tabularx}		% tableaux
\usepackage{colortbl}		% tableaux en couleur
\usepackage{fontawesome}		% Fontawesome
\usepackage{environ}			% environment with command
\usepackage{fp}				% calculs pour ps-tricks
\usepackage{multido}			% pour ps tricks
\usepackage[np]{numprint}	% formattage nombre
\usepackage{tikz,tkz-tab} 			% package principal TikZ
\usepackage{pgfplots}   % axes
\usepackage{mathrsfs}    % cursives
\usepackage{calc}			% calcul taille boites
\usepackage[scaled=0.875]{helvet} % font sans serif
\usepackage{svg} % svg
\usepackage{scrextend} % local margin
\usepackage{scratch} %scratch
\usepackage{multicol} % colonnes
%\usepackage{infix-RPN,pst-func} % formule en notation polanaise inversée
\usepackage{listings}

%================================================================================================================================
%
% Réglages de base
%
%================================================================================================================================

\lstset{
language=Python,   % R code
literate=
{á}{{\'a}}1
{à}{{\`a}}1
{ã}{{\~a}}1
{é}{{\'e}}1
{è}{{\`e}}1
{ê}{{\^e}}1
{í}{{\'i}}1
{ó}{{\'o}}1
{õ}{{\~o}}1
{ú}{{\'u}}1
{ü}{{\"u}}1
{ç}{{\c{c}}}1
{~}{{ }}1
}


\definecolor{codegreen}{rgb}{0,0.6,0}
\definecolor{codegray}{rgb}{0.5,0.5,0.5}
\definecolor{codepurple}{rgb}{0.58,0,0.82}
\definecolor{backcolour}{rgb}{0.95,0.95,0.92}

\lstdefinestyle{mystyle}{
    backgroundcolor=\color{backcolour},   
    commentstyle=\color{codegreen},
    keywordstyle=\color{magenta},
    numberstyle=\tiny\color{codegray},
    stringstyle=\color{codepurple},
    basicstyle=\ttfamily\footnotesize,
    breakatwhitespace=false,         
    breaklines=true,                 
    captionpos=b,                    
    keepspaces=true,                 
    numbers=left,                    
xleftmargin=2em,
framexleftmargin=2em,            
    showspaces=false,                
    showstringspaces=false,
    showtabs=false,                  
    tabsize=2,
    upquote=true
}

\lstset{style=mystyle}


\lstset{style=mystyle}
\newcommand{\imgdir}{C:/laragon/www/newmc/assets/imgsvg/}
\newcommand{\imgsvgdir}{C:/laragon/www/newmc/assets/imgsvg/}

\definecolor{mcgris}{RGB}{220, 220, 220}% ancien~; pour compatibilité
\definecolor{mcbleu}{RGB}{52, 152, 219}
\definecolor{mcvert}{RGB}{125, 194, 70}
\definecolor{mcmauve}{RGB}{154, 0, 215}
\definecolor{mcorange}{RGB}{255, 96, 0}
\definecolor{mcturquoise}{RGB}{0, 153, 153}
\definecolor{mcrouge}{RGB}{255, 0, 0}
\definecolor{mclightvert}{RGB}{205, 234, 190}

\definecolor{gris}{RGB}{220, 220, 220}
\definecolor{bleu}{RGB}{52, 152, 219}
\definecolor{vert}{RGB}{125, 194, 70}
\definecolor{mauve}{RGB}{154, 0, 215}
\definecolor{orange}{RGB}{255, 96, 0}
\definecolor{turquoise}{RGB}{0, 153, 153}
\definecolor{rouge}{RGB}{255, 0, 0}
\definecolor{lightvert}{RGB}{205, 234, 190}
\setitemize[0]{label=\color{lightvert}  $\bullet$}

\pagestyle{fancy}
\renewcommand{\headrulewidth}{0.2pt}
\fancyhead[L]{maths-cours.fr}
\fancyhead[R]{\thepage}
\renewcommand{\footrulewidth}{0.2pt}
\fancyfoot[C]{}

\newcolumntype{C}{>{\centering\arraybackslash}X}
\newcolumntype{s}{>{\hsize=.35\hsize\arraybackslash}X}

\setlength{\parindent}{0pt}		 
\setlength{\parskip}{3mm}
\setlength{\headheight}{1cm}

\def\ebook{ebook}
\def\book{book}
\def\web{web}
\def\type{web}

\newcommand{\vect}[1]{\overrightarrow{\,\mathstrut#1\,}}

\def\Oij{$\left(\text{O}~;~\vect{\imath},~\vect{\jmath}\right)$}
\def\Oijk{$\left(\text{O}~;~\vect{\imath},~\vect{\jmath},~\vect{k}\right)$}
\def\Ouv{$\left(\text{O}~;~\vect{u},~\vect{v}\right)$}

\hypersetup{breaklinks=true, colorlinks = true, linkcolor = OliveGreen, urlcolor = OliveGreen, citecolor = OliveGreen, pdfauthor={Didier BONNEL - https://www.maths-cours.fr} } % supprime les bordures autour des liens

\renewcommand{\arg}[0]{\text{arg}}

\everymath{\displaystyle}

%================================================================================================================================
%
% Macros - Commandes
%
%================================================================================================================================

\newcommand\meta[2]{    			% Utilisé pour créer le post HTML.
	\def\titre{titre}
	\def\url{url}
	\def\arg{#1}
	\ifx\titre\arg
		\newcommand\maintitle{#2}
		\fancyhead[L]{#2}
		{\Large\sffamily \MakeUppercase{#2}}
		\vspace{1mm}\textcolor{mcvert}{\hrule}
	\fi 
	\ifx\url\arg
		\fancyfoot[L]{\href{https://www.maths-cours.fr#2}{\black \footnotesize{https://www.maths-cours.fr#2}}}
	\fi 
}


\newcommand\TitreC[1]{    		% Titre centré
     \needspace{3\baselineskip}
     \begin{center}\textbf{#1}\end{center}
}

\newcommand\newpar{    		% paragraphe
     \par
}

\newcommand\nosp {    		% commande vide (pas d'espace)
}
\newcommand{\id}[1]{} %ignore

\newcommand\boite[2]{				% Boite simple sans titre
	\vspace{5mm}
	\setlength{\fboxrule}{0.2mm}
	\setlength{\fboxsep}{5mm}	
	\fcolorbox{#1}{#1!3}{\makebox[\linewidth-2\fboxrule-2\fboxsep]{
  		\begin{minipage}[t]{\linewidth-2\fboxrule-4\fboxsep}\setlength{\parskip}{3mm}
  			 #2
  		\end{minipage}
	}}
	\vspace{5mm}
}

\newcommand\CBox[4]{				% Boites
	\vspace{5mm}
	\setlength{\fboxrule}{0.2mm}
	\setlength{\fboxsep}{5mm}
	
	\fcolorbox{#1}{#1!3}{\makebox[\linewidth-2\fboxrule-2\fboxsep]{
		\begin{minipage}[t]{1cm}\setlength{\parskip}{3mm}
	  		\textcolor{#1}{\LARGE{#2}}    
 	 	\end{minipage}  
  		\begin{minipage}[t]{\linewidth-2\fboxrule-4\fboxsep}\setlength{\parskip}{3mm}
			\raisebox{1.2mm}{\normalsize\sffamily{\textcolor{#1}{#3}}}						
  			 #4
  		\end{minipage}
	}}
	\vspace{5mm}
}

\newcommand\cadre[3]{				% Boites convertible html
	\par
	\vspace{2mm}
	\setlength{\fboxrule}{0.1mm}
	\setlength{\fboxsep}{5mm}
	\fcolorbox{#1}{white}{\makebox[\linewidth-2\fboxrule-2\fboxsep]{
  		\begin{minipage}[t]{\linewidth-2\fboxrule-4\fboxsep}\setlength{\parskip}{3mm}
			\raisebox{-2.5mm}{\sffamily \small{\textcolor{#1}{\MakeUppercase{#2}}}}		
			\par		
  			 #3
 	 		\end{minipage}
	}}
		\vspace{2mm}
	\par
}

\newcommand\bloc[3]{				% Boites convertible html sans bordure
     \needspace{2\baselineskip}
     {\sffamily \small{\textcolor{#1}{\MakeUppercase{#2}}}}    
		\par		
  			 #3
		\par
}

\newcommand\CHelp[1]{
     \CBox{Plum}{\faInfoCircle}{À RETENIR}{#1}
}

\newcommand\CUp[1]{
     \CBox{NavyBlue}{\faThumbsOUp}{EN PRATIQUE}{#1}
}

\newcommand\CInfo[1]{
     \CBox{Sepia}{\faArrowCircleRight}{REMARQUE}{#1}
}

\newcommand\CRedac[1]{
     \CBox{PineGreen}{\faEdit}{BIEN R\'EDIGER}{#1}
}

\newcommand\CError[1]{
     \CBox{Red}{\faExclamationTriangle}{ATTENTION}{#1}
}

\newcommand\TitreExo[2]{
\needspace{4\baselineskip}
 {\sffamily\large EXERCICE #1\ (\emph{#2 points})}
\vspace{5mm}
}

\newcommand\img[2]{
          \includegraphics[width=#2\paperwidth]{\imgdir#1}
}

\newcommand\imgsvg[2]{
       \begin{center}   \includegraphics[width=#2\paperwidth]{\imgsvgdir#1} \end{center}
}


\newcommand\Lien[2]{
     \href{#1}{#2 \tiny \faExternalLink}
}
\newcommand\mcLien[2]{
     \href{https~://www.maths-cours.fr/#1}{#2 \tiny \faExternalLink}
}

\newcommand{\euro}{\eurologo{}}

%================================================================================================================================
%
% Macros - Environement
%
%================================================================================================================================

\newenvironment{tex}{ %
}
{%
}

\newenvironment{indente}{ %
	\setlength\parindent{10mm}
}

{
	\setlength\parindent{0mm}
}

\newenvironment{corrige}{%
     \needspace{3\baselineskip}
     \medskip
     \textbf{\textsc{Corrigé}}
     \medskip
}
{
}

\newenvironment{extern}{%
     \begin{center}
     }
     {
     \end{center}
}

\NewEnviron{code}{%
	\par
     \boite{gray}{\texttt{%
     \BODY
     }}
     \par
}

\newenvironment{vbloc}{% boite sans cadre empeche saut de page
     \begin{minipage}[t]{\linewidth}
     }
     {
     \end{minipage}
}
\NewEnviron{h2}{%
    \needspace{3\baselineskip}
    \vspace{0.6cm}
	\noindent \MakeUppercase{\sffamily \large \BODY}
	\vspace{1mm}\textcolor{mcgris}{\hrule}\vspace{0.4cm}
	\par
}{}

\NewEnviron{h3}{%
    \needspace{3\baselineskip}
	\vspace{5mm}
	\textsc{\BODY}
	\par
}

\NewEnviron{margeneg}{ %
\begin{addmargin}[-1cm]{0cm}
\BODY
\end{addmargin}
}

\NewEnviron{html}{%
}

\begin{document}
\meta{url}{/exercices/fonctions-bac-es-l-pondichery-2018/}
\meta{pid}{7105}
\meta{titre}{Fonctions – Bac ES/L Pondichéry 2018}
\meta{type}{exercice}
\begin{h2}Exercice 4 (5 points)\end{h2}
\textbf{Commun à tous les candidats}
\medskip
\emph{Dans cet exercice, si nécessaire, les valeurs numériques approchées seront données à 0,01~près}.
\medskip
On considère la fonction $f$ définie sur l'intervalle [0~;~4] par~:
\[f(x) = (3,6x + 2,4)\text{e}^{-0,6x}- 1,4.\]
\begin{center}\begin{h3}Partie A \end{h3}\end{center}
On admet que la fonction $f$ est dérivable sur l'intervalle [0~;~4] et on note $f'$ sa fonction dérivée.
\medskip
\begin{enumerate}
     \item Justifier que pour tout nombre réel $x$ de l'intervalle [0~;~4] on a~:
     \[f'(x) = (- 2,16 x + 2,16)\text{e}^{-0,6x}.\]
     \item
     \begin{enumerate}[label=\alph*.]
          \item Étudier le signe de $f'(x)$ sur l'intervalle [0~;~4].
          \item Dresser le tableau de variation de la fonction $f$ sur cet intervalle.
          \par
          On donnera les valeurs numériques qui apparaissent dans le tableau de variation sous
          forme approchée.
     \end{enumerate}
     \item  On admet que la fonction $F$ définie par~:
     \[F(x) = (- 6x - 14)\text{e}^{-0,6x} - 1,4x\]
     est une primitive de la fonction $f$ sur l'intervalle [0~;~4].
     \par
     Calculer la valeur exacte de $\displaystyle\int_0^4  f(x)\:\text{d}x$ puis en donner une valeur numérique approchée.
\end{enumerate}
\begin{center}\begin{h3}Partie B \end{h3}\end{center}
On note $\mathscr{C}_f$ la courbe représentative de la fonction $f$ sur l'intervalle [0~;~4].
\par
On considère la fonction $g$ définie par~:
\[g(x) = 4x^2 - 4x + 1.\]
On note $\mathscr{C}_g$ la courbe représentative de cette fonction sur l'intervalle [0~;~0,5].
\par
On a tracé ci-dessous les courbes $\mathscr{C}_f$ et $\mathscr{C}_g$ dans un repère d'origine O et, en pointillés, les courbes obtenues par symétrie de $\mathscr{C}_f$ et $\mathscr{C}_g$ par rapport à l'axe des abscisses~:
\begin{center}
     \begin{extern}%width="350px"
          \psset{unit=1.25cm}
          \begin{pspicture}(-1,-3)(5,3)
               \psgrid[gridlabels=0pt,subgriddiv=1,gridwidth=0.15pt]
               \pscustom[fillstyle=solid,fillcolor=mcvert!20,linewidth=0.75pt,linestyle=dashed]{
                    \psplot[plotpoints=3000]{0.5}{0}{x dup mul 4 mul 4 x mul sub 1 add}
                    \psplot[plotpoints=3000]{0}{4}{3.6 x mul 2.4 add 2.71828 0.6 x mul exp div 1.4 sub}
                    \psline(4,0.124)(4,-0.124)
                    \psplot[plotpoints=3000]{4}{0}{3.6 x mul 2.4 add 2.71828 0.6 x mul exp div 1.4 sub neg}
                    \psplot[plotpoints=3000]{0}{0.5}{x dup mul 4 mul 4 x mul sub 1 add neg}
               }
               \psaxes[linewidth=1pt]{->}(0,0)(-1,-3)(5,3)
               \psplot[plotpoints=3000,linewidth=0.75pt,linecolor=black]{0}{4}{3.6 x mul 2.4 add 2.71828 0.6 x mul exp div 1.4 sub}
               \psplot[plotpoints=3000,linewidth=0.75pt,linecolor=black]{0}{0.5}{x dup mul 4 mul 4 x mul sub 1 add}
               \psline(4,0.124)(4,-0.124)
               \uput[ur](2,1.5){\black $\mathcal{C}_f$}
               \uput[l](0.1,0.5){\black $\mathcal{C}_g$}
               \uput[dl](0,0){O}
               \psgrid[gridlabels=0pt,subgriddiv=1,gridwidth=0.15pt]
          \end{pspicture}
     \end{extern}
\end{center}
\medskip
\begin{enumerate}
     \item Montrer que $\displaystyle\int_0^{0,5}g(x)\:\text{d}x = \dfrac{1}{6}$.
     \item On considère le domaine plan délimité par les courbes $\mathscr{C}_f$, \:$\mathscr{C}_g$, leurs courbes symétriques (en pointillés) ainsi que la droite d'équation $x = 4$.
     \par
     Ce domaine apparaît coloré sur la figure ci-dessus.
     \par
     Calculer une valeur approchée de l'aire, en unités d'aire, de ce domaine.
\end{enumerate}

\end{document}
µ
\documentclass[a4paper]{article}

%================================================================================================================================
%
% Packages
%
%================================================================================================================================

\usepackage[T1]{fontenc} 	% pour caractères accentués
\usepackage[utf8]{inputenc}  % encodage utf8
\usepackage[french]{babel}	% langue : français
\usepackage{fourier}			% caractères plus lisibles
\usepackage[dvipsnames]{xcolor} % couleurs
\usepackage{fancyhdr}		% réglage header footer
\usepackage{needspace}		% empêcher sauts de page mal placés
\usepackage{graphicx}		% pour inclure des graphiques
\usepackage{enumitem,cprotect}		% personnalise les listes d'items (nécessaire pour ol, al ...)
\usepackage{hyperref}		% Liens hypertexte
\usepackage{pstricks,pst-all,pst-node,pstricks-add,pst-math,pst-plot,pst-tree,pst-eucl} % pstricks
\usepackage[a4paper,includeheadfoot,top=2cm,left=3cm, bottom=2cm,right=3cm]{geometry} % marges etc.
\usepackage{comment}			% commentaires multilignes
\usepackage{amsmath,environ} % maths (matrices, etc.)
\usepackage{amssymb,makeidx}
\usepackage{bm}				% bold maths
\usepackage{tabularx}		% tableaux
\usepackage{colortbl}		% tableaux en couleur
\usepackage{fontawesome}		% Fontawesome
\usepackage{environ}			% environment with command
\usepackage{fp}				% calculs pour ps-tricks
\usepackage{multido}			% pour ps tricks
\usepackage[np]{numprint}	% formattage nombre
\usepackage{tikz,tkz-tab} 			% package principal TikZ
\usepackage{pgfplots}   % axes
\usepackage{mathrsfs}    % cursives
\usepackage{calc}			% calcul taille boites
\usepackage[scaled=0.875]{helvet} % font sans serif
\usepackage{svg} % svg
\usepackage{scrextend} % local margin
\usepackage{scratch} %scratch
\usepackage{multicol} % colonnes
%\usepackage{infix-RPN,pst-func} % formule en notation polanaise inversée
\usepackage{listings}

%================================================================================================================================
%
% Réglages de base
%
%================================================================================================================================

\lstset{
language=Python,   % R code
literate=
{á}{{\'a}}1
{à}{{\`a}}1
{ã}{{\~a}}1
{é}{{\'e}}1
{è}{{\`e}}1
{ê}{{\^e}}1
{í}{{\'i}}1
{ó}{{\'o}}1
{õ}{{\~o}}1
{ú}{{\'u}}1
{ü}{{\"u}}1
{ç}{{\c{c}}}1
{~}{{ }}1
}


\definecolor{codegreen}{rgb}{0,0.6,0}
\definecolor{codegray}{rgb}{0.5,0.5,0.5}
\definecolor{codepurple}{rgb}{0.58,0,0.82}
\definecolor{backcolour}{rgb}{0.95,0.95,0.92}

\lstdefinestyle{mystyle}{
    backgroundcolor=\color{backcolour},   
    commentstyle=\color{codegreen},
    keywordstyle=\color{magenta},
    numberstyle=\tiny\color{codegray},
    stringstyle=\color{codepurple},
    basicstyle=\ttfamily\footnotesize,
    breakatwhitespace=false,         
    breaklines=true,                 
    captionpos=b,                    
    keepspaces=true,                 
    numbers=left,                    
xleftmargin=2em,
framexleftmargin=2em,            
    showspaces=false,                
    showstringspaces=false,
    showtabs=false,                  
    tabsize=2,
    upquote=true
}

\lstset{style=mystyle}


\lstset{style=mystyle}
\newcommand{\imgdir}{C:/laragon/www/newmc/assets/imgsvg/}
\newcommand{\imgsvgdir}{C:/laragon/www/newmc/assets/imgsvg/}

\definecolor{mcgris}{RGB}{220, 220, 220}% ancien~; pour compatibilité
\definecolor{mcbleu}{RGB}{52, 152, 219}
\definecolor{mcvert}{RGB}{125, 194, 70}
\definecolor{mcmauve}{RGB}{154, 0, 215}
\definecolor{mcorange}{RGB}{255, 96, 0}
\definecolor{mcturquoise}{RGB}{0, 153, 153}
\definecolor{mcrouge}{RGB}{255, 0, 0}
\definecolor{mclightvert}{RGB}{205, 234, 190}

\definecolor{gris}{RGB}{220, 220, 220}
\definecolor{bleu}{RGB}{52, 152, 219}
\definecolor{vert}{RGB}{125, 194, 70}
\definecolor{mauve}{RGB}{154, 0, 215}
\definecolor{orange}{RGB}{255, 96, 0}
\definecolor{turquoise}{RGB}{0, 153, 153}
\definecolor{rouge}{RGB}{255, 0, 0}
\definecolor{lightvert}{RGB}{205, 234, 190}
\setitemize[0]{label=\color{lightvert}  $\bullet$}

\pagestyle{fancy}
\renewcommand{\headrulewidth}{0.2pt}
\fancyhead[L]{maths-cours.fr}
\fancyhead[R]{\thepage}
\renewcommand{\footrulewidth}{0.2pt}
\fancyfoot[C]{}

\newcolumntype{C}{>{\centering\arraybackslash}X}
\newcolumntype{s}{>{\hsize=.35\hsize\arraybackslash}X}

\setlength{\parindent}{0pt}		 
\setlength{\parskip}{3mm}
\setlength{\headheight}{1cm}

\def\ebook{ebook}
\def\book{book}
\def\web{web}
\def\type{web}

\newcommand{\vect}[1]{\overrightarrow{\,\mathstrut#1\,}}

\def\Oij{$\left(\text{O}~;~\vect{\imath},~\vect{\jmath}\right)$}
\def\Oijk{$\left(\text{O}~;~\vect{\imath},~\vect{\jmath},~\vect{k}\right)$}
\def\Ouv{$\left(\text{O}~;~\vect{u},~\vect{v}\right)$}

\hypersetup{breaklinks=true, colorlinks = true, linkcolor = OliveGreen, urlcolor = OliveGreen, citecolor = OliveGreen, pdfauthor={Didier BONNEL - https://www.maths-cours.fr} } % supprime les bordures autour des liens

\renewcommand{\arg}[0]{\text{arg}}

\everymath{\displaystyle}

%================================================================================================================================
%
% Macros - Commandes
%
%================================================================================================================================

\newcommand\meta[2]{    			% Utilisé pour créer le post HTML.
	\def\titre{titre}
	\def\url{url}
	\def\arg{#1}
	\ifx\titre\arg
		\newcommand\maintitle{#2}
		\fancyhead[L]{#2}
		{\Large\sffamily \MakeUppercase{#2}}
		\vspace{1mm}\textcolor{mcvert}{\hrule}
	\fi 
	\ifx\url\arg
		\fancyfoot[L]{\href{https://www.maths-cours.fr#2}{\black \footnotesize{https://www.maths-cours.fr#2}}}
	\fi 
}


\newcommand\TitreC[1]{    		% Titre centré
     \needspace{3\baselineskip}
     \begin{center}\textbf{#1}\end{center}
}

\newcommand\newpar{    		% paragraphe
     \par
}

\newcommand\nosp {    		% commande vide (pas d'espace)
}
\newcommand{\id}[1]{} %ignore

\newcommand\boite[2]{				% Boite simple sans titre
	\vspace{5mm}
	\setlength{\fboxrule}{0.2mm}
	\setlength{\fboxsep}{5mm}	
	\fcolorbox{#1}{#1!3}{\makebox[\linewidth-2\fboxrule-2\fboxsep]{
  		\begin{minipage}[t]{\linewidth-2\fboxrule-4\fboxsep}\setlength{\parskip}{3mm}
  			 #2
  		\end{minipage}
	}}
	\vspace{5mm}
}

\newcommand\CBox[4]{				% Boites
	\vspace{5mm}
	\setlength{\fboxrule}{0.2mm}
	\setlength{\fboxsep}{5mm}
	
	\fcolorbox{#1}{#1!3}{\makebox[\linewidth-2\fboxrule-2\fboxsep]{
		\begin{minipage}[t]{1cm}\setlength{\parskip}{3mm}
	  		\textcolor{#1}{\LARGE{#2}}    
 	 	\end{minipage}  
  		\begin{minipage}[t]{\linewidth-2\fboxrule-4\fboxsep}\setlength{\parskip}{3mm}
			\raisebox{1.2mm}{\normalsize\sffamily{\textcolor{#1}{#3}}}						
  			 #4
  		\end{minipage}
	}}
	\vspace{5mm}
}

\newcommand\cadre[3]{				% Boites convertible html
	\par
	\vspace{2mm}
	\setlength{\fboxrule}{0.1mm}
	\setlength{\fboxsep}{5mm}
	\fcolorbox{#1}{white}{\makebox[\linewidth-2\fboxrule-2\fboxsep]{
  		\begin{minipage}[t]{\linewidth-2\fboxrule-4\fboxsep}\setlength{\parskip}{3mm}
			\raisebox{-2.5mm}{\sffamily \small{\textcolor{#1}{\MakeUppercase{#2}}}}		
			\par		
  			 #3
 	 		\end{minipage}
	}}
		\vspace{2mm}
	\par
}

\newcommand\bloc[3]{				% Boites convertible html sans bordure
     \needspace{2\baselineskip}
     {\sffamily \small{\textcolor{#1}{\MakeUppercase{#2}}}}    
		\par		
  			 #3
		\par
}

\newcommand\CHelp[1]{
     \CBox{Plum}{\faInfoCircle}{À RETENIR}{#1}
}

\newcommand\CUp[1]{
     \CBox{NavyBlue}{\faThumbsOUp}{EN PRATIQUE}{#1}
}

\newcommand\CInfo[1]{
     \CBox{Sepia}{\faArrowCircleRight}{REMARQUE}{#1}
}

\newcommand\CRedac[1]{
     \CBox{PineGreen}{\faEdit}{BIEN R\'EDIGER}{#1}
}

\newcommand\CError[1]{
     \CBox{Red}{\faExclamationTriangle}{ATTENTION}{#1}
}

\newcommand\TitreExo[2]{
\needspace{4\baselineskip}
 {\sffamily\large EXERCICE #1\ (\emph{#2 points})}
\vspace{5mm}
}

\newcommand\img[2]{
          \includegraphics[width=#2\paperwidth]{\imgdir#1}
}

\newcommand\imgsvg[2]{
       \begin{center}   \includegraphics[width=#2\paperwidth]{\imgsvgdir#1} \end{center}
}


\newcommand\Lien[2]{
     \href{#1}{#2 \tiny \faExternalLink}
}
\newcommand\mcLien[2]{
     \href{https~://www.maths-cours.fr/#1}{#2 \tiny \faExternalLink}
}

\newcommand{\euro}{\eurologo{}}

%================================================================================================================================
%
% Macros - Environement
%
%================================================================================================================================

\newenvironment{tex}{ %
}
{%
}

\newenvironment{indente}{ %
	\setlength\parindent{10mm}
}

{
	\setlength\parindent{0mm}
}

\newenvironment{corrige}{%
     \needspace{3\baselineskip}
     \medskip
     \textbf{\textsc{Corrigé}}
     \medskip
}
{
}

\newenvironment{extern}{%
     \begin{center}
     }
     {
     \end{center}
}

\NewEnviron{code}{%
	\par
     \boite{gray}{\texttt{%
     \BODY
     }}
     \par
}

\newenvironment{vbloc}{% boite sans cadre empeche saut de page
     \begin{minipage}[t]{\linewidth}
     }
     {
     \end{minipage}
}
\NewEnviron{h2}{%
    \needspace{3\baselineskip}
    \vspace{0.6cm}
	\noindent \MakeUppercase{\sffamily \large \BODY}
	\vspace{1mm}\textcolor{mcgris}{\hrule}\vspace{0.4cm}
	\par
}{}

\NewEnviron{h3}{%
    \needspace{3\baselineskip}
	\vspace{5mm}
	\textsc{\BODY}
	\par
}

\NewEnviron{margeneg}{ %
\begin{addmargin}[-1cm]{0cm}
\BODY
\end{addmargin}
}

\NewEnviron{html}{%
}

\begin{document}
\meta{url}{/exercices/graphes-bac-es-l-pondichery-2018-spe/}
\meta{pid}{7116}
\meta{titre}{Graphes – Bac ES/L Pondichéry 2018 (spé)}
\meta{type}{exercice}
\begin{h2}Exercice 3 (5 points)\end{h2}
\textbf{Candidats ayant suivi l'enseignement de spécialité}
\medskip
Les différentes parties de cet exercice peuvent être traitées de façon indépendante.
\begin{center}\begin{h3}Partie A \end{h3}\end{center}
Le graphe pondéré ci-dessous représente les différents lieux A, B, C, D, E, F, G et H dans lesquels
Louis est susceptible de se rendre chaque jour. Le lieu A désigne son domicile et G le lieu de son
site de travail.
\par
Le poids de chaque arête représente la distance, en kilomètres, entre les deux lieux
reliés par l'arête.
\par
\begin{center}
     \begin{extern}%width="380px" alt="graphe pondéré"
          \psset{unit=0.6cm}
          \begin{pspicture}(15,11.5)
               %\psgrid
               % <html:width="380px">
               \rput(2.3,9.7){\circlenode{C}{C}}
               \rput(3.5,6.4){\circlenode{B}{B}}
               \rput(10.2,8.3){\circlenode{D}{D}}
               \rput(8,5){\circlenode{E}{E}}
               \rput(1.4,1.5){\circlenode{A}{A}}
               \rput(7,0.5){\circlenode{F}{F}}
               \rput(14,6){\circlenode{G}{G}}
               \rput(12.5,1.8){\circlenode{H}{H}}
               \ncarc[arcangle=-30]{C}{A}\ncput*[nrot=:U]{56}
               \ncarc[arcangle=20]{A}{B}\ncput*[nrot=:U]{47}
               \ncarc[arcangle=20]{A}{E}\ncput*[nrot=:U]{23}
               \ncarc[arcangle=20]{A}{F}\ncput*[nrot=:U]{30}
               \ncarc[arcangle=20]{C}{B}\ncput*[nrot=:U]{10}
               \ncarc[arcangle=20]{B}{E}\ncput*[nrot=:U]{20}
               \ncarc[arcangle=20]{C}{D}\ncput*[nrot=:U]{15}
               \ncarc[arcangle=20]{E}{D}\ncput*[nrot=:U]{42}
               \ncarc[arcangle=20]{D}{G}\ncput*[nrot=:U]{15}
               \ncarc[arcangle=20]{E}{F}\ncput*[nrot=:U]{28}
               \ncarc[arcangle=20]{E}{H}\ncput*[nrot=:U]{40}
               \ncarc[arcangle=20]{F}{H}\ncput*[nrot=:U]{28}
               \ncarc[arcangle=20]{G}{H}\ncput*[nrot=:U]{23}
          \end{pspicture}
     \end{extern}
\end{center}
\medskip
Déterminer le chemin le plus court qui permet à Louis de relier son domicile à son travail. On
pourra utiliser un algorithme. Préciser la distance, en kilomètres, de ce chemin.
\begin{center}\begin{h3}Partie B \end{h3}\end{center}
Afin de réduire son empreinte énergétique, Louis décide d'utiliser lors de ses trajets quotidiens
soit les transports en commun, soit le covoiturage.
\begin{itemize}
     \item s'il a utilisé les transports en commun lors d'un trajet, il utilisera le covoiturage lors de son
     prochain déplacement avec une probabilité de $0,53$~;
     \item s'il a utilisé le covoiturage lors d'un trajet, il effectuera le prochain déplacement en transport en commun avec une probabilité de $0,78$.
\end{itemize}
\medskip
Louis décide de mettre en place ces résolutions au 1$^{\text{er}}$ janvier 2018.
\smallskip
Pour tout entier naturel $n$, on note~:
\begin{itemize}
     \item $c_n$ la probabilité que Louis utilise le covoiturage $n$ jour(s) après le 1$^{\text{er}}$ janvier 2018~;
     \item $t_n$ la probabilité que Louis utilise les transports en commun $n$ jour(s) après le 1$^{\text{er}}$ janvier 2018~;
\end{itemize}
\medskip
La matrice ligne $P_n = \left(c_n \quad t_n\right)$ traduit l'état probabiliste $n$ jour(s) après le 1$^{\text{er}}$  janvier 2018.
\smallskip
Le 1$^{\text{er}}$ janvier 2018, Louis décide d'utiliser le covoiturage.
\medskip
\begin{enumerate}
     \item
     \begin{enumerate}[label=\alph*.]
          \item Préciser l'état probabiliste initial $P_0$.
          \item Traduire les données de l'énoncé par un graphe probabiliste.
          On notera \og C \fg{} et \og T \fg{} ses deux sommets~:
          \begin{itemize}
               \item \og C\fg{} pour indiquer que Louis utilise le covoiturage~;
               \item \og T\fg{} pour indiquer que Louis utilise les transports en commun.
          \end{itemize}
     \end{enumerate}
     \item  Déterminer la matrice de transition du graphe probabiliste en considérant ses sommets dans
     l'ordre alphabétique.
     \item  Calculer l'état probabiliste $P_2$ et interpréter ce résultat dans le cadre de l'exercice.
     \item  Soit la matrice ligne $P = (x \quad y)$ associée à l'état stable du graphe probabiliste.
     \begin{enumerate}[label=\alph*.]
          \item Calculer les valeurs exactes de $x$ et de $y$ puis en donner une valeur approchée à $0,01$ près.
          \item Selon ce modèle, peut-on dire qu'à long terme, Louis utilisera aussi souvent le covoiturage
          que les transports en commun~? Justifier la réponse.
     \end{enumerate}
\end{enumerate}

\end{document}
µ
\documentclass[a4paper]{article}

%================================================================================================================================
%
% Packages
%
%================================================================================================================================

\usepackage[T1]{fontenc} 	% pour caractères accentués
\usepackage[utf8]{inputenc}  % encodage utf8
\usepackage[french]{babel}	% langue : français
\usepackage{fourier}			% caractères plus lisibles
\usepackage[dvipsnames]{xcolor} % couleurs
\usepackage{fancyhdr}		% réglage header footer
\usepackage{needspace}		% empêcher sauts de page mal placés
\usepackage{graphicx}		% pour inclure des graphiques
\usepackage{enumitem,cprotect}		% personnalise les listes d'items (nécessaire pour ol, al ...)
\usepackage{hyperref}		% Liens hypertexte
\usepackage{pstricks,pst-all,pst-node,pstricks-add,pst-math,pst-plot,pst-tree,pst-eucl} % pstricks
\usepackage[a4paper,includeheadfoot,top=2cm,left=3cm, bottom=2cm,right=3cm]{geometry} % marges etc.
\usepackage{comment}			% commentaires multilignes
\usepackage{amsmath,environ} % maths (matrices, etc.)
\usepackage{amssymb,makeidx}
\usepackage{bm}				% bold maths
\usepackage{tabularx}		% tableaux
\usepackage{colortbl}		% tableaux en couleur
\usepackage{fontawesome}		% Fontawesome
\usepackage{environ}			% environment with command
\usepackage{fp}				% calculs pour ps-tricks
\usepackage{multido}			% pour ps tricks
\usepackage[np]{numprint}	% formattage nombre
\usepackage{tikz,tkz-tab} 			% package principal TikZ
\usepackage{pgfplots}   % axes
\usepackage{mathrsfs}    % cursives
\usepackage{calc}			% calcul taille boites
\usepackage[scaled=0.875]{helvet} % font sans serif
\usepackage{svg} % svg
\usepackage{scrextend} % local margin
\usepackage{scratch} %scratch
\usepackage{multicol} % colonnes
%\usepackage{infix-RPN,pst-func} % formule en notation polanaise inversée
\usepackage{listings}

%================================================================================================================================
%
% Réglages de base
%
%================================================================================================================================

\lstset{
language=Python,   % R code
literate=
{á}{{\'a}}1
{à}{{\`a}}1
{ã}{{\~a}}1
{é}{{\'e}}1
{è}{{\`e}}1
{ê}{{\^e}}1
{í}{{\'i}}1
{ó}{{\'o}}1
{õ}{{\~o}}1
{ú}{{\'u}}1
{ü}{{\"u}}1
{ç}{{\c{c}}}1
{~}{{ }}1
}


\definecolor{codegreen}{rgb}{0,0.6,0}
\definecolor{codegray}{rgb}{0.5,0.5,0.5}
\definecolor{codepurple}{rgb}{0.58,0,0.82}
\definecolor{backcolour}{rgb}{0.95,0.95,0.92}

\lstdefinestyle{mystyle}{
    backgroundcolor=\color{backcolour},   
    commentstyle=\color{codegreen},
    keywordstyle=\color{magenta},
    numberstyle=\tiny\color{codegray},
    stringstyle=\color{codepurple},
    basicstyle=\ttfamily\footnotesize,
    breakatwhitespace=false,         
    breaklines=true,                 
    captionpos=b,                    
    keepspaces=true,                 
    numbers=left,                    
xleftmargin=2em,
framexleftmargin=2em,            
    showspaces=false,                
    showstringspaces=false,
    showtabs=false,                  
    tabsize=2,
    upquote=true
}

\lstset{style=mystyle}


\lstset{style=mystyle}
\newcommand{\imgdir}{C:/laragon/www/newmc/assets/imgsvg/}
\newcommand{\imgsvgdir}{C:/laragon/www/newmc/assets/imgsvg/}

\definecolor{mcgris}{RGB}{220, 220, 220}% ancien~; pour compatibilité
\definecolor{mcbleu}{RGB}{52, 152, 219}
\definecolor{mcvert}{RGB}{125, 194, 70}
\definecolor{mcmauve}{RGB}{154, 0, 215}
\definecolor{mcorange}{RGB}{255, 96, 0}
\definecolor{mcturquoise}{RGB}{0, 153, 153}
\definecolor{mcrouge}{RGB}{255, 0, 0}
\definecolor{mclightvert}{RGB}{205, 234, 190}

\definecolor{gris}{RGB}{220, 220, 220}
\definecolor{bleu}{RGB}{52, 152, 219}
\definecolor{vert}{RGB}{125, 194, 70}
\definecolor{mauve}{RGB}{154, 0, 215}
\definecolor{orange}{RGB}{255, 96, 0}
\definecolor{turquoise}{RGB}{0, 153, 153}
\definecolor{rouge}{RGB}{255, 0, 0}
\definecolor{lightvert}{RGB}{205, 234, 190}
\setitemize[0]{label=\color{lightvert}  $\bullet$}

\pagestyle{fancy}
\renewcommand{\headrulewidth}{0.2pt}
\fancyhead[L]{maths-cours.fr}
\fancyhead[R]{\thepage}
\renewcommand{\footrulewidth}{0.2pt}
\fancyfoot[C]{}

\newcolumntype{C}{>{\centering\arraybackslash}X}
\newcolumntype{s}{>{\hsize=.35\hsize\arraybackslash}X}

\setlength{\parindent}{0pt}		 
\setlength{\parskip}{3mm}
\setlength{\headheight}{1cm}

\def\ebook{ebook}
\def\book{book}
\def\web{web}
\def\type{web}

\newcommand{\vect}[1]{\overrightarrow{\,\mathstrut#1\,}}

\def\Oij{$\left(\text{O}~;~\vect{\imath},~\vect{\jmath}\right)$}
\def\Oijk{$\left(\text{O}~;~\vect{\imath},~\vect{\jmath},~\vect{k}\right)$}
\def\Ouv{$\left(\text{O}~;~\vect{u},~\vect{v}\right)$}

\hypersetup{breaklinks=true, colorlinks = true, linkcolor = OliveGreen, urlcolor = OliveGreen, citecolor = OliveGreen, pdfauthor={Didier BONNEL - https://www.maths-cours.fr} } % supprime les bordures autour des liens

\renewcommand{\arg}[0]{\text{arg}}

\everymath{\displaystyle}

%================================================================================================================================
%
% Macros - Commandes
%
%================================================================================================================================

\newcommand\meta[2]{    			% Utilisé pour créer le post HTML.
	\def\titre{titre}
	\def\url{url}
	\def\arg{#1}
	\ifx\titre\arg
		\newcommand\maintitle{#2}
		\fancyhead[L]{#2}
		{\Large\sffamily \MakeUppercase{#2}}
		\vspace{1mm}\textcolor{mcvert}{\hrule}
	\fi 
	\ifx\url\arg
		\fancyfoot[L]{\href{https://www.maths-cours.fr#2}{\black \footnotesize{https://www.maths-cours.fr#2}}}
	\fi 
}


\newcommand\TitreC[1]{    		% Titre centré
     \needspace{3\baselineskip}
     \begin{center}\textbf{#1}\end{center}
}

\newcommand\newpar{    		% paragraphe
     \par
}

\newcommand\nosp {    		% commande vide (pas d'espace)
}
\newcommand{\id}[1]{} %ignore

\newcommand\boite[2]{				% Boite simple sans titre
	\vspace{5mm}
	\setlength{\fboxrule}{0.2mm}
	\setlength{\fboxsep}{5mm}	
	\fcolorbox{#1}{#1!3}{\makebox[\linewidth-2\fboxrule-2\fboxsep]{
  		\begin{minipage}[t]{\linewidth-2\fboxrule-4\fboxsep}\setlength{\parskip}{3mm}
  			 #2
  		\end{minipage}
	}}
	\vspace{5mm}
}

\newcommand\CBox[4]{				% Boites
	\vspace{5mm}
	\setlength{\fboxrule}{0.2mm}
	\setlength{\fboxsep}{5mm}
	
	\fcolorbox{#1}{#1!3}{\makebox[\linewidth-2\fboxrule-2\fboxsep]{
		\begin{minipage}[t]{1cm}\setlength{\parskip}{3mm}
	  		\textcolor{#1}{\LARGE{#2}}    
 	 	\end{minipage}  
  		\begin{minipage}[t]{\linewidth-2\fboxrule-4\fboxsep}\setlength{\parskip}{3mm}
			\raisebox{1.2mm}{\normalsize\sffamily{\textcolor{#1}{#3}}}						
  			 #4
  		\end{minipage}
	}}
	\vspace{5mm}
}

\newcommand\cadre[3]{				% Boites convertible html
	\par
	\vspace{2mm}
	\setlength{\fboxrule}{0.1mm}
	\setlength{\fboxsep}{5mm}
	\fcolorbox{#1}{white}{\makebox[\linewidth-2\fboxrule-2\fboxsep]{
  		\begin{minipage}[t]{\linewidth-2\fboxrule-4\fboxsep}\setlength{\parskip}{3mm}
			\raisebox{-2.5mm}{\sffamily \small{\textcolor{#1}{\MakeUppercase{#2}}}}		
			\par		
  			 #3
 	 		\end{minipage}
	}}
		\vspace{2mm}
	\par
}

\newcommand\bloc[3]{				% Boites convertible html sans bordure
     \needspace{2\baselineskip}
     {\sffamily \small{\textcolor{#1}{\MakeUppercase{#2}}}}    
		\par		
  			 #3
		\par
}

\newcommand\CHelp[1]{
     \CBox{Plum}{\faInfoCircle}{À RETENIR}{#1}
}

\newcommand\CUp[1]{
     \CBox{NavyBlue}{\faThumbsOUp}{EN PRATIQUE}{#1}
}

\newcommand\CInfo[1]{
     \CBox{Sepia}{\faArrowCircleRight}{REMARQUE}{#1}
}

\newcommand\CRedac[1]{
     \CBox{PineGreen}{\faEdit}{BIEN R\'EDIGER}{#1}
}

\newcommand\CError[1]{
     \CBox{Red}{\faExclamationTriangle}{ATTENTION}{#1}
}

\newcommand\TitreExo[2]{
\needspace{4\baselineskip}
 {\sffamily\large EXERCICE #1\ (\emph{#2 points})}
\vspace{5mm}
}

\newcommand\img[2]{
          \includegraphics[width=#2\paperwidth]{\imgdir#1}
}

\newcommand\imgsvg[2]{
       \begin{center}   \includegraphics[width=#2\paperwidth]{\imgsvgdir#1} \end{center}
}


\newcommand\Lien[2]{
     \href{#1}{#2 \tiny \faExternalLink}
}
\newcommand\mcLien[2]{
     \href{https~://www.maths-cours.fr/#1}{#2 \tiny \faExternalLink}
}

\newcommand{\euro}{\eurologo{}}

%================================================================================================================================
%
% Macros - Environement
%
%================================================================================================================================

\newenvironment{tex}{ %
}
{%
}

\newenvironment{indente}{ %
	\setlength\parindent{10mm}
}

{
	\setlength\parindent{0mm}
}

\newenvironment{corrige}{%
     \needspace{3\baselineskip}
     \medskip
     \textbf{\textsc{Corrigé}}
     \medskip
}
{
}

\newenvironment{extern}{%
     \begin{center}
     }
     {
     \end{center}
}

\NewEnviron{code}{%
	\par
     \boite{gray}{\texttt{%
     \BODY
     }}
     \par
}

\newenvironment{vbloc}{% boite sans cadre empeche saut de page
     \begin{minipage}[t]{\linewidth}
     }
     {
     \end{minipage}
}
\NewEnviron{h2}{%
    \needspace{3\baselineskip}
    \vspace{0.6cm}
	\noindent \MakeUppercase{\sffamily \large \BODY}
	\vspace{1mm}\textcolor{mcgris}{\hrule}\vspace{0.4cm}
	\par
}{}

\NewEnviron{h3}{%
    \needspace{3\baselineskip}
	\vspace{5mm}
	\textsc{\BODY}
	\par
}

\NewEnviron{margeneg}{ %
\begin{addmargin}[-1cm]{0cm}
\BODY
\end{addmargin}
}

\NewEnviron{html}{%
}

\begin{document}
\meta{url}{/exercices/suites-et-fonctions-bac-s-pondichery-2018/}
\meta{pid}{7127}
\meta{titre}{Géométrie dans l'espace – Bac S Pondichéry 2018}
\meta{type}{exercice}
\begin{h2}Exercice 1 (6 points)\end{h2}
\textbf{Commun  à tous les candidats}
\medskip
\emph{Les parties } A \emph{et}   B \emph{peuvent être traitées de façon indépendante.}
\bigskip
Dans une usine, un four cuit des céramiques à la température de 1~000~\degres C. À la fin de la
cuisson, il est éteint et il refroidit.
\smallskip
On s'intéresse à la phase de refroidissement du four, qui débute dès l'instant où il est éteint.
\smallskip
La température du four est exprimée en degré Celsius (\degres~C).
\smallskip
La porte du four peut être ouverte sans risque pour les céramiques dès que sa température est
inférieure à $70$\degres~C. Sinon les céramiques peuvent se fissurer, voire se casser.
\begin{center}\begin{h3}Partie A \end{h3}\end{center}
Pour un nombre entier naturel $n$, on note $T_n$ la température en degré Celsius du four au bout
de $n$ heures écoulées à partir de l'instant où il a été éteint. On a donc $T_0 = 1~000$.
\par
La température $T_n$ est calculée par l'algorithme suivant~:
\begin{center}
     \begin{extern}%width="230px" alt="algorithme suite arithmético-géométrique"
          \begin{tabularx}{0.35\linewidth}{|X|}\hline
               $T \gets 1~000$\\
               Pour $i$ allant de $1$ à $n$\\
               \hspace{1cm}$T \gets 0,82 \times T + 3,6$\\
               Fin Pour\\\hline
          \end{tabularx}
     \end{extern}
\end{center}
\begin{enumerate}
     \item Déterminer la température du four, arrondie à l'unité, au bout de $4$ heures de
     refroidissement.
     \item  Démontrer que, pour tout nombre entier naturel $n$, on a~: $T_n = 980 \times 0,82^n + 20$.
     \item  Au bout de combien d'heures le four peut-il être ouvert sans risque pour les céramiques~?
\end{enumerate}
\begin{center}\begin{h3}Partie B \end{h3}\end{center}
Dans cette partie, on note $t$ le temps (en heure) écoulé depuis l'instant où le four a été éteint.
\par
La température du four (en degré Celsius) à l'instant $t$ est donnée par la fonction $f$ définie,
pour tout nombre réel $t$ positif, par~:
\[f(t) = a\text{e}^{- \frac{t}{5}} + b,\]
où $a$ et $b$ sont deux nombres réels.
\par
On admet que $f$ vérifie la relation suivante~: $f'(t) + \dfrac{1}{5}f(t) = 4$.
\medskip
\begin{enumerate}
     \item Déterminer les valeurs de $a$ et $b$ sachant qu'initialement, la température du four est de
     $1~000$~\degres C, c'est-à-dire que $f(0) = 1~000$.
     \item  Pour la suite, on admet que, pour tout nombre réel positif $t$~:
     \par
     \[f(t) = 980\text{e}^{- \frac{t}{5}} + 20.\]
     \begin{enumerate}[label=\alph*.]
          \item Déterminer la limite de $f$ lorsque $t$ tend vers $+ \infty$.
          \item Étudier les variations de $f$ sur $[0~;~+ \infty[$.
          \par
          En déduire son tableau de variations complet.
          \item Avec ce modèle, après combien de minutes le four peut-il être ouvert sans risque pour
          les céramiques~?
     \end{enumerate}
     \item  La température moyenne (en degré Celsius) du four entre deux instants $t_1$ et $t_2$ est donnée
     par~: $\dfrac{1}{t_2 - t_1}\displaystyle\int_{t_1}^{t_2} f(t)\:\text{d}t$.
     \begin{enumerate}[label=\alph*.]
          \item À l'aide de la représentation graphique de $f$ ci-dessous, donner une estimation de la
          température moyenne $\theta$ du four sur les $15$ premières heures de refroidissement.
          \par
          Expliquer votre démarche.
          \begin{center}
               \begin{extern}%alt="fonction température du four"
                    \definecolor{darkgreen}{rgb}{0.,0.4,0.}
                    \psset{xunit=0.6cm,yunit=0.01cm}
                    \begin{pspicture}(-1,-50)(19,1120)
                         \multido{\n=0+1}{20}{\psline[linecolor=lightgray,linewidth=0.5pt](\n,0)(\n,1100)}
                         \multido{\n=0+100}{12}{\psline[linecolor=lightgray,linewidth=0.5pt](0,\n)(19,\n)}
                         \psaxes[linewidth=0.5pt,Dy=200]{->}(0,0)(0,0)(19,1120)
                         \psaxes[linewidth=0.5pt,Dy=200](0,0)(0,0)(19,1120)
                         \uput[d](16.2,-60){temps écoulé (en heures)}
                         \uput[r](0,1050){température (en degrés Celsius)}
                         \psplot[plotpoints=3000,linewidth=0.75pt,linecolor=darkgreen]{0}{19}{980 2.71828 0.2 x mul exp div 20 add}
                    \end{pspicture}
               \end{extern}
          \end{center}
          \item  Calculer la valeur exacte de cette température moyenne $\theta$ et en donner la valeur
          arrondie au degré Celsius.
     \end{enumerate}
     \item  Dans cette question, on s'intéresse à l'abaissement de température (en degré Celsius) du
     four au cours d'une heure, soit entre deux instants $t$ et $(t + 1)$. Cet abaissement est donné
     par la fonction $d$ définie, pour tout nombre réel $t$ positif, par~: $d(t) = f(t) - f(t + 1)$.
     \begin{enumerate}[label=\alph*.]
          \item Vérifier que. pour tout nombre réel $t$ positif~: $d(t) = 980\left(1 - \text{e}^{- \frac{1}{5}}\right)\text{e}^{- \frac{t}{5}}$.
          \item Déterminer la limite de $d(t)$ lorsque $t$ tend vers $+ \infty$.
          \par
          Quelle interprétation peut-on en donner~?
     \end{enumerate}
\end{enumerate}
\begin{corrige}
     \begin{center}\begin{h3}Partie A \end{h3}\end{center}
     \begin{enumerate}
          \item La température du four au bout de 4 heures de
          refroidissement est égale à $T_4$.
          \par
          L'algorithme fourni par l'énoncé conduit à la relation de récurrence~:
          \begin{center}$T_{n+1}=0,82T_n+3,6 $\end{center}
          Pour trouver $T_4$ on peut~:
          \begin{itemize}[label=---]
               \item %
               programmer l'algorithme sur sa calculatrice
               \item %
               utiliser le menu suite de la calculatrice
               \item %
               calculer directement $T_1, T_2, T_3, T_4$.
          \end{itemize}
          $T_1=0,82 T_0 + 3,6 $\nosp$= 0,82 \times 1\ 000+3,6$\nosp$ \approx 824$\\
          $T_2=0,82 T_1 + 3,6 \approx 679$\\
          $T_3=0,82 T_2 + 3,6 \approx 560$\\
          $T_4=0,82 T_3 + 3,6 \approx 463$
          \par
          La température du four au bout de 4 heures de
          refroidissement est $T_4\approx 463$~\degres~C (arrondie à l'unité).
          \item
          Montrons par récurrence que pour tout entier naturel $n$~:
          \begin{center}$ T_n=980 \times 0,82^n + 20 $\end{center}
          \smallskip
          \textbf{\textit{Initialisation}}
          \smallskip
          $T_0=1~000$ et $980 \times 0,82^0+20=1~000$\\
          \par
          La propriété est donc vraie au rang 0.
          \smallskip
          \textbf{\textit{Hérédité}}
          \smallskip
          Supposons que pour un certain entier naturel $n$~: $T_n=980 \times 0,82^n + 20$ et montrons que $T_{n+1} = 980 \times 0,82^{n+1} + 20 $.\\
          \par
          $T_{n+1}=0,82T_n+3,6$\\
          $\phantom{T_{n+1}}=0,82\times \left( 980 \times 0,82^n + 20\right) +3,6$\\
          $\phantom{T_{n+1}}=980 \times 0,82^{n+1} + 16,4+3,6 $\\
          $\phantom{T_{n+1}}=980 \times 0,82^{n+1} + 20 $\\
          \par
          La propriété est donc héréditaire.
          \smallskip
          \textbf{\textit{Conclusion}}
          \smallskip
          Pour tout entier naturel $n$~:
          \begin{center}$ T_n=980 \times 0,82^n + 20 $\end{center}
          \item
          Le four peut être ouvert sans risque pour les céramiques si et seulement si $T_n \leqslant 70$.
          \par
          $T_n \leqslant 70 \Leftrightarrow 980 \times 0,82^n + 20 \leqslant 70$\\
          $\phantom{T_n \leqslant 70} \Leftrightarrow 0,82^n \leqslant \dfrac{50}{980}$
          \par
          La fonction $\ln$ étant strictement croissante sur $]0~;~+\infty[$~:
          \par
          $\phantom{T_n \leqslant 70} \Leftrightarrow n\ln(0,82) \leqslant \ln\left( \dfrac{5}{98}\right)$
          \par
          Comme $\ln(0,82)<0$ on divise par $\ln(0,82)$ en changeant le sens de l'inégalité~:
          \par
          $\phantom{T_n \leqslant 70} \Leftrightarrow  n \geqslant \dfrac{\ln\left( \dfrac{5}{98}\right)}{\ln(0,82)}\approx 14,99.$
          \par
          Le four peut être ouvert sans risque pour les céramiques au bout de 15 heures.
     \end{enumerate}
     \begin{center}\begin{h3}Partie B \end{h3}\end{center}
     \begin{enumerate}
          \item
          $f$ est dérivable sur $[0~;~+\infty[$ et~:
          \par
          $f'(t)=-\dfrac{a}{5}\text{e}^{-\frac{t}{5}}$.
          \par
          La condition $f'(t)+\dfrac{1}{5}f(t)=4$ s'écrit alors~:
          \par
          $-\dfrac{a}{5}\text{e}^{-\frac{t}{5}} + \dfrac{a}{5}\text{e}^{-\frac{t}{5}} + \dfrac{b}{5} = 4$
          \par
          Donc  $ \dfrac{b}{5} = 4$ et $b=20$.
          \par
          De plus $f(0) = 1~000$ donc $a\text{e}^0+20=1~000$ et $a=980$.
          \par
          $f$ est donc définie sur $[0~;~+\infty[$ par $f(t)=980\text{e}^{-\frac{t}{5}} + 20$.
          \item
          \begin{enumerate}[label=\alph*.]
               \item
               $\lim\limits_{t \rightarrow +\infty}-\dfrac{t}{5} = -\infty$\\
               \par
               $\lim\limits_{x \rightarrow -\infty}\text{e}^{x} = 0$\\
               \par
               Donc par composition~:\\
               \par
               $\lim\limits_{x \rightarrow -\infty}\text{e}^{-\frac{t}{5}} = 0$\\
               \par
               et\\
               \par
               $\lim\limits_{x \rightarrow -\infty}f(t) = \lim\limits_{x \rightarrow -\infty}980\text{e}^{-\frac{t}{5}} + 20 = 20.$\\
               \item
               Pour tout réel positif $t$, $f'(t)=-196\text{e}^{-\frac{t}{5}}$ est strictement négatif. La fonction $f$ est donc strictement décroissante sur $[0~;~+\infty[$.\\
               \par
               On obtient alors le tableau de variation suivant~:
               \begin{center}
                    \begin{extern}%width="260px" alt="tableau de variations"
                         \begin{tikzpicture}[scale=0.875]
                              % source : tikz
                              % Styles
                              \tikzstyle{cadre}=[thin]
                              \tikzstyle{fleche}=[->,>=latex,thin]
                              \tikzstyle{nondefini}=[lightgray]
                              % Dimensions Modifiables
                              \def\Lrg{1.5}
                              \def\HtX{1}
                              \def\HtY{0.5}
                              % Dimensions Calculées
                              \def\lignex{-0.5*\HtX}
                              \def\lignef{-1.5*\HtX}
                              \def\separateur{-0.5*\Lrg}
                              % Largeur du tableau
                              \def\gauche{-1.5*\Lrg}
                              \def\droite{2.5*\Lrg}
                              % Hauteur du tableau
                              \def\haut{0.5*\HtX}
                              \def\bas{-2.5*\HtX-2*\HtY}
                              % Ligne de l'abscisse : x
                              \node at (-1*\Lrg,0) {$t$};
                              \node at (0*\Lrg,0) {$0$};
                              \node at (2*\Lrg,0) {$+\infty$};
                              % Ligne de la dérivée : f'(x)
                              \node at (-1*\Lrg,-1*\HtX) {$f'(t)$};
                              \node at (0*\Lrg,-1*\HtX) {$$};
                              \node at (1*\Lrg,-1*\HtX) {$-$};
                              \node at (2*\Lrg,-1*\HtX) {$$};
                              % Ligne de la fonction : f(x)
                              \node  at (-1*\Lrg,{-2*\HtX+(-1)*\HtY}) {$f(t)$};
                              \node (f1) at (0*\Lrg,{-2*\HtX+(0)*\HtY}) {$1~000$};
                              \node (f2) at (2*\Lrg,{-2*\HtX+(-2)*\HtY}) {$20$};
                              % Flèches
                              \draw[fleche] (f1) -- (f2);
                              % Encadrement
                              \draw[cadre] (\separateur,\haut) -- (\separateur,\bas);
                              \draw[cadre] (\gauche,\haut) rectangle  (\droite,\bas);
                              \draw[cadre] (\gauche,\lignex) -- (\droite,\lignex);
                              \draw[cadre] (\gauche,\lignef) -- (\droite,\lignef);
                         \end{tikzpicture}
                    \end{extern}
               \end{center}
               \item
               Le four peut être ouvert sans risque pour
               les céramiques si et seulement si $f(t) \leqslant 70$.\\
               \par
               $f(t) \leqslant 70 \Leftrightarrow 980\text{e}^{-\frac{t}{5}} + 20  \leqslant 70$\\
               $\phantom{f(t) \leqslant 70} \Leftrightarrow \text{e}^{-\frac{t}{5}} \leqslant \dfrac{50}{980}$\\
               \par
               La fonction $\ln$ est strictement croissante sur $]0~;~+\infty[$, donc~:\\
               \par
               $\phantom{f(t) \leqslant 70} \Leftrightarrow \ln \text{e}^{-\frac{t}{5}} \leqslant \ln\dfrac{5}{98}$\\
               $\phantom{f(t) \leqslant 70} \Leftrightarrow t \geqslant -5\ln\dfrac{5}{98} \approx 14,88$\\
               \par
               On peut ouvrir le four sans risque au bout de 14,88 heures soit 14,88$\times$60$\approx$893 minutes. \\
          \end{enumerate}
          \item
          \begin{enumerate}[label=\alph*.]
               \item %
               L'intégrale $\displaystyle\int_{0}^{15}f(t)\text{d}t$ est égale à l'aire, en unités d'aire, du domaine délimité par la courbe $\mathcal{C}_f$, l'axe des abscisses, l'axe des ordonnées et la droite d'équation $x=15$. \\
               \begin{center}
                    \begin{extern}%alt="courbe et intégrale"
                         \definecolor{darkgreen}{rgb}{0.,0.4,0.}
                         \psset{xunit=0.6cm,yunit=0.01cm}
                         \begin{pspicture}(-1,-50)(19,1120)
                              \multido{\n=0+1}{20}{\psline[linecolor=lightgray,linewidth=0.5pt](\n,0)(\n,1100)}
                              \multido{\n=0+100}{12}{\psline[linecolor=lightgray,linewidth=0.5pt](0,\n)(19,\n)}
                              \psaxes[linewidth=0.5pt,Dy=200]{->}(0,0)(0,0)(19,1120)
                              \psaxes[linewidth=0.5pt,Dy=200](0,0)(0,0)(19,1120)
                              \uput[d](16.2,-60){temps écoulé (en heures)}
                              \uput[r](0,1050){température (en degrés Celsius)}
                              \pscustom[fillstyle=solid,fillcolor=darkgreen,opacity=0.1]{
                                   \psplot[plotpoints=3000,linewidth=0.75pt,linecolor=darkgreen]{0}{15}{980 2.71828 0.2 x mul exp div 20 add}
                              \psline[linewidth=0.75pt,linecolor=darkgreen](15,0)(0,0)(0,1000)}
                              \psplot[plotpoints=3000,linewidth=0.75pt,linecolor=darkgreen]{0}{19}{980 2.71828 0.2 x mul exp div 20 add}
                         \end{pspicture}
                    \end{extern}
               \end{center}
               En comptant les carreaux rectangulaires du graphique, on peut estimer cette aire a environ 50 carreaux soit 5~000 U.A.
               \par
               Une valeur approchée de la température moyenne $\theta$ est donc~:
               \par
               $\theta \approx \dfrac{1}{15} \times 500 \approx 333$\degres~C
               \item
               Une primitive de la fonction $f$ est la fonction $F$ définie sur $[0~;~+\infty[$ par $F(t)=-5 \times 980\text{e}^{-\frac{t}{5}} + 20t$ $=-4~900\text{e}^{-\frac{t}{5}} + 20t$.\\
               \par
               On a donc~:
               $\theta =\dfrac{1}{15}\displaystyle\int_{0}^{15}f(t)\text{d}t=\dfrac{1}{15}\left(F(15)-F(0)\right)$\\
               $\theta =  \dfrac{-980\text{e}^{-3}+1~040}{3}$\\
               $\theta \approx 330$\degres~C  (arrondi au degré).
          \end{enumerate}
          \item
          \begin{enumerate}[label=\alph*.]
               \item
               Pour tout réel $t$ positif ou nul~:\\
               \par
               $d(t) = f(t) -f(t + 1)$\\
               $\phantom{d(t)}=\left(980\text{e}^{-\frac{t}{5}} + 20 \right)$\nosp$ -\left(980\text{e}^{-\frac{t+1}{5}} + 20 \right)$\\
               $\phantom{d(t)} =980\left(\text{e}^{-\frac{t}{5}}-\text{e}^{-\frac{t+1}{5}}\right) $\\
               $\phantom{d(t)} = 980\left(1 -\text{e}^{-\frac{1}{5}}\right)\text{e}^{-\frac{t}{5}}$
               \item
               $\lim\limits_{t \rightarrow +\infty} -\dfrac{t}{5}=~-\infty$
               \par
               Donc~:
               \par
               $\lim\limits_{t \rightarrow +\infty} \text{e}^{-\frac{t}{5}} = 0$
               \par
               et $\lim\limits_{t \rightarrow +\infty} d(t)=0$.
               \par
               Il est difficile de trouver une interprétation correcte de ce résultat.
               \par
               On peut certes dire que la température se stabilise au cours du temps mais cela résulte du fait que $\lim\limits_{t \rightarrow +\infty} f(t)=20$ et non du fait que  $\lim\limits_{t \rightarrow +\infty} d(t)=0$.
          \end{enumerate}
     \end{enumerate}
\end{corrige}

\end{document}
µ
\documentclass[a4paper]{article}

%================================================================================================================================
%
% Packages
%
%================================================================================================================================

\usepackage[T1]{fontenc} 	% pour caractères accentués
\usepackage[utf8]{inputenc}  % encodage utf8
\usepackage[french]{babel}	% langue : français
\usepackage{fourier}			% caractères plus lisibles
\usepackage[dvipsnames]{xcolor} % couleurs
\usepackage{fancyhdr}		% réglage header footer
\usepackage{needspace}		% empêcher sauts de page mal placés
\usepackage{graphicx}		% pour inclure des graphiques
\usepackage{enumitem,cprotect}		% personnalise les listes d'items (nécessaire pour ol, al ...)
\usepackage{hyperref}		% Liens hypertexte
\usepackage{pstricks,pst-all,pst-node,pstricks-add,pst-math,pst-plot,pst-tree,pst-eucl} % pstricks
\usepackage[a4paper,includeheadfoot,top=2cm,left=3cm, bottom=2cm,right=3cm]{geometry} % marges etc.
\usepackage{comment}			% commentaires multilignes
\usepackage{amsmath,environ} % maths (matrices, etc.)
\usepackage{amssymb,makeidx}
\usepackage{bm}				% bold maths
\usepackage{tabularx}		% tableaux
\usepackage{colortbl}		% tableaux en couleur
\usepackage{fontawesome}		% Fontawesome
\usepackage{environ}			% environment with command
\usepackage{fp}				% calculs pour ps-tricks
\usepackage{multido}			% pour ps tricks
\usepackage[np]{numprint}	% formattage nombre
\usepackage{tikz,tkz-tab} 			% package principal TikZ
\usepackage{pgfplots}   % axes
\usepackage{mathrsfs}    % cursives
\usepackage{calc}			% calcul taille boites
\usepackage[scaled=0.875]{helvet} % font sans serif
\usepackage{svg} % svg
\usepackage{scrextend} % local margin
\usepackage{scratch} %scratch
\usepackage{multicol} % colonnes
%\usepackage{infix-RPN,pst-func} % formule en notation polanaise inversée
\usepackage{listings}

%================================================================================================================================
%
% Réglages de base
%
%================================================================================================================================

\lstset{
language=Python,   % R code
literate=
{á}{{\'a}}1
{à}{{\`a}}1
{ã}{{\~a}}1
{é}{{\'e}}1
{è}{{\`e}}1
{ê}{{\^e}}1
{í}{{\'i}}1
{ó}{{\'o}}1
{õ}{{\~o}}1
{ú}{{\'u}}1
{ü}{{\"u}}1
{ç}{{\c{c}}}1
{~}{{ }}1
}


\definecolor{codegreen}{rgb}{0,0.6,0}
\definecolor{codegray}{rgb}{0.5,0.5,0.5}
\definecolor{codepurple}{rgb}{0.58,0,0.82}
\definecolor{backcolour}{rgb}{0.95,0.95,0.92}

\lstdefinestyle{mystyle}{
    backgroundcolor=\color{backcolour},   
    commentstyle=\color{codegreen},
    keywordstyle=\color{magenta},
    numberstyle=\tiny\color{codegray},
    stringstyle=\color{codepurple},
    basicstyle=\ttfamily\footnotesize,
    breakatwhitespace=false,         
    breaklines=true,                 
    captionpos=b,                    
    keepspaces=true,                 
    numbers=left,                    
xleftmargin=2em,
framexleftmargin=2em,            
    showspaces=false,                
    showstringspaces=false,
    showtabs=false,                  
    tabsize=2,
    upquote=true
}

\lstset{style=mystyle}


\lstset{style=mystyle}
\newcommand{\imgdir}{C:/laragon/www/newmc/assets/imgsvg/}
\newcommand{\imgsvgdir}{C:/laragon/www/newmc/assets/imgsvg/}

\definecolor{mcgris}{RGB}{220, 220, 220}% ancien~; pour compatibilité
\definecolor{mcbleu}{RGB}{52, 152, 219}
\definecolor{mcvert}{RGB}{125, 194, 70}
\definecolor{mcmauve}{RGB}{154, 0, 215}
\definecolor{mcorange}{RGB}{255, 96, 0}
\definecolor{mcturquoise}{RGB}{0, 153, 153}
\definecolor{mcrouge}{RGB}{255, 0, 0}
\definecolor{mclightvert}{RGB}{205, 234, 190}

\definecolor{gris}{RGB}{220, 220, 220}
\definecolor{bleu}{RGB}{52, 152, 219}
\definecolor{vert}{RGB}{125, 194, 70}
\definecolor{mauve}{RGB}{154, 0, 215}
\definecolor{orange}{RGB}{255, 96, 0}
\definecolor{turquoise}{RGB}{0, 153, 153}
\definecolor{rouge}{RGB}{255, 0, 0}
\definecolor{lightvert}{RGB}{205, 234, 190}
\setitemize[0]{label=\color{lightvert}  $\bullet$}

\pagestyle{fancy}
\renewcommand{\headrulewidth}{0.2pt}
\fancyhead[L]{maths-cours.fr}
\fancyhead[R]{\thepage}
\renewcommand{\footrulewidth}{0.2pt}
\fancyfoot[C]{}

\newcolumntype{C}{>{\centering\arraybackslash}X}
\newcolumntype{s}{>{\hsize=.35\hsize\arraybackslash}X}

\setlength{\parindent}{0pt}		 
\setlength{\parskip}{3mm}
\setlength{\headheight}{1cm}

\def\ebook{ebook}
\def\book{book}
\def\web{web}
\def\type{web}

\newcommand{\vect}[1]{\overrightarrow{\,\mathstrut#1\,}}

\def\Oij{$\left(\text{O}~;~\vect{\imath},~\vect{\jmath}\right)$}
\def\Oijk{$\left(\text{O}~;~\vect{\imath},~\vect{\jmath},~\vect{k}\right)$}
\def\Ouv{$\left(\text{O}~;~\vect{u},~\vect{v}\right)$}

\hypersetup{breaklinks=true, colorlinks = true, linkcolor = OliveGreen, urlcolor = OliveGreen, citecolor = OliveGreen, pdfauthor={Didier BONNEL - https://www.maths-cours.fr} } % supprime les bordures autour des liens

\renewcommand{\arg}[0]{\text{arg}}

\everymath{\displaystyle}

%================================================================================================================================
%
% Macros - Commandes
%
%================================================================================================================================

\newcommand\meta[2]{    			% Utilisé pour créer le post HTML.
	\def\titre{titre}
	\def\url{url}
	\def\arg{#1}
	\ifx\titre\arg
		\newcommand\maintitle{#2}
		\fancyhead[L]{#2}
		{\Large\sffamily \MakeUppercase{#2}}
		\vspace{1mm}\textcolor{mcvert}{\hrule}
	\fi 
	\ifx\url\arg
		\fancyfoot[L]{\href{https://www.maths-cours.fr#2}{\black \footnotesize{https://www.maths-cours.fr#2}}}
	\fi 
}


\newcommand\TitreC[1]{    		% Titre centré
     \needspace{3\baselineskip}
     \begin{center}\textbf{#1}\end{center}
}

\newcommand\newpar{    		% paragraphe
     \par
}

\newcommand\nosp {    		% commande vide (pas d'espace)
}
\newcommand{\id}[1]{} %ignore

\newcommand\boite[2]{				% Boite simple sans titre
	\vspace{5mm}
	\setlength{\fboxrule}{0.2mm}
	\setlength{\fboxsep}{5mm}	
	\fcolorbox{#1}{#1!3}{\makebox[\linewidth-2\fboxrule-2\fboxsep]{
  		\begin{minipage}[t]{\linewidth-2\fboxrule-4\fboxsep}\setlength{\parskip}{3mm}
  			 #2
  		\end{minipage}
	}}
	\vspace{5mm}
}

\newcommand\CBox[4]{				% Boites
	\vspace{5mm}
	\setlength{\fboxrule}{0.2mm}
	\setlength{\fboxsep}{5mm}
	
	\fcolorbox{#1}{#1!3}{\makebox[\linewidth-2\fboxrule-2\fboxsep]{
		\begin{minipage}[t]{1cm}\setlength{\parskip}{3mm}
	  		\textcolor{#1}{\LARGE{#2}}    
 	 	\end{minipage}  
  		\begin{minipage}[t]{\linewidth-2\fboxrule-4\fboxsep}\setlength{\parskip}{3mm}
			\raisebox{1.2mm}{\normalsize\sffamily{\textcolor{#1}{#3}}}						
  			 #4
  		\end{minipage}
	}}
	\vspace{5mm}
}

\newcommand\cadre[3]{				% Boites convertible html
	\par
	\vspace{2mm}
	\setlength{\fboxrule}{0.1mm}
	\setlength{\fboxsep}{5mm}
	\fcolorbox{#1}{white}{\makebox[\linewidth-2\fboxrule-2\fboxsep]{
  		\begin{minipage}[t]{\linewidth-2\fboxrule-4\fboxsep}\setlength{\parskip}{3mm}
			\raisebox{-2.5mm}{\sffamily \small{\textcolor{#1}{\MakeUppercase{#2}}}}		
			\par		
  			 #3
 	 		\end{minipage}
	}}
		\vspace{2mm}
	\par
}

\newcommand\bloc[3]{				% Boites convertible html sans bordure
     \needspace{2\baselineskip}
     {\sffamily \small{\textcolor{#1}{\MakeUppercase{#2}}}}    
		\par		
  			 #3
		\par
}

\newcommand\CHelp[1]{
     \CBox{Plum}{\faInfoCircle}{À RETENIR}{#1}
}

\newcommand\CUp[1]{
     \CBox{NavyBlue}{\faThumbsOUp}{EN PRATIQUE}{#1}
}

\newcommand\CInfo[1]{
     \CBox{Sepia}{\faArrowCircleRight}{REMARQUE}{#1}
}

\newcommand\CRedac[1]{
     \CBox{PineGreen}{\faEdit}{BIEN R\'EDIGER}{#1}
}

\newcommand\CError[1]{
     \CBox{Red}{\faExclamationTriangle}{ATTENTION}{#1}
}

\newcommand\TitreExo[2]{
\needspace{4\baselineskip}
 {\sffamily\large EXERCICE #1\ (\emph{#2 points})}
\vspace{5mm}
}

\newcommand\img[2]{
          \includegraphics[width=#2\paperwidth]{\imgdir#1}
}

\newcommand\imgsvg[2]{
       \begin{center}   \includegraphics[width=#2\paperwidth]{\imgsvgdir#1} \end{center}
}


\newcommand\Lien[2]{
     \href{#1}{#2 \tiny \faExternalLink}
}
\newcommand\mcLien[2]{
     \href{https~://www.maths-cours.fr/#1}{#2 \tiny \faExternalLink}
}

\newcommand{\euro}{\eurologo{}}

%================================================================================================================================
%
% Macros - Environement
%
%================================================================================================================================

\newenvironment{tex}{ %
}
{%
}

\newenvironment{indente}{ %
	\setlength\parindent{10mm}
}

{
	\setlength\parindent{0mm}
}

\newenvironment{corrige}{%
     \needspace{3\baselineskip}
     \medskip
     \textbf{\textsc{Corrigé}}
     \medskip
}
{
}

\newenvironment{extern}{%
     \begin{center}
     }
     {
     \end{center}
}

\NewEnviron{code}{%
	\par
     \boite{gray}{\texttt{%
     \BODY
     }}
     \par
}

\newenvironment{vbloc}{% boite sans cadre empeche saut de page
     \begin{minipage}[t]{\linewidth}
     }
     {
     \end{minipage}
}
\NewEnviron{h2}{%
    \needspace{3\baselineskip}
    \vspace{0.6cm}
	\noindent \MakeUppercase{\sffamily \large \BODY}
	\vspace{1mm}\textcolor{mcgris}{\hrule}\vspace{0.4cm}
	\par
}{}

\NewEnviron{h3}{%
    \needspace{3\baselineskip}
	\vspace{5mm}
	\textsc{\BODY}
	\par
}

\NewEnviron{margeneg}{ %
\begin{addmargin}[-1cm]{0cm}
\BODY
\end{addmargin}
}

\NewEnviron{html}{%
}

\begin{document}
\meta{url}{/exercices/nombres-complexes-bac-s-pondichery-2018/}
\meta{pid}{7157}
\meta{titre}{Nombres complexes – Bac S Pondichéry 2018}
\meta{type}{exercice}
\begin{h2}Exercice 2 (4 points)\end{h2}
\textbf{Commun  à tous les candidats}
\medskip
Le plan est muni d'un repère orthonormé $(O~;~\overrightarrow{u},~\overrightarrow{v})$.
\smallskip
Les points A, B et C ont pour affixes respectives $a = -4,\: b = 2$ et $c = 4$.
\medskip
\begin{enumerate}
     \item On considère les trois points A$'$, B$'$ et C$'$ d'affixes respectives $a'= ja$, $b'= jb$ et $c'= jc$ où $j$ est le nombre complexe $-\dfrac{1}{2} + \text{i}\dfrac{\sqrt{3}}{2}$.
     \begin{enumerate}[label=\alph*.]
          \item Donner la forme trigonométrique et la forme exponentielle de $j$.
          \par
          En déduire les formes algébriques et exponentielles de $a'$, $b'$ et $c'$.
          \item Les points A, B et C ainsi que les cercles de centre O et de rayon 2, 3 et 4 sont
          représentés sur le graphique fourni en \textbf{Annexe}.
          \par
          Placer les points A$'$, B$'$ et C$'$ sur ce graphique.
     \end{enumerate}
     \item  Montrer que les points A$'$, B$'$ et C$'$ sont alignés.
     \item  On note M le milieu du segment [A$'$C], N le milieu du segment [C$'$C] et P le milieu du
     segment $[\text{C}'\text{A}]$.
     \par
     Démontrer que le triangle MNP est isocèle.
\end{enumerate}
\begin{center}
     \bigskip
     \textbf{ANNEXE}
     \par
     \textit{À compléter et à remettre avec la copie}
     \bigskip
     \begin{extern}%width="400px"
          % src~:nombres-complexes-bac-s-pondichery-2018-1.ggb
          \psset{xunit=1.0cm,yunit=1.0cm,algebraic=true,dimen=middle,dotstyle=*,dotsize=3pt 0,linewidth=0.5pt,arrowsize=3pt 2,arrowinset=0.25}
          \begin{pspicture*}(-4.74,-4.846666666666671)(4.86,4.713333333333333)
               \multips(0,-4)(0,1.0){10}{\psline[linestyle=dashed,linecap=1,dash=1.5pt 1.5pt,linewidth=0.4pt,linecolor=lightgray]{c-c}(-4.74,0)(4.86,0)}
               \multips(-4,0)(1.0,0){10}{\psline[linestyle=dashed,linecap=1,dash=1.5pt 1.5pt,linewidth=0.4pt,linecolor=lightgray]{c-c}(0,-4.846666666666671)(0,4.713333333333333)}
               \psaxes[labelFontSize=\scriptstyle,xAxis=true,yAxis=true,labels=none,Dx=1.,Dy=1.,ticksize=-0.5pt 0,subticks=2]{->}(0,0)(-4.74,-4.846666666666671)(4.86,4.713333333333333)
               \pscircle[linewidth=0.5pt](0.,0.){2.}
               \pscircle[linewidth=0.5pt](0.,0.){4.}
               \psline[linewidth=0.5pt]{->}(0.,0.)(1.,0.)
               \psline[linewidth=0.5pt]{->}(0.,0.)(0.,1.)
               \pscircle[linewidth=0.5pt](0.,0.){3.}
               \begin{scriptsize}
                    \psdots[dotsize=3pt 0,dotstyle=*,linecolor=darkgray](2.,0.)
                    \rput[bl](2.1,0.19333333333333094){\normalsize \darkgray{$B$}}
                    \psdots[dotsize=3pt 0,dotstyle=*,linecolor=darkgray](4.,0.)
                    \rput[bl](4.14,0.21333333333333096){\normalsize \darkgray{$C$}}
                    \psdots[dotsize=3pt 0,dotstyle=*,linecolor=darkgray](0.,0.)
                    \rput[bl](-0.64,-0.5266666666666694){\normalsize \darkgray{$O$}}
                    \rput[bl](0.24,-0.6266666666666693){\normalsize $\overrightarrow{u}$}
                    \rput[bl](-0.6,0.25333333333333097){\normalsize $\overrightarrow{v}$}
                    \psdots[dotsize=3pt 0,dotstyle=*,linecolor=darkgray](-4.,0.)
                    \rput[bl](-4.48,0.21333333333333096){\normalsize \darkgray{$A$}}
               \end{scriptsize}
          \end{pspicture*}
     \end{extern}
\end{center}
\begin{corrige}
     \begin{enumerate}
          \item
          \begin{enumerate}[label=\alph*.]
               \item
               $j=-\dfrac{1}{2} + \text{i}\dfrac{\sqrt{3}}{2}$\\
               $\left| j \right| = \sqrt{\left(-\dfrac{1}{2}\right)^2+\left(\dfrac{\sqrt{3}}{2}\right)^2} = \sqrt{\dfrac{1}{4}+\dfrac{3}{4}}=1$\\
               $\theta$ est un argument de $j$ si et seulement si $\cos \theta = -\dfrac{1}{2}$ et $\sin \theta = \dfrac{\sqrt{3}}{2}$. Donc $\dfrac{2\pi}{3}$ est un argument de $j$.
               \par
               La forme trigonométrique de $j$ est~:
               \par
               $j=\cos\left(\dfrac{2\pi}{3}\right) + \text{i}\sin\left(\dfrac{2\pi}{3}\right)$
               \par
               et sa forme exponentielle~:
               \par
               $j= \text{e}^{\frac{2\text{i}\pi}{3}}$.
               \par
               La forme algébrique de $a'$ est~:
               \par
               $a'=aj=-4j=2-2\text{i}\sqrt{3}$.
               \par
               Par ailleurs~:
               \par
               $a'=-4j= -4\text{e}^{\frac{2\text{i}\pi}{3}}$
               \par
               Toutefois $-4$ étant négatif, l'écriture ci-dessus n'est pas la forme exponentielle de $a'$.
               \par
               Pour obtenir la forme exponentielle de $a'$ on utilise le fait que $-1=\text{e}^{\text{i}\pi}$~; par conséquent~:
               \par
               $a'=-4\left( \text{e}^{\frac{2\text{i}\pi}{3}}\right)$\\
               $\phantom{a'}=4 \text{e}^{\text{i}\pi}\text{e}^{\frac{2\text{i}\pi}{3}}$\\
               $\phantom{a'}=4\text{e}^{\text{i}\left( \pi+\frac{2\pi}{3}\right)  }$.
               \par
               La forme exponentielle de $a'$ est donc~:
               \par
               $a'=4\text{e}^{\frac{5\text{i}\pi}{3}}$.
               \smallskip
               La forme algébrique de $b'$ est~:
               \par
               $b'= bj=2j=-1+\text{i}\sqrt{3}$
               \par
               et sa forme exponentielle~:
               \par
               $b'=2j=2\text{e}^{\frac{2\text{i}\pi}{3}}$.
               \par
               Enfin, la forme algébrique de $c'$ est~:
               \par
               $c'= cj=4j=-2+2\text{i}\sqrt{3}$
               \par
               et sa forme exponentielle~:
               \par
               $c'=4j=4\text{e}^{\frac{2\text{i}\pi}{3}}$.
               \item
               Voir figure ci-après.
          \end{enumerate}
          \item
          L'affixe du vecteur $\overrightarrow{A'B'}$ est~:
          \par
          $b'-a'=2j-(-4j)=6j$.
          \par
          L'affixe du vecteur $\overrightarrow{B'C'}$ est~:
          \par
          $c'-b'=4j-2j=2j$.
          \par
          Par conséquent $\overrightarrow{A'B'}$ =3$\overrightarrow{B'C'}$.
          \par
          Les vecteurs $\overrightarrow{A'B'}$ et $\overrightarrow{B'C'}$ sont colinéaires donc les points $A'$, $B'$ et $C'$ sont alignés.
          \item
          ~
          \begin{center}
               \begin{extern}%width="400px"
                    % src~:nombres-complexes-bac-s-pondichery-2018-2.ggb
                    \newrgbcolor{qqwuqq}{0. 0.39215686274509803 0.}
                    \psset{xunit=1.0cm,yunit=1.0cm,algebraic=true,dimen=middle,dotstyle=o,dotsize=3pt 0,linewidth=0.5pt,arrowsize=3pt 2,arrowinset=0.25}
                    \begin{pspicture*}(-4.74,-4.846666666666671)(4.86,4.713333333333333)
                         \multips(0,-4)(0,1.0){10}{\psline[linestyle=dashed,linecap=1,dash=1.5pt 1.5pt,linewidth=0.4pt,linecolor=lightgray]{c-c}(-4.74,0)(4.86,0)}
                         \multips(-4,0)(1.0,0){10}{\psline[linestyle=dashed,linecap=1,dash=1.5pt 1.5pt,linewidth=0.4pt,linecolor=lightgray]{c-c}(0,-4.846666666666671)(0,4.713333333333333)}
                         \psaxes[labelFontSize=\scriptstyle,xAxis=true,yAxis=true,labels=none,Dx=1.,Dy=1.,ticksize=-2pt 0,subticks=2]{->}(0,0)(-4.74,-4.846666666666671)(4.86,4.713333333333333)
                         \pscircle[linewidth=0.5pt](0.,0.){2.}
                         \pscircle[linewidth=0.5pt](0.,0.){4.}
                         \psline[linewidth=0.5pt]{->}(0.,0.)(1.,0.)
                         \psline[linewidth=0.5pt]{->}(0.,0.)(0.,1.)
                         \pscircle[linewidth=0.5pt](0.,0.){3.}
                         \psline[linecolor=qqwuqq](-2.995923759375434,1.734397839456537)(1.004076240624566,1.7343978394565367)
                         \psline[linecolor=qqwuqq](1.004076240624566,1.7343978394565367)(2.9810026048630416,-1.7428809165436194)
                         \psline[linecolor=qqwuqq](2.9810026048630416,-1.7428809165436194)(-2.995923759375434,1.734397839456537)
                         \begin{scriptsize}
                              \normalsize
                              \psdots[dotsize=3pt 0,dotstyle=*,linecolor=darkgray](2.,0.)
                              \rput[bl](2.1,0.19333333333333094){\darkgray{$B$}}
                              \psdots[dotsize=3pt 0,dotstyle=*,linecolor=darkgray](4.,0.)
                              \rput[bl](4.14,0.21333333333333096){\darkgray{$C$}}
                              \psdots[dotsize=3pt 0,dotstyle=*,linecolor=darkgray](0.,0.)
                              \rput[bl](-0.74,-0.6266666666666694){\darkgray{$O$}}
                              \rput[bl](0.24,-0.7266666666666693){$\overrightarrow{u}$}
                              \rput[bl](-0.6,0.25333333333333097){$\overrightarrow{v}$}
                              \psdots[dotsize=3pt 0,dotstyle=*,linecolor=darkgray](-4.,0.)
                              \rput[bl](-4.48,0.21333333333333096){\darkgray{$A$}}
                              \psdots[dotsize=3pt 0,dotstyle=*,linecolor=blue](1.9620052097260836,-3.485761833087239)
                              \rput[bl](1.78,-3.24666666666667){\blue{$A'$}}
                              \psdots[dotsize=3pt 0,dotstyle=*,linecolor=blue](-0.9508987680858454,1.7594861559139423)
                              \rput[bl](-1.02,2.0733333333333315){\blue{$B'$}}
                              \psdots[dotsize=3pt 0,dotstyle=*,linecolor=blue](-1.9918475187508682,3.4687956789130734)
                              \rput[bl](-2.08,3.753333333333332){\blue{$C'$}}
                              \psdots[dotsize=3pt 0,dotstyle=*,linecolor=qqwuqq](2.9810026048630416,-1.7428809165436194)
                              \rput[bl](3.06,-1.6666666666666698){\qqwuqq{$M$}}
                              \psdots[dotsize=3pt 0,dotstyle=*,linecolor=qqwuqq](1.004076240624566,1.7343978394565367)
                              \rput[bl](1.08,1.8133333333333315){\qqwuqq{$N$}}
                              \psdots[dotsize=3pt 0,dotstyle=*,linecolor=qqwuqq](-2.995923759375434,1.734397839456537)
                              \rput[bl](-3.02,1.9133333333333316){\qqwuqq{$P$}}
                         \end{scriptsize}
                    \end{pspicture*}
               \end{extern}
          \end{center}
          L'affixe de M est~:
          \par
          $m=\dfrac{a'+c}{2}=3-\text{i}\sqrt{3}$
          \par
          L'affixe de N est~:
          \par
          $n=\dfrac{c'+c}{2}=1+\text{i}\sqrt{3}$
          \par
          L'affixe de P est~:
          \par
          $p=\dfrac{c'+a}{2}=-3+\text{i}\sqrt{3}$
          \par
          Montrons que $MN=PN$\\
          $MN=\left|m-n \right| = \left|2-2\text{i}\sqrt{3} \right| $\\
          $\phantom{MN}=\sqrt{2^2+\left(2 \sqrt{3}\right)^2}=\sqrt{4+12}=4$\\
          $PN=\left|n-p \right| =\left|4 \right| = 4$
          \par
          Le triangle $MNP$ est donc isocèle en $N$.
     \end{enumerate}
\end{corrige}
\par

\end{document}
µ
\documentclass[a4paper]{article}

%================================================================================================================================
%
% Packages
%
%================================================================================================================================

\usepackage[T1]{fontenc} 	% pour caractères accentués
\usepackage[utf8]{inputenc}  % encodage utf8
\usepackage[french]{babel}	% langue : français
\usepackage{fourier}			% caractères plus lisibles
\usepackage[dvipsnames]{xcolor} % couleurs
\usepackage{fancyhdr}		% réglage header footer
\usepackage{needspace}		% empêcher sauts de page mal placés
\usepackage{graphicx}		% pour inclure des graphiques
\usepackage{enumitem,cprotect}		% personnalise les listes d'items (nécessaire pour ol, al ...)
\usepackage{hyperref}		% Liens hypertexte
\usepackage{pstricks,pst-all,pst-node,pstricks-add,pst-math,pst-plot,pst-tree,pst-eucl} % pstricks
\usepackage[a4paper,includeheadfoot,top=2cm,left=3cm, bottom=2cm,right=3cm]{geometry} % marges etc.
\usepackage{comment}			% commentaires multilignes
\usepackage{amsmath,environ} % maths (matrices, etc.)
\usepackage{amssymb,makeidx}
\usepackage{bm}				% bold maths
\usepackage{tabularx}		% tableaux
\usepackage{colortbl}		% tableaux en couleur
\usepackage{fontawesome}		% Fontawesome
\usepackage{environ}			% environment with command
\usepackage{fp}				% calculs pour ps-tricks
\usepackage{multido}			% pour ps tricks
\usepackage[np]{numprint}	% formattage nombre
\usepackage{tikz,tkz-tab} 			% package principal TikZ
\usepackage{pgfplots}   % axes
\usepackage{mathrsfs}    % cursives
\usepackage{calc}			% calcul taille boites
\usepackage[scaled=0.875]{helvet} % font sans serif
\usepackage{svg} % svg
\usepackage{scrextend} % local margin
\usepackage{scratch} %scratch
\usepackage{multicol} % colonnes
%\usepackage{infix-RPN,pst-func} % formule en notation polanaise inversée
\usepackage{listings}

%================================================================================================================================
%
% Réglages de base
%
%================================================================================================================================

\lstset{
language=Python,   % R code
literate=
{á}{{\'a}}1
{à}{{\`a}}1
{ã}{{\~a}}1
{é}{{\'e}}1
{è}{{\`e}}1
{ê}{{\^e}}1
{í}{{\'i}}1
{ó}{{\'o}}1
{õ}{{\~o}}1
{ú}{{\'u}}1
{ü}{{\"u}}1
{ç}{{\c{c}}}1
{~}{{ }}1
}


\definecolor{codegreen}{rgb}{0,0.6,0}
\definecolor{codegray}{rgb}{0.5,0.5,0.5}
\definecolor{codepurple}{rgb}{0.58,0,0.82}
\definecolor{backcolour}{rgb}{0.95,0.95,0.92}

\lstdefinestyle{mystyle}{
    backgroundcolor=\color{backcolour},   
    commentstyle=\color{codegreen},
    keywordstyle=\color{magenta},
    numberstyle=\tiny\color{codegray},
    stringstyle=\color{codepurple},
    basicstyle=\ttfamily\footnotesize,
    breakatwhitespace=false,         
    breaklines=true,                 
    captionpos=b,                    
    keepspaces=true,                 
    numbers=left,                    
xleftmargin=2em,
framexleftmargin=2em,            
    showspaces=false,                
    showstringspaces=false,
    showtabs=false,                  
    tabsize=2,
    upquote=true
}

\lstset{style=mystyle}


\lstset{style=mystyle}
\newcommand{\imgdir}{C:/laragon/www/newmc/assets/imgsvg/}
\newcommand{\imgsvgdir}{C:/laragon/www/newmc/assets/imgsvg/}

\definecolor{mcgris}{RGB}{220, 220, 220}% ancien~; pour compatibilité
\definecolor{mcbleu}{RGB}{52, 152, 219}
\definecolor{mcvert}{RGB}{125, 194, 70}
\definecolor{mcmauve}{RGB}{154, 0, 215}
\definecolor{mcorange}{RGB}{255, 96, 0}
\definecolor{mcturquoise}{RGB}{0, 153, 153}
\definecolor{mcrouge}{RGB}{255, 0, 0}
\definecolor{mclightvert}{RGB}{205, 234, 190}

\definecolor{gris}{RGB}{220, 220, 220}
\definecolor{bleu}{RGB}{52, 152, 219}
\definecolor{vert}{RGB}{125, 194, 70}
\definecolor{mauve}{RGB}{154, 0, 215}
\definecolor{orange}{RGB}{255, 96, 0}
\definecolor{turquoise}{RGB}{0, 153, 153}
\definecolor{rouge}{RGB}{255, 0, 0}
\definecolor{lightvert}{RGB}{205, 234, 190}
\setitemize[0]{label=\color{lightvert}  $\bullet$}

\pagestyle{fancy}
\renewcommand{\headrulewidth}{0.2pt}
\fancyhead[L]{maths-cours.fr}
\fancyhead[R]{\thepage}
\renewcommand{\footrulewidth}{0.2pt}
\fancyfoot[C]{}

\newcolumntype{C}{>{\centering\arraybackslash}X}
\newcolumntype{s}{>{\hsize=.35\hsize\arraybackslash}X}

\setlength{\parindent}{0pt}		 
\setlength{\parskip}{3mm}
\setlength{\headheight}{1cm}

\def\ebook{ebook}
\def\book{book}
\def\web{web}
\def\type{web}

\newcommand{\vect}[1]{\overrightarrow{\,\mathstrut#1\,}}

\def\Oij{$\left(\text{O}~;~\vect{\imath},~\vect{\jmath}\right)$}
\def\Oijk{$\left(\text{O}~;~\vect{\imath},~\vect{\jmath},~\vect{k}\right)$}
\def\Ouv{$\left(\text{O}~;~\vect{u},~\vect{v}\right)$}

\hypersetup{breaklinks=true, colorlinks = true, linkcolor = OliveGreen, urlcolor = OliveGreen, citecolor = OliveGreen, pdfauthor={Didier BONNEL - https://www.maths-cours.fr} } % supprime les bordures autour des liens

\renewcommand{\arg}[0]{\text{arg}}

\everymath{\displaystyle}

%================================================================================================================================
%
% Macros - Commandes
%
%================================================================================================================================

\newcommand\meta[2]{    			% Utilisé pour créer le post HTML.
	\def\titre{titre}
	\def\url{url}
	\def\arg{#1}
	\ifx\titre\arg
		\newcommand\maintitle{#2}
		\fancyhead[L]{#2}
		{\Large\sffamily \MakeUppercase{#2}}
		\vspace{1mm}\textcolor{mcvert}{\hrule}
	\fi 
	\ifx\url\arg
		\fancyfoot[L]{\href{https://www.maths-cours.fr#2}{\black \footnotesize{https://www.maths-cours.fr#2}}}
	\fi 
}


\newcommand\TitreC[1]{    		% Titre centré
     \needspace{3\baselineskip}
     \begin{center}\textbf{#1}\end{center}
}

\newcommand\newpar{    		% paragraphe
     \par
}

\newcommand\nosp {    		% commande vide (pas d'espace)
}
\newcommand{\id}[1]{} %ignore

\newcommand\boite[2]{				% Boite simple sans titre
	\vspace{5mm}
	\setlength{\fboxrule}{0.2mm}
	\setlength{\fboxsep}{5mm}	
	\fcolorbox{#1}{#1!3}{\makebox[\linewidth-2\fboxrule-2\fboxsep]{
  		\begin{minipage}[t]{\linewidth-2\fboxrule-4\fboxsep}\setlength{\parskip}{3mm}
  			 #2
  		\end{minipage}
	}}
	\vspace{5mm}
}

\newcommand\CBox[4]{				% Boites
	\vspace{5mm}
	\setlength{\fboxrule}{0.2mm}
	\setlength{\fboxsep}{5mm}
	
	\fcolorbox{#1}{#1!3}{\makebox[\linewidth-2\fboxrule-2\fboxsep]{
		\begin{minipage}[t]{1cm}\setlength{\parskip}{3mm}
	  		\textcolor{#1}{\LARGE{#2}}    
 	 	\end{minipage}  
  		\begin{minipage}[t]{\linewidth-2\fboxrule-4\fboxsep}\setlength{\parskip}{3mm}
			\raisebox{1.2mm}{\normalsize\sffamily{\textcolor{#1}{#3}}}						
  			 #4
  		\end{minipage}
	}}
	\vspace{5mm}
}

\newcommand\cadre[3]{				% Boites convertible html
	\par
	\vspace{2mm}
	\setlength{\fboxrule}{0.1mm}
	\setlength{\fboxsep}{5mm}
	\fcolorbox{#1}{white}{\makebox[\linewidth-2\fboxrule-2\fboxsep]{
  		\begin{minipage}[t]{\linewidth-2\fboxrule-4\fboxsep}\setlength{\parskip}{3mm}
			\raisebox{-2.5mm}{\sffamily \small{\textcolor{#1}{\MakeUppercase{#2}}}}		
			\par		
  			 #3
 	 		\end{minipage}
	}}
		\vspace{2mm}
	\par
}

\newcommand\bloc[3]{				% Boites convertible html sans bordure
     \needspace{2\baselineskip}
     {\sffamily \small{\textcolor{#1}{\MakeUppercase{#2}}}}    
		\par		
  			 #3
		\par
}

\newcommand\CHelp[1]{
     \CBox{Plum}{\faInfoCircle}{À RETENIR}{#1}
}

\newcommand\CUp[1]{
     \CBox{NavyBlue}{\faThumbsOUp}{EN PRATIQUE}{#1}
}

\newcommand\CInfo[1]{
     \CBox{Sepia}{\faArrowCircleRight}{REMARQUE}{#1}
}

\newcommand\CRedac[1]{
     \CBox{PineGreen}{\faEdit}{BIEN R\'EDIGER}{#1}
}

\newcommand\CError[1]{
     \CBox{Red}{\faExclamationTriangle}{ATTENTION}{#1}
}

\newcommand\TitreExo[2]{
\needspace{4\baselineskip}
 {\sffamily\large EXERCICE #1\ (\emph{#2 points})}
\vspace{5mm}
}

\newcommand\img[2]{
          \includegraphics[width=#2\paperwidth]{\imgdir#1}
}

\newcommand\imgsvg[2]{
       \begin{center}   \includegraphics[width=#2\paperwidth]{\imgsvgdir#1} \end{center}
}


\newcommand\Lien[2]{
     \href{#1}{#2 \tiny \faExternalLink}
}
\newcommand\mcLien[2]{
     \href{https~://www.maths-cours.fr/#1}{#2 \tiny \faExternalLink}
}

\newcommand{\euro}{\eurologo{}}

%================================================================================================================================
%
% Macros - Environement
%
%================================================================================================================================

\newenvironment{tex}{ %
}
{%
}

\newenvironment{indente}{ %
	\setlength\parindent{10mm}
}

{
	\setlength\parindent{0mm}
}

\newenvironment{corrige}{%
     \needspace{3\baselineskip}
     \medskip
     \textbf{\textsc{Corrigé}}
     \medskip
}
{
}

\newenvironment{extern}{%
     \begin{center}
     }
     {
     \end{center}
}

\NewEnviron{code}{%
	\par
     \boite{gray}{\texttt{%
     \BODY
     }}
     \par
}

\newenvironment{vbloc}{% boite sans cadre empeche saut de page
     \begin{minipage}[t]{\linewidth}
     }
     {
     \end{minipage}
}
\NewEnviron{h2}{%
    \needspace{3\baselineskip}
    \vspace{0.6cm}
	\noindent \MakeUppercase{\sffamily \large \BODY}
	\vspace{1mm}\textcolor{mcgris}{\hrule}\vspace{0.4cm}
	\par
}{}

\NewEnviron{h3}{%
    \needspace{3\baselineskip}
	\vspace{5mm}
	\textsc{\BODY}
	\par
}

\NewEnviron{margeneg}{ %
\begin{addmargin}[-1cm]{0cm}
\BODY
\end{addmargin}
}

\NewEnviron{html}{%
}

\begin{document}
\meta{url}{/exercices/geometrie-dans-lespace-bac-s-pondichery-2018/}
\meta{pid}{7171}
\meta{titre}{Géométrie dans l'espace – Bac S Pondichéry 2018}
\meta{type}{exercice}
\begin{h2}Exercice 4 (5 points)\end{h2}
\textbf{Candidats n'ayant pas suivi l'enseignement de spécialité}
\medskip
Dans l'espace muni du repère orthonormé $(O~;~\overrightarrow{i},~\overrightarrow{j}~,~\overrightarrow{k})$ d'unité 1~cm, on considère les points
A, B, C et D de coordonnées respectives $(2~;~1~;~4)$, $(4~;~-1~;~0)$, $(0~;~3~;~2)$ et $(4~;~3~;~-2)$.
\medskip
\begin{enumerate}
     \item Déterminer une représentation paramétrique de la droite (CD).
     \item Soit M un point de la droite (CD).
     \begin{enumerate}[label=\alph*.]
          \item Déterminer les coordonnées du point M tel que la distance BM soit minimale.
          \item On note H le point de la droite (CD) ayant pour coordonnées $(3~;~3~;~- 1)$.
          Vérifier que les droites (BH) et (CD) sont perpendiculaires.
          \item Montrer que l'aire du triangle BCD est égale à 12 cm$^2$.
     \end{enumerate}
     \item
     \begin{enumerate}[label=\alph*.]
          \item Démontrer que le vecteur $\overrightarrow{n}\begin{pmatrix}2\\1\\2\end{pmatrix}$  est un vecteur normal au plan (BCD).
          \item Déterminer une équation cartésienne du plan (BCD).
          \item Déterminer une représentation paramétrique de la droite $\Delta$ passant par A et orthogonale
          au plan (BCD).
          \item Démontrer que le point I, intersection de la droite $\Delta$ et du plan (BCD) a pour
          coordonnées $\left(\dfrac{2}{3}~;~\dfrac{1}{3}~;~\dfrac{8}{3}\right)$.
     \end{enumerate}
     \item  Calculer le volume du tétraèdre ABCD.
\end{enumerate}
\begin{corrige}
     \begin{enumerate}
          \item
          \par
          Un vecteur directeur de la droite $(CD)$ est le vecteur $\overrightarrow{CD}$ de coordonnées $\begin{pmatrix} 4\\0\\-4 \end{pmatrix}$. Cette droite passe par le point  $C(0~;~3~;~2)$.
          \par
          Une représentation paramétrique de la droite $(CD)$ est donc~:
          \begin{center}
               $\begin{cases}
                    x=t\\y=3\\z=2-t
               \end{cases}~~(t \in  \mathbb{R})$
          \end{center}
          \item
          \begin{enumerate}[label=\alph*.]
               \item
               Si $M$ est un point de la droite $(CD)$, ses coordonnées sont de la forme $(t~;~3~;~2-t)$ où $t \in \mathbb{R}$
               \par
               La distance $BM$ vaut alors~:
               \par
               $BM=\sqrt{\left(t-4 \right)^2+\left(4 \right)^2+\left(2-t \right)^2}$\\
               $BM=\sqrt{2t^2-12t+36} $
               \par
               La distance $BM$ est minimale lorsque $2t^2-12t+36$ est minimal, c'est à dire pour $t= -\dfrac{b}{2a}=-\dfrac{-12}{4}=3$
               \par
               En remplaçant $t$ par $3$ dans les coordonnées du point $M$, on obtient que la distance  $BM$ est minimale pour $M(3~;~3~;~-1)$.
               \item
               Le vecteur $\overrightarrow{BH}$ a pour coordonnées $ \begin{pmatrix}-1\\4\\-1\end{pmatrix}$.
               \par
               Le vecteur  $\overrightarrow{CD}$ a pour coordonnées $\begin{pmatrix}4\\0\\-4\end{pmatrix}$.
               \par
               Le produit scalaire $\overrightarrow{HB} \cdot \overrightarrow{CD} $ vaut donc~:
               \par
               $\overrightarrow{HB}\cdot \overrightarrow{CD} = -1 \times 4+ 4 \times 0-1 \times (-4)= 0$
               \par
               Les droites $(BH)$ et $(CD)$ sont donc orthogonales et comme elles sont sécantes en $H$, elles sont perpendiculaires.
               \item
               D'après la question précédente, $(BH)$ est la hauteur issue de $B$ dans le triangle $BCD$.
               \par
               Par conséquent, l'aire du triangle $BCD$ est égale à~:
               \par
               $\mathscr{A}=\dfrac{1}{2} \times  CD \times  BH$\nosp$=\dfrac{1}{2}\times \sqrt{32} \times  \sqrt{18}$\nosp$=\dfrac{1}{2}\sqrt{576}=12$cm$^2$
          \end{enumerate}
          \item
          \begin{enumerate}[label=\alph*.]
               \item
               Le vecteur $\overrightarrow{n}$ est un vecteur normal au plan $(BCD)$ si et seulement s'il  est orthogonal à deux vecteurs non colinéaires de ce plan.
               \par
               Les vecteurs
               $\overrightarrow{BC}\begin{pmatrix}
                    -4\\4\\2
                    \end{pmatrix}$ et $\overrightarrow{CD}\begin{pmatrix}
                    4\\0\\-4
               \end{pmatrix}$ ne sont
               pas colinéaires et~:
               \par
               $\overrightarrow{n}\cdot\overrightarrow{BC}=-4 \times 2+4 \times 1+2\times 2=0$\\
               $\overrightarrow{n}\cdot\overrightarrow{CD}=4 \times 2+0\times 1-4\times 2=0$
               \par
               Le vecteur$\overrightarrow{n}$ est donc bien normal au plan $(BCD)$.
               \item
               Le vecteur $\overrightarrow{n}\begin{pmatrix}2\\1\\2\end{pmatrix}$ est normal au plan $(BCD)$ donc ce plan admet une équation cartésienne de la forme~: $2x+y+2z+d=0$ où $d \in \mathbb{R}$.
               \par
               Par ailleurs, le point $B(4~;~-1~;~0)$ appartient à ce plan donc ses coordonnées vérifient l'équation du plan.
               \par
               Par conséquent $2 \times 4 -1+2 \times 0+d=0$ donc $d=-7$.
               \par
               Une équation cartésienne du plan $(BCD)$ est donc $2x+y+2z-7=0$.
               \item
               $\Delta$ étant orthogonale au plan $(BCD)$, le vecteur $\overrightarrow{n}$ est un vecteur directeur de $\Delta$. Comme par ailleurs la droite $\Delta$ passe par le point $A(2~;~1~;~4)$, une représentation paramétrique de la droite  $\Delta$ est~:
               \begin{center}
                    $\begin{cases}
                         x=2+2t\\y=1+t\\z=4+2t
                    \end{cases}~~(t\in \mathbb{R})$
               \end{center}
               \item
               Soient $(x~;~y~;~z)$ les coordonnées du point $I$, intersection de la droite $\Delta$ et du plan $(BCD)$.
               \par
               Il existe une valeur de $t$ telle que les coordonnées de $I$ vérifient simultanément les équations~:
               \par
               \begin{center}
                    $\begin{cases}
                         x=2+2t\\y=1+t\\z=4+2t\\2x+y+2z-7=0
                    \end{cases}$
               \end{center}
               On a alors~:
               \par
               $2(2+2t)+(1+t)+2(4+2t)-7=0$
               \par
               soit $9t=-6$ et donc $t=-\dfrac{2}{3}$.
               \par
               Les coordonnées de $I$ sont donc~:
               \par
               $x=2+2t=\dfrac{2}{3}$\\
               $y=1+t=\dfrac{1}{3}$\\
               $z=4+2t=~\dfrac{8}{3}$\\
          \end{enumerate}
          \item
          D'après les questions précédentes, la droite $(AI)$ est la perpendiculaire au plan $(BCD)$ passant par $A$.
          \par
          Les coordonnées du vecteur $\overrightarrow{AI}$ sont $\begin{pmatrix}-4/3\\-2/3\\-4/3\end{pmatrix}$.
          \par
          La hauteur du tétraèdre $ABCD$ associée à la base $BCD$ est donc~:
          \par
          $AI=\sqrt{\left(-\dfrac{4}{3} \right)^2+\left(-\dfrac{2}{3} \right)^2+\left(-\dfrac{4}{3} \right)^2}=2$cm.
          \par
          Le volume du tétraèdre $ABCD$ est alors~:
          \par
          $\mathscr{V}=\dfrac{1}{3} \times \mathscr{A} \times  AI =\dfrac{1}{3} \times 12 \times 2=8$cm$^3$.
     \end{enumerate}
\end{corrige}
\par

\end{document}
µ
\documentclass[a4paper]{article}

%================================================================================================================================
%
% Packages
%
%================================================================================================================================

\usepackage[T1]{fontenc} 	% pour caractères accentués
\usepackage[utf8]{inputenc}  % encodage utf8
\usepackage[french]{babel}	% langue : français
\usepackage{fourier}			% caractères plus lisibles
\usepackage[dvipsnames]{xcolor} % couleurs
\usepackage{fancyhdr}		% réglage header footer
\usepackage{needspace}		% empêcher sauts de page mal placés
\usepackage{graphicx}		% pour inclure des graphiques
\usepackage{enumitem,cprotect}		% personnalise les listes d'items (nécessaire pour ol, al ...)
\usepackage{hyperref}		% Liens hypertexte
\usepackage{pstricks,pst-all,pst-node,pstricks-add,pst-math,pst-plot,pst-tree,pst-eucl} % pstricks
\usepackage[a4paper,includeheadfoot,top=2cm,left=3cm, bottom=2cm,right=3cm]{geometry} % marges etc.
\usepackage{comment}			% commentaires multilignes
\usepackage{amsmath,environ} % maths (matrices, etc.)
\usepackage{amssymb,makeidx}
\usepackage{bm}				% bold maths
\usepackage{tabularx}		% tableaux
\usepackage{colortbl}		% tableaux en couleur
\usepackage{fontawesome}		% Fontawesome
\usepackage{environ}			% environment with command
\usepackage{fp}				% calculs pour ps-tricks
\usepackage{multido}			% pour ps tricks
\usepackage[np]{numprint}	% formattage nombre
\usepackage{tikz,tkz-tab} 			% package principal TikZ
\usepackage{pgfplots}   % axes
\usepackage{mathrsfs}    % cursives
\usepackage{calc}			% calcul taille boites
\usepackage[scaled=0.875]{helvet} % font sans serif
\usepackage{svg} % svg
\usepackage{scrextend} % local margin
\usepackage{scratch} %scratch
\usepackage{multicol} % colonnes
%\usepackage{infix-RPN,pst-func} % formule en notation polanaise inversée
\usepackage{listings}

%================================================================================================================================
%
% Réglages de base
%
%================================================================================================================================

\lstset{
language=Python,   % R code
literate=
{á}{{\'a}}1
{à}{{\`a}}1
{ã}{{\~a}}1
{é}{{\'e}}1
{è}{{\`e}}1
{ê}{{\^e}}1
{í}{{\'i}}1
{ó}{{\'o}}1
{õ}{{\~o}}1
{ú}{{\'u}}1
{ü}{{\"u}}1
{ç}{{\c{c}}}1
{~}{{ }}1
}


\definecolor{codegreen}{rgb}{0,0.6,0}
\definecolor{codegray}{rgb}{0.5,0.5,0.5}
\definecolor{codepurple}{rgb}{0.58,0,0.82}
\definecolor{backcolour}{rgb}{0.95,0.95,0.92}

\lstdefinestyle{mystyle}{
    backgroundcolor=\color{backcolour},   
    commentstyle=\color{codegreen},
    keywordstyle=\color{magenta},
    numberstyle=\tiny\color{codegray},
    stringstyle=\color{codepurple},
    basicstyle=\ttfamily\footnotesize,
    breakatwhitespace=false,         
    breaklines=true,                 
    captionpos=b,                    
    keepspaces=true,                 
    numbers=left,                    
xleftmargin=2em,
framexleftmargin=2em,            
    showspaces=false,                
    showstringspaces=false,
    showtabs=false,                  
    tabsize=2,
    upquote=true
}

\lstset{style=mystyle}


\lstset{style=mystyle}
\newcommand{\imgdir}{C:/laragon/www/newmc/assets/imgsvg/}
\newcommand{\imgsvgdir}{C:/laragon/www/newmc/assets/imgsvg/}

\definecolor{mcgris}{RGB}{220, 220, 220}% ancien~; pour compatibilité
\definecolor{mcbleu}{RGB}{52, 152, 219}
\definecolor{mcvert}{RGB}{125, 194, 70}
\definecolor{mcmauve}{RGB}{154, 0, 215}
\definecolor{mcorange}{RGB}{255, 96, 0}
\definecolor{mcturquoise}{RGB}{0, 153, 153}
\definecolor{mcrouge}{RGB}{255, 0, 0}
\definecolor{mclightvert}{RGB}{205, 234, 190}

\definecolor{gris}{RGB}{220, 220, 220}
\definecolor{bleu}{RGB}{52, 152, 219}
\definecolor{vert}{RGB}{125, 194, 70}
\definecolor{mauve}{RGB}{154, 0, 215}
\definecolor{orange}{RGB}{255, 96, 0}
\definecolor{turquoise}{RGB}{0, 153, 153}
\definecolor{rouge}{RGB}{255, 0, 0}
\definecolor{lightvert}{RGB}{205, 234, 190}
\setitemize[0]{label=\color{lightvert}  $\bullet$}

\pagestyle{fancy}
\renewcommand{\headrulewidth}{0.2pt}
\fancyhead[L]{maths-cours.fr}
\fancyhead[R]{\thepage}
\renewcommand{\footrulewidth}{0.2pt}
\fancyfoot[C]{}

\newcolumntype{C}{>{\centering\arraybackslash}X}
\newcolumntype{s}{>{\hsize=.35\hsize\arraybackslash}X}

\setlength{\parindent}{0pt}		 
\setlength{\parskip}{3mm}
\setlength{\headheight}{1cm}

\def\ebook{ebook}
\def\book{book}
\def\web{web}
\def\type{web}

\newcommand{\vect}[1]{\overrightarrow{\,\mathstrut#1\,}}

\def\Oij{$\left(\text{O}~;~\vect{\imath},~\vect{\jmath}\right)$}
\def\Oijk{$\left(\text{O}~;~\vect{\imath},~\vect{\jmath},~\vect{k}\right)$}
\def\Ouv{$\left(\text{O}~;~\vect{u},~\vect{v}\right)$}

\hypersetup{breaklinks=true, colorlinks = true, linkcolor = OliveGreen, urlcolor = OliveGreen, citecolor = OliveGreen, pdfauthor={Didier BONNEL - https://www.maths-cours.fr} } % supprime les bordures autour des liens

\renewcommand{\arg}[0]{\text{arg}}

\everymath{\displaystyle}

%================================================================================================================================
%
% Macros - Commandes
%
%================================================================================================================================

\newcommand\meta[2]{    			% Utilisé pour créer le post HTML.
	\def\titre{titre}
	\def\url{url}
	\def\arg{#1}
	\ifx\titre\arg
		\newcommand\maintitle{#2}
		\fancyhead[L]{#2}
		{\Large\sffamily \MakeUppercase{#2}}
		\vspace{1mm}\textcolor{mcvert}{\hrule}
	\fi 
	\ifx\url\arg
		\fancyfoot[L]{\href{https://www.maths-cours.fr#2}{\black \footnotesize{https://www.maths-cours.fr#2}}}
	\fi 
}


\newcommand\TitreC[1]{    		% Titre centré
     \needspace{3\baselineskip}
     \begin{center}\textbf{#1}\end{center}
}

\newcommand\newpar{    		% paragraphe
     \par
}

\newcommand\nosp {    		% commande vide (pas d'espace)
}
\newcommand{\id}[1]{} %ignore

\newcommand\boite[2]{				% Boite simple sans titre
	\vspace{5mm}
	\setlength{\fboxrule}{0.2mm}
	\setlength{\fboxsep}{5mm}	
	\fcolorbox{#1}{#1!3}{\makebox[\linewidth-2\fboxrule-2\fboxsep]{
  		\begin{minipage}[t]{\linewidth-2\fboxrule-4\fboxsep}\setlength{\parskip}{3mm}
  			 #2
  		\end{minipage}
	}}
	\vspace{5mm}
}

\newcommand\CBox[4]{				% Boites
	\vspace{5mm}
	\setlength{\fboxrule}{0.2mm}
	\setlength{\fboxsep}{5mm}
	
	\fcolorbox{#1}{#1!3}{\makebox[\linewidth-2\fboxrule-2\fboxsep]{
		\begin{minipage}[t]{1cm}\setlength{\parskip}{3mm}
	  		\textcolor{#1}{\LARGE{#2}}    
 	 	\end{minipage}  
  		\begin{minipage}[t]{\linewidth-2\fboxrule-4\fboxsep}\setlength{\parskip}{3mm}
			\raisebox{1.2mm}{\normalsize\sffamily{\textcolor{#1}{#3}}}						
  			 #4
  		\end{minipage}
	}}
	\vspace{5mm}
}

\newcommand\cadre[3]{				% Boites convertible html
	\par
	\vspace{2mm}
	\setlength{\fboxrule}{0.1mm}
	\setlength{\fboxsep}{5mm}
	\fcolorbox{#1}{white}{\makebox[\linewidth-2\fboxrule-2\fboxsep]{
  		\begin{minipage}[t]{\linewidth-2\fboxrule-4\fboxsep}\setlength{\parskip}{3mm}
			\raisebox{-2.5mm}{\sffamily \small{\textcolor{#1}{\MakeUppercase{#2}}}}		
			\par		
  			 #3
 	 		\end{minipage}
	}}
		\vspace{2mm}
	\par
}

\newcommand\bloc[3]{				% Boites convertible html sans bordure
     \needspace{2\baselineskip}
     {\sffamily \small{\textcolor{#1}{\MakeUppercase{#2}}}}    
		\par		
  			 #3
		\par
}

\newcommand\CHelp[1]{
     \CBox{Plum}{\faInfoCircle}{À RETENIR}{#1}
}

\newcommand\CUp[1]{
     \CBox{NavyBlue}{\faThumbsOUp}{EN PRATIQUE}{#1}
}

\newcommand\CInfo[1]{
     \CBox{Sepia}{\faArrowCircleRight}{REMARQUE}{#1}
}

\newcommand\CRedac[1]{
     \CBox{PineGreen}{\faEdit}{BIEN R\'EDIGER}{#1}
}

\newcommand\CError[1]{
     \CBox{Red}{\faExclamationTriangle}{ATTENTION}{#1}
}

\newcommand\TitreExo[2]{
\needspace{4\baselineskip}
 {\sffamily\large EXERCICE #1\ (\emph{#2 points})}
\vspace{5mm}
}

\newcommand\img[2]{
          \includegraphics[width=#2\paperwidth]{\imgdir#1}
}

\newcommand\imgsvg[2]{
       \begin{center}   \includegraphics[width=#2\paperwidth]{\imgsvgdir#1} \end{center}
}


\newcommand\Lien[2]{
     \href{#1}{#2 \tiny \faExternalLink}
}
\newcommand\mcLien[2]{
     \href{https~://www.maths-cours.fr/#1}{#2 \tiny \faExternalLink}
}

\newcommand{\euro}{\eurologo{}}

%================================================================================================================================
%
% Macros - Environement
%
%================================================================================================================================

\newenvironment{tex}{ %
}
{%
}

\newenvironment{indente}{ %
	\setlength\parindent{10mm}
}

{
	\setlength\parindent{0mm}
}

\newenvironment{corrige}{%
     \needspace{3\baselineskip}
     \medskip
     \textbf{\textsc{Corrigé}}
     \medskip
}
{
}

\newenvironment{extern}{%
     \begin{center}
     }
     {
     \end{center}
}

\NewEnviron{code}{%
	\par
     \boite{gray}{\texttt{%
     \BODY
     }}
     \par
}

\newenvironment{vbloc}{% boite sans cadre empeche saut de page
     \begin{minipage}[t]{\linewidth}
     }
     {
     \end{minipage}
}
\NewEnviron{h2}{%
    \needspace{3\baselineskip}
    \vspace{0.6cm}
	\noindent \MakeUppercase{\sffamily \large \BODY}
	\vspace{1mm}\textcolor{mcgris}{\hrule}\vspace{0.4cm}
	\par
}{}

\NewEnviron{h3}{%
    \needspace{3\baselineskip}
	\vspace{5mm}
	\textsc{\BODY}
	\par
}

\NewEnviron{margeneg}{ %
\begin{addmargin}[-1cm]{0cm}
\BODY
\end{addmargin}
}

\NewEnviron{html}{%
}

\begin{document}
\meta{url}{/exercices/probabilites-bac-s-pondichery-2018/}
\meta{pid}{7178}
\meta{titre}{Probabilités – Bac S Pondichéry 2018}
\meta{type}{exercice}
\begin{h2}Exercice 3 (5 points)\end{h2}
\textbf{Commun  à tous les candidats}
\medskip
Une entreprise conditionne du sucre blanc provenant de deux exploitations U et V en paquets
de 1 kg et de différentes qualités.
\smallskip
Le sucre extra fin est conditionné séparément dans des paquets portant le label \og  extra fin \fg.
\smallskip
\emph{Les parties A, B et C peuvent être traitées de façon indépendante.}
\smallskip
Dans tout l'exercice, les résultats seront arrondis, si nécessaire, au millième.
\begin{center}\begin{h3}Partie A \end{h3}\end{center}
Pour calibrer le sucre en fonction de la taille de ses cristaux, on le fait passer au travers d'une
série de trois tamis positionnés les uns au-dessus des autres et posés sur un récipient à fond
étanche.
Les ouvertures des mailles sont les suivantes~:
\begin{center}
     \begin{extern}%alt="tamis"
          \psset{unit=1cm}
          \begin{pspicture}(12,4)
               %\psgrid
               \uput[r](0.5,3){Tamis 1 : 0,8 mm} \psline[linewidth=1pt]{->}(4,1)(6.8,1)
               \uput[r](0.5,2){Tamis 2 : 0,5 mm} \psline[linewidth=1pt]{->}(4,2)(6.8,2)
               \uput[r](0.5,1){Tamis 3 : 0,2 mm} \psline[linewidth=1pt]{->}(4,3)(6.8,3)
               \uput[r](-0.95,0.2){Récipient à fond étanche}  \psline[linewidth=1pt]{->}(4,0.2)(7,0.2)
               \psline[linewidth=1pt](7,3.75)(7,0)(12,0)(12,3.75)
               \multido{\n=7.00+0.13}{39}{\psframe(\n,2.94)(\n,3)}
               \multido{\n=7.00+0.12}{42}{\psframe(\n,1.96)(\n,2.02)}
               \multido{\n=7.00+0.1}{50}{\psframe(\n,0.97)(\n,1.03)}
          \end{pspicture}
     \end{extern}
\end{center}
Les cristaux de sucre dont la taille est inférieure à $0,2$ mm se trouvent dans le récipient à fond
étanche à la fin du calibrage. Ils seront conditionnés dans des paquets portant le label \og  sucre
extra fin \fg.
\medskip
\begin{enumerate}
     \item On prélève au hasard un cristal de sucre de l'exploitation U. La taille de ce cristal,
     exprimée en millimètre, est modélisée par la variable aléatoire $X_{\text{ U}}$ qui suit la loi normale
     de moyenne $\mu_{\text{ U}} = 0,58$~mm et d'écart type $\sigma_{\text{ U}} = 0,21$~mm.
     \begin{enumerate}[label=\alph*.]
          \item Calculer les probabilités des événements suivants~: $X_{\text{ U}} < 0,2 $ et $ 0,5 \leqslant X_{\text{ U}} < 0,8$.
          \item On fait passer 1~800 grammes de sucre provenant de l'exploitation U au travers de la
          série de tamis.
          \par
          Déduire de la question précédente une estimation de la masse de sucre récupérée dans
          le récipient à fond étanche et une estimation de la masse de sucre récupérée dans le
          tamis 2.
     \end{enumerate}
     \item On prélève au hasard un cristal de sucre de l'exploitation V. La taille de ce cristal,
     exprimée en millimètre, est modélisée par la variable aléatoire $X_{\text{V}}$ qui suit la loi normale
     de moyenne $\mu_{\text{V}} = 0,65$ mm et d'écart type $\sigma_{\text{V}}$ à déterminer.
     \par
     Lors du calibrage d'une grande quantité de cristaux de sucre provenant de l'exploitation V,
     on constate que 40\,\% de ces cristaux se retrouvent dans le tamis 2.
     \par
     Quelle est la valeur de l'écart type $\sigma_{\text{V}}$ de la variable aléatoire $X_{\text{V}}$~?
\end{enumerate}
\begin{center}\begin{h3}Partie B \end{h3}\end{center}
Dans cette partie, on admet que 3\,\% du sucre provenant de l'exploitation U est extra fin et que
5\,\% du sucre provenant de l'exploitation V est extra fin.
\par
On prélève au hasard un paquet de sucre dans la production de l'entreprise et, dans un souci
de traçabilité, on s'intéresse à la provenance de ce paquet.
\par
On considère les événements suivants~:
\begin{indent}
     \begin{itemize}
          \item $U$~: \og  Le paquet contient du sucre provenant de l'exploitation U \fg{}~;
          \item $V$~: \og Le paquet contient du sucre provenant de l'exploitation V \fg{}~;
          \item $E$~: \og Le paquet porte le label "extra fin" \fg{}.
     \end{itemize}
\end{indent}
\medskip
\begin{enumerate}
     \item Dans cette question, on admet que l'entreprise fabrique 30\,\% de ses paquets avec du sucre
     provenant de l'exploitation U et les autres avec du sucre provenant de l'exploitation V,
     sans mélanger les sucres des deux exploitations.
     \begin{enumerate}[label=\alph*.]
          \item Quelle est la probabilité que le paquet prélevé porte le label \og extra fin \fg{}~?
          \item Sachant qu'un paquet porte le label \og extra fin \fg, quelle est la probabilité que le sucre
          qu'il contient provienne de l'exploitation U~?
     \end{enumerate}
     \item L'entreprise souhaite modifier son approvisionnement auprès des deux exploitations afin
     que parmi les paquets portant le label « extra fin », 30\,\% d'entre eux contiennent du sucre
     provenant de l'exploitation U.
     \par
     Comment doit-elle s'approvisionner auprès des exploitations U et V~?
     \par
     \emph{Toute trace de recherche sera valorisée dans cette question}.
\end{enumerate}
\begin{center}\begin{h3}Partie C \end{h3}\end{center}
\begin{enumerate}
     \item L'entreprise annonce que 30\,\% des paquets de sucre portant le label « extra fin » qu'elle
     conditionne contiennent du sucre provenant de l'exploitation U.
     \par
     Avant de valider une commande, un acheteur veut vérifier cette proportion annoncée. Il
     prélève $150$ paquets pris au hasard dans la production de paquets labellisés \og extra fin \fg{} de
     l'entreprise. Parmi ces paquets, $30$ contiennent du sucre provenant de l'exploitation U.
     \par
     A-t-il des raisons de remettre en question l'annonce de l'entreprise~?
     \item  L'année suivante, l'entreprise déclare avoir modifié sa production. L'acheteur souhaite
     estimer la nouvelle proportion de paquets de sucre provenant de l'exploitation U parmi les
     paquets portant le label \og extra fin \fg.
     \par
     Il prélève 150 paquets pris au hasard dans la production de paquets labellisés \og extra fin \fg{} de l'entreprise. Parmi ces paquets 42\,\% contiennent du sucre provenant de l'exploitation U.
     \par
     Donner un intervalle de confiance, au niveau de confiance 95\,\%, de la nouvelle proportion
     de paquets labellisés \og extra fin \fg{} contenant du sucre provenant de l'exploitation U.
\end{enumerate}
\begin{corrige}
     \begin{center}\begin{h3}Partie A \end{h3}\end{center}
     \begin{enumerate}
          \item
          \begin{enumerate}[label=\alph*.]
               \item
               $X_{\text{ U}}$ suit la loi normale de moyenne $\mu_{\text{ U}} = 0,58$ et d'écart type $\sigma_{\text{ U}} = 0,21$.
               \par
               \`A la calculatrice, on trouve~:
               \par
               $p\left(X_{\text{ U}}<0,2 \right)\approx 0,035$  (au millième).
               \par
               $p\left(0,5 \leqslant  X_{\text{ U}} < 0,8 \right)\approx 0,501$  (au millième).
               \item
               $p\left(X_{\text{ U}}<0,2 \right)\approx 0,035$ donc 3,5\% des cristaux se retrouvent dans le récipient à fond étanche.
               \par
               Pour 1~800g de sucre, cela correspond à une masse d'environ $1~800 \times  0,035~=~63$ grammes (arrondie au gramme).
               \smallskip
               $p\left(0,5 \leqslant  X_{\text{ U}}<0,8 \right)\approx 0,501$ donc 50,1\% des cristaux se retrouvent dans le tamis 2 .
               \par
               Pour 1~800g de sucre, cela correspond à une masse d'environ $1~800 \times   0,501~=~902 $ grammes (arrondie au gramme).
          \end{enumerate}
          \item
          Posons  $Z=\dfrac{X_{\text{V}}-\mu_{\text{V}}}{\sigma_{\text{V}}} = \dfrac{X_{\text{V}}-0,65}{\sigma_{\text{V}}}$
          \par
          Puisque $X_{\text{V}}$ suit la loi normale de moyenne $\mu_{\text{ V}} = 0,58$ et d'écart type $\sigma_{\text{V}}$, $Z$ suit la loi normale centrée réduite.
          \par
          D'après l'énoncé, 40\,\% des cristaux se retrouvent dans le tamis 2 donc~:
          \par
          $p\left(0,5 \leqslant X_{\text{ V}}<0,8 \right)= 0,4$.
          \par
          Or~:
          \par
          $0,5 \leqslant X_{\text{ V}}<0,8~ \Leftrightarrow ~0,5 -0,65\leqslant X_{\text{ V}}-0,65<0,8-0,65$\\
          $\phantom{0,5 \leqslant X_{\text{ V}}<0,8}~ \Leftrightarrow  ~ -0,15\leqslant X_{\text{ V}}-0,65<0,15$\\
          $\phantom{0,5 \leqslant X_{\text{ V}}<0,8}~ \Leftrightarrow  ~ -\dfrac{0,15}{\sigma_{\text{V}}} \leqslant Z<\dfrac{0,15}{\sigma_{\text{V}}} $
          \par
          Par conséquent~:
          \par
          $p\left(-\dfrac{0,15}{\sigma_{\text{V}}} \leqslant Z<\dfrac{0,15}{\sigma_{\text{V}}} \right) =0,4$
          \par
          où $Z$ suit la loi normale centrée réduite.
          \par
          \`A  la calculatrice on trouve $p\left(-0,524 \leqslant Z<0,524\right) =0,4$.
          \par
          Donc $\dfrac{0,15}{\sigma_{\text{V}}}\approx 0,524$  et $\sigma_{\text{V}} \approx \dfrac{0,15}{0,524}\approx 0,286$.
     \end{enumerate}
     \begin{center}\begin{h3}Partie B \end{h3}\end{center}
     \begin{enumerate}
          \item
          \begin{enumerate}[label=\alph*.]
               \item
               On cherche à calculer $p(E)$.
               \par
               On peut modéliser la situation à l'aide de l'arbre ci-dessous~:
               \begin{center}
                    \begin{extern}%width="350" alt="arbre pondéré probabilités"
                         % Racine à Gauche, développement vers la droite
                         \begin{tikzpicture}[xscale=1,yscale=1]
                              % Styles (MODIFIABLES)
                              \tikzstyle{fleche}=[-,>=latex,thick]
                              \tikzstyle{noeud}=[fill=white]
                              \tikzstyle{feuille}=[fill=white]
                              \tikzstyle{etiquette}=[midway,fill=white]
                              % Dimensions (MODIFIABLES)
                              \def\DistanceInterNiveaux{3}
                              \def\DistanceInterFeuilles{2}
                              % Dimensions calculées (NON MODIFIABLES)
                              \def\NiveauA{(0)*\DistanceInterNiveaux}
                              \def\NiveauB{(1.5)*\DistanceInterNiveaux}
                              \def\NiveauC{(2.5)*\DistanceInterNiveaux}
                              \def\InterFeuilles{(-1)*\DistanceInterFeuilles}
                              % Noeuds (MODIFIABLES : Styles et Coefficients d'InterFeuilles)
                              \node[noeud] (R) at ({\NiveauA},{(1.5)*\InterFeuilles}) {$ $};
                              \node[noeud] (Ra) at ({\NiveauB},{(0.5)*\InterFeuilles}) {$U$};
                              \node[feuille] (Raa) at ({\NiveauC},{(0)*\InterFeuilles}) {$E$};
                              \node[feuille] (Rab) at ({\NiveauC},{(1)*\InterFeuilles}) {$\overline{E}$};
                              \node[noeud] (Rb) at ({\NiveauB},{(2.5)*\InterFeuilles}) {$V$};
                              \node[feuille] (Rba) at ({\NiveauC},{(2)*\InterFeuilles}) {$E$};
                              \node[feuille] (Rbb) at ({\NiveauC},{(3)*\InterFeuilles}) {$\overline{E}$};
                              % Arcs (MODIFIABLES : Styles)
                              \draw[fleche] (R)--(Ra) node[etiquette] {$0,3$};
                              \draw[fleche] (Ra)--(Raa) node[etiquette] {$0,03$};
                              \draw[fleche] (Ra)--(Rab) node[etiquette] {$0,97$};
                              \draw[fleche] (R)--(Rb) node[etiquette] {$0,7$};
                              \draw[fleche] (Rb)--(Rba) node[etiquette] {$0,05$};
                              \draw[fleche] (Rb)--(Rbb) node[etiquette] {$0,95$};
                         \end{tikzpicture}
                    \end{extern}
               \end{center}
               %~:-+-+-+-+- Fin
               \par
               Le sucre provenant soit de l'exploitation U, soit de l'exploitation V, les événements $U$ et $V$ forment une partition de l'univers.
               \par
               D'après la formule des probabilités totales~:
               \par
               $p(E)=p(U) \times  p_U(E) +p(V) \times  p_V(E)$\\
               $\phantom{P(E)}=0,3 \times 0,03 + 0,7 \times 0,05 $ $= 0,009 +0,035=0,044$
               \item
               La probabilité cherchée est $P_E(U)$.
               \par
               D'après la formule des probabilités conditionnelles~:
               \par
               $P_E(U)=\dfrac{P(E\cap U)}{P(E)}$ $=\dfrac{0,009}{0,044}=\dfrac{9}{44}\approx 0,205$
          \end{enumerate}
          \item
          Posons  $p(U)=x$. Alors $p(V)=1-x$.
          \par
          L'arbre obtenu est alors~:
          \par
          %~:-+-+-+- Engendré par~: http~://math.et.info.free.fr/TikZ/Arbre/
          \begin{center}
               \begin{extern}%width="350" alt="arbre pondéré Pondichery 2018"
                    % Racine à Gauche, développement vers la droite
                    \begin{tikzpicture}[xscale=1,yscale=1]
                         % Styles (MODIFIABLES)
                         \tikzstyle{fleche}=[-,>=latex,thick]
                         \tikzstyle{noeud}=[fill=white]
                         \tikzstyle{feuille}=[fill=white]
                         \tikzstyle{etiquette}=[midway,fill=white]
                         % Dimensions (MODIFIABLES)
                         \def\DistanceInterNiveaux{3}
                         \def\DistanceInterFeuilles{2}
                         % Dimensions calculées (NON MODIFIABLES)
                         \def\NiveauA{(0)*\DistanceInterNiveaux}
                         \def\NiveauB{(1.5)*\DistanceInterNiveaux}
                         \def\NiveauC{(2.5)*\DistanceInterNiveaux}
                         \def\InterFeuilles{(-1)*\DistanceInterFeuilles}
                         % Noeuds (MODIFIABLES : Styles et Coefficients d'InterFeuilles)
                         \node[noeud] (R) at ({\NiveauA},{(1.5)*\InterFeuilles}) {$ $};
                         \node[noeud] (Ra) at ({\NiveauB},{(0.5)*\InterFeuilles}) {$U$};
                         \node[feuille] (Raa) at ({\NiveauC},{(0)*\InterFeuilles}) {$E$};
                         \node[feuille] (Rab) at ({\NiveauC},{(1)*\InterFeuilles}) {$\overline{E}$};
                         \node[noeud] (Rb) at ({\NiveauB},{(2.5)*\InterFeuilles}) {$V$};
                         \node[feuille] (Rba) at ({\NiveauC},{(2)*\InterFeuilles}) {$E$};
                         \node[feuille] (Rbb) at ({\NiveauC},{(3)*\InterFeuilles}) {$\overline{E}$};
                         % Arcs (MODIFIABLES : Styles)
                         \draw[fleche] (R)--(Ra) node[etiquette] {$x$};
                         \draw[fleche] (Ra)--(Raa) node[etiquette] {$0,03$};
                         \draw[fleche] (Ra)--(Rab) node[etiquette] {$0,97$};
                         \draw[fleche] (R)--(Rb) node[etiquette] {$1-x$};
                         \draw[fleche] (Rb)--(Rba) node[etiquette] {$0,05$};
                         \draw[fleche] (Rb)--(Rbb) node[etiquette] {$0,95$};
                    \end{tikzpicture}
               \end{extern}
          \end{center}
          %~:-+-+-+-+- Fin
          \par
          On a donc~:
          \par
          $p_E(U)=\dfrac{p(E\cap U)}{p(E)} = \dfrac{x\times p_U(E)}{x\times p_U(E) +(1-x)\times p_V(E)}$\\
          $\phantom{p_E(U)}=\dfrac{0,03x}{0,03x +0,05(1-x)}=\dfrac{0,03x}{0,05 -0,02x}$
          \medskip
          Par conséquent~:
          \par
          $p_E(U)=0,3 \Leftrightarrow \dfrac{0,03x}{0,05 -0,02x}=0,3$\\
          $\phantom{p_E(U)=0,3} \Leftrightarrow 0,03x=0,015-0,006x$\\
          $\phantom{p_E(U)=0,3} \Leftrightarrow 0,036x=0,015$\\
          $\phantom{p_E(U)=0,3} \Leftrightarrow  x=\dfrac{0,015}{0,036}=\dfrac{5}{12}\approx 0,417 $
          \par
          Pour que, parmi les paquets portant le label « extra fin », 30\,\%  contiennent du sucre provenant de l'exploitation U, l'entreprise doit s'approvisionner à  41,7\% auprès de l'exploitation U.
     \end{enumerate}
     \begin{center}\begin{h3}Partie C \end{h3}\end{center}
     \begin{enumerate}
          \item
          D'après l'entreprise, la proportion de paquets de sucre portant le label «extra fin» provenant de l’exploitation U est $p = 0,3$.
          \par
          La taille  de l'échantillon est $n=150$.
          \par
          On vérifie que~:
          \begin{itbullet}
               \item $n=150 \geqslant 30$~;
               \item $np=150 \times 0,3=45\geqslant 5$~;
               \item $n(1-p)=150 \times 0,7=105\geqslant 5$.
          \end{itbullet}
          Les conditions de validité étant remplies, l'intervalle de fluctuation asymptotique au seuil de $95\%$ est~:
          \par
          \[ I=\left[p-1,96\dfrac{\sqrt{p(1-p)}}{\sqrt{n}}~;~p+1,96\dfrac{\sqrt{p(1-p)}}{\sqrt{n}}\right]. \]
          \medskip
          $p-1,96\dfrac{\sqrt{p(1-p)}}{\sqrt{n}}=0,3-1,96\dfrac{\sqrt{0,3(1-0,3)}}{\sqrt{150}}$\\
          $\phantom{p-1,96\dfrac{\sqrt{p(1-p)}}{\sqrt{n}}}  \approx 0,226$ (arrondi au millième par défaut).
          \medskip
          $p+1,96\dfrac{\sqrt{p(1-p)}}{\sqrt{n}}=0,3+1,96\dfrac{\sqrt{0,3(1-0,3)}}{\sqrt{150}}$\\
          $\phantom{p+1,96\dfrac{\sqrt{p(1-p)}}{\sqrt{n}}} \approx 0,374 $ (arrondi au millième par excès).
          \medskip
          L'intervalle de fluctuation asymptotique au seuil de $95\%$ de la proportion de paquets  provenant de l’exploitation U est donc~:
          \[ I=[0,226~;~0,374]. \]
          \par
          La fréquence observée des paquets  provenant de l’exploitation U  est $f=\dfrac{30}{150}=0,2$.
          \par
          Comme $0,2 \notin I$, on peut donc rejeter l'affirmation de l'entreprise avec un risque d'erreur inférieur à 5\%.
          \item
          La taille  de l'échantillon est $n=150$.
          \par
          La fréquence observée de paquets provenant de l'exploitation U est $f=0,42$
          \par
          On vérifie que~:
          \begin{itbullet}
               \item $n=150 \geqslant 30$~;
               \item $nf=150 \times 0,42=63\geqslant 5$~;
               \item $n(1-f)=150 \times 0,58=87\geqslant 5$.
          \end{itbullet}
          Les conditions de validité étant remplies, un intervalle de confiance, au niveau de confiance 95\,\% est~:
          \[ I=\left[ f-\dfrac{1}{\sqrt{n}}~; f+\dfrac{1}{\sqrt{n}}  \right]. \]
          \par
          $f-\dfrac{1}{\sqrt{n}} \approx 0,338$ (arrondi au millième par défaut).
          \par
          $f+\dfrac{1}{\sqrt{n}} \approx 0,502$ (arrondi au millième par excès).
          \par
          Un intervalle de confiance, au niveau de confiance 95\,\% est donc~:
          \[ I=[0,338~;~0,502] .\]
     \end{enumerate}
\end{corrige}

\end{document}
µ
\documentclass[a4paper]{article}

%================================================================================================================================
%
% Packages
%
%================================================================================================================================

\usepackage[T1]{fontenc} 	% pour caractères accentués
\usepackage[utf8]{inputenc}  % encodage utf8
\usepackage[french]{babel}	% langue : français
\usepackage{fourier}			% caractères plus lisibles
\usepackage[dvipsnames]{xcolor} % couleurs
\usepackage{fancyhdr}		% réglage header footer
\usepackage{needspace}		% empêcher sauts de page mal placés
\usepackage{graphicx}		% pour inclure des graphiques
\usepackage{enumitem,cprotect}		% personnalise les listes d'items (nécessaire pour ol, al ...)
\usepackage{hyperref}		% Liens hypertexte
\usepackage{pstricks,pst-all,pst-node,pstricks-add,pst-math,pst-plot,pst-tree,pst-eucl} % pstricks
\usepackage[a4paper,includeheadfoot,top=2cm,left=3cm, bottom=2cm,right=3cm]{geometry} % marges etc.
\usepackage{comment}			% commentaires multilignes
\usepackage{amsmath,environ} % maths (matrices, etc.)
\usepackage{amssymb,makeidx}
\usepackage{bm}				% bold maths
\usepackage{tabularx}		% tableaux
\usepackage{colortbl}		% tableaux en couleur
\usepackage{fontawesome}		% Fontawesome
\usepackage{environ}			% environment with command
\usepackage{fp}				% calculs pour ps-tricks
\usepackage{multido}			% pour ps tricks
\usepackage[np]{numprint}	% formattage nombre
\usepackage{tikz,tkz-tab} 			% package principal TikZ
\usepackage{pgfplots}   % axes
\usepackage{mathrsfs}    % cursives
\usepackage{calc}			% calcul taille boites
\usepackage[scaled=0.875]{helvet} % font sans serif
\usepackage{svg} % svg
\usepackage{scrextend} % local margin
\usepackage{scratch} %scratch
\usepackage{multicol} % colonnes
%\usepackage{infix-RPN,pst-func} % formule en notation polanaise inversée
\usepackage{listings}

%================================================================================================================================
%
% Réglages de base
%
%================================================================================================================================

\lstset{
language=Python,   % R code
literate=
{á}{{\'a}}1
{à}{{\`a}}1
{ã}{{\~a}}1
{é}{{\'e}}1
{è}{{\`e}}1
{ê}{{\^e}}1
{í}{{\'i}}1
{ó}{{\'o}}1
{õ}{{\~o}}1
{ú}{{\'u}}1
{ü}{{\"u}}1
{ç}{{\c{c}}}1
{~}{{ }}1
}


\definecolor{codegreen}{rgb}{0,0.6,0}
\definecolor{codegray}{rgb}{0.5,0.5,0.5}
\definecolor{codepurple}{rgb}{0.58,0,0.82}
\definecolor{backcolour}{rgb}{0.95,0.95,0.92}

\lstdefinestyle{mystyle}{
    backgroundcolor=\color{backcolour},   
    commentstyle=\color{codegreen},
    keywordstyle=\color{magenta},
    numberstyle=\tiny\color{codegray},
    stringstyle=\color{codepurple},
    basicstyle=\ttfamily\footnotesize,
    breakatwhitespace=false,         
    breaklines=true,                 
    captionpos=b,                    
    keepspaces=true,                 
    numbers=left,                    
xleftmargin=2em,
framexleftmargin=2em,            
    showspaces=false,                
    showstringspaces=false,
    showtabs=false,                  
    tabsize=2,
    upquote=true
}

\lstset{style=mystyle}


\lstset{style=mystyle}
\newcommand{\imgdir}{C:/laragon/www/newmc/assets/imgsvg/}
\newcommand{\imgsvgdir}{C:/laragon/www/newmc/assets/imgsvg/}

\definecolor{mcgris}{RGB}{220, 220, 220}% ancien~; pour compatibilité
\definecolor{mcbleu}{RGB}{52, 152, 219}
\definecolor{mcvert}{RGB}{125, 194, 70}
\definecolor{mcmauve}{RGB}{154, 0, 215}
\definecolor{mcorange}{RGB}{255, 96, 0}
\definecolor{mcturquoise}{RGB}{0, 153, 153}
\definecolor{mcrouge}{RGB}{255, 0, 0}
\definecolor{mclightvert}{RGB}{205, 234, 190}

\definecolor{gris}{RGB}{220, 220, 220}
\definecolor{bleu}{RGB}{52, 152, 219}
\definecolor{vert}{RGB}{125, 194, 70}
\definecolor{mauve}{RGB}{154, 0, 215}
\definecolor{orange}{RGB}{255, 96, 0}
\definecolor{turquoise}{RGB}{0, 153, 153}
\definecolor{rouge}{RGB}{255, 0, 0}
\definecolor{lightvert}{RGB}{205, 234, 190}
\setitemize[0]{label=\color{lightvert}  $\bullet$}

\pagestyle{fancy}
\renewcommand{\headrulewidth}{0.2pt}
\fancyhead[L]{maths-cours.fr}
\fancyhead[R]{\thepage}
\renewcommand{\footrulewidth}{0.2pt}
\fancyfoot[C]{}

\newcolumntype{C}{>{\centering\arraybackslash}X}
\newcolumntype{s}{>{\hsize=.35\hsize\arraybackslash}X}

\setlength{\parindent}{0pt}		 
\setlength{\parskip}{3mm}
\setlength{\headheight}{1cm}

\def\ebook{ebook}
\def\book{book}
\def\web{web}
\def\type{web}

\newcommand{\vect}[1]{\overrightarrow{\,\mathstrut#1\,}}

\def\Oij{$\left(\text{O}~;~\vect{\imath},~\vect{\jmath}\right)$}
\def\Oijk{$\left(\text{O}~;~\vect{\imath},~\vect{\jmath},~\vect{k}\right)$}
\def\Ouv{$\left(\text{O}~;~\vect{u},~\vect{v}\right)$}

\hypersetup{breaklinks=true, colorlinks = true, linkcolor = OliveGreen, urlcolor = OliveGreen, citecolor = OliveGreen, pdfauthor={Didier BONNEL - https://www.maths-cours.fr} } % supprime les bordures autour des liens

\renewcommand{\arg}[0]{\text{arg}}

\everymath{\displaystyle}

%================================================================================================================================
%
% Macros - Commandes
%
%================================================================================================================================

\newcommand\meta[2]{    			% Utilisé pour créer le post HTML.
	\def\titre{titre}
	\def\url{url}
	\def\arg{#1}
	\ifx\titre\arg
		\newcommand\maintitle{#2}
		\fancyhead[L]{#2}
		{\Large\sffamily \MakeUppercase{#2}}
		\vspace{1mm}\textcolor{mcvert}{\hrule}
	\fi 
	\ifx\url\arg
		\fancyfoot[L]{\href{https://www.maths-cours.fr#2}{\black \footnotesize{https://www.maths-cours.fr#2}}}
	\fi 
}


\newcommand\TitreC[1]{    		% Titre centré
     \needspace{3\baselineskip}
     \begin{center}\textbf{#1}\end{center}
}

\newcommand\newpar{    		% paragraphe
     \par
}

\newcommand\nosp {    		% commande vide (pas d'espace)
}
\newcommand{\id}[1]{} %ignore

\newcommand\boite[2]{				% Boite simple sans titre
	\vspace{5mm}
	\setlength{\fboxrule}{0.2mm}
	\setlength{\fboxsep}{5mm}	
	\fcolorbox{#1}{#1!3}{\makebox[\linewidth-2\fboxrule-2\fboxsep]{
  		\begin{minipage}[t]{\linewidth-2\fboxrule-4\fboxsep}\setlength{\parskip}{3mm}
  			 #2
  		\end{minipage}
	}}
	\vspace{5mm}
}

\newcommand\CBox[4]{				% Boites
	\vspace{5mm}
	\setlength{\fboxrule}{0.2mm}
	\setlength{\fboxsep}{5mm}
	
	\fcolorbox{#1}{#1!3}{\makebox[\linewidth-2\fboxrule-2\fboxsep]{
		\begin{minipage}[t]{1cm}\setlength{\parskip}{3mm}
	  		\textcolor{#1}{\LARGE{#2}}    
 	 	\end{minipage}  
  		\begin{minipage}[t]{\linewidth-2\fboxrule-4\fboxsep}\setlength{\parskip}{3mm}
			\raisebox{1.2mm}{\normalsize\sffamily{\textcolor{#1}{#3}}}						
  			 #4
  		\end{minipage}
	}}
	\vspace{5mm}
}

\newcommand\cadre[3]{				% Boites convertible html
	\par
	\vspace{2mm}
	\setlength{\fboxrule}{0.1mm}
	\setlength{\fboxsep}{5mm}
	\fcolorbox{#1}{white}{\makebox[\linewidth-2\fboxrule-2\fboxsep]{
  		\begin{minipage}[t]{\linewidth-2\fboxrule-4\fboxsep}\setlength{\parskip}{3mm}
			\raisebox{-2.5mm}{\sffamily \small{\textcolor{#1}{\MakeUppercase{#2}}}}		
			\par		
  			 #3
 	 		\end{minipage}
	}}
		\vspace{2mm}
	\par
}

\newcommand\bloc[3]{				% Boites convertible html sans bordure
     \needspace{2\baselineskip}
     {\sffamily \small{\textcolor{#1}{\MakeUppercase{#2}}}}    
		\par		
  			 #3
		\par
}

\newcommand\CHelp[1]{
     \CBox{Plum}{\faInfoCircle}{À RETENIR}{#1}
}

\newcommand\CUp[1]{
     \CBox{NavyBlue}{\faThumbsOUp}{EN PRATIQUE}{#1}
}

\newcommand\CInfo[1]{
     \CBox{Sepia}{\faArrowCircleRight}{REMARQUE}{#1}
}

\newcommand\CRedac[1]{
     \CBox{PineGreen}{\faEdit}{BIEN R\'EDIGER}{#1}
}

\newcommand\CError[1]{
     \CBox{Red}{\faExclamationTriangle}{ATTENTION}{#1}
}

\newcommand\TitreExo[2]{
\needspace{4\baselineskip}
 {\sffamily\large EXERCICE #1\ (\emph{#2 points})}
\vspace{5mm}
}

\newcommand\img[2]{
          \includegraphics[width=#2\paperwidth]{\imgdir#1}
}

\newcommand\imgsvg[2]{
       \begin{center}   \includegraphics[width=#2\paperwidth]{\imgsvgdir#1} \end{center}
}


\newcommand\Lien[2]{
     \href{#1}{#2 \tiny \faExternalLink}
}
\newcommand\mcLien[2]{
     \href{https~://www.maths-cours.fr/#1}{#2 \tiny \faExternalLink}
}

\newcommand{\euro}{\eurologo{}}

%================================================================================================================================
%
% Macros - Environement
%
%================================================================================================================================

\newenvironment{tex}{ %
}
{%
}

\newenvironment{indente}{ %
	\setlength\parindent{10mm}
}

{
	\setlength\parindent{0mm}
}

\newenvironment{corrige}{%
     \needspace{3\baselineskip}
     \medskip
     \textbf{\textsc{Corrigé}}
     \medskip
}
{
}

\newenvironment{extern}{%
     \begin{center}
     }
     {
     \end{center}
}

\NewEnviron{code}{%
	\par
     \boite{gray}{\texttt{%
     \BODY
     }}
     \par
}

\newenvironment{vbloc}{% boite sans cadre empeche saut de page
     \begin{minipage}[t]{\linewidth}
     }
     {
     \end{minipage}
}
\NewEnviron{h2}{%
    \needspace{3\baselineskip}
    \vspace{0.6cm}
	\noindent \MakeUppercase{\sffamily \large \BODY}
	\vspace{1mm}\textcolor{mcgris}{\hrule}\vspace{0.4cm}
	\par
}{}

\NewEnviron{h3}{%
    \needspace{3\baselineskip}
	\vspace{5mm}
	\textsc{\BODY}
	\par
}

\NewEnviron{margeneg}{ %
\begin{addmargin}[-1cm]{0cm}
\BODY
\end{addmargin}
}

\NewEnviron{html}{%
}

\begin{document}
\meta{url}{/exercices/chiffrement-bac-s-pondichery-2018-spe/}
\meta{pid}{7213}
\meta{titre}{Cryptage – Bac S Pondichéry 2018 (spé)}
\meta{type}{exercice}
\begin{h2}Exercice 4 (5 points)\end{h2}
\textbf{Candidats ayant suivi l'enseignement de spécialité}
\medskip
À toute lettre de l'alphabet on associe un nombre entier $x$ compris entre 0 et 25 comme
indiqué dans le tableau ci-dessous~:
\begin{center}
     \begin{tabularx}{\linewidth}{|c|*{13}{>{\centering \arraybackslash}X|}}\hline %class="compact"
          Lettre 	&A &B &C &D &E &F &G &H &I &J &K 	&L 	&M\\ \hline
          $x$ 	&0 &1 &2 &3 &4 &5 &6 &7 &8 &9 &10 	&11 &12\\ \hline\hline
          Lettre 	&N &O &P &Q &R &S &T &U &V &W &X 	&Y 	&Z\\ \hline
          $x$ 	&13&14&15&16&17&18&19&20&21&22&23 	&24 &25\\ \hline
     \end{tabularx}
\end{center}
\medskip
Le \og chiffre de RABIN \fg{} est un dispositif de cryptage asymétrique inventé en 1979 par
l'informaticien Michael Rabin.
\smallskip
Alice veut communiquer de manière sécurisée en utilisant ce cryptosystème. Elle choisit deux
nombres premiers distincts $p$ et $q$. Ce couple de nombres est sa clé privée qu'elle garde
secrète.
\par
Elle calcule ensuite $n = p \times q$ et elle choisit un nombre entier naturel $B$ tel que $0 \leqslant B \leqslant n -1$.
\smallskip
Si Bob veut envoyer un message secret à Alice, il le code lettre par lettre.
\par
Le codage d'une lettre représentée par le nombre entier $x$ est le nombre $y$ tel que~:
\par
\[y \equiv  x(x + B)\:\: [n] \:\text{ avec }\: 0 \leqslant y \leqslant n.\]
\smallskip
Dans tout l'exercice on prend $p = 3,\: q = 11$ donc $n = p \times q = 33$ et $B = 13$.
\begin{center}\begin{h3}Partie A~: Cryptage \end{h3}\end{center}
Bob veut envoyer le mot \og  NO \fg{} à Alice.
\begin{enumerate}
     \item Montrer que Bob code la lettre \og N \fg{} avec le nombre 8.
     \item Déterminer le nombre qui code la lettre \og O \fg.
\end{enumerate}
\begin{center}\begin{h3}Partie B~: Décryptage \end{h3}\end{center}
Alice a reçu un message crypté qui commence par le nombre 3.
\par
Pour décoder ce premier nombre, elle doit déterminer le nombre entier $x$ tel que~:
\par
\[x(x + 13) \equiv  3 \:\: [33]\:  \text{ avec }\: 0 \leqslant  x < 26.\]
\medskip
\begin{enumerate}
     \item Montrer que $x(x + 13) \equiv 3\:\: [33]$ équivaut à $(x + 23)^2 \equiv 4\:\: [33].$
     \smallskip
     \item
     \begin{enumerate}[label=\alph*.]
          \item Montrer que si $(x + 23)^2 \equiv 4\:\: [33]$ alors le système d'équations
          \par
          \[\left\{\begin{array}{l c l}
                    (x + 23)^2 &\equiv &4 \:\: [3]\\
                    (x + 23)^2 &\equiv &4 \:\: [11]
          \end{array}\right.\]
          est vérifié.
          \item Réciproquement, montrer que si
          \par
          \[\left\{\begin{array}{l c l}
                    (x + 23)^2 &\equiv &4\:\: [3]\\
                    (x + 23)^2 &\equiv &4 \:\: [11]
          \end{array}\right.\]
          alors $(x + 23)^2 \equiv 4\:\: [33].$
          \item En déduire que~:
          \begin{center}
               $x(x + 13) \equiv 3\:\: [33] \Leftrightarrow $\nosp$ \left\{\begin{array}{l c l}
                         (x + 23)^2 &\equiv&1 \:\: [3]\\
                         (x + 23)^2 &\equiv& 4 \:\: [11]
               \end{array}\right.$
          \end{center}
     \end{enumerate}
     \item
     \begin{enumerate}[label=\alph*.]
          \item Déterminer les nombres entiers naturels $a$ tels que $0 \leqslant a < 3$ et $a^2 \equiv 1 \:\:  [3]$.
          \item Déterminer les nombres entiers naturels $b$ tels que $0 \leqslant b < 11$ et $b^2 \equiv 4\:\: [11]$.
     \end{enumerate}
     \item
     \begin{enumerate}[label=\alph*.]
          \item En déduire que $x(x + 13) \equiv 3 \quad[33]$ équivaut aux quatre systèmes suivants~:
          \medskip
          \begin{center}
               $\left\{\begin{array}{l c l}
                         x &\equiv&2\quad [3]\\
                         x&\equiv &8\quad[11]
               \end{array}\right. $\\
               ou\\
               $\left\{\begin{array}{l c l}
                         x &\equiv& 0\quad[3]\\
                         x &\equiv& 1 \quad[11]
               \end{array}\right.$\\
               ou\\
               $\left\{\begin{array}{l c l}
                         x  &\equiv& 2\quad[3]\\
                         x &\equiv&1 \quad[11]
               \end{array}\right.$\\
               ou\\
               $\left\{\begin{array}{l c l}
                         x &\equiv& 0\quad [3]\\
                         x &\equiv& 8 \quad [11]
               \end{array}\right.$
          \end{center}
          \item On admet que chacun de ces systèmes admet une unique solution entière $x$ telle que $0 \leqslant x < 33$.
          \par
          Déterminer, sans justification, chacune de ces solutions.
     \end{enumerate}
     \item Compléter l'algorithme ci-dessous pour qu'il affiche les quatre solutions trouvées dans la
     question précédente.
     \begin{center}
          \begin{extern}%width="450" alt="algorithme décodage Rabin"
               \begin{tabularx}{0.7\linewidth}{|X|}\hline
                    Pour ...... allant de ......à .......\\
                    \quad Si le reste de la division de ....... par ....... est égal à ....... alors\\
                    \qquad Afficher .......\\
                    \quad Fin Si\\
                    Fin Pour\\ \hline
               \end{tabularx}
          \end{extern}
     \end{center}
     \item Alice peut-elle connaître la première lettre du message envoyé par Bob~? \\
     \par
     Le \og chiffre de RABIN \fg{} est-il utilisable pour décoder un message lettre par lettre~?
\end{enumerate}

\end{document}
µ
\documentclass[a4paper]{article}

%================================================================================================================================
%
% Packages
%
%================================================================================================================================

\usepackage[T1]{fontenc} 	% pour caractères accentués
\usepackage[utf8]{inputenc}  % encodage utf8
\usepackage[french]{babel}	% langue : français
\usepackage{fourier}			% caractères plus lisibles
\usepackage[dvipsnames]{xcolor} % couleurs
\usepackage{fancyhdr}		% réglage header footer
\usepackage{needspace}		% empêcher sauts de page mal placés
\usepackage{graphicx}		% pour inclure des graphiques
\usepackage{enumitem,cprotect}		% personnalise les listes d'items (nécessaire pour ol, al ...)
\usepackage{hyperref}		% Liens hypertexte
\usepackage{pstricks,pst-all,pst-node,pstricks-add,pst-math,pst-plot,pst-tree,pst-eucl} % pstricks
\usepackage[a4paper,includeheadfoot,top=2cm,left=3cm, bottom=2cm,right=3cm]{geometry} % marges etc.
\usepackage{comment}			% commentaires multilignes
\usepackage{amsmath,environ} % maths (matrices, etc.)
\usepackage{amssymb,makeidx}
\usepackage{bm}				% bold maths
\usepackage{tabularx}		% tableaux
\usepackage{colortbl}		% tableaux en couleur
\usepackage{fontawesome}		% Fontawesome
\usepackage{environ}			% environment with command
\usepackage{fp}				% calculs pour ps-tricks
\usepackage{multido}			% pour ps tricks
\usepackage[np]{numprint}	% formattage nombre
\usepackage{tikz,tkz-tab} 			% package principal TikZ
\usepackage{pgfplots}   % axes
\usepackage{mathrsfs}    % cursives
\usepackage{calc}			% calcul taille boites
\usepackage[scaled=0.875]{helvet} % font sans serif
\usepackage{svg} % svg
\usepackage{scrextend} % local margin
\usepackage{scratch} %scratch
\usepackage{multicol} % colonnes
%\usepackage{infix-RPN,pst-func} % formule en notation polanaise inversée
\usepackage{listings}

%================================================================================================================================
%
% Réglages de base
%
%================================================================================================================================

\lstset{
language=Python,   % R code
literate=
{á}{{\'a}}1
{à}{{\`a}}1
{ã}{{\~a}}1
{é}{{\'e}}1
{è}{{\`e}}1
{ê}{{\^e}}1
{í}{{\'i}}1
{ó}{{\'o}}1
{õ}{{\~o}}1
{ú}{{\'u}}1
{ü}{{\"u}}1
{ç}{{\c{c}}}1
{~}{{ }}1
}


\definecolor{codegreen}{rgb}{0,0.6,0}
\definecolor{codegray}{rgb}{0.5,0.5,0.5}
\definecolor{codepurple}{rgb}{0.58,0,0.82}
\definecolor{backcolour}{rgb}{0.95,0.95,0.92}

\lstdefinestyle{mystyle}{
    backgroundcolor=\color{backcolour},   
    commentstyle=\color{codegreen},
    keywordstyle=\color{magenta},
    numberstyle=\tiny\color{codegray},
    stringstyle=\color{codepurple},
    basicstyle=\ttfamily\footnotesize,
    breakatwhitespace=false,         
    breaklines=true,                 
    captionpos=b,                    
    keepspaces=true,                 
    numbers=left,                    
xleftmargin=2em,
framexleftmargin=2em,            
    showspaces=false,                
    showstringspaces=false,
    showtabs=false,                  
    tabsize=2,
    upquote=true
}

\lstset{style=mystyle}


\lstset{style=mystyle}
\newcommand{\imgdir}{C:/laragon/www/newmc/assets/imgsvg/}
\newcommand{\imgsvgdir}{C:/laragon/www/newmc/assets/imgsvg/}

\definecolor{mcgris}{RGB}{220, 220, 220}% ancien~; pour compatibilité
\definecolor{mcbleu}{RGB}{52, 152, 219}
\definecolor{mcvert}{RGB}{125, 194, 70}
\definecolor{mcmauve}{RGB}{154, 0, 215}
\definecolor{mcorange}{RGB}{255, 96, 0}
\definecolor{mcturquoise}{RGB}{0, 153, 153}
\definecolor{mcrouge}{RGB}{255, 0, 0}
\definecolor{mclightvert}{RGB}{205, 234, 190}

\definecolor{gris}{RGB}{220, 220, 220}
\definecolor{bleu}{RGB}{52, 152, 219}
\definecolor{vert}{RGB}{125, 194, 70}
\definecolor{mauve}{RGB}{154, 0, 215}
\definecolor{orange}{RGB}{255, 96, 0}
\definecolor{turquoise}{RGB}{0, 153, 153}
\definecolor{rouge}{RGB}{255, 0, 0}
\definecolor{lightvert}{RGB}{205, 234, 190}
\setitemize[0]{label=\color{lightvert}  $\bullet$}

\pagestyle{fancy}
\renewcommand{\headrulewidth}{0.2pt}
\fancyhead[L]{maths-cours.fr}
\fancyhead[R]{\thepage}
\renewcommand{\footrulewidth}{0.2pt}
\fancyfoot[C]{}

\newcolumntype{C}{>{\centering\arraybackslash}X}
\newcolumntype{s}{>{\hsize=.35\hsize\arraybackslash}X}

\setlength{\parindent}{0pt}		 
\setlength{\parskip}{3mm}
\setlength{\headheight}{1cm}

\def\ebook{ebook}
\def\book{book}
\def\web{web}
\def\type{web}

\newcommand{\vect}[1]{\overrightarrow{\,\mathstrut#1\,}}

\def\Oij{$\left(\text{O}~;~\vect{\imath},~\vect{\jmath}\right)$}
\def\Oijk{$\left(\text{O}~;~\vect{\imath},~\vect{\jmath},~\vect{k}\right)$}
\def\Ouv{$\left(\text{O}~;~\vect{u},~\vect{v}\right)$}

\hypersetup{breaklinks=true, colorlinks = true, linkcolor = OliveGreen, urlcolor = OliveGreen, citecolor = OliveGreen, pdfauthor={Didier BONNEL - https://www.maths-cours.fr} } % supprime les bordures autour des liens

\renewcommand{\arg}[0]{\text{arg}}

\everymath{\displaystyle}

%================================================================================================================================
%
% Macros - Commandes
%
%================================================================================================================================

\newcommand\meta[2]{    			% Utilisé pour créer le post HTML.
	\def\titre{titre}
	\def\url{url}
	\def\arg{#1}
	\ifx\titre\arg
		\newcommand\maintitle{#2}
		\fancyhead[L]{#2}
		{\Large\sffamily \MakeUppercase{#2}}
		\vspace{1mm}\textcolor{mcvert}{\hrule}
	\fi 
	\ifx\url\arg
		\fancyfoot[L]{\href{https://www.maths-cours.fr#2}{\black \footnotesize{https://www.maths-cours.fr#2}}}
	\fi 
}


\newcommand\TitreC[1]{    		% Titre centré
     \needspace{3\baselineskip}
     \begin{center}\textbf{#1}\end{center}
}

\newcommand\newpar{    		% paragraphe
     \par
}

\newcommand\nosp {    		% commande vide (pas d'espace)
}
\newcommand{\id}[1]{} %ignore

\newcommand\boite[2]{				% Boite simple sans titre
	\vspace{5mm}
	\setlength{\fboxrule}{0.2mm}
	\setlength{\fboxsep}{5mm}	
	\fcolorbox{#1}{#1!3}{\makebox[\linewidth-2\fboxrule-2\fboxsep]{
  		\begin{minipage}[t]{\linewidth-2\fboxrule-4\fboxsep}\setlength{\parskip}{3mm}
  			 #2
  		\end{minipage}
	}}
	\vspace{5mm}
}

\newcommand\CBox[4]{				% Boites
	\vspace{5mm}
	\setlength{\fboxrule}{0.2mm}
	\setlength{\fboxsep}{5mm}
	
	\fcolorbox{#1}{#1!3}{\makebox[\linewidth-2\fboxrule-2\fboxsep]{
		\begin{minipage}[t]{1cm}\setlength{\parskip}{3mm}
	  		\textcolor{#1}{\LARGE{#2}}    
 	 	\end{minipage}  
  		\begin{minipage}[t]{\linewidth-2\fboxrule-4\fboxsep}\setlength{\parskip}{3mm}
			\raisebox{1.2mm}{\normalsize\sffamily{\textcolor{#1}{#3}}}						
  			 #4
  		\end{minipage}
	}}
	\vspace{5mm}
}

\newcommand\cadre[3]{				% Boites convertible html
	\par
	\vspace{2mm}
	\setlength{\fboxrule}{0.1mm}
	\setlength{\fboxsep}{5mm}
	\fcolorbox{#1}{white}{\makebox[\linewidth-2\fboxrule-2\fboxsep]{
  		\begin{minipage}[t]{\linewidth-2\fboxrule-4\fboxsep}\setlength{\parskip}{3mm}
			\raisebox{-2.5mm}{\sffamily \small{\textcolor{#1}{\MakeUppercase{#2}}}}		
			\par		
  			 #3
 	 		\end{minipage}
	}}
		\vspace{2mm}
	\par
}

\newcommand\bloc[3]{				% Boites convertible html sans bordure
     \needspace{2\baselineskip}
     {\sffamily \small{\textcolor{#1}{\MakeUppercase{#2}}}}    
		\par		
  			 #3
		\par
}

\newcommand\CHelp[1]{
     \CBox{Plum}{\faInfoCircle}{À RETENIR}{#1}
}

\newcommand\CUp[1]{
     \CBox{NavyBlue}{\faThumbsOUp}{EN PRATIQUE}{#1}
}

\newcommand\CInfo[1]{
     \CBox{Sepia}{\faArrowCircleRight}{REMARQUE}{#1}
}

\newcommand\CRedac[1]{
     \CBox{PineGreen}{\faEdit}{BIEN R\'EDIGER}{#1}
}

\newcommand\CError[1]{
     \CBox{Red}{\faExclamationTriangle}{ATTENTION}{#1}
}

\newcommand\TitreExo[2]{
\needspace{4\baselineskip}
 {\sffamily\large EXERCICE #1\ (\emph{#2 points})}
\vspace{5mm}
}

\newcommand\img[2]{
          \includegraphics[width=#2\paperwidth]{\imgdir#1}
}

\newcommand\imgsvg[2]{
       \begin{center}   \includegraphics[width=#2\paperwidth]{\imgsvgdir#1} \end{center}
}


\newcommand\Lien[2]{
     \href{#1}{#2 \tiny \faExternalLink}
}
\newcommand\mcLien[2]{
     \href{https~://www.maths-cours.fr/#1}{#2 \tiny \faExternalLink}
}

\newcommand{\euro}{\eurologo{}}

%================================================================================================================================
%
% Macros - Environement
%
%================================================================================================================================

\newenvironment{tex}{ %
}
{%
}

\newenvironment{indente}{ %
	\setlength\parindent{10mm}
}

{
	\setlength\parindent{0mm}
}

\newenvironment{corrige}{%
     \needspace{3\baselineskip}
     \medskip
     \textbf{\textsc{Corrigé}}
     \medskip
}
{
}

\newenvironment{extern}{%
     \begin{center}
     }
     {
     \end{center}
}

\NewEnviron{code}{%
	\par
     \boite{gray}{\texttt{%
     \BODY
     }}
     \par
}

\newenvironment{vbloc}{% boite sans cadre empeche saut de page
     \begin{minipage}[t]{\linewidth}
     }
     {
     \end{minipage}
}
\NewEnviron{h2}{%
    \needspace{3\baselineskip}
    \vspace{0.6cm}
	\noindent \MakeUppercase{\sffamily \large \BODY}
	\vspace{1mm}\textcolor{mcgris}{\hrule}\vspace{0.4cm}
	\par
}{}

\NewEnviron{h3}{%
    \needspace{3\baselineskip}
	\vspace{5mm}
	\textsc{\BODY}
	\par
}

\NewEnviron{margeneg}{ %
\begin{addmargin}[-1cm]{0cm}
\BODY
\end{addmargin}
}

\NewEnviron{html}{%
}

\begin{document}
\meta{url}{/exercices/roc-esperance-mathematique-dune-loi-exponentielle/}
\meta{pid}{7676}
\meta{titre}{[ROC] Espérance mathématique d'une loi exponentielle}
\meta{type}{exercices}
%
\boite{mcvert}{
     L'objectif de cet exercice est de démontrer que l'espérance mathématique de la loi exponentielle de paramètre $\lambda$ est $\dfrac{1}{\lambda }$.
}
\par
Soient $a$ et $b$ deux réels quelconques et $\lambda$ un réel strictement positif. On considère la fonction $f$ définie sur l'intervalle $[0~;~+\infty[$ par :
\[f(x)=(ax+b)\text{e}^{-\lambda x}.\]
\begin{enumerate}
     \item %
     Calculer $f'(x)$.
     \item %
     Montrer qu'il existe une valeur de $a$ et une valeur de $b$ pour lesquelles, pour tout réel $x \geqslant 0$, $f'(x)=x\text{e}^{-\lambda x}$ et déterminer ces valeurs.
     \item %
     L'espérance mathématique d'une variable aléatoire continue $X$ qui suit une loi de densité $f$ sur l'intervalle $\left[a;b\right]$ est $E\left(X\right)=\int_{a}^{b}xf\left(x\right)dx$.
     \par
     En particulier, dans le cas de la loi exponentielle de paramètre $\lambda > 0$ :
     \begin{center}
          $E(X)=\displaystyle\int_{0}^{+\infty}\lambda x \text{e}^{-\lambda x}dx$\nosp$=\lim\limits_{t \rightarrow +\infty}\displaystyle\int_{0}^{t}\lambda x \text{e}^{-\lambda x}dx$.
     \end{center}
     \begin{enumerate}[label=\alph*.]
          \item %
          Calculer, en fonction de $t$,  $I(t)=\displaystyle\int_{0}^{t}\lambda x \text{e}^{-\lambda x}dx$.
          \item %
          En déduire que, pour la loi exponentielle de paramètre $\lambda$ :
          \begin{center}
               $E(X)=\dfrac{1}{\lambda }$.
          \end{center}
     \end{enumerate}
\end{enumerate}
\begin{corrige}
     \begin{enumerate}
          \item %
          Posons $u(x)=ax+b$ et $v(x)=\text{e}^{-\lambda x}$ ; alors :
          \par
          $u'(x)=a$ et $v'(x)=-\lambda \text{e}^{-\lambda x}$.
          \par
          Par conséquent :
          \par
          $f'(x)= u'(x)v(x)+u(x)v'(x)$\\
          $\phantom{f'(x)}=a\text{e}^{-\lambda x}-\lambda (ax+b)\text{e}^{-\lambda x}$\\
          $\phantom{f'(x)}=(-\lambda ax+a-\lambda b)\text{e}^{-\lambda x}$.
          \item %
          $f'(x)=x\text{e}^{-\lambda x}$, pour tout réel $x \geqslant 0$, si et seulement si $-\lambda ax+a-\lambda b$ est identique à $x$, c'est à dire si et seulement si le couple $(a~;~b)$ est solution du système :
          \par
          $\begin{cases}
               -\lambda a = 1\\
               a-\lambda b = 0
          \end{cases}$
          \par
          La première équation donne immédiatement $a=-\dfrac{1}{\lambda }$ ; puis, en remplaçant $a$ dans la seconde, on obtient $b=-\dfrac{1}{\lambda^2 }$.
          \par
          Finalement, la fonction $f$ définie par :
          \begin{center}
               $f(x)=\left(-\dfrac{1}{\lambda }x-\dfrac{1}{\lambda ^2}\right)\text{e}^{-\lambda x}$
          \end{center}
          a pour dérivée la fonction $x \longmapsto x\text{e}^{-\lambda x}$.
          \item %
          \begin{enumerate}[label=\alph*.]
               \item %
               D'après la question précédente, la fonction $x \longmapsto \left(-\dfrac{1}{\lambda }x-\dfrac{1}{\lambda ^2}\right)\text{e}^{-\lambda x}$ est une primitive sur $[0~;~+\infty[$ de la fonction $x \longmapsto x\text{e}^{-\lambda x}$.
               \par
               On en déduit que :\\
               $I(t)=\displaystyle\int_{0}^{t}\lambda x \text{e}^{-\lambda x}dx$\\
               $\phantom{I(t)} = \lambda\displaystyle\int_{0}^{t} x \text{e}^{-\lambda x}dx$\\
               $\phantom{I(t)} = \lambda \left[\left(-\dfrac{1}{\lambda }x-\dfrac{1}{\lambda ^2}\right)\text{e}^{-\lambda x}\right]_0^t$\\
               $\phantom{I(t)} = \lambda \left(-\dfrac{1}{\lambda }t-\dfrac{1}{\lambda ^2}\right)\text{e}^{-\lambda t} - \lambda\left(-\dfrac{1}{\lambda ^2}\right)\text{e}^{-\lambda 0}$\\
               $\phantom{I(t)} = -t\text{e}^{-\lambda t} - \dfrac{1}{\lambda } \text{e}^{-\lambda t} +\dfrac{1}{\lambda }$.
               \item %
               Lorsque  $t$ tend vers $+\infty $, comme $\lambda $ est strictement positif $-\lambda t$ tend vers $-\infty$.
               \par
               Alors :
               \begin{itemize}
                    \item %
                    $\lim\limits_{t \rightarrow +\infty} \text{e}^{-\lambda t} =0$ (par composition)
                    \item %
                    $\lim\limits_{t \rightarrow +\infty} t\text{e}^{-\lambda t} =0$ (croissance comparée)
               \end{itemize}
               donc, par somme :
               $\lim\limits_{t \rightarrow +\infty}-t\text{e}^{-\lambda t} - \dfrac{1}{\lambda } \text{e}^{-\lambda t} +\dfrac{1}{\lambda } = \dfrac{1}{\lambda }$.
               \par
               On a donc bien :
               \[ E(X) = \dfrac{1}{\lambda }. \]
          \end{enumerate}
     \end{enumerate}
\end{corrige}

\end{document}
µ
\documentclass[a4paper]{article}

%================================================================================================================================
%
% Packages
%
%================================================================================================================================

\usepackage[T1]{fontenc} 	% pour caractères accentués
\usepackage[utf8]{inputenc}  % encodage utf8
\usepackage[french]{babel}	% langue : français
\usepackage{fourier}			% caractères plus lisibles
\usepackage[dvipsnames]{xcolor} % couleurs
\usepackage{fancyhdr}		% réglage header footer
\usepackage{needspace}		% empêcher sauts de page mal placés
\usepackage{graphicx}		% pour inclure des graphiques
\usepackage{enumitem,cprotect}		% personnalise les listes d'items (nécessaire pour ol, al ...)
\usepackage{hyperref}		% Liens hypertexte
\usepackage{pstricks,pst-all,pst-node,pstricks-add,pst-math,pst-plot,pst-tree,pst-eucl} % pstricks
\usepackage[a4paper,includeheadfoot,top=2cm,left=3cm, bottom=2cm,right=3cm]{geometry} % marges etc.
\usepackage{comment}			% commentaires multilignes
\usepackage{amsmath,environ} % maths (matrices, etc.)
\usepackage{amssymb,makeidx}
\usepackage{bm}				% bold maths
\usepackage{tabularx}		% tableaux
\usepackage{colortbl}		% tableaux en couleur
\usepackage{fontawesome}		% Fontawesome
\usepackage{environ}			% environment with command
\usepackage{fp}				% calculs pour ps-tricks
\usepackage{multido}			% pour ps tricks
\usepackage[np]{numprint}	% formattage nombre
\usepackage{tikz,tkz-tab} 			% package principal TikZ
\usepackage{pgfplots}   % axes
\usepackage{mathrsfs}    % cursives
\usepackage{calc}			% calcul taille boites
\usepackage[scaled=0.875]{helvet} % font sans serif
\usepackage{svg} % svg
\usepackage{scrextend} % local margin
\usepackage{scratch} %scratch
\usepackage{multicol} % colonnes
%\usepackage{infix-RPN,pst-func} % formule en notation polanaise inversée
\usepackage{listings}

%================================================================================================================================
%
% Réglages de base
%
%================================================================================================================================

\lstset{
language=Python,   % R code
literate=
{á}{{\'a}}1
{à}{{\`a}}1
{ã}{{\~a}}1
{é}{{\'e}}1
{è}{{\`e}}1
{ê}{{\^e}}1
{í}{{\'i}}1
{ó}{{\'o}}1
{õ}{{\~o}}1
{ú}{{\'u}}1
{ü}{{\"u}}1
{ç}{{\c{c}}}1
{~}{{ }}1
}


\definecolor{codegreen}{rgb}{0,0.6,0}
\definecolor{codegray}{rgb}{0.5,0.5,0.5}
\definecolor{codepurple}{rgb}{0.58,0,0.82}
\definecolor{backcolour}{rgb}{0.95,0.95,0.92}

\lstdefinestyle{mystyle}{
    backgroundcolor=\color{backcolour},   
    commentstyle=\color{codegreen},
    keywordstyle=\color{magenta},
    numberstyle=\tiny\color{codegray},
    stringstyle=\color{codepurple},
    basicstyle=\ttfamily\footnotesize,
    breakatwhitespace=false,         
    breaklines=true,                 
    captionpos=b,                    
    keepspaces=true,                 
    numbers=left,                    
xleftmargin=2em,
framexleftmargin=2em,            
    showspaces=false,                
    showstringspaces=false,
    showtabs=false,                  
    tabsize=2,
    upquote=true
}

\lstset{style=mystyle}


\lstset{style=mystyle}
\newcommand{\imgdir}{C:/laragon/www/newmc/assets/imgsvg/}
\newcommand{\imgsvgdir}{C:/laragon/www/newmc/assets/imgsvg/}

\definecolor{mcgris}{RGB}{220, 220, 220}% ancien~; pour compatibilité
\definecolor{mcbleu}{RGB}{52, 152, 219}
\definecolor{mcvert}{RGB}{125, 194, 70}
\definecolor{mcmauve}{RGB}{154, 0, 215}
\definecolor{mcorange}{RGB}{255, 96, 0}
\definecolor{mcturquoise}{RGB}{0, 153, 153}
\definecolor{mcrouge}{RGB}{255, 0, 0}
\definecolor{mclightvert}{RGB}{205, 234, 190}

\definecolor{gris}{RGB}{220, 220, 220}
\definecolor{bleu}{RGB}{52, 152, 219}
\definecolor{vert}{RGB}{125, 194, 70}
\definecolor{mauve}{RGB}{154, 0, 215}
\definecolor{orange}{RGB}{255, 96, 0}
\definecolor{turquoise}{RGB}{0, 153, 153}
\definecolor{rouge}{RGB}{255, 0, 0}
\definecolor{lightvert}{RGB}{205, 234, 190}
\setitemize[0]{label=\color{lightvert}  $\bullet$}

\pagestyle{fancy}
\renewcommand{\headrulewidth}{0.2pt}
\fancyhead[L]{maths-cours.fr}
\fancyhead[R]{\thepage}
\renewcommand{\footrulewidth}{0.2pt}
\fancyfoot[C]{}

\newcolumntype{C}{>{\centering\arraybackslash}X}
\newcolumntype{s}{>{\hsize=.35\hsize\arraybackslash}X}

\setlength{\parindent}{0pt}		 
\setlength{\parskip}{3mm}
\setlength{\headheight}{1cm}

\def\ebook{ebook}
\def\book{book}
\def\web{web}
\def\type{web}

\newcommand{\vect}[1]{\overrightarrow{\,\mathstrut#1\,}}

\def\Oij{$\left(\text{O}~;~\vect{\imath},~\vect{\jmath}\right)$}
\def\Oijk{$\left(\text{O}~;~\vect{\imath},~\vect{\jmath},~\vect{k}\right)$}
\def\Ouv{$\left(\text{O}~;~\vect{u},~\vect{v}\right)$}

\hypersetup{breaklinks=true, colorlinks = true, linkcolor = OliveGreen, urlcolor = OliveGreen, citecolor = OliveGreen, pdfauthor={Didier BONNEL - https://www.maths-cours.fr} } % supprime les bordures autour des liens

\renewcommand{\arg}[0]{\text{arg}}

\everymath{\displaystyle}

%================================================================================================================================
%
% Macros - Commandes
%
%================================================================================================================================

\newcommand\meta[2]{    			% Utilisé pour créer le post HTML.
	\def\titre{titre}
	\def\url{url}
	\def\arg{#1}
	\ifx\titre\arg
		\newcommand\maintitle{#2}
		\fancyhead[L]{#2}
		{\Large\sffamily \MakeUppercase{#2}}
		\vspace{1mm}\textcolor{mcvert}{\hrule}
	\fi 
	\ifx\url\arg
		\fancyfoot[L]{\href{https://www.maths-cours.fr#2}{\black \footnotesize{https://www.maths-cours.fr#2}}}
	\fi 
}


\newcommand\TitreC[1]{    		% Titre centré
     \needspace{3\baselineskip}
     \begin{center}\textbf{#1}\end{center}
}

\newcommand\newpar{    		% paragraphe
     \par
}

\newcommand\nosp {    		% commande vide (pas d'espace)
}
\newcommand{\id}[1]{} %ignore

\newcommand\boite[2]{				% Boite simple sans titre
	\vspace{5mm}
	\setlength{\fboxrule}{0.2mm}
	\setlength{\fboxsep}{5mm}	
	\fcolorbox{#1}{#1!3}{\makebox[\linewidth-2\fboxrule-2\fboxsep]{
  		\begin{minipage}[t]{\linewidth-2\fboxrule-4\fboxsep}\setlength{\parskip}{3mm}
  			 #2
  		\end{minipage}
	}}
	\vspace{5mm}
}

\newcommand\CBox[4]{				% Boites
	\vspace{5mm}
	\setlength{\fboxrule}{0.2mm}
	\setlength{\fboxsep}{5mm}
	
	\fcolorbox{#1}{#1!3}{\makebox[\linewidth-2\fboxrule-2\fboxsep]{
		\begin{minipage}[t]{1cm}\setlength{\parskip}{3mm}
	  		\textcolor{#1}{\LARGE{#2}}    
 	 	\end{minipage}  
  		\begin{minipage}[t]{\linewidth-2\fboxrule-4\fboxsep}\setlength{\parskip}{3mm}
			\raisebox{1.2mm}{\normalsize\sffamily{\textcolor{#1}{#3}}}						
  			 #4
  		\end{minipage}
	}}
	\vspace{5mm}
}

\newcommand\cadre[3]{				% Boites convertible html
	\par
	\vspace{2mm}
	\setlength{\fboxrule}{0.1mm}
	\setlength{\fboxsep}{5mm}
	\fcolorbox{#1}{white}{\makebox[\linewidth-2\fboxrule-2\fboxsep]{
  		\begin{minipage}[t]{\linewidth-2\fboxrule-4\fboxsep}\setlength{\parskip}{3mm}
			\raisebox{-2.5mm}{\sffamily \small{\textcolor{#1}{\MakeUppercase{#2}}}}		
			\par		
  			 #3
 	 		\end{minipage}
	}}
		\vspace{2mm}
	\par
}

\newcommand\bloc[3]{				% Boites convertible html sans bordure
     \needspace{2\baselineskip}
     {\sffamily \small{\textcolor{#1}{\MakeUppercase{#2}}}}    
		\par		
  			 #3
		\par
}

\newcommand\CHelp[1]{
     \CBox{Plum}{\faInfoCircle}{À RETENIR}{#1}
}

\newcommand\CUp[1]{
     \CBox{NavyBlue}{\faThumbsOUp}{EN PRATIQUE}{#1}
}

\newcommand\CInfo[1]{
     \CBox{Sepia}{\faArrowCircleRight}{REMARQUE}{#1}
}

\newcommand\CRedac[1]{
     \CBox{PineGreen}{\faEdit}{BIEN R\'EDIGER}{#1}
}

\newcommand\CError[1]{
     \CBox{Red}{\faExclamationTriangle}{ATTENTION}{#1}
}

\newcommand\TitreExo[2]{
\needspace{4\baselineskip}
 {\sffamily\large EXERCICE #1\ (\emph{#2 points})}
\vspace{5mm}
}

\newcommand\img[2]{
          \includegraphics[width=#2\paperwidth]{\imgdir#1}
}

\newcommand\imgsvg[2]{
       \begin{center}   \includegraphics[width=#2\paperwidth]{\imgsvgdir#1} \end{center}
}


\newcommand\Lien[2]{
     \href{#1}{#2 \tiny \faExternalLink}
}
\newcommand\mcLien[2]{
     \href{https~://www.maths-cours.fr/#1}{#2 \tiny \faExternalLink}
}

\newcommand{\euro}{\eurologo{}}

%================================================================================================================================
%
% Macros - Environement
%
%================================================================================================================================

\newenvironment{tex}{ %
}
{%
}

\newenvironment{indente}{ %
	\setlength\parindent{10mm}
}

{
	\setlength\parindent{0mm}
}

\newenvironment{corrige}{%
     \needspace{3\baselineskip}
     \medskip
     \textbf{\textsc{Corrigé}}
     \medskip
}
{
}

\newenvironment{extern}{%
     \begin{center}
     }
     {
     \end{center}
}

\NewEnviron{code}{%
	\par
     \boite{gray}{\texttt{%
     \BODY
     }}
     \par
}

\newenvironment{vbloc}{% boite sans cadre empeche saut de page
     \begin{minipage}[t]{\linewidth}
     }
     {
     \end{minipage}
}
\NewEnviron{h2}{%
    \needspace{3\baselineskip}
    \vspace{0.6cm}
	\noindent \MakeUppercase{\sffamily \large \BODY}
	\vspace{1mm}\textcolor{mcgris}{\hrule}\vspace{0.4cm}
	\par
}{}

\NewEnviron{h3}{%
    \needspace{3\baselineskip}
	\vspace{5mm}
	\textsc{\BODY}
	\par
}

\NewEnviron{margeneg}{ %
\begin{addmargin}[-1cm]{0cm}
\BODY
\end{addmargin}
}

\NewEnviron{html}{%
}

\begin{document}
\meta{url}{/exercices/probabilites-bac-s-liban-2018/}
\meta{pid}{7716}
\meta{titre}{Lois continues – Bac S Liban 2018}
\meta{type}{exercices}
%
\begin{h2}Exercice 1 (3 points)\end{h2}
\textbf{Commun à  tous les candidats}
\medskip
Les quinze jours précédant la rentrée universitaire, le standard téléphonique d'une mutuelle étudiante
enregistre un nombre record d'appels.
\par
Les appelants sont d'abord mis en attente et entendent une musique d'ambiance et un message
pré-enregistré.
\par
Lors de cette première phase, le temps d'attente, exprimé en secondes, est modélisé par la variable
aléatoire $X$ qui suit la loi exponentielle de paramètre $\lambda = 0,02$s$^{-1}$.
\par
Les appelants sont ensuite mis en relation avec un chargé de clientèle qui répond à leurs questions.
\par
Le temps d'échange, exprimé en secondes, lors de cette deuxième phase est modélisé par la
variable aléatoire $Y$, exprimée en secondes, qui suit la loi normale d'espérance $\mu = 96$s et d'écart-type
$\sigma = 26$s.
\begin{enumerate}
     \item Quelle est la durée totale moyenne d'un appel au standard téléphonique (temps d'attente et
     temps d'échange avec le chargé de clientèle)~?
     \item  Un étudiant est choisi au hasard parmi les appelants du standard téléphonique.
     \begin{enumerate}[label=\alph*.]
          \item Calculer la probabilité que l'étudiant soit mis en attente plus de 2 minutes.
          \item Calculer la probabilité pour que le temps d'échange avec le conseiller soit inférieur à 90~secondes.
     \end{enumerate}
     \item  Une étudiante, choisie au hasard parmi les appelants, attend depuis plus d'une minute d'être
     mise en relation avec le service clientèle. Lasse, elle raccroche et recompose le numéro. Elle
     espère attendre moins de trente secondes cette fois-ci.
     \par
     Le fait de raccrocher puis de rappeler augmente-t-il ses chances de limiter à 30~secondes
     l'attente supplémentaire ou bien aurait-elle mieux fait de rester en ligne~?
\end{enumerate}

\end{document}
µ
\documentclass[a4paper]{article}

%================================================================================================================================
%
% Packages
%
%================================================================================================================================

\usepackage[T1]{fontenc} 	% pour caractères accentués
\usepackage[utf8]{inputenc}  % encodage utf8
\usepackage[french]{babel}	% langue : français
\usepackage{fourier}			% caractères plus lisibles
\usepackage[dvipsnames]{xcolor} % couleurs
\usepackage{fancyhdr}		% réglage header footer
\usepackage{needspace}		% empêcher sauts de page mal placés
\usepackage{graphicx}		% pour inclure des graphiques
\usepackage{enumitem,cprotect}		% personnalise les listes d'items (nécessaire pour ol, al ...)
\usepackage{hyperref}		% Liens hypertexte
\usepackage{pstricks,pst-all,pst-node,pstricks-add,pst-math,pst-plot,pst-tree,pst-eucl} % pstricks
\usepackage[a4paper,includeheadfoot,top=2cm,left=3cm, bottom=2cm,right=3cm]{geometry} % marges etc.
\usepackage{comment}			% commentaires multilignes
\usepackage{amsmath,environ} % maths (matrices, etc.)
\usepackage{amssymb,makeidx}
\usepackage{bm}				% bold maths
\usepackage{tabularx}		% tableaux
\usepackage{colortbl}		% tableaux en couleur
\usepackage{fontawesome}		% Fontawesome
\usepackage{environ}			% environment with command
\usepackage{fp}				% calculs pour ps-tricks
\usepackage{multido}			% pour ps tricks
\usepackage[np]{numprint}	% formattage nombre
\usepackage{tikz,tkz-tab} 			% package principal TikZ
\usepackage{pgfplots}   % axes
\usepackage{mathrsfs}    % cursives
\usepackage{calc}			% calcul taille boites
\usepackage[scaled=0.875]{helvet} % font sans serif
\usepackage{svg} % svg
\usepackage{scrextend} % local margin
\usepackage{scratch} %scratch
\usepackage{multicol} % colonnes
%\usepackage{infix-RPN,pst-func} % formule en notation polanaise inversée
\usepackage{listings}

%================================================================================================================================
%
% Réglages de base
%
%================================================================================================================================

\lstset{
language=Python,   % R code
literate=
{á}{{\'a}}1
{à}{{\`a}}1
{ã}{{\~a}}1
{é}{{\'e}}1
{è}{{\`e}}1
{ê}{{\^e}}1
{í}{{\'i}}1
{ó}{{\'o}}1
{õ}{{\~o}}1
{ú}{{\'u}}1
{ü}{{\"u}}1
{ç}{{\c{c}}}1
{~}{{ }}1
}


\definecolor{codegreen}{rgb}{0,0.6,0}
\definecolor{codegray}{rgb}{0.5,0.5,0.5}
\definecolor{codepurple}{rgb}{0.58,0,0.82}
\definecolor{backcolour}{rgb}{0.95,0.95,0.92}

\lstdefinestyle{mystyle}{
    backgroundcolor=\color{backcolour},   
    commentstyle=\color{codegreen},
    keywordstyle=\color{magenta},
    numberstyle=\tiny\color{codegray},
    stringstyle=\color{codepurple},
    basicstyle=\ttfamily\footnotesize,
    breakatwhitespace=false,         
    breaklines=true,                 
    captionpos=b,                    
    keepspaces=true,                 
    numbers=left,                    
xleftmargin=2em,
framexleftmargin=2em,            
    showspaces=false,                
    showstringspaces=false,
    showtabs=false,                  
    tabsize=2,
    upquote=true
}

\lstset{style=mystyle}


\lstset{style=mystyle}
\newcommand{\imgdir}{C:/laragon/www/newmc/assets/imgsvg/}
\newcommand{\imgsvgdir}{C:/laragon/www/newmc/assets/imgsvg/}

\definecolor{mcgris}{RGB}{220, 220, 220}% ancien~; pour compatibilité
\definecolor{mcbleu}{RGB}{52, 152, 219}
\definecolor{mcvert}{RGB}{125, 194, 70}
\definecolor{mcmauve}{RGB}{154, 0, 215}
\definecolor{mcorange}{RGB}{255, 96, 0}
\definecolor{mcturquoise}{RGB}{0, 153, 153}
\definecolor{mcrouge}{RGB}{255, 0, 0}
\definecolor{mclightvert}{RGB}{205, 234, 190}

\definecolor{gris}{RGB}{220, 220, 220}
\definecolor{bleu}{RGB}{52, 152, 219}
\definecolor{vert}{RGB}{125, 194, 70}
\definecolor{mauve}{RGB}{154, 0, 215}
\definecolor{orange}{RGB}{255, 96, 0}
\definecolor{turquoise}{RGB}{0, 153, 153}
\definecolor{rouge}{RGB}{255, 0, 0}
\definecolor{lightvert}{RGB}{205, 234, 190}
\setitemize[0]{label=\color{lightvert}  $\bullet$}

\pagestyle{fancy}
\renewcommand{\headrulewidth}{0.2pt}
\fancyhead[L]{maths-cours.fr}
\fancyhead[R]{\thepage}
\renewcommand{\footrulewidth}{0.2pt}
\fancyfoot[C]{}

\newcolumntype{C}{>{\centering\arraybackslash}X}
\newcolumntype{s}{>{\hsize=.35\hsize\arraybackslash}X}

\setlength{\parindent}{0pt}		 
\setlength{\parskip}{3mm}
\setlength{\headheight}{1cm}

\def\ebook{ebook}
\def\book{book}
\def\web{web}
\def\type{web}

\newcommand{\vect}[1]{\overrightarrow{\,\mathstrut#1\,}}

\def\Oij{$\left(\text{O}~;~\vect{\imath},~\vect{\jmath}\right)$}
\def\Oijk{$\left(\text{O}~;~\vect{\imath},~\vect{\jmath},~\vect{k}\right)$}
\def\Ouv{$\left(\text{O}~;~\vect{u},~\vect{v}\right)$}

\hypersetup{breaklinks=true, colorlinks = true, linkcolor = OliveGreen, urlcolor = OliveGreen, citecolor = OliveGreen, pdfauthor={Didier BONNEL - https://www.maths-cours.fr} } % supprime les bordures autour des liens

\renewcommand{\arg}[0]{\text{arg}}

\everymath{\displaystyle}

%================================================================================================================================
%
% Macros - Commandes
%
%================================================================================================================================

\newcommand\meta[2]{    			% Utilisé pour créer le post HTML.
	\def\titre{titre}
	\def\url{url}
	\def\arg{#1}
	\ifx\titre\arg
		\newcommand\maintitle{#2}
		\fancyhead[L]{#2}
		{\Large\sffamily \MakeUppercase{#2}}
		\vspace{1mm}\textcolor{mcvert}{\hrule}
	\fi 
	\ifx\url\arg
		\fancyfoot[L]{\href{https://www.maths-cours.fr#2}{\black \footnotesize{https://www.maths-cours.fr#2}}}
	\fi 
}


\newcommand\TitreC[1]{    		% Titre centré
     \needspace{3\baselineskip}
     \begin{center}\textbf{#1}\end{center}
}

\newcommand\newpar{    		% paragraphe
     \par
}

\newcommand\nosp {    		% commande vide (pas d'espace)
}
\newcommand{\id}[1]{} %ignore

\newcommand\boite[2]{				% Boite simple sans titre
	\vspace{5mm}
	\setlength{\fboxrule}{0.2mm}
	\setlength{\fboxsep}{5mm}	
	\fcolorbox{#1}{#1!3}{\makebox[\linewidth-2\fboxrule-2\fboxsep]{
  		\begin{minipage}[t]{\linewidth-2\fboxrule-4\fboxsep}\setlength{\parskip}{3mm}
  			 #2
  		\end{minipage}
	}}
	\vspace{5mm}
}

\newcommand\CBox[4]{				% Boites
	\vspace{5mm}
	\setlength{\fboxrule}{0.2mm}
	\setlength{\fboxsep}{5mm}
	
	\fcolorbox{#1}{#1!3}{\makebox[\linewidth-2\fboxrule-2\fboxsep]{
		\begin{minipage}[t]{1cm}\setlength{\parskip}{3mm}
	  		\textcolor{#1}{\LARGE{#2}}    
 	 	\end{minipage}  
  		\begin{minipage}[t]{\linewidth-2\fboxrule-4\fboxsep}\setlength{\parskip}{3mm}
			\raisebox{1.2mm}{\normalsize\sffamily{\textcolor{#1}{#3}}}						
  			 #4
  		\end{minipage}
	}}
	\vspace{5mm}
}

\newcommand\cadre[3]{				% Boites convertible html
	\par
	\vspace{2mm}
	\setlength{\fboxrule}{0.1mm}
	\setlength{\fboxsep}{5mm}
	\fcolorbox{#1}{white}{\makebox[\linewidth-2\fboxrule-2\fboxsep]{
  		\begin{minipage}[t]{\linewidth-2\fboxrule-4\fboxsep}\setlength{\parskip}{3mm}
			\raisebox{-2.5mm}{\sffamily \small{\textcolor{#1}{\MakeUppercase{#2}}}}		
			\par		
  			 #3
 	 		\end{minipage}
	}}
		\vspace{2mm}
	\par
}

\newcommand\bloc[3]{				% Boites convertible html sans bordure
     \needspace{2\baselineskip}
     {\sffamily \small{\textcolor{#1}{\MakeUppercase{#2}}}}    
		\par		
  			 #3
		\par
}

\newcommand\CHelp[1]{
     \CBox{Plum}{\faInfoCircle}{À RETENIR}{#1}
}

\newcommand\CUp[1]{
     \CBox{NavyBlue}{\faThumbsOUp}{EN PRATIQUE}{#1}
}

\newcommand\CInfo[1]{
     \CBox{Sepia}{\faArrowCircleRight}{REMARQUE}{#1}
}

\newcommand\CRedac[1]{
     \CBox{PineGreen}{\faEdit}{BIEN R\'EDIGER}{#1}
}

\newcommand\CError[1]{
     \CBox{Red}{\faExclamationTriangle}{ATTENTION}{#1}
}

\newcommand\TitreExo[2]{
\needspace{4\baselineskip}
 {\sffamily\large EXERCICE #1\ (\emph{#2 points})}
\vspace{5mm}
}

\newcommand\img[2]{
          \includegraphics[width=#2\paperwidth]{\imgdir#1}
}

\newcommand\imgsvg[2]{
       \begin{center}   \includegraphics[width=#2\paperwidth]{\imgsvgdir#1} \end{center}
}


\newcommand\Lien[2]{
     \href{#1}{#2 \tiny \faExternalLink}
}
\newcommand\mcLien[2]{
     \href{https~://www.maths-cours.fr/#1}{#2 \tiny \faExternalLink}
}

\newcommand{\euro}{\eurologo{}}

%================================================================================================================================
%
% Macros - Environement
%
%================================================================================================================================

\newenvironment{tex}{ %
}
{%
}

\newenvironment{indente}{ %
	\setlength\parindent{10mm}
}

{
	\setlength\parindent{0mm}
}

\newenvironment{corrige}{%
     \needspace{3\baselineskip}
     \medskip
     \textbf{\textsc{Corrigé}}
     \medskip
}
{
}

\newenvironment{extern}{%
     \begin{center}
     }
     {
     \end{center}
}

\NewEnviron{code}{%
	\par
     \boite{gray}{\texttt{%
     \BODY
     }}
     \par
}

\newenvironment{vbloc}{% boite sans cadre empeche saut de page
     \begin{minipage}[t]{\linewidth}
     }
     {
     \end{minipage}
}
\NewEnviron{h2}{%
    \needspace{3\baselineskip}
    \vspace{0.6cm}
	\noindent \MakeUppercase{\sffamily \large \BODY}
	\vspace{1mm}\textcolor{mcgris}{\hrule}\vspace{0.4cm}
	\par
}{}

\NewEnviron{h3}{%
    \needspace{3\baselineskip}
	\vspace{5mm}
	\textsc{\BODY}
	\par
}

\NewEnviron{margeneg}{ %
\begin{addmargin}[-1cm]{0cm}
\BODY
\end{addmargin}
}

\NewEnviron{html}{%
}

\begin{document}
\meta{url}{/exercices/nombres-complexes-bac-s-liban-2018/}
\meta{pid}{7722}
\meta{titre}{Nombres complexes – Bac S Liban 2018}
\meta{type}{exercices}
%
\begin{h2}EXERCICE 2 (3 points)\end{h2}
\textbf{Commun à tous les candidats}
\medskip
\begin{enumerate}
     \item Donner les formes exponentielle et trigonométrique des nombres complexes $1 + \text{i}$ et $1 - \text{i}$.
     \item  Pour tout entier naturel $n$, on pose
     \[S_n = (1 + \text{i})^n + (1 - \text{i})^n.\]
     \begin{enumerate}
          \item Déterminer la forme trigonométrique de $S_n$.
          \item Pour chacune des deux affirmations suivantes, dire si elle est vraie ou fausse en justifiant la réponse. \\
          \emph{Une réponse non justifiée ne sera pas prise en compte et l'absence de réponse n'est
          pas pénalisée.}
          \par
          \textbf{Affirmation A }: Pour tout entier naturel $n$, le nombre complexe $S_n$ est un nombre réel.
          \par
          \textbf{Affirmation B }: Il existe une infinité d'entiers naturels $n$ tels que $S_n = 0$.
     \end{enumerate}
\end{enumerate}

\end{document}
µ
\documentclass[a4paper]{article}

%================================================================================================================================
%
% Packages
%
%================================================================================================================================

\usepackage[T1]{fontenc} 	% pour caractères accentués
\usepackage[utf8]{inputenc}  % encodage utf8
\usepackage[french]{babel}	% langue : français
\usepackage{fourier}			% caractères plus lisibles
\usepackage[dvipsnames]{xcolor} % couleurs
\usepackage{fancyhdr}		% réglage header footer
\usepackage{needspace}		% empêcher sauts de page mal placés
\usepackage{graphicx}		% pour inclure des graphiques
\usepackage{enumitem,cprotect}		% personnalise les listes d'items (nécessaire pour ol, al ...)
\usepackage{hyperref}		% Liens hypertexte
\usepackage{pstricks,pst-all,pst-node,pstricks-add,pst-math,pst-plot,pst-tree,pst-eucl} % pstricks
\usepackage[a4paper,includeheadfoot,top=2cm,left=3cm, bottom=2cm,right=3cm]{geometry} % marges etc.
\usepackage{comment}			% commentaires multilignes
\usepackage{amsmath,environ} % maths (matrices, etc.)
\usepackage{amssymb,makeidx}
\usepackage{bm}				% bold maths
\usepackage{tabularx}		% tableaux
\usepackage{colortbl}		% tableaux en couleur
\usepackage{fontawesome}		% Fontawesome
\usepackage{environ}			% environment with command
\usepackage{fp}				% calculs pour ps-tricks
\usepackage{multido}			% pour ps tricks
\usepackage[np]{numprint}	% formattage nombre
\usepackage{tikz,tkz-tab} 			% package principal TikZ
\usepackage{pgfplots}   % axes
\usepackage{mathrsfs}    % cursives
\usepackage{calc}			% calcul taille boites
\usepackage[scaled=0.875]{helvet} % font sans serif
\usepackage{svg} % svg
\usepackage{scrextend} % local margin
\usepackage{scratch} %scratch
\usepackage{multicol} % colonnes
%\usepackage{infix-RPN,pst-func} % formule en notation polanaise inversée
\usepackage{listings}

%================================================================================================================================
%
% Réglages de base
%
%================================================================================================================================

\lstset{
language=Python,   % R code
literate=
{á}{{\'a}}1
{à}{{\`a}}1
{ã}{{\~a}}1
{é}{{\'e}}1
{è}{{\`e}}1
{ê}{{\^e}}1
{í}{{\'i}}1
{ó}{{\'o}}1
{õ}{{\~o}}1
{ú}{{\'u}}1
{ü}{{\"u}}1
{ç}{{\c{c}}}1
{~}{{ }}1
}


\definecolor{codegreen}{rgb}{0,0.6,0}
\definecolor{codegray}{rgb}{0.5,0.5,0.5}
\definecolor{codepurple}{rgb}{0.58,0,0.82}
\definecolor{backcolour}{rgb}{0.95,0.95,0.92}

\lstdefinestyle{mystyle}{
    backgroundcolor=\color{backcolour},   
    commentstyle=\color{codegreen},
    keywordstyle=\color{magenta},
    numberstyle=\tiny\color{codegray},
    stringstyle=\color{codepurple},
    basicstyle=\ttfamily\footnotesize,
    breakatwhitespace=false,         
    breaklines=true,                 
    captionpos=b,                    
    keepspaces=true,                 
    numbers=left,                    
xleftmargin=2em,
framexleftmargin=2em,            
    showspaces=false,                
    showstringspaces=false,
    showtabs=false,                  
    tabsize=2,
    upquote=true
}

\lstset{style=mystyle}


\lstset{style=mystyle}
\newcommand{\imgdir}{C:/laragon/www/newmc/assets/imgsvg/}
\newcommand{\imgsvgdir}{C:/laragon/www/newmc/assets/imgsvg/}

\definecolor{mcgris}{RGB}{220, 220, 220}% ancien~; pour compatibilité
\definecolor{mcbleu}{RGB}{52, 152, 219}
\definecolor{mcvert}{RGB}{125, 194, 70}
\definecolor{mcmauve}{RGB}{154, 0, 215}
\definecolor{mcorange}{RGB}{255, 96, 0}
\definecolor{mcturquoise}{RGB}{0, 153, 153}
\definecolor{mcrouge}{RGB}{255, 0, 0}
\definecolor{mclightvert}{RGB}{205, 234, 190}

\definecolor{gris}{RGB}{220, 220, 220}
\definecolor{bleu}{RGB}{52, 152, 219}
\definecolor{vert}{RGB}{125, 194, 70}
\definecolor{mauve}{RGB}{154, 0, 215}
\definecolor{orange}{RGB}{255, 96, 0}
\definecolor{turquoise}{RGB}{0, 153, 153}
\definecolor{rouge}{RGB}{255, 0, 0}
\definecolor{lightvert}{RGB}{205, 234, 190}
\setitemize[0]{label=\color{lightvert}  $\bullet$}

\pagestyle{fancy}
\renewcommand{\headrulewidth}{0.2pt}
\fancyhead[L]{maths-cours.fr}
\fancyhead[R]{\thepage}
\renewcommand{\footrulewidth}{0.2pt}
\fancyfoot[C]{}

\newcolumntype{C}{>{\centering\arraybackslash}X}
\newcolumntype{s}{>{\hsize=.35\hsize\arraybackslash}X}

\setlength{\parindent}{0pt}		 
\setlength{\parskip}{3mm}
\setlength{\headheight}{1cm}

\def\ebook{ebook}
\def\book{book}
\def\web{web}
\def\type{web}

\newcommand{\vect}[1]{\overrightarrow{\,\mathstrut#1\,}}

\def\Oij{$\left(\text{O}~;~\vect{\imath},~\vect{\jmath}\right)$}
\def\Oijk{$\left(\text{O}~;~\vect{\imath},~\vect{\jmath},~\vect{k}\right)$}
\def\Ouv{$\left(\text{O}~;~\vect{u},~\vect{v}\right)$}

\hypersetup{breaklinks=true, colorlinks = true, linkcolor = OliveGreen, urlcolor = OliveGreen, citecolor = OliveGreen, pdfauthor={Didier BONNEL - https://www.maths-cours.fr} } % supprime les bordures autour des liens

\renewcommand{\arg}[0]{\text{arg}}

\everymath{\displaystyle}

%================================================================================================================================
%
% Macros - Commandes
%
%================================================================================================================================

\newcommand\meta[2]{    			% Utilisé pour créer le post HTML.
	\def\titre{titre}
	\def\url{url}
	\def\arg{#1}
	\ifx\titre\arg
		\newcommand\maintitle{#2}
		\fancyhead[L]{#2}
		{\Large\sffamily \MakeUppercase{#2}}
		\vspace{1mm}\textcolor{mcvert}{\hrule}
	\fi 
	\ifx\url\arg
		\fancyfoot[L]{\href{https://www.maths-cours.fr#2}{\black \footnotesize{https://www.maths-cours.fr#2}}}
	\fi 
}


\newcommand\TitreC[1]{    		% Titre centré
     \needspace{3\baselineskip}
     \begin{center}\textbf{#1}\end{center}
}

\newcommand\newpar{    		% paragraphe
     \par
}

\newcommand\nosp {    		% commande vide (pas d'espace)
}
\newcommand{\id}[1]{} %ignore

\newcommand\boite[2]{				% Boite simple sans titre
	\vspace{5mm}
	\setlength{\fboxrule}{0.2mm}
	\setlength{\fboxsep}{5mm}	
	\fcolorbox{#1}{#1!3}{\makebox[\linewidth-2\fboxrule-2\fboxsep]{
  		\begin{minipage}[t]{\linewidth-2\fboxrule-4\fboxsep}\setlength{\parskip}{3mm}
  			 #2
  		\end{minipage}
	}}
	\vspace{5mm}
}

\newcommand\CBox[4]{				% Boites
	\vspace{5mm}
	\setlength{\fboxrule}{0.2mm}
	\setlength{\fboxsep}{5mm}
	
	\fcolorbox{#1}{#1!3}{\makebox[\linewidth-2\fboxrule-2\fboxsep]{
		\begin{minipage}[t]{1cm}\setlength{\parskip}{3mm}
	  		\textcolor{#1}{\LARGE{#2}}    
 	 	\end{minipage}  
  		\begin{minipage}[t]{\linewidth-2\fboxrule-4\fboxsep}\setlength{\parskip}{3mm}
			\raisebox{1.2mm}{\normalsize\sffamily{\textcolor{#1}{#3}}}						
  			 #4
  		\end{minipage}
	}}
	\vspace{5mm}
}

\newcommand\cadre[3]{				% Boites convertible html
	\par
	\vspace{2mm}
	\setlength{\fboxrule}{0.1mm}
	\setlength{\fboxsep}{5mm}
	\fcolorbox{#1}{white}{\makebox[\linewidth-2\fboxrule-2\fboxsep]{
  		\begin{minipage}[t]{\linewidth-2\fboxrule-4\fboxsep}\setlength{\parskip}{3mm}
			\raisebox{-2.5mm}{\sffamily \small{\textcolor{#1}{\MakeUppercase{#2}}}}		
			\par		
  			 #3
 	 		\end{minipage}
	}}
		\vspace{2mm}
	\par
}

\newcommand\bloc[3]{				% Boites convertible html sans bordure
     \needspace{2\baselineskip}
     {\sffamily \small{\textcolor{#1}{\MakeUppercase{#2}}}}    
		\par		
  			 #3
		\par
}

\newcommand\CHelp[1]{
     \CBox{Plum}{\faInfoCircle}{À RETENIR}{#1}
}

\newcommand\CUp[1]{
     \CBox{NavyBlue}{\faThumbsOUp}{EN PRATIQUE}{#1}
}

\newcommand\CInfo[1]{
     \CBox{Sepia}{\faArrowCircleRight}{REMARQUE}{#1}
}

\newcommand\CRedac[1]{
     \CBox{PineGreen}{\faEdit}{BIEN R\'EDIGER}{#1}
}

\newcommand\CError[1]{
     \CBox{Red}{\faExclamationTriangle}{ATTENTION}{#1}
}

\newcommand\TitreExo[2]{
\needspace{4\baselineskip}
 {\sffamily\large EXERCICE #1\ (\emph{#2 points})}
\vspace{5mm}
}

\newcommand\img[2]{
          \includegraphics[width=#2\paperwidth]{\imgdir#1}
}

\newcommand\imgsvg[2]{
       \begin{center}   \includegraphics[width=#2\paperwidth]{\imgsvgdir#1} \end{center}
}


\newcommand\Lien[2]{
     \href{#1}{#2 \tiny \faExternalLink}
}
\newcommand\mcLien[2]{
     \href{https~://www.maths-cours.fr/#1}{#2 \tiny \faExternalLink}
}

\newcommand{\euro}{\eurologo{}}

%================================================================================================================================
%
% Macros - Environement
%
%================================================================================================================================

\newenvironment{tex}{ %
}
{%
}

\newenvironment{indente}{ %
	\setlength\parindent{10mm}
}

{
	\setlength\parindent{0mm}
}

\newenvironment{corrige}{%
     \needspace{3\baselineskip}
     \medskip
     \textbf{\textsc{Corrigé}}
     \medskip
}
{
}

\newenvironment{extern}{%
     \begin{center}
     }
     {
     \end{center}
}

\NewEnviron{code}{%
	\par
     \boite{gray}{\texttt{%
     \BODY
     }}
     \par
}

\newenvironment{vbloc}{% boite sans cadre empeche saut de page
     \begin{minipage}[t]{\linewidth}
     }
     {
     \end{minipage}
}
\NewEnviron{h2}{%
    \needspace{3\baselineskip}
    \vspace{0.6cm}
	\noindent \MakeUppercase{\sffamily \large \BODY}
	\vspace{1mm}\textcolor{mcgris}{\hrule}\vspace{0.4cm}
	\par
}{}

\NewEnviron{h3}{%
    \needspace{3\baselineskip}
	\vspace{5mm}
	\textsc{\BODY}
	\par
}

\NewEnviron{margeneg}{ %
\begin{addmargin}[-1cm]{0cm}
\BODY
\end{addmargin}
}

\NewEnviron{html}{%
}

\begin{document}
\meta{url}{/exercices/geometrie-dans-lespace-bac-s-liban-2018/}
\meta{pid}{7729}
\meta{titre}{Géométrie dans l'espace – Bac S Liban 2018}
\meta{type}{exercices}
%
\begin{h2}EXERCICE 3 (4 points)\end{h2}
\textbf{Commun à tous les candidats}
\medskip
L'objectif de cet exercice est d'étudier les trajectoires de deux sous-marins en phase de plongée.
\par
On considère que ces sous-marins se déplacent en ligne droite, chacun à vitesse constante.
\par
À chaque instant $t$, exprimé en minutes, le premier sous-marin est repéré par le point $S_1(t)$ et le second
sous-marin est repéré par le point $S_2(t)$ dans un repère orthonormé $\left(\text{O}~;~\vec{i},~\vec{j},~\vec{k}\right)$ dont l'unité est le mètre.
\par
Le plan défini par $\left(\text{O}~;~\vec{i},~\vec{j}\right)$ représente la surface de la mer. La cote $z$ est nulle au niveau de la
mer, négative sous l'eau.
\medskip
\begin{enumerate}
     \item On admet que, pour tout réel $t \geqslant 0$, le point $S_1(t)$ a pour coordonnées:
     \begin{center}
          $\begin{cases}
               x(t) = \phantom{-}140 - 60t\\
               y(t) = \phantom{-}105 - 90t\\
               z(t) = -170 - 30 t
          \end{cases}$
     \end{center}
     \begin{enumerate}[label=\alph*.]
          \item Donner les coordonnées du sous- marin au début de l'observation.
          \item Quelle est la vitesse du sous-marin ?
          \item  On se place dans le plan vertical contenant la trajectoire du premier sous-marin.
          \begin{center}
               \begin{extern}%width="420"
                    \psset{xunit=1.0cm,yunit=1.0cm,algebraic=true,dimen=middle,dotstyle=o,dotsize=5pt 0,linewidth=1.6pt,arrowsize=3pt 2,arrowinset=0.25}
                    \begin{pspicture*}(0,-3)(7,3)
                         \psscalebox{0.5}{
                              \psplot[linewidth=0.75pt,linecolor=blue]{-0.38}{13.42}{(--46.12-0.02*x)/11.48}
                              \pscustom[linewidth=0.75pt,linecolor=blue]{
                                   \parametricplot{-0.6324}{0.}{0.6*cos(t)+1.|0.6*sin(t)+4.}
                              \lineto(1.,4.)\closepath}
                              \psline[linewidth=1pt,linestyle=dashed,dash=2pt 2pt](1.,4.)(12,-4.)
                              \rput[tl](9.22,4.5){\fontsize{15pt}{15pt}\selectfont{\blue{niveau de la mer}}}
                              \rput[tl](1.82,3.78){$\blue{\alpha}$}
                              \def\sm{\pscustom[linestyle=none,fillstyle=solid,fillcolor=black]
                                   {
                                        \newpath
                                        \moveto(0.04237557,31.8862686)
                                        \curveto(-0.14149443,31.65805592)(0.28549936,31.08408035)(1.08023317,30.56566562)
                                        \curveto(1.81265886,29.99169006)(2.48583077,28.96037077)(2.48583077,28.21420253)
                                        \curveto(2.60739266,26.3193944)(3.28056457,25.45843106)(4.62385406,25.45843106)
                                        \curveto(5.23471786,25.45843106)(5.84558166,25.11289777)(6.02945167,24.71157406)
                                        \curveto(6.57800736,23.39097108)(33.57757641,3.76146599)(37.48893731,1.81132661)
                                        \curveto(40.6025101,0.31761261)(41.33799011,0.14496076)(42.12966959,0.71893632)
                                        \curveto(43.47601341,1.75255152)(43.35139719,3.18749043)(41.8865458,4.96819221)
                                        \curveto(41.21337389,5.77038046)(40.6648182,6.6901189)(40.6648182,7.03266752)
                                        \curveto(40.6648182,7.37820081)(40.42169441,7.66518859)(40.11626251,7.66518859)
                                        \curveto(39.81083061,7.66518859)(38.1590549,8.69903338)(36.45107972,9.96109084)
                                        \curveto(33.76144641,11.97000532)(33.39676072,12.42987454)(33.94531641,13.40494423)
                                        \curveto(34.74005022,14.78133763)(33.76144641,16.84879761)(32.41815692,16.84879761)
                                        \curveto(31.92885501,16.84879761)(31.31799122,17.30866683)(30.95330553,17.88264239)
                                        \curveto(30.52325741,18.5707243)(29.66926982,18.85771208)(27.59355463,18.79916657)
                                        \curveto(25.08779133,18.74360574)(24.47692753,19.03059352)(19.46845524,22.93362735)
                                        \curveto(16.41413625,25.34432472)(12.80820724,27.58145187)(11.22179396,28.15542744)
                                        \curveto(8.65677686,29.01639078)(8.35134497,29.30337856)(8.65677686,30.45132969)
                                        \curveto(9.14302445,32.11723637)(7.37274116,32.92011338)(6.27257546,31.5432608)
                                        \curveto(5.54014976,30.62421113)(5.23471786,30.62421113)(2.97513267,31.37106813)
                                        \curveto(1.56953507,31.83024858)(0.22624557,32.05915004)(0.04237557,31.8862686)
                                        \closepath
                              }}
                              \rput[tl](9.25,-3){\psscalebox{0.03}{\sm}}
                         }
                    \end{pspicture*}
               \end{extern}
          \end{center}
          Déterminer l'angle $\alpha$ que forme la trajectoire du sous-marin avec le plan horizontal.\\
          On donnera l'arrondi de $\alpha$ à $0,1$ degré près.
     \end{enumerate}
     \item  Au début de l'observation, le second sous-marin est situé au point $S_2(0)$ de coordonnées
     $(68~;~135~;~- 68)$ et atteint au bout de trois minutes le point $S_2(3)$ de coordonnées $(-202~;~-405~;~ - 248)$ avec une vitesse constante.
     \par
     \`A quel instant $t$, exprimé en minutes, les deux sous-marins sont-ils à la même profondeur ?
\end{enumerate}

\end{document}
µ
\documentclass[a4paper]{article}

%================================================================================================================================
%
% Packages
%
%================================================================================================================================

\usepackage[T1]{fontenc} 	% pour caractères accentués
\usepackage[utf8]{inputenc}  % encodage utf8
\usepackage[french]{babel}	% langue : français
\usepackage{fourier}			% caractères plus lisibles
\usepackage[dvipsnames]{xcolor} % couleurs
\usepackage{fancyhdr}		% réglage header footer
\usepackage{needspace}		% empêcher sauts de page mal placés
\usepackage{graphicx}		% pour inclure des graphiques
\usepackage{enumitem,cprotect}		% personnalise les listes d'items (nécessaire pour ol, al ...)
\usepackage{hyperref}		% Liens hypertexte
\usepackage{pstricks,pst-all,pst-node,pstricks-add,pst-math,pst-plot,pst-tree,pst-eucl} % pstricks
\usepackage[a4paper,includeheadfoot,top=2cm,left=3cm, bottom=2cm,right=3cm]{geometry} % marges etc.
\usepackage{comment}			% commentaires multilignes
\usepackage{amsmath,environ} % maths (matrices, etc.)
\usepackage{amssymb,makeidx}
\usepackage{bm}				% bold maths
\usepackage{tabularx}		% tableaux
\usepackage{colortbl}		% tableaux en couleur
\usepackage{fontawesome}		% Fontawesome
\usepackage{environ}			% environment with command
\usepackage{fp}				% calculs pour ps-tricks
\usepackage{multido}			% pour ps tricks
\usepackage[np]{numprint}	% formattage nombre
\usepackage{tikz,tkz-tab} 			% package principal TikZ
\usepackage{pgfplots}   % axes
\usepackage{mathrsfs}    % cursives
\usepackage{calc}			% calcul taille boites
\usepackage[scaled=0.875]{helvet} % font sans serif
\usepackage{svg} % svg
\usepackage{scrextend} % local margin
\usepackage{scratch} %scratch
\usepackage{multicol} % colonnes
%\usepackage{infix-RPN,pst-func} % formule en notation polanaise inversée
\usepackage{listings}

%================================================================================================================================
%
% Réglages de base
%
%================================================================================================================================

\lstset{
language=Python,   % R code
literate=
{á}{{\'a}}1
{à}{{\`a}}1
{ã}{{\~a}}1
{é}{{\'e}}1
{è}{{\`e}}1
{ê}{{\^e}}1
{í}{{\'i}}1
{ó}{{\'o}}1
{õ}{{\~o}}1
{ú}{{\'u}}1
{ü}{{\"u}}1
{ç}{{\c{c}}}1
{~}{{ }}1
}


\definecolor{codegreen}{rgb}{0,0.6,0}
\definecolor{codegray}{rgb}{0.5,0.5,0.5}
\definecolor{codepurple}{rgb}{0.58,0,0.82}
\definecolor{backcolour}{rgb}{0.95,0.95,0.92}

\lstdefinestyle{mystyle}{
    backgroundcolor=\color{backcolour},   
    commentstyle=\color{codegreen},
    keywordstyle=\color{magenta},
    numberstyle=\tiny\color{codegray},
    stringstyle=\color{codepurple},
    basicstyle=\ttfamily\footnotesize,
    breakatwhitespace=false,         
    breaklines=true,                 
    captionpos=b,                    
    keepspaces=true,                 
    numbers=left,                    
xleftmargin=2em,
framexleftmargin=2em,            
    showspaces=false,                
    showstringspaces=false,
    showtabs=false,                  
    tabsize=2,
    upquote=true
}

\lstset{style=mystyle}


\lstset{style=mystyle}
\newcommand{\imgdir}{C:/laragon/www/newmc/assets/imgsvg/}
\newcommand{\imgsvgdir}{C:/laragon/www/newmc/assets/imgsvg/}

\definecolor{mcgris}{RGB}{220, 220, 220}% ancien~; pour compatibilité
\definecolor{mcbleu}{RGB}{52, 152, 219}
\definecolor{mcvert}{RGB}{125, 194, 70}
\definecolor{mcmauve}{RGB}{154, 0, 215}
\definecolor{mcorange}{RGB}{255, 96, 0}
\definecolor{mcturquoise}{RGB}{0, 153, 153}
\definecolor{mcrouge}{RGB}{255, 0, 0}
\definecolor{mclightvert}{RGB}{205, 234, 190}

\definecolor{gris}{RGB}{220, 220, 220}
\definecolor{bleu}{RGB}{52, 152, 219}
\definecolor{vert}{RGB}{125, 194, 70}
\definecolor{mauve}{RGB}{154, 0, 215}
\definecolor{orange}{RGB}{255, 96, 0}
\definecolor{turquoise}{RGB}{0, 153, 153}
\definecolor{rouge}{RGB}{255, 0, 0}
\definecolor{lightvert}{RGB}{205, 234, 190}
\setitemize[0]{label=\color{lightvert}  $\bullet$}

\pagestyle{fancy}
\renewcommand{\headrulewidth}{0.2pt}
\fancyhead[L]{maths-cours.fr}
\fancyhead[R]{\thepage}
\renewcommand{\footrulewidth}{0.2pt}
\fancyfoot[C]{}

\newcolumntype{C}{>{\centering\arraybackslash}X}
\newcolumntype{s}{>{\hsize=.35\hsize\arraybackslash}X}

\setlength{\parindent}{0pt}		 
\setlength{\parskip}{3mm}
\setlength{\headheight}{1cm}

\def\ebook{ebook}
\def\book{book}
\def\web{web}
\def\type{web}

\newcommand{\vect}[1]{\overrightarrow{\,\mathstrut#1\,}}

\def\Oij{$\left(\text{O}~;~\vect{\imath},~\vect{\jmath}\right)$}
\def\Oijk{$\left(\text{O}~;~\vect{\imath},~\vect{\jmath},~\vect{k}\right)$}
\def\Ouv{$\left(\text{O}~;~\vect{u},~\vect{v}\right)$}

\hypersetup{breaklinks=true, colorlinks = true, linkcolor = OliveGreen, urlcolor = OliveGreen, citecolor = OliveGreen, pdfauthor={Didier BONNEL - https://www.maths-cours.fr} } % supprime les bordures autour des liens

\renewcommand{\arg}[0]{\text{arg}}

\everymath{\displaystyle}

%================================================================================================================================
%
% Macros - Commandes
%
%================================================================================================================================

\newcommand\meta[2]{    			% Utilisé pour créer le post HTML.
	\def\titre{titre}
	\def\url{url}
	\def\arg{#1}
	\ifx\titre\arg
		\newcommand\maintitle{#2}
		\fancyhead[L]{#2}
		{\Large\sffamily \MakeUppercase{#2}}
		\vspace{1mm}\textcolor{mcvert}{\hrule}
	\fi 
	\ifx\url\arg
		\fancyfoot[L]{\href{https://www.maths-cours.fr#2}{\black \footnotesize{https://www.maths-cours.fr#2}}}
	\fi 
}


\newcommand\TitreC[1]{    		% Titre centré
     \needspace{3\baselineskip}
     \begin{center}\textbf{#1}\end{center}
}

\newcommand\newpar{    		% paragraphe
     \par
}

\newcommand\nosp {    		% commande vide (pas d'espace)
}
\newcommand{\id}[1]{} %ignore

\newcommand\boite[2]{				% Boite simple sans titre
	\vspace{5mm}
	\setlength{\fboxrule}{0.2mm}
	\setlength{\fboxsep}{5mm}	
	\fcolorbox{#1}{#1!3}{\makebox[\linewidth-2\fboxrule-2\fboxsep]{
  		\begin{minipage}[t]{\linewidth-2\fboxrule-4\fboxsep}\setlength{\parskip}{3mm}
  			 #2
  		\end{minipage}
	}}
	\vspace{5mm}
}

\newcommand\CBox[4]{				% Boites
	\vspace{5mm}
	\setlength{\fboxrule}{0.2mm}
	\setlength{\fboxsep}{5mm}
	
	\fcolorbox{#1}{#1!3}{\makebox[\linewidth-2\fboxrule-2\fboxsep]{
		\begin{minipage}[t]{1cm}\setlength{\parskip}{3mm}
	  		\textcolor{#1}{\LARGE{#2}}    
 	 	\end{minipage}  
  		\begin{minipage}[t]{\linewidth-2\fboxrule-4\fboxsep}\setlength{\parskip}{3mm}
			\raisebox{1.2mm}{\normalsize\sffamily{\textcolor{#1}{#3}}}						
  			 #4
  		\end{minipage}
	}}
	\vspace{5mm}
}

\newcommand\cadre[3]{				% Boites convertible html
	\par
	\vspace{2mm}
	\setlength{\fboxrule}{0.1mm}
	\setlength{\fboxsep}{5mm}
	\fcolorbox{#1}{white}{\makebox[\linewidth-2\fboxrule-2\fboxsep]{
  		\begin{minipage}[t]{\linewidth-2\fboxrule-4\fboxsep}\setlength{\parskip}{3mm}
			\raisebox{-2.5mm}{\sffamily \small{\textcolor{#1}{\MakeUppercase{#2}}}}		
			\par		
  			 #3
 	 		\end{minipage}
	}}
		\vspace{2mm}
	\par
}

\newcommand\bloc[3]{				% Boites convertible html sans bordure
     \needspace{2\baselineskip}
     {\sffamily \small{\textcolor{#1}{\MakeUppercase{#2}}}}    
		\par		
  			 #3
		\par
}

\newcommand\CHelp[1]{
     \CBox{Plum}{\faInfoCircle}{À RETENIR}{#1}
}

\newcommand\CUp[1]{
     \CBox{NavyBlue}{\faThumbsOUp}{EN PRATIQUE}{#1}
}

\newcommand\CInfo[1]{
     \CBox{Sepia}{\faArrowCircleRight}{REMARQUE}{#1}
}

\newcommand\CRedac[1]{
     \CBox{PineGreen}{\faEdit}{BIEN R\'EDIGER}{#1}
}

\newcommand\CError[1]{
     \CBox{Red}{\faExclamationTriangle}{ATTENTION}{#1}
}

\newcommand\TitreExo[2]{
\needspace{4\baselineskip}
 {\sffamily\large EXERCICE #1\ (\emph{#2 points})}
\vspace{5mm}
}

\newcommand\img[2]{
          \includegraphics[width=#2\paperwidth]{\imgdir#1}
}

\newcommand\imgsvg[2]{
       \begin{center}   \includegraphics[width=#2\paperwidth]{\imgsvgdir#1} \end{center}
}


\newcommand\Lien[2]{
     \href{#1}{#2 \tiny \faExternalLink}
}
\newcommand\mcLien[2]{
     \href{https~://www.maths-cours.fr/#1}{#2 \tiny \faExternalLink}
}

\newcommand{\euro}{\eurologo{}}

%================================================================================================================================
%
% Macros - Environement
%
%================================================================================================================================

\newenvironment{tex}{ %
}
{%
}

\newenvironment{indente}{ %
	\setlength\parindent{10mm}
}

{
	\setlength\parindent{0mm}
}

\newenvironment{corrige}{%
     \needspace{3\baselineskip}
     \medskip
     \textbf{\textsc{Corrigé}}
     \medskip
}
{
}

\newenvironment{extern}{%
     \begin{center}
     }
     {
     \end{center}
}

\NewEnviron{code}{%
	\par
     \boite{gray}{\texttt{%
     \BODY
     }}
     \par
}

\newenvironment{vbloc}{% boite sans cadre empeche saut de page
     \begin{minipage}[t]{\linewidth}
     }
     {
     \end{minipage}
}
\NewEnviron{h2}{%
    \needspace{3\baselineskip}
    \vspace{0.6cm}
	\noindent \MakeUppercase{\sffamily \large \BODY}
	\vspace{1mm}\textcolor{mcgris}{\hrule}\vspace{0.4cm}
	\par
}{}

\NewEnviron{h3}{%
    \needspace{3\baselineskip}
	\vspace{5mm}
	\textsc{\BODY}
	\par
}

\NewEnviron{margeneg}{ %
\begin{addmargin}[-1cm]{0cm}
\BODY
\end{addmargin}
}

\NewEnviron{html}{%
}

\begin{document}
\meta{url}{/exercices/suite-de-fonctions-bac-s-liban-2018/}
\meta{pid}{7740}
\meta{titre}{Suite de fonctions – Bac S Liban 2018}
\meta{type}{exercices}
%
\begin{h2}EXERCICE 4 (5 points)\end{h2}
\textbf{Commun à tous les candidats}
\medskip
On considère, pour tout entier $n > 0$, les fonctions $f_n$ définies sur l'intervalle [1~;~5] par:
\[f_n(x) = \dfrac{\ln x}{x^n}.\]
\par
Pour tout entier $n > 0$, on note $\mathscr{C}_n$ la courbe représentative de la fonction $f_n$ dans un repère orthogonal.
\par
Sur le graphique ci-dessous sont représentées les courbes $\mathscr{C}_n$ pour $n$ appartenant à
$\{1~;~2~;~3~;~4\}$.
\begin{center}
     \begin{extern}
          \psset{xunit=2cm,yunit=5cm,comma=true}
          \begin{pspicture}(-0.2,-0.1)(5.2,0.65)
               \psaxes[linewidth=1pt,Dy=0.5]{->}(0,0)(0,0)(5.3,0.6)
               \psplot[plotpoints=300,linewidth=0.75pt]{1}{5}{x ln x div}
               \psplot[plotpoints=300,linewidth=0.75pt]{1}{5}{x ln x dup mul div}
               \psplot[plotpoints=300,linewidth=0.75pt]{1}{5}{x ln x 3 exp  div}
               \psplot[plotpoints=300,linewidth=0.75pt]{1}{5}{x ln x 4 exp div}
          \end{pspicture}
     \end{extern}
\end{center}
\medskip
\begin{enumerate}
     \item Montrer que, pour tout entier $n > 0$ et tout réel $x$ de l'intervalle [1~;~5] :
     \[f'_n(x) = \dfrac{1- n\ln (x)}{x^{n+1}}.\]
     \item  Pour tout entier $n > 0$, on admet que la fonction $f_n$ admet un maximum sur l'intervalle [1~;~5].
     \par
     On note $A_n$ le point de la courbe $\mathscr{C}_n$ ayant pour ordonnée ce maximum.
     \par
     Montrer que tous les points $A_n$ appartiennent à une même courbe $\Gamma$ d'équation
     \[y = \dfrac{1}{\text{e}} \ln (x).\]
     \item
     \begin{enumerate}[label=\alph*.]
          \item Montrer que, pour tout entier $n > 1$ et tout réel $x$ de l'intervalle [1~;~5] :
          \[0 \leqslant \dfrac{\ln (x)}{x^n} \leqslant \dfrac{\ln (5)}{x^n}.\]
          \item  Montrer que pour tout entier $n > 1$ :
          \[\displaystyle\int_1^5 \dfrac{1}{x^n} \:\text{d}x = \dfrac{1}{n - 1}\left(1 - \dfrac{1}{5^{n - 1}} \right).\]
          \item  Pour tout entier $n > 0$, on s'intéresse à l'aire, exprimée en unités d'aire, de la surface sous la courbe $f_n$,
          c'est-à-dire l'aire du domaine du plan délimité par les droites d'équations $x = 1$, $x = 5$, $y = 0$
          et la courbe $\mathscr{C}_n$.
          \par
          Déterminer la valeur limite de cette aire quand $n$ tend vers $+ \infty$.
     \end{enumerate}
\end{enumerate}

\end{document}
µ
\documentclass[a4paper]{article}

%================================================================================================================================
%
% Packages
%
%================================================================================================================================

\usepackage[T1]{fontenc} 	% pour caractères accentués
\usepackage[utf8]{inputenc}  % encodage utf8
\usepackage[french]{babel}	% langue : français
\usepackage{fourier}			% caractères plus lisibles
\usepackage[dvipsnames]{xcolor} % couleurs
\usepackage{fancyhdr}		% réglage header footer
\usepackage{needspace}		% empêcher sauts de page mal placés
\usepackage{graphicx}		% pour inclure des graphiques
\usepackage{enumitem,cprotect}		% personnalise les listes d'items (nécessaire pour ol, al ...)
\usepackage{hyperref}		% Liens hypertexte
\usepackage{pstricks,pst-all,pst-node,pstricks-add,pst-math,pst-plot,pst-tree,pst-eucl} % pstricks
\usepackage[a4paper,includeheadfoot,top=2cm,left=3cm, bottom=2cm,right=3cm]{geometry} % marges etc.
\usepackage{comment}			% commentaires multilignes
\usepackage{amsmath,environ} % maths (matrices, etc.)
\usepackage{amssymb,makeidx}
\usepackage{bm}				% bold maths
\usepackage{tabularx}		% tableaux
\usepackage{colortbl}		% tableaux en couleur
\usepackage{fontawesome}		% Fontawesome
\usepackage{environ}			% environment with command
\usepackage{fp}				% calculs pour ps-tricks
\usepackage{multido}			% pour ps tricks
\usepackage[np]{numprint}	% formattage nombre
\usepackage{tikz,tkz-tab} 			% package principal TikZ
\usepackage{pgfplots}   % axes
\usepackage{mathrsfs}    % cursives
\usepackage{calc}			% calcul taille boites
\usepackage[scaled=0.875]{helvet} % font sans serif
\usepackage{svg} % svg
\usepackage{scrextend} % local margin
\usepackage{scratch} %scratch
\usepackage{multicol} % colonnes
%\usepackage{infix-RPN,pst-func} % formule en notation polanaise inversée
\usepackage{listings}

%================================================================================================================================
%
% Réglages de base
%
%================================================================================================================================

\lstset{
language=Python,   % R code
literate=
{á}{{\'a}}1
{à}{{\`a}}1
{ã}{{\~a}}1
{é}{{\'e}}1
{è}{{\`e}}1
{ê}{{\^e}}1
{í}{{\'i}}1
{ó}{{\'o}}1
{õ}{{\~o}}1
{ú}{{\'u}}1
{ü}{{\"u}}1
{ç}{{\c{c}}}1
{~}{{ }}1
}


\definecolor{codegreen}{rgb}{0,0.6,0}
\definecolor{codegray}{rgb}{0.5,0.5,0.5}
\definecolor{codepurple}{rgb}{0.58,0,0.82}
\definecolor{backcolour}{rgb}{0.95,0.95,0.92}

\lstdefinestyle{mystyle}{
    backgroundcolor=\color{backcolour},   
    commentstyle=\color{codegreen},
    keywordstyle=\color{magenta},
    numberstyle=\tiny\color{codegray},
    stringstyle=\color{codepurple},
    basicstyle=\ttfamily\footnotesize,
    breakatwhitespace=false,         
    breaklines=true,                 
    captionpos=b,                    
    keepspaces=true,                 
    numbers=left,                    
xleftmargin=2em,
framexleftmargin=2em,            
    showspaces=false,                
    showstringspaces=false,
    showtabs=false,                  
    tabsize=2,
    upquote=true
}

\lstset{style=mystyle}


\lstset{style=mystyle}
\newcommand{\imgdir}{C:/laragon/www/newmc/assets/imgsvg/}
\newcommand{\imgsvgdir}{C:/laragon/www/newmc/assets/imgsvg/}

\definecolor{mcgris}{RGB}{220, 220, 220}% ancien~; pour compatibilité
\definecolor{mcbleu}{RGB}{52, 152, 219}
\definecolor{mcvert}{RGB}{125, 194, 70}
\definecolor{mcmauve}{RGB}{154, 0, 215}
\definecolor{mcorange}{RGB}{255, 96, 0}
\definecolor{mcturquoise}{RGB}{0, 153, 153}
\definecolor{mcrouge}{RGB}{255, 0, 0}
\definecolor{mclightvert}{RGB}{205, 234, 190}

\definecolor{gris}{RGB}{220, 220, 220}
\definecolor{bleu}{RGB}{52, 152, 219}
\definecolor{vert}{RGB}{125, 194, 70}
\definecolor{mauve}{RGB}{154, 0, 215}
\definecolor{orange}{RGB}{255, 96, 0}
\definecolor{turquoise}{RGB}{0, 153, 153}
\definecolor{rouge}{RGB}{255, 0, 0}
\definecolor{lightvert}{RGB}{205, 234, 190}
\setitemize[0]{label=\color{lightvert}  $\bullet$}

\pagestyle{fancy}
\renewcommand{\headrulewidth}{0.2pt}
\fancyhead[L]{maths-cours.fr}
\fancyhead[R]{\thepage}
\renewcommand{\footrulewidth}{0.2pt}
\fancyfoot[C]{}

\newcolumntype{C}{>{\centering\arraybackslash}X}
\newcolumntype{s}{>{\hsize=.35\hsize\arraybackslash}X}

\setlength{\parindent}{0pt}		 
\setlength{\parskip}{3mm}
\setlength{\headheight}{1cm}

\def\ebook{ebook}
\def\book{book}
\def\web{web}
\def\type{web}

\newcommand{\vect}[1]{\overrightarrow{\,\mathstrut#1\,}}

\def\Oij{$\left(\text{O}~;~\vect{\imath},~\vect{\jmath}\right)$}
\def\Oijk{$\left(\text{O}~;~\vect{\imath},~\vect{\jmath},~\vect{k}\right)$}
\def\Ouv{$\left(\text{O}~;~\vect{u},~\vect{v}\right)$}

\hypersetup{breaklinks=true, colorlinks = true, linkcolor = OliveGreen, urlcolor = OliveGreen, citecolor = OliveGreen, pdfauthor={Didier BONNEL - https://www.maths-cours.fr} } % supprime les bordures autour des liens

\renewcommand{\arg}[0]{\text{arg}}

\everymath{\displaystyle}

%================================================================================================================================
%
% Macros - Commandes
%
%================================================================================================================================

\newcommand\meta[2]{    			% Utilisé pour créer le post HTML.
	\def\titre{titre}
	\def\url{url}
	\def\arg{#1}
	\ifx\titre\arg
		\newcommand\maintitle{#2}
		\fancyhead[L]{#2}
		{\Large\sffamily \MakeUppercase{#2}}
		\vspace{1mm}\textcolor{mcvert}{\hrule}
	\fi 
	\ifx\url\arg
		\fancyfoot[L]{\href{https://www.maths-cours.fr#2}{\black \footnotesize{https://www.maths-cours.fr#2}}}
	\fi 
}


\newcommand\TitreC[1]{    		% Titre centré
     \needspace{3\baselineskip}
     \begin{center}\textbf{#1}\end{center}
}

\newcommand\newpar{    		% paragraphe
     \par
}

\newcommand\nosp {    		% commande vide (pas d'espace)
}
\newcommand{\id}[1]{} %ignore

\newcommand\boite[2]{				% Boite simple sans titre
	\vspace{5mm}
	\setlength{\fboxrule}{0.2mm}
	\setlength{\fboxsep}{5mm}	
	\fcolorbox{#1}{#1!3}{\makebox[\linewidth-2\fboxrule-2\fboxsep]{
  		\begin{minipage}[t]{\linewidth-2\fboxrule-4\fboxsep}\setlength{\parskip}{3mm}
  			 #2
  		\end{minipage}
	}}
	\vspace{5mm}
}

\newcommand\CBox[4]{				% Boites
	\vspace{5mm}
	\setlength{\fboxrule}{0.2mm}
	\setlength{\fboxsep}{5mm}
	
	\fcolorbox{#1}{#1!3}{\makebox[\linewidth-2\fboxrule-2\fboxsep]{
		\begin{minipage}[t]{1cm}\setlength{\parskip}{3mm}
	  		\textcolor{#1}{\LARGE{#2}}    
 	 	\end{minipage}  
  		\begin{minipage}[t]{\linewidth-2\fboxrule-4\fboxsep}\setlength{\parskip}{3mm}
			\raisebox{1.2mm}{\normalsize\sffamily{\textcolor{#1}{#3}}}						
  			 #4
  		\end{minipage}
	}}
	\vspace{5mm}
}

\newcommand\cadre[3]{				% Boites convertible html
	\par
	\vspace{2mm}
	\setlength{\fboxrule}{0.1mm}
	\setlength{\fboxsep}{5mm}
	\fcolorbox{#1}{white}{\makebox[\linewidth-2\fboxrule-2\fboxsep]{
  		\begin{minipage}[t]{\linewidth-2\fboxrule-4\fboxsep}\setlength{\parskip}{3mm}
			\raisebox{-2.5mm}{\sffamily \small{\textcolor{#1}{\MakeUppercase{#2}}}}		
			\par		
  			 #3
 	 		\end{minipage}
	}}
		\vspace{2mm}
	\par
}

\newcommand\bloc[3]{				% Boites convertible html sans bordure
     \needspace{2\baselineskip}
     {\sffamily \small{\textcolor{#1}{\MakeUppercase{#2}}}}    
		\par		
  			 #3
		\par
}

\newcommand\CHelp[1]{
     \CBox{Plum}{\faInfoCircle}{À RETENIR}{#1}
}

\newcommand\CUp[1]{
     \CBox{NavyBlue}{\faThumbsOUp}{EN PRATIQUE}{#1}
}

\newcommand\CInfo[1]{
     \CBox{Sepia}{\faArrowCircleRight}{REMARQUE}{#1}
}

\newcommand\CRedac[1]{
     \CBox{PineGreen}{\faEdit}{BIEN R\'EDIGER}{#1}
}

\newcommand\CError[1]{
     \CBox{Red}{\faExclamationTriangle}{ATTENTION}{#1}
}

\newcommand\TitreExo[2]{
\needspace{4\baselineskip}
 {\sffamily\large EXERCICE #1\ (\emph{#2 points})}
\vspace{5mm}
}

\newcommand\img[2]{
          \includegraphics[width=#2\paperwidth]{\imgdir#1}
}

\newcommand\imgsvg[2]{
       \begin{center}   \includegraphics[width=#2\paperwidth]{\imgsvgdir#1} \end{center}
}


\newcommand\Lien[2]{
     \href{#1}{#2 \tiny \faExternalLink}
}
\newcommand\mcLien[2]{
     \href{https~://www.maths-cours.fr/#1}{#2 \tiny \faExternalLink}
}

\newcommand{\euro}{\eurologo{}}

%================================================================================================================================
%
% Macros - Environement
%
%================================================================================================================================

\newenvironment{tex}{ %
}
{%
}

\newenvironment{indente}{ %
	\setlength\parindent{10mm}
}

{
	\setlength\parindent{0mm}
}

\newenvironment{corrige}{%
     \needspace{3\baselineskip}
     \medskip
     \textbf{\textsc{Corrigé}}
     \medskip
}
{
}

\newenvironment{extern}{%
     \begin{center}
     }
     {
     \end{center}
}

\NewEnviron{code}{%
	\par
     \boite{gray}{\texttt{%
     \BODY
     }}
     \par
}

\newenvironment{vbloc}{% boite sans cadre empeche saut de page
     \begin{minipage}[t]{\linewidth}
     }
     {
     \end{minipage}
}
\NewEnviron{h2}{%
    \needspace{3\baselineskip}
    \vspace{0.6cm}
	\noindent \MakeUppercase{\sffamily \large \BODY}
	\vspace{1mm}\textcolor{mcgris}{\hrule}\vspace{0.4cm}
	\par
}{}

\NewEnviron{h3}{%
    \needspace{3\baselineskip}
	\vspace{5mm}
	\textsc{\BODY}
	\par
}

\NewEnviron{margeneg}{ %
\begin{addmargin}[-1cm]{0cm}
\BODY
\end{addmargin}
}

\NewEnviron{html}{%
}

\begin{document}
\meta{url}{/exercices/suites-et-probabilites-bac-s-liban-2018/}
\meta{pid}{7752}
\meta{titre}{Suites et probabilités - Bac S Liban 2018}
\meta{type}{exercices}
%
\begin{h2}EXERCICE 5 (5 points)\end{h2}
\textbf{Candidats n'ayant pas suivi l'enseignement de spécialité}
\medskip
Un jeu de hasard sur ordinateur est paramétré de la façon suivante :
\medskip
\begin{itemize}
     \item  Si le joueur gagne une partie, la probabilité qu'il gagne la partie suivante est
     $\dfrac{1}{4}$ ;
     \item Si le joueur perd une partie, la probabilité qu'il perde la partie suivante est $\dfrac{1}{2}$ ;
     \item La probabilité de gagner la première partie est $\dfrac{1}{4}$ .
\end{itemize}
Pour tout entier naturel $n$ non nul, on note $G_n$ l'événement \og la $n^{\text{e}}$ partie est gagnée \fg{} et on note $p_n$ la probabilité de cet événement. On a donc $p_1 = \dfrac{1}{4}$.
\medskip
\begin{enumerate}
     \item Montrer que $p_2 = \dfrac{7}{16}$.
     \item Montrer que, pour tout entier naturel $n$ non nul, $p_{n+1} = - \dfrac{1}{4}p_n + \dfrac{1}{2}$.
     \item  On obtient ainsi les premières valeurs de $p_n$ :
     \begin{center}
          \begin{extern}%width="570"
               \begin{tabularx}{0.8\linewidth}{|c|*{7}{>{\centering \arraybackslash}X|}}\hline
                    $n$ &1 &2 &3 &4 &5 &6 &7\\ \hline
                    $p_n$& 1 &\np{0,4375} &\np{0,3906} &\np{0,4023} &\np{0,3994} &\np{0,4001} &\np{0,3999}\\ \hline
               \end{tabularx}
          \end{extern}
     \end{center}
     \par
     Quelle conjecture peut -on émettre ?
     \item  On définit, pour tout entier naturel $n$ non nul, la suite $\left(u_n\right)$ par $u_n = p_n - \dfrac{2}{5}$.
     \begin{enumerate}[label=\alph*.]
          \item Démontrer que la suite $\left(u_n\right)$ est une suite géométrique dont on précisera la raison.
          \item En déduire que, pour tout entier naturel $n$ non nul, $p_n = \dfrac{2}{5} - \dfrac{3}{20}\left(- \dfrac{1}{4}\right)^{n-1}$.
          \item La suite $\left(p_n\right)$ converge-t-elle ? Interpréter ce résultat.
     \end{enumerate}
\end{enumerate}

\end{document}
µ
\documentclass[a4paper]{article}

%================================================================================================================================
%
% Packages
%
%================================================================================================================================

\usepackage[T1]{fontenc} 	% pour caractères accentués
\usepackage[utf8]{inputenc}  % encodage utf8
\usepackage[french]{babel}	% langue : français
\usepackage{fourier}			% caractères plus lisibles
\usepackage[dvipsnames]{xcolor} % couleurs
\usepackage{fancyhdr}		% réglage header footer
\usepackage{needspace}		% empêcher sauts de page mal placés
\usepackage{graphicx}		% pour inclure des graphiques
\usepackage{enumitem,cprotect}		% personnalise les listes d'items (nécessaire pour ol, al ...)
\usepackage{hyperref}		% Liens hypertexte
\usepackage{pstricks,pst-all,pst-node,pstricks-add,pst-math,pst-plot,pst-tree,pst-eucl} % pstricks
\usepackage[a4paper,includeheadfoot,top=2cm,left=3cm, bottom=2cm,right=3cm]{geometry} % marges etc.
\usepackage{comment}			% commentaires multilignes
\usepackage{amsmath,environ} % maths (matrices, etc.)
\usepackage{amssymb,makeidx}
\usepackage{bm}				% bold maths
\usepackage{tabularx}		% tableaux
\usepackage{colortbl}		% tableaux en couleur
\usepackage{fontawesome}		% Fontawesome
\usepackage{environ}			% environment with command
\usepackage{fp}				% calculs pour ps-tricks
\usepackage{multido}			% pour ps tricks
\usepackage[np]{numprint}	% formattage nombre
\usepackage{tikz,tkz-tab} 			% package principal TikZ
\usepackage{pgfplots}   % axes
\usepackage{mathrsfs}    % cursives
\usepackage{calc}			% calcul taille boites
\usepackage[scaled=0.875]{helvet} % font sans serif
\usepackage{svg} % svg
\usepackage{scrextend} % local margin
\usepackage{scratch} %scratch
\usepackage{multicol} % colonnes
%\usepackage{infix-RPN,pst-func} % formule en notation polanaise inversée
\usepackage{listings}

%================================================================================================================================
%
% Réglages de base
%
%================================================================================================================================

\lstset{
language=Python,   % R code
literate=
{á}{{\'a}}1
{à}{{\`a}}1
{ã}{{\~a}}1
{é}{{\'e}}1
{è}{{\`e}}1
{ê}{{\^e}}1
{í}{{\'i}}1
{ó}{{\'o}}1
{õ}{{\~o}}1
{ú}{{\'u}}1
{ü}{{\"u}}1
{ç}{{\c{c}}}1
{~}{{ }}1
}


\definecolor{codegreen}{rgb}{0,0.6,0}
\definecolor{codegray}{rgb}{0.5,0.5,0.5}
\definecolor{codepurple}{rgb}{0.58,0,0.82}
\definecolor{backcolour}{rgb}{0.95,0.95,0.92}

\lstdefinestyle{mystyle}{
    backgroundcolor=\color{backcolour},   
    commentstyle=\color{codegreen},
    keywordstyle=\color{magenta},
    numberstyle=\tiny\color{codegray},
    stringstyle=\color{codepurple},
    basicstyle=\ttfamily\footnotesize,
    breakatwhitespace=false,         
    breaklines=true,                 
    captionpos=b,                    
    keepspaces=true,                 
    numbers=left,                    
xleftmargin=2em,
framexleftmargin=2em,            
    showspaces=false,                
    showstringspaces=false,
    showtabs=false,                  
    tabsize=2,
    upquote=true
}

\lstset{style=mystyle}


\lstset{style=mystyle}
\newcommand{\imgdir}{C:/laragon/www/newmc/assets/imgsvg/}
\newcommand{\imgsvgdir}{C:/laragon/www/newmc/assets/imgsvg/}

\definecolor{mcgris}{RGB}{220, 220, 220}% ancien~; pour compatibilité
\definecolor{mcbleu}{RGB}{52, 152, 219}
\definecolor{mcvert}{RGB}{125, 194, 70}
\definecolor{mcmauve}{RGB}{154, 0, 215}
\definecolor{mcorange}{RGB}{255, 96, 0}
\definecolor{mcturquoise}{RGB}{0, 153, 153}
\definecolor{mcrouge}{RGB}{255, 0, 0}
\definecolor{mclightvert}{RGB}{205, 234, 190}

\definecolor{gris}{RGB}{220, 220, 220}
\definecolor{bleu}{RGB}{52, 152, 219}
\definecolor{vert}{RGB}{125, 194, 70}
\definecolor{mauve}{RGB}{154, 0, 215}
\definecolor{orange}{RGB}{255, 96, 0}
\definecolor{turquoise}{RGB}{0, 153, 153}
\definecolor{rouge}{RGB}{255, 0, 0}
\definecolor{lightvert}{RGB}{205, 234, 190}
\setitemize[0]{label=\color{lightvert}  $\bullet$}

\pagestyle{fancy}
\renewcommand{\headrulewidth}{0.2pt}
\fancyhead[L]{maths-cours.fr}
\fancyhead[R]{\thepage}
\renewcommand{\footrulewidth}{0.2pt}
\fancyfoot[C]{}

\newcolumntype{C}{>{\centering\arraybackslash}X}
\newcolumntype{s}{>{\hsize=.35\hsize\arraybackslash}X}

\setlength{\parindent}{0pt}		 
\setlength{\parskip}{3mm}
\setlength{\headheight}{1cm}

\def\ebook{ebook}
\def\book{book}
\def\web{web}
\def\type{web}

\newcommand{\vect}[1]{\overrightarrow{\,\mathstrut#1\,}}

\def\Oij{$\left(\text{O}~;~\vect{\imath},~\vect{\jmath}\right)$}
\def\Oijk{$\left(\text{O}~;~\vect{\imath},~\vect{\jmath},~\vect{k}\right)$}
\def\Ouv{$\left(\text{O}~;~\vect{u},~\vect{v}\right)$}

\hypersetup{breaklinks=true, colorlinks = true, linkcolor = OliveGreen, urlcolor = OliveGreen, citecolor = OliveGreen, pdfauthor={Didier BONNEL - https://www.maths-cours.fr} } % supprime les bordures autour des liens

\renewcommand{\arg}[0]{\text{arg}}

\everymath{\displaystyle}

%================================================================================================================================
%
% Macros - Commandes
%
%================================================================================================================================

\newcommand\meta[2]{    			% Utilisé pour créer le post HTML.
	\def\titre{titre}
	\def\url{url}
	\def\arg{#1}
	\ifx\titre\arg
		\newcommand\maintitle{#2}
		\fancyhead[L]{#2}
		{\Large\sffamily \MakeUppercase{#2}}
		\vspace{1mm}\textcolor{mcvert}{\hrule}
	\fi 
	\ifx\url\arg
		\fancyfoot[L]{\href{https://www.maths-cours.fr#2}{\black \footnotesize{https://www.maths-cours.fr#2}}}
	\fi 
}


\newcommand\TitreC[1]{    		% Titre centré
     \needspace{3\baselineskip}
     \begin{center}\textbf{#1}\end{center}
}

\newcommand\newpar{    		% paragraphe
     \par
}

\newcommand\nosp {    		% commande vide (pas d'espace)
}
\newcommand{\id}[1]{} %ignore

\newcommand\boite[2]{				% Boite simple sans titre
	\vspace{5mm}
	\setlength{\fboxrule}{0.2mm}
	\setlength{\fboxsep}{5mm}	
	\fcolorbox{#1}{#1!3}{\makebox[\linewidth-2\fboxrule-2\fboxsep]{
  		\begin{minipage}[t]{\linewidth-2\fboxrule-4\fboxsep}\setlength{\parskip}{3mm}
  			 #2
  		\end{minipage}
	}}
	\vspace{5mm}
}

\newcommand\CBox[4]{				% Boites
	\vspace{5mm}
	\setlength{\fboxrule}{0.2mm}
	\setlength{\fboxsep}{5mm}
	
	\fcolorbox{#1}{#1!3}{\makebox[\linewidth-2\fboxrule-2\fboxsep]{
		\begin{minipage}[t]{1cm}\setlength{\parskip}{3mm}
	  		\textcolor{#1}{\LARGE{#2}}    
 	 	\end{minipage}  
  		\begin{minipage}[t]{\linewidth-2\fboxrule-4\fboxsep}\setlength{\parskip}{3mm}
			\raisebox{1.2mm}{\normalsize\sffamily{\textcolor{#1}{#3}}}						
  			 #4
  		\end{minipage}
	}}
	\vspace{5mm}
}

\newcommand\cadre[3]{				% Boites convertible html
	\par
	\vspace{2mm}
	\setlength{\fboxrule}{0.1mm}
	\setlength{\fboxsep}{5mm}
	\fcolorbox{#1}{white}{\makebox[\linewidth-2\fboxrule-2\fboxsep]{
  		\begin{minipage}[t]{\linewidth-2\fboxrule-4\fboxsep}\setlength{\parskip}{3mm}
			\raisebox{-2.5mm}{\sffamily \small{\textcolor{#1}{\MakeUppercase{#2}}}}		
			\par		
  			 #3
 	 		\end{minipage}
	}}
		\vspace{2mm}
	\par
}

\newcommand\bloc[3]{				% Boites convertible html sans bordure
     \needspace{2\baselineskip}
     {\sffamily \small{\textcolor{#1}{\MakeUppercase{#2}}}}    
		\par		
  			 #3
		\par
}

\newcommand\CHelp[1]{
     \CBox{Plum}{\faInfoCircle}{À RETENIR}{#1}
}

\newcommand\CUp[1]{
     \CBox{NavyBlue}{\faThumbsOUp}{EN PRATIQUE}{#1}
}

\newcommand\CInfo[1]{
     \CBox{Sepia}{\faArrowCircleRight}{REMARQUE}{#1}
}

\newcommand\CRedac[1]{
     \CBox{PineGreen}{\faEdit}{BIEN R\'EDIGER}{#1}
}

\newcommand\CError[1]{
     \CBox{Red}{\faExclamationTriangle}{ATTENTION}{#1}
}

\newcommand\TitreExo[2]{
\needspace{4\baselineskip}
 {\sffamily\large EXERCICE #1\ (\emph{#2 points})}
\vspace{5mm}
}

\newcommand\img[2]{
          \includegraphics[width=#2\paperwidth]{\imgdir#1}
}

\newcommand\imgsvg[2]{
       \begin{center}   \includegraphics[width=#2\paperwidth]{\imgsvgdir#1} \end{center}
}


\newcommand\Lien[2]{
     \href{#1}{#2 \tiny \faExternalLink}
}
\newcommand\mcLien[2]{
     \href{https~://www.maths-cours.fr/#1}{#2 \tiny \faExternalLink}
}

\newcommand{\euro}{\eurologo{}}

%================================================================================================================================
%
% Macros - Environement
%
%================================================================================================================================

\newenvironment{tex}{ %
}
{%
}

\newenvironment{indente}{ %
	\setlength\parindent{10mm}
}

{
	\setlength\parindent{0mm}
}

\newenvironment{corrige}{%
     \needspace{3\baselineskip}
     \medskip
     \textbf{\textsc{Corrigé}}
     \medskip
}
{
}

\newenvironment{extern}{%
     \begin{center}
     }
     {
     \end{center}
}

\NewEnviron{code}{%
	\par
     \boite{gray}{\texttt{%
     \BODY
     }}
     \par
}

\newenvironment{vbloc}{% boite sans cadre empeche saut de page
     \begin{minipage}[t]{\linewidth}
     }
     {
     \end{minipage}
}
\NewEnviron{h2}{%
    \needspace{3\baselineskip}
    \vspace{0.6cm}
	\noindent \MakeUppercase{\sffamily \large \BODY}
	\vspace{1mm}\textcolor{mcgris}{\hrule}\vspace{0.4cm}
	\par
}{}

\NewEnviron{h3}{%
    \needspace{3\baselineskip}
	\vspace{5mm}
	\textsc{\BODY}
	\par
}

\NewEnviron{margeneg}{ %
\begin{addmargin}[-1cm]{0cm}
\BODY
\end{addmargin}
}

\NewEnviron{html}{%
}

\begin{document}
\meta{url}{/exercices/suite-de-fibonacci-bac-s-liban-2018-spe/}
\meta{pid}{7761}
\meta{titre}{Matrices et arithmétique - Bac S Liban 2018 (spé)}
\meta{type}{exercices}
%
\begin{h2}EXERCICE 5 (5 points)\end{h2}
\textbf{Candidats ayant suivi l'enseignement de spécialité}
\medskip
On définit la suite de réels $\left(a_n\right)$ par :
\[\left\{\begin{array}{l c l}
          a_0 &= &0\\
          a_1 &= &1\\
          a_{n+1} &=& a_n + a_{n-1}\: \text{ pour }\: n \geqslant 1.
\end{array}\right.\]
On appelle cette suite la suite de Fibonacci.
\medskip
\begin{enumerate}
     \item Recopier et compléter l'algorithme ci-dessous pour qu'à la fin de son exécution la variable $A$
     contienne le terme $a_n$.
     \begin{center}
          \begin{extern}%width="340"
               \begin{tabularx}{0.5\linewidth}{|c X|}\hline
                    1&$A \gets 0$\\
                    2& $B \gets 1$\\
                    3& Pour $i$ allant de 2 à $n$ :\\
                    4& {\hspace{0.6cm} $C \gets A + B$}\\
                    5& {\hspace{0.6cm} $A \gets \ldots$}\\
                    6& {\hspace{0.6cm} $B \gets \ldots$}\\
                    7& Fin Pour\\ \hline
               \end{tabularx}
          \end{extern}
     \end{center}
     On obtient ainsi les premières valeurs de la suite $a_n$ :
     \begin{center}
          \begin{extern}%width="670"
               \begin{tabularx}{\linewidth}{|c|*{11}{>{\centering \arraybackslash}X|}}\hline
                    $n$		&0 	&1 	&2 	&3 	&4 	&5 &6 	&7 		&8 	&9 &10\\ \hline
                    $a_n$	&0	& 1 &1	&2 	&3	&5 &8 	&13 	&21 &34 &55\\ \hline
               \end{tabularx}
          \end{extern}
     \end{center}
     \item  Soit la matrice $A = \begin{pmatrix}1&1\\1&0\end{pmatrix}$.
     \par
     Calculer $A^2$, $A^3$ et $A^4$.
     \par
     Vérifier que $A^5 = \begin{pmatrix}8&5\\5&3\end{pmatrix}$.
     \item On peut démontrer, et nous admettrons, que pour tout entier naturel $n$ non nul,
     \[A^n = \begin{pmatrix}a_{n+1}&a_n\\a_n&a_{n-1}\end{pmatrix}.\]
     \begin{enumerate}[label=\alph*.]
          \item Soit $p$ et $q$ deux entiers naturels non nuls. Calculer le produit $A^p \times A^q$ et en déduire que
          \[a_{p+q} = a_p \times a_{q+1} + a_{p-1} \times a_q.\]
          \item  En déduire que si un entier $r$ divise les entiers $a_p$ et $a_q$, alors $r$ divise également $a_{p+q}$.
          \item  Soit $p$ un entier naturel non nul.
          \par
          Démontrer, en utilisant un raisonnement par récurrence sur $n$, que pour tout entier naturel
          $n$ non nul, $a_p$ divise $a_{np}$.
     \end{enumerate}
     \item
     \begin{enumerate}[label=\alph*.]
          \item Soit $n$ un entier supérieur ou égal à 5. Montrer que si $n$ est un entier naturel qui n'est pas premier, alors $a_n$ n'est pas un nombre premier.
          \item On peut calculer $a_{19} = 4~181 = 37 \times 113$.
          \par
          Que penser de la réciproque de la propriété obtenue dans la question 4. a. ?
     \end{enumerate}
\end{enumerate}

\end{document}
µ
\documentclass[a4paper]{article}

%================================================================================================================================
%
% Packages
%
%================================================================================================================================

\usepackage[T1]{fontenc} 	% pour caractères accentués
\usepackage[utf8]{inputenc}  % encodage utf8
\usepackage[french]{babel}	% langue : français
\usepackage{fourier}			% caractères plus lisibles
\usepackage[dvipsnames]{xcolor} % couleurs
\usepackage{fancyhdr}		% réglage header footer
\usepackage{needspace}		% empêcher sauts de page mal placés
\usepackage{graphicx}		% pour inclure des graphiques
\usepackage{enumitem,cprotect}		% personnalise les listes d'items (nécessaire pour ol, al ...)
\usepackage{hyperref}		% Liens hypertexte
\usepackage{pstricks,pst-all,pst-node,pstricks-add,pst-math,pst-plot,pst-tree,pst-eucl} % pstricks
\usepackage[a4paper,includeheadfoot,top=2cm,left=3cm, bottom=2cm,right=3cm]{geometry} % marges etc.
\usepackage{comment}			% commentaires multilignes
\usepackage{amsmath,environ} % maths (matrices, etc.)
\usepackage{amssymb,makeidx}
\usepackage{bm}				% bold maths
\usepackage{tabularx}		% tableaux
\usepackage{colortbl}		% tableaux en couleur
\usepackage{fontawesome}		% Fontawesome
\usepackage{environ}			% environment with command
\usepackage{fp}				% calculs pour ps-tricks
\usepackage{multido}			% pour ps tricks
\usepackage[np]{numprint}	% formattage nombre
\usepackage{tikz,tkz-tab} 			% package principal TikZ
\usepackage{pgfplots}   % axes
\usepackage{mathrsfs}    % cursives
\usepackage{calc}			% calcul taille boites
\usepackage[scaled=0.875]{helvet} % font sans serif
\usepackage{svg} % svg
\usepackage{scrextend} % local margin
\usepackage{scratch} %scratch
\usepackage{multicol} % colonnes
%\usepackage{infix-RPN,pst-func} % formule en notation polanaise inversée
\usepackage{listings}

%================================================================================================================================
%
% Réglages de base
%
%================================================================================================================================

\lstset{
language=Python,   % R code
literate=
{á}{{\'a}}1
{à}{{\`a}}1
{ã}{{\~a}}1
{é}{{\'e}}1
{è}{{\`e}}1
{ê}{{\^e}}1
{í}{{\'i}}1
{ó}{{\'o}}1
{õ}{{\~o}}1
{ú}{{\'u}}1
{ü}{{\"u}}1
{ç}{{\c{c}}}1
{~}{{ }}1
}


\definecolor{codegreen}{rgb}{0,0.6,0}
\definecolor{codegray}{rgb}{0.5,0.5,0.5}
\definecolor{codepurple}{rgb}{0.58,0,0.82}
\definecolor{backcolour}{rgb}{0.95,0.95,0.92}

\lstdefinestyle{mystyle}{
    backgroundcolor=\color{backcolour},   
    commentstyle=\color{codegreen},
    keywordstyle=\color{magenta},
    numberstyle=\tiny\color{codegray},
    stringstyle=\color{codepurple},
    basicstyle=\ttfamily\footnotesize,
    breakatwhitespace=false,         
    breaklines=true,                 
    captionpos=b,                    
    keepspaces=true,                 
    numbers=left,                    
xleftmargin=2em,
framexleftmargin=2em,            
    showspaces=false,                
    showstringspaces=false,
    showtabs=false,                  
    tabsize=2,
    upquote=true
}

\lstset{style=mystyle}


\lstset{style=mystyle}
\newcommand{\imgdir}{C:/laragon/www/newmc/assets/imgsvg/}
\newcommand{\imgsvgdir}{C:/laragon/www/newmc/assets/imgsvg/}

\definecolor{mcgris}{RGB}{220, 220, 220}% ancien~; pour compatibilité
\definecolor{mcbleu}{RGB}{52, 152, 219}
\definecolor{mcvert}{RGB}{125, 194, 70}
\definecolor{mcmauve}{RGB}{154, 0, 215}
\definecolor{mcorange}{RGB}{255, 96, 0}
\definecolor{mcturquoise}{RGB}{0, 153, 153}
\definecolor{mcrouge}{RGB}{255, 0, 0}
\definecolor{mclightvert}{RGB}{205, 234, 190}

\definecolor{gris}{RGB}{220, 220, 220}
\definecolor{bleu}{RGB}{52, 152, 219}
\definecolor{vert}{RGB}{125, 194, 70}
\definecolor{mauve}{RGB}{154, 0, 215}
\definecolor{orange}{RGB}{255, 96, 0}
\definecolor{turquoise}{RGB}{0, 153, 153}
\definecolor{rouge}{RGB}{255, 0, 0}
\definecolor{lightvert}{RGB}{205, 234, 190}
\setitemize[0]{label=\color{lightvert}  $\bullet$}

\pagestyle{fancy}
\renewcommand{\headrulewidth}{0.2pt}
\fancyhead[L]{maths-cours.fr}
\fancyhead[R]{\thepage}
\renewcommand{\footrulewidth}{0.2pt}
\fancyfoot[C]{}

\newcolumntype{C}{>{\centering\arraybackslash}X}
\newcolumntype{s}{>{\hsize=.35\hsize\arraybackslash}X}

\setlength{\parindent}{0pt}		 
\setlength{\parskip}{3mm}
\setlength{\headheight}{1cm}

\def\ebook{ebook}
\def\book{book}
\def\web{web}
\def\type{web}

\newcommand{\vect}[1]{\overrightarrow{\,\mathstrut#1\,}}

\def\Oij{$\left(\text{O}~;~\vect{\imath},~\vect{\jmath}\right)$}
\def\Oijk{$\left(\text{O}~;~\vect{\imath},~\vect{\jmath},~\vect{k}\right)$}
\def\Ouv{$\left(\text{O}~;~\vect{u},~\vect{v}\right)$}

\hypersetup{breaklinks=true, colorlinks = true, linkcolor = OliveGreen, urlcolor = OliveGreen, citecolor = OliveGreen, pdfauthor={Didier BONNEL - https://www.maths-cours.fr} } % supprime les bordures autour des liens

\renewcommand{\arg}[0]{\text{arg}}

\everymath{\displaystyle}

%================================================================================================================================
%
% Macros - Commandes
%
%================================================================================================================================

\newcommand\meta[2]{    			% Utilisé pour créer le post HTML.
	\def\titre{titre}
	\def\url{url}
	\def\arg{#1}
	\ifx\titre\arg
		\newcommand\maintitle{#2}
		\fancyhead[L]{#2}
		{\Large\sffamily \MakeUppercase{#2}}
		\vspace{1mm}\textcolor{mcvert}{\hrule}
	\fi 
	\ifx\url\arg
		\fancyfoot[L]{\href{https://www.maths-cours.fr#2}{\black \footnotesize{https://www.maths-cours.fr#2}}}
	\fi 
}


\newcommand\TitreC[1]{    		% Titre centré
     \needspace{3\baselineskip}
     \begin{center}\textbf{#1}\end{center}
}

\newcommand\newpar{    		% paragraphe
     \par
}

\newcommand\nosp {    		% commande vide (pas d'espace)
}
\newcommand{\id}[1]{} %ignore

\newcommand\boite[2]{				% Boite simple sans titre
	\vspace{5mm}
	\setlength{\fboxrule}{0.2mm}
	\setlength{\fboxsep}{5mm}	
	\fcolorbox{#1}{#1!3}{\makebox[\linewidth-2\fboxrule-2\fboxsep]{
  		\begin{minipage}[t]{\linewidth-2\fboxrule-4\fboxsep}\setlength{\parskip}{3mm}
  			 #2
  		\end{minipage}
	}}
	\vspace{5mm}
}

\newcommand\CBox[4]{				% Boites
	\vspace{5mm}
	\setlength{\fboxrule}{0.2mm}
	\setlength{\fboxsep}{5mm}
	
	\fcolorbox{#1}{#1!3}{\makebox[\linewidth-2\fboxrule-2\fboxsep]{
		\begin{minipage}[t]{1cm}\setlength{\parskip}{3mm}
	  		\textcolor{#1}{\LARGE{#2}}    
 	 	\end{minipage}  
  		\begin{minipage}[t]{\linewidth-2\fboxrule-4\fboxsep}\setlength{\parskip}{3mm}
			\raisebox{1.2mm}{\normalsize\sffamily{\textcolor{#1}{#3}}}						
  			 #4
  		\end{minipage}
	}}
	\vspace{5mm}
}

\newcommand\cadre[3]{				% Boites convertible html
	\par
	\vspace{2mm}
	\setlength{\fboxrule}{0.1mm}
	\setlength{\fboxsep}{5mm}
	\fcolorbox{#1}{white}{\makebox[\linewidth-2\fboxrule-2\fboxsep]{
  		\begin{minipage}[t]{\linewidth-2\fboxrule-4\fboxsep}\setlength{\parskip}{3mm}
			\raisebox{-2.5mm}{\sffamily \small{\textcolor{#1}{\MakeUppercase{#2}}}}		
			\par		
  			 #3
 	 		\end{minipage}
	}}
		\vspace{2mm}
	\par
}

\newcommand\bloc[3]{				% Boites convertible html sans bordure
     \needspace{2\baselineskip}
     {\sffamily \small{\textcolor{#1}{\MakeUppercase{#2}}}}    
		\par		
  			 #3
		\par
}

\newcommand\CHelp[1]{
     \CBox{Plum}{\faInfoCircle}{À RETENIR}{#1}
}

\newcommand\CUp[1]{
     \CBox{NavyBlue}{\faThumbsOUp}{EN PRATIQUE}{#1}
}

\newcommand\CInfo[1]{
     \CBox{Sepia}{\faArrowCircleRight}{REMARQUE}{#1}
}

\newcommand\CRedac[1]{
     \CBox{PineGreen}{\faEdit}{BIEN R\'EDIGER}{#1}
}

\newcommand\CError[1]{
     \CBox{Red}{\faExclamationTriangle}{ATTENTION}{#1}
}

\newcommand\TitreExo[2]{
\needspace{4\baselineskip}
 {\sffamily\large EXERCICE #1\ (\emph{#2 points})}
\vspace{5mm}
}

\newcommand\img[2]{
          \includegraphics[width=#2\paperwidth]{\imgdir#1}
}

\newcommand\imgsvg[2]{
       \begin{center}   \includegraphics[width=#2\paperwidth]{\imgsvgdir#1} \end{center}
}


\newcommand\Lien[2]{
     \href{#1}{#2 \tiny \faExternalLink}
}
\newcommand\mcLien[2]{
     \href{https~://www.maths-cours.fr/#1}{#2 \tiny \faExternalLink}
}

\newcommand{\euro}{\eurologo{}}

%================================================================================================================================
%
% Macros - Environement
%
%================================================================================================================================

\newenvironment{tex}{ %
}
{%
}

\newenvironment{indente}{ %
	\setlength\parindent{10mm}
}

{
	\setlength\parindent{0mm}
}

\newenvironment{corrige}{%
     \needspace{3\baselineskip}
     \medskip
     \textbf{\textsc{Corrigé}}
     \medskip
}
{
}

\newenvironment{extern}{%
     \begin{center}
     }
     {
     \end{center}
}

\NewEnviron{code}{%
	\par
     \boite{gray}{\texttt{%
     \BODY
     }}
     \par
}

\newenvironment{vbloc}{% boite sans cadre empeche saut de page
     \begin{minipage}[t]{\linewidth}
     }
     {
     \end{minipage}
}
\NewEnviron{h2}{%
    \needspace{3\baselineskip}
    \vspace{0.6cm}
	\noindent \MakeUppercase{\sffamily \large \BODY}
	\vspace{1mm}\textcolor{mcgris}{\hrule}\vspace{0.4cm}
	\par
}{}

\NewEnviron{h3}{%
    \needspace{3\baselineskip}
	\vspace{5mm}
	\textsc{\BODY}
	\par
}

\NewEnviron{margeneg}{ %
\begin{addmargin}[-1cm]{0cm}
\BODY
\end{addmargin}
}

\NewEnviron{html}{%
}

\begin{document}
\meta{url}{/exercices/probabilites-bac-es-l-liban-2018/}
\meta{pid}{7771}
\meta{titre}{Probabilités - Bac ES/L Liban 2018}
\meta{type}{exercices}
%
\begin{h2}EXERCICE 1 (6 points)\end{h2}
\textbf{Commun à  tous les candidats}
\medskip
Dans un aéroport, les portiques de sécurité servent à détecter les objets métalliques que peuvent emporter les voyageurs.
\par
On choisit au hasard un voyageur franchissant un portique.
\par
On note :
\begin{itemize}
     \item $S$ l'événement \og le voyageur fait sonner le portique \fg{};
     \item $M$ l'événement \og le voyageur porte un objet métallique \fg{}.
\end{itemize}
On considère qu'un voyageur sur 500 porte sur lui un objet métallique.
\begin{enumerate}
     \item
     On admet que :
     \begin{itemize}
          \item Lorsqu'un voyageur franchit le portique avec un objet métallique, la probabilité que le portique sonne est égale à $0,98$;
          \item Lorsqu'un voyageur franchit le portique sans objet métallique, la probabilité que le portique ne sonne pas est aussi égale à $0,98$.
     \end{itemize}
     \begin{enumerate}[label=\alph*.]
          \item  À l'aide des données de l'énoncé, préciser les valeurs de $P(M)$, $P_{M}(S)$ et $P_{\overline{M}}(\overline{S})$.
          \item Recopier et compléter l'arbre pondéré ci-dessous illustrant cette situation.
          %:-+-+-+- Engendré par : http://math.et.info.free.fr/TikZ/Arbre/
          \begin{center}
               \begin{extern}%width="350"
                    % Racine à Gauche, développement vers la droite
                    \begin{tikzpicture}[xscale=1,yscale=1]
                         % Styles (MODIFIABLES)
                         \tikzstyle{fleche}=[-,>=latex,thick]
                         \tikzstyle{noeud}=[fill=white]
                         \tikzstyle{feuille}=[fill=white]
                         \tikzstyle{etiquette}=[midway,fill=white]
                         % Dimensions (MODIFIABLES)
                         \def\DistanceInterNiveaux{3}
                         \def\DistanceInterFeuilles{2}
                         % Dimensions calculées (NON MODIFIABLES)
                         \def\NiveauA{(0)*\DistanceInterNiveaux}
                         \def\NiveauB{(1.5)*\DistanceInterNiveaux}
                         \def\NiveauC{(2.5)*\DistanceInterNiveaux}
                         \def\InterFeuilles{(-1)*\DistanceInterFeuilles}
                         % Noeuds (MODIFIABLES : Styles et Coefficients d'InterFeuilles)
                         \node[noeud] (R) at ({\NiveauA},{(1.5)*\InterFeuilles}) {$ $};
                         \node[noeud] (Ra) at ({\NiveauB},{(0.5)*\InterFeuilles}) {$M$};
                         \node[feuille] (Raa) at ({\NiveauC},{(0)*\InterFeuilles}) {$S$};
                         \node[feuille] (Rab) at ({\NiveauC},{(1)*\InterFeuilles}) {$\overline{S}$};
                         \node[noeud] (Rb) at ({\NiveauB},{(2.5)*\InterFeuilles}) {$\overline{M}$};
                         \node[feuille] (Rba) at ({\NiveauC},{(2)*\InterFeuilles}) {$S$};
                         \node[feuille] (Rbb) at ({\NiveauC},{(3)*\InterFeuilles}) {$\overline{S}$};
                         % Arcs (MODIFIABLES : Styles)
                         \draw[fleche] (R)--(Ra) node[etiquette] {$\cdots$};
                         \draw[fleche] (Ra)--(Raa) node[etiquette] {$\cdots$};
                         \draw[fleche] (Ra)--(Rab) node[etiquette] {$\cdots$};
                         \draw[fleche] (R)--(Rb) node[etiquette] {$\cdots$};
                         \draw[fleche] (Rb)--(Rba) node[etiquette] {$\cdots$};
                         \draw[fleche] (Rb)--(Rbb) node[etiquette] {$\cdots$};
                    \end{tikzpicture}
               \end{extern}
          \end{center}
          %:-+-+-+-+- Fin
          \item Montrer que~: $P(S)=0,021~92$.
          \item En déduire la probabilité qu'un voyageur porte un objet métallique sachant qu'il a fait sonner le portique. (On arrondira le résultat à $10^{-3}$.)
     \end{enumerate}
     \item 80 personnes s'apprêtent à passer le portique de sécurité. On suppose que pour chaque personne la probabilité que le portique sonne est égale à $0,021~92$.
     \par
     Soit $X$ la variable aléatoire donnant le nombre de personnes faisant sonner le portique, parmi les 80 personnes de ce groupe.
     \begin{enumerate}[label=\alph*.]
          \item Justifier que $X$ suit une loi binomiale dont on précisera les paramètres.
          \item Calculer l'espérance de $X$ et interpréter le résultat.
          \item
          Sans le justifier, donner la valeur arrondie à $10^{-3}$ de~:
          \begin{itemize}
               \item la probabilité qu'au moins une personne du groupe fasse sonner le portique;
               \item la probabilité qu'au maximum 5 personnes fassent sonner le portique.
          \end{itemize}
          \item Sans le justifier, donner la valeur du plus petit entier $n$ tel que $P(X \leqslant n) \geqslant 0,9$.
     \end{enumerate}
\end{enumerate}

\end{document}
µ
\documentclass[a4paper]{article}

%================================================================================================================================
%
% Packages
%
%================================================================================================================================

\usepackage[T1]{fontenc} 	% pour caractères accentués
\usepackage[utf8]{inputenc}  % encodage utf8
\usepackage[french]{babel}	% langue : français
\usepackage{fourier}			% caractères plus lisibles
\usepackage[dvipsnames]{xcolor} % couleurs
\usepackage{fancyhdr}		% réglage header footer
\usepackage{needspace}		% empêcher sauts de page mal placés
\usepackage{graphicx}		% pour inclure des graphiques
\usepackage{enumitem,cprotect}		% personnalise les listes d'items (nécessaire pour ol, al ...)
\usepackage{hyperref}		% Liens hypertexte
\usepackage{pstricks,pst-all,pst-node,pstricks-add,pst-math,pst-plot,pst-tree,pst-eucl} % pstricks
\usepackage[a4paper,includeheadfoot,top=2cm,left=3cm, bottom=2cm,right=3cm]{geometry} % marges etc.
\usepackage{comment}			% commentaires multilignes
\usepackage{amsmath,environ} % maths (matrices, etc.)
\usepackage{amssymb,makeidx}
\usepackage{bm}				% bold maths
\usepackage{tabularx}		% tableaux
\usepackage{colortbl}		% tableaux en couleur
\usepackage{fontawesome}		% Fontawesome
\usepackage{environ}			% environment with command
\usepackage{fp}				% calculs pour ps-tricks
\usepackage{multido}			% pour ps tricks
\usepackage[np]{numprint}	% formattage nombre
\usepackage{tikz,tkz-tab} 			% package principal TikZ
\usepackage{pgfplots}   % axes
\usepackage{mathrsfs}    % cursives
\usepackage{calc}			% calcul taille boites
\usepackage[scaled=0.875]{helvet} % font sans serif
\usepackage{svg} % svg
\usepackage{scrextend} % local margin
\usepackage{scratch} %scratch
\usepackage{multicol} % colonnes
%\usepackage{infix-RPN,pst-func} % formule en notation polanaise inversée
\usepackage{listings}

%================================================================================================================================
%
% Réglages de base
%
%================================================================================================================================

\lstset{
language=Python,   % R code
literate=
{á}{{\'a}}1
{à}{{\`a}}1
{ã}{{\~a}}1
{é}{{\'e}}1
{è}{{\`e}}1
{ê}{{\^e}}1
{í}{{\'i}}1
{ó}{{\'o}}1
{õ}{{\~o}}1
{ú}{{\'u}}1
{ü}{{\"u}}1
{ç}{{\c{c}}}1
{~}{{ }}1
}


\definecolor{codegreen}{rgb}{0,0.6,0}
\definecolor{codegray}{rgb}{0.5,0.5,0.5}
\definecolor{codepurple}{rgb}{0.58,0,0.82}
\definecolor{backcolour}{rgb}{0.95,0.95,0.92}

\lstdefinestyle{mystyle}{
    backgroundcolor=\color{backcolour},   
    commentstyle=\color{codegreen},
    keywordstyle=\color{magenta},
    numberstyle=\tiny\color{codegray},
    stringstyle=\color{codepurple},
    basicstyle=\ttfamily\footnotesize,
    breakatwhitespace=false,         
    breaklines=true,                 
    captionpos=b,                    
    keepspaces=true,                 
    numbers=left,                    
xleftmargin=2em,
framexleftmargin=2em,            
    showspaces=false,                
    showstringspaces=false,
    showtabs=false,                  
    tabsize=2,
    upquote=true
}

\lstset{style=mystyle}


\lstset{style=mystyle}
\newcommand{\imgdir}{C:/laragon/www/newmc/assets/imgsvg/}
\newcommand{\imgsvgdir}{C:/laragon/www/newmc/assets/imgsvg/}

\definecolor{mcgris}{RGB}{220, 220, 220}% ancien~; pour compatibilité
\definecolor{mcbleu}{RGB}{52, 152, 219}
\definecolor{mcvert}{RGB}{125, 194, 70}
\definecolor{mcmauve}{RGB}{154, 0, 215}
\definecolor{mcorange}{RGB}{255, 96, 0}
\definecolor{mcturquoise}{RGB}{0, 153, 153}
\definecolor{mcrouge}{RGB}{255, 0, 0}
\definecolor{mclightvert}{RGB}{205, 234, 190}

\definecolor{gris}{RGB}{220, 220, 220}
\definecolor{bleu}{RGB}{52, 152, 219}
\definecolor{vert}{RGB}{125, 194, 70}
\definecolor{mauve}{RGB}{154, 0, 215}
\definecolor{orange}{RGB}{255, 96, 0}
\definecolor{turquoise}{RGB}{0, 153, 153}
\definecolor{rouge}{RGB}{255, 0, 0}
\definecolor{lightvert}{RGB}{205, 234, 190}
\setitemize[0]{label=\color{lightvert}  $\bullet$}

\pagestyle{fancy}
\renewcommand{\headrulewidth}{0.2pt}
\fancyhead[L]{maths-cours.fr}
\fancyhead[R]{\thepage}
\renewcommand{\footrulewidth}{0.2pt}
\fancyfoot[C]{}

\newcolumntype{C}{>{\centering\arraybackslash}X}
\newcolumntype{s}{>{\hsize=.35\hsize\arraybackslash}X}

\setlength{\parindent}{0pt}		 
\setlength{\parskip}{3mm}
\setlength{\headheight}{1cm}

\def\ebook{ebook}
\def\book{book}
\def\web{web}
\def\type{web}

\newcommand{\vect}[1]{\overrightarrow{\,\mathstrut#1\,}}

\def\Oij{$\left(\text{O}~;~\vect{\imath},~\vect{\jmath}\right)$}
\def\Oijk{$\left(\text{O}~;~\vect{\imath},~\vect{\jmath},~\vect{k}\right)$}
\def\Ouv{$\left(\text{O}~;~\vect{u},~\vect{v}\right)$}

\hypersetup{breaklinks=true, colorlinks = true, linkcolor = OliveGreen, urlcolor = OliveGreen, citecolor = OliveGreen, pdfauthor={Didier BONNEL - https://www.maths-cours.fr} } % supprime les bordures autour des liens

\renewcommand{\arg}[0]{\text{arg}}

\everymath{\displaystyle}

%================================================================================================================================
%
% Macros - Commandes
%
%================================================================================================================================

\newcommand\meta[2]{    			% Utilisé pour créer le post HTML.
	\def\titre{titre}
	\def\url{url}
	\def\arg{#1}
	\ifx\titre\arg
		\newcommand\maintitle{#2}
		\fancyhead[L]{#2}
		{\Large\sffamily \MakeUppercase{#2}}
		\vspace{1mm}\textcolor{mcvert}{\hrule}
	\fi 
	\ifx\url\arg
		\fancyfoot[L]{\href{https://www.maths-cours.fr#2}{\black \footnotesize{https://www.maths-cours.fr#2}}}
	\fi 
}


\newcommand\TitreC[1]{    		% Titre centré
     \needspace{3\baselineskip}
     \begin{center}\textbf{#1}\end{center}
}

\newcommand\newpar{    		% paragraphe
     \par
}

\newcommand\nosp {    		% commande vide (pas d'espace)
}
\newcommand{\id}[1]{} %ignore

\newcommand\boite[2]{				% Boite simple sans titre
	\vspace{5mm}
	\setlength{\fboxrule}{0.2mm}
	\setlength{\fboxsep}{5mm}	
	\fcolorbox{#1}{#1!3}{\makebox[\linewidth-2\fboxrule-2\fboxsep]{
  		\begin{minipage}[t]{\linewidth-2\fboxrule-4\fboxsep}\setlength{\parskip}{3mm}
  			 #2
  		\end{minipage}
	}}
	\vspace{5mm}
}

\newcommand\CBox[4]{				% Boites
	\vspace{5mm}
	\setlength{\fboxrule}{0.2mm}
	\setlength{\fboxsep}{5mm}
	
	\fcolorbox{#1}{#1!3}{\makebox[\linewidth-2\fboxrule-2\fboxsep]{
		\begin{minipage}[t]{1cm}\setlength{\parskip}{3mm}
	  		\textcolor{#1}{\LARGE{#2}}    
 	 	\end{minipage}  
  		\begin{minipage}[t]{\linewidth-2\fboxrule-4\fboxsep}\setlength{\parskip}{3mm}
			\raisebox{1.2mm}{\normalsize\sffamily{\textcolor{#1}{#3}}}						
  			 #4
  		\end{minipage}
	}}
	\vspace{5mm}
}

\newcommand\cadre[3]{				% Boites convertible html
	\par
	\vspace{2mm}
	\setlength{\fboxrule}{0.1mm}
	\setlength{\fboxsep}{5mm}
	\fcolorbox{#1}{white}{\makebox[\linewidth-2\fboxrule-2\fboxsep]{
  		\begin{minipage}[t]{\linewidth-2\fboxrule-4\fboxsep}\setlength{\parskip}{3mm}
			\raisebox{-2.5mm}{\sffamily \small{\textcolor{#1}{\MakeUppercase{#2}}}}		
			\par		
  			 #3
 	 		\end{minipage}
	}}
		\vspace{2mm}
	\par
}

\newcommand\bloc[3]{				% Boites convertible html sans bordure
     \needspace{2\baselineskip}
     {\sffamily \small{\textcolor{#1}{\MakeUppercase{#2}}}}    
		\par		
  			 #3
		\par
}

\newcommand\CHelp[1]{
     \CBox{Plum}{\faInfoCircle}{À RETENIR}{#1}
}

\newcommand\CUp[1]{
     \CBox{NavyBlue}{\faThumbsOUp}{EN PRATIQUE}{#1}
}

\newcommand\CInfo[1]{
     \CBox{Sepia}{\faArrowCircleRight}{REMARQUE}{#1}
}

\newcommand\CRedac[1]{
     \CBox{PineGreen}{\faEdit}{BIEN R\'EDIGER}{#1}
}

\newcommand\CError[1]{
     \CBox{Red}{\faExclamationTriangle}{ATTENTION}{#1}
}

\newcommand\TitreExo[2]{
\needspace{4\baselineskip}
 {\sffamily\large EXERCICE #1\ (\emph{#2 points})}
\vspace{5mm}
}

\newcommand\img[2]{
          \includegraphics[width=#2\paperwidth]{\imgdir#1}
}

\newcommand\imgsvg[2]{
       \begin{center}   \includegraphics[width=#2\paperwidth]{\imgsvgdir#1} \end{center}
}


\newcommand\Lien[2]{
     \href{#1}{#2 \tiny \faExternalLink}
}
\newcommand\mcLien[2]{
     \href{https~://www.maths-cours.fr/#1}{#2 \tiny \faExternalLink}
}

\newcommand{\euro}{\eurologo{}}

%================================================================================================================================
%
% Macros - Environement
%
%================================================================================================================================

\newenvironment{tex}{ %
}
{%
}

\newenvironment{indente}{ %
	\setlength\parindent{10mm}
}

{
	\setlength\parindent{0mm}
}

\newenvironment{corrige}{%
     \needspace{3\baselineskip}
     \medskip
     \textbf{\textsc{Corrigé}}
     \medskip
}
{
}

\newenvironment{extern}{%
     \begin{center}
     }
     {
     \end{center}
}

\NewEnviron{code}{%
	\par
     \boite{gray}{\texttt{%
     \BODY
     }}
     \par
}

\newenvironment{vbloc}{% boite sans cadre empeche saut de page
     \begin{minipage}[t]{\linewidth}
     }
     {
     \end{minipage}
}
\NewEnviron{h2}{%
    \needspace{3\baselineskip}
    \vspace{0.6cm}
	\noindent \MakeUppercase{\sffamily \large \BODY}
	\vspace{1mm}\textcolor{mcgris}{\hrule}\vspace{0.4cm}
	\par
}{}

\NewEnviron{h3}{%
    \needspace{3\baselineskip}
	\vspace{5mm}
	\textsc{\BODY}
	\par
}

\NewEnviron{margeneg}{ %
\begin{addmargin}[-1cm]{0cm}
\BODY
\end{addmargin}
}

\NewEnviron{html}{%
}

\begin{document}
\meta{url}{/exercices/suites-bac-es-l-liban-2018/}
\meta{pid}{7781}
\meta{titre}{Suites - Bac ES/L Liban 2018}
\meta{type}{exercices}
%
\begin{h2}Exercice 2 (5 points)\end{h2}
\textbf{Candidats de ES n'ayant pas suivi la spécialité et candidats de L}
\medskip
Maya possède 20~\euro{} dans sa tirelire au 1${^\text{er}}$ juin 2018.
\par
À partir de cette date, chaque mois elle dépense un quart du contenu de sa tirelire puis y place 20~\euro{} supplémentaires.
\par
Pour tout entier naturel $n$, on note $u_n$ la somme d'argent contenue dans la tirelire de Maya à la fin du $n$-ième mois. On a $u_0=20$.
\begin{enumerate}
     \item
     \begin{enumerate}[label=\alph*.]
          \item Montrer que la somme d'argent contenue dans la tirelire de Maya à la fin du 1${^\text{er}}$ mois est de 35~\euro.
          \item Calculer $u_2$.
     \end{enumerate}
     \item On admet que pour tout entier naturel $n$, $u_{n+1}=0,75 u_n + 20$.
     \par
     On considère l'algorithme suivant:
     \begin{center}
          \begin{extern}%width="350"
               \begin{tabularx}{0.5\linewidth}{|X|}
                    \hline
                    $U \longleftarrow 20$\\
                    $N \longleftarrow 0$\\
                    Tant que $U < 70$\\
                    \hspace*{1cm} $U \longleftarrow 0,75 \times U + 20$\\
                    \hspace*{1cm} $N \longleftarrow N+1$\\
                    Fin Tant que\\
                    Afficher $N$\\
                    \hline
               \end{tabularx}
          \end{extern}
     \end{center}
     \begin{enumerate}[label=\alph*.]
          \item Recopier et compléter le tableau ci-dessous qui retrace les différentes étapes de l'exécution de l'algorithme. On ajoutera autant de colonnes que nécessaire à la place de celle contenant les pointillés. Arrondir les résultats au centième.
          \begin{center}
               \begin{extern}%width="480"
                    \begin{tabular}{|c|c|c|c|c|} \hline
                         Valeur de $U$ & 20 & \hspace*{1cm} $\cdots$  \hspace*{1cm}&  & \\ \hline
                         Valeur de $N$ & 0 &  $\cdots$ & & \\ \hline
                         Condition $U<70$ & vrai & $\cdots$  & vrai & faux \\ \hline
                    \end{tabular}
               \end{extern}
          \end{center}
          \item Quelle valeur est affichée à la fin de l'exécution de cet algorithme?
          \par
          Interpréter cette valeur dans le contexte de l'exercice.
     \end{enumerate}
     \item Pour tout entier $n$, on pose $v_n=u_n-80$.
     \begin{enumerate}[label=\alph*.]
          \item  Montrer que la suite $(v_n)$ est une suite géométrique de raison $0,75$.
          \item Préciser son premier terme $v_0$.
          \item En déduire que, pour tout entier $n$, $u_n=80-60\times 0,75^{n}$.
          \item Déterminer, au centime près, le montant que Maya possèdera dans sa tirelire au 1${^\text{er}}$ juin 2019.
          \item Déterminer la limite de la suite $(v_n)$.
          \item En déduire la limite de la suite $(u_n)$ et interpréter le résultat dans le contexte de l'exercice.
     \end{enumerate}
\end{enumerate}

\end{document}
µ
\documentclass[a4paper]{article}

%================================================================================================================================
%
% Packages
%
%================================================================================================================================

\usepackage[T1]{fontenc} 	% pour caractères accentués
\usepackage[utf8]{inputenc}  % encodage utf8
\usepackage[french]{babel}	% langue : français
\usepackage{fourier}			% caractères plus lisibles
\usepackage[dvipsnames]{xcolor} % couleurs
\usepackage{fancyhdr}		% réglage header footer
\usepackage{needspace}		% empêcher sauts de page mal placés
\usepackage{graphicx}		% pour inclure des graphiques
\usepackage{enumitem,cprotect}		% personnalise les listes d'items (nécessaire pour ol, al ...)
\usepackage{hyperref}		% Liens hypertexte
\usepackage{pstricks,pst-all,pst-node,pstricks-add,pst-math,pst-plot,pst-tree,pst-eucl} % pstricks
\usepackage[a4paper,includeheadfoot,top=2cm,left=3cm, bottom=2cm,right=3cm]{geometry} % marges etc.
\usepackage{comment}			% commentaires multilignes
\usepackage{amsmath,environ} % maths (matrices, etc.)
\usepackage{amssymb,makeidx}
\usepackage{bm}				% bold maths
\usepackage{tabularx}		% tableaux
\usepackage{colortbl}		% tableaux en couleur
\usepackage{fontawesome}		% Fontawesome
\usepackage{environ}			% environment with command
\usepackage{fp}				% calculs pour ps-tricks
\usepackage{multido}			% pour ps tricks
\usepackage[np]{numprint}	% formattage nombre
\usepackage{tikz,tkz-tab} 			% package principal TikZ
\usepackage{pgfplots}   % axes
\usepackage{mathrsfs}    % cursives
\usepackage{calc}			% calcul taille boites
\usepackage[scaled=0.875]{helvet} % font sans serif
\usepackage{svg} % svg
\usepackage{scrextend} % local margin
\usepackage{scratch} %scratch
\usepackage{multicol} % colonnes
%\usepackage{infix-RPN,pst-func} % formule en notation polanaise inversée
\usepackage{listings}

%================================================================================================================================
%
% Réglages de base
%
%================================================================================================================================

\lstset{
language=Python,   % R code
literate=
{á}{{\'a}}1
{à}{{\`a}}1
{ã}{{\~a}}1
{é}{{\'e}}1
{è}{{\`e}}1
{ê}{{\^e}}1
{í}{{\'i}}1
{ó}{{\'o}}1
{õ}{{\~o}}1
{ú}{{\'u}}1
{ü}{{\"u}}1
{ç}{{\c{c}}}1
{~}{{ }}1
}


\definecolor{codegreen}{rgb}{0,0.6,0}
\definecolor{codegray}{rgb}{0.5,0.5,0.5}
\definecolor{codepurple}{rgb}{0.58,0,0.82}
\definecolor{backcolour}{rgb}{0.95,0.95,0.92}

\lstdefinestyle{mystyle}{
    backgroundcolor=\color{backcolour},   
    commentstyle=\color{codegreen},
    keywordstyle=\color{magenta},
    numberstyle=\tiny\color{codegray},
    stringstyle=\color{codepurple},
    basicstyle=\ttfamily\footnotesize,
    breakatwhitespace=false,         
    breaklines=true,                 
    captionpos=b,                    
    keepspaces=true,                 
    numbers=left,                    
xleftmargin=2em,
framexleftmargin=2em,            
    showspaces=false,                
    showstringspaces=false,
    showtabs=false,                  
    tabsize=2,
    upquote=true
}

\lstset{style=mystyle}


\lstset{style=mystyle}
\newcommand{\imgdir}{C:/laragon/www/newmc/assets/imgsvg/}
\newcommand{\imgsvgdir}{C:/laragon/www/newmc/assets/imgsvg/}

\definecolor{mcgris}{RGB}{220, 220, 220}% ancien~; pour compatibilité
\definecolor{mcbleu}{RGB}{52, 152, 219}
\definecolor{mcvert}{RGB}{125, 194, 70}
\definecolor{mcmauve}{RGB}{154, 0, 215}
\definecolor{mcorange}{RGB}{255, 96, 0}
\definecolor{mcturquoise}{RGB}{0, 153, 153}
\definecolor{mcrouge}{RGB}{255, 0, 0}
\definecolor{mclightvert}{RGB}{205, 234, 190}

\definecolor{gris}{RGB}{220, 220, 220}
\definecolor{bleu}{RGB}{52, 152, 219}
\definecolor{vert}{RGB}{125, 194, 70}
\definecolor{mauve}{RGB}{154, 0, 215}
\definecolor{orange}{RGB}{255, 96, 0}
\definecolor{turquoise}{RGB}{0, 153, 153}
\definecolor{rouge}{RGB}{255, 0, 0}
\definecolor{lightvert}{RGB}{205, 234, 190}
\setitemize[0]{label=\color{lightvert}  $\bullet$}

\pagestyle{fancy}
\renewcommand{\headrulewidth}{0.2pt}
\fancyhead[L]{maths-cours.fr}
\fancyhead[R]{\thepage}
\renewcommand{\footrulewidth}{0.2pt}
\fancyfoot[C]{}

\newcolumntype{C}{>{\centering\arraybackslash}X}
\newcolumntype{s}{>{\hsize=.35\hsize\arraybackslash}X}

\setlength{\parindent}{0pt}		 
\setlength{\parskip}{3mm}
\setlength{\headheight}{1cm}

\def\ebook{ebook}
\def\book{book}
\def\web{web}
\def\type{web}

\newcommand{\vect}[1]{\overrightarrow{\,\mathstrut#1\,}}

\def\Oij{$\left(\text{O}~;~\vect{\imath},~\vect{\jmath}\right)$}
\def\Oijk{$\left(\text{O}~;~\vect{\imath},~\vect{\jmath},~\vect{k}\right)$}
\def\Ouv{$\left(\text{O}~;~\vect{u},~\vect{v}\right)$}

\hypersetup{breaklinks=true, colorlinks = true, linkcolor = OliveGreen, urlcolor = OliveGreen, citecolor = OliveGreen, pdfauthor={Didier BONNEL - https://www.maths-cours.fr} } % supprime les bordures autour des liens

\renewcommand{\arg}[0]{\text{arg}}

\everymath{\displaystyle}

%================================================================================================================================
%
% Macros - Commandes
%
%================================================================================================================================

\newcommand\meta[2]{    			% Utilisé pour créer le post HTML.
	\def\titre{titre}
	\def\url{url}
	\def\arg{#1}
	\ifx\titre\arg
		\newcommand\maintitle{#2}
		\fancyhead[L]{#2}
		{\Large\sffamily \MakeUppercase{#2}}
		\vspace{1mm}\textcolor{mcvert}{\hrule}
	\fi 
	\ifx\url\arg
		\fancyfoot[L]{\href{https://www.maths-cours.fr#2}{\black \footnotesize{https://www.maths-cours.fr#2}}}
	\fi 
}


\newcommand\TitreC[1]{    		% Titre centré
     \needspace{3\baselineskip}
     \begin{center}\textbf{#1}\end{center}
}

\newcommand\newpar{    		% paragraphe
     \par
}

\newcommand\nosp {    		% commande vide (pas d'espace)
}
\newcommand{\id}[1]{} %ignore

\newcommand\boite[2]{				% Boite simple sans titre
	\vspace{5mm}
	\setlength{\fboxrule}{0.2mm}
	\setlength{\fboxsep}{5mm}	
	\fcolorbox{#1}{#1!3}{\makebox[\linewidth-2\fboxrule-2\fboxsep]{
  		\begin{minipage}[t]{\linewidth-2\fboxrule-4\fboxsep}\setlength{\parskip}{3mm}
  			 #2
  		\end{minipage}
	}}
	\vspace{5mm}
}

\newcommand\CBox[4]{				% Boites
	\vspace{5mm}
	\setlength{\fboxrule}{0.2mm}
	\setlength{\fboxsep}{5mm}
	
	\fcolorbox{#1}{#1!3}{\makebox[\linewidth-2\fboxrule-2\fboxsep]{
		\begin{minipage}[t]{1cm}\setlength{\parskip}{3mm}
	  		\textcolor{#1}{\LARGE{#2}}    
 	 	\end{minipage}  
  		\begin{minipage}[t]{\linewidth-2\fboxrule-4\fboxsep}\setlength{\parskip}{3mm}
			\raisebox{1.2mm}{\normalsize\sffamily{\textcolor{#1}{#3}}}						
  			 #4
  		\end{minipage}
	}}
	\vspace{5mm}
}

\newcommand\cadre[3]{				% Boites convertible html
	\par
	\vspace{2mm}
	\setlength{\fboxrule}{0.1mm}
	\setlength{\fboxsep}{5mm}
	\fcolorbox{#1}{white}{\makebox[\linewidth-2\fboxrule-2\fboxsep]{
  		\begin{minipage}[t]{\linewidth-2\fboxrule-4\fboxsep}\setlength{\parskip}{3mm}
			\raisebox{-2.5mm}{\sffamily \small{\textcolor{#1}{\MakeUppercase{#2}}}}		
			\par		
  			 #3
 	 		\end{minipage}
	}}
		\vspace{2mm}
	\par
}

\newcommand\bloc[3]{				% Boites convertible html sans bordure
     \needspace{2\baselineskip}
     {\sffamily \small{\textcolor{#1}{\MakeUppercase{#2}}}}    
		\par		
  			 #3
		\par
}

\newcommand\CHelp[1]{
     \CBox{Plum}{\faInfoCircle}{À RETENIR}{#1}
}

\newcommand\CUp[1]{
     \CBox{NavyBlue}{\faThumbsOUp}{EN PRATIQUE}{#1}
}

\newcommand\CInfo[1]{
     \CBox{Sepia}{\faArrowCircleRight}{REMARQUE}{#1}
}

\newcommand\CRedac[1]{
     \CBox{PineGreen}{\faEdit}{BIEN R\'EDIGER}{#1}
}

\newcommand\CError[1]{
     \CBox{Red}{\faExclamationTriangle}{ATTENTION}{#1}
}

\newcommand\TitreExo[2]{
\needspace{4\baselineskip}
 {\sffamily\large EXERCICE #1\ (\emph{#2 points})}
\vspace{5mm}
}

\newcommand\img[2]{
          \includegraphics[width=#2\paperwidth]{\imgdir#1}
}

\newcommand\imgsvg[2]{
       \begin{center}   \includegraphics[width=#2\paperwidth]{\imgsvgdir#1} \end{center}
}


\newcommand\Lien[2]{
     \href{#1}{#2 \tiny \faExternalLink}
}
\newcommand\mcLien[2]{
     \href{https~://www.maths-cours.fr/#1}{#2 \tiny \faExternalLink}
}

\newcommand{\euro}{\eurologo{}}

%================================================================================================================================
%
% Macros - Environement
%
%================================================================================================================================

\newenvironment{tex}{ %
}
{%
}

\newenvironment{indente}{ %
	\setlength\parindent{10mm}
}

{
	\setlength\parindent{0mm}
}

\newenvironment{corrige}{%
     \needspace{3\baselineskip}
     \medskip
     \textbf{\textsc{Corrigé}}
     \medskip
}
{
}

\newenvironment{extern}{%
     \begin{center}
     }
     {
     \end{center}
}

\NewEnviron{code}{%
	\par
     \boite{gray}{\texttt{%
     \BODY
     }}
     \par
}

\newenvironment{vbloc}{% boite sans cadre empeche saut de page
     \begin{minipage}[t]{\linewidth}
     }
     {
     \end{minipage}
}
\NewEnviron{h2}{%
    \needspace{3\baselineskip}
    \vspace{0.6cm}
	\noindent \MakeUppercase{\sffamily \large \BODY}
	\vspace{1mm}\textcolor{mcgris}{\hrule}\vspace{0.4cm}
	\par
}{}

\NewEnviron{h3}{%
    \needspace{3\baselineskip}
	\vspace{5mm}
	\textsc{\BODY}
	\par
}

\NewEnviron{margeneg}{ %
\begin{addmargin}[-1cm]{0cm}
\BODY
\end{addmargin}
}

\NewEnviron{html}{%
}

\begin{document}
\meta{url}{/exercices/qcm-bac-es-l-liban-2018/}
\meta{pid}{7793}
\meta{titre}{QCM - Bac ES/L Liban 2018}
\meta{type}{exercices}
%
\begin{h2}Exercice 3 (4 points)\end{h2}
\textbf{Commun à  tous les candidats}
\medskip
\emph{Cet exercice est un questionnaire à choix multiples. Pour chacune des questions suivantes, une seule des quatre propositions est exacte. Aucune justification n'est demandée. Une bonne réponse rapporte un point. Une mauvaise réponse, plusieurs réponses ou l'absence de réponse à une question ne rapportent ni n'enlèvent de point. Pour répondre, vous recopierez \textbf{sur votre copie} le numéro de la question et indiquerez la seule bonne réponse.}
\par
Pour les questions \textbf{1.} et \textbf{2.} et \textbf{3.}, on a représenté ci-dessous la courbe représentative d'une fonction $f$ ainsi que deux de ses tangentes aux points d'abscisses respectives 2 et 4.
\begin{center}
     \begin{extern}
          \psset{unit=0.8cm}
          \def\xmin {-5}   \def\xmax {9}
          \def\ymin {-4}   \def\ymax {7}
          \begin{pspicture*}(\xmin,\ymin)(\xmax,\ymax)
               \psgrid[subgriddiv=1,  gridlabels=0, gridcolor=lightgray]
               \psaxes[arrowsize=3pt 3, ticksize=-2pt 2pt]{->}(0,0)(\xmin,\ymin)(\xmax,\ymax)
               \def\f{x x x 6 sub mul 4 add mul 32 add 8 div}                           % définition de la fonction
               \psplot[plotpoints=2000,linecolor=red]{\xmin}{\xmax}{\f}
               \rput(-3,-3.5){$\red \mathcal{C}_f$}
               \psplot[plotpoints=2000,linecolor=mcvert]{\xmin}{\xmax}{x 2 div}% Tangente en x = 4
               \psplot[plotpoints=2000,linecolor=mcvert]{\xmin}{\xmax}{x neg 5 add}% Tangente en x = 2
               \psdots[dotstyle=*, dotscale=0.5, linecolor=mcvert](4,2)(2,3)
          \end{pspicture*}
     \end{extern}
\end{center}
\begin{enumerate}
     \item $f'(4)$ est égal à~:
     \begin{center}
          \begin{tabularx}{0.8\linewidth}{|X|X|}%class="cel50 noborder"
               \hline
               \textbf{A.} \quad $2$ & \textbf{B.} \quad $-1$\\
               \hline
               \textbf{C.} \quad $0,5$ & \textbf{D.} \quad $0$\\
               \hline
          \end{tabularx}
     \end{center}
     \item $f$ est convexe sur l'intervalle~:
     \begin{center}
          \begin{tabularx}{0.8\linewidth}{|X|X|}%class="cel50 noborder"
               \hline
               \textbf{A.} \quad $]-\infty\,;\,2]$ & \textbf{B.} \quad $]-\infty\,;\,0,5]$\\
               \hline
               \textbf{C.} \quad $[0\,;\,4]$ & \textbf{D.} \quad $[2\,;\,5]$\\
               \hline
          \end{tabularx}
     \end{center}
     \item Une valeur approchée au dixième de la valeur moyenne de $f$ sur l'intervalle $[0\,;\,5]$ est~:
     \begin{center}
          \begin{tabularx}{0.8\linewidth}{|X|X|}%class="cel50 noborder"
               \hline
               \textbf{A.} \quad $-0,1$ & \textbf{B.} \quad $2,5$\\
               \hline
               \textbf{C.} \quad $2,9$ & \textbf{D.} \quad $14,5$\\
               \hline
          \end{tabularx}
     \end{center}
     \item Dans le repère ci-dessous, on a tracé la courbe représentative de la fonction de densité de probabilité d'une variable aléatoire $X$ qui suit une loi normale et telle que
     \par
     \[P(X \leqslant 649) \approx 0,158~7.\]
     \par
     On note respectivement $\mu$ et $\sigma$ l'espérance et l'écart-type de cette loi normale.
     \begin{center}
          \begin{extern}
               \psset{xunit=0.07cm, yunit=200cm, runit=1cm, arrowsize=3pt 3, algebraic=true}
               \def\xmin {-10}   \def\xmax {170}
               \def\ymin {-0.005} \def\ymax {0.025}
               \begin{pspicture*}(\xmin,\ymin)(\xmax,\ymax)
                    \psline(-5,0)(165,0)
                    \def\m{80}% moyenne
                    \def\s{20}% écart type
                    \def\f{1/(\s*sqrt(2*PI))*EXP((-((x-\m)/\s)^2)/2)}
                    \def\inf{\xmin} \def\sup{60}
                    \pscustom[fillstyle=solid,fillcolor=mcvert!20]
                    {
                         \psplot{\inf}{\sup}{\f} % courbe de f sur [inf ; sup]
                         \lineto(\sup,0)\lineto(\inf,0)
                         \closepath % indispensable !
                    }
                    \psline[linecolor=mcvert,linestyle=dashed,dash=2pt 2pt](80,0)(80,0.4)
                    \psplot[plotpoints=1000,linecolor=red]{\xmin}{\xmax}{\f}
                    \multido{\n=646+1,\i=0+20}{9}
                    {
                         \uput[d](\i,0){\n}
                         \psline(\i,0.0005)(\i,-0.0005)
                    }
               \end{pspicture*}
          \end{extern}
     \end{center}
     \textbf{A.} \quad $P(X\leqslant 651) $\nosp$\approx 0,658~7$
     \par
     \textbf{B.} \quad $P( 649  \leqslant X\leqslant 651) $\nosp$ \approx 0,683$
     \par
     \textbf{C.} \quad $\sigma=650$
     \par
     \textbf{D.} \quad $\mu=649$
\end{enumerate}

\end{document}
µ
\documentclass[a4paper]{article}

%================================================================================================================================
%
% Packages
%
%================================================================================================================================

\usepackage[T1]{fontenc} 	% pour caractères accentués
\usepackage[utf8]{inputenc}  % encodage utf8
\usepackage[french]{babel}	% langue : français
\usepackage{fourier}			% caractères plus lisibles
\usepackage[dvipsnames]{xcolor} % couleurs
\usepackage{fancyhdr}		% réglage header footer
\usepackage{needspace}		% empêcher sauts de page mal placés
\usepackage{graphicx}		% pour inclure des graphiques
\usepackage{enumitem,cprotect}		% personnalise les listes d'items (nécessaire pour ol, al ...)
\usepackage{hyperref}		% Liens hypertexte
\usepackage{pstricks,pst-all,pst-node,pstricks-add,pst-math,pst-plot,pst-tree,pst-eucl} % pstricks
\usepackage[a4paper,includeheadfoot,top=2cm,left=3cm, bottom=2cm,right=3cm]{geometry} % marges etc.
\usepackage{comment}			% commentaires multilignes
\usepackage{amsmath,environ} % maths (matrices, etc.)
\usepackage{amssymb,makeidx}
\usepackage{bm}				% bold maths
\usepackage{tabularx}		% tableaux
\usepackage{colortbl}		% tableaux en couleur
\usepackage{fontawesome}		% Fontawesome
\usepackage{environ}			% environment with command
\usepackage{fp}				% calculs pour ps-tricks
\usepackage{multido}			% pour ps tricks
\usepackage[np]{numprint}	% formattage nombre
\usepackage{tikz,tkz-tab} 			% package principal TikZ
\usepackage{pgfplots}   % axes
\usepackage{mathrsfs}    % cursives
\usepackage{calc}			% calcul taille boites
\usepackage[scaled=0.875]{helvet} % font sans serif
\usepackage{svg} % svg
\usepackage{scrextend} % local margin
\usepackage{scratch} %scratch
\usepackage{multicol} % colonnes
%\usepackage{infix-RPN,pst-func} % formule en notation polanaise inversée
\usepackage{listings}

%================================================================================================================================
%
% Réglages de base
%
%================================================================================================================================

\lstset{
language=Python,   % R code
literate=
{á}{{\'a}}1
{à}{{\`a}}1
{ã}{{\~a}}1
{é}{{\'e}}1
{è}{{\`e}}1
{ê}{{\^e}}1
{í}{{\'i}}1
{ó}{{\'o}}1
{õ}{{\~o}}1
{ú}{{\'u}}1
{ü}{{\"u}}1
{ç}{{\c{c}}}1
{~}{{ }}1
}


\definecolor{codegreen}{rgb}{0,0.6,0}
\definecolor{codegray}{rgb}{0.5,0.5,0.5}
\definecolor{codepurple}{rgb}{0.58,0,0.82}
\definecolor{backcolour}{rgb}{0.95,0.95,0.92}

\lstdefinestyle{mystyle}{
    backgroundcolor=\color{backcolour},   
    commentstyle=\color{codegreen},
    keywordstyle=\color{magenta},
    numberstyle=\tiny\color{codegray},
    stringstyle=\color{codepurple},
    basicstyle=\ttfamily\footnotesize,
    breakatwhitespace=false,         
    breaklines=true,                 
    captionpos=b,                    
    keepspaces=true,                 
    numbers=left,                    
xleftmargin=2em,
framexleftmargin=2em,            
    showspaces=false,                
    showstringspaces=false,
    showtabs=false,                  
    tabsize=2,
    upquote=true
}

\lstset{style=mystyle}


\lstset{style=mystyle}
\newcommand{\imgdir}{C:/laragon/www/newmc/assets/imgsvg/}
\newcommand{\imgsvgdir}{C:/laragon/www/newmc/assets/imgsvg/}

\definecolor{mcgris}{RGB}{220, 220, 220}% ancien~; pour compatibilité
\definecolor{mcbleu}{RGB}{52, 152, 219}
\definecolor{mcvert}{RGB}{125, 194, 70}
\definecolor{mcmauve}{RGB}{154, 0, 215}
\definecolor{mcorange}{RGB}{255, 96, 0}
\definecolor{mcturquoise}{RGB}{0, 153, 153}
\definecolor{mcrouge}{RGB}{255, 0, 0}
\definecolor{mclightvert}{RGB}{205, 234, 190}

\definecolor{gris}{RGB}{220, 220, 220}
\definecolor{bleu}{RGB}{52, 152, 219}
\definecolor{vert}{RGB}{125, 194, 70}
\definecolor{mauve}{RGB}{154, 0, 215}
\definecolor{orange}{RGB}{255, 96, 0}
\definecolor{turquoise}{RGB}{0, 153, 153}
\definecolor{rouge}{RGB}{255, 0, 0}
\definecolor{lightvert}{RGB}{205, 234, 190}
\setitemize[0]{label=\color{lightvert}  $\bullet$}

\pagestyle{fancy}
\renewcommand{\headrulewidth}{0.2pt}
\fancyhead[L]{maths-cours.fr}
\fancyhead[R]{\thepage}
\renewcommand{\footrulewidth}{0.2pt}
\fancyfoot[C]{}

\newcolumntype{C}{>{\centering\arraybackslash}X}
\newcolumntype{s}{>{\hsize=.35\hsize\arraybackslash}X}

\setlength{\parindent}{0pt}		 
\setlength{\parskip}{3mm}
\setlength{\headheight}{1cm}

\def\ebook{ebook}
\def\book{book}
\def\web{web}
\def\type{web}

\newcommand{\vect}[1]{\overrightarrow{\,\mathstrut#1\,}}

\def\Oij{$\left(\text{O}~;~\vect{\imath},~\vect{\jmath}\right)$}
\def\Oijk{$\left(\text{O}~;~\vect{\imath},~\vect{\jmath},~\vect{k}\right)$}
\def\Ouv{$\left(\text{O}~;~\vect{u},~\vect{v}\right)$}

\hypersetup{breaklinks=true, colorlinks = true, linkcolor = OliveGreen, urlcolor = OliveGreen, citecolor = OliveGreen, pdfauthor={Didier BONNEL - https://www.maths-cours.fr} } % supprime les bordures autour des liens

\renewcommand{\arg}[0]{\text{arg}}

\everymath{\displaystyle}

%================================================================================================================================
%
% Macros - Commandes
%
%================================================================================================================================

\newcommand\meta[2]{    			% Utilisé pour créer le post HTML.
	\def\titre{titre}
	\def\url{url}
	\def\arg{#1}
	\ifx\titre\arg
		\newcommand\maintitle{#2}
		\fancyhead[L]{#2}
		{\Large\sffamily \MakeUppercase{#2}}
		\vspace{1mm}\textcolor{mcvert}{\hrule}
	\fi 
	\ifx\url\arg
		\fancyfoot[L]{\href{https://www.maths-cours.fr#2}{\black \footnotesize{https://www.maths-cours.fr#2}}}
	\fi 
}


\newcommand\TitreC[1]{    		% Titre centré
     \needspace{3\baselineskip}
     \begin{center}\textbf{#1}\end{center}
}

\newcommand\newpar{    		% paragraphe
     \par
}

\newcommand\nosp {    		% commande vide (pas d'espace)
}
\newcommand{\id}[1]{} %ignore

\newcommand\boite[2]{				% Boite simple sans titre
	\vspace{5mm}
	\setlength{\fboxrule}{0.2mm}
	\setlength{\fboxsep}{5mm}	
	\fcolorbox{#1}{#1!3}{\makebox[\linewidth-2\fboxrule-2\fboxsep]{
  		\begin{minipage}[t]{\linewidth-2\fboxrule-4\fboxsep}\setlength{\parskip}{3mm}
  			 #2
  		\end{minipage}
	}}
	\vspace{5mm}
}

\newcommand\CBox[4]{				% Boites
	\vspace{5mm}
	\setlength{\fboxrule}{0.2mm}
	\setlength{\fboxsep}{5mm}
	
	\fcolorbox{#1}{#1!3}{\makebox[\linewidth-2\fboxrule-2\fboxsep]{
		\begin{minipage}[t]{1cm}\setlength{\parskip}{3mm}
	  		\textcolor{#1}{\LARGE{#2}}    
 	 	\end{minipage}  
  		\begin{minipage}[t]{\linewidth-2\fboxrule-4\fboxsep}\setlength{\parskip}{3mm}
			\raisebox{1.2mm}{\normalsize\sffamily{\textcolor{#1}{#3}}}						
  			 #4
  		\end{minipage}
	}}
	\vspace{5mm}
}

\newcommand\cadre[3]{				% Boites convertible html
	\par
	\vspace{2mm}
	\setlength{\fboxrule}{0.1mm}
	\setlength{\fboxsep}{5mm}
	\fcolorbox{#1}{white}{\makebox[\linewidth-2\fboxrule-2\fboxsep]{
  		\begin{minipage}[t]{\linewidth-2\fboxrule-4\fboxsep}\setlength{\parskip}{3mm}
			\raisebox{-2.5mm}{\sffamily \small{\textcolor{#1}{\MakeUppercase{#2}}}}		
			\par		
  			 #3
 	 		\end{minipage}
	}}
		\vspace{2mm}
	\par
}

\newcommand\bloc[3]{				% Boites convertible html sans bordure
     \needspace{2\baselineskip}
     {\sffamily \small{\textcolor{#1}{\MakeUppercase{#2}}}}    
		\par		
  			 #3
		\par
}

\newcommand\CHelp[1]{
     \CBox{Plum}{\faInfoCircle}{À RETENIR}{#1}
}

\newcommand\CUp[1]{
     \CBox{NavyBlue}{\faThumbsOUp}{EN PRATIQUE}{#1}
}

\newcommand\CInfo[1]{
     \CBox{Sepia}{\faArrowCircleRight}{REMARQUE}{#1}
}

\newcommand\CRedac[1]{
     \CBox{PineGreen}{\faEdit}{BIEN R\'EDIGER}{#1}
}

\newcommand\CError[1]{
     \CBox{Red}{\faExclamationTriangle}{ATTENTION}{#1}
}

\newcommand\TitreExo[2]{
\needspace{4\baselineskip}
 {\sffamily\large EXERCICE #1\ (\emph{#2 points})}
\vspace{5mm}
}

\newcommand\img[2]{
          \includegraphics[width=#2\paperwidth]{\imgdir#1}
}

\newcommand\imgsvg[2]{
       \begin{center}   \includegraphics[width=#2\paperwidth]{\imgsvgdir#1} \end{center}
}


\newcommand\Lien[2]{
     \href{#1}{#2 \tiny \faExternalLink}
}
\newcommand\mcLien[2]{
     \href{https~://www.maths-cours.fr/#1}{#2 \tiny \faExternalLink}
}

\newcommand{\euro}{\eurologo{}}

%================================================================================================================================
%
% Macros - Environement
%
%================================================================================================================================

\newenvironment{tex}{ %
}
{%
}

\newenvironment{indente}{ %
	\setlength\parindent{10mm}
}

{
	\setlength\parindent{0mm}
}

\newenvironment{corrige}{%
     \needspace{3\baselineskip}
     \medskip
     \textbf{\textsc{Corrigé}}
     \medskip
}
{
}

\newenvironment{extern}{%
     \begin{center}
     }
     {
     \end{center}
}

\NewEnviron{code}{%
	\par
     \boite{gray}{\texttt{%
     \BODY
     }}
     \par
}

\newenvironment{vbloc}{% boite sans cadre empeche saut de page
     \begin{minipage}[t]{\linewidth}
     }
     {
     \end{minipage}
}
\NewEnviron{h2}{%
    \needspace{3\baselineskip}
    \vspace{0.6cm}
	\noindent \MakeUppercase{\sffamily \large \BODY}
	\vspace{1mm}\textcolor{mcgris}{\hrule}\vspace{0.4cm}
	\par
}{}

\NewEnviron{h3}{%
    \needspace{3\baselineskip}
	\vspace{5mm}
	\textsc{\BODY}
	\par
}

\NewEnviron{margeneg}{ %
\begin{addmargin}[-1cm]{0cm}
\BODY
\end{addmargin}
}

\NewEnviron{html}{%
}

\begin{document}
\meta{url}{/exercices/fonctions-bac-es-l-liban-2018/}
\meta{pid}{7817}
\meta{titre}{Fonctions - Bac ES/L Liban 2018}
\meta{type}{exercices}
%
\begin{h2}Exercice 4 (5 points)\end{h2}
\textbf{Commun à tous les candidats}
\medskip
\begin{enumerate}
     \item Soit $f$ la fonction définie sur l'intervalle $[1\,;\,25]$ par
     \par
     \[f(x)=\dfrac{x+2-\ln(x)}{x}.\]
     \begin{enumerate}[label=\alph*.]
          \item  On admet que $f$ est dérivable sur $[1\,;\,25]$.
          \par
          Démontrer que pour tout réel $x$ appartient à l'intervalle $[1\,;\,25]$,
          \par
          \[f'(x)=\dfrac{-3+\ln(x)}{x^2}.\]
          \medskip
          \item Résoudre dans $[1\,;\,25]$ l'inéquation $-3+\ln(x)>0$.
          \item Dresser le tableau des variations de la fonction $f$ sur $[1\,;\,25]$.
          \item Démontrer que dans l'intervalle $[1\,;\,25]$, l'équation $f(x)=1,5$ admet une seule solution. On notera $\alpha$ cette solution.
          \item Déterminer un encadrement d'amplitude $0,01$ de $\alpha$ à l'aide de la calculatrice.
     \end{enumerate}
     \item Une entreprise fabrique chaque jour entre 100 et 2~500 pièces électroniques pour des vidéo-projecteurs. Toutes les pièces fabriquées sont identiques.
     \par
     On admet que lorsque $x$ centaines de pièces sont fabriquées, avec $1 \leqslant x \leqslant 25$, le coût moyen de fabrication d'une pièce est de $f(x)$ euros.
     \par
     En utilisant les résultats obtenus à la question \textbf{1.}:
     \begin{enumerate}[label=\alph*.]
          \item Déterminer, à l'unité près, le nombre de pièces à fabriquer pour que le coût moyen de fabrication d'une pièce soit minimal.
          \par
          Déterminer alors ce coût moyen, au centime d'euro près.
          \par
          \item Déterminer le nombre minimal de pièces à fabriquer pour que le coût moyen de fabrication d'une pièce soit inférieur ou égal à $1,50$ euro.
          \par
          \item Est-il possible que le coût moyen d'une pièce soit de 50 centimes? Justifier.
     \end{enumerate}
\end{enumerate}

\end{document}
µ
\documentclass[a4paper]{article}

%================================================================================================================================
%
% Packages
%
%================================================================================================================================

\usepackage[T1]{fontenc} 	% pour caractères accentués
\usepackage[utf8]{inputenc}  % encodage utf8
\usepackage[french]{babel}	% langue : français
\usepackage{fourier}			% caractères plus lisibles
\usepackage[dvipsnames]{xcolor} % couleurs
\usepackage{fancyhdr}		% réglage header footer
\usepackage{needspace}		% empêcher sauts de page mal placés
\usepackage{graphicx}		% pour inclure des graphiques
\usepackage{enumitem,cprotect}		% personnalise les listes d'items (nécessaire pour ol, al ...)
\usepackage{hyperref}		% Liens hypertexte
\usepackage{pstricks,pst-all,pst-node,pstricks-add,pst-math,pst-plot,pst-tree,pst-eucl} % pstricks
\usepackage[a4paper,includeheadfoot,top=2cm,left=3cm, bottom=2cm,right=3cm]{geometry} % marges etc.
\usepackage{comment}			% commentaires multilignes
\usepackage{amsmath,environ} % maths (matrices, etc.)
\usepackage{amssymb,makeidx}
\usepackage{bm}				% bold maths
\usepackage{tabularx}		% tableaux
\usepackage{colortbl}		% tableaux en couleur
\usepackage{fontawesome}		% Fontawesome
\usepackage{environ}			% environment with command
\usepackage{fp}				% calculs pour ps-tricks
\usepackage{multido}			% pour ps tricks
\usepackage[np]{numprint}	% formattage nombre
\usepackage{tikz,tkz-tab} 			% package principal TikZ
\usepackage{pgfplots}   % axes
\usepackage{mathrsfs}    % cursives
\usepackage{calc}			% calcul taille boites
\usepackage[scaled=0.875]{helvet} % font sans serif
\usepackage{svg} % svg
\usepackage{scrextend} % local margin
\usepackage{scratch} %scratch
\usepackage{multicol} % colonnes
%\usepackage{infix-RPN,pst-func} % formule en notation polanaise inversée
\usepackage{listings}

%================================================================================================================================
%
% Réglages de base
%
%================================================================================================================================

\lstset{
language=Python,   % R code
literate=
{á}{{\'a}}1
{à}{{\`a}}1
{ã}{{\~a}}1
{é}{{\'e}}1
{è}{{\`e}}1
{ê}{{\^e}}1
{í}{{\'i}}1
{ó}{{\'o}}1
{õ}{{\~o}}1
{ú}{{\'u}}1
{ü}{{\"u}}1
{ç}{{\c{c}}}1
{~}{{ }}1
}


\definecolor{codegreen}{rgb}{0,0.6,0}
\definecolor{codegray}{rgb}{0.5,0.5,0.5}
\definecolor{codepurple}{rgb}{0.58,0,0.82}
\definecolor{backcolour}{rgb}{0.95,0.95,0.92}

\lstdefinestyle{mystyle}{
    backgroundcolor=\color{backcolour},   
    commentstyle=\color{codegreen},
    keywordstyle=\color{magenta},
    numberstyle=\tiny\color{codegray},
    stringstyle=\color{codepurple},
    basicstyle=\ttfamily\footnotesize,
    breakatwhitespace=false,         
    breaklines=true,                 
    captionpos=b,                    
    keepspaces=true,                 
    numbers=left,                    
xleftmargin=2em,
framexleftmargin=2em,            
    showspaces=false,                
    showstringspaces=false,
    showtabs=false,                  
    tabsize=2,
    upquote=true
}

\lstset{style=mystyle}


\lstset{style=mystyle}
\newcommand{\imgdir}{C:/laragon/www/newmc/assets/imgsvg/}
\newcommand{\imgsvgdir}{C:/laragon/www/newmc/assets/imgsvg/}

\definecolor{mcgris}{RGB}{220, 220, 220}% ancien~; pour compatibilité
\definecolor{mcbleu}{RGB}{52, 152, 219}
\definecolor{mcvert}{RGB}{125, 194, 70}
\definecolor{mcmauve}{RGB}{154, 0, 215}
\definecolor{mcorange}{RGB}{255, 96, 0}
\definecolor{mcturquoise}{RGB}{0, 153, 153}
\definecolor{mcrouge}{RGB}{255, 0, 0}
\definecolor{mclightvert}{RGB}{205, 234, 190}

\definecolor{gris}{RGB}{220, 220, 220}
\definecolor{bleu}{RGB}{52, 152, 219}
\definecolor{vert}{RGB}{125, 194, 70}
\definecolor{mauve}{RGB}{154, 0, 215}
\definecolor{orange}{RGB}{255, 96, 0}
\definecolor{turquoise}{RGB}{0, 153, 153}
\definecolor{rouge}{RGB}{255, 0, 0}
\definecolor{lightvert}{RGB}{205, 234, 190}
\setitemize[0]{label=\color{lightvert}  $\bullet$}

\pagestyle{fancy}
\renewcommand{\headrulewidth}{0.2pt}
\fancyhead[L]{maths-cours.fr}
\fancyhead[R]{\thepage}
\renewcommand{\footrulewidth}{0.2pt}
\fancyfoot[C]{}

\newcolumntype{C}{>{\centering\arraybackslash}X}
\newcolumntype{s}{>{\hsize=.35\hsize\arraybackslash}X}

\setlength{\parindent}{0pt}		 
\setlength{\parskip}{3mm}
\setlength{\headheight}{1cm}

\def\ebook{ebook}
\def\book{book}
\def\web{web}
\def\type{web}

\newcommand{\vect}[1]{\overrightarrow{\,\mathstrut#1\,}}

\def\Oij{$\left(\text{O}~;~\vect{\imath},~\vect{\jmath}\right)$}
\def\Oijk{$\left(\text{O}~;~\vect{\imath},~\vect{\jmath},~\vect{k}\right)$}
\def\Ouv{$\left(\text{O}~;~\vect{u},~\vect{v}\right)$}

\hypersetup{breaklinks=true, colorlinks = true, linkcolor = OliveGreen, urlcolor = OliveGreen, citecolor = OliveGreen, pdfauthor={Didier BONNEL - https://www.maths-cours.fr} } % supprime les bordures autour des liens

\renewcommand{\arg}[0]{\text{arg}}

\everymath{\displaystyle}

%================================================================================================================================
%
% Macros - Commandes
%
%================================================================================================================================

\newcommand\meta[2]{    			% Utilisé pour créer le post HTML.
	\def\titre{titre}
	\def\url{url}
	\def\arg{#1}
	\ifx\titre\arg
		\newcommand\maintitle{#2}
		\fancyhead[L]{#2}
		{\Large\sffamily \MakeUppercase{#2}}
		\vspace{1mm}\textcolor{mcvert}{\hrule}
	\fi 
	\ifx\url\arg
		\fancyfoot[L]{\href{https://www.maths-cours.fr#2}{\black \footnotesize{https://www.maths-cours.fr#2}}}
	\fi 
}


\newcommand\TitreC[1]{    		% Titre centré
     \needspace{3\baselineskip}
     \begin{center}\textbf{#1}\end{center}
}

\newcommand\newpar{    		% paragraphe
     \par
}

\newcommand\nosp {    		% commande vide (pas d'espace)
}
\newcommand{\id}[1]{} %ignore

\newcommand\boite[2]{				% Boite simple sans titre
	\vspace{5mm}
	\setlength{\fboxrule}{0.2mm}
	\setlength{\fboxsep}{5mm}	
	\fcolorbox{#1}{#1!3}{\makebox[\linewidth-2\fboxrule-2\fboxsep]{
  		\begin{minipage}[t]{\linewidth-2\fboxrule-4\fboxsep}\setlength{\parskip}{3mm}
  			 #2
  		\end{minipage}
	}}
	\vspace{5mm}
}

\newcommand\CBox[4]{				% Boites
	\vspace{5mm}
	\setlength{\fboxrule}{0.2mm}
	\setlength{\fboxsep}{5mm}
	
	\fcolorbox{#1}{#1!3}{\makebox[\linewidth-2\fboxrule-2\fboxsep]{
		\begin{minipage}[t]{1cm}\setlength{\parskip}{3mm}
	  		\textcolor{#1}{\LARGE{#2}}    
 	 	\end{minipage}  
  		\begin{minipage}[t]{\linewidth-2\fboxrule-4\fboxsep}\setlength{\parskip}{3mm}
			\raisebox{1.2mm}{\normalsize\sffamily{\textcolor{#1}{#3}}}						
  			 #4
  		\end{minipage}
	}}
	\vspace{5mm}
}

\newcommand\cadre[3]{				% Boites convertible html
	\par
	\vspace{2mm}
	\setlength{\fboxrule}{0.1mm}
	\setlength{\fboxsep}{5mm}
	\fcolorbox{#1}{white}{\makebox[\linewidth-2\fboxrule-2\fboxsep]{
  		\begin{minipage}[t]{\linewidth-2\fboxrule-4\fboxsep}\setlength{\parskip}{3mm}
			\raisebox{-2.5mm}{\sffamily \small{\textcolor{#1}{\MakeUppercase{#2}}}}		
			\par		
  			 #3
 	 		\end{minipage}
	}}
		\vspace{2mm}
	\par
}

\newcommand\bloc[3]{				% Boites convertible html sans bordure
     \needspace{2\baselineskip}
     {\sffamily \small{\textcolor{#1}{\MakeUppercase{#2}}}}    
		\par		
  			 #3
		\par
}

\newcommand\CHelp[1]{
     \CBox{Plum}{\faInfoCircle}{À RETENIR}{#1}
}

\newcommand\CUp[1]{
     \CBox{NavyBlue}{\faThumbsOUp}{EN PRATIQUE}{#1}
}

\newcommand\CInfo[1]{
     \CBox{Sepia}{\faArrowCircleRight}{REMARQUE}{#1}
}

\newcommand\CRedac[1]{
     \CBox{PineGreen}{\faEdit}{BIEN R\'EDIGER}{#1}
}

\newcommand\CError[1]{
     \CBox{Red}{\faExclamationTriangle}{ATTENTION}{#1}
}

\newcommand\TitreExo[2]{
\needspace{4\baselineskip}
 {\sffamily\large EXERCICE #1\ (\emph{#2 points})}
\vspace{5mm}
}

\newcommand\img[2]{
          \includegraphics[width=#2\paperwidth]{\imgdir#1}
}

\newcommand\imgsvg[2]{
       \begin{center}   \includegraphics[width=#2\paperwidth]{\imgsvgdir#1} \end{center}
}


\newcommand\Lien[2]{
     \href{#1}{#2 \tiny \faExternalLink}
}
\newcommand\mcLien[2]{
     \href{https~://www.maths-cours.fr/#1}{#2 \tiny \faExternalLink}
}

\newcommand{\euro}{\eurologo{}}

%================================================================================================================================
%
% Macros - Environement
%
%================================================================================================================================

\newenvironment{tex}{ %
}
{%
}

\newenvironment{indente}{ %
	\setlength\parindent{10mm}
}

{
	\setlength\parindent{0mm}
}

\newenvironment{corrige}{%
     \needspace{3\baselineskip}
     \medskip
     \textbf{\textsc{Corrigé}}
     \medskip
}
{
}

\newenvironment{extern}{%
     \begin{center}
     }
     {
     \end{center}
}

\NewEnviron{code}{%
	\par
     \boite{gray}{\texttt{%
     \BODY
     }}
     \par
}

\newenvironment{vbloc}{% boite sans cadre empeche saut de page
     \begin{minipage}[t]{\linewidth}
     }
     {
     \end{minipage}
}
\NewEnviron{h2}{%
    \needspace{3\baselineskip}
    \vspace{0.6cm}
	\noindent \MakeUppercase{\sffamily \large \BODY}
	\vspace{1mm}\textcolor{mcgris}{\hrule}\vspace{0.4cm}
	\par
}{}

\NewEnviron{h3}{%
    \needspace{3\baselineskip}
	\vspace{5mm}
	\textsc{\BODY}
	\par
}

\NewEnviron{margeneg}{ %
\begin{addmargin}[-1cm]{0cm}
\BODY
\end{addmargin}
}

\NewEnviron{html}{%
}

\begin{document}
\meta{url}{/exercices/graphes-probabilistes-bac-es-liban-2018-spe/}
\meta{pid}{7823}
\meta{titre}{Graphes probabilistes - Bac ES Liban 2018 (spé)}
\meta{type}{exercices}
%
\begin{h2}Exercice 2 (5 points)\end{h2}
\textbf{Candidats de ES ayant choisi la spécialité \og mathématiques \fg{} }
\medskip
Dans un pays deux opérateurs se partagent le marché des télécommunications mobiles. Une étude révèle que chaque année~:
\begin{itemize}
     \item parmi les clients de l'opérateur \emph{EfficaceRéseau}, 70\,\% se réabonnent à ce même opérateur et 30\,\% souscrivent un contrat avec l'opérateur \emph{GenialPhone};
     \item parmi les clients de l'opérateur \emph{GenialPhone}, 55\,\% se réabonnent à ce même opérateur et 45\,\% souscrivent un contrat avec l'opérateur \emph{EfficaceRéseau}.
\end{itemize}
\bigskip
On note $E$ l'état~: \og la personne possède un contrat chez l'opérateur \emph{EfficaceRéseau} \fg{} et $G$ l'état~: \og la personne possède un contrat chez l'opérateur \emph{GenialPhone} \fg{}.
\par
À partir de 2018, on choisit au hasard un client de l'un des deux opérateurs.
\bigskip
On note également~:
\begin{itemize}
     \item $e_n$ la probabilité que le client possède un contrat avec l'opérateur \emph{EfficaceRéseau} au 1${^\text{er}}$ janvier $(2018+n)$;
     \item $g_n$ la probabilité que le client possède un contrat avec l'opérateur \emph{GenialPhone} au 1${^\text{er}}$ janvier $(2018+n)$;
     \item $P_n = \begin{pmatrix} e_n & g_n \end{pmatrix}$ désigne la matrice ligne traduisant l'état probabiliste du système au 1${^\text{er}}$ janvier $(2018+n)$.
\end{itemize}
\medskip
Au 1${^\text{er}}$ janvier 2018, on suppose que 10\,\% des clients possèdent un contrat chez \emph{EfficaceRéseau}, ainsi $P_0 = \begin{pmatrix} 0,1 & 0,9 \end{pmatrix}$.
\bigskip
\begin{enumerate}
     \item Représenter cette situation par un graphe probabiliste de sommets $E$ et $G$.
     \par
     \item
     \begin{enumerate}[label=\alph*.]
          \item Déterminer la matrice de transition $M$ associée au graphe en rangeant les sommets dans l'ordre alphabétique.
          \item Vérifier qu'au 1${^\text{er}}$ janvier 2020, environ 57\,\% des clients ont un contrat avec l'opérateur \emph{EfficaceRéseau}.
     \end{enumerate}
     \par
     \item
     \begin{enumerate}[label=\alph*.]
          \item On rappelle que pour tout entier naturel $n$, $P_{n+1} = P_n\times M$.
          \par
          Exprimer $e_{n+1}$ en fonction de $e_n$ et $g_n$.
          \par
          \item En déduire que pour tout entier naturel $n$, $e_{n+1} = 0,25 e_n + 0,45$.
     \end{enumerate}
     \par
     \item
     \begin{enumerate}[label=\alph*.]
          \item Recopier et compléter l'algorithme ci-dessous de façon à ce qu'il affiche l'état probabiliste au 1${^\text{er}}$ janvier $(2018+n)$~:
          \begin{center}
               \begin{extern}%width="320"
                    \renewcommand{\arraystretch}{1.3}
                    \begin{tabular}{|l|}
                         \hline
                         $E\longleftarrow 0,1$\\
                         $G \longleftarrow 0,9$\\
                         Pour $I$ allant de 1 à $N$\hspace*{3cm}\\
                         \hspace*{1cm} $E \longleftarrow \cdots \times E + \cdots $\\
                         \hspace*{1cm} $G \longleftarrow \cdots$\\
                         Fin Pour\\
                         Afficher $E$ et $G$\\
                         \hline
                    \end{tabular}
               \end{extern}
          \end{center}
          \item Déterminer l'affichage de cet algorithme pour $N=3$. Arrondir au centième.
          \par
          \item Déterminer l'état stable du système et interpréter votre réponse dans le contexte de l'exercice.
     \end{enumerate}
\end{enumerate}
\par

\end{document}
µ
\documentclass[a4paper]{article}

%================================================================================================================================
%
% Packages
%
%================================================================================================================================

\usepackage[T1]{fontenc} 	% pour caractères accentués
\usepackage[utf8]{inputenc}  % encodage utf8
\usepackage[french]{babel}	% langue : français
\usepackage{fourier}			% caractères plus lisibles
\usepackage[dvipsnames]{xcolor} % couleurs
\usepackage{fancyhdr}		% réglage header footer
\usepackage{needspace}		% empêcher sauts de page mal placés
\usepackage{graphicx}		% pour inclure des graphiques
\usepackage{enumitem,cprotect}		% personnalise les listes d'items (nécessaire pour ol, al ...)
\usepackage{hyperref}		% Liens hypertexte
\usepackage{pstricks,pst-all,pst-node,pstricks-add,pst-math,pst-plot,pst-tree,pst-eucl} % pstricks
\usepackage[a4paper,includeheadfoot,top=2cm,left=3cm, bottom=2cm,right=3cm]{geometry} % marges etc.
\usepackage{comment}			% commentaires multilignes
\usepackage{amsmath,environ} % maths (matrices, etc.)
\usepackage{amssymb,makeidx}
\usepackage{bm}				% bold maths
\usepackage{tabularx}		% tableaux
\usepackage{colortbl}		% tableaux en couleur
\usepackage{fontawesome}		% Fontawesome
\usepackage{environ}			% environment with command
\usepackage{fp}				% calculs pour ps-tricks
\usepackage{multido}			% pour ps tricks
\usepackage[np]{numprint}	% formattage nombre
\usepackage{tikz,tkz-tab} 			% package principal TikZ
\usepackage{pgfplots}   % axes
\usepackage{mathrsfs}    % cursives
\usepackage{calc}			% calcul taille boites
\usepackage[scaled=0.875]{helvet} % font sans serif
\usepackage{svg} % svg
\usepackage{scrextend} % local margin
\usepackage{scratch} %scratch
\usepackage{multicol} % colonnes
%\usepackage{infix-RPN,pst-func} % formule en notation polanaise inversée
\usepackage{listings}

%================================================================================================================================
%
% Réglages de base
%
%================================================================================================================================

\lstset{
language=Python,   % R code
literate=
{á}{{\'a}}1
{à}{{\`a}}1
{ã}{{\~a}}1
{é}{{\'e}}1
{è}{{\`e}}1
{ê}{{\^e}}1
{í}{{\'i}}1
{ó}{{\'o}}1
{õ}{{\~o}}1
{ú}{{\'u}}1
{ü}{{\"u}}1
{ç}{{\c{c}}}1
{~}{{ }}1
}


\definecolor{codegreen}{rgb}{0,0.6,0}
\definecolor{codegray}{rgb}{0.5,0.5,0.5}
\definecolor{codepurple}{rgb}{0.58,0,0.82}
\definecolor{backcolour}{rgb}{0.95,0.95,0.92}

\lstdefinestyle{mystyle}{
    backgroundcolor=\color{backcolour},   
    commentstyle=\color{codegreen},
    keywordstyle=\color{magenta},
    numberstyle=\tiny\color{codegray},
    stringstyle=\color{codepurple},
    basicstyle=\ttfamily\footnotesize,
    breakatwhitespace=false,         
    breaklines=true,                 
    captionpos=b,                    
    keepspaces=true,                 
    numbers=left,                    
xleftmargin=2em,
framexleftmargin=2em,            
    showspaces=false,                
    showstringspaces=false,
    showtabs=false,                  
    tabsize=2,
    upquote=true
}

\lstset{style=mystyle}


\lstset{style=mystyle}
\newcommand{\imgdir}{C:/laragon/www/newmc/assets/imgsvg/}
\newcommand{\imgsvgdir}{C:/laragon/www/newmc/assets/imgsvg/}

\definecolor{mcgris}{RGB}{220, 220, 220}% ancien~; pour compatibilité
\definecolor{mcbleu}{RGB}{52, 152, 219}
\definecolor{mcvert}{RGB}{125, 194, 70}
\definecolor{mcmauve}{RGB}{154, 0, 215}
\definecolor{mcorange}{RGB}{255, 96, 0}
\definecolor{mcturquoise}{RGB}{0, 153, 153}
\definecolor{mcrouge}{RGB}{255, 0, 0}
\definecolor{mclightvert}{RGB}{205, 234, 190}

\definecolor{gris}{RGB}{220, 220, 220}
\definecolor{bleu}{RGB}{52, 152, 219}
\definecolor{vert}{RGB}{125, 194, 70}
\definecolor{mauve}{RGB}{154, 0, 215}
\definecolor{orange}{RGB}{255, 96, 0}
\definecolor{turquoise}{RGB}{0, 153, 153}
\definecolor{rouge}{RGB}{255, 0, 0}
\definecolor{lightvert}{RGB}{205, 234, 190}
\setitemize[0]{label=\color{lightvert}  $\bullet$}

\pagestyle{fancy}
\renewcommand{\headrulewidth}{0.2pt}
\fancyhead[L]{maths-cours.fr}
\fancyhead[R]{\thepage}
\renewcommand{\footrulewidth}{0.2pt}
\fancyfoot[C]{}

\newcolumntype{C}{>{\centering\arraybackslash}X}
\newcolumntype{s}{>{\hsize=.35\hsize\arraybackslash}X}

\setlength{\parindent}{0pt}		 
\setlength{\parskip}{3mm}
\setlength{\headheight}{1cm}

\def\ebook{ebook}
\def\book{book}
\def\web{web}
\def\type{web}

\newcommand{\vect}[1]{\overrightarrow{\,\mathstrut#1\,}}

\def\Oij{$\left(\text{O}~;~\vect{\imath},~\vect{\jmath}\right)$}
\def\Oijk{$\left(\text{O}~;~\vect{\imath},~\vect{\jmath},~\vect{k}\right)$}
\def\Ouv{$\left(\text{O}~;~\vect{u},~\vect{v}\right)$}

\hypersetup{breaklinks=true, colorlinks = true, linkcolor = OliveGreen, urlcolor = OliveGreen, citecolor = OliveGreen, pdfauthor={Didier BONNEL - https://www.maths-cours.fr} } % supprime les bordures autour des liens

\renewcommand{\arg}[0]{\text{arg}}

\everymath{\displaystyle}

%================================================================================================================================
%
% Macros - Commandes
%
%================================================================================================================================

\newcommand\meta[2]{    			% Utilisé pour créer le post HTML.
	\def\titre{titre}
	\def\url{url}
	\def\arg{#1}
	\ifx\titre\arg
		\newcommand\maintitle{#2}
		\fancyhead[L]{#2}
		{\Large\sffamily \MakeUppercase{#2}}
		\vspace{1mm}\textcolor{mcvert}{\hrule}
	\fi 
	\ifx\url\arg
		\fancyfoot[L]{\href{https://www.maths-cours.fr#2}{\black \footnotesize{https://www.maths-cours.fr#2}}}
	\fi 
}


\newcommand\TitreC[1]{    		% Titre centré
     \needspace{3\baselineskip}
     \begin{center}\textbf{#1}\end{center}
}

\newcommand\newpar{    		% paragraphe
     \par
}

\newcommand\nosp {    		% commande vide (pas d'espace)
}
\newcommand{\id}[1]{} %ignore

\newcommand\boite[2]{				% Boite simple sans titre
	\vspace{5mm}
	\setlength{\fboxrule}{0.2mm}
	\setlength{\fboxsep}{5mm}	
	\fcolorbox{#1}{#1!3}{\makebox[\linewidth-2\fboxrule-2\fboxsep]{
  		\begin{minipage}[t]{\linewidth-2\fboxrule-4\fboxsep}\setlength{\parskip}{3mm}
  			 #2
  		\end{minipage}
	}}
	\vspace{5mm}
}

\newcommand\CBox[4]{				% Boites
	\vspace{5mm}
	\setlength{\fboxrule}{0.2mm}
	\setlength{\fboxsep}{5mm}
	
	\fcolorbox{#1}{#1!3}{\makebox[\linewidth-2\fboxrule-2\fboxsep]{
		\begin{minipage}[t]{1cm}\setlength{\parskip}{3mm}
	  		\textcolor{#1}{\LARGE{#2}}    
 	 	\end{minipage}  
  		\begin{minipage}[t]{\linewidth-2\fboxrule-4\fboxsep}\setlength{\parskip}{3mm}
			\raisebox{1.2mm}{\normalsize\sffamily{\textcolor{#1}{#3}}}						
  			 #4
  		\end{minipage}
	}}
	\vspace{5mm}
}

\newcommand\cadre[3]{				% Boites convertible html
	\par
	\vspace{2mm}
	\setlength{\fboxrule}{0.1mm}
	\setlength{\fboxsep}{5mm}
	\fcolorbox{#1}{white}{\makebox[\linewidth-2\fboxrule-2\fboxsep]{
  		\begin{minipage}[t]{\linewidth-2\fboxrule-4\fboxsep}\setlength{\parskip}{3mm}
			\raisebox{-2.5mm}{\sffamily \small{\textcolor{#1}{\MakeUppercase{#2}}}}		
			\par		
  			 #3
 	 		\end{minipage}
	}}
		\vspace{2mm}
	\par
}

\newcommand\bloc[3]{				% Boites convertible html sans bordure
     \needspace{2\baselineskip}
     {\sffamily \small{\textcolor{#1}{\MakeUppercase{#2}}}}    
		\par		
  			 #3
		\par
}

\newcommand\CHelp[1]{
     \CBox{Plum}{\faInfoCircle}{À RETENIR}{#1}
}

\newcommand\CUp[1]{
     \CBox{NavyBlue}{\faThumbsOUp}{EN PRATIQUE}{#1}
}

\newcommand\CInfo[1]{
     \CBox{Sepia}{\faArrowCircleRight}{REMARQUE}{#1}
}

\newcommand\CRedac[1]{
     \CBox{PineGreen}{\faEdit}{BIEN R\'EDIGER}{#1}
}

\newcommand\CError[1]{
     \CBox{Red}{\faExclamationTriangle}{ATTENTION}{#1}
}

\newcommand\TitreExo[2]{
\needspace{4\baselineskip}
 {\sffamily\large EXERCICE #1\ (\emph{#2 points})}
\vspace{5mm}
}

\newcommand\img[2]{
          \includegraphics[width=#2\paperwidth]{\imgdir#1}
}

\newcommand\imgsvg[2]{
       \begin{center}   \includegraphics[width=#2\paperwidth]{\imgsvgdir#1} \end{center}
}


\newcommand\Lien[2]{
     \href{#1}{#2 \tiny \faExternalLink}
}
\newcommand\mcLien[2]{
     \href{https~://www.maths-cours.fr/#1}{#2 \tiny \faExternalLink}
}

\newcommand{\euro}{\eurologo{}}

%================================================================================================================================
%
% Macros - Environement
%
%================================================================================================================================

\newenvironment{tex}{ %
}
{%
}

\newenvironment{indente}{ %
	\setlength\parindent{10mm}
}

{
	\setlength\parindent{0mm}
}

\newenvironment{corrige}{%
     \needspace{3\baselineskip}
     \medskip
     \textbf{\textsc{Corrigé}}
     \medskip
}
{
}

\newenvironment{extern}{%
     \begin{center}
     }
     {
     \end{center}
}

\NewEnviron{code}{%
	\par
     \boite{gray}{\texttt{%
     \BODY
     }}
     \par
}

\newenvironment{vbloc}{% boite sans cadre empeche saut de page
     \begin{minipage}[t]{\linewidth}
     }
     {
     \end{minipage}
}
\NewEnviron{h2}{%
    \needspace{3\baselineskip}
    \vspace{0.6cm}
	\noindent \MakeUppercase{\sffamily \large \BODY}
	\vspace{1mm}\textcolor{mcgris}{\hrule}\vspace{0.4cm}
	\par
}{}

\NewEnviron{h3}{%
    \needspace{3\baselineskip}
	\vspace{5mm}
	\textsc{\BODY}
	\par
}

\NewEnviron{margeneg}{ %
\begin{addmargin}[-1cm]{0cm}
\BODY
\end{addmargin}
}

\NewEnviron{html}{%
}

\begin{document}
\meta{url}{/exercices/qcm-bac-es-l-amerique-du-nord-2018/}
\meta{pid}{7845}
\meta{titre}{QCM - Bac ES/L Amérique du Nord  2018}
\meta{type}{exercices}
%
\begin{h2}Exercice 1 (4 points)\end{h2}
\textbf{Commun à  tous les candidats}
\medskip
\emph{Cet exercice est un questionnaire à choix multiples. Pour chacune des questions suivantes, une seule des quatre propositions est exacte. Aucune justification n'est demandée. Une bonne réponse rapporte un point. Une mauvaise réponse, plusieurs réponses ou l'absence de réponse à une question ne rapportent ni n'enlèvent de point. Pour répondre, vous recopierez sur votre copie le numéro de la question et indiquerez la seule réponse choisie.}
\bigskip
\begin{enumerate}
     \item Un pépiniériste cultive des bulbes de fleurs. La probabilité qu'un bulbe germe, c'est-à-dire qu'il donne naissance à une plante qui fleurit, est de $0,85$.
     \par
     Il prélève au hasard 20 bulbes du lot. La production est assez grande pour que l'on puisse assimiler ce prélèvement à un tirage avec remise de 20 bulbes.
     \par
     On peut affirmer que:
     \par
     \textbf{A.} \quad La probabilité qu'au maximum 15 bulbes germent est proche de $0,103$.\\
     \textbf{B.} \quad La probabilité qu'au maximum 15 bulbes germent est proche de $0,067$.\\
     \textbf{C.} \quad La probabilité qu'au minimum 15 bulbes germent est proche de $0,830$.\\
     \textbf{D.} \quad La probabilité qu'au minimum 15 bulbes germent est proche de $0,933$.\\
     \item On considère une fonction $f$ définie sur $[0\,;\,8]$ dont $\mathscr{C}_f$ est la courbe représentative dessinée ci-dessous:
     \begin{center}
          \begin{extern}
               \psset{unit=1cm}
               \def\xmin {-1}   \def\xmax {9}
               \def\ymin {-1}   \def\ymax {6}
               \begin{pspicture*}(\xmin,\ymin)(\xmax,\ymax)
                    \psgrid[subgriddiv=1,  gridlabels=0, gridcolor=lightgray]
                    \psaxes[arrowsize=3pt 3, ticksize=-2pt 2pt]{->}(0,0)(\xmin,\ymin)(\xmax,\ymax)[$x$,-110][$y$,200]
                    \def\f{x x neg 10 add mul 16 add 8 div}                           % définition de la fonction
                    \psplot[plotpoints=2000,linecolor=red,linewidth=0.75pt]{0}{8}{\f}
                    \uput[ul](1,3){\color{red} $\mathcal{C}_f$}
                    \uput[dl](0,0){$0$}
               \end{pspicture*}
          \end{extern}
     \end{center}
     \begin{center}
          \begin{tabularx}{0.9\linewidth}{|X|X|} %class="noborder cel50 mw400"
               \hline
               \textbf{A.}~~$8 \leqslant \displaystyle\int_2^4 f(x) \text{d}x \leqslant 9$ \rule[-15pt]{0pt}{0pt} & \textbf{B.}~~$9 \leqslant \displaystyle\int_2^4 f(x) \text{d}x \leqslant 10$\\
               \hline
               \textbf{C.}~~$\displaystyle\int_2^4 f(x) \text{d}x=f(4)-f(2)$ \rule[-15pt]{0pt}{0pt} &    \textbf{D.}~~$\displaystyle\int_2^4 f(x) \text{d}x = 9$\\
               \hline
          \end{tabularx}
     \end{center}
     \par
     \item On considère la fonction $g$ définie sur $]0\,;\,+\infty[$ par $g(x)=\ln(x)$.
     \par
     Une primitive de $g$ sur $]0\,;\,+\infty[$ est la fonction $G$ définie par:
     \begin{center}
          \begin{tabularx}{0.9\linewidth}{|X|X|} %class="noborder cel50 mw400"
               \hline
               \textbf{A.}~~$G(x)=\ln(x)$ & \textbf{B.}~~$G(x)=x \ln(x)$\\
               \hline
               \textbf{C.}~~$G(x)= x \ln(x) - x$ & \textbf{D.}~~$G(x)= \dfrac{1}{x}$ \rule[-10pt]{0pt}{0pt}\\
               \hline
          \end{tabularx}
     \end{center}
     \item L'ensemble des solutions de l'inéquation $\ln(x)>0$ est:
     \begin{center}
          \begin{tabularx}{0.9\linewidth}{|X|X|}\hline %class="noborder cel50"
               \textbf{A.}~~$]0\,;\,+\infty[$ & \textbf{B.}~~$]0\,;\,1[$\\
               \hline
               \textbf{C.}~~$]1\,;\,+\infty[$ & \textbf{D.}~~$]\text{e}\,;\,+\infty[$ \\
               \hline
          \end{tabularx}
     \end{center}
\end{enumerate}

\end{document}
µ
\documentclass[a4paper]{article}

%================================================================================================================================
%
% Packages
%
%================================================================================================================================

\usepackage[T1]{fontenc} 	% pour caractères accentués
\usepackage[utf8]{inputenc}  % encodage utf8
\usepackage[french]{babel}	% langue : français
\usepackage{fourier}			% caractères plus lisibles
\usepackage[dvipsnames]{xcolor} % couleurs
\usepackage{fancyhdr}		% réglage header footer
\usepackage{needspace}		% empêcher sauts de page mal placés
\usepackage{graphicx}		% pour inclure des graphiques
\usepackage{enumitem,cprotect}		% personnalise les listes d'items (nécessaire pour ol, al ...)
\usepackage{hyperref}		% Liens hypertexte
\usepackage{pstricks,pst-all,pst-node,pstricks-add,pst-math,pst-plot,pst-tree,pst-eucl} % pstricks
\usepackage[a4paper,includeheadfoot,top=2cm,left=3cm, bottom=2cm,right=3cm]{geometry} % marges etc.
\usepackage{comment}			% commentaires multilignes
\usepackage{amsmath,environ} % maths (matrices, etc.)
\usepackage{amssymb,makeidx}
\usepackage{bm}				% bold maths
\usepackage{tabularx}		% tableaux
\usepackage{colortbl}		% tableaux en couleur
\usepackage{fontawesome}		% Fontawesome
\usepackage{environ}			% environment with command
\usepackage{fp}				% calculs pour ps-tricks
\usepackage{multido}			% pour ps tricks
\usepackage[np]{numprint}	% formattage nombre
\usepackage{tikz,tkz-tab} 			% package principal TikZ
\usepackage{pgfplots}   % axes
\usepackage{mathrsfs}    % cursives
\usepackage{calc}			% calcul taille boites
\usepackage[scaled=0.875]{helvet} % font sans serif
\usepackage{svg} % svg
\usepackage{scrextend} % local margin
\usepackage{scratch} %scratch
\usepackage{multicol} % colonnes
%\usepackage{infix-RPN,pst-func} % formule en notation polanaise inversée
\usepackage{listings}

%================================================================================================================================
%
% Réglages de base
%
%================================================================================================================================

\lstset{
language=Python,   % R code
literate=
{á}{{\'a}}1
{à}{{\`a}}1
{ã}{{\~a}}1
{é}{{\'e}}1
{è}{{\`e}}1
{ê}{{\^e}}1
{í}{{\'i}}1
{ó}{{\'o}}1
{õ}{{\~o}}1
{ú}{{\'u}}1
{ü}{{\"u}}1
{ç}{{\c{c}}}1
{~}{{ }}1
}


\definecolor{codegreen}{rgb}{0,0.6,0}
\definecolor{codegray}{rgb}{0.5,0.5,0.5}
\definecolor{codepurple}{rgb}{0.58,0,0.82}
\definecolor{backcolour}{rgb}{0.95,0.95,0.92}

\lstdefinestyle{mystyle}{
    backgroundcolor=\color{backcolour},   
    commentstyle=\color{codegreen},
    keywordstyle=\color{magenta},
    numberstyle=\tiny\color{codegray},
    stringstyle=\color{codepurple},
    basicstyle=\ttfamily\footnotesize,
    breakatwhitespace=false,         
    breaklines=true,                 
    captionpos=b,                    
    keepspaces=true,                 
    numbers=left,                    
xleftmargin=2em,
framexleftmargin=2em,            
    showspaces=false,                
    showstringspaces=false,
    showtabs=false,                  
    tabsize=2,
    upquote=true
}

\lstset{style=mystyle}


\lstset{style=mystyle}
\newcommand{\imgdir}{C:/laragon/www/newmc/assets/imgsvg/}
\newcommand{\imgsvgdir}{C:/laragon/www/newmc/assets/imgsvg/}

\definecolor{mcgris}{RGB}{220, 220, 220}% ancien~; pour compatibilité
\definecolor{mcbleu}{RGB}{52, 152, 219}
\definecolor{mcvert}{RGB}{125, 194, 70}
\definecolor{mcmauve}{RGB}{154, 0, 215}
\definecolor{mcorange}{RGB}{255, 96, 0}
\definecolor{mcturquoise}{RGB}{0, 153, 153}
\definecolor{mcrouge}{RGB}{255, 0, 0}
\definecolor{mclightvert}{RGB}{205, 234, 190}

\definecolor{gris}{RGB}{220, 220, 220}
\definecolor{bleu}{RGB}{52, 152, 219}
\definecolor{vert}{RGB}{125, 194, 70}
\definecolor{mauve}{RGB}{154, 0, 215}
\definecolor{orange}{RGB}{255, 96, 0}
\definecolor{turquoise}{RGB}{0, 153, 153}
\definecolor{rouge}{RGB}{255, 0, 0}
\definecolor{lightvert}{RGB}{205, 234, 190}
\setitemize[0]{label=\color{lightvert}  $\bullet$}

\pagestyle{fancy}
\renewcommand{\headrulewidth}{0.2pt}
\fancyhead[L]{maths-cours.fr}
\fancyhead[R]{\thepage}
\renewcommand{\footrulewidth}{0.2pt}
\fancyfoot[C]{}

\newcolumntype{C}{>{\centering\arraybackslash}X}
\newcolumntype{s}{>{\hsize=.35\hsize\arraybackslash}X}

\setlength{\parindent}{0pt}		 
\setlength{\parskip}{3mm}
\setlength{\headheight}{1cm}

\def\ebook{ebook}
\def\book{book}
\def\web{web}
\def\type{web}

\newcommand{\vect}[1]{\overrightarrow{\,\mathstrut#1\,}}

\def\Oij{$\left(\text{O}~;~\vect{\imath},~\vect{\jmath}\right)$}
\def\Oijk{$\left(\text{O}~;~\vect{\imath},~\vect{\jmath},~\vect{k}\right)$}
\def\Ouv{$\left(\text{O}~;~\vect{u},~\vect{v}\right)$}

\hypersetup{breaklinks=true, colorlinks = true, linkcolor = OliveGreen, urlcolor = OliveGreen, citecolor = OliveGreen, pdfauthor={Didier BONNEL - https://www.maths-cours.fr} } % supprime les bordures autour des liens

\renewcommand{\arg}[0]{\text{arg}}

\everymath{\displaystyle}

%================================================================================================================================
%
% Macros - Commandes
%
%================================================================================================================================

\newcommand\meta[2]{    			% Utilisé pour créer le post HTML.
	\def\titre{titre}
	\def\url{url}
	\def\arg{#1}
	\ifx\titre\arg
		\newcommand\maintitle{#2}
		\fancyhead[L]{#2}
		{\Large\sffamily \MakeUppercase{#2}}
		\vspace{1mm}\textcolor{mcvert}{\hrule}
	\fi 
	\ifx\url\arg
		\fancyfoot[L]{\href{https://www.maths-cours.fr#2}{\black \footnotesize{https://www.maths-cours.fr#2}}}
	\fi 
}


\newcommand\TitreC[1]{    		% Titre centré
     \needspace{3\baselineskip}
     \begin{center}\textbf{#1}\end{center}
}

\newcommand\newpar{    		% paragraphe
     \par
}

\newcommand\nosp {    		% commande vide (pas d'espace)
}
\newcommand{\id}[1]{} %ignore

\newcommand\boite[2]{				% Boite simple sans titre
	\vspace{5mm}
	\setlength{\fboxrule}{0.2mm}
	\setlength{\fboxsep}{5mm}	
	\fcolorbox{#1}{#1!3}{\makebox[\linewidth-2\fboxrule-2\fboxsep]{
  		\begin{minipage}[t]{\linewidth-2\fboxrule-4\fboxsep}\setlength{\parskip}{3mm}
  			 #2
  		\end{minipage}
	}}
	\vspace{5mm}
}

\newcommand\CBox[4]{				% Boites
	\vspace{5mm}
	\setlength{\fboxrule}{0.2mm}
	\setlength{\fboxsep}{5mm}
	
	\fcolorbox{#1}{#1!3}{\makebox[\linewidth-2\fboxrule-2\fboxsep]{
		\begin{minipage}[t]{1cm}\setlength{\parskip}{3mm}
	  		\textcolor{#1}{\LARGE{#2}}    
 	 	\end{minipage}  
  		\begin{minipage}[t]{\linewidth-2\fboxrule-4\fboxsep}\setlength{\parskip}{3mm}
			\raisebox{1.2mm}{\normalsize\sffamily{\textcolor{#1}{#3}}}						
  			 #4
  		\end{minipage}
	}}
	\vspace{5mm}
}

\newcommand\cadre[3]{				% Boites convertible html
	\par
	\vspace{2mm}
	\setlength{\fboxrule}{0.1mm}
	\setlength{\fboxsep}{5mm}
	\fcolorbox{#1}{white}{\makebox[\linewidth-2\fboxrule-2\fboxsep]{
  		\begin{minipage}[t]{\linewidth-2\fboxrule-4\fboxsep}\setlength{\parskip}{3mm}
			\raisebox{-2.5mm}{\sffamily \small{\textcolor{#1}{\MakeUppercase{#2}}}}		
			\par		
  			 #3
 	 		\end{minipage}
	}}
		\vspace{2mm}
	\par
}

\newcommand\bloc[3]{				% Boites convertible html sans bordure
     \needspace{2\baselineskip}
     {\sffamily \small{\textcolor{#1}{\MakeUppercase{#2}}}}    
		\par		
  			 #3
		\par
}

\newcommand\CHelp[1]{
     \CBox{Plum}{\faInfoCircle}{À RETENIR}{#1}
}

\newcommand\CUp[1]{
     \CBox{NavyBlue}{\faThumbsOUp}{EN PRATIQUE}{#1}
}

\newcommand\CInfo[1]{
     \CBox{Sepia}{\faArrowCircleRight}{REMARQUE}{#1}
}

\newcommand\CRedac[1]{
     \CBox{PineGreen}{\faEdit}{BIEN R\'EDIGER}{#1}
}

\newcommand\CError[1]{
     \CBox{Red}{\faExclamationTriangle}{ATTENTION}{#1}
}

\newcommand\TitreExo[2]{
\needspace{4\baselineskip}
 {\sffamily\large EXERCICE #1\ (\emph{#2 points})}
\vspace{5mm}
}

\newcommand\img[2]{
          \includegraphics[width=#2\paperwidth]{\imgdir#1}
}

\newcommand\imgsvg[2]{
       \begin{center}   \includegraphics[width=#2\paperwidth]{\imgsvgdir#1} \end{center}
}


\newcommand\Lien[2]{
     \href{#1}{#2 \tiny \faExternalLink}
}
\newcommand\mcLien[2]{
     \href{https~://www.maths-cours.fr/#1}{#2 \tiny \faExternalLink}
}

\newcommand{\euro}{\eurologo{}}

%================================================================================================================================
%
% Macros - Environement
%
%================================================================================================================================

\newenvironment{tex}{ %
}
{%
}

\newenvironment{indente}{ %
	\setlength\parindent{10mm}
}

{
	\setlength\parindent{0mm}
}

\newenvironment{corrige}{%
     \needspace{3\baselineskip}
     \medskip
     \textbf{\textsc{Corrigé}}
     \medskip
}
{
}

\newenvironment{extern}{%
     \begin{center}
     }
     {
     \end{center}
}

\NewEnviron{code}{%
	\par
     \boite{gray}{\texttt{%
     \BODY
     }}
     \par
}

\newenvironment{vbloc}{% boite sans cadre empeche saut de page
     \begin{minipage}[t]{\linewidth}
     }
     {
     \end{minipage}
}
\NewEnviron{h2}{%
    \needspace{3\baselineskip}
    \vspace{0.6cm}
	\noindent \MakeUppercase{\sffamily \large \BODY}
	\vspace{1mm}\textcolor{mcgris}{\hrule}\vspace{0.4cm}
	\par
}{}

\NewEnviron{h3}{%
    \needspace{3\baselineskip}
	\vspace{5mm}
	\textsc{\BODY}
	\par
}

\NewEnviron{margeneg}{ %
\begin{addmargin}[-1cm]{0cm}
\BODY
\end{addmargin}
}

\NewEnviron{html}{%
}

\begin{document}
\meta{url}{/exercices/probabilites-bac-es-l-amerique-du-nord-2018/}
\meta{pid}{7864}
\meta{titre}{Probabilités - Bac ES/L Amérique du Nord  2018}
\meta{type}{exercices}
%
\begin{h2}Exercice 2 (5 points)\end{h2}
\textbf{Commun à  tous les candidats}
\medskip
\textbf{Tous les résultats demandés dans cet exercice seront arrondis au millième.}
\par
\textbf{Les parties A, B et C sont indépendantes.}
\medskip
Le site internet \og ledislight.com \fg{} spécialisé dans la vente de matériel lumineux vend deux sortes de rubans LED flexibles~: un premier modèle dit d'\og intérieur \fg{} et un deuxième modèle dit d'\og extérieur \fg{}. Le site internet dispose d'un grand stock de ces rubans LED.
\begin{center}\begin{h3}Partie A \end{h3}\end{center}
\begin{enumerate}
     \item Le fournisseur affirme que, parmi les rubans LED d'extérieur expédiés au site internet, 5\,\% sont défectueux. Le responsable internet désire vérifier la validité de cette affirmation. Dans son stock, il prélève au hasard $400$ rubans LED d'extérieur parmi lesquels $25$ sont défectueux.
     \par
     Ce contrôle remet-il en cause l'affirmation du fournisseur~?
     \medskip
     \emph{\textbf{Rappel~:} lorsque la proportion $p$ d'un caractère dans la population est connue, l'intervalle $I$ de fluctuation asymptotique au seuil de 95\,\% d'une fréquence d'apparition de ce caractère obtenue sur un échantillon de taille $n$ est donnée par~:}
     \begin{center}
     $I= \left [ p-1,96 \sqrt{\dfrac{p(1-p)}{n}}\,;\right.$\nosp$\left.p+1,96 \sqrt{\dfrac{p(1-p)}{n}}\right ]$
\end{center}
\bigskip
\item Le fournisseur n'a donné aucune information concernant la fiabilité des rubans LED d'intérieur. Le directeur du site souhaite estimer la proportion de rubans LED d'intérieur défectueux. Pour cela, il prélève un échantillon aléatoire de $400$ rubans d'intérieur, parmi lesquels $38$ sont défectueux.
\par
Donner un intervalle de confiance de cette proportion au seuil de confiance de 95\,\%.
\end{enumerate}
\begin{center}\begin{h3}Partie B \end{h3}\end{center}
À partir d'une étude statistique réalisée sur de nombreux mois, on peut modéliser le nombre de rubans LED d'intérieur vendus chaque mois par le site à l'aide d'une variable aléatoire $X$ qui suit la loi normale de moyenne $\mu = 2~500$ et d'écart-type $\sigma=400$.
\begin{enumerate}
     \item Quelle est la probabilité que le site internet vende entre $2~100$ et $2~900$ rubans LED d'intérieur en un mois~?
     \item
     \begin{enumerate}[label=\alph*.]
          \item Trouver, arrondie à l'entier, la valeur de $a$ telle que $P(X \leqslant a)=0,95$.
          \item Interpréter la valeur de $a$ obtenue ci-dessus en termes de probabilité de rupture de stock.
     \end{enumerate}
\end{enumerate}
\begin{center}\begin{h3}Partie C \end{h3}\end{center}
On admet maintenant que~:
\begin{itemize}
     \item 20\,\% des rubans LED proposés à la vente sont d'extérieur~;
     \item 5\,\% des rubans LED d'extérieur sont défectueux.
\end{itemize}
On prélève au hasard un ruban LED dans le stock.
\par
On appelle~:
\begin{itemize}
     \item $E$ l'événement~: \og le ruban LED est d'extérieur \fg{}~;
     \item $D$ l'événement~: \og le ruban LED est défectueux \fg{}.
\end{itemize}
\begin{enumerate}
     \item Représenter la situation à l'aide d'un arbre pondéré, que l'on complètera au fur et à mesure.
     \item Déterminer la probabilité que le ruban LED soit d'extérieur et défectueux.
     \item D'autre part on sait que 6\,\% de tous les rubans LED sont défectueux. \\
     Calculer puis interpréter $P_{\overline{E}}(D)$.
\end{enumerate}

\end{document}
µ
\documentclass[a4paper]{article}

%================================================================================================================================
%
% Packages
%
%================================================================================================================================

\usepackage[T1]{fontenc} 	% pour caractères accentués
\usepackage[utf8]{inputenc}  % encodage utf8
\usepackage[french]{babel}	% langue : français
\usepackage{fourier}			% caractères plus lisibles
\usepackage[dvipsnames]{xcolor} % couleurs
\usepackage{fancyhdr}		% réglage header footer
\usepackage{needspace}		% empêcher sauts de page mal placés
\usepackage{graphicx}		% pour inclure des graphiques
\usepackage{enumitem,cprotect}		% personnalise les listes d'items (nécessaire pour ol, al ...)
\usepackage{hyperref}		% Liens hypertexte
\usepackage{pstricks,pst-all,pst-node,pstricks-add,pst-math,pst-plot,pst-tree,pst-eucl} % pstricks
\usepackage[a4paper,includeheadfoot,top=2cm,left=3cm, bottom=2cm,right=3cm]{geometry} % marges etc.
\usepackage{comment}			% commentaires multilignes
\usepackage{amsmath,environ} % maths (matrices, etc.)
\usepackage{amssymb,makeidx}
\usepackage{bm}				% bold maths
\usepackage{tabularx}		% tableaux
\usepackage{colortbl}		% tableaux en couleur
\usepackage{fontawesome}		% Fontawesome
\usepackage{environ}			% environment with command
\usepackage{fp}				% calculs pour ps-tricks
\usepackage{multido}			% pour ps tricks
\usepackage[np]{numprint}	% formattage nombre
\usepackage{tikz,tkz-tab} 			% package principal TikZ
\usepackage{pgfplots}   % axes
\usepackage{mathrsfs}    % cursives
\usepackage{calc}			% calcul taille boites
\usepackage[scaled=0.875]{helvet} % font sans serif
\usepackage{svg} % svg
\usepackage{scrextend} % local margin
\usepackage{scratch} %scratch
\usepackage{multicol} % colonnes
%\usepackage{infix-RPN,pst-func} % formule en notation polanaise inversée
\usepackage{listings}

%================================================================================================================================
%
% Réglages de base
%
%================================================================================================================================

\lstset{
language=Python,   % R code
literate=
{á}{{\'a}}1
{à}{{\`a}}1
{ã}{{\~a}}1
{é}{{\'e}}1
{è}{{\`e}}1
{ê}{{\^e}}1
{í}{{\'i}}1
{ó}{{\'o}}1
{õ}{{\~o}}1
{ú}{{\'u}}1
{ü}{{\"u}}1
{ç}{{\c{c}}}1
{~}{{ }}1
}


\definecolor{codegreen}{rgb}{0,0.6,0}
\definecolor{codegray}{rgb}{0.5,0.5,0.5}
\definecolor{codepurple}{rgb}{0.58,0,0.82}
\definecolor{backcolour}{rgb}{0.95,0.95,0.92}

\lstdefinestyle{mystyle}{
    backgroundcolor=\color{backcolour},   
    commentstyle=\color{codegreen},
    keywordstyle=\color{magenta},
    numberstyle=\tiny\color{codegray},
    stringstyle=\color{codepurple},
    basicstyle=\ttfamily\footnotesize,
    breakatwhitespace=false,         
    breaklines=true,                 
    captionpos=b,                    
    keepspaces=true,                 
    numbers=left,                    
xleftmargin=2em,
framexleftmargin=2em,            
    showspaces=false,                
    showstringspaces=false,
    showtabs=false,                  
    tabsize=2,
    upquote=true
}

\lstset{style=mystyle}


\lstset{style=mystyle}
\newcommand{\imgdir}{C:/laragon/www/newmc/assets/imgsvg/}
\newcommand{\imgsvgdir}{C:/laragon/www/newmc/assets/imgsvg/}

\definecolor{mcgris}{RGB}{220, 220, 220}% ancien~; pour compatibilité
\definecolor{mcbleu}{RGB}{52, 152, 219}
\definecolor{mcvert}{RGB}{125, 194, 70}
\definecolor{mcmauve}{RGB}{154, 0, 215}
\definecolor{mcorange}{RGB}{255, 96, 0}
\definecolor{mcturquoise}{RGB}{0, 153, 153}
\definecolor{mcrouge}{RGB}{255, 0, 0}
\definecolor{mclightvert}{RGB}{205, 234, 190}

\definecolor{gris}{RGB}{220, 220, 220}
\definecolor{bleu}{RGB}{52, 152, 219}
\definecolor{vert}{RGB}{125, 194, 70}
\definecolor{mauve}{RGB}{154, 0, 215}
\definecolor{orange}{RGB}{255, 96, 0}
\definecolor{turquoise}{RGB}{0, 153, 153}
\definecolor{rouge}{RGB}{255, 0, 0}
\definecolor{lightvert}{RGB}{205, 234, 190}
\setitemize[0]{label=\color{lightvert}  $\bullet$}

\pagestyle{fancy}
\renewcommand{\headrulewidth}{0.2pt}
\fancyhead[L]{maths-cours.fr}
\fancyhead[R]{\thepage}
\renewcommand{\footrulewidth}{0.2pt}
\fancyfoot[C]{}

\newcolumntype{C}{>{\centering\arraybackslash}X}
\newcolumntype{s}{>{\hsize=.35\hsize\arraybackslash}X}

\setlength{\parindent}{0pt}		 
\setlength{\parskip}{3mm}
\setlength{\headheight}{1cm}

\def\ebook{ebook}
\def\book{book}
\def\web{web}
\def\type{web}

\newcommand{\vect}[1]{\overrightarrow{\,\mathstrut#1\,}}

\def\Oij{$\left(\text{O}~;~\vect{\imath},~\vect{\jmath}\right)$}
\def\Oijk{$\left(\text{O}~;~\vect{\imath},~\vect{\jmath},~\vect{k}\right)$}
\def\Ouv{$\left(\text{O}~;~\vect{u},~\vect{v}\right)$}

\hypersetup{breaklinks=true, colorlinks = true, linkcolor = OliveGreen, urlcolor = OliveGreen, citecolor = OliveGreen, pdfauthor={Didier BONNEL - https://www.maths-cours.fr} } % supprime les bordures autour des liens

\renewcommand{\arg}[0]{\text{arg}}

\everymath{\displaystyle}

%================================================================================================================================
%
% Macros - Commandes
%
%================================================================================================================================

\newcommand\meta[2]{    			% Utilisé pour créer le post HTML.
	\def\titre{titre}
	\def\url{url}
	\def\arg{#1}
	\ifx\titre\arg
		\newcommand\maintitle{#2}
		\fancyhead[L]{#2}
		{\Large\sffamily \MakeUppercase{#2}}
		\vspace{1mm}\textcolor{mcvert}{\hrule}
	\fi 
	\ifx\url\arg
		\fancyfoot[L]{\href{https://www.maths-cours.fr#2}{\black \footnotesize{https://www.maths-cours.fr#2}}}
	\fi 
}


\newcommand\TitreC[1]{    		% Titre centré
     \needspace{3\baselineskip}
     \begin{center}\textbf{#1}\end{center}
}

\newcommand\newpar{    		% paragraphe
     \par
}

\newcommand\nosp {    		% commande vide (pas d'espace)
}
\newcommand{\id}[1]{} %ignore

\newcommand\boite[2]{				% Boite simple sans titre
	\vspace{5mm}
	\setlength{\fboxrule}{0.2mm}
	\setlength{\fboxsep}{5mm}	
	\fcolorbox{#1}{#1!3}{\makebox[\linewidth-2\fboxrule-2\fboxsep]{
  		\begin{minipage}[t]{\linewidth-2\fboxrule-4\fboxsep}\setlength{\parskip}{3mm}
  			 #2
  		\end{minipage}
	}}
	\vspace{5mm}
}

\newcommand\CBox[4]{				% Boites
	\vspace{5mm}
	\setlength{\fboxrule}{0.2mm}
	\setlength{\fboxsep}{5mm}
	
	\fcolorbox{#1}{#1!3}{\makebox[\linewidth-2\fboxrule-2\fboxsep]{
		\begin{minipage}[t]{1cm}\setlength{\parskip}{3mm}
	  		\textcolor{#1}{\LARGE{#2}}    
 	 	\end{minipage}  
  		\begin{minipage}[t]{\linewidth-2\fboxrule-4\fboxsep}\setlength{\parskip}{3mm}
			\raisebox{1.2mm}{\normalsize\sffamily{\textcolor{#1}{#3}}}						
  			 #4
  		\end{minipage}
	}}
	\vspace{5mm}
}

\newcommand\cadre[3]{				% Boites convertible html
	\par
	\vspace{2mm}
	\setlength{\fboxrule}{0.1mm}
	\setlength{\fboxsep}{5mm}
	\fcolorbox{#1}{white}{\makebox[\linewidth-2\fboxrule-2\fboxsep]{
  		\begin{minipage}[t]{\linewidth-2\fboxrule-4\fboxsep}\setlength{\parskip}{3mm}
			\raisebox{-2.5mm}{\sffamily \small{\textcolor{#1}{\MakeUppercase{#2}}}}		
			\par		
  			 #3
 	 		\end{minipage}
	}}
		\vspace{2mm}
	\par
}

\newcommand\bloc[3]{				% Boites convertible html sans bordure
     \needspace{2\baselineskip}
     {\sffamily \small{\textcolor{#1}{\MakeUppercase{#2}}}}    
		\par		
  			 #3
		\par
}

\newcommand\CHelp[1]{
     \CBox{Plum}{\faInfoCircle}{À RETENIR}{#1}
}

\newcommand\CUp[1]{
     \CBox{NavyBlue}{\faThumbsOUp}{EN PRATIQUE}{#1}
}

\newcommand\CInfo[1]{
     \CBox{Sepia}{\faArrowCircleRight}{REMARQUE}{#1}
}

\newcommand\CRedac[1]{
     \CBox{PineGreen}{\faEdit}{BIEN R\'EDIGER}{#1}
}

\newcommand\CError[1]{
     \CBox{Red}{\faExclamationTriangle}{ATTENTION}{#1}
}

\newcommand\TitreExo[2]{
\needspace{4\baselineskip}
 {\sffamily\large EXERCICE #1\ (\emph{#2 points})}
\vspace{5mm}
}

\newcommand\img[2]{
          \includegraphics[width=#2\paperwidth]{\imgdir#1}
}

\newcommand\imgsvg[2]{
       \begin{center}   \includegraphics[width=#2\paperwidth]{\imgsvgdir#1} \end{center}
}


\newcommand\Lien[2]{
     \href{#1}{#2 \tiny \faExternalLink}
}
\newcommand\mcLien[2]{
     \href{https~://www.maths-cours.fr/#1}{#2 \tiny \faExternalLink}
}

\newcommand{\euro}{\eurologo{}}

%================================================================================================================================
%
% Macros - Environement
%
%================================================================================================================================

\newenvironment{tex}{ %
}
{%
}

\newenvironment{indente}{ %
	\setlength\parindent{10mm}
}

{
	\setlength\parindent{0mm}
}

\newenvironment{corrige}{%
     \needspace{3\baselineskip}
     \medskip
     \textbf{\textsc{Corrigé}}
     \medskip
}
{
}

\newenvironment{extern}{%
     \begin{center}
     }
     {
     \end{center}
}

\NewEnviron{code}{%
	\par
     \boite{gray}{\texttt{%
     \BODY
     }}
     \par
}

\newenvironment{vbloc}{% boite sans cadre empeche saut de page
     \begin{minipage}[t]{\linewidth}
     }
     {
     \end{minipage}
}
\NewEnviron{h2}{%
    \needspace{3\baselineskip}
    \vspace{0.6cm}
	\noindent \MakeUppercase{\sffamily \large \BODY}
	\vspace{1mm}\textcolor{mcgris}{\hrule}\vspace{0.4cm}
	\par
}{}

\NewEnviron{h3}{%
    \needspace{3\baselineskip}
	\vspace{5mm}
	\textsc{\BODY}
	\par
}

\NewEnviron{margeneg}{ %
\begin{addmargin}[-1cm]{0cm}
\BODY
\end{addmargin}
}

\NewEnviron{html}{%
}

\begin{document}
\meta{url}{/exercices/suites-bac-es-l-amerique-du-nord-2018/}
\meta{pid}{7879}
\meta{titre}{Suites - Bac ES/L Amérique du Nord  2018}
\meta{type}{exercices}
%
\begin{h2}Exercice 3 (5 points)\end{h2}
\textbf{Candidats de ES n'ayant pas choisi la spécialité \og mathématiques \fg{} et candidats de L.}
\medskip
Une société propose des contrats annuels d'entretien de photocopieurs. Le directeur de cette société remarque que, chaque année, $14\,\%$ des contrats supplémentaires sont souscrits et $7$ sont résiliés.
\par
En $2017$, l'entreprise dénombrait $120$ contrats souscrits.
\par
On modélise la situation par une suite $(u_n)$ où $u_n$ est le nombre de contrats souscrits l'année $2017+n$.
\par
Ainsi on a $u_0=120$.
\begin{enumerate}
     \item
     \begin{enumerate}[label=\alph*.]
          \item Justifier que, pour tout entier naturel $n$, on a $u_{n+1}=1,14 u_n-7$.
          \item Estimer le nombre de contrats d'entretien en 2018.
     \end{enumerate}
     \item Compte tenu de ses capacités structurelles actuelles, l'entreprise ne peut prendre ne charge que 190 contrats. Au-delà, l'entreprise devra embaucher davantage de personnel.
     \par
     On cherche donc à savoir en quelle année l'entreprise devra embaucher.
     \par
     Pour cela, on utilise l'algorithme suivant~:
     \begin{center}
          \begin{extern}%width="200"
               \renewcommand{\arraystretch}{1.25}
               \begin{tabularx}{0.3\linewidth}{|X|}
                    \hline
                    $n\longleftarrow 0$\\
                    $u\longleftarrow 120$\\
                    Tant que $\ldots\ldots\ldots$\\
                    \hspace*{1cm} $n\longleftarrow n+1$\\
                    \hspace*{1cm} $\ldots\ldots\ldots$\\
                    Fin Tant que\\
                    Afficher $2017+n$\\
                    \hline
               \end{tabularx}
          \end{extern}
     \end{center}
     \begin{enumerate}[label=\alph*.]
          \item Recopier et compléter l'algorithme ci-dessus.
          \item Quelle est l'année affichée en sortie d'algorithme~? Interpréter cette valeur dans le contexte de l'exercice.
     \end{enumerate}
     \item On définit la suite $(v_n)$ par $v_n=u_n-50$ pour tout entier naturel $n$.
     \begin{enumerate}[label=\alph*.]
          \item Démontrer que la suite $(v_n)$ est une suite géométrique dont on précisera la raison et le premier terme $v_0$.
          \item Exprimer $v_n$ en fonction de $n$ puis démonter que, pour tout entier naturel $n$,
          \par
          \[u_n=70\times 1,14^n+50.\]
          \par
          \item Résoudre par le calcul l'inéquation $u_n>190$.
          \par
          Quel résultat de la question \textbf{2.} retrouve-t-on~?
     \end{enumerate}
\end{enumerate}

\end{document}
µ
\documentclass[a4paper]{article}

%================================================================================================================================
%
% Packages
%
%================================================================================================================================

\usepackage[T1]{fontenc} 	% pour caractères accentués
\usepackage[utf8]{inputenc}  % encodage utf8
\usepackage[french]{babel}	% langue : français
\usepackage{fourier}			% caractères plus lisibles
\usepackage[dvipsnames]{xcolor} % couleurs
\usepackage{fancyhdr}		% réglage header footer
\usepackage{needspace}		% empêcher sauts de page mal placés
\usepackage{graphicx}		% pour inclure des graphiques
\usepackage{enumitem,cprotect}		% personnalise les listes d'items (nécessaire pour ol, al ...)
\usepackage{hyperref}		% Liens hypertexte
\usepackage{pstricks,pst-all,pst-node,pstricks-add,pst-math,pst-plot,pst-tree,pst-eucl} % pstricks
\usepackage[a4paper,includeheadfoot,top=2cm,left=3cm, bottom=2cm,right=3cm]{geometry} % marges etc.
\usepackage{comment}			% commentaires multilignes
\usepackage{amsmath,environ} % maths (matrices, etc.)
\usepackage{amssymb,makeidx}
\usepackage{bm}				% bold maths
\usepackage{tabularx}		% tableaux
\usepackage{colortbl}		% tableaux en couleur
\usepackage{fontawesome}		% Fontawesome
\usepackage{environ}			% environment with command
\usepackage{fp}				% calculs pour ps-tricks
\usepackage{multido}			% pour ps tricks
\usepackage[np]{numprint}	% formattage nombre
\usepackage{tikz,tkz-tab} 			% package principal TikZ
\usepackage{pgfplots}   % axes
\usepackage{mathrsfs}    % cursives
\usepackage{calc}			% calcul taille boites
\usepackage[scaled=0.875]{helvet} % font sans serif
\usepackage{svg} % svg
\usepackage{scrextend} % local margin
\usepackage{scratch} %scratch
\usepackage{multicol} % colonnes
%\usepackage{infix-RPN,pst-func} % formule en notation polanaise inversée
\usepackage{listings}

%================================================================================================================================
%
% Réglages de base
%
%================================================================================================================================

\lstset{
language=Python,   % R code
literate=
{á}{{\'a}}1
{à}{{\`a}}1
{ã}{{\~a}}1
{é}{{\'e}}1
{è}{{\`e}}1
{ê}{{\^e}}1
{í}{{\'i}}1
{ó}{{\'o}}1
{õ}{{\~o}}1
{ú}{{\'u}}1
{ü}{{\"u}}1
{ç}{{\c{c}}}1
{~}{{ }}1
}


\definecolor{codegreen}{rgb}{0,0.6,0}
\definecolor{codegray}{rgb}{0.5,0.5,0.5}
\definecolor{codepurple}{rgb}{0.58,0,0.82}
\definecolor{backcolour}{rgb}{0.95,0.95,0.92}

\lstdefinestyle{mystyle}{
    backgroundcolor=\color{backcolour},   
    commentstyle=\color{codegreen},
    keywordstyle=\color{magenta},
    numberstyle=\tiny\color{codegray},
    stringstyle=\color{codepurple},
    basicstyle=\ttfamily\footnotesize,
    breakatwhitespace=false,         
    breaklines=true,                 
    captionpos=b,                    
    keepspaces=true,                 
    numbers=left,                    
xleftmargin=2em,
framexleftmargin=2em,            
    showspaces=false,                
    showstringspaces=false,
    showtabs=false,                  
    tabsize=2,
    upquote=true
}

\lstset{style=mystyle}


\lstset{style=mystyle}
\newcommand{\imgdir}{C:/laragon/www/newmc/assets/imgsvg/}
\newcommand{\imgsvgdir}{C:/laragon/www/newmc/assets/imgsvg/}

\definecolor{mcgris}{RGB}{220, 220, 220}% ancien~; pour compatibilité
\definecolor{mcbleu}{RGB}{52, 152, 219}
\definecolor{mcvert}{RGB}{125, 194, 70}
\definecolor{mcmauve}{RGB}{154, 0, 215}
\definecolor{mcorange}{RGB}{255, 96, 0}
\definecolor{mcturquoise}{RGB}{0, 153, 153}
\definecolor{mcrouge}{RGB}{255, 0, 0}
\definecolor{mclightvert}{RGB}{205, 234, 190}

\definecolor{gris}{RGB}{220, 220, 220}
\definecolor{bleu}{RGB}{52, 152, 219}
\definecolor{vert}{RGB}{125, 194, 70}
\definecolor{mauve}{RGB}{154, 0, 215}
\definecolor{orange}{RGB}{255, 96, 0}
\definecolor{turquoise}{RGB}{0, 153, 153}
\definecolor{rouge}{RGB}{255, 0, 0}
\definecolor{lightvert}{RGB}{205, 234, 190}
\setitemize[0]{label=\color{lightvert}  $\bullet$}

\pagestyle{fancy}
\renewcommand{\headrulewidth}{0.2pt}
\fancyhead[L]{maths-cours.fr}
\fancyhead[R]{\thepage}
\renewcommand{\footrulewidth}{0.2pt}
\fancyfoot[C]{}

\newcolumntype{C}{>{\centering\arraybackslash}X}
\newcolumntype{s}{>{\hsize=.35\hsize\arraybackslash}X}

\setlength{\parindent}{0pt}		 
\setlength{\parskip}{3mm}
\setlength{\headheight}{1cm}

\def\ebook{ebook}
\def\book{book}
\def\web{web}
\def\type{web}

\newcommand{\vect}[1]{\overrightarrow{\,\mathstrut#1\,}}

\def\Oij{$\left(\text{O}~;~\vect{\imath},~\vect{\jmath}\right)$}
\def\Oijk{$\left(\text{O}~;~\vect{\imath},~\vect{\jmath},~\vect{k}\right)$}
\def\Ouv{$\left(\text{O}~;~\vect{u},~\vect{v}\right)$}

\hypersetup{breaklinks=true, colorlinks = true, linkcolor = OliveGreen, urlcolor = OliveGreen, citecolor = OliveGreen, pdfauthor={Didier BONNEL - https://www.maths-cours.fr} } % supprime les bordures autour des liens

\renewcommand{\arg}[0]{\text{arg}}

\everymath{\displaystyle}

%================================================================================================================================
%
% Macros - Commandes
%
%================================================================================================================================

\newcommand\meta[2]{    			% Utilisé pour créer le post HTML.
	\def\titre{titre}
	\def\url{url}
	\def\arg{#1}
	\ifx\titre\arg
		\newcommand\maintitle{#2}
		\fancyhead[L]{#2}
		{\Large\sffamily \MakeUppercase{#2}}
		\vspace{1mm}\textcolor{mcvert}{\hrule}
	\fi 
	\ifx\url\arg
		\fancyfoot[L]{\href{https://www.maths-cours.fr#2}{\black \footnotesize{https://www.maths-cours.fr#2}}}
	\fi 
}


\newcommand\TitreC[1]{    		% Titre centré
     \needspace{3\baselineskip}
     \begin{center}\textbf{#1}\end{center}
}

\newcommand\newpar{    		% paragraphe
     \par
}

\newcommand\nosp {    		% commande vide (pas d'espace)
}
\newcommand{\id}[1]{} %ignore

\newcommand\boite[2]{				% Boite simple sans titre
	\vspace{5mm}
	\setlength{\fboxrule}{0.2mm}
	\setlength{\fboxsep}{5mm}	
	\fcolorbox{#1}{#1!3}{\makebox[\linewidth-2\fboxrule-2\fboxsep]{
  		\begin{minipage}[t]{\linewidth-2\fboxrule-4\fboxsep}\setlength{\parskip}{3mm}
  			 #2
  		\end{minipage}
	}}
	\vspace{5mm}
}

\newcommand\CBox[4]{				% Boites
	\vspace{5mm}
	\setlength{\fboxrule}{0.2mm}
	\setlength{\fboxsep}{5mm}
	
	\fcolorbox{#1}{#1!3}{\makebox[\linewidth-2\fboxrule-2\fboxsep]{
		\begin{minipage}[t]{1cm}\setlength{\parskip}{3mm}
	  		\textcolor{#1}{\LARGE{#2}}    
 	 	\end{minipage}  
  		\begin{minipage}[t]{\linewidth-2\fboxrule-4\fboxsep}\setlength{\parskip}{3mm}
			\raisebox{1.2mm}{\normalsize\sffamily{\textcolor{#1}{#3}}}						
  			 #4
  		\end{minipage}
	}}
	\vspace{5mm}
}

\newcommand\cadre[3]{				% Boites convertible html
	\par
	\vspace{2mm}
	\setlength{\fboxrule}{0.1mm}
	\setlength{\fboxsep}{5mm}
	\fcolorbox{#1}{white}{\makebox[\linewidth-2\fboxrule-2\fboxsep]{
  		\begin{minipage}[t]{\linewidth-2\fboxrule-4\fboxsep}\setlength{\parskip}{3mm}
			\raisebox{-2.5mm}{\sffamily \small{\textcolor{#1}{\MakeUppercase{#2}}}}		
			\par		
  			 #3
 	 		\end{minipage}
	}}
		\vspace{2mm}
	\par
}

\newcommand\bloc[3]{				% Boites convertible html sans bordure
     \needspace{2\baselineskip}
     {\sffamily \small{\textcolor{#1}{\MakeUppercase{#2}}}}    
		\par		
  			 #3
		\par
}

\newcommand\CHelp[1]{
     \CBox{Plum}{\faInfoCircle}{À RETENIR}{#1}
}

\newcommand\CUp[1]{
     \CBox{NavyBlue}{\faThumbsOUp}{EN PRATIQUE}{#1}
}

\newcommand\CInfo[1]{
     \CBox{Sepia}{\faArrowCircleRight}{REMARQUE}{#1}
}

\newcommand\CRedac[1]{
     \CBox{PineGreen}{\faEdit}{BIEN R\'EDIGER}{#1}
}

\newcommand\CError[1]{
     \CBox{Red}{\faExclamationTriangle}{ATTENTION}{#1}
}

\newcommand\TitreExo[2]{
\needspace{4\baselineskip}
 {\sffamily\large EXERCICE #1\ (\emph{#2 points})}
\vspace{5mm}
}

\newcommand\img[2]{
          \includegraphics[width=#2\paperwidth]{\imgdir#1}
}

\newcommand\imgsvg[2]{
       \begin{center}   \includegraphics[width=#2\paperwidth]{\imgsvgdir#1} \end{center}
}


\newcommand\Lien[2]{
     \href{#1}{#2 \tiny \faExternalLink}
}
\newcommand\mcLien[2]{
     \href{https~://www.maths-cours.fr/#1}{#2 \tiny \faExternalLink}
}

\newcommand{\euro}{\eurologo{}}

%================================================================================================================================
%
% Macros - Environement
%
%================================================================================================================================

\newenvironment{tex}{ %
}
{%
}

\newenvironment{indente}{ %
	\setlength\parindent{10mm}
}

{
	\setlength\parindent{0mm}
}

\newenvironment{corrige}{%
     \needspace{3\baselineskip}
     \medskip
     \textbf{\textsc{Corrigé}}
     \medskip
}
{
}

\newenvironment{extern}{%
     \begin{center}
     }
     {
     \end{center}
}

\NewEnviron{code}{%
	\par
     \boite{gray}{\texttt{%
     \BODY
     }}
     \par
}

\newenvironment{vbloc}{% boite sans cadre empeche saut de page
     \begin{minipage}[t]{\linewidth}
     }
     {
     \end{minipage}
}
\NewEnviron{h2}{%
    \needspace{3\baselineskip}
    \vspace{0.6cm}
	\noindent \MakeUppercase{\sffamily \large \BODY}
	\vspace{1mm}\textcolor{mcgris}{\hrule}\vspace{0.4cm}
	\par
}{}

\NewEnviron{h3}{%
    \needspace{3\baselineskip}
	\vspace{5mm}
	\textsc{\BODY}
	\par
}

\NewEnviron{margeneg}{ %
\begin{addmargin}[-1cm]{0cm}
\BODY
\end{addmargin}
}

\NewEnviron{html}{%
}

\begin{document}
\meta{url}{/exercices/graphes-probabilistes-bac-es-l-amerique-du-nord-2018/}
\meta{pid}{7887}
\meta{titre}{Graphes probabilistes - Bac ES/L Amérique du Nord  2018}
\meta{type}{exercices}
%
\begin{h2}Exercice 3 (5 points)\end{h2}
\textbf{Candidats de ES ayant choisi la spécialité \og mathématiques \fg{}}
\medskip
Deux entreprises concurrentes \og Alphacopy \fg{} et \og Betacopy \fg{} proposent des contrats annuels d'entretien de photocopieurs. Ces deux entreprises se partagent le marché des contrats d'entretien sur un secteur donné.
\par
\smallskip
\par
Le patron de Alphacopy remarque que, chaque année~:
\begin{itemize}
     \item 15\,\% des clients qui avaient souscrit un contrat d'entretien chez Alphacopy décident de souscrire un contrat d'entretien chez Betacopy. Les autres restent fidèles à Alphacopy~:
     \item 25\,\% des clients qui avaient souscrit un contrat d'entretien chez Betacopy décident de souscrire un contrat d'entretien chez Alphacopy. Les autres restent fidèles à Betacopy.
\end{itemize}
\smallskip
\par
On définit les événements suivants~:
\begin{itemize}
     \item $A$~: \og le client est sous contrat avec  l'entreprise Alphacopy \fg{}~:
     \item $B$~: \og le client est sous contrat avec l'entreprise Betacopy \fg{}.
\end{itemize}
\smallskip
\par
À partir de 2017, on choisit au hasard un client ayant un contrat d'entretien de photocopieurs et on note, pour tout entier naturel $n$~:
\par
\smallskip
\par
\begin{itemize}
     \item $a_n$ la probabilité que le client soit sous contrat avec l'entreprise Alphacopy l'année $2017+n$~:
     \item $b_n$ la probabilité que le client soit sous contrat avec l'entreprise Betacopy l'année $2017+ n$.
\end{itemize}
On note $P_n=\begin{pmatrix} a_n & b_n\end{pmatrix}$ la matrice ligne de l'état probabiliste pour l'année $2017+n$.
\par
\smallskip
\par
L'objectif de l'entreprise Alphacopy est d'obtenir au moins 62\,\% des contrats d'entretien des photocopieurs.
\begin{center}\begin{h3}Partie A \end{h3}\end{center}
\begin{enumerate}
     \item Représenter le graphe probabiliste de cette situation et donner la matrice de transition $M$ associée à ce graphe dont les sommets sont pris dans l'ordre alphabétique.
     \item Montrer que $P = \begin{pmatrix} 0,625 & 0,375\end{pmatrix}$ est un état stable de la matrice.
     \item À votre avis, l'entreprise Alphacopy peut-elle espérer atteindre son objectif~?
\end{enumerate}
\begin{center}\begin{h3}Partie B \end{h3}\end{center}
En 2017, on sait que 46\,\% des clients ayant un contrat d'entretien de photocopieurs étaient sous contrat avec l'entreprise Alphacopy.
\par
On a ainsi $P_0=\begin{pmatrix} 0,46 & 0,54\end{pmatrix}$.
\begin{enumerate}
     \item On rappelle que pour tout entier naturel $n$, $P_{n+1}=P_n \times M$.
     \par
     Démontrer que, pour tout entier naturel $n$, $a_{n+1}= 0,85 a_n + 0,25 b_n$ puis que
     \par
     \[a_{n+1} = 0,60 a_n + 0,25.\]
     \par
     \item À l'aide de l'algorithme ci-dessous, on cherche à déterminer en quelle année l'entreprise Alphacopy atteindra son objectif.
     \begin{center}
          \begin{extern}%width="200"
               \begin{tabular}{|p{3.5cm}|}
                    \hline
                    $n \leftarrow 0$\\
                    $a\leftarrow 0,46$\\
                    Tant que \ldots\ldots\ldots\\
                    \hspace*{1cm} $n\leftarrow n+1$\\
                    \hspace*{1cm} \ldots\ldots\ldots\\
                    Fin Tant que\\
                    Afficher $2017+n$\\
                    \hline
               \end{tabular}
          \end{extern}
     \end{center}
     \begin{enumerate}[label=\alph*.]
          \item Recopier et compléter l'algorithme ci-dessus.
          \item Quelle est l'année en sortie de l'algorithme~? Interpréter cette valeur dans le contexte de l'exercice.
     \end{enumerate}
     \item On définit la suite $\left(u_n\right)$ par $u_n=a_n-0,625$ pour tout entier naturel $n$.
     \begin{enumerate}[label=\alph*.]
          \item Démontrer que la suite $\left(u_n\right)$ est une suite géométrique dont on précisera la raison et le premier terme $u_0$.
          \item Exprimer $u_n$ en fonction de $n$ puis démontrer que, pour tout entier $n$,
          \par
          \[a_n=-0,165 \times 0,60^n + 0,625.\]
          \par
          \item Résoudre par le calcul l'inéquation $a_n \geqslant 0,62$.
          \par
          Quel résultat de la question \textbf{2.} retrouve-t-on~?
     \end{enumerate}
\end{enumerate}

\end{document}
µ
\documentclass[a4paper]{article}

%================================================================================================================================
%
% Packages
%
%================================================================================================================================

\usepackage[T1]{fontenc} 	% pour caractères accentués
\usepackage[utf8]{inputenc}  % encodage utf8
\usepackage[french]{babel}	% langue : français
\usepackage{fourier}			% caractères plus lisibles
\usepackage[dvipsnames]{xcolor} % couleurs
\usepackage{fancyhdr}		% réglage header footer
\usepackage{needspace}		% empêcher sauts de page mal placés
\usepackage{graphicx}		% pour inclure des graphiques
\usepackage{enumitem,cprotect}		% personnalise les listes d'items (nécessaire pour ol, al ...)
\usepackage{hyperref}		% Liens hypertexte
\usepackage{pstricks,pst-all,pst-node,pstricks-add,pst-math,pst-plot,pst-tree,pst-eucl} % pstricks
\usepackage[a4paper,includeheadfoot,top=2cm,left=3cm, bottom=2cm,right=3cm]{geometry} % marges etc.
\usepackage{comment}			% commentaires multilignes
\usepackage{amsmath,environ} % maths (matrices, etc.)
\usepackage{amssymb,makeidx}
\usepackage{bm}				% bold maths
\usepackage{tabularx}		% tableaux
\usepackage{colortbl}		% tableaux en couleur
\usepackage{fontawesome}		% Fontawesome
\usepackage{environ}			% environment with command
\usepackage{fp}				% calculs pour ps-tricks
\usepackage{multido}			% pour ps tricks
\usepackage[np]{numprint}	% formattage nombre
\usepackage{tikz,tkz-tab} 			% package principal TikZ
\usepackage{pgfplots}   % axes
\usepackage{mathrsfs}    % cursives
\usepackage{calc}			% calcul taille boites
\usepackage[scaled=0.875]{helvet} % font sans serif
\usepackage{svg} % svg
\usepackage{scrextend} % local margin
\usepackage{scratch} %scratch
\usepackage{multicol} % colonnes
%\usepackage{infix-RPN,pst-func} % formule en notation polanaise inversée
\usepackage{listings}

%================================================================================================================================
%
% Réglages de base
%
%================================================================================================================================

\lstset{
language=Python,   % R code
literate=
{á}{{\'a}}1
{à}{{\`a}}1
{ã}{{\~a}}1
{é}{{\'e}}1
{è}{{\`e}}1
{ê}{{\^e}}1
{í}{{\'i}}1
{ó}{{\'o}}1
{õ}{{\~o}}1
{ú}{{\'u}}1
{ü}{{\"u}}1
{ç}{{\c{c}}}1
{~}{{ }}1
}


\definecolor{codegreen}{rgb}{0,0.6,0}
\definecolor{codegray}{rgb}{0.5,0.5,0.5}
\definecolor{codepurple}{rgb}{0.58,0,0.82}
\definecolor{backcolour}{rgb}{0.95,0.95,0.92}

\lstdefinestyle{mystyle}{
    backgroundcolor=\color{backcolour},   
    commentstyle=\color{codegreen},
    keywordstyle=\color{magenta},
    numberstyle=\tiny\color{codegray},
    stringstyle=\color{codepurple},
    basicstyle=\ttfamily\footnotesize,
    breakatwhitespace=false,         
    breaklines=true,                 
    captionpos=b,                    
    keepspaces=true,                 
    numbers=left,                    
xleftmargin=2em,
framexleftmargin=2em,            
    showspaces=false,                
    showstringspaces=false,
    showtabs=false,                  
    tabsize=2,
    upquote=true
}

\lstset{style=mystyle}


\lstset{style=mystyle}
\newcommand{\imgdir}{C:/laragon/www/newmc/assets/imgsvg/}
\newcommand{\imgsvgdir}{C:/laragon/www/newmc/assets/imgsvg/}

\definecolor{mcgris}{RGB}{220, 220, 220}% ancien~; pour compatibilité
\definecolor{mcbleu}{RGB}{52, 152, 219}
\definecolor{mcvert}{RGB}{125, 194, 70}
\definecolor{mcmauve}{RGB}{154, 0, 215}
\definecolor{mcorange}{RGB}{255, 96, 0}
\definecolor{mcturquoise}{RGB}{0, 153, 153}
\definecolor{mcrouge}{RGB}{255, 0, 0}
\definecolor{mclightvert}{RGB}{205, 234, 190}

\definecolor{gris}{RGB}{220, 220, 220}
\definecolor{bleu}{RGB}{52, 152, 219}
\definecolor{vert}{RGB}{125, 194, 70}
\definecolor{mauve}{RGB}{154, 0, 215}
\definecolor{orange}{RGB}{255, 96, 0}
\definecolor{turquoise}{RGB}{0, 153, 153}
\definecolor{rouge}{RGB}{255, 0, 0}
\definecolor{lightvert}{RGB}{205, 234, 190}
\setitemize[0]{label=\color{lightvert}  $\bullet$}

\pagestyle{fancy}
\renewcommand{\headrulewidth}{0.2pt}
\fancyhead[L]{maths-cours.fr}
\fancyhead[R]{\thepage}
\renewcommand{\footrulewidth}{0.2pt}
\fancyfoot[C]{}

\newcolumntype{C}{>{\centering\arraybackslash}X}
\newcolumntype{s}{>{\hsize=.35\hsize\arraybackslash}X}

\setlength{\parindent}{0pt}		 
\setlength{\parskip}{3mm}
\setlength{\headheight}{1cm}

\def\ebook{ebook}
\def\book{book}
\def\web{web}
\def\type{web}

\newcommand{\vect}[1]{\overrightarrow{\,\mathstrut#1\,}}

\def\Oij{$\left(\text{O}~;~\vect{\imath},~\vect{\jmath}\right)$}
\def\Oijk{$\left(\text{O}~;~\vect{\imath},~\vect{\jmath},~\vect{k}\right)$}
\def\Ouv{$\left(\text{O}~;~\vect{u},~\vect{v}\right)$}

\hypersetup{breaklinks=true, colorlinks = true, linkcolor = OliveGreen, urlcolor = OliveGreen, citecolor = OliveGreen, pdfauthor={Didier BONNEL - https://www.maths-cours.fr} } % supprime les bordures autour des liens

\renewcommand{\arg}[0]{\text{arg}}

\everymath{\displaystyle}

%================================================================================================================================
%
% Macros - Commandes
%
%================================================================================================================================

\newcommand\meta[2]{    			% Utilisé pour créer le post HTML.
	\def\titre{titre}
	\def\url{url}
	\def\arg{#1}
	\ifx\titre\arg
		\newcommand\maintitle{#2}
		\fancyhead[L]{#2}
		{\Large\sffamily \MakeUppercase{#2}}
		\vspace{1mm}\textcolor{mcvert}{\hrule}
	\fi 
	\ifx\url\arg
		\fancyfoot[L]{\href{https://www.maths-cours.fr#2}{\black \footnotesize{https://www.maths-cours.fr#2}}}
	\fi 
}


\newcommand\TitreC[1]{    		% Titre centré
     \needspace{3\baselineskip}
     \begin{center}\textbf{#1}\end{center}
}

\newcommand\newpar{    		% paragraphe
     \par
}

\newcommand\nosp {    		% commande vide (pas d'espace)
}
\newcommand{\id}[1]{} %ignore

\newcommand\boite[2]{				% Boite simple sans titre
	\vspace{5mm}
	\setlength{\fboxrule}{0.2mm}
	\setlength{\fboxsep}{5mm}	
	\fcolorbox{#1}{#1!3}{\makebox[\linewidth-2\fboxrule-2\fboxsep]{
  		\begin{minipage}[t]{\linewidth-2\fboxrule-4\fboxsep}\setlength{\parskip}{3mm}
  			 #2
  		\end{minipage}
	}}
	\vspace{5mm}
}

\newcommand\CBox[4]{				% Boites
	\vspace{5mm}
	\setlength{\fboxrule}{0.2mm}
	\setlength{\fboxsep}{5mm}
	
	\fcolorbox{#1}{#1!3}{\makebox[\linewidth-2\fboxrule-2\fboxsep]{
		\begin{minipage}[t]{1cm}\setlength{\parskip}{3mm}
	  		\textcolor{#1}{\LARGE{#2}}    
 	 	\end{minipage}  
  		\begin{minipage}[t]{\linewidth-2\fboxrule-4\fboxsep}\setlength{\parskip}{3mm}
			\raisebox{1.2mm}{\normalsize\sffamily{\textcolor{#1}{#3}}}						
  			 #4
  		\end{minipage}
	}}
	\vspace{5mm}
}

\newcommand\cadre[3]{				% Boites convertible html
	\par
	\vspace{2mm}
	\setlength{\fboxrule}{0.1mm}
	\setlength{\fboxsep}{5mm}
	\fcolorbox{#1}{white}{\makebox[\linewidth-2\fboxrule-2\fboxsep]{
  		\begin{minipage}[t]{\linewidth-2\fboxrule-4\fboxsep}\setlength{\parskip}{3mm}
			\raisebox{-2.5mm}{\sffamily \small{\textcolor{#1}{\MakeUppercase{#2}}}}		
			\par		
  			 #3
 	 		\end{minipage}
	}}
		\vspace{2mm}
	\par
}

\newcommand\bloc[3]{				% Boites convertible html sans bordure
     \needspace{2\baselineskip}
     {\sffamily \small{\textcolor{#1}{\MakeUppercase{#2}}}}    
		\par		
  			 #3
		\par
}

\newcommand\CHelp[1]{
     \CBox{Plum}{\faInfoCircle}{À RETENIR}{#1}
}

\newcommand\CUp[1]{
     \CBox{NavyBlue}{\faThumbsOUp}{EN PRATIQUE}{#1}
}

\newcommand\CInfo[1]{
     \CBox{Sepia}{\faArrowCircleRight}{REMARQUE}{#1}
}

\newcommand\CRedac[1]{
     \CBox{PineGreen}{\faEdit}{BIEN R\'EDIGER}{#1}
}

\newcommand\CError[1]{
     \CBox{Red}{\faExclamationTriangle}{ATTENTION}{#1}
}

\newcommand\TitreExo[2]{
\needspace{4\baselineskip}
 {\sffamily\large EXERCICE #1\ (\emph{#2 points})}
\vspace{5mm}
}

\newcommand\img[2]{
          \includegraphics[width=#2\paperwidth]{\imgdir#1}
}

\newcommand\imgsvg[2]{
       \begin{center}   \includegraphics[width=#2\paperwidth]{\imgsvgdir#1} \end{center}
}


\newcommand\Lien[2]{
     \href{#1}{#2 \tiny \faExternalLink}
}
\newcommand\mcLien[2]{
     \href{https~://www.maths-cours.fr/#1}{#2 \tiny \faExternalLink}
}

\newcommand{\euro}{\eurologo{}}

%================================================================================================================================
%
% Macros - Environement
%
%================================================================================================================================

\newenvironment{tex}{ %
}
{%
}

\newenvironment{indente}{ %
	\setlength\parindent{10mm}
}

{
	\setlength\parindent{0mm}
}

\newenvironment{corrige}{%
     \needspace{3\baselineskip}
     \medskip
     \textbf{\textsc{Corrigé}}
     \medskip
}
{
}

\newenvironment{extern}{%
     \begin{center}
     }
     {
     \end{center}
}

\NewEnviron{code}{%
	\par
     \boite{gray}{\texttt{%
     \BODY
     }}
     \par
}

\newenvironment{vbloc}{% boite sans cadre empeche saut de page
     \begin{minipage}[t]{\linewidth}
     }
     {
     \end{minipage}
}
\NewEnviron{h2}{%
    \needspace{3\baselineskip}
    \vspace{0.6cm}
	\noindent \MakeUppercase{\sffamily \large \BODY}
	\vspace{1mm}\textcolor{mcgris}{\hrule}\vspace{0.4cm}
	\par
}{}

\NewEnviron{h3}{%
    \needspace{3\baselineskip}
	\vspace{5mm}
	\textsc{\BODY}
	\par
}

\NewEnviron{margeneg}{ %
\begin{addmargin}[-1cm]{0cm}
\BODY
\end{addmargin}
}

\NewEnviron{html}{%
}

\begin{document}
\meta{url}{/exercices/fonctions-bac-es-l-amerique-du-nord-2018/}
\meta{pid}{7892}
\meta{titre}{Fonctions - Bac ES/L Amérique du Nord  2018}
\meta{type}{exercices}
%
\begin{h2}Exercice 4 (6 points)\end{h2}
\textbf{Commun à  tous les candidats}
\medskip
On appelle fonction \og \emph{satisfaction} \fg{} toute fonction dérivable qui prend ses valeurs entre 0 et 100. Lorsque la fonction  \og \emph{satisfaction} \fg{} atteint la valeur $100$, on dit qu'il y a \og \emph{saturation} \fg{}.
\par
On définit aussi la fonction \og \emph{envie} \fg{} comme la fonction dérivée de la fonction \og \emph{satisfaction} \fg{}. On dira qu'il y a \og \emph{souhait} \fg{} lorsque la fonction \og \emph{envie} \fg{} est positive ou nulle et qu'il y a \og \emph{rejet} \fg{}  lorsque la fonction \og \emph{envie} \fg{} est strictement négative.
\medskip
\textbf{Dans chaque partie, on teste un modèle de fonction \og \emph{\textbf{satisfaction}} \fg{} différent.} \\
\textbf{Les parties A, B et C sont indépendantes.}
\begin{center}\begin{h3}Partie A \end{h3}\end{center}
Un étudiant prépare un concours, pour lequel sa durée de travail varie entre 0 et 6 heures par jour. Il modélise sa satisfaction en fonction de son temps de travail quotidien par la fonction \og \emph{satisfaction} \fg{} $f$ dont la courbe représentative est donnée ci-dessous ($x$ est exprimé en heures).
\begin{center}
     \begin{extern}%width="370"
          \psset{xunit=1cm,yunit=0.05cm,arrowsize=2pt 2}
          \def\xmin {-1}   \def\xmax {7}
          \def\ymin {-20}   \def\ymax {120}
          \begin{pspicture*}(\xmin,\ymin)(\xmax,\ymax)
               \psgrid[unit=1cm,subgriddiv=1,  gridlabels=0, gridcolor=lightgray](0,0)(8,6)
               \psaxes[ticksize=-2pt 2pt,Dy=20]{->}(0,0)(0,0)(\xmax,\ymax)%[$x$,-110][$y$,200]
               \def\f{x x -100 mul 600 add mul 9 div}                           % définition de la fonction
               \psplot[linewidth=0.75pt,plotpoints=2000,linecolor=red]{0}{6}{\f}
               \uput[ur](5.7,20){\red $\mathcal{C}_f$}
               \psdots[dotstyle=*,dotscale=1,linecolor=red](3,100)
               \psline[linewidth=0.75pt,linecolor=red]{<->}(2,100)(4,100)
          \end{pspicture*}
     \end{extern}
\end{center}
\par
\textbf{Par lecture graphique, répondre aux questions suivantes.}
\begin{enumerate}
     \item Lire la durée de travail quotidien menant à \og \emph{saturation} \fg{}.
     \item Déterminer à partir de quelle durée de travail il y a \og \emph{rejet} \fg{}.
\end{enumerate}
\begin{center}\begin{h3}Partie B \end{h3}\end{center}
Le directeur d'une agence de trekking modélise la satisfaction de ses clients en fonction de la durée de leur séjour. On admet que la fonction \og \emph{satisfaction} \fg{} $g$ est définie sur l'intervalle $[0\,~:\,30]$ par
$g(x)=12,5 x \text{e}^{-0,125x+1}$ ($x$ est exprimé en jour).
\begin{enumerate}
     \item Démontrer que, pour tout $x$ de l'intervalle   $[0~;~30]$,
     \par
     \[g'(x)=(12,5-1,562~5 x) \text{e}^{-0,125x+1}.\]
     \par
     \item \'Etudier le signe de $g'(x)$ sur l'intervalle  $[0~;~30]$ puis dresser le tableau des variations de $g$ sur cet intervalle.
     \item Quelle durée de séjour correspond-elle à l'effet \og \emph{saturation} \fg{}~?
\end{enumerate}
\begin{center}\begin{h3}Partie C \end{h3}\end{center}
La direction des ressources humaines d'une entreprise modélise la satisfaction d'un salarié en fonction du salaire annuel qu'il perçoit. On admet que la fonction \og \emph{satisfaction} \fg{} $h$, est définie sur l'intervalle $[10~;~50]$ par
\par
\[h(x)=\dfrac{90}{1+\text{e}^{-0,25 x +6}}\]
\par
($x$ est exprimé en millier d'euros).
\par
La courbe $\mathcal{C}_h$ de la fonction $h$ est représentée ci-dessous~:
\begin{center}
     \begin{extern}
          \psset{xunit=0.2cm,yunit=0.1cm,arrowsize=3pt 3}
          \def\xmin {-3}   \def\xmax {55}
          \def\ymin {-5}   \def\ymax {100}
          \begin{pspicture*}(\xmin,\ymin)(\xmax,\ymax)
               \psgrid[unit=1cm,subgriddiv=1,  gridlabels=0, gridcolor=lightgray](0,0)(0,0)(11,10)
               \psaxes[ticksize=-2pt 2pt,Dy=10,Dx=5]{->}(0,0)(0,0)(\xmax,\ymax)
               \def\f{90 1 2.7183 x -0.25 mul 6 add exp add div}                           % définition de la fonction
               \psplot[plotpoints=2000,linecolor=red,linewidth=0.75pt]{10}{50}{\f}
               \uput[ur](30,70){\red $\mathcal{C}_h$}
          \end{pspicture*}
     \end{extern}
\end{center}
Un logiciel de calcul formel donne les résultats suivants~:
\begin{center}
     \begin{extern}%width="600"
          \renewcommand{\arraystretch}{1.5}
          \begin{tabular}{|l|l|}
               \hline
               & \verb!Dériver(90/(1 + exp(-0.25 * x + 6))) !\\
               1 & \\
               & \hfill{}$\dfrac{22,5 \text{e}^{-0,25x+6}}{\left(1+\text{e}^{-0,25x+6}\right)^2}$\rule[-20pt]{0pt}{0pt}\\
               \hline
               & \verb!Dériver(22.5 * exp(-0,25 x + 6)/(1 + exp(-0,25 * x + 6))^2) !\\
               2 & \\
               & \hfill{}$\dfrac{5,625 \text{e}^{-0,25x+6} \left(\text{e}^{-0,25x + 6} -1\right)}{\left(1+\text{e}^{-0,25x+6}\right)^3}$\rule[-20pt]{0pt}{0pt}\\
               \hline
          \end{tabular}
     \end{extern}
\end{center}
\begin{enumerate}
     \item Donner sans justification une expression de $h''(x)$.
     \item Résoudre dans l'intervalle $[10\,~:\,50]$ l'inéquation $\text{e}^{-0,25x+6}-1>0$.
     \item \'Etudier la convexité de la fonction $h$ sur l'intervalle $[10\,~:\,50]$.
     \item À partir de quel salaire annuel peut-on estimer que la fonction \og \emph{envie} \fg{} décroît~? Justifier.
     \item Déterminer, en le justifiant, pour quel salaire annuel la fonction \og \emph{satisfaction} \fg{} atteint 80.\\ \emph{Arrondir au millier d'euros.}
\end{enumerate}

\end{document}
µ
\documentclass[a4paper]{article}

%================================================================================================================================
%
% Packages
%
%================================================================================================================================

\usepackage[T1]{fontenc} 	% pour caractères accentués
\usepackage[utf8]{inputenc}  % encodage utf8
\usepackage[french]{babel}	% langue : français
\usepackage{fourier}			% caractères plus lisibles
\usepackage[dvipsnames]{xcolor} % couleurs
\usepackage{fancyhdr}		% réglage header footer
\usepackage{needspace}		% empêcher sauts de page mal placés
\usepackage{graphicx}		% pour inclure des graphiques
\usepackage{enumitem,cprotect}		% personnalise les listes d'items (nécessaire pour ol, al ...)
\usepackage{hyperref}		% Liens hypertexte
\usepackage{pstricks,pst-all,pst-node,pstricks-add,pst-math,pst-plot,pst-tree,pst-eucl} % pstricks
\usepackage[a4paper,includeheadfoot,top=2cm,left=3cm, bottom=2cm,right=3cm]{geometry} % marges etc.
\usepackage{comment}			% commentaires multilignes
\usepackage{amsmath,environ} % maths (matrices, etc.)
\usepackage{amssymb,makeidx}
\usepackage{bm}				% bold maths
\usepackage{tabularx}		% tableaux
\usepackage{colortbl}		% tableaux en couleur
\usepackage{fontawesome}		% Fontawesome
\usepackage{environ}			% environment with command
\usepackage{fp}				% calculs pour ps-tricks
\usepackage{multido}			% pour ps tricks
\usepackage[np]{numprint}	% formattage nombre
\usepackage{tikz,tkz-tab} 			% package principal TikZ
\usepackage{pgfplots}   % axes
\usepackage{mathrsfs}    % cursives
\usepackage{calc}			% calcul taille boites
\usepackage[scaled=0.875]{helvet} % font sans serif
\usepackage{svg} % svg
\usepackage{scrextend} % local margin
\usepackage{scratch} %scratch
\usepackage{multicol} % colonnes
%\usepackage{infix-RPN,pst-func} % formule en notation polanaise inversée
\usepackage{listings}

%================================================================================================================================
%
% Réglages de base
%
%================================================================================================================================

\lstset{
language=Python,   % R code
literate=
{á}{{\'a}}1
{à}{{\`a}}1
{ã}{{\~a}}1
{é}{{\'e}}1
{è}{{\`e}}1
{ê}{{\^e}}1
{í}{{\'i}}1
{ó}{{\'o}}1
{õ}{{\~o}}1
{ú}{{\'u}}1
{ü}{{\"u}}1
{ç}{{\c{c}}}1
{~}{{ }}1
}


\definecolor{codegreen}{rgb}{0,0.6,0}
\definecolor{codegray}{rgb}{0.5,0.5,0.5}
\definecolor{codepurple}{rgb}{0.58,0,0.82}
\definecolor{backcolour}{rgb}{0.95,0.95,0.92}

\lstdefinestyle{mystyle}{
    backgroundcolor=\color{backcolour},   
    commentstyle=\color{codegreen},
    keywordstyle=\color{magenta},
    numberstyle=\tiny\color{codegray},
    stringstyle=\color{codepurple},
    basicstyle=\ttfamily\footnotesize,
    breakatwhitespace=false,         
    breaklines=true,                 
    captionpos=b,                    
    keepspaces=true,                 
    numbers=left,                    
xleftmargin=2em,
framexleftmargin=2em,            
    showspaces=false,                
    showstringspaces=false,
    showtabs=false,                  
    tabsize=2,
    upquote=true
}

\lstset{style=mystyle}


\lstset{style=mystyle}
\newcommand{\imgdir}{C:/laragon/www/newmc/assets/imgsvg/}
\newcommand{\imgsvgdir}{C:/laragon/www/newmc/assets/imgsvg/}

\definecolor{mcgris}{RGB}{220, 220, 220}% ancien~; pour compatibilité
\definecolor{mcbleu}{RGB}{52, 152, 219}
\definecolor{mcvert}{RGB}{125, 194, 70}
\definecolor{mcmauve}{RGB}{154, 0, 215}
\definecolor{mcorange}{RGB}{255, 96, 0}
\definecolor{mcturquoise}{RGB}{0, 153, 153}
\definecolor{mcrouge}{RGB}{255, 0, 0}
\definecolor{mclightvert}{RGB}{205, 234, 190}

\definecolor{gris}{RGB}{220, 220, 220}
\definecolor{bleu}{RGB}{52, 152, 219}
\definecolor{vert}{RGB}{125, 194, 70}
\definecolor{mauve}{RGB}{154, 0, 215}
\definecolor{orange}{RGB}{255, 96, 0}
\definecolor{turquoise}{RGB}{0, 153, 153}
\definecolor{rouge}{RGB}{255, 0, 0}
\definecolor{lightvert}{RGB}{205, 234, 190}
\setitemize[0]{label=\color{lightvert}  $\bullet$}

\pagestyle{fancy}
\renewcommand{\headrulewidth}{0.2pt}
\fancyhead[L]{maths-cours.fr}
\fancyhead[R]{\thepage}
\renewcommand{\footrulewidth}{0.2pt}
\fancyfoot[C]{}

\newcolumntype{C}{>{\centering\arraybackslash}X}
\newcolumntype{s}{>{\hsize=.35\hsize\arraybackslash}X}

\setlength{\parindent}{0pt}		 
\setlength{\parskip}{3mm}
\setlength{\headheight}{1cm}

\def\ebook{ebook}
\def\book{book}
\def\web{web}
\def\type{web}

\newcommand{\vect}[1]{\overrightarrow{\,\mathstrut#1\,}}

\def\Oij{$\left(\text{O}~;~\vect{\imath},~\vect{\jmath}\right)$}
\def\Oijk{$\left(\text{O}~;~\vect{\imath},~\vect{\jmath},~\vect{k}\right)$}
\def\Ouv{$\left(\text{O}~;~\vect{u},~\vect{v}\right)$}

\hypersetup{breaklinks=true, colorlinks = true, linkcolor = OliveGreen, urlcolor = OliveGreen, citecolor = OliveGreen, pdfauthor={Didier BONNEL - https://www.maths-cours.fr} } % supprime les bordures autour des liens

\renewcommand{\arg}[0]{\text{arg}}

\everymath{\displaystyle}

%================================================================================================================================
%
% Macros - Commandes
%
%================================================================================================================================

\newcommand\meta[2]{    			% Utilisé pour créer le post HTML.
	\def\titre{titre}
	\def\url{url}
	\def\arg{#1}
	\ifx\titre\arg
		\newcommand\maintitle{#2}
		\fancyhead[L]{#2}
		{\Large\sffamily \MakeUppercase{#2}}
		\vspace{1mm}\textcolor{mcvert}{\hrule}
	\fi 
	\ifx\url\arg
		\fancyfoot[L]{\href{https://www.maths-cours.fr#2}{\black \footnotesize{https://www.maths-cours.fr#2}}}
	\fi 
}


\newcommand\TitreC[1]{    		% Titre centré
     \needspace{3\baselineskip}
     \begin{center}\textbf{#1}\end{center}
}

\newcommand\newpar{    		% paragraphe
     \par
}

\newcommand\nosp {    		% commande vide (pas d'espace)
}
\newcommand{\id}[1]{} %ignore

\newcommand\boite[2]{				% Boite simple sans titre
	\vspace{5mm}
	\setlength{\fboxrule}{0.2mm}
	\setlength{\fboxsep}{5mm}	
	\fcolorbox{#1}{#1!3}{\makebox[\linewidth-2\fboxrule-2\fboxsep]{
  		\begin{minipage}[t]{\linewidth-2\fboxrule-4\fboxsep}\setlength{\parskip}{3mm}
  			 #2
  		\end{minipage}
	}}
	\vspace{5mm}
}

\newcommand\CBox[4]{				% Boites
	\vspace{5mm}
	\setlength{\fboxrule}{0.2mm}
	\setlength{\fboxsep}{5mm}
	
	\fcolorbox{#1}{#1!3}{\makebox[\linewidth-2\fboxrule-2\fboxsep]{
		\begin{minipage}[t]{1cm}\setlength{\parskip}{3mm}
	  		\textcolor{#1}{\LARGE{#2}}    
 	 	\end{minipage}  
  		\begin{minipage}[t]{\linewidth-2\fboxrule-4\fboxsep}\setlength{\parskip}{3mm}
			\raisebox{1.2mm}{\normalsize\sffamily{\textcolor{#1}{#3}}}						
  			 #4
  		\end{minipage}
	}}
	\vspace{5mm}
}

\newcommand\cadre[3]{				% Boites convertible html
	\par
	\vspace{2mm}
	\setlength{\fboxrule}{0.1mm}
	\setlength{\fboxsep}{5mm}
	\fcolorbox{#1}{white}{\makebox[\linewidth-2\fboxrule-2\fboxsep]{
  		\begin{minipage}[t]{\linewidth-2\fboxrule-4\fboxsep}\setlength{\parskip}{3mm}
			\raisebox{-2.5mm}{\sffamily \small{\textcolor{#1}{\MakeUppercase{#2}}}}		
			\par		
  			 #3
 	 		\end{minipage}
	}}
		\vspace{2mm}
	\par
}

\newcommand\bloc[3]{				% Boites convertible html sans bordure
     \needspace{2\baselineskip}
     {\sffamily \small{\textcolor{#1}{\MakeUppercase{#2}}}}    
		\par		
  			 #3
		\par
}

\newcommand\CHelp[1]{
     \CBox{Plum}{\faInfoCircle}{À RETENIR}{#1}
}

\newcommand\CUp[1]{
     \CBox{NavyBlue}{\faThumbsOUp}{EN PRATIQUE}{#1}
}

\newcommand\CInfo[1]{
     \CBox{Sepia}{\faArrowCircleRight}{REMARQUE}{#1}
}

\newcommand\CRedac[1]{
     \CBox{PineGreen}{\faEdit}{BIEN R\'EDIGER}{#1}
}

\newcommand\CError[1]{
     \CBox{Red}{\faExclamationTriangle}{ATTENTION}{#1}
}

\newcommand\TitreExo[2]{
\needspace{4\baselineskip}
 {\sffamily\large EXERCICE #1\ (\emph{#2 points})}
\vspace{5mm}
}

\newcommand\img[2]{
          \includegraphics[width=#2\paperwidth]{\imgdir#1}
}

\newcommand\imgsvg[2]{
       \begin{center}   \includegraphics[width=#2\paperwidth]{\imgsvgdir#1} \end{center}
}


\newcommand\Lien[2]{
     \href{#1}{#2 \tiny \faExternalLink}
}
\newcommand\mcLien[2]{
     \href{https~://www.maths-cours.fr/#1}{#2 \tiny \faExternalLink}
}

\newcommand{\euro}{\eurologo{}}

%================================================================================================================================
%
% Macros - Environement
%
%================================================================================================================================

\newenvironment{tex}{ %
}
{%
}

\newenvironment{indente}{ %
	\setlength\parindent{10mm}
}

{
	\setlength\parindent{0mm}
}

\newenvironment{corrige}{%
     \needspace{3\baselineskip}
     \medskip
     \textbf{\textsc{Corrigé}}
     \medskip
}
{
}

\newenvironment{extern}{%
     \begin{center}
     }
     {
     \end{center}
}

\NewEnviron{code}{%
	\par
     \boite{gray}{\texttt{%
     \BODY
     }}
     \par
}

\newenvironment{vbloc}{% boite sans cadre empeche saut de page
     \begin{minipage}[t]{\linewidth}
     }
     {
     \end{minipage}
}
\NewEnviron{h2}{%
    \needspace{3\baselineskip}
    \vspace{0.6cm}
	\noindent \MakeUppercase{\sffamily \large \BODY}
	\vspace{1mm}\textcolor{mcgris}{\hrule}\vspace{0.4cm}
	\par
}{}

\NewEnviron{h3}{%
    \needspace{3\baselineskip}
	\vspace{5mm}
	\textsc{\BODY}
	\par
}

\NewEnviron{margeneg}{ %
\begin{addmargin}[-1cm]{0cm}
\BODY
\end{addmargin}
}

\NewEnviron{html}{%
}

\begin{document}
\meta{url}{/exercices/probabilites-bac-s-amerique-du-nord-2018/}
\meta{pid}{7916}
\meta{titre}{Probabilités - Bac S Amérique du Nord  2018}
\meta{type}{exercices}
%
\begin{h2}Exercice 1 (6 points)\end{h2}
\textbf{Commun à  tous les candidats}
\medskip
On étudie certaines caractéristiques d'un supermarché d'une petite ville.
\bigskip
\begin{center}\begin{h3}Partie A - Démonstration préliminaire \end{h3}\end{center}
\medskip
Soit $X$ une variable aléatoire qui suit la loi exponentielle de paramètre $0,2$.
\par
On rappelle que l'espérance de la variable aléatoire $X$, notée $E(X)$, est égale à~:
\par
\[\displaystyle\lim_{x \to + \infty}\displaystyle\int_{0}^{x}  0,2t\text{e}^{-0,2t}\:\text{d}t.\]
\par
Le but de cette partie est de démontrer que $E(X) = 5$.
\medskip
\begin{enumerate}
     \item On note $g$ la fonction définie sur l'intervalle $[0~:~+\infty[$ par $g(t) = 0,2t\text{e}^{-0,2t}$.
     \par
     On définit la fonction $G$ sur l'intervalle $[0~:~+\infty[$ par $G(t) = (- t - 5)\text{e}^{-0,2t}$.
     \par
     Vérifier que $G$ est une primitive de $g$ sur l'intervalle $[0~:~+\infty[$.
     \item  En déduire que la valeur exacte de $E(X)$ est 5.
     \par
     \emph{Indication~: on pourra utiliser, sans le démontrer, le résultat suivant }~:
     \par
     \[\displaystyle\lim_{x \to + \infty} x \text{e}^{- 0,2x} = 0.\]
\end{enumerate}
\bigskip
\begin{center}\begin{h3}Partie B - Étude de la durée de présence d'un client dans le supermarché \end{h3}\end{center}
\medskip
Une étude commandée par le gérant du supermarché permet de modéliser la durée, exprimée en
minutes, passée dans le supermarché par un client choisi au hasard par une variable aléatoire $T$.
\par
Cette variable $T$ suit une loi normale d'espérance $40$ minutes et d'écart type un réel positif noté $\sigma$.
\par
Grâce à cette étude, on estime que $P(T < 10) = 0,067$.
\medskip
\begin{enumerate}
     \item Déterminer une valeur arrondie du réel $\sigma$ à la seconde près.
     \item Dans cette question, on prend $\sigma = 20$~minutes. Quelle est alors la proportion de clients qui
     passent plus d'une heure dans le supermarché~?
\end{enumerate}
\bigskip
\begin{center}\begin{h3}Partie C - Durée d'attente pour le paiement \end{h3}\end{center}
\medskip
Ce supermarché laisse le choix au client d'utiliser seul des bornes automatiques de paiement ou
bien de passer par une caisse gérée par un opérateur.
\medskip
\begin{enumerate}
     \item La durée d'attente à une borne automatique, exprimée en minutes, est modélisée par une
     variable aléatoire qui suit la loi exponentielle de paramètre $0,2$~min$^{-1}$.
     \begin{enumerate}[label=\alph*.]
          \item Donner la durée moyenne d'attente d'un client à une borne automatique de paiement.
          \item Calculer la probabilité, arrondie à $10^{-3}$, que la durée d'attente d'un client à une borne automatique de paiement soit supérieure à $10$ minutes.
     \end{enumerate}
     \item L'étude commandée par le gérant conduit à la modélisation suivante~:
     \begin{indent}
          \begin{itemize}
               \item parmi les clients ayant choisi de passer à une borne automatique, 86\,\% attendent moins de $10$ minutes~:
               \item parmi les clients passant en caisse, 63\,\% attendent moins de $10$ minutes.
          \end{itemize}
     \end{indent}
     \medskip
     On choisit un client du magasin au hasard et on définit les événements suivants~:
     \par
     $B$~: \og le client paye à une borne automatique \fg{}~:
     \par
     $\overline{B}$~: \og le client paye à une caisse avec opérateur \fg{}~:
     \par
     $S$~: \og la durée d'attente du client lors du paiement est inférieure à $10$ minutes \fg.
     \par
     Une attente supérieure à dix minutes à une caisse avec opérateur ou à une borne automatique
     engendre chez le client une perception négative du magasin. Le gérant souhaite que
     plus de 75\,\% des clients attendent moins de $10$ minutes.
     \par
     Quelle est la proportion minimale de clients qui doivent choisir une borne automatique de
     paiement pour que cet objectif soit atteint~?
\end{enumerate}
\bigskip
\begin{center}\begin{h3}Partie D - Bons d'achat \end{h3}\end{center}
\medskip
Lors du paiement, des cartes à gratter, gagnantes ou perdantes, sont distribuées aux clients. Le
nombre de cartes distribuées dépend du montant des achats. Chaque client a droit à une carte à
gratter par tranche de $10$~\euro{} d'achats.
\par
Par exemple, si le montant des achats est 58,64~\euro, alors le client obtient $5$ cartes~: si le montant est
124,31~\euro, le client obtient $12$~cartes.
\par
Les cartes gagnantes représentent $0,5$\,\% de l'ensemble du stock de cartes. De plus, ce stock est
suffisamment grand pour assimiler la distribution d'une carte à un tirage avec remise.
\medskip
\begin{enumerate}
     \item Un client effectue des achats pour un montant de 158,02~\euro.
     \par
     Quelle est la probabilité, arrondie à $10^{-2}$, qu'il obtienne au moins une carte gagnante~?
     \item  À partir de quel montant d'achats, arrondi à 10~\euro, la probabilité d'obtenir au moins une carte
     gagnante est-elle supérieure à 50\,\%~?
\end{enumerate}

\end{document}
µ
\documentclass[a4paper]{article}

%================================================================================================================================
%
% Packages
%
%================================================================================================================================

\usepackage[T1]{fontenc} 	% pour caractères accentués
\usepackage[utf8]{inputenc}  % encodage utf8
\usepackage[french]{babel}	% langue : français
\usepackage{fourier}			% caractères plus lisibles
\usepackage[dvipsnames]{xcolor} % couleurs
\usepackage{fancyhdr}		% réglage header footer
\usepackage{needspace}		% empêcher sauts de page mal placés
\usepackage{graphicx}		% pour inclure des graphiques
\usepackage{enumitem,cprotect}		% personnalise les listes d'items (nécessaire pour ol, al ...)
\usepackage{hyperref}		% Liens hypertexte
\usepackage{pstricks,pst-all,pst-node,pstricks-add,pst-math,pst-plot,pst-tree,pst-eucl} % pstricks
\usepackage[a4paper,includeheadfoot,top=2cm,left=3cm, bottom=2cm,right=3cm]{geometry} % marges etc.
\usepackage{comment}			% commentaires multilignes
\usepackage{amsmath,environ} % maths (matrices, etc.)
\usepackage{amssymb,makeidx}
\usepackage{bm}				% bold maths
\usepackage{tabularx}		% tableaux
\usepackage{colortbl}		% tableaux en couleur
\usepackage{fontawesome}		% Fontawesome
\usepackage{environ}			% environment with command
\usepackage{fp}				% calculs pour ps-tricks
\usepackage{multido}			% pour ps tricks
\usepackage[np]{numprint}	% formattage nombre
\usepackage{tikz,tkz-tab} 			% package principal TikZ
\usepackage{pgfplots}   % axes
\usepackage{mathrsfs}    % cursives
\usepackage{calc}			% calcul taille boites
\usepackage[scaled=0.875]{helvet} % font sans serif
\usepackage{svg} % svg
\usepackage{scrextend} % local margin
\usepackage{scratch} %scratch
\usepackage{multicol} % colonnes
%\usepackage{infix-RPN,pst-func} % formule en notation polanaise inversée
\usepackage{listings}

%================================================================================================================================
%
% Réglages de base
%
%================================================================================================================================

\lstset{
language=Python,   % R code
literate=
{á}{{\'a}}1
{à}{{\`a}}1
{ã}{{\~a}}1
{é}{{\'e}}1
{è}{{\`e}}1
{ê}{{\^e}}1
{í}{{\'i}}1
{ó}{{\'o}}1
{õ}{{\~o}}1
{ú}{{\'u}}1
{ü}{{\"u}}1
{ç}{{\c{c}}}1
{~}{{ }}1
}


\definecolor{codegreen}{rgb}{0,0.6,0}
\definecolor{codegray}{rgb}{0.5,0.5,0.5}
\definecolor{codepurple}{rgb}{0.58,0,0.82}
\definecolor{backcolour}{rgb}{0.95,0.95,0.92}

\lstdefinestyle{mystyle}{
    backgroundcolor=\color{backcolour},   
    commentstyle=\color{codegreen},
    keywordstyle=\color{magenta},
    numberstyle=\tiny\color{codegray},
    stringstyle=\color{codepurple},
    basicstyle=\ttfamily\footnotesize,
    breakatwhitespace=false,         
    breaklines=true,                 
    captionpos=b,                    
    keepspaces=true,                 
    numbers=left,                    
xleftmargin=2em,
framexleftmargin=2em,            
    showspaces=false,                
    showstringspaces=false,
    showtabs=false,                  
    tabsize=2,
    upquote=true
}

\lstset{style=mystyle}


\lstset{style=mystyle}
\newcommand{\imgdir}{C:/laragon/www/newmc/assets/imgsvg/}
\newcommand{\imgsvgdir}{C:/laragon/www/newmc/assets/imgsvg/}

\definecolor{mcgris}{RGB}{220, 220, 220}% ancien~; pour compatibilité
\definecolor{mcbleu}{RGB}{52, 152, 219}
\definecolor{mcvert}{RGB}{125, 194, 70}
\definecolor{mcmauve}{RGB}{154, 0, 215}
\definecolor{mcorange}{RGB}{255, 96, 0}
\definecolor{mcturquoise}{RGB}{0, 153, 153}
\definecolor{mcrouge}{RGB}{255, 0, 0}
\definecolor{mclightvert}{RGB}{205, 234, 190}

\definecolor{gris}{RGB}{220, 220, 220}
\definecolor{bleu}{RGB}{52, 152, 219}
\definecolor{vert}{RGB}{125, 194, 70}
\definecolor{mauve}{RGB}{154, 0, 215}
\definecolor{orange}{RGB}{255, 96, 0}
\definecolor{turquoise}{RGB}{0, 153, 153}
\definecolor{rouge}{RGB}{255, 0, 0}
\definecolor{lightvert}{RGB}{205, 234, 190}
\setitemize[0]{label=\color{lightvert}  $\bullet$}

\pagestyle{fancy}
\renewcommand{\headrulewidth}{0.2pt}
\fancyhead[L]{maths-cours.fr}
\fancyhead[R]{\thepage}
\renewcommand{\footrulewidth}{0.2pt}
\fancyfoot[C]{}

\newcolumntype{C}{>{\centering\arraybackslash}X}
\newcolumntype{s}{>{\hsize=.35\hsize\arraybackslash}X}

\setlength{\parindent}{0pt}		 
\setlength{\parskip}{3mm}
\setlength{\headheight}{1cm}

\def\ebook{ebook}
\def\book{book}
\def\web{web}
\def\type{web}

\newcommand{\vect}[1]{\overrightarrow{\,\mathstrut#1\,}}

\def\Oij{$\left(\text{O}~;~\vect{\imath},~\vect{\jmath}\right)$}
\def\Oijk{$\left(\text{O}~;~\vect{\imath},~\vect{\jmath},~\vect{k}\right)$}
\def\Ouv{$\left(\text{O}~;~\vect{u},~\vect{v}\right)$}

\hypersetup{breaklinks=true, colorlinks = true, linkcolor = OliveGreen, urlcolor = OliveGreen, citecolor = OliveGreen, pdfauthor={Didier BONNEL - https://www.maths-cours.fr} } % supprime les bordures autour des liens

\renewcommand{\arg}[0]{\text{arg}}

\everymath{\displaystyle}

%================================================================================================================================
%
% Macros - Commandes
%
%================================================================================================================================

\newcommand\meta[2]{    			% Utilisé pour créer le post HTML.
	\def\titre{titre}
	\def\url{url}
	\def\arg{#1}
	\ifx\titre\arg
		\newcommand\maintitle{#2}
		\fancyhead[L]{#2}
		{\Large\sffamily \MakeUppercase{#2}}
		\vspace{1mm}\textcolor{mcvert}{\hrule}
	\fi 
	\ifx\url\arg
		\fancyfoot[L]{\href{https://www.maths-cours.fr#2}{\black \footnotesize{https://www.maths-cours.fr#2}}}
	\fi 
}


\newcommand\TitreC[1]{    		% Titre centré
     \needspace{3\baselineskip}
     \begin{center}\textbf{#1}\end{center}
}

\newcommand\newpar{    		% paragraphe
     \par
}

\newcommand\nosp {    		% commande vide (pas d'espace)
}
\newcommand{\id}[1]{} %ignore

\newcommand\boite[2]{				% Boite simple sans titre
	\vspace{5mm}
	\setlength{\fboxrule}{0.2mm}
	\setlength{\fboxsep}{5mm}	
	\fcolorbox{#1}{#1!3}{\makebox[\linewidth-2\fboxrule-2\fboxsep]{
  		\begin{minipage}[t]{\linewidth-2\fboxrule-4\fboxsep}\setlength{\parskip}{3mm}
  			 #2
  		\end{minipage}
	}}
	\vspace{5mm}
}

\newcommand\CBox[4]{				% Boites
	\vspace{5mm}
	\setlength{\fboxrule}{0.2mm}
	\setlength{\fboxsep}{5mm}
	
	\fcolorbox{#1}{#1!3}{\makebox[\linewidth-2\fboxrule-2\fboxsep]{
		\begin{minipage}[t]{1cm}\setlength{\parskip}{3mm}
	  		\textcolor{#1}{\LARGE{#2}}    
 	 	\end{minipage}  
  		\begin{minipage}[t]{\linewidth-2\fboxrule-4\fboxsep}\setlength{\parskip}{3mm}
			\raisebox{1.2mm}{\normalsize\sffamily{\textcolor{#1}{#3}}}						
  			 #4
  		\end{minipage}
	}}
	\vspace{5mm}
}

\newcommand\cadre[3]{				% Boites convertible html
	\par
	\vspace{2mm}
	\setlength{\fboxrule}{0.1mm}
	\setlength{\fboxsep}{5mm}
	\fcolorbox{#1}{white}{\makebox[\linewidth-2\fboxrule-2\fboxsep]{
  		\begin{minipage}[t]{\linewidth-2\fboxrule-4\fboxsep}\setlength{\parskip}{3mm}
			\raisebox{-2.5mm}{\sffamily \small{\textcolor{#1}{\MakeUppercase{#2}}}}		
			\par		
  			 #3
 	 		\end{minipage}
	}}
		\vspace{2mm}
	\par
}

\newcommand\bloc[3]{				% Boites convertible html sans bordure
     \needspace{2\baselineskip}
     {\sffamily \small{\textcolor{#1}{\MakeUppercase{#2}}}}    
		\par		
  			 #3
		\par
}

\newcommand\CHelp[1]{
     \CBox{Plum}{\faInfoCircle}{À RETENIR}{#1}
}

\newcommand\CUp[1]{
     \CBox{NavyBlue}{\faThumbsOUp}{EN PRATIQUE}{#1}
}

\newcommand\CInfo[1]{
     \CBox{Sepia}{\faArrowCircleRight}{REMARQUE}{#1}
}

\newcommand\CRedac[1]{
     \CBox{PineGreen}{\faEdit}{BIEN R\'EDIGER}{#1}
}

\newcommand\CError[1]{
     \CBox{Red}{\faExclamationTriangle}{ATTENTION}{#1}
}

\newcommand\TitreExo[2]{
\needspace{4\baselineskip}
 {\sffamily\large EXERCICE #1\ (\emph{#2 points})}
\vspace{5mm}
}

\newcommand\img[2]{
          \includegraphics[width=#2\paperwidth]{\imgdir#1}
}

\newcommand\imgsvg[2]{
       \begin{center}   \includegraphics[width=#2\paperwidth]{\imgsvgdir#1} \end{center}
}


\newcommand\Lien[2]{
     \href{#1}{#2 \tiny \faExternalLink}
}
\newcommand\mcLien[2]{
     \href{https~://www.maths-cours.fr/#1}{#2 \tiny \faExternalLink}
}

\newcommand{\euro}{\eurologo{}}

%================================================================================================================================
%
% Macros - Environement
%
%================================================================================================================================

\newenvironment{tex}{ %
}
{%
}

\newenvironment{indente}{ %
	\setlength\parindent{10mm}
}

{
	\setlength\parindent{0mm}
}

\newenvironment{corrige}{%
     \needspace{3\baselineskip}
     \medskip
     \textbf{\textsc{Corrigé}}
     \medskip
}
{
}

\newenvironment{extern}{%
     \begin{center}
     }
     {
     \end{center}
}

\NewEnviron{code}{%
	\par
     \boite{gray}{\texttt{%
     \BODY
     }}
     \par
}

\newenvironment{vbloc}{% boite sans cadre empeche saut de page
     \begin{minipage}[t]{\linewidth}
     }
     {
     \end{minipage}
}
\NewEnviron{h2}{%
    \needspace{3\baselineskip}
    \vspace{0.6cm}
	\noindent \MakeUppercase{\sffamily \large \BODY}
	\vspace{1mm}\textcolor{mcgris}{\hrule}\vspace{0.4cm}
	\par
}{}

\NewEnviron{h3}{%
    \needspace{3\baselineskip}
	\vspace{5mm}
	\textsc{\BODY}
	\par
}

\NewEnviron{margeneg}{ %
\begin{addmargin}[-1cm]{0cm}
\BODY
\end{addmargin}
}

\NewEnviron{html}{%
}

\begin{document}
\meta{url}{/exercices/fonctions-bac-s-amerique-du-nord-2018/}
\meta{pid}{7931}
\meta{titre}{Fonctions - Bac S Amérique du Nord  2018}
\meta{type}{exercices}
%
\begin{h2}Exercice 2 (4 points)\end{h2}
\textbf{Commun à  tous les candidats}
\medskip
Lors d'une expérience en laboratoire, on lance un projectile dans un milieu fluide. L'objectif est de déterminer pour quel angle de tir
$\theta$ par rapport à l'horizontale la hauteur du projectile ne dépasse
pas $1,6$ mètre.
\par
Comme le projectile ne se déplace pas dans l'air mais dans un
fluide, le modèle parabolique usuel n'est pas adopté.
\par
On modélise ici le projectile par un point qui se déplace, dans un
plan vertical, sur la courbe représentative de la fonction $f$ définie
sur l'intervalle [0~:~1[ par~:
\par
\[f(x) = bx + 2\ln (1- x)\]
\par
où $b$ est un paramètre réel supérieur ou égal à $2$, $x$ est l'abscisse
du projectile, $f(x)$ son ordonnée, toutes les deux exprimées en mètres.
\begin{center}
     \begin{extern}%width="300"
          \psset{unit=4cm,comma=true}
          \begin{pspicture*}(-0.15,-0.15)(1.1,1.8)
               \psgrid[gridlabels=0pt,subgriddiv=10,gridwidth=0.15pt,subgridwidth=0.15pt,gridcolor=gray!20,subgridcolor=gray](0,0)(1.1,1.8)
               \psaxes[linewidth=0.75pt,Dx=0.5,Dy=0.5,labelFontSize=\scriptstyle]{->}(0,0)(0,0)(1.1,1.8)
               \psplot[plotpoints=3000,linewidth=0.75pt,linecolor=red]{0}{0.932}{5.69 x mul 1 x sub ln 2 mul add}
               \psline[linewidth=0.75pt,linecolor=blue](0.4,1.5)
               \psarc[linewidth=0.75pt,linecolor=blue](0,0){0.15}{0}{72}
          \end{pspicture*}
     \end{extern}
\end{center}
\medskip
\begin{enumerate}
     \item La fonction $f$ est dérivable sur l'intervalle [0~:~1[. On note $f'$ sa fonction dérivée.
     \par
     On admet que la fonction $f$ possède un maximum sur l'intervalle [0~:~1[ et que, pour tout réel
     $x$ de l'intervalle [0~:~1[~:
     \par
     \[f'(x) = \dfrac{- bx + b - 2}{1 - x}.\]
     \par
     Montrer que le maximum de la fonction $f$ est égal à $b - 2 + 2\ln \left(\dfrac{2}{b}\right)$.
     \item  Déterminer pour quelles valeurs du paramètre $b$ la hauteur maximale du projectile ne dépasse
     pas $1,6$~mètre.
     \item  Dans cette question, on choisit $b = 5,69$.
     \par
     L'angle de tir $\theta$ correspond à l'angle entre l'axe des abscisses et la tangente à la courbe de la
     fonction $f$ au point d'abscisse $0$ comme indiqué sur le schéma donné ci-dessus.
     \par
     Déterminer une valeur approchée au dixième de degré près de l'angle $\theta$.
\end{enumerate}

\end{document}
µ
\documentclass[a4paper]{article}

%================================================================================================================================
%
% Packages
%
%================================================================================================================================

\usepackage[T1]{fontenc} 	% pour caractères accentués
\usepackage[utf8]{inputenc}  % encodage utf8
\usepackage[french]{babel}	% langue : français
\usepackage{fourier}			% caractères plus lisibles
\usepackage[dvipsnames]{xcolor} % couleurs
\usepackage{fancyhdr}		% réglage header footer
\usepackage{needspace}		% empêcher sauts de page mal placés
\usepackage{graphicx}		% pour inclure des graphiques
\usepackage{enumitem,cprotect}		% personnalise les listes d'items (nécessaire pour ol, al ...)
\usepackage{hyperref}		% Liens hypertexte
\usepackage{pstricks,pst-all,pst-node,pstricks-add,pst-math,pst-plot,pst-tree,pst-eucl} % pstricks
\usepackage[a4paper,includeheadfoot,top=2cm,left=3cm, bottom=2cm,right=3cm]{geometry} % marges etc.
\usepackage{comment}			% commentaires multilignes
\usepackage{amsmath,environ} % maths (matrices, etc.)
\usepackage{amssymb,makeidx}
\usepackage{bm}				% bold maths
\usepackage{tabularx}		% tableaux
\usepackage{colortbl}		% tableaux en couleur
\usepackage{fontawesome}		% Fontawesome
\usepackage{environ}			% environment with command
\usepackage{fp}				% calculs pour ps-tricks
\usepackage{multido}			% pour ps tricks
\usepackage[np]{numprint}	% formattage nombre
\usepackage{tikz,tkz-tab} 			% package principal TikZ
\usepackage{pgfplots}   % axes
\usepackage{mathrsfs}    % cursives
\usepackage{calc}			% calcul taille boites
\usepackage[scaled=0.875]{helvet} % font sans serif
\usepackage{svg} % svg
\usepackage{scrextend} % local margin
\usepackage{scratch} %scratch
\usepackage{multicol} % colonnes
%\usepackage{infix-RPN,pst-func} % formule en notation polanaise inversée
\usepackage{listings}

%================================================================================================================================
%
% Réglages de base
%
%================================================================================================================================

\lstset{
language=Python,   % R code
literate=
{á}{{\'a}}1
{à}{{\`a}}1
{ã}{{\~a}}1
{é}{{\'e}}1
{è}{{\`e}}1
{ê}{{\^e}}1
{í}{{\'i}}1
{ó}{{\'o}}1
{õ}{{\~o}}1
{ú}{{\'u}}1
{ü}{{\"u}}1
{ç}{{\c{c}}}1
{~}{{ }}1
}


\definecolor{codegreen}{rgb}{0,0.6,0}
\definecolor{codegray}{rgb}{0.5,0.5,0.5}
\definecolor{codepurple}{rgb}{0.58,0,0.82}
\definecolor{backcolour}{rgb}{0.95,0.95,0.92}

\lstdefinestyle{mystyle}{
    backgroundcolor=\color{backcolour},   
    commentstyle=\color{codegreen},
    keywordstyle=\color{magenta},
    numberstyle=\tiny\color{codegray},
    stringstyle=\color{codepurple},
    basicstyle=\ttfamily\footnotesize,
    breakatwhitespace=false,         
    breaklines=true,                 
    captionpos=b,                    
    keepspaces=true,                 
    numbers=left,                    
xleftmargin=2em,
framexleftmargin=2em,            
    showspaces=false,                
    showstringspaces=false,
    showtabs=false,                  
    tabsize=2,
    upquote=true
}

\lstset{style=mystyle}


\lstset{style=mystyle}
\newcommand{\imgdir}{C:/laragon/www/newmc/assets/imgsvg/}
\newcommand{\imgsvgdir}{C:/laragon/www/newmc/assets/imgsvg/}

\definecolor{mcgris}{RGB}{220, 220, 220}% ancien~; pour compatibilité
\definecolor{mcbleu}{RGB}{52, 152, 219}
\definecolor{mcvert}{RGB}{125, 194, 70}
\definecolor{mcmauve}{RGB}{154, 0, 215}
\definecolor{mcorange}{RGB}{255, 96, 0}
\definecolor{mcturquoise}{RGB}{0, 153, 153}
\definecolor{mcrouge}{RGB}{255, 0, 0}
\definecolor{mclightvert}{RGB}{205, 234, 190}

\definecolor{gris}{RGB}{220, 220, 220}
\definecolor{bleu}{RGB}{52, 152, 219}
\definecolor{vert}{RGB}{125, 194, 70}
\definecolor{mauve}{RGB}{154, 0, 215}
\definecolor{orange}{RGB}{255, 96, 0}
\definecolor{turquoise}{RGB}{0, 153, 153}
\definecolor{rouge}{RGB}{255, 0, 0}
\definecolor{lightvert}{RGB}{205, 234, 190}
\setitemize[0]{label=\color{lightvert}  $\bullet$}

\pagestyle{fancy}
\renewcommand{\headrulewidth}{0.2pt}
\fancyhead[L]{maths-cours.fr}
\fancyhead[R]{\thepage}
\renewcommand{\footrulewidth}{0.2pt}
\fancyfoot[C]{}

\newcolumntype{C}{>{\centering\arraybackslash}X}
\newcolumntype{s}{>{\hsize=.35\hsize\arraybackslash}X}

\setlength{\parindent}{0pt}		 
\setlength{\parskip}{3mm}
\setlength{\headheight}{1cm}

\def\ebook{ebook}
\def\book{book}
\def\web{web}
\def\type{web}

\newcommand{\vect}[1]{\overrightarrow{\,\mathstrut#1\,}}

\def\Oij{$\left(\text{O}~;~\vect{\imath},~\vect{\jmath}\right)$}
\def\Oijk{$\left(\text{O}~;~\vect{\imath},~\vect{\jmath},~\vect{k}\right)$}
\def\Ouv{$\left(\text{O}~;~\vect{u},~\vect{v}\right)$}

\hypersetup{breaklinks=true, colorlinks = true, linkcolor = OliveGreen, urlcolor = OliveGreen, citecolor = OliveGreen, pdfauthor={Didier BONNEL - https://www.maths-cours.fr} } % supprime les bordures autour des liens

\renewcommand{\arg}[0]{\text{arg}}

\everymath{\displaystyle}

%================================================================================================================================
%
% Macros - Commandes
%
%================================================================================================================================

\newcommand\meta[2]{    			% Utilisé pour créer le post HTML.
	\def\titre{titre}
	\def\url{url}
	\def\arg{#1}
	\ifx\titre\arg
		\newcommand\maintitle{#2}
		\fancyhead[L]{#2}
		{\Large\sffamily \MakeUppercase{#2}}
		\vspace{1mm}\textcolor{mcvert}{\hrule}
	\fi 
	\ifx\url\arg
		\fancyfoot[L]{\href{https://www.maths-cours.fr#2}{\black \footnotesize{https://www.maths-cours.fr#2}}}
	\fi 
}


\newcommand\TitreC[1]{    		% Titre centré
     \needspace{3\baselineskip}
     \begin{center}\textbf{#1}\end{center}
}

\newcommand\newpar{    		% paragraphe
     \par
}

\newcommand\nosp {    		% commande vide (pas d'espace)
}
\newcommand{\id}[1]{} %ignore

\newcommand\boite[2]{				% Boite simple sans titre
	\vspace{5mm}
	\setlength{\fboxrule}{0.2mm}
	\setlength{\fboxsep}{5mm}	
	\fcolorbox{#1}{#1!3}{\makebox[\linewidth-2\fboxrule-2\fboxsep]{
  		\begin{minipage}[t]{\linewidth-2\fboxrule-4\fboxsep}\setlength{\parskip}{3mm}
  			 #2
  		\end{minipage}
	}}
	\vspace{5mm}
}

\newcommand\CBox[4]{				% Boites
	\vspace{5mm}
	\setlength{\fboxrule}{0.2mm}
	\setlength{\fboxsep}{5mm}
	
	\fcolorbox{#1}{#1!3}{\makebox[\linewidth-2\fboxrule-2\fboxsep]{
		\begin{minipage}[t]{1cm}\setlength{\parskip}{3mm}
	  		\textcolor{#1}{\LARGE{#2}}    
 	 	\end{minipage}  
  		\begin{minipage}[t]{\linewidth-2\fboxrule-4\fboxsep}\setlength{\parskip}{3mm}
			\raisebox{1.2mm}{\normalsize\sffamily{\textcolor{#1}{#3}}}						
  			 #4
  		\end{minipage}
	}}
	\vspace{5mm}
}

\newcommand\cadre[3]{				% Boites convertible html
	\par
	\vspace{2mm}
	\setlength{\fboxrule}{0.1mm}
	\setlength{\fboxsep}{5mm}
	\fcolorbox{#1}{white}{\makebox[\linewidth-2\fboxrule-2\fboxsep]{
  		\begin{minipage}[t]{\linewidth-2\fboxrule-4\fboxsep}\setlength{\parskip}{3mm}
			\raisebox{-2.5mm}{\sffamily \small{\textcolor{#1}{\MakeUppercase{#2}}}}		
			\par		
  			 #3
 	 		\end{minipage}
	}}
		\vspace{2mm}
	\par
}

\newcommand\bloc[3]{				% Boites convertible html sans bordure
     \needspace{2\baselineskip}
     {\sffamily \small{\textcolor{#1}{\MakeUppercase{#2}}}}    
		\par		
  			 #3
		\par
}

\newcommand\CHelp[1]{
     \CBox{Plum}{\faInfoCircle}{À RETENIR}{#1}
}

\newcommand\CUp[1]{
     \CBox{NavyBlue}{\faThumbsOUp}{EN PRATIQUE}{#1}
}

\newcommand\CInfo[1]{
     \CBox{Sepia}{\faArrowCircleRight}{REMARQUE}{#1}
}

\newcommand\CRedac[1]{
     \CBox{PineGreen}{\faEdit}{BIEN R\'EDIGER}{#1}
}

\newcommand\CError[1]{
     \CBox{Red}{\faExclamationTriangle}{ATTENTION}{#1}
}

\newcommand\TitreExo[2]{
\needspace{4\baselineskip}
 {\sffamily\large EXERCICE #1\ (\emph{#2 points})}
\vspace{5mm}
}

\newcommand\img[2]{
          \includegraphics[width=#2\paperwidth]{\imgdir#1}
}

\newcommand\imgsvg[2]{
       \begin{center}   \includegraphics[width=#2\paperwidth]{\imgsvgdir#1} \end{center}
}


\newcommand\Lien[2]{
     \href{#1}{#2 \tiny \faExternalLink}
}
\newcommand\mcLien[2]{
     \href{https~://www.maths-cours.fr/#1}{#2 \tiny \faExternalLink}
}

\newcommand{\euro}{\eurologo{}}

%================================================================================================================================
%
% Macros - Environement
%
%================================================================================================================================

\newenvironment{tex}{ %
}
{%
}

\newenvironment{indente}{ %
	\setlength\parindent{10mm}
}

{
	\setlength\parindent{0mm}
}

\newenvironment{corrige}{%
     \needspace{3\baselineskip}
     \medskip
     \textbf{\textsc{Corrigé}}
     \medskip
}
{
}

\newenvironment{extern}{%
     \begin{center}
     }
     {
     \end{center}
}

\NewEnviron{code}{%
	\par
     \boite{gray}{\texttt{%
     \BODY
     }}
     \par
}

\newenvironment{vbloc}{% boite sans cadre empeche saut de page
     \begin{minipage}[t]{\linewidth}
     }
     {
     \end{minipage}
}
\NewEnviron{h2}{%
    \needspace{3\baselineskip}
    \vspace{0.6cm}
	\noindent \MakeUppercase{\sffamily \large \BODY}
	\vspace{1mm}\textcolor{mcgris}{\hrule}\vspace{0.4cm}
	\par
}{}

\NewEnviron{h3}{%
    \needspace{3\baselineskip}
	\vspace{5mm}
	\textsc{\BODY}
	\par
}

\NewEnviron{margeneg}{ %
\begin{addmargin}[-1cm]{0cm}
\BODY
\end{addmargin}
}

\NewEnviron{html}{%
}

\begin{document}
\meta{url}{/exercices/geometrie-dans-lespace-bac-s-amerique-du-nord-2018/}
\meta{pid}{7949}
\meta{titre}{Géométrie dans l'espace - Bac S Amérique du Nord  2018}
\meta{type}{exercices}
%
\begin{h2}Exercice 3 (5 points)\end{h2}
\textbf{Commun à  tous les candidats}
\medskip
On se place dans l'espace muni d'un repère orthonormé dont l'origine est le point A.
\par
On considère les points B(10~:~-8~:~2), C(-1~:~-8~:~5) et D(14~:~4~:~8).
\medskip
\begin{enumerate}
     \item
     \begin{enumerate}[label=\alph*.]
          \item Déterminer un système d'équations paramétriques de chacune des droites (AB) et (CD).
          \item Vérifier que les droites (AB) et (CD) ne sont pas coplanaires.
     \end{enumerate}
     \item On considère le point I de la droite (AB) d'abscisse 5 et le point J de la droite (CD) d'abscisse
     4.
     \begin{enumerate}[label=\alph*.]
          \item Déterminer les coordonnées des points I et J et en déduire la distance IJ.
          \item Démontrer que la droite (IJ) est perpendiculaire aux droites (AB) et (CD).
          \par
          La droite (IJ) est appelée perpendiculaire commune aux droites (AB) et (CD).
     \end{enumerate}
     \item Cette question a pour but de vérifier que la distance IJ est la distance minimale entre les
     droites (AB) et (CD).
     \par
     Sur le schéma ci -dessous on a représenté les droites (AB) et (CD), les points I et J, et la droite
     $\Delta$ parallèle à la droite (CD) passant par I.
     \par
     On considère un point M de la droite (AB) distinct du point I.
     \par
     On considère un point M' de la droite (CD) distinct du point J.
     \begin{center}
          \begin{extern}
               \psset{unit=1cm}
               \begin{pspicture*}(12.3,7.8)
                    \pspolygon[linecolor=blue,fillcolor=blue,fillstyle=solid,opacity=0.15](0.7,1.2)(5.8,0.5)(11.5,3.8)(6.4,4.5)
                    %\psgrid
                    \psline(0,2)(12.3,2)%(AB)
                    \psline(1.5,0)(12.3,6)%$\Delta$
                    \psline(0,2.8)(10.8,8.8)%(CD)
                    \psline(5.1,2)(4.3,5.22)%IJ
                    \psline(7.3,3.22)(6.5,6.45)%PM'
                    \psline(7.3,3.22)(6.5,2)(6.5,6.45)%PMM'
                    \uput[u](4.3,5.22){J}\uput[dr](5.1,2){I}
                    \uput[d](6.5,2){M}\uput[dr](7.3,3.22){P}
                    \uput[u](6.5,6.45){M'}
                    \uput[d](10.5,2){(AB)}\uput[ul](2.3,4){(CD)}
                    \uput[u](10,4.7){$\Delta$}
                    \psline(4.38,5)(4.52,5.08)(4.47,5.28)
                    \psline(5.05,2.2)(4.85,2.2)(4.9,2.)
               \end{pspicture*}
          \end{extern}
     \end{center}
     \medskip
     \begin{enumerate}[label=\alph*.]
          \item Justifier que la parallèle à la droite (IJ) passant par le point M' coupe la droite $\Delta$ en un point que l'on notera P.
          \item Démontrer que le triangle MPM' est rectangle en P.
          \item Justifier que MM' > IJ et conclure.
     \end{enumerate}
\end{enumerate}

\end{document}
µ
\documentclass[a4paper]{article}

%================================================================================================================================
%
% Packages
%
%================================================================================================================================

\usepackage[T1]{fontenc} 	% pour caractères accentués
\usepackage[utf8]{inputenc}  % encodage utf8
\usepackage[french]{babel}	% langue : français
\usepackage{fourier}			% caractères plus lisibles
\usepackage[dvipsnames]{xcolor} % couleurs
\usepackage{fancyhdr}		% réglage header footer
\usepackage{needspace}		% empêcher sauts de page mal placés
\usepackage{graphicx}		% pour inclure des graphiques
\usepackage{enumitem,cprotect}		% personnalise les listes d'items (nécessaire pour ol, al ...)
\usepackage{hyperref}		% Liens hypertexte
\usepackage{pstricks,pst-all,pst-node,pstricks-add,pst-math,pst-plot,pst-tree,pst-eucl} % pstricks
\usepackage[a4paper,includeheadfoot,top=2cm,left=3cm, bottom=2cm,right=3cm]{geometry} % marges etc.
\usepackage{comment}			% commentaires multilignes
\usepackage{amsmath,environ} % maths (matrices, etc.)
\usepackage{amssymb,makeidx}
\usepackage{bm}				% bold maths
\usepackage{tabularx}		% tableaux
\usepackage{colortbl}		% tableaux en couleur
\usepackage{fontawesome}		% Fontawesome
\usepackage{environ}			% environment with command
\usepackage{fp}				% calculs pour ps-tricks
\usepackage{multido}			% pour ps tricks
\usepackage[np]{numprint}	% formattage nombre
\usepackage{tikz,tkz-tab} 			% package principal TikZ
\usepackage{pgfplots}   % axes
\usepackage{mathrsfs}    % cursives
\usepackage{calc}			% calcul taille boites
\usepackage[scaled=0.875]{helvet} % font sans serif
\usepackage{svg} % svg
\usepackage{scrextend} % local margin
\usepackage{scratch} %scratch
\usepackage{multicol} % colonnes
%\usepackage{infix-RPN,pst-func} % formule en notation polanaise inversée
\usepackage{listings}

%================================================================================================================================
%
% Réglages de base
%
%================================================================================================================================

\lstset{
language=Python,   % R code
literate=
{á}{{\'a}}1
{à}{{\`a}}1
{ã}{{\~a}}1
{é}{{\'e}}1
{è}{{\`e}}1
{ê}{{\^e}}1
{í}{{\'i}}1
{ó}{{\'o}}1
{õ}{{\~o}}1
{ú}{{\'u}}1
{ü}{{\"u}}1
{ç}{{\c{c}}}1
{~}{{ }}1
}


\definecolor{codegreen}{rgb}{0,0.6,0}
\definecolor{codegray}{rgb}{0.5,0.5,0.5}
\definecolor{codepurple}{rgb}{0.58,0,0.82}
\definecolor{backcolour}{rgb}{0.95,0.95,0.92}

\lstdefinestyle{mystyle}{
    backgroundcolor=\color{backcolour},   
    commentstyle=\color{codegreen},
    keywordstyle=\color{magenta},
    numberstyle=\tiny\color{codegray},
    stringstyle=\color{codepurple},
    basicstyle=\ttfamily\footnotesize,
    breakatwhitespace=false,         
    breaklines=true,                 
    captionpos=b,                    
    keepspaces=true,                 
    numbers=left,                    
xleftmargin=2em,
framexleftmargin=2em,            
    showspaces=false,                
    showstringspaces=false,
    showtabs=false,                  
    tabsize=2,
    upquote=true
}

\lstset{style=mystyle}


\lstset{style=mystyle}
\newcommand{\imgdir}{C:/laragon/www/newmc/assets/imgsvg/}
\newcommand{\imgsvgdir}{C:/laragon/www/newmc/assets/imgsvg/}

\definecolor{mcgris}{RGB}{220, 220, 220}% ancien~; pour compatibilité
\definecolor{mcbleu}{RGB}{52, 152, 219}
\definecolor{mcvert}{RGB}{125, 194, 70}
\definecolor{mcmauve}{RGB}{154, 0, 215}
\definecolor{mcorange}{RGB}{255, 96, 0}
\definecolor{mcturquoise}{RGB}{0, 153, 153}
\definecolor{mcrouge}{RGB}{255, 0, 0}
\definecolor{mclightvert}{RGB}{205, 234, 190}

\definecolor{gris}{RGB}{220, 220, 220}
\definecolor{bleu}{RGB}{52, 152, 219}
\definecolor{vert}{RGB}{125, 194, 70}
\definecolor{mauve}{RGB}{154, 0, 215}
\definecolor{orange}{RGB}{255, 96, 0}
\definecolor{turquoise}{RGB}{0, 153, 153}
\definecolor{rouge}{RGB}{255, 0, 0}
\definecolor{lightvert}{RGB}{205, 234, 190}
\setitemize[0]{label=\color{lightvert}  $\bullet$}

\pagestyle{fancy}
\renewcommand{\headrulewidth}{0.2pt}
\fancyhead[L]{maths-cours.fr}
\fancyhead[R]{\thepage}
\renewcommand{\footrulewidth}{0.2pt}
\fancyfoot[C]{}

\newcolumntype{C}{>{\centering\arraybackslash}X}
\newcolumntype{s}{>{\hsize=.35\hsize\arraybackslash}X}

\setlength{\parindent}{0pt}		 
\setlength{\parskip}{3mm}
\setlength{\headheight}{1cm}

\def\ebook{ebook}
\def\book{book}
\def\web{web}
\def\type{web}

\newcommand{\vect}[1]{\overrightarrow{\,\mathstrut#1\,}}

\def\Oij{$\left(\text{O}~;~\vect{\imath},~\vect{\jmath}\right)$}
\def\Oijk{$\left(\text{O}~;~\vect{\imath},~\vect{\jmath},~\vect{k}\right)$}
\def\Ouv{$\left(\text{O}~;~\vect{u},~\vect{v}\right)$}

\hypersetup{breaklinks=true, colorlinks = true, linkcolor = OliveGreen, urlcolor = OliveGreen, citecolor = OliveGreen, pdfauthor={Didier BONNEL - https://www.maths-cours.fr} } % supprime les bordures autour des liens

\renewcommand{\arg}[0]{\text{arg}}

\everymath{\displaystyle}

%================================================================================================================================
%
% Macros - Commandes
%
%================================================================================================================================

\newcommand\meta[2]{    			% Utilisé pour créer le post HTML.
	\def\titre{titre}
	\def\url{url}
	\def\arg{#1}
	\ifx\titre\arg
		\newcommand\maintitle{#2}
		\fancyhead[L]{#2}
		{\Large\sffamily \MakeUppercase{#2}}
		\vspace{1mm}\textcolor{mcvert}{\hrule}
	\fi 
	\ifx\url\arg
		\fancyfoot[L]{\href{https://www.maths-cours.fr#2}{\black \footnotesize{https://www.maths-cours.fr#2}}}
	\fi 
}


\newcommand\TitreC[1]{    		% Titre centré
     \needspace{3\baselineskip}
     \begin{center}\textbf{#1}\end{center}
}

\newcommand\newpar{    		% paragraphe
     \par
}

\newcommand\nosp {    		% commande vide (pas d'espace)
}
\newcommand{\id}[1]{} %ignore

\newcommand\boite[2]{				% Boite simple sans titre
	\vspace{5mm}
	\setlength{\fboxrule}{0.2mm}
	\setlength{\fboxsep}{5mm}	
	\fcolorbox{#1}{#1!3}{\makebox[\linewidth-2\fboxrule-2\fboxsep]{
  		\begin{minipage}[t]{\linewidth-2\fboxrule-4\fboxsep}\setlength{\parskip}{3mm}
  			 #2
  		\end{minipage}
	}}
	\vspace{5mm}
}

\newcommand\CBox[4]{				% Boites
	\vspace{5mm}
	\setlength{\fboxrule}{0.2mm}
	\setlength{\fboxsep}{5mm}
	
	\fcolorbox{#1}{#1!3}{\makebox[\linewidth-2\fboxrule-2\fboxsep]{
		\begin{minipage}[t]{1cm}\setlength{\parskip}{3mm}
	  		\textcolor{#1}{\LARGE{#2}}    
 	 	\end{minipage}  
  		\begin{minipage}[t]{\linewidth-2\fboxrule-4\fboxsep}\setlength{\parskip}{3mm}
			\raisebox{1.2mm}{\normalsize\sffamily{\textcolor{#1}{#3}}}						
  			 #4
  		\end{minipage}
	}}
	\vspace{5mm}
}

\newcommand\cadre[3]{				% Boites convertible html
	\par
	\vspace{2mm}
	\setlength{\fboxrule}{0.1mm}
	\setlength{\fboxsep}{5mm}
	\fcolorbox{#1}{white}{\makebox[\linewidth-2\fboxrule-2\fboxsep]{
  		\begin{minipage}[t]{\linewidth-2\fboxrule-4\fboxsep}\setlength{\parskip}{3mm}
			\raisebox{-2.5mm}{\sffamily \small{\textcolor{#1}{\MakeUppercase{#2}}}}		
			\par		
  			 #3
 	 		\end{minipage}
	}}
		\vspace{2mm}
	\par
}

\newcommand\bloc[3]{				% Boites convertible html sans bordure
     \needspace{2\baselineskip}
     {\sffamily \small{\textcolor{#1}{\MakeUppercase{#2}}}}    
		\par		
  			 #3
		\par
}

\newcommand\CHelp[1]{
     \CBox{Plum}{\faInfoCircle}{À RETENIR}{#1}
}

\newcommand\CUp[1]{
     \CBox{NavyBlue}{\faThumbsOUp}{EN PRATIQUE}{#1}
}

\newcommand\CInfo[1]{
     \CBox{Sepia}{\faArrowCircleRight}{REMARQUE}{#1}
}

\newcommand\CRedac[1]{
     \CBox{PineGreen}{\faEdit}{BIEN R\'EDIGER}{#1}
}

\newcommand\CError[1]{
     \CBox{Red}{\faExclamationTriangle}{ATTENTION}{#1}
}

\newcommand\TitreExo[2]{
\needspace{4\baselineskip}
 {\sffamily\large EXERCICE #1\ (\emph{#2 points})}
\vspace{5mm}
}

\newcommand\img[2]{
          \includegraphics[width=#2\paperwidth]{\imgdir#1}
}

\newcommand\imgsvg[2]{
       \begin{center}   \includegraphics[width=#2\paperwidth]{\imgsvgdir#1} \end{center}
}


\newcommand\Lien[2]{
     \href{#1}{#2 \tiny \faExternalLink}
}
\newcommand\mcLien[2]{
     \href{https~://www.maths-cours.fr/#1}{#2 \tiny \faExternalLink}
}

\newcommand{\euro}{\eurologo{}}

%================================================================================================================================
%
% Macros - Environement
%
%================================================================================================================================

\newenvironment{tex}{ %
}
{%
}

\newenvironment{indente}{ %
	\setlength\parindent{10mm}
}

{
	\setlength\parindent{0mm}
}

\newenvironment{corrige}{%
     \needspace{3\baselineskip}
     \medskip
     \textbf{\textsc{Corrigé}}
     \medskip
}
{
}

\newenvironment{extern}{%
     \begin{center}
     }
     {
     \end{center}
}

\NewEnviron{code}{%
	\par
     \boite{gray}{\texttt{%
     \BODY
     }}
     \par
}

\newenvironment{vbloc}{% boite sans cadre empeche saut de page
     \begin{minipage}[t]{\linewidth}
     }
     {
     \end{minipage}
}
\NewEnviron{h2}{%
    \needspace{3\baselineskip}
    \vspace{0.6cm}
	\noindent \MakeUppercase{\sffamily \large \BODY}
	\vspace{1mm}\textcolor{mcgris}{\hrule}\vspace{0.4cm}
	\par
}{}

\NewEnviron{h3}{%
    \needspace{3\baselineskip}
	\vspace{5mm}
	\textsc{\BODY}
	\par
}

\NewEnviron{margeneg}{ %
\begin{addmargin}[-1cm]{0cm}
\BODY
\end{addmargin}
}

\NewEnviron{html}{%
}

\begin{document}
\meta{url}{/exercices/suites-fonctions-bac-s-amerique-du-nord-2018/}
\meta{pid}{7974}
\meta{titre}{Suites - Fonctions - Bac S Amérique du Nord  2018}
\meta{type}{exercices}
%
\begin{h2}Exercice 4 (5 points)\end{h2}
\textbf{Candidats n'ayant pas suivi l'enseignement de spécialité}
\medskip
\textbf{Les deux graphiques donnés en annexe seront à compléter et à rendre avec la copie}
\medskip
Un scooter radio commandé se déplace en ligne droite à la vitesse constante de 1 m.s$^{-1}$. Il est poursuivi
par un chien qui se déplace à la même vitesse.
\par
On représente la situation vue de dessus dans un repère orthonormé du plan d'unité 1 mètre. L'origine de ce repère est la position initiale du chien. Le scooter est représenté par un point appartenant à la droite d'équation $x = 5$. Il se déplace sur cette droite dans le sens des ordonnées croissantes.
\par
\smallskip
\par
Dans la suite de l'exercice, on étudie deux modélisations différentes de la trajectoire du chien.
\bigskip
\begin{center}\begin{h3}Partie A - Modélisation à l'aide d'une suite \end{h3}\end{center}
\medskip
La situation est représentée par le graphique n° 1 donné en annexe.
\par
À l'instant initial, le scooter est représenté par le point $S_0$. Le chien qui le poursuit est représenté
par le point $M_0$. On considère qu'à chaque seconde, le chien s'oriente instantanément en direction
du scooter et se déplace en ligne droite sur une distance de 1 mètre.
\par
Ainsi, à l'instant initial, le chien s'oriente en direction du point $S_0$, et une seconde plus tard il se
trouve un mètre plus loin au point $M_1$. À cet instant, le scooter est au point $S_1$. Le chien s'oriente
en direction de $S_1$ et se déplace en ligne droite en parcourant 1 mètre, et ainsi de suite.
\par
On modélise alors les trajectoires du chien et du scooter par deux suites de points notées $\left(M_n\right)$ et $\left(S_n\right)$.
\par
Au bout de $n$ secondes, les coordonnées du point $S_n$ sont $(5~:~n)$. On note $\left(x_n~:~y_n\right)$ les coordonnées du point $M_n$.
\medskip
\begin{enumerate}
     \item Construire sur le graphique n° 1 donné en annexe les points $M_2$ et $M_3$.
     \item  On note $d_n$ la distance entre le chien et le scooter $n$ secondes après le début de la poursuite.
     \par
     On a donc $d_n = M_nS_n$.
     \par
     Calculer $d_0$ et $d_1$.
     \item  Justifier que le point $M_2$ a pour coordonnées $\left(1 + \dfrac{4}{\sqrt{17}}~:~\dfrac{1}{\sqrt{17}}\right)$.
     \item  On admet que, pour tout entier naturel $n$~:
     \par
     \[\left\{\begin{array}{l c l}
               x_{n+1}& = &x_n + \dfrac{5 - x_n}{d_n}\\
               y_{n+1}&=&y_n + \dfrac{n - y_n}{d_n}
     \end{array}\right.\]
     \begin{enumerate}[label=\alph*.]
          \item Le tableau ci-dessous, obtenu à l'aide d'un tableur, donne les coordonnées des points $M_n$
          et $S_n$ ainsi que la distance $d_n$ en fonction de $n$. Quelles formules doit-on écrire dans les
          cellules C5 et F5 et recopier vers le bas pour remplir les colonnes C et F~?
          \begin{center}
               \begin{extern}%width="700" class="mw500"
                    \begin{tabularx}{\linewidth}{|>{\columncolor[gray]{0.9}}c|*{6}{>{\centering \arraybackslash}X|}}\hline
                         \rowcolor[gray]{0.9}&A &B &C &D &E &F\\ \hline
                         1 &$n$& \multicolumn{2}{|c|}{$M_n$} & \multicolumn{2}{|c|}{$S_n$} &$d_n$\\ \hline
                         2 &&$x_n$& $y_n$& 5 &n&\\ \hline
                         3 &0& 0& 0& 5 &0& 5\\ \hline
                         4 &1 &1 &0 &5 &1 &\np{4,12310563}\\ \hline
                         5 &2 &\np{1,9701425} &\np{0,24253563} &5 &2 &\np{3,50267291}\\ \hline
                         6 &3 &\np{2,83515547} &\np{0,74428512} &5 &3 &\np{3,12646789}\\ \hline
                         7 &4 &\np{3,52758047} &\np{1,46577498} &5 &4 &\np{2,93092404}\\ \hline
                         \ldots&\ldots&\ldots&\ldots&\ldots&\ldots&\ldots\\ \hline
                         28 &24 &\np{4,99979751} &\np{21,2268342} &5 &24 &\np{2,7731658}\\ \hline
                         29 &25 &\np{4,99987053} &\np{22,2268342} &5 &25 &\np{2,7731658}\\ \hline
                    \end{tabularx}
               \end{extern}
          \end{center}
          \medskip
          \item On admet que la suite $\left(d_n\right)$ est strictement décroissante.
          \par
          Justifier que cette suite est convergente et conjecturer sa limite à l'aide du tableau.
     \end{enumerate}
\end{enumerate}
\bigskip
\begin{center}\begin{h3}Partie B - Modélisation à l'aide d'une fonction \end{h3}\end{center}
\par
\medskip
\par
On modélise maintenant la trajectoire du chien à l'aide de la courbe $\mathscr{F}$ de la fonction $f$ définie
pour tout réel $x$ de l'intervalle [0~:~5[ par~:
\par
\[f(x) = -2,5\ln (1 - 0, 2x) - 0,5x + 0,05x^2.\]
\par
\medskip
\par
Cela signifie que le chien se déplace sur la courbe $\mathscr{F}$ de la fonction $f$.
\par
\medskip
\begin{enumerate}
     \item Lorsque le chien se trouve au point $M$ de coordonnées $(x~:~f(x))$ de la courbe $\mathscr{F}$, où $x$  appartient à l'intervalle [0~:~5[, le scooter se trouve au point $S$, d'ordonnée notée $y_S$. Ainsi le point $S$
     a pour coordonnées $\left(5~:~y_S\right)$. La tangente à la courbe $\mathscr{F}$ au point $M$ passe par le point $S$. Cela traduit le fait que le chien s'oriente toujours en direction du scooter. On note $d(x)$ la distance $MS$ entre le chien et le scooter lorsque $M$ a pour abscisse $x$.
     \begin{enumerate}[label=\alph*.]
          \item Sur le graphique n° 2 donné en annexe, construire, sans calcul, le point $S$ donnant la position du scooter lorsque le chien se trouve au point d'abscisse 3 de la courbe $\mathscr{F}$ et lire les
          coordonnées du point $S$.
          \item On note $f'$ la fonction dérivée de la fonction $f$ sur l'intervalle [0~:~5[ et on admet que, pour tout réel $x$ de l'intervalle [0~:~5[~:
          \par
          \[f'(x) = \dfrac{x(1  - 0,1x)}{5 - x}.\]
          \par
          Déterminer par le calcul une valeur approchée au centième de l'ordonnée du point $S$ lorsque
          le chien se trouve au point d'abscisse 3 de la courbe $\mathscr{F}$.
     \end{enumerate}
     \item  On admet que $d(x) = 0,1x^2 - x + 5$ pour tout réel $x$ de l'intervalle [0~:~5[.
     \par
     Justifier qu'au cours du temps la distance $MS$ se rapproche d'une valeur limite que l'on déterminera.
\end{enumerate}
\newpage
\begin{center}
     \begin{h3}Annexe\end{h3}
     \par
     \textbf{À rendre avec la copie}
     \bigskip
     \textbf{Partie A - question 1}
     \par
     Graphique n° 1
     \par          \begin{extern}
          \psset{unit=1.4cm}
          \begin{pspicture*}(-0.2,-0.2)(5.5,4.5)
               \psgrid[gridlabels=0pt,subgriddiv=1,gridwidth=0.3pt](0,0)(6,5)
               \psaxes[linewidth=0.75pt,labelFontSize=\scriptstyle]{->}(0,0)(0,0)(5.5,4.5)
               \psline[linecolor=blue](5,0)(5,6)
               \uput[dr](0,0){$M_0$}\uput[dr](1,0){$M_1$}\uput[ur](5,0){$S_0$}
               \uput[ur](5,1){$S_1$}\uput[ur](5,2){$S_2$}\uput[ur](5,3){$S_3$}
          \end{pspicture*}
     \end{extern}
     \bigskip
     \textbf{Partie B - question 1 }
     \par
     Graphique n° 2
     \par
     \begin{extern}
          \psset{unit=1.4cm}
          \begin{pspicture*}(-0.5,-0.5)(5.5,5.5)
               \psgrid[gridlabels=0pt,subgriddiv=1,gridwidth=0.3pt](0,0)(6,6)
               \psaxes[linewidth=0.75pt,labelFontSize=\scriptstyle]{->}(0,0)(0,0)(5.5,5.5)
               \psline[linecolor=blue](5,0)(5,6)
               \psplot[plotpoints=3000,linewidth=0.75pt,linecolor=red]{0}{4.8}{0.05 x dup mul mul 0.5 x mul sub 1 0.2 x mul sub ln 2.5 mul sub}
               \psdots[dotsize=3pt,dotstyle=+](3,1.24)\uput[ul](3,1.24){$M$}\uput[l](4.5,4.5){\red $\mathcal{F}$}
          \end{pspicture*}
     \end{extern}
\end{center}

\end{document}
µ
\documentclass[a4paper]{article}

%================================================================================================================================
%
% Packages
%
%================================================================================================================================

\usepackage[T1]{fontenc} 	% pour caractères accentués
\usepackage[utf8]{inputenc}  % encodage utf8
\usepackage[french]{babel}	% langue : français
\usepackage{fourier}			% caractères plus lisibles
\usepackage[dvipsnames]{xcolor} % couleurs
\usepackage{fancyhdr}		% réglage header footer
\usepackage{needspace}		% empêcher sauts de page mal placés
\usepackage{graphicx}		% pour inclure des graphiques
\usepackage{enumitem,cprotect}		% personnalise les listes d'items (nécessaire pour ol, al ...)
\usepackage{hyperref}		% Liens hypertexte
\usepackage{pstricks,pst-all,pst-node,pstricks-add,pst-math,pst-plot,pst-tree,pst-eucl} % pstricks
\usepackage[a4paper,includeheadfoot,top=2cm,left=3cm, bottom=2cm,right=3cm]{geometry} % marges etc.
\usepackage{comment}			% commentaires multilignes
\usepackage{amsmath,environ} % maths (matrices, etc.)
\usepackage{amssymb,makeidx}
\usepackage{bm}				% bold maths
\usepackage{tabularx}		% tableaux
\usepackage{colortbl}		% tableaux en couleur
\usepackage{fontawesome}		% Fontawesome
\usepackage{environ}			% environment with command
\usepackage{fp}				% calculs pour ps-tricks
\usepackage{multido}			% pour ps tricks
\usepackage[np]{numprint}	% formattage nombre
\usepackage{tikz,tkz-tab} 			% package principal TikZ
\usepackage{pgfplots}   % axes
\usepackage{mathrsfs}    % cursives
\usepackage{calc}			% calcul taille boites
\usepackage[scaled=0.875]{helvet} % font sans serif
\usepackage{svg} % svg
\usepackage{scrextend} % local margin
\usepackage{scratch} %scratch
\usepackage{multicol} % colonnes
%\usepackage{infix-RPN,pst-func} % formule en notation polanaise inversée
\usepackage{listings}

%================================================================================================================================
%
% Réglages de base
%
%================================================================================================================================

\lstset{
language=Python,   % R code
literate=
{á}{{\'a}}1
{à}{{\`a}}1
{ã}{{\~a}}1
{é}{{\'e}}1
{è}{{\`e}}1
{ê}{{\^e}}1
{í}{{\'i}}1
{ó}{{\'o}}1
{õ}{{\~o}}1
{ú}{{\'u}}1
{ü}{{\"u}}1
{ç}{{\c{c}}}1
{~}{{ }}1
}


\definecolor{codegreen}{rgb}{0,0.6,0}
\definecolor{codegray}{rgb}{0.5,0.5,0.5}
\definecolor{codepurple}{rgb}{0.58,0,0.82}
\definecolor{backcolour}{rgb}{0.95,0.95,0.92}

\lstdefinestyle{mystyle}{
    backgroundcolor=\color{backcolour},   
    commentstyle=\color{codegreen},
    keywordstyle=\color{magenta},
    numberstyle=\tiny\color{codegray},
    stringstyle=\color{codepurple},
    basicstyle=\ttfamily\footnotesize,
    breakatwhitespace=false,         
    breaklines=true,                 
    captionpos=b,                    
    keepspaces=true,                 
    numbers=left,                    
xleftmargin=2em,
framexleftmargin=2em,            
    showspaces=false,                
    showstringspaces=false,
    showtabs=false,                  
    tabsize=2,
    upquote=true
}

\lstset{style=mystyle}


\lstset{style=mystyle}
\newcommand{\imgdir}{C:/laragon/www/newmc/assets/imgsvg/}
\newcommand{\imgsvgdir}{C:/laragon/www/newmc/assets/imgsvg/}

\definecolor{mcgris}{RGB}{220, 220, 220}% ancien~; pour compatibilité
\definecolor{mcbleu}{RGB}{52, 152, 219}
\definecolor{mcvert}{RGB}{125, 194, 70}
\definecolor{mcmauve}{RGB}{154, 0, 215}
\definecolor{mcorange}{RGB}{255, 96, 0}
\definecolor{mcturquoise}{RGB}{0, 153, 153}
\definecolor{mcrouge}{RGB}{255, 0, 0}
\definecolor{mclightvert}{RGB}{205, 234, 190}

\definecolor{gris}{RGB}{220, 220, 220}
\definecolor{bleu}{RGB}{52, 152, 219}
\definecolor{vert}{RGB}{125, 194, 70}
\definecolor{mauve}{RGB}{154, 0, 215}
\definecolor{orange}{RGB}{255, 96, 0}
\definecolor{turquoise}{RGB}{0, 153, 153}
\definecolor{rouge}{RGB}{255, 0, 0}
\definecolor{lightvert}{RGB}{205, 234, 190}
\setitemize[0]{label=\color{lightvert}  $\bullet$}

\pagestyle{fancy}
\renewcommand{\headrulewidth}{0.2pt}
\fancyhead[L]{maths-cours.fr}
\fancyhead[R]{\thepage}
\renewcommand{\footrulewidth}{0.2pt}
\fancyfoot[C]{}

\newcolumntype{C}{>{\centering\arraybackslash}X}
\newcolumntype{s}{>{\hsize=.35\hsize\arraybackslash}X}

\setlength{\parindent}{0pt}		 
\setlength{\parskip}{3mm}
\setlength{\headheight}{1cm}

\def\ebook{ebook}
\def\book{book}
\def\web{web}
\def\type{web}

\newcommand{\vect}[1]{\overrightarrow{\,\mathstrut#1\,}}

\def\Oij{$\left(\text{O}~;~\vect{\imath},~\vect{\jmath}\right)$}
\def\Oijk{$\left(\text{O}~;~\vect{\imath},~\vect{\jmath},~\vect{k}\right)$}
\def\Ouv{$\left(\text{O}~;~\vect{u},~\vect{v}\right)$}

\hypersetup{breaklinks=true, colorlinks = true, linkcolor = OliveGreen, urlcolor = OliveGreen, citecolor = OliveGreen, pdfauthor={Didier BONNEL - https://www.maths-cours.fr} } % supprime les bordures autour des liens

\renewcommand{\arg}[0]{\text{arg}}

\everymath{\displaystyle}

%================================================================================================================================
%
% Macros - Commandes
%
%================================================================================================================================

\newcommand\meta[2]{    			% Utilisé pour créer le post HTML.
	\def\titre{titre}
	\def\url{url}
	\def\arg{#1}
	\ifx\titre\arg
		\newcommand\maintitle{#2}
		\fancyhead[L]{#2}
		{\Large\sffamily \MakeUppercase{#2}}
		\vspace{1mm}\textcolor{mcvert}{\hrule}
	\fi 
	\ifx\url\arg
		\fancyfoot[L]{\href{https://www.maths-cours.fr#2}{\black \footnotesize{https://www.maths-cours.fr#2}}}
	\fi 
}


\newcommand\TitreC[1]{    		% Titre centré
     \needspace{3\baselineskip}
     \begin{center}\textbf{#1}\end{center}
}

\newcommand\newpar{    		% paragraphe
     \par
}

\newcommand\nosp {    		% commande vide (pas d'espace)
}
\newcommand{\id}[1]{} %ignore

\newcommand\boite[2]{				% Boite simple sans titre
	\vspace{5mm}
	\setlength{\fboxrule}{0.2mm}
	\setlength{\fboxsep}{5mm}	
	\fcolorbox{#1}{#1!3}{\makebox[\linewidth-2\fboxrule-2\fboxsep]{
  		\begin{minipage}[t]{\linewidth-2\fboxrule-4\fboxsep}\setlength{\parskip}{3mm}
  			 #2
  		\end{minipage}
	}}
	\vspace{5mm}
}

\newcommand\CBox[4]{				% Boites
	\vspace{5mm}
	\setlength{\fboxrule}{0.2mm}
	\setlength{\fboxsep}{5mm}
	
	\fcolorbox{#1}{#1!3}{\makebox[\linewidth-2\fboxrule-2\fboxsep]{
		\begin{minipage}[t]{1cm}\setlength{\parskip}{3mm}
	  		\textcolor{#1}{\LARGE{#2}}    
 	 	\end{minipage}  
  		\begin{minipage}[t]{\linewidth-2\fboxrule-4\fboxsep}\setlength{\parskip}{3mm}
			\raisebox{1.2mm}{\normalsize\sffamily{\textcolor{#1}{#3}}}						
  			 #4
  		\end{minipage}
	}}
	\vspace{5mm}
}

\newcommand\cadre[3]{				% Boites convertible html
	\par
	\vspace{2mm}
	\setlength{\fboxrule}{0.1mm}
	\setlength{\fboxsep}{5mm}
	\fcolorbox{#1}{white}{\makebox[\linewidth-2\fboxrule-2\fboxsep]{
  		\begin{minipage}[t]{\linewidth-2\fboxrule-4\fboxsep}\setlength{\parskip}{3mm}
			\raisebox{-2.5mm}{\sffamily \small{\textcolor{#1}{\MakeUppercase{#2}}}}		
			\par		
  			 #3
 	 		\end{minipage}
	}}
		\vspace{2mm}
	\par
}

\newcommand\bloc[3]{				% Boites convertible html sans bordure
     \needspace{2\baselineskip}
     {\sffamily \small{\textcolor{#1}{\MakeUppercase{#2}}}}    
		\par		
  			 #3
		\par
}

\newcommand\CHelp[1]{
     \CBox{Plum}{\faInfoCircle}{À RETENIR}{#1}
}

\newcommand\CUp[1]{
     \CBox{NavyBlue}{\faThumbsOUp}{EN PRATIQUE}{#1}
}

\newcommand\CInfo[1]{
     \CBox{Sepia}{\faArrowCircleRight}{REMARQUE}{#1}
}

\newcommand\CRedac[1]{
     \CBox{PineGreen}{\faEdit}{BIEN R\'EDIGER}{#1}
}

\newcommand\CError[1]{
     \CBox{Red}{\faExclamationTriangle}{ATTENTION}{#1}
}

\newcommand\TitreExo[2]{
\needspace{4\baselineskip}
 {\sffamily\large EXERCICE #1\ (\emph{#2 points})}
\vspace{5mm}
}

\newcommand\img[2]{
          \includegraphics[width=#2\paperwidth]{\imgdir#1}
}

\newcommand\imgsvg[2]{
       \begin{center}   \includegraphics[width=#2\paperwidth]{\imgsvgdir#1} \end{center}
}


\newcommand\Lien[2]{
     \href{#1}{#2 \tiny \faExternalLink}
}
\newcommand\mcLien[2]{
     \href{https~://www.maths-cours.fr/#1}{#2 \tiny \faExternalLink}
}

\newcommand{\euro}{\eurologo{}}

%================================================================================================================================
%
% Macros - Environement
%
%================================================================================================================================

\newenvironment{tex}{ %
}
{%
}

\newenvironment{indente}{ %
	\setlength\parindent{10mm}
}

{
	\setlength\parindent{0mm}
}

\newenvironment{corrige}{%
     \needspace{3\baselineskip}
     \medskip
     \textbf{\textsc{Corrigé}}
     \medskip
}
{
}

\newenvironment{extern}{%
     \begin{center}
     }
     {
     \end{center}
}

\NewEnviron{code}{%
	\par
     \boite{gray}{\texttt{%
     \BODY
     }}
     \par
}

\newenvironment{vbloc}{% boite sans cadre empeche saut de page
     \begin{minipage}[t]{\linewidth}
     }
     {
     \end{minipage}
}
\NewEnviron{h2}{%
    \needspace{3\baselineskip}
    \vspace{0.6cm}
	\noindent \MakeUppercase{\sffamily \large \BODY}
	\vspace{1mm}\textcolor{mcgris}{\hrule}\vspace{0.4cm}
	\par
}{}

\NewEnviron{h3}{%
    \needspace{3\baselineskip}
	\vspace{5mm}
	\textsc{\BODY}
	\par
}

\NewEnviron{margeneg}{ %
\begin{addmargin}[-1cm]{0cm}
\BODY
\end{addmargin}
}

\NewEnviron{html}{%
}

\begin{document}
\meta{url}{/exercices/suites-matrices-bac-s-amerique-du-nord-2018-spe/}
\meta{pid}{8011}
\meta{titre}{Suites - Matrices - Bac S Amérique du Nord  2018 (spé)}
\meta{type}{exercices}
%
\begin{h2}Exercice 4 (5 points)\end{h2}
\textbf{Candidats ayant suivi l'enseignement de spécialité}
\medskip
Dans une région, on s'intéresse à la cohabitation de deux espèces animales~: les campagnols et les
renards, les renards étant les prédateurs des campagnols.
\par
Au 1$^{\text{er}}$ juillet 2012, on estime qu'il y a dans cette région approximativement deux millions de campagnols et cent-vingt renards.
\par
On note $u_n$ le nombre de campagnols et $v_n$ le nombre de renards au 1\up{er} juillet de l'année $2012+ n$.
\bigskip
\begin{center}\begin{h3}Partie A - Un modèle simple \end{h3}\end{center}
\medskip
On modélise l'évolution des populations par les relations suivantes~:
\par
\[\left\{\begin{array}{l}
          u_{n+1} = 1,1u_n - 2~000v_n\\
          v_{n+1}= 2 \times 10^{-5}u_n + 0,6v_n
\end{array}\right.\]
pour tout entier $n \geqslant 0$, avec $u_0 = 2~000~000$ et $v_0 = 120$.
\medskip
\begin{enumerate}
     \item
     \begin{enumerate}[label=\alph*.]
          \item On considère la matrice colonne $U_n = \begin{pmatrix}u_n\\v_n\end{pmatrix}$ pour tout entier $n \geqslant 0$.
          \par
          Déterminer la matrice $A$ telle que $U_{n+1} = A \times U_n$ pour tout entier $n$ et donner la matrice $U_0$.
          \item Calculer le nombre de campagnols et de renards estimés grâce à ce modèle au 1\up{er} juillet
          2018.
     \end{enumerate}
     \item Soit les matrices $P = \begin{pmatrix}20~000&5~000\\1&1\end{pmatrix}$, \:$D = \begin{pmatrix}1&0\\0&0,7\end{pmatrix}$ et $P^{- 1}  = \dfrac{1}{15~000}\begin{pmatrix}1& -5~000\\- 1&20~000\end{pmatrix}$.
     \par
     On admet que $P^{- 1}$ est la matrice inverse de la matrice $P$ et que $A = P \times D \times P^{- 1}$.
     \begin{enumerate}[label=\alph*.]
          \item Montrer que pour tout entier naturel $n$,\: $U_n = P \times D^n \times P^{- 1} \times U_0$.
          \item Donner sans justification l'expression de la matrice $D^n$ en fonction de $n$.
          \item On admet que, pour tout entier naturel $n$~:
          \par
          \[\left\{\begin{array}{l}
                    u_n = \dfrac{2,8 \times 10^7 + 2 \times 10^6 \times 0,7^n}{15}\\
                    \\
                    v_n =\dfrac{1~400 + 400 \times 0,7^n}{15}\\
          \end{array}\right.\]
          Décrire l'évolution des deux populations.
     \end{enumerate}
\end{enumerate}
\bigskip
\begin{center}\begin{h3}Partie B - Un modèle plus conforme à la réalité \end{h3}\end{center}
\medskip
Dans la réalité, on observe que si le nombre de renards a suffisamment baissé, alors le nombre de
campagnols augmente à nouveau, ce qui n'est pas le cas avec le modèle précédent.
\par
On construit donc un autre modèle, plus précis, qui tient compte de ce type d'observations à l'aide des relations suivantes~:
\par
\[\left\{\begin{array}{l}
          u_{n+1} = 1,1u_n - 0,001ru_n \times v_n\\
          v_{n+1} = 2 \times 10^{-7} u_n \times v_n + 0,6v_n
\end{array}\right.\]
pour tout entier $n \geqslant 0$, avec $u_0 = 2~000~000$ et $v_0 = 120$.
\medskip
Le tableau ci-dessous présente ce nouveau modèle sur les $25$ premières années en donnant les
effectifs des populations arrondis à l'unité~:
\begin{center}
     \begin{extern}%width="400"
          \begin{tabularx}{0.7\linewidth}{|>{\columncolor[gray]{0.7}}c|*{3}{>{\centering \arraybackslash}X|}}\hline
               \rowcolor[gray]{0.7}&A &B &C\\ \hline
               1& \multicolumn{3}{c|}{Modèle de la \textbf{partie B}}\\ \hline
               2& $n$ 	&$u_n$ 			&$v_n$\\ \hline
               3&0		& 2~000~000 	&120\\ \hline
               4&1		& 1~960~000 	&120\\ \hline
               5&2		& 1~920~800 	&119\\ \hline
               6&3		& 1~884~228 	&117\\ \hline
               7&4		& 1~851~905 	&114\\ \hline
               8&5		& 1~825~160 	&111\\ \hline
               9&6		& 1~804~988 	&107\\ \hline
               10&7	& 1~792~049 	&103\\ \hline
               11&8	& 1~786~692 	&99\\ \hline
               12&9	& 1~789~005 	&94\\ \hline
               13&10	& 1~798~854 	&91\\ \hline
               14&11	& 1~815~930 	&87\\ \hline
               15&12	& 1~839~780 	&84\\ \hline
               16&13	& 1~869~827 	&81\\ \hline
               17&14	& 1~905~378 	&79\\ \hline
               18&15	& 1~945~622 	&77\\ \hline
               19&16	& 1~989~620 	&77\\ \hline
               20&17	& 2~036~288 	&76\\ \hline
               21&18	& 2~084~374 	&77\\ \hline
               22&19	& 2~132~440 	&78\\ \hline
               23&20	& 2~178~846 	&80\\ \hline
               24&21	& 2~221~746 	&83\\ \hline
               25&22	& 2~259~109 	&87\\ \hline
               26&23	& 2~288~766 	&91\\ \hline
               27&24	& 2~308~508 	&97\\ \hline
          \end{tabularx}
     \end{extern}
\end{center}
\medskip
\begin{enumerate}
     \item Quelles formules faut-il écrire dans les cellules B4 et C4 et recopier vers le bas pour remplir
     les colonnes B et C~?
     \item  Avec le deuxième modèle, à partir de quelle année observe-t-on le phénomène décrit (baisse
     des renards et hausse des campagnols)~?
\end{enumerate}
\bigskip
\begin{center}\begin{h3}Partie C \end{h3}\end{center}
\medskip
Dans cette partie on utilise le modèle de la partie B.
\par
Est-il possible de donner à $u_0$ et $v_0$ des valeurs afin que les deux populations restent stables d'une
année sur l'autre, c'est-à-dire telles que pour tout entier naturel $n$ on ait $u_{n+1} = u_n$ et $v_{n+1} = v_n$~? (On parle alors d'état stable.)
\par
\par

\end{document}
µ
\documentclass[a4paper]{article}

%================================================================================================================================
%
% Packages
%
%================================================================================================================================

\usepackage[T1]{fontenc} 	% pour caractères accentués
\usepackage[utf8]{inputenc}  % encodage utf8
\usepackage[french]{babel}	% langue : français
\usepackage{fourier}			% caractères plus lisibles
\usepackage[dvipsnames]{xcolor} % couleurs
\usepackage{fancyhdr}		% réglage header footer
\usepackage{needspace}		% empêcher sauts de page mal placés
\usepackage{graphicx}		% pour inclure des graphiques
\usepackage{enumitem,cprotect}		% personnalise les listes d'items (nécessaire pour ol, al ...)
\usepackage{hyperref}		% Liens hypertexte
\usepackage{pstricks,pst-all,pst-node,pstricks-add,pst-math,pst-plot,pst-tree,pst-eucl} % pstricks
\usepackage[a4paper,includeheadfoot,top=2cm,left=3cm, bottom=2cm,right=3cm]{geometry} % marges etc.
\usepackage{comment}			% commentaires multilignes
\usepackage{amsmath,environ} % maths (matrices, etc.)
\usepackage{amssymb,makeidx}
\usepackage{bm}				% bold maths
\usepackage{tabularx}		% tableaux
\usepackage{colortbl}		% tableaux en couleur
\usepackage{fontawesome}		% Fontawesome
\usepackage{environ}			% environment with command
\usepackage{fp}				% calculs pour ps-tricks
\usepackage{multido}			% pour ps tricks
\usepackage[np]{numprint}	% formattage nombre
\usepackage{tikz,tkz-tab} 			% package principal TikZ
\usepackage{pgfplots}   % axes
\usepackage{mathrsfs}    % cursives
\usepackage{calc}			% calcul taille boites
\usepackage[scaled=0.875]{helvet} % font sans serif
\usepackage{svg} % svg
\usepackage{scrextend} % local margin
\usepackage{scratch} %scratch
\usepackage{multicol} % colonnes
%\usepackage{infix-RPN,pst-func} % formule en notation polanaise inversée
\usepackage{listings}

%================================================================================================================================
%
% Réglages de base
%
%================================================================================================================================

\lstset{
language=Python,   % R code
literate=
{á}{{\'a}}1
{à}{{\`a}}1
{ã}{{\~a}}1
{é}{{\'e}}1
{è}{{\`e}}1
{ê}{{\^e}}1
{í}{{\'i}}1
{ó}{{\'o}}1
{õ}{{\~o}}1
{ú}{{\'u}}1
{ü}{{\"u}}1
{ç}{{\c{c}}}1
{~}{{ }}1
}


\definecolor{codegreen}{rgb}{0,0.6,0}
\definecolor{codegray}{rgb}{0.5,0.5,0.5}
\definecolor{codepurple}{rgb}{0.58,0,0.82}
\definecolor{backcolour}{rgb}{0.95,0.95,0.92}

\lstdefinestyle{mystyle}{
    backgroundcolor=\color{backcolour},   
    commentstyle=\color{codegreen},
    keywordstyle=\color{magenta},
    numberstyle=\tiny\color{codegray},
    stringstyle=\color{codepurple},
    basicstyle=\ttfamily\footnotesize,
    breakatwhitespace=false,         
    breaklines=true,                 
    captionpos=b,                    
    keepspaces=true,                 
    numbers=left,                    
xleftmargin=2em,
framexleftmargin=2em,            
    showspaces=false,                
    showstringspaces=false,
    showtabs=false,                  
    tabsize=2,
    upquote=true
}

\lstset{style=mystyle}


\lstset{style=mystyle}
\newcommand{\imgdir}{C:/laragon/www/newmc/assets/imgsvg/}
\newcommand{\imgsvgdir}{C:/laragon/www/newmc/assets/imgsvg/}

\definecolor{mcgris}{RGB}{220, 220, 220}% ancien~; pour compatibilité
\definecolor{mcbleu}{RGB}{52, 152, 219}
\definecolor{mcvert}{RGB}{125, 194, 70}
\definecolor{mcmauve}{RGB}{154, 0, 215}
\definecolor{mcorange}{RGB}{255, 96, 0}
\definecolor{mcturquoise}{RGB}{0, 153, 153}
\definecolor{mcrouge}{RGB}{255, 0, 0}
\definecolor{mclightvert}{RGB}{205, 234, 190}

\definecolor{gris}{RGB}{220, 220, 220}
\definecolor{bleu}{RGB}{52, 152, 219}
\definecolor{vert}{RGB}{125, 194, 70}
\definecolor{mauve}{RGB}{154, 0, 215}
\definecolor{orange}{RGB}{255, 96, 0}
\definecolor{turquoise}{RGB}{0, 153, 153}
\definecolor{rouge}{RGB}{255, 0, 0}
\definecolor{lightvert}{RGB}{205, 234, 190}
\setitemize[0]{label=\color{lightvert}  $\bullet$}

\pagestyle{fancy}
\renewcommand{\headrulewidth}{0.2pt}
\fancyhead[L]{maths-cours.fr}
\fancyhead[R]{\thepage}
\renewcommand{\footrulewidth}{0.2pt}
\fancyfoot[C]{}

\newcolumntype{C}{>{\centering\arraybackslash}X}
\newcolumntype{s}{>{\hsize=.35\hsize\arraybackslash}X}

\setlength{\parindent}{0pt}		 
\setlength{\parskip}{3mm}
\setlength{\headheight}{1cm}

\def\ebook{ebook}
\def\book{book}
\def\web{web}
\def\type{web}

\newcommand{\vect}[1]{\overrightarrow{\,\mathstrut#1\,}}

\def\Oij{$\left(\text{O}~;~\vect{\imath},~\vect{\jmath}\right)$}
\def\Oijk{$\left(\text{O}~;~\vect{\imath},~\vect{\jmath},~\vect{k}\right)$}
\def\Ouv{$\left(\text{O}~;~\vect{u},~\vect{v}\right)$}

\hypersetup{breaklinks=true, colorlinks = true, linkcolor = OliveGreen, urlcolor = OliveGreen, citecolor = OliveGreen, pdfauthor={Didier BONNEL - https://www.maths-cours.fr} } % supprime les bordures autour des liens

\renewcommand{\arg}[0]{\text{arg}}

\everymath{\displaystyle}

%================================================================================================================================
%
% Macros - Commandes
%
%================================================================================================================================

\newcommand\meta[2]{    			% Utilisé pour créer le post HTML.
	\def\titre{titre}
	\def\url{url}
	\def\arg{#1}
	\ifx\titre\arg
		\newcommand\maintitle{#2}
		\fancyhead[L]{#2}
		{\Large\sffamily \MakeUppercase{#2}}
		\vspace{1mm}\textcolor{mcvert}{\hrule}
	\fi 
	\ifx\url\arg
		\fancyfoot[L]{\href{https://www.maths-cours.fr#2}{\black \footnotesize{https://www.maths-cours.fr#2}}}
	\fi 
}


\newcommand\TitreC[1]{    		% Titre centré
     \needspace{3\baselineskip}
     \begin{center}\textbf{#1}\end{center}
}

\newcommand\newpar{    		% paragraphe
     \par
}

\newcommand\nosp {    		% commande vide (pas d'espace)
}
\newcommand{\id}[1]{} %ignore

\newcommand\boite[2]{				% Boite simple sans titre
	\vspace{5mm}
	\setlength{\fboxrule}{0.2mm}
	\setlength{\fboxsep}{5mm}	
	\fcolorbox{#1}{#1!3}{\makebox[\linewidth-2\fboxrule-2\fboxsep]{
  		\begin{minipage}[t]{\linewidth-2\fboxrule-4\fboxsep}\setlength{\parskip}{3mm}
  			 #2
  		\end{minipage}
	}}
	\vspace{5mm}
}

\newcommand\CBox[4]{				% Boites
	\vspace{5mm}
	\setlength{\fboxrule}{0.2mm}
	\setlength{\fboxsep}{5mm}
	
	\fcolorbox{#1}{#1!3}{\makebox[\linewidth-2\fboxrule-2\fboxsep]{
		\begin{minipage}[t]{1cm}\setlength{\parskip}{3mm}
	  		\textcolor{#1}{\LARGE{#2}}    
 	 	\end{minipage}  
  		\begin{minipage}[t]{\linewidth-2\fboxrule-4\fboxsep}\setlength{\parskip}{3mm}
			\raisebox{1.2mm}{\normalsize\sffamily{\textcolor{#1}{#3}}}						
  			 #4
  		\end{minipage}
	}}
	\vspace{5mm}
}

\newcommand\cadre[3]{				% Boites convertible html
	\par
	\vspace{2mm}
	\setlength{\fboxrule}{0.1mm}
	\setlength{\fboxsep}{5mm}
	\fcolorbox{#1}{white}{\makebox[\linewidth-2\fboxrule-2\fboxsep]{
  		\begin{minipage}[t]{\linewidth-2\fboxrule-4\fboxsep}\setlength{\parskip}{3mm}
			\raisebox{-2.5mm}{\sffamily \small{\textcolor{#1}{\MakeUppercase{#2}}}}		
			\par		
  			 #3
 	 		\end{minipage}
	}}
		\vspace{2mm}
	\par
}

\newcommand\bloc[3]{				% Boites convertible html sans bordure
     \needspace{2\baselineskip}
     {\sffamily \small{\textcolor{#1}{\MakeUppercase{#2}}}}    
		\par		
  			 #3
		\par
}

\newcommand\CHelp[1]{
     \CBox{Plum}{\faInfoCircle}{À RETENIR}{#1}
}

\newcommand\CUp[1]{
     \CBox{NavyBlue}{\faThumbsOUp}{EN PRATIQUE}{#1}
}

\newcommand\CInfo[1]{
     \CBox{Sepia}{\faArrowCircleRight}{REMARQUE}{#1}
}

\newcommand\CRedac[1]{
     \CBox{PineGreen}{\faEdit}{BIEN R\'EDIGER}{#1}
}

\newcommand\CError[1]{
     \CBox{Red}{\faExclamationTriangle}{ATTENTION}{#1}
}

\newcommand\TitreExo[2]{
\needspace{4\baselineskip}
 {\sffamily\large EXERCICE #1\ (\emph{#2 points})}
\vspace{5mm}
}

\newcommand\img[2]{
          \includegraphics[width=#2\paperwidth]{\imgdir#1}
}

\newcommand\imgsvg[2]{
       \begin{center}   \includegraphics[width=#2\paperwidth]{\imgsvgdir#1} \end{center}
}


\newcommand\Lien[2]{
     \href{#1}{#2 \tiny \faExternalLink}
}
\newcommand\mcLien[2]{
     \href{https~://www.maths-cours.fr/#1}{#2 \tiny \faExternalLink}
}

\newcommand{\euro}{\eurologo{}}

%================================================================================================================================
%
% Macros - Environement
%
%================================================================================================================================

\newenvironment{tex}{ %
}
{%
}

\newenvironment{indente}{ %
	\setlength\parindent{10mm}
}

{
	\setlength\parindent{0mm}
}

\newenvironment{corrige}{%
     \needspace{3\baselineskip}
     \medskip
     \textbf{\textsc{Corrigé}}
     \medskip
}
{
}

\newenvironment{extern}{%
     \begin{center}
     }
     {
     \end{center}
}

\NewEnviron{code}{%
	\par
     \boite{gray}{\texttt{%
     \BODY
     }}
     \par
}

\newenvironment{vbloc}{% boite sans cadre empeche saut de page
     \begin{minipage}[t]{\linewidth}
     }
     {
     \end{minipage}
}
\NewEnviron{h2}{%
    \needspace{3\baselineskip}
    \vspace{0.6cm}
	\noindent \MakeUppercase{\sffamily \large \BODY}
	\vspace{1mm}\textcolor{mcgris}{\hrule}\vspace{0.4cm}
	\par
}{}

\NewEnviron{h3}{%
    \needspace{3\baselineskip}
	\vspace{5mm}
	\textsc{\BODY}
	\par
}

\NewEnviron{margeneg}{ %
\begin{addmargin}[-1cm]{0cm}
\BODY
\end{addmargin}
}

\NewEnviron{html}{%
}

\begin{document}
\meta{url}{/exercices/fonctions-bac-s-centres-etrangers-2018/}
\meta{pid}{8595}
\meta{titre}{Fonctions - Bac S Centres étrangers  2018}
\meta{type}{exercices}
%
\begin{h2}Exercice 1 (4 points)\end{h2}
\textbf{Commun à  tous les candidats}
\medskip
Dans une usine, on se propose de tester un prototype de hotte aspirante pour un local industriel.
\par
Avant de lancer la fabrication en série, on réalise l'expérience suivante : dans un local clos équipé
du prototype de hotte aspirante, on diffuse du dioxyde de carbone (CO$_2$) à débit constant.
\par
Dans ce qui suit, $t$ est le temps exprimé en minute.
\par
À l'instant $t = 0$, la hotte est mise en marche et on la laisse fonctionner pendant $20$ minutes. Les
mesures réalisées permettent de modéliser le taux (en pourcentage) de CO$_2$ contenu dans le local au
bout de $t$ minutes de fonctionnement de la hotte par l'expression $f(t)$, où $f$ est la fonction définie
pour tout réel $t$ de l'intervalle [0~;~20] par :
\[f(t) = (0,8t + 0,2)\text{e}^{-0,5t} + 0,03.\]
On donne ci-dessous le tableau de variation de la fonction $f$ sur l'intervalle [0~;~20].
\par
Ainsi, la valeur $f(0) = 0,23$ traduit le fait que le taux
de CO$_2$ à l'instant $0$ est égal à 23\,\%.
%:-+-+-+-+- Engendré par : http://math.et.info.free.fr/TikZ/TableauxVariations/
\begin{center}
     \begin{extern}%width="350" alt="tableau de variations bac centres étrangers"
          \begin{tikzpicture}[scale=0.875]
               % Styles
               \tikzstyle{cadre}=[thin]
               \tikzstyle{fleche}=[->,>=latex,thin]
               \tikzstyle{nondefini}=[lightgray]
               % Dimensions Modifiables
               \def\Lrg{1.5}
               \def\HtX{1}
               \def\HtY{0.5}
               % Dimensions Calculées
               \def\lignex{-0.5*\HtX}
               \def\lignef{-1.5*\HtX}
               \def\separateur{-0.5*\Lrg}
               % Largeur du tableau
               \def\gauche{-1.5*\Lrg}
               \def\droite{4.5*\Lrg}
               % Hauteur du tableau
               \def\haut{0.5*\HtX}
               \def\bas{-2.5*\HtX-2*\HtY}
               % Ligne de l'abscisse : x
               \node at (-1*\Lrg,0) {$t$};
               \node at (0*\Lrg,0) {$0$};
               \node at (2*\Lrg,0) {$1,75$};
               \node at (4*\Lrg,0) {$20$};
               % Ligne de la dérivée : f'(x)
               \node at (-1*\Lrg,-1*\HtX) {$f'(t)$};
               \node at (0*\Lrg,-1*\HtX) {$$};
               \node at (1*\Lrg,-1*\HtX) {$+$};
               \node at (2*\Lrg,-1*\HtX) {$0$};
               \node at (3*\Lrg,-1*\HtX) {$-$};
               \node at (4*\Lrg,-1*\HtX) {$$};
               % Ligne de la fonction : f(x)
               \node  at (-1*\Lrg,{-2*\HtX+(-1)*\HtY}) {$f(t)$};
               \node (f1) at (0*\Lrg,{-2*\HtX+(-2)*\HtY}) {$0,23$};
               \node (f2) at (2*\Lrg,{-2*\HtX+(0)*\HtY}) {$$};
               \node (f3) at (4*\Lrg,{-2*\HtX+(-2)*\HtY}) {$$};
               % Flèches
               \draw[fleche] (f1) -- (f2);
               \draw[fleche] (f2) -- (f3);
               % Encadrement
               \draw[cadre] (\separateur,\haut) -- (\separateur,\bas);
               \draw[cadre] (\gauche,\haut) rectangle  (\droite,\bas);
               \draw[cadre] (\gauche,\lignex) -- (\droite,\lignex);
               \draw[cadre] (\gauche,\lignef) -- (\droite,\lignef);
          \end{tikzpicture}
     \end{extern}
\end{center}
\begin{enumerate}
     \item Dans cette question, on arrondira les deux résultats au millième.
     \begin{enumerate}[label=\alph*.]
          \item Calculer $f (20)$.
          \item Déterminer le taux maximal de CO$_2$ présent dans le local pendant l'expérience.
     \end{enumerate}
     \item  On souhaite que le taux de CO$_2$ dans le local retrouve une valeur $V$ inférieure ou égale à $3,5$\,\%.
     \begin{enumerate}[label=\alph*.]
          \item Justifier qu'il existe un unique instant $T$ satisfaisant cette condition.
          \item  On considère l'algorithme suivant :
          \begin{center}
               \begin{extern}%width="400" alt="algorithme bac centres étrangers"
                    \begin{tabularx}{0.6\linewidth}{|X|}\hline
                         $t \gets 1,75$\\
                         $p \gets 0,1$\\
                         $V \gets 0,7$\\
                         Tant que $V > 0,035$\\
                         \hspace{0.75cm}$t \gets t + p$\\
                         \hspace{0.75cm}$V \gets (0,8t + 0,2)\text{e}^{-0,5t} + 0,03$\\
                         Fin Tant que\\ \hline
                    \end{tabularx}
               \end{extern}
          \end{center}
          Quelle est la valeur de la variable $t$ à la fin de l'algorithme ?
          \par
          Que représente cette valeur dans le contexte de l'exercice ?
     \end{enumerate}
     \item  On désigne par $V_m$ le taux moyen (en pourcentage) de CO$_2$ présent dans le local pendant les 11 premières minutes de fonctionnement de la hotte aspirante.
     \begin{enumerate}[label=\alph*.]
          \item Soit $F$ la fonction définie sur l'intervalle [0~;~11] par :
          \[F(t) = (-1,6t -3,6)\text{e}^{-0,5t} +0,03t.\]
          Montrer que la fonction $F$ est une primitive de la fonction $f$ sur l'intervalle [0~;~11].
          \item En déduire le taux moyen $V_m$, valeur moyenne de la fonction $f$ sur l'intervalle [0~;~11].
          Arrondir le résultat au millième, soit à $0,1$\,\%.
     \end{enumerate}
\end{enumerate}

\end{document}
µ
\documentclass[a4paper]{article}

%================================================================================================================================
%
% Packages
%
%================================================================================================================================

\usepackage[T1]{fontenc} 	% pour caractères accentués
\usepackage[utf8]{inputenc}  % encodage utf8
\usepackage[french]{babel}	% langue : français
\usepackage{fourier}			% caractères plus lisibles
\usepackage[dvipsnames]{xcolor} % couleurs
\usepackage{fancyhdr}		% réglage header footer
\usepackage{needspace}		% empêcher sauts de page mal placés
\usepackage{graphicx}		% pour inclure des graphiques
\usepackage{enumitem,cprotect}		% personnalise les listes d'items (nécessaire pour ol, al ...)
\usepackage{hyperref}		% Liens hypertexte
\usepackage{pstricks,pst-all,pst-node,pstricks-add,pst-math,pst-plot,pst-tree,pst-eucl} % pstricks
\usepackage[a4paper,includeheadfoot,top=2cm,left=3cm, bottom=2cm,right=3cm]{geometry} % marges etc.
\usepackage{comment}			% commentaires multilignes
\usepackage{amsmath,environ} % maths (matrices, etc.)
\usepackage{amssymb,makeidx}
\usepackage{bm}				% bold maths
\usepackage{tabularx}		% tableaux
\usepackage{colortbl}		% tableaux en couleur
\usepackage{fontawesome}		% Fontawesome
\usepackage{environ}			% environment with command
\usepackage{fp}				% calculs pour ps-tricks
\usepackage{multido}			% pour ps tricks
\usepackage[np]{numprint}	% formattage nombre
\usepackage{tikz,tkz-tab} 			% package principal TikZ
\usepackage{pgfplots}   % axes
\usepackage{mathrsfs}    % cursives
\usepackage{calc}			% calcul taille boites
\usepackage[scaled=0.875]{helvet} % font sans serif
\usepackage{svg} % svg
\usepackage{scrextend} % local margin
\usepackage{scratch} %scratch
\usepackage{multicol} % colonnes
%\usepackage{infix-RPN,pst-func} % formule en notation polanaise inversée
\usepackage{listings}

%================================================================================================================================
%
% Réglages de base
%
%================================================================================================================================

\lstset{
language=Python,   % R code
literate=
{á}{{\'a}}1
{à}{{\`a}}1
{ã}{{\~a}}1
{é}{{\'e}}1
{è}{{\`e}}1
{ê}{{\^e}}1
{í}{{\'i}}1
{ó}{{\'o}}1
{õ}{{\~o}}1
{ú}{{\'u}}1
{ü}{{\"u}}1
{ç}{{\c{c}}}1
{~}{{ }}1
}


\definecolor{codegreen}{rgb}{0,0.6,0}
\definecolor{codegray}{rgb}{0.5,0.5,0.5}
\definecolor{codepurple}{rgb}{0.58,0,0.82}
\definecolor{backcolour}{rgb}{0.95,0.95,0.92}

\lstdefinestyle{mystyle}{
    backgroundcolor=\color{backcolour},   
    commentstyle=\color{codegreen},
    keywordstyle=\color{magenta},
    numberstyle=\tiny\color{codegray},
    stringstyle=\color{codepurple},
    basicstyle=\ttfamily\footnotesize,
    breakatwhitespace=false,         
    breaklines=true,                 
    captionpos=b,                    
    keepspaces=true,                 
    numbers=left,                    
xleftmargin=2em,
framexleftmargin=2em,            
    showspaces=false,                
    showstringspaces=false,
    showtabs=false,                  
    tabsize=2,
    upquote=true
}

\lstset{style=mystyle}


\lstset{style=mystyle}
\newcommand{\imgdir}{C:/laragon/www/newmc/assets/imgsvg/}
\newcommand{\imgsvgdir}{C:/laragon/www/newmc/assets/imgsvg/}

\definecolor{mcgris}{RGB}{220, 220, 220}% ancien~; pour compatibilité
\definecolor{mcbleu}{RGB}{52, 152, 219}
\definecolor{mcvert}{RGB}{125, 194, 70}
\definecolor{mcmauve}{RGB}{154, 0, 215}
\definecolor{mcorange}{RGB}{255, 96, 0}
\definecolor{mcturquoise}{RGB}{0, 153, 153}
\definecolor{mcrouge}{RGB}{255, 0, 0}
\definecolor{mclightvert}{RGB}{205, 234, 190}

\definecolor{gris}{RGB}{220, 220, 220}
\definecolor{bleu}{RGB}{52, 152, 219}
\definecolor{vert}{RGB}{125, 194, 70}
\definecolor{mauve}{RGB}{154, 0, 215}
\definecolor{orange}{RGB}{255, 96, 0}
\definecolor{turquoise}{RGB}{0, 153, 153}
\definecolor{rouge}{RGB}{255, 0, 0}
\definecolor{lightvert}{RGB}{205, 234, 190}
\setitemize[0]{label=\color{lightvert}  $\bullet$}

\pagestyle{fancy}
\renewcommand{\headrulewidth}{0.2pt}
\fancyhead[L]{maths-cours.fr}
\fancyhead[R]{\thepage}
\renewcommand{\footrulewidth}{0.2pt}
\fancyfoot[C]{}

\newcolumntype{C}{>{\centering\arraybackslash}X}
\newcolumntype{s}{>{\hsize=.35\hsize\arraybackslash}X}

\setlength{\parindent}{0pt}		 
\setlength{\parskip}{3mm}
\setlength{\headheight}{1cm}

\def\ebook{ebook}
\def\book{book}
\def\web{web}
\def\type{web}

\newcommand{\vect}[1]{\overrightarrow{\,\mathstrut#1\,}}

\def\Oij{$\left(\text{O}~;~\vect{\imath},~\vect{\jmath}\right)$}
\def\Oijk{$\left(\text{O}~;~\vect{\imath},~\vect{\jmath},~\vect{k}\right)$}
\def\Ouv{$\left(\text{O}~;~\vect{u},~\vect{v}\right)$}

\hypersetup{breaklinks=true, colorlinks = true, linkcolor = OliveGreen, urlcolor = OliveGreen, citecolor = OliveGreen, pdfauthor={Didier BONNEL - https://www.maths-cours.fr} } % supprime les bordures autour des liens

\renewcommand{\arg}[0]{\text{arg}}

\everymath{\displaystyle}

%================================================================================================================================
%
% Macros - Commandes
%
%================================================================================================================================

\newcommand\meta[2]{    			% Utilisé pour créer le post HTML.
	\def\titre{titre}
	\def\url{url}
	\def\arg{#1}
	\ifx\titre\arg
		\newcommand\maintitle{#2}
		\fancyhead[L]{#2}
		{\Large\sffamily \MakeUppercase{#2}}
		\vspace{1mm}\textcolor{mcvert}{\hrule}
	\fi 
	\ifx\url\arg
		\fancyfoot[L]{\href{https://www.maths-cours.fr#2}{\black \footnotesize{https://www.maths-cours.fr#2}}}
	\fi 
}


\newcommand\TitreC[1]{    		% Titre centré
     \needspace{3\baselineskip}
     \begin{center}\textbf{#1}\end{center}
}

\newcommand\newpar{    		% paragraphe
     \par
}

\newcommand\nosp {    		% commande vide (pas d'espace)
}
\newcommand{\id}[1]{} %ignore

\newcommand\boite[2]{				% Boite simple sans titre
	\vspace{5mm}
	\setlength{\fboxrule}{0.2mm}
	\setlength{\fboxsep}{5mm}	
	\fcolorbox{#1}{#1!3}{\makebox[\linewidth-2\fboxrule-2\fboxsep]{
  		\begin{minipage}[t]{\linewidth-2\fboxrule-4\fboxsep}\setlength{\parskip}{3mm}
  			 #2
  		\end{minipage}
	}}
	\vspace{5mm}
}

\newcommand\CBox[4]{				% Boites
	\vspace{5mm}
	\setlength{\fboxrule}{0.2mm}
	\setlength{\fboxsep}{5mm}
	
	\fcolorbox{#1}{#1!3}{\makebox[\linewidth-2\fboxrule-2\fboxsep]{
		\begin{minipage}[t]{1cm}\setlength{\parskip}{3mm}
	  		\textcolor{#1}{\LARGE{#2}}    
 	 	\end{minipage}  
  		\begin{minipage}[t]{\linewidth-2\fboxrule-4\fboxsep}\setlength{\parskip}{3mm}
			\raisebox{1.2mm}{\normalsize\sffamily{\textcolor{#1}{#3}}}						
  			 #4
  		\end{minipage}
	}}
	\vspace{5mm}
}

\newcommand\cadre[3]{				% Boites convertible html
	\par
	\vspace{2mm}
	\setlength{\fboxrule}{0.1mm}
	\setlength{\fboxsep}{5mm}
	\fcolorbox{#1}{white}{\makebox[\linewidth-2\fboxrule-2\fboxsep]{
  		\begin{minipage}[t]{\linewidth-2\fboxrule-4\fboxsep}\setlength{\parskip}{3mm}
			\raisebox{-2.5mm}{\sffamily \small{\textcolor{#1}{\MakeUppercase{#2}}}}		
			\par		
  			 #3
 	 		\end{minipage}
	}}
		\vspace{2mm}
	\par
}

\newcommand\bloc[3]{				% Boites convertible html sans bordure
     \needspace{2\baselineskip}
     {\sffamily \small{\textcolor{#1}{\MakeUppercase{#2}}}}    
		\par		
  			 #3
		\par
}

\newcommand\CHelp[1]{
     \CBox{Plum}{\faInfoCircle}{À RETENIR}{#1}
}

\newcommand\CUp[1]{
     \CBox{NavyBlue}{\faThumbsOUp}{EN PRATIQUE}{#1}
}

\newcommand\CInfo[1]{
     \CBox{Sepia}{\faArrowCircleRight}{REMARQUE}{#1}
}

\newcommand\CRedac[1]{
     \CBox{PineGreen}{\faEdit}{BIEN R\'EDIGER}{#1}
}

\newcommand\CError[1]{
     \CBox{Red}{\faExclamationTriangle}{ATTENTION}{#1}
}

\newcommand\TitreExo[2]{
\needspace{4\baselineskip}
 {\sffamily\large EXERCICE #1\ (\emph{#2 points})}
\vspace{5mm}
}

\newcommand\img[2]{
          \includegraphics[width=#2\paperwidth]{\imgdir#1}
}

\newcommand\imgsvg[2]{
       \begin{center}   \includegraphics[width=#2\paperwidth]{\imgsvgdir#1} \end{center}
}


\newcommand\Lien[2]{
     \href{#1}{#2 \tiny \faExternalLink}
}
\newcommand\mcLien[2]{
     \href{https~://www.maths-cours.fr/#1}{#2 \tiny \faExternalLink}
}

\newcommand{\euro}{\eurologo{}}

%================================================================================================================================
%
% Macros - Environement
%
%================================================================================================================================

\newenvironment{tex}{ %
}
{%
}

\newenvironment{indente}{ %
	\setlength\parindent{10mm}
}

{
	\setlength\parindent{0mm}
}

\newenvironment{corrige}{%
     \needspace{3\baselineskip}
     \medskip
     \textbf{\textsc{Corrigé}}
     \medskip
}
{
}

\newenvironment{extern}{%
     \begin{center}
     }
     {
     \end{center}
}

\NewEnviron{code}{%
	\par
     \boite{gray}{\texttt{%
     \BODY
     }}
     \par
}

\newenvironment{vbloc}{% boite sans cadre empeche saut de page
     \begin{minipage}[t]{\linewidth}
     }
     {
     \end{minipage}
}
\NewEnviron{h2}{%
    \needspace{3\baselineskip}
    \vspace{0.6cm}
	\noindent \MakeUppercase{\sffamily \large \BODY}
	\vspace{1mm}\textcolor{mcgris}{\hrule}\vspace{0.4cm}
	\par
}{}

\NewEnviron{h3}{%
    \needspace{3\baselineskip}
	\vspace{5mm}
	\textsc{\BODY}
	\par
}

\NewEnviron{margeneg}{ %
\begin{addmargin}[-1cm]{0cm}
\BODY
\end{addmargin}
}

\NewEnviron{html}{%
}

\begin{document}
\meta{url}{/exercices/qcm-bac-s-centres-etrangers-2018/}
\meta{pid}{8604}
\meta{titre}{QCM - Bac S Centres étrangers  2018}
\meta{type}{exercices}
%
\begin{h2}Exercice 2 (4 points)\end{h2}
\textbf{Commun à  tous les candidats}
\medbreak
Pour chacune des quatre affirmations suivantes, indiquer si elle est vraie ou fausse, en justifiant la
réponse. Il est attribué un point par réponse exacte correctement justifiée. Une réponse inexacte ou
non justifiée ne rapporte ni n'enlève aucun point.
\par
\begin{enumerate}
     \item Un type d'oscilloscope a une durée de vie, exprimée en année, qui peut être modélisée par une
     variable aléatoire $D$ qui suit une loi exponentielle de paramètre $\lambda$.
     \par
     On sait que la durée de vie moyenne de ce type d'oscilloscope est de $8$ ans.
     \par
     \textbf{Affirmation 1~:} pour un oscilloscope de ce type choisi au hasard et ayant déjà fonctionné $3$ ans,
     la probabilité que la durée de vie soit supérieure ou égale à $10$ ans, arrondie au centième, est
     égale à $0,42$.
     \par
     \emph{On rappelle que si $X$ est une variable aléatoire qui suit une loi exponentielle de paramètre $\lambda$, on a pour tout réel $t$ positif~:}
     \begin{center}
          $P(X \leqslant t) = 1 - \text{e}^{-\lambda t}$.
     \end{center}
     \item  En 2016, en France, les forces de l'ordre ont réalisé $9,8$ millions de dépistages d'alcoolémie
     auprès des automobilistes, et 3,1\,\% de ces dépistages étaient positifs.
     \par
     Source~: \emph{OFDT (Observatoire Français des Drogues et des Toxicomanies)}
     \par
     Dans une région donnée, le 15 juin 2016, une brigade de gendarmerie a effectué un dépistage
     sur $200$ automobilistes.
     \par
     \textbf{Affirmation 2~:} en arrondissant au centième, la probabilité que, sur les $200$ dépistages, il y ait
     eu strictement plus de $5$ dépistages positifs, est égale à $0,59$.
     \item  On considère dans $\mathbb{R}$ l'équation~:
     \[\ln (6 x - 2) + \ln (2x - 1) = \ln (x).\]
     \par
     \textbf{Affirmation 3~:} l'équation admet deux solutions dans l'intervalle $\left]\dfrac{1}{2}~;~+ \infty\right[$.
     \item  On considère dans $\mathbb{C}$ l'équation~:
     \par
     \[\left(4z^2 - 20z + 37\right)(2z -7 + 2\text{i}) = 0.\]
     \par
     \textbf{Affirmation 4~:} les solutions de l'équation sont les affixes de points appartenant à un même
     cercle de centre le point P d'affixe $2$.
\end{enumerate}

\end{document}
µ
\documentclass[a4paper]{article}

%================================================================================================================================
%
% Packages
%
%================================================================================================================================

\usepackage[T1]{fontenc} 	% pour caractères accentués
\usepackage[utf8]{inputenc}  % encodage utf8
\usepackage[french]{babel}	% langue : français
\usepackage{fourier}			% caractères plus lisibles
\usepackage[dvipsnames]{xcolor} % couleurs
\usepackage{fancyhdr}		% réglage header footer
\usepackage{needspace}		% empêcher sauts de page mal placés
\usepackage{graphicx}		% pour inclure des graphiques
\usepackage{enumitem,cprotect}		% personnalise les listes d'items (nécessaire pour ol, al ...)
\usepackage{hyperref}		% Liens hypertexte
\usepackage{pstricks,pst-all,pst-node,pstricks-add,pst-math,pst-plot,pst-tree,pst-eucl} % pstricks
\usepackage[a4paper,includeheadfoot,top=2cm,left=3cm, bottom=2cm,right=3cm]{geometry} % marges etc.
\usepackage{comment}			% commentaires multilignes
\usepackage{amsmath,environ} % maths (matrices, etc.)
\usepackage{amssymb,makeidx}
\usepackage{bm}				% bold maths
\usepackage{tabularx}		% tableaux
\usepackage{colortbl}		% tableaux en couleur
\usepackage{fontawesome}		% Fontawesome
\usepackage{environ}			% environment with command
\usepackage{fp}				% calculs pour ps-tricks
\usepackage{multido}			% pour ps tricks
\usepackage[np]{numprint}	% formattage nombre
\usepackage{tikz,tkz-tab} 			% package principal TikZ
\usepackage{pgfplots}   % axes
\usepackage{mathrsfs}    % cursives
\usepackage{calc}			% calcul taille boites
\usepackage[scaled=0.875]{helvet} % font sans serif
\usepackage{svg} % svg
\usepackage{scrextend} % local margin
\usepackage{scratch} %scratch
\usepackage{multicol} % colonnes
%\usepackage{infix-RPN,pst-func} % formule en notation polanaise inversée
\usepackage{listings}

%================================================================================================================================
%
% Réglages de base
%
%================================================================================================================================

\lstset{
language=Python,   % R code
literate=
{á}{{\'a}}1
{à}{{\`a}}1
{ã}{{\~a}}1
{é}{{\'e}}1
{è}{{\`e}}1
{ê}{{\^e}}1
{í}{{\'i}}1
{ó}{{\'o}}1
{õ}{{\~o}}1
{ú}{{\'u}}1
{ü}{{\"u}}1
{ç}{{\c{c}}}1
{~}{{ }}1
}


\definecolor{codegreen}{rgb}{0,0.6,0}
\definecolor{codegray}{rgb}{0.5,0.5,0.5}
\definecolor{codepurple}{rgb}{0.58,0,0.82}
\definecolor{backcolour}{rgb}{0.95,0.95,0.92}

\lstdefinestyle{mystyle}{
    backgroundcolor=\color{backcolour},   
    commentstyle=\color{codegreen},
    keywordstyle=\color{magenta},
    numberstyle=\tiny\color{codegray},
    stringstyle=\color{codepurple},
    basicstyle=\ttfamily\footnotesize,
    breakatwhitespace=false,         
    breaklines=true,                 
    captionpos=b,                    
    keepspaces=true,                 
    numbers=left,                    
xleftmargin=2em,
framexleftmargin=2em,            
    showspaces=false,                
    showstringspaces=false,
    showtabs=false,                  
    tabsize=2,
    upquote=true
}

\lstset{style=mystyle}


\lstset{style=mystyle}
\newcommand{\imgdir}{C:/laragon/www/newmc/assets/imgsvg/}
\newcommand{\imgsvgdir}{C:/laragon/www/newmc/assets/imgsvg/}

\definecolor{mcgris}{RGB}{220, 220, 220}% ancien~; pour compatibilité
\definecolor{mcbleu}{RGB}{52, 152, 219}
\definecolor{mcvert}{RGB}{125, 194, 70}
\definecolor{mcmauve}{RGB}{154, 0, 215}
\definecolor{mcorange}{RGB}{255, 96, 0}
\definecolor{mcturquoise}{RGB}{0, 153, 153}
\definecolor{mcrouge}{RGB}{255, 0, 0}
\definecolor{mclightvert}{RGB}{205, 234, 190}

\definecolor{gris}{RGB}{220, 220, 220}
\definecolor{bleu}{RGB}{52, 152, 219}
\definecolor{vert}{RGB}{125, 194, 70}
\definecolor{mauve}{RGB}{154, 0, 215}
\definecolor{orange}{RGB}{255, 96, 0}
\definecolor{turquoise}{RGB}{0, 153, 153}
\definecolor{rouge}{RGB}{255, 0, 0}
\definecolor{lightvert}{RGB}{205, 234, 190}
\setitemize[0]{label=\color{lightvert}  $\bullet$}

\pagestyle{fancy}
\renewcommand{\headrulewidth}{0.2pt}
\fancyhead[L]{maths-cours.fr}
\fancyhead[R]{\thepage}
\renewcommand{\footrulewidth}{0.2pt}
\fancyfoot[C]{}

\newcolumntype{C}{>{\centering\arraybackslash}X}
\newcolumntype{s}{>{\hsize=.35\hsize\arraybackslash}X}

\setlength{\parindent}{0pt}		 
\setlength{\parskip}{3mm}
\setlength{\headheight}{1cm}

\def\ebook{ebook}
\def\book{book}
\def\web{web}
\def\type{web}

\newcommand{\vect}[1]{\overrightarrow{\,\mathstrut#1\,}}

\def\Oij{$\left(\text{O}~;~\vect{\imath},~\vect{\jmath}\right)$}
\def\Oijk{$\left(\text{O}~;~\vect{\imath},~\vect{\jmath},~\vect{k}\right)$}
\def\Ouv{$\left(\text{O}~;~\vect{u},~\vect{v}\right)$}

\hypersetup{breaklinks=true, colorlinks = true, linkcolor = OliveGreen, urlcolor = OliveGreen, citecolor = OliveGreen, pdfauthor={Didier BONNEL - https://www.maths-cours.fr} } % supprime les bordures autour des liens

\renewcommand{\arg}[0]{\text{arg}}

\everymath{\displaystyle}

%================================================================================================================================
%
% Macros - Commandes
%
%================================================================================================================================

\newcommand\meta[2]{    			% Utilisé pour créer le post HTML.
	\def\titre{titre}
	\def\url{url}
	\def\arg{#1}
	\ifx\titre\arg
		\newcommand\maintitle{#2}
		\fancyhead[L]{#2}
		{\Large\sffamily \MakeUppercase{#2}}
		\vspace{1mm}\textcolor{mcvert}{\hrule}
	\fi 
	\ifx\url\arg
		\fancyfoot[L]{\href{https://www.maths-cours.fr#2}{\black \footnotesize{https://www.maths-cours.fr#2}}}
	\fi 
}


\newcommand\TitreC[1]{    		% Titre centré
     \needspace{3\baselineskip}
     \begin{center}\textbf{#1}\end{center}
}

\newcommand\newpar{    		% paragraphe
     \par
}

\newcommand\nosp {    		% commande vide (pas d'espace)
}
\newcommand{\id}[1]{} %ignore

\newcommand\boite[2]{				% Boite simple sans titre
	\vspace{5mm}
	\setlength{\fboxrule}{0.2mm}
	\setlength{\fboxsep}{5mm}	
	\fcolorbox{#1}{#1!3}{\makebox[\linewidth-2\fboxrule-2\fboxsep]{
  		\begin{minipage}[t]{\linewidth-2\fboxrule-4\fboxsep}\setlength{\parskip}{3mm}
  			 #2
  		\end{minipage}
	}}
	\vspace{5mm}
}

\newcommand\CBox[4]{				% Boites
	\vspace{5mm}
	\setlength{\fboxrule}{0.2mm}
	\setlength{\fboxsep}{5mm}
	
	\fcolorbox{#1}{#1!3}{\makebox[\linewidth-2\fboxrule-2\fboxsep]{
		\begin{minipage}[t]{1cm}\setlength{\parskip}{3mm}
	  		\textcolor{#1}{\LARGE{#2}}    
 	 	\end{minipage}  
  		\begin{minipage}[t]{\linewidth-2\fboxrule-4\fboxsep}\setlength{\parskip}{3mm}
			\raisebox{1.2mm}{\normalsize\sffamily{\textcolor{#1}{#3}}}						
  			 #4
  		\end{minipage}
	}}
	\vspace{5mm}
}

\newcommand\cadre[3]{				% Boites convertible html
	\par
	\vspace{2mm}
	\setlength{\fboxrule}{0.1mm}
	\setlength{\fboxsep}{5mm}
	\fcolorbox{#1}{white}{\makebox[\linewidth-2\fboxrule-2\fboxsep]{
  		\begin{minipage}[t]{\linewidth-2\fboxrule-4\fboxsep}\setlength{\parskip}{3mm}
			\raisebox{-2.5mm}{\sffamily \small{\textcolor{#1}{\MakeUppercase{#2}}}}		
			\par		
  			 #3
 	 		\end{minipage}
	}}
		\vspace{2mm}
	\par
}

\newcommand\bloc[3]{				% Boites convertible html sans bordure
     \needspace{2\baselineskip}
     {\sffamily \small{\textcolor{#1}{\MakeUppercase{#2}}}}    
		\par		
  			 #3
		\par
}

\newcommand\CHelp[1]{
     \CBox{Plum}{\faInfoCircle}{À RETENIR}{#1}
}

\newcommand\CUp[1]{
     \CBox{NavyBlue}{\faThumbsOUp}{EN PRATIQUE}{#1}
}

\newcommand\CInfo[1]{
     \CBox{Sepia}{\faArrowCircleRight}{REMARQUE}{#1}
}

\newcommand\CRedac[1]{
     \CBox{PineGreen}{\faEdit}{BIEN R\'EDIGER}{#1}
}

\newcommand\CError[1]{
     \CBox{Red}{\faExclamationTriangle}{ATTENTION}{#1}
}

\newcommand\TitreExo[2]{
\needspace{4\baselineskip}
 {\sffamily\large EXERCICE #1\ (\emph{#2 points})}
\vspace{5mm}
}

\newcommand\img[2]{
          \includegraphics[width=#2\paperwidth]{\imgdir#1}
}

\newcommand\imgsvg[2]{
       \begin{center}   \includegraphics[width=#2\paperwidth]{\imgsvgdir#1} \end{center}
}


\newcommand\Lien[2]{
     \href{#1}{#2 \tiny \faExternalLink}
}
\newcommand\mcLien[2]{
     \href{https~://www.maths-cours.fr/#1}{#2 \tiny \faExternalLink}
}

\newcommand{\euro}{\eurologo{}}

%================================================================================================================================
%
% Macros - Environement
%
%================================================================================================================================

\newenvironment{tex}{ %
}
{%
}

\newenvironment{indente}{ %
	\setlength\parindent{10mm}
}

{
	\setlength\parindent{0mm}
}

\newenvironment{corrige}{%
     \needspace{3\baselineskip}
     \medskip
     \textbf{\textsc{Corrigé}}
     \medskip
}
{
}

\newenvironment{extern}{%
     \begin{center}
     }
     {
     \end{center}
}

\NewEnviron{code}{%
	\par
     \boite{gray}{\texttt{%
     \BODY
     }}
     \par
}

\newenvironment{vbloc}{% boite sans cadre empeche saut de page
     \begin{minipage}[t]{\linewidth}
     }
     {
     \end{minipage}
}
\NewEnviron{h2}{%
    \needspace{3\baselineskip}
    \vspace{0.6cm}
	\noindent \MakeUppercase{\sffamily \large \BODY}
	\vspace{1mm}\textcolor{mcgris}{\hrule}\vspace{0.4cm}
	\par
}{}

\NewEnviron{h3}{%
    \needspace{3\baselineskip}
	\vspace{5mm}
	\textsc{\BODY}
	\par
}

\NewEnviron{margeneg}{ %
\begin{addmargin}[-1cm]{0cm}
\BODY
\end{addmargin}
}

\NewEnviron{html}{%
}

\begin{document}
\meta{url}{/exercices/probabilites-bac-s-centres-etrangers-2018/}
\meta{pid}{8610}
\meta{titre}{Probabilités - Bac S Centres étrangers  2018}
\meta{type}{exercices}
%
\begin{h2}Exercice 3 (7 points)\end{h2}
\textbf{Commun à  tous les candidats}
\medskip
\emph{Les parties A  et B sont indépendantes}
\medskip
Un détaillant en fruits et légumes étudie l'évolution de ses ventes de melons afin de pouvoir
anticiper ses commandes.
\bigskip
\TitreC{Partie A}
\medskip
Le détaillant constate que ses melons se vendent bien lorsque leur masse est comprise entre $900$ g et
1~200~g. Dans la suite, de tels melons sont qualifiés \og conformes \fg.
\par
Le détaillant achète ses melons auprès de trois maraîchers, notés respectivement A, B et C.
\par
Pour les melons du maraîcher A, on modélise la masse en gramme par une variable aléatoire $M_{\text{A}}$
qui suit une loi uniforme sur l'intervalle $[850~;~x]$, où $x$ est un nombre réel supérieur à 1~200.
\par
La masse en gramme des melons du maraîcher B est modélisée par une variable aléatoire $M_{\text{B}}$ qui
suit une loi normale de moyenne 1~050 et d'écart-type inconnu $\sigma$.
\par
Le maraîcher C affirme, quant à lui, que 80\,\% des melons de sa production sont conformes.
\medskip
\begin{enumerate}
     \item Le détaillant constate que 75\,\% des melons du maraîcher A sont conformes. Déterminer $x$.
     \item Il constate que 85\,\% des melons fournis par le maraîcher B sont conformes.
     \par
     Déterminer l'écart-type $\sigma$ de la variable aléatoire $M_{\text{B}}$. En donner la valeur arrondie à l'unité.
     \item  Le détaillant doute de l'affirmation du maraîcher C. Il constate que sur $400$ melons livrés par ce
     maraîcher au cours d'une semaine, seulement $294$ sont conformes.
     \par
     Le détaillant a-t-il raison de douter de l'affirmation du maraîcher C~?
\end{enumerate}
\bigskip
\TitreC{Partie B}
\medskip
Le détaillant réalise une étude sur ses clients. Il constate que~:
\begin{itemize}
     \item parmi les clients qui achètent un melon une semaine donnée, 90\,\% d'entre eux achètent un
     melon la semaine suivante~;
     \item parmi les clients qui n'achètent pas de melon une semaine donnée, 60\,\% d'entre eux n'achètent
     pas de melon la semaine suivante.
\end{itemize}
\smallskip
On choisit au hasard un client ayant acheté un melon au cours de la semaine 1 et, pour $n \geqslant 1$, on
note $A_n$ l'événement~: \og le client achète un melon au cours de la semaine $n$ \fg.
\par
On a ainsi $p\left(A_1\right) = 1$.
\medskip
\begin{enumerate}
     \item
     \begin{enumerate}[label=\alph*.]
          \item Reproduire et compléter l'arbre de probabilités
          ci-dessous, relatif aux trois premières semaines.
          \begin{center}
               \begin{extern}%width="220" alt="Arbre Probabilité Bac S Centres étrangers  2018"
                    \pstree[treemode=R,nodesepA=0pt,nodesepB=3pt]{\TR{$A_1$~}}
                    {
                         \pstree{\TR{$A_2$~}}
                         {
                              \TR{$A_3$}
                              \TR{$\overline{A_3}$}
                         }
                         \pstree{\TR{$\overline{A_2}$~} }
                         {
                              \TR{$A_3$}
                              \TR{$\overline{A_3}$}
                         }
                    }
               \end{extern}
          \end{center}
          \item Démontrer que $p\left(A_3\right) = 0,85$.
          \item Sachant que le client achète un melon au cours
          de la semaine 3, quelle est la probabilité qu'il en ait acheté un au cours de la semaine 2~?
          \par
          Arrondir au centième.
     \end{enumerate}
     \medskip
     Dans la suite, on pose pour tout entier $n \geqslant 1$~: \:$p_n = P\left(A_n\right)$. \\
     On a ainsi $p_1 = 1$.
     \medskip
     \item Démontrer que, pour tout entier $n \geqslant 1$~: $p_{n+1} = 0,5p_n + 0,4$.
     \item
     \begin{enumerate}[label=\alph*.]
          \item Démontrer par récurrence que, pour tout entier $n \geqslant 1$~: $p_n > 0,8$.
          \item  Démontrer que la suite $(p_n)$ est décroissante.
          \item  La suite $\left(p_n\right)$ est-elle convergente~?
     \end{enumerate}
     \item On pose pour tout entier $n \geqslant 1$~: $v_n = p_n - 0,8$.
     \begin{enumerate}[label=\alph*.]
          \item Démontrer que $\left(v_n\right)$ est une suite géométrique dont on donnera le premier terme $v_1$ et la raison.
          \item  Exprimer $v_n$ en fonction de $n$.
          \par
          En déduire que, pour tout $n \geqslant 1$,\: $p_n = 0,8 + 0,2 \times  0,5^{n-1}$.
          \item  Déterminer la limite de la suite $\left(p_n\right)$.
     \end{enumerate}
\end{enumerate}

\end{document}
µ
\documentclass[a4paper]{article}

%================================================================================================================================
%
% Packages
%
%================================================================================================================================

\usepackage[T1]{fontenc} 	% pour caractères accentués
\usepackage[utf8]{inputenc}  % encodage utf8
\usepackage[french]{babel}	% langue : français
\usepackage{fourier}			% caractères plus lisibles
\usepackage[dvipsnames]{xcolor} % couleurs
\usepackage{fancyhdr}		% réglage header footer
\usepackage{needspace}		% empêcher sauts de page mal placés
\usepackage{graphicx}		% pour inclure des graphiques
\usepackage{enumitem,cprotect}		% personnalise les listes d'items (nécessaire pour ol, al ...)
\usepackage{hyperref}		% Liens hypertexte
\usepackage{pstricks,pst-all,pst-node,pstricks-add,pst-math,pst-plot,pst-tree,pst-eucl} % pstricks
\usepackage[a4paper,includeheadfoot,top=2cm,left=3cm, bottom=2cm,right=3cm]{geometry} % marges etc.
\usepackage{comment}			% commentaires multilignes
\usepackage{amsmath,environ} % maths (matrices, etc.)
\usepackage{amssymb,makeidx}
\usepackage{bm}				% bold maths
\usepackage{tabularx}		% tableaux
\usepackage{colortbl}		% tableaux en couleur
\usepackage{fontawesome}		% Fontawesome
\usepackage{environ}			% environment with command
\usepackage{fp}				% calculs pour ps-tricks
\usepackage{multido}			% pour ps tricks
\usepackage[np]{numprint}	% formattage nombre
\usepackage{tikz,tkz-tab} 			% package principal TikZ
\usepackage{pgfplots}   % axes
\usepackage{mathrsfs}    % cursives
\usepackage{calc}			% calcul taille boites
\usepackage[scaled=0.875]{helvet} % font sans serif
\usepackage{svg} % svg
\usepackage{scrextend} % local margin
\usepackage{scratch} %scratch
\usepackage{multicol} % colonnes
%\usepackage{infix-RPN,pst-func} % formule en notation polanaise inversée
\usepackage{listings}

%================================================================================================================================
%
% Réglages de base
%
%================================================================================================================================

\lstset{
language=Python,   % R code
literate=
{á}{{\'a}}1
{à}{{\`a}}1
{ã}{{\~a}}1
{é}{{\'e}}1
{è}{{\`e}}1
{ê}{{\^e}}1
{í}{{\'i}}1
{ó}{{\'o}}1
{õ}{{\~o}}1
{ú}{{\'u}}1
{ü}{{\"u}}1
{ç}{{\c{c}}}1
{~}{{ }}1
}


\definecolor{codegreen}{rgb}{0,0.6,0}
\definecolor{codegray}{rgb}{0.5,0.5,0.5}
\definecolor{codepurple}{rgb}{0.58,0,0.82}
\definecolor{backcolour}{rgb}{0.95,0.95,0.92}

\lstdefinestyle{mystyle}{
    backgroundcolor=\color{backcolour},   
    commentstyle=\color{codegreen},
    keywordstyle=\color{magenta},
    numberstyle=\tiny\color{codegray},
    stringstyle=\color{codepurple},
    basicstyle=\ttfamily\footnotesize,
    breakatwhitespace=false,         
    breaklines=true,                 
    captionpos=b,                    
    keepspaces=true,                 
    numbers=left,                    
xleftmargin=2em,
framexleftmargin=2em,            
    showspaces=false,                
    showstringspaces=false,
    showtabs=false,                  
    tabsize=2,
    upquote=true
}

\lstset{style=mystyle}


\lstset{style=mystyle}
\newcommand{\imgdir}{C:/laragon/www/newmc/assets/imgsvg/}
\newcommand{\imgsvgdir}{C:/laragon/www/newmc/assets/imgsvg/}

\definecolor{mcgris}{RGB}{220, 220, 220}% ancien~; pour compatibilité
\definecolor{mcbleu}{RGB}{52, 152, 219}
\definecolor{mcvert}{RGB}{125, 194, 70}
\definecolor{mcmauve}{RGB}{154, 0, 215}
\definecolor{mcorange}{RGB}{255, 96, 0}
\definecolor{mcturquoise}{RGB}{0, 153, 153}
\definecolor{mcrouge}{RGB}{255, 0, 0}
\definecolor{mclightvert}{RGB}{205, 234, 190}

\definecolor{gris}{RGB}{220, 220, 220}
\definecolor{bleu}{RGB}{52, 152, 219}
\definecolor{vert}{RGB}{125, 194, 70}
\definecolor{mauve}{RGB}{154, 0, 215}
\definecolor{orange}{RGB}{255, 96, 0}
\definecolor{turquoise}{RGB}{0, 153, 153}
\definecolor{rouge}{RGB}{255, 0, 0}
\definecolor{lightvert}{RGB}{205, 234, 190}
\setitemize[0]{label=\color{lightvert}  $\bullet$}

\pagestyle{fancy}
\renewcommand{\headrulewidth}{0.2pt}
\fancyhead[L]{maths-cours.fr}
\fancyhead[R]{\thepage}
\renewcommand{\footrulewidth}{0.2pt}
\fancyfoot[C]{}

\newcolumntype{C}{>{\centering\arraybackslash}X}
\newcolumntype{s}{>{\hsize=.35\hsize\arraybackslash}X}

\setlength{\parindent}{0pt}		 
\setlength{\parskip}{3mm}
\setlength{\headheight}{1cm}

\def\ebook{ebook}
\def\book{book}
\def\web{web}
\def\type{web}

\newcommand{\vect}[1]{\overrightarrow{\,\mathstrut#1\,}}

\def\Oij{$\left(\text{O}~;~\vect{\imath},~\vect{\jmath}\right)$}
\def\Oijk{$\left(\text{O}~;~\vect{\imath},~\vect{\jmath},~\vect{k}\right)$}
\def\Ouv{$\left(\text{O}~;~\vect{u},~\vect{v}\right)$}

\hypersetup{breaklinks=true, colorlinks = true, linkcolor = OliveGreen, urlcolor = OliveGreen, citecolor = OliveGreen, pdfauthor={Didier BONNEL - https://www.maths-cours.fr} } % supprime les bordures autour des liens

\renewcommand{\arg}[0]{\text{arg}}

\everymath{\displaystyle}

%================================================================================================================================
%
% Macros - Commandes
%
%================================================================================================================================

\newcommand\meta[2]{    			% Utilisé pour créer le post HTML.
	\def\titre{titre}
	\def\url{url}
	\def\arg{#1}
	\ifx\titre\arg
		\newcommand\maintitle{#2}
		\fancyhead[L]{#2}
		{\Large\sffamily \MakeUppercase{#2}}
		\vspace{1mm}\textcolor{mcvert}{\hrule}
	\fi 
	\ifx\url\arg
		\fancyfoot[L]{\href{https://www.maths-cours.fr#2}{\black \footnotesize{https://www.maths-cours.fr#2}}}
	\fi 
}


\newcommand\TitreC[1]{    		% Titre centré
     \needspace{3\baselineskip}
     \begin{center}\textbf{#1}\end{center}
}

\newcommand\newpar{    		% paragraphe
     \par
}

\newcommand\nosp {    		% commande vide (pas d'espace)
}
\newcommand{\id}[1]{} %ignore

\newcommand\boite[2]{				% Boite simple sans titre
	\vspace{5mm}
	\setlength{\fboxrule}{0.2mm}
	\setlength{\fboxsep}{5mm}	
	\fcolorbox{#1}{#1!3}{\makebox[\linewidth-2\fboxrule-2\fboxsep]{
  		\begin{minipage}[t]{\linewidth-2\fboxrule-4\fboxsep}\setlength{\parskip}{3mm}
  			 #2
  		\end{minipage}
	}}
	\vspace{5mm}
}

\newcommand\CBox[4]{				% Boites
	\vspace{5mm}
	\setlength{\fboxrule}{0.2mm}
	\setlength{\fboxsep}{5mm}
	
	\fcolorbox{#1}{#1!3}{\makebox[\linewidth-2\fboxrule-2\fboxsep]{
		\begin{minipage}[t]{1cm}\setlength{\parskip}{3mm}
	  		\textcolor{#1}{\LARGE{#2}}    
 	 	\end{minipage}  
  		\begin{minipage}[t]{\linewidth-2\fboxrule-4\fboxsep}\setlength{\parskip}{3mm}
			\raisebox{1.2mm}{\normalsize\sffamily{\textcolor{#1}{#3}}}						
  			 #4
  		\end{minipage}
	}}
	\vspace{5mm}
}

\newcommand\cadre[3]{				% Boites convertible html
	\par
	\vspace{2mm}
	\setlength{\fboxrule}{0.1mm}
	\setlength{\fboxsep}{5mm}
	\fcolorbox{#1}{white}{\makebox[\linewidth-2\fboxrule-2\fboxsep]{
  		\begin{minipage}[t]{\linewidth-2\fboxrule-4\fboxsep}\setlength{\parskip}{3mm}
			\raisebox{-2.5mm}{\sffamily \small{\textcolor{#1}{\MakeUppercase{#2}}}}		
			\par		
  			 #3
 	 		\end{minipage}
	}}
		\vspace{2mm}
	\par
}

\newcommand\bloc[3]{				% Boites convertible html sans bordure
     \needspace{2\baselineskip}
     {\sffamily \small{\textcolor{#1}{\MakeUppercase{#2}}}}    
		\par		
  			 #3
		\par
}

\newcommand\CHelp[1]{
     \CBox{Plum}{\faInfoCircle}{À RETENIR}{#1}
}

\newcommand\CUp[1]{
     \CBox{NavyBlue}{\faThumbsOUp}{EN PRATIQUE}{#1}
}

\newcommand\CInfo[1]{
     \CBox{Sepia}{\faArrowCircleRight}{REMARQUE}{#1}
}

\newcommand\CRedac[1]{
     \CBox{PineGreen}{\faEdit}{BIEN R\'EDIGER}{#1}
}

\newcommand\CError[1]{
     \CBox{Red}{\faExclamationTriangle}{ATTENTION}{#1}
}

\newcommand\TitreExo[2]{
\needspace{4\baselineskip}
 {\sffamily\large EXERCICE #1\ (\emph{#2 points})}
\vspace{5mm}
}

\newcommand\img[2]{
          \includegraphics[width=#2\paperwidth]{\imgdir#1}
}

\newcommand\imgsvg[2]{
       \begin{center}   \includegraphics[width=#2\paperwidth]{\imgsvgdir#1} \end{center}
}


\newcommand\Lien[2]{
     \href{#1}{#2 \tiny \faExternalLink}
}
\newcommand\mcLien[2]{
     \href{https~://www.maths-cours.fr/#1}{#2 \tiny \faExternalLink}
}

\newcommand{\euro}{\eurologo{}}

%================================================================================================================================
%
% Macros - Environement
%
%================================================================================================================================

\newenvironment{tex}{ %
}
{%
}

\newenvironment{indente}{ %
	\setlength\parindent{10mm}
}

{
	\setlength\parindent{0mm}
}

\newenvironment{corrige}{%
     \needspace{3\baselineskip}
     \medskip
     \textbf{\textsc{Corrigé}}
     \medskip
}
{
}

\newenvironment{extern}{%
     \begin{center}
     }
     {
     \end{center}
}

\NewEnviron{code}{%
	\par
     \boite{gray}{\texttt{%
     \BODY
     }}
     \par
}

\newenvironment{vbloc}{% boite sans cadre empeche saut de page
     \begin{minipage}[t]{\linewidth}
     }
     {
     \end{minipage}
}
\NewEnviron{h2}{%
    \needspace{3\baselineskip}
    \vspace{0.6cm}
	\noindent \MakeUppercase{\sffamily \large \BODY}
	\vspace{1mm}\textcolor{mcgris}{\hrule}\vspace{0.4cm}
	\par
}{}

\NewEnviron{h3}{%
    \needspace{3\baselineskip}
	\vspace{5mm}
	\textsc{\BODY}
	\par
}

\NewEnviron{margeneg}{ %
\begin{addmargin}[-1cm]{0cm}
\BODY
\end{addmargin}
}

\NewEnviron{html}{%
}

\begin{document}
\meta{url}{/exercices/geometrie-dans-lespace-bac-s-centres-etrangers-2018/}
\meta{pid}{8625}
\meta{titre}{Géométrie dans l'espace - Bac S Centres étrangers  2018}
\meta{type}{exercices}
%
\begin{h2}Exercice 4 (5 points)\end{h2}
\textbf{Candidats n'ayant pas choisi la spécialité \og mathématiques \fg{} }
\medskip
La figure ci-dessous représente un cube ABCDEFGH.
\begin{center}
     \begin{extern} %width="250" alt="Géométrie dans l'espace Section d'un cube"
          \psset{unit=0.8cm}
          \begin{pspicture}(-0.5,-0.5)(8,7.8)
               \psframe(0,0)(4.5,4.5)%ABFE
               \psline(4.5,0)(6.7,2.3)(6.7,6.8)(4.5,4.5)%BCGF
               \psline(6.7,6.8)(2.2,6.8)(0,4.5)%GHE
               \psline[linestyle=dashed](0,0)(2.2,2.3)(6.7,2.3)
               \psline[linestyle=dashed](2.2,2.3)(2.2,6.8)
               \uput[dl](0,0){A} \uput[dr](4.5,0){B} \uput[r](6.7,2.3){C}
               \uput[ur](2.2,2.3){D} \uput[l](0,4.5){E} \uput[r](4.5,4.5){F}
               \uput[r](6.7,6.8){G} \uput[u](2.2,6.8){H} \uput[ul](1.1,1.15){I}
               \uput[l](0,3.375){J} \uput[dr](5.6,5.65){K}
               \psdots(1.1,1.15)(0,3.375)(5.6,5.65)
          \end{pspicture}
     \end{extern}
\end{center}
Les trois points I, J, K sont définis par les conditions
suivantes~:
\begin{itemize}
     \item I est le milieu du segment [AD]~;
     \item J est tel que $\overrightarrow{\text{AJ}} = \dfrac{3}{4} \overrightarrow{\text{AE}}$~;
     \item K est le milieu du segment [FG].
\end{itemize}
\bigskip
\TitreC{Partie A}
\medskip
\begin{enumerate}
     \item Sur la figure donnée en annexe, construire sans justifier le point d'intersection P du plan (IJK) et
     de la droite (EH). On laissera les traits de construction sur la figure.
     \item  En déduire, en justifiant, l'intersection du plan (IJK) et du plan (EFG).
\end{enumerate}
\bigskip
\TitreC{Partie B}
\medskip
On se place désormais dans le repère orthonormé $\left(\text{A}~;~\overrightarrow{\text{AB}}, \overrightarrow{\text{AD}}, \overrightarrow{\text{AE}}\right)$.
\medskip
\begin{enumerate}
     \item
     \begin{enumerate}[label=\alph*.]
          \item Donner sans justification les coordonnées des points I, J et K.
          \item Déterminer les réels $a$ et $b$ tels que le vecteur $\overrightarrow{n} (4~;~a~;~b)$ soit orthogonal aux vecteurs $\overrightarrow{\text{IJ}}$ et $\overrightarrow{\text{IK}}$.
          \item  En déduire qu'une équation cartésienne du plan (IJK) est~: $4x - 6y - 4z + 3 = 0$.
     \end{enumerate}
     \item
     \begin{enumerate}[label=\alph*.]
          \item Donner une représentation paramétrique de la droite (CG).
          \item Calculer les coordonnées du point N, intersection du plan (IJK) et de la droite (CG).
          \item Placer le point N sur la figure et construire en couleur la section du cube par le plan (IJK).
     \end{enumerate}
\end{enumerate}
\bigskip
\TitreC{Partie C}
\medskip
On note R le projeté orthogonal du point F sur le plan (IJK). Le point R est donc l'unique point du
plan (IJK) tel que la droite (FR) est orthogonale au plan (IJK).
\par
On définit l'intérieur du cube comme l'ensemble des points $M(x~;~y~;~z)$ tels que :
\begin{center}
     $\left\{\begin{array}{l}
               0 < x < 1\\
               0 < y < 1\\
               0 < z < 1
     \end{array}\right.$
\end{center}
Le point R est-il à l'intérieur du cube~?
\bigskip
\newpage
\TitreC{Annexe}
\begin{center}
     \textit{(À rendre avec la copie)}
\end{center}
\bigskip
\begin{center}
     \begin{extern} %width="400" alt="Géométrie dans l'espace Section d'un cube"
          \psset{unit=1.2cm}
          \begin{pspicture}(-0.5,-0.5)(8,7.8)
               \psframe(0,0)(4.5,4.5)%ABFE
               \psline(4.5,0)(6.7,2.3)(6.7,6.8)(4.5,4.5)%BCGF
               \psline(6.7,6.8)(2.2,6.8)(0,4.5)%GHE
               \psline[linestyle=dashed](0,0)(2.2,2.3)(6.7,2.3)
               \psline[linestyle=dashed](2.2,2.3)(2.2,6.8)
               \uput[dl](0,0){A} \uput[dr](4.5,0){B} \uput[r](6.7,2.3){C}
               \uput[ur](2.2,2.3){D} \uput[l](0,4.5){E} \uput[r](4.5,4.5){F}
               \uput[r](6.7,6.8){G} \uput[u](2.2,6.8){H} \uput[ul](1.1,1.15){I}
               \uput[l](0,3.375){J} \uput[dr](5.6,5.65){K}
               \psdots(1.1,1.15)(0,3.375)(5.6,5.65)
          \end{pspicture}
     \end{extern}
\end{center}

\end{document}
µ
\documentclass[a4paper]{article}

%================================================================================================================================
%
% Packages
%
%================================================================================================================================

\usepackage[T1]{fontenc} 	% pour caractères accentués
\usepackage[utf8]{inputenc}  % encodage utf8
\usepackage[french]{babel}	% langue : français
\usepackage{fourier}			% caractères plus lisibles
\usepackage[dvipsnames]{xcolor} % couleurs
\usepackage{fancyhdr}		% réglage header footer
\usepackage{needspace}		% empêcher sauts de page mal placés
\usepackage{graphicx}		% pour inclure des graphiques
\usepackage{enumitem,cprotect}		% personnalise les listes d'items (nécessaire pour ol, al ...)
\usepackage{hyperref}		% Liens hypertexte
\usepackage{pstricks,pst-all,pst-node,pstricks-add,pst-math,pst-plot,pst-tree,pst-eucl} % pstricks
\usepackage[a4paper,includeheadfoot,top=2cm,left=3cm, bottom=2cm,right=3cm]{geometry} % marges etc.
\usepackage{comment}			% commentaires multilignes
\usepackage{amsmath,environ} % maths (matrices, etc.)
\usepackage{amssymb,makeidx}
\usepackage{bm}				% bold maths
\usepackage{tabularx}		% tableaux
\usepackage{colortbl}		% tableaux en couleur
\usepackage{fontawesome}		% Fontawesome
\usepackage{environ}			% environment with command
\usepackage{fp}				% calculs pour ps-tricks
\usepackage{multido}			% pour ps tricks
\usepackage[np]{numprint}	% formattage nombre
\usepackage{tikz,tkz-tab} 			% package principal TikZ
\usepackage{pgfplots}   % axes
\usepackage{mathrsfs}    % cursives
\usepackage{calc}			% calcul taille boites
\usepackage[scaled=0.875]{helvet} % font sans serif
\usepackage{svg} % svg
\usepackage{scrextend} % local margin
\usepackage{scratch} %scratch
\usepackage{multicol} % colonnes
%\usepackage{infix-RPN,pst-func} % formule en notation polanaise inversée
\usepackage{listings}

%================================================================================================================================
%
% Réglages de base
%
%================================================================================================================================

\lstset{
language=Python,   % R code
literate=
{á}{{\'a}}1
{à}{{\`a}}1
{ã}{{\~a}}1
{é}{{\'e}}1
{è}{{\`e}}1
{ê}{{\^e}}1
{í}{{\'i}}1
{ó}{{\'o}}1
{õ}{{\~o}}1
{ú}{{\'u}}1
{ü}{{\"u}}1
{ç}{{\c{c}}}1
{~}{{ }}1
}


\definecolor{codegreen}{rgb}{0,0.6,0}
\definecolor{codegray}{rgb}{0.5,0.5,0.5}
\definecolor{codepurple}{rgb}{0.58,0,0.82}
\definecolor{backcolour}{rgb}{0.95,0.95,0.92}

\lstdefinestyle{mystyle}{
    backgroundcolor=\color{backcolour},   
    commentstyle=\color{codegreen},
    keywordstyle=\color{magenta},
    numberstyle=\tiny\color{codegray},
    stringstyle=\color{codepurple},
    basicstyle=\ttfamily\footnotesize,
    breakatwhitespace=false,         
    breaklines=true,                 
    captionpos=b,                    
    keepspaces=true,                 
    numbers=left,                    
xleftmargin=2em,
framexleftmargin=2em,            
    showspaces=false,                
    showstringspaces=false,
    showtabs=false,                  
    tabsize=2,
    upquote=true
}

\lstset{style=mystyle}


\lstset{style=mystyle}
\newcommand{\imgdir}{C:/laragon/www/newmc/assets/imgsvg/}
\newcommand{\imgsvgdir}{C:/laragon/www/newmc/assets/imgsvg/}

\definecolor{mcgris}{RGB}{220, 220, 220}% ancien~; pour compatibilité
\definecolor{mcbleu}{RGB}{52, 152, 219}
\definecolor{mcvert}{RGB}{125, 194, 70}
\definecolor{mcmauve}{RGB}{154, 0, 215}
\definecolor{mcorange}{RGB}{255, 96, 0}
\definecolor{mcturquoise}{RGB}{0, 153, 153}
\definecolor{mcrouge}{RGB}{255, 0, 0}
\definecolor{mclightvert}{RGB}{205, 234, 190}

\definecolor{gris}{RGB}{220, 220, 220}
\definecolor{bleu}{RGB}{52, 152, 219}
\definecolor{vert}{RGB}{125, 194, 70}
\definecolor{mauve}{RGB}{154, 0, 215}
\definecolor{orange}{RGB}{255, 96, 0}
\definecolor{turquoise}{RGB}{0, 153, 153}
\definecolor{rouge}{RGB}{255, 0, 0}
\definecolor{lightvert}{RGB}{205, 234, 190}
\setitemize[0]{label=\color{lightvert}  $\bullet$}

\pagestyle{fancy}
\renewcommand{\headrulewidth}{0.2pt}
\fancyhead[L]{maths-cours.fr}
\fancyhead[R]{\thepage}
\renewcommand{\footrulewidth}{0.2pt}
\fancyfoot[C]{}

\newcolumntype{C}{>{\centering\arraybackslash}X}
\newcolumntype{s}{>{\hsize=.35\hsize\arraybackslash}X}

\setlength{\parindent}{0pt}		 
\setlength{\parskip}{3mm}
\setlength{\headheight}{1cm}

\def\ebook{ebook}
\def\book{book}
\def\web{web}
\def\type{web}

\newcommand{\vect}[1]{\overrightarrow{\,\mathstrut#1\,}}

\def\Oij{$\left(\text{O}~;~\vect{\imath},~\vect{\jmath}\right)$}
\def\Oijk{$\left(\text{O}~;~\vect{\imath},~\vect{\jmath},~\vect{k}\right)$}
\def\Ouv{$\left(\text{O}~;~\vect{u},~\vect{v}\right)$}

\hypersetup{breaklinks=true, colorlinks = true, linkcolor = OliveGreen, urlcolor = OliveGreen, citecolor = OliveGreen, pdfauthor={Didier BONNEL - https://www.maths-cours.fr} } % supprime les bordures autour des liens

\renewcommand{\arg}[0]{\text{arg}}

\everymath{\displaystyle}

%================================================================================================================================
%
% Macros - Commandes
%
%================================================================================================================================

\newcommand\meta[2]{    			% Utilisé pour créer le post HTML.
	\def\titre{titre}
	\def\url{url}
	\def\arg{#1}
	\ifx\titre\arg
		\newcommand\maintitle{#2}
		\fancyhead[L]{#2}
		{\Large\sffamily \MakeUppercase{#2}}
		\vspace{1mm}\textcolor{mcvert}{\hrule}
	\fi 
	\ifx\url\arg
		\fancyfoot[L]{\href{https://www.maths-cours.fr#2}{\black \footnotesize{https://www.maths-cours.fr#2}}}
	\fi 
}


\newcommand\TitreC[1]{    		% Titre centré
     \needspace{3\baselineskip}
     \begin{center}\textbf{#1}\end{center}
}

\newcommand\newpar{    		% paragraphe
     \par
}

\newcommand\nosp {    		% commande vide (pas d'espace)
}
\newcommand{\id}[1]{} %ignore

\newcommand\boite[2]{				% Boite simple sans titre
	\vspace{5mm}
	\setlength{\fboxrule}{0.2mm}
	\setlength{\fboxsep}{5mm}	
	\fcolorbox{#1}{#1!3}{\makebox[\linewidth-2\fboxrule-2\fboxsep]{
  		\begin{minipage}[t]{\linewidth-2\fboxrule-4\fboxsep}\setlength{\parskip}{3mm}
  			 #2
  		\end{minipage}
	}}
	\vspace{5mm}
}

\newcommand\CBox[4]{				% Boites
	\vspace{5mm}
	\setlength{\fboxrule}{0.2mm}
	\setlength{\fboxsep}{5mm}
	
	\fcolorbox{#1}{#1!3}{\makebox[\linewidth-2\fboxrule-2\fboxsep]{
		\begin{minipage}[t]{1cm}\setlength{\parskip}{3mm}
	  		\textcolor{#1}{\LARGE{#2}}    
 	 	\end{minipage}  
  		\begin{minipage}[t]{\linewidth-2\fboxrule-4\fboxsep}\setlength{\parskip}{3mm}
			\raisebox{1.2mm}{\normalsize\sffamily{\textcolor{#1}{#3}}}						
  			 #4
  		\end{minipage}
	}}
	\vspace{5mm}
}

\newcommand\cadre[3]{				% Boites convertible html
	\par
	\vspace{2mm}
	\setlength{\fboxrule}{0.1mm}
	\setlength{\fboxsep}{5mm}
	\fcolorbox{#1}{white}{\makebox[\linewidth-2\fboxrule-2\fboxsep]{
  		\begin{minipage}[t]{\linewidth-2\fboxrule-4\fboxsep}\setlength{\parskip}{3mm}
			\raisebox{-2.5mm}{\sffamily \small{\textcolor{#1}{\MakeUppercase{#2}}}}		
			\par		
  			 #3
 	 		\end{minipage}
	}}
		\vspace{2mm}
	\par
}

\newcommand\bloc[3]{				% Boites convertible html sans bordure
     \needspace{2\baselineskip}
     {\sffamily \small{\textcolor{#1}{\MakeUppercase{#2}}}}    
		\par		
  			 #3
		\par
}

\newcommand\CHelp[1]{
     \CBox{Plum}{\faInfoCircle}{À RETENIR}{#1}
}

\newcommand\CUp[1]{
     \CBox{NavyBlue}{\faThumbsOUp}{EN PRATIQUE}{#1}
}

\newcommand\CInfo[1]{
     \CBox{Sepia}{\faArrowCircleRight}{REMARQUE}{#1}
}

\newcommand\CRedac[1]{
     \CBox{PineGreen}{\faEdit}{BIEN R\'EDIGER}{#1}
}

\newcommand\CError[1]{
     \CBox{Red}{\faExclamationTriangle}{ATTENTION}{#1}
}

\newcommand\TitreExo[2]{
\needspace{4\baselineskip}
 {\sffamily\large EXERCICE #1\ (\emph{#2 points})}
\vspace{5mm}
}

\newcommand\img[2]{
          \includegraphics[width=#2\paperwidth]{\imgdir#1}
}

\newcommand\imgsvg[2]{
       \begin{center}   \includegraphics[width=#2\paperwidth]{\imgsvgdir#1} \end{center}
}


\newcommand\Lien[2]{
     \href{#1}{#2 \tiny \faExternalLink}
}
\newcommand\mcLien[2]{
     \href{https~://www.maths-cours.fr/#1}{#2 \tiny \faExternalLink}
}

\newcommand{\euro}{\eurologo{}}

%================================================================================================================================
%
% Macros - Environement
%
%================================================================================================================================

\newenvironment{tex}{ %
}
{%
}

\newenvironment{indente}{ %
	\setlength\parindent{10mm}
}

{
	\setlength\parindent{0mm}
}

\newenvironment{corrige}{%
     \needspace{3\baselineskip}
     \medskip
     \textbf{\textsc{Corrigé}}
     \medskip
}
{
}

\newenvironment{extern}{%
     \begin{center}
     }
     {
     \end{center}
}

\NewEnviron{code}{%
	\par
     \boite{gray}{\texttt{%
     \BODY
     }}
     \par
}

\newenvironment{vbloc}{% boite sans cadre empeche saut de page
     \begin{minipage}[t]{\linewidth}
     }
     {
     \end{minipage}
}
\NewEnviron{h2}{%
    \needspace{3\baselineskip}
    \vspace{0.6cm}
	\noindent \MakeUppercase{\sffamily \large \BODY}
	\vspace{1mm}\textcolor{mcgris}{\hrule}\vspace{0.4cm}
	\par
}{}

\NewEnviron{h3}{%
    \needspace{3\baselineskip}
	\vspace{5mm}
	\textsc{\BODY}
	\par
}

\NewEnviron{margeneg}{ %
\begin{addmargin}[-1cm]{0cm}
\BODY
\end{addmargin}
}

\NewEnviron{html}{%
}

\begin{document}
\meta{url}{/exercices/cryptage-rsa-bac-s-centres-etrangers-2018-spe/}
\meta{pid}{8638}
\meta{titre}{Cryptage RSA - Bac S Centres étrangers  2018 (spé)}
\meta{type}{exercices}
%
\begin{h2}Exercice 4 (5 points)\end{h2}
\textbf{Candidats ayant choisi la spécialité \og mathématiques \fg{}}
\medskip
Le but de cet exercice est d'envisager une méthode de cryptage à clé publique d'une information
numérique, appelée système RSA, en l'honneur des mathématiciens Ronald Rivest, Adi Shamir et
Leonard Adleman, qui ont inventé cette méthode de cryptage en 1977 et l'ont publiée en 1978.
\smallskip
Les questions 1 et 2 sont des questions préparatoires, la question 3 aborde le cryptage, la question 4
le décryptage.
\bigskip
\begin{enumerate}
     \item Cette question envisage de calculer le reste dans la division euclidienne par $55$ de certaines
     puissances de l'entier $8$.
     \begin{enumerate}[label=\alph*.]
          \item Vérifier que $8^7 \equiv 2 \mod 55$.
          \par
          En déduire le reste dans la division euclidienne par $55$ du nombre $8^{21}$.
          \item Vérifier que $8^2 \equiv 9 \mod 55$, puis déduire de la question \textbf{a.} le reste dans la division
          euclidienne par $55$ de $8^{23}$.
     \end{enumerate}
     \item  Dans cette question, on considère l'équation $(E)$\: $23 x - 40 y = 1$, dont les solutions sont des
     couples $(x~;~y)$ d'entiers relatifs.
     \begin{enumerate}[label=\alph*.]
          \item Justifier le fait que l'équation $(E)$ admet au moins un couple solution.
          \item  Donner un couple, solution particulière de l'équation $(E)$.
          \item  Déterminer tous les couples d'entiers relatifs solutions de l'équation $(E)$.
          \item  En déduire qu'il existe un unique entier $d$ vérifiant les conditions $0 \leqslant d < 40$ et
          $23 d \equiv  1 \mod 40$.
     \end{enumerate}
     \item  \textbf{Cryptage dans le système RSA}
     \par
     Une personne A choisit deux nombres premiers $p$ et $q$, puis calcule les produits $N = p q$ et
     $n = (p - 1)(q - 1)$. Elle choisit également un entier naturel $c$ premier avec $n$.
     \par
     La personne A publie le couple $(N~;~c)$, qui est une clé publique permettant à quiconque de lui
     envoyer un nombre crypté.
     \par
     Les messages sont numérisés et transformés en une suite d'entiers compris entre $0$ et $N -1$.
     \par
     Pour crypter un entier $a$ de cette suite, on procède ainsi~: on calcule le reste $b$ dans la division
     euclidienne par $N$ du nombre $a^c$, et le nombre crypté est l'entier $b$.
     \smallskip
     Dans la pratique, cette méthode est sûre si la personne A choisit des nombres premiers $p$ et $q$
     très grands, s'écrivant avec plusieurs dizaines de chiffres.
     \par
     On va l'envisager ici avec des nombres plus simples~: $p = 5$ et $q = 11$.
     \par
     La personne A choisit également $c = 23$.
     \begin{enumerate}[label=\alph*.]
          \item Calculer les nombres $N$ et $n$, puis justifier que la valeur de $c$ vérifie la condition voulue.
          \item  Un émetteur souhaite envoyer à la personne A le nombre $a = 8$.
          \par
          Déterminer la valeur du nombre crypté $b$.
     \end{enumerate}
     \item  \textbf{Décryptage dans le système RSA}
     \par
     La personne A calcule dans un premier temps l'unique entier naturel $d$ vérifiant les conditions
     $0 \leqslant d < n$ et $cd \equiv 1 \mod n$.
     \par
     Elle garde secret ce nombre $d$ qui lui permet, et à elle seule, de
     décrypter les nombres qui lui ont été envoyés cryptés avec sa clé publique.
     \par
     Pour décrypter un nombre crypté $b$, la personne A calcule le reste $a$ dans la division euclidienne
     par $N$ du nombre $bd$, et le nombre en clair -- c'est-à-dire le nombre avant cryptage -- est le
     nombre $a$.
     \par
     On admet l'existence et l'unicité de l'entier $d$, et le fait que le décryptage fonctionne.
     \par
     Les nombres choisis par A sont encore $p = 5$, $q = 11$ et $c = 23$.
     \begin{enumerate}[label=\alph*.]
          \item Quelle est la valeur de $d$~?
          \item  En appliquant la règle de décryptage, retrouver le nombre en clair lorsque le nombre crypté
          est $b = 17$.
     \end{enumerate}
\end{enumerate}

\end{document}
µ
\documentclass[a4paper]{article}

%================================================================================================================================
%
% Packages
%
%================================================================================================================================

\usepackage[T1]{fontenc} 	% pour caractères accentués
\usepackage[utf8]{inputenc}  % encodage utf8
\usepackage[french]{babel}	% langue : français
\usepackage{fourier}			% caractères plus lisibles
\usepackage[dvipsnames]{xcolor} % couleurs
\usepackage{fancyhdr}		% réglage header footer
\usepackage{needspace}		% empêcher sauts de page mal placés
\usepackage{graphicx}		% pour inclure des graphiques
\usepackage{enumitem,cprotect}		% personnalise les listes d'items (nécessaire pour ol, al ...)
\usepackage{hyperref}		% Liens hypertexte
\usepackage{pstricks,pst-all,pst-node,pstricks-add,pst-math,pst-plot,pst-tree,pst-eucl} % pstricks
\usepackage[a4paper,includeheadfoot,top=2cm,left=3cm, bottom=2cm,right=3cm]{geometry} % marges etc.
\usepackage{comment}			% commentaires multilignes
\usepackage{amsmath,environ} % maths (matrices, etc.)
\usepackage{amssymb,makeidx}
\usepackage{bm}				% bold maths
\usepackage{tabularx}		% tableaux
\usepackage{colortbl}		% tableaux en couleur
\usepackage{fontawesome}		% Fontawesome
\usepackage{environ}			% environment with command
\usepackage{fp}				% calculs pour ps-tricks
\usepackage{multido}			% pour ps tricks
\usepackage[np]{numprint}	% formattage nombre
\usepackage{tikz,tkz-tab} 			% package principal TikZ
\usepackage{pgfplots}   % axes
\usepackage{mathrsfs}    % cursives
\usepackage{calc}			% calcul taille boites
\usepackage[scaled=0.875]{helvet} % font sans serif
\usepackage{svg} % svg
\usepackage{scrextend} % local margin
\usepackage{scratch} %scratch
\usepackage{multicol} % colonnes
%\usepackage{infix-RPN,pst-func} % formule en notation polanaise inversée
\usepackage{listings}

%================================================================================================================================
%
% Réglages de base
%
%================================================================================================================================

\lstset{
language=Python,   % R code
literate=
{á}{{\'a}}1
{à}{{\`a}}1
{ã}{{\~a}}1
{é}{{\'e}}1
{è}{{\`e}}1
{ê}{{\^e}}1
{í}{{\'i}}1
{ó}{{\'o}}1
{õ}{{\~o}}1
{ú}{{\'u}}1
{ü}{{\"u}}1
{ç}{{\c{c}}}1
{~}{{ }}1
}


\definecolor{codegreen}{rgb}{0,0.6,0}
\definecolor{codegray}{rgb}{0.5,0.5,0.5}
\definecolor{codepurple}{rgb}{0.58,0,0.82}
\definecolor{backcolour}{rgb}{0.95,0.95,0.92}

\lstdefinestyle{mystyle}{
    backgroundcolor=\color{backcolour},   
    commentstyle=\color{codegreen},
    keywordstyle=\color{magenta},
    numberstyle=\tiny\color{codegray},
    stringstyle=\color{codepurple},
    basicstyle=\ttfamily\footnotesize,
    breakatwhitespace=false,         
    breaklines=true,                 
    captionpos=b,                    
    keepspaces=true,                 
    numbers=left,                    
xleftmargin=2em,
framexleftmargin=2em,            
    showspaces=false,                
    showstringspaces=false,
    showtabs=false,                  
    tabsize=2,
    upquote=true
}

\lstset{style=mystyle}


\lstset{style=mystyle}
\newcommand{\imgdir}{C:/laragon/www/newmc/assets/imgsvg/}
\newcommand{\imgsvgdir}{C:/laragon/www/newmc/assets/imgsvg/}

\definecolor{mcgris}{RGB}{220, 220, 220}% ancien~; pour compatibilité
\definecolor{mcbleu}{RGB}{52, 152, 219}
\definecolor{mcvert}{RGB}{125, 194, 70}
\definecolor{mcmauve}{RGB}{154, 0, 215}
\definecolor{mcorange}{RGB}{255, 96, 0}
\definecolor{mcturquoise}{RGB}{0, 153, 153}
\definecolor{mcrouge}{RGB}{255, 0, 0}
\definecolor{mclightvert}{RGB}{205, 234, 190}

\definecolor{gris}{RGB}{220, 220, 220}
\definecolor{bleu}{RGB}{52, 152, 219}
\definecolor{vert}{RGB}{125, 194, 70}
\definecolor{mauve}{RGB}{154, 0, 215}
\definecolor{orange}{RGB}{255, 96, 0}
\definecolor{turquoise}{RGB}{0, 153, 153}
\definecolor{rouge}{RGB}{255, 0, 0}
\definecolor{lightvert}{RGB}{205, 234, 190}
\setitemize[0]{label=\color{lightvert}  $\bullet$}

\pagestyle{fancy}
\renewcommand{\headrulewidth}{0.2pt}
\fancyhead[L]{maths-cours.fr}
\fancyhead[R]{\thepage}
\renewcommand{\footrulewidth}{0.2pt}
\fancyfoot[C]{}

\newcolumntype{C}{>{\centering\arraybackslash}X}
\newcolumntype{s}{>{\hsize=.35\hsize\arraybackslash}X}

\setlength{\parindent}{0pt}		 
\setlength{\parskip}{3mm}
\setlength{\headheight}{1cm}

\def\ebook{ebook}
\def\book{book}
\def\web{web}
\def\type{web}

\newcommand{\vect}[1]{\overrightarrow{\,\mathstrut#1\,}}

\def\Oij{$\left(\text{O}~;~\vect{\imath},~\vect{\jmath}\right)$}
\def\Oijk{$\left(\text{O}~;~\vect{\imath},~\vect{\jmath},~\vect{k}\right)$}
\def\Ouv{$\left(\text{O}~;~\vect{u},~\vect{v}\right)$}

\hypersetup{breaklinks=true, colorlinks = true, linkcolor = OliveGreen, urlcolor = OliveGreen, citecolor = OliveGreen, pdfauthor={Didier BONNEL - https://www.maths-cours.fr} } % supprime les bordures autour des liens

\renewcommand{\arg}[0]{\text{arg}}

\everymath{\displaystyle}

%================================================================================================================================
%
% Macros - Commandes
%
%================================================================================================================================

\newcommand\meta[2]{    			% Utilisé pour créer le post HTML.
	\def\titre{titre}
	\def\url{url}
	\def\arg{#1}
	\ifx\titre\arg
		\newcommand\maintitle{#2}
		\fancyhead[L]{#2}
		{\Large\sffamily \MakeUppercase{#2}}
		\vspace{1mm}\textcolor{mcvert}{\hrule}
	\fi 
	\ifx\url\arg
		\fancyfoot[L]{\href{https://www.maths-cours.fr#2}{\black \footnotesize{https://www.maths-cours.fr#2}}}
	\fi 
}


\newcommand\TitreC[1]{    		% Titre centré
     \needspace{3\baselineskip}
     \begin{center}\textbf{#1}\end{center}
}

\newcommand\newpar{    		% paragraphe
     \par
}

\newcommand\nosp {    		% commande vide (pas d'espace)
}
\newcommand{\id}[1]{} %ignore

\newcommand\boite[2]{				% Boite simple sans titre
	\vspace{5mm}
	\setlength{\fboxrule}{0.2mm}
	\setlength{\fboxsep}{5mm}	
	\fcolorbox{#1}{#1!3}{\makebox[\linewidth-2\fboxrule-2\fboxsep]{
  		\begin{minipage}[t]{\linewidth-2\fboxrule-4\fboxsep}\setlength{\parskip}{3mm}
  			 #2
  		\end{minipage}
	}}
	\vspace{5mm}
}

\newcommand\CBox[4]{				% Boites
	\vspace{5mm}
	\setlength{\fboxrule}{0.2mm}
	\setlength{\fboxsep}{5mm}
	
	\fcolorbox{#1}{#1!3}{\makebox[\linewidth-2\fboxrule-2\fboxsep]{
		\begin{minipage}[t]{1cm}\setlength{\parskip}{3mm}
	  		\textcolor{#1}{\LARGE{#2}}    
 	 	\end{minipage}  
  		\begin{minipage}[t]{\linewidth-2\fboxrule-4\fboxsep}\setlength{\parskip}{3mm}
			\raisebox{1.2mm}{\normalsize\sffamily{\textcolor{#1}{#3}}}						
  			 #4
  		\end{minipage}
	}}
	\vspace{5mm}
}

\newcommand\cadre[3]{				% Boites convertible html
	\par
	\vspace{2mm}
	\setlength{\fboxrule}{0.1mm}
	\setlength{\fboxsep}{5mm}
	\fcolorbox{#1}{white}{\makebox[\linewidth-2\fboxrule-2\fboxsep]{
  		\begin{minipage}[t]{\linewidth-2\fboxrule-4\fboxsep}\setlength{\parskip}{3mm}
			\raisebox{-2.5mm}{\sffamily \small{\textcolor{#1}{\MakeUppercase{#2}}}}		
			\par		
  			 #3
 	 		\end{minipage}
	}}
		\vspace{2mm}
	\par
}

\newcommand\bloc[3]{				% Boites convertible html sans bordure
     \needspace{2\baselineskip}
     {\sffamily \small{\textcolor{#1}{\MakeUppercase{#2}}}}    
		\par		
  			 #3
		\par
}

\newcommand\CHelp[1]{
     \CBox{Plum}{\faInfoCircle}{À RETENIR}{#1}
}

\newcommand\CUp[1]{
     \CBox{NavyBlue}{\faThumbsOUp}{EN PRATIQUE}{#1}
}

\newcommand\CInfo[1]{
     \CBox{Sepia}{\faArrowCircleRight}{REMARQUE}{#1}
}

\newcommand\CRedac[1]{
     \CBox{PineGreen}{\faEdit}{BIEN R\'EDIGER}{#1}
}

\newcommand\CError[1]{
     \CBox{Red}{\faExclamationTriangle}{ATTENTION}{#1}
}

\newcommand\TitreExo[2]{
\needspace{4\baselineskip}
 {\sffamily\large EXERCICE #1\ (\emph{#2 points})}
\vspace{5mm}
}

\newcommand\img[2]{
          \includegraphics[width=#2\paperwidth]{\imgdir#1}
}

\newcommand\imgsvg[2]{
       \begin{center}   \includegraphics[width=#2\paperwidth]{\imgsvgdir#1} \end{center}
}


\newcommand\Lien[2]{
     \href{#1}{#2 \tiny \faExternalLink}
}
\newcommand\mcLien[2]{
     \href{https~://www.maths-cours.fr/#1}{#2 \tiny \faExternalLink}
}

\newcommand{\euro}{\eurologo{}}

%================================================================================================================================
%
% Macros - Environement
%
%================================================================================================================================

\newenvironment{tex}{ %
}
{%
}

\newenvironment{indente}{ %
	\setlength\parindent{10mm}
}

{
	\setlength\parindent{0mm}
}

\newenvironment{corrige}{%
     \needspace{3\baselineskip}
     \medskip
     \textbf{\textsc{Corrigé}}
     \medskip
}
{
}

\newenvironment{extern}{%
     \begin{center}
     }
     {
     \end{center}
}

\NewEnviron{code}{%
	\par
     \boite{gray}{\texttt{%
     \BODY
     }}
     \par
}

\newenvironment{vbloc}{% boite sans cadre empeche saut de page
     \begin{minipage}[t]{\linewidth}
     }
     {
     \end{minipage}
}
\NewEnviron{h2}{%
    \needspace{3\baselineskip}
    \vspace{0.6cm}
	\noindent \MakeUppercase{\sffamily \large \BODY}
	\vspace{1mm}\textcolor{mcgris}{\hrule}\vspace{0.4cm}
	\par
}{}

\NewEnviron{h3}{%
    \needspace{3\baselineskip}
	\vspace{5mm}
	\textsc{\BODY}
	\par
}

\NewEnviron{margeneg}{ %
\begin{addmargin}[-1cm]{0cm}
\BODY
\end{addmargin}
}

\NewEnviron{html}{%
}

\begin{document}
\meta{url}{/exercices/qcm-bac-es-l-centres-etrangers-2018/}
\meta{pid}{8643}
\meta{titre}{QCM - Bac ES/L Centres étrangers 2018}
\meta{type}{exercices}
%
\begin{h2}Exercice 1 (4 points)\end{h2}
\textbf{Commun à  tous les candidats}
\par
\emph{Cet exercice est un questionnaire à choix multiples. Pour chacune des questions
     suivantes, une seule des quatre réponses proposées est exacte. Aucune justification
n'est demandée.}
\par
\textit{Une bonne réponse rapporte un point. Une mauvaise réponse,
     plusieurs réponses ou l'absence de réponse à une question ne rapportent ni
n'enlèvent de point.}
\par
\textit{Pour répondre, vous recopierez sur votre copie le numéro de la question et indiquerez la seule réponse choisie.}
\bigskip
\begin{enumerate}
     \item Soit $f$ la fonction définie pour tout réel $x$ par $f(x) = \text{e}^{-3x} + \text{e}^2$.
     \par
     \begin{tabularx}{\linewidth}{*{2}{X}}%class="noborder"
          \textbf{A.~~} $f'(x) = $\nosp$- \text{e}^{-3x} + 2\text{e}$ 	&\textbf{B.~~} $f'(x) =  $\nosp$-3\text{e}^{-3x}+ \text{e}^2$\\
          \textbf{C.~~} $f'(x) = $\nosp$ - 3\text{e}^{-3x}$ 		&\textbf{D.~~} $f'(x) =  $\nosp$\text{e}^{-3x}$
     \end{tabularx}
     \par
     \item  D'après une étude, le nombre d'objets connectés à Internet à travers le monde
     est passé de $4$ milliards en 2010 à $15$ milliards en 2017. L'arrondi au dixième
     du taux d'évolution annuel moyen est de~:
     \par
     \begin{tabularx}{\linewidth}{*{2}{X}}%class="noborder"
          \textbf{A.~~} 10,5\,\% &\textbf{B.~~} 68,8\,\%\\
          \textbf{C.~~} 39,3\,\% &\textbf{D.~~} 20,8\,\%
     \end{tabularx}
     \par
     \item  Soit $X$ une variable aléatoire qui suit la loi normale d'espérance $\mu = 13$ et
     d'écart-type $\sigma = 2,4$. L'arrondi au centième de $P(X \geqslant 12,5)$ est~:
     \par
     \begin{tabularx}{\linewidth}{*{2}{X}}%class="noborder"
          \textbf{A.~~} 0,58&\textbf{B.~~} 0,42\\
          \textbf{C.~~} 0,54&\textbf{D.~~} 0,63
     \end{tabularx}
     \par
     \item  Soit $Y$ une variable aléatoire qui suit la loi uniforme sur l'intervalle [14~;~16].
     $P(X \leqslant 15,5)$ est égal à~:
     \par
     \begin{tabularx}{\linewidth}{*{2}{X}}%class="noborder"
          \textbf{A.~~}0,97 &\textbf{B.~~} 0,75\\
          \textbf{C.~~}0,5&\textbf{D.~~}$\dfrac{1}{4}$
     \end{tabularx}
     \par
\end{enumerate}

\end{document}
µ
\documentclass[a4paper]{article}

%================================================================================================================================
%
% Packages
%
%================================================================================================================================

\usepackage[T1]{fontenc} 	% pour caractères accentués
\usepackage[utf8]{inputenc}  % encodage utf8
\usepackage[french]{babel}	% langue : français
\usepackage{fourier}			% caractères plus lisibles
\usepackage[dvipsnames]{xcolor} % couleurs
\usepackage{fancyhdr}		% réglage header footer
\usepackage{needspace}		% empêcher sauts de page mal placés
\usepackage{graphicx}		% pour inclure des graphiques
\usepackage{enumitem,cprotect}		% personnalise les listes d'items (nécessaire pour ol, al ...)
\usepackage{hyperref}		% Liens hypertexte
\usepackage{pstricks,pst-all,pst-node,pstricks-add,pst-math,pst-plot,pst-tree,pst-eucl} % pstricks
\usepackage[a4paper,includeheadfoot,top=2cm,left=3cm, bottom=2cm,right=3cm]{geometry} % marges etc.
\usepackage{comment}			% commentaires multilignes
\usepackage{amsmath,environ} % maths (matrices, etc.)
\usepackage{amssymb,makeidx}
\usepackage{bm}				% bold maths
\usepackage{tabularx}		% tableaux
\usepackage{colortbl}		% tableaux en couleur
\usepackage{fontawesome}		% Fontawesome
\usepackage{environ}			% environment with command
\usepackage{fp}				% calculs pour ps-tricks
\usepackage{multido}			% pour ps tricks
\usepackage[np]{numprint}	% formattage nombre
\usepackage{tikz,tkz-tab} 			% package principal TikZ
\usepackage{pgfplots}   % axes
\usepackage{mathrsfs}    % cursives
\usepackage{calc}			% calcul taille boites
\usepackage[scaled=0.875]{helvet} % font sans serif
\usepackage{svg} % svg
\usepackage{scrextend} % local margin
\usepackage{scratch} %scratch
\usepackage{multicol} % colonnes
%\usepackage{infix-RPN,pst-func} % formule en notation polanaise inversée
\usepackage{listings}

%================================================================================================================================
%
% Réglages de base
%
%================================================================================================================================

\lstset{
language=Python,   % R code
literate=
{á}{{\'a}}1
{à}{{\`a}}1
{ã}{{\~a}}1
{é}{{\'e}}1
{è}{{\`e}}1
{ê}{{\^e}}1
{í}{{\'i}}1
{ó}{{\'o}}1
{õ}{{\~o}}1
{ú}{{\'u}}1
{ü}{{\"u}}1
{ç}{{\c{c}}}1
{~}{{ }}1
}


\definecolor{codegreen}{rgb}{0,0.6,0}
\definecolor{codegray}{rgb}{0.5,0.5,0.5}
\definecolor{codepurple}{rgb}{0.58,0,0.82}
\definecolor{backcolour}{rgb}{0.95,0.95,0.92}

\lstdefinestyle{mystyle}{
    backgroundcolor=\color{backcolour},   
    commentstyle=\color{codegreen},
    keywordstyle=\color{magenta},
    numberstyle=\tiny\color{codegray},
    stringstyle=\color{codepurple},
    basicstyle=\ttfamily\footnotesize,
    breakatwhitespace=false,         
    breaklines=true,                 
    captionpos=b,                    
    keepspaces=true,                 
    numbers=left,                    
xleftmargin=2em,
framexleftmargin=2em,            
    showspaces=false,                
    showstringspaces=false,
    showtabs=false,                  
    tabsize=2,
    upquote=true
}

\lstset{style=mystyle}


\lstset{style=mystyle}
\newcommand{\imgdir}{C:/laragon/www/newmc/assets/imgsvg/}
\newcommand{\imgsvgdir}{C:/laragon/www/newmc/assets/imgsvg/}

\definecolor{mcgris}{RGB}{220, 220, 220}% ancien~; pour compatibilité
\definecolor{mcbleu}{RGB}{52, 152, 219}
\definecolor{mcvert}{RGB}{125, 194, 70}
\definecolor{mcmauve}{RGB}{154, 0, 215}
\definecolor{mcorange}{RGB}{255, 96, 0}
\definecolor{mcturquoise}{RGB}{0, 153, 153}
\definecolor{mcrouge}{RGB}{255, 0, 0}
\definecolor{mclightvert}{RGB}{205, 234, 190}

\definecolor{gris}{RGB}{220, 220, 220}
\definecolor{bleu}{RGB}{52, 152, 219}
\definecolor{vert}{RGB}{125, 194, 70}
\definecolor{mauve}{RGB}{154, 0, 215}
\definecolor{orange}{RGB}{255, 96, 0}
\definecolor{turquoise}{RGB}{0, 153, 153}
\definecolor{rouge}{RGB}{255, 0, 0}
\definecolor{lightvert}{RGB}{205, 234, 190}
\setitemize[0]{label=\color{lightvert}  $\bullet$}

\pagestyle{fancy}
\renewcommand{\headrulewidth}{0.2pt}
\fancyhead[L]{maths-cours.fr}
\fancyhead[R]{\thepage}
\renewcommand{\footrulewidth}{0.2pt}
\fancyfoot[C]{}

\newcolumntype{C}{>{\centering\arraybackslash}X}
\newcolumntype{s}{>{\hsize=.35\hsize\arraybackslash}X}

\setlength{\parindent}{0pt}		 
\setlength{\parskip}{3mm}
\setlength{\headheight}{1cm}

\def\ebook{ebook}
\def\book{book}
\def\web{web}
\def\type{web}

\newcommand{\vect}[1]{\overrightarrow{\,\mathstrut#1\,}}

\def\Oij{$\left(\text{O}~;~\vect{\imath},~\vect{\jmath}\right)$}
\def\Oijk{$\left(\text{O}~;~\vect{\imath},~\vect{\jmath},~\vect{k}\right)$}
\def\Ouv{$\left(\text{O}~;~\vect{u},~\vect{v}\right)$}

\hypersetup{breaklinks=true, colorlinks = true, linkcolor = OliveGreen, urlcolor = OliveGreen, citecolor = OliveGreen, pdfauthor={Didier BONNEL - https://www.maths-cours.fr} } % supprime les bordures autour des liens

\renewcommand{\arg}[0]{\text{arg}}

\everymath{\displaystyle}

%================================================================================================================================
%
% Macros - Commandes
%
%================================================================================================================================

\newcommand\meta[2]{    			% Utilisé pour créer le post HTML.
	\def\titre{titre}
	\def\url{url}
	\def\arg{#1}
	\ifx\titre\arg
		\newcommand\maintitle{#2}
		\fancyhead[L]{#2}
		{\Large\sffamily \MakeUppercase{#2}}
		\vspace{1mm}\textcolor{mcvert}{\hrule}
	\fi 
	\ifx\url\arg
		\fancyfoot[L]{\href{https://www.maths-cours.fr#2}{\black \footnotesize{https://www.maths-cours.fr#2}}}
	\fi 
}


\newcommand\TitreC[1]{    		% Titre centré
     \needspace{3\baselineskip}
     \begin{center}\textbf{#1}\end{center}
}

\newcommand\newpar{    		% paragraphe
     \par
}

\newcommand\nosp {    		% commande vide (pas d'espace)
}
\newcommand{\id}[1]{} %ignore

\newcommand\boite[2]{				% Boite simple sans titre
	\vspace{5mm}
	\setlength{\fboxrule}{0.2mm}
	\setlength{\fboxsep}{5mm}	
	\fcolorbox{#1}{#1!3}{\makebox[\linewidth-2\fboxrule-2\fboxsep]{
  		\begin{minipage}[t]{\linewidth-2\fboxrule-4\fboxsep}\setlength{\parskip}{3mm}
  			 #2
  		\end{minipage}
	}}
	\vspace{5mm}
}

\newcommand\CBox[4]{				% Boites
	\vspace{5mm}
	\setlength{\fboxrule}{0.2mm}
	\setlength{\fboxsep}{5mm}
	
	\fcolorbox{#1}{#1!3}{\makebox[\linewidth-2\fboxrule-2\fboxsep]{
		\begin{minipage}[t]{1cm}\setlength{\parskip}{3mm}
	  		\textcolor{#1}{\LARGE{#2}}    
 	 	\end{minipage}  
  		\begin{minipage}[t]{\linewidth-2\fboxrule-4\fboxsep}\setlength{\parskip}{3mm}
			\raisebox{1.2mm}{\normalsize\sffamily{\textcolor{#1}{#3}}}						
  			 #4
  		\end{minipage}
	}}
	\vspace{5mm}
}

\newcommand\cadre[3]{				% Boites convertible html
	\par
	\vspace{2mm}
	\setlength{\fboxrule}{0.1mm}
	\setlength{\fboxsep}{5mm}
	\fcolorbox{#1}{white}{\makebox[\linewidth-2\fboxrule-2\fboxsep]{
  		\begin{minipage}[t]{\linewidth-2\fboxrule-4\fboxsep}\setlength{\parskip}{3mm}
			\raisebox{-2.5mm}{\sffamily \small{\textcolor{#1}{\MakeUppercase{#2}}}}		
			\par		
  			 #3
 	 		\end{minipage}
	}}
		\vspace{2mm}
	\par
}

\newcommand\bloc[3]{				% Boites convertible html sans bordure
     \needspace{2\baselineskip}
     {\sffamily \small{\textcolor{#1}{\MakeUppercase{#2}}}}    
		\par		
  			 #3
		\par
}

\newcommand\CHelp[1]{
     \CBox{Plum}{\faInfoCircle}{À RETENIR}{#1}
}

\newcommand\CUp[1]{
     \CBox{NavyBlue}{\faThumbsOUp}{EN PRATIQUE}{#1}
}

\newcommand\CInfo[1]{
     \CBox{Sepia}{\faArrowCircleRight}{REMARQUE}{#1}
}

\newcommand\CRedac[1]{
     \CBox{PineGreen}{\faEdit}{BIEN R\'EDIGER}{#1}
}

\newcommand\CError[1]{
     \CBox{Red}{\faExclamationTriangle}{ATTENTION}{#1}
}

\newcommand\TitreExo[2]{
\needspace{4\baselineskip}
 {\sffamily\large EXERCICE #1\ (\emph{#2 points})}
\vspace{5mm}
}

\newcommand\img[2]{
          \includegraphics[width=#2\paperwidth]{\imgdir#1}
}

\newcommand\imgsvg[2]{
       \begin{center}   \includegraphics[width=#2\paperwidth]{\imgsvgdir#1} \end{center}
}


\newcommand\Lien[2]{
     \href{#1}{#2 \tiny \faExternalLink}
}
\newcommand\mcLien[2]{
     \href{https~://www.maths-cours.fr/#1}{#2 \tiny \faExternalLink}
}

\newcommand{\euro}{\eurologo{}}

%================================================================================================================================
%
% Macros - Environement
%
%================================================================================================================================

\newenvironment{tex}{ %
}
{%
}

\newenvironment{indente}{ %
	\setlength\parindent{10mm}
}

{
	\setlength\parindent{0mm}
}

\newenvironment{corrige}{%
     \needspace{3\baselineskip}
     \medskip
     \textbf{\textsc{Corrigé}}
     \medskip
}
{
}

\newenvironment{extern}{%
     \begin{center}
     }
     {
     \end{center}
}

\NewEnviron{code}{%
	\par
     \boite{gray}{\texttt{%
     \BODY
     }}
     \par
}

\newenvironment{vbloc}{% boite sans cadre empeche saut de page
     \begin{minipage}[t]{\linewidth}
     }
     {
     \end{minipage}
}
\NewEnviron{h2}{%
    \needspace{3\baselineskip}
    \vspace{0.6cm}
	\noindent \MakeUppercase{\sffamily \large \BODY}
	\vspace{1mm}\textcolor{mcgris}{\hrule}\vspace{0.4cm}
	\par
}{}

\NewEnviron{h3}{%
    \needspace{3\baselineskip}
	\vspace{5mm}
	\textsc{\BODY}
	\par
}

\NewEnviron{margeneg}{ %
\begin{addmargin}[-1cm]{0cm}
\BODY
\end{addmargin}
}

\NewEnviron{html}{%
}

\begin{document}
\meta{url}{/exercices/suites-bac-es-l-centres-etrangers-2018/}
\meta{pid}{8649}
\meta{titre}{Suites - Bac ES/L Centres étrangers 2018}
\meta{type}{exercices}
%
\begin{h2}Exercice 2 (5 points)\end{h2}
\textbf{Candidats n'ayant pas choisi la spécialité \og mathématiques \fg{}}
\medskip
Des algues prolifèrent dans un étang. Pour s'en débarrasser, le propriétaire installe
un système de filtration.
\par
En journée, la masse d'algues augmente de 2\,\%, puis à la nuit tombée, le propriétaire
actionne pendant une heure le système de filtration qui retire 1$00$~kg d'algues. On
admet que les algues ne prolifèrent pas la nuit.
\par
Le propriétaire estime que la masse d'algues dans l'étang au matin de l'installation
du système de filtration est de 2~000~kg.
\par
On modélise par $a_n$ la masse d'algues dans l'étang, exprimée en kg, après utilisation
du système de filtration pendant $n$ jours~; ainsi, $a_0 = 2~000$. On admet que cette
modélisation demeure valable tant que $a_n$ reste positif.
\bigskip
\begin{enumerate}
     \item Vérifier par le calcul que la masse $a_2$ d'algues après deux jours de
     fonctionnement du système de filtration est de 1~878,8 kg.
     \item On affirme que pour tout entier naturel $n$,\: $a_{n+1} = 1,02a_n - 100$.
     \begin{enumerate}[label=\alph*.]
          \item Justifier à l'aide de l'énoncé la relation précédente.
          \item  On considère la suite $\left(b_n\right)$ définie pour tout nombre entier naturel $n$ par~:
          \[b_n = a_n - 5~000.\]
          Démontrer que la suite $\left(b_n\right)$ est géométrique. Préciser son premier terme
          $b_0$ et sa raison.
          \item  En déduire pour tout entier naturel $n$, une expression de $b_n$ en fonction de
          $n$, puis montrer que $a_n = 5~000 - 3~000 \times 1,02^n$.
          \item  En déterminant la limite de la suite $\left(a_n\right)$, justifier que les algues finissent
          par disparaître.
     \end{enumerate}
     \item
     \begin{enumerate}[label=\alph*.]
          \item Recopier et compléter l'algorithme suivant afin qu'il détermine le nombre
          de jours nécessaire à la disparition des algues.
          \begin{center}
               \begin{extern}%width="180" alt="Algorithme Bac ES/L Centres étrangers 2018"
                    \begin{tabularx}{0.25\linewidth}{|X|}\hline
                         $N \gets 0$\\
                         $A \gets 2~000$\\
                         Tant que \ldots\\
                         \hspace{1cm}$A \gets \ldots$\\
                         \hspace{1cm}$N \gets N + 1$\\
                         Fin Tant que\\
                         Afficher \ldots\\ \hline
                    \end{tabularx}
               \end{extern}
          \end{center}
          \item Quel est le résultat renvoyé par l'algorithme~?
     \end{enumerate}
     \item
     \begin{enumerate}[label=\alph*.]
          \item Résoudre par le calcul l'inéquation $5~000 - 3~000 \times 1,02^n \leqslant 0$.
          \item Quel résultat précédemment obtenu retrouve-t-on~?
     \end{enumerate}
\end{enumerate}

\end{document}
µ
\documentclass[a4paper]{article}

%================================================================================================================================
%
% Packages
%
%================================================================================================================================

\usepackage[T1]{fontenc} 	% pour caractères accentués
\usepackage[utf8]{inputenc}  % encodage utf8
\usepackage[french]{babel}	% langue : français
\usepackage{fourier}			% caractères plus lisibles
\usepackage[dvipsnames]{xcolor} % couleurs
\usepackage{fancyhdr}		% réglage header footer
\usepackage{needspace}		% empêcher sauts de page mal placés
\usepackage{graphicx}		% pour inclure des graphiques
\usepackage{enumitem,cprotect}		% personnalise les listes d'items (nécessaire pour ol, al ...)
\usepackage{hyperref}		% Liens hypertexte
\usepackage{pstricks,pst-all,pst-node,pstricks-add,pst-math,pst-plot,pst-tree,pst-eucl} % pstricks
\usepackage[a4paper,includeheadfoot,top=2cm,left=3cm, bottom=2cm,right=3cm]{geometry} % marges etc.
\usepackage{comment}			% commentaires multilignes
\usepackage{amsmath,environ} % maths (matrices, etc.)
\usepackage{amssymb,makeidx}
\usepackage{bm}				% bold maths
\usepackage{tabularx}		% tableaux
\usepackage{colortbl}		% tableaux en couleur
\usepackage{fontawesome}		% Fontawesome
\usepackage{environ}			% environment with command
\usepackage{fp}				% calculs pour ps-tricks
\usepackage{multido}			% pour ps tricks
\usepackage[np]{numprint}	% formattage nombre
\usepackage{tikz,tkz-tab} 			% package principal TikZ
\usepackage{pgfplots}   % axes
\usepackage{mathrsfs}    % cursives
\usepackage{calc}			% calcul taille boites
\usepackage[scaled=0.875]{helvet} % font sans serif
\usepackage{svg} % svg
\usepackage{scrextend} % local margin
\usepackage{scratch} %scratch
\usepackage{multicol} % colonnes
%\usepackage{infix-RPN,pst-func} % formule en notation polanaise inversée
\usepackage{listings}

%================================================================================================================================
%
% Réglages de base
%
%================================================================================================================================

\lstset{
language=Python,   % R code
literate=
{á}{{\'a}}1
{à}{{\`a}}1
{ã}{{\~a}}1
{é}{{\'e}}1
{è}{{\`e}}1
{ê}{{\^e}}1
{í}{{\'i}}1
{ó}{{\'o}}1
{õ}{{\~o}}1
{ú}{{\'u}}1
{ü}{{\"u}}1
{ç}{{\c{c}}}1
{~}{{ }}1
}


\definecolor{codegreen}{rgb}{0,0.6,0}
\definecolor{codegray}{rgb}{0.5,0.5,0.5}
\definecolor{codepurple}{rgb}{0.58,0,0.82}
\definecolor{backcolour}{rgb}{0.95,0.95,0.92}

\lstdefinestyle{mystyle}{
    backgroundcolor=\color{backcolour},   
    commentstyle=\color{codegreen},
    keywordstyle=\color{magenta},
    numberstyle=\tiny\color{codegray},
    stringstyle=\color{codepurple},
    basicstyle=\ttfamily\footnotesize,
    breakatwhitespace=false,         
    breaklines=true,                 
    captionpos=b,                    
    keepspaces=true,                 
    numbers=left,                    
xleftmargin=2em,
framexleftmargin=2em,            
    showspaces=false,                
    showstringspaces=false,
    showtabs=false,                  
    tabsize=2,
    upquote=true
}

\lstset{style=mystyle}


\lstset{style=mystyle}
\newcommand{\imgdir}{C:/laragon/www/newmc/assets/imgsvg/}
\newcommand{\imgsvgdir}{C:/laragon/www/newmc/assets/imgsvg/}

\definecolor{mcgris}{RGB}{220, 220, 220}% ancien~; pour compatibilité
\definecolor{mcbleu}{RGB}{52, 152, 219}
\definecolor{mcvert}{RGB}{125, 194, 70}
\definecolor{mcmauve}{RGB}{154, 0, 215}
\definecolor{mcorange}{RGB}{255, 96, 0}
\definecolor{mcturquoise}{RGB}{0, 153, 153}
\definecolor{mcrouge}{RGB}{255, 0, 0}
\definecolor{mclightvert}{RGB}{205, 234, 190}

\definecolor{gris}{RGB}{220, 220, 220}
\definecolor{bleu}{RGB}{52, 152, 219}
\definecolor{vert}{RGB}{125, 194, 70}
\definecolor{mauve}{RGB}{154, 0, 215}
\definecolor{orange}{RGB}{255, 96, 0}
\definecolor{turquoise}{RGB}{0, 153, 153}
\definecolor{rouge}{RGB}{255, 0, 0}
\definecolor{lightvert}{RGB}{205, 234, 190}
\setitemize[0]{label=\color{lightvert}  $\bullet$}

\pagestyle{fancy}
\renewcommand{\headrulewidth}{0.2pt}
\fancyhead[L]{maths-cours.fr}
\fancyhead[R]{\thepage}
\renewcommand{\footrulewidth}{0.2pt}
\fancyfoot[C]{}

\newcolumntype{C}{>{\centering\arraybackslash}X}
\newcolumntype{s}{>{\hsize=.35\hsize\arraybackslash}X}

\setlength{\parindent}{0pt}		 
\setlength{\parskip}{3mm}
\setlength{\headheight}{1cm}

\def\ebook{ebook}
\def\book{book}
\def\web{web}
\def\type{web}

\newcommand{\vect}[1]{\overrightarrow{\,\mathstrut#1\,}}

\def\Oij{$\left(\text{O}~;~\vect{\imath},~\vect{\jmath}\right)$}
\def\Oijk{$\left(\text{O}~;~\vect{\imath},~\vect{\jmath},~\vect{k}\right)$}
\def\Ouv{$\left(\text{O}~;~\vect{u},~\vect{v}\right)$}

\hypersetup{breaklinks=true, colorlinks = true, linkcolor = OliveGreen, urlcolor = OliveGreen, citecolor = OliveGreen, pdfauthor={Didier BONNEL - https://www.maths-cours.fr} } % supprime les bordures autour des liens

\renewcommand{\arg}[0]{\text{arg}}

\everymath{\displaystyle}

%================================================================================================================================
%
% Macros - Commandes
%
%================================================================================================================================

\newcommand\meta[2]{    			% Utilisé pour créer le post HTML.
	\def\titre{titre}
	\def\url{url}
	\def\arg{#1}
	\ifx\titre\arg
		\newcommand\maintitle{#2}
		\fancyhead[L]{#2}
		{\Large\sffamily \MakeUppercase{#2}}
		\vspace{1mm}\textcolor{mcvert}{\hrule}
	\fi 
	\ifx\url\arg
		\fancyfoot[L]{\href{https://www.maths-cours.fr#2}{\black \footnotesize{https://www.maths-cours.fr#2}}}
	\fi 
}


\newcommand\TitreC[1]{    		% Titre centré
     \needspace{3\baselineskip}
     \begin{center}\textbf{#1}\end{center}
}

\newcommand\newpar{    		% paragraphe
     \par
}

\newcommand\nosp {    		% commande vide (pas d'espace)
}
\newcommand{\id}[1]{} %ignore

\newcommand\boite[2]{				% Boite simple sans titre
	\vspace{5mm}
	\setlength{\fboxrule}{0.2mm}
	\setlength{\fboxsep}{5mm}	
	\fcolorbox{#1}{#1!3}{\makebox[\linewidth-2\fboxrule-2\fboxsep]{
  		\begin{minipage}[t]{\linewidth-2\fboxrule-4\fboxsep}\setlength{\parskip}{3mm}
  			 #2
  		\end{minipage}
	}}
	\vspace{5mm}
}

\newcommand\CBox[4]{				% Boites
	\vspace{5mm}
	\setlength{\fboxrule}{0.2mm}
	\setlength{\fboxsep}{5mm}
	
	\fcolorbox{#1}{#1!3}{\makebox[\linewidth-2\fboxrule-2\fboxsep]{
		\begin{minipage}[t]{1cm}\setlength{\parskip}{3mm}
	  		\textcolor{#1}{\LARGE{#2}}    
 	 	\end{minipage}  
  		\begin{minipage}[t]{\linewidth-2\fboxrule-4\fboxsep}\setlength{\parskip}{3mm}
			\raisebox{1.2mm}{\normalsize\sffamily{\textcolor{#1}{#3}}}						
  			 #4
  		\end{minipage}
	}}
	\vspace{5mm}
}

\newcommand\cadre[3]{				% Boites convertible html
	\par
	\vspace{2mm}
	\setlength{\fboxrule}{0.1mm}
	\setlength{\fboxsep}{5mm}
	\fcolorbox{#1}{white}{\makebox[\linewidth-2\fboxrule-2\fboxsep]{
  		\begin{minipage}[t]{\linewidth-2\fboxrule-4\fboxsep}\setlength{\parskip}{3mm}
			\raisebox{-2.5mm}{\sffamily \small{\textcolor{#1}{\MakeUppercase{#2}}}}		
			\par		
  			 #3
 	 		\end{minipage}
	}}
		\vspace{2mm}
	\par
}

\newcommand\bloc[3]{				% Boites convertible html sans bordure
     \needspace{2\baselineskip}
     {\sffamily \small{\textcolor{#1}{\MakeUppercase{#2}}}}    
		\par		
  			 #3
		\par
}

\newcommand\CHelp[1]{
     \CBox{Plum}{\faInfoCircle}{À RETENIR}{#1}
}

\newcommand\CUp[1]{
     \CBox{NavyBlue}{\faThumbsOUp}{EN PRATIQUE}{#1}
}

\newcommand\CInfo[1]{
     \CBox{Sepia}{\faArrowCircleRight}{REMARQUE}{#1}
}

\newcommand\CRedac[1]{
     \CBox{PineGreen}{\faEdit}{BIEN R\'EDIGER}{#1}
}

\newcommand\CError[1]{
     \CBox{Red}{\faExclamationTriangle}{ATTENTION}{#1}
}

\newcommand\TitreExo[2]{
\needspace{4\baselineskip}
 {\sffamily\large EXERCICE #1\ (\emph{#2 points})}
\vspace{5mm}
}

\newcommand\img[2]{
          \includegraphics[width=#2\paperwidth]{\imgdir#1}
}

\newcommand\imgsvg[2]{
       \begin{center}   \includegraphics[width=#2\paperwidth]{\imgsvgdir#1} \end{center}
}


\newcommand\Lien[2]{
     \href{#1}{#2 \tiny \faExternalLink}
}
\newcommand\mcLien[2]{
     \href{https~://www.maths-cours.fr/#1}{#2 \tiny \faExternalLink}
}

\newcommand{\euro}{\eurologo{}}

%================================================================================================================================
%
% Macros - Environement
%
%================================================================================================================================

\newenvironment{tex}{ %
}
{%
}

\newenvironment{indente}{ %
	\setlength\parindent{10mm}
}

{
	\setlength\parindent{0mm}
}

\newenvironment{corrige}{%
     \needspace{3\baselineskip}
     \medskip
     \textbf{\textsc{Corrigé}}
     \medskip
}
{
}

\newenvironment{extern}{%
     \begin{center}
     }
     {
     \end{center}
}

\NewEnviron{code}{%
	\par
     \boite{gray}{\texttt{%
     \BODY
     }}
     \par
}

\newenvironment{vbloc}{% boite sans cadre empeche saut de page
     \begin{minipage}[t]{\linewidth}
     }
     {
     \end{minipage}
}
\NewEnviron{h2}{%
    \needspace{3\baselineskip}
    \vspace{0.6cm}
	\noindent \MakeUppercase{\sffamily \large \BODY}
	\vspace{1mm}\textcolor{mcgris}{\hrule}\vspace{0.4cm}
	\par
}{}

\NewEnviron{h3}{%
    \needspace{3\baselineskip}
	\vspace{5mm}
	\textsc{\BODY}
	\par
}

\NewEnviron{margeneg}{ %
\begin{addmargin}[-1cm]{0cm}
\BODY
\end{addmargin}
}

\NewEnviron{html}{%
}

\begin{document}
\meta{url}{/exercices/probabilites-bac-es-l-centres-etrangers-2018/}
\meta{pid}{8660}
\meta{titre}{Probabilités - Bac ES/L Centres étrangers 2018}
\meta{type}{exercices}
%
\begin{h2}Exercice 3 (5 points)\end{h2}
\textbf{Commun à  tous les candidats}
\medskip
Une entreprise dispose d'un stock de guirlandes électriques. On sait que 40\,\% des
guirlandes proviennent d'un fournisseur A et le reste d'un fournisseur B.
\par
Un quart des guirlandes provenant du fournisseur A et un tiers des guirlandes
provenant du fournisseur B peuvent être utilisées uniquement en intérieur pour des
raisons de sécurité. Les autres guirlandes peuvent être utilisées aussi bien en
intérieur qu'en extérieur.
\bigskip
\begin{enumerate}
     \item On choisit au hasard une guirlande dans le stock.
     \begin{indent}
          \begin{itemize}
               \item On note $A$ l'événement \og la guirlande provient du fournisseur A\fg{} et
               $B$ l'événement \og la guirlande provient du fournisseur B \fg.
               \item On note $I$ l'événement \og la guirlande peut être utilisée uniquement en
               intérieur \fg.
          \end{itemize}
     \end{indent}
     \begin{enumerate}[label=\alph*.]
          \item Construire un arbre pondéré décrivant la situation.
          \item Montrer que la probabilité $P(I)$ de l'événement $I$ est $0,3$.
          \item On choisit une guirlande pouvant être utilisée aussi bien en intérieur qu'en
          extérieur. Le responsable de l'entreprise estime qu'il y a autant de chance
          qu'elle provienne du fournisseur A que du fournisseur B.
          \par
          Le responsable a-t-il raison~? Justifier.
     \end{enumerate}
     \item  Une guirlande pouvant être utilisée aussi bien en intérieur qu'en extérieur est
     vendue $5$~\euro{} et une guirlande pouvant être utilisée uniquement en intérieur est
     vendue $3$~\euro.
     \par
     Calculer le prix moyen d'une guirlande prélevée au hasard dans le stock.
     \item  Lors d'un contrôle qualité, on prélève au hasard $50$ guirlandes dans le stock. Le
     stock est suffisamment grand pour que l'on puisse assimiler ce prélèvement à un
     tirage aléatoire avec remise. On admet que la proportion de guirlandes
     défectueuses est égale à $0,02$.
     \par
     Calculer la probabilité qu'au moins une guirlande soit défectueuse. Arrondir le
     résultat à $10^{-3}$.
     \item L'entreprise souhaite connaître l'opinion de ses clients quant à la qualité de ses
     guirlandes électriques. Pour cela elle souhaite obtenir, à partir d'un échantillon
     aléatoire, une estimation de la proportion de clients satisfaits au niveau de
     confiance de 95\,\% à l'aide d'un intervalle de confiance d'amplitude inférieure ou
     égale à 8\,\%.
     \par
     Combien l'entreprise doit-elle interroger de clients au minimum~?
\end{enumerate}

\end{document}

µ
\documentclass[a4paper]{article}

%================================================================================================================================
%
% Packages
%
%================================================================================================================================

\usepackage[T1]{fontenc} 	% pour caractères accentués
\usepackage[utf8]{inputenc}  % encodage utf8
\usepackage[french]{babel}	% langue : français
\usepackage{fourier}			% caractères plus lisibles
\usepackage[dvipsnames]{xcolor} % couleurs
\usepackage{fancyhdr}		% réglage header footer
\usepackage{needspace}		% empêcher sauts de page mal placés
\usepackage{graphicx}		% pour inclure des graphiques
\usepackage{enumitem,cprotect}		% personnalise les listes d'items (nécessaire pour ol, al ...)
\usepackage{hyperref}		% Liens hypertexte
\usepackage{pstricks,pst-all,pst-node,pstricks-add,pst-math,pst-plot,pst-tree,pst-eucl} % pstricks
\usepackage[a4paper,includeheadfoot,top=2cm,left=3cm, bottom=2cm,right=3cm]{geometry} % marges etc.
\usepackage{comment}			% commentaires multilignes
\usepackage{amsmath,environ} % maths (matrices, etc.)
\usepackage{amssymb,makeidx}
\usepackage{bm}				% bold maths
\usepackage{tabularx}		% tableaux
\usepackage{colortbl}		% tableaux en couleur
\usepackage{fontawesome}		% Fontawesome
\usepackage{environ}			% environment with command
\usepackage{fp}				% calculs pour ps-tricks
\usepackage{multido}			% pour ps tricks
\usepackage[np]{numprint}	% formattage nombre
\usepackage{tikz,tkz-tab} 			% package principal TikZ
\usepackage{pgfplots}   % axes
\usepackage{mathrsfs}    % cursives
\usepackage{calc}			% calcul taille boites
\usepackage[scaled=0.875]{helvet} % font sans serif
\usepackage{svg} % svg
\usepackage{scrextend} % local margin
\usepackage{scratch} %scratch
\usepackage{multicol} % colonnes
%\usepackage{infix-RPN,pst-func} % formule en notation polanaise inversée
\usepackage{listings}

%================================================================================================================================
%
% Réglages de base
%
%================================================================================================================================

\lstset{
language=Python,   % R code
literate=
{á}{{\'a}}1
{à}{{\`a}}1
{ã}{{\~a}}1
{é}{{\'e}}1
{è}{{\`e}}1
{ê}{{\^e}}1
{í}{{\'i}}1
{ó}{{\'o}}1
{õ}{{\~o}}1
{ú}{{\'u}}1
{ü}{{\"u}}1
{ç}{{\c{c}}}1
{~}{{ }}1
}


\definecolor{codegreen}{rgb}{0,0.6,0}
\definecolor{codegray}{rgb}{0.5,0.5,0.5}
\definecolor{codepurple}{rgb}{0.58,0,0.82}
\definecolor{backcolour}{rgb}{0.95,0.95,0.92}

\lstdefinestyle{mystyle}{
    backgroundcolor=\color{backcolour},   
    commentstyle=\color{codegreen},
    keywordstyle=\color{magenta},
    numberstyle=\tiny\color{codegray},
    stringstyle=\color{codepurple},
    basicstyle=\ttfamily\footnotesize,
    breakatwhitespace=false,         
    breaklines=true,                 
    captionpos=b,                    
    keepspaces=true,                 
    numbers=left,                    
xleftmargin=2em,
framexleftmargin=2em,            
    showspaces=false,                
    showstringspaces=false,
    showtabs=false,                  
    tabsize=2,
    upquote=true
}

\lstset{style=mystyle}


\lstset{style=mystyle}
\newcommand{\imgdir}{C:/laragon/www/newmc/assets/imgsvg/}
\newcommand{\imgsvgdir}{C:/laragon/www/newmc/assets/imgsvg/}

\definecolor{mcgris}{RGB}{220, 220, 220}% ancien~; pour compatibilité
\definecolor{mcbleu}{RGB}{52, 152, 219}
\definecolor{mcvert}{RGB}{125, 194, 70}
\definecolor{mcmauve}{RGB}{154, 0, 215}
\definecolor{mcorange}{RGB}{255, 96, 0}
\definecolor{mcturquoise}{RGB}{0, 153, 153}
\definecolor{mcrouge}{RGB}{255, 0, 0}
\definecolor{mclightvert}{RGB}{205, 234, 190}

\definecolor{gris}{RGB}{220, 220, 220}
\definecolor{bleu}{RGB}{52, 152, 219}
\definecolor{vert}{RGB}{125, 194, 70}
\definecolor{mauve}{RGB}{154, 0, 215}
\definecolor{orange}{RGB}{255, 96, 0}
\definecolor{turquoise}{RGB}{0, 153, 153}
\definecolor{rouge}{RGB}{255, 0, 0}
\definecolor{lightvert}{RGB}{205, 234, 190}
\setitemize[0]{label=\color{lightvert}  $\bullet$}

\pagestyle{fancy}
\renewcommand{\headrulewidth}{0.2pt}
\fancyhead[L]{maths-cours.fr}
\fancyhead[R]{\thepage}
\renewcommand{\footrulewidth}{0.2pt}
\fancyfoot[C]{}

\newcolumntype{C}{>{\centering\arraybackslash}X}
\newcolumntype{s}{>{\hsize=.35\hsize\arraybackslash}X}

\setlength{\parindent}{0pt}		 
\setlength{\parskip}{3mm}
\setlength{\headheight}{1cm}

\def\ebook{ebook}
\def\book{book}
\def\web{web}
\def\type{web}

\newcommand{\vect}[1]{\overrightarrow{\,\mathstrut#1\,}}

\def\Oij{$\left(\text{O}~;~\vect{\imath},~\vect{\jmath}\right)$}
\def\Oijk{$\left(\text{O}~;~\vect{\imath},~\vect{\jmath},~\vect{k}\right)$}
\def\Ouv{$\left(\text{O}~;~\vect{u},~\vect{v}\right)$}

\hypersetup{breaklinks=true, colorlinks = true, linkcolor = OliveGreen, urlcolor = OliveGreen, citecolor = OliveGreen, pdfauthor={Didier BONNEL - https://www.maths-cours.fr} } % supprime les bordures autour des liens

\renewcommand{\arg}[0]{\text{arg}}

\everymath{\displaystyle}

%================================================================================================================================
%
% Macros - Commandes
%
%================================================================================================================================

\newcommand\meta[2]{    			% Utilisé pour créer le post HTML.
	\def\titre{titre}
	\def\url{url}
	\def\arg{#1}
	\ifx\titre\arg
		\newcommand\maintitle{#2}
		\fancyhead[L]{#2}
		{\Large\sffamily \MakeUppercase{#2}}
		\vspace{1mm}\textcolor{mcvert}{\hrule}
	\fi 
	\ifx\url\arg
		\fancyfoot[L]{\href{https://www.maths-cours.fr#2}{\black \footnotesize{https://www.maths-cours.fr#2}}}
	\fi 
}


\newcommand\TitreC[1]{    		% Titre centré
     \needspace{3\baselineskip}
     \begin{center}\textbf{#1}\end{center}
}

\newcommand\newpar{    		% paragraphe
     \par
}

\newcommand\nosp {    		% commande vide (pas d'espace)
}
\newcommand{\id}[1]{} %ignore

\newcommand\boite[2]{				% Boite simple sans titre
	\vspace{5mm}
	\setlength{\fboxrule}{0.2mm}
	\setlength{\fboxsep}{5mm}	
	\fcolorbox{#1}{#1!3}{\makebox[\linewidth-2\fboxrule-2\fboxsep]{
  		\begin{minipage}[t]{\linewidth-2\fboxrule-4\fboxsep}\setlength{\parskip}{3mm}
  			 #2
  		\end{minipage}
	}}
	\vspace{5mm}
}

\newcommand\CBox[4]{				% Boites
	\vspace{5mm}
	\setlength{\fboxrule}{0.2mm}
	\setlength{\fboxsep}{5mm}
	
	\fcolorbox{#1}{#1!3}{\makebox[\linewidth-2\fboxrule-2\fboxsep]{
		\begin{minipage}[t]{1cm}\setlength{\parskip}{3mm}
	  		\textcolor{#1}{\LARGE{#2}}    
 	 	\end{minipage}  
  		\begin{minipage}[t]{\linewidth-2\fboxrule-4\fboxsep}\setlength{\parskip}{3mm}
			\raisebox{1.2mm}{\normalsize\sffamily{\textcolor{#1}{#3}}}						
  			 #4
  		\end{minipage}
	}}
	\vspace{5mm}
}

\newcommand\cadre[3]{				% Boites convertible html
	\par
	\vspace{2mm}
	\setlength{\fboxrule}{0.1mm}
	\setlength{\fboxsep}{5mm}
	\fcolorbox{#1}{white}{\makebox[\linewidth-2\fboxrule-2\fboxsep]{
  		\begin{minipage}[t]{\linewidth-2\fboxrule-4\fboxsep}\setlength{\parskip}{3mm}
			\raisebox{-2.5mm}{\sffamily \small{\textcolor{#1}{\MakeUppercase{#2}}}}		
			\par		
  			 #3
 	 		\end{minipage}
	}}
		\vspace{2mm}
	\par
}

\newcommand\bloc[3]{				% Boites convertible html sans bordure
     \needspace{2\baselineskip}
     {\sffamily \small{\textcolor{#1}{\MakeUppercase{#2}}}}    
		\par		
  			 #3
		\par
}

\newcommand\CHelp[1]{
     \CBox{Plum}{\faInfoCircle}{À RETENIR}{#1}
}

\newcommand\CUp[1]{
     \CBox{NavyBlue}{\faThumbsOUp}{EN PRATIQUE}{#1}
}

\newcommand\CInfo[1]{
     \CBox{Sepia}{\faArrowCircleRight}{REMARQUE}{#1}
}

\newcommand\CRedac[1]{
     \CBox{PineGreen}{\faEdit}{BIEN R\'EDIGER}{#1}
}

\newcommand\CError[1]{
     \CBox{Red}{\faExclamationTriangle}{ATTENTION}{#1}
}

\newcommand\TitreExo[2]{
\needspace{4\baselineskip}
 {\sffamily\large EXERCICE #1\ (\emph{#2 points})}
\vspace{5mm}
}

\newcommand\img[2]{
          \includegraphics[width=#2\paperwidth]{\imgdir#1}
}

\newcommand\imgsvg[2]{
       \begin{center}   \includegraphics[width=#2\paperwidth]{\imgsvgdir#1} \end{center}
}


\newcommand\Lien[2]{
     \href{#1}{#2 \tiny \faExternalLink}
}
\newcommand\mcLien[2]{
     \href{https~://www.maths-cours.fr/#1}{#2 \tiny \faExternalLink}
}

\newcommand{\euro}{\eurologo{}}

%================================================================================================================================
%
% Macros - Environement
%
%================================================================================================================================

\newenvironment{tex}{ %
}
{%
}

\newenvironment{indente}{ %
	\setlength\parindent{10mm}
}

{
	\setlength\parindent{0mm}
}

\newenvironment{corrige}{%
     \needspace{3\baselineskip}
     \medskip
     \textbf{\textsc{Corrigé}}
     \medskip
}
{
}

\newenvironment{extern}{%
     \begin{center}
     }
     {
     \end{center}
}

\NewEnviron{code}{%
	\par
     \boite{gray}{\texttt{%
     \BODY
     }}
     \par
}

\newenvironment{vbloc}{% boite sans cadre empeche saut de page
     \begin{minipage}[t]{\linewidth}
     }
     {
     \end{minipage}
}
\NewEnviron{h2}{%
    \needspace{3\baselineskip}
    \vspace{0.6cm}
	\noindent \MakeUppercase{\sffamily \large \BODY}
	\vspace{1mm}\textcolor{mcgris}{\hrule}\vspace{0.4cm}
	\par
}{}

\NewEnviron{h3}{%
    \needspace{3\baselineskip}
	\vspace{5mm}
	\textsc{\BODY}
	\par
}

\NewEnviron{margeneg}{ %
\begin{addmargin}[-1cm]{0cm}
\BODY
\end{addmargin}
}

\NewEnviron{html}{%
}

\begin{document}
\meta{url}{/exercices/fonctions-bac-es-l-centres-etrangers-2018/}
\meta{pid}{8664}
\meta{titre}{Fonctions - Bac ES/L Centres étrangers 2018}
\meta{type}{exercices}
%
\begin{h2}Exercice 5 (6 points)\end{h2}
\textbf{Commun à  tous les candidats}
\par
On considère la fonction dérivable $f$ définie sur $I = [0~;~20]$ par~:
\par
\[f(x) = 1~000(x + 5)\text{e}^{- 0,2x}.\]
\medskip
\TitreC{Partie A - Étude graphique}
\medskip
On a représenté sur le graphique ci-dessous, la courbe représentative de la
fonction $f$.
\par
\emph{Répondre aux questions suivantes par lecture graphique.}
\begin{center}
     \begin{extern}%width="600" alt="courbe représentative fonction Bac ES/L Centres étrangers 2018"
          \psset{xunit=0.6cm,yunit=0.0012cm,comma=true}
          \begin{pspicture}(-1,-200)(22,6000)
               \multido{\n=0.0+0.4}{56}{\psline[linecolor=gris,linewidth=0.4pt](\n,0)(\n,6000)}
               \multido{\n=0+2}{12}{\psline[linecolor=gris,linewidth=0.8pt](\n,0)(\n,6000)}
               \multido{\n=0+200}{31}{\psline[linecolor=gris,linewidth=0.4pt](0,\n)(22,\n)}
               \multido{\n=0+1000}{7}{\psline[linecolor=gris,linewidth=0.8pt](0,\n)(22,\n)}
               \multido{\n=0+1000}{6}{\uput[l](0,\n){\np{\n}}}
               \psaxes[linewidth=0.6pt,Dx=2,Dy=6000]{->}(0,0)(0,0)(22,6000)
               \psplot[plotpoints=3000,linewidth=0.75pt,linecolor=red]{0}{20}{x 5 add 1000 mul 2.71828 0.2 x mul exp div}
          \end{pspicture}
     \end{extern}
\end{center}
\bigskip
\begin{enumerate}
     \item Résoudre graphiquement et de façon approchée l'équation $f(x) = 3~000$.
     \item Donner graphiquement une valeur approchée de l'intégrale de [ entre 2 et 8 à
     une unité d'aire près. Justifier la démarche.
\end{enumerate}
\bigskip
\TitreC{Partie B - Étude théorique}
\medskip
\begin{enumerate}
     \item On note $f'$ la dérivée de la fonction [sur [0~;~20].
     \par
     Démontrer que pour tout $x$ de [0~;~20], $f'(x) = - 200x\text{e}^{-0,2x}$.
     \item  En déduire le sens de variation de $f$ et dresser son tableau des variations sur
     l'intervalle [0~;~20]. Si nécessaire, arrondir à l'unité les valeurs présentes dans
     le tableau.
     \item  Démontrer que l'équation $f(x) = 3~000$ admet une unique solution $\alpha$ sur
     [0~;~20], puis donner une valeur approchée de $\alpha$ à $10^{-2}$ près à l'aide de la
     calculatrice.
     \item  On admet que la fonction $F$ définie sur l'intervalle [0~;~20] par l'expression
     \par
     $F(x) = - 5~000(x + 10)\text{e}^{-0,2x}$ est une primitive de la fonction $f$ sur [0~;~20].
     \par
     Calculer $\displaystyle\int_2^8 f(x)\:\text{d}x$. On donnera la valeur exacte, puis la valeur arrondie à
     l'unité.
\end{enumerate}
\bigskip
\TitreC{Partie C - Application économique}
\medskip
La fonction de demande d'un produit est modélisée sur l'intervalle [0~;~20] par la
fonction $f$ étudiée dans les parties A et B.
\par
Le nombre $f(x)$ représente la quantité d'objets demandés lorsque le prix unitaire est
égal à $x$ euros.
\par
Utiliser les résultats de la partie B afin de répondre aux questions suivantes~:
\bigskip
\begin{enumerate}
     \item En-dessous de quel prix unitaire, arrondi au centime, la demande est-elle
     supérieure à 3~000~objets~?
     \item Déterminer la valeur moyenne de la fonction $f$ sur l'intervalle
     [2~;~8]. Interpréter ce résultat.
\end{enumerate}

\end{document}
µ
\documentclass[a4paper]{article}

%================================================================================================================================
%
% Packages
%
%================================================================================================================================

\usepackage[T1]{fontenc} 	% pour caractères accentués
\usepackage[utf8]{inputenc}  % encodage utf8
\usepackage[french]{babel}	% langue : français
\usepackage{fourier}			% caractères plus lisibles
\usepackage[dvipsnames]{xcolor} % couleurs
\usepackage{fancyhdr}		% réglage header footer
\usepackage{needspace}		% empêcher sauts de page mal placés
\usepackage{graphicx}		% pour inclure des graphiques
\usepackage{enumitem,cprotect}		% personnalise les listes d'items (nécessaire pour ol, al ...)
\usepackage{hyperref}		% Liens hypertexte
\usepackage{pstricks,pst-all,pst-node,pstricks-add,pst-math,pst-plot,pst-tree,pst-eucl} % pstricks
\usepackage[a4paper,includeheadfoot,top=2cm,left=3cm, bottom=2cm,right=3cm]{geometry} % marges etc.
\usepackage{comment}			% commentaires multilignes
\usepackage{amsmath,environ} % maths (matrices, etc.)
\usepackage{amssymb,makeidx}
\usepackage{bm}				% bold maths
\usepackage{tabularx}		% tableaux
\usepackage{colortbl}		% tableaux en couleur
\usepackage{fontawesome}		% Fontawesome
\usepackage{environ}			% environment with command
\usepackage{fp}				% calculs pour ps-tricks
\usepackage{multido}			% pour ps tricks
\usepackage[np]{numprint}	% formattage nombre
\usepackage{tikz,tkz-tab} 			% package principal TikZ
\usepackage{pgfplots}   % axes
\usepackage{mathrsfs}    % cursives
\usepackage{calc}			% calcul taille boites
\usepackage[scaled=0.875]{helvet} % font sans serif
\usepackage{svg} % svg
\usepackage{scrextend} % local margin
\usepackage{scratch} %scratch
\usepackage{multicol} % colonnes
%\usepackage{infix-RPN,pst-func} % formule en notation polanaise inversée
\usepackage{listings}

%================================================================================================================================
%
% Réglages de base
%
%================================================================================================================================

\lstset{
language=Python,   % R code
literate=
{á}{{\'a}}1
{à}{{\`a}}1
{ã}{{\~a}}1
{é}{{\'e}}1
{è}{{\`e}}1
{ê}{{\^e}}1
{í}{{\'i}}1
{ó}{{\'o}}1
{õ}{{\~o}}1
{ú}{{\'u}}1
{ü}{{\"u}}1
{ç}{{\c{c}}}1
{~}{{ }}1
}


\definecolor{codegreen}{rgb}{0,0.6,0}
\definecolor{codegray}{rgb}{0.5,0.5,0.5}
\definecolor{codepurple}{rgb}{0.58,0,0.82}
\definecolor{backcolour}{rgb}{0.95,0.95,0.92}

\lstdefinestyle{mystyle}{
    backgroundcolor=\color{backcolour},   
    commentstyle=\color{codegreen},
    keywordstyle=\color{magenta},
    numberstyle=\tiny\color{codegray},
    stringstyle=\color{codepurple},
    basicstyle=\ttfamily\footnotesize,
    breakatwhitespace=false,         
    breaklines=true,                 
    captionpos=b,                    
    keepspaces=true,                 
    numbers=left,                    
xleftmargin=2em,
framexleftmargin=2em,            
    showspaces=false,                
    showstringspaces=false,
    showtabs=false,                  
    tabsize=2,
    upquote=true
}

\lstset{style=mystyle}


\lstset{style=mystyle}
\newcommand{\imgdir}{C:/laragon/www/newmc/assets/imgsvg/}
\newcommand{\imgsvgdir}{C:/laragon/www/newmc/assets/imgsvg/}

\definecolor{mcgris}{RGB}{220, 220, 220}% ancien~; pour compatibilité
\definecolor{mcbleu}{RGB}{52, 152, 219}
\definecolor{mcvert}{RGB}{125, 194, 70}
\definecolor{mcmauve}{RGB}{154, 0, 215}
\definecolor{mcorange}{RGB}{255, 96, 0}
\definecolor{mcturquoise}{RGB}{0, 153, 153}
\definecolor{mcrouge}{RGB}{255, 0, 0}
\definecolor{mclightvert}{RGB}{205, 234, 190}

\definecolor{gris}{RGB}{220, 220, 220}
\definecolor{bleu}{RGB}{52, 152, 219}
\definecolor{vert}{RGB}{125, 194, 70}
\definecolor{mauve}{RGB}{154, 0, 215}
\definecolor{orange}{RGB}{255, 96, 0}
\definecolor{turquoise}{RGB}{0, 153, 153}
\definecolor{rouge}{RGB}{255, 0, 0}
\definecolor{lightvert}{RGB}{205, 234, 190}
\setitemize[0]{label=\color{lightvert}  $\bullet$}

\pagestyle{fancy}
\renewcommand{\headrulewidth}{0.2pt}
\fancyhead[L]{maths-cours.fr}
\fancyhead[R]{\thepage}
\renewcommand{\footrulewidth}{0.2pt}
\fancyfoot[C]{}

\newcolumntype{C}{>{\centering\arraybackslash}X}
\newcolumntype{s}{>{\hsize=.35\hsize\arraybackslash}X}

\setlength{\parindent}{0pt}		 
\setlength{\parskip}{3mm}
\setlength{\headheight}{1cm}

\def\ebook{ebook}
\def\book{book}
\def\web{web}
\def\type{web}

\newcommand{\vect}[1]{\overrightarrow{\,\mathstrut#1\,}}

\def\Oij{$\left(\text{O}~;~\vect{\imath},~\vect{\jmath}\right)$}
\def\Oijk{$\left(\text{O}~;~\vect{\imath},~\vect{\jmath},~\vect{k}\right)$}
\def\Ouv{$\left(\text{O}~;~\vect{u},~\vect{v}\right)$}

\hypersetup{breaklinks=true, colorlinks = true, linkcolor = OliveGreen, urlcolor = OliveGreen, citecolor = OliveGreen, pdfauthor={Didier BONNEL - https://www.maths-cours.fr} } % supprime les bordures autour des liens

\renewcommand{\arg}[0]{\text{arg}}

\everymath{\displaystyle}

%================================================================================================================================
%
% Macros - Commandes
%
%================================================================================================================================

\newcommand\meta[2]{    			% Utilisé pour créer le post HTML.
	\def\titre{titre}
	\def\url{url}
	\def\arg{#1}
	\ifx\titre\arg
		\newcommand\maintitle{#2}
		\fancyhead[L]{#2}
		{\Large\sffamily \MakeUppercase{#2}}
		\vspace{1mm}\textcolor{mcvert}{\hrule}
	\fi 
	\ifx\url\arg
		\fancyfoot[L]{\href{https://www.maths-cours.fr#2}{\black \footnotesize{https://www.maths-cours.fr#2}}}
	\fi 
}


\newcommand\TitreC[1]{    		% Titre centré
     \needspace{3\baselineskip}
     \begin{center}\textbf{#1}\end{center}
}

\newcommand\newpar{    		% paragraphe
     \par
}

\newcommand\nosp {    		% commande vide (pas d'espace)
}
\newcommand{\id}[1]{} %ignore

\newcommand\boite[2]{				% Boite simple sans titre
	\vspace{5mm}
	\setlength{\fboxrule}{0.2mm}
	\setlength{\fboxsep}{5mm}	
	\fcolorbox{#1}{#1!3}{\makebox[\linewidth-2\fboxrule-2\fboxsep]{
  		\begin{minipage}[t]{\linewidth-2\fboxrule-4\fboxsep}\setlength{\parskip}{3mm}
  			 #2
  		\end{minipage}
	}}
	\vspace{5mm}
}

\newcommand\CBox[4]{				% Boites
	\vspace{5mm}
	\setlength{\fboxrule}{0.2mm}
	\setlength{\fboxsep}{5mm}
	
	\fcolorbox{#1}{#1!3}{\makebox[\linewidth-2\fboxrule-2\fboxsep]{
		\begin{minipage}[t]{1cm}\setlength{\parskip}{3mm}
	  		\textcolor{#1}{\LARGE{#2}}    
 	 	\end{minipage}  
  		\begin{minipage}[t]{\linewidth-2\fboxrule-4\fboxsep}\setlength{\parskip}{3mm}
			\raisebox{1.2mm}{\normalsize\sffamily{\textcolor{#1}{#3}}}						
  			 #4
  		\end{minipage}
	}}
	\vspace{5mm}
}

\newcommand\cadre[3]{				% Boites convertible html
	\par
	\vspace{2mm}
	\setlength{\fboxrule}{0.1mm}
	\setlength{\fboxsep}{5mm}
	\fcolorbox{#1}{white}{\makebox[\linewidth-2\fboxrule-2\fboxsep]{
  		\begin{minipage}[t]{\linewidth-2\fboxrule-4\fboxsep}\setlength{\parskip}{3mm}
			\raisebox{-2.5mm}{\sffamily \small{\textcolor{#1}{\MakeUppercase{#2}}}}		
			\par		
  			 #3
 	 		\end{minipage}
	}}
		\vspace{2mm}
	\par
}

\newcommand\bloc[3]{				% Boites convertible html sans bordure
     \needspace{2\baselineskip}
     {\sffamily \small{\textcolor{#1}{\MakeUppercase{#2}}}}    
		\par		
  			 #3
		\par
}

\newcommand\CHelp[1]{
     \CBox{Plum}{\faInfoCircle}{À RETENIR}{#1}
}

\newcommand\CUp[1]{
     \CBox{NavyBlue}{\faThumbsOUp}{EN PRATIQUE}{#1}
}

\newcommand\CInfo[1]{
     \CBox{Sepia}{\faArrowCircleRight}{REMARQUE}{#1}
}

\newcommand\CRedac[1]{
     \CBox{PineGreen}{\faEdit}{BIEN R\'EDIGER}{#1}
}

\newcommand\CError[1]{
     \CBox{Red}{\faExclamationTriangle}{ATTENTION}{#1}
}

\newcommand\TitreExo[2]{
\needspace{4\baselineskip}
 {\sffamily\large EXERCICE #1\ (\emph{#2 points})}
\vspace{5mm}
}

\newcommand\img[2]{
          \includegraphics[width=#2\paperwidth]{\imgdir#1}
}

\newcommand\imgsvg[2]{
       \begin{center}   \includegraphics[width=#2\paperwidth]{\imgsvgdir#1} \end{center}
}


\newcommand\Lien[2]{
     \href{#1}{#2 \tiny \faExternalLink}
}
\newcommand\mcLien[2]{
     \href{https~://www.maths-cours.fr/#1}{#2 \tiny \faExternalLink}
}

\newcommand{\euro}{\eurologo{}}

%================================================================================================================================
%
% Macros - Environement
%
%================================================================================================================================

\newenvironment{tex}{ %
}
{%
}

\newenvironment{indente}{ %
	\setlength\parindent{10mm}
}

{
	\setlength\parindent{0mm}
}

\newenvironment{corrige}{%
     \needspace{3\baselineskip}
     \medskip
     \textbf{\textsc{Corrigé}}
     \medskip
}
{
}

\newenvironment{extern}{%
     \begin{center}
     }
     {
     \end{center}
}

\NewEnviron{code}{%
	\par
     \boite{gray}{\texttt{%
     \BODY
     }}
     \par
}

\newenvironment{vbloc}{% boite sans cadre empeche saut de page
     \begin{minipage}[t]{\linewidth}
     }
     {
     \end{minipage}
}
\NewEnviron{h2}{%
    \needspace{3\baselineskip}
    \vspace{0.6cm}
	\noindent \MakeUppercase{\sffamily \large \BODY}
	\vspace{1mm}\textcolor{mcgris}{\hrule}\vspace{0.4cm}
	\par
}{}

\NewEnviron{h3}{%
    \needspace{3\baselineskip}
	\vspace{5mm}
	\textsc{\BODY}
	\par
}

\NewEnviron{margeneg}{ %
\begin{addmargin}[-1cm]{0cm}
\BODY
\end{addmargin}
}

\NewEnviron{html}{%
}

\begin{document}
\meta{url}{/exercices/graphes-probabilistes-bac-es-l-centres-etrangers-2018-spe/}
\meta{pid}{8683}
\meta{titre}{Graphes probabilistes - Bac ES/L Centres étrangers  2018 (spé)}
\meta{type}{exercices}
%
\begin{h2}Exercice 2 (5 points)\end{h2}
\textbf{Candidats ayant choisi la spécialité \og mathématiques \fg{}}
\medskip
Une société d'autoroute étudie l'évolution de l'état de ses automates de péage en
l'absence de maintenance.
\par
Un automate peut se trouver dans l'un des états suivants~:
\begin{itemize}
     \item fonctionnel (F)~;
     \item en sursis (S) s'il fonctionne encore, mais montre des signes de faiblesse~;
     \item défaillant (D) s'il ne fonctionne plus.
\end{itemize}
La société a observé que d'un jour sur l'autre~:
\begin{itemize}
     \item concernant les automates fonctionnels, 90\,\% le restent et 10\,\% deviennent
     en sursis~;
     \item concernant les automates en sursis, 80\,\% le restent et 20\,\% deviennent
     défaillants.
\end{itemize}
\bigskip
\begin{enumerate}
     \item
     \begin{enumerate}[label=\alph*.]
          \item Reproduire et compléter le graphe probabiliste ci-après qui représente les
          évolutions possibles de l'état d'un automate.
          \begin{center}
               \begin{extern}%width="300" alt="Automate Graphe d'états" style="margin: -3em 0 0 -6em;"
                    \begin{pspicture*}(-1,-1)(7,1.5)%
                         \cnodeput(0,0){A}{\large F}%
                         \cnodeput(3,0){B}{\large S}%
                         \cnodeput(6,0){C}{\large D}%
                         \ncline{->}{A}{B}%
                         \ncline{->}{B}{C}%
                         %\ncloop{A}{A}\ncloop{B}{B}\ncloop{C}{C}
                         \nccircle[angleA=0]{->}{A}{0.4cm}\Bput{\ldots}%
                         \nccircle[angleA=0]{->}{C}{0.4cm}\Bput{1}%
                         \nccircle[angleA=0]{->}{B}{0.4cm}\Bput{\ldots}%
                         \uput[u](1.5,0){\ldots}%
                         \uput[u](4.5,0){\ldots}%
                    \end{pspicture*}
               \end{extern}
          \end{center}
          \item  Interpréter le nombre 1 qui apparaît sur ce graphe.
          \item  Voici la matrice de transition $M = \begin{pmatrix}0,9 &0,1 &0\\0 &0,8 &0,2\\0&0&1\end{pmatrix}$ associée à ce graphe en prenant les sommets dans l'ordre F, S, D.
          Préciser la signification du coefficient $0,2$ dans cette matrice.
     \end{enumerate}
     \item  À compter d'une certaine date, la société relève chaque jour à midi l'état de
     ses automates. On note ainsi pour tout entier naturel $n$~:
     \begin{itemize}
          \item$f_n$ la probabilité qu'un automate soit fonctionnelle $n$-ième jour~;
          \item $s_n$ la probabilité qu'un automate soit en sursis le $n$-ième jour~;
          \item $d_n$ la probabilité qu'un automate soit défaillant le $n$-ième jour.
     \end{itemize}
     On note alors $P_n = \begin{pmatrix}f_n &s_n&d_n \end{pmatrix}$ la matrice ligne de l'état probabiliste le $n$-ième jour.
     Enfin, la société observe qu'au début de l'expérience tous ses automates sont fonctionnels~: on a donc $P_0 = \begin{pmatrix}1 &0 &0\end{pmatrix}$.
     \begin{enumerate}[label=\alph*.]
          \item Calculer $P_1$.
          \item Montrer que, le 3\up{e} jour, l'état probabiliste est $\begin{pmatrix}0,729 &0,217 &0,054\end{pmatrix}$.
          \item  Vérifier que ce graphe possède un unique état stable $P = \begin{pmatrix}0 &0 &1\end{pmatrix}$.
          Quelle est la signification de ce résultat pour la situation étudiée~?
     \end{enumerate}
     \item
     \begin{enumerate}[label=\alph*.]
          \item Justifier que pour tout entier naturel $n$, $s_{n+1} = 0,1f_n + 0,8s_n$.
          \item On vérifierait de même que pour tout entier naturel $n$,
          \[d_{n+1} = 0,2s_n + d_n \quad \text{et } \:f_{n+1} = 0,9f_n.\]
          Compléter l'algorithme ci-dessous de sorte qu'il affiche le nombre de jours au
          bout duquel 30\,\% des automates ne fonctionnent plus.
          \begin{center}
               \begin{extern}%width="280" alt="Algorithme automates Bac ES/L Centres étrangers  2018"
                    \begin{tabularx}{0.4\linewidth}{|X|}\hline
                         $D \gets 0$\\
                         $S \gets ...$\\
                         $F \gets 1$\\
                         $N \gets 0$\\
                         Tant que \ldots\\
                         \hspace{1cm}$D \gets 0,2 \times S + D$\\
                         \hspace{1cm}$S \gets 0,1 \times F + 0,8 \times S$\\
                         \hspace{1cm}$F \gets 0,9 \times F$\\
                         \hspace{1cm}$N \gets ...$\\
                         Fin Tant que\\
                         Afficher \ldots\\ \hline
                    \end{tabularx}
               \end{extern}
          \end{center}
          \item  Au bout de combien de jours la proportion d'automates défaillants devient-elle
          supérieure à 30\,\%~?
          \item  Dans le codage de la boucle \og Tant que \fg, l'ordre d'affectation des
          variables $D$, $S$ et $F$ est-il important~? Justifier.
     \end{enumerate}
\end{enumerate}

\end{document}
µ
\documentclass[a4paper]{article}

%================================================================================================================================
%
% Packages
%
%================================================================================================================================

\usepackage[T1]{fontenc} 	% pour caractères accentués
\usepackage[utf8]{inputenc}  % encodage utf8
\usepackage[french]{babel}	% langue : français
\usepackage{fourier}			% caractères plus lisibles
\usepackage[dvipsnames]{xcolor} % couleurs
\usepackage{fancyhdr}		% réglage header footer
\usepackage{needspace}		% empêcher sauts de page mal placés
\usepackage{graphicx}		% pour inclure des graphiques
\usepackage{enumitem,cprotect}		% personnalise les listes d'items (nécessaire pour ol, al ...)
\usepackage{hyperref}		% Liens hypertexte
\usepackage{pstricks,pst-all,pst-node,pstricks-add,pst-math,pst-plot,pst-tree,pst-eucl} % pstricks
\usepackage[a4paper,includeheadfoot,top=2cm,left=3cm, bottom=2cm,right=3cm]{geometry} % marges etc.
\usepackage{comment}			% commentaires multilignes
\usepackage{amsmath,environ} % maths (matrices, etc.)
\usepackage{amssymb,makeidx}
\usepackage{bm}				% bold maths
\usepackage{tabularx}		% tableaux
\usepackage{colortbl}		% tableaux en couleur
\usepackage{fontawesome}		% Fontawesome
\usepackage{environ}			% environment with command
\usepackage{fp}				% calculs pour ps-tricks
\usepackage{multido}			% pour ps tricks
\usepackage[np]{numprint}	% formattage nombre
\usepackage{tikz,tkz-tab} 			% package principal TikZ
\usepackage{pgfplots}   % axes
\usepackage{mathrsfs}    % cursives
\usepackage{calc}			% calcul taille boites
\usepackage[scaled=0.875]{helvet} % font sans serif
\usepackage{svg} % svg
\usepackage{scrextend} % local margin
\usepackage{scratch} %scratch
\usepackage{multicol} % colonnes
%\usepackage{infix-RPN,pst-func} % formule en notation polanaise inversée
\usepackage{listings}

%================================================================================================================================
%
% Réglages de base
%
%================================================================================================================================

\lstset{
language=Python,   % R code
literate=
{á}{{\'a}}1
{à}{{\`a}}1
{ã}{{\~a}}1
{é}{{\'e}}1
{è}{{\`e}}1
{ê}{{\^e}}1
{í}{{\'i}}1
{ó}{{\'o}}1
{õ}{{\~o}}1
{ú}{{\'u}}1
{ü}{{\"u}}1
{ç}{{\c{c}}}1
{~}{{ }}1
}


\definecolor{codegreen}{rgb}{0,0.6,0}
\definecolor{codegray}{rgb}{0.5,0.5,0.5}
\definecolor{codepurple}{rgb}{0.58,0,0.82}
\definecolor{backcolour}{rgb}{0.95,0.95,0.92}

\lstdefinestyle{mystyle}{
    backgroundcolor=\color{backcolour},   
    commentstyle=\color{codegreen},
    keywordstyle=\color{magenta},
    numberstyle=\tiny\color{codegray},
    stringstyle=\color{codepurple},
    basicstyle=\ttfamily\footnotesize,
    breakatwhitespace=false,         
    breaklines=true,                 
    captionpos=b,                    
    keepspaces=true,                 
    numbers=left,                    
xleftmargin=2em,
framexleftmargin=2em,            
    showspaces=false,                
    showstringspaces=false,
    showtabs=false,                  
    tabsize=2,
    upquote=true
}

\lstset{style=mystyle}


\lstset{style=mystyle}
\newcommand{\imgdir}{C:/laragon/www/newmc/assets/imgsvg/}
\newcommand{\imgsvgdir}{C:/laragon/www/newmc/assets/imgsvg/}

\definecolor{mcgris}{RGB}{220, 220, 220}% ancien~; pour compatibilité
\definecolor{mcbleu}{RGB}{52, 152, 219}
\definecolor{mcvert}{RGB}{125, 194, 70}
\definecolor{mcmauve}{RGB}{154, 0, 215}
\definecolor{mcorange}{RGB}{255, 96, 0}
\definecolor{mcturquoise}{RGB}{0, 153, 153}
\definecolor{mcrouge}{RGB}{255, 0, 0}
\definecolor{mclightvert}{RGB}{205, 234, 190}

\definecolor{gris}{RGB}{220, 220, 220}
\definecolor{bleu}{RGB}{52, 152, 219}
\definecolor{vert}{RGB}{125, 194, 70}
\definecolor{mauve}{RGB}{154, 0, 215}
\definecolor{orange}{RGB}{255, 96, 0}
\definecolor{turquoise}{RGB}{0, 153, 153}
\definecolor{rouge}{RGB}{255, 0, 0}
\definecolor{lightvert}{RGB}{205, 234, 190}
\setitemize[0]{label=\color{lightvert}  $\bullet$}

\pagestyle{fancy}
\renewcommand{\headrulewidth}{0.2pt}
\fancyhead[L]{maths-cours.fr}
\fancyhead[R]{\thepage}
\renewcommand{\footrulewidth}{0.2pt}
\fancyfoot[C]{}

\newcolumntype{C}{>{\centering\arraybackslash}X}
\newcolumntype{s}{>{\hsize=.35\hsize\arraybackslash}X}

\setlength{\parindent}{0pt}		 
\setlength{\parskip}{3mm}
\setlength{\headheight}{1cm}

\def\ebook{ebook}
\def\book{book}
\def\web{web}
\def\type{web}

\newcommand{\vect}[1]{\overrightarrow{\,\mathstrut#1\,}}

\def\Oij{$\left(\text{O}~;~\vect{\imath},~\vect{\jmath}\right)$}
\def\Oijk{$\left(\text{O}~;~\vect{\imath},~\vect{\jmath},~\vect{k}\right)$}
\def\Ouv{$\left(\text{O}~;~\vect{u},~\vect{v}\right)$}

\hypersetup{breaklinks=true, colorlinks = true, linkcolor = OliveGreen, urlcolor = OliveGreen, citecolor = OliveGreen, pdfauthor={Didier BONNEL - https://www.maths-cours.fr} } % supprime les bordures autour des liens

\renewcommand{\arg}[0]{\text{arg}}

\everymath{\displaystyle}

%================================================================================================================================
%
% Macros - Commandes
%
%================================================================================================================================

\newcommand\meta[2]{    			% Utilisé pour créer le post HTML.
	\def\titre{titre}
	\def\url{url}
	\def\arg{#1}
	\ifx\titre\arg
		\newcommand\maintitle{#2}
		\fancyhead[L]{#2}
		{\Large\sffamily \MakeUppercase{#2}}
		\vspace{1mm}\textcolor{mcvert}{\hrule}
	\fi 
	\ifx\url\arg
		\fancyfoot[L]{\href{https://www.maths-cours.fr#2}{\black \footnotesize{https://www.maths-cours.fr#2}}}
	\fi 
}


\newcommand\TitreC[1]{    		% Titre centré
     \needspace{3\baselineskip}
     \begin{center}\textbf{#1}\end{center}
}

\newcommand\newpar{    		% paragraphe
     \par
}

\newcommand\nosp {    		% commande vide (pas d'espace)
}
\newcommand{\id}[1]{} %ignore

\newcommand\boite[2]{				% Boite simple sans titre
	\vspace{5mm}
	\setlength{\fboxrule}{0.2mm}
	\setlength{\fboxsep}{5mm}	
	\fcolorbox{#1}{#1!3}{\makebox[\linewidth-2\fboxrule-2\fboxsep]{
  		\begin{minipage}[t]{\linewidth-2\fboxrule-4\fboxsep}\setlength{\parskip}{3mm}
  			 #2
  		\end{minipage}
	}}
	\vspace{5mm}
}

\newcommand\CBox[4]{				% Boites
	\vspace{5mm}
	\setlength{\fboxrule}{0.2mm}
	\setlength{\fboxsep}{5mm}
	
	\fcolorbox{#1}{#1!3}{\makebox[\linewidth-2\fboxrule-2\fboxsep]{
		\begin{minipage}[t]{1cm}\setlength{\parskip}{3mm}
	  		\textcolor{#1}{\LARGE{#2}}    
 	 	\end{minipage}  
  		\begin{minipage}[t]{\linewidth-2\fboxrule-4\fboxsep}\setlength{\parskip}{3mm}
			\raisebox{1.2mm}{\normalsize\sffamily{\textcolor{#1}{#3}}}						
  			 #4
  		\end{minipage}
	}}
	\vspace{5mm}
}

\newcommand\cadre[3]{				% Boites convertible html
	\par
	\vspace{2mm}
	\setlength{\fboxrule}{0.1mm}
	\setlength{\fboxsep}{5mm}
	\fcolorbox{#1}{white}{\makebox[\linewidth-2\fboxrule-2\fboxsep]{
  		\begin{minipage}[t]{\linewidth-2\fboxrule-4\fboxsep}\setlength{\parskip}{3mm}
			\raisebox{-2.5mm}{\sffamily \small{\textcolor{#1}{\MakeUppercase{#2}}}}		
			\par		
  			 #3
 	 		\end{minipage}
	}}
		\vspace{2mm}
	\par
}

\newcommand\bloc[3]{				% Boites convertible html sans bordure
     \needspace{2\baselineskip}
     {\sffamily \small{\textcolor{#1}{\MakeUppercase{#2}}}}    
		\par		
  			 #3
		\par
}

\newcommand\CHelp[1]{
     \CBox{Plum}{\faInfoCircle}{À RETENIR}{#1}
}

\newcommand\CUp[1]{
     \CBox{NavyBlue}{\faThumbsOUp}{EN PRATIQUE}{#1}
}

\newcommand\CInfo[1]{
     \CBox{Sepia}{\faArrowCircleRight}{REMARQUE}{#1}
}

\newcommand\CRedac[1]{
     \CBox{PineGreen}{\faEdit}{BIEN R\'EDIGER}{#1}
}

\newcommand\CError[1]{
     \CBox{Red}{\faExclamationTriangle}{ATTENTION}{#1}
}

\newcommand\TitreExo[2]{
\needspace{4\baselineskip}
 {\sffamily\large EXERCICE #1\ (\emph{#2 points})}
\vspace{5mm}
}

\newcommand\img[2]{
          \includegraphics[width=#2\paperwidth]{\imgdir#1}
}

\newcommand\imgsvg[2]{
       \begin{center}   \includegraphics[width=#2\paperwidth]{\imgsvgdir#1} \end{center}
}


\newcommand\Lien[2]{
     \href{#1}{#2 \tiny \faExternalLink}
}
\newcommand\mcLien[2]{
     \href{https~://www.maths-cours.fr/#1}{#2 \tiny \faExternalLink}
}

\newcommand{\euro}{\eurologo{}}

%================================================================================================================================
%
% Macros - Environement
%
%================================================================================================================================

\newenvironment{tex}{ %
}
{%
}

\newenvironment{indente}{ %
	\setlength\parindent{10mm}
}

{
	\setlength\parindent{0mm}
}

\newenvironment{corrige}{%
     \needspace{3\baselineskip}
     \medskip
     \textbf{\textsc{Corrigé}}
     \medskip
}
{
}

\newenvironment{extern}{%
     \begin{center}
     }
     {
     \end{center}
}

\NewEnviron{code}{%
	\par
     \boite{gray}{\texttt{%
     \BODY
     }}
     \par
}

\newenvironment{vbloc}{% boite sans cadre empeche saut de page
     \begin{minipage}[t]{\linewidth}
     }
     {
     \end{minipage}
}
\NewEnviron{h2}{%
    \needspace{3\baselineskip}
    \vspace{0.6cm}
	\noindent \MakeUppercase{\sffamily \large \BODY}
	\vspace{1mm}\textcolor{mcgris}{\hrule}\vspace{0.4cm}
	\par
}{}

\NewEnviron{h3}{%
    \needspace{3\baselineskip}
	\vspace{5mm}
	\textsc{\BODY}
	\par
}

\NewEnviron{margeneg}{ %
\begin{addmargin}[-1cm]{0cm}
\BODY
\end{addmargin}
}

\NewEnviron{html}{%
}

\begin{document}
\meta{url}{/exercices/probabilites-bac-s-antilles-guyane-2018/}
\meta{pid}{9063}
\meta{titre}{Probabilités - Bac S Antilles-Guyane 2018}
\meta{type}{exercices}
%
\begin{h2}Exercice 1 (5 points)\end{h2}
\smallbreak
\textit{Commun à tous les candidats}
\bigbreak
L'exploitant d'une forêt communale décide d'abattre des arbres afin de les vendre, soit aux habitants, soit à des entreprises. On admet que~:
\begin{itemize}
     \item
     parmi les arbres abattus, 30~\% sont des chênes, 50~\% sont des sapins et les autres sont des arbres d'essence secondaire (ce qui signifie qu'ils sont de moindre valeur)~;
     \item 45,9~\% des chênes et 80~\% des sapins abattus sont vendus aux habitants de la commune~;
     \item les trois quarts des arbres d'essence secondaire abattus sont vendus à des entreprises.
\end{itemize}
\medbreak
\TitreC{Partie A}
\smallbreak
Parmi les arbres abattus, on en choisit un au hasard.
\par
On considère les événements suivants~:
\begin{itemize}
     \item $C$~: \og l'arbre abattu est un chêne\fg{}~;
     \item $S$~: \og l'arbre abattu est un sapin\fg{}~;
     \item $E$~: \og l'arbre abattu est un arbre d'essence secondaire\fg{}~;
     \item $H$~: \og l'arbre abattu est vendu à un habitant de la commune\fg{}.
\end{itemize}
\bigbreak
\begin{enumerate}
     \item Construire un arbre pondéré complet traduisant la situation.
     \item Calculer la probabilité que l'arbre abattu soit un chêne  vendu à un habitant de la commune.
     \item Justifier que la probabilité que l'arbre abattu soit vendu à un habitant de la commune est égale à 0,587~7.
     \item Quelle est la probabilité qu'un arbre abattu vendu à un habitant de la commune soit un sapin~?\\On donnera le résultat arrondi  à $10^{-3}$.
\end{enumerate}
\medbreak
\TitreC{Partie B}
\smallbreak
Le nombre d'arbres sur un hectare de cette forêt peut être modélisé par une variable aléatoire $X$ suivant une loi normale d'espérance $\mu=4~000$ et d'écart-type $\sigma=300$.
\begin{enumerate}
     \item Déterminer la probabilité qu'il y ait entre 3~400 et 4~600 arbres sur un hectaure donné de cette forêt. On donnera le résultat arrondi à $10^{-3}$.
     \item Calculer la probabilité qu'il y ait plus de 4~500 arbres sur un hectare donné de cette forêt.
     On donnera le résultat arrondi à $10^{-3}$.
\end{enumerate}
\medbreak
\TitreC{Partie C}
\smallbreak
L'exploitant affirme que la densité de sapins dans cette forêt communale est de 1 sapin pour 2 arbres.
\smallbreak
Sur une parcelle, on a compté 106 sapins dans un échantillon de 200 arbres.\\
Ce résultat remet-il en cause l'affirmation de l'exploitant~?
\par

\end{document}
µ
\documentclass[a4paper]{article}

%================================================================================================================================
%
% Packages
%
%================================================================================================================================

\usepackage[T1]{fontenc} 	% pour caractères accentués
\usepackage[utf8]{inputenc}  % encodage utf8
\usepackage[french]{babel}	% langue : français
\usepackage{fourier}			% caractères plus lisibles
\usepackage[dvipsnames]{xcolor} % couleurs
\usepackage{fancyhdr}		% réglage header footer
\usepackage{needspace}		% empêcher sauts de page mal placés
\usepackage{graphicx}		% pour inclure des graphiques
\usepackage{enumitem,cprotect}		% personnalise les listes d'items (nécessaire pour ol, al ...)
\usepackage{hyperref}		% Liens hypertexte
\usepackage{pstricks,pst-all,pst-node,pstricks-add,pst-math,pst-plot,pst-tree,pst-eucl} % pstricks
\usepackage[a4paper,includeheadfoot,top=2cm,left=3cm, bottom=2cm,right=3cm]{geometry} % marges etc.
\usepackage{comment}			% commentaires multilignes
\usepackage{amsmath,environ} % maths (matrices, etc.)
\usepackage{amssymb,makeidx}
\usepackage{bm}				% bold maths
\usepackage{tabularx}		% tableaux
\usepackage{colortbl}		% tableaux en couleur
\usepackage{fontawesome}		% Fontawesome
\usepackage{environ}			% environment with command
\usepackage{fp}				% calculs pour ps-tricks
\usepackage{multido}			% pour ps tricks
\usepackage[np]{numprint}	% formattage nombre
\usepackage{tikz,tkz-tab} 			% package principal TikZ
\usepackage{pgfplots}   % axes
\usepackage{mathrsfs}    % cursives
\usepackage{calc}			% calcul taille boites
\usepackage[scaled=0.875]{helvet} % font sans serif
\usepackage{svg} % svg
\usepackage{scrextend} % local margin
\usepackage{scratch} %scratch
\usepackage{multicol} % colonnes
%\usepackage{infix-RPN,pst-func} % formule en notation polanaise inversée
\usepackage{listings}

%================================================================================================================================
%
% Réglages de base
%
%================================================================================================================================

\lstset{
language=Python,   % R code
literate=
{á}{{\'a}}1
{à}{{\`a}}1
{ã}{{\~a}}1
{é}{{\'e}}1
{è}{{\`e}}1
{ê}{{\^e}}1
{í}{{\'i}}1
{ó}{{\'o}}1
{õ}{{\~o}}1
{ú}{{\'u}}1
{ü}{{\"u}}1
{ç}{{\c{c}}}1
{~}{{ }}1
}


\definecolor{codegreen}{rgb}{0,0.6,0}
\definecolor{codegray}{rgb}{0.5,0.5,0.5}
\definecolor{codepurple}{rgb}{0.58,0,0.82}
\definecolor{backcolour}{rgb}{0.95,0.95,0.92}

\lstdefinestyle{mystyle}{
    backgroundcolor=\color{backcolour},   
    commentstyle=\color{codegreen},
    keywordstyle=\color{magenta},
    numberstyle=\tiny\color{codegray},
    stringstyle=\color{codepurple},
    basicstyle=\ttfamily\footnotesize,
    breakatwhitespace=false,         
    breaklines=true,                 
    captionpos=b,                    
    keepspaces=true,                 
    numbers=left,                    
xleftmargin=2em,
framexleftmargin=2em,            
    showspaces=false,                
    showstringspaces=false,
    showtabs=false,                  
    tabsize=2,
    upquote=true
}

\lstset{style=mystyle}


\lstset{style=mystyle}
\newcommand{\imgdir}{C:/laragon/www/newmc/assets/imgsvg/}
\newcommand{\imgsvgdir}{C:/laragon/www/newmc/assets/imgsvg/}

\definecolor{mcgris}{RGB}{220, 220, 220}% ancien~; pour compatibilité
\definecolor{mcbleu}{RGB}{52, 152, 219}
\definecolor{mcvert}{RGB}{125, 194, 70}
\definecolor{mcmauve}{RGB}{154, 0, 215}
\definecolor{mcorange}{RGB}{255, 96, 0}
\definecolor{mcturquoise}{RGB}{0, 153, 153}
\definecolor{mcrouge}{RGB}{255, 0, 0}
\definecolor{mclightvert}{RGB}{205, 234, 190}

\definecolor{gris}{RGB}{220, 220, 220}
\definecolor{bleu}{RGB}{52, 152, 219}
\definecolor{vert}{RGB}{125, 194, 70}
\definecolor{mauve}{RGB}{154, 0, 215}
\definecolor{orange}{RGB}{255, 96, 0}
\definecolor{turquoise}{RGB}{0, 153, 153}
\definecolor{rouge}{RGB}{255, 0, 0}
\definecolor{lightvert}{RGB}{205, 234, 190}
\setitemize[0]{label=\color{lightvert}  $\bullet$}

\pagestyle{fancy}
\renewcommand{\headrulewidth}{0.2pt}
\fancyhead[L]{maths-cours.fr}
\fancyhead[R]{\thepage}
\renewcommand{\footrulewidth}{0.2pt}
\fancyfoot[C]{}

\newcolumntype{C}{>{\centering\arraybackslash}X}
\newcolumntype{s}{>{\hsize=.35\hsize\arraybackslash}X}

\setlength{\parindent}{0pt}		 
\setlength{\parskip}{3mm}
\setlength{\headheight}{1cm}

\def\ebook{ebook}
\def\book{book}
\def\web{web}
\def\type{web}

\newcommand{\vect}[1]{\overrightarrow{\,\mathstrut#1\,}}

\def\Oij{$\left(\text{O}~;~\vect{\imath},~\vect{\jmath}\right)$}
\def\Oijk{$\left(\text{O}~;~\vect{\imath},~\vect{\jmath},~\vect{k}\right)$}
\def\Ouv{$\left(\text{O}~;~\vect{u},~\vect{v}\right)$}

\hypersetup{breaklinks=true, colorlinks = true, linkcolor = OliveGreen, urlcolor = OliveGreen, citecolor = OliveGreen, pdfauthor={Didier BONNEL - https://www.maths-cours.fr} } % supprime les bordures autour des liens

\renewcommand{\arg}[0]{\text{arg}}

\everymath{\displaystyle}

%================================================================================================================================
%
% Macros - Commandes
%
%================================================================================================================================

\newcommand\meta[2]{    			% Utilisé pour créer le post HTML.
	\def\titre{titre}
	\def\url{url}
	\def\arg{#1}
	\ifx\titre\arg
		\newcommand\maintitle{#2}
		\fancyhead[L]{#2}
		{\Large\sffamily \MakeUppercase{#2}}
		\vspace{1mm}\textcolor{mcvert}{\hrule}
	\fi 
	\ifx\url\arg
		\fancyfoot[L]{\href{https://www.maths-cours.fr#2}{\black \footnotesize{https://www.maths-cours.fr#2}}}
	\fi 
}


\newcommand\TitreC[1]{    		% Titre centré
     \needspace{3\baselineskip}
     \begin{center}\textbf{#1}\end{center}
}

\newcommand\newpar{    		% paragraphe
     \par
}

\newcommand\nosp {    		% commande vide (pas d'espace)
}
\newcommand{\id}[1]{} %ignore

\newcommand\boite[2]{				% Boite simple sans titre
	\vspace{5mm}
	\setlength{\fboxrule}{0.2mm}
	\setlength{\fboxsep}{5mm}	
	\fcolorbox{#1}{#1!3}{\makebox[\linewidth-2\fboxrule-2\fboxsep]{
  		\begin{minipage}[t]{\linewidth-2\fboxrule-4\fboxsep}\setlength{\parskip}{3mm}
  			 #2
  		\end{minipage}
	}}
	\vspace{5mm}
}

\newcommand\CBox[4]{				% Boites
	\vspace{5mm}
	\setlength{\fboxrule}{0.2mm}
	\setlength{\fboxsep}{5mm}
	
	\fcolorbox{#1}{#1!3}{\makebox[\linewidth-2\fboxrule-2\fboxsep]{
		\begin{minipage}[t]{1cm}\setlength{\parskip}{3mm}
	  		\textcolor{#1}{\LARGE{#2}}    
 	 	\end{minipage}  
  		\begin{minipage}[t]{\linewidth-2\fboxrule-4\fboxsep}\setlength{\parskip}{3mm}
			\raisebox{1.2mm}{\normalsize\sffamily{\textcolor{#1}{#3}}}						
  			 #4
  		\end{minipage}
	}}
	\vspace{5mm}
}

\newcommand\cadre[3]{				% Boites convertible html
	\par
	\vspace{2mm}
	\setlength{\fboxrule}{0.1mm}
	\setlength{\fboxsep}{5mm}
	\fcolorbox{#1}{white}{\makebox[\linewidth-2\fboxrule-2\fboxsep]{
  		\begin{minipage}[t]{\linewidth-2\fboxrule-4\fboxsep}\setlength{\parskip}{3mm}
			\raisebox{-2.5mm}{\sffamily \small{\textcolor{#1}{\MakeUppercase{#2}}}}		
			\par		
  			 #3
 	 		\end{minipage}
	}}
		\vspace{2mm}
	\par
}

\newcommand\bloc[3]{				% Boites convertible html sans bordure
     \needspace{2\baselineskip}
     {\sffamily \small{\textcolor{#1}{\MakeUppercase{#2}}}}    
		\par		
  			 #3
		\par
}

\newcommand\CHelp[1]{
     \CBox{Plum}{\faInfoCircle}{À RETENIR}{#1}
}

\newcommand\CUp[1]{
     \CBox{NavyBlue}{\faThumbsOUp}{EN PRATIQUE}{#1}
}

\newcommand\CInfo[1]{
     \CBox{Sepia}{\faArrowCircleRight}{REMARQUE}{#1}
}

\newcommand\CRedac[1]{
     \CBox{PineGreen}{\faEdit}{BIEN R\'EDIGER}{#1}
}

\newcommand\CError[1]{
     \CBox{Red}{\faExclamationTriangle}{ATTENTION}{#1}
}

\newcommand\TitreExo[2]{
\needspace{4\baselineskip}
 {\sffamily\large EXERCICE #1\ (\emph{#2 points})}
\vspace{5mm}
}

\newcommand\img[2]{
          \includegraphics[width=#2\paperwidth]{\imgdir#1}
}

\newcommand\imgsvg[2]{
       \begin{center}   \includegraphics[width=#2\paperwidth]{\imgsvgdir#1} \end{center}
}


\newcommand\Lien[2]{
     \href{#1}{#2 \tiny \faExternalLink}
}
\newcommand\mcLien[2]{
     \href{https~://www.maths-cours.fr/#1}{#2 \tiny \faExternalLink}
}

\newcommand{\euro}{\eurologo{}}

%================================================================================================================================
%
% Macros - Environement
%
%================================================================================================================================

\newenvironment{tex}{ %
}
{%
}

\newenvironment{indente}{ %
	\setlength\parindent{10mm}
}

{
	\setlength\parindent{0mm}
}

\newenvironment{corrige}{%
     \needspace{3\baselineskip}
     \medskip
     \textbf{\textsc{Corrigé}}
     \medskip
}
{
}

\newenvironment{extern}{%
     \begin{center}
     }
     {
     \end{center}
}

\NewEnviron{code}{%
	\par
     \boite{gray}{\texttt{%
     \BODY
     }}
     \par
}

\newenvironment{vbloc}{% boite sans cadre empeche saut de page
     \begin{minipage}[t]{\linewidth}
     }
     {
     \end{minipage}
}
\NewEnviron{h2}{%
    \needspace{3\baselineskip}
    \vspace{0.6cm}
	\noindent \MakeUppercase{\sffamily \large \BODY}
	\vspace{1mm}\textcolor{mcgris}{\hrule}\vspace{0.4cm}
	\par
}{}

\NewEnviron{h3}{%
    \needspace{3\baselineskip}
	\vspace{5mm}
	\textsc{\BODY}
	\par
}

\NewEnviron{margeneg}{ %
\begin{addmargin}[-1cm]{0cm}
\BODY
\end{addmargin}
}

\NewEnviron{html}{%
}

\begin{document}
\meta{url}{/exercices/geometrie-dans-lespace-bac-s-antilles-guyane-2018/}
\meta{pid}{9070}
\meta{titre}{Géométrie dans l'espace - Bac S Antilles-Guyane 2018}
\meta{type}{exercices}
%
\begin{h2}Exercice 2 (5 points)\end{h2}
\smallbreak
\textit{Commun à tous les candidats}
\bigbreak
Un artiste souhaite réaliser une sculpture composée d'un tétraèdre posé sur un cube de 6 mètres d'arête.
\par
Ces deux solides sont représentés par le cube $ABCDEFGH$ et par le tétraèdre $SELM$ ci-dessous.
\begin{center}
     \begin{extern}%width="300" alt="Section cube Bac S Antilles-Guyane"
          \psset{unit=1.3cm}
          \resizebox{7cm}{!}{
               \begin{pspicture}(0,0)(8,8)
                    \pstGeonode[PosAngle=-135, dotscale=0.6](0,0){B}
                    \pstGeonode[PosAngle=135, dotscale=0.6](0,4){F}
                    \pstGeonode[PosAngle=-45, dotscale=0.6](4,0){C}
                    \pstGeonode[PosAngle=0, dotscale=0.6](4,4){G}
                    \pstGeonode[PosAngle=45, dotscale=0.6](2,2){A}
                    \pstTranslation[PosAngle=-45, dotscale=0.6]{B}{A}{C}[D]
                    \pstTranslation[PosAngle=45, dotscale=0.6]{B}{A}{G}[H]
                    \pstTranslation[PosAngle=45, dotscale=0.6]{B}{A}{F}[E]
                    \pstHomO[HomCoef=0.166666666667,PosAngle=-45, dotscale=0.6]{A}{B}[I]
                    \pstHomO[HomCoef=0.166666666667,PosAngle=-45, dotscale=0.6]{A}{D}[J]
                    \pstHomO[HomCoef=0.166666666667,PosAngle=135, dotscale=0.6]{A}{E}[K]
                    \par
                    \psline(B)(C)(G)(F)(B)
                    \psline[linestyle=dashed](B)(A)(E)
                    \psline[linestyle=dashed](A)(D)
                    \psline(G)(H)(D)(C)
                    \pstHomO[HomCoef=.3333333,PosAngle=45, dotscale=0.6]{E}{H}[M]
                    \pstHomO[HomCoef=.3333333,PosAngle=135, dotscale=0.6]{E}{F}[L]
                    \pstInterLL[PosAngle=90]{B}{L}{M}{D}{S}
                    \psline[linecolor=red](S)(L)(M)(S)
                    \psline(F)(L)
                    \psline(M)(H)
                    \psline[linestyle=dashed](L)(E)(S)
                    \psline[linestyle=dashed](E)(M)
                    \psline[linestyle=dashed,linecolor=red](L)(B)(D)(M)
                    \psline[linecolor=blue]{->}(A)(I)
                    \psline[linecolor=blue]{->}(A)(J)
                    \psline[linecolor=blue]{->}(A)(K)
               \end{pspicture}
          }
     \end{extern}
\end{center}
On munit l'espace du repère orthonormé $\left(A~;\overrightarrow{AI},\overrightarrow{AJ},\overrightarrow{AK}\right)$ tel que~: $I\in[AB]$, $J\in[AD]$, $K\in[AE]$ et
\par
$AI=AJ=AK=1$, l'unité graphique représentant 1 mètre.
\par
Les points $L$, $M$ et $S$ sont définis de la façon suivante~:
\begin{itemize}
     \item $L$ est le point tel que $\overrightarrow{FL}=\frac23\overrightarrow{FE}$~;
     \item $M$ est le point d'intersection du plan $(BDL)$ et de la droite $(EH)$~;
     \item $S$ est le point d'intersection des droites $(BL)$ et $(AK)$.
\end{itemize}
\begin{enumerate}
     \item Démontrer, sans calcul de coordonnées, que les droites $(LM)$ et $(BD)$ sont parallèles.
     \item Démontrer que les coordonnées du point $L$ sont $(2~;~0~;~6)$.
     \item \begin{enumerate}[label=\alph*.]
          \item Donner une représentation paramétrique de la droite $(BL)$.
          \item Vérifier que les coordonnées du point $S$ sont $(0~;~0~;~9)$.
     \end{enumerate}
     \item Soit $\overrightarrow{n}$ le vecteur de coordonnées $(3~;~3~;~2)$.
     \begin{enumerate}[label=\alph*.]
          \item Vérifier que $\overrightarrow{n}$ est normal au plan $(BDL)$.
          \item Démontrer qu'une équation cartésienne du plan $(BDL)$ est~:
          \[
          3x+3y+2z-18=0.
          \]
          \item On admet que la droite $(EH)$ a pour représentation paramétrique~:
          \begin{center}
               $ \left\{
                    \begin{array}{rcl}
                         x&=&0\\
                         y&=&s\\
                         z&=&6
                    \end{array}
               \right.~~~~s\in\mathbb{R}$
          \end{center}
          Calculer les coordonnées du point $M$.
     \end{enumerate}
     \item Calculer le volume du tétraèdre $SELM$. On rappelle que le volume $V$ d'un tétraèdre est donné par la formule suivante~:
     \[
     V=\frac13\times\text{Aire de la base}\times\text{Hauteur}.
     \]
     \item L'artiste souhaite que la mesure de l'angle $\widehat{SLE}$ soit comprise entre 55° et 60°.\\
     Cette contrainte d'angle est-elle respectée~?
\end{enumerate}

\end{document}
µ
\documentclass[a4paper]{article}

%================================================================================================================================
%
% Packages
%
%================================================================================================================================

\usepackage[T1]{fontenc} 	% pour caractères accentués
\usepackage[utf8]{inputenc}  % encodage utf8
\usepackage[french]{babel}	% langue : français
\usepackage{fourier}			% caractères plus lisibles
\usepackage[dvipsnames]{xcolor} % couleurs
\usepackage{fancyhdr}		% réglage header footer
\usepackage{needspace}		% empêcher sauts de page mal placés
\usepackage{graphicx}		% pour inclure des graphiques
\usepackage{enumitem,cprotect}		% personnalise les listes d'items (nécessaire pour ol, al ...)
\usepackage{hyperref}		% Liens hypertexte
\usepackage{pstricks,pst-all,pst-node,pstricks-add,pst-math,pst-plot,pst-tree,pst-eucl} % pstricks
\usepackage[a4paper,includeheadfoot,top=2cm,left=3cm, bottom=2cm,right=3cm]{geometry} % marges etc.
\usepackage{comment}			% commentaires multilignes
\usepackage{amsmath,environ} % maths (matrices, etc.)
\usepackage{amssymb,makeidx}
\usepackage{bm}				% bold maths
\usepackage{tabularx}		% tableaux
\usepackage{colortbl}		% tableaux en couleur
\usepackage{fontawesome}		% Fontawesome
\usepackage{environ}			% environment with command
\usepackage{fp}				% calculs pour ps-tricks
\usepackage{multido}			% pour ps tricks
\usepackage[np]{numprint}	% formattage nombre
\usepackage{tikz,tkz-tab} 			% package principal TikZ
\usepackage{pgfplots}   % axes
\usepackage{mathrsfs}    % cursives
\usepackage{calc}			% calcul taille boites
\usepackage[scaled=0.875]{helvet} % font sans serif
\usepackage{svg} % svg
\usepackage{scrextend} % local margin
\usepackage{scratch} %scratch
\usepackage{multicol} % colonnes
%\usepackage{infix-RPN,pst-func} % formule en notation polanaise inversée
\usepackage{listings}

%================================================================================================================================
%
% Réglages de base
%
%================================================================================================================================

\lstset{
language=Python,   % R code
literate=
{á}{{\'a}}1
{à}{{\`a}}1
{ã}{{\~a}}1
{é}{{\'e}}1
{è}{{\`e}}1
{ê}{{\^e}}1
{í}{{\'i}}1
{ó}{{\'o}}1
{õ}{{\~o}}1
{ú}{{\'u}}1
{ü}{{\"u}}1
{ç}{{\c{c}}}1
{~}{{ }}1
}


\definecolor{codegreen}{rgb}{0,0.6,0}
\definecolor{codegray}{rgb}{0.5,0.5,0.5}
\definecolor{codepurple}{rgb}{0.58,0,0.82}
\definecolor{backcolour}{rgb}{0.95,0.95,0.92}

\lstdefinestyle{mystyle}{
    backgroundcolor=\color{backcolour},   
    commentstyle=\color{codegreen},
    keywordstyle=\color{magenta},
    numberstyle=\tiny\color{codegray},
    stringstyle=\color{codepurple},
    basicstyle=\ttfamily\footnotesize,
    breakatwhitespace=false,         
    breaklines=true,                 
    captionpos=b,                    
    keepspaces=true,                 
    numbers=left,                    
xleftmargin=2em,
framexleftmargin=2em,            
    showspaces=false,                
    showstringspaces=false,
    showtabs=false,                  
    tabsize=2,
    upquote=true
}

\lstset{style=mystyle}


\lstset{style=mystyle}
\newcommand{\imgdir}{C:/laragon/www/newmc/assets/imgsvg/}
\newcommand{\imgsvgdir}{C:/laragon/www/newmc/assets/imgsvg/}

\definecolor{mcgris}{RGB}{220, 220, 220}% ancien~; pour compatibilité
\definecolor{mcbleu}{RGB}{52, 152, 219}
\definecolor{mcvert}{RGB}{125, 194, 70}
\definecolor{mcmauve}{RGB}{154, 0, 215}
\definecolor{mcorange}{RGB}{255, 96, 0}
\definecolor{mcturquoise}{RGB}{0, 153, 153}
\definecolor{mcrouge}{RGB}{255, 0, 0}
\definecolor{mclightvert}{RGB}{205, 234, 190}

\definecolor{gris}{RGB}{220, 220, 220}
\definecolor{bleu}{RGB}{52, 152, 219}
\definecolor{vert}{RGB}{125, 194, 70}
\definecolor{mauve}{RGB}{154, 0, 215}
\definecolor{orange}{RGB}{255, 96, 0}
\definecolor{turquoise}{RGB}{0, 153, 153}
\definecolor{rouge}{RGB}{255, 0, 0}
\definecolor{lightvert}{RGB}{205, 234, 190}
\setitemize[0]{label=\color{lightvert}  $\bullet$}

\pagestyle{fancy}
\renewcommand{\headrulewidth}{0.2pt}
\fancyhead[L]{maths-cours.fr}
\fancyhead[R]{\thepage}
\renewcommand{\footrulewidth}{0.2pt}
\fancyfoot[C]{}

\newcolumntype{C}{>{\centering\arraybackslash}X}
\newcolumntype{s}{>{\hsize=.35\hsize\arraybackslash}X}

\setlength{\parindent}{0pt}		 
\setlength{\parskip}{3mm}
\setlength{\headheight}{1cm}

\def\ebook{ebook}
\def\book{book}
\def\web{web}
\def\type{web}

\newcommand{\vect}[1]{\overrightarrow{\,\mathstrut#1\,}}

\def\Oij{$\left(\text{O}~;~\vect{\imath},~\vect{\jmath}\right)$}
\def\Oijk{$\left(\text{O}~;~\vect{\imath},~\vect{\jmath},~\vect{k}\right)$}
\def\Ouv{$\left(\text{O}~;~\vect{u},~\vect{v}\right)$}

\hypersetup{breaklinks=true, colorlinks = true, linkcolor = OliveGreen, urlcolor = OliveGreen, citecolor = OliveGreen, pdfauthor={Didier BONNEL - https://www.maths-cours.fr} } % supprime les bordures autour des liens

\renewcommand{\arg}[0]{\text{arg}}

\everymath{\displaystyle}

%================================================================================================================================
%
% Macros - Commandes
%
%================================================================================================================================

\newcommand\meta[2]{    			% Utilisé pour créer le post HTML.
	\def\titre{titre}
	\def\url{url}
	\def\arg{#1}
	\ifx\titre\arg
		\newcommand\maintitle{#2}
		\fancyhead[L]{#2}
		{\Large\sffamily \MakeUppercase{#2}}
		\vspace{1mm}\textcolor{mcvert}{\hrule}
	\fi 
	\ifx\url\arg
		\fancyfoot[L]{\href{https://www.maths-cours.fr#2}{\black \footnotesize{https://www.maths-cours.fr#2}}}
	\fi 
}


\newcommand\TitreC[1]{    		% Titre centré
     \needspace{3\baselineskip}
     \begin{center}\textbf{#1}\end{center}
}

\newcommand\newpar{    		% paragraphe
     \par
}

\newcommand\nosp {    		% commande vide (pas d'espace)
}
\newcommand{\id}[1]{} %ignore

\newcommand\boite[2]{				% Boite simple sans titre
	\vspace{5mm}
	\setlength{\fboxrule}{0.2mm}
	\setlength{\fboxsep}{5mm}	
	\fcolorbox{#1}{#1!3}{\makebox[\linewidth-2\fboxrule-2\fboxsep]{
  		\begin{minipage}[t]{\linewidth-2\fboxrule-4\fboxsep}\setlength{\parskip}{3mm}
  			 #2
  		\end{minipage}
	}}
	\vspace{5mm}
}

\newcommand\CBox[4]{				% Boites
	\vspace{5mm}
	\setlength{\fboxrule}{0.2mm}
	\setlength{\fboxsep}{5mm}
	
	\fcolorbox{#1}{#1!3}{\makebox[\linewidth-2\fboxrule-2\fboxsep]{
		\begin{minipage}[t]{1cm}\setlength{\parskip}{3mm}
	  		\textcolor{#1}{\LARGE{#2}}    
 	 	\end{minipage}  
  		\begin{minipage}[t]{\linewidth-2\fboxrule-4\fboxsep}\setlength{\parskip}{3mm}
			\raisebox{1.2mm}{\normalsize\sffamily{\textcolor{#1}{#3}}}						
  			 #4
  		\end{minipage}
	}}
	\vspace{5mm}
}

\newcommand\cadre[3]{				% Boites convertible html
	\par
	\vspace{2mm}
	\setlength{\fboxrule}{0.1mm}
	\setlength{\fboxsep}{5mm}
	\fcolorbox{#1}{white}{\makebox[\linewidth-2\fboxrule-2\fboxsep]{
  		\begin{minipage}[t]{\linewidth-2\fboxrule-4\fboxsep}\setlength{\parskip}{3mm}
			\raisebox{-2.5mm}{\sffamily \small{\textcolor{#1}{\MakeUppercase{#2}}}}		
			\par		
  			 #3
 	 		\end{minipage}
	}}
		\vspace{2mm}
	\par
}

\newcommand\bloc[3]{				% Boites convertible html sans bordure
     \needspace{2\baselineskip}
     {\sffamily \small{\textcolor{#1}{\MakeUppercase{#2}}}}    
		\par		
  			 #3
		\par
}

\newcommand\CHelp[1]{
     \CBox{Plum}{\faInfoCircle}{À RETENIR}{#1}
}

\newcommand\CUp[1]{
     \CBox{NavyBlue}{\faThumbsOUp}{EN PRATIQUE}{#1}
}

\newcommand\CInfo[1]{
     \CBox{Sepia}{\faArrowCircleRight}{REMARQUE}{#1}
}

\newcommand\CRedac[1]{
     \CBox{PineGreen}{\faEdit}{BIEN R\'EDIGER}{#1}
}

\newcommand\CError[1]{
     \CBox{Red}{\faExclamationTriangle}{ATTENTION}{#1}
}

\newcommand\TitreExo[2]{
\needspace{4\baselineskip}
 {\sffamily\large EXERCICE #1\ (\emph{#2 points})}
\vspace{5mm}
}

\newcommand\img[2]{
          \includegraphics[width=#2\paperwidth]{\imgdir#1}
}

\newcommand\imgsvg[2]{
       \begin{center}   \includegraphics[width=#2\paperwidth]{\imgsvgdir#1} \end{center}
}


\newcommand\Lien[2]{
     \href{#1}{#2 \tiny \faExternalLink}
}
\newcommand\mcLien[2]{
     \href{https~://www.maths-cours.fr/#1}{#2 \tiny \faExternalLink}
}

\newcommand{\euro}{\eurologo{}}

%================================================================================================================================
%
% Macros - Environement
%
%================================================================================================================================

\newenvironment{tex}{ %
}
{%
}

\newenvironment{indente}{ %
	\setlength\parindent{10mm}
}

{
	\setlength\parindent{0mm}
}

\newenvironment{corrige}{%
     \needspace{3\baselineskip}
     \medskip
     \textbf{\textsc{Corrigé}}
     \medskip
}
{
}

\newenvironment{extern}{%
     \begin{center}
     }
     {
     \end{center}
}

\NewEnviron{code}{%
	\par
     \boite{gray}{\texttt{%
     \BODY
     }}
     \par
}

\newenvironment{vbloc}{% boite sans cadre empeche saut de page
     \begin{minipage}[t]{\linewidth}
     }
     {
     \end{minipage}
}
\NewEnviron{h2}{%
    \needspace{3\baselineskip}
    \vspace{0.6cm}
	\noindent \MakeUppercase{\sffamily \large \BODY}
	\vspace{1mm}\textcolor{mcgris}{\hrule}\vspace{0.4cm}
	\par
}{}

\NewEnviron{h3}{%
    \needspace{3\baselineskip}
	\vspace{5mm}
	\textsc{\BODY}
	\par
}

\NewEnviron{margeneg}{ %
\begin{addmargin}[-1cm]{0cm}
\BODY
\end{addmargin}
}

\NewEnviron{html}{%
}

\begin{document}
\meta{url}{/exercices/fonctions-bac-s-antilles-guyane-2018/}
\meta{pid}{9074}
\meta{titre}{Fonctions - Bac S Antilles-Guyane 2018}
\meta{type}{exercices}
%
\begin{h2}Exercice 3 (5 points)\end{h2}
\smallbreak
\textit{Commun à tous les candidats}
\bigbreak
Un publicitaire souhaite imprimer le logo ci-dessous sur un T-shirt~:
\begin{center}
     \begin{extern}%width="400" alt="Logo  Bac S Antilles-Guyane 2018"
          \resizebox{8cm}{!}{
               \psset{xunit=2.0cm,yunit=2.0cm,algebraic=true,dimen=middle,dotstyle=o,dotsize=5pt 0,linewidth=1pt,arrowsize=3pt 2,arrowinset=0.25}
               \begin{pspicture*}(-2.,-2.)(5.,1.)
                    \pscustom[linewidth=0.8pt,fillcolor=lightgray!30,fillstyle=solid]{\psplot{-1.57}{4.6}{EXP(-x)*(-COS(x)+SIN(x)+1.0)}\lineto(4.6,0)\psplot{4.6}{-1.57}{-EXP(-x)*COS(x)}\lineto(-1.57,0)\closepath}
                    \psplot[linewidth=1.2pt,plotpoints=200]{-1.57}{4.6}{EXP(-x)*(-COS(x)+SIN(x)+1.0)}
                    \psplot[linewidth=1.2pt,plotpoints=200]{-1.57}{4.6}{-EXP(-x)*COS(x)}
               \end{pspicture*}
          }
     \end{extern}
\end{center}
Il dessine ce logo à l'aide des courbes de deux fonctions $f$ et $g$ définies sur $\mathbb{R}$ par~:
\[
f(x)=\text{e}^{-x}(-\cos x+\sin x+1)\text{~~et~~}
g(x)=-\text{e}^{x}\cos x.
\]
On admet que les fonctions $f$ et $g$ sont dérivables sur $\mathbb{R}$.
\par
\medbreak
\TitreC{Partie A — Étude de la fonction $f$}
\smallbreak
\begin{enumerate}
     \item Justifier que, pour tout $x\in\mathbb{R}$~:\[
     -\text{e}^{-x}\leqslant f(x)\leqslant 3\text{e}^{-x}.\]
     \item En déduire la limite de $f$ en $+\infty$.
     \item Démontrer que, pour tout $x\in\mathbb{R}$, $f'(x)=\text{e}^{-x}(2\cos x-1)$ où $f'$ est la fonction dérivée de $f$.
     \item Dans cette question, on étudie la fonction $f$ sur l'intervalle $[-\pi~;\pi]$.
     \begin{enumerate}[label=\alph*.]
          \item Déterminer le signe de $f'(x)$ pour $x$ appartenant à l'intervalle $[-\pi~;\pi]$.
          \item En déduire les variations de $f$ sur $[-\pi~;\pi]$.
     \end{enumerate}
\end{enumerate}
\medbreak
\TitreC{Partie B — Aire du logo}
\smallbreak
\par
\par
On note $\mathscr{C}_f$ et $\mathscr{C}_g$ les représentations graphiques des fonctions $f$ et $g$ dans un repère orthonormé $(O~;~\overrightarrow{u},~\overrightarrow{v})$. L'unité graphique est de 2 centimètres. Ces deux courbes sont tracées en Annexe.
\begin{enumerate}
     \item Étudier la position relative de la courbe $\mathscr{C}_f$ par rapport à la courbe $\mathscr{C}_g$ sur $\mathbb{R}$.
     \item Soit $H$ la fonction définie sur $\mathbb{R}$ par~:
     \[
     H(x)=\left(-\frac{\cos x}{2}-\frac{\sin x}{2}-1\right)\text{e}^{-x}.
     \]
     On admet que $H$ est une primitive de la fonction $x\mapsto (\sin x+1)\text{e}^{-x}$ sur $\mathbb{R}$.
     \par
     On note $\mathscr{D}$ le domaine délimité par la courbe $\mathscr{C}_f$, la courbe $\mathscr{C}_g$ est les droites d'équation $x=-\frac{\pi}{2}$ et $x=\frac{3\pi}{2}$.
     \begin{enumerate}[label=\alph*.]
          \item Hachurer le domaine $\mathscr{D}$ sur le graphique en annexe à rendre avec la copie.
          \item Calculer, une unité d'aire, l'aire du domaine $\mathscr{D}$, puis en donner une valeur approchée à $10^{-2}$ près en cm$^2$.
     \end{enumerate}
\end{enumerate}
\newpage
\begin{center}
     \bigbreak
     \textbf{ANNEXE}
     \par
     \textit{À compléter et à remettre avec la copie}
     \bigbreak
     %    \vspace{5cm}
     \begin{extern} %width="500" alt="Annexe Bac S Antilles-Guyane 2018"
          \resizebox{10cm}{!}{%
               % -+-+-+ variables modifiables
               \def\fonction{2.71828^(-x)*(-cos( x)+sin( x)+1) }
               \def\g{-2.71828^(-x)*cos( x) }
               \def\xmin{-2.5}
               \def\xmax{5.5}
               \def\ymin{-2.2}
               \def\ymax{2.8}
               \def\xunit{2}  % unités en cm
               \def\yunit{2}
               \psset{xunit=\xunit,yunit=\yunit,algebraic=true}
               \fontsize{18pt}{18pt}\selectfont
               \begin{pspicture*}[linewidth=1pt](\xmin,\ymin)(\xmax,\ymax)
                    \psgrid[gridcolor=mcgris, subgriddiv=1, gridlabels=0pt](-3,-3)(\xmax,\ymax)
                    \psaxes[linewidth=0.75pt]{->}(0,0)(\xmin,\ymin)(\xmax,\ymax)
                    \psplot[plotpoints=2000,linecolor=blue]{\xmin}{\xmax}{\fonction}
                    \psplot[plotpoints=2000,linecolor=red]{\xmin}{\xmax}{\g}
                    \rput[tr](-0.1,-0.1){$O$}
                    \rput[bl](1.5,0.5){$\color{blue} \mathcal{C}_f$}
                    \rput[bl](1.5,-0.5){$\color{red} \mathcal{C}_g$}
               \end{pspicture*}
          }
     \end{extern}
\end{center}

\end{document}
µ
\documentclass[a4paper]{article}

%================================================================================================================================
%
% Packages
%
%================================================================================================================================

\usepackage[T1]{fontenc} 	% pour caractères accentués
\usepackage[utf8]{inputenc}  % encodage utf8
\usepackage[french]{babel}	% langue : français
\usepackage{fourier}			% caractères plus lisibles
\usepackage[dvipsnames]{xcolor} % couleurs
\usepackage{fancyhdr}		% réglage header footer
\usepackage{needspace}		% empêcher sauts de page mal placés
\usepackage{graphicx}		% pour inclure des graphiques
\usepackage{enumitem,cprotect}		% personnalise les listes d'items (nécessaire pour ol, al ...)
\usepackage{hyperref}		% Liens hypertexte
\usepackage{pstricks,pst-all,pst-node,pstricks-add,pst-math,pst-plot,pst-tree,pst-eucl} % pstricks
\usepackage[a4paper,includeheadfoot,top=2cm,left=3cm, bottom=2cm,right=3cm]{geometry} % marges etc.
\usepackage{comment}			% commentaires multilignes
\usepackage{amsmath,environ} % maths (matrices, etc.)
\usepackage{amssymb,makeidx}
\usepackage{bm}				% bold maths
\usepackage{tabularx}		% tableaux
\usepackage{colortbl}		% tableaux en couleur
\usepackage{fontawesome}		% Fontawesome
\usepackage{environ}			% environment with command
\usepackage{fp}				% calculs pour ps-tricks
\usepackage{multido}			% pour ps tricks
\usepackage[np]{numprint}	% formattage nombre
\usepackage{tikz,tkz-tab} 			% package principal TikZ
\usepackage{pgfplots}   % axes
\usepackage{mathrsfs}    % cursives
\usepackage{calc}			% calcul taille boites
\usepackage[scaled=0.875]{helvet} % font sans serif
\usepackage{svg} % svg
\usepackage{scrextend} % local margin
\usepackage{scratch} %scratch
\usepackage{multicol} % colonnes
%\usepackage{infix-RPN,pst-func} % formule en notation polanaise inversée
\usepackage{listings}

%================================================================================================================================
%
% Réglages de base
%
%================================================================================================================================

\lstset{
language=Python,   % R code
literate=
{á}{{\'a}}1
{à}{{\`a}}1
{ã}{{\~a}}1
{é}{{\'e}}1
{è}{{\`e}}1
{ê}{{\^e}}1
{í}{{\'i}}1
{ó}{{\'o}}1
{õ}{{\~o}}1
{ú}{{\'u}}1
{ü}{{\"u}}1
{ç}{{\c{c}}}1
{~}{{ }}1
}


\definecolor{codegreen}{rgb}{0,0.6,0}
\definecolor{codegray}{rgb}{0.5,0.5,0.5}
\definecolor{codepurple}{rgb}{0.58,0,0.82}
\definecolor{backcolour}{rgb}{0.95,0.95,0.92}

\lstdefinestyle{mystyle}{
    backgroundcolor=\color{backcolour},   
    commentstyle=\color{codegreen},
    keywordstyle=\color{magenta},
    numberstyle=\tiny\color{codegray},
    stringstyle=\color{codepurple},
    basicstyle=\ttfamily\footnotesize,
    breakatwhitespace=false,         
    breaklines=true,                 
    captionpos=b,                    
    keepspaces=true,                 
    numbers=left,                    
xleftmargin=2em,
framexleftmargin=2em,            
    showspaces=false,                
    showstringspaces=false,
    showtabs=false,                  
    tabsize=2,
    upquote=true
}

\lstset{style=mystyle}


\lstset{style=mystyle}
\newcommand{\imgdir}{C:/laragon/www/newmc/assets/imgsvg/}
\newcommand{\imgsvgdir}{C:/laragon/www/newmc/assets/imgsvg/}

\definecolor{mcgris}{RGB}{220, 220, 220}% ancien~; pour compatibilité
\definecolor{mcbleu}{RGB}{52, 152, 219}
\definecolor{mcvert}{RGB}{125, 194, 70}
\definecolor{mcmauve}{RGB}{154, 0, 215}
\definecolor{mcorange}{RGB}{255, 96, 0}
\definecolor{mcturquoise}{RGB}{0, 153, 153}
\definecolor{mcrouge}{RGB}{255, 0, 0}
\definecolor{mclightvert}{RGB}{205, 234, 190}

\definecolor{gris}{RGB}{220, 220, 220}
\definecolor{bleu}{RGB}{52, 152, 219}
\definecolor{vert}{RGB}{125, 194, 70}
\definecolor{mauve}{RGB}{154, 0, 215}
\definecolor{orange}{RGB}{255, 96, 0}
\definecolor{turquoise}{RGB}{0, 153, 153}
\definecolor{rouge}{RGB}{255, 0, 0}
\definecolor{lightvert}{RGB}{205, 234, 190}
\setitemize[0]{label=\color{lightvert}  $\bullet$}

\pagestyle{fancy}
\renewcommand{\headrulewidth}{0.2pt}
\fancyhead[L]{maths-cours.fr}
\fancyhead[R]{\thepage}
\renewcommand{\footrulewidth}{0.2pt}
\fancyfoot[C]{}

\newcolumntype{C}{>{\centering\arraybackslash}X}
\newcolumntype{s}{>{\hsize=.35\hsize\arraybackslash}X}

\setlength{\parindent}{0pt}		 
\setlength{\parskip}{3mm}
\setlength{\headheight}{1cm}

\def\ebook{ebook}
\def\book{book}
\def\web{web}
\def\type{web}

\newcommand{\vect}[1]{\overrightarrow{\,\mathstrut#1\,}}

\def\Oij{$\left(\text{O}~;~\vect{\imath},~\vect{\jmath}\right)$}
\def\Oijk{$\left(\text{O}~;~\vect{\imath},~\vect{\jmath},~\vect{k}\right)$}
\def\Ouv{$\left(\text{O}~;~\vect{u},~\vect{v}\right)$}

\hypersetup{breaklinks=true, colorlinks = true, linkcolor = OliveGreen, urlcolor = OliveGreen, citecolor = OliveGreen, pdfauthor={Didier BONNEL - https://www.maths-cours.fr} } % supprime les bordures autour des liens

\renewcommand{\arg}[0]{\text{arg}}

\everymath{\displaystyle}

%================================================================================================================================
%
% Macros - Commandes
%
%================================================================================================================================

\newcommand\meta[2]{    			% Utilisé pour créer le post HTML.
	\def\titre{titre}
	\def\url{url}
	\def\arg{#1}
	\ifx\titre\arg
		\newcommand\maintitle{#2}
		\fancyhead[L]{#2}
		{\Large\sffamily \MakeUppercase{#2}}
		\vspace{1mm}\textcolor{mcvert}{\hrule}
	\fi 
	\ifx\url\arg
		\fancyfoot[L]{\href{https://www.maths-cours.fr#2}{\black \footnotesize{https://www.maths-cours.fr#2}}}
	\fi 
}


\newcommand\TitreC[1]{    		% Titre centré
     \needspace{3\baselineskip}
     \begin{center}\textbf{#1}\end{center}
}

\newcommand\newpar{    		% paragraphe
     \par
}

\newcommand\nosp {    		% commande vide (pas d'espace)
}
\newcommand{\id}[1]{} %ignore

\newcommand\boite[2]{				% Boite simple sans titre
	\vspace{5mm}
	\setlength{\fboxrule}{0.2mm}
	\setlength{\fboxsep}{5mm}	
	\fcolorbox{#1}{#1!3}{\makebox[\linewidth-2\fboxrule-2\fboxsep]{
  		\begin{minipage}[t]{\linewidth-2\fboxrule-4\fboxsep}\setlength{\parskip}{3mm}
  			 #2
  		\end{minipage}
	}}
	\vspace{5mm}
}

\newcommand\CBox[4]{				% Boites
	\vspace{5mm}
	\setlength{\fboxrule}{0.2mm}
	\setlength{\fboxsep}{5mm}
	
	\fcolorbox{#1}{#1!3}{\makebox[\linewidth-2\fboxrule-2\fboxsep]{
		\begin{minipage}[t]{1cm}\setlength{\parskip}{3mm}
	  		\textcolor{#1}{\LARGE{#2}}    
 	 	\end{minipage}  
  		\begin{minipage}[t]{\linewidth-2\fboxrule-4\fboxsep}\setlength{\parskip}{3mm}
			\raisebox{1.2mm}{\normalsize\sffamily{\textcolor{#1}{#3}}}						
  			 #4
  		\end{minipage}
	}}
	\vspace{5mm}
}

\newcommand\cadre[3]{				% Boites convertible html
	\par
	\vspace{2mm}
	\setlength{\fboxrule}{0.1mm}
	\setlength{\fboxsep}{5mm}
	\fcolorbox{#1}{white}{\makebox[\linewidth-2\fboxrule-2\fboxsep]{
  		\begin{minipage}[t]{\linewidth-2\fboxrule-4\fboxsep}\setlength{\parskip}{3mm}
			\raisebox{-2.5mm}{\sffamily \small{\textcolor{#1}{\MakeUppercase{#2}}}}		
			\par		
  			 #3
 	 		\end{minipage}
	}}
		\vspace{2mm}
	\par
}

\newcommand\bloc[3]{				% Boites convertible html sans bordure
     \needspace{2\baselineskip}
     {\sffamily \small{\textcolor{#1}{\MakeUppercase{#2}}}}    
		\par		
  			 #3
		\par
}

\newcommand\CHelp[1]{
     \CBox{Plum}{\faInfoCircle}{À RETENIR}{#1}
}

\newcommand\CUp[1]{
     \CBox{NavyBlue}{\faThumbsOUp}{EN PRATIQUE}{#1}
}

\newcommand\CInfo[1]{
     \CBox{Sepia}{\faArrowCircleRight}{REMARQUE}{#1}
}

\newcommand\CRedac[1]{
     \CBox{PineGreen}{\faEdit}{BIEN R\'EDIGER}{#1}
}

\newcommand\CError[1]{
     \CBox{Red}{\faExclamationTriangle}{ATTENTION}{#1}
}

\newcommand\TitreExo[2]{
\needspace{4\baselineskip}
 {\sffamily\large EXERCICE #1\ (\emph{#2 points})}
\vspace{5mm}
}

\newcommand\img[2]{
          \includegraphics[width=#2\paperwidth]{\imgdir#1}
}

\newcommand\imgsvg[2]{
       \begin{center}   \includegraphics[width=#2\paperwidth]{\imgsvgdir#1} \end{center}
}


\newcommand\Lien[2]{
     \href{#1}{#2 \tiny \faExternalLink}
}
\newcommand\mcLien[2]{
     \href{https~://www.maths-cours.fr/#1}{#2 \tiny \faExternalLink}
}

\newcommand{\euro}{\eurologo{}}

%================================================================================================================================
%
% Macros - Environement
%
%================================================================================================================================

\newenvironment{tex}{ %
}
{%
}

\newenvironment{indente}{ %
	\setlength\parindent{10mm}
}

{
	\setlength\parindent{0mm}
}

\newenvironment{corrige}{%
     \needspace{3\baselineskip}
     \medskip
     \textbf{\textsc{Corrigé}}
     \medskip
}
{
}

\newenvironment{extern}{%
     \begin{center}
     }
     {
     \end{center}
}

\NewEnviron{code}{%
	\par
     \boite{gray}{\texttt{%
     \BODY
     }}
     \par
}

\newenvironment{vbloc}{% boite sans cadre empeche saut de page
     \begin{minipage}[t]{\linewidth}
     }
     {
     \end{minipage}
}
\NewEnviron{h2}{%
    \needspace{3\baselineskip}
    \vspace{0.6cm}
	\noindent \MakeUppercase{\sffamily \large \BODY}
	\vspace{1mm}\textcolor{mcgris}{\hrule}\vspace{0.4cm}
	\par
}{}

\NewEnviron{h3}{%
    \needspace{3\baselineskip}
	\vspace{5mm}
	\textsc{\BODY}
	\par
}

\NewEnviron{margeneg}{ %
\begin{addmargin}[-1cm]{0cm}
\BODY
\end{addmargin}
}

\NewEnviron{html}{%
}

\begin{document}
\meta{url}{/exercices/suites-bac-s-antilles-guyane-2018/}
\meta{pid}{9080}
\meta{titre}{Suites - Bac S Antilles-Guyane 2018}
\meta{type}{exercices}
%
\begin{h2}Exercice 4 (5 points)\end{h2}
\smallbreak
\textit{Candidats n'ayant pas choisi l'enseignement de spécialité \og Mathématiques \fg{} }
\bigbreak
Le directeur d'une réserve marine a recensé 3~000 cétacés dans cette réserve au 1$^{\text{er}}$ juin 2017. Il est inquiet car il sait que le classement de la zone en \og réserve marine\fg{} ne sera pas reconduit si le nombre de cétacés de cette réserve devient inférieur à 2~000.
\smallbreak
Une étude lui permet d'élaborer un modèle selon lequel, chaque année~:
\begin{itemize}
     \item entre le 1$^{\text{er}}$ juin et le 31 octobre, 80 cétacés arrivent dans la réserve marine~;
     \item entre le 1$^{\text{er}}$ novembre et le 31 mai, la réserve subit une baisse de 5~\% de son effectif par rapport à celui du 31 octobre qui précède.
\end{itemize}
On modélise l'évolution du nombre de cétacés par une suite $(u_n)$. Selon ce modèle, pour tout entier naturel $n$, $u_n$ désigne le nombre de cétacés au 1$^{\text{er}}$ juin de l'année $2017+n$. On a donc $u_0=3~000$.
\begin{enumerate}
     \item
     Justifier que $u_1=2~926$.
     \item Justifier que, pour tout entier naturel $n$, $u_{n+1}=0,95u_n+76$.
     \item À l'aide d'un tableur, on a calculé les 8 premiers termes de la suite $(u_n)$. Le directeur a configuré le format des cellules pour que ne soient affichés que des nombres arrondis à l'unité.
     \begin{center}
          \begin{extern}%width="520" alt="Tableur Bac S Antilles-Guyane 2018"
               \begin{tabular}{|c|c|c|c|c|c|c|c|c|c|}
                    \hline
                    \rowcolor{lightgray!20}&A&B&C&D&E&F&G&H&I\\
                    \hline
                    \cellcolor{lightgray!20}1&$n$&0&1&2&3&4&5&6&7\\
                    \hline
                    \cellcolor{lightgray!20}2&$u_n$&3~000&2~926&2~856&2~789&2~725&2~665&2~608&2~553\\
                    \hline
               \end{tabular}
          \end{extern}
     \end{center}
     Quelle formule peut-on entrer dans la cellule C2 afin d'obtenir, par recopie vers la droite, les termes de la suite $(u_n)$~?
     \item \begin{enumerate}[label=\alph*.]
          \item Démontrer que, pour tout entier naturel $n$, $u_n\geqslant1~520$.
          \item Démontrer que la suite $(u_n)$ est décroissante.
          \item Justifier que la suite $(u_n)$ est convergente. On ne cherchera pas ici la valeur de la limite.
     \end{enumerate}
     \item On désigne par $(v_n)$ la suite définie par, pour tout entier naturel $n$, $v_n=u_n-1~520$.
     \begin{enumerate}[label=\alph*.]
          \item Démontrer que la suite $(v_n)$ est une suite géométrique de raison 0,95 dont on précisera le premier terme.
          \item En déduire que, pour tout entier naturel $n$, $u_n=1~480\times0,95^n+1~520$.
          \item Déterminer la limite de la suite $(u_n)$.
     \end{enumerate}
     \item Recopier et compléter l'algorithme suivant pour déterminer l'année à partir de laquelle le nombre de cétacés présents dans la réserve marine sera inférieur à 2~000.
     \begin{center}
          \begin{extern}%width="200" alt="Algorithme Bac S Antilles-Guyane 2018"
               \fbox{
                    \begin{minipage}{3.8cm}
                         $n\leftarrow 0$\\
                         $u\leftarrow3~000$\\
                         Tant que $\ldots$\\
                         \phantom{xxxx}$n\leftarrow \ldots$\\
                         \phantom{xxxx}$u\leftarrow \ldots$\\
                         Fin de Tant que
                    \end{minipage}
               }
          \end{extern}
     \end{center}
     La notation \og$\leftarrow$\fg{} correspond à une affectation de valeur, ainsi \og $n\leftarrow 0$\fg{} signifie \og~Affecter à $n$ la valeur $0$~\fg{}.
     \item La réserve marine fermera-t-elle un jour~? Si oui, déterminer l'année de la fermeture.
\end{enumerate}

\end{document}
µ
\documentclass[a4paper]{article}

%================================================================================================================================
%
% Packages
%
%================================================================================================================================

\usepackage[T1]{fontenc} 	% pour caractères accentués
\usepackage[utf8]{inputenc}  % encodage utf8
\usepackage[french]{babel}	% langue : français
\usepackage{fourier}			% caractères plus lisibles
\usepackage[dvipsnames]{xcolor} % couleurs
\usepackage{fancyhdr}		% réglage header footer
\usepackage{needspace}		% empêcher sauts de page mal placés
\usepackage{graphicx}		% pour inclure des graphiques
\usepackage{enumitem,cprotect}		% personnalise les listes d'items (nécessaire pour ol, al ...)
\usepackage{hyperref}		% Liens hypertexte
\usepackage{pstricks,pst-all,pst-node,pstricks-add,pst-math,pst-plot,pst-tree,pst-eucl} % pstricks
\usepackage[a4paper,includeheadfoot,top=2cm,left=3cm, bottom=2cm,right=3cm]{geometry} % marges etc.
\usepackage{comment}			% commentaires multilignes
\usepackage{amsmath,environ} % maths (matrices, etc.)
\usepackage{amssymb,makeidx}
\usepackage{bm}				% bold maths
\usepackage{tabularx}		% tableaux
\usepackage{colortbl}		% tableaux en couleur
\usepackage{fontawesome}		% Fontawesome
\usepackage{environ}			% environment with command
\usepackage{fp}				% calculs pour ps-tricks
\usepackage{multido}			% pour ps tricks
\usepackage[np]{numprint}	% formattage nombre
\usepackage{tikz,tkz-tab} 			% package principal TikZ
\usepackage{pgfplots}   % axes
\usepackage{mathrsfs}    % cursives
\usepackage{calc}			% calcul taille boites
\usepackage[scaled=0.875]{helvet} % font sans serif
\usepackage{svg} % svg
\usepackage{scrextend} % local margin
\usepackage{scratch} %scratch
\usepackage{multicol} % colonnes
%\usepackage{infix-RPN,pst-func} % formule en notation polanaise inversée
\usepackage{listings}

%================================================================================================================================
%
% Réglages de base
%
%================================================================================================================================

\lstset{
language=Python,   % R code
literate=
{á}{{\'a}}1
{à}{{\`a}}1
{ã}{{\~a}}1
{é}{{\'e}}1
{è}{{\`e}}1
{ê}{{\^e}}1
{í}{{\'i}}1
{ó}{{\'o}}1
{õ}{{\~o}}1
{ú}{{\'u}}1
{ü}{{\"u}}1
{ç}{{\c{c}}}1
{~}{{ }}1
}


\definecolor{codegreen}{rgb}{0,0.6,0}
\definecolor{codegray}{rgb}{0.5,0.5,0.5}
\definecolor{codepurple}{rgb}{0.58,0,0.82}
\definecolor{backcolour}{rgb}{0.95,0.95,0.92}

\lstdefinestyle{mystyle}{
    backgroundcolor=\color{backcolour},   
    commentstyle=\color{codegreen},
    keywordstyle=\color{magenta},
    numberstyle=\tiny\color{codegray},
    stringstyle=\color{codepurple},
    basicstyle=\ttfamily\footnotesize,
    breakatwhitespace=false,         
    breaklines=true,                 
    captionpos=b,                    
    keepspaces=true,                 
    numbers=left,                    
xleftmargin=2em,
framexleftmargin=2em,            
    showspaces=false,                
    showstringspaces=false,
    showtabs=false,                  
    tabsize=2,
    upquote=true
}

\lstset{style=mystyle}


\lstset{style=mystyle}
\newcommand{\imgdir}{C:/laragon/www/newmc/assets/imgsvg/}
\newcommand{\imgsvgdir}{C:/laragon/www/newmc/assets/imgsvg/}

\definecolor{mcgris}{RGB}{220, 220, 220}% ancien~; pour compatibilité
\definecolor{mcbleu}{RGB}{52, 152, 219}
\definecolor{mcvert}{RGB}{125, 194, 70}
\definecolor{mcmauve}{RGB}{154, 0, 215}
\definecolor{mcorange}{RGB}{255, 96, 0}
\definecolor{mcturquoise}{RGB}{0, 153, 153}
\definecolor{mcrouge}{RGB}{255, 0, 0}
\definecolor{mclightvert}{RGB}{205, 234, 190}

\definecolor{gris}{RGB}{220, 220, 220}
\definecolor{bleu}{RGB}{52, 152, 219}
\definecolor{vert}{RGB}{125, 194, 70}
\definecolor{mauve}{RGB}{154, 0, 215}
\definecolor{orange}{RGB}{255, 96, 0}
\definecolor{turquoise}{RGB}{0, 153, 153}
\definecolor{rouge}{RGB}{255, 0, 0}
\definecolor{lightvert}{RGB}{205, 234, 190}
\setitemize[0]{label=\color{lightvert}  $\bullet$}

\pagestyle{fancy}
\renewcommand{\headrulewidth}{0.2pt}
\fancyhead[L]{maths-cours.fr}
\fancyhead[R]{\thepage}
\renewcommand{\footrulewidth}{0.2pt}
\fancyfoot[C]{}

\newcolumntype{C}{>{\centering\arraybackslash}X}
\newcolumntype{s}{>{\hsize=.35\hsize\arraybackslash}X}

\setlength{\parindent}{0pt}		 
\setlength{\parskip}{3mm}
\setlength{\headheight}{1cm}

\def\ebook{ebook}
\def\book{book}
\def\web{web}
\def\type{web}

\newcommand{\vect}[1]{\overrightarrow{\,\mathstrut#1\,}}

\def\Oij{$\left(\text{O}~;~\vect{\imath},~\vect{\jmath}\right)$}
\def\Oijk{$\left(\text{O}~;~\vect{\imath},~\vect{\jmath},~\vect{k}\right)$}
\def\Ouv{$\left(\text{O}~;~\vect{u},~\vect{v}\right)$}

\hypersetup{breaklinks=true, colorlinks = true, linkcolor = OliveGreen, urlcolor = OliveGreen, citecolor = OliveGreen, pdfauthor={Didier BONNEL - https://www.maths-cours.fr} } % supprime les bordures autour des liens

\renewcommand{\arg}[0]{\text{arg}}

\everymath{\displaystyle}

%================================================================================================================================
%
% Macros - Commandes
%
%================================================================================================================================

\newcommand\meta[2]{    			% Utilisé pour créer le post HTML.
	\def\titre{titre}
	\def\url{url}
	\def\arg{#1}
	\ifx\titre\arg
		\newcommand\maintitle{#2}
		\fancyhead[L]{#2}
		{\Large\sffamily \MakeUppercase{#2}}
		\vspace{1mm}\textcolor{mcvert}{\hrule}
	\fi 
	\ifx\url\arg
		\fancyfoot[L]{\href{https://www.maths-cours.fr#2}{\black \footnotesize{https://www.maths-cours.fr#2}}}
	\fi 
}


\newcommand\TitreC[1]{    		% Titre centré
     \needspace{3\baselineskip}
     \begin{center}\textbf{#1}\end{center}
}

\newcommand\newpar{    		% paragraphe
     \par
}

\newcommand\nosp {    		% commande vide (pas d'espace)
}
\newcommand{\id}[1]{} %ignore

\newcommand\boite[2]{				% Boite simple sans titre
	\vspace{5mm}
	\setlength{\fboxrule}{0.2mm}
	\setlength{\fboxsep}{5mm}	
	\fcolorbox{#1}{#1!3}{\makebox[\linewidth-2\fboxrule-2\fboxsep]{
  		\begin{minipage}[t]{\linewidth-2\fboxrule-4\fboxsep}\setlength{\parskip}{3mm}
  			 #2
  		\end{minipage}
	}}
	\vspace{5mm}
}

\newcommand\CBox[4]{				% Boites
	\vspace{5mm}
	\setlength{\fboxrule}{0.2mm}
	\setlength{\fboxsep}{5mm}
	
	\fcolorbox{#1}{#1!3}{\makebox[\linewidth-2\fboxrule-2\fboxsep]{
		\begin{minipage}[t]{1cm}\setlength{\parskip}{3mm}
	  		\textcolor{#1}{\LARGE{#2}}    
 	 	\end{minipage}  
  		\begin{minipage}[t]{\linewidth-2\fboxrule-4\fboxsep}\setlength{\parskip}{3mm}
			\raisebox{1.2mm}{\normalsize\sffamily{\textcolor{#1}{#3}}}						
  			 #4
  		\end{minipage}
	}}
	\vspace{5mm}
}

\newcommand\cadre[3]{				% Boites convertible html
	\par
	\vspace{2mm}
	\setlength{\fboxrule}{0.1mm}
	\setlength{\fboxsep}{5mm}
	\fcolorbox{#1}{white}{\makebox[\linewidth-2\fboxrule-2\fboxsep]{
  		\begin{minipage}[t]{\linewidth-2\fboxrule-4\fboxsep}\setlength{\parskip}{3mm}
			\raisebox{-2.5mm}{\sffamily \small{\textcolor{#1}{\MakeUppercase{#2}}}}		
			\par		
  			 #3
 	 		\end{minipage}
	}}
		\vspace{2mm}
	\par
}

\newcommand\bloc[3]{				% Boites convertible html sans bordure
     \needspace{2\baselineskip}
     {\sffamily \small{\textcolor{#1}{\MakeUppercase{#2}}}}    
		\par		
  			 #3
		\par
}

\newcommand\CHelp[1]{
     \CBox{Plum}{\faInfoCircle}{À RETENIR}{#1}
}

\newcommand\CUp[1]{
     \CBox{NavyBlue}{\faThumbsOUp}{EN PRATIQUE}{#1}
}

\newcommand\CInfo[1]{
     \CBox{Sepia}{\faArrowCircleRight}{REMARQUE}{#1}
}

\newcommand\CRedac[1]{
     \CBox{PineGreen}{\faEdit}{BIEN R\'EDIGER}{#1}
}

\newcommand\CError[1]{
     \CBox{Red}{\faExclamationTriangle}{ATTENTION}{#1}
}

\newcommand\TitreExo[2]{
\needspace{4\baselineskip}
 {\sffamily\large EXERCICE #1\ (\emph{#2 points})}
\vspace{5mm}
}

\newcommand\img[2]{
          \includegraphics[width=#2\paperwidth]{\imgdir#1}
}

\newcommand\imgsvg[2]{
       \begin{center}   \includegraphics[width=#2\paperwidth]{\imgsvgdir#1} \end{center}
}


\newcommand\Lien[2]{
     \href{#1}{#2 \tiny \faExternalLink}
}
\newcommand\mcLien[2]{
     \href{https~://www.maths-cours.fr/#1}{#2 \tiny \faExternalLink}
}

\newcommand{\euro}{\eurologo{}}

%================================================================================================================================
%
% Macros - Environement
%
%================================================================================================================================

\newenvironment{tex}{ %
}
{%
}

\newenvironment{indente}{ %
	\setlength\parindent{10mm}
}

{
	\setlength\parindent{0mm}
}

\newenvironment{corrige}{%
     \needspace{3\baselineskip}
     \medskip
     \textbf{\textsc{Corrigé}}
     \medskip
}
{
}

\newenvironment{extern}{%
     \begin{center}
     }
     {
     \end{center}
}

\NewEnviron{code}{%
	\par
     \boite{gray}{\texttt{%
     \BODY
     }}
     \par
}

\newenvironment{vbloc}{% boite sans cadre empeche saut de page
     \begin{minipage}[t]{\linewidth}
     }
     {
     \end{minipage}
}
\NewEnviron{h2}{%
    \needspace{3\baselineskip}
    \vspace{0.6cm}
	\noindent \MakeUppercase{\sffamily \large \BODY}
	\vspace{1mm}\textcolor{mcgris}{\hrule}\vspace{0.4cm}
	\par
}{}

\NewEnviron{h3}{%
    \needspace{3\baselineskip}
	\vspace{5mm}
	\textsc{\BODY}
	\par
}

\NewEnviron{margeneg}{ %
\begin{addmargin}[-1cm]{0cm}
\BODY
\end{addmargin}
}

\NewEnviron{html}{%
}

\begin{document}
\meta{url}{/exercices/matrices-de-transition-bac-s-antilles-guyane-2018-spe/}
\meta{pid}{9091}
\meta{titre}{Matrices de transition - Bac S Antilles-Guyane 2018 (spé)}
\meta{type}{exercices}
%
\begin{h2}Exercice 4 (5 points)\end{h2}
\smallbreak
\textit{Candidats ayant choisi l'enseignement de spécialité \og Mathématiques \fg{} }
\bigbreak
Le droit de pêche dans une réserve marine est réglementé~: chaque pêcheur doit posséder une carte d'accréditation annuelle. Il existe deux types de cartes~:
\begin{itemize}
     \item une carte de pêche dite « libre » (le pêcheur n'est pas limité en nombre de poissons pêchés)~;
     \item une carte de pêche dite « avec quota » (le pêcheur ne doit pas dépasser une certaine quantité hebdomadaire de poissons).
\end{itemize}
On suppose que le nombre total de pêcheurs reste constant d'année en année.\\
On note, pour l'année $2017+n$~:
\begin{itemize}
     \item $l_n$ la proportion de pêcheurs possédant la carte de pêche libre~;
     \item $q_n$ la proportion de pêcheurs possédant la carte de pêche avec quota.
\end{itemize}
On observe que~:
\begin{itemize}
     \item chaque année, 65~\% des possesseurs de la carte de pêche libres achète de nouveaux une carte de pêche libre l'année suivante~;
     \item Chaque année, 45~\% des possesseurs de la carte de pêche avec quota acheté une carte de pêche libre l'année suivante~;
     \item En 2017, 40~\% des pêcheurs ont acheté une carte de pêche libre. On a donc $l_0=0,4$ et $q_0=0,6$.
\end{itemize}
On note, pour tout entier naturel $n$, $P_n=\begin{pmatrix}
     l_n\\q_n
\end{pmatrix}$.
\begin{enumerate}
     \item Démontrer que, pour tout entier naturel $n$, $P_{n+1}=MP_n$, où $M$ est la matrice carrée $\begin{pmatrix}
          0,65&0,45\\
          0,35&0,55
     \end{pmatrix}$.
     \item Calculer la proportion de pêcheurs achetant une carte de pêche avec quota en 2019.
     \item Un logiciel de calcul formel donne les résultats ci-dessous~:
     \begin{center}
          \begin{extern}%width="320" alt="Calcul formel Bac S Antilles-Guyane 2018 "
               \def\arraystretch{1.8}%
               \begin{tabular}{cc}
                    \begin{tabular}{|c|l|}
                         \hline
                         \cellcolor{lightgray!50}\begin{tabular}{c}1\end{tabular} &
                         \begin{tabular}{l}
                              $M~:=\{\{0.65,0,45\},\{0.35,0.55\}\}$\\
                              $\qquad M~:=\begin{pmatrix}
                                   0.65&0.45\\0.35&0.55
                              \end{pmatrix}$\\
                         \end{tabular}
                         \\
                         \hline
                         \cellcolor{lightgray!50}\begin{tabular}{c}2\end{tabular} &
                         \begin{tabular}{l}
                              $P_0~:=\{\{ 0.4 \},\{ 0.6 \}\}$\\
                              $\qquad P_0~:=\begin{pmatrix}
                                   0.4\\0.6
                              \end{pmatrix}$\\
                         \end{tabular}
                         \\
                         \hline
                         \cellcolor{lightgray!50}\begin{tabular}{c}3\end{tabular} &
                         \begin{tabular}{l}
                              $Q~:=\{\{ 9,1 \},\{ 7,$-1$ \}\}$\\
                              $\qquad Q~:=\begin{pmatrix}
                                   9&1\\7&-1
                              \end{pmatrix}$\\
                         \end{tabular}
                         \\
                         \hline
                         \cellcolor{lightgray!50}\begin{tabular}{c}4\end{tabular} &
                         \begin{tabular}{l}
                              $T~:=\{\{ 1/16,1/16 \},\{ 7/16,$-9/16$ \}\}$\\
                              $\qquad T~:=\begin{pmatrix}
                                   \frac{1}{16}&\frac{1}{16}\\\frac{7}{16}&-\frac{9}{16}
                              \end{pmatrix}$\\
                         \end{tabular}
                         \\
                         \hline
                         \cellcolor{lightgray!50}\begin{tabular}{c}5\end{tabular} &
                         \begin{tabular}{l}
                              $TQ$\\
                              $\rightarrow~~ \begin{pmatrix}
                                   1&0\\0&1
                              \end{pmatrix}$\\
                         \end{tabular}
                         \\
                         \hline
                         \cellcolor{lightgray!50}\begin{tabular}{c}6\end{tabular} &
                         \begin{tabular}{l}
                              $QT$\\
                              $\rightarrow~~ \begin{pmatrix}
                                   1&0\\0&1
                              \end{pmatrix}$\\
                         \end{tabular}
                         \\
                         \hline
                         \cellcolor{lightgray!50}\begin{tabular}{c}7\end{tabular} &
                         \begin{tabular}{l}
                              $D~:=TMQ$\\
                              $\rightarrow~~ D~:=\begin{pmatrix}
                                   1&0\\0&\frac15
                              \end{pmatrix}$\\
                         \end{tabular}
                         \\
                         \hline
                    \end{tabular}
               \end{tabular}
          \end{extern}
     \end{center}
     En vous appuyant sur les résultats précédents, répondre aux deux questions suivantes~:
     \begin{enumerate}[label=\alph*.]
          \item Justifier que $Q$ est une matrice inversible et préciser sa matrice inverse.\\
          On notera $Q^{-1}$ la matrice inverse de $Q$.
          \item Justifier que $M=QDQ^{-1}$ et démontrer que, pour tout entier naturel $n$ non nul~:
          \[M^n=QD^nQ^{-1}.\]
     \end{enumerate}
     \item On admet que, pour tout entier naturel $n$ non nul,
     \[
     M^n=\frac{1}{16}\begin{pmatrix}
          9+7\times0,2^n&9-9\times0,2^n\\
          7-7\times0,2^n&7+9\times0,2^n
     \end{pmatrix}.
     \]
     \begin{enumerate}[label=\alph*.]
          \item
          Démontrer que pour tout entier naturel $n$, $P_n=M^nP_0$.
          \item Justifier que, pour tout entier naturel $n$~:
          \[
          l_n=\frac{9}{16}-\frac{13}{80}\times0,2^n.
          \]
     \end{enumerate}
     \item La proportion de pêcheurs achetant la carte de pêche libre dépassera-t-elle 60~\%~?
\end{enumerate}

\end{document}
µ
\documentclass[a4paper]{article}

%================================================================================================================================
%
% Packages
%
%================================================================================================================================

\usepackage[T1]{fontenc} 	% pour caractères accentués
\usepackage[utf8]{inputenc}  % encodage utf8
\usepackage[french]{babel}	% langue : français
\usepackage{fourier}			% caractères plus lisibles
\usepackage[dvipsnames]{xcolor} % couleurs
\usepackage{fancyhdr}		% réglage header footer
\usepackage{needspace}		% empêcher sauts de page mal placés
\usepackage{graphicx}		% pour inclure des graphiques
\usepackage{enumitem,cprotect}		% personnalise les listes d'items (nécessaire pour ol, al ...)
\usepackage{hyperref}		% Liens hypertexte
\usepackage{pstricks,pst-all,pst-node,pstricks-add,pst-math,pst-plot,pst-tree,pst-eucl} % pstricks
\usepackage[a4paper,includeheadfoot,top=2cm,left=3cm, bottom=2cm,right=3cm]{geometry} % marges etc.
\usepackage{comment}			% commentaires multilignes
\usepackage{amsmath,environ} % maths (matrices, etc.)
\usepackage{amssymb,makeidx}
\usepackage{bm}				% bold maths
\usepackage{tabularx}		% tableaux
\usepackage{colortbl}		% tableaux en couleur
\usepackage{fontawesome}		% Fontawesome
\usepackage{environ}			% environment with command
\usepackage{fp}				% calculs pour ps-tricks
\usepackage{multido}			% pour ps tricks
\usepackage[np]{numprint}	% formattage nombre
\usepackage{tikz,tkz-tab} 			% package principal TikZ
\usepackage{pgfplots}   % axes
\usepackage{mathrsfs}    % cursives
\usepackage{calc}			% calcul taille boites
\usepackage[scaled=0.875]{helvet} % font sans serif
\usepackage{svg} % svg
\usepackage{scrextend} % local margin
\usepackage{scratch} %scratch
\usepackage{multicol} % colonnes
%\usepackage{infix-RPN,pst-func} % formule en notation polanaise inversée
\usepackage{listings}

%================================================================================================================================
%
% Réglages de base
%
%================================================================================================================================

\lstset{
language=Python,   % R code
literate=
{á}{{\'a}}1
{à}{{\`a}}1
{ã}{{\~a}}1
{é}{{\'e}}1
{è}{{\`e}}1
{ê}{{\^e}}1
{í}{{\'i}}1
{ó}{{\'o}}1
{õ}{{\~o}}1
{ú}{{\'u}}1
{ü}{{\"u}}1
{ç}{{\c{c}}}1
{~}{{ }}1
}


\definecolor{codegreen}{rgb}{0,0.6,0}
\definecolor{codegray}{rgb}{0.5,0.5,0.5}
\definecolor{codepurple}{rgb}{0.58,0,0.82}
\definecolor{backcolour}{rgb}{0.95,0.95,0.92}

\lstdefinestyle{mystyle}{
    backgroundcolor=\color{backcolour},   
    commentstyle=\color{codegreen},
    keywordstyle=\color{magenta},
    numberstyle=\tiny\color{codegray},
    stringstyle=\color{codepurple},
    basicstyle=\ttfamily\footnotesize,
    breakatwhitespace=false,         
    breaklines=true,                 
    captionpos=b,                    
    keepspaces=true,                 
    numbers=left,                    
xleftmargin=2em,
framexleftmargin=2em,            
    showspaces=false,                
    showstringspaces=false,
    showtabs=false,                  
    tabsize=2,
    upquote=true
}

\lstset{style=mystyle}


\lstset{style=mystyle}
\newcommand{\imgdir}{C:/laragon/www/newmc/assets/imgsvg/}
\newcommand{\imgsvgdir}{C:/laragon/www/newmc/assets/imgsvg/}

\definecolor{mcgris}{RGB}{220, 220, 220}% ancien~; pour compatibilité
\definecolor{mcbleu}{RGB}{52, 152, 219}
\definecolor{mcvert}{RGB}{125, 194, 70}
\definecolor{mcmauve}{RGB}{154, 0, 215}
\definecolor{mcorange}{RGB}{255, 96, 0}
\definecolor{mcturquoise}{RGB}{0, 153, 153}
\definecolor{mcrouge}{RGB}{255, 0, 0}
\definecolor{mclightvert}{RGB}{205, 234, 190}

\definecolor{gris}{RGB}{220, 220, 220}
\definecolor{bleu}{RGB}{52, 152, 219}
\definecolor{vert}{RGB}{125, 194, 70}
\definecolor{mauve}{RGB}{154, 0, 215}
\definecolor{orange}{RGB}{255, 96, 0}
\definecolor{turquoise}{RGB}{0, 153, 153}
\definecolor{rouge}{RGB}{255, 0, 0}
\definecolor{lightvert}{RGB}{205, 234, 190}
\setitemize[0]{label=\color{lightvert}  $\bullet$}

\pagestyle{fancy}
\renewcommand{\headrulewidth}{0.2pt}
\fancyhead[L]{maths-cours.fr}
\fancyhead[R]{\thepage}
\renewcommand{\footrulewidth}{0.2pt}
\fancyfoot[C]{}

\newcolumntype{C}{>{\centering\arraybackslash}X}
\newcolumntype{s}{>{\hsize=.35\hsize\arraybackslash}X}

\setlength{\parindent}{0pt}		 
\setlength{\parskip}{3mm}
\setlength{\headheight}{1cm}

\def\ebook{ebook}
\def\book{book}
\def\web{web}
\def\type{web}

\newcommand{\vect}[1]{\overrightarrow{\,\mathstrut#1\,}}

\def\Oij{$\left(\text{O}~;~\vect{\imath},~\vect{\jmath}\right)$}
\def\Oijk{$\left(\text{O}~;~\vect{\imath},~\vect{\jmath},~\vect{k}\right)$}
\def\Ouv{$\left(\text{O}~;~\vect{u},~\vect{v}\right)$}

\hypersetup{breaklinks=true, colorlinks = true, linkcolor = OliveGreen, urlcolor = OliveGreen, citecolor = OliveGreen, pdfauthor={Didier BONNEL - https://www.maths-cours.fr} } % supprime les bordures autour des liens

\renewcommand{\arg}[0]{\text{arg}}

\everymath{\displaystyle}

%================================================================================================================================
%
% Macros - Commandes
%
%================================================================================================================================

\newcommand\meta[2]{    			% Utilisé pour créer le post HTML.
	\def\titre{titre}
	\def\url{url}
	\def\arg{#1}
	\ifx\titre\arg
		\newcommand\maintitle{#2}
		\fancyhead[L]{#2}
		{\Large\sffamily \MakeUppercase{#2}}
		\vspace{1mm}\textcolor{mcvert}{\hrule}
	\fi 
	\ifx\url\arg
		\fancyfoot[L]{\href{https://www.maths-cours.fr#2}{\black \footnotesize{https://www.maths-cours.fr#2}}}
	\fi 
}


\newcommand\TitreC[1]{    		% Titre centré
     \needspace{3\baselineskip}
     \begin{center}\textbf{#1}\end{center}
}

\newcommand\newpar{    		% paragraphe
     \par
}

\newcommand\nosp {    		% commande vide (pas d'espace)
}
\newcommand{\id}[1]{} %ignore

\newcommand\boite[2]{				% Boite simple sans titre
	\vspace{5mm}
	\setlength{\fboxrule}{0.2mm}
	\setlength{\fboxsep}{5mm}	
	\fcolorbox{#1}{#1!3}{\makebox[\linewidth-2\fboxrule-2\fboxsep]{
  		\begin{minipage}[t]{\linewidth-2\fboxrule-4\fboxsep}\setlength{\parskip}{3mm}
  			 #2
  		\end{minipage}
	}}
	\vspace{5mm}
}

\newcommand\CBox[4]{				% Boites
	\vspace{5mm}
	\setlength{\fboxrule}{0.2mm}
	\setlength{\fboxsep}{5mm}
	
	\fcolorbox{#1}{#1!3}{\makebox[\linewidth-2\fboxrule-2\fboxsep]{
		\begin{minipage}[t]{1cm}\setlength{\parskip}{3mm}
	  		\textcolor{#1}{\LARGE{#2}}    
 	 	\end{minipage}  
  		\begin{minipage}[t]{\linewidth-2\fboxrule-4\fboxsep}\setlength{\parskip}{3mm}
			\raisebox{1.2mm}{\normalsize\sffamily{\textcolor{#1}{#3}}}						
  			 #4
  		\end{minipage}
	}}
	\vspace{5mm}
}

\newcommand\cadre[3]{				% Boites convertible html
	\par
	\vspace{2mm}
	\setlength{\fboxrule}{0.1mm}
	\setlength{\fboxsep}{5mm}
	\fcolorbox{#1}{white}{\makebox[\linewidth-2\fboxrule-2\fboxsep]{
  		\begin{minipage}[t]{\linewidth-2\fboxrule-4\fboxsep}\setlength{\parskip}{3mm}
			\raisebox{-2.5mm}{\sffamily \small{\textcolor{#1}{\MakeUppercase{#2}}}}		
			\par		
  			 #3
 	 		\end{minipage}
	}}
		\vspace{2mm}
	\par
}

\newcommand\bloc[3]{				% Boites convertible html sans bordure
     \needspace{2\baselineskip}
     {\sffamily \small{\textcolor{#1}{\MakeUppercase{#2}}}}    
		\par		
  			 #3
		\par
}

\newcommand\CHelp[1]{
     \CBox{Plum}{\faInfoCircle}{À RETENIR}{#1}
}

\newcommand\CUp[1]{
     \CBox{NavyBlue}{\faThumbsOUp}{EN PRATIQUE}{#1}
}

\newcommand\CInfo[1]{
     \CBox{Sepia}{\faArrowCircleRight}{REMARQUE}{#1}
}

\newcommand\CRedac[1]{
     \CBox{PineGreen}{\faEdit}{BIEN R\'EDIGER}{#1}
}

\newcommand\CError[1]{
     \CBox{Red}{\faExclamationTriangle}{ATTENTION}{#1}
}

\newcommand\TitreExo[2]{
\needspace{4\baselineskip}
 {\sffamily\large EXERCICE #1\ (\emph{#2 points})}
\vspace{5mm}
}

\newcommand\img[2]{
          \includegraphics[width=#2\paperwidth]{\imgdir#1}
}

\newcommand\imgsvg[2]{
       \begin{center}   \includegraphics[width=#2\paperwidth]{\imgsvgdir#1} \end{center}
}


\newcommand\Lien[2]{
     \href{#1}{#2 \tiny \faExternalLink}
}
\newcommand\mcLien[2]{
     \href{https~://www.maths-cours.fr/#1}{#2 \tiny \faExternalLink}
}

\newcommand{\euro}{\eurologo{}}

%================================================================================================================================
%
% Macros - Environement
%
%================================================================================================================================

\newenvironment{tex}{ %
}
{%
}

\newenvironment{indente}{ %
	\setlength\parindent{10mm}
}

{
	\setlength\parindent{0mm}
}

\newenvironment{corrige}{%
     \needspace{3\baselineskip}
     \medskip
     \textbf{\textsc{Corrigé}}
     \medskip
}
{
}

\newenvironment{extern}{%
     \begin{center}
     }
     {
     \end{center}
}

\NewEnviron{code}{%
	\par
     \boite{gray}{\texttt{%
     \BODY
     }}
     \par
}

\newenvironment{vbloc}{% boite sans cadre empeche saut de page
     \begin{minipage}[t]{\linewidth}
     }
     {
     \end{minipage}
}
\NewEnviron{h2}{%
    \needspace{3\baselineskip}
    \vspace{0.6cm}
	\noindent \MakeUppercase{\sffamily \large \BODY}
	\vspace{1mm}\textcolor{mcgris}{\hrule}\vspace{0.4cm}
	\par
}{}

\NewEnviron{h3}{%
    \needspace{3\baselineskip}
	\vspace{5mm}
	\textsc{\BODY}
	\par
}

\NewEnviron{margeneg}{ %
\begin{addmargin}[-1cm]{0cm}
\BODY
\end{addmargin}
}

\NewEnviron{html}{%
}

\begin{document}
\meta{url}{/exercices/qcm-bac-es-l-antilles-guyane-2018/}
\meta{pid}{9102}
\meta{titre}{QCM - Bac ES/L Antilles-Guyane 2018}
\meta{type}{exercices}
%
\begin{h2}Exercice 1 (4 points)\end{h2}
\textbf{Commun à  tous les candidats}
\medbreak
\emph{Pour chacune des questions suivantes, une seule des quatre réponses proposées est exacte. Aucune justification n'est demandée. Une bonne réponse rapporte un point. Une mauvaise réponse, plusieurs réponses ou l'absence de réponse ne rapportent, ni n'enlèvent aucun point.}
\medbreak
\emph{Indiquer sur la copie le numéro de la question et recopier la réponse choisie.}
\medbreak
\begin{enumerate}
     \item Soit la fonction $f$ définie sur l'intervalle $[-10~;~10]$ par $f(x) = (2x - 3)\text{e}^{-3x}$.
     \par
     L'équation $f(x) = 0$ admet sur l'intervalle $[-10~;~10]$~:
     \begin{tabularx}{\linewidth}{XX} %class="cel50 noborder"
          \textbf{a.~~} 0 solution&\textbf{b.~~}1 solution\\
          \textbf{c.~~} 2 solutions&\textbf{d.~~}3 solutions ou plus\\
     \end{tabularx}
     \item  Dans un repère $(O~;~\overrightarrow{u},~\overrightarrow{v})$ on considère la courbe représentative de la fonction $x \longmapsto \ln (x)$~; l'équation de sa tangente au point d'abscisse $1$ est~:
     \begin{tabularx}{\linewidth}{XX} %class="cel50 noborder"
          \textbf{a.~~} $y= 1$ &\textbf{b.~~} $y = x-1$ \\
          \textbf{c.~~} $y = 1- x$ &\textbf{d.~~} $y = x+ 1$\\
     \end{tabularx}
     \item  Soit $X$ une variable aléatoire qui suit la loi normale de paramètres $\mu = 25$ et $\sigma = 3$.
     \par
     La meilleure valeur approchée du réel $t$ tel que $P(X > t) = 0,025$ est~:
     \begin{tabularx}{\linewidth}{XX} %class="cel50 noborder"
          \textbf{a.~~} $t \approx 0,97$ &\textbf{b.~~} $t \approx 19,12$ \\
          \textbf{c.~~} $t \approx 28$ &\textbf{d.~~} $t \approx 30,88$\\
     \end{tabularx}
     \item  Anne prévoit d'appeler Benoît par téléphone à un moment choisi au hasard entre
     8~h 30 et 10~h. Benoît sera dans un train à partir de 9~h pour un trajet de plusieurs
     heures.
     \par
     Quelle est la probabilité qu'Anne appelle Benoît alors qu'il est dans le train~?
     \begin{tabularx}{\linewidth}{XX} %class="cel50 noborder"
          \textbf{a.~~}$\dfrac{60}{150}$&\textbf{b.~~}$\dfrac{2}{3}$\\
          \textbf{c.~~}$\dfrac{6}{13}$&\textbf{d.~~}$\dfrac{1}{3}$\\
     \end{tabularx}
\end{enumerate}

\end{document}
µ
\documentclass[a4paper]{article}

%================================================================================================================================
%
% Packages
%
%================================================================================================================================

\usepackage[T1]{fontenc} 	% pour caractères accentués
\usepackage[utf8]{inputenc}  % encodage utf8
\usepackage[french]{babel}	% langue : français
\usepackage{fourier}			% caractères plus lisibles
\usepackage[dvipsnames]{xcolor} % couleurs
\usepackage{fancyhdr}		% réglage header footer
\usepackage{needspace}		% empêcher sauts de page mal placés
\usepackage{graphicx}		% pour inclure des graphiques
\usepackage{enumitem,cprotect}		% personnalise les listes d'items (nécessaire pour ol, al ...)
\usepackage{hyperref}		% Liens hypertexte
\usepackage{pstricks,pst-all,pst-node,pstricks-add,pst-math,pst-plot,pst-tree,pst-eucl} % pstricks
\usepackage[a4paper,includeheadfoot,top=2cm,left=3cm, bottom=2cm,right=3cm]{geometry} % marges etc.
\usepackage{comment}			% commentaires multilignes
\usepackage{amsmath,environ} % maths (matrices, etc.)
\usepackage{amssymb,makeidx}
\usepackage{bm}				% bold maths
\usepackage{tabularx}		% tableaux
\usepackage{colortbl}		% tableaux en couleur
\usepackage{fontawesome}		% Fontawesome
\usepackage{environ}			% environment with command
\usepackage{fp}				% calculs pour ps-tricks
\usepackage{multido}			% pour ps tricks
\usepackage[np]{numprint}	% formattage nombre
\usepackage{tikz,tkz-tab} 			% package principal TikZ
\usepackage{pgfplots}   % axes
\usepackage{mathrsfs}    % cursives
\usepackage{calc}			% calcul taille boites
\usepackage[scaled=0.875]{helvet} % font sans serif
\usepackage{svg} % svg
\usepackage{scrextend} % local margin
\usepackage{scratch} %scratch
\usepackage{multicol} % colonnes
%\usepackage{infix-RPN,pst-func} % formule en notation polanaise inversée
\usepackage{listings}

%================================================================================================================================
%
% Réglages de base
%
%================================================================================================================================

\lstset{
language=Python,   % R code
literate=
{á}{{\'a}}1
{à}{{\`a}}1
{ã}{{\~a}}1
{é}{{\'e}}1
{è}{{\`e}}1
{ê}{{\^e}}1
{í}{{\'i}}1
{ó}{{\'o}}1
{õ}{{\~o}}1
{ú}{{\'u}}1
{ü}{{\"u}}1
{ç}{{\c{c}}}1
{~}{{ }}1
}


\definecolor{codegreen}{rgb}{0,0.6,0}
\definecolor{codegray}{rgb}{0.5,0.5,0.5}
\definecolor{codepurple}{rgb}{0.58,0,0.82}
\definecolor{backcolour}{rgb}{0.95,0.95,0.92}

\lstdefinestyle{mystyle}{
    backgroundcolor=\color{backcolour},   
    commentstyle=\color{codegreen},
    keywordstyle=\color{magenta},
    numberstyle=\tiny\color{codegray},
    stringstyle=\color{codepurple},
    basicstyle=\ttfamily\footnotesize,
    breakatwhitespace=false,         
    breaklines=true,                 
    captionpos=b,                    
    keepspaces=true,                 
    numbers=left,                    
xleftmargin=2em,
framexleftmargin=2em,            
    showspaces=false,                
    showstringspaces=false,
    showtabs=false,                  
    tabsize=2,
    upquote=true
}

\lstset{style=mystyle}


\lstset{style=mystyle}
\newcommand{\imgdir}{C:/laragon/www/newmc/assets/imgsvg/}
\newcommand{\imgsvgdir}{C:/laragon/www/newmc/assets/imgsvg/}

\definecolor{mcgris}{RGB}{220, 220, 220}% ancien~; pour compatibilité
\definecolor{mcbleu}{RGB}{52, 152, 219}
\definecolor{mcvert}{RGB}{125, 194, 70}
\definecolor{mcmauve}{RGB}{154, 0, 215}
\definecolor{mcorange}{RGB}{255, 96, 0}
\definecolor{mcturquoise}{RGB}{0, 153, 153}
\definecolor{mcrouge}{RGB}{255, 0, 0}
\definecolor{mclightvert}{RGB}{205, 234, 190}

\definecolor{gris}{RGB}{220, 220, 220}
\definecolor{bleu}{RGB}{52, 152, 219}
\definecolor{vert}{RGB}{125, 194, 70}
\definecolor{mauve}{RGB}{154, 0, 215}
\definecolor{orange}{RGB}{255, 96, 0}
\definecolor{turquoise}{RGB}{0, 153, 153}
\definecolor{rouge}{RGB}{255, 0, 0}
\definecolor{lightvert}{RGB}{205, 234, 190}
\setitemize[0]{label=\color{lightvert}  $\bullet$}

\pagestyle{fancy}
\renewcommand{\headrulewidth}{0.2pt}
\fancyhead[L]{maths-cours.fr}
\fancyhead[R]{\thepage}
\renewcommand{\footrulewidth}{0.2pt}
\fancyfoot[C]{}

\newcolumntype{C}{>{\centering\arraybackslash}X}
\newcolumntype{s}{>{\hsize=.35\hsize\arraybackslash}X}

\setlength{\parindent}{0pt}		 
\setlength{\parskip}{3mm}
\setlength{\headheight}{1cm}

\def\ebook{ebook}
\def\book{book}
\def\web{web}
\def\type{web}

\newcommand{\vect}[1]{\overrightarrow{\,\mathstrut#1\,}}

\def\Oij{$\left(\text{O}~;~\vect{\imath},~\vect{\jmath}\right)$}
\def\Oijk{$\left(\text{O}~;~\vect{\imath},~\vect{\jmath},~\vect{k}\right)$}
\def\Ouv{$\left(\text{O}~;~\vect{u},~\vect{v}\right)$}

\hypersetup{breaklinks=true, colorlinks = true, linkcolor = OliveGreen, urlcolor = OliveGreen, citecolor = OliveGreen, pdfauthor={Didier BONNEL - https://www.maths-cours.fr} } % supprime les bordures autour des liens

\renewcommand{\arg}[0]{\text{arg}}

\everymath{\displaystyle}

%================================================================================================================================
%
% Macros - Commandes
%
%================================================================================================================================

\newcommand\meta[2]{    			% Utilisé pour créer le post HTML.
	\def\titre{titre}
	\def\url{url}
	\def\arg{#1}
	\ifx\titre\arg
		\newcommand\maintitle{#2}
		\fancyhead[L]{#2}
		{\Large\sffamily \MakeUppercase{#2}}
		\vspace{1mm}\textcolor{mcvert}{\hrule}
	\fi 
	\ifx\url\arg
		\fancyfoot[L]{\href{https://www.maths-cours.fr#2}{\black \footnotesize{https://www.maths-cours.fr#2}}}
	\fi 
}


\newcommand\TitreC[1]{    		% Titre centré
     \needspace{3\baselineskip}
     \begin{center}\textbf{#1}\end{center}
}

\newcommand\newpar{    		% paragraphe
     \par
}

\newcommand\nosp {    		% commande vide (pas d'espace)
}
\newcommand{\id}[1]{} %ignore

\newcommand\boite[2]{				% Boite simple sans titre
	\vspace{5mm}
	\setlength{\fboxrule}{0.2mm}
	\setlength{\fboxsep}{5mm}	
	\fcolorbox{#1}{#1!3}{\makebox[\linewidth-2\fboxrule-2\fboxsep]{
  		\begin{minipage}[t]{\linewidth-2\fboxrule-4\fboxsep}\setlength{\parskip}{3mm}
  			 #2
  		\end{minipage}
	}}
	\vspace{5mm}
}

\newcommand\CBox[4]{				% Boites
	\vspace{5mm}
	\setlength{\fboxrule}{0.2mm}
	\setlength{\fboxsep}{5mm}
	
	\fcolorbox{#1}{#1!3}{\makebox[\linewidth-2\fboxrule-2\fboxsep]{
		\begin{minipage}[t]{1cm}\setlength{\parskip}{3mm}
	  		\textcolor{#1}{\LARGE{#2}}    
 	 	\end{minipage}  
  		\begin{minipage}[t]{\linewidth-2\fboxrule-4\fboxsep}\setlength{\parskip}{3mm}
			\raisebox{1.2mm}{\normalsize\sffamily{\textcolor{#1}{#3}}}						
  			 #4
  		\end{minipage}
	}}
	\vspace{5mm}
}

\newcommand\cadre[3]{				% Boites convertible html
	\par
	\vspace{2mm}
	\setlength{\fboxrule}{0.1mm}
	\setlength{\fboxsep}{5mm}
	\fcolorbox{#1}{white}{\makebox[\linewidth-2\fboxrule-2\fboxsep]{
  		\begin{minipage}[t]{\linewidth-2\fboxrule-4\fboxsep}\setlength{\parskip}{3mm}
			\raisebox{-2.5mm}{\sffamily \small{\textcolor{#1}{\MakeUppercase{#2}}}}		
			\par		
  			 #3
 	 		\end{minipage}
	}}
		\vspace{2mm}
	\par
}

\newcommand\bloc[3]{				% Boites convertible html sans bordure
     \needspace{2\baselineskip}
     {\sffamily \small{\textcolor{#1}{\MakeUppercase{#2}}}}    
		\par		
  			 #3
		\par
}

\newcommand\CHelp[1]{
     \CBox{Plum}{\faInfoCircle}{À RETENIR}{#1}
}

\newcommand\CUp[1]{
     \CBox{NavyBlue}{\faThumbsOUp}{EN PRATIQUE}{#1}
}

\newcommand\CInfo[1]{
     \CBox{Sepia}{\faArrowCircleRight}{REMARQUE}{#1}
}

\newcommand\CRedac[1]{
     \CBox{PineGreen}{\faEdit}{BIEN R\'EDIGER}{#1}
}

\newcommand\CError[1]{
     \CBox{Red}{\faExclamationTriangle}{ATTENTION}{#1}
}

\newcommand\TitreExo[2]{
\needspace{4\baselineskip}
 {\sffamily\large EXERCICE #1\ (\emph{#2 points})}
\vspace{5mm}
}

\newcommand\img[2]{
          \includegraphics[width=#2\paperwidth]{\imgdir#1}
}

\newcommand\imgsvg[2]{
       \begin{center}   \includegraphics[width=#2\paperwidth]{\imgsvgdir#1} \end{center}
}


\newcommand\Lien[2]{
     \href{#1}{#2 \tiny \faExternalLink}
}
\newcommand\mcLien[2]{
     \href{https~://www.maths-cours.fr/#1}{#2 \tiny \faExternalLink}
}

\newcommand{\euro}{\eurologo{}}

%================================================================================================================================
%
% Macros - Environement
%
%================================================================================================================================

\newenvironment{tex}{ %
}
{%
}

\newenvironment{indente}{ %
	\setlength\parindent{10mm}
}

{
	\setlength\parindent{0mm}
}

\newenvironment{corrige}{%
     \needspace{3\baselineskip}
     \medskip
     \textbf{\textsc{Corrigé}}
     \medskip
}
{
}

\newenvironment{extern}{%
     \begin{center}
     }
     {
     \end{center}
}

\NewEnviron{code}{%
	\par
     \boite{gray}{\texttt{%
     \BODY
     }}
     \par
}

\newenvironment{vbloc}{% boite sans cadre empeche saut de page
     \begin{minipage}[t]{\linewidth}
     }
     {
     \end{minipage}
}
\NewEnviron{h2}{%
    \needspace{3\baselineskip}
    \vspace{0.6cm}
	\noindent \MakeUppercase{\sffamily \large \BODY}
	\vspace{1mm}\textcolor{mcgris}{\hrule}\vspace{0.4cm}
	\par
}{}

\NewEnviron{h3}{%
    \needspace{3\baselineskip}
	\vspace{5mm}
	\textsc{\BODY}
	\par
}

\NewEnviron{margeneg}{ %
\begin{addmargin}[-1cm]{0cm}
\BODY
\end{addmargin}
}

\NewEnviron{html}{%
}

\begin{document}
\meta{url}{/exercices/probabilites-bac-es-l-antilles-guyane-2018/}
\meta{pid}{9104}
\meta{titre}{Probabilités - Bac ES/L Antilles-Guyane 2018}
\meta{type}{exercices}
%
\begin{h2}Exercice 2 (5 points)\end{h2}
\textbf{Candidats ES n'ayant pas choisi l'enseignement de spécialité \og Mathématiques \fg{}  et candidats L}
\par
\medbreak
\par
\emph{Dans tout cet exercice les résultats seront arrondis au centième si nécessaire.}
\par
\medbreak
\par
\textbf{Les parties A et B sont indépendantes.}
\par
\medbreak
\TitreC{Partie A}
\medbreak
\par
Victor a téléchargé un jeu sur son téléphone. Le but de ce jeu est d'affronter des obstacles à l'aide de personnages qui peuvent être de trois types~: \og Terre \fg, \og Air\fg{} ou \og Feu \fg.
\par
Au début de chaque partie, Victor obtient de façon aléatoire un personnage d'un des trois
types et peut, en cours de partie, conserver ce personnage ou changer une seule fois de type de personnage.
\par
Le jeu a été programmé de telle sorte que~:
\begin{itemize}
     \item la probabilité que la partie débute avec un personnage de type \og Terre\fg{} est $0,3$~;
     \item la probabilité que la partie débute avec un personnage de type \og Air\fg{} est $0,5$~;
     \item si la partie débute avec un personnage de type \og Terre \fg, la probabilité que celui-ci soit conservé est $0,5$~;
     \item si la partie débute avec un personnage de type \og Air\fg, la probabilité que celui-ci soit conservé est $0,4$~;
     \item si la partie débute avec un personnage de type \og Feu \fg, la probabilité que celui-ci soit conservé est $0,9$.
\end{itemize}
\par
On note les événements suivants~:
\begin{itemize}
     \item $T$~: la partie débute avec un personnage de type \og Terre \fg{}~;
     \item $A$~: la partie débute avec un personnage de type \og Air\fg{}~;
     \item $F$~: la partie débute avec un personnage de type \og Feu \fg{}~;
     \item $C$~: Victor conserve le même personnage tout au long de la partie.
\end{itemize}
\medbreak
\begin{enumerate}
     \item Recopier et compléter l'arbre de probabilités ci-dessous.
     \begin{center}
          \begin{extern}%width="250" alt="Arbre de probabilités Bac ES/L Antilles-Guyane 2018"
               %:-+-+-+- Engendré par : http://math.et.info.free.fr/TikZ/Arbre/
               % Racine à Gauche, développement vers la droite
               \begin{tikzpicture}[xscale=1,yscale=1]
                    % Styles (MODIFIABLES)
                    \tikzstyle{fleche}=[-,>=latex,thick]
                    \tikzstyle{noeud}=[circle,draw]
                    \tikzstyle{feuille}=[circle,draw]
                    \tikzstyle{etiquette}=[midway,fill=white]
                    % Dimensions (MODIFIABLES)
                    \def\DistanceInterNiveaux{2.2}
                    \def\DistanceInterFeuilles{1}
                    % Dimensions calculées (NON MODIFIABLES)
                    \def\NiveauA{(0)*\DistanceInterNiveaux}
                    \def\NiveauB{(1.5)*\DistanceInterNiveaux}
                    \def\NiveauC{(2.5)*\DistanceInterNiveaux}
                    \def\InterFeuilles{(-1)*\DistanceInterFeuilles}
                    % Noeuds (MODIFIABLES : Styles et Coefficients d'InterFeuilles)
                    \node[noeud] (R) at ({\NiveauA},{(2.5)*\InterFeuilles}) {$ $};
                    \node[noeud] (Ra) at ({\NiveauB},{(0.5)*\InterFeuilles}) {$T$};
                    \node[feuille] (Raa) at ({\NiveauC},{(0)*\InterFeuilles}) {$C$};
                    \node[feuille] (Rab) at ({\NiveauC},{(1)*\InterFeuilles}) {$\overline{C}$};
                    \node[noeud] (Rb) at ({\NiveauB},{(2.5)*\InterFeuilles}) {$A$};
                    \node[feuille] (Rba) at ({\NiveauC},{(2)*\InterFeuilles}) {$C$};
                    \node[feuille] (Rbb) at ({\NiveauC},{(3)*\InterFeuilles}) {$\overline{C}$};
                    \node[noeud] (Rc) at ({\NiveauB},{(4.5)*\InterFeuilles}) {$F$};
                    \node[feuille] (Rca) at ({\NiveauC},{(4)*\InterFeuilles}) {$C$};
                    \node[feuille] (Rcb) at ({\NiveauC},{(5)*\InterFeuilles}) {$\overline{C}$};
                    % Arcs (MODIFIABLES : Styles)
                    \draw[fleche] (R)--(Ra) ;
                    \draw[fleche] (Ra)--(Raa) ;
                    \draw[fleche] (Ra)--(Rab) ;
                    \draw[fleche] (R)--(Rb) ;
                    \draw[fleche] (Rb)--(Rba) ;
                    \draw[fleche] (Rb)--(Rbb) ;
                    \draw[fleche] (R)--(Rc) ;
                    \draw[fleche] (Rc)--(Rca) ;
                    \draw[fleche] (Rc)--(Rcb) ;
               \end{tikzpicture}
          \end{extern}
     \end{center}
     \item Calculer la probabilité que Victor obtienne et conserve un personnage de type \og Air \fg.
     \item  Justifier que la probabilité que Victor conserve le personnage obtenu en début de partie est $0,53$.
     \item  On considère une partie au cours de laquelle Victor a conservé le personnage obtenu
     en début de partie.
     \par
     Quelle est la probabilité que ce soit un personnage de type \og Air\fg{}~?
\end{enumerate}
\bigbreak
\TitreC{Partie B}
\medbreak
\par
On considère $10$ parties jouées par Victor, prises indépendamment les unes des autres.
On rappelle que la probabilité que Victor obtienne un personnage de type \og Terre\fg{} est $0,3$.
\par
$Y$ désigne la variable aléatoire qui compte le nombre de personnages de type \og Terre\fg{} obtenus au début de ses $10$ parties.
\par
\medbreak
\begin{enumerate}
     \item Justifier que cette situation peut être modélisée par une loi binomiale dont on précisera les paramètres.
     \item Calculer la probabilité que Victor ait obtenu exactement 3 personnages de type \og Terre\fg{} au début de ses $10$ parties.
     \item Calculer la probabilité que Victor ait obtenu au moins une fois un personnage de type
     \og Terre\fg au début de ses $10$parties.
\end{enumerate}

\end{document}
µ
\documentclass[a4paper]{article}

%================================================================================================================================
%
% Packages
%
%================================================================================================================================

\usepackage[T1]{fontenc} 	% pour caractères accentués
\usepackage[utf8]{inputenc}  % encodage utf8
\usepackage[french]{babel}	% langue : français
\usepackage{fourier}			% caractères plus lisibles
\usepackage[dvipsnames]{xcolor} % couleurs
\usepackage{fancyhdr}		% réglage header footer
\usepackage{needspace}		% empêcher sauts de page mal placés
\usepackage{graphicx}		% pour inclure des graphiques
\usepackage{enumitem,cprotect}		% personnalise les listes d'items (nécessaire pour ol, al ...)
\usepackage{hyperref}		% Liens hypertexte
\usepackage{pstricks,pst-all,pst-node,pstricks-add,pst-math,pst-plot,pst-tree,pst-eucl} % pstricks
\usepackage[a4paper,includeheadfoot,top=2cm,left=3cm, bottom=2cm,right=3cm]{geometry} % marges etc.
\usepackage{comment}			% commentaires multilignes
\usepackage{amsmath,environ} % maths (matrices, etc.)
\usepackage{amssymb,makeidx}
\usepackage{bm}				% bold maths
\usepackage{tabularx}		% tableaux
\usepackage{colortbl}		% tableaux en couleur
\usepackage{fontawesome}		% Fontawesome
\usepackage{environ}			% environment with command
\usepackage{fp}				% calculs pour ps-tricks
\usepackage{multido}			% pour ps tricks
\usepackage[np]{numprint}	% formattage nombre
\usepackage{tikz,tkz-tab} 			% package principal TikZ
\usepackage{pgfplots}   % axes
\usepackage{mathrsfs}    % cursives
\usepackage{calc}			% calcul taille boites
\usepackage[scaled=0.875]{helvet} % font sans serif
\usepackage{svg} % svg
\usepackage{scrextend} % local margin
\usepackage{scratch} %scratch
\usepackage{multicol} % colonnes
%\usepackage{infix-RPN,pst-func} % formule en notation polanaise inversée
\usepackage{listings}

%================================================================================================================================
%
% Réglages de base
%
%================================================================================================================================

\lstset{
language=Python,   % R code
literate=
{á}{{\'a}}1
{à}{{\`a}}1
{ã}{{\~a}}1
{é}{{\'e}}1
{è}{{\`e}}1
{ê}{{\^e}}1
{í}{{\'i}}1
{ó}{{\'o}}1
{õ}{{\~o}}1
{ú}{{\'u}}1
{ü}{{\"u}}1
{ç}{{\c{c}}}1
{~}{{ }}1
}


\definecolor{codegreen}{rgb}{0,0.6,0}
\definecolor{codegray}{rgb}{0.5,0.5,0.5}
\definecolor{codepurple}{rgb}{0.58,0,0.82}
\definecolor{backcolour}{rgb}{0.95,0.95,0.92}

\lstdefinestyle{mystyle}{
    backgroundcolor=\color{backcolour},   
    commentstyle=\color{codegreen},
    keywordstyle=\color{magenta},
    numberstyle=\tiny\color{codegray},
    stringstyle=\color{codepurple},
    basicstyle=\ttfamily\footnotesize,
    breakatwhitespace=false,         
    breaklines=true,                 
    captionpos=b,                    
    keepspaces=true,                 
    numbers=left,                    
xleftmargin=2em,
framexleftmargin=2em,            
    showspaces=false,                
    showstringspaces=false,
    showtabs=false,                  
    tabsize=2,
    upquote=true
}

\lstset{style=mystyle}


\lstset{style=mystyle}
\newcommand{\imgdir}{C:/laragon/www/newmc/assets/imgsvg/}
\newcommand{\imgsvgdir}{C:/laragon/www/newmc/assets/imgsvg/}

\definecolor{mcgris}{RGB}{220, 220, 220}% ancien~; pour compatibilité
\definecolor{mcbleu}{RGB}{52, 152, 219}
\definecolor{mcvert}{RGB}{125, 194, 70}
\definecolor{mcmauve}{RGB}{154, 0, 215}
\definecolor{mcorange}{RGB}{255, 96, 0}
\definecolor{mcturquoise}{RGB}{0, 153, 153}
\definecolor{mcrouge}{RGB}{255, 0, 0}
\definecolor{mclightvert}{RGB}{205, 234, 190}

\definecolor{gris}{RGB}{220, 220, 220}
\definecolor{bleu}{RGB}{52, 152, 219}
\definecolor{vert}{RGB}{125, 194, 70}
\definecolor{mauve}{RGB}{154, 0, 215}
\definecolor{orange}{RGB}{255, 96, 0}
\definecolor{turquoise}{RGB}{0, 153, 153}
\definecolor{rouge}{RGB}{255, 0, 0}
\definecolor{lightvert}{RGB}{205, 234, 190}
\setitemize[0]{label=\color{lightvert}  $\bullet$}

\pagestyle{fancy}
\renewcommand{\headrulewidth}{0.2pt}
\fancyhead[L]{maths-cours.fr}
\fancyhead[R]{\thepage}
\renewcommand{\footrulewidth}{0.2pt}
\fancyfoot[C]{}

\newcolumntype{C}{>{\centering\arraybackslash}X}
\newcolumntype{s}{>{\hsize=.35\hsize\arraybackslash}X}

\setlength{\parindent}{0pt}		 
\setlength{\parskip}{3mm}
\setlength{\headheight}{1cm}

\def\ebook{ebook}
\def\book{book}
\def\web{web}
\def\type{web}

\newcommand{\vect}[1]{\overrightarrow{\,\mathstrut#1\,}}

\def\Oij{$\left(\text{O}~;~\vect{\imath},~\vect{\jmath}\right)$}
\def\Oijk{$\left(\text{O}~;~\vect{\imath},~\vect{\jmath},~\vect{k}\right)$}
\def\Ouv{$\left(\text{O}~;~\vect{u},~\vect{v}\right)$}

\hypersetup{breaklinks=true, colorlinks = true, linkcolor = OliveGreen, urlcolor = OliveGreen, citecolor = OliveGreen, pdfauthor={Didier BONNEL - https://www.maths-cours.fr} } % supprime les bordures autour des liens

\renewcommand{\arg}[0]{\text{arg}}

\everymath{\displaystyle}

%================================================================================================================================
%
% Macros - Commandes
%
%================================================================================================================================

\newcommand\meta[2]{    			% Utilisé pour créer le post HTML.
	\def\titre{titre}
	\def\url{url}
	\def\arg{#1}
	\ifx\titre\arg
		\newcommand\maintitle{#2}
		\fancyhead[L]{#2}
		{\Large\sffamily \MakeUppercase{#2}}
		\vspace{1mm}\textcolor{mcvert}{\hrule}
	\fi 
	\ifx\url\arg
		\fancyfoot[L]{\href{https://www.maths-cours.fr#2}{\black \footnotesize{https://www.maths-cours.fr#2}}}
	\fi 
}


\newcommand\TitreC[1]{    		% Titre centré
     \needspace{3\baselineskip}
     \begin{center}\textbf{#1}\end{center}
}

\newcommand\newpar{    		% paragraphe
     \par
}

\newcommand\nosp {    		% commande vide (pas d'espace)
}
\newcommand{\id}[1]{} %ignore

\newcommand\boite[2]{				% Boite simple sans titre
	\vspace{5mm}
	\setlength{\fboxrule}{0.2mm}
	\setlength{\fboxsep}{5mm}	
	\fcolorbox{#1}{#1!3}{\makebox[\linewidth-2\fboxrule-2\fboxsep]{
  		\begin{minipage}[t]{\linewidth-2\fboxrule-4\fboxsep}\setlength{\parskip}{3mm}
  			 #2
  		\end{minipage}
	}}
	\vspace{5mm}
}

\newcommand\CBox[4]{				% Boites
	\vspace{5mm}
	\setlength{\fboxrule}{0.2mm}
	\setlength{\fboxsep}{5mm}
	
	\fcolorbox{#1}{#1!3}{\makebox[\linewidth-2\fboxrule-2\fboxsep]{
		\begin{minipage}[t]{1cm}\setlength{\parskip}{3mm}
	  		\textcolor{#1}{\LARGE{#2}}    
 	 	\end{minipage}  
  		\begin{minipage}[t]{\linewidth-2\fboxrule-4\fboxsep}\setlength{\parskip}{3mm}
			\raisebox{1.2mm}{\normalsize\sffamily{\textcolor{#1}{#3}}}						
  			 #4
  		\end{minipage}
	}}
	\vspace{5mm}
}

\newcommand\cadre[3]{				% Boites convertible html
	\par
	\vspace{2mm}
	\setlength{\fboxrule}{0.1mm}
	\setlength{\fboxsep}{5mm}
	\fcolorbox{#1}{white}{\makebox[\linewidth-2\fboxrule-2\fboxsep]{
  		\begin{minipage}[t]{\linewidth-2\fboxrule-4\fboxsep}\setlength{\parskip}{3mm}
			\raisebox{-2.5mm}{\sffamily \small{\textcolor{#1}{\MakeUppercase{#2}}}}		
			\par		
  			 #3
 	 		\end{minipage}
	}}
		\vspace{2mm}
	\par
}

\newcommand\bloc[3]{				% Boites convertible html sans bordure
     \needspace{2\baselineskip}
     {\sffamily \small{\textcolor{#1}{\MakeUppercase{#2}}}}    
		\par		
  			 #3
		\par
}

\newcommand\CHelp[1]{
     \CBox{Plum}{\faInfoCircle}{À RETENIR}{#1}
}

\newcommand\CUp[1]{
     \CBox{NavyBlue}{\faThumbsOUp}{EN PRATIQUE}{#1}
}

\newcommand\CInfo[1]{
     \CBox{Sepia}{\faArrowCircleRight}{REMARQUE}{#1}
}

\newcommand\CRedac[1]{
     \CBox{PineGreen}{\faEdit}{BIEN R\'EDIGER}{#1}
}

\newcommand\CError[1]{
     \CBox{Red}{\faExclamationTriangle}{ATTENTION}{#1}
}

\newcommand\TitreExo[2]{
\needspace{4\baselineskip}
 {\sffamily\large EXERCICE #1\ (\emph{#2 points})}
\vspace{5mm}
}

\newcommand\img[2]{
          \includegraphics[width=#2\paperwidth]{\imgdir#1}
}

\newcommand\imgsvg[2]{
       \begin{center}   \includegraphics[width=#2\paperwidth]{\imgsvgdir#1} \end{center}
}


\newcommand\Lien[2]{
     \href{#1}{#2 \tiny \faExternalLink}
}
\newcommand\mcLien[2]{
     \href{https~://www.maths-cours.fr/#1}{#2 \tiny \faExternalLink}
}

\newcommand{\euro}{\eurologo{}}

%================================================================================================================================
%
% Macros - Environement
%
%================================================================================================================================

\newenvironment{tex}{ %
}
{%
}

\newenvironment{indente}{ %
	\setlength\parindent{10mm}
}

{
	\setlength\parindent{0mm}
}

\newenvironment{corrige}{%
     \needspace{3\baselineskip}
     \medskip
     \textbf{\textsc{Corrigé}}
     \medskip
}
{
}

\newenvironment{extern}{%
     \begin{center}
     }
     {
     \end{center}
}

\NewEnviron{code}{%
	\par
     \boite{gray}{\texttt{%
     \BODY
     }}
     \par
}

\newenvironment{vbloc}{% boite sans cadre empeche saut de page
     \begin{minipage}[t]{\linewidth}
     }
     {
     \end{minipage}
}
\NewEnviron{h2}{%
    \needspace{3\baselineskip}
    \vspace{0.6cm}
	\noindent \MakeUppercase{\sffamily \large \BODY}
	\vspace{1mm}\textcolor{mcgris}{\hrule}\vspace{0.4cm}
	\par
}{}

\NewEnviron{h3}{%
    \needspace{3\baselineskip}
	\vspace{5mm}
	\textsc{\BODY}
	\par
}

\NewEnviron{margeneg}{ %
\begin{addmargin}[-1cm]{0cm}
\BODY
\end{addmargin}
}

\NewEnviron{html}{%
}

\begin{document}
\meta{url}{/exercices/suites-bac-es-l-antilles-guyane-2018/}
\meta{pid}{9107}
\meta{titre}{Suites - Bac ES/L Antilles-Guyane 2018}
\meta{type}{exercices}
%
\begin{h2}Exercice 3 (5 points)\end{h2}
\textbf{Commun à tous les candidats}
\par
\medbreak
\par
On définit deux suites $\left(u_n\right)$ et $\left(v_n\right)$ par, pour tout entier naturel $n$
\begin{center}
     $\left\{\begin{array}{l c l}
               u_0 &=& 10\\
               u_{n+1}& =& u_n + 0,4
               \end{array}\right.$\quad  et \quad $ \left\{\begin{array}{l c l} v_0 &=& 8\\
     v_{n+1}& =& 1,028v_n\end{array}\right.$
\end{center}
\begin{enumerate}
     \item
     \begin{enumerate}[label=\alph*.]
          \item Parmi ces deux suites, préciser laquelle est arithmétique et laquelle est géométrique~; donner leurs raisons respectives.
          \item Exprimer $u_n$ et $v_n$ en fonction de l'entier naturel $n$.
     \end{enumerate}
     \item On donne l'algorithme suivant dans lequel $n$ est un entier naturel, et $U$ et $V$ sont des réels qui désignent respectivement les termes de rang $n$ des suites $\left(u_n\right)$ et $\left(v_n\right)$~:
     \begin{center}
          \begin{extern}%width="200"
               \begin{tabularx}{0.3\linewidth}{|X|}\hline
                    $n \gets 0$\\
                    $U \gets 10$\\
                    $V \gets 8$\\
                    Tant que $U > V$\\
                    \hspace{0.8cm}$U \gets U + 0,4$\\
                    \hspace{0.8cm}$V \gets  V \times 1,028$\\
                    \hspace{0.8cm}$n \gets n+1$\\
                    Fin Tant que\\ \hline
               \end{tabularx}
          \end{extern}
     \end{center}
     En sortie de cet algorithme, $n$ a pour valeur 46.
     \par
     Interpréter ce résultat.
     \item En 1798, l'économiste anglais Thomas Malthus publie \og An essay on the principle of population \fg{} dans lequel il émet l'hypothèse que l'accroissement de la population, beaucoup plus rapide que celui des ressources alimentaires, conduira son pays à la famine.
     \par
     Il écrit~:
     \begin{center}
          \boite{black}{
               \og \emph{Nous pouvons donc tenir pour certain que, lorsque la population n'est arrêtée par aucun obstacle, elle va doublant tous les vingt-cinq ans, et croît de période en période selon une progression géométrique...
               Nous sommes donc en état de prononcer, en partant de l'état actuel de la terre habitée, que les moyens de subsistance, dans les circonstances les plus favorables de l'industrie, ne peuvent jamais augmenter plus rapidement que selon une progression arithmétique.}\fg
          }
     \end{center}
     En 1800, la population de l'Angleterre était estimée à $8$ millions d'habitants et l'agriculture anglaise pouvait nourrir $10$ millions de personnes. Le modèle de Malthus admet que la population augmente de $2,8$\,\% chaque année et que les progrès de l'agriculture permettent de nourrir $0,4$~million de personnes de plus chaque année.
     \par
     On utilisera ce modèle pour répondre aux questions suivantes.
     \begin{enumerate}[label=\alph*.]
          \item Quelle aurait été, en million d'habitants, la population de l'Angleterre en 1810~?
          \par
          On arrondira le résultat au millième.
          \item À partir de quelle année la population de l'Angleterre aurait-elle dépassé 16 millions
          d'habitants~?
          \item À partir de quelle année la population de l'Angleterre serait -elle devenue trop grande pour ne plus être suffisamment nourrie par son agriculture~?
     \end{enumerate}
\end{enumerate}

\end{document}
µ
\documentclass[a4paper]{article}

%================================================================================================================================
%
% Packages
%
%================================================================================================================================

\usepackage[T1]{fontenc} 	% pour caractères accentués
\usepackage[utf8]{inputenc}  % encodage utf8
\usepackage[french]{babel}	% langue : français
\usepackage{fourier}			% caractères plus lisibles
\usepackage[dvipsnames]{xcolor} % couleurs
\usepackage{fancyhdr}		% réglage header footer
\usepackage{needspace}		% empêcher sauts de page mal placés
\usepackage{graphicx}		% pour inclure des graphiques
\usepackage{enumitem,cprotect}		% personnalise les listes d'items (nécessaire pour ol, al ...)
\usepackage{hyperref}		% Liens hypertexte
\usepackage{pstricks,pst-all,pst-node,pstricks-add,pst-math,pst-plot,pst-tree,pst-eucl} % pstricks
\usepackage[a4paper,includeheadfoot,top=2cm,left=3cm, bottom=2cm,right=3cm]{geometry} % marges etc.
\usepackage{comment}			% commentaires multilignes
\usepackage{amsmath,environ} % maths (matrices, etc.)
\usepackage{amssymb,makeidx}
\usepackage{bm}				% bold maths
\usepackage{tabularx}		% tableaux
\usepackage{colortbl}		% tableaux en couleur
\usepackage{fontawesome}		% Fontawesome
\usepackage{environ}			% environment with command
\usepackage{fp}				% calculs pour ps-tricks
\usepackage{multido}			% pour ps tricks
\usepackage[np]{numprint}	% formattage nombre
\usepackage{tikz,tkz-tab} 			% package principal TikZ
\usepackage{pgfplots}   % axes
\usepackage{mathrsfs}    % cursives
\usepackage{calc}			% calcul taille boites
\usepackage[scaled=0.875]{helvet} % font sans serif
\usepackage{svg} % svg
\usepackage{scrextend} % local margin
\usepackage{scratch} %scratch
\usepackage{multicol} % colonnes
%\usepackage{infix-RPN,pst-func} % formule en notation polanaise inversée
\usepackage{listings}

%================================================================================================================================
%
% Réglages de base
%
%================================================================================================================================

\lstset{
language=Python,   % R code
literate=
{á}{{\'a}}1
{à}{{\`a}}1
{ã}{{\~a}}1
{é}{{\'e}}1
{è}{{\`e}}1
{ê}{{\^e}}1
{í}{{\'i}}1
{ó}{{\'o}}1
{õ}{{\~o}}1
{ú}{{\'u}}1
{ü}{{\"u}}1
{ç}{{\c{c}}}1
{~}{{ }}1
}


\definecolor{codegreen}{rgb}{0,0.6,0}
\definecolor{codegray}{rgb}{0.5,0.5,0.5}
\definecolor{codepurple}{rgb}{0.58,0,0.82}
\definecolor{backcolour}{rgb}{0.95,0.95,0.92}

\lstdefinestyle{mystyle}{
    backgroundcolor=\color{backcolour},   
    commentstyle=\color{codegreen},
    keywordstyle=\color{magenta},
    numberstyle=\tiny\color{codegray},
    stringstyle=\color{codepurple},
    basicstyle=\ttfamily\footnotesize,
    breakatwhitespace=false,         
    breaklines=true,                 
    captionpos=b,                    
    keepspaces=true,                 
    numbers=left,                    
xleftmargin=2em,
framexleftmargin=2em,            
    showspaces=false,                
    showstringspaces=false,
    showtabs=false,                  
    tabsize=2,
    upquote=true
}

\lstset{style=mystyle}


\lstset{style=mystyle}
\newcommand{\imgdir}{C:/laragon/www/newmc/assets/imgsvg/}
\newcommand{\imgsvgdir}{C:/laragon/www/newmc/assets/imgsvg/}

\definecolor{mcgris}{RGB}{220, 220, 220}% ancien~; pour compatibilité
\definecolor{mcbleu}{RGB}{52, 152, 219}
\definecolor{mcvert}{RGB}{125, 194, 70}
\definecolor{mcmauve}{RGB}{154, 0, 215}
\definecolor{mcorange}{RGB}{255, 96, 0}
\definecolor{mcturquoise}{RGB}{0, 153, 153}
\definecolor{mcrouge}{RGB}{255, 0, 0}
\definecolor{mclightvert}{RGB}{205, 234, 190}

\definecolor{gris}{RGB}{220, 220, 220}
\definecolor{bleu}{RGB}{52, 152, 219}
\definecolor{vert}{RGB}{125, 194, 70}
\definecolor{mauve}{RGB}{154, 0, 215}
\definecolor{orange}{RGB}{255, 96, 0}
\definecolor{turquoise}{RGB}{0, 153, 153}
\definecolor{rouge}{RGB}{255, 0, 0}
\definecolor{lightvert}{RGB}{205, 234, 190}
\setitemize[0]{label=\color{lightvert}  $\bullet$}

\pagestyle{fancy}
\renewcommand{\headrulewidth}{0.2pt}
\fancyhead[L]{maths-cours.fr}
\fancyhead[R]{\thepage}
\renewcommand{\footrulewidth}{0.2pt}
\fancyfoot[C]{}

\newcolumntype{C}{>{\centering\arraybackslash}X}
\newcolumntype{s}{>{\hsize=.35\hsize\arraybackslash}X}

\setlength{\parindent}{0pt}		 
\setlength{\parskip}{3mm}
\setlength{\headheight}{1cm}

\def\ebook{ebook}
\def\book{book}
\def\web{web}
\def\type{web}

\newcommand{\vect}[1]{\overrightarrow{\,\mathstrut#1\,}}

\def\Oij{$\left(\text{O}~;~\vect{\imath},~\vect{\jmath}\right)$}
\def\Oijk{$\left(\text{O}~;~\vect{\imath},~\vect{\jmath},~\vect{k}\right)$}
\def\Ouv{$\left(\text{O}~;~\vect{u},~\vect{v}\right)$}

\hypersetup{breaklinks=true, colorlinks = true, linkcolor = OliveGreen, urlcolor = OliveGreen, citecolor = OliveGreen, pdfauthor={Didier BONNEL - https://www.maths-cours.fr} } % supprime les bordures autour des liens

\renewcommand{\arg}[0]{\text{arg}}

\everymath{\displaystyle}

%================================================================================================================================
%
% Macros - Commandes
%
%================================================================================================================================

\newcommand\meta[2]{    			% Utilisé pour créer le post HTML.
	\def\titre{titre}
	\def\url{url}
	\def\arg{#1}
	\ifx\titre\arg
		\newcommand\maintitle{#2}
		\fancyhead[L]{#2}
		{\Large\sffamily \MakeUppercase{#2}}
		\vspace{1mm}\textcolor{mcvert}{\hrule}
	\fi 
	\ifx\url\arg
		\fancyfoot[L]{\href{https://www.maths-cours.fr#2}{\black \footnotesize{https://www.maths-cours.fr#2}}}
	\fi 
}


\newcommand\TitreC[1]{    		% Titre centré
     \needspace{3\baselineskip}
     \begin{center}\textbf{#1}\end{center}
}

\newcommand\newpar{    		% paragraphe
     \par
}

\newcommand\nosp {    		% commande vide (pas d'espace)
}
\newcommand{\id}[1]{} %ignore

\newcommand\boite[2]{				% Boite simple sans titre
	\vspace{5mm}
	\setlength{\fboxrule}{0.2mm}
	\setlength{\fboxsep}{5mm}	
	\fcolorbox{#1}{#1!3}{\makebox[\linewidth-2\fboxrule-2\fboxsep]{
  		\begin{minipage}[t]{\linewidth-2\fboxrule-4\fboxsep}\setlength{\parskip}{3mm}
  			 #2
  		\end{minipage}
	}}
	\vspace{5mm}
}

\newcommand\CBox[4]{				% Boites
	\vspace{5mm}
	\setlength{\fboxrule}{0.2mm}
	\setlength{\fboxsep}{5mm}
	
	\fcolorbox{#1}{#1!3}{\makebox[\linewidth-2\fboxrule-2\fboxsep]{
		\begin{minipage}[t]{1cm}\setlength{\parskip}{3mm}
	  		\textcolor{#1}{\LARGE{#2}}    
 	 	\end{minipage}  
  		\begin{minipage}[t]{\linewidth-2\fboxrule-4\fboxsep}\setlength{\parskip}{3mm}
			\raisebox{1.2mm}{\normalsize\sffamily{\textcolor{#1}{#3}}}						
  			 #4
  		\end{minipage}
	}}
	\vspace{5mm}
}

\newcommand\cadre[3]{				% Boites convertible html
	\par
	\vspace{2mm}
	\setlength{\fboxrule}{0.1mm}
	\setlength{\fboxsep}{5mm}
	\fcolorbox{#1}{white}{\makebox[\linewidth-2\fboxrule-2\fboxsep]{
  		\begin{minipage}[t]{\linewidth-2\fboxrule-4\fboxsep}\setlength{\parskip}{3mm}
			\raisebox{-2.5mm}{\sffamily \small{\textcolor{#1}{\MakeUppercase{#2}}}}		
			\par		
  			 #3
 	 		\end{minipage}
	}}
		\vspace{2mm}
	\par
}

\newcommand\bloc[3]{				% Boites convertible html sans bordure
     \needspace{2\baselineskip}
     {\sffamily \small{\textcolor{#1}{\MakeUppercase{#2}}}}    
		\par		
  			 #3
		\par
}

\newcommand\CHelp[1]{
     \CBox{Plum}{\faInfoCircle}{À RETENIR}{#1}
}

\newcommand\CUp[1]{
     \CBox{NavyBlue}{\faThumbsOUp}{EN PRATIQUE}{#1}
}

\newcommand\CInfo[1]{
     \CBox{Sepia}{\faArrowCircleRight}{REMARQUE}{#1}
}

\newcommand\CRedac[1]{
     \CBox{PineGreen}{\faEdit}{BIEN R\'EDIGER}{#1}
}

\newcommand\CError[1]{
     \CBox{Red}{\faExclamationTriangle}{ATTENTION}{#1}
}

\newcommand\TitreExo[2]{
\needspace{4\baselineskip}
 {\sffamily\large EXERCICE #1\ (\emph{#2 points})}
\vspace{5mm}
}

\newcommand\img[2]{
          \includegraphics[width=#2\paperwidth]{\imgdir#1}
}

\newcommand\imgsvg[2]{
       \begin{center}   \includegraphics[width=#2\paperwidth]{\imgsvgdir#1} \end{center}
}


\newcommand\Lien[2]{
     \href{#1}{#2 \tiny \faExternalLink}
}
\newcommand\mcLien[2]{
     \href{https~://www.maths-cours.fr/#1}{#2 \tiny \faExternalLink}
}

\newcommand{\euro}{\eurologo{}}

%================================================================================================================================
%
% Macros - Environement
%
%================================================================================================================================

\newenvironment{tex}{ %
}
{%
}

\newenvironment{indente}{ %
	\setlength\parindent{10mm}
}

{
	\setlength\parindent{0mm}
}

\newenvironment{corrige}{%
     \needspace{3\baselineskip}
     \medskip
     \textbf{\textsc{Corrigé}}
     \medskip
}
{
}

\newenvironment{extern}{%
     \begin{center}
     }
     {
     \end{center}
}

\NewEnviron{code}{%
	\par
     \boite{gray}{\texttt{%
     \BODY
     }}
     \par
}

\newenvironment{vbloc}{% boite sans cadre empeche saut de page
     \begin{minipage}[t]{\linewidth}
     }
     {
     \end{minipage}
}
\NewEnviron{h2}{%
    \needspace{3\baselineskip}
    \vspace{0.6cm}
	\noindent \MakeUppercase{\sffamily \large \BODY}
	\vspace{1mm}\textcolor{mcgris}{\hrule}\vspace{0.4cm}
	\par
}{}

\NewEnviron{h3}{%
    \needspace{3\baselineskip}
	\vspace{5mm}
	\textsc{\BODY}
	\par
}

\NewEnviron{margeneg}{ %
\begin{addmargin}[-1cm]{0cm}
\BODY
\end{addmargin}
}

\NewEnviron{html}{%
}

\begin{document}
\meta{url}{/exercices/graphes-bac-es-antilles-guyane-2018-spe/}
\meta{pid}{9109}
\meta{titre}{Graphes - Bac ES Antilles-Guyane 2018 (spé)}
\meta{type}{exercices}
%
\begin{h2}Exercice 2 (5 points)\end{h2}
\textbf{Candidats de ES ayant  choisi l'enseignement de spécialité}
\medbreak
\emph{Les parties} A \emph{et} B \emph{sont indépendantes}
\medbreak
Franck joue en ligne sur internet.
\medbreak
\TitreC{Partie A}
\medbreak
Après plusieurs semaines, des statistiques données par le logiciel lui permettent de dire que~:
\begin{indent}
     \begin{itemize}
          \item quand il gagne une partie, la probabilité qu'il gagne la suivante est égale à $0,65$~;
          \item quand il perd une partie, la probabilité qu'il gagne la suivante est égale à $0,42$.
     \end{itemize}
\end{indent}
\medbreak
On note G l'état~: \og Franck gagne la partie \fg{} et P l'état~: \og Franck perd la partie \fg.
\par
Sur une période donnée, on note, pour tout entier naturel $n$ non nul~:
\begin{indent}
     \begin{itemize}
          \item $g_n$ la probabilité que Franck gagne la $n$-ième partie~;
          \item $p_n$ la probabilité que Franck perde la $n$-ième partie.
     \end{itemize}
\end{indent}
\medbreak
Dans cette période, Franck a gagné la première partie.
\medbreak
\begin{enumerate}
     \item Représenter la situation par un graphe probabiliste de sommets notés G et P.
     \item
     \begin{enumerate}[label=\alph*.]
          \item Écrire la matrice de transition $M$ dans l'ordre G-P.
          \item Calculer la probabilité que Franck gagne la troisième partie.
     \end{enumerate}
     \item  Déterminer l'état stable du système et interpréter le résultat dans le contexte de l'exercice,
\end{enumerate}
\bigbreak
\TitreC{Partie B}
\medbreak
Dans ce jeu vidéo, Franck circule dans des catacombes infestées de monstres qu'il doit combattre.
\smallbreak
On a représenté ci-dessous le graphe modélisant ces
catacombes.
\par
Les sommets représentent les salles et les arêtes représentent les couloirs.
\par
Les étiquettes du graphe correspondent au nombre de monstres présents dans chaque couloir.
\begin{center}
     \begin{extern}%width="320" alt="Graphe probabiliste Bac ES Antilles-Guyane 2018"
          {\psset{unit=0.8cm}
               \begin{pspicture}(8,6)
                    \psset{arcangle=15} \cnodeput(0.2,3){A}{A} \cnodeput(2.2,5){B}{B} \cnodeput(2.5,0.5){C}{C} \cnodeput(4.5,3){D}{D} \cnodeput(6,0.6){E}{E} \cnodeput(6.8,4.5){F}{F} \cnodeput(8.6,2.8){G}{G}
                    \ncarc{A}{B} \ncput*{11}\ncarc{C}{A} \ncput*{12}\ncarc{C}{B} \ncput*{18}\ncarc{B}{D} \ncput*{7}\ncarc{B}{F} \ncput*{22}\ncarc{D}{E} \ncput*{4}\ncarc{D}{F} \ncput*{3}\ncarc{F}{E} \ncput*{11}\ncarc{G}{E} \ncput*{7}
                    \ncarc{F}{G} \ncput*{5}\ncarc{C}{D} \ncput*{5}\ncarc{E}{C} \ncput*{19}
          \end{pspicture}}
     \end{extern}
\end{center}
\begin{enumerate}
     \item
     \begin{enumerate}[label=\alph*.]
          \item Justifier qu'il est possible, au départ d'une salle quelconque, d'y revenir après
          avoir parcouru tous les couloirs une et une seule fois.
          \item Donner un tel chemin.
     \end{enumerate}
     \item Franck débute le jeu dans la salle A et doit atteindre l'adversaire final en salle G.
     \par
     Existe-t-il un chemin permettant de se rendre de la salle A à la salle G en passant une
     et une seule fois par tous les couloirs~?
     \item Une fois arrivé en salle G, Franck souhaite revenir en salle A en affrontant le moins de
     monstres possible afin de recommencer une nouvelle partie.
     \par
     Déterminer ce trajet minimal et préciser le nombre de monstres affrontés.
\end{enumerate}

\end{document}
µ
\documentclass[a4paper]{article}

%================================================================================================================================
%
% Packages
%
%================================================================================================================================

\usepackage[T1]{fontenc} 	% pour caractères accentués
\usepackage[utf8]{inputenc}  % encodage utf8
\usepackage[french]{babel}	% langue : français
\usepackage{fourier}			% caractères plus lisibles
\usepackage[dvipsnames]{xcolor} % couleurs
\usepackage{fancyhdr}		% réglage header footer
\usepackage{needspace}		% empêcher sauts de page mal placés
\usepackage{graphicx}		% pour inclure des graphiques
\usepackage{enumitem,cprotect}		% personnalise les listes d'items (nécessaire pour ol, al ...)
\usepackage{hyperref}		% Liens hypertexte
\usepackage{pstricks,pst-all,pst-node,pstricks-add,pst-math,pst-plot,pst-tree,pst-eucl} % pstricks
\usepackage[a4paper,includeheadfoot,top=2cm,left=3cm, bottom=2cm,right=3cm]{geometry} % marges etc.
\usepackage{comment}			% commentaires multilignes
\usepackage{amsmath,environ} % maths (matrices, etc.)
\usepackage{amssymb,makeidx}
\usepackage{bm}				% bold maths
\usepackage{tabularx}		% tableaux
\usepackage{colortbl}		% tableaux en couleur
\usepackage{fontawesome}		% Fontawesome
\usepackage{environ}			% environment with command
\usepackage{fp}				% calculs pour ps-tricks
\usepackage{multido}			% pour ps tricks
\usepackage[np]{numprint}	% formattage nombre
\usepackage{tikz,tkz-tab} 			% package principal TikZ
\usepackage{pgfplots}   % axes
\usepackage{mathrsfs}    % cursives
\usepackage{calc}			% calcul taille boites
\usepackage[scaled=0.875]{helvet} % font sans serif
\usepackage{svg} % svg
\usepackage{scrextend} % local margin
\usepackage{scratch} %scratch
\usepackage{multicol} % colonnes
%\usepackage{infix-RPN,pst-func} % formule en notation polanaise inversée
\usepackage{listings}

%================================================================================================================================
%
% Réglages de base
%
%================================================================================================================================

\lstset{
language=Python,   % R code
literate=
{á}{{\'a}}1
{à}{{\`a}}1
{ã}{{\~a}}1
{é}{{\'e}}1
{è}{{\`e}}1
{ê}{{\^e}}1
{í}{{\'i}}1
{ó}{{\'o}}1
{õ}{{\~o}}1
{ú}{{\'u}}1
{ü}{{\"u}}1
{ç}{{\c{c}}}1
{~}{{ }}1
}


\definecolor{codegreen}{rgb}{0,0.6,0}
\definecolor{codegray}{rgb}{0.5,0.5,0.5}
\definecolor{codepurple}{rgb}{0.58,0,0.82}
\definecolor{backcolour}{rgb}{0.95,0.95,0.92}

\lstdefinestyle{mystyle}{
    backgroundcolor=\color{backcolour},   
    commentstyle=\color{codegreen},
    keywordstyle=\color{magenta},
    numberstyle=\tiny\color{codegray},
    stringstyle=\color{codepurple},
    basicstyle=\ttfamily\footnotesize,
    breakatwhitespace=false,         
    breaklines=true,                 
    captionpos=b,                    
    keepspaces=true,                 
    numbers=left,                    
xleftmargin=2em,
framexleftmargin=2em,            
    showspaces=false,                
    showstringspaces=false,
    showtabs=false,                  
    tabsize=2,
    upquote=true
}

\lstset{style=mystyle}


\lstset{style=mystyle}
\newcommand{\imgdir}{C:/laragon/www/newmc/assets/imgsvg/}
\newcommand{\imgsvgdir}{C:/laragon/www/newmc/assets/imgsvg/}

\definecolor{mcgris}{RGB}{220, 220, 220}% ancien~; pour compatibilité
\definecolor{mcbleu}{RGB}{52, 152, 219}
\definecolor{mcvert}{RGB}{125, 194, 70}
\definecolor{mcmauve}{RGB}{154, 0, 215}
\definecolor{mcorange}{RGB}{255, 96, 0}
\definecolor{mcturquoise}{RGB}{0, 153, 153}
\definecolor{mcrouge}{RGB}{255, 0, 0}
\definecolor{mclightvert}{RGB}{205, 234, 190}

\definecolor{gris}{RGB}{220, 220, 220}
\definecolor{bleu}{RGB}{52, 152, 219}
\definecolor{vert}{RGB}{125, 194, 70}
\definecolor{mauve}{RGB}{154, 0, 215}
\definecolor{orange}{RGB}{255, 96, 0}
\definecolor{turquoise}{RGB}{0, 153, 153}
\definecolor{rouge}{RGB}{255, 0, 0}
\definecolor{lightvert}{RGB}{205, 234, 190}
\setitemize[0]{label=\color{lightvert}  $\bullet$}

\pagestyle{fancy}
\renewcommand{\headrulewidth}{0.2pt}
\fancyhead[L]{maths-cours.fr}
\fancyhead[R]{\thepage}
\renewcommand{\footrulewidth}{0.2pt}
\fancyfoot[C]{}

\newcolumntype{C}{>{\centering\arraybackslash}X}
\newcolumntype{s}{>{\hsize=.35\hsize\arraybackslash}X}

\setlength{\parindent}{0pt}		 
\setlength{\parskip}{3mm}
\setlength{\headheight}{1cm}

\def\ebook{ebook}
\def\book{book}
\def\web{web}
\def\type{web}

\newcommand{\vect}[1]{\overrightarrow{\,\mathstrut#1\,}}

\def\Oij{$\left(\text{O}~;~\vect{\imath},~\vect{\jmath}\right)$}
\def\Oijk{$\left(\text{O}~;~\vect{\imath},~\vect{\jmath},~\vect{k}\right)$}
\def\Ouv{$\left(\text{O}~;~\vect{u},~\vect{v}\right)$}

\hypersetup{breaklinks=true, colorlinks = true, linkcolor = OliveGreen, urlcolor = OliveGreen, citecolor = OliveGreen, pdfauthor={Didier BONNEL - https://www.maths-cours.fr} } % supprime les bordures autour des liens

\renewcommand{\arg}[0]{\text{arg}}

\everymath{\displaystyle}

%================================================================================================================================
%
% Macros - Commandes
%
%================================================================================================================================

\newcommand\meta[2]{    			% Utilisé pour créer le post HTML.
	\def\titre{titre}
	\def\url{url}
	\def\arg{#1}
	\ifx\titre\arg
		\newcommand\maintitle{#2}
		\fancyhead[L]{#2}
		{\Large\sffamily \MakeUppercase{#2}}
		\vspace{1mm}\textcolor{mcvert}{\hrule}
	\fi 
	\ifx\url\arg
		\fancyfoot[L]{\href{https://www.maths-cours.fr#2}{\black \footnotesize{https://www.maths-cours.fr#2}}}
	\fi 
}


\newcommand\TitreC[1]{    		% Titre centré
     \needspace{3\baselineskip}
     \begin{center}\textbf{#1}\end{center}
}

\newcommand\newpar{    		% paragraphe
     \par
}

\newcommand\nosp {    		% commande vide (pas d'espace)
}
\newcommand{\id}[1]{} %ignore

\newcommand\boite[2]{				% Boite simple sans titre
	\vspace{5mm}
	\setlength{\fboxrule}{0.2mm}
	\setlength{\fboxsep}{5mm}	
	\fcolorbox{#1}{#1!3}{\makebox[\linewidth-2\fboxrule-2\fboxsep]{
  		\begin{minipage}[t]{\linewidth-2\fboxrule-4\fboxsep}\setlength{\parskip}{3mm}
  			 #2
  		\end{minipage}
	}}
	\vspace{5mm}
}

\newcommand\CBox[4]{				% Boites
	\vspace{5mm}
	\setlength{\fboxrule}{0.2mm}
	\setlength{\fboxsep}{5mm}
	
	\fcolorbox{#1}{#1!3}{\makebox[\linewidth-2\fboxrule-2\fboxsep]{
		\begin{minipage}[t]{1cm}\setlength{\parskip}{3mm}
	  		\textcolor{#1}{\LARGE{#2}}    
 	 	\end{minipage}  
  		\begin{minipage}[t]{\linewidth-2\fboxrule-4\fboxsep}\setlength{\parskip}{3mm}
			\raisebox{1.2mm}{\normalsize\sffamily{\textcolor{#1}{#3}}}						
  			 #4
  		\end{minipage}
	}}
	\vspace{5mm}
}

\newcommand\cadre[3]{				% Boites convertible html
	\par
	\vspace{2mm}
	\setlength{\fboxrule}{0.1mm}
	\setlength{\fboxsep}{5mm}
	\fcolorbox{#1}{white}{\makebox[\linewidth-2\fboxrule-2\fboxsep]{
  		\begin{minipage}[t]{\linewidth-2\fboxrule-4\fboxsep}\setlength{\parskip}{3mm}
			\raisebox{-2.5mm}{\sffamily \small{\textcolor{#1}{\MakeUppercase{#2}}}}		
			\par		
  			 #3
 	 		\end{minipage}
	}}
		\vspace{2mm}
	\par
}

\newcommand\bloc[3]{				% Boites convertible html sans bordure
     \needspace{2\baselineskip}
     {\sffamily \small{\textcolor{#1}{\MakeUppercase{#2}}}}    
		\par		
  			 #3
		\par
}

\newcommand\CHelp[1]{
     \CBox{Plum}{\faInfoCircle}{À RETENIR}{#1}
}

\newcommand\CUp[1]{
     \CBox{NavyBlue}{\faThumbsOUp}{EN PRATIQUE}{#1}
}

\newcommand\CInfo[1]{
     \CBox{Sepia}{\faArrowCircleRight}{REMARQUE}{#1}
}

\newcommand\CRedac[1]{
     \CBox{PineGreen}{\faEdit}{BIEN R\'EDIGER}{#1}
}

\newcommand\CError[1]{
     \CBox{Red}{\faExclamationTriangle}{ATTENTION}{#1}
}

\newcommand\TitreExo[2]{
\needspace{4\baselineskip}
 {\sffamily\large EXERCICE #1\ (\emph{#2 points})}
\vspace{5mm}
}

\newcommand\img[2]{
          \includegraphics[width=#2\paperwidth]{\imgdir#1}
}

\newcommand\imgsvg[2]{
       \begin{center}   \includegraphics[width=#2\paperwidth]{\imgsvgdir#1} \end{center}
}


\newcommand\Lien[2]{
     \href{#1}{#2 \tiny \faExternalLink}
}
\newcommand\mcLien[2]{
     \href{https~://www.maths-cours.fr/#1}{#2 \tiny \faExternalLink}
}

\newcommand{\euro}{\eurologo{}}

%================================================================================================================================
%
% Macros - Environement
%
%================================================================================================================================

\newenvironment{tex}{ %
}
{%
}

\newenvironment{indente}{ %
	\setlength\parindent{10mm}
}

{
	\setlength\parindent{0mm}
}

\newenvironment{corrige}{%
     \needspace{3\baselineskip}
     \medskip
     \textbf{\textsc{Corrigé}}
     \medskip
}
{
}

\newenvironment{extern}{%
     \begin{center}
     }
     {
     \end{center}
}

\NewEnviron{code}{%
	\par
     \boite{gray}{\texttt{%
     \BODY
     }}
     \par
}

\newenvironment{vbloc}{% boite sans cadre empeche saut de page
     \begin{minipage}[t]{\linewidth}
     }
     {
     \end{minipage}
}
\NewEnviron{h2}{%
    \needspace{3\baselineskip}
    \vspace{0.6cm}
	\noindent \MakeUppercase{\sffamily \large \BODY}
	\vspace{1mm}\textcolor{mcgris}{\hrule}\vspace{0.4cm}
	\par
}{}

\NewEnviron{h3}{%
    \needspace{3\baselineskip}
	\vspace{5mm}
	\textsc{\BODY}
	\par
}

\NewEnviron{margeneg}{ %
\begin{addmargin}[-1cm]{0cm}
\BODY
\end{addmargin}
}

\NewEnviron{html}{%
}

\begin{document}
\meta{url}{/exercices/fonctions-bac-es-antilles-guyane-2018/}
\meta{pid}{9111}
\meta{titre}{Fonctions - Bac STMG Antilles-Guyane 2018}
\meta{type}{exercices}
%
Une entreprise produit des panneaux solaires. Une étude de marché permet d'estimer que la production
pour le mois à venir est comprise entre 1~500 et 3~000 panneaux solaires. On s'intéresse au
bénéfice de l'entreprise sur la vente des panneaux solaires produits.
\par
On décide de modéliser l'évolution du bénéfice de l'entreprise, exprimé en centaine d'euros, par la
fonction $f$ définie ci-dessous~:
\begin{center}
     $f(x) = - 2x^2 + 90x - 400$, ~ ~ pour~$x \in  [15~;~30]$.
\end{center}
On admet que la fonction $f$ est dérivable sur l'intervalle [15~;~30] et on note $f'$ sa fonction dérivée.
\medbreak
\begin{enumerate}
     \item Étudier les variations de la fonction $f$ sur l'intervalle [15~;~30].
     \item  Calculer son maximum.
     \par
     Les valeurs de $x$, arrondies au centième, représentent le nombre de centaines de panneaux solaires
     produits.
     \item  Pour quelle production le bénéfice est-il maximal~? Quelle est alors sa valeur~?
\end{enumerate}

\end{document}
µ
\documentclass[a4paper]{article}

%================================================================================================================================
%
% Packages
%
%================================================================================================================================

\usepackage[T1]{fontenc} 	% pour caractères accentués
\usepackage[utf8]{inputenc}  % encodage utf8
\usepackage[french]{babel}	% langue : français
\usepackage{fourier}			% caractères plus lisibles
\usepackage[dvipsnames]{xcolor} % couleurs
\usepackage{fancyhdr}		% réglage header footer
\usepackage{needspace}		% empêcher sauts de page mal placés
\usepackage{graphicx}		% pour inclure des graphiques
\usepackage{enumitem,cprotect}		% personnalise les listes d'items (nécessaire pour ol, al ...)
\usepackage{hyperref}		% Liens hypertexte
\usepackage{pstricks,pst-all,pst-node,pstricks-add,pst-math,pst-plot,pst-tree,pst-eucl} % pstricks
\usepackage[a4paper,includeheadfoot,top=2cm,left=3cm, bottom=2cm,right=3cm]{geometry} % marges etc.
\usepackage{comment}			% commentaires multilignes
\usepackage{amsmath,environ} % maths (matrices, etc.)
\usepackage{amssymb,makeidx}
\usepackage{bm}				% bold maths
\usepackage{tabularx}		% tableaux
\usepackage{colortbl}		% tableaux en couleur
\usepackage{fontawesome}		% Fontawesome
\usepackage{environ}			% environment with command
\usepackage{fp}				% calculs pour ps-tricks
\usepackage{multido}			% pour ps tricks
\usepackage[np]{numprint}	% formattage nombre
\usepackage{tikz,tkz-tab} 			% package principal TikZ
\usepackage{pgfplots}   % axes
\usepackage{mathrsfs}    % cursives
\usepackage{calc}			% calcul taille boites
\usepackage[scaled=0.875]{helvet} % font sans serif
\usepackage{svg} % svg
\usepackage{scrextend} % local margin
\usepackage{scratch} %scratch
\usepackage{multicol} % colonnes
%\usepackage{infix-RPN,pst-func} % formule en notation polanaise inversée
\usepackage{listings}

%================================================================================================================================
%
% Réglages de base
%
%================================================================================================================================

\lstset{
language=Python,   % R code
literate=
{á}{{\'a}}1
{à}{{\`a}}1
{ã}{{\~a}}1
{é}{{\'e}}1
{è}{{\`e}}1
{ê}{{\^e}}1
{í}{{\'i}}1
{ó}{{\'o}}1
{õ}{{\~o}}1
{ú}{{\'u}}1
{ü}{{\"u}}1
{ç}{{\c{c}}}1
{~}{{ }}1
}


\definecolor{codegreen}{rgb}{0,0.6,0}
\definecolor{codegray}{rgb}{0.5,0.5,0.5}
\definecolor{codepurple}{rgb}{0.58,0,0.82}
\definecolor{backcolour}{rgb}{0.95,0.95,0.92}

\lstdefinestyle{mystyle}{
    backgroundcolor=\color{backcolour},   
    commentstyle=\color{codegreen},
    keywordstyle=\color{magenta},
    numberstyle=\tiny\color{codegray},
    stringstyle=\color{codepurple},
    basicstyle=\ttfamily\footnotesize,
    breakatwhitespace=false,         
    breaklines=true,                 
    captionpos=b,                    
    keepspaces=true,                 
    numbers=left,                    
xleftmargin=2em,
framexleftmargin=2em,            
    showspaces=false,                
    showstringspaces=false,
    showtabs=false,                  
    tabsize=2,
    upquote=true
}

\lstset{style=mystyle}


\lstset{style=mystyle}
\newcommand{\imgdir}{C:/laragon/www/newmc/assets/imgsvg/}
\newcommand{\imgsvgdir}{C:/laragon/www/newmc/assets/imgsvg/}

\definecolor{mcgris}{RGB}{220, 220, 220}% ancien~; pour compatibilité
\definecolor{mcbleu}{RGB}{52, 152, 219}
\definecolor{mcvert}{RGB}{125, 194, 70}
\definecolor{mcmauve}{RGB}{154, 0, 215}
\definecolor{mcorange}{RGB}{255, 96, 0}
\definecolor{mcturquoise}{RGB}{0, 153, 153}
\definecolor{mcrouge}{RGB}{255, 0, 0}
\definecolor{mclightvert}{RGB}{205, 234, 190}

\definecolor{gris}{RGB}{220, 220, 220}
\definecolor{bleu}{RGB}{52, 152, 219}
\definecolor{vert}{RGB}{125, 194, 70}
\definecolor{mauve}{RGB}{154, 0, 215}
\definecolor{orange}{RGB}{255, 96, 0}
\definecolor{turquoise}{RGB}{0, 153, 153}
\definecolor{rouge}{RGB}{255, 0, 0}
\definecolor{lightvert}{RGB}{205, 234, 190}
\setitemize[0]{label=\color{lightvert}  $\bullet$}

\pagestyle{fancy}
\renewcommand{\headrulewidth}{0.2pt}
\fancyhead[L]{maths-cours.fr}
\fancyhead[R]{\thepage}
\renewcommand{\footrulewidth}{0.2pt}
\fancyfoot[C]{}

\newcolumntype{C}{>{\centering\arraybackslash}X}
\newcolumntype{s}{>{\hsize=.35\hsize\arraybackslash}X}

\setlength{\parindent}{0pt}		 
\setlength{\parskip}{3mm}
\setlength{\headheight}{1cm}

\def\ebook{ebook}
\def\book{book}
\def\web{web}
\def\type{web}

\newcommand{\vect}[1]{\overrightarrow{\,\mathstrut#1\,}}

\def\Oij{$\left(\text{O}~;~\vect{\imath},~\vect{\jmath}\right)$}
\def\Oijk{$\left(\text{O}~;~\vect{\imath},~\vect{\jmath},~\vect{k}\right)$}
\def\Ouv{$\left(\text{O}~;~\vect{u},~\vect{v}\right)$}

\hypersetup{breaklinks=true, colorlinks = true, linkcolor = OliveGreen, urlcolor = OliveGreen, citecolor = OliveGreen, pdfauthor={Didier BONNEL - https://www.maths-cours.fr} } % supprime les bordures autour des liens

\renewcommand{\arg}[0]{\text{arg}}

\everymath{\displaystyle}

%================================================================================================================================
%
% Macros - Commandes
%
%================================================================================================================================

\newcommand\meta[2]{    			% Utilisé pour créer le post HTML.
	\def\titre{titre}
	\def\url{url}
	\def\arg{#1}
	\ifx\titre\arg
		\newcommand\maintitle{#2}
		\fancyhead[L]{#2}
		{\Large\sffamily \MakeUppercase{#2}}
		\vspace{1mm}\textcolor{mcvert}{\hrule}
	\fi 
	\ifx\url\arg
		\fancyfoot[L]{\href{https://www.maths-cours.fr#2}{\black \footnotesize{https://www.maths-cours.fr#2}}}
	\fi 
}


\newcommand\TitreC[1]{    		% Titre centré
     \needspace{3\baselineskip}
     \begin{center}\textbf{#1}\end{center}
}

\newcommand\newpar{    		% paragraphe
     \par
}

\newcommand\nosp {    		% commande vide (pas d'espace)
}
\newcommand{\id}[1]{} %ignore

\newcommand\boite[2]{				% Boite simple sans titre
	\vspace{5mm}
	\setlength{\fboxrule}{0.2mm}
	\setlength{\fboxsep}{5mm}	
	\fcolorbox{#1}{#1!3}{\makebox[\linewidth-2\fboxrule-2\fboxsep]{
  		\begin{minipage}[t]{\linewidth-2\fboxrule-4\fboxsep}\setlength{\parskip}{3mm}
  			 #2
  		\end{minipage}
	}}
	\vspace{5mm}
}

\newcommand\CBox[4]{				% Boites
	\vspace{5mm}
	\setlength{\fboxrule}{0.2mm}
	\setlength{\fboxsep}{5mm}
	
	\fcolorbox{#1}{#1!3}{\makebox[\linewidth-2\fboxrule-2\fboxsep]{
		\begin{minipage}[t]{1cm}\setlength{\parskip}{3mm}
	  		\textcolor{#1}{\LARGE{#2}}    
 	 	\end{minipage}  
  		\begin{minipage}[t]{\linewidth-2\fboxrule-4\fboxsep}\setlength{\parskip}{3mm}
			\raisebox{1.2mm}{\normalsize\sffamily{\textcolor{#1}{#3}}}						
  			 #4
  		\end{minipage}
	}}
	\vspace{5mm}
}

\newcommand\cadre[3]{				% Boites convertible html
	\par
	\vspace{2mm}
	\setlength{\fboxrule}{0.1mm}
	\setlength{\fboxsep}{5mm}
	\fcolorbox{#1}{white}{\makebox[\linewidth-2\fboxrule-2\fboxsep]{
  		\begin{minipage}[t]{\linewidth-2\fboxrule-4\fboxsep}\setlength{\parskip}{3mm}
			\raisebox{-2.5mm}{\sffamily \small{\textcolor{#1}{\MakeUppercase{#2}}}}		
			\par		
  			 #3
 	 		\end{minipage}
	}}
		\vspace{2mm}
	\par
}

\newcommand\bloc[3]{				% Boites convertible html sans bordure
     \needspace{2\baselineskip}
     {\sffamily \small{\textcolor{#1}{\MakeUppercase{#2}}}}    
		\par		
  			 #3
		\par
}

\newcommand\CHelp[1]{
     \CBox{Plum}{\faInfoCircle}{À RETENIR}{#1}
}

\newcommand\CUp[1]{
     \CBox{NavyBlue}{\faThumbsOUp}{EN PRATIQUE}{#1}
}

\newcommand\CInfo[1]{
     \CBox{Sepia}{\faArrowCircleRight}{REMARQUE}{#1}
}

\newcommand\CRedac[1]{
     \CBox{PineGreen}{\faEdit}{BIEN R\'EDIGER}{#1}
}

\newcommand\CError[1]{
     \CBox{Red}{\faExclamationTriangle}{ATTENTION}{#1}
}

\newcommand\TitreExo[2]{
\needspace{4\baselineskip}
 {\sffamily\large EXERCICE #1\ (\emph{#2 points})}
\vspace{5mm}
}

\newcommand\img[2]{
          \includegraphics[width=#2\paperwidth]{\imgdir#1}
}

\newcommand\imgsvg[2]{
       \begin{center}   \includegraphics[width=#2\paperwidth]{\imgsvgdir#1} \end{center}
}


\newcommand\Lien[2]{
     \href{#1}{#2 \tiny \faExternalLink}
}
\newcommand\mcLien[2]{
     \href{https~://www.maths-cours.fr/#1}{#2 \tiny \faExternalLink}
}

\newcommand{\euro}{\eurologo{}}

%================================================================================================================================
%
% Macros - Environement
%
%================================================================================================================================

\newenvironment{tex}{ %
}
{%
}

\newenvironment{indente}{ %
	\setlength\parindent{10mm}
}

{
	\setlength\parindent{0mm}
}

\newenvironment{corrige}{%
     \needspace{3\baselineskip}
     \medskip
     \textbf{\textsc{Corrigé}}
     \medskip
}
{
}

\newenvironment{extern}{%
     \begin{center}
     }
     {
     \end{center}
}

\NewEnviron{code}{%
	\par
     \boite{gray}{\texttt{%
     \BODY
     }}
     \par
}

\newenvironment{vbloc}{% boite sans cadre empeche saut de page
     \begin{minipage}[t]{\linewidth}
     }
     {
     \end{minipage}
}
\NewEnviron{h2}{%
    \needspace{3\baselineskip}
    \vspace{0.6cm}
	\noindent \MakeUppercase{\sffamily \large \BODY}
	\vspace{1mm}\textcolor{mcgris}{\hrule}\vspace{0.4cm}
	\par
}{}

\NewEnviron{h3}{%
    \needspace{3\baselineskip}
	\vspace{5mm}
	\textsc{\BODY}
	\par
}

\NewEnviron{margeneg}{ %
\begin{addmargin}[-1cm]{0cm}
\BODY
\end{addmargin}
}

\NewEnviron{html}{%
}

\begin{document}
\meta{url}{/exercices/fonctions-bac-es-l-antilles-guyane-2018/}
\meta{pid}{9138}
\meta{titre}{Fonctions - Bac ES/L Antilles-Guyane 2018}
\meta{type}{exercices}
%
\begin{h2}Exercice 4 (6 points)\end{h2}
\textbf{Commun à tous les candidats}
\medbreak
\begin{center}
     \begin{extern}%width="350" alt="courbe représentative de f"
          \resizebox{10cm}{!}{
               \begin{pspicture}(-0.8,-2.3)(5.8,3.2)
                    \psaxes[linewidth=0.75pt,labelFontSize=\scriptstyle]{->}(0,0)(-0.8,-2.3)(5.8,3.2)
                    \psplot[linecolor=blue,plotpoints=3000,linewidth=0.75pt]{0}{5}{2 x mul x dup mul add 2.71828 x exp div}
                    \uput[u](3,0.9){$\blue \mathcal{C}_f$}
               \end{pspicture}
          }
     \end{extern}
     \par
     \textit{Courbe} $\mathscr{C}_f$
     \bigbreak
     \begin{extern}%width="350" alt="courbe représentative de la dérivée de f"
          \resizebox{10cm}{!}{
               \begin{pspicture}(-0.8,-2.3)(5.8,3.2)
                    \psaxes[linewidth=0.75pt,labelFontSize=\scriptstyle]{->}(0,0)(-0.8,-2.3)(5.8,3.2)
                    \psplot[linecolor=blue,plotpoints=3000,linewidth=0.75pt]{0}{5}{2 x dup mul sub 2.71828 x exp div}
                    \uput[d](3.5,-0.4){$\blue \mathcal{C}_{f'}$}
               \end{pspicture}
          }
     \end{extern}
     \par
     \textit{Courbe} $\mathscr{C}_{f'}$
     \bigbreak
     \begin{extern}%width="350" alt="courbe représentative de la dérivée seconde de f"
          \resizebox{10cm}{!}{
               \begin{pspicture}(-0.8,-2.3)(5.8,3.2)
                    \psaxes[linewidth=0.75pt,labelFontSize=\scriptstyle]{->}(0,0)(-0.8,-2.3)(5.8,3.2)
                    \psplot[linecolor=blue,plotpoints=3000,linewidth=0.75pt]{0}{5}{x dup mul 2 x mul  sub 2 sub  2.71828 x exp div}
                    \uput[u](3.5,0.4){$\blue \mathcal{C}_{f''}$}
               \end{pspicture}
          }
     \end{extern}
     \par
     \textit{Courbe} $\mathscr{C}_{f''}$
     \bigbreak
\end{center}
\medbreak
On donne ci-dessus la courbe $\mathscr{C}_f$ représentative dans un repère donné d'une fonction $f$ définie
et dérivable sur l'intervalle [0~;~5] ainsi que les courbes représentatives $\mathscr{C}_{f'}$ et $\mathscr{C}_{f''}$ respectivement
de la dérivée $f'$ et de la dérivée seconde $f''$ de la fonction $f$.
\medbreak
\TitreC{Partie A}
\medbreak
Dans cette partie les réponses seront obtenues à l'aide de lectures graphiques.
\medbreak
\begin{enumerate}
     \item Donner un encadrement par deux entiers consécutifs du nombre réel pour lequel la
     fonction $f$ semble atteindre son maximum.
     \item
     \begin{enumerate}[label=\alph*.]
          \item Donner un intervalle défini par deux entiers sur lequel la fonction $f$ semble convexe.
          \item Expliquer pourquoi on peut conjecturer que la courbe $\mathscr{C}_f$ admet un point d'inflexion.
          \par
          Donner un encadrement par deux entiers consécutifs de l'abscisse de ce point d'inflexion.
     \end{enumerate}
     \item Parmi les équations suivantes quelle est l'équation de la tangente à la courbe $\mathscr{C}_f$ au point d'abscisse $0$~?
     \begin{center}
          \begin{tabularx}{\linewidth}{XX}%class="noborder"
               \textbf{a.~~}$y=x$&\textbf{b.~~} $y = 2x+ 1$\\
               \textbf{c.~~} $y= 2x$&\textbf{d.~~} $y= \dfrac{3}{4}x$\\
          \end{tabularx}
     \end{center}
     \item On note $I = \displaystyle\int_0^1 f'(x)\:\text{d}x$ où $f'$ est la fonction dérivée de $f$.
     \par
     Comment s'interprète graphiquement ce nombre $I$~?
\end{enumerate}
\bigbreak
\TitreC{Partie B}
\medbreak
La fonction $f$ représentée ci-dessus est définie sur l'intervalle [0~;~5] par $f(x) = \left(x^2 + 2x\right)\text{e}^{-x}$.
\medbreak
\begin{enumerate}
     \item
     \begin{enumerate}[label=\alph*.]
          \item Montrer que la dérivée $f'$ de $f$ est définie par $f'(x) = \left(- x^2 + 2\right)\text{e}^{-x}$ pour tout réel $x$ de l'intervalle [0~;~5].
          \item Déterminer les variations de $f$ sur [0~;~5] et préciser l'abscisse de son maximum.
          \item Donner la valeur arrondie au millième du maximum de $f$.
     \end{enumerate}
     \item Avec un outil de calcul on obtient, pour $\displaystyle\int_0^1 f'(x)\:\text{d}x$ et $f(1)$, la même valeur approchée 1,10364.
     \par
     Ces deux valeurs sont-elles égales~?
\end{enumerate}

\end{document}
µ
\documentclass[a4paper]{article}

%================================================================================================================================
%
% Packages
%
%================================================================================================================================

\usepackage[T1]{fontenc} 	% pour caractères accentués
\usepackage[utf8]{inputenc}  % encodage utf8
\usepackage[french]{babel}	% langue : français
\usepackage{fourier}			% caractères plus lisibles
\usepackage[dvipsnames]{xcolor} % couleurs
\usepackage{fancyhdr}		% réglage header footer
\usepackage{needspace}		% empêcher sauts de page mal placés
\usepackage{graphicx}		% pour inclure des graphiques
\usepackage{enumitem,cprotect}		% personnalise les listes d'items (nécessaire pour ol, al ...)
\usepackage{hyperref}		% Liens hypertexte
\usepackage{pstricks,pst-all,pst-node,pstricks-add,pst-math,pst-plot,pst-tree,pst-eucl} % pstricks
\usepackage[a4paper,includeheadfoot,top=2cm,left=3cm, bottom=2cm,right=3cm]{geometry} % marges etc.
\usepackage{comment}			% commentaires multilignes
\usepackage{amsmath,environ} % maths (matrices, etc.)
\usepackage{amssymb,makeidx}
\usepackage{bm}				% bold maths
\usepackage{tabularx}		% tableaux
\usepackage{colortbl}		% tableaux en couleur
\usepackage{fontawesome}		% Fontawesome
\usepackage{environ}			% environment with command
\usepackage{fp}				% calculs pour ps-tricks
\usepackage{multido}			% pour ps tricks
\usepackage[np]{numprint}	% formattage nombre
\usepackage{tikz,tkz-tab} 			% package principal TikZ
\usepackage{pgfplots}   % axes
\usepackage{mathrsfs}    % cursives
\usepackage{calc}			% calcul taille boites
\usepackage[scaled=0.875]{helvet} % font sans serif
\usepackage{svg} % svg
\usepackage{scrextend} % local margin
\usepackage{scratch} %scratch
\usepackage{multicol} % colonnes
%\usepackage{infix-RPN,pst-func} % formule en notation polanaise inversée
\usepackage{listings}

%================================================================================================================================
%
% Réglages de base
%
%================================================================================================================================

\lstset{
language=Python,   % R code
literate=
{á}{{\'a}}1
{à}{{\`a}}1
{ã}{{\~a}}1
{é}{{\'e}}1
{è}{{\`e}}1
{ê}{{\^e}}1
{í}{{\'i}}1
{ó}{{\'o}}1
{õ}{{\~o}}1
{ú}{{\'u}}1
{ü}{{\"u}}1
{ç}{{\c{c}}}1
{~}{{ }}1
}


\definecolor{codegreen}{rgb}{0,0.6,0}
\definecolor{codegray}{rgb}{0.5,0.5,0.5}
\definecolor{codepurple}{rgb}{0.58,0,0.82}
\definecolor{backcolour}{rgb}{0.95,0.95,0.92}

\lstdefinestyle{mystyle}{
    backgroundcolor=\color{backcolour},   
    commentstyle=\color{codegreen},
    keywordstyle=\color{magenta},
    numberstyle=\tiny\color{codegray},
    stringstyle=\color{codepurple},
    basicstyle=\ttfamily\footnotesize,
    breakatwhitespace=false,         
    breaklines=true,                 
    captionpos=b,                    
    keepspaces=true,                 
    numbers=left,                    
xleftmargin=2em,
framexleftmargin=2em,            
    showspaces=false,                
    showstringspaces=false,
    showtabs=false,                  
    tabsize=2,
    upquote=true
}

\lstset{style=mystyle}


\lstset{style=mystyle}
\newcommand{\imgdir}{C:/laragon/www/newmc/assets/imgsvg/}
\newcommand{\imgsvgdir}{C:/laragon/www/newmc/assets/imgsvg/}

\definecolor{mcgris}{RGB}{220, 220, 220}% ancien~; pour compatibilité
\definecolor{mcbleu}{RGB}{52, 152, 219}
\definecolor{mcvert}{RGB}{125, 194, 70}
\definecolor{mcmauve}{RGB}{154, 0, 215}
\definecolor{mcorange}{RGB}{255, 96, 0}
\definecolor{mcturquoise}{RGB}{0, 153, 153}
\definecolor{mcrouge}{RGB}{255, 0, 0}
\definecolor{mclightvert}{RGB}{205, 234, 190}

\definecolor{gris}{RGB}{220, 220, 220}
\definecolor{bleu}{RGB}{52, 152, 219}
\definecolor{vert}{RGB}{125, 194, 70}
\definecolor{mauve}{RGB}{154, 0, 215}
\definecolor{orange}{RGB}{255, 96, 0}
\definecolor{turquoise}{RGB}{0, 153, 153}
\definecolor{rouge}{RGB}{255, 0, 0}
\definecolor{lightvert}{RGB}{205, 234, 190}
\setitemize[0]{label=\color{lightvert}  $\bullet$}

\pagestyle{fancy}
\renewcommand{\headrulewidth}{0.2pt}
\fancyhead[L]{maths-cours.fr}
\fancyhead[R]{\thepage}
\renewcommand{\footrulewidth}{0.2pt}
\fancyfoot[C]{}

\newcolumntype{C}{>{\centering\arraybackslash}X}
\newcolumntype{s}{>{\hsize=.35\hsize\arraybackslash}X}

\setlength{\parindent}{0pt}		 
\setlength{\parskip}{3mm}
\setlength{\headheight}{1cm}

\def\ebook{ebook}
\def\book{book}
\def\web{web}
\def\type{web}

\newcommand{\vect}[1]{\overrightarrow{\,\mathstrut#1\,}}

\def\Oij{$\left(\text{O}~;~\vect{\imath},~\vect{\jmath}\right)$}
\def\Oijk{$\left(\text{O}~;~\vect{\imath},~\vect{\jmath},~\vect{k}\right)$}
\def\Ouv{$\left(\text{O}~;~\vect{u},~\vect{v}\right)$}

\hypersetup{breaklinks=true, colorlinks = true, linkcolor = OliveGreen, urlcolor = OliveGreen, citecolor = OliveGreen, pdfauthor={Didier BONNEL - https://www.maths-cours.fr} } % supprime les bordures autour des liens

\renewcommand{\arg}[0]{\text{arg}}

\everymath{\displaystyle}

%================================================================================================================================
%
% Macros - Commandes
%
%================================================================================================================================

\newcommand\meta[2]{    			% Utilisé pour créer le post HTML.
	\def\titre{titre}
	\def\url{url}
	\def\arg{#1}
	\ifx\titre\arg
		\newcommand\maintitle{#2}
		\fancyhead[L]{#2}
		{\Large\sffamily \MakeUppercase{#2}}
		\vspace{1mm}\textcolor{mcvert}{\hrule}
	\fi 
	\ifx\url\arg
		\fancyfoot[L]{\href{https://www.maths-cours.fr#2}{\black \footnotesize{https://www.maths-cours.fr#2}}}
	\fi 
}


\newcommand\TitreC[1]{    		% Titre centré
     \needspace{3\baselineskip}
     \begin{center}\textbf{#1}\end{center}
}

\newcommand\newpar{    		% paragraphe
     \par
}

\newcommand\nosp {    		% commande vide (pas d'espace)
}
\newcommand{\id}[1]{} %ignore

\newcommand\boite[2]{				% Boite simple sans titre
	\vspace{5mm}
	\setlength{\fboxrule}{0.2mm}
	\setlength{\fboxsep}{5mm}	
	\fcolorbox{#1}{#1!3}{\makebox[\linewidth-2\fboxrule-2\fboxsep]{
  		\begin{minipage}[t]{\linewidth-2\fboxrule-4\fboxsep}\setlength{\parskip}{3mm}
  			 #2
  		\end{minipage}
	}}
	\vspace{5mm}
}

\newcommand\CBox[4]{				% Boites
	\vspace{5mm}
	\setlength{\fboxrule}{0.2mm}
	\setlength{\fboxsep}{5mm}
	
	\fcolorbox{#1}{#1!3}{\makebox[\linewidth-2\fboxrule-2\fboxsep]{
		\begin{minipage}[t]{1cm}\setlength{\parskip}{3mm}
	  		\textcolor{#1}{\LARGE{#2}}    
 	 	\end{minipage}  
  		\begin{minipage}[t]{\linewidth-2\fboxrule-4\fboxsep}\setlength{\parskip}{3mm}
			\raisebox{1.2mm}{\normalsize\sffamily{\textcolor{#1}{#3}}}						
  			 #4
  		\end{minipage}
	}}
	\vspace{5mm}
}

\newcommand\cadre[3]{				% Boites convertible html
	\par
	\vspace{2mm}
	\setlength{\fboxrule}{0.1mm}
	\setlength{\fboxsep}{5mm}
	\fcolorbox{#1}{white}{\makebox[\linewidth-2\fboxrule-2\fboxsep]{
  		\begin{minipage}[t]{\linewidth-2\fboxrule-4\fboxsep}\setlength{\parskip}{3mm}
			\raisebox{-2.5mm}{\sffamily \small{\textcolor{#1}{\MakeUppercase{#2}}}}		
			\par		
  			 #3
 	 		\end{minipage}
	}}
		\vspace{2mm}
	\par
}

\newcommand\bloc[3]{				% Boites convertible html sans bordure
     \needspace{2\baselineskip}
     {\sffamily \small{\textcolor{#1}{\MakeUppercase{#2}}}}    
		\par		
  			 #3
		\par
}

\newcommand\CHelp[1]{
     \CBox{Plum}{\faInfoCircle}{À RETENIR}{#1}
}

\newcommand\CUp[1]{
     \CBox{NavyBlue}{\faThumbsOUp}{EN PRATIQUE}{#1}
}

\newcommand\CInfo[1]{
     \CBox{Sepia}{\faArrowCircleRight}{REMARQUE}{#1}
}

\newcommand\CRedac[1]{
     \CBox{PineGreen}{\faEdit}{BIEN R\'EDIGER}{#1}
}

\newcommand\CError[1]{
     \CBox{Red}{\faExclamationTriangle}{ATTENTION}{#1}
}

\newcommand\TitreExo[2]{
\needspace{4\baselineskip}
 {\sffamily\large EXERCICE #1\ (\emph{#2 points})}
\vspace{5mm}
}

\newcommand\img[2]{
          \includegraphics[width=#2\paperwidth]{\imgdir#1}
}

\newcommand\imgsvg[2]{
       \begin{center}   \includegraphics[width=#2\paperwidth]{\imgsvgdir#1} \end{center}
}


\newcommand\Lien[2]{
     \href{#1}{#2 \tiny \faExternalLink}
}
\newcommand\mcLien[2]{
     \href{https~://www.maths-cours.fr/#1}{#2 \tiny \faExternalLink}
}

\newcommand{\euro}{\eurologo{}}

%================================================================================================================================
%
% Macros - Environement
%
%================================================================================================================================

\newenvironment{tex}{ %
}
{%
}

\newenvironment{indente}{ %
	\setlength\parindent{10mm}
}

{
	\setlength\parindent{0mm}
}

\newenvironment{corrige}{%
     \needspace{3\baselineskip}
     \medskip
     \textbf{\textsc{Corrigé}}
     \medskip
}
{
}

\newenvironment{extern}{%
     \begin{center}
     }
     {
     \end{center}
}

\NewEnviron{code}{%
	\par
     \boite{gray}{\texttt{%
     \BODY
     }}
     \par
}

\newenvironment{vbloc}{% boite sans cadre empeche saut de page
     \begin{minipage}[t]{\linewidth}
     }
     {
     \end{minipage}
}
\NewEnviron{h2}{%
    \needspace{3\baselineskip}
    \vspace{0.6cm}
	\noindent \MakeUppercase{\sffamily \large \BODY}
	\vspace{1mm}\textcolor{mcgris}{\hrule}\vspace{0.4cm}
	\par
}{}

\NewEnviron{h3}{%
    \needspace{3\baselineskip}
	\vspace{5mm}
	\textsc{\BODY}
	\par
}

\NewEnviron{margeneg}{ %
\begin{addmargin}[-1cm]{0cm}
\BODY
\end{addmargin}
}

\NewEnviron{html}{%
}

\begin{document}
\meta{url}{/exercices/fonctions-bac-s-metropole-2018/}
\meta{pid}{9167}
\meta{titre}{Fonctions - Bac S Métropole 2018}
\meta{type}{exercices}
%
\begin{h2}Exercice 1 (6 points)\end{h2}
\textbf{Commun à tous les candidats }
\bigbreak
\emph{Dans cet exercice, on munit le plan d'un repère orthonormé.}
\par
On a représenté ci-dessous la courbe d'équation~:
\[y = \dfrac{1}{2}\left(\text{e}^x + \text{e}^{-x} - 2\right).\]
\par
Cette courbe est appelée une \og chaînette \fg.
\par
On s'intéresse ici aux \og arcs de chaînette\fg{} délimités par deux points de cette courbe
symétriques par rapport à l'axe des ordonnées.
\par
Un tel arc est représenté sur le graphique ci-dessous en trait plein.
\par
On définit la \og largeur \fg{} et la \og hauteur \fg{} de l'arc de chaînette délimité par les points $M$ et $M'$ comme indiqué sur le graphique.
\begin{center}
     \begin{extern}%width="400" alt=""
          \psset{unit=1.5cm}
          \begin{pspicture}(-2,-0.8)(2,2)
               \psaxes[linewidth=1pt,Dx=4,Dy=4]{->}(0,0)(-2,0)(2,2)
               \psplot[plotpoints=3000,linewidth=1pt]{-1.8}{1.8}{2.71828 x exp 2.71828 x neg exp add 2 sub 0.5 mul}
               \psline[linestyle=dashed](1.4,0)(1.4,1.1509)(-1.4,1.1509)(-1.4,0)
               \uput[d](1.4,0){$x$} \uput[d](-1.4,0){$- x$}
               \psline{<->}(-1.4,-0.5)(1.4,-0.5)
               \uput[d](0,-0.5){largeur}
               \psline{<->}(-1.6,0)(-1.6,1.1509)
               \uput[l](-1.6,0.56){hauteur}
               \uput[ur](1.5,0.9){$M\left(x~;~\dfrac{1}{2}\left(\text{e}^x + \text{e}^{- x} - 2\right)\right)$}
               \uput[ur](-1.4,1.15){$M'$}
          \end{pspicture}
     \end{extern}
\end{center}
\medbreak
Le but de l'exercice est d'étudier les positions possibles sur la courbe du point $M$ d'abscisse $x$ strictement positive afin que la largeur de l'arc de chaînette soit égale à sa hauteur.
\medbreak
\begin{enumerate}
     \item Justifier que le problème étudié se ramène à la recherche des solutions strictement
     positives de l'équation
     \par
     \[(E)~: \text{e}^x + \text{e}^{- x} - 4x - 2 = 0.\]
     \item  On note $f$ la fonction définie sur l'intervalle $[0~;~+\infty[$ par~:
     \par
     \[f(x) = \text{e}^x + \text{e}^{- x} - 4x - 2.\]
     \begin{enumerate}[label=\alph*.]
          \item Vérifier que pour tout $x > 0,\: f(x) = x \left(\dfrac{\text{e}^x}{x}- 4\right) + \text{e}^{- x} - 2$.
          \item Déterminer $\displaystyle\lim_{x \to + \infty} f(x)$.
     \end{enumerate}
     \item
     \begin{enumerate}[label=\alph*.]
          \item On note $f'$ la fonction dérivée de la fonction $f$. Calculer $f'(x)$, où $x$ appartient à l'intervalle $[0~;~ +\infty[$.
          \item Montrer que l'équation $f'(x) = 0$ équivaut à l'équation~: $\left(\text{e}^x\right)^2 - 4\text{e}^x - 1 = 0$.
          \item En posant $X = \text{e}^x$, montrer que l'équation $f'(x) = 0$ admet pour unique solution réelle le nombre $\ln \left(2 + \sqrt{5}\right)$.
     \end{enumerate}
     \item  On donne ci-dessous le tableau de signes de la fonction dérivée $f'$ de $f$~:
     \begin{center}
          \begin{extern}%width="300" alt=""
               \psset{unit=1cm}
               \begin{pspicture}(7,1.5)
                    \psframe(7,1.5)\psline(0,0.75)(7,0.75)\psline(1,0)(1,1.5)
                    \uput[u](0.5,0.85){$x$}\uput[u](1.2,0.85){$0$}
                    \uput[u](4,0.7){$\ln \left(2 + \sqrt{5} \right)$}\uput[u](6.5,0.85){$+ \infty$}
                    \rput(0.5,0.375){$f'(x)$}\rput(2,0.375){$-$}
                    \rput(4,0.375){$0$}\rput(5,0.375){$+$}
               \end{pspicture}
          \end{extern}
     \end{center}
     \begin{enumerate}[label=\alph*.]
          \item Dresser le tableau de variations de la fonction $f$.
          \item Démontrer que l'équation $f(x) = 0$ admet une unique solution strictement positive que l'on notera $\alpha$.
     \end{enumerate}
     \item On considère l'algorithme suivant où les variables $a$, $b$ et $m$ sont des nombres réels~:
     \begin{center}
          \begin{extern}%width="350" alt=""
               \begin{tabularx}{0.5\linewidth}{|X|}\hline
                    Tant que $b - a > 0,1$ faire~:\\
                    \hspace{1cm}$m \gets \dfrac{a+b}{2}$\\
                    \hspace{1cm}Si $\text{e}^m + \text{e}^{-m} - 4m - 2 > 0$, alors~:\\
                    \hspace{2cm}$b \gets m$\\
                    \hspace{1cm}Sinon~:\\
                    \hspace{2cm}$a\gets m$\\
                    \hspace{1cm}Fin Si\\
                    Fin Tant que\\ \hline
               \end{tabularx}
          \end{extern}
     \end{center}
     \begin{enumerate}[label=\alph*.]
          \item Avant l'exécution de cet algorithme, les variables $a$ et $b$
          contiennent respectivement les valeurs $2$ et $3$.
          \par
          Que contiennent-elles à la fin de l'exécution de l'algorithme~?
          \par
          On justifiera la réponse en reproduisant et en complétant le tableau ci-contre avec les différentes valeurs prises par les variables, à chaque étape de l'algorithme.
          \item Comment peut-on utiliser les valeurs obtenues en fin d'algorithme à la question
          précédente~?
     \end{enumerate}
     \begin{center}
          \begin{extern}%width="550" alt=""
               \begin{tabularx}{0.8\linewidth}{|*{4}{>{\centering \arraybackslash}X|}}\hline
                    $m$ 	&$a$ &$b$ &$b - a$\\ \hline
                    \cellcolor{lightgray}	&2& 3 &1\\ \hline
                    2,5		&&&\\ \hline
                    \ldots	&\ldots&\ldots&\\ \hline
                    ~		&&&\\ \hline
               \end{tabularx}
          \end{extern}
     \end{center}
     \item La \emph{Gateway Arch}, édifiée dans la ville de Saint-Louis aux États-Unis, a l'allure ci-dessous.
     \begin{center}
          \begin{extern}%width="180" alt=""
               \psset{unit=1.3cm,arrowsize=2pt 3}
               \begin{pspicture}(-2,-1)(2,2)
                    %\psaxes[linewidth=1.25pt,Dx=2,Dy=2]{->}(0,0)(-2,0)(2,2)
                    \psplot[plotpoints=3000,linewidth=0.75pt]{-1.8}{1.8}{2.71828 x exp 2.71828 x neg exp add 2 sub 0.5 mul neg 1.5 add}
                    \psline{<->}(-1.8,-0.6)(1.8,-0.6)
                    \uput[d](0,-0.6){largeur}
                    \psline{<->}(0,-0.6)(0,1.5)
                    \uput[r](0,0.45){hauteur}
               \end{pspicture}
          \end{extern}
     \end{center}
     Son profil peut être approché par un arc de chaînette renversé dont la largeur est égale à  la hauteur.
     \par
     La largeur de cet arc, exprimée en mètre, est égale au double de la solution strictement
     positive de l'équation~:
     \par
     \[\left(E'\right)~: \text{e}^{\tfrac{t}{39}} + \text{e}^{-\tfrac{t}{39}} - 4\frac{t}{39} - 2 = 0.\]
     \par
     Donner un encadrement de la hauteur de la \emph{Gateway Arch}.
\end{enumerate}

\end{document}
µ
\documentclass[a4paper]{article}

%================================================================================================================================
%
% Packages
%
%================================================================================================================================

\usepackage[T1]{fontenc} 	% pour caractères accentués
\usepackage[utf8]{inputenc}  % encodage utf8
\usepackage[french]{babel}	% langue : français
\usepackage{fourier}			% caractères plus lisibles
\usepackage[dvipsnames]{xcolor} % couleurs
\usepackage{fancyhdr}		% réglage header footer
\usepackage{needspace}		% empêcher sauts de page mal placés
\usepackage{graphicx}		% pour inclure des graphiques
\usepackage{enumitem,cprotect}		% personnalise les listes d'items (nécessaire pour ol, al ...)
\usepackage{hyperref}		% Liens hypertexte
\usepackage{pstricks,pst-all,pst-node,pstricks-add,pst-math,pst-plot,pst-tree,pst-eucl} % pstricks
\usepackage[a4paper,includeheadfoot,top=2cm,left=3cm, bottom=2cm,right=3cm]{geometry} % marges etc.
\usepackage{comment}			% commentaires multilignes
\usepackage{amsmath,environ} % maths (matrices, etc.)
\usepackage{amssymb,makeidx}
\usepackage{bm}				% bold maths
\usepackage{tabularx}		% tableaux
\usepackage{colortbl}		% tableaux en couleur
\usepackage{fontawesome}		% Fontawesome
\usepackage{environ}			% environment with command
\usepackage{fp}				% calculs pour ps-tricks
\usepackage{multido}			% pour ps tricks
\usepackage[np]{numprint}	% formattage nombre
\usepackage{tikz,tkz-tab} 			% package principal TikZ
\usepackage{pgfplots}   % axes
\usepackage{mathrsfs}    % cursives
\usepackage{calc}			% calcul taille boites
\usepackage[scaled=0.875]{helvet} % font sans serif
\usepackage{svg} % svg
\usepackage{scrextend} % local margin
\usepackage{scratch} %scratch
\usepackage{multicol} % colonnes
%\usepackage{infix-RPN,pst-func} % formule en notation polanaise inversée
\usepackage{listings}

%================================================================================================================================
%
% Réglages de base
%
%================================================================================================================================

\lstset{
language=Python,   % R code
literate=
{á}{{\'a}}1
{à}{{\`a}}1
{ã}{{\~a}}1
{é}{{\'e}}1
{è}{{\`e}}1
{ê}{{\^e}}1
{í}{{\'i}}1
{ó}{{\'o}}1
{õ}{{\~o}}1
{ú}{{\'u}}1
{ü}{{\"u}}1
{ç}{{\c{c}}}1
{~}{{ }}1
}


\definecolor{codegreen}{rgb}{0,0.6,0}
\definecolor{codegray}{rgb}{0.5,0.5,0.5}
\definecolor{codepurple}{rgb}{0.58,0,0.82}
\definecolor{backcolour}{rgb}{0.95,0.95,0.92}

\lstdefinestyle{mystyle}{
    backgroundcolor=\color{backcolour},   
    commentstyle=\color{codegreen},
    keywordstyle=\color{magenta},
    numberstyle=\tiny\color{codegray},
    stringstyle=\color{codepurple},
    basicstyle=\ttfamily\footnotesize,
    breakatwhitespace=false,         
    breaklines=true,                 
    captionpos=b,                    
    keepspaces=true,                 
    numbers=left,                    
xleftmargin=2em,
framexleftmargin=2em,            
    showspaces=false,                
    showstringspaces=false,
    showtabs=false,                  
    tabsize=2,
    upquote=true
}

\lstset{style=mystyle}


\lstset{style=mystyle}
\newcommand{\imgdir}{C:/laragon/www/newmc/assets/imgsvg/}
\newcommand{\imgsvgdir}{C:/laragon/www/newmc/assets/imgsvg/}

\definecolor{mcgris}{RGB}{220, 220, 220}% ancien~; pour compatibilité
\definecolor{mcbleu}{RGB}{52, 152, 219}
\definecolor{mcvert}{RGB}{125, 194, 70}
\definecolor{mcmauve}{RGB}{154, 0, 215}
\definecolor{mcorange}{RGB}{255, 96, 0}
\definecolor{mcturquoise}{RGB}{0, 153, 153}
\definecolor{mcrouge}{RGB}{255, 0, 0}
\definecolor{mclightvert}{RGB}{205, 234, 190}

\definecolor{gris}{RGB}{220, 220, 220}
\definecolor{bleu}{RGB}{52, 152, 219}
\definecolor{vert}{RGB}{125, 194, 70}
\definecolor{mauve}{RGB}{154, 0, 215}
\definecolor{orange}{RGB}{255, 96, 0}
\definecolor{turquoise}{RGB}{0, 153, 153}
\definecolor{rouge}{RGB}{255, 0, 0}
\definecolor{lightvert}{RGB}{205, 234, 190}
\setitemize[0]{label=\color{lightvert}  $\bullet$}

\pagestyle{fancy}
\renewcommand{\headrulewidth}{0.2pt}
\fancyhead[L]{maths-cours.fr}
\fancyhead[R]{\thepage}
\renewcommand{\footrulewidth}{0.2pt}
\fancyfoot[C]{}

\newcolumntype{C}{>{\centering\arraybackslash}X}
\newcolumntype{s}{>{\hsize=.35\hsize\arraybackslash}X}

\setlength{\parindent}{0pt}		 
\setlength{\parskip}{3mm}
\setlength{\headheight}{1cm}

\def\ebook{ebook}
\def\book{book}
\def\web{web}
\def\type{web}

\newcommand{\vect}[1]{\overrightarrow{\,\mathstrut#1\,}}

\def\Oij{$\left(\text{O}~;~\vect{\imath},~\vect{\jmath}\right)$}
\def\Oijk{$\left(\text{O}~;~\vect{\imath},~\vect{\jmath},~\vect{k}\right)$}
\def\Ouv{$\left(\text{O}~;~\vect{u},~\vect{v}\right)$}

\hypersetup{breaklinks=true, colorlinks = true, linkcolor = OliveGreen, urlcolor = OliveGreen, citecolor = OliveGreen, pdfauthor={Didier BONNEL - https://www.maths-cours.fr} } % supprime les bordures autour des liens

\renewcommand{\arg}[0]{\text{arg}}

\everymath{\displaystyle}

%================================================================================================================================
%
% Macros - Commandes
%
%================================================================================================================================

\newcommand\meta[2]{    			% Utilisé pour créer le post HTML.
	\def\titre{titre}
	\def\url{url}
	\def\arg{#1}
	\ifx\titre\arg
		\newcommand\maintitle{#2}
		\fancyhead[L]{#2}
		{\Large\sffamily \MakeUppercase{#2}}
		\vspace{1mm}\textcolor{mcvert}{\hrule}
	\fi 
	\ifx\url\arg
		\fancyfoot[L]{\href{https://www.maths-cours.fr#2}{\black \footnotesize{https://www.maths-cours.fr#2}}}
	\fi 
}


\newcommand\TitreC[1]{    		% Titre centré
     \needspace{3\baselineskip}
     \begin{center}\textbf{#1}\end{center}
}

\newcommand\newpar{    		% paragraphe
     \par
}

\newcommand\nosp {    		% commande vide (pas d'espace)
}
\newcommand{\id}[1]{} %ignore

\newcommand\boite[2]{				% Boite simple sans titre
	\vspace{5mm}
	\setlength{\fboxrule}{0.2mm}
	\setlength{\fboxsep}{5mm}	
	\fcolorbox{#1}{#1!3}{\makebox[\linewidth-2\fboxrule-2\fboxsep]{
  		\begin{minipage}[t]{\linewidth-2\fboxrule-4\fboxsep}\setlength{\parskip}{3mm}
  			 #2
  		\end{minipage}
	}}
	\vspace{5mm}
}

\newcommand\CBox[4]{				% Boites
	\vspace{5mm}
	\setlength{\fboxrule}{0.2mm}
	\setlength{\fboxsep}{5mm}
	
	\fcolorbox{#1}{#1!3}{\makebox[\linewidth-2\fboxrule-2\fboxsep]{
		\begin{minipage}[t]{1cm}\setlength{\parskip}{3mm}
	  		\textcolor{#1}{\LARGE{#2}}    
 	 	\end{minipage}  
  		\begin{minipage}[t]{\linewidth-2\fboxrule-4\fboxsep}\setlength{\parskip}{3mm}
			\raisebox{1.2mm}{\normalsize\sffamily{\textcolor{#1}{#3}}}						
  			 #4
  		\end{minipage}
	}}
	\vspace{5mm}
}

\newcommand\cadre[3]{				% Boites convertible html
	\par
	\vspace{2mm}
	\setlength{\fboxrule}{0.1mm}
	\setlength{\fboxsep}{5mm}
	\fcolorbox{#1}{white}{\makebox[\linewidth-2\fboxrule-2\fboxsep]{
  		\begin{minipage}[t]{\linewidth-2\fboxrule-4\fboxsep}\setlength{\parskip}{3mm}
			\raisebox{-2.5mm}{\sffamily \small{\textcolor{#1}{\MakeUppercase{#2}}}}		
			\par		
  			 #3
 	 		\end{minipage}
	}}
		\vspace{2mm}
	\par
}

\newcommand\bloc[3]{				% Boites convertible html sans bordure
     \needspace{2\baselineskip}
     {\sffamily \small{\textcolor{#1}{\MakeUppercase{#2}}}}    
		\par		
  			 #3
		\par
}

\newcommand\CHelp[1]{
     \CBox{Plum}{\faInfoCircle}{À RETENIR}{#1}
}

\newcommand\CUp[1]{
     \CBox{NavyBlue}{\faThumbsOUp}{EN PRATIQUE}{#1}
}

\newcommand\CInfo[1]{
     \CBox{Sepia}{\faArrowCircleRight}{REMARQUE}{#1}
}

\newcommand\CRedac[1]{
     \CBox{PineGreen}{\faEdit}{BIEN R\'EDIGER}{#1}
}

\newcommand\CError[1]{
     \CBox{Red}{\faExclamationTriangle}{ATTENTION}{#1}
}

\newcommand\TitreExo[2]{
\needspace{4\baselineskip}
 {\sffamily\large EXERCICE #1\ (\emph{#2 points})}
\vspace{5mm}
}

\newcommand\img[2]{
          \includegraphics[width=#2\paperwidth]{\imgdir#1}
}

\newcommand\imgsvg[2]{
       \begin{center}   \includegraphics[width=#2\paperwidth]{\imgsvgdir#1} \end{center}
}


\newcommand\Lien[2]{
     \href{#1}{#2 \tiny \faExternalLink}
}
\newcommand\mcLien[2]{
     \href{https~://www.maths-cours.fr/#1}{#2 \tiny \faExternalLink}
}

\newcommand{\euro}{\eurologo{}}

%================================================================================================================================
%
% Macros - Environement
%
%================================================================================================================================

\newenvironment{tex}{ %
}
{%
}

\newenvironment{indente}{ %
	\setlength\parindent{10mm}
}

{
	\setlength\parindent{0mm}
}

\newenvironment{corrige}{%
     \needspace{3\baselineskip}
     \medskip
     \textbf{\textsc{Corrigé}}
     \medskip
}
{
}

\newenvironment{extern}{%
     \begin{center}
     }
     {
     \end{center}
}

\NewEnviron{code}{%
	\par
     \boite{gray}{\texttt{%
     \BODY
     }}
     \par
}

\newenvironment{vbloc}{% boite sans cadre empeche saut de page
     \begin{minipage}[t]{\linewidth}
     }
     {
     \end{minipage}
}
\NewEnviron{h2}{%
    \needspace{3\baselineskip}
    \vspace{0.6cm}
	\noindent \MakeUppercase{\sffamily \large \BODY}
	\vspace{1mm}\textcolor{mcgris}{\hrule}\vspace{0.4cm}
	\par
}{}

\NewEnviron{h3}{%
    \needspace{3\baselineskip}
	\vspace{5mm}
	\textsc{\BODY}
	\par
}

\NewEnviron{margeneg}{ %
\begin{addmargin}[-1cm]{0cm}
\BODY
\end{addmargin}
}

\NewEnviron{html}{%
}

\begin{document}
\meta{url}{/exercices/probabilites-bac-s-metropole-2018/}
\meta{pid}{9171}
\meta{titre}{Probabilités- Bac S Métropole 2018}
\meta{type}{exercices}
%
\begin{h2}Exercice 2 (4 points)\end{h2}
\textbf{Commun à tous les candidats }
\bigbreak
\emph{Les parties A et B de cet exercice sont indépendantes.}
\medbreak
Le virus de la grippe atteint chaque année, en période hivernale, une partie de la population d'une ville.
\par
La vaccination contre la grippe est possible~; elle doit être renouvelée chaque année.
\bigbreak
\TitreC{Partie A}
\medbreak
L'efficacité du vaccin contre la grippe peut être diminuée en fonction des caractéristiques
individuelles des personnes vaccinées, ou en raison du vaccin, qui n'est pas toujours
totalement adapté aux souches du virus qui circulent. Il est donc possible de contracter la
grippe tout en étant vacciné.
\par
Une étude menée dans la population de la ville à l'issue de la période hivernale a permis de constater que~:
\begin{itemize}
     \item40\,\% de la population est vaccinée~;
     \item8\,\% des personnes vaccinées ont contracté la grippe~;
     \item20\,\% de la population a contracté la grippe.
\end{itemize}
\smallbreak
On choisit une personne au hasard dans la population de la ville et on considère les événements~:
\begin{itemize}[label=---]
     \item $V$~: \og la personne est vaccinée contre la grippe \fg{}~;
     \item $G$~: \og la personne a contracté la grippe \fg.
\end{itemize}
\medbreak
\begin{enumerate}
     \item
     \begin{enumerate}[label=\alph*.]
          \item Donner la probabilité de l'événement $G$.
          \item Reproduire l'arbre pondéré ci-dessous et compléter les pointillés indiqués sur quatre de ses branches.
          %:-+-+-+- Engendré par : http://math.et.info.free.fr/TikZ/Arbre/
          \begin{center}
               \begin{extern}%width="350"
                    % Racine à Gauche, développement vers la droite
                    \begin{tikzpicture}[xscale=1,yscale=1]
                         % Styles (MODIFIABLES)
                         \tikzstyle{fleche}=[-,>=latex,thick]
                         \tikzstyle{noeud}=[circle,draw]
                         \tikzstyle{feuille}=[circle,draw]
                         \tikzstyle{etiquette}=[midway,fill=white]
                         % Dimensions (MODIFIABLES)
                         \def\DistanceInterNiveaux{3}
                         \def\DistanceInterFeuilles{2}
                         % Dimensions calculées (NON MODIFIABLES)
                         \def\NiveauA{(0)*\DistanceInterNiveaux}
                         \def\NiveauB{(1.5)*\DistanceInterNiveaux}
                         \def\NiveauC{(2.5)*\DistanceInterNiveaux}
                         \def\InterFeuilles{(-1)*\DistanceInterFeuilles}
                         % Noeuds (MODIFIABLES : Styles et Coefficients d'InterFeuilles)
                         \node[noeud] (R) at ({\NiveauA},{(1.5)*\InterFeuilles}) {$ $};
                         \node[noeud] (Ra) at ({\NiveauB},{(0.5)*\InterFeuilles}) {$V$};
                         \node[feuille] (Raa) at ({\NiveauC},{(0)*\InterFeuilles}) {$G$};
                         \node[feuille] (Rab) at ({\NiveauC},{(1)*\InterFeuilles}) {$\overline{G}$};
                         \node[noeud] (Rb) at ({\NiveauB},{(2.5)*\InterFeuilles}) {$\overline{V}$};
                         \node[feuille] (Rba) at ({\NiveauC},{(2)*\InterFeuilles}) {$G$};
                         \node[feuille] (Rbb) at ({\NiveauC},{(3)*\InterFeuilles}) {$\overline{G}$};
                         % Arcs (MODIFIABLES : Styles)
                         \draw[fleche] (R)--(Ra) node[etiquette] {$\cdots$};
                         \draw[fleche] (Ra)--(Raa) node[etiquette] {$\cdots$};
                         \draw[fleche] (Ra)--(Rab) node[etiquette] {$\cdots$};
                         \draw[fleche] (R)--(Rb) node[etiquette] {$\cdots$};
                         \draw[fleche] (Rb)--(Rba) node[etiquette] {$\cdots$};
                         \draw[fleche] (Rb)--(Rbb) node[etiquette] {$\cdots$};
                    \end{tikzpicture}
               \end{extern}
          \end{center}
     \end{enumerate}
     \item Déterminer la probabilité que la personne choisie ait contracté la grippe et soit vaccinée.
     \item La personne choisie n'est pas vaccinée. Montrer que la probabilité qu'elle ait contracté la grippe est égale à $0,28$.
\end{enumerate}
\bigbreak
\TitreC{Partie B}
\medbreak
\emph{Dans cette partie, les probabilités demandées seront données à $10^{-3}$ près.}
\medbreak
Un laboratoire pharmaceutique mène une étude sur la vaccination contre la grippe dans cette
ville.
\medbreak
Après la période hivernale, on interroge au hasard $n$ habitants de la ville, en admettant que ce choix se ramène à $n$ tirages successifs indépendants et avec remise. On suppose que la probabilité qu'une personne choisie au hasard dans la ville soit vaccinée contre la grippe est égale à $0,4$.
\par
On note $X$ la variable aléatoire égale au nombre de personnes vaccinées parmi les $n$
interrogées.
\medbreak
\begin{enumerate}
     \item Quelle est la loi de probabilité suivie par la variable aléatoire $X$~?
     \item Dans cette question, on suppose que $n = 40$.
     \begin{enumerate}[label=\alph*.]
          \item Déterminer la probabilité qu'exactement $15$ des $40$ personnes interrogées soient vaccinées.
          \item Déterminer la probabilité qu'au moins la moitié des personnes interrogées soit vaccinée.
     \end{enumerate}
     \item  On interroge un échantillon de 3~750 habitants de la ville, c'est-à-dire que l'on suppose ici que $n = 3~750$.
     \par
     On note $Z$ la variable aléatoire définie par~: $Z = \dfrac{X - 1~500}{30}$.
     \par
     On admet que la loi de probabilité de la variable aléatoire $Z$ peut être approchée par la
     loi normale centrée réduite.
     \par
     En utilisant cette approximation, déterminer la probabilité qu'il y ait entre 1~450 et 1~550 individus vaccinés dans l'échantillon interrogé.
\end{enumerate}

\end{document}
µ
\documentclass[a4paper]{article}

%================================================================================================================================
%
% Packages
%
%================================================================================================================================

\usepackage[T1]{fontenc} 	% pour caractères accentués
\usepackage[utf8]{inputenc}  % encodage utf8
\usepackage[french]{babel}	% langue : français
\usepackage{fourier}			% caractères plus lisibles
\usepackage[dvipsnames]{xcolor} % couleurs
\usepackage{fancyhdr}		% réglage header footer
\usepackage{needspace}		% empêcher sauts de page mal placés
\usepackage{graphicx}		% pour inclure des graphiques
\usepackage{enumitem,cprotect}		% personnalise les listes d'items (nécessaire pour ol, al ...)
\usepackage{hyperref}		% Liens hypertexte
\usepackage{pstricks,pst-all,pst-node,pstricks-add,pst-math,pst-plot,pst-tree,pst-eucl} % pstricks
\usepackage[a4paper,includeheadfoot,top=2cm,left=3cm, bottom=2cm,right=3cm]{geometry} % marges etc.
\usepackage{comment}			% commentaires multilignes
\usepackage{amsmath,environ} % maths (matrices, etc.)
\usepackage{amssymb,makeidx}
\usepackage{bm}				% bold maths
\usepackage{tabularx}		% tableaux
\usepackage{colortbl}		% tableaux en couleur
\usepackage{fontawesome}		% Fontawesome
\usepackage{environ}			% environment with command
\usepackage{fp}				% calculs pour ps-tricks
\usepackage{multido}			% pour ps tricks
\usepackage[np]{numprint}	% formattage nombre
\usepackage{tikz,tkz-tab} 			% package principal TikZ
\usepackage{pgfplots}   % axes
\usepackage{mathrsfs}    % cursives
\usepackage{calc}			% calcul taille boites
\usepackage[scaled=0.875]{helvet} % font sans serif
\usepackage{svg} % svg
\usepackage{scrextend} % local margin
\usepackage{scratch} %scratch
\usepackage{multicol} % colonnes
%\usepackage{infix-RPN,pst-func} % formule en notation polanaise inversée
\usepackage{listings}

%================================================================================================================================
%
% Réglages de base
%
%================================================================================================================================

\lstset{
language=Python,   % R code
literate=
{á}{{\'a}}1
{à}{{\`a}}1
{ã}{{\~a}}1
{é}{{\'e}}1
{è}{{\`e}}1
{ê}{{\^e}}1
{í}{{\'i}}1
{ó}{{\'o}}1
{õ}{{\~o}}1
{ú}{{\'u}}1
{ü}{{\"u}}1
{ç}{{\c{c}}}1
{~}{{ }}1
}


\definecolor{codegreen}{rgb}{0,0.6,0}
\definecolor{codegray}{rgb}{0.5,0.5,0.5}
\definecolor{codepurple}{rgb}{0.58,0,0.82}
\definecolor{backcolour}{rgb}{0.95,0.95,0.92}

\lstdefinestyle{mystyle}{
    backgroundcolor=\color{backcolour},   
    commentstyle=\color{codegreen},
    keywordstyle=\color{magenta},
    numberstyle=\tiny\color{codegray},
    stringstyle=\color{codepurple},
    basicstyle=\ttfamily\footnotesize,
    breakatwhitespace=false,         
    breaklines=true,                 
    captionpos=b,                    
    keepspaces=true,                 
    numbers=left,                    
xleftmargin=2em,
framexleftmargin=2em,            
    showspaces=false,                
    showstringspaces=false,
    showtabs=false,                  
    tabsize=2,
    upquote=true
}

\lstset{style=mystyle}


\lstset{style=mystyle}
\newcommand{\imgdir}{C:/laragon/www/newmc/assets/imgsvg/}
\newcommand{\imgsvgdir}{C:/laragon/www/newmc/assets/imgsvg/}

\definecolor{mcgris}{RGB}{220, 220, 220}% ancien~; pour compatibilité
\definecolor{mcbleu}{RGB}{52, 152, 219}
\definecolor{mcvert}{RGB}{125, 194, 70}
\definecolor{mcmauve}{RGB}{154, 0, 215}
\definecolor{mcorange}{RGB}{255, 96, 0}
\definecolor{mcturquoise}{RGB}{0, 153, 153}
\definecolor{mcrouge}{RGB}{255, 0, 0}
\definecolor{mclightvert}{RGB}{205, 234, 190}

\definecolor{gris}{RGB}{220, 220, 220}
\definecolor{bleu}{RGB}{52, 152, 219}
\definecolor{vert}{RGB}{125, 194, 70}
\definecolor{mauve}{RGB}{154, 0, 215}
\definecolor{orange}{RGB}{255, 96, 0}
\definecolor{turquoise}{RGB}{0, 153, 153}
\definecolor{rouge}{RGB}{255, 0, 0}
\definecolor{lightvert}{RGB}{205, 234, 190}
\setitemize[0]{label=\color{lightvert}  $\bullet$}

\pagestyle{fancy}
\renewcommand{\headrulewidth}{0.2pt}
\fancyhead[L]{maths-cours.fr}
\fancyhead[R]{\thepage}
\renewcommand{\footrulewidth}{0.2pt}
\fancyfoot[C]{}

\newcolumntype{C}{>{\centering\arraybackslash}X}
\newcolumntype{s}{>{\hsize=.35\hsize\arraybackslash}X}

\setlength{\parindent}{0pt}		 
\setlength{\parskip}{3mm}
\setlength{\headheight}{1cm}

\def\ebook{ebook}
\def\book{book}
\def\web{web}
\def\type{web}

\newcommand{\vect}[1]{\overrightarrow{\,\mathstrut#1\,}}

\def\Oij{$\left(\text{O}~;~\vect{\imath},~\vect{\jmath}\right)$}
\def\Oijk{$\left(\text{O}~;~\vect{\imath},~\vect{\jmath},~\vect{k}\right)$}
\def\Ouv{$\left(\text{O}~;~\vect{u},~\vect{v}\right)$}

\hypersetup{breaklinks=true, colorlinks = true, linkcolor = OliveGreen, urlcolor = OliveGreen, citecolor = OliveGreen, pdfauthor={Didier BONNEL - https://www.maths-cours.fr} } % supprime les bordures autour des liens

\renewcommand{\arg}[0]{\text{arg}}

\everymath{\displaystyle}

%================================================================================================================================
%
% Macros - Commandes
%
%================================================================================================================================

\newcommand\meta[2]{    			% Utilisé pour créer le post HTML.
	\def\titre{titre}
	\def\url{url}
	\def\arg{#1}
	\ifx\titre\arg
		\newcommand\maintitle{#2}
		\fancyhead[L]{#2}
		{\Large\sffamily \MakeUppercase{#2}}
		\vspace{1mm}\textcolor{mcvert}{\hrule}
	\fi 
	\ifx\url\arg
		\fancyfoot[L]{\href{https://www.maths-cours.fr#2}{\black \footnotesize{https://www.maths-cours.fr#2}}}
	\fi 
}


\newcommand\TitreC[1]{    		% Titre centré
     \needspace{3\baselineskip}
     \begin{center}\textbf{#1}\end{center}
}

\newcommand\newpar{    		% paragraphe
     \par
}

\newcommand\nosp {    		% commande vide (pas d'espace)
}
\newcommand{\id}[1]{} %ignore

\newcommand\boite[2]{				% Boite simple sans titre
	\vspace{5mm}
	\setlength{\fboxrule}{0.2mm}
	\setlength{\fboxsep}{5mm}	
	\fcolorbox{#1}{#1!3}{\makebox[\linewidth-2\fboxrule-2\fboxsep]{
  		\begin{minipage}[t]{\linewidth-2\fboxrule-4\fboxsep}\setlength{\parskip}{3mm}
  			 #2
  		\end{minipage}
	}}
	\vspace{5mm}
}

\newcommand\CBox[4]{				% Boites
	\vspace{5mm}
	\setlength{\fboxrule}{0.2mm}
	\setlength{\fboxsep}{5mm}
	
	\fcolorbox{#1}{#1!3}{\makebox[\linewidth-2\fboxrule-2\fboxsep]{
		\begin{minipage}[t]{1cm}\setlength{\parskip}{3mm}
	  		\textcolor{#1}{\LARGE{#2}}    
 	 	\end{minipage}  
  		\begin{minipage}[t]{\linewidth-2\fboxrule-4\fboxsep}\setlength{\parskip}{3mm}
			\raisebox{1.2mm}{\normalsize\sffamily{\textcolor{#1}{#3}}}						
  			 #4
  		\end{minipage}
	}}
	\vspace{5mm}
}

\newcommand\cadre[3]{				% Boites convertible html
	\par
	\vspace{2mm}
	\setlength{\fboxrule}{0.1mm}
	\setlength{\fboxsep}{5mm}
	\fcolorbox{#1}{white}{\makebox[\linewidth-2\fboxrule-2\fboxsep]{
  		\begin{minipage}[t]{\linewidth-2\fboxrule-4\fboxsep}\setlength{\parskip}{3mm}
			\raisebox{-2.5mm}{\sffamily \small{\textcolor{#1}{\MakeUppercase{#2}}}}		
			\par		
  			 #3
 	 		\end{minipage}
	}}
		\vspace{2mm}
	\par
}

\newcommand\bloc[3]{				% Boites convertible html sans bordure
     \needspace{2\baselineskip}
     {\sffamily \small{\textcolor{#1}{\MakeUppercase{#2}}}}    
		\par		
  			 #3
		\par
}

\newcommand\CHelp[1]{
     \CBox{Plum}{\faInfoCircle}{À RETENIR}{#1}
}

\newcommand\CUp[1]{
     \CBox{NavyBlue}{\faThumbsOUp}{EN PRATIQUE}{#1}
}

\newcommand\CInfo[1]{
     \CBox{Sepia}{\faArrowCircleRight}{REMARQUE}{#1}
}

\newcommand\CRedac[1]{
     \CBox{PineGreen}{\faEdit}{BIEN R\'EDIGER}{#1}
}

\newcommand\CError[1]{
     \CBox{Red}{\faExclamationTriangle}{ATTENTION}{#1}
}

\newcommand\TitreExo[2]{
\needspace{4\baselineskip}
 {\sffamily\large EXERCICE #1\ (\emph{#2 points})}
\vspace{5mm}
}

\newcommand\img[2]{
          \includegraphics[width=#2\paperwidth]{\imgdir#1}
}

\newcommand\imgsvg[2]{
       \begin{center}   \includegraphics[width=#2\paperwidth]{\imgsvgdir#1} \end{center}
}


\newcommand\Lien[2]{
     \href{#1}{#2 \tiny \faExternalLink}
}
\newcommand\mcLien[2]{
     \href{https~://www.maths-cours.fr/#1}{#2 \tiny \faExternalLink}
}

\newcommand{\euro}{\eurologo{}}

%================================================================================================================================
%
% Macros - Environement
%
%================================================================================================================================

\newenvironment{tex}{ %
}
{%
}

\newenvironment{indente}{ %
	\setlength\parindent{10mm}
}

{
	\setlength\parindent{0mm}
}

\newenvironment{corrige}{%
     \needspace{3\baselineskip}
     \medskip
     \textbf{\textsc{Corrigé}}
     \medskip
}
{
}

\newenvironment{extern}{%
     \begin{center}
     }
     {
     \end{center}
}

\NewEnviron{code}{%
	\par
     \boite{gray}{\texttt{%
     \BODY
     }}
     \par
}

\newenvironment{vbloc}{% boite sans cadre empeche saut de page
     \begin{minipage}[t]{\linewidth}
     }
     {
     \end{minipage}
}
\NewEnviron{h2}{%
    \needspace{3\baselineskip}
    \vspace{0.6cm}
	\noindent \MakeUppercase{\sffamily \large \BODY}
	\vspace{1mm}\textcolor{mcgris}{\hrule}\vspace{0.4cm}
	\par
}{}

\NewEnviron{h3}{%
    \needspace{3\baselineskip}
	\vspace{5mm}
	\textsc{\BODY}
	\par
}

\NewEnviron{margeneg}{ %
\begin{addmargin}[-1cm]{0cm}
\BODY
\end{addmargin}
}

\NewEnviron{html}{%
}

\begin{document}
\meta{url}{/exercices/geometrie-dans-lespace-bac-s-metropole-2018/}
\meta{pid}{9174}
\meta{titre}{Géométrie dans l'espace - Bac S Métropole 2018}
\meta{type}{exercices}
%
\begin{h2}Exercice 3 (5 points)\end{h2}
\par
\textbf{Commun à tous les candidats }
\bigbreak
Le but de cet exercice est d'examiner, dans différents cas, si les hauteurs d'un tétraèdre sont concourantes, c'est-à-dire d'étudier l'existence d'un point d'intersection de ses quatre hauteurs.
\par
\emph{On rappelle que dans un tétraèdre} MNPQ, \emph{la hauteur issue de} M \emph{est la droite passant par} M \emph{orthogonale au plan} (NPQ).
\bigbreak
\TitreC{Partie A - Étude de cas particuliers}
\medbreak
On considère un cube ABCDEFGH.
\begin{center}
     \begin{extern}%width="280" alt="Cube Géométrie Bac S Métropole 2018"
          \psset{unit=0.8cm}
          \begin{pspicture}(8,7.3)
               \psline(0.5,1)(4.8,0.5)(6.8,1.3)(6.8,6.3)(4.8,5.5)(4.8,0.5)%ABCGFB
               \psline(6.8,6.3)(2.5,6.8)(0.5,6)(4.8,5.5)%GHEF
               \psline(0.5,1)(0.5,6)%AE
               \psline[linestyle=dashed,linewidth=1pt](0.5,1)(2.5,1.8)(2.5,6.8)
               \psline[linestyle=dashed,linewidth=1pt](6.8,1.3)(2.5,1.8)
               \uput[dl](0.5,1){A} \uput[d](4.8,0.5){B} \uput[r](6.8,1.3){C} \uput[ur](2.5,1.8){D}
               \uput[l](0.5,6){E} \uput[u](4.8,5.5){F} \uput[ur](6.8,6.3){G} \uput[u](2.5,6.8){H}
          \end{pspicture}
     \end{extern}
\end{center}
\medbreak
On admet que les droites (AG), (BH), (CE) et (DF), appelées \og grandes diagonales\fg{} du cube, sont concourantes.
\medbreak
\begin{enumerate}
     \item On considère le tétraèdre ABCE.
     \begin{enumerate}[label=\alph*.]
          \item Préciser la hauteur issue de E et la hauteur issue de C dans ce tétraèdre.
          \item Les quatre hauteurs du tétraèdre ABCE sont-elles concourantes~?
     \end{enumerate}
     \item On considère le tétraèdre ACHF et on travaille dans le repère $\left(\text{A}~;~ \overrightarrow{\text{AB}},~ \overrightarrow{\text{AD}},~ \overrightarrow{\text{AE}}\right)$.
     \begin{enumerate}[label=\alph*.]
          \item Vérifier qu'une équation cartésienne du plan (ACH) est~: $x - y + z = 0$.
          \item En déduire que (FD) est la hauteur issue de F du tétraèdre ACHF{}.
          \item Par analogie avec le résultat précédent, préciser les hauteurs du tétraèdre ACHF issues respectivement des sommets A, C et H.
          \par
          Les quatre hauteurs du tétraèdre ACHF sont-elles concourantes~?
     \end{enumerate}
\end{enumerate}
\emph{Dans la suite de cet exercice, un tétraèdre dont les quatre hauteurs sont concourantes sera appelé un tétraèdre orthocentrique.}
\bigbreak
\TitreC{Partie B - Une propriété des tétraèdres orthocentriques}
\medbreak
Dans cette partie, on considère un tétraèdre MNPQ dont les hauteurs issues des sommets M et
N sont sécantes en un point K. Les droites (MK) et (NK) sont donc orthogonales aux plans
(NPQ) et (MPQ) respectivement.
\begin{center}
     \begin{extern}%width="280" alt="Tétraèdre Géométrie Bac S Métropole 2018"
          \psset{unit=0.8cm}
          \begin{pspicture}(8,8.2)
               \pspolygon(3.9,8)(0.5,0.5)(7.8,0.5)%MNP
               \psline[linestyle=dashed](0.5,0.5)(5,2)(3.9,8)%NQM
               \psline[linestyle=dashed](5,2)(7.8,0.5)%QP
               \psline[linestyle=dotted](3.9,8)(3.9,0.8) \psframe(3.9,0.8)(4.1,1)
               \psline[linestyle=dotted](0.5,0.5)(6,3.2) \rput{-60}(6,3.2){\psframe(0.2,0.2)}
               \uput[u](3.9,8){M} \uput[dl](0.5,0.5){N} \uput[dr](7.8,0.5){P}
               \uput[ur](5,2){Q} \uput[ul](3.9,2.2){K}
          \end{pspicture}
     \end{extern}
\end{center}
\begin{enumerate}
     \item
     \begin{enumerate}[label=\alph*.]
          \item Justifier que la droite (PQ) est orthogonale à la droite (MK)~; on admet de même que les droites (PQ) et (NK) sont orthogonales.
          \item Que peut-on déduire de la question précédente relativement à la droite (PQ) et au plan (MNK)~? Justifier la réponse.
     \end{enumerate}
     \item Montrer que les arêtes [MN] et [PQ] sont orthogonales.
     \par
     Ainsi, on obtient la propriété suivante~:
     \par
     Si un tétraèdre est orthocentrique, alors ses arêtes opposées sont orthogonales deux à deux.
     \par
     (On dit que deux arêtes d'un tétraèdre sont \og opposées\fg{} lorsqu'elles n'ont pas de sommet commun.)
\end{enumerate}
\bigbreak
\TitreC{Partie C - Application}
\medbreak
Dans un repère orthonormé, on considère les points~:
\par
$\text{R}(-3~;~5~;~2)$ , $\text{S}(1~;~4~;~-2)$ , $\text{T}(4~;~-1~;~5)$ et $\text{U}(4~;~7~;~3).$
\par
Le tétraèdre RSTU est-il orthocentrique~? Justifier.

\end{document}
µ
\documentclass[a4paper]{article}

%================================================================================================================================
%
% Packages
%
%================================================================================================================================

\usepackage[T1]{fontenc} 	% pour caractères accentués
\usepackage[utf8]{inputenc}  % encodage utf8
\usepackage[french]{babel}	% langue : français
\usepackage{fourier}			% caractères plus lisibles
\usepackage[dvipsnames]{xcolor} % couleurs
\usepackage{fancyhdr}		% réglage header footer
\usepackage{needspace}		% empêcher sauts de page mal placés
\usepackage{graphicx}		% pour inclure des graphiques
\usepackage{enumitem,cprotect}		% personnalise les listes d'items (nécessaire pour ol, al ...)
\usepackage{hyperref}		% Liens hypertexte
\usepackage{pstricks,pst-all,pst-node,pstricks-add,pst-math,pst-plot,pst-tree,pst-eucl} % pstricks
\usepackage[a4paper,includeheadfoot,top=2cm,left=3cm, bottom=2cm,right=3cm]{geometry} % marges etc.
\usepackage{comment}			% commentaires multilignes
\usepackage{amsmath,environ} % maths (matrices, etc.)
\usepackage{amssymb,makeidx}
\usepackage{bm}				% bold maths
\usepackage{tabularx}		% tableaux
\usepackage{colortbl}		% tableaux en couleur
\usepackage{fontawesome}		% Fontawesome
\usepackage{environ}			% environment with command
\usepackage{fp}				% calculs pour ps-tricks
\usepackage{multido}			% pour ps tricks
\usepackage[np]{numprint}	% formattage nombre
\usepackage{tikz,tkz-tab} 			% package principal TikZ
\usepackage{pgfplots}   % axes
\usepackage{mathrsfs}    % cursives
\usepackage{calc}			% calcul taille boites
\usepackage[scaled=0.875]{helvet} % font sans serif
\usepackage{svg} % svg
\usepackage{scrextend} % local margin
\usepackage{scratch} %scratch
\usepackage{multicol} % colonnes
%\usepackage{infix-RPN,pst-func} % formule en notation polanaise inversée
\usepackage{listings}

%================================================================================================================================
%
% Réglages de base
%
%================================================================================================================================

\lstset{
language=Python,   % R code
literate=
{á}{{\'a}}1
{à}{{\`a}}1
{ã}{{\~a}}1
{é}{{\'e}}1
{è}{{\`e}}1
{ê}{{\^e}}1
{í}{{\'i}}1
{ó}{{\'o}}1
{õ}{{\~o}}1
{ú}{{\'u}}1
{ü}{{\"u}}1
{ç}{{\c{c}}}1
{~}{{ }}1
}


\definecolor{codegreen}{rgb}{0,0.6,0}
\definecolor{codegray}{rgb}{0.5,0.5,0.5}
\definecolor{codepurple}{rgb}{0.58,0,0.82}
\definecolor{backcolour}{rgb}{0.95,0.95,0.92}

\lstdefinestyle{mystyle}{
    backgroundcolor=\color{backcolour},   
    commentstyle=\color{codegreen},
    keywordstyle=\color{magenta},
    numberstyle=\tiny\color{codegray},
    stringstyle=\color{codepurple},
    basicstyle=\ttfamily\footnotesize,
    breakatwhitespace=false,         
    breaklines=true,                 
    captionpos=b,                    
    keepspaces=true,                 
    numbers=left,                    
xleftmargin=2em,
framexleftmargin=2em,            
    showspaces=false,                
    showstringspaces=false,
    showtabs=false,                  
    tabsize=2,
    upquote=true
}

\lstset{style=mystyle}


\lstset{style=mystyle}
\newcommand{\imgdir}{C:/laragon/www/newmc/assets/imgsvg/}
\newcommand{\imgsvgdir}{C:/laragon/www/newmc/assets/imgsvg/}

\definecolor{mcgris}{RGB}{220, 220, 220}% ancien~; pour compatibilité
\definecolor{mcbleu}{RGB}{52, 152, 219}
\definecolor{mcvert}{RGB}{125, 194, 70}
\definecolor{mcmauve}{RGB}{154, 0, 215}
\definecolor{mcorange}{RGB}{255, 96, 0}
\definecolor{mcturquoise}{RGB}{0, 153, 153}
\definecolor{mcrouge}{RGB}{255, 0, 0}
\definecolor{mclightvert}{RGB}{205, 234, 190}

\definecolor{gris}{RGB}{220, 220, 220}
\definecolor{bleu}{RGB}{52, 152, 219}
\definecolor{vert}{RGB}{125, 194, 70}
\definecolor{mauve}{RGB}{154, 0, 215}
\definecolor{orange}{RGB}{255, 96, 0}
\definecolor{turquoise}{RGB}{0, 153, 153}
\definecolor{rouge}{RGB}{255, 0, 0}
\definecolor{lightvert}{RGB}{205, 234, 190}
\setitemize[0]{label=\color{lightvert}  $\bullet$}

\pagestyle{fancy}
\renewcommand{\headrulewidth}{0.2pt}
\fancyhead[L]{maths-cours.fr}
\fancyhead[R]{\thepage}
\renewcommand{\footrulewidth}{0.2pt}
\fancyfoot[C]{}

\newcolumntype{C}{>{\centering\arraybackslash}X}
\newcolumntype{s}{>{\hsize=.35\hsize\arraybackslash}X}

\setlength{\parindent}{0pt}		 
\setlength{\parskip}{3mm}
\setlength{\headheight}{1cm}

\def\ebook{ebook}
\def\book{book}
\def\web{web}
\def\type{web}

\newcommand{\vect}[1]{\overrightarrow{\,\mathstrut#1\,}}

\def\Oij{$\left(\text{O}~;~\vect{\imath},~\vect{\jmath}\right)$}
\def\Oijk{$\left(\text{O}~;~\vect{\imath},~\vect{\jmath},~\vect{k}\right)$}
\def\Ouv{$\left(\text{O}~;~\vect{u},~\vect{v}\right)$}

\hypersetup{breaklinks=true, colorlinks = true, linkcolor = OliveGreen, urlcolor = OliveGreen, citecolor = OliveGreen, pdfauthor={Didier BONNEL - https://www.maths-cours.fr} } % supprime les bordures autour des liens

\renewcommand{\arg}[0]{\text{arg}}

\everymath{\displaystyle}

%================================================================================================================================
%
% Macros - Commandes
%
%================================================================================================================================

\newcommand\meta[2]{    			% Utilisé pour créer le post HTML.
	\def\titre{titre}
	\def\url{url}
	\def\arg{#1}
	\ifx\titre\arg
		\newcommand\maintitle{#2}
		\fancyhead[L]{#2}
		{\Large\sffamily \MakeUppercase{#2}}
		\vspace{1mm}\textcolor{mcvert}{\hrule}
	\fi 
	\ifx\url\arg
		\fancyfoot[L]{\href{https://www.maths-cours.fr#2}{\black \footnotesize{https://www.maths-cours.fr#2}}}
	\fi 
}


\newcommand\TitreC[1]{    		% Titre centré
     \needspace{3\baselineskip}
     \begin{center}\textbf{#1}\end{center}
}

\newcommand\newpar{    		% paragraphe
     \par
}

\newcommand\nosp {    		% commande vide (pas d'espace)
}
\newcommand{\id}[1]{} %ignore

\newcommand\boite[2]{				% Boite simple sans titre
	\vspace{5mm}
	\setlength{\fboxrule}{0.2mm}
	\setlength{\fboxsep}{5mm}	
	\fcolorbox{#1}{#1!3}{\makebox[\linewidth-2\fboxrule-2\fboxsep]{
  		\begin{minipage}[t]{\linewidth-2\fboxrule-4\fboxsep}\setlength{\parskip}{3mm}
  			 #2
  		\end{minipage}
	}}
	\vspace{5mm}
}

\newcommand\CBox[4]{				% Boites
	\vspace{5mm}
	\setlength{\fboxrule}{0.2mm}
	\setlength{\fboxsep}{5mm}
	
	\fcolorbox{#1}{#1!3}{\makebox[\linewidth-2\fboxrule-2\fboxsep]{
		\begin{minipage}[t]{1cm}\setlength{\parskip}{3mm}
	  		\textcolor{#1}{\LARGE{#2}}    
 	 	\end{minipage}  
  		\begin{minipage}[t]{\linewidth-2\fboxrule-4\fboxsep}\setlength{\parskip}{3mm}
			\raisebox{1.2mm}{\normalsize\sffamily{\textcolor{#1}{#3}}}						
  			 #4
  		\end{minipage}
	}}
	\vspace{5mm}
}

\newcommand\cadre[3]{				% Boites convertible html
	\par
	\vspace{2mm}
	\setlength{\fboxrule}{0.1mm}
	\setlength{\fboxsep}{5mm}
	\fcolorbox{#1}{white}{\makebox[\linewidth-2\fboxrule-2\fboxsep]{
  		\begin{minipage}[t]{\linewidth-2\fboxrule-4\fboxsep}\setlength{\parskip}{3mm}
			\raisebox{-2.5mm}{\sffamily \small{\textcolor{#1}{\MakeUppercase{#2}}}}		
			\par		
  			 #3
 	 		\end{minipage}
	}}
		\vspace{2mm}
	\par
}

\newcommand\bloc[3]{				% Boites convertible html sans bordure
     \needspace{2\baselineskip}
     {\sffamily \small{\textcolor{#1}{\MakeUppercase{#2}}}}    
		\par		
  			 #3
		\par
}

\newcommand\CHelp[1]{
     \CBox{Plum}{\faInfoCircle}{À RETENIR}{#1}
}

\newcommand\CUp[1]{
     \CBox{NavyBlue}{\faThumbsOUp}{EN PRATIQUE}{#1}
}

\newcommand\CInfo[1]{
     \CBox{Sepia}{\faArrowCircleRight}{REMARQUE}{#1}
}

\newcommand\CRedac[1]{
     \CBox{PineGreen}{\faEdit}{BIEN R\'EDIGER}{#1}
}

\newcommand\CError[1]{
     \CBox{Red}{\faExclamationTriangle}{ATTENTION}{#1}
}

\newcommand\TitreExo[2]{
\needspace{4\baselineskip}
 {\sffamily\large EXERCICE #1\ (\emph{#2 points})}
\vspace{5mm}
}

\newcommand\img[2]{
          \includegraphics[width=#2\paperwidth]{\imgdir#1}
}

\newcommand\imgsvg[2]{
       \begin{center}   \includegraphics[width=#2\paperwidth]{\imgsvgdir#1} \end{center}
}


\newcommand\Lien[2]{
     \href{#1}{#2 \tiny \faExternalLink}
}
\newcommand\mcLien[2]{
     \href{https~://www.maths-cours.fr/#1}{#2 \tiny \faExternalLink}
}

\newcommand{\euro}{\eurologo{}}

%================================================================================================================================
%
% Macros - Environement
%
%================================================================================================================================

\newenvironment{tex}{ %
}
{%
}

\newenvironment{indente}{ %
	\setlength\parindent{10mm}
}

{
	\setlength\parindent{0mm}
}

\newenvironment{corrige}{%
     \needspace{3\baselineskip}
     \medskip
     \textbf{\textsc{Corrigé}}
     \medskip
}
{
}

\newenvironment{extern}{%
     \begin{center}
     }
     {
     \end{center}
}

\NewEnviron{code}{%
	\par
     \boite{gray}{\texttt{%
     \BODY
     }}
     \par
}

\newenvironment{vbloc}{% boite sans cadre empeche saut de page
     \begin{minipage}[t]{\linewidth}
     }
     {
     \end{minipage}
}
\NewEnviron{h2}{%
    \needspace{3\baselineskip}
    \vspace{0.6cm}
	\noindent \MakeUppercase{\sffamily \large \BODY}
	\vspace{1mm}\textcolor{mcgris}{\hrule}\vspace{0.4cm}
	\par
}{}

\NewEnviron{h3}{%
    \needspace{3\baselineskip}
	\vspace{5mm}
	\textsc{\BODY}
	\par
}

\NewEnviron{margeneg}{ %
\begin{addmargin}[-1cm]{0cm}
\BODY
\end{addmargin}
}

\NewEnviron{html}{%
}

\begin{document}
\meta{url}{/exercices/nombres-complexes-bac-s-metropole-2018/}
\meta{pid}{9176}
\meta{titre}{Nombres complexes - Bac S Métropole 2018}
\meta{type}{exercices}
%
\begin{h2}Exercice 4 (5 points)\end{h2}
\textbf{Pour les candidats n'ayant pas choisi l'enseignement de spécialité \og Mathématiques \fg{} }
\bigbreak
Le plan complexe est muni d'un repère orthonormé direct $(O~;~\overrightarrow{i},~\overrightarrow{j}~,~\overrightarrow{k})$.
\par
On pose $z_0 = 8$ et, pour tout entier naturel $n$~:
\par
\[z_{n+1} = \dfrac{3 - \text{i}\sqrt{3}}{4}z_n.\]
\par
On note $A_n$ le point du plan d'affixe $z_n$.
\medbreak
\begin{enumerate}
     \item
     \begin{enumerate}[label=\alph*.]
          \item Vérifier que~:
          \par
          \[\dfrac{3 - \text{i}\sqrt{3}}{4} = \dfrac{\sqrt{3}}{2}\text{e}^{- \text{i}\frac{\pi}{6}}.\]
          \item En déduire l'écriture de chacun des nombres complexes $z_1$,  $z_2$ et $z_3$ sous forme exponentielle et vérifier que $z_3$ est un imaginaire pur dont on précisera la partie imaginaire.
          \item Représenter graphiquement les points $A_0$ , $A_1$ , $A_2$ et $A_3$~; on prendra pour unité le centimètre.
     \end{enumerate}
     \item
     \begin{enumerate}[label=\alph*.]
          \item Démontrer par récurrence que, pour tout entier naturel $n$,
          \par
          \[z_n = 8 \times \left(\dfrac{\sqrt{3}}{2}\right)^n \text{e}^{- \text{i}\frac{n\pi}{6}}.\]
          \item Pour tout entier naturel $n$, on pose $u_n = \left|z_n\right|$.
          \par
          Déterminer la nature et la limite de la suite $\left(u_n\right)$.
     \end{enumerate}
     \item
     \begin{enumerate}[label=\alph*.]
          \item Démontrer que, pour tout entier naturel $k$,
          \par
          \[\dfrac{z_{k+1} - z_{k}}{z_{k+1}} = - \dfrac{1}{\sqrt{3}}\text{i}.\]
          \par
          En déduire que, pour tout entier naturel $k$, on a l'égalité~: $A_kA_{k+1} = \dfrac{1}{\sqrt{3}} \text{O}A_{k+1}$.
          \item Pour tout entier naturel $n$, on appelle $l_n$ la longueur de la ligne brisée reliant dans cet ordre les points $A_0$,\: $A_1$,\: $A_2$, $\ldots$ , $A_n$.
          \par
          On a ainsi~: $l_n = A_0A_1 + A_1A_2 + \ldots + A_{n-1}A_n$.
          \par
          Démontrer que la suite $\left(l_n\right)$ est convergente et calculer sa limite.
     \end{enumerate}
\end{enumerate}

\end{document}
µ
\documentclass[a4paper]{article}

%================================================================================================================================
%
% Packages
%
%================================================================================================================================

\usepackage[T1]{fontenc} 	% pour caractères accentués
\usepackage[utf8]{inputenc}  % encodage utf8
\usepackage[french]{babel}	% langue : français
\usepackage{fourier}			% caractères plus lisibles
\usepackage[dvipsnames]{xcolor} % couleurs
\usepackage{fancyhdr}		% réglage header footer
\usepackage{needspace}		% empêcher sauts de page mal placés
\usepackage{graphicx}		% pour inclure des graphiques
\usepackage{enumitem,cprotect}		% personnalise les listes d'items (nécessaire pour ol, al ...)
\usepackage{hyperref}		% Liens hypertexte
\usepackage{pstricks,pst-all,pst-node,pstricks-add,pst-math,pst-plot,pst-tree,pst-eucl} % pstricks
\usepackage[a4paper,includeheadfoot,top=2cm,left=3cm, bottom=2cm,right=3cm]{geometry} % marges etc.
\usepackage{comment}			% commentaires multilignes
\usepackage{amsmath,environ} % maths (matrices, etc.)
\usepackage{amssymb,makeidx}
\usepackage{bm}				% bold maths
\usepackage{tabularx}		% tableaux
\usepackage{colortbl}		% tableaux en couleur
\usepackage{fontawesome}		% Fontawesome
\usepackage{environ}			% environment with command
\usepackage{fp}				% calculs pour ps-tricks
\usepackage{multido}			% pour ps tricks
\usepackage[np]{numprint}	% formattage nombre
\usepackage{tikz,tkz-tab} 			% package principal TikZ
\usepackage{pgfplots}   % axes
\usepackage{mathrsfs}    % cursives
\usepackage{calc}			% calcul taille boites
\usepackage[scaled=0.875]{helvet} % font sans serif
\usepackage{svg} % svg
\usepackage{scrextend} % local margin
\usepackage{scratch} %scratch
\usepackage{multicol} % colonnes
%\usepackage{infix-RPN,pst-func} % formule en notation polanaise inversée
\usepackage{listings}

%================================================================================================================================
%
% Réglages de base
%
%================================================================================================================================

\lstset{
language=Python,   % R code
literate=
{á}{{\'a}}1
{à}{{\`a}}1
{ã}{{\~a}}1
{é}{{\'e}}1
{è}{{\`e}}1
{ê}{{\^e}}1
{í}{{\'i}}1
{ó}{{\'o}}1
{õ}{{\~o}}1
{ú}{{\'u}}1
{ü}{{\"u}}1
{ç}{{\c{c}}}1
{~}{{ }}1
}


\definecolor{codegreen}{rgb}{0,0.6,0}
\definecolor{codegray}{rgb}{0.5,0.5,0.5}
\definecolor{codepurple}{rgb}{0.58,0,0.82}
\definecolor{backcolour}{rgb}{0.95,0.95,0.92}

\lstdefinestyle{mystyle}{
    backgroundcolor=\color{backcolour},   
    commentstyle=\color{codegreen},
    keywordstyle=\color{magenta},
    numberstyle=\tiny\color{codegray},
    stringstyle=\color{codepurple},
    basicstyle=\ttfamily\footnotesize,
    breakatwhitespace=false,         
    breaklines=true,                 
    captionpos=b,                    
    keepspaces=true,                 
    numbers=left,                    
xleftmargin=2em,
framexleftmargin=2em,            
    showspaces=false,                
    showstringspaces=false,
    showtabs=false,                  
    tabsize=2,
    upquote=true
}

\lstset{style=mystyle}


\lstset{style=mystyle}
\newcommand{\imgdir}{C:/laragon/www/newmc/assets/imgsvg/}
\newcommand{\imgsvgdir}{C:/laragon/www/newmc/assets/imgsvg/}

\definecolor{mcgris}{RGB}{220, 220, 220}% ancien~; pour compatibilité
\definecolor{mcbleu}{RGB}{52, 152, 219}
\definecolor{mcvert}{RGB}{125, 194, 70}
\definecolor{mcmauve}{RGB}{154, 0, 215}
\definecolor{mcorange}{RGB}{255, 96, 0}
\definecolor{mcturquoise}{RGB}{0, 153, 153}
\definecolor{mcrouge}{RGB}{255, 0, 0}
\definecolor{mclightvert}{RGB}{205, 234, 190}

\definecolor{gris}{RGB}{220, 220, 220}
\definecolor{bleu}{RGB}{52, 152, 219}
\definecolor{vert}{RGB}{125, 194, 70}
\definecolor{mauve}{RGB}{154, 0, 215}
\definecolor{orange}{RGB}{255, 96, 0}
\definecolor{turquoise}{RGB}{0, 153, 153}
\definecolor{rouge}{RGB}{255, 0, 0}
\definecolor{lightvert}{RGB}{205, 234, 190}
\setitemize[0]{label=\color{lightvert}  $\bullet$}

\pagestyle{fancy}
\renewcommand{\headrulewidth}{0.2pt}
\fancyhead[L]{maths-cours.fr}
\fancyhead[R]{\thepage}
\renewcommand{\footrulewidth}{0.2pt}
\fancyfoot[C]{}

\newcolumntype{C}{>{\centering\arraybackslash}X}
\newcolumntype{s}{>{\hsize=.35\hsize\arraybackslash}X}

\setlength{\parindent}{0pt}		 
\setlength{\parskip}{3mm}
\setlength{\headheight}{1cm}

\def\ebook{ebook}
\def\book{book}
\def\web{web}
\def\type{web}

\newcommand{\vect}[1]{\overrightarrow{\,\mathstrut#1\,}}

\def\Oij{$\left(\text{O}~;~\vect{\imath},~\vect{\jmath}\right)$}
\def\Oijk{$\left(\text{O}~;~\vect{\imath},~\vect{\jmath},~\vect{k}\right)$}
\def\Ouv{$\left(\text{O}~;~\vect{u},~\vect{v}\right)$}

\hypersetup{breaklinks=true, colorlinks = true, linkcolor = OliveGreen, urlcolor = OliveGreen, citecolor = OliveGreen, pdfauthor={Didier BONNEL - https://www.maths-cours.fr} } % supprime les bordures autour des liens

\renewcommand{\arg}[0]{\text{arg}}

\everymath{\displaystyle}

%================================================================================================================================
%
% Macros - Commandes
%
%================================================================================================================================

\newcommand\meta[2]{    			% Utilisé pour créer le post HTML.
	\def\titre{titre}
	\def\url{url}
	\def\arg{#1}
	\ifx\titre\arg
		\newcommand\maintitle{#2}
		\fancyhead[L]{#2}
		{\Large\sffamily \MakeUppercase{#2}}
		\vspace{1mm}\textcolor{mcvert}{\hrule}
	\fi 
	\ifx\url\arg
		\fancyfoot[L]{\href{https://www.maths-cours.fr#2}{\black \footnotesize{https://www.maths-cours.fr#2}}}
	\fi 
}


\newcommand\TitreC[1]{    		% Titre centré
     \needspace{3\baselineskip}
     \begin{center}\textbf{#1}\end{center}
}

\newcommand\newpar{    		% paragraphe
     \par
}

\newcommand\nosp {    		% commande vide (pas d'espace)
}
\newcommand{\id}[1]{} %ignore

\newcommand\boite[2]{				% Boite simple sans titre
	\vspace{5mm}
	\setlength{\fboxrule}{0.2mm}
	\setlength{\fboxsep}{5mm}	
	\fcolorbox{#1}{#1!3}{\makebox[\linewidth-2\fboxrule-2\fboxsep]{
  		\begin{minipage}[t]{\linewidth-2\fboxrule-4\fboxsep}\setlength{\parskip}{3mm}
  			 #2
  		\end{minipage}
	}}
	\vspace{5mm}
}

\newcommand\CBox[4]{				% Boites
	\vspace{5mm}
	\setlength{\fboxrule}{0.2mm}
	\setlength{\fboxsep}{5mm}
	
	\fcolorbox{#1}{#1!3}{\makebox[\linewidth-2\fboxrule-2\fboxsep]{
		\begin{minipage}[t]{1cm}\setlength{\parskip}{3mm}
	  		\textcolor{#1}{\LARGE{#2}}    
 	 	\end{minipage}  
  		\begin{minipage}[t]{\linewidth-2\fboxrule-4\fboxsep}\setlength{\parskip}{3mm}
			\raisebox{1.2mm}{\normalsize\sffamily{\textcolor{#1}{#3}}}						
  			 #4
  		\end{minipage}
	}}
	\vspace{5mm}
}

\newcommand\cadre[3]{				% Boites convertible html
	\par
	\vspace{2mm}
	\setlength{\fboxrule}{0.1mm}
	\setlength{\fboxsep}{5mm}
	\fcolorbox{#1}{white}{\makebox[\linewidth-2\fboxrule-2\fboxsep]{
  		\begin{minipage}[t]{\linewidth-2\fboxrule-4\fboxsep}\setlength{\parskip}{3mm}
			\raisebox{-2.5mm}{\sffamily \small{\textcolor{#1}{\MakeUppercase{#2}}}}		
			\par		
  			 #3
 	 		\end{minipage}
	}}
		\vspace{2mm}
	\par
}

\newcommand\bloc[3]{				% Boites convertible html sans bordure
     \needspace{2\baselineskip}
     {\sffamily \small{\textcolor{#1}{\MakeUppercase{#2}}}}    
		\par		
  			 #3
		\par
}

\newcommand\CHelp[1]{
     \CBox{Plum}{\faInfoCircle}{À RETENIR}{#1}
}

\newcommand\CUp[1]{
     \CBox{NavyBlue}{\faThumbsOUp}{EN PRATIQUE}{#1}
}

\newcommand\CInfo[1]{
     \CBox{Sepia}{\faArrowCircleRight}{REMARQUE}{#1}
}

\newcommand\CRedac[1]{
     \CBox{PineGreen}{\faEdit}{BIEN R\'EDIGER}{#1}
}

\newcommand\CError[1]{
     \CBox{Red}{\faExclamationTriangle}{ATTENTION}{#1}
}

\newcommand\TitreExo[2]{
\needspace{4\baselineskip}
 {\sffamily\large EXERCICE #1\ (\emph{#2 points})}
\vspace{5mm}
}

\newcommand\img[2]{
          \includegraphics[width=#2\paperwidth]{\imgdir#1}
}

\newcommand\imgsvg[2]{
       \begin{center}   \includegraphics[width=#2\paperwidth]{\imgsvgdir#1} \end{center}
}


\newcommand\Lien[2]{
     \href{#1}{#2 \tiny \faExternalLink}
}
\newcommand\mcLien[2]{
     \href{https~://www.maths-cours.fr/#1}{#2 \tiny \faExternalLink}
}

\newcommand{\euro}{\eurologo{}}

%================================================================================================================================
%
% Macros - Environement
%
%================================================================================================================================

\newenvironment{tex}{ %
}
{%
}

\newenvironment{indente}{ %
	\setlength\parindent{10mm}
}

{
	\setlength\parindent{0mm}
}

\newenvironment{corrige}{%
     \needspace{3\baselineskip}
     \medskip
     \textbf{\textsc{Corrigé}}
     \medskip
}
{
}

\newenvironment{extern}{%
     \begin{center}
     }
     {
     \end{center}
}

\NewEnviron{code}{%
	\par
     \boite{gray}{\texttt{%
     \BODY
     }}
     \par
}

\newenvironment{vbloc}{% boite sans cadre empeche saut de page
     \begin{minipage}[t]{\linewidth}
     }
     {
     \end{minipage}
}
\NewEnviron{h2}{%
    \needspace{3\baselineskip}
    \vspace{0.6cm}
	\noindent \MakeUppercase{\sffamily \large \BODY}
	\vspace{1mm}\textcolor{mcgris}{\hrule}\vspace{0.4cm}
	\par
}{}

\NewEnviron{h3}{%
    \needspace{3\baselineskip}
	\vspace{5mm}
	\textsc{\BODY}
	\par
}

\NewEnviron{margeneg}{ %
\begin{addmargin}[-1cm]{0cm}
\BODY
\end{addmargin}
}

\NewEnviron{html}{%
}

\begin{document}
\meta{url}{/exercices/matrices-et-arithmetique-bac-s-metropole-2018/}
\meta{pid}{9178}
\meta{titre}{Matrices et arithmétique - Bac S Métropole 2018}
\meta{type}{exercices}
%
\begin{h2}Exercice 4 (5 points)\end{h2}
\par
\textbf{Pour les candidats ayant choisi l'enseignement de spécialité \og Mathématiques \fg{} }
\bigbreak
\TitreC{Partie A}
\medbreak
On considère l'équation suivante dont les inconnues $x$ et $y$ sont des entiers naturels~:
\par
\[x^2 - 8y^2 = 1 . \quad(E)\]
\medbreak
\begin{enumerate}
     \item Déterminer un couple solution $(x~;~y)$ où $x$ et $y$ sont deux entiers naturels.
     \item  On considère la matrice $A = \begin{pmatrix}3&8\\1&3\end{pmatrix}$.
     On définit les suites d'entiers naturels $\left(x_n\right)$ et $\left(y_n\right)$ par~:
     \par
     \begin{center}
          $x_0 = 1,\: y_0 = 0,$ et pour tout entier naturel $n$, $\begin{pmatrix} x_{n+1}\\y_{n+1}\end{pmatrix} = A\begin{pmatrix}x_{n}\\y_{n}\end{pmatrix}.$
     \end{center}
     \begin{enumerate}[label=\alph*.]
          \item Démontrer par récurrence que pour tout entier naturel $n$, le couple
          $\left(x_n~;~y_n\right)$ est solution de l'équation $(E)$.
          \item En admettant que la suite $\left(x_n\right)$ est à valeurs strictement positives, démontrer que pour tout entier naturel $n$, on a~: $x_{n+1} > x_n$.
     \end{enumerate}
     \item  En déduire que l'équation $(E)$ admet une infinité de couples solutions.
\end{enumerate}
\bigbreak
\TitreC{Partie B}
\medbreak
Un entier naturel $n$ est appelé un nombre puissant lorsque, pour tout diviseur premier $p$ de $n$,\: $p^2$ divise $n$.
\medbreak
\begin{enumerate}
     \item Vérifier qu'il existe deux nombres entiers consécutifs inférieurs à $10$ qui sont puissants.
     \medbreak
     \begin{margeneg}
          L'objectif de cette partie est de démontrer, à l'aide des résultats de la partie A, qu'il existe une infinité de couples de nombres entiers naturels consécutifs puissants et d'en trouver quelques exemples.
     \end{margeneg}
     \medbreak
     \item  Soient $a$ et $b$ deux entiers naturels.
     \par
     Montrer que l'entier naturel $n = a^2 b^3$ est un nombre puissant.
     \item  Montrer que si $(x~;~y)$ est un couple solution de l'équation $(E)$ définie dans la partie A, alors $x^2 - 1$ et $x^2$ sont des entiers consécutifs puissants.
     \item  Conclure quant à l'objectif fixé pour cette partie, en démontrant qu'il existe une infinité de couples de nombres entiers consécutifs puissants.
     \par
     Déterminer deux nombres entiers consécutifs puissants supérieurs à $2018$.
\end{enumerate}

\end{document}
µ
\documentclass[a4paper]{article}

%================================================================================================================================
%
% Packages
%
%================================================================================================================================

\usepackage[T1]{fontenc} 	% pour caractères accentués
\usepackage[utf8]{inputenc}  % encodage utf8
\usepackage[french]{babel}	% langue : français
\usepackage{fourier}			% caractères plus lisibles
\usepackage[dvipsnames]{xcolor} % couleurs
\usepackage{fancyhdr}		% réglage header footer
\usepackage{needspace}		% empêcher sauts de page mal placés
\usepackage{graphicx}		% pour inclure des graphiques
\usepackage{enumitem,cprotect}		% personnalise les listes d'items (nécessaire pour ol, al ...)
\usepackage{hyperref}		% Liens hypertexte
\usepackage{pstricks,pst-all,pst-node,pstricks-add,pst-math,pst-plot,pst-tree,pst-eucl} % pstricks
\usepackage[a4paper,includeheadfoot,top=2cm,left=3cm, bottom=2cm,right=3cm]{geometry} % marges etc.
\usepackage{comment}			% commentaires multilignes
\usepackage{amsmath,environ} % maths (matrices, etc.)
\usepackage{amssymb,makeidx}
\usepackage{bm}				% bold maths
\usepackage{tabularx}		% tableaux
\usepackage{colortbl}		% tableaux en couleur
\usepackage{fontawesome}		% Fontawesome
\usepackage{environ}			% environment with command
\usepackage{fp}				% calculs pour ps-tricks
\usepackage{multido}			% pour ps tricks
\usepackage[np]{numprint}	% formattage nombre
\usepackage{tikz,tkz-tab} 			% package principal TikZ
\usepackage{pgfplots}   % axes
\usepackage{mathrsfs}    % cursives
\usepackage{calc}			% calcul taille boites
\usepackage[scaled=0.875]{helvet} % font sans serif
\usepackage{svg} % svg
\usepackage{scrextend} % local margin
\usepackage{scratch} %scratch
\usepackage{multicol} % colonnes
%\usepackage{infix-RPN,pst-func} % formule en notation polanaise inversée
\usepackage{listings}

%================================================================================================================================
%
% Réglages de base
%
%================================================================================================================================

\lstset{
language=Python,   % R code
literate=
{á}{{\'a}}1
{à}{{\`a}}1
{ã}{{\~a}}1
{é}{{\'e}}1
{è}{{\`e}}1
{ê}{{\^e}}1
{í}{{\'i}}1
{ó}{{\'o}}1
{õ}{{\~o}}1
{ú}{{\'u}}1
{ü}{{\"u}}1
{ç}{{\c{c}}}1
{~}{{ }}1
}


\definecolor{codegreen}{rgb}{0,0.6,0}
\definecolor{codegray}{rgb}{0.5,0.5,0.5}
\definecolor{codepurple}{rgb}{0.58,0,0.82}
\definecolor{backcolour}{rgb}{0.95,0.95,0.92}

\lstdefinestyle{mystyle}{
    backgroundcolor=\color{backcolour},   
    commentstyle=\color{codegreen},
    keywordstyle=\color{magenta},
    numberstyle=\tiny\color{codegray},
    stringstyle=\color{codepurple},
    basicstyle=\ttfamily\footnotesize,
    breakatwhitespace=false,         
    breaklines=true,                 
    captionpos=b,                    
    keepspaces=true,                 
    numbers=left,                    
xleftmargin=2em,
framexleftmargin=2em,            
    showspaces=false,                
    showstringspaces=false,
    showtabs=false,                  
    tabsize=2,
    upquote=true
}

\lstset{style=mystyle}


\lstset{style=mystyle}
\newcommand{\imgdir}{C:/laragon/www/newmc/assets/imgsvg/}
\newcommand{\imgsvgdir}{C:/laragon/www/newmc/assets/imgsvg/}

\definecolor{mcgris}{RGB}{220, 220, 220}% ancien~; pour compatibilité
\definecolor{mcbleu}{RGB}{52, 152, 219}
\definecolor{mcvert}{RGB}{125, 194, 70}
\definecolor{mcmauve}{RGB}{154, 0, 215}
\definecolor{mcorange}{RGB}{255, 96, 0}
\definecolor{mcturquoise}{RGB}{0, 153, 153}
\definecolor{mcrouge}{RGB}{255, 0, 0}
\definecolor{mclightvert}{RGB}{205, 234, 190}

\definecolor{gris}{RGB}{220, 220, 220}
\definecolor{bleu}{RGB}{52, 152, 219}
\definecolor{vert}{RGB}{125, 194, 70}
\definecolor{mauve}{RGB}{154, 0, 215}
\definecolor{orange}{RGB}{255, 96, 0}
\definecolor{turquoise}{RGB}{0, 153, 153}
\definecolor{rouge}{RGB}{255, 0, 0}
\definecolor{lightvert}{RGB}{205, 234, 190}
\setitemize[0]{label=\color{lightvert}  $\bullet$}

\pagestyle{fancy}
\renewcommand{\headrulewidth}{0.2pt}
\fancyhead[L]{maths-cours.fr}
\fancyhead[R]{\thepage}
\renewcommand{\footrulewidth}{0.2pt}
\fancyfoot[C]{}

\newcolumntype{C}{>{\centering\arraybackslash}X}
\newcolumntype{s}{>{\hsize=.35\hsize\arraybackslash}X}

\setlength{\parindent}{0pt}		 
\setlength{\parskip}{3mm}
\setlength{\headheight}{1cm}

\def\ebook{ebook}
\def\book{book}
\def\web{web}
\def\type{web}

\newcommand{\vect}[1]{\overrightarrow{\,\mathstrut#1\,}}

\def\Oij{$\left(\text{O}~;~\vect{\imath},~\vect{\jmath}\right)$}
\def\Oijk{$\left(\text{O}~;~\vect{\imath},~\vect{\jmath},~\vect{k}\right)$}
\def\Ouv{$\left(\text{O}~;~\vect{u},~\vect{v}\right)$}

\hypersetup{breaklinks=true, colorlinks = true, linkcolor = OliveGreen, urlcolor = OliveGreen, citecolor = OliveGreen, pdfauthor={Didier BONNEL - https://www.maths-cours.fr} } % supprime les bordures autour des liens

\renewcommand{\arg}[0]{\text{arg}}

\everymath{\displaystyle}

%================================================================================================================================
%
% Macros - Commandes
%
%================================================================================================================================

\newcommand\meta[2]{    			% Utilisé pour créer le post HTML.
	\def\titre{titre}
	\def\url{url}
	\def\arg{#1}
	\ifx\titre\arg
		\newcommand\maintitle{#2}
		\fancyhead[L]{#2}
		{\Large\sffamily \MakeUppercase{#2}}
		\vspace{1mm}\textcolor{mcvert}{\hrule}
	\fi 
	\ifx\url\arg
		\fancyfoot[L]{\href{https://www.maths-cours.fr#2}{\black \footnotesize{https://www.maths-cours.fr#2}}}
	\fi 
}


\newcommand\TitreC[1]{    		% Titre centré
     \needspace{3\baselineskip}
     \begin{center}\textbf{#1}\end{center}
}

\newcommand\newpar{    		% paragraphe
     \par
}

\newcommand\nosp {    		% commande vide (pas d'espace)
}
\newcommand{\id}[1]{} %ignore

\newcommand\boite[2]{				% Boite simple sans titre
	\vspace{5mm}
	\setlength{\fboxrule}{0.2mm}
	\setlength{\fboxsep}{5mm}	
	\fcolorbox{#1}{#1!3}{\makebox[\linewidth-2\fboxrule-2\fboxsep]{
  		\begin{minipage}[t]{\linewidth-2\fboxrule-4\fboxsep}\setlength{\parskip}{3mm}
  			 #2
  		\end{minipage}
	}}
	\vspace{5mm}
}

\newcommand\CBox[4]{				% Boites
	\vspace{5mm}
	\setlength{\fboxrule}{0.2mm}
	\setlength{\fboxsep}{5mm}
	
	\fcolorbox{#1}{#1!3}{\makebox[\linewidth-2\fboxrule-2\fboxsep]{
		\begin{minipage}[t]{1cm}\setlength{\parskip}{3mm}
	  		\textcolor{#1}{\LARGE{#2}}    
 	 	\end{minipage}  
  		\begin{minipage}[t]{\linewidth-2\fboxrule-4\fboxsep}\setlength{\parskip}{3mm}
			\raisebox{1.2mm}{\normalsize\sffamily{\textcolor{#1}{#3}}}						
  			 #4
  		\end{minipage}
	}}
	\vspace{5mm}
}

\newcommand\cadre[3]{				% Boites convertible html
	\par
	\vspace{2mm}
	\setlength{\fboxrule}{0.1mm}
	\setlength{\fboxsep}{5mm}
	\fcolorbox{#1}{white}{\makebox[\linewidth-2\fboxrule-2\fboxsep]{
  		\begin{minipage}[t]{\linewidth-2\fboxrule-4\fboxsep}\setlength{\parskip}{3mm}
			\raisebox{-2.5mm}{\sffamily \small{\textcolor{#1}{\MakeUppercase{#2}}}}		
			\par		
  			 #3
 	 		\end{minipage}
	}}
		\vspace{2mm}
	\par
}

\newcommand\bloc[3]{				% Boites convertible html sans bordure
     \needspace{2\baselineskip}
     {\sffamily \small{\textcolor{#1}{\MakeUppercase{#2}}}}    
		\par		
  			 #3
		\par
}

\newcommand\CHelp[1]{
     \CBox{Plum}{\faInfoCircle}{À RETENIR}{#1}
}

\newcommand\CUp[1]{
     \CBox{NavyBlue}{\faThumbsOUp}{EN PRATIQUE}{#1}
}

\newcommand\CInfo[1]{
     \CBox{Sepia}{\faArrowCircleRight}{REMARQUE}{#1}
}

\newcommand\CRedac[1]{
     \CBox{PineGreen}{\faEdit}{BIEN R\'EDIGER}{#1}
}

\newcommand\CError[1]{
     \CBox{Red}{\faExclamationTriangle}{ATTENTION}{#1}
}

\newcommand\TitreExo[2]{
\needspace{4\baselineskip}
 {\sffamily\large EXERCICE #1\ (\emph{#2 points})}
\vspace{5mm}
}

\newcommand\img[2]{
          \includegraphics[width=#2\paperwidth]{\imgdir#1}
}

\newcommand\imgsvg[2]{
       \begin{center}   \includegraphics[width=#2\paperwidth]{\imgsvgdir#1} \end{center}
}


\newcommand\Lien[2]{
     \href{#1}{#2 \tiny \faExternalLink}
}
\newcommand\mcLien[2]{
     \href{https~://www.maths-cours.fr/#1}{#2 \tiny \faExternalLink}
}

\newcommand{\euro}{\eurologo{}}

%================================================================================================================================
%
% Macros - Environement
%
%================================================================================================================================

\newenvironment{tex}{ %
}
{%
}

\newenvironment{indente}{ %
	\setlength\parindent{10mm}
}

{
	\setlength\parindent{0mm}
}

\newenvironment{corrige}{%
     \needspace{3\baselineskip}
     \medskip
     \textbf{\textsc{Corrigé}}
     \medskip
}
{
}

\newenvironment{extern}{%
     \begin{center}
     }
     {
     \end{center}
}

\NewEnviron{code}{%
	\par
     \boite{gray}{\texttt{%
     \BODY
     }}
     \par
}

\newenvironment{vbloc}{% boite sans cadre empeche saut de page
     \begin{minipage}[t]{\linewidth}
     }
     {
     \end{minipage}
}
\NewEnviron{h2}{%
    \needspace{3\baselineskip}
    \vspace{0.6cm}
	\noindent \MakeUppercase{\sffamily \large \BODY}
	\vspace{1mm}\textcolor{mcgris}{\hrule}\vspace{0.4cm}
	\par
}{}

\NewEnviron{h3}{%
    \needspace{3\baselineskip}
	\vspace{5mm}
	\textsc{\BODY}
	\par
}

\NewEnviron{margeneg}{ %
\begin{addmargin}[-1cm]{0cm}
\BODY
\end{addmargin}
}

\NewEnviron{html}{%
}

\begin{document}
\meta{url}{/exercices/probabilites-bac-es-l-metropole-2018/}
\meta{pid}{9223}
\meta{titre}{Probabilités - Bac ES/L Métropole 2018}
\meta{type}{exercices}
%
\begin{h2}Exercice 1 (6 points)\end{h2}
\textbf{Commun à  tous les candidats}
\medbreak
\emph{Les parties \emph{A} et \emph{B} sont indépendantes.}
\TitreC{Partie A}
Le temps passé par un client, en minute, dans un supermarché peut être modélisé par une variable aléatoire $X$ suivant la loi normale d'espérance $\mu=45$ et d'écart-type $\sigma=12$.
\smallbreak
\emph{Pour tout événement $E$, on note $p(E)$ sa probabilité.}
\begin{enumerate}
     \item Déterminer, en justifiant~:
     \begin{enumerate}[label=\alph*.]
          \item $p(X=10)$
          \item $p(X\geqslant 45)$
          \item $p(21 \leqslant X \leqslant 69)$
          \item $p(21 \leqslant X \leqslant 45)$
     \end{enumerate}
     \item Calculer la probabilité, arrondie au millième, qu'un client passe entre 30 et 60 minutes dans ce supermarché.
     \item Déterminer la valeur de $a$, arrondie à l'unité, telle que $P(X\leqslant a)=0,30$. Interpréter la valeur de $a$ dans le contexte de l'énoncé.
\end{enumerate}
\TitreC{Partie B}
En 2013, une étude a montré que 89\,\% des clients étaient satisfaits des produits de ce supermarché.
\begin{enumerate}
     \item Déterminer un intervalle de fluctuation au seuil de 95\,\% de la proportion de clients satisfaits pour un échantillon de 300 clients pris au hasard en 2013.
     \begin{margeneg}
          Lors d'une enquête réalisée en 2018 auprès de 300 clients choisis au hasard, 280 ont déclaré être satisfaits.
     \end{margeneg}
     \item Calculer la fréquence de clients satisfaits dans l'enquête réalisée en 2018.
     \item Peut-on affirmer, au seuil de 95\,\%, que le taux de satisfaction des clients est resté stable entre 2013 et 2018~? Justifier.
\end{enumerate}

\end{document}
µ
\documentclass[a4paper]{article}

%================================================================================================================================
%
% Packages
%
%================================================================================================================================

\usepackage[T1]{fontenc} 	% pour caractères accentués
\usepackage[utf8]{inputenc}  % encodage utf8
\usepackage[french]{babel}	% langue : français
\usepackage{fourier}			% caractères plus lisibles
\usepackage[dvipsnames]{xcolor} % couleurs
\usepackage{fancyhdr}		% réglage header footer
\usepackage{needspace}		% empêcher sauts de page mal placés
\usepackage{graphicx}		% pour inclure des graphiques
\usepackage{enumitem,cprotect}		% personnalise les listes d'items (nécessaire pour ol, al ...)
\usepackage{hyperref}		% Liens hypertexte
\usepackage{pstricks,pst-all,pst-node,pstricks-add,pst-math,pst-plot,pst-tree,pst-eucl} % pstricks
\usepackage[a4paper,includeheadfoot,top=2cm,left=3cm, bottom=2cm,right=3cm]{geometry} % marges etc.
\usepackage{comment}			% commentaires multilignes
\usepackage{amsmath,environ} % maths (matrices, etc.)
\usepackage{amssymb,makeidx}
\usepackage{bm}				% bold maths
\usepackage{tabularx}		% tableaux
\usepackage{colortbl}		% tableaux en couleur
\usepackage{fontawesome}		% Fontawesome
\usepackage{environ}			% environment with command
\usepackage{fp}				% calculs pour ps-tricks
\usepackage{multido}			% pour ps tricks
\usepackage[np]{numprint}	% formattage nombre
\usepackage{tikz,tkz-tab} 			% package principal TikZ
\usepackage{pgfplots}   % axes
\usepackage{mathrsfs}    % cursives
\usepackage{calc}			% calcul taille boites
\usepackage[scaled=0.875]{helvet} % font sans serif
\usepackage{svg} % svg
\usepackage{scrextend} % local margin
\usepackage{scratch} %scratch
\usepackage{multicol} % colonnes
%\usepackage{infix-RPN,pst-func} % formule en notation polanaise inversée
\usepackage{listings}

%================================================================================================================================
%
% Réglages de base
%
%================================================================================================================================

\lstset{
language=Python,   % R code
literate=
{á}{{\'a}}1
{à}{{\`a}}1
{ã}{{\~a}}1
{é}{{\'e}}1
{è}{{\`e}}1
{ê}{{\^e}}1
{í}{{\'i}}1
{ó}{{\'o}}1
{õ}{{\~o}}1
{ú}{{\'u}}1
{ü}{{\"u}}1
{ç}{{\c{c}}}1
{~}{{ }}1
}


\definecolor{codegreen}{rgb}{0,0.6,0}
\definecolor{codegray}{rgb}{0.5,0.5,0.5}
\definecolor{codepurple}{rgb}{0.58,0,0.82}
\definecolor{backcolour}{rgb}{0.95,0.95,0.92}

\lstdefinestyle{mystyle}{
    backgroundcolor=\color{backcolour},   
    commentstyle=\color{codegreen},
    keywordstyle=\color{magenta},
    numberstyle=\tiny\color{codegray},
    stringstyle=\color{codepurple},
    basicstyle=\ttfamily\footnotesize,
    breakatwhitespace=false,         
    breaklines=true,                 
    captionpos=b,                    
    keepspaces=true,                 
    numbers=left,                    
xleftmargin=2em,
framexleftmargin=2em,            
    showspaces=false,                
    showstringspaces=false,
    showtabs=false,                  
    tabsize=2,
    upquote=true
}

\lstset{style=mystyle}


\lstset{style=mystyle}
\newcommand{\imgdir}{C:/laragon/www/newmc/assets/imgsvg/}
\newcommand{\imgsvgdir}{C:/laragon/www/newmc/assets/imgsvg/}

\definecolor{mcgris}{RGB}{220, 220, 220}% ancien~; pour compatibilité
\definecolor{mcbleu}{RGB}{52, 152, 219}
\definecolor{mcvert}{RGB}{125, 194, 70}
\definecolor{mcmauve}{RGB}{154, 0, 215}
\definecolor{mcorange}{RGB}{255, 96, 0}
\definecolor{mcturquoise}{RGB}{0, 153, 153}
\definecolor{mcrouge}{RGB}{255, 0, 0}
\definecolor{mclightvert}{RGB}{205, 234, 190}

\definecolor{gris}{RGB}{220, 220, 220}
\definecolor{bleu}{RGB}{52, 152, 219}
\definecolor{vert}{RGB}{125, 194, 70}
\definecolor{mauve}{RGB}{154, 0, 215}
\definecolor{orange}{RGB}{255, 96, 0}
\definecolor{turquoise}{RGB}{0, 153, 153}
\definecolor{rouge}{RGB}{255, 0, 0}
\definecolor{lightvert}{RGB}{205, 234, 190}
\setitemize[0]{label=\color{lightvert}  $\bullet$}

\pagestyle{fancy}
\renewcommand{\headrulewidth}{0.2pt}
\fancyhead[L]{maths-cours.fr}
\fancyhead[R]{\thepage}
\renewcommand{\footrulewidth}{0.2pt}
\fancyfoot[C]{}

\newcolumntype{C}{>{\centering\arraybackslash}X}
\newcolumntype{s}{>{\hsize=.35\hsize\arraybackslash}X}

\setlength{\parindent}{0pt}		 
\setlength{\parskip}{3mm}
\setlength{\headheight}{1cm}

\def\ebook{ebook}
\def\book{book}
\def\web{web}
\def\type{web}

\newcommand{\vect}[1]{\overrightarrow{\,\mathstrut#1\,}}

\def\Oij{$\left(\text{O}~;~\vect{\imath},~\vect{\jmath}\right)$}
\def\Oijk{$\left(\text{O}~;~\vect{\imath},~\vect{\jmath},~\vect{k}\right)$}
\def\Ouv{$\left(\text{O}~;~\vect{u},~\vect{v}\right)$}

\hypersetup{breaklinks=true, colorlinks = true, linkcolor = OliveGreen, urlcolor = OliveGreen, citecolor = OliveGreen, pdfauthor={Didier BONNEL - https://www.maths-cours.fr} } % supprime les bordures autour des liens

\renewcommand{\arg}[0]{\text{arg}}

\everymath{\displaystyle}

%================================================================================================================================
%
% Macros - Commandes
%
%================================================================================================================================

\newcommand\meta[2]{    			% Utilisé pour créer le post HTML.
	\def\titre{titre}
	\def\url{url}
	\def\arg{#1}
	\ifx\titre\arg
		\newcommand\maintitle{#2}
		\fancyhead[L]{#2}
		{\Large\sffamily \MakeUppercase{#2}}
		\vspace{1mm}\textcolor{mcvert}{\hrule}
	\fi 
	\ifx\url\arg
		\fancyfoot[L]{\href{https://www.maths-cours.fr#2}{\black \footnotesize{https://www.maths-cours.fr#2}}}
	\fi 
}


\newcommand\TitreC[1]{    		% Titre centré
     \needspace{3\baselineskip}
     \begin{center}\textbf{#1}\end{center}
}

\newcommand\newpar{    		% paragraphe
     \par
}

\newcommand\nosp {    		% commande vide (pas d'espace)
}
\newcommand{\id}[1]{} %ignore

\newcommand\boite[2]{				% Boite simple sans titre
	\vspace{5mm}
	\setlength{\fboxrule}{0.2mm}
	\setlength{\fboxsep}{5mm}	
	\fcolorbox{#1}{#1!3}{\makebox[\linewidth-2\fboxrule-2\fboxsep]{
  		\begin{minipage}[t]{\linewidth-2\fboxrule-4\fboxsep}\setlength{\parskip}{3mm}
  			 #2
  		\end{minipage}
	}}
	\vspace{5mm}
}

\newcommand\CBox[4]{				% Boites
	\vspace{5mm}
	\setlength{\fboxrule}{0.2mm}
	\setlength{\fboxsep}{5mm}
	
	\fcolorbox{#1}{#1!3}{\makebox[\linewidth-2\fboxrule-2\fboxsep]{
		\begin{minipage}[t]{1cm}\setlength{\parskip}{3mm}
	  		\textcolor{#1}{\LARGE{#2}}    
 	 	\end{minipage}  
  		\begin{minipage}[t]{\linewidth-2\fboxrule-4\fboxsep}\setlength{\parskip}{3mm}
			\raisebox{1.2mm}{\normalsize\sffamily{\textcolor{#1}{#3}}}						
  			 #4
  		\end{minipage}
	}}
	\vspace{5mm}
}

\newcommand\cadre[3]{				% Boites convertible html
	\par
	\vspace{2mm}
	\setlength{\fboxrule}{0.1mm}
	\setlength{\fboxsep}{5mm}
	\fcolorbox{#1}{white}{\makebox[\linewidth-2\fboxrule-2\fboxsep]{
  		\begin{minipage}[t]{\linewidth-2\fboxrule-4\fboxsep}\setlength{\parskip}{3mm}
			\raisebox{-2.5mm}{\sffamily \small{\textcolor{#1}{\MakeUppercase{#2}}}}		
			\par		
  			 #3
 	 		\end{minipage}
	}}
		\vspace{2mm}
	\par
}

\newcommand\bloc[3]{				% Boites convertible html sans bordure
     \needspace{2\baselineskip}
     {\sffamily \small{\textcolor{#1}{\MakeUppercase{#2}}}}    
		\par		
  			 #3
		\par
}

\newcommand\CHelp[1]{
     \CBox{Plum}{\faInfoCircle}{À RETENIR}{#1}
}

\newcommand\CUp[1]{
     \CBox{NavyBlue}{\faThumbsOUp}{EN PRATIQUE}{#1}
}

\newcommand\CInfo[1]{
     \CBox{Sepia}{\faArrowCircleRight}{REMARQUE}{#1}
}

\newcommand\CRedac[1]{
     \CBox{PineGreen}{\faEdit}{BIEN R\'EDIGER}{#1}
}

\newcommand\CError[1]{
     \CBox{Red}{\faExclamationTriangle}{ATTENTION}{#1}
}

\newcommand\TitreExo[2]{
\needspace{4\baselineskip}
 {\sffamily\large EXERCICE #1\ (\emph{#2 points})}
\vspace{5mm}
}

\newcommand\img[2]{
          \includegraphics[width=#2\paperwidth]{\imgdir#1}
}

\newcommand\imgsvg[2]{
       \begin{center}   \includegraphics[width=#2\paperwidth]{\imgsvgdir#1} \end{center}
}


\newcommand\Lien[2]{
     \href{#1}{#2 \tiny \faExternalLink}
}
\newcommand\mcLien[2]{
     \href{https~://www.maths-cours.fr/#1}{#2 \tiny \faExternalLink}
}

\newcommand{\euro}{\eurologo{}}

%================================================================================================================================
%
% Macros - Environement
%
%================================================================================================================================

\newenvironment{tex}{ %
}
{%
}

\newenvironment{indente}{ %
	\setlength\parindent{10mm}
}

{
	\setlength\parindent{0mm}
}

\newenvironment{corrige}{%
     \needspace{3\baselineskip}
     \medskip
     \textbf{\textsc{Corrigé}}
     \medskip
}
{
}

\newenvironment{extern}{%
     \begin{center}
     }
     {
     \end{center}
}

\NewEnviron{code}{%
	\par
     \boite{gray}{\texttt{%
     \BODY
     }}
     \par
}

\newenvironment{vbloc}{% boite sans cadre empeche saut de page
     \begin{minipage}[t]{\linewidth}
     }
     {
     \end{minipage}
}
\NewEnviron{h2}{%
    \needspace{3\baselineskip}
    \vspace{0.6cm}
	\noindent \MakeUppercase{\sffamily \large \BODY}
	\vspace{1mm}\textcolor{mcgris}{\hrule}\vspace{0.4cm}
	\par
}{}

\NewEnviron{h3}{%
    \needspace{3\baselineskip}
	\vspace{5mm}
	\textsc{\BODY}
	\par
}

\NewEnviron{margeneg}{ %
\begin{addmargin}[-1cm]{0cm}
\BODY
\end{addmargin}
}

\NewEnviron{html}{%
}

\begin{document}
\meta{url}{/exercices/qcm-bac-es-l-metropole-2018/}
\meta{pid}{9225}
\meta{titre}{QCM - Bac ES/L Métropole 2018}
\meta{type}{exercices}
%
\begin{h2}Exercice 2 (4 points)\end{h2}
\textbf{Commun à  tous les candidats}
\medbreak
\emph{Cet exercice est un questionnaire à choix multiples. Pour chaque question, une seule des quatre réponses proposées est correcte.\\
     Reporter sur la copie le numéro de la question ainsi que la lettre correspondant à la réponse choisie.\\
     Une réponse exacte rapporte 1 point. Une réponse fausse, une réponse multiple ou l'absence de réponse ne rappoete ni n'enlève aucun point. Aucune justification n'est demandée.\\
Les partie \emph{A} et \emph{B} sont indépendantes.}
\TitreC{Partie A}
Dans un établissement scolaire, 30\,\% des élèves sont inscrits dans un club de sport, et parmi eux, 40\,\% sont des filles. Parmi ceux n'étant pas inscrits dans un club de sport, 50\,\% sont des garçons.
\smallbreak
\emph{Pour tout événement $E$, on note $\overline{E}$ l'événement contraire de $E$ et $p(E)$ sa probabilité. Pour tout événement $F$ de probabilité non nulle, on note $P_F(E)$ la probabilité de $E$ sachant que $F$ est réalisé.}
\medbreak
On interroge un élève au hasard et on considère les événements suivants~:
\begin{itemize}
     \item $S$~: \og l'élève est inscrit dans un club de sport \fg{}
     \item $F$~: \og l'élève est une fille \fg{}
\end{itemize}
La situation est représentée par l'arbre pondéré ci-dessous :
%:-+-+-+- Engendré par : http://math.et.info.free.fr/TikZ/Arbre/
\begin{center}
     \begin{extern}%width="320"
          % Racine à Gauche, développement vers la droite
          \begin{tikzpicture}[xscale=1,yscale=1]
               % Styles (MODIFIABLES)
               \tikzstyle{fleche}=[-,>=latex,thick]
               \tikzstyle{noeud}=[circle,draw]
               \tikzstyle{feuille}=[circle,draw]
               \tikzstyle{etiquette}=[midway,fill=white]
               % Dimensions (MODIFIABLES)
               \def\DistanceInterNiveaux{3}
               \def\DistanceInterFeuilles{2}
               % Dimensions calculées (NON MODIFIABLES)
               \def\NiveauA{(0)*\DistanceInterNiveaux}
               \def\NiveauB{(1.5)*\DistanceInterNiveaux}
               \def\NiveauC{(2.5)*\DistanceInterNiveaux}
               \def\InterFeuilles{(-1)*\DistanceInterFeuilles}
               % Noeuds (MODIFIABLES : Styles et Coefficients d'InterFeuilles)
               \node[noeud] (R) at ({\NiveauA},{(1.5)*\InterFeuilles}) {$ $};
               \node[noeud] (Ra) at ({\NiveauB},{(0.5)*\InterFeuilles}) {$S$};
               \node[feuille] (Raa) at ({\NiveauC},{(0)*\InterFeuilles}) {$F$};
               \node[feuille] (Rab) at ({\NiveauC},{(1)*\InterFeuilles}) {$\overline{F}$};
               \node[noeud] (Rb) at ({\NiveauB},{(2.5)*\InterFeuilles}) {$\overline{S}$};
               \node[feuille] (Rba) at ({\NiveauC},{(2)*\InterFeuilles}) {$F$};
               \node[feuille] (Rbb) at ({\NiveauC},{(3)*\InterFeuilles}) {$\overline{F}$};
               % Arcs (MODIFIABLES : Styles)
               \draw[fleche] (R)--(Ra) node[etiquette] {$0,3$};
               \draw[fleche] (Ra)--(Raa) node[etiquette] {$0,4$};
               \draw[fleche] (Ra)--(Rab) ;
               \draw[fleche] (R)--(Rb) ;
               \draw[fleche] (Rb)--(Rba) ;
               \draw[fleche] (Rb)--(Rbb) node[etiquette] {$0,5$};
          \end{tikzpicture}
     \end{extern}
\end{center}
\begin{enumerate}
     \item La probabilité $p_{\overline{F}}(S)$ est la probabilité que l'élève soit~:
     \begin{enumerate}[label=\alph*.]
          \item inscrit dans un club de sport sachant que c'est un garçon~;
          \item un garçon inscrit dans un club de sport~;
          \item inscrit dans un club de sport ou un garçon~;
          \item un garçon sachant qu'il est inscrit dans un club de sport.
     \end{enumerate}
     \item On admet que $P(F)=0,47$. La valeur arrondie de $P_{F}(S)$ est~:
     \begin{tabularx}{\linewidth}{XX}%class="cel50 noborder"
          \textbf{a.}~~$0,141$ & \textbf{b.}~~$0,255$ \\
          \textbf{c.}~~$0,400$ & \textbf{d.}~~$0,638$
     \end{tabularx}
     \medbreak
\end{enumerate}
\TitreC{Partie B}
Soit $g$ la fonction définie sur $[-1~;~4]$ par $g(x) = - x^3 + 3x^2 - 1$ et $\mathscr{C}_g$ sa courbe représentative dans un repère.
\begin{enumerate}
     \item La tangente à la courbe $\mathscr{C}_g$ au point d'abscisse 1 a pour équation~:
     \begin{center}
          \begin{tabularx}{\linewidth}{XX}%class="cel50 noborder"
               \textbf{a.}~~$y = - 3x^2 + 6x$ & \textbf{b.}~~$y= 3x-2$ \\
               \textbf{c.}~~$y= 3x - 3$ & \textbf{d.}~~$y = 2x - 1$
          \end{tabularx}
     \end{center}
     \item La valeur moyenne de la fonction $g$ sur l'intervalle $[-1~; a]$ est nulle pour~:
     \begin{center}
          \begin{tabularx}{\linewidth}{XX}%class="cel50 noborder"
               \textbf{a.}~~$a=0$ & \textbf{b.}~~$a=1$\\
               \textbf{c.}~~$a=2$ & \textbf{d.}~~$a=3$
          \end{tabularx}
     \end{center}
\end{enumerate}

\end{document}
µ
\documentclass[a4paper]{article}

%================================================================================================================================
%
% Packages
%
%================================================================================================================================

\usepackage[T1]{fontenc} 	% pour caractères accentués
\usepackage[utf8]{inputenc}  % encodage utf8
\usepackage[french]{babel}	% langue : français
\usepackage{fourier}			% caractères plus lisibles
\usepackage[dvipsnames]{xcolor} % couleurs
\usepackage{fancyhdr}		% réglage header footer
\usepackage{needspace}		% empêcher sauts de page mal placés
\usepackage{graphicx}		% pour inclure des graphiques
\usepackage{enumitem,cprotect}		% personnalise les listes d'items (nécessaire pour ol, al ...)
\usepackage{hyperref}		% Liens hypertexte
\usepackage{pstricks,pst-all,pst-node,pstricks-add,pst-math,pst-plot,pst-tree,pst-eucl} % pstricks
\usepackage[a4paper,includeheadfoot,top=2cm,left=3cm, bottom=2cm,right=3cm]{geometry} % marges etc.
\usepackage{comment}			% commentaires multilignes
\usepackage{amsmath,environ} % maths (matrices, etc.)
\usepackage{amssymb,makeidx}
\usepackage{bm}				% bold maths
\usepackage{tabularx}		% tableaux
\usepackage{colortbl}		% tableaux en couleur
\usepackage{fontawesome}		% Fontawesome
\usepackage{environ}			% environment with command
\usepackage{fp}				% calculs pour ps-tricks
\usepackage{multido}			% pour ps tricks
\usepackage[np]{numprint}	% formattage nombre
\usepackage{tikz,tkz-tab} 			% package principal TikZ
\usepackage{pgfplots}   % axes
\usepackage{mathrsfs}    % cursives
\usepackage{calc}			% calcul taille boites
\usepackage[scaled=0.875]{helvet} % font sans serif
\usepackage{svg} % svg
\usepackage{scrextend} % local margin
\usepackage{scratch} %scratch
\usepackage{multicol} % colonnes
%\usepackage{infix-RPN,pst-func} % formule en notation polanaise inversée
\usepackage{listings}

%================================================================================================================================
%
% Réglages de base
%
%================================================================================================================================

\lstset{
language=Python,   % R code
literate=
{á}{{\'a}}1
{à}{{\`a}}1
{ã}{{\~a}}1
{é}{{\'e}}1
{è}{{\`e}}1
{ê}{{\^e}}1
{í}{{\'i}}1
{ó}{{\'o}}1
{õ}{{\~o}}1
{ú}{{\'u}}1
{ü}{{\"u}}1
{ç}{{\c{c}}}1
{~}{{ }}1
}


\definecolor{codegreen}{rgb}{0,0.6,0}
\definecolor{codegray}{rgb}{0.5,0.5,0.5}
\definecolor{codepurple}{rgb}{0.58,0,0.82}
\definecolor{backcolour}{rgb}{0.95,0.95,0.92}

\lstdefinestyle{mystyle}{
    backgroundcolor=\color{backcolour},   
    commentstyle=\color{codegreen},
    keywordstyle=\color{magenta},
    numberstyle=\tiny\color{codegray},
    stringstyle=\color{codepurple},
    basicstyle=\ttfamily\footnotesize,
    breakatwhitespace=false,         
    breaklines=true,                 
    captionpos=b,                    
    keepspaces=true,                 
    numbers=left,                    
xleftmargin=2em,
framexleftmargin=2em,            
    showspaces=false,                
    showstringspaces=false,
    showtabs=false,                  
    tabsize=2,
    upquote=true
}

\lstset{style=mystyle}


\lstset{style=mystyle}
\newcommand{\imgdir}{C:/laragon/www/newmc/assets/imgsvg/}
\newcommand{\imgsvgdir}{C:/laragon/www/newmc/assets/imgsvg/}

\definecolor{mcgris}{RGB}{220, 220, 220}% ancien~; pour compatibilité
\definecolor{mcbleu}{RGB}{52, 152, 219}
\definecolor{mcvert}{RGB}{125, 194, 70}
\definecolor{mcmauve}{RGB}{154, 0, 215}
\definecolor{mcorange}{RGB}{255, 96, 0}
\definecolor{mcturquoise}{RGB}{0, 153, 153}
\definecolor{mcrouge}{RGB}{255, 0, 0}
\definecolor{mclightvert}{RGB}{205, 234, 190}

\definecolor{gris}{RGB}{220, 220, 220}
\definecolor{bleu}{RGB}{52, 152, 219}
\definecolor{vert}{RGB}{125, 194, 70}
\definecolor{mauve}{RGB}{154, 0, 215}
\definecolor{orange}{RGB}{255, 96, 0}
\definecolor{turquoise}{RGB}{0, 153, 153}
\definecolor{rouge}{RGB}{255, 0, 0}
\definecolor{lightvert}{RGB}{205, 234, 190}
\setitemize[0]{label=\color{lightvert}  $\bullet$}

\pagestyle{fancy}
\renewcommand{\headrulewidth}{0.2pt}
\fancyhead[L]{maths-cours.fr}
\fancyhead[R]{\thepage}
\renewcommand{\footrulewidth}{0.2pt}
\fancyfoot[C]{}

\newcolumntype{C}{>{\centering\arraybackslash}X}
\newcolumntype{s}{>{\hsize=.35\hsize\arraybackslash}X}

\setlength{\parindent}{0pt}		 
\setlength{\parskip}{3mm}
\setlength{\headheight}{1cm}

\def\ebook{ebook}
\def\book{book}
\def\web{web}
\def\type{web}

\newcommand{\vect}[1]{\overrightarrow{\,\mathstrut#1\,}}

\def\Oij{$\left(\text{O}~;~\vect{\imath},~\vect{\jmath}\right)$}
\def\Oijk{$\left(\text{O}~;~\vect{\imath},~\vect{\jmath},~\vect{k}\right)$}
\def\Ouv{$\left(\text{O}~;~\vect{u},~\vect{v}\right)$}

\hypersetup{breaklinks=true, colorlinks = true, linkcolor = OliveGreen, urlcolor = OliveGreen, citecolor = OliveGreen, pdfauthor={Didier BONNEL - https://www.maths-cours.fr} } % supprime les bordures autour des liens

\renewcommand{\arg}[0]{\text{arg}}

\everymath{\displaystyle}

%================================================================================================================================
%
% Macros - Commandes
%
%================================================================================================================================

\newcommand\meta[2]{    			% Utilisé pour créer le post HTML.
	\def\titre{titre}
	\def\url{url}
	\def\arg{#1}
	\ifx\titre\arg
		\newcommand\maintitle{#2}
		\fancyhead[L]{#2}
		{\Large\sffamily \MakeUppercase{#2}}
		\vspace{1mm}\textcolor{mcvert}{\hrule}
	\fi 
	\ifx\url\arg
		\fancyfoot[L]{\href{https://www.maths-cours.fr#2}{\black \footnotesize{https://www.maths-cours.fr#2}}}
	\fi 
}


\newcommand\TitreC[1]{    		% Titre centré
     \needspace{3\baselineskip}
     \begin{center}\textbf{#1}\end{center}
}

\newcommand\newpar{    		% paragraphe
     \par
}

\newcommand\nosp {    		% commande vide (pas d'espace)
}
\newcommand{\id}[1]{} %ignore

\newcommand\boite[2]{				% Boite simple sans titre
	\vspace{5mm}
	\setlength{\fboxrule}{0.2mm}
	\setlength{\fboxsep}{5mm}	
	\fcolorbox{#1}{#1!3}{\makebox[\linewidth-2\fboxrule-2\fboxsep]{
  		\begin{minipage}[t]{\linewidth-2\fboxrule-4\fboxsep}\setlength{\parskip}{3mm}
  			 #2
  		\end{minipage}
	}}
	\vspace{5mm}
}

\newcommand\CBox[4]{				% Boites
	\vspace{5mm}
	\setlength{\fboxrule}{0.2mm}
	\setlength{\fboxsep}{5mm}
	
	\fcolorbox{#1}{#1!3}{\makebox[\linewidth-2\fboxrule-2\fboxsep]{
		\begin{minipage}[t]{1cm}\setlength{\parskip}{3mm}
	  		\textcolor{#1}{\LARGE{#2}}    
 	 	\end{minipage}  
  		\begin{minipage}[t]{\linewidth-2\fboxrule-4\fboxsep}\setlength{\parskip}{3mm}
			\raisebox{1.2mm}{\normalsize\sffamily{\textcolor{#1}{#3}}}						
  			 #4
  		\end{minipage}
	}}
	\vspace{5mm}
}

\newcommand\cadre[3]{				% Boites convertible html
	\par
	\vspace{2mm}
	\setlength{\fboxrule}{0.1mm}
	\setlength{\fboxsep}{5mm}
	\fcolorbox{#1}{white}{\makebox[\linewidth-2\fboxrule-2\fboxsep]{
  		\begin{minipage}[t]{\linewidth-2\fboxrule-4\fboxsep}\setlength{\parskip}{3mm}
			\raisebox{-2.5mm}{\sffamily \small{\textcolor{#1}{\MakeUppercase{#2}}}}		
			\par		
  			 #3
 	 		\end{minipage}
	}}
		\vspace{2mm}
	\par
}

\newcommand\bloc[3]{				% Boites convertible html sans bordure
     \needspace{2\baselineskip}
     {\sffamily \small{\textcolor{#1}{\MakeUppercase{#2}}}}    
		\par		
  			 #3
		\par
}

\newcommand\CHelp[1]{
     \CBox{Plum}{\faInfoCircle}{À RETENIR}{#1}
}

\newcommand\CUp[1]{
     \CBox{NavyBlue}{\faThumbsOUp}{EN PRATIQUE}{#1}
}

\newcommand\CInfo[1]{
     \CBox{Sepia}{\faArrowCircleRight}{REMARQUE}{#1}
}

\newcommand\CRedac[1]{
     \CBox{PineGreen}{\faEdit}{BIEN R\'EDIGER}{#1}
}

\newcommand\CError[1]{
     \CBox{Red}{\faExclamationTriangle}{ATTENTION}{#1}
}

\newcommand\TitreExo[2]{
\needspace{4\baselineskip}
 {\sffamily\large EXERCICE #1\ (\emph{#2 points})}
\vspace{5mm}
}

\newcommand\img[2]{
          \includegraphics[width=#2\paperwidth]{\imgdir#1}
}

\newcommand\imgsvg[2]{
       \begin{center}   \includegraphics[width=#2\paperwidth]{\imgsvgdir#1} \end{center}
}


\newcommand\Lien[2]{
     \href{#1}{#2 \tiny \faExternalLink}
}
\newcommand\mcLien[2]{
     \href{https~://www.maths-cours.fr/#1}{#2 \tiny \faExternalLink}
}

\newcommand{\euro}{\eurologo{}}

%================================================================================================================================
%
% Macros - Environement
%
%================================================================================================================================

\newenvironment{tex}{ %
}
{%
}

\newenvironment{indente}{ %
	\setlength\parindent{10mm}
}

{
	\setlength\parindent{0mm}
}

\newenvironment{corrige}{%
     \needspace{3\baselineskip}
     \medskip
     \textbf{\textsc{Corrigé}}
     \medskip
}
{
}

\newenvironment{extern}{%
     \begin{center}
     }
     {
     \end{center}
}

\NewEnviron{code}{%
	\par
     \boite{gray}{\texttt{%
     \BODY
     }}
     \par
}

\newenvironment{vbloc}{% boite sans cadre empeche saut de page
     \begin{minipage}[t]{\linewidth}
     }
     {
     \end{minipage}
}
\NewEnviron{h2}{%
    \needspace{3\baselineskip}
    \vspace{0.6cm}
	\noindent \MakeUppercase{\sffamily \large \BODY}
	\vspace{1mm}\textcolor{mcgris}{\hrule}\vspace{0.4cm}
	\par
}{}

\NewEnviron{h3}{%
    \needspace{3\baselineskip}
	\vspace{5mm}
	\textsc{\BODY}
	\par
}

\NewEnviron{margeneg}{ %
\begin{addmargin}[-1cm]{0cm}
\BODY
\end{addmargin}
}

\NewEnviron{html}{%
}

\begin{document}
\meta{url}{/exercices/suites-bac-es-l-metropole-2018/}
\meta{pid}{9227}
\meta{titre}{Suites - Bac ES/L Métropole 2018}
\meta{type}{exercices}
%
\begin{h2}Exercice 3 (5 points)\end{h2}
\textbf{Candidats de ES n'ayant pas choisi l'enseignement de spécialité et candidats de L}
\medbreak
Un lac de montagne est alimenté par une rivière et régulé par un barrage, situé en aval, d'une hauteur de 10~m.
\par
On mesure le niveau de l'eau chaque jour à midi.
\par
Le 1${^\text{er}}$ janvier 2018, à midi, le niveau du lac était de $6,05$~m.
\smallbreak
Entre deux mesures successives, le niveau d'eau du lac évolue de la façon suivante~:
\begin{itemize}[label=---]
     \item d'abord une augmentation de 6\,\% (apport de la rivière)~;
     \item ensuite une baisse de 15~cm (écoulement à travers le barrage).
\end{itemize}
\begin{enumerate}
     \item On modélise l'évolution du niveau d'eau du lac par une suite $(u_n)_{n\in \mathbb{N}}$, le terme $u_n$ représentant le niveau d'eau du lac à midi, en cm, $n$ jours après le 1${^\text{er}}$ janvier 2018.
     \par
     Ainsi le niveau d'eau du lac, en cm, le 1${^\text{er}}$ janvier 2018 est donné par $u_0=605$.
     \begin{enumerate}[label=\alph*.]
          \item Calculer le niveau du lac, en cm, le 2 janvier 2018 à midi.
          \item Montrer que, pour tout $n\in\mathbb{N}$, $u_{n+1}=1,06 u_n - 15$.
     \end{enumerate}
     \item On pose, pour tout $n\in \mathbb{N}$, $v_n=u_n-250$.
     \begin{enumerate}[label=\alph*.]
          \item Montrer que la suite $(v_n)$ est géométrique de raison $1,06$.
          \par
          Préciser son terme initial.
          \item Montrer que, pour tout $n\in\mathbb{N}$, $u_n=355\times 1,06^{n}+250$.
     \end{enumerate}
     \item Lorsque le niveau du lac dépasse 10~m, l'équipe d'entretien doit agrandir l'ouverture des vannes du barrage.
     \begin{enumerate}[label=\alph*.]
          \item Déterminer la limite de la suite $(u_n)$.
          \item L'équipe d'entretien devra-t-elle ouvrir les vannes afin de réguler le niveau d'eau~? Justifier la réponse.
     \end{enumerate}
     \item Afin de déterminer la première date d'intervention des techniciens, on souhaite utiliser l'algorithme incomplet ci-dessous.
     \begin{center}
          \begin{extern}%width="240" alt=""
               \renewcommand{\arraystretch}{1.2}
               \begin{tabular}{|p{5cm}|}
                    \hline
                    $N \leftarrow 0$\\
                    $U \leftarrow 605$\\
                    Tant que ..................... faire\\
                    \hspace{1cm} $U \leftarrow ....................$\\
                    \hspace{1cm} $N\leftarrow N+1$\\
                    Fin Tant que\\
                    \hline
               \end{tabular}
          \end{extern}
     \end{center}
     \begin{enumerate}[label=\alph*.]
          \item Recopier et compléter l'algorithme.
          \item À la fin de l'exécution de l'algorithme, que contient la variable $N$~?
          \item En déduire la première date d'intervention des techniciens sur les vannes du barrage.
     \end{enumerate}
\end{enumerate}

\end{document}
µ
\documentclass[a4paper]{article}

%================================================================================================================================
%
% Packages
%
%================================================================================================================================

\usepackage[T1]{fontenc} 	% pour caractères accentués
\usepackage[utf8]{inputenc}  % encodage utf8
\usepackage[french]{babel}	% langue : français
\usepackage{fourier}			% caractères plus lisibles
\usepackage[dvipsnames]{xcolor} % couleurs
\usepackage{fancyhdr}		% réglage header footer
\usepackage{needspace}		% empêcher sauts de page mal placés
\usepackage{graphicx}		% pour inclure des graphiques
\usepackage{enumitem,cprotect}		% personnalise les listes d'items (nécessaire pour ol, al ...)
\usepackage{hyperref}		% Liens hypertexte
\usepackage{pstricks,pst-all,pst-node,pstricks-add,pst-math,pst-plot,pst-tree,pst-eucl} % pstricks
\usepackage[a4paper,includeheadfoot,top=2cm,left=3cm, bottom=2cm,right=3cm]{geometry} % marges etc.
\usepackage{comment}			% commentaires multilignes
\usepackage{amsmath,environ} % maths (matrices, etc.)
\usepackage{amssymb,makeidx}
\usepackage{bm}				% bold maths
\usepackage{tabularx}		% tableaux
\usepackage{colortbl}		% tableaux en couleur
\usepackage{fontawesome}		% Fontawesome
\usepackage{environ}			% environment with command
\usepackage{fp}				% calculs pour ps-tricks
\usepackage{multido}			% pour ps tricks
\usepackage[np]{numprint}	% formattage nombre
\usepackage{tikz,tkz-tab} 			% package principal TikZ
\usepackage{pgfplots}   % axes
\usepackage{mathrsfs}    % cursives
\usepackage{calc}			% calcul taille boites
\usepackage[scaled=0.875]{helvet} % font sans serif
\usepackage{svg} % svg
\usepackage{scrextend} % local margin
\usepackage{scratch} %scratch
\usepackage{multicol} % colonnes
%\usepackage{infix-RPN,pst-func} % formule en notation polanaise inversée
\usepackage{listings}

%================================================================================================================================
%
% Réglages de base
%
%================================================================================================================================

\lstset{
language=Python,   % R code
literate=
{á}{{\'a}}1
{à}{{\`a}}1
{ã}{{\~a}}1
{é}{{\'e}}1
{è}{{\`e}}1
{ê}{{\^e}}1
{í}{{\'i}}1
{ó}{{\'o}}1
{õ}{{\~o}}1
{ú}{{\'u}}1
{ü}{{\"u}}1
{ç}{{\c{c}}}1
{~}{{ }}1
}


\definecolor{codegreen}{rgb}{0,0.6,0}
\definecolor{codegray}{rgb}{0.5,0.5,0.5}
\definecolor{codepurple}{rgb}{0.58,0,0.82}
\definecolor{backcolour}{rgb}{0.95,0.95,0.92}

\lstdefinestyle{mystyle}{
    backgroundcolor=\color{backcolour},   
    commentstyle=\color{codegreen},
    keywordstyle=\color{magenta},
    numberstyle=\tiny\color{codegray},
    stringstyle=\color{codepurple},
    basicstyle=\ttfamily\footnotesize,
    breakatwhitespace=false,         
    breaklines=true,                 
    captionpos=b,                    
    keepspaces=true,                 
    numbers=left,                    
xleftmargin=2em,
framexleftmargin=2em,            
    showspaces=false,                
    showstringspaces=false,
    showtabs=false,                  
    tabsize=2,
    upquote=true
}

\lstset{style=mystyle}


\lstset{style=mystyle}
\newcommand{\imgdir}{C:/laragon/www/newmc/assets/imgsvg/}
\newcommand{\imgsvgdir}{C:/laragon/www/newmc/assets/imgsvg/}

\definecolor{mcgris}{RGB}{220, 220, 220}% ancien~; pour compatibilité
\definecolor{mcbleu}{RGB}{52, 152, 219}
\definecolor{mcvert}{RGB}{125, 194, 70}
\definecolor{mcmauve}{RGB}{154, 0, 215}
\definecolor{mcorange}{RGB}{255, 96, 0}
\definecolor{mcturquoise}{RGB}{0, 153, 153}
\definecolor{mcrouge}{RGB}{255, 0, 0}
\definecolor{mclightvert}{RGB}{205, 234, 190}

\definecolor{gris}{RGB}{220, 220, 220}
\definecolor{bleu}{RGB}{52, 152, 219}
\definecolor{vert}{RGB}{125, 194, 70}
\definecolor{mauve}{RGB}{154, 0, 215}
\definecolor{orange}{RGB}{255, 96, 0}
\definecolor{turquoise}{RGB}{0, 153, 153}
\definecolor{rouge}{RGB}{255, 0, 0}
\definecolor{lightvert}{RGB}{205, 234, 190}
\setitemize[0]{label=\color{lightvert}  $\bullet$}

\pagestyle{fancy}
\renewcommand{\headrulewidth}{0.2pt}
\fancyhead[L]{maths-cours.fr}
\fancyhead[R]{\thepage}
\renewcommand{\footrulewidth}{0.2pt}
\fancyfoot[C]{}

\newcolumntype{C}{>{\centering\arraybackslash}X}
\newcolumntype{s}{>{\hsize=.35\hsize\arraybackslash}X}

\setlength{\parindent}{0pt}		 
\setlength{\parskip}{3mm}
\setlength{\headheight}{1cm}

\def\ebook{ebook}
\def\book{book}
\def\web{web}
\def\type{web}

\newcommand{\vect}[1]{\overrightarrow{\,\mathstrut#1\,}}

\def\Oij{$\left(\text{O}~;~\vect{\imath},~\vect{\jmath}\right)$}
\def\Oijk{$\left(\text{O}~;~\vect{\imath},~\vect{\jmath},~\vect{k}\right)$}
\def\Ouv{$\left(\text{O}~;~\vect{u},~\vect{v}\right)$}

\hypersetup{breaklinks=true, colorlinks = true, linkcolor = OliveGreen, urlcolor = OliveGreen, citecolor = OliveGreen, pdfauthor={Didier BONNEL - https://www.maths-cours.fr} } % supprime les bordures autour des liens

\renewcommand{\arg}[0]{\text{arg}}

\everymath{\displaystyle}

%================================================================================================================================
%
% Macros - Commandes
%
%================================================================================================================================

\newcommand\meta[2]{    			% Utilisé pour créer le post HTML.
	\def\titre{titre}
	\def\url{url}
	\def\arg{#1}
	\ifx\titre\arg
		\newcommand\maintitle{#2}
		\fancyhead[L]{#2}
		{\Large\sffamily \MakeUppercase{#2}}
		\vspace{1mm}\textcolor{mcvert}{\hrule}
	\fi 
	\ifx\url\arg
		\fancyfoot[L]{\href{https://www.maths-cours.fr#2}{\black \footnotesize{https://www.maths-cours.fr#2}}}
	\fi 
}


\newcommand\TitreC[1]{    		% Titre centré
     \needspace{3\baselineskip}
     \begin{center}\textbf{#1}\end{center}
}

\newcommand\newpar{    		% paragraphe
     \par
}

\newcommand\nosp {    		% commande vide (pas d'espace)
}
\newcommand{\id}[1]{} %ignore

\newcommand\boite[2]{				% Boite simple sans titre
	\vspace{5mm}
	\setlength{\fboxrule}{0.2mm}
	\setlength{\fboxsep}{5mm}	
	\fcolorbox{#1}{#1!3}{\makebox[\linewidth-2\fboxrule-2\fboxsep]{
  		\begin{minipage}[t]{\linewidth-2\fboxrule-4\fboxsep}\setlength{\parskip}{3mm}
  			 #2
  		\end{minipage}
	}}
	\vspace{5mm}
}

\newcommand\CBox[4]{				% Boites
	\vspace{5mm}
	\setlength{\fboxrule}{0.2mm}
	\setlength{\fboxsep}{5mm}
	
	\fcolorbox{#1}{#1!3}{\makebox[\linewidth-2\fboxrule-2\fboxsep]{
		\begin{minipage}[t]{1cm}\setlength{\parskip}{3mm}
	  		\textcolor{#1}{\LARGE{#2}}    
 	 	\end{minipage}  
  		\begin{minipage}[t]{\linewidth-2\fboxrule-4\fboxsep}\setlength{\parskip}{3mm}
			\raisebox{1.2mm}{\normalsize\sffamily{\textcolor{#1}{#3}}}						
  			 #4
  		\end{minipage}
	}}
	\vspace{5mm}
}

\newcommand\cadre[3]{				% Boites convertible html
	\par
	\vspace{2mm}
	\setlength{\fboxrule}{0.1mm}
	\setlength{\fboxsep}{5mm}
	\fcolorbox{#1}{white}{\makebox[\linewidth-2\fboxrule-2\fboxsep]{
  		\begin{minipage}[t]{\linewidth-2\fboxrule-4\fboxsep}\setlength{\parskip}{3mm}
			\raisebox{-2.5mm}{\sffamily \small{\textcolor{#1}{\MakeUppercase{#2}}}}		
			\par		
  			 #3
 	 		\end{minipage}
	}}
		\vspace{2mm}
	\par
}

\newcommand\bloc[3]{				% Boites convertible html sans bordure
     \needspace{2\baselineskip}
     {\sffamily \small{\textcolor{#1}{\MakeUppercase{#2}}}}    
		\par		
  			 #3
		\par
}

\newcommand\CHelp[1]{
     \CBox{Plum}{\faInfoCircle}{À RETENIR}{#1}
}

\newcommand\CUp[1]{
     \CBox{NavyBlue}{\faThumbsOUp}{EN PRATIQUE}{#1}
}

\newcommand\CInfo[1]{
     \CBox{Sepia}{\faArrowCircleRight}{REMARQUE}{#1}
}

\newcommand\CRedac[1]{
     \CBox{PineGreen}{\faEdit}{BIEN R\'EDIGER}{#1}
}

\newcommand\CError[1]{
     \CBox{Red}{\faExclamationTriangle}{ATTENTION}{#1}
}

\newcommand\TitreExo[2]{
\needspace{4\baselineskip}
 {\sffamily\large EXERCICE #1\ (\emph{#2 points})}
\vspace{5mm}
}

\newcommand\img[2]{
          \includegraphics[width=#2\paperwidth]{\imgdir#1}
}

\newcommand\imgsvg[2]{
       \begin{center}   \includegraphics[width=#2\paperwidth]{\imgsvgdir#1} \end{center}
}


\newcommand\Lien[2]{
     \href{#1}{#2 \tiny \faExternalLink}
}
\newcommand\mcLien[2]{
     \href{https~://www.maths-cours.fr/#1}{#2 \tiny \faExternalLink}
}

\newcommand{\euro}{\eurologo{}}

%================================================================================================================================
%
% Macros - Environement
%
%================================================================================================================================

\newenvironment{tex}{ %
}
{%
}

\newenvironment{indente}{ %
	\setlength\parindent{10mm}
}

{
	\setlength\parindent{0mm}
}

\newenvironment{corrige}{%
     \needspace{3\baselineskip}
     \medskip
     \textbf{\textsc{Corrigé}}
     \medskip
}
{
}

\newenvironment{extern}{%
     \begin{center}
     }
     {
     \end{center}
}

\NewEnviron{code}{%
	\par
     \boite{gray}{\texttt{%
     \BODY
     }}
     \par
}

\newenvironment{vbloc}{% boite sans cadre empeche saut de page
     \begin{minipage}[t]{\linewidth}
     }
     {
     \end{minipage}
}
\NewEnviron{h2}{%
    \needspace{3\baselineskip}
    \vspace{0.6cm}
	\noindent \MakeUppercase{\sffamily \large \BODY}
	\vspace{1mm}\textcolor{mcgris}{\hrule}\vspace{0.4cm}
	\par
}{}

\NewEnviron{h3}{%
    \needspace{3\baselineskip}
	\vspace{5mm}
	\textsc{\BODY}
	\par
}

\NewEnviron{margeneg}{ %
\begin{addmargin}[-1cm]{0cm}
\BODY
\end{addmargin}
}

\NewEnviron{html}{%
}

\begin{document}
\meta{url}{/exercices/fonctions-bac-es-l-metropole-2018/}
\meta{pid}{9229}
\meta{titre}{Fonctions - Bac ES/L Métropole 2018}
\meta{type}{exercices}
%
\begin{h2}Exercice 4 (6 points)\end{h2}
\textbf{Commun à  tous les candidats}
\medbreak
On désigne par $f$ la fonction définie sur l'intervalle $[-2~;~4]$ par
\par
\[f(x) = (2x+1)\text{e}^{-2x}+3.\]
\par
On note $\mathscr{C}_f$ la courbe représentative de $f$ dans une repère. Une représentation graphique est donnée en annexe.
\begin{enumerate}
     \item On note $f'$ la fonction dérivée de $f$. Montrer que, pour tout $x\in  [-2~;~4]$,
     \par
     \[f'(x)=-4x\text{e}^{-2x}.\]
     \item Étudier les variations de $f$.
     \item Montrer que l'équation $f(x) = 0$ admet une unique solution sur $[-2~;~0]$ et donner une valeur approchée au dixième de cette solution.
     \item On note $f''$ la fonction dérivée de $f'$. On admet que, pour tout $x\in [-2~;~4]$,
     \[f''(x)=(8x-4)\text{e}^{-2x}.\]
     \begin{enumerate}[label=\alph*.]
          \item Étudier le signe de $f''$ sur l'intervalle $[-2~;~4]$.
          \item En déduire le plus grand intervalle sur lequel $f$ est convexe.
     \end{enumerate}
     \item On note $g$ la fonction définie sur l'intervalle $[-2~; 4]$ par $g(x)=(2x+1)\text{e}^{-2x}$.
     \begin{enumerate}[label=\alph*.]
          \item Vérifier que la fonction $G$ définie pour tout $x\in [-2~;~4]$ par $G(x)=(-x-1)^{-2x}$ est une primitive de la fonction $g$.
          \item En déduire une primitive $F$ de $f$.
     \end{enumerate}
     \item On note $\mathscr{A}$ l'aire du domaine $\mathscr{D}$ compris entre la courbe $\mathscr{C}_f$, l'axe des abscisses et les droites d'équations $x=0$ et $x=1$.
     \begin{enumerate}[label=\alph*.]
          \item Hachurer le domaine $\mathscr{D}$ sur le graphique donné en annexe, à rendre avec la copie.
          \item Par lecture graphique, donner un encadrement de $\mathscr{A}$, en unité d'aire, par deux entiers consécutifs.
          \item Calculer la valeur exacte de $\mathscr{A}$, puis une valeur approchée au centième.
     \end{enumerate}
\end{enumerate}
\bigbreak
\newpage
\begin{center}
     \TitreC{Annexe}
     \`A rendre avec la copie
     \par
     \begin{extern}%width="500" alt=""
          \psset{xunit=1.4cm,yunit=0.7cm}
          \def\xmin {-3.5}   \def\xmax {4.6}
          \def\ymin {-7}   \def\ymax {5.5}
          \begin{pspicture*}(\xmin,\ymin)(\xmax,\ymax)
               \psgrid[unit=1.4cm,subgriddiv=1,  gridlabels=0, gridcolor=lightgray](0,0)(-4,-4)(\xmax,5)
               \psaxes[arrowsize=3pt 3, ticksize=-2pt 2pt,Dy=2]{->}(0,0)(\xmin,\ymin)(\xmax,\ymax)
               \uput{10pt}[dl](0,0){$0$}
               %\psaxes[ linewidth=1.8pt]{->}(0,0)(1,1)
               %\uput[d](0.5,0){$\vec{\imath}$} \uput[l](0,0.5){$\vec{\jmath}$}
               \def\f{2 x mul 1 add 2.7183 -2 x mul exp mul 3 add}                           % définition de la fonction
               \psplot[plotpoints=3000,linecolor=red,linewidth=1pt]{-2}{4}{\f}
               \uput[u](3.5,3){\red $\mathcal{C}_f$}
          \end{pspicture*}
     \end{extern}
\end{center}

\end{document}
µ
\documentclass[a4paper]{article}

%================================================================================================================================
%
% Packages
%
%================================================================================================================================

\usepackage[T1]{fontenc} 	% pour caractères accentués
\usepackage[utf8]{inputenc}  % encodage utf8
\usepackage[french]{babel}	% langue : français
\usepackage{fourier}			% caractères plus lisibles
\usepackage[dvipsnames]{xcolor} % couleurs
\usepackage{fancyhdr}		% réglage header footer
\usepackage{needspace}		% empêcher sauts de page mal placés
\usepackage{graphicx}		% pour inclure des graphiques
\usepackage{enumitem,cprotect}		% personnalise les listes d'items (nécessaire pour ol, al ...)
\usepackage{hyperref}		% Liens hypertexte
\usepackage{pstricks,pst-all,pst-node,pstricks-add,pst-math,pst-plot,pst-tree,pst-eucl} % pstricks
\usepackage[a4paper,includeheadfoot,top=2cm,left=3cm, bottom=2cm,right=3cm]{geometry} % marges etc.
\usepackage{comment}			% commentaires multilignes
\usepackage{amsmath,environ} % maths (matrices, etc.)
\usepackage{amssymb,makeidx}
\usepackage{bm}				% bold maths
\usepackage{tabularx}		% tableaux
\usepackage{colortbl}		% tableaux en couleur
\usepackage{fontawesome}		% Fontawesome
\usepackage{environ}			% environment with command
\usepackage{fp}				% calculs pour ps-tricks
\usepackage{multido}			% pour ps tricks
\usepackage[np]{numprint}	% formattage nombre
\usepackage{tikz,tkz-tab} 			% package principal TikZ
\usepackage{pgfplots}   % axes
\usepackage{mathrsfs}    % cursives
\usepackage{calc}			% calcul taille boites
\usepackage[scaled=0.875]{helvet} % font sans serif
\usepackage{svg} % svg
\usepackage{scrextend} % local margin
\usepackage{scratch} %scratch
\usepackage{multicol} % colonnes
%\usepackage{infix-RPN,pst-func} % formule en notation polanaise inversée
\usepackage{listings}

%================================================================================================================================
%
% Réglages de base
%
%================================================================================================================================

\lstset{
language=Python,   % R code
literate=
{á}{{\'a}}1
{à}{{\`a}}1
{ã}{{\~a}}1
{é}{{\'e}}1
{è}{{\`e}}1
{ê}{{\^e}}1
{í}{{\'i}}1
{ó}{{\'o}}1
{õ}{{\~o}}1
{ú}{{\'u}}1
{ü}{{\"u}}1
{ç}{{\c{c}}}1
{~}{{ }}1
}


\definecolor{codegreen}{rgb}{0,0.6,0}
\definecolor{codegray}{rgb}{0.5,0.5,0.5}
\definecolor{codepurple}{rgb}{0.58,0,0.82}
\definecolor{backcolour}{rgb}{0.95,0.95,0.92}

\lstdefinestyle{mystyle}{
    backgroundcolor=\color{backcolour},   
    commentstyle=\color{codegreen},
    keywordstyle=\color{magenta},
    numberstyle=\tiny\color{codegray},
    stringstyle=\color{codepurple},
    basicstyle=\ttfamily\footnotesize,
    breakatwhitespace=false,         
    breaklines=true,                 
    captionpos=b,                    
    keepspaces=true,                 
    numbers=left,                    
xleftmargin=2em,
framexleftmargin=2em,            
    showspaces=false,                
    showstringspaces=false,
    showtabs=false,                  
    tabsize=2,
    upquote=true
}

\lstset{style=mystyle}


\lstset{style=mystyle}
\newcommand{\imgdir}{C:/laragon/www/newmc/assets/imgsvg/}
\newcommand{\imgsvgdir}{C:/laragon/www/newmc/assets/imgsvg/}

\definecolor{mcgris}{RGB}{220, 220, 220}% ancien~; pour compatibilité
\definecolor{mcbleu}{RGB}{52, 152, 219}
\definecolor{mcvert}{RGB}{125, 194, 70}
\definecolor{mcmauve}{RGB}{154, 0, 215}
\definecolor{mcorange}{RGB}{255, 96, 0}
\definecolor{mcturquoise}{RGB}{0, 153, 153}
\definecolor{mcrouge}{RGB}{255, 0, 0}
\definecolor{mclightvert}{RGB}{205, 234, 190}

\definecolor{gris}{RGB}{220, 220, 220}
\definecolor{bleu}{RGB}{52, 152, 219}
\definecolor{vert}{RGB}{125, 194, 70}
\definecolor{mauve}{RGB}{154, 0, 215}
\definecolor{orange}{RGB}{255, 96, 0}
\definecolor{turquoise}{RGB}{0, 153, 153}
\definecolor{rouge}{RGB}{255, 0, 0}
\definecolor{lightvert}{RGB}{205, 234, 190}
\setitemize[0]{label=\color{lightvert}  $\bullet$}

\pagestyle{fancy}
\renewcommand{\headrulewidth}{0.2pt}
\fancyhead[L]{maths-cours.fr}
\fancyhead[R]{\thepage}
\renewcommand{\footrulewidth}{0.2pt}
\fancyfoot[C]{}

\newcolumntype{C}{>{\centering\arraybackslash}X}
\newcolumntype{s}{>{\hsize=.35\hsize\arraybackslash}X}

\setlength{\parindent}{0pt}		 
\setlength{\parskip}{3mm}
\setlength{\headheight}{1cm}

\def\ebook{ebook}
\def\book{book}
\def\web{web}
\def\type{web}

\newcommand{\vect}[1]{\overrightarrow{\,\mathstrut#1\,}}

\def\Oij{$\left(\text{O}~;~\vect{\imath},~\vect{\jmath}\right)$}
\def\Oijk{$\left(\text{O}~;~\vect{\imath},~\vect{\jmath},~\vect{k}\right)$}
\def\Ouv{$\left(\text{O}~;~\vect{u},~\vect{v}\right)$}

\hypersetup{breaklinks=true, colorlinks = true, linkcolor = OliveGreen, urlcolor = OliveGreen, citecolor = OliveGreen, pdfauthor={Didier BONNEL - https://www.maths-cours.fr} } % supprime les bordures autour des liens

\renewcommand{\arg}[0]{\text{arg}}

\everymath{\displaystyle}

%================================================================================================================================
%
% Macros - Commandes
%
%================================================================================================================================

\newcommand\meta[2]{    			% Utilisé pour créer le post HTML.
	\def\titre{titre}
	\def\url{url}
	\def\arg{#1}
	\ifx\titre\arg
		\newcommand\maintitle{#2}
		\fancyhead[L]{#2}
		{\Large\sffamily \MakeUppercase{#2}}
		\vspace{1mm}\textcolor{mcvert}{\hrule}
	\fi 
	\ifx\url\arg
		\fancyfoot[L]{\href{https://www.maths-cours.fr#2}{\black \footnotesize{https://www.maths-cours.fr#2}}}
	\fi 
}


\newcommand\TitreC[1]{    		% Titre centré
     \needspace{3\baselineskip}
     \begin{center}\textbf{#1}\end{center}
}

\newcommand\newpar{    		% paragraphe
     \par
}

\newcommand\nosp {    		% commande vide (pas d'espace)
}
\newcommand{\id}[1]{} %ignore

\newcommand\boite[2]{				% Boite simple sans titre
	\vspace{5mm}
	\setlength{\fboxrule}{0.2mm}
	\setlength{\fboxsep}{5mm}	
	\fcolorbox{#1}{#1!3}{\makebox[\linewidth-2\fboxrule-2\fboxsep]{
  		\begin{minipage}[t]{\linewidth-2\fboxrule-4\fboxsep}\setlength{\parskip}{3mm}
  			 #2
  		\end{minipage}
	}}
	\vspace{5mm}
}

\newcommand\CBox[4]{				% Boites
	\vspace{5mm}
	\setlength{\fboxrule}{0.2mm}
	\setlength{\fboxsep}{5mm}
	
	\fcolorbox{#1}{#1!3}{\makebox[\linewidth-2\fboxrule-2\fboxsep]{
		\begin{minipage}[t]{1cm}\setlength{\parskip}{3mm}
	  		\textcolor{#1}{\LARGE{#2}}    
 	 	\end{minipage}  
  		\begin{minipage}[t]{\linewidth-2\fboxrule-4\fboxsep}\setlength{\parskip}{3mm}
			\raisebox{1.2mm}{\normalsize\sffamily{\textcolor{#1}{#3}}}						
  			 #4
  		\end{minipage}
	}}
	\vspace{5mm}
}

\newcommand\cadre[3]{				% Boites convertible html
	\par
	\vspace{2mm}
	\setlength{\fboxrule}{0.1mm}
	\setlength{\fboxsep}{5mm}
	\fcolorbox{#1}{white}{\makebox[\linewidth-2\fboxrule-2\fboxsep]{
  		\begin{minipage}[t]{\linewidth-2\fboxrule-4\fboxsep}\setlength{\parskip}{3mm}
			\raisebox{-2.5mm}{\sffamily \small{\textcolor{#1}{\MakeUppercase{#2}}}}		
			\par		
  			 #3
 	 		\end{minipage}
	}}
		\vspace{2mm}
	\par
}

\newcommand\bloc[3]{				% Boites convertible html sans bordure
     \needspace{2\baselineskip}
     {\sffamily \small{\textcolor{#1}{\MakeUppercase{#2}}}}    
		\par		
  			 #3
		\par
}

\newcommand\CHelp[1]{
     \CBox{Plum}{\faInfoCircle}{À RETENIR}{#1}
}

\newcommand\CUp[1]{
     \CBox{NavyBlue}{\faThumbsOUp}{EN PRATIQUE}{#1}
}

\newcommand\CInfo[1]{
     \CBox{Sepia}{\faArrowCircleRight}{REMARQUE}{#1}
}

\newcommand\CRedac[1]{
     \CBox{PineGreen}{\faEdit}{BIEN R\'EDIGER}{#1}
}

\newcommand\CError[1]{
     \CBox{Red}{\faExclamationTriangle}{ATTENTION}{#1}
}

\newcommand\TitreExo[2]{
\needspace{4\baselineskip}
 {\sffamily\large EXERCICE #1\ (\emph{#2 points})}
\vspace{5mm}
}

\newcommand\img[2]{
          \includegraphics[width=#2\paperwidth]{\imgdir#1}
}

\newcommand\imgsvg[2]{
       \begin{center}   \includegraphics[width=#2\paperwidth]{\imgsvgdir#1} \end{center}
}


\newcommand\Lien[2]{
     \href{#1}{#2 \tiny \faExternalLink}
}
\newcommand\mcLien[2]{
     \href{https~://www.maths-cours.fr/#1}{#2 \tiny \faExternalLink}
}

\newcommand{\euro}{\eurologo{}}

%================================================================================================================================
%
% Macros - Environement
%
%================================================================================================================================

\newenvironment{tex}{ %
}
{%
}

\newenvironment{indente}{ %
	\setlength\parindent{10mm}
}

{
	\setlength\parindent{0mm}
}

\newenvironment{corrige}{%
     \needspace{3\baselineskip}
     \medskip
     \textbf{\textsc{Corrigé}}
     \medskip
}
{
}

\newenvironment{extern}{%
     \begin{center}
     }
     {
     \end{center}
}

\NewEnviron{code}{%
	\par
     \boite{gray}{\texttt{%
     \BODY
     }}
     \par
}

\newenvironment{vbloc}{% boite sans cadre empeche saut de page
     \begin{minipage}[t]{\linewidth}
     }
     {
     \end{minipage}
}
\NewEnviron{h2}{%
    \needspace{3\baselineskip}
    \vspace{0.6cm}
	\noindent \MakeUppercase{\sffamily \large \BODY}
	\vspace{1mm}\textcolor{mcgris}{\hrule}\vspace{0.4cm}
	\par
}{}

\NewEnviron{h3}{%
    \needspace{3\baselineskip}
	\vspace{5mm}
	\textsc{\BODY}
	\par
}

\NewEnviron{margeneg}{ %
\begin{addmargin}[-1cm]{0cm}
\BODY
\end{addmargin}
}

\NewEnviron{html}{%
}

\begin{document}
\meta{url}{/exercices/graphes-bac-es-l-metropole-2018-spe/}
\meta{pid}{9231}
\meta{titre}{Graphes - Bac ES/L Métropole 2018 (spé)}
\meta{type}{exercices}
%
\begin{h2}Exercice 3 (5 points)\end{h2}
\textbf{Candidats de ES ayant choisi l'enseignement de spécialité}
\TitreC{Partie A}
Un parcours sportif est composé d'un banc pour abdominaux, de haies et d'anneaux. Le graphe orienté ci-dessous indique les différents parcours conseillés partant de D et terminant à F.
\smallbreak
Les sommets sont~: D (départ), B (banc pour abdominaux), H (haies), A (anneaux) et F (fin du parcours).
\par
Les arêtes représentent les différents sentiers reliant les sommets.
\par
\begin{center}
     \begin{extern}%width="300" alt=""
          \psset{unit=1.2cm}
          \def\xmin {-3.5}   \def\xmax {3.5}
          \def\ymin {-3.5}   \def\ymax {3.5}
          \begin{pspicture}(\xmin,\ymin)(\xmax,\ymax)
               %\psgrid[subgriddiv=1,  gridlabels=0, gridcolor=lightgray]
               %\psaxes[arrowsize=3pt 3, ticksize=-2pt 2pt]{->}(0,0)(\xmin,\ymin)(\xmax,\ymax)
               \psset{ArrowInside=->,arrowsize=3pt 3}
               %%% définitions des points
               \psnode*(3;108){D}{}    \psnode*(3;36){H}{}   \psnode*(3;-36){A}{}
               \psnode*(3;-108){F}{}   \psnode*(3;-180){B}{}
               %%% tracés des arêtes et des étiquettes
               \ncline{D}{H} % \naput{19} %%% DH
               \ncline{D}{A} % \naput{28} %%% DA
               \ncline{D}{B} % \nbput{40} %%% DB
               \ncline{H}{B} % \nbput[npos=0.6]{16} %%% HB
               \ncline{H}{F} % \nbput{32} %%% HF
               \ncline{H}{A} % \naput{26} %%% HA
               \ncline{A}{B} % \naput[npos=0.6]{12} %%% AB
               \ncline{A}{F} % \naput{25} %%% AF
               \ncline{B}{F} % \nbput{14} %%% BF
               %%% noms des points
               \uput[-36](3;-36){A}       \uput[-180](3;-180){B}    \uput[108](3;108){D}
               \uput[-108](3;-108){F}   \uput[36](3;36){H}
          \end{pspicture}
     \end{extern}
     \end{center}\begin{enumerate}
     \item Quel est l'ordre du graphe~?
     \item On note $M$ la matrice d'adjacence de ce graphe où les sommets sont rangés dans l'ordre alphabétique.
     \begin{enumerate}[label=\alph*.]
          \item Déterminer $M$.
          \item On donne
          $M^3 =
          \begin{pmatrix}
               0 & 0 & 0 & 0 & 0 \\
               0 & 0 & 0 & 0 & 0 \\
               0 & 1 & 0 & 3 & 0 \\
               0 & 0 & 0 & 0 & 0 \\
               0 & 0 & 0 & 1 & 0
          \end{pmatrix}$
          \par
          Assia souhaite aller de D à F en faisant un parcours constitué de 3 arêtes.
          \par
          Est-ce possible~? Si oui, combien de parcours différents pourra-t-elle emprunter~?
          \par
          Préciser ces trajets.
     \end{enumerate}
     \item Assia a relevé ses temps de course en minute entre les différents sommets. Ces durées sont portées sur le graphe ci-dessous.
     \par
     Lors d'un entraînement, Assia souhaite courir le moins longtemps possible en allant de D à F. Déterminer le trajet pour lequel le temps de course est minimal et préciser la durée de sa course.
     \begin{center}
          \begin{extern}%width="300" alt=""
               \psset{unit=1.2cm}
               \def\xmin {-3.5}   \def\xmax {3.5}
               \def\ymin {-3.5}   \def\ymax {3.5}
               \begin{pspicture}(\xmin,\ymin)(\xmax,\ymax)
                    %\psgrid[subgriddiv=1,  gridlabels=0, gridcolor=lightgray]
                    %\psaxes[arrowsize=3pt 3, ticksize=-2pt 2pt]{->}(0,0)(\xmin,\ymin)(\xmax,\ymax)
                    \psset{ArrowInside=->,arrowsize=3pt 3}
                    %%% définitions des points
                    \psnode*(3;108){D}{}    \psnode*(3;36){H}{}   \psnode*(3;-36){A}{}
                    \psnode*(3;-108){F}{}   \psnode*(3;-180){B}{}
                    %%% tracés des arêtes et des étiquettes
                    \ncline{D}{H} \naput{19} %%% DH
                    \ncline{D}{A} \naput{28} %%% DA
                    \ncline{D}{B} \nbput{40} %%% DB
                    \ncline{H}{B} \nbput[npos=0.6]{16} %%% HB
                    \ncline{H}{F} \nbput{32} %%% HF
                    \ncline{H}{A} \naput{26} %%% HA
                    \ncline{A}{B} \naput[npos=0.6]{12} %%% AB
                    \ncline{A}{F} \naput{25} %%% AF
                    \ncline{B}{F} \nbput{14} %%% BF
                    %%% noms des points
                    \uput[-36](3;-36){A}       \uput[-180](3;-180){B}    \uput[108](3;108){D}
                    \uput[-108](3;-108){F}   \uput[36](3;36){H}
               \end{pspicture}
          \end{extern}
     \end{center}
\end{enumerate}
\TitreC{Partie B}
Le responsable souhaite ajouter une barre de traction notée T. De nouveaux sentiers sont construits et de nouveaux parcours sont possibles.
\smallbreak
La matrice d'adjacence $N$ associée au graphe représentant les nouveaux parcours, dans lequel les sommets sont classés en ordre alphabétique, est
\par
\[N =
\begin{pmatrix}
     0 & 1 & 0 & 1 & 0 & 1 \\
     0 & 0 & 0 & 1 & 0 & 0 \\
     1 & 1 & 0 & 0 & 1 & 0 \\
     0 & 0 & 0 & 0 & 0 & 0 \\
     1 & 1 & 0 & 1 & 0 & 1 \\
     0 & 0 & 0 & 1 & 0 & 0
\end{pmatrix}
\]
\par
Compléter l'annexe à rendre avec la copie, en ajoutant les arêtes nécessaires au graphe orienté correspondant à la matrice $N$.
\par
\newpage
\begin{center}
     \TitreC{Annexe}
     \par
     \`A rendre avec la copie
\end{center}
\begin{center}
     \begin{extern}%width="380" alt=""
          \psset{unit=1.8cm}
          \def\xmin {-3.5}   \def\xmax {3.5}
          \def\ymin {-3.5}   \def\ymax {4.5}
          \begin{pspicture}(\xmin,\ymin)(\xmax,\ymax)
               %\psgrid[subgriddiv=1,  gridlabels=0, gridcolor=lightgray]
               %\psaxes[arrowsize=3pt 3, ticksize=-2pt 2pt]{->}(0,0)(\xmin,\ymin)(\xmax,\ymax)
               \psset{ArrowInside=->,arrowsize=3pt 3}
               %%% définitions des points
               \psnode*(3;108){D}{}    \psnode*(3;36){H}{}   \psnode*(3;-36){A}{}
               \psnode*(3;-108){F}{}   \psnode*(3;-180){B}{}
               %%% tracés des arêtes et des étiquettes
               \ncline{D}{H} % \naput{19} %%% DH
               \ncline{D}{A} % \naput{28} %%% DA
               \ncline{D}{B} % \nbput{40} %%% DB
               \ncline{H}{B} % \nbput[npos=0.6]{16} %%% HB
               \ncline{H}{F} % \nbput{32} %%% HF
               \ncline{H}{A} % \naput{26} %%% HA
               \ncline{A}{B} % \naput[npos=0.6]{12} %%% AB
               \ncline{A}{F} % \naput{25} %%% AF
               \ncline{B}{F} % \nbput{14} %%% BF
               %%% noms des points
               \uput[-36](3;-36){A} \uput[-180](3;-180){B} \uput[108](3;108){D}
               \uput[-108](3;-108){F} \uput[36](3;36){H}
               \psdots[dotscale=1](0,0) \uput[ur](0,0){T}
          \end{pspicture}
     \end{extern}
\end{center}

\end{document}
µ
\documentclass[a4paper]{article}

%================================================================================================================================
%
% Packages
%
%================================================================================================================================

\usepackage[T1]{fontenc} 	% pour caractères accentués
\usepackage[utf8]{inputenc}  % encodage utf8
\usepackage[french]{babel}	% langue : français
\usepackage{fourier}			% caractères plus lisibles
\usepackage[dvipsnames]{xcolor} % couleurs
\usepackage{fancyhdr}		% réglage header footer
\usepackage{needspace}		% empêcher sauts de page mal placés
\usepackage{graphicx}		% pour inclure des graphiques
\usepackage{enumitem,cprotect}		% personnalise les listes d'items (nécessaire pour ol, al ...)
\usepackage{hyperref}		% Liens hypertexte
\usepackage{pstricks,pst-all,pst-node,pstricks-add,pst-math,pst-plot,pst-tree,pst-eucl} % pstricks
\usepackage[a4paper,includeheadfoot,top=2cm,left=3cm, bottom=2cm,right=3cm]{geometry} % marges etc.
\usepackage{comment}			% commentaires multilignes
\usepackage{amsmath,environ} % maths (matrices, etc.)
\usepackage{amssymb,makeidx}
\usepackage{bm}				% bold maths
\usepackage{tabularx}		% tableaux
\usepackage{colortbl}		% tableaux en couleur
\usepackage{fontawesome}		% Fontawesome
\usepackage{environ}			% environment with command
\usepackage{fp}				% calculs pour ps-tricks
\usepackage{multido}			% pour ps tricks
\usepackage[np]{numprint}	% formattage nombre
\usepackage{tikz,tkz-tab} 			% package principal TikZ
\usepackage{pgfplots}   % axes
\usepackage{mathrsfs}    % cursives
\usepackage{calc}			% calcul taille boites
\usepackage[scaled=0.875]{helvet} % font sans serif
\usepackage{svg} % svg
\usepackage{scrextend} % local margin
\usepackage{scratch} %scratch
\usepackage{multicol} % colonnes
%\usepackage{infix-RPN,pst-func} % formule en notation polanaise inversée
\usepackage{listings}

%================================================================================================================================
%
% Réglages de base
%
%================================================================================================================================

\lstset{
language=Python,   % R code
literate=
{á}{{\'a}}1
{à}{{\`a}}1
{ã}{{\~a}}1
{é}{{\'e}}1
{è}{{\`e}}1
{ê}{{\^e}}1
{í}{{\'i}}1
{ó}{{\'o}}1
{õ}{{\~o}}1
{ú}{{\'u}}1
{ü}{{\"u}}1
{ç}{{\c{c}}}1
{~}{{ }}1
}


\definecolor{codegreen}{rgb}{0,0.6,0}
\definecolor{codegray}{rgb}{0.5,0.5,0.5}
\definecolor{codepurple}{rgb}{0.58,0,0.82}
\definecolor{backcolour}{rgb}{0.95,0.95,0.92}

\lstdefinestyle{mystyle}{
    backgroundcolor=\color{backcolour},   
    commentstyle=\color{codegreen},
    keywordstyle=\color{magenta},
    numberstyle=\tiny\color{codegray},
    stringstyle=\color{codepurple},
    basicstyle=\ttfamily\footnotesize,
    breakatwhitespace=false,         
    breaklines=true,                 
    captionpos=b,                    
    keepspaces=true,                 
    numbers=left,                    
xleftmargin=2em,
framexleftmargin=2em,            
    showspaces=false,                
    showstringspaces=false,
    showtabs=false,                  
    tabsize=2,
    upquote=true
}

\lstset{style=mystyle}


\lstset{style=mystyle}
\newcommand{\imgdir}{C:/laragon/www/newmc/assets/imgsvg/}
\newcommand{\imgsvgdir}{C:/laragon/www/newmc/assets/imgsvg/}

\definecolor{mcgris}{RGB}{220, 220, 220}% ancien~; pour compatibilité
\definecolor{mcbleu}{RGB}{52, 152, 219}
\definecolor{mcvert}{RGB}{125, 194, 70}
\definecolor{mcmauve}{RGB}{154, 0, 215}
\definecolor{mcorange}{RGB}{255, 96, 0}
\definecolor{mcturquoise}{RGB}{0, 153, 153}
\definecolor{mcrouge}{RGB}{255, 0, 0}
\definecolor{mclightvert}{RGB}{205, 234, 190}

\definecolor{gris}{RGB}{220, 220, 220}
\definecolor{bleu}{RGB}{52, 152, 219}
\definecolor{vert}{RGB}{125, 194, 70}
\definecolor{mauve}{RGB}{154, 0, 215}
\definecolor{orange}{RGB}{255, 96, 0}
\definecolor{turquoise}{RGB}{0, 153, 153}
\definecolor{rouge}{RGB}{255, 0, 0}
\definecolor{lightvert}{RGB}{205, 234, 190}
\setitemize[0]{label=\color{lightvert}  $\bullet$}

\pagestyle{fancy}
\renewcommand{\headrulewidth}{0.2pt}
\fancyhead[L]{maths-cours.fr}
\fancyhead[R]{\thepage}
\renewcommand{\footrulewidth}{0.2pt}
\fancyfoot[C]{}

\newcolumntype{C}{>{\centering\arraybackslash}X}
\newcolumntype{s}{>{\hsize=.35\hsize\arraybackslash}X}

\setlength{\parindent}{0pt}		 
\setlength{\parskip}{3mm}
\setlength{\headheight}{1cm}

\def\ebook{ebook}
\def\book{book}
\def\web{web}
\def\type{web}

\newcommand{\vect}[1]{\overrightarrow{\,\mathstrut#1\,}}

\def\Oij{$\left(\text{O}~;~\vect{\imath},~\vect{\jmath}\right)$}
\def\Oijk{$\left(\text{O}~;~\vect{\imath},~\vect{\jmath},~\vect{k}\right)$}
\def\Ouv{$\left(\text{O}~;~\vect{u},~\vect{v}\right)$}

\hypersetup{breaklinks=true, colorlinks = true, linkcolor = OliveGreen, urlcolor = OliveGreen, citecolor = OliveGreen, pdfauthor={Didier BONNEL - https://www.maths-cours.fr} } % supprime les bordures autour des liens

\renewcommand{\arg}[0]{\text{arg}}

\everymath{\displaystyle}

%================================================================================================================================
%
% Macros - Commandes
%
%================================================================================================================================

\newcommand\meta[2]{    			% Utilisé pour créer le post HTML.
	\def\titre{titre}
	\def\url{url}
	\def\arg{#1}
	\ifx\titre\arg
		\newcommand\maintitle{#2}
		\fancyhead[L]{#2}
		{\Large\sffamily \MakeUppercase{#2}}
		\vspace{1mm}\textcolor{mcvert}{\hrule}
	\fi 
	\ifx\url\arg
		\fancyfoot[L]{\href{https://www.maths-cours.fr#2}{\black \footnotesize{https://www.maths-cours.fr#2}}}
	\fi 
}


\newcommand\TitreC[1]{    		% Titre centré
     \needspace{3\baselineskip}
     \begin{center}\textbf{#1}\end{center}
}

\newcommand\newpar{    		% paragraphe
     \par
}

\newcommand\nosp {    		% commande vide (pas d'espace)
}
\newcommand{\id}[1]{} %ignore

\newcommand\boite[2]{				% Boite simple sans titre
	\vspace{5mm}
	\setlength{\fboxrule}{0.2mm}
	\setlength{\fboxsep}{5mm}	
	\fcolorbox{#1}{#1!3}{\makebox[\linewidth-2\fboxrule-2\fboxsep]{
  		\begin{minipage}[t]{\linewidth-2\fboxrule-4\fboxsep}\setlength{\parskip}{3mm}
  			 #2
  		\end{minipage}
	}}
	\vspace{5mm}
}

\newcommand\CBox[4]{				% Boites
	\vspace{5mm}
	\setlength{\fboxrule}{0.2mm}
	\setlength{\fboxsep}{5mm}
	
	\fcolorbox{#1}{#1!3}{\makebox[\linewidth-2\fboxrule-2\fboxsep]{
		\begin{minipage}[t]{1cm}\setlength{\parskip}{3mm}
	  		\textcolor{#1}{\LARGE{#2}}    
 	 	\end{minipage}  
  		\begin{minipage}[t]{\linewidth-2\fboxrule-4\fboxsep}\setlength{\parskip}{3mm}
			\raisebox{1.2mm}{\normalsize\sffamily{\textcolor{#1}{#3}}}						
  			 #4
  		\end{minipage}
	}}
	\vspace{5mm}
}

\newcommand\cadre[3]{				% Boites convertible html
	\par
	\vspace{2mm}
	\setlength{\fboxrule}{0.1mm}
	\setlength{\fboxsep}{5mm}
	\fcolorbox{#1}{white}{\makebox[\linewidth-2\fboxrule-2\fboxsep]{
  		\begin{minipage}[t]{\linewidth-2\fboxrule-4\fboxsep}\setlength{\parskip}{3mm}
			\raisebox{-2.5mm}{\sffamily \small{\textcolor{#1}{\MakeUppercase{#2}}}}		
			\par		
  			 #3
 	 		\end{minipage}
	}}
		\vspace{2mm}
	\par
}

\newcommand\bloc[3]{				% Boites convertible html sans bordure
     \needspace{2\baselineskip}
     {\sffamily \small{\textcolor{#1}{\MakeUppercase{#2}}}}    
		\par		
  			 #3
		\par
}

\newcommand\CHelp[1]{
     \CBox{Plum}{\faInfoCircle}{À RETENIR}{#1}
}

\newcommand\CUp[1]{
     \CBox{NavyBlue}{\faThumbsOUp}{EN PRATIQUE}{#1}
}

\newcommand\CInfo[1]{
     \CBox{Sepia}{\faArrowCircleRight}{REMARQUE}{#1}
}

\newcommand\CRedac[1]{
     \CBox{PineGreen}{\faEdit}{BIEN R\'EDIGER}{#1}
}

\newcommand\CError[1]{
     \CBox{Red}{\faExclamationTriangle}{ATTENTION}{#1}
}

\newcommand\TitreExo[2]{
\needspace{4\baselineskip}
 {\sffamily\large EXERCICE #1\ (\emph{#2 points})}
\vspace{5mm}
}

\newcommand\img[2]{
          \includegraphics[width=#2\paperwidth]{\imgdir#1}
}

\newcommand\imgsvg[2]{
       \begin{center}   \includegraphics[width=#2\paperwidth]{\imgsvgdir#1} \end{center}
}


\newcommand\Lien[2]{
     \href{#1}{#2 \tiny \faExternalLink}
}
\newcommand\mcLien[2]{
     \href{https~://www.maths-cours.fr/#1}{#2 \tiny \faExternalLink}
}

\newcommand{\euro}{\eurologo{}}

%================================================================================================================================
%
% Macros - Environement
%
%================================================================================================================================

\newenvironment{tex}{ %
}
{%
}

\newenvironment{indente}{ %
	\setlength\parindent{10mm}
}

{
	\setlength\parindent{0mm}
}

\newenvironment{corrige}{%
     \needspace{3\baselineskip}
     \medskip
     \textbf{\textsc{Corrigé}}
     \medskip
}
{
}

\newenvironment{extern}{%
     \begin{center}
     }
     {
     \end{center}
}

\NewEnviron{code}{%
	\par
     \boite{gray}{\texttt{%
     \BODY
     }}
     \par
}

\newenvironment{vbloc}{% boite sans cadre empeche saut de page
     \begin{minipage}[t]{\linewidth}
     }
     {
     \end{minipage}
}
\NewEnviron{h2}{%
    \needspace{3\baselineskip}
    \vspace{0.6cm}
	\noindent \MakeUppercase{\sffamily \large \BODY}
	\vspace{1mm}\textcolor{mcgris}{\hrule}\vspace{0.4cm}
	\par
}{}

\NewEnviron{h3}{%
    \needspace{3\baselineskip}
	\vspace{5mm}
	\textsc{\BODY}
	\par
}

\NewEnviron{margeneg}{ %
\begin{addmargin}[-1cm]{0cm}
\BODY
\end{addmargin}
}

\NewEnviron{html}{%
}

\begin{document}
\meta{url}{/exercices/probabilites-bac-s-polynesie-2018/}
\meta{pid}{9274}
\meta{titre}{Probabilités – Bac S Polynésie 2018}
\meta{type}{exercices}
%
\begin{h2}Exercice 1 (5 points)\end{h2}
\textbf{Commun à tous les candidats}
\medbreak
\textbf{Rappel de connaissances~:}
\medbreak
L'intervalle de fluctuation asymptotique au seuil de 95\,\% est donné par la formule~:
\par
\[\left[p- 1,96\dfrac{\sqrt{p(1 - p)}}{\sqrt{n}}~;~p + 1,96\dfrac{\sqrt{p(1 + p)}}{\sqrt{n}}\right]\]
\par
où $n$ désigne la taille de l'échantillon et $p$ la proportion des individus possédant le caractère étudié dans cette population. Les conditions de validité de cet intervalle sont les suivantes~:
\par
\[n \geqslant 30,\: np \geqslant 5,\: n(1 - p) \geqslant 5.\]
\medbreak
\newpage
\medbreak
La municipalité d'une grande ville dispose d'un stock de DVD qu'elle propose en location aux usagers des différentes médiathèques de cette ville.
\par
Afin de renouveler son offre de location, la municipalité décide de retirer des DVD de son stock.
\par
Parmi les DVD retirés, certains sont défectueux, d'autres non.
\par
Parmi les 6\,\% de DVD défectueux sur l'ensemble du stock, 98\,\% sont retirés.
\par
On admet par ailleurs que parmi les DVD non défectueux, 92\,\% sont maintenus dans le stock~; les autres sont retirés.
\medbreak
\textbf{Les trois parties sont indépendantes.}
\bigbreak
\TitreC{Partie A}
\medbreak
On choisit un DVD au hasard dans le stock de la municipalité.
\par
On considère les événements suivants~:
\begin{indent}
     \begin{itemize}
          \item $D$~: \og le DVD est défectueux \fg{}~;
          \item $R$~: \og le DVD est retiré du stock \fg\fg.
     \end{itemize}
\end{indent}
On note $\overline{D}$ et $\overline{R}$ les événements contraires respectifs des événements $D$ et $R$.
\medbreak
\begin{enumerate}
     \item Démontrer que la probabilité de l'événement $R$ est $0,134$.
     \item Une association caritative contacte la municipalité dans l'objectif de récupérer l'ensemble des DVD qui sont retirés du stock. Un responsable de la ville affirme alors que parmi ces DVD retirés, plus de la moitié est composée de DVD défectueux.
     \par
     Cette affirmation est-elle vraie~?
\end{enumerate}
\bigbreak
\TitreC{Partie B}
\medbreak
Une des médiathèques de la ville se demande si le nombre de DVD défectueux qu'elle possède
n'est pas anormalement élevé. Pour cela, elle effectue des tests sur un échantillon de $150 $~DVD de son propre stock qui est suffisamment important pour que cet échantillon soit assimilé à un tirage successif avec remise. Sur cet échantillon, on détecte $14$ DVD défectueux.
\par
Peut-on rejeter l'hypothèse selon laquelle, dans cette médiathèque, 6\,\% des DVD sont défectueux~?
\bigbreak
\TitreC{Partie C}
\medbreak
Une partie du stock de DVD de la ville est constituée de DVD de films d'animation destinés au jeune public. On choisit un film d'animation au hasard et on note $X$ la variable aléatoire qui donne la durée, en minutes, de ce film. $X$ suit une loi normale d'espérance $\mu = 80$ min et d'écart-type $\sigma$.
\par
De plus, on estime que $P(X \geqslant 92) = 0,10$.
\medbreak
\begin{enumerate}
     \item Déterminer le réel $\sigma$ et en donner une valeur approchée à $0,01$.
     \item Un enfant regarde un film d'animation dont il ne connaît pas la durée. Sachant qu'il en a déjà vu une heure et demie, quelle est la probabilité que le film se termine dans les cinq minutes qui suivent~?
\end{enumerate}

\end{document}
µ
\documentclass[a4paper]{article}

%================================================================================================================================
%
% Packages
%
%================================================================================================================================

\usepackage[T1]{fontenc} 	% pour caractères accentués
\usepackage[utf8]{inputenc}  % encodage utf8
\usepackage[french]{babel}	% langue : français
\usepackage{fourier}			% caractères plus lisibles
\usepackage[dvipsnames]{xcolor} % couleurs
\usepackage{fancyhdr}		% réglage header footer
\usepackage{needspace}		% empêcher sauts de page mal placés
\usepackage{graphicx}		% pour inclure des graphiques
\usepackage{enumitem,cprotect}		% personnalise les listes d'items (nécessaire pour ol, al ...)
\usepackage{hyperref}		% Liens hypertexte
\usepackage{pstricks,pst-all,pst-node,pstricks-add,pst-math,pst-plot,pst-tree,pst-eucl} % pstricks
\usepackage[a4paper,includeheadfoot,top=2cm,left=3cm, bottom=2cm,right=3cm]{geometry} % marges etc.
\usepackage{comment}			% commentaires multilignes
\usepackage{amsmath,environ} % maths (matrices, etc.)
\usepackage{amssymb,makeidx}
\usepackage{bm}				% bold maths
\usepackage{tabularx}		% tableaux
\usepackage{colortbl}		% tableaux en couleur
\usepackage{fontawesome}		% Fontawesome
\usepackage{environ}			% environment with command
\usepackage{fp}				% calculs pour ps-tricks
\usepackage{multido}			% pour ps tricks
\usepackage[np]{numprint}	% formattage nombre
\usepackage{tikz,tkz-tab} 			% package principal TikZ
\usepackage{pgfplots}   % axes
\usepackage{mathrsfs}    % cursives
\usepackage{calc}			% calcul taille boites
\usepackage[scaled=0.875]{helvet} % font sans serif
\usepackage{svg} % svg
\usepackage{scrextend} % local margin
\usepackage{scratch} %scratch
\usepackage{multicol} % colonnes
%\usepackage{infix-RPN,pst-func} % formule en notation polanaise inversée
\usepackage{listings}

%================================================================================================================================
%
% Réglages de base
%
%================================================================================================================================

\lstset{
language=Python,   % R code
literate=
{á}{{\'a}}1
{à}{{\`a}}1
{ã}{{\~a}}1
{é}{{\'e}}1
{è}{{\`e}}1
{ê}{{\^e}}1
{í}{{\'i}}1
{ó}{{\'o}}1
{õ}{{\~o}}1
{ú}{{\'u}}1
{ü}{{\"u}}1
{ç}{{\c{c}}}1
{~}{{ }}1
}


\definecolor{codegreen}{rgb}{0,0.6,0}
\definecolor{codegray}{rgb}{0.5,0.5,0.5}
\definecolor{codepurple}{rgb}{0.58,0,0.82}
\definecolor{backcolour}{rgb}{0.95,0.95,0.92}

\lstdefinestyle{mystyle}{
    backgroundcolor=\color{backcolour},   
    commentstyle=\color{codegreen},
    keywordstyle=\color{magenta},
    numberstyle=\tiny\color{codegray},
    stringstyle=\color{codepurple},
    basicstyle=\ttfamily\footnotesize,
    breakatwhitespace=false,         
    breaklines=true,                 
    captionpos=b,                    
    keepspaces=true,                 
    numbers=left,                    
xleftmargin=2em,
framexleftmargin=2em,            
    showspaces=false,                
    showstringspaces=false,
    showtabs=false,                  
    tabsize=2,
    upquote=true
}

\lstset{style=mystyle}


\lstset{style=mystyle}
\newcommand{\imgdir}{C:/laragon/www/newmc/assets/imgsvg/}
\newcommand{\imgsvgdir}{C:/laragon/www/newmc/assets/imgsvg/}

\definecolor{mcgris}{RGB}{220, 220, 220}% ancien~; pour compatibilité
\definecolor{mcbleu}{RGB}{52, 152, 219}
\definecolor{mcvert}{RGB}{125, 194, 70}
\definecolor{mcmauve}{RGB}{154, 0, 215}
\definecolor{mcorange}{RGB}{255, 96, 0}
\definecolor{mcturquoise}{RGB}{0, 153, 153}
\definecolor{mcrouge}{RGB}{255, 0, 0}
\definecolor{mclightvert}{RGB}{205, 234, 190}

\definecolor{gris}{RGB}{220, 220, 220}
\definecolor{bleu}{RGB}{52, 152, 219}
\definecolor{vert}{RGB}{125, 194, 70}
\definecolor{mauve}{RGB}{154, 0, 215}
\definecolor{orange}{RGB}{255, 96, 0}
\definecolor{turquoise}{RGB}{0, 153, 153}
\definecolor{rouge}{RGB}{255, 0, 0}
\definecolor{lightvert}{RGB}{205, 234, 190}
\setitemize[0]{label=\color{lightvert}  $\bullet$}

\pagestyle{fancy}
\renewcommand{\headrulewidth}{0.2pt}
\fancyhead[L]{maths-cours.fr}
\fancyhead[R]{\thepage}
\renewcommand{\footrulewidth}{0.2pt}
\fancyfoot[C]{}

\newcolumntype{C}{>{\centering\arraybackslash}X}
\newcolumntype{s}{>{\hsize=.35\hsize\arraybackslash}X}

\setlength{\parindent}{0pt}		 
\setlength{\parskip}{3mm}
\setlength{\headheight}{1cm}

\def\ebook{ebook}
\def\book{book}
\def\web{web}
\def\type{web}

\newcommand{\vect}[1]{\overrightarrow{\,\mathstrut#1\,}}

\def\Oij{$\left(\text{O}~;~\vect{\imath},~\vect{\jmath}\right)$}
\def\Oijk{$\left(\text{O}~;~\vect{\imath},~\vect{\jmath},~\vect{k}\right)$}
\def\Ouv{$\left(\text{O}~;~\vect{u},~\vect{v}\right)$}

\hypersetup{breaklinks=true, colorlinks = true, linkcolor = OliveGreen, urlcolor = OliveGreen, citecolor = OliveGreen, pdfauthor={Didier BONNEL - https://www.maths-cours.fr} } % supprime les bordures autour des liens

\renewcommand{\arg}[0]{\text{arg}}

\everymath{\displaystyle}

%================================================================================================================================
%
% Macros - Commandes
%
%================================================================================================================================

\newcommand\meta[2]{    			% Utilisé pour créer le post HTML.
	\def\titre{titre}
	\def\url{url}
	\def\arg{#1}
	\ifx\titre\arg
		\newcommand\maintitle{#2}
		\fancyhead[L]{#2}
		{\Large\sffamily \MakeUppercase{#2}}
		\vspace{1mm}\textcolor{mcvert}{\hrule}
	\fi 
	\ifx\url\arg
		\fancyfoot[L]{\href{https://www.maths-cours.fr#2}{\black \footnotesize{https://www.maths-cours.fr#2}}}
	\fi 
}


\newcommand\TitreC[1]{    		% Titre centré
     \needspace{3\baselineskip}
     \begin{center}\textbf{#1}\end{center}
}

\newcommand\newpar{    		% paragraphe
     \par
}

\newcommand\nosp {    		% commande vide (pas d'espace)
}
\newcommand{\id}[1]{} %ignore

\newcommand\boite[2]{				% Boite simple sans titre
	\vspace{5mm}
	\setlength{\fboxrule}{0.2mm}
	\setlength{\fboxsep}{5mm}	
	\fcolorbox{#1}{#1!3}{\makebox[\linewidth-2\fboxrule-2\fboxsep]{
  		\begin{minipage}[t]{\linewidth-2\fboxrule-4\fboxsep}\setlength{\parskip}{3mm}
  			 #2
  		\end{minipage}
	}}
	\vspace{5mm}
}

\newcommand\CBox[4]{				% Boites
	\vspace{5mm}
	\setlength{\fboxrule}{0.2mm}
	\setlength{\fboxsep}{5mm}
	
	\fcolorbox{#1}{#1!3}{\makebox[\linewidth-2\fboxrule-2\fboxsep]{
		\begin{minipage}[t]{1cm}\setlength{\parskip}{3mm}
	  		\textcolor{#1}{\LARGE{#2}}    
 	 	\end{minipage}  
  		\begin{minipage}[t]{\linewidth-2\fboxrule-4\fboxsep}\setlength{\parskip}{3mm}
			\raisebox{1.2mm}{\normalsize\sffamily{\textcolor{#1}{#3}}}						
  			 #4
  		\end{minipage}
	}}
	\vspace{5mm}
}

\newcommand\cadre[3]{				% Boites convertible html
	\par
	\vspace{2mm}
	\setlength{\fboxrule}{0.1mm}
	\setlength{\fboxsep}{5mm}
	\fcolorbox{#1}{white}{\makebox[\linewidth-2\fboxrule-2\fboxsep]{
  		\begin{minipage}[t]{\linewidth-2\fboxrule-4\fboxsep}\setlength{\parskip}{3mm}
			\raisebox{-2.5mm}{\sffamily \small{\textcolor{#1}{\MakeUppercase{#2}}}}		
			\par		
  			 #3
 	 		\end{minipage}
	}}
		\vspace{2mm}
	\par
}

\newcommand\bloc[3]{				% Boites convertible html sans bordure
     \needspace{2\baselineskip}
     {\sffamily \small{\textcolor{#1}{\MakeUppercase{#2}}}}    
		\par		
  			 #3
		\par
}

\newcommand\CHelp[1]{
     \CBox{Plum}{\faInfoCircle}{À RETENIR}{#1}
}

\newcommand\CUp[1]{
     \CBox{NavyBlue}{\faThumbsOUp}{EN PRATIQUE}{#1}
}

\newcommand\CInfo[1]{
     \CBox{Sepia}{\faArrowCircleRight}{REMARQUE}{#1}
}

\newcommand\CRedac[1]{
     \CBox{PineGreen}{\faEdit}{BIEN R\'EDIGER}{#1}
}

\newcommand\CError[1]{
     \CBox{Red}{\faExclamationTriangle}{ATTENTION}{#1}
}

\newcommand\TitreExo[2]{
\needspace{4\baselineskip}
 {\sffamily\large EXERCICE #1\ (\emph{#2 points})}
\vspace{5mm}
}

\newcommand\img[2]{
          \includegraphics[width=#2\paperwidth]{\imgdir#1}
}

\newcommand\imgsvg[2]{
       \begin{center}   \includegraphics[width=#2\paperwidth]{\imgsvgdir#1} \end{center}
}


\newcommand\Lien[2]{
     \href{#1}{#2 \tiny \faExternalLink}
}
\newcommand\mcLien[2]{
     \href{https~://www.maths-cours.fr/#1}{#2 \tiny \faExternalLink}
}

\newcommand{\euro}{\eurologo{}}

%================================================================================================================================
%
% Macros - Environement
%
%================================================================================================================================

\newenvironment{tex}{ %
}
{%
}

\newenvironment{indente}{ %
	\setlength\parindent{10mm}
}

{
	\setlength\parindent{0mm}
}

\newenvironment{corrige}{%
     \needspace{3\baselineskip}
     \medskip
     \textbf{\textsc{Corrigé}}
     \medskip
}
{
}

\newenvironment{extern}{%
     \begin{center}
     }
     {
     \end{center}
}

\NewEnviron{code}{%
	\par
     \boite{gray}{\texttt{%
     \BODY
     }}
     \par
}

\newenvironment{vbloc}{% boite sans cadre empeche saut de page
     \begin{minipage}[t]{\linewidth}
     }
     {
     \end{minipage}
}
\NewEnviron{h2}{%
    \needspace{3\baselineskip}
    \vspace{0.6cm}
	\noindent \MakeUppercase{\sffamily \large \BODY}
	\vspace{1mm}\textcolor{mcgris}{\hrule}\vspace{0.4cm}
	\par
}{}

\NewEnviron{h3}{%
    \needspace{3\baselineskip}
	\vspace{5mm}
	\textsc{\BODY}
	\par
}

\NewEnviron{margeneg}{ %
\begin{addmargin}[-1cm]{0cm}
\BODY
\end{addmargin}
}

\NewEnviron{html}{%
}

\begin{document}
\meta{url}{/exercices/fonctions-et-volumes-bac-s-polynesie-2018/}
\meta{pid}{9276}
\meta{titre}{Fonctions et Volumes – Bac S Polynésie 2018}
\meta{type}{exercices}
%
\textbf{\textit{Exercice 2} \hfill 6 points}
\par
\textbf{Commun  à tous les candidats}
\medbreak
Dans cet exercice, on s'intéresse au volume d'une ampoule basse consommation.
\bigbreak
\TitreC{Partie A - Modélisation de la forme de l'ampoule}
\medbreak
Le plan est muni d'un repère orthonormé $(O~;~\overrightarrow{u},~\overrightarrow{v})$.
\par
On considère les points A$(-1~;~1)$, B$(0~;~1)$, C$(4~;~3)$, D$(7~;~0)$, E$(4~;~-3)$, F$(O~;~-1)$ et G$(- 1~;~- 1)$.
\par
On modélise la section de l'ampoule par un plan passant par son axe de révolution à l'aide de la figure ci-dessous~:
\begin{center}
     \begin{extern}%width="400" alt="Coupe ampoule Bac S Polynésie 2018"
          \psset{unit=1.2cm,algebraic=true}
          \begin{pspicture*}(-1.8,-3.5)(7.8,3.5)
               \psgrid[gridlabels=0pt,subgriddiv=1,gridwidth=0.3pt](-2,-4)(8,4)
               \psaxes[linewidth=1pt,Dx=10,Dy=10]{->}(0,0)(-1.8,-3.5)(7.8,3.5)
               \psaxes[linewidth=1.2pt,Dx=10,Dy=10]{->}(0,0)(1,1)
               \uput[d](0.5,0){$\overrightarrow{u}$}
               \uput[l](0,0.5){$\overrightarrow{v}$}
               \psline[linecolor=blue](-1,1)(0,1)\psline[linecolor=blue](-1,-1)(0,-1)
               \psplot[linewidth=1pt,plotpoints=1000,linecolor=blue]{0}{4}{2-cos(x*3.141659/4)}
               \psplot[linewidth=1pt,plotpoints=1000,linecolor=blue]{0}{4}{cos(x*3.141659/4)-2}
               \psarc[linewidth=1pt,linecolor=blue](4,0){3}{-90}{90}
               \psdots(-1,1)(0,1)(4,3)(7,0)(4,-3)(0,-1)(-1,-1)
               \uput[dl](0,0){\small O}\uput[ul](-1,1){\small A}\uput[ur](0,1){\small B}\uput[u](4,3){\small C}
               \uput[ur](7,0){\small D}\uput[d](4,-3){\small E}\uput[dr](0,-1){\small F}\uput[d](-1,-1){\small G}
          \end{pspicture*}
     \end{extern}
\end{center}
\medbreak
La partie de la courbe située au-dessus de l'axe des abscisses se décompose de la manière suivante~:
\begin{indent}
     \begin{itemize}
          \item la portion située entre les points A et B est la représentation graphique de la fonction constante
          $h$ définie sur l'intervalle $[-1~;~0]$ par $h(x) = 1$~;
          \item la portion située entre les points B et C est la représentation graphique d'une fonction $f$ définie sur l'intervalle [0~;~4] par $f(x) = a + b \sin \left(c + \frac{\pi}{4} x\right)$, où $a$, $b$ et $c$ sont des réels non nuls
          fixés et où le réel $c$ appartient à l'intervalle $\left[0~;~\frac{\pi}{2}\right]$~;
          \item la portion située entre les points C et D est un quart de cercle de diamètre [CE].
     \end{itemize}
\end{indent}
La partie de la courbe située en-dessous de l'axe des abscisses est obtenue par symétrie par rapport à l'axe des abscisses.
\medbreak
\begin{enumerate}
     \item
     \begin{enumerate}[label=\alph*.]
          \item On appelle $f'$ la fonction dérivée de la fonction $f$. Pour tout réel $x$ de l'intervalle [0~;~4], déterminer $f'(x)$.
          \item On impose que les tangentes aux points B et C à la représentation graphique de la fonction $f$ soient parallèles à l'axe des abscisses. Déterminer la valeur du réel $c$.
     \end{enumerate}
     \item  Déterminer les réels $a$ et $b$.
\end{enumerate}
\bigbreak
\TitreC{Partie B - Approximation du volume de l'ampoule}
\medbreak
Par rotation de la figure précédente autour de l'axe des abscisses, on obtient un modèle de l'ampoule.
\par
Afin d'en calculer le volume, on la décompose en trois parties comme illustré ci-dessous~:
\begin{center}
     \begin{extern}%width="400" alt="Volume ampoule Bac S Polynésie 2018"
          \psset{unit=1.2cm,algebraic=true}
          \begin{pspicture*}(-2,-4)(8,3.5)
               %\psgrid[gridlabels=0pt,subgriddiv=1,gridwidth=0.3pt]
               \par
               \uput[d](0.5,0){$\overrightarrow{u}$}
               \uput[l](0,0.5){$\overrightarrow{v}$}
               \psline(-1,1)(0,1)\psline(-1,-1)(0,-1)
               \psplot[linewidth=1pt,plotpoints=1000]{0}{4}{2-cos(x*3.141659/4)}
               \psplot[linewidth=1pt,plotpoints=1000]{0}{4}{cos(x*3.141659/4)-2}
               \psarc[linewidth=1pt](4,0){3}{-90}{90}
               \psdots(-1,1)(0,1)(4,3)(7,0)(4,-3)(0,-1)(-1,-1)
               \uput[dl](0,0){\small O}\uput[ul](-1,1){\small A}\uput[ur](0,1){\small B}\uput[u](4,3){\small C}
               \uput[ur](7,0){\small D}\uput[d](4,-3){\small E}\uput[dr](0,-1){\small F}\uput[d](-1,-1){\small G}
               \pscustom[fillstyle=solid,fillcolor=mcmauve]{
                    \psplot[linewidth=1pt,plotpoints=1000]{0}{4}{2-cos(x*3.141659/4)}
               \psplot[linewidth=1pt,plotpoints=1000]{4}{0}{cos(x*3.141659/4)-2}}
               \pscustom[fillstyle=solid,fillcolor=mcvert]{
               \psline(4,3)(4,-3)\psarc[linewidth=1pt](4,0){3}{-90}{90}}
               \psframe[fillstyle=solid,fillcolor=red](-1,-1)(0,1)
               \psaxes[linewidth=1pt,Dx=10,Dy=10](0,0)(-2,-3.5)(8,3.5)
               \psaxes[linewidth=1.5pt,Dx=10,Dy=10]{->}(0,0)(1,1)
          \end{pspicture*}
     \end{extern}
\end{center}
\begin{center}
     Vue dans le plan (BCE)
\end{center}
\medbreak
On rappelle que~:
\begin{indent}
     \begin{itemize}
          \item le volume d'un cylindre est donné par la formule $\pi r^2 h$ où $r$ est le rayon du disque de base et $h$ est la hauteur~;
          \item le volume d'une boule de rayon $r$ est donné par la formule $\dfrac{4}{3}\pi r^3$.
     \end{itemize}
\end{indent}
On admet également que, pour tout réel $x$ de l'intervalle [0~;~4], $f(x) = 2 - \cos \left(\frac{\pi}{4}x\right)$.
\medbreak
\begin{enumerate}
     \item Calculer le volume du cylindre de section le rectangle ABFG.
     \item Calculer le volume de la demi-sphère de section le demi -disque de diamètre [CE].
     \item Pour approcher le volume du solide de section la zone colorée en mauve BCEF, on partage le segment [OO$'$] en $n$ segments de même longueur $\dfrac{4}{n}$ puis on construit $n$ cylindres de même hauteur $\dfrac{4}{n}$.
     \begin{enumerate}[label=\alph*.]
          \item \textbf{Cas particulier~:} dans cette question uniquement on choisit $n = 5$.
          \par
          Calculer le volume du troisième cylindre, grisé dans les figures ci-dessous, puis en donner
          la valeur arrondie à $10^{-2}$.
          \begin{center}
               \begin{extern}%width="400" alt="Cylindres vue plan Bac S Polynésie 2018"
                    \psset{unit=1.2cm,algebraic=true}
                    \begin{pspicture*}(-2,-4)(8,3.5)
                         %\psgrid[gridlabels=0pt,subgriddiv=1,gridwidth=0.3pt]
                         \psaxes[linewidth=1pt,Dx=10,Dy=10](0,0)(-2,-3.5)(8,3.5)
                         \psaxes[linewidth=1.5pt,Dx=10,Dy=10]{->}(0,0)(1,1)
                         %\uput[d](0.5,0){$\overrightarrow{i}$}
                         %\uput[l](0,0.5){$\overrightarrow{j}$}
                         \psline(-1,1)(0,1)\psline(-1,-1)(0,-1)
                         \psplot[linewidth=1pt,plotpoints=1000]{0}{4}{2-cos(x*3.141659/4)}
                         \psplot[linewidth=1pt,plotpoints=1000]{0}{4}{cos(x*3.141659/4)-2}
                         \psarc[linewidth=1pt](4,0){3}{-90}{90}
                         \psdots(-1,1)(0,1)(4,3)(7,0)(4,-3)(0,-1)(-1,-1)
                         \uput[dl](0,0){\small O}
                         %\uput[ul](-1,1){\small A}
                         \uput[ur](0,1){\small B}\uput[u](4,3){\small C}
                         \uput[ur](7,0){\small D}\uput[d](4,-3){\small E}\uput[dr](0,-1){\small F}
                         %\uput[d](-1,-1){\small G}
                         \psframe[fillstyle=solid,fillcolor=mcmauve](0,-1)(0.8,1)
                         \psframe[fillstyle=solid,fillcolor=mcmauve](0.8,-1.2)(1.6,1.2)
                         \psframe[fillstyle=solid,fillcolor=blue](1.6,-1.7)(2.4,1.7)
                         \psframe[fillstyle=solid,fillcolor=mcmauve](2.4,-2.3)(3.2,2.3)
                         \psframe[fillstyle=solid,fillcolor=mcmauve](3.2,-2.8)(4,2.8)
                         \psline(-1,-1)(-1,1)\psline(4,3)(4,-3)
                    \end{pspicture*}
               \end{extern}
          \end{center}
          \begin{center}
               Vue dans le plan (BCE)
          \end{center}
          \begin{center}
               \begin{extern}%width="360" alt="Cylindres vue espace Bac S Polynésie 2018"
                    \psset{unit=1.2cm,algebraic=true}
                    \begin{pspicture*}(-2,-4)(8,3.5)
                         %\psgrid
                         \psline(-1,1)(0,1)\psline(-1,-1)(0,-1)
                         \psplot[linewidth=1pt,plotpoints=1000]{0}{4}{2-cos(x*3.141659/4)}
                         \psplot[linewidth=1pt,plotpoints=1000]{0}{4}{cos(x*3.141659/4)-2}
                         \psarc[linewidth=1pt](4,0){3}{-90}{90}
                         \psline(-1,-1)(-1,1)
                         \psframe[fillstyle=solid,fillcolor=mcmauve,linewidth=0pt](3.2,-2.8)(4,2.8)
                         \psellipse[fillstyle=solid,fillcolor=mcmauve,linewidth=0pt](3.2,0)(0.3,2.8)
                         \psellipse[fillstyle=solid,fillcolor=mcmauve,linewidth=0pt](4,0)(0.3,2.8)
                         \psframe[fillstyle=solid,fillcolor=mcmauve,linewidth=0pt](2.4,-2.3)(3.2,2.3)
                         \psellipse[fillstyle=solid,fillcolor=mcmauve,linewidth=0pt](2.4,0)(0.3,2.3)
                         \psellipse[fillstyle=solid,fillcolor=mcmauve,linewidth=0pt](3.2,0)(0.3,2.3)
                         \psframe[fillstyle=solid,fillcolor=blue,linewidth=0pt](1.6,-1.7)(2.4,1.7)
                         \psellipse[fillstyle=solid,fillcolor=blue,linewidth=0pt](1.6,0)(0.3,1.7)
                         \psellipse[fillstyle=solid,fillcolor=blue,linewidth=0pt](2.4,0)(0.3,1.7)
                         \psframe[fillstyle=solid,fillcolor=mcmauve,linewidth=0pt](0.8,-1.2)(1.6,1.2)
                         \psellipse[fillstyle=solid,fillcolor=mcmauve,linewidth=0pt](0.8,0)(0.3,1.2)
                         \psellipse[fillstyle=solid,fillcolor=mcmauve,linewidth=0pt](1.6,0)(0.3,1.2)
                         \psframe[fillstyle=solid,fillcolor=mcmauve,linewidth=0pt](0,-1)(0.8,1)
                         \psellipse[fillstyle=solid,fillcolor=mcmauve,linewidth=0pt](0,0)(0.3,1)
                         \psellipse[fillstyle=solid,fillcolor=mcmauve,linewidth=0pt](0.8,0)(0.3,1)
                    \end{pspicture*}
               \end{extern}
          \end{center}
          \begin{center}
               Vue dans l'espace
          \end{center}
          \item \textbf{Cas général~:} dans cette question, $n$ désigne un entier naturel quelconque non nul.
          \par
          On approche le volume du solide de section BCEF par la somme des volumes des $n$ cylindres
          ainsi créés en choisissant une valeur de $n$ suffisamment grande.
          \par
          Recopier et compléter l'algorithme suivant de sorte qu'à la fin de son exécution, la variable $V$ contienne la somme des volumes des $n$ cylindres créés lorsque l'on saisit $n$.
          \begin{center}
               \begin{extern}%width="300" alt="Algorithme Bac S Polynésie 2018"
                    \begin{tabularx}{0.45\linewidth}{|l X|}\hline
                         1&$V \gets 0$\\
                         2& Pour $k$ allant de \ldots à \ldots~:\\
                         3& \hspace{0.5cm} $V \gets \ldots$\\
                         4& Fin Pour\\ \hline
                    \end{tabularx}
               \end{extern}
          \end{center}
     \end{enumerate}
\end{enumerate}

\end{document}
µ
\documentclass[a4paper]{article}

%================================================================================================================================
%
% Packages
%
%================================================================================================================================

\usepackage[T1]{fontenc} 	% pour caractères accentués
\usepackage[utf8]{inputenc}  % encodage utf8
\usepackage[french]{babel}	% langue : français
\usepackage{fourier}			% caractères plus lisibles
\usepackage[dvipsnames]{xcolor} % couleurs
\usepackage{fancyhdr}		% réglage header footer
\usepackage{needspace}		% empêcher sauts de page mal placés
\usepackage{graphicx}		% pour inclure des graphiques
\usepackage{enumitem,cprotect}		% personnalise les listes d'items (nécessaire pour ol, al ...)
\usepackage{hyperref}		% Liens hypertexte
\usepackage{pstricks,pst-all,pst-node,pstricks-add,pst-math,pst-plot,pst-tree,pst-eucl} % pstricks
\usepackage[a4paper,includeheadfoot,top=2cm,left=3cm, bottom=2cm,right=3cm]{geometry} % marges etc.
\usepackage{comment}			% commentaires multilignes
\usepackage{amsmath,environ} % maths (matrices, etc.)
\usepackage{amssymb,makeidx}
\usepackage{bm}				% bold maths
\usepackage{tabularx}		% tableaux
\usepackage{colortbl}		% tableaux en couleur
\usepackage{fontawesome}		% Fontawesome
\usepackage{environ}			% environment with command
\usepackage{fp}				% calculs pour ps-tricks
\usepackage{multido}			% pour ps tricks
\usepackage[np]{numprint}	% formattage nombre
\usepackage{tikz,tkz-tab} 			% package principal TikZ
\usepackage{pgfplots}   % axes
\usepackage{mathrsfs}    % cursives
\usepackage{calc}			% calcul taille boites
\usepackage[scaled=0.875]{helvet} % font sans serif
\usepackage{svg} % svg
\usepackage{scrextend} % local margin
\usepackage{scratch} %scratch
\usepackage{multicol} % colonnes
%\usepackage{infix-RPN,pst-func} % formule en notation polanaise inversée
\usepackage{listings}

%================================================================================================================================
%
% Réglages de base
%
%================================================================================================================================

\lstset{
language=Python,   % R code
literate=
{á}{{\'a}}1
{à}{{\`a}}1
{ã}{{\~a}}1
{é}{{\'e}}1
{è}{{\`e}}1
{ê}{{\^e}}1
{í}{{\'i}}1
{ó}{{\'o}}1
{õ}{{\~o}}1
{ú}{{\'u}}1
{ü}{{\"u}}1
{ç}{{\c{c}}}1
{~}{{ }}1
}


\definecolor{codegreen}{rgb}{0,0.6,0}
\definecolor{codegray}{rgb}{0.5,0.5,0.5}
\definecolor{codepurple}{rgb}{0.58,0,0.82}
\definecolor{backcolour}{rgb}{0.95,0.95,0.92}

\lstdefinestyle{mystyle}{
    backgroundcolor=\color{backcolour},   
    commentstyle=\color{codegreen},
    keywordstyle=\color{magenta},
    numberstyle=\tiny\color{codegray},
    stringstyle=\color{codepurple},
    basicstyle=\ttfamily\footnotesize,
    breakatwhitespace=false,         
    breaklines=true,                 
    captionpos=b,                    
    keepspaces=true,                 
    numbers=left,                    
xleftmargin=2em,
framexleftmargin=2em,            
    showspaces=false,                
    showstringspaces=false,
    showtabs=false,                  
    tabsize=2,
    upquote=true
}

\lstset{style=mystyle}


\lstset{style=mystyle}
\newcommand{\imgdir}{C:/laragon/www/newmc/assets/imgsvg/}
\newcommand{\imgsvgdir}{C:/laragon/www/newmc/assets/imgsvg/}

\definecolor{mcgris}{RGB}{220, 220, 220}% ancien~; pour compatibilité
\definecolor{mcbleu}{RGB}{52, 152, 219}
\definecolor{mcvert}{RGB}{125, 194, 70}
\definecolor{mcmauve}{RGB}{154, 0, 215}
\definecolor{mcorange}{RGB}{255, 96, 0}
\definecolor{mcturquoise}{RGB}{0, 153, 153}
\definecolor{mcrouge}{RGB}{255, 0, 0}
\definecolor{mclightvert}{RGB}{205, 234, 190}

\definecolor{gris}{RGB}{220, 220, 220}
\definecolor{bleu}{RGB}{52, 152, 219}
\definecolor{vert}{RGB}{125, 194, 70}
\definecolor{mauve}{RGB}{154, 0, 215}
\definecolor{orange}{RGB}{255, 96, 0}
\definecolor{turquoise}{RGB}{0, 153, 153}
\definecolor{rouge}{RGB}{255, 0, 0}
\definecolor{lightvert}{RGB}{205, 234, 190}
\setitemize[0]{label=\color{lightvert}  $\bullet$}

\pagestyle{fancy}
\renewcommand{\headrulewidth}{0.2pt}
\fancyhead[L]{maths-cours.fr}
\fancyhead[R]{\thepage}
\renewcommand{\footrulewidth}{0.2pt}
\fancyfoot[C]{}

\newcolumntype{C}{>{\centering\arraybackslash}X}
\newcolumntype{s}{>{\hsize=.35\hsize\arraybackslash}X}

\setlength{\parindent}{0pt}		 
\setlength{\parskip}{3mm}
\setlength{\headheight}{1cm}

\def\ebook{ebook}
\def\book{book}
\def\web{web}
\def\type{web}

\newcommand{\vect}[1]{\overrightarrow{\,\mathstrut#1\,}}

\def\Oij{$\left(\text{O}~;~\vect{\imath},~\vect{\jmath}\right)$}
\def\Oijk{$\left(\text{O}~;~\vect{\imath},~\vect{\jmath},~\vect{k}\right)$}
\def\Ouv{$\left(\text{O}~;~\vect{u},~\vect{v}\right)$}

\hypersetup{breaklinks=true, colorlinks = true, linkcolor = OliveGreen, urlcolor = OliveGreen, citecolor = OliveGreen, pdfauthor={Didier BONNEL - https://www.maths-cours.fr} } % supprime les bordures autour des liens

\renewcommand{\arg}[0]{\text{arg}}

\everymath{\displaystyle}

%================================================================================================================================
%
% Macros - Commandes
%
%================================================================================================================================

\newcommand\meta[2]{    			% Utilisé pour créer le post HTML.
	\def\titre{titre}
	\def\url{url}
	\def\arg{#1}
	\ifx\titre\arg
		\newcommand\maintitle{#2}
		\fancyhead[L]{#2}
		{\Large\sffamily \MakeUppercase{#2}}
		\vspace{1mm}\textcolor{mcvert}{\hrule}
	\fi 
	\ifx\url\arg
		\fancyfoot[L]{\href{https://www.maths-cours.fr#2}{\black \footnotesize{https://www.maths-cours.fr#2}}}
	\fi 
}


\newcommand\TitreC[1]{    		% Titre centré
     \needspace{3\baselineskip}
     \begin{center}\textbf{#1}\end{center}
}

\newcommand\newpar{    		% paragraphe
     \par
}

\newcommand\nosp {    		% commande vide (pas d'espace)
}
\newcommand{\id}[1]{} %ignore

\newcommand\boite[2]{				% Boite simple sans titre
	\vspace{5mm}
	\setlength{\fboxrule}{0.2mm}
	\setlength{\fboxsep}{5mm}	
	\fcolorbox{#1}{#1!3}{\makebox[\linewidth-2\fboxrule-2\fboxsep]{
  		\begin{minipage}[t]{\linewidth-2\fboxrule-4\fboxsep}\setlength{\parskip}{3mm}
  			 #2
  		\end{minipage}
	}}
	\vspace{5mm}
}

\newcommand\CBox[4]{				% Boites
	\vspace{5mm}
	\setlength{\fboxrule}{0.2mm}
	\setlength{\fboxsep}{5mm}
	
	\fcolorbox{#1}{#1!3}{\makebox[\linewidth-2\fboxrule-2\fboxsep]{
		\begin{minipage}[t]{1cm}\setlength{\parskip}{3mm}
	  		\textcolor{#1}{\LARGE{#2}}    
 	 	\end{minipage}  
  		\begin{minipage}[t]{\linewidth-2\fboxrule-4\fboxsep}\setlength{\parskip}{3mm}
			\raisebox{1.2mm}{\normalsize\sffamily{\textcolor{#1}{#3}}}						
  			 #4
  		\end{minipage}
	}}
	\vspace{5mm}
}

\newcommand\cadre[3]{				% Boites convertible html
	\par
	\vspace{2mm}
	\setlength{\fboxrule}{0.1mm}
	\setlength{\fboxsep}{5mm}
	\fcolorbox{#1}{white}{\makebox[\linewidth-2\fboxrule-2\fboxsep]{
  		\begin{minipage}[t]{\linewidth-2\fboxrule-4\fboxsep}\setlength{\parskip}{3mm}
			\raisebox{-2.5mm}{\sffamily \small{\textcolor{#1}{\MakeUppercase{#2}}}}		
			\par		
  			 #3
 	 		\end{minipage}
	}}
		\vspace{2mm}
	\par
}

\newcommand\bloc[3]{				% Boites convertible html sans bordure
     \needspace{2\baselineskip}
     {\sffamily \small{\textcolor{#1}{\MakeUppercase{#2}}}}    
		\par		
  			 #3
		\par
}

\newcommand\CHelp[1]{
     \CBox{Plum}{\faInfoCircle}{À RETENIR}{#1}
}

\newcommand\CUp[1]{
     \CBox{NavyBlue}{\faThumbsOUp}{EN PRATIQUE}{#1}
}

\newcommand\CInfo[1]{
     \CBox{Sepia}{\faArrowCircleRight}{REMARQUE}{#1}
}

\newcommand\CRedac[1]{
     \CBox{PineGreen}{\faEdit}{BIEN R\'EDIGER}{#1}
}

\newcommand\CError[1]{
     \CBox{Red}{\faExclamationTriangle}{ATTENTION}{#1}
}

\newcommand\TitreExo[2]{
\needspace{4\baselineskip}
 {\sffamily\large EXERCICE #1\ (\emph{#2 points})}
\vspace{5mm}
}

\newcommand\img[2]{
          \includegraphics[width=#2\paperwidth]{\imgdir#1}
}

\newcommand\imgsvg[2]{
       \begin{center}   \includegraphics[width=#2\paperwidth]{\imgsvgdir#1} \end{center}
}


\newcommand\Lien[2]{
     \href{#1}{#2 \tiny \faExternalLink}
}
\newcommand\mcLien[2]{
     \href{https~://www.maths-cours.fr/#1}{#2 \tiny \faExternalLink}
}

\newcommand{\euro}{\eurologo{}}

%================================================================================================================================
%
% Macros - Environement
%
%================================================================================================================================

\newenvironment{tex}{ %
}
{%
}

\newenvironment{indente}{ %
	\setlength\parindent{10mm}
}

{
	\setlength\parindent{0mm}
}

\newenvironment{corrige}{%
     \needspace{3\baselineskip}
     \medskip
     \textbf{\textsc{Corrigé}}
     \medskip
}
{
}

\newenvironment{extern}{%
     \begin{center}
     }
     {
     \end{center}
}

\NewEnviron{code}{%
	\par
     \boite{gray}{\texttt{%
     \BODY
     }}
     \par
}

\newenvironment{vbloc}{% boite sans cadre empeche saut de page
     \begin{minipage}[t]{\linewidth}
     }
     {
     \end{minipage}
}
\NewEnviron{h2}{%
    \needspace{3\baselineskip}
    \vspace{0.6cm}
	\noindent \MakeUppercase{\sffamily \large \BODY}
	\vspace{1mm}\textcolor{mcgris}{\hrule}\vspace{0.4cm}
	\par
}{}

\NewEnviron{h3}{%
    \needspace{3\baselineskip}
	\vspace{5mm}
	\textsc{\BODY}
	\par
}

\NewEnviron{margeneg}{ %
\begin{addmargin}[-1cm]{0cm}
\BODY
\end{addmargin}
}

\NewEnviron{html}{%
}

\begin{document}
\meta{url}{/exercices/fonctions-et-aires-bac-s-polynesie-2018/}
\meta{pid}{9278}
\meta{titre}{Fonctions et Aires – Bac S Polynésie 2018}
\meta{type}{exercices}
%
\begin{h2}Exercice 3 (4 points)\end{h2}
\par
\textbf{Commun  à tous les candidats}
\medbreak
On considère la fonction $f$ définie sur l'intervalle $[0~;~ +\infty[$ par $f(x) = k\text{e}^{-kx}$  où $k$ est un nombre réel strictement positif.
\par
On appelle $\mathscr{C}_f$ sa représentation graphique dans le repère orthonormé $(O~;~\overrightarrow{u},~\overrightarrow{v})$.
\par
On considère le point A de la courbe $\mathscr{C}_f$ d'abscisse 0 et le point B de la courbe $\mathscr{C}_f$ d'abscisse 1.
\par
Le point C a pour coordonnées (1~;~0).
\par
\begin{center}
     \begin{extern}%width="300" alt=""
          \psset{unit=3cm}
          \begin{pspicture}(-0.2,-0.1)(2.2,2)
               \psaxes[linewidth=1.pt,Dx=10,Dy=10](0,0)(0,0)(2.2,2)
               \psaxes[linewidth=1.5pt,Dx=10,Dy=10]{->}(0,0)(1,1)
               \par
               \pscustom[fillstyle=solid,fillcolor=blue,opacity=0.1]{
                    \psplot[plotpoints=1000,linewidth=1.25pt,linecolor=blue]{0}{1}{1.5 2.71828 1.5 x mul exp div}
               \psline(1,0.335)(0,0)}
               \rput(0.3,0.5){\blue $\mathcal{D}$}
               \rput(0.2,1.5){\blue $\mathcal{C}_f$}
               \uput[d](0.5,0){$\overrightarrow{u}$}
               \uput[l](0,0.5){$\overrightarrow{v}$}
               \psdots(0,1.5)(1,0.335)(1,0)%ABC
               \uput[l](0,1.5){A}\uput[u](1,0.335){B}\uput[d](1,0){C}\uput[l](0,0){O}
               \psline(1,0.335)(1,0)
               \psplot[plotpoints=3000,linewidth=1.25pt,linecolor=blue]{0}{2.2}{1.5 2.71828 1.5 x mul exp div}
          \end{pspicture}
     \end{extern}
\end{center}
\medbreak
\begin{enumerate}
     \item Déterminer une primitive de la fonction $f$ sur l'intervalle $[0~;~ +\infty[$.
     \item Exprimer, en fonction de $k$, l'aire du triangle OCB et celle du domaine $\mathscr{D}$ délimité par l'axe des ordonnées, la courbe $\mathscr{C}_f$ et le segment [OB].
     \item Montrer qu'il existe une unique valeur du réel $k$ strictement positive telle que l'aire du domaine $\mathscr{D}$ vaut le double de celle du triangle OCB.
\end{enumerate}

\end{document}
µ
\documentclass[a4paper]{article}

%================================================================================================================================
%
% Packages
%
%================================================================================================================================

\usepackage[T1]{fontenc} 	% pour caractères accentués
\usepackage[utf8]{inputenc}  % encodage utf8
\usepackage[french]{babel}	% langue : français
\usepackage{fourier}			% caractères plus lisibles
\usepackage[dvipsnames]{xcolor} % couleurs
\usepackage{fancyhdr}		% réglage header footer
\usepackage{needspace}		% empêcher sauts de page mal placés
\usepackage{graphicx}		% pour inclure des graphiques
\usepackage{enumitem,cprotect}		% personnalise les listes d'items (nécessaire pour ol, al ...)
\usepackage{hyperref}		% Liens hypertexte
\usepackage{pstricks,pst-all,pst-node,pstricks-add,pst-math,pst-plot,pst-tree,pst-eucl} % pstricks
\usepackage[a4paper,includeheadfoot,top=2cm,left=3cm, bottom=2cm,right=3cm]{geometry} % marges etc.
\usepackage{comment}			% commentaires multilignes
\usepackage{amsmath,environ} % maths (matrices, etc.)
\usepackage{amssymb,makeidx}
\usepackage{bm}				% bold maths
\usepackage{tabularx}		% tableaux
\usepackage{colortbl}		% tableaux en couleur
\usepackage{fontawesome}		% Fontawesome
\usepackage{environ}			% environment with command
\usepackage{fp}				% calculs pour ps-tricks
\usepackage{multido}			% pour ps tricks
\usepackage[np]{numprint}	% formattage nombre
\usepackage{tikz,tkz-tab} 			% package principal TikZ
\usepackage{pgfplots}   % axes
\usepackage{mathrsfs}    % cursives
\usepackage{calc}			% calcul taille boites
\usepackage[scaled=0.875]{helvet} % font sans serif
\usepackage{svg} % svg
\usepackage{scrextend} % local margin
\usepackage{scratch} %scratch
\usepackage{multicol} % colonnes
%\usepackage{infix-RPN,pst-func} % formule en notation polanaise inversée
\usepackage{listings}

%================================================================================================================================
%
% Réglages de base
%
%================================================================================================================================

\lstset{
language=Python,   % R code
literate=
{á}{{\'a}}1
{à}{{\`a}}1
{ã}{{\~a}}1
{é}{{\'e}}1
{è}{{\`e}}1
{ê}{{\^e}}1
{í}{{\'i}}1
{ó}{{\'o}}1
{õ}{{\~o}}1
{ú}{{\'u}}1
{ü}{{\"u}}1
{ç}{{\c{c}}}1
{~}{{ }}1
}


\definecolor{codegreen}{rgb}{0,0.6,0}
\definecolor{codegray}{rgb}{0.5,0.5,0.5}
\definecolor{codepurple}{rgb}{0.58,0,0.82}
\definecolor{backcolour}{rgb}{0.95,0.95,0.92}

\lstdefinestyle{mystyle}{
    backgroundcolor=\color{backcolour},   
    commentstyle=\color{codegreen},
    keywordstyle=\color{magenta},
    numberstyle=\tiny\color{codegray},
    stringstyle=\color{codepurple},
    basicstyle=\ttfamily\footnotesize,
    breakatwhitespace=false,         
    breaklines=true,                 
    captionpos=b,                    
    keepspaces=true,                 
    numbers=left,                    
xleftmargin=2em,
framexleftmargin=2em,            
    showspaces=false,                
    showstringspaces=false,
    showtabs=false,                  
    tabsize=2,
    upquote=true
}

\lstset{style=mystyle}


\lstset{style=mystyle}
\newcommand{\imgdir}{C:/laragon/www/newmc/assets/imgsvg/}
\newcommand{\imgsvgdir}{C:/laragon/www/newmc/assets/imgsvg/}

\definecolor{mcgris}{RGB}{220, 220, 220}% ancien~; pour compatibilité
\definecolor{mcbleu}{RGB}{52, 152, 219}
\definecolor{mcvert}{RGB}{125, 194, 70}
\definecolor{mcmauve}{RGB}{154, 0, 215}
\definecolor{mcorange}{RGB}{255, 96, 0}
\definecolor{mcturquoise}{RGB}{0, 153, 153}
\definecolor{mcrouge}{RGB}{255, 0, 0}
\definecolor{mclightvert}{RGB}{205, 234, 190}

\definecolor{gris}{RGB}{220, 220, 220}
\definecolor{bleu}{RGB}{52, 152, 219}
\definecolor{vert}{RGB}{125, 194, 70}
\definecolor{mauve}{RGB}{154, 0, 215}
\definecolor{orange}{RGB}{255, 96, 0}
\definecolor{turquoise}{RGB}{0, 153, 153}
\definecolor{rouge}{RGB}{255, 0, 0}
\definecolor{lightvert}{RGB}{205, 234, 190}
\setitemize[0]{label=\color{lightvert}  $\bullet$}

\pagestyle{fancy}
\renewcommand{\headrulewidth}{0.2pt}
\fancyhead[L]{maths-cours.fr}
\fancyhead[R]{\thepage}
\renewcommand{\footrulewidth}{0.2pt}
\fancyfoot[C]{}

\newcolumntype{C}{>{\centering\arraybackslash}X}
\newcolumntype{s}{>{\hsize=.35\hsize\arraybackslash}X}

\setlength{\parindent}{0pt}		 
\setlength{\parskip}{3mm}
\setlength{\headheight}{1cm}

\def\ebook{ebook}
\def\book{book}
\def\web{web}
\def\type{web}

\newcommand{\vect}[1]{\overrightarrow{\,\mathstrut#1\,}}

\def\Oij{$\left(\text{O}~;~\vect{\imath},~\vect{\jmath}\right)$}
\def\Oijk{$\left(\text{O}~;~\vect{\imath},~\vect{\jmath},~\vect{k}\right)$}
\def\Ouv{$\left(\text{O}~;~\vect{u},~\vect{v}\right)$}

\hypersetup{breaklinks=true, colorlinks = true, linkcolor = OliveGreen, urlcolor = OliveGreen, citecolor = OliveGreen, pdfauthor={Didier BONNEL - https://www.maths-cours.fr} } % supprime les bordures autour des liens

\renewcommand{\arg}[0]{\text{arg}}

\everymath{\displaystyle}

%================================================================================================================================
%
% Macros - Commandes
%
%================================================================================================================================

\newcommand\meta[2]{    			% Utilisé pour créer le post HTML.
	\def\titre{titre}
	\def\url{url}
	\def\arg{#1}
	\ifx\titre\arg
		\newcommand\maintitle{#2}
		\fancyhead[L]{#2}
		{\Large\sffamily \MakeUppercase{#2}}
		\vspace{1mm}\textcolor{mcvert}{\hrule}
	\fi 
	\ifx\url\arg
		\fancyfoot[L]{\href{https://www.maths-cours.fr#2}{\black \footnotesize{https://www.maths-cours.fr#2}}}
	\fi 
}


\newcommand\TitreC[1]{    		% Titre centré
     \needspace{3\baselineskip}
     \begin{center}\textbf{#1}\end{center}
}

\newcommand\newpar{    		% paragraphe
     \par
}

\newcommand\nosp {    		% commande vide (pas d'espace)
}
\newcommand{\id}[1]{} %ignore

\newcommand\boite[2]{				% Boite simple sans titre
	\vspace{5mm}
	\setlength{\fboxrule}{0.2mm}
	\setlength{\fboxsep}{5mm}	
	\fcolorbox{#1}{#1!3}{\makebox[\linewidth-2\fboxrule-2\fboxsep]{
  		\begin{minipage}[t]{\linewidth-2\fboxrule-4\fboxsep}\setlength{\parskip}{3mm}
  			 #2
  		\end{minipage}
	}}
	\vspace{5mm}
}

\newcommand\CBox[4]{				% Boites
	\vspace{5mm}
	\setlength{\fboxrule}{0.2mm}
	\setlength{\fboxsep}{5mm}
	
	\fcolorbox{#1}{#1!3}{\makebox[\linewidth-2\fboxrule-2\fboxsep]{
		\begin{minipage}[t]{1cm}\setlength{\parskip}{3mm}
	  		\textcolor{#1}{\LARGE{#2}}    
 	 	\end{minipage}  
  		\begin{minipage}[t]{\linewidth-2\fboxrule-4\fboxsep}\setlength{\parskip}{3mm}
			\raisebox{1.2mm}{\normalsize\sffamily{\textcolor{#1}{#3}}}						
  			 #4
  		\end{minipage}
	}}
	\vspace{5mm}
}

\newcommand\cadre[3]{				% Boites convertible html
	\par
	\vspace{2mm}
	\setlength{\fboxrule}{0.1mm}
	\setlength{\fboxsep}{5mm}
	\fcolorbox{#1}{white}{\makebox[\linewidth-2\fboxrule-2\fboxsep]{
  		\begin{minipage}[t]{\linewidth-2\fboxrule-4\fboxsep}\setlength{\parskip}{3mm}
			\raisebox{-2.5mm}{\sffamily \small{\textcolor{#1}{\MakeUppercase{#2}}}}		
			\par		
  			 #3
 	 		\end{minipage}
	}}
		\vspace{2mm}
	\par
}

\newcommand\bloc[3]{				% Boites convertible html sans bordure
     \needspace{2\baselineskip}
     {\sffamily \small{\textcolor{#1}{\MakeUppercase{#2}}}}    
		\par		
  			 #3
		\par
}

\newcommand\CHelp[1]{
     \CBox{Plum}{\faInfoCircle}{À RETENIR}{#1}
}

\newcommand\CUp[1]{
     \CBox{NavyBlue}{\faThumbsOUp}{EN PRATIQUE}{#1}
}

\newcommand\CInfo[1]{
     \CBox{Sepia}{\faArrowCircleRight}{REMARQUE}{#1}
}

\newcommand\CRedac[1]{
     \CBox{PineGreen}{\faEdit}{BIEN R\'EDIGER}{#1}
}

\newcommand\CError[1]{
     \CBox{Red}{\faExclamationTriangle}{ATTENTION}{#1}
}

\newcommand\TitreExo[2]{
\needspace{4\baselineskip}
 {\sffamily\large EXERCICE #1\ (\emph{#2 points})}
\vspace{5mm}
}

\newcommand\img[2]{
          \includegraphics[width=#2\paperwidth]{\imgdir#1}
}

\newcommand\imgsvg[2]{
       \begin{center}   \includegraphics[width=#2\paperwidth]{\imgsvgdir#1} \end{center}
}


\newcommand\Lien[2]{
     \href{#1}{#2 \tiny \faExternalLink}
}
\newcommand\mcLien[2]{
     \href{https~://www.maths-cours.fr/#1}{#2 \tiny \faExternalLink}
}

\newcommand{\euro}{\eurologo{}}

%================================================================================================================================
%
% Macros - Environement
%
%================================================================================================================================

\newenvironment{tex}{ %
}
{%
}

\newenvironment{indente}{ %
	\setlength\parindent{10mm}
}

{
	\setlength\parindent{0mm}
}

\newenvironment{corrige}{%
     \needspace{3\baselineskip}
     \medskip
     \textbf{\textsc{Corrigé}}
     \medskip
}
{
}

\newenvironment{extern}{%
     \begin{center}
     }
     {
     \end{center}
}

\NewEnviron{code}{%
	\par
     \boite{gray}{\texttt{%
     \BODY
     }}
     \par
}

\newenvironment{vbloc}{% boite sans cadre empeche saut de page
     \begin{minipage}[t]{\linewidth}
     }
     {
     \end{minipage}
}
\NewEnviron{h2}{%
    \needspace{3\baselineskip}
    \vspace{0.6cm}
	\noindent \MakeUppercase{\sffamily \large \BODY}
	\vspace{1mm}\textcolor{mcgris}{\hrule}\vspace{0.4cm}
	\par
}{}

\NewEnviron{h3}{%
    \needspace{3\baselineskip}
	\vspace{5mm}
	\textsc{\BODY}
	\par
}

\NewEnviron{margeneg}{ %
\begin{addmargin}[-1cm]{0cm}
\BODY
\end{addmargin}
}

\NewEnviron{html}{%
}

\begin{document}
\meta{url}{/exercices/suites-bac-s-polynesie-2018/}
\meta{pid}{9280}
\meta{titre}{Suites – Bac S Polynésie 2018}
\meta{type}{exercices}
%
\begin{h2}Exercice 4 (5 points)\end{h2}
\textbf{Candidats n'ayant pas choisi  l'enseignement de spécialité \og mathématiques \fg{} }
\medbreak
Un lapin se déplace dans un terrier composé de trois galeries, notées A, B et C, dans chacune desquelles il est confronté à un stimulus particulier.
\par
À chaque fois qu'il est soumis à un stimulus, le lapin reste dans la galerie où il se trouve ou change de galerie. Cela constitue une étape.
\smallbreak
Soit $n$ un entier naturel.
\par
On note $a_n$ la probabilité de l'événement~: \og le lapin est dans la galerie A à l'étape $n$ \fg.\\
On note $b_n$ la probabilité de l'événement~: \og le lapin est dans la galerie B à l'étape $n $\fg.\\
On note $c_n$ la probabilité de l'événement~: \og le lapin est dans la galerie C à l'étape $n $\fg.
\par
À l'étape $n = 0$, le lapin est dans la galerie A.
\par
Une étude antérieure des réactions du lapin face aux différents stimuli permet de modéliser ses déplacements par le système suivant~:
\par
\[\left\{\begin{array}{l c r}
          a_{n+1}&=&\dfrac{1}{3}a_n + \dfrac{1}{4} b_n \phantom{+ \dfrac{2}{3}c_n}\\
          \\
          b_{n+1}&=&\dfrac{2}{3}a_n + \dfrac{1}{2} b_n + \dfrac{2}{3}c_n\\
          \\
          c_{n+1}&=&\dfrac{1}{4}b_n + \dfrac{1}{3} c_n
\end{array}\right.\]
\medbreak
L'objectif de cet exercice est d'estimer dans quelle galerie le lapin a la plus grande probabilité de se trouver à long terme.
\bigbreak
\TitreC{Partie A}
\medbreak
À l'aide d'un tableur, on obtient le tableau de valeurs suivant~:
\begin{center}
     \begin{tabularx}{0.75\linewidth}{|c|*{4}{>{\centering \arraybackslash}X|}}\hline%class="compact"
          &A &B &C &D\\ \hline
          1 	&$n$ &$a_n$ &$b_n$ &$c_n$\\ \hline
          2 	&0	& 1 	&0	&0\\ \hline
          3	&1	&0,333 	&0,667 	&0\\ \hline
          4 	&2 	&0,278 &0,556 &0,167\\ \hline
          5 	&3 &0,231 &0,574 &0,194\\ \hline
          6 	&4 &0,221 &0,571 &0,208\\ \hline
          7 	&5 &0,216 &0,572 &0,212\\ \hline
          8 	&6 &0,215 &0,571 &0,214\\ \hline
          9 	&7 &0,215 &0,571 &0,214\\ \hline
          10 	&8 &0,214 &0,571 &0,214\\ \hline
          11 	&9 &0,214 &0,571 &0,214\\ \hline
          12 	&10 &0,214 &0,571 &0,214\\ \hline
     \end{tabularx}
\end{center}
\medbreak
\begin{enumerate}
     \item Quelle formule faut-il entrer dans la cellule C3 et recopier vers le bas pour remplir la colonne C~?
     \item  Quelle conjecture peut-on émettre~?
\end{enumerate}
\bigbreak
\TitreC{Partie B}
\medbreak
\begin{enumerate}
     \item On définit la suite $\left(u_n\right)$, pour tout entier naturel $n$, par $u_n = a_n - c_n$.
     \begin{enumerate}[label=\alph*.]
          \item Démontrer que la suite $\left(u_n\right)$ est géométrique en précisant sa raison.
          \item Donner, pour tout entier naturel $n$, l'expression de $u_n$ en fonction de $n$.
     \end{enumerate}
     \item  On définit la suite $\left(v_n\right)$ par $v_n = b_n - \dfrac{4}{7}$ pour tout entier naturel $n$.
     \begin{enumerate}[label=\alph*.]
          \item Expliquer pourquoi pour tout entier naturel $n$,\: $a_n + b_n + c_n = 1$ et en déduire que pour tout
          entier naturel $n$,\: $v_{n+1} = - \dfrac{1}{6}v_n$.
          \item En déduire, pour tout entier naturel $n$, l'expression de $v_n$ en fonction de $n$.
     \end{enumerate}
     \item  En déduire que pour tout entier naturel $n$, on a~:
     \begin{center}
          $a_{n} = \dfrac{3}{14} +\dfrac{1}{2}\left(\dfrac{1}{3}\right)^n + \dfrac{2}{7}\left(- \dfrac{1}{6}\right)^n$, \\ $b_{n} = \dfrac{4}{7} - \dfrac{4}{7}\left(- \dfrac{1}{6}\right)^n $ \\et $c_{n} = \dfrac{3}{14} -\dfrac{1}{2}\left(\dfrac{1}{3}\right)^n + \dfrac{2}{7}\left(- \dfrac{1}{6}\right)^n.$
     \end{center}
     \item  Que peut-on en déduire sur la position du lapin après un très grand nombre d'étapes~?
\end{enumerate}

\end{document}
µ
\documentclass[a4paper]{article}

%================================================================================================================================
%
% Packages
%
%================================================================================================================================

\usepackage[T1]{fontenc} 	% pour caractères accentués
\usepackage[utf8]{inputenc}  % encodage utf8
\usepackage[french]{babel}	% langue : français
\usepackage{fourier}			% caractères plus lisibles
\usepackage[dvipsnames]{xcolor} % couleurs
\usepackage{fancyhdr}		% réglage header footer
\usepackage{needspace}		% empêcher sauts de page mal placés
\usepackage{graphicx}		% pour inclure des graphiques
\usepackage{enumitem,cprotect}		% personnalise les listes d'items (nécessaire pour ol, al ...)
\usepackage{hyperref}		% Liens hypertexte
\usepackage{pstricks,pst-all,pst-node,pstricks-add,pst-math,pst-plot,pst-tree,pst-eucl} % pstricks
\usepackage[a4paper,includeheadfoot,top=2cm,left=3cm, bottom=2cm,right=3cm]{geometry} % marges etc.
\usepackage{comment}			% commentaires multilignes
\usepackage{amsmath,environ} % maths (matrices, etc.)
\usepackage{amssymb,makeidx}
\usepackage{bm}				% bold maths
\usepackage{tabularx}		% tableaux
\usepackage{colortbl}		% tableaux en couleur
\usepackage{fontawesome}		% Fontawesome
\usepackage{environ}			% environment with command
\usepackage{fp}				% calculs pour ps-tricks
\usepackage{multido}			% pour ps tricks
\usepackage[np]{numprint}	% formattage nombre
\usepackage{tikz,tkz-tab} 			% package principal TikZ
\usepackage{pgfplots}   % axes
\usepackage{mathrsfs}    % cursives
\usepackage{calc}			% calcul taille boites
\usepackage[scaled=0.875]{helvet} % font sans serif
\usepackage{svg} % svg
\usepackage{scrextend} % local margin
\usepackage{scratch} %scratch
\usepackage{multicol} % colonnes
%\usepackage{infix-RPN,pst-func} % formule en notation polanaise inversée
\usepackage{listings}

%================================================================================================================================
%
% Réglages de base
%
%================================================================================================================================

\lstset{
language=Python,   % R code
literate=
{á}{{\'a}}1
{à}{{\`a}}1
{ã}{{\~a}}1
{é}{{\'e}}1
{è}{{\`e}}1
{ê}{{\^e}}1
{í}{{\'i}}1
{ó}{{\'o}}1
{õ}{{\~o}}1
{ú}{{\'u}}1
{ü}{{\"u}}1
{ç}{{\c{c}}}1
{~}{{ }}1
}


\definecolor{codegreen}{rgb}{0,0.6,0}
\definecolor{codegray}{rgb}{0.5,0.5,0.5}
\definecolor{codepurple}{rgb}{0.58,0,0.82}
\definecolor{backcolour}{rgb}{0.95,0.95,0.92}

\lstdefinestyle{mystyle}{
    backgroundcolor=\color{backcolour},   
    commentstyle=\color{codegreen},
    keywordstyle=\color{magenta},
    numberstyle=\tiny\color{codegray},
    stringstyle=\color{codepurple},
    basicstyle=\ttfamily\footnotesize,
    breakatwhitespace=false,         
    breaklines=true,                 
    captionpos=b,                    
    keepspaces=true,                 
    numbers=left,                    
xleftmargin=2em,
framexleftmargin=2em,            
    showspaces=false,                
    showstringspaces=false,
    showtabs=false,                  
    tabsize=2,
    upquote=true
}

\lstset{style=mystyle}


\lstset{style=mystyle}
\newcommand{\imgdir}{C:/laragon/www/newmc/assets/imgsvg/}
\newcommand{\imgsvgdir}{C:/laragon/www/newmc/assets/imgsvg/}

\definecolor{mcgris}{RGB}{220, 220, 220}% ancien~; pour compatibilité
\definecolor{mcbleu}{RGB}{52, 152, 219}
\definecolor{mcvert}{RGB}{125, 194, 70}
\definecolor{mcmauve}{RGB}{154, 0, 215}
\definecolor{mcorange}{RGB}{255, 96, 0}
\definecolor{mcturquoise}{RGB}{0, 153, 153}
\definecolor{mcrouge}{RGB}{255, 0, 0}
\definecolor{mclightvert}{RGB}{205, 234, 190}

\definecolor{gris}{RGB}{220, 220, 220}
\definecolor{bleu}{RGB}{52, 152, 219}
\definecolor{vert}{RGB}{125, 194, 70}
\definecolor{mauve}{RGB}{154, 0, 215}
\definecolor{orange}{RGB}{255, 96, 0}
\definecolor{turquoise}{RGB}{0, 153, 153}
\definecolor{rouge}{RGB}{255, 0, 0}
\definecolor{lightvert}{RGB}{205, 234, 190}
\setitemize[0]{label=\color{lightvert}  $\bullet$}

\pagestyle{fancy}
\renewcommand{\headrulewidth}{0.2pt}
\fancyhead[L]{maths-cours.fr}
\fancyhead[R]{\thepage}
\renewcommand{\footrulewidth}{0.2pt}
\fancyfoot[C]{}

\newcolumntype{C}{>{\centering\arraybackslash}X}
\newcolumntype{s}{>{\hsize=.35\hsize\arraybackslash}X}

\setlength{\parindent}{0pt}		 
\setlength{\parskip}{3mm}
\setlength{\headheight}{1cm}

\def\ebook{ebook}
\def\book{book}
\def\web{web}
\def\type{web}

\newcommand{\vect}[1]{\overrightarrow{\,\mathstrut#1\,}}

\def\Oij{$\left(\text{O}~;~\vect{\imath},~\vect{\jmath}\right)$}
\def\Oijk{$\left(\text{O}~;~\vect{\imath},~\vect{\jmath},~\vect{k}\right)$}
\def\Ouv{$\left(\text{O}~;~\vect{u},~\vect{v}\right)$}

\hypersetup{breaklinks=true, colorlinks = true, linkcolor = OliveGreen, urlcolor = OliveGreen, citecolor = OliveGreen, pdfauthor={Didier BONNEL - https://www.maths-cours.fr} } % supprime les bordures autour des liens

\renewcommand{\arg}[0]{\text{arg}}

\everymath{\displaystyle}

%================================================================================================================================
%
% Macros - Commandes
%
%================================================================================================================================

\newcommand\meta[2]{    			% Utilisé pour créer le post HTML.
	\def\titre{titre}
	\def\url{url}
	\def\arg{#1}
	\ifx\titre\arg
		\newcommand\maintitle{#2}
		\fancyhead[L]{#2}
		{\Large\sffamily \MakeUppercase{#2}}
		\vspace{1mm}\textcolor{mcvert}{\hrule}
	\fi 
	\ifx\url\arg
		\fancyfoot[L]{\href{https://www.maths-cours.fr#2}{\black \footnotesize{https://www.maths-cours.fr#2}}}
	\fi 
}


\newcommand\TitreC[1]{    		% Titre centré
     \needspace{3\baselineskip}
     \begin{center}\textbf{#1}\end{center}
}

\newcommand\newpar{    		% paragraphe
     \par
}

\newcommand\nosp {    		% commande vide (pas d'espace)
}
\newcommand{\id}[1]{} %ignore

\newcommand\boite[2]{				% Boite simple sans titre
	\vspace{5mm}
	\setlength{\fboxrule}{0.2mm}
	\setlength{\fboxsep}{5mm}	
	\fcolorbox{#1}{#1!3}{\makebox[\linewidth-2\fboxrule-2\fboxsep]{
  		\begin{minipage}[t]{\linewidth-2\fboxrule-4\fboxsep}\setlength{\parskip}{3mm}
  			 #2
  		\end{minipage}
	}}
	\vspace{5mm}
}

\newcommand\CBox[4]{				% Boites
	\vspace{5mm}
	\setlength{\fboxrule}{0.2mm}
	\setlength{\fboxsep}{5mm}
	
	\fcolorbox{#1}{#1!3}{\makebox[\linewidth-2\fboxrule-2\fboxsep]{
		\begin{minipage}[t]{1cm}\setlength{\parskip}{3mm}
	  		\textcolor{#1}{\LARGE{#2}}    
 	 	\end{minipage}  
  		\begin{minipage}[t]{\linewidth-2\fboxrule-4\fboxsep}\setlength{\parskip}{3mm}
			\raisebox{1.2mm}{\normalsize\sffamily{\textcolor{#1}{#3}}}						
  			 #4
  		\end{minipage}
	}}
	\vspace{5mm}
}

\newcommand\cadre[3]{				% Boites convertible html
	\par
	\vspace{2mm}
	\setlength{\fboxrule}{0.1mm}
	\setlength{\fboxsep}{5mm}
	\fcolorbox{#1}{white}{\makebox[\linewidth-2\fboxrule-2\fboxsep]{
  		\begin{minipage}[t]{\linewidth-2\fboxrule-4\fboxsep}\setlength{\parskip}{3mm}
			\raisebox{-2.5mm}{\sffamily \small{\textcolor{#1}{\MakeUppercase{#2}}}}		
			\par		
  			 #3
 	 		\end{minipage}
	}}
		\vspace{2mm}
	\par
}

\newcommand\bloc[3]{				% Boites convertible html sans bordure
     \needspace{2\baselineskip}
     {\sffamily \small{\textcolor{#1}{\MakeUppercase{#2}}}}    
		\par		
  			 #3
		\par
}

\newcommand\CHelp[1]{
     \CBox{Plum}{\faInfoCircle}{À RETENIR}{#1}
}

\newcommand\CUp[1]{
     \CBox{NavyBlue}{\faThumbsOUp}{EN PRATIQUE}{#1}
}

\newcommand\CInfo[1]{
     \CBox{Sepia}{\faArrowCircleRight}{REMARQUE}{#1}
}

\newcommand\CRedac[1]{
     \CBox{PineGreen}{\faEdit}{BIEN R\'EDIGER}{#1}
}

\newcommand\CError[1]{
     \CBox{Red}{\faExclamationTriangle}{ATTENTION}{#1}
}

\newcommand\TitreExo[2]{
\needspace{4\baselineskip}
 {\sffamily\large EXERCICE #1\ (\emph{#2 points})}
\vspace{5mm}
}

\newcommand\img[2]{
          \includegraphics[width=#2\paperwidth]{\imgdir#1}
}

\newcommand\imgsvg[2]{
       \begin{center}   \includegraphics[width=#2\paperwidth]{\imgsvgdir#1} \end{center}
}


\newcommand\Lien[2]{
     \href{#1}{#2 \tiny \faExternalLink}
}
\newcommand\mcLien[2]{
     \href{https~://www.maths-cours.fr/#1}{#2 \tiny \faExternalLink}
}

\newcommand{\euro}{\eurologo{}}

%================================================================================================================================
%
% Macros - Environement
%
%================================================================================================================================

\newenvironment{tex}{ %
}
{%
}

\newenvironment{indente}{ %
	\setlength\parindent{10mm}
}

{
	\setlength\parindent{0mm}
}

\newenvironment{corrige}{%
     \needspace{3\baselineskip}
     \medskip
     \textbf{\textsc{Corrigé}}
     \medskip
}
{
}

\newenvironment{extern}{%
     \begin{center}
     }
     {
     \end{center}
}

\NewEnviron{code}{%
	\par
     \boite{gray}{\texttt{%
     \BODY
     }}
     \par
}

\newenvironment{vbloc}{% boite sans cadre empeche saut de page
     \begin{minipage}[t]{\linewidth}
     }
     {
     \end{minipage}
}
\NewEnviron{h2}{%
    \needspace{3\baselineskip}
    \vspace{0.6cm}
	\noindent \MakeUppercase{\sffamily \large \BODY}
	\vspace{1mm}\textcolor{mcgris}{\hrule}\vspace{0.4cm}
	\par
}{}

\NewEnviron{h3}{%
    \needspace{3\baselineskip}
	\vspace{5mm}
	\textsc{\BODY}
	\par
}

\NewEnviron{margeneg}{ %
\begin{addmargin}[-1cm]{0cm}
\BODY
\end{addmargin}
}

\NewEnviron{html}{%
}

\begin{document}
\meta{url}{/exercices/matrices-de-transition-bac-s-polynesie-2018-spe/}
\meta{pid}{9282}
\meta{titre}{Matrices de transition – Bac S Polynésie 2018 (spé)}
\meta{type}{exercices}
%
\begin{h2}Exercice 4 (5 points)\end{h2}
\textbf{Candidats ayant  choisi l'enseignement de spécialité \og mathématiques \fg{} }
\medbreak
Un atome d'hydrogène peut se trouver dans deux états différents, l'état stable et l'état excité. À chaque nanoseconde, l'atome peut changer d'état.
\bigbreak
\TitreC{Partie A - Étude d'un premier milieu}
\medbreak
Dans cette partie, on se place dans un premier milieu (milieu 1) où, à chaque nanoseconde, la probabilité qu'un atome passe de l'état stable à l'état excité est $0,005$, et la probabilité qu'il passe de l'état excité à l'état stable est $0,6$.
\par
On observe un atome d'hydrogène initialement à l'état stable.
\par
On note $a_n$ la probabilité que l'atome soit dans un état stable et $b_n$ la probabilité qu'il se trouve dans un état excité, $n$ nanosecondes après le début de l'observation.
\par
On a donc $a_0 = 1$ et $b_0 = 0$.
\par
On appelle $X_n$ la matrice ligne $X_n = \begin{pmatrix}a_n& b_n\end{pmatrix}$.
L'objectif est de savoir dans quel état se trouvera l'atome d'hydrogène à long terme.
\medbreak
\begin{enumerate}
     \item Calculer $a_1$ puis $b_1$ et montrer que $a_2 = 0,993025$ et $b_2 = 0,006975$.
     \item Déterminer la matrice $A$ telle que, pour tout entier naturel $n$,\: $X_{n+1} = X_n A$.
     \par
     $A$ est appelée matrice de transition dans le milieu 1.
     \par
     On admet alors que, pour tout entier naturel $n$,\: $X_n = X_0A^n$.
     \item On définit la matrice $P$ par $P = \begin{pmatrix}1&-1\\ 1&120\end{pmatrix}$.
     On admet que $P$ est inversible et que
     \[P^{-1} = \dfrac{1}{121}\begin{pmatrix}120&1\\- 1&1\end{pmatrix}.\]
     Déterminer la matrice $D$ définie par $D = P^{-1} AP$.
     \item Démontrer que, pour tout entier naturel $n$,\: $A^n = P D^n P^{-1}$.
     \item On admet par la suite que, pour tout entier naturel $n$,
     \par
     \[A^n = \dfrac{1}{121}\begin{pmatrix}120 + 0,395^n&1 - 0,395^n\\120\left(1 - 0,395^n\right)&1 + 120 \times 0,395^n\end{pmatrix}.\]
     En déduire une expression de $a_n$ en fonction de $n$.
     \item Déterminer la limite de la suite $\left(a_n\right)$. Conclure.
\end{enumerate}
\bigbreak
\TitreC{Partie B - Étude d'un second milieu}
\medbreak
Dans cette partie, on se place dans un second milieu (milieu 2), dans lequel on ne connaît pas la probabilité que l'atome passe de l'état excité à l'état stable. On note $a$ cette probabilité supposée constante. On sait, en revanche, qu'à chaque nanoseconde, la probabilité qu'un atome passe de l'état stable à l'état excité est $0,01$.
\medbreak
\begin{enumerate}
     \item Donner, en fonction de $a$, la matrice de transition $M$ dans le milieu 2.
     \item Après un temps très long, dans le milieu 2, la proportion d'atomes excités se stabilise autour de 2\,\%.
     \par
     On admet qu'il existe un unique vecteur $X$, appelé état stationnaire, tel que $XM = X$, et que $X = \begin{pmatrix}0,98& 0,02\end{pmatrix}$.
     Déterminer la valeur de $a$.
\end{enumerate}

\end{document}
µ
\documentclass[a4paper]{article}

%================================================================================================================================
%
% Packages
%
%================================================================================================================================

\usepackage[T1]{fontenc} 	% pour caractères accentués
\usepackage[utf8]{inputenc}  % encodage utf8
\usepackage[french]{babel}	% langue : français
\usepackage{fourier}			% caractères plus lisibles
\usepackage[dvipsnames]{xcolor} % couleurs
\usepackage{fancyhdr}		% réglage header footer
\usepackage{needspace}		% empêcher sauts de page mal placés
\usepackage{graphicx}		% pour inclure des graphiques
\usepackage{enumitem,cprotect}		% personnalise les listes d'items (nécessaire pour ol, al ...)
\usepackage{hyperref}		% Liens hypertexte
\usepackage{pstricks,pst-all,pst-node,pstricks-add,pst-math,pst-plot,pst-tree,pst-eucl} % pstricks
\usepackage[a4paper,includeheadfoot,top=2cm,left=3cm, bottom=2cm,right=3cm]{geometry} % marges etc.
\usepackage{comment}			% commentaires multilignes
\usepackage{amsmath,environ} % maths (matrices, etc.)
\usepackage{amssymb,makeidx}
\usepackage{bm}				% bold maths
\usepackage{tabularx}		% tableaux
\usepackage{colortbl}		% tableaux en couleur
\usepackage{fontawesome}		% Fontawesome
\usepackage{environ}			% environment with command
\usepackage{fp}				% calculs pour ps-tricks
\usepackage{multido}			% pour ps tricks
\usepackage[np]{numprint}	% formattage nombre
\usepackage{tikz,tkz-tab} 			% package principal TikZ
\usepackage{pgfplots}   % axes
\usepackage{mathrsfs}    % cursives
\usepackage{calc}			% calcul taille boites
\usepackage[scaled=0.875]{helvet} % font sans serif
\usepackage{svg} % svg
\usepackage{scrextend} % local margin
\usepackage{scratch} %scratch
\usepackage{multicol} % colonnes
%\usepackage{infix-RPN,pst-func} % formule en notation polanaise inversée
\usepackage{listings}

%================================================================================================================================
%
% Réglages de base
%
%================================================================================================================================

\lstset{
language=Python,   % R code
literate=
{á}{{\'a}}1
{à}{{\`a}}1
{ã}{{\~a}}1
{é}{{\'e}}1
{è}{{\`e}}1
{ê}{{\^e}}1
{í}{{\'i}}1
{ó}{{\'o}}1
{õ}{{\~o}}1
{ú}{{\'u}}1
{ü}{{\"u}}1
{ç}{{\c{c}}}1
{~}{{ }}1
}


\definecolor{codegreen}{rgb}{0,0.6,0}
\definecolor{codegray}{rgb}{0.5,0.5,0.5}
\definecolor{codepurple}{rgb}{0.58,0,0.82}
\definecolor{backcolour}{rgb}{0.95,0.95,0.92}

\lstdefinestyle{mystyle}{
    backgroundcolor=\color{backcolour},   
    commentstyle=\color{codegreen},
    keywordstyle=\color{magenta},
    numberstyle=\tiny\color{codegray},
    stringstyle=\color{codepurple},
    basicstyle=\ttfamily\footnotesize,
    breakatwhitespace=false,         
    breaklines=true,                 
    captionpos=b,                    
    keepspaces=true,                 
    numbers=left,                    
xleftmargin=2em,
framexleftmargin=2em,            
    showspaces=false,                
    showstringspaces=false,
    showtabs=false,                  
    tabsize=2,
    upquote=true
}

\lstset{style=mystyle}


\lstset{style=mystyle}
\newcommand{\imgdir}{C:/laragon/www/newmc/assets/imgsvg/}
\newcommand{\imgsvgdir}{C:/laragon/www/newmc/assets/imgsvg/}

\definecolor{mcgris}{RGB}{220, 220, 220}% ancien~; pour compatibilité
\definecolor{mcbleu}{RGB}{52, 152, 219}
\definecolor{mcvert}{RGB}{125, 194, 70}
\definecolor{mcmauve}{RGB}{154, 0, 215}
\definecolor{mcorange}{RGB}{255, 96, 0}
\definecolor{mcturquoise}{RGB}{0, 153, 153}
\definecolor{mcrouge}{RGB}{255, 0, 0}
\definecolor{mclightvert}{RGB}{205, 234, 190}

\definecolor{gris}{RGB}{220, 220, 220}
\definecolor{bleu}{RGB}{52, 152, 219}
\definecolor{vert}{RGB}{125, 194, 70}
\definecolor{mauve}{RGB}{154, 0, 215}
\definecolor{orange}{RGB}{255, 96, 0}
\definecolor{turquoise}{RGB}{0, 153, 153}
\definecolor{rouge}{RGB}{255, 0, 0}
\definecolor{lightvert}{RGB}{205, 234, 190}
\setitemize[0]{label=\color{lightvert}  $\bullet$}

\pagestyle{fancy}
\renewcommand{\headrulewidth}{0.2pt}
\fancyhead[L]{maths-cours.fr}
\fancyhead[R]{\thepage}
\renewcommand{\footrulewidth}{0.2pt}
\fancyfoot[C]{}

\newcolumntype{C}{>{\centering\arraybackslash}X}
\newcolumntype{s}{>{\hsize=.35\hsize\arraybackslash}X}

\setlength{\parindent}{0pt}		 
\setlength{\parskip}{3mm}
\setlength{\headheight}{1cm}

\def\ebook{ebook}
\def\book{book}
\def\web{web}
\def\type{web}

\newcommand{\vect}[1]{\overrightarrow{\,\mathstrut#1\,}}

\def\Oij{$\left(\text{O}~;~\vect{\imath},~\vect{\jmath}\right)$}
\def\Oijk{$\left(\text{O}~;~\vect{\imath},~\vect{\jmath},~\vect{k}\right)$}
\def\Ouv{$\left(\text{O}~;~\vect{u},~\vect{v}\right)$}

\hypersetup{breaklinks=true, colorlinks = true, linkcolor = OliveGreen, urlcolor = OliveGreen, citecolor = OliveGreen, pdfauthor={Didier BONNEL - https://www.maths-cours.fr} } % supprime les bordures autour des liens

\renewcommand{\arg}[0]{\text{arg}}

\everymath{\displaystyle}

%================================================================================================================================
%
% Macros - Commandes
%
%================================================================================================================================

\newcommand\meta[2]{    			% Utilisé pour créer le post HTML.
	\def\titre{titre}
	\def\url{url}
	\def\arg{#1}
	\ifx\titre\arg
		\newcommand\maintitle{#2}
		\fancyhead[L]{#2}
		{\Large\sffamily \MakeUppercase{#2}}
		\vspace{1mm}\textcolor{mcvert}{\hrule}
	\fi 
	\ifx\url\arg
		\fancyfoot[L]{\href{https://www.maths-cours.fr#2}{\black \footnotesize{https://www.maths-cours.fr#2}}}
	\fi 
}


\newcommand\TitreC[1]{    		% Titre centré
     \needspace{3\baselineskip}
     \begin{center}\textbf{#1}\end{center}
}

\newcommand\newpar{    		% paragraphe
     \par
}

\newcommand\nosp {    		% commande vide (pas d'espace)
}
\newcommand{\id}[1]{} %ignore

\newcommand\boite[2]{				% Boite simple sans titre
	\vspace{5mm}
	\setlength{\fboxrule}{0.2mm}
	\setlength{\fboxsep}{5mm}	
	\fcolorbox{#1}{#1!3}{\makebox[\linewidth-2\fboxrule-2\fboxsep]{
  		\begin{minipage}[t]{\linewidth-2\fboxrule-4\fboxsep}\setlength{\parskip}{3mm}
  			 #2
  		\end{minipage}
	}}
	\vspace{5mm}
}

\newcommand\CBox[4]{				% Boites
	\vspace{5mm}
	\setlength{\fboxrule}{0.2mm}
	\setlength{\fboxsep}{5mm}
	
	\fcolorbox{#1}{#1!3}{\makebox[\linewidth-2\fboxrule-2\fboxsep]{
		\begin{minipage}[t]{1cm}\setlength{\parskip}{3mm}
	  		\textcolor{#1}{\LARGE{#2}}    
 	 	\end{minipage}  
  		\begin{minipage}[t]{\linewidth-2\fboxrule-4\fboxsep}\setlength{\parskip}{3mm}
			\raisebox{1.2mm}{\normalsize\sffamily{\textcolor{#1}{#3}}}						
  			 #4
  		\end{minipage}
	}}
	\vspace{5mm}
}

\newcommand\cadre[3]{				% Boites convertible html
	\par
	\vspace{2mm}
	\setlength{\fboxrule}{0.1mm}
	\setlength{\fboxsep}{5mm}
	\fcolorbox{#1}{white}{\makebox[\linewidth-2\fboxrule-2\fboxsep]{
  		\begin{minipage}[t]{\linewidth-2\fboxrule-4\fboxsep}\setlength{\parskip}{3mm}
			\raisebox{-2.5mm}{\sffamily \small{\textcolor{#1}{\MakeUppercase{#2}}}}		
			\par		
  			 #3
 	 		\end{minipage}
	}}
		\vspace{2mm}
	\par
}

\newcommand\bloc[3]{				% Boites convertible html sans bordure
     \needspace{2\baselineskip}
     {\sffamily \small{\textcolor{#1}{\MakeUppercase{#2}}}}    
		\par		
  			 #3
		\par
}

\newcommand\CHelp[1]{
     \CBox{Plum}{\faInfoCircle}{À RETENIR}{#1}
}

\newcommand\CUp[1]{
     \CBox{NavyBlue}{\faThumbsOUp}{EN PRATIQUE}{#1}
}

\newcommand\CInfo[1]{
     \CBox{Sepia}{\faArrowCircleRight}{REMARQUE}{#1}
}

\newcommand\CRedac[1]{
     \CBox{PineGreen}{\faEdit}{BIEN R\'EDIGER}{#1}
}

\newcommand\CError[1]{
     \CBox{Red}{\faExclamationTriangle}{ATTENTION}{#1}
}

\newcommand\TitreExo[2]{
\needspace{4\baselineskip}
 {\sffamily\large EXERCICE #1\ (\emph{#2 points})}
\vspace{5mm}
}

\newcommand\img[2]{
          \includegraphics[width=#2\paperwidth]{\imgdir#1}
}

\newcommand\imgsvg[2]{
       \begin{center}   \includegraphics[width=#2\paperwidth]{\imgsvgdir#1} \end{center}
}


\newcommand\Lien[2]{
     \href{#1}{#2 \tiny \faExternalLink}
}
\newcommand\mcLien[2]{
     \href{https~://www.maths-cours.fr/#1}{#2 \tiny \faExternalLink}
}

\newcommand{\euro}{\eurologo{}}

%================================================================================================================================
%
% Macros - Environement
%
%================================================================================================================================

\newenvironment{tex}{ %
}
{%
}

\newenvironment{indente}{ %
	\setlength\parindent{10mm}
}

{
	\setlength\parindent{0mm}
}

\newenvironment{corrige}{%
     \needspace{3\baselineskip}
     \medskip
     \textbf{\textsc{Corrigé}}
     \medskip
}
{
}

\newenvironment{extern}{%
     \begin{center}
     }
     {
     \end{center}
}

\NewEnviron{code}{%
	\par
     \boite{gray}{\texttt{%
     \BODY
     }}
     \par
}

\newenvironment{vbloc}{% boite sans cadre empeche saut de page
     \begin{minipage}[t]{\linewidth}
     }
     {
     \end{minipage}
}
\NewEnviron{h2}{%
    \needspace{3\baselineskip}
    \vspace{0.6cm}
	\noindent \MakeUppercase{\sffamily \large \BODY}
	\vspace{1mm}\textcolor{mcgris}{\hrule}\vspace{0.4cm}
	\par
}{}

\NewEnviron{h3}{%
    \needspace{3\baselineskip}
	\vspace{5mm}
	\textsc{\BODY}
	\par
}

\NewEnviron{margeneg}{ %
\begin{addmargin}[-1cm]{0cm}
\BODY
\end{addmargin}
}

\NewEnviron{html}{%
}

\begin{document}
\meta{url}{/exercices/qcm-bac-es-l-polynesie-2018/}
\meta{pid}{9320}
\meta{titre}{QCM – Bac ES/L Polynésie 2018}
\meta{type}{exercices}
%
\begin{h2}Exercice 1 (5 points)\end{h2}
\textbf{Commun à  tous les candidats}
\medbreak
On considère la fonction $f$ définie sur l'intervalle ]0~;~3] par
\par
\[f(x) = x^2(1 - \ln x).\]
\par
On donne ci-dessous sa courbe représentative $\mathscr{C}$.
\begin{center}
     \begin{extern}%width="400" alt=""
          \psset{unit=3cm,comma=true}
          \begin{pspicture}(-1,-2)(4,2)
               \psgrid[Dx=1,Dy=1,subgriddiv=10,gridcolor=gray,gridlabels=0pt](-1,-2)(4,2)
               \psaxes[linewidth=1.2pt,Dx=0.5,Dy=0.5]{->}(0,0)(-1,-2)(4,2)
               \psaxes[linewidth=1.3pt]{->}(0,0)(1,1)
               \psplot[plotpoints=2000,linewidth=1.25pt,linecolor=blue]{0.01}{3.1}{1 x ln sub x dup mul mul}
               \uput[u](2.2,1.2){\blue $\mathcal{C}$}
          \end{pspicture}
     \end{extern}
\end{center}
\medbreak
On admet que $f$ est deux fois dérivable sur ]0~;~3], on note $f'$ sa fonction dérivée et on admet que sa dérivée seconde $f''$  est définie sur ]0~;~3] par~: $f''(x) = - 1- 2 \ln x$.
\medbreak
Cet exercice est un questionnaire à choix multiples. Pour chacune des questions posées, une seule réponse est exacte. Aucune justification n'est demandée.
\par
Une réponse exacte rapporte $1$ point, une réponse fausse ou l'absence de réponse ne rapporte ni n'enlève de point. Une réponse multiple ne rapporte aucun point.
\par
\medbreak
\begin{enumerate}
     \item Sur ]0~;~3], $\mathscr{C}$ coupe l'axe des abscisses au point d'abscisse~:
     \smallbreak
     \textbf{a.~~} e \\ \textbf{b.~~} 2,72 \\ \textbf{c.~~} $\dfrac{1}{2}\text{e} + 1$
     \medbreak
     \item  $\mathscr{C}$ admet un point d'inflexion d'abscisse~:
     \smallbreak
     \textbf{a.~~} e \\ \textbf{b.~~} $\dfrac{1}{\sqrt{\text{e}}}$\\ \textbf{c.~~} $\sqrt{\text{e}}$
     \medbreak
     \item  Pour tout nombre réel $x$ de l'intervalle ]0~;~3] on a~:
     \smallbreak
     \textbf{a.~~} $f'(x) = x(1 - 2\ln x)$\\ \textbf{b.~~} $f'(x)= - \dfrac{2}{x}$
     \\ \textbf{c.~~} $f'(x) = - 2$
     \medbreak
     \item  Sur l'intervalle [1~;~3]~:
     \smallbreak
     \textbf{a.~~}$f$ est convexe \\ \textbf{b.~~} $f$ est décroissante \\ \textbf{c.~~} $f'$ est décroissante
     \medbreak
     \item  Une équation de la tangente à $\mathscr{C}$  au point d'abscisse e s'écrit~:
     \smallbreak
     \textbf{a.~~} $y= -x+ \text{e}$\\ \textbf{b.~~} $y= - \text{e}x$ \\ \textbf{c.~~} $y = - \text{e}x + \text{e}^2$
     \medbreak
\end{enumerate}

\end{document}
µ
\documentclass[a4paper]{article}

%================================================================================================================================
%
% Packages
%
%================================================================================================================================

\usepackage[T1]{fontenc} 	% pour caractères accentués
\usepackage[utf8]{inputenc}  % encodage utf8
\usepackage[french]{babel}	% langue : français
\usepackage{fourier}			% caractères plus lisibles
\usepackage[dvipsnames]{xcolor} % couleurs
\usepackage{fancyhdr}		% réglage header footer
\usepackage{needspace}		% empêcher sauts de page mal placés
\usepackage{graphicx}		% pour inclure des graphiques
\usepackage{enumitem,cprotect}		% personnalise les listes d'items (nécessaire pour ol, al ...)
\usepackage{hyperref}		% Liens hypertexte
\usepackage{pstricks,pst-all,pst-node,pstricks-add,pst-math,pst-plot,pst-tree,pst-eucl} % pstricks
\usepackage[a4paper,includeheadfoot,top=2cm,left=3cm, bottom=2cm,right=3cm]{geometry} % marges etc.
\usepackage{comment}			% commentaires multilignes
\usepackage{amsmath,environ} % maths (matrices, etc.)
\usepackage{amssymb,makeidx}
\usepackage{bm}				% bold maths
\usepackage{tabularx}		% tableaux
\usepackage{colortbl}		% tableaux en couleur
\usepackage{fontawesome}		% Fontawesome
\usepackage{environ}			% environment with command
\usepackage{fp}				% calculs pour ps-tricks
\usepackage{multido}			% pour ps tricks
\usepackage[np]{numprint}	% formattage nombre
\usepackage{tikz,tkz-tab} 			% package principal TikZ
\usepackage{pgfplots}   % axes
\usepackage{mathrsfs}    % cursives
\usepackage{calc}			% calcul taille boites
\usepackage[scaled=0.875]{helvet} % font sans serif
\usepackage{svg} % svg
\usepackage{scrextend} % local margin
\usepackage{scratch} %scratch
\usepackage{multicol} % colonnes
%\usepackage{infix-RPN,pst-func} % formule en notation polanaise inversée
\usepackage{listings}

%================================================================================================================================
%
% Réglages de base
%
%================================================================================================================================

\lstset{
language=Python,   % R code
literate=
{á}{{\'a}}1
{à}{{\`a}}1
{ã}{{\~a}}1
{é}{{\'e}}1
{è}{{\`e}}1
{ê}{{\^e}}1
{í}{{\'i}}1
{ó}{{\'o}}1
{õ}{{\~o}}1
{ú}{{\'u}}1
{ü}{{\"u}}1
{ç}{{\c{c}}}1
{~}{{ }}1
}


\definecolor{codegreen}{rgb}{0,0.6,0}
\definecolor{codegray}{rgb}{0.5,0.5,0.5}
\definecolor{codepurple}{rgb}{0.58,0,0.82}
\definecolor{backcolour}{rgb}{0.95,0.95,0.92}

\lstdefinestyle{mystyle}{
    backgroundcolor=\color{backcolour},   
    commentstyle=\color{codegreen},
    keywordstyle=\color{magenta},
    numberstyle=\tiny\color{codegray},
    stringstyle=\color{codepurple},
    basicstyle=\ttfamily\footnotesize,
    breakatwhitespace=false,         
    breaklines=true,                 
    captionpos=b,                    
    keepspaces=true,                 
    numbers=left,                    
xleftmargin=2em,
framexleftmargin=2em,            
    showspaces=false,                
    showstringspaces=false,
    showtabs=false,                  
    tabsize=2,
    upquote=true
}

\lstset{style=mystyle}


\lstset{style=mystyle}
\newcommand{\imgdir}{C:/laragon/www/newmc/assets/imgsvg/}
\newcommand{\imgsvgdir}{C:/laragon/www/newmc/assets/imgsvg/}

\definecolor{mcgris}{RGB}{220, 220, 220}% ancien~; pour compatibilité
\definecolor{mcbleu}{RGB}{52, 152, 219}
\definecolor{mcvert}{RGB}{125, 194, 70}
\definecolor{mcmauve}{RGB}{154, 0, 215}
\definecolor{mcorange}{RGB}{255, 96, 0}
\definecolor{mcturquoise}{RGB}{0, 153, 153}
\definecolor{mcrouge}{RGB}{255, 0, 0}
\definecolor{mclightvert}{RGB}{205, 234, 190}

\definecolor{gris}{RGB}{220, 220, 220}
\definecolor{bleu}{RGB}{52, 152, 219}
\definecolor{vert}{RGB}{125, 194, 70}
\definecolor{mauve}{RGB}{154, 0, 215}
\definecolor{orange}{RGB}{255, 96, 0}
\definecolor{turquoise}{RGB}{0, 153, 153}
\definecolor{rouge}{RGB}{255, 0, 0}
\definecolor{lightvert}{RGB}{205, 234, 190}
\setitemize[0]{label=\color{lightvert}  $\bullet$}

\pagestyle{fancy}
\renewcommand{\headrulewidth}{0.2pt}
\fancyhead[L]{maths-cours.fr}
\fancyhead[R]{\thepage}
\renewcommand{\footrulewidth}{0.2pt}
\fancyfoot[C]{}

\newcolumntype{C}{>{\centering\arraybackslash}X}
\newcolumntype{s}{>{\hsize=.35\hsize\arraybackslash}X}

\setlength{\parindent}{0pt}		 
\setlength{\parskip}{3mm}
\setlength{\headheight}{1cm}

\def\ebook{ebook}
\def\book{book}
\def\web{web}
\def\type{web}

\newcommand{\vect}[1]{\overrightarrow{\,\mathstrut#1\,}}

\def\Oij{$\left(\text{O}~;~\vect{\imath},~\vect{\jmath}\right)$}
\def\Oijk{$\left(\text{O}~;~\vect{\imath},~\vect{\jmath},~\vect{k}\right)$}
\def\Ouv{$\left(\text{O}~;~\vect{u},~\vect{v}\right)$}

\hypersetup{breaklinks=true, colorlinks = true, linkcolor = OliveGreen, urlcolor = OliveGreen, citecolor = OliveGreen, pdfauthor={Didier BONNEL - https://www.maths-cours.fr} } % supprime les bordures autour des liens

\renewcommand{\arg}[0]{\text{arg}}

\everymath{\displaystyle}

%================================================================================================================================
%
% Macros - Commandes
%
%================================================================================================================================

\newcommand\meta[2]{    			% Utilisé pour créer le post HTML.
	\def\titre{titre}
	\def\url{url}
	\def\arg{#1}
	\ifx\titre\arg
		\newcommand\maintitle{#2}
		\fancyhead[L]{#2}
		{\Large\sffamily \MakeUppercase{#2}}
		\vspace{1mm}\textcolor{mcvert}{\hrule}
	\fi 
	\ifx\url\arg
		\fancyfoot[L]{\href{https://www.maths-cours.fr#2}{\black \footnotesize{https://www.maths-cours.fr#2}}}
	\fi 
}


\newcommand\TitreC[1]{    		% Titre centré
     \needspace{3\baselineskip}
     \begin{center}\textbf{#1}\end{center}
}

\newcommand\newpar{    		% paragraphe
     \par
}

\newcommand\nosp {    		% commande vide (pas d'espace)
}
\newcommand{\id}[1]{} %ignore

\newcommand\boite[2]{				% Boite simple sans titre
	\vspace{5mm}
	\setlength{\fboxrule}{0.2mm}
	\setlength{\fboxsep}{5mm}	
	\fcolorbox{#1}{#1!3}{\makebox[\linewidth-2\fboxrule-2\fboxsep]{
  		\begin{minipage}[t]{\linewidth-2\fboxrule-4\fboxsep}\setlength{\parskip}{3mm}
  			 #2
  		\end{minipage}
	}}
	\vspace{5mm}
}

\newcommand\CBox[4]{				% Boites
	\vspace{5mm}
	\setlength{\fboxrule}{0.2mm}
	\setlength{\fboxsep}{5mm}
	
	\fcolorbox{#1}{#1!3}{\makebox[\linewidth-2\fboxrule-2\fboxsep]{
		\begin{minipage}[t]{1cm}\setlength{\parskip}{3mm}
	  		\textcolor{#1}{\LARGE{#2}}    
 	 	\end{minipage}  
  		\begin{minipage}[t]{\linewidth-2\fboxrule-4\fboxsep}\setlength{\parskip}{3mm}
			\raisebox{1.2mm}{\normalsize\sffamily{\textcolor{#1}{#3}}}						
  			 #4
  		\end{minipage}
	}}
	\vspace{5mm}
}

\newcommand\cadre[3]{				% Boites convertible html
	\par
	\vspace{2mm}
	\setlength{\fboxrule}{0.1mm}
	\setlength{\fboxsep}{5mm}
	\fcolorbox{#1}{white}{\makebox[\linewidth-2\fboxrule-2\fboxsep]{
  		\begin{minipage}[t]{\linewidth-2\fboxrule-4\fboxsep}\setlength{\parskip}{3mm}
			\raisebox{-2.5mm}{\sffamily \small{\textcolor{#1}{\MakeUppercase{#2}}}}		
			\par		
  			 #3
 	 		\end{minipage}
	}}
		\vspace{2mm}
	\par
}

\newcommand\bloc[3]{				% Boites convertible html sans bordure
     \needspace{2\baselineskip}
     {\sffamily \small{\textcolor{#1}{\MakeUppercase{#2}}}}    
		\par		
  			 #3
		\par
}

\newcommand\CHelp[1]{
     \CBox{Plum}{\faInfoCircle}{À RETENIR}{#1}
}

\newcommand\CUp[1]{
     \CBox{NavyBlue}{\faThumbsOUp}{EN PRATIQUE}{#1}
}

\newcommand\CInfo[1]{
     \CBox{Sepia}{\faArrowCircleRight}{REMARQUE}{#1}
}

\newcommand\CRedac[1]{
     \CBox{PineGreen}{\faEdit}{BIEN R\'EDIGER}{#1}
}

\newcommand\CError[1]{
     \CBox{Red}{\faExclamationTriangle}{ATTENTION}{#1}
}

\newcommand\TitreExo[2]{
\needspace{4\baselineskip}
 {\sffamily\large EXERCICE #1\ (\emph{#2 points})}
\vspace{5mm}
}

\newcommand\img[2]{
          \includegraphics[width=#2\paperwidth]{\imgdir#1}
}

\newcommand\imgsvg[2]{
       \begin{center}   \includegraphics[width=#2\paperwidth]{\imgsvgdir#1} \end{center}
}


\newcommand\Lien[2]{
     \href{#1}{#2 \tiny \faExternalLink}
}
\newcommand\mcLien[2]{
     \href{https~://www.maths-cours.fr/#1}{#2 \tiny \faExternalLink}
}

\newcommand{\euro}{\eurologo{}}

%================================================================================================================================
%
% Macros - Environement
%
%================================================================================================================================

\newenvironment{tex}{ %
}
{%
}

\newenvironment{indente}{ %
	\setlength\parindent{10mm}
}

{
	\setlength\parindent{0mm}
}

\newenvironment{corrige}{%
     \needspace{3\baselineskip}
     \medskip
     \textbf{\textsc{Corrigé}}
     \medskip
}
{
}

\newenvironment{extern}{%
     \begin{center}
     }
     {
     \end{center}
}

\NewEnviron{code}{%
	\par
     \boite{gray}{\texttt{%
     \BODY
     }}
     \par
}

\newenvironment{vbloc}{% boite sans cadre empeche saut de page
     \begin{minipage}[t]{\linewidth}
     }
     {
     \end{minipage}
}
\NewEnviron{h2}{%
    \needspace{3\baselineskip}
    \vspace{0.6cm}
	\noindent \MakeUppercase{\sffamily \large \BODY}
	\vspace{1mm}\textcolor{mcgris}{\hrule}\vspace{0.4cm}
	\par
}{}

\NewEnviron{h3}{%
    \needspace{3\baselineskip}
	\vspace{5mm}
	\textsc{\BODY}
	\par
}

\NewEnviron{margeneg}{ %
\begin{addmargin}[-1cm]{0cm}
\BODY
\end{addmargin}
}

\NewEnviron{html}{%
}

\begin{document}
\meta{url}{/exercices/probabilites-bac-es-l-polynesie-2018/}
\meta{pid}{9322}
\meta{titre}{Probabilités – Bac ES/L Polynésie 2018}
\meta{type}{exercices}
%
\begin{h2}Exercice 2 (5 points)\end{h2}
\textbf{Commun à  tous les candidats}
\medbreak
Les parties de cet exercice peuvent être traitées de façon indépendante.
\par
Les résultats numériques seront donnés, si nécessaire, sous forme approchée à 0,001 près.
\TitreC{Partie A}
Une entreprise est composée de 3 services A, B et C d'effectifs respectifs $450$, $230$ et $320$ employés.
\par
Une enquête effectuée sur le temps de parcours quotidien entre le domicile des employés et
l'entreprise a montré que~:
\begin{itemize}
     \item 40\,\% des employés du service A résident à moins de 30 minutes de l'entreprise~;
     \item 20\,\% des employés du service B résident à moins de 30 minutes de l'entreprise~;
     \item 80\,\% des employés du service C résident à moins de 30 minutes de l'entreprise.
\end{itemize}
On choisit au hasard un employé de cette entreprise et on considère les événements suivants~:
\begin{itemize}
     \item $A$~: \og l'employé fait partie du service A \fg{}~;
     \item $B$~: \og l'employé fait partie du service B \fg{}~;
     \item $C$~: \og l'employé fait partie du service C \fg{}~;
     \item $T$~: \og l'employé réside à moins de 30 minutes de l'entreprise\fg.
\end{itemize}
On rappelle que si $E$ et $F$ sont deux événements, la probabilité d'un événement $E$ est notée $P(E)$ et celle de $E$ sachant $F$ est notée $P_F(E)$.
\medbreak
\begin{enumerate}
     \item
     \begin{enumerate}[label=\alph*.]
          \item Justifier que $P(A) = 0,45$.
          \item Donner $P_A(T)$.
          \item Représenter la situation à l'aide d'un arbre pondéré en indiquant les probabilités associées à chaque branche.
     \end{enumerate}
     \item Déterminer la probabilité que l'employé choisi soit du service A et qu'il réside à moins de $30$ minutes de son lieu de travail.
     \item Montrer que $P(T) = 0,482$.
     \item Sachant qu'un employé de l'entreprise réside à plus de $30$ minutes de son lieu de travail, déterminer la probabilité qu'il fasse partie du service C.
     \item On choisit successivement de manière indépendante $5$ employés de l'entreprise. On considère que le nombre d'employés est suffisamment grand pour que ce tirage soit assimilé à un tirage avec remise.
     \par
     Déterminer la probabilité qu'exactement $2$ d'entre eux résident à moins de 30 minutes de leur lieu de travail.
\end{enumerate}
\TitreC{Partie B}
Soit $X$ la variable aléatoire qui, à chaque employé en France, associe son temps de trajet quotidien, en minutes, entre son domicile et l'entreprise. Une enquête montre que $X$ suit une loi normale d'espérance $40$ et d'écart type $10$.
\medbreak
\begin{enumerate}
     \item Calculer la probabilité que le trajet dure entre 20 minutes et 40 minutes.
     \item  Déterminer$P(X > 50)$.
     \item  À l'aide de la méthode de votre choix, déterminer une valeur approchée du nombre $a$ à l'unité près, tel que $P(X > a) = 0,2$. Interpréter ce résultat dans le contexte de l'exercice.
\end{enumerate}
\TitreC{Partie C}
Cette entreprise souhaite faire une offre de transport auprès de ses employés. Un sondage auprès de quelques employés est effectué afin d'estimer la proportion d'employés dans l'entreprise intéressés par cette offre de transport. \\On souhaite ainsi obtenir un intervalle de confiance d'amplitude strictement inférieure à $0,15$ avec un niveau de confiance de $0,95$.
\par
Quel est le nombre minimal d'employés à consulter~?

\end{document}
µ
\documentclass[a4paper]{article}

%================================================================================================================================
%
% Packages
%
%================================================================================================================================

\usepackage[T1]{fontenc} 	% pour caractères accentués
\usepackage[utf8]{inputenc}  % encodage utf8
\usepackage[french]{babel}	% langue : français
\usepackage{fourier}			% caractères plus lisibles
\usepackage[dvipsnames]{xcolor} % couleurs
\usepackage{fancyhdr}		% réglage header footer
\usepackage{needspace}		% empêcher sauts de page mal placés
\usepackage{graphicx}		% pour inclure des graphiques
\usepackage{enumitem,cprotect}		% personnalise les listes d'items (nécessaire pour ol, al ...)
\usepackage{hyperref}		% Liens hypertexte
\usepackage{pstricks,pst-all,pst-node,pstricks-add,pst-math,pst-plot,pst-tree,pst-eucl} % pstricks
\usepackage[a4paper,includeheadfoot,top=2cm,left=3cm, bottom=2cm,right=3cm]{geometry} % marges etc.
\usepackage{comment}			% commentaires multilignes
\usepackage{amsmath,environ} % maths (matrices, etc.)
\usepackage{amssymb,makeidx}
\usepackage{bm}				% bold maths
\usepackage{tabularx}		% tableaux
\usepackage{colortbl}		% tableaux en couleur
\usepackage{fontawesome}		% Fontawesome
\usepackage{environ}			% environment with command
\usepackage{fp}				% calculs pour ps-tricks
\usepackage{multido}			% pour ps tricks
\usepackage[np]{numprint}	% formattage nombre
\usepackage{tikz,tkz-tab} 			% package principal TikZ
\usepackage{pgfplots}   % axes
\usepackage{mathrsfs}    % cursives
\usepackage{calc}			% calcul taille boites
\usepackage[scaled=0.875]{helvet} % font sans serif
\usepackage{svg} % svg
\usepackage{scrextend} % local margin
\usepackage{scratch} %scratch
\usepackage{multicol} % colonnes
%\usepackage{infix-RPN,pst-func} % formule en notation polanaise inversée
\usepackage{listings}

%================================================================================================================================
%
% Réglages de base
%
%================================================================================================================================

\lstset{
language=Python,   % R code
literate=
{á}{{\'a}}1
{à}{{\`a}}1
{ã}{{\~a}}1
{é}{{\'e}}1
{è}{{\`e}}1
{ê}{{\^e}}1
{í}{{\'i}}1
{ó}{{\'o}}1
{õ}{{\~o}}1
{ú}{{\'u}}1
{ü}{{\"u}}1
{ç}{{\c{c}}}1
{~}{{ }}1
}


\definecolor{codegreen}{rgb}{0,0.6,0}
\definecolor{codegray}{rgb}{0.5,0.5,0.5}
\definecolor{codepurple}{rgb}{0.58,0,0.82}
\definecolor{backcolour}{rgb}{0.95,0.95,0.92}

\lstdefinestyle{mystyle}{
    backgroundcolor=\color{backcolour},   
    commentstyle=\color{codegreen},
    keywordstyle=\color{magenta},
    numberstyle=\tiny\color{codegray},
    stringstyle=\color{codepurple},
    basicstyle=\ttfamily\footnotesize,
    breakatwhitespace=false,         
    breaklines=true,                 
    captionpos=b,                    
    keepspaces=true,                 
    numbers=left,                    
xleftmargin=2em,
framexleftmargin=2em,            
    showspaces=false,                
    showstringspaces=false,
    showtabs=false,                  
    tabsize=2,
    upquote=true
}

\lstset{style=mystyle}


\lstset{style=mystyle}
\newcommand{\imgdir}{C:/laragon/www/newmc/assets/imgsvg/}
\newcommand{\imgsvgdir}{C:/laragon/www/newmc/assets/imgsvg/}

\definecolor{mcgris}{RGB}{220, 220, 220}% ancien~; pour compatibilité
\definecolor{mcbleu}{RGB}{52, 152, 219}
\definecolor{mcvert}{RGB}{125, 194, 70}
\definecolor{mcmauve}{RGB}{154, 0, 215}
\definecolor{mcorange}{RGB}{255, 96, 0}
\definecolor{mcturquoise}{RGB}{0, 153, 153}
\definecolor{mcrouge}{RGB}{255, 0, 0}
\definecolor{mclightvert}{RGB}{205, 234, 190}

\definecolor{gris}{RGB}{220, 220, 220}
\definecolor{bleu}{RGB}{52, 152, 219}
\definecolor{vert}{RGB}{125, 194, 70}
\definecolor{mauve}{RGB}{154, 0, 215}
\definecolor{orange}{RGB}{255, 96, 0}
\definecolor{turquoise}{RGB}{0, 153, 153}
\definecolor{rouge}{RGB}{255, 0, 0}
\definecolor{lightvert}{RGB}{205, 234, 190}
\setitemize[0]{label=\color{lightvert}  $\bullet$}

\pagestyle{fancy}
\renewcommand{\headrulewidth}{0.2pt}
\fancyhead[L]{maths-cours.fr}
\fancyhead[R]{\thepage}
\renewcommand{\footrulewidth}{0.2pt}
\fancyfoot[C]{}

\newcolumntype{C}{>{\centering\arraybackslash}X}
\newcolumntype{s}{>{\hsize=.35\hsize\arraybackslash}X}

\setlength{\parindent}{0pt}		 
\setlength{\parskip}{3mm}
\setlength{\headheight}{1cm}

\def\ebook{ebook}
\def\book{book}
\def\web{web}
\def\type{web}

\newcommand{\vect}[1]{\overrightarrow{\,\mathstrut#1\,}}

\def\Oij{$\left(\text{O}~;~\vect{\imath},~\vect{\jmath}\right)$}
\def\Oijk{$\left(\text{O}~;~\vect{\imath},~\vect{\jmath},~\vect{k}\right)$}
\def\Ouv{$\left(\text{O}~;~\vect{u},~\vect{v}\right)$}

\hypersetup{breaklinks=true, colorlinks = true, linkcolor = OliveGreen, urlcolor = OliveGreen, citecolor = OliveGreen, pdfauthor={Didier BONNEL - https://www.maths-cours.fr} } % supprime les bordures autour des liens

\renewcommand{\arg}[0]{\text{arg}}

\everymath{\displaystyle}

%================================================================================================================================
%
% Macros - Commandes
%
%================================================================================================================================

\newcommand\meta[2]{    			% Utilisé pour créer le post HTML.
	\def\titre{titre}
	\def\url{url}
	\def\arg{#1}
	\ifx\titre\arg
		\newcommand\maintitle{#2}
		\fancyhead[L]{#2}
		{\Large\sffamily \MakeUppercase{#2}}
		\vspace{1mm}\textcolor{mcvert}{\hrule}
	\fi 
	\ifx\url\arg
		\fancyfoot[L]{\href{https://www.maths-cours.fr#2}{\black \footnotesize{https://www.maths-cours.fr#2}}}
	\fi 
}


\newcommand\TitreC[1]{    		% Titre centré
     \needspace{3\baselineskip}
     \begin{center}\textbf{#1}\end{center}
}

\newcommand\newpar{    		% paragraphe
     \par
}

\newcommand\nosp {    		% commande vide (pas d'espace)
}
\newcommand{\id}[1]{} %ignore

\newcommand\boite[2]{				% Boite simple sans titre
	\vspace{5mm}
	\setlength{\fboxrule}{0.2mm}
	\setlength{\fboxsep}{5mm}	
	\fcolorbox{#1}{#1!3}{\makebox[\linewidth-2\fboxrule-2\fboxsep]{
  		\begin{minipage}[t]{\linewidth-2\fboxrule-4\fboxsep}\setlength{\parskip}{3mm}
  			 #2
  		\end{minipage}
	}}
	\vspace{5mm}
}

\newcommand\CBox[4]{				% Boites
	\vspace{5mm}
	\setlength{\fboxrule}{0.2mm}
	\setlength{\fboxsep}{5mm}
	
	\fcolorbox{#1}{#1!3}{\makebox[\linewidth-2\fboxrule-2\fboxsep]{
		\begin{minipage}[t]{1cm}\setlength{\parskip}{3mm}
	  		\textcolor{#1}{\LARGE{#2}}    
 	 	\end{minipage}  
  		\begin{minipage}[t]{\linewidth-2\fboxrule-4\fboxsep}\setlength{\parskip}{3mm}
			\raisebox{1.2mm}{\normalsize\sffamily{\textcolor{#1}{#3}}}						
  			 #4
  		\end{minipage}
	}}
	\vspace{5mm}
}

\newcommand\cadre[3]{				% Boites convertible html
	\par
	\vspace{2mm}
	\setlength{\fboxrule}{0.1mm}
	\setlength{\fboxsep}{5mm}
	\fcolorbox{#1}{white}{\makebox[\linewidth-2\fboxrule-2\fboxsep]{
  		\begin{minipage}[t]{\linewidth-2\fboxrule-4\fboxsep}\setlength{\parskip}{3mm}
			\raisebox{-2.5mm}{\sffamily \small{\textcolor{#1}{\MakeUppercase{#2}}}}		
			\par		
  			 #3
 	 		\end{minipage}
	}}
		\vspace{2mm}
	\par
}

\newcommand\bloc[3]{				% Boites convertible html sans bordure
     \needspace{2\baselineskip}
     {\sffamily \small{\textcolor{#1}{\MakeUppercase{#2}}}}    
		\par		
  			 #3
		\par
}

\newcommand\CHelp[1]{
     \CBox{Plum}{\faInfoCircle}{À RETENIR}{#1}
}

\newcommand\CUp[1]{
     \CBox{NavyBlue}{\faThumbsOUp}{EN PRATIQUE}{#1}
}

\newcommand\CInfo[1]{
     \CBox{Sepia}{\faArrowCircleRight}{REMARQUE}{#1}
}

\newcommand\CRedac[1]{
     \CBox{PineGreen}{\faEdit}{BIEN R\'EDIGER}{#1}
}

\newcommand\CError[1]{
     \CBox{Red}{\faExclamationTriangle}{ATTENTION}{#1}
}

\newcommand\TitreExo[2]{
\needspace{4\baselineskip}
 {\sffamily\large EXERCICE #1\ (\emph{#2 points})}
\vspace{5mm}
}

\newcommand\img[2]{
          \includegraphics[width=#2\paperwidth]{\imgdir#1}
}

\newcommand\imgsvg[2]{
       \begin{center}   \includegraphics[width=#2\paperwidth]{\imgsvgdir#1} \end{center}
}


\newcommand\Lien[2]{
     \href{#1}{#2 \tiny \faExternalLink}
}
\newcommand\mcLien[2]{
     \href{https~://www.maths-cours.fr/#1}{#2 \tiny \faExternalLink}
}

\newcommand{\euro}{\eurologo{}}

%================================================================================================================================
%
% Macros - Environement
%
%================================================================================================================================

\newenvironment{tex}{ %
}
{%
}

\newenvironment{indente}{ %
	\setlength\parindent{10mm}
}

{
	\setlength\parindent{0mm}
}

\newenvironment{corrige}{%
     \needspace{3\baselineskip}
     \medskip
     \textbf{\textsc{Corrigé}}
     \medskip
}
{
}

\newenvironment{extern}{%
     \begin{center}
     }
     {
     \end{center}
}

\NewEnviron{code}{%
	\par
     \boite{gray}{\texttt{%
     \BODY
     }}
     \par
}

\newenvironment{vbloc}{% boite sans cadre empeche saut de page
     \begin{minipage}[t]{\linewidth}
     }
     {
     \end{minipage}
}
\NewEnviron{h2}{%
    \needspace{3\baselineskip}
    \vspace{0.6cm}
	\noindent \MakeUppercase{\sffamily \large \BODY}
	\vspace{1mm}\textcolor{mcgris}{\hrule}\vspace{0.4cm}
	\par
}{}

\NewEnviron{h3}{%
    \needspace{3\baselineskip}
	\vspace{5mm}
	\textsc{\BODY}
	\par
}

\NewEnviron{margeneg}{ %
\begin{addmargin}[-1cm]{0cm}
\BODY
\end{addmargin}
}

\NewEnviron{html}{%
}

\begin{document}
\meta{url}{/exercices/suites-bac-es-l-polynesie-2018/}
\meta{pid}{9324}
\meta{titre}{Suites – Bac ES/L Polynésie 2018}
\meta{type}{exercices}
%
\begin{h2}Exercice 3 (5 points)\end{h2}
\textbf{Candidats n'ayant pas choisi la spécialité \og mathématiques \fg{}}
\medbreak
En économie le résultat net désigne la différence entre la recette et les charges d'une entreprise sur une période donnée. Lorsqu'il est strictement positif, c'est un bénéfice.
\par
Propriétaire d'une société, Pierre veut estimer son résultat net à la fin de chaque mois.
\par
À la fin du mois de janvier 2018, celui-ci était de 10~000 euros.
\par
Pierre modélise ce résultat net par une suite $\left(u_n\right)$ de premier terme $u_0 = 10~000$ et de terme général $u_n$ tel que :
\[u_{n+1} = 1,02u_n - 500\]
où $n$ désigne le nombre de mois écoulés depuis janvier 2018.
\medbreak
\begin{enumerate}
     \item Quel est le montant du résultat net réalisé à la fin du mois de mars 2018~?
     \item Pour tout entier naturel $n$, on pose $a_n = u_n  - 25~000$.
     \begin{enumerate}[label=\alph*.]
          \item Montrer que la suite $\left(u_n\right)$ est une suite géométrique dont on précisera le premier terme $a_0$ et la raison.
          \item  Exprimer $a_n$ en fonction de $n$ et montrer que, pour tout entier naturel $n$,\:
          \par
          $u_n = 25~000 - 15~000 \times 1,02^n$.
          \item  Résoudre l'inéquation $25~000 - 15~000 \times 1,02^n > 0$ où $n$ désigne un entier naturel.
          \par
          Interpréter le résultat obtenu dans le contexte de l'exercice.
     \end{enumerate}
     \item  À l'aide d'un algorithme, Pierre souhaite déterminer le cumul total des résultats nets mensuels de la société jusqu'au dernier mois où l'entreprise est bénéficiaire.
     \par
     Recopier et compléter l'algorithme pour qu'à la fin de son exécution, la variable $N$ contienne le nombre de mois pendant lesquels l'entreprise est bénéficiaire et la variable $S$ le cumul total des résultats nets mensuels sur cette période.
     \begin{center}
          \begin{extern}%width="220" alt="Algorithme Bac ES/L Polynésie 2018"
               \begin{tabularx}{0.3\linewidth}{|X|}\hline
                    $U \gets 10~000$\\
                    $S \gets 0$\\
                    $N \gets 0$\\
                    Tant que \ldots\ldots\\
                    \hspace{1cm}$S$  \ldots\ldots\\
                    \hspace{1cm}$U$  \ldots\ldots\\
                    \hspace{1cm}$N$  \ldots\ldots\\
                    Fin Tant que\\ \hline
               \end{tabularx}
          \end{extern}
     \end{center}
\end{enumerate}

\end{document}
µ
\documentclass[a4paper]{article}

%================================================================================================================================
%
% Packages
%
%================================================================================================================================

\usepackage[T1]{fontenc} 	% pour caractères accentués
\usepackage[utf8]{inputenc}  % encodage utf8
\usepackage[french]{babel}	% langue : français
\usepackage{fourier}			% caractères plus lisibles
\usepackage[dvipsnames]{xcolor} % couleurs
\usepackage{fancyhdr}		% réglage header footer
\usepackage{needspace}		% empêcher sauts de page mal placés
\usepackage{graphicx}		% pour inclure des graphiques
\usepackage{enumitem,cprotect}		% personnalise les listes d'items (nécessaire pour ol, al ...)
\usepackage{hyperref}		% Liens hypertexte
\usepackage{pstricks,pst-all,pst-node,pstricks-add,pst-math,pst-plot,pst-tree,pst-eucl} % pstricks
\usepackage[a4paper,includeheadfoot,top=2cm,left=3cm, bottom=2cm,right=3cm]{geometry} % marges etc.
\usepackage{comment}			% commentaires multilignes
\usepackage{amsmath,environ} % maths (matrices, etc.)
\usepackage{amssymb,makeidx}
\usepackage{bm}				% bold maths
\usepackage{tabularx}		% tableaux
\usepackage{colortbl}		% tableaux en couleur
\usepackage{fontawesome}		% Fontawesome
\usepackage{environ}			% environment with command
\usepackage{fp}				% calculs pour ps-tricks
\usepackage{multido}			% pour ps tricks
\usepackage[np]{numprint}	% formattage nombre
\usepackage{tikz,tkz-tab} 			% package principal TikZ
\usepackage{pgfplots}   % axes
\usepackage{mathrsfs}    % cursives
\usepackage{calc}			% calcul taille boites
\usepackage[scaled=0.875]{helvet} % font sans serif
\usepackage{svg} % svg
\usepackage{scrextend} % local margin
\usepackage{scratch} %scratch
\usepackage{multicol} % colonnes
%\usepackage{infix-RPN,pst-func} % formule en notation polanaise inversée
\usepackage{listings}

%================================================================================================================================
%
% Réglages de base
%
%================================================================================================================================

\lstset{
language=Python,   % R code
literate=
{á}{{\'a}}1
{à}{{\`a}}1
{ã}{{\~a}}1
{é}{{\'e}}1
{è}{{\`e}}1
{ê}{{\^e}}1
{í}{{\'i}}1
{ó}{{\'o}}1
{õ}{{\~o}}1
{ú}{{\'u}}1
{ü}{{\"u}}1
{ç}{{\c{c}}}1
{~}{{ }}1
}


\definecolor{codegreen}{rgb}{0,0.6,0}
\definecolor{codegray}{rgb}{0.5,0.5,0.5}
\definecolor{codepurple}{rgb}{0.58,0,0.82}
\definecolor{backcolour}{rgb}{0.95,0.95,0.92}

\lstdefinestyle{mystyle}{
    backgroundcolor=\color{backcolour},   
    commentstyle=\color{codegreen},
    keywordstyle=\color{magenta},
    numberstyle=\tiny\color{codegray},
    stringstyle=\color{codepurple},
    basicstyle=\ttfamily\footnotesize,
    breakatwhitespace=false,         
    breaklines=true,                 
    captionpos=b,                    
    keepspaces=true,                 
    numbers=left,                    
xleftmargin=2em,
framexleftmargin=2em,            
    showspaces=false,                
    showstringspaces=false,
    showtabs=false,                  
    tabsize=2,
    upquote=true
}

\lstset{style=mystyle}


\lstset{style=mystyle}
\newcommand{\imgdir}{C:/laragon/www/newmc/assets/imgsvg/}
\newcommand{\imgsvgdir}{C:/laragon/www/newmc/assets/imgsvg/}

\definecolor{mcgris}{RGB}{220, 220, 220}% ancien~; pour compatibilité
\definecolor{mcbleu}{RGB}{52, 152, 219}
\definecolor{mcvert}{RGB}{125, 194, 70}
\definecolor{mcmauve}{RGB}{154, 0, 215}
\definecolor{mcorange}{RGB}{255, 96, 0}
\definecolor{mcturquoise}{RGB}{0, 153, 153}
\definecolor{mcrouge}{RGB}{255, 0, 0}
\definecolor{mclightvert}{RGB}{205, 234, 190}

\definecolor{gris}{RGB}{220, 220, 220}
\definecolor{bleu}{RGB}{52, 152, 219}
\definecolor{vert}{RGB}{125, 194, 70}
\definecolor{mauve}{RGB}{154, 0, 215}
\definecolor{orange}{RGB}{255, 96, 0}
\definecolor{turquoise}{RGB}{0, 153, 153}
\definecolor{rouge}{RGB}{255, 0, 0}
\definecolor{lightvert}{RGB}{205, 234, 190}
\setitemize[0]{label=\color{lightvert}  $\bullet$}

\pagestyle{fancy}
\renewcommand{\headrulewidth}{0.2pt}
\fancyhead[L]{maths-cours.fr}
\fancyhead[R]{\thepage}
\renewcommand{\footrulewidth}{0.2pt}
\fancyfoot[C]{}

\newcolumntype{C}{>{\centering\arraybackslash}X}
\newcolumntype{s}{>{\hsize=.35\hsize\arraybackslash}X}

\setlength{\parindent}{0pt}		 
\setlength{\parskip}{3mm}
\setlength{\headheight}{1cm}

\def\ebook{ebook}
\def\book{book}
\def\web{web}
\def\type{web}

\newcommand{\vect}[1]{\overrightarrow{\,\mathstrut#1\,}}

\def\Oij{$\left(\text{O}~;~\vect{\imath},~\vect{\jmath}\right)$}
\def\Oijk{$\left(\text{O}~;~\vect{\imath},~\vect{\jmath},~\vect{k}\right)$}
\def\Ouv{$\left(\text{O}~;~\vect{u},~\vect{v}\right)$}

\hypersetup{breaklinks=true, colorlinks = true, linkcolor = OliveGreen, urlcolor = OliveGreen, citecolor = OliveGreen, pdfauthor={Didier BONNEL - https://www.maths-cours.fr} } % supprime les bordures autour des liens

\renewcommand{\arg}[0]{\text{arg}}

\everymath{\displaystyle}

%================================================================================================================================
%
% Macros - Commandes
%
%================================================================================================================================

\newcommand\meta[2]{    			% Utilisé pour créer le post HTML.
	\def\titre{titre}
	\def\url{url}
	\def\arg{#1}
	\ifx\titre\arg
		\newcommand\maintitle{#2}
		\fancyhead[L]{#2}
		{\Large\sffamily \MakeUppercase{#2}}
		\vspace{1mm}\textcolor{mcvert}{\hrule}
	\fi 
	\ifx\url\arg
		\fancyfoot[L]{\href{https://www.maths-cours.fr#2}{\black \footnotesize{https://www.maths-cours.fr#2}}}
	\fi 
}


\newcommand\TitreC[1]{    		% Titre centré
     \needspace{3\baselineskip}
     \begin{center}\textbf{#1}\end{center}
}

\newcommand\newpar{    		% paragraphe
     \par
}

\newcommand\nosp {    		% commande vide (pas d'espace)
}
\newcommand{\id}[1]{} %ignore

\newcommand\boite[2]{				% Boite simple sans titre
	\vspace{5mm}
	\setlength{\fboxrule}{0.2mm}
	\setlength{\fboxsep}{5mm}	
	\fcolorbox{#1}{#1!3}{\makebox[\linewidth-2\fboxrule-2\fboxsep]{
  		\begin{minipage}[t]{\linewidth-2\fboxrule-4\fboxsep}\setlength{\parskip}{3mm}
  			 #2
  		\end{minipage}
	}}
	\vspace{5mm}
}

\newcommand\CBox[4]{				% Boites
	\vspace{5mm}
	\setlength{\fboxrule}{0.2mm}
	\setlength{\fboxsep}{5mm}
	
	\fcolorbox{#1}{#1!3}{\makebox[\linewidth-2\fboxrule-2\fboxsep]{
		\begin{minipage}[t]{1cm}\setlength{\parskip}{3mm}
	  		\textcolor{#1}{\LARGE{#2}}    
 	 	\end{minipage}  
  		\begin{minipage}[t]{\linewidth-2\fboxrule-4\fboxsep}\setlength{\parskip}{3mm}
			\raisebox{1.2mm}{\normalsize\sffamily{\textcolor{#1}{#3}}}						
  			 #4
  		\end{minipage}
	}}
	\vspace{5mm}
}

\newcommand\cadre[3]{				% Boites convertible html
	\par
	\vspace{2mm}
	\setlength{\fboxrule}{0.1mm}
	\setlength{\fboxsep}{5mm}
	\fcolorbox{#1}{white}{\makebox[\linewidth-2\fboxrule-2\fboxsep]{
  		\begin{minipage}[t]{\linewidth-2\fboxrule-4\fboxsep}\setlength{\parskip}{3mm}
			\raisebox{-2.5mm}{\sffamily \small{\textcolor{#1}{\MakeUppercase{#2}}}}		
			\par		
  			 #3
 	 		\end{minipage}
	}}
		\vspace{2mm}
	\par
}

\newcommand\bloc[3]{				% Boites convertible html sans bordure
     \needspace{2\baselineskip}
     {\sffamily \small{\textcolor{#1}{\MakeUppercase{#2}}}}    
		\par		
  			 #3
		\par
}

\newcommand\CHelp[1]{
     \CBox{Plum}{\faInfoCircle}{À RETENIR}{#1}
}

\newcommand\CUp[1]{
     \CBox{NavyBlue}{\faThumbsOUp}{EN PRATIQUE}{#1}
}

\newcommand\CInfo[1]{
     \CBox{Sepia}{\faArrowCircleRight}{REMARQUE}{#1}
}

\newcommand\CRedac[1]{
     \CBox{PineGreen}{\faEdit}{BIEN R\'EDIGER}{#1}
}

\newcommand\CError[1]{
     \CBox{Red}{\faExclamationTriangle}{ATTENTION}{#1}
}

\newcommand\TitreExo[2]{
\needspace{4\baselineskip}
 {\sffamily\large EXERCICE #1\ (\emph{#2 points})}
\vspace{5mm}
}

\newcommand\img[2]{
          \includegraphics[width=#2\paperwidth]{\imgdir#1}
}

\newcommand\imgsvg[2]{
       \begin{center}   \includegraphics[width=#2\paperwidth]{\imgsvgdir#1} \end{center}
}


\newcommand\Lien[2]{
     \href{#1}{#2 \tiny \faExternalLink}
}
\newcommand\mcLien[2]{
     \href{https~://www.maths-cours.fr/#1}{#2 \tiny \faExternalLink}
}

\newcommand{\euro}{\eurologo{}}

%================================================================================================================================
%
% Macros - Environement
%
%================================================================================================================================

\newenvironment{tex}{ %
}
{%
}

\newenvironment{indente}{ %
	\setlength\parindent{10mm}
}

{
	\setlength\parindent{0mm}
}

\newenvironment{corrige}{%
     \needspace{3\baselineskip}
     \medskip
     \textbf{\textsc{Corrigé}}
     \medskip
}
{
}

\newenvironment{extern}{%
     \begin{center}
     }
     {
     \end{center}
}

\NewEnviron{code}{%
	\par
     \boite{gray}{\texttt{%
     \BODY
     }}
     \par
}

\newenvironment{vbloc}{% boite sans cadre empeche saut de page
     \begin{minipage}[t]{\linewidth}
     }
     {
     \end{minipage}
}
\NewEnviron{h2}{%
    \needspace{3\baselineskip}
    \vspace{0.6cm}
	\noindent \MakeUppercase{\sffamily \large \BODY}
	\vspace{1mm}\textcolor{mcgris}{\hrule}\vspace{0.4cm}
	\par
}{}

\NewEnviron{h3}{%
    \needspace{3\baselineskip}
	\vspace{5mm}
	\textsc{\BODY}
	\par
}

\NewEnviron{margeneg}{ %
\begin{addmargin}[-1cm]{0cm}
\BODY
\end{addmargin}
}

\NewEnviron{html}{%
}

\begin{document}
\meta{url}{/exercices/fonctions-bac-es-l-polynesie-2018/}
\meta{pid}{9326}
\meta{titre}{Fonctions – Bac ES/L Polynésie 2018}
\meta{type}{exercices}
%
\begin{h2}Exercice 4 (5 points)\end{h2}
\textbf{Commun à  tous les candidats}
\medbreak
\begin{center}
     \textit{ Les parties de cet exercice peuvent être traitées indépendamment.}
\end{center}
\medbreak
Une usine qui fabrique un produit A, décide de fabriquer un nouveau produit B afin d'augmenter son chiffre d'affaires. La quantité, exprimée en tonnes, fabriquée par jour par l'usine est modélisée par~:
\begin{itemize}
     \item la fonction $f$ définie sur [0~;~14] par
     \par
     \[f(x) = 2~000\text{e}^{-0,2x}\]
     \par
     pour le produit A~;
     \item  la fonction $g$ définie sur [0~;~14] par
     \par
     \[g (x)= 15x^2 + 50 x\]
     \par
     pour le produit B, où $x$ est la durée écoulée depuis le lancement du nouveau produit B exprimée en mois.
\end{itemize}
Leurs courbes représentatives respectives $\mathscr{C}_f$ et $\mathscr{C}_g$ sont données ci-dessous.
\begin{center}
     \begin{extern}%width="600" alt="Fonctions quantités Bac ES/L Polynésie 2018"
          \resizebox{10cm}{!}{
               \psset{xunit=0.8cm,yunit=0.0025cm}
               \begin{pspicture}(-1,-200)(17.5,3700)
                    \multido{\n=0+1}{18}{\psline[linewidth=0.2pt,linecolor=gray](\n,0)(\n,3500)}
                    \multido{\n=0+100}{36}{\psline[linewidth=0.1pt,linecolor=lightgray](0,\n)(17,\n)}
                    \multido{\n=0+500}{7}{\psline[linewidth=0.2pt,linecolor=gray](0,\n)(17,\n)}
                    \psaxes[linewidth=1.25pt,Dy=500]{->}(0,0)(0,0)(17,3500)
                    \uput[r](-1.9,3600){$y$ en tonnes}
                    \uput[u](16,-450){$x$ en mois}
                    \psplot[plotpoints=3000,linewidth=1.25pt,linecolor=blue]{0}{14}{2000 2.71828 0.2 x mul exp div}\uput[u](1,1750){\blue $\mathcal{C}_f$}
                    \psplot[plotpoints=3000,linewidth=1.25pt,linecolor=red]{0}{14}{x dup mul 15 mul 50 x mul add}\uput[u](1,100){\red $\mathcal{C}_g$}
               \end{pspicture}
          }
     \end{extern}
\end{center}
\TitreC{Partie A}
\medbreak
Par lecture graphique, sans justification et avec la précision permise par le graphique~:
\medbreak
\begin{enumerate}
     \item Déterminer la durée nécessaire pour que la quantité de produit B dépasse celle du produit A.
     \item L'usine ne peut pas fabriquer une quantité journalière de produit B supérieure à 3~000~tonnes.
     \par
     Au bout de combien de mois cette quantité journalière sera atteinte~?
\end{enumerate}
\bigbreak
\TitreC{Partie B}
\medbreak
Pour tout nombre réel $x$ de l'intervalle [0~;~14] on pose $h(x) = f(x) + g(x)$.
\par
On admet que la fonction $h$ ainsi définie est dérivable sur [0~;~14].
\medbreak
\begin{enumerate}
     \item
     \begin{enumerate}[label=\alph*.]
          \item Que modélise cette fonction dans le contexte de l'exercice~?
          \item Montrer que, pour tout nombre réel $x$ de l'intervalle [0~;~14]
          $h'(x) = - 400\text{e}^{-0,2x} + 30x + 50$.
     \end{enumerate}
     \item On admet que le tableau de variation de la fonction $h'$ sur l'intervalle [0~;~14] est~:
     %:-+-+-+-+- Engendré par : http://math.et.info.free.fr/TikZ/TableauxVariations/
     \begin{center}
          \begin{extern}%width="350" alt=""
               \begin{tikzpicture}[scale=0.875]
                    % Styles
                    \tikzstyle{cadre}=[thin]
                    \tikzstyle{fleche}=[->,>=latex,thin]
                    \tikzstyle{nondefini}=[lightgray]
                    % Dimensions Modifiables
                    \def\Lrg{1.5}
                    \def\HtX{0.8}
                    \def\HtY{0.5}
                    % Dimensions Calculées
                    \def\lignex{-0.5*\HtX}
                    \def\lignef{-1.5*\HtX}
                    \def\separateur{-0.5*\Lrg}
                    % Largeur du tableau
                    \def\gauche{-3*\Lrg}
                    \def\droite{3*\Lrg}
                    % Hauteur du tableau
                    \def\haut{0.5*\HtX}
                    \def\bas{-1.5*\HtX-2*\HtY}
                    % Ligne de l'abscisse : x
                    \node at (-1.8*\Lrg,0) {$x$};
                    \node at (0*\Lrg,0) {$0$};
                    \node at (2*\Lrg,0) {$14$};
                    % Ligne de la fonction : f(x)
                    \node  at (-1.69*\Lrg,{-1*\HtX+(-1)*\HtY}) {variation de $h'$};
                    \node (f1) at (0*\Lrg,{-1*\HtX+(-2)*\HtY}) {$-350$};
                    \node (f2) at (2*\Lrg,{-1*\HtX+(0)*\HtY}) {$h'(14) \approx 446$};
                    % Flèches
                    \draw[fleche] (f1) -- (f2);
                    % Encadrement
                    \draw[cadre] (\separateur,\haut) -- (\separateur,\bas);
                    \draw[cadre] (\gauche,\haut) rectangle  (\droite,\bas);
                    \draw[cadre] (\gauche,\lignex) -- (\droite,\lignex);
               \end{tikzpicture}
          \end{extern}
     \end{center}
     %:-+-+-+-+- Fin
     \begin{enumerate}[label=\alph*.]
          \item Justifier que l'équation $h'(x)= 0$ admet une unique solution $\alpha$ sur l'intervalle [0~;~14] et donner un encadrement d'amplitude $0,1$ de $\alpha$.
          \item  En déduire les variations de la fonction $h$ sur l'intervalle [0~;~14].
     \end{enumerate}
     \item Voici un algorithme~:
     \begin{center}
          \begin{extern}%width="320" alt="Algorithme Bac ES/L Polynésie 2018"
               \begin{tabularx}{0.5\linewidth}{|X|}\hline
                    $Y \gets -400 \text{exp}(- 0,2X) + 30X + 50$\\
                    Tant que $Y \leqslant 0$\\
                    \hspace{0.7cm}$X \gets X + 0,1$\\
                    \hspace{0.7cm}$Y \gets  -400 \text{exp}(- 0,2X)+ 30X +50$\\
                    Fin Tant que\\ \hline
               \end{tabularx}
          \end{extern}
     \end{center}
     \begin{enumerate}[label=\alph*.]
          \item Si la variable $X$ contient la valeur 3 avant l'exécution de cet algorithme, que contient la variable $X$ après l'exécution de cet algorithme~?
          \item En supposant toujours que la variable $X$ contient la valeur $3$ avant l'exécution de cet algorithme, modifier l'algorithme de façon à ce que X contienne une valeur approchée à $0,001$ près de a après l'exécution de l'algorithme.
     \end{enumerate}
     \item
     \begin{enumerate}[label=\alph*.]
          \item Vérifier qu'une primitive $H$ de la fonction $h$ sur [0~;~14] est~:
          \par
          \[H(x) = - 10~000 \text{e}^{- 0,2x} + 5x^3 + 25x^2.\]
          \item  Calculer une valeur approchée à l'unité près de
          $\dfrac{1}{12} \displaystyle\int_0^{12}  h(x)\:\text{d}x$.
          \item  Donner une interprétation dans le contexte de l'exercice.
     \end{enumerate}
\end{enumerate}

\end{document}
µ
\documentclass[a4paper]{article}

%================================================================================================================================
%
% Packages
%
%================================================================================================================================

\usepackage[T1]{fontenc} 	% pour caractères accentués
\usepackage[utf8]{inputenc}  % encodage utf8
\usepackage[french]{babel}	% langue : français
\usepackage{fourier}			% caractères plus lisibles
\usepackage[dvipsnames]{xcolor} % couleurs
\usepackage{fancyhdr}		% réglage header footer
\usepackage{needspace}		% empêcher sauts de page mal placés
\usepackage{graphicx}		% pour inclure des graphiques
\usepackage{enumitem,cprotect}		% personnalise les listes d'items (nécessaire pour ol, al ...)
\usepackage{hyperref}		% Liens hypertexte
\usepackage{pstricks,pst-all,pst-node,pstricks-add,pst-math,pst-plot,pst-tree,pst-eucl} % pstricks
\usepackage[a4paper,includeheadfoot,top=2cm,left=3cm, bottom=2cm,right=3cm]{geometry} % marges etc.
\usepackage{comment}			% commentaires multilignes
\usepackage{amsmath,environ} % maths (matrices, etc.)
\usepackage{amssymb,makeidx}
\usepackage{bm}				% bold maths
\usepackage{tabularx}		% tableaux
\usepackage{colortbl}		% tableaux en couleur
\usepackage{fontawesome}		% Fontawesome
\usepackage{environ}			% environment with command
\usepackage{fp}				% calculs pour ps-tricks
\usepackage{multido}			% pour ps tricks
\usepackage[np]{numprint}	% formattage nombre
\usepackage{tikz,tkz-tab} 			% package principal TikZ
\usepackage{pgfplots}   % axes
\usepackage{mathrsfs}    % cursives
\usepackage{calc}			% calcul taille boites
\usepackage[scaled=0.875]{helvet} % font sans serif
\usepackage{svg} % svg
\usepackage{scrextend} % local margin
\usepackage{scratch} %scratch
\usepackage{multicol} % colonnes
%\usepackage{infix-RPN,pst-func} % formule en notation polanaise inversée
\usepackage{listings}

%================================================================================================================================
%
% Réglages de base
%
%================================================================================================================================

\lstset{
language=Python,   % R code
literate=
{á}{{\'a}}1
{à}{{\`a}}1
{ã}{{\~a}}1
{é}{{\'e}}1
{è}{{\`e}}1
{ê}{{\^e}}1
{í}{{\'i}}1
{ó}{{\'o}}1
{õ}{{\~o}}1
{ú}{{\'u}}1
{ü}{{\"u}}1
{ç}{{\c{c}}}1
{~}{{ }}1
}


\definecolor{codegreen}{rgb}{0,0.6,0}
\definecolor{codegray}{rgb}{0.5,0.5,0.5}
\definecolor{codepurple}{rgb}{0.58,0,0.82}
\definecolor{backcolour}{rgb}{0.95,0.95,0.92}

\lstdefinestyle{mystyle}{
    backgroundcolor=\color{backcolour},   
    commentstyle=\color{codegreen},
    keywordstyle=\color{magenta},
    numberstyle=\tiny\color{codegray},
    stringstyle=\color{codepurple},
    basicstyle=\ttfamily\footnotesize,
    breakatwhitespace=false,         
    breaklines=true,                 
    captionpos=b,                    
    keepspaces=true,                 
    numbers=left,                    
xleftmargin=2em,
framexleftmargin=2em,            
    showspaces=false,                
    showstringspaces=false,
    showtabs=false,                  
    tabsize=2,
    upquote=true
}

\lstset{style=mystyle}


\lstset{style=mystyle}
\newcommand{\imgdir}{C:/laragon/www/newmc/assets/imgsvg/}
\newcommand{\imgsvgdir}{C:/laragon/www/newmc/assets/imgsvg/}

\definecolor{mcgris}{RGB}{220, 220, 220}% ancien~; pour compatibilité
\definecolor{mcbleu}{RGB}{52, 152, 219}
\definecolor{mcvert}{RGB}{125, 194, 70}
\definecolor{mcmauve}{RGB}{154, 0, 215}
\definecolor{mcorange}{RGB}{255, 96, 0}
\definecolor{mcturquoise}{RGB}{0, 153, 153}
\definecolor{mcrouge}{RGB}{255, 0, 0}
\definecolor{mclightvert}{RGB}{205, 234, 190}

\definecolor{gris}{RGB}{220, 220, 220}
\definecolor{bleu}{RGB}{52, 152, 219}
\definecolor{vert}{RGB}{125, 194, 70}
\definecolor{mauve}{RGB}{154, 0, 215}
\definecolor{orange}{RGB}{255, 96, 0}
\definecolor{turquoise}{RGB}{0, 153, 153}
\definecolor{rouge}{RGB}{255, 0, 0}
\definecolor{lightvert}{RGB}{205, 234, 190}
\setitemize[0]{label=\color{lightvert}  $\bullet$}

\pagestyle{fancy}
\renewcommand{\headrulewidth}{0.2pt}
\fancyhead[L]{maths-cours.fr}
\fancyhead[R]{\thepage}
\renewcommand{\footrulewidth}{0.2pt}
\fancyfoot[C]{}

\newcolumntype{C}{>{\centering\arraybackslash}X}
\newcolumntype{s}{>{\hsize=.35\hsize\arraybackslash}X}

\setlength{\parindent}{0pt}		 
\setlength{\parskip}{3mm}
\setlength{\headheight}{1cm}

\def\ebook{ebook}
\def\book{book}
\def\web{web}
\def\type{web}

\newcommand{\vect}[1]{\overrightarrow{\,\mathstrut#1\,}}

\def\Oij{$\left(\text{O}~;~\vect{\imath},~\vect{\jmath}\right)$}
\def\Oijk{$\left(\text{O}~;~\vect{\imath},~\vect{\jmath},~\vect{k}\right)$}
\def\Ouv{$\left(\text{O}~;~\vect{u},~\vect{v}\right)$}

\hypersetup{breaklinks=true, colorlinks = true, linkcolor = OliveGreen, urlcolor = OliveGreen, citecolor = OliveGreen, pdfauthor={Didier BONNEL - https://www.maths-cours.fr} } % supprime les bordures autour des liens

\renewcommand{\arg}[0]{\text{arg}}

\everymath{\displaystyle}

%================================================================================================================================
%
% Macros - Commandes
%
%================================================================================================================================

\newcommand\meta[2]{    			% Utilisé pour créer le post HTML.
	\def\titre{titre}
	\def\url{url}
	\def\arg{#1}
	\ifx\titre\arg
		\newcommand\maintitle{#2}
		\fancyhead[L]{#2}
		{\Large\sffamily \MakeUppercase{#2}}
		\vspace{1mm}\textcolor{mcvert}{\hrule}
	\fi 
	\ifx\url\arg
		\fancyfoot[L]{\href{https://www.maths-cours.fr#2}{\black \footnotesize{https://www.maths-cours.fr#2}}}
	\fi 
}


\newcommand\TitreC[1]{    		% Titre centré
     \needspace{3\baselineskip}
     \begin{center}\textbf{#1}\end{center}
}

\newcommand\newpar{    		% paragraphe
     \par
}

\newcommand\nosp {    		% commande vide (pas d'espace)
}
\newcommand{\id}[1]{} %ignore

\newcommand\boite[2]{				% Boite simple sans titre
	\vspace{5mm}
	\setlength{\fboxrule}{0.2mm}
	\setlength{\fboxsep}{5mm}	
	\fcolorbox{#1}{#1!3}{\makebox[\linewidth-2\fboxrule-2\fboxsep]{
  		\begin{minipage}[t]{\linewidth-2\fboxrule-4\fboxsep}\setlength{\parskip}{3mm}
  			 #2
  		\end{minipage}
	}}
	\vspace{5mm}
}

\newcommand\CBox[4]{				% Boites
	\vspace{5mm}
	\setlength{\fboxrule}{0.2mm}
	\setlength{\fboxsep}{5mm}
	
	\fcolorbox{#1}{#1!3}{\makebox[\linewidth-2\fboxrule-2\fboxsep]{
		\begin{minipage}[t]{1cm}\setlength{\parskip}{3mm}
	  		\textcolor{#1}{\LARGE{#2}}    
 	 	\end{minipage}  
  		\begin{minipage}[t]{\linewidth-2\fboxrule-4\fboxsep}\setlength{\parskip}{3mm}
			\raisebox{1.2mm}{\normalsize\sffamily{\textcolor{#1}{#3}}}						
  			 #4
  		\end{minipage}
	}}
	\vspace{5mm}
}

\newcommand\cadre[3]{				% Boites convertible html
	\par
	\vspace{2mm}
	\setlength{\fboxrule}{0.1mm}
	\setlength{\fboxsep}{5mm}
	\fcolorbox{#1}{white}{\makebox[\linewidth-2\fboxrule-2\fboxsep]{
  		\begin{minipage}[t]{\linewidth-2\fboxrule-4\fboxsep}\setlength{\parskip}{3mm}
			\raisebox{-2.5mm}{\sffamily \small{\textcolor{#1}{\MakeUppercase{#2}}}}		
			\par		
  			 #3
 	 		\end{minipage}
	}}
		\vspace{2mm}
	\par
}

\newcommand\bloc[3]{				% Boites convertible html sans bordure
     \needspace{2\baselineskip}
     {\sffamily \small{\textcolor{#1}{\MakeUppercase{#2}}}}    
		\par		
  			 #3
		\par
}

\newcommand\CHelp[1]{
     \CBox{Plum}{\faInfoCircle}{À RETENIR}{#1}
}

\newcommand\CUp[1]{
     \CBox{NavyBlue}{\faThumbsOUp}{EN PRATIQUE}{#1}
}

\newcommand\CInfo[1]{
     \CBox{Sepia}{\faArrowCircleRight}{REMARQUE}{#1}
}

\newcommand\CRedac[1]{
     \CBox{PineGreen}{\faEdit}{BIEN R\'EDIGER}{#1}
}

\newcommand\CError[1]{
     \CBox{Red}{\faExclamationTriangle}{ATTENTION}{#1}
}

\newcommand\TitreExo[2]{
\needspace{4\baselineskip}
 {\sffamily\large EXERCICE #1\ (\emph{#2 points})}
\vspace{5mm}
}

\newcommand\img[2]{
          \includegraphics[width=#2\paperwidth]{\imgdir#1}
}

\newcommand\imgsvg[2]{
       \begin{center}   \includegraphics[width=#2\paperwidth]{\imgsvgdir#1} \end{center}
}


\newcommand\Lien[2]{
     \href{#1}{#2 \tiny \faExternalLink}
}
\newcommand\mcLien[2]{
     \href{https~://www.maths-cours.fr/#1}{#2 \tiny \faExternalLink}
}

\newcommand{\euro}{\eurologo{}}

%================================================================================================================================
%
% Macros - Environement
%
%================================================================================================================================

\newenvironment{tex}{ %
}
{%
}

\newenvironment{indente}{ %
	\setlength\parindent{10mm}
}

{
	\setlength\parindent{0mm}
}

\newenvironment{corrige}{%
     \needspace{3\baselineskip}
     \medskip
     \textbf{\textsc{Corrigé}}
     \medskip
}
{
}

\newenvironment{extern}{%
     \begin{center}
     }
     {
     \end{center}
}

\NewEnviron{code}{%
	\par
     \boite{gray}{\texttt{%
     \BODY
     }}
     \par
}

\newenvironment{vbloc}{% boite sans cadre empeche saut de page
     \begin{minipage}[t]{\linewidth}
     }
     {
     \end{minipage}
}
\NewEnviron{h2}{%
    \needspace{3\baselineskip}
    \vspace{0.6cm}
	\noindent \MakeUppercase{\sffamily \large \BODY}
	\vspace{1mm}\textcolor{mcgris}{\hrule}\vspace{0.4cm}
	\par
}{}

\NewEnviron{h3}{%
    \needspace{3\baselineskip}
	\vspace{5mm}
	\textsc{\BODY}
	\par
}

\NewEnviron{margeneg}{ %
\begin{addmargin}[-1cm]{0cm}
\BODY
\end{addmargin}
}

\NewEnviron{html}{%
}

\begin{document}
\meta{url}{/exercices/graphes-bac-es-polynesie-2018-spe/}
\meta{pid}{9328}
\meta{titre}{Graphes – Bac ES Polynésie 2018 (spé)}
\meta{type}{exercices}
%
\begin{h2}Exercice 3 (5 points)\end{h2}
\textbf{Candidats ayant choisi la spécialité \og mathématiques \fg{}}
\medbreak
Un journaliste britannique d'une revue consacrée à l'automobile doit tester les autoroutes
françaises. Pour remplir sa mission, il décide de louer une voiture et de circuler entre six grandes villes françaises~: Bordeaux $(B)$, Lyon $(L)$, Marseille $(M)$, Nantes $(N)$, Paris $(P)$ et Toulouse $(T)$.
\smallbreak
Le réseau autoroutier reliant ces six villes est modélisé par le graphe ci-dessous sur lequel les sommets représentent les villes et les arêtes les liaisons autoroutières entre ces villes.
\begin{center}
     \begin{extern}%width="300" alt="Graphe Bac ES/L Polynésie 2018"
          \resizebox{8cm}{!}{
               \begin{pspicture}(7,6.5)
                    \rput[t](2.6,2.8){\cnode*{2pt}{A}}\rput[r](2.6,2.8){~~$N$~~}
                    \rput[tr](2.4,5.7){\cnode*{2pt}{B}} \rput[br](2.4,5.8){$P$}
                    \rput[tr](3.6,3.4){\cnode*{2pt}{C}}\rput[br](3.6,3.5){$L$~~}
                    \rput[ur](5.7,3){\cnode*{2pt}{D}}\rput[tr](5.7,3){$M$~~}
                    \rput[tr](2,1.2){\cnode*{2pt}{E}}\rput[tr](2,1.1){$B$~~}
                    \rput[br](6.3,0.7){\cnode*{2pt}{F}}  \rput[tl](6.3,0.7){~~$T$}
                    \ncarc[arcangle=30]{A}{B} \ncarc[arcangle=-30]{C}{B}
                    \ncarc[arcangle=70]{B}{F}\ncarc[arcangle=-60]{B}{E}
                    \ncarc[arcangle=-30]{C}{A}\ncarc[arcangle=30]{C}{E}
                    \ncarc[arcangle=-30]{C}{F}\ncarc[arcangle=-30]{D}{C}
                    \ncarc[arcangle=30]{D}{F}\ncarc[arcangle=-30]{A}{E}
                    \ncarc[arcangle=30]{F}{E}
               \end{pspicture}
          }
     \end{extern}
\end{center}
\bigbreak
\TitreC{Partie A}
\medbreak
\begin{enumerate}
     \item
     \begin{enumerate}[label=\alph*.]
          \item Quel est l'ordre du graphe~?
          \item Le graphe est-il complet~? Justifier la réponse.
     \end{enumerate}
     \item
     \begin{enumerate}[label=\alph*.]
          \item On admet que le graphe est connexe. Le journaliste envisage de parcourir chacune des
          liaisons modélisées sur le graphe une fois et une seule. Est-ce possible~? Justifier la réponse.
          \item Le journaliste va-t-il pouvoir louer sa voiture dans un aéroport parisien, parcourir chacune des liaisons une et une seule fois puis rendre la voiture dans le même aéroport~? Justifier la réponse.
     \end{enumerate}
     \item  On nomme $G$ la matrice d'adjacence du graphe (les villes étant rangées dans l'ordre
     alphabétique). On donne~:
     \par
     \begin{center}
          $G = \begin{pmatrix}
               0 &\ldots &0 &1 &1 &1 \\
               \ldots&0 &1 &1 &1 &1\\
               0 &1 &\ldots &0 &\ldots &1\\
               1 &1 &0 &0 &1 &0 \\
               1 &1 &\ldots &1 &0 &1\\
               1 &1 &1 &0 &1 &0
          \end{pmatrix} $
          ~  et ~ $G^3 = \begin{pmatrix}
               10 &13 &5 &10 &11 &12\\
               13 &12 &8 &11 &13 &12\\
               5 &8 &2 &5 &5 &7\\
               10 &11 &5 &6 &10 &7\\
               11 &13& 5 &10 &10 &12\\
               12 &12 &7 &7 &12 &8
          \end{pmatrix}$
     \end{center}
     \begin{enumerate}[label=\alph*.]
          \item Recopier et compléter la matrice d'adjacence.
          \item Alors qu'il se trouve à Paris, le rédacteur en chef demande au journaliste d'être à Marseille exactement trois jours plus tard pour assister à une course automobile. Le journaliste décidechaque jour de s'arrêter dans une ville différente.
          \par
          Déterminer le nombre de trajets possibles.
     \end{enumerate}
\end{enumerate}
\bigbreak
\TitreC{Partie B}
\medbreak
On a indiqué sur le graphe ci-dessous le temps nécessaire en minutes pour parcourir chacune des liaisons autoroutières.
\begin{center}
     \begin{extern}%width="300" alt="Graphe pondéré Bac ES/L Polynésie 2018"
          \resizebox{8cm}{!}{
               \begin{pspicture}(7,6.5)
                    \rput[t](2.6,2.8){\cnode*{2pt}{A}}\rput[r](2.6,2.8){~~$N$~~}
                    \rput[tr](2.4,5.7){\cnode*{2pt}{B}} \rput[br](2.4,5.8){$P$}
                    \rput[tr](3.6,3.4){\cnode*{2pt}{C}}\rput[br](3.6,3.5){$L$~~}
                    \rput[ur](5.7,3){\cnode*{2pt}{D}}\rput[tr](5.7,3){$M$~~}
                    \rput[tr](2,1.2){\cnode*{2pt}{E}}\rput[tr](2,1.1){$B$~~}
                    \rput[br](6.3,0.7){\cnode*{2pt}{F}}  \rput[tl](6.3,0.7){~~$T$}
                    \ncarc[arcangle=30]{A}{B}\ncput*{222} \ncarc[arcangle=-30]{C}{B}\ncput*{268}
                    \ncarc[arcangle=60]{B}{F}\ncput*{391}\ncarc[arcangle=-60]{B}{E}\ncput*{340}
                    \ncarc[arcangle=-30]{C}{A}\ncput*{396}\ncarc[arcangle=30]{C}{E}\ncput*{336}
                    \ncarc[arcangle=-30]{C}{F}\ncput*{305}\ncarc[arcangle=-30]{D}{C}\ncput*{214}
                    \ncarc[arcangle=30]{D}{F}\ncput*{236}\ncarc[arcangle=-30]{A}{E}\ncput*{206}
                    \ncarc[arcangle=30]{F}{E}\ncput*{153}
               \end{pspicture}
          }
     \end{extern}
\end{center}
Le journaliste se trouve à Nantes et désire se rendre le plus rapidement possible à Marseille.
\par
Déterminer un trajet qui minimise son temps de parcours.

\end{document}
µ
\documentclass[a4paper]{article}

%================================================================================================================================
%
% Packages
%
%================================================================================================================================

\usepackage[T1]{fontenc} 	% pour caractères accentués
\usepackage[utf8]{inputenc}  % encodage utf8
\usepackage[french]{babel}	% langue : français
\usepackage{fourier}			% caractères plus lisibles
\usepackage[dvipsnames]{xcolor} % couleurs
\usepackage{fancyhdr}		% réglage header footer
\usepackage{needspace}		% empêcher sauts de page mal placés
\usepackage{graphicx}		% pour inclure des graphiques
\usepackage{enumitem,cprotect}		% personnalise les listes d'items (nécessaire pour ol, al ...)
\usepackage{hyperref}		% Liens hypertexte
\usepackage{pstricks,pst-all,pst-node,pstricks-add,pst-math,pst-plot,pst-tree,pst-eucl} % pstricks
\usepackage[a4paper,includeheadfoot,top=2cm,left=3cm, bottom=2cm,right=3cm]{geometry} % marges etc.
\usepackage{comment}			% commentaires multilignes
\usepackage{amsmath,environ} % maths (matrices, etc.)
\usepackage{amssymb,makeidx}
\usepackage{bm}				% bold maths
\usepackage{tabularx}		% tableaux
\usepackage{colortbl}		% tableaux en couleur
\usepackage{fontawesome}		% Fontawesome
\usepackage{environ}			% environment with command
\usepackage{fp}				% calculs pour ps-tricks
\usepackage{multido}			% pour ps tricks
\usepackage[np]{numprint}	% formattage nombre
\usepackage{tikz,tkz-tab} 			% package principal TikZ
\usepackage{pgfplots}   % axes
\usepackage{mathrsfs}    % cursives
\usepackage{calc}			% calcul taille boites
\usepackage[scaled=0.875]{helvet} % font sans serif
\usepackage{svg} % svg
\usepackage{scrextend} % local margin
\usepackage{scratch} %scratch
\usepackage{multicol} % colonnes
%\usepackage{infix-RPN,pst-func} % formule en notation polanaise inversée
\usepackage{listings}

%================================================================================================================================
%
% Réglages de base
%
%================================================================================================================================

\lstset{
language=Python,   % R code
literate=
{á}{{\'a}}1
{à}{{\`a}}1
{ã}{{\~a}}1
{é}{{\'e}}1
{è}{{\`e}}1
{ê}{{\^e}}1
{í}{{\'i}}1
{ó}{{\'o}}1
{õ}{{\~o}}1
{ú}{{\'u}}1
{ü}{{\"u}}1
{ç}{{\c{c}}}1
{~}{{ }}1
}


\definecolor{codegreen}{rgb}{0,0.6,0}
\definecolor{codegray}{rgb}{0.5,0.5,0.5}
\definecolor{codepurple}{rgb}{0.58,0,0.82}
\definecolor{backcolour}{rgb}{0.95,0.95,0.92}

\lstdefinestyle{mystyle}{
    backgroundcolor=\color{backcolour},   
    commentstyle=\color{codegreen},
    keywordstyle=\color{magenta},
    numberstyle=\tiny\color{codegray},
    stringstyle=\color{codepurple},
    basicstyle=\ttfamily\footnotesize,
    breakatwhitespace=false,         
    breaklines=true,                 
    captionpos=b,                    
    keepspaces=true,                 
    numbers=left,                    
xleftmargin=2em,
framexleftmargin=2em,            
    showspaces=false,                
    showstringspaces=false,
    showtabs=false,                  
    tabsize=2,
    upquote=true
}

\lstset{style=mystyle}


\lstset{style=mystyle}
\newcommand{\imgdir}{C:/laragon/www/newmc/assets/imgsvg/}
\newcommand{\imgsvgdir}{C:/laragon/www/newmc/assets/imgsvg/}

\definecolor{mcgris}{RGB}{220, 220, 220}% ancien~; pour compatibilité
\definecolor{mcbleu}{RGB}{52, 152, 219}
\definecolor{mcvert}{RGB}{125, 194, 70}
\definecolor{mcmauve}{RGB}{154, 0, 215}
\definecolor{mcorange}{RGB}{255, 96, 0}
\definecolor{mcturquoise}{RGB}{0, 153, 153}
\definecolor{mcrouge}{RGB}{255, 0, 0}
\definecolor{mclightvert}{RGB}{205, 234, 190}

\definecolor{gris}{RGB}{220, 220, 220}
\definecolor{bleu}{RGB}{52, 152, 219}
\definecolor{vert}{RGB}{125, 194, 70}
\definecolor{mauve}{RGB}{154, 0, 215}
\definecolor{orange}{RGB}{255, 96, 0}
\definecolor{turquoise}{RGB}{0, 153, 153}
\definecolor{rouge}{RGB}{255, 0, 0}
\definecolor{lightvert}{RGB}{205, 234, 190}
\setitemize[0]{label=\color{lightvert}  $\bullet$}

\pagestyle{fancy}
\renewcommand{\headrulewidth}{0.2pt}
\fancyhead[L]{maths-cours.fr}
\fancyhead[R]{\thepage}
\renewcommand{\footrulewidth}{0.2pt}
\fancyfoot[C]{}

\newcolumntype{C}{>{\centering\arraybackslash}X}
\newcolumntype{s}{>{\hsize=.35\hsize\arraybackslash}X}

\setlength{\parindent}{0pt}		 
\setlength{\parskip}{3mm}
\setlength{\headheight}{1cm}

\def\ebook{ebook}
\def\book{book}
\def\web{web}
\def\type{web}

\newcommand{\vect}[1]{\overrightarrow{\,\mathstrut#1\,}}

\def\Oij{$\left(\text{O}~;~\vect{\imath},~\vect{\jmath}\right)$}
\def\Oijk{$\left(\text{O}~;~\vect{\imath},~\vect{\jmath},~\vect{k}\right)$}
\def\Ouv{$\left(\text{O}~;~\vect{u},~\vect{v}\right)$}

\hypersetup{breaklinks=true, colorlinks = true, linkcolor = OliveGreen, urlcolor = OliveGreen, citecolor = OliveGreen, pdfauthor={Didier BONNEL - https://www.maths-cours.fr} } % supprime les bordures autour des liens

\renewcommand{\arg}[0]{\text{arg}}

\everymath{\displaystyle}

%================================================================================================================================
%
% Macros - Commandes
%
%================================================================================================================================

\newcommand\meta[2]{    			% Utilisé pour créer le post HTML.
	\def\titre{titre}
	\def\url{url}
	\def\arg{#1}
	\ifx\titre\arg
		\newcommand\maintitle{#2}
		\fancyhead[L]{#2}
		{\Large\sffamily \MakeUppercase{#2}}
		\vspace{1mm}\textcolor{mcvert}{\hrule}
	\fi 
	\ifx\url\arg
		\fancyfoot[L]{\href{https://www.maths-cours.fr#2}{\black \footnotesize{https://www.maths-cours.fr#2}}}
	\fi 
}


\newcommand\TitreC[1]{    		% Titre centré
     \needspace{3\baselineskip}
     \begin{center}\textbf{#1}\end{center}
}

\newcommand\newpar{    		% paragraphe
     \par
}

\newcommand\nosp {    		% commande vide (pas d'espace)
}
\newcommand{\id}[1]{} %ignore

\newcommand\boite[2]{				% Boite simple sans titre
	\vspace{5mm}
	\setlength{\fboxrule}{0.2mm}
	\setlength{\fboxsep}{5mm}	
	\fcolorbox{#1}{#1!3}{\makebox[\linewidth-2\fboxrule-2\fboxsep]{
  		\begin{minipage}[t]{\linewidth-2\fboxrule-4\fboxsep}\setlength{\parskip}{3mm}
  			 #2
  		\end{minipage}
	}}
	\vspace{5mm}
}

\newcommand\CBox[4]{				% Boites
	\vspace{5mm}
	\setlength{\fboxrule}{0.2mm}
	\setlength{\fboxsep}{5mm}
	
	\fcolorbox{#1}{#1!3}{\makebox[\linewidth-2\fboxrule-2\fboxsep]{
		\begin{minipage}[t]{1cm}\setlength{\parskip}{3mm}
	  		\textcolor{#1}{\LARGE{#2}}    
 	 	\end{minipage}  
  		\begin{minipage}[t]{\linewidth-2\fboxrule-4\fboxsep}\setlength{\parskip}{3mm}
			\raisebox{1.2mm}{\normalsize\sffamily{\textcolor{#1}{#3}}}						
  			 #4
  		\end{minipage}
	}}
	\vspace{5mm}
}

\newcommand\cadre[3]{				% Boites convertible html
	\par
	\vspace{2mm}
	\setlength{\fboxrule}{0.1mm}
	\setlength{\fboxsep}{5mm}
	\fcolorbox{#1}{white}{\makebox[\linewidth-2\fboxrule-2\fboxsep]{
  		\begin{minipage}[t]{\linewidth-2\fboxrule-4\fboxsep}\setlength{\parskip}{3mm}
			\raisebox{-2.5mm}{\sffamily \small{\textcolor{#1}{\MakeUppercase{#2}}}}		
			\par		
  			 #3
 	 		\end{minipage}
	}}
		\vspace{2mm}
	\par
}

\newcommand\bloc[3]{				% Boites convertible html sans bordure
     \needspace{2\baselineskip}
     {\sffamily \small{\textcolor{#1}{\MakeUppercase{#2}}}}    
		\par		
  			 #3
		\par
}

\newcommand\CHelp[1]{
     \CBox{Plum}{\faInfoCircle}{À RETENIR}{#1}
}

\newcommand\CUp[1]{
     \CBox{NavyBlue}{\faThumbsOUp}{EN PRATIQUE}{#1}
}

\newcommand\CInfo[1]{
     \CBox{Sepia}{\faArrowCircleRight}{REMARQUE}{#1}
}

\newcommand\CRedac[1]{
     \CBox{PineGreen}{\faEdit}{BIEN R\'EDIGER}{#1}
}

\newcommand\CError[1]{
     \CBox{Red}{\faExclamationTriangle}{ATTENTION}{#1}
}

\newcommand\TitreExo[2]{
\needspace{4\baselineskip}
 {\sffamily\large EXERCICE #1\ (\emph{#2 points})}
\vspace{5mm}
}

\newcommand\img[2]{
          \includegraphics[width=#2\paperwidth]{\imgdir#1}
}

\newcommand\imgsvg[2]{
       \begin{center}   \includegraphics[width=#2\paperwidth]{\imgsvgdir#1} \end{center}
}


\newcommand\Lien[2]{
     \href{#1}{#2 \tiny \faExternalLink}
}
\newcommand\mcLien[2]{
     \href{https~://www.maths-cours.fr/#1}{#2 \tiny \faExternalLink}
}

\newcommand{\euro}{\eurologo{}}

%================================================================================================================================
%
% Macros - Environement
%
%================================================================================================================================

\newenvironment{tex}{ %
}
{%
}

\newenvironment{indente}{ %
	\setlength\parindent{10mm}
}

{
	\setlength\parindent{0mm}
}

\newenvironment{corrige}{%
     \needspace{3\baselineskip}
     \medskip
     \textbf{\textsc{Corrigé}}
     \medskip
}
{
}

\newenvironment{extern}{%
     \begin{center}
     }
     {
     \end{center}
}

\NewEnviron{code}{%
	\par
     \boite{gray}{\texttt{%
     \BODY
     }}
     \par
}

\newenvironment{vbloc}{% boite sans cadre empeche saut de page
     \begin{minipage}[t]{\linewidth}
     }
     {
     \end{minipage}
}
\NewEnviron{h2}{%
    \needspace{3\baselineskip}
    \vspace{0.6cm}
	\noindent \MakeUppercase{\sffamily \large \BODY}
	\vspace{1mm}\textcolor{mcgris}{\hrule}\vspace{0.4cm}
	\par
}{}

\NewEnviron{h3}{%
    \needspace{3\baselineskip}
	\vspace{5mm}
	\textsc{\BODY}
	\par
}

\NewEnviron{margeneg}{ %
\begin{addmargin}[-1cm]{0cm}
\BODY
\end{addmargin}
}

\NewEnviron{html}{%
}

\begin{document}
\meta{url}{/exercices/qcm-bac-es-l-asie-2018/}
\meta{pid}{9371}
\meta{titre}{QCM – Bac ES/L Asie 2018}
\meta{type}{exercices}
%
\begin{h2}Exercice 1 (5 points)\end{h2}
\par
\textbf{Commun  à tous les candidats}
\bigbreak
\par
\emph{Cet exercice est un QCM (questionnaire à choix multiples). Pour chacune des questions
     posées, une seule des quatre réponses proposées est exacte. Indiquer sur la copie le
numéro de la question et recopier la lettre de la réponse choisie.}
\medbreak
\textbf{Aucune justification n'est demandée.}
\medbreak
\emph{Une réponse exacte rapporte 1 point~; une réponse fausse, une réponse multiple ou l'absence de réponse ne rapporte ni n'enlève de point.}
\medbreak
\begin{enumerate}
     \item Pour la recherche d'un emploi, une personne envoie sa candidature à $25$ entreprises.
     \par
     La probabilité qu'une entreprise lui réponde est de $0,2$ et on suppose que ces réponses
     sont indépendantes.
     \par
     Quelle est la probabilité, arrondie au centième, que la personne reçoive au moins $5$
     réponses~?
     \medbreak
     \begin{tabularx}{\linewidth}{*{2}{X}}%class="noborder"
          \textbf{a.~~} 0,20 &\textbf{b.~~} 0,62\\
          \textbf{c.~~}0,38  &\textbf{d.~~} 0,58
     \end{tabularx}
     \medbreak
     \item Pour tout événement $E$ on note $P(E)$ sa probabilité. $X$ est une variable aléatoire suivant la loi normale d'espérance $30$ et d'écart type $\sigma$. Alors~:
     \medbreak
     \textbf{a.~~} $P(X = 30) = 0,5$ \\
     \textbf{b.~~} $P(X < 40 ) < 0,5$\\
     \textbf{c.~~} $P(X < 20) = P(X > 40)$\\
     \textbf{d.~~} $P(X) < 20) > P(X < 30)$\\
     \medbreak
     \item En France, les ventes de tablettes numériques sont passées de 6,2 millions d'unités en 2014 à 4,3 millions d'unités en 2016. Les ventes ont diminué, entre 2014 et 2016,
     d'environ~:
     \medbreak
     \begin{tabularx}{\linewidth}{*{2}{X}}%class="noborder"
          \textbf{a.~~} 65\,\%&\textbf{b.~~}31\,\% \\
          \textbf{c.~~} 20\,\%&\textbf{d.~~}  17\,\%
     \end{tabularx}
     \medbreak
     Pour les questions 4 et 5, on donne ci-dessous
     la représentation graphique d'une
     fonction $f$ définie sur $\mathbb{R}$.
     \medbreak
     \begin{center}
          \begin{extern}%width="600" alt="représentation graphique de la  fonction f"
               \resizebox{8cm}{!}{
                    \begin{pspicture*}(-5.3,-3.5)(8.3,8.5)
                         \psgrid[gridlabels=0pt,subgriddiv=1,gridwidth=0.2pt](-5.6,-3.5)(8.6,8.5)
                         \psaxes[linewidth=0.8pt]{->}(0,0)(-5.3,-3.5)(8.3,8.5)
                         \psaxes[linewidth=1pt]{->}(0,0)(1,1)
                         \psplot[plotpoints=3000,linewidth=1pt,linecolor=blue]{-4.5}{7.2}{x 4 exp 40 div x 3 exp 6 div sub x dup mul 5 div sub 2 x mul add}
                    \end{pspicture*}
               }
          \end{extern}
     \end{center}
     \item Soit $f'$ la dérivée de $f$ et $F$ une
     primitive de $f$ sur $\mathbb{R}$.
     \begin{enumerate}[label=\alph*.]
          \item $f'$ est positive sur [2~;~4].
          \item $f'$ est négative sur [-3~;~-1]-
          \item $F$ est décroissante sur [2~;~4].
          \item $F$ est décroissante sur [-3~;~-1].
     \end{enumerate}
     \item Une des courbes ci-dessous
     représente la fonction $f''$. Laquelle~?
     \medbreak
     \textbf{a.}
     \begin{center}
          \begin{extern}%width="600" alt="représentation graphique de la dérivée seconde - 1"
               \resizebox{8cm}{!}{
                    \begin{pspicture*}(-5.3,-3.5)(8.3,8.5)
                         \psgrid[gridlabels=0pt,subgriddiv=1,gridwidth=0.2pt](-5.6,-3.5)(8.6,8.5)
                         \psaxes[linewidth=0.8pt]{->}(0,0)(-5.3,-3.5)(8.3,8.5)
                         \psaxes[linewidth=1pt]{->}(0,0)(1,1)
                         \psplot[plotpoints=3000,linewidth=1pt,linecolor=blue]{-5}{7.2}{x 5 exp 5 div x 4 exp 1.7 mul sub x 3 exp 2.41667 mul sub x dup mul 39.6 mul add 35 div}
                    \end{pspicture*}
               }
          \end{extern}
     \end{center}
     \textbf{b.}
     \begin{center}
          \begin{extern}%width="600" alt="représentation graphique de la dérivée seconde - 2"
               \resizebox{8cm}{!}{
                    \begin{pspicture*}(-5.3,-3.5)(8.3,8.5)
                         \psgrid[gridlabels=0pt,subgriddiv=1,gridwidth=0.2pt](-5.6,-3.5)(8.6,8.5)
                         \psaxes[linewidth=0.8pt]{->}(0,0)(-5.3,-3.5)(8.3,8.5)
                         \psaxes[linewidth=1pt]{->}(0,0)(1,1)
                         \psplot[plotpoints=3000,linewidth=1pt,linecolor=blue]{-4.5}{7.2}{x 3 exp x dup mul 5 mul sub 4 x mul sub  20 add 10 div}
                    \end{pspicture*}
               }
          \end{extern}
     \end{center}
     \textbf{c.}
     \begin{center}
          \begin{extern}%width="600" alt="représentation graphique de la dérivée seconde - 3"
               \resizebox{8cm}{!}{
                    \begin{pspicture*}(-5.3,-3.5)(8.3,8.5)
                         \psgrid[gridlabels=0pt,subgriddiv=1,gridwidth=0.2pt](-5.6,-3.5)(8.6,8.5)
                         \psaxes[linewidth=0.8pt]{->}(0,0)(-5.3,-3.5)(8.3,8.5)
                         \psaxes[linewidth=1pt]{->}(0,0)(1,1)
                         \psplot[plotpoints=3000,linewidth=1pt,linecolor=blue]{-4.5}{7.2}{x x 3.4 add mul x 4.5 sub mul x 5.5 sub mul  68 neg div}
                    \end{pspicture*}
               }
          \end{extern}
     \end{center}
     \textbf{d.}
     \begin{center}
          \begin{extern}%width="600" alt="représentation graphique de la dérivée seconde - 4"
               \resizebox{8cm}{!}{
                    \begin{pspicture*}(-5.3,-3.5)(8.3,8.5)
                         \psgrid[gridlabels=0pt,subgriddiv=1,gridwidth=0.2pt](-5.6,-3.5)(8.6,8.5)
                         \psaxes[linewidth=0.8pt]{->}(0,0)(-5.3,-3.5)(8.3,8.5)
                         \psaxes[linewidth=1pt]{->}(0,0)(1,1)
                         \psplot[plotpoints=3000,linewidth=1pt,linecolor=blue]{-4.5}{7.2}{x 1.8 sub 2 exp 4 sub 0.35 mul}
                    \end{pspicture*}
               }
          \end{extern}
     \end{center}
\end{enumerate}

\end{document}
µ
\documentclass[a4paper]{article}

%================================================================================================================================
%
% Packages
%
%================================================================================================================================

\usepackage[T1]{fontenc} 	% pour caractères accentués
\usepackage[utf8]{inputenc}  % encodage utf8
\usepackage[french]{babel}	% langue : français
\usepackage{fourier}			% caractères plus lisibles
\usepackage[dvipsnames]{xcolor} % couleurs
\usepackage{fancyhdr}		% réglage header footer
\usepackage{needspace}		% empêcher sauts de page mal placés
\usepackage{graphicx}		% pour inclure des graphiques
\usepackage{enumitem,cprotect}		% personnalise les listes d'items (nécessaire pour ol, al ...)
\usepackage{hyperref}		% Liens hypertexte
\usepackage{pstricks,pst-all,pst-node,pstricks-add,pst-math,pst-plot,pst-tree,pst-eucl} % pstricks
\usepackage[a4paper,includeheadfoot,top=2cm,left=3cm, bottom=2cm,right=3cm]{geometry} % marges etc.
\usepackage{comment}			% commentaires multilignes
\usepackage{amsmath,environ} % maths (matrices, etc.)
\usepackage{amssymb,makeidx}
\usepackage{bm}				% bold maths
\usepackage{tabularx}		% tableaux
\usepackage{colortbl}		% tableaux en couleur
\usepackage{fontawesome}		% Fontawesome
\usepackage{environ}			% environment with command
\usepackage{fp}				% calculs pour ps-tricks
\usepackage{multido}			% pour ps tricks
\usepackage[np]{numprint}	% formattage nombre
\usepackage{tikz,tkz-tab} 			% package principal TikZ
\usepackage{pgfplots}   % axes
\usepackage{mathrsfs}    % cursives
\usepackage{calc}			% calcul taille boites
\usepackage[scaled=0.875]{helvet} % font sans serif
\usepackage{svg} % svg
\usepackage{scrextend} % local margin
\usepackage{scratch} %scratch
\usepackage{multicol} % colonnes
%\usepackage{infix-RPN,pst-func} % formule en notation polanaise inversée
\usepackage{listings}

%================================================================================================================================
%
% Réglages de base
%
%================================================================================================================================

\lstset{
language=Python,   % R code
literate=
{á}{{\'a}}1
{à}{{\`a}}1
{ã}{{\~a}}1
{é}{{\'e}}1
{è}{{\`e}}1
{ê}{{\^e}}1
{í}{{\'i}}1
{ó}{{\'o}}1
{õ}{{\~o}}1
{ú}{{\'u}}1
{ü}{{\"u}}1
{ç}{{\c{c}}}1
{~}{{ }}1
}


\definecolor{codegreen}{rgb}{0,0.6,0}
\definecolor{codegray}{rgb}{0.5,0.5,0.5}
\definecolor{codepurple}{rgb}{0.58,0,0.82}
\definecolor{backcolour}{rgb}{0.95,0.95,0.92}

\lstdefinestyle{mystyle}{
    backgroundcolor=\color{backcolour},   
    commentstyle=\color{codegreen},
    keywordstyle=\color{magenta},
    numberstyle=\tiny\color{codegray},
    stringstyle=\color{codepurple},
    basicstyle=\ttfamily\footnotesize,
    breakatwhitespace=false,         
    breaklines=true,                 
    captionpos=b,                    
    keepspaces=true,                 
    numbers=left,                    
xleftmargin=2em,
framexleftmargin=2em,            
    showspaces=false,                
    showstringspaces=false,
    showtabs=false,                  
    tabsize=2,
    upquote=true
}

\lstset{style=mystyle}


\lstset{style=mystyle}
\newcommand{\imgdir}{C:/laragon/www/newmc/assets/imgsvg/}
\newcommand{\imgsvgdir}{C:/laragon/www/newmc/assets/imgsvg/}

\definecolor{mcgris}{RGB}{220, 220, 220}% ancien~; pour compatibilité
\definecolor{mcbleu}{RGB}{52, 152, 219}
\definecolor{mcvert}{RGB}{125, 194, 70}
\definecolor{mcmauve}{RGB}{154, 0, 215}
\definecolor{mcorange}{RGB}{255, 96, 0}
\definecolor{mcturquoise}{RGB}{0, 153, 153}
\definecolor{mcrouge}{RGB}{255, 0, 0}
\definecolor{mclightvert}{RGB}{205, 234, 190}

\definecolor{gris}{RGB}{220, 220, 220}
\definecolor{bleu}{RGB}{52, 152, 219}
\definecolor{vert}{RGB}{125, 194, 70}
\definecolor{mauve}{RGB}{154, 0, 215}
\definecolor{orange}{RGB}{255, 96, 0}
\definecolor{turquoise}{RGB}{0, 153, 153}
\definecolor{rouge}{RGB}{255, 0, 0}
\definecolor{lightvert}{RGB}{205, 234, 190}
\setitemize[0]{label=\color{lightvert}  $\bullet$}

\pagestyle{fancy}
\renewcommand{\headrulewidth}{0.2pt}
\fancyhead[L]{maths-cours.fr}
\fancyhead[R]{\thepage}
\renewcommand{\footrulewidth}{0.2pt}
\fancyfoot[C]{}

\newcolumntype{C}{>{\centering\arraybackslash}X}
\newcolumntype{s}{>{\hsize=.35\hsize\arraybackslash}X}

\setlength{\parindent}{0pt}		 
\setlength{\parskip}{3mm}
\setlength{\headheight}{1cm}

\def\ebook{ebook}
\def\book{book}
\def\web{web}
\def\type{web}

\newcommand{\vect}[1]{\overrightarrow{\,\mathstrut#1\,}}

\def\Oij{$\left(\text{O}~;~\vect{\imath},~\vect{\jmath}\right)$}
\def\Oijk{$\left(\text{O}~;~\vect{\imath},~\vect{\jmath},~\vect{k}\right)$}
\def\Ouv{$\left(\text{O}~;~\vect{u},~\vect{v}\right)$}

\hypersetup{breaklinks=true, colorlinks = true, linkcolor = OliveGreen, urlcolor = OliveGreen, citecolor = OliveGreen, pdfauthor={Didier BONNEL - https://www.maths-cours.fr} } % supprime les bordures autour des liens

\renewcommand{\arg}[0]{\text{arg}}

\everymath{\displaystyle}

%================================================================================================================================
%
% Macros - Commandes
%
%================================================================================================================================

\newcommand\meta[2]{    			% Utilisé pour créer le post HTML.
	\def\titre{titre}
	\def\url{url}
	\def\arg{#1}
	\ifx\titre\arg
		\newcommand\maintitle{#2}
		\fancyhead[L]{#2}
		{\Large\sffamily \MakeUppercase{#2}}
		\vspace{1mm}\textcolor{mcvert}{\hrule}
	\fi 
	\ifx\url\arg
		\fancyfoot[L]{\href{https://www.maths-cours.fr#2}{\black \footnotesize{https://www.maths-cours.fr#2}}}
	\fi 
}


\newcommand\TitreC[1]{    		% Titre centré
     \needspace{3\baselineskip}
     \begin{center}\textbf{#1}\end{center}
}

\newcommand\newpar{    		% paragraphe
     \par
}

\newcommand\nosp {    		% commande vide (pas d'espace)
}
\newcommand{\id}[1]{} %ignore

\newcommand\boite[2]{				% Boite simple sans titre
	\vspace{5mm}
	\setlength{\fboxrule}{0.2mm}
	\setlength{\fboxsep}{5mm}	
	\fcolorbox{#1}{#1!3}{\makebox[\linewidth-2\fboxrule-2\fboxsep]{
  		\begin{minipage}[t]{\linewidth-2\fboxrule-4\fboxsep}\setlength{\parskip}{3mm}
  			 #2
  		\end{minipage}
	}}
	\vspace{5mm}
}

\newcommand\CBox[4]{				% Boites
	\vspace{5mm}
	\setlength{\fboxrule}{0.2mm}
	\setlength{\fboxsep}{5mm}
	
	\fcolorbox{#1}{#1!3}{\makebox[\linewidth-2\fboxrule-2\fboxsep]{
		\begin{minipage}[t]{1cm}\setlength{\parskip}{3mm}
	  		\textcolor{#1}{\LARGE{#2}}    
 	 	\end{minipage}  
  		\begin{minipage}[t]{\linewidth-2\fboxrule-4\fboxsep}\setlength{\parskip}{3mm}
			\raisebox{1.2mm}{\normalsize\sffamily{\textcolor{#1}{#3}}}						
  			 #4
  		\end{minipage}
	}}
	\vspace{5mm}
}

\newcommand\cadre[3]{				% Boites convertible html
	\par
	\vspace{2mm}
	\setlength{\fboxrule}{0.1mm}
	\setlength{\fboxsep}{5mm}
	\fcolorbox{#1}{white}{\makebox[\linewidth-2\fboxrule-2\fboxsep]{
  		\begin{minipage}[t]{\linewidth-2\fboxrule-4\fboxsep}\setlength{\parskip}{3mm}
			\raisebox{-2.5mm}{\sffamily \small{\textcolor{#1}{\MakeUppercase{#2}}}}		
			\par		
  			 #3
 	 		\end{minipage}
	}}
		\vspace{2mm}
	\par
}

\newcommand\bloc[3]{				% Boites convertible html sans bordure
     \needspace{2\baselineskip}
     {\sffamily \small{\textcolor{#1}{\MakeUppercase{#2}}}}    
		\par		
  			 #3
		\par
}

\newcommand\CHelp[1]{
     \CBox{Plum}{\faInfoCircle}{À RETENIR}{#1}
}

\newcommand\CUp[1]{
     \CBox{NavyBlue}{\faThumbsOUp}{EN PRATIQUE}{#1}
}

\newcommand\CInfo[1]{
     \CBox{Sepia}{\faArrowCircleRight}{REMARQUE}{#1}
}

\newcommand\CRedac[1]{
     \CBox{PineGreen}{\faEdit}{BIEN R\'EDIGER}{#1}
}

\newcommand\CError[1]{
     \CBox{Red}{\faExclamationTriangle}{ATTENTION}{#1}
}

\newcommand\TitreExo[2]{
\needspace{4\baselineskip}
 {\sffamily\large EXERCICE #1\ (\emph{#2 points})}
\vspace{5mm}
}

\newcommand\img[2]{
          \includegraphics[width=#2\paperwidth]{\imgdir#1}
}

\newcommand\imgsvg[2]{
       \begin{center}   \includegraphics[width=#2\paperwidth]{\imgsvgdir#1} \end{center}
}


\newcommand\Lien[2]{
     \href{#1}{#2 \tiny \faExternalLink}
}
\newcommand\mcLien[2]{
     \href{https~://www.maths-cours.fr/#1}{#2 \tiny \faExternalLink}
}

\newcommand{\euro}{\eurologo{}}

%================================================================================================================================
%
% Macros - Environement
%
%================================================================================================================================

\newenvironment{tex}{ %
}
{%
}

\newenvironment{indente}{ %
	\setlength\parindent{10mm}
}

{
	\setlength\parindent{0mm}
}

\newenvironment{corrige}{%
     \needspace{3\baselineskip}
     \medskip
     \textbf{\textsc{Corrigé}}
     \medskip
}
{
}

\newenvironment{extern}{%
     \begin{center}
     }
     {
     \end{center}
}

\NewEnviron{code}{%
	\par
     \boite{gray}{\texttt{%
     \BODY
     }}
     \par
}

\newenvironment{vbloc}{% boite sans cadre empeche saut de page
     \begin{minipage}[t]{\linewidth}
     }
     {
     \end{minipage}
}
\NewEnviron{h2}{%
    \needspace{3\baselineskip}
    \vspace{0.6cm}
	\noindent \MakeUppercase{\sffamily \large \BODY}
	\vspace{1mm}\textcolor{mcgris}{\hrule}\vspace{0.4cm}
	\par
}{}

\NewEnviron{h3}{%
    \needspace{3\baselineskip}
	\vspace{5mm}
	\textsc{\BODY}
	\par
}

\NewEnviron{margeneg}{ %
\begin{addmargin}[-1cm]{0cm}
\BODY
\end{addmargin}
}

\NewEnviron{html}{%
}

\begin{document}
\meta{url}{/exercices/probabilites-bac-es-l-asie-2018/}
\meta{pid}{9373}
\meta{titre}{Probabilités – Bac ES/L Asie 2018}
\meta{type}{exercices}
%
\begin{h2}Exercice 2 (4 points)\end{h2}
\par
\textbf{Commun  à tous les candidats}
\bigbreak
Un navigateur s'entraîne régulièrement dans le but de battre le record du monde de
traversée de l'Atlantique à la voile.
\medbreak
\emph{Dans cet exercice, les résultats seront arrondis au millième si nécessaire.}
\medbreak
\emph{Pour tous événements $A$ et $B$, on note $\overline{A}$ l'événement contraire de $A$, $P(A)$ la probabilité de $A$
et si $B$ est de probabilité non nulle, $P_B(A)$ la probabilité de $A$ sachant $B$.}
\bigbreak
\TitreC{Partie A}
\medbreak
Le navigateur décide de modéliser la durée de sa traversée en jour par une loi normale de
paramètres $\mu = 7$ et $\sigma= 1$.
\medbreak
\begin{enumerate}
     \item Quelle est la probabilité que le navigateur termine sa course entre $5$ et $8$ jours après le départ~?
     \item Dans sa catégorie de voilier, le record du monde actuel est de $5$ jours.
     Quelle est la probabilité que le navigateur batte le record du monde~?
\end{enumerate}
\bigbreak
\TitreC{Partie B}
\medbreak
Une entreprise nommée \og Régate \fg, s'intéresse aux résultats de ce navigateur.
\par
La probabilité qu'il réalise la traversée en moins de 6 jours est de 0,16.
\par
Si le navigateur réalise la traversée en moins de 6 jours, l'entreprise le sponsorise avec une probabilité de 0,95.
\par
Sinon, l'entreprise hésite et le sponsorise avec une probabilité de 0,50.
\par
On note~:
\begin{itemize}
     \item $M$ l'événement \og la traversée est réalisée par le navigateur en moins de 6 jours \fg~;
     \item $F$ l'événement \og l'entreprise sponsorise le navigateur \fg.
\end{itemize}
\begin{enumerate}
     \item Représenter cette situation à l'aide d'un arbre pondéré.
     \item  Montrer que la probabilité que l'entreprise ne sponsorise pas le navigateur à la
     prochaine course est $0,428$.
     \item  L'entreprise a finalement choisi de ne pas financer le navigateur.
     \par
     Calculer la probabilité que le navigateur ait tout de même réalisé la traversée en moins
     de $6$ jours.
\end{enumerate}
\bigbreak
\TitreC{Partie C}
\medbreak
L'entreprise \og Régate \fg sponsorise plusieurs catégories de sportifs dans le monde nautique.
\par
Ces derniers doivent afficher le slogan \og Avec Régate. j'ai 97\,\% de chance d'être sur le podium! \fg.
\par
L'étude des résultats sportifs de l'année a révélé que, parmi $280$ sportifs de chez \og Régate \fg, $263$ sont montés sur le podium. Que penser du slogan~?
\medbreak

\end{document}
µ
\documentclass[a4paper]{article}

%================================================================================================================================
%
% Packages
%
%================================================================================================================================

\usepackage[T1]{fontenc} 	% pour caractères accentués
\usepackage[utf8]{inputenc}  % encodage utf8
\usepackage[french]{babel}	% langue : français
\usepackage{fourier}			% caractères plus lisibles
\usepackage[dvipsnames]{xcolor} % couleurs
\usepackage{fancyhdr}		% réglage header footer
\usepackage{needspace}		% empêcher sauts de page mal placés
\usepackage{graphicx}		% pour inclure des graphiques
\usepackage{enumitem,cprotect}		% personnalise les listes d'items (nécessaire pour ol, al ...)
\usepackage{hyperref}		% Liens hypertexte
\usepackage{pstricks,pst-all,pst-node,pstricks-add,pst-math,pst-plot,pst-tree,pst-eucl} % pstricks
\usepackage[a4paper,includeheadfoot,top=2cm,left=3cm, bottom=2cm,right=3cm]{geometry} % marges etc.
\usepackage{comment}			% commentaires multilignes
\usepackage{amsmath,environ} % maths (matrices, etc.)
\usepackage{amssymb,makeidx}
\usepackage{bm}				% bold maths
\usepackage{tabularx}		% tableaux
\usepackage{colortbl}		% tableaux en couleur
\usepackage{fontawesome}		% Fontawesome
\usepackage{environ}			% environment with command
\usepackage{fp}				% calculs pour ps-tricks
\usepackage{multido}			% pour ps tricks
\usepackage[np]{numprint}	% formattage nombre
\usepackage{tikz,tkz-tab} 			% package principal TikZ
\usepackage{pgfplots}   % axes
\usepackage{mathrsfs}    % cursives
\usepackage{calc}			% calcul taille boites
\usepackage[scaled=0.875]{helvet} % font sans serif
\usepackage{svg} % svg
\usepackage{scrextend} % local margin
\usepackage{scratch} %scratch
\usepackage{multicol} % colonnes
%\usepackage{infix-RPN,pst-func} % formule en notation polanaise inversée
\usepackage{listings}

%================================================================================================================================
%
% Réglages de base
%
%================================================================================================================================

\lstset{
language=Python,   % R code
literate=
{á}{{\'a}}1
{à}{{\`a}}1
{ã}{{\~a}}1
{é}{{\'e}}1
{è}{{\`e}}1
{ê}{{\^e}}1
{í}{{\'i}}1
{ó}{{\'o}}1
{õ}{{\~o}}1
{ú}{{\'u}}1
{ü}{{\"u}}1
{ç}{{\c{c}}}1
{~}{{ }}1
}


\definecolor{codegreen}{rgb}{0,0.6,0}
\definecolor{codegray}{rgb}{0.5,0.5,0.5}
\definecolor{codepurple}{rgb}{0.58,0,0.82}
\definecolor{backcolour}{rgb}{0.95,0.95,0.92}

\lstdefinestyle{mystyle}{
    backgroundcolor=\color{backcolour},   
    commentstyle=\color{codegreen},
    keywordstyle=\color{magenta},
    numberstyle=\tiny\color{codegray},
    stringstyle=\color{codepurple},
    basicstyle=\ttfamily\footnotesize,
    breakatwhitespace=false,         
    breaklines=true,                 
    captionpos=b,                    
    keepspaces=true,                 
    numbers=left,                    
xleftmargin=2em,
framexleftmargin=2em,            
    showspaces=false,                
    showstringspaces=false,
    showtabs=false,                  
    tabsize=2,
    upquote=true
}

\lstset{style=mystyle}


\lstset{style=mystyle}
\newcommand{\imgdir}{C:/laragon/www/newmc/assets/imgsvg/}
\newcommand{\imgsvgdir}{C:/laragon/www/newmc/assets/imgsvg/}

\definecolor{mcgris}{RGB}{220, 220, 220}% ancien~; pour compatibilité
\definecolor{mcbleu}{RGB}{52, 152, 219}
\definecolor{mcvert}{RGB}{125, 194, 70}
\definecolor{mcmauve}{RGB}{154, 0, 215}
\definecolor{mcorange}{RGB}{255, 96, 0}
\definecolor{mcturquoise}{RGB}{0, 153, 153}
\definecolor{mcrouge}{RGB}{255, 0, 0}
\definecolor{mclightvert}{RGB}{205, 234, 190}

\definecolor{gris}{RGB}{220, 220, 220}
\definecolor{bleu}{RGB}{52, 152, 219}
\definecolor{vert}{RGB}{125, 194, 70}
\definecolor{mauve}{RGB}{154, 0, 215}
\definecolor{orange}{RGB}{255, 96, 0}
\definecolor{turquoise}{RGB}{0, 153, 153}
\definecolor{rouge}{RGB}{255, 0, 0}
\definecolor{lightvert}{RGB}{205, 234, 190}
\setitemize[0]{label=\color{lightvert}  $\bullet$}

\pagestyle{fancy}
\renewcommand{\headrulewidth}{0.2pt}
\fancyhead[L]{maths-cours.fr}
\fancyhead[R]{\thepage}
\renewcommand{\footrulewidth}{0.2pt}
\fancyfoot[C]{}

\newcolumntype{C}{>{\centering\arraybackslash}X}
\newcolumntype{s}{>{\hsize=.35\hsize\arraybackslash}X}

\setlength{\parindent}{0pt}		 
\setlength{\parskip}{3mm}
\setlength{\headheight}{1cm}

\def\ebook{ebook}
\def\book{book}
\def\web{web}
\def\type{web}

\newcommand{\vect}[1]{\overrightarrow{\,\mathstrut#1\,}}

\def\Oij{$\left(\text{O}~;~\vect{\imath},~\vect{\jmath}\right)$}
\def\Oijk{$\left(\text{O}~;~\vect{\imath},~\vect{\jmath},~\vect{k}\right)$}
\def\Ouv{$\left(\text{O}~;~\vect{u},~\vect{v}\right)$}

\hypersetup{breaklinks=true, colorlinks = true, linkcolor = OliveGreen, urlcolor = OliveGreen, citecolor = OliveGreen, pdfauthor={Didier BONNEL - https://www.maths-cours.fr} } % supprime les bordures autour des liens

\renewcommand{\arg}[0]{\text{arg}}

\everymath{\displaystyle}

%================================================================================================================================
%
% Macros - Commandes
%
%================================================================================================================================

\newcommand\meta[2]{    			% Utilisé pour créer le post HTML.
	\def\titre{titre}
	\def\url{url}
	\def\arg{#1}
	\ifx\titre\arg
		\newcommand\maintitle{#2}
		\fancyhead[L]{#2}
		{\Large\sffamily \MakeUppercase{#2}}
		\vspace{1mm}\textcolor{mcvert}{\hrule}
	\fi 
	\ifx\url\arg
		\fancyfoot[L]{\href{https://www.maths-cours.fr#2}{\black \footnotesize{https://www.maths-cours.fr#2}}}
	\fi 
}


\newcommand\TitreC[1]{    		% Titre centré
     \needspace{3\baselineskip}
     \begin{center}\textbf{#1}\end{center}
}

\newcommand\newpar{    		% paragraphe
     \par
}

\newcommand\nosp {    		% commande vide (pas d'espace)
}
\newcommand{\id}[1]{} %ignore

\newcommand\boite[2]{				% Boite simple sans titre
	\vspace{5mm}
	\setlength{\fboxrule}{0.2mm}
	\setlength{\fboxsep}{5mm}	
	\fcolorbox{#1}{#1!3}{\makebox[\linewidth-2\fboxrule-2\fboxsep]{
  		\begin{minipage}[t]{\linewidth-2\fboxrule-4\fboxsep}\setlength{\parskip}{3mm}
  			 #2
  		\end{minipage}
	}}
	\vspace{5mm}
}

\newcommand\CBox[4]{				% Boites
	\vspace{5mm}
	\setlength{\fboxrule}{0.2mm}
	\setlength{\fboxsep}{5mm}
	
	\fcolorbox{#1}{#1!3}{\makebox[\linewidth-2\fboxrule-2\fboxsep]{
		\begin{minipage}[t]{1cm}\setlength{\parskip}{3mm}
	  		\textcolor{#1}{\LARGE{#2}}    
 	 	\end{minipage}  
  		\begin{minipage}[t]{\linewidth-2\fboxrule-4\fboxsep}\setlength{\parskip}{3mm}
			\raisebox{1.2mm}{\normalsize\sffamily{\textcolor{#1}{#3}}}						
  			 #4
  		\end{minipage}
	}}
	\vspace{5mm}
}

\newcommand\cadre[3]{				% Boites convertible html
	\par
	\vspace{2mm}
	\setlength{\fboxrule}{0.1mm}
	\setlength{\fboxsep}{5mm}
	\fcolorbox{#1}{white}{\makebox[\linewidth-2\fboxrule-2\fboxsep]{
  		\begin{minipage}[t]{\linewidth-2\fboxrule-4\fboxsep}\setlength{\parskip}{3mm}
			\raisebox{-2.5mm}{\sffamily \small{\textcolor{#1}{\MakeUppercase{#2}}}}		
			\par		
  			 #3
 	 		\end{minipage}
	}}
		\vspace{2mm}
	\par
}

\newcommand\bloc[3]{				% Boites convertible html sans bordure
     \needspace{2\baselineskip}
     {\sffamily \small{\textcolor{#1}{\MakeUppercase{#2}}}}    
		\par		
  			 #3
		\par
}

\newcommand\CHelp[1]{
     \CBox{Plum}{\faInfoCircle}{À RETENIR}{#1}
}

\newcommand\CUp[1]{
     \CBox{NavyBlue}{\faThumbsOUp}{EN PRATIQUE}{#1}
}

\newcommand\CInfo[1]{
     \CBox{Sepia}{\faArrowCircleRight}{REMARQUE}{#1}
}

\newcommand\CRedac[1]{
     \CBox{PineGreen}{\faEdit}{BIEN R\'EDIGER}{#1}
}

\newcommand\CError[1]{
     \CBox{Red}{\faExclamationTriangle}{ATTENTION}{#1}
}

\newcommand\TitreExo[2]{
\needspace{4\baselineskip}
 {\sffamily\large EXERCICE #1\ (\emph{#2 points})}
\vspace{5mm}
}

\newcommand\img[2]{
          \includegraphics[width=#2\paperwidth]{\imgdir#1}
}

\newcommand\imgsvg[2]{
       \begin{center}   \includegraphics[width=#2\paperwidth]{\imgsvgdir#1} \end{center}
}


\newcommand\Lien[2]{
     \href{#1}{#2 \tiny \faExternalLink}
}
\newcommand\mcLien[2]{
     \href{https~://www.maths-cours.fr/#1}{#2 \tiny \faExternalLink}
}

\newcommand{\euro}{\eurologo{}}

%================================================================================================================================
%
% Macros - Environement
%
%================================================================================================================================

\newenvironment{tex}{ %
}
{%
}

\newenvironment{indente}{ %
	\setlength\parindent{10mm}
}

{
	\setlength\parindent{0mm}
}

\newenvironment{corrige}{%
     \needspace{3\baselineskip}
     \medskip
     \textbf{\textsc{Corrigé}}
     \medskip
}
{
}

\newenvironment{extern}{%
     \begin{center}
     }
     {
     \end{center}
}

\NewEnviron{code}{%
	\par
     \boite{gray}{\texttt{%
     \BODY
     }}
     \par
}

\newenvironment{vbloc}{% boite sans cadre empeche saut de page
     \begin{minipage}[t]{\linewidth}
     }
     {
     \end{minipage}
}
\NewEnviron{h2}{%
    \needspace{3\baselineskip}
    \vspace{0.6cm}
	\noindent \MakeUppercase{\sffamily \large \BODY}
	\vspace{1mm}\textcolor{mcgris}{\hrule}\vspace{0.4cm}
	\par
}{}

\NewEnviron{h3}{%
    \needspace{3\baselineskip}
	\vspace{5mm}
	\textsc{\BODY}
	\par
}

\NewEnviron{margeneg}{ %
\begin{addmargin}[-1cm]{0cm}
\BODY
\end{addmargin}
}

\NewEnviron{html}{%
}

\begin{document}
\meta{url}{/exercices/suites-bac-es-l-asie-2018/}
\meta{pid}{9375}
\meta{titre}{Suites – Bac ES/L Asie 2018}
\meta{type}{exercices}
%
\begin{h2}Exercice 3 (5 points)\end{h2}
\textbf{Candidats de la série ES n'ayant pas choisi la spécialité \og mathématiques \fg{} et candidats de la série L}
\bigbreak
Un pays compte $300$ loups en 2017. On estime que la population des loups croit
naturellement au rythme de 12\,\% par an. Pour réguler la population des loups. le
gouvernement autorise les chasseurs à tuer un quota de $18$ loups par an.
\par
On modélise la population par une suite $\left(u_n\right)$ le terme $u_n$ représentant le nombre de loups de ce pays en $2017+n$.
\medbreak
\begin{enumerate}
     \item
     \begin{enumerate}[label=\alph*.]
          \item Avec ce modèle. vérifier que le nombre de loups de ce pays en 2018 sera de $318$.
          \item  Justifier que, pour tout entier $n \in \mathbb{N}$, \:  $u_{n+1} = 1,12u_n - 18$.
     \end{enumerate}
     \item Recopier et compléter l'algorithme suivant pour qu'il détermine au bout de combien
     d'années la population de loups aura doublé.
     \begin{center}
          \begin{extern}%width="260" alt="Algorithme Bac ES/L Asie 2018"
               \begin{tabularx}{0.4\linewidth}{|X|}\hline
                    $N \gets 0$\\
                    $U \gets 300$\\
                    Tant que \ldots faire\\
                    \hspace{1cm}$U \gets \ldots$\\
                    \hspace{1cm}$N \gets \ldots$\\
                    Fin Tant que\\ \hline
               \end{tabularx}
          \end{extern}
     \end{center}
     \item On définit la suite $\left(v_n\right)$ par~: $v_n = u_n - 150$ pour tout $n \in \mathbb{N}$.
     \begin{enumerate}[label=\alph*.]
          \item Montrer que la suite $\left(v_n\right)$ est une suite géométrique de raison $1,12$.
          \par
          Préciser son terme initial.
          \item Exprimer, pour tout $n \in \mathbb{N}$,  $v_n$ en fonction de $n$.
          \par
          En déduire $u_n$ en fonction de $n$.
          \item Quelle est la limite de la suite $\left(u_n\right)$~? Justifier.
          \par
          Que peut-on en déduire~?
     \end{enumerate}
     \item
     \begin{enumerate}[label=\alph*.]
          \item Résoudre dans l'ensemble des entiers naturels l'inéquation~:
          \[150 + 1,12^n \times 150 > 600.\]
          \item Interpréter le résultat précédent dans le contexte de l'énoncé.
     \end{enumerate}
     \item En 2023. avec ce modèle, la population de loups est estimée à $446$ loups et le rythme de croissance annuel de la population reste identique. Dans ce cas, une nouvelle
     décision sera prise par le gouvernement~: afin de gérer le nombre de loups dans le
     pays, il autorisera les chasseurs à tuer un quota de $35$ loups par an.
     \par
     En quelle année la population de loups dépassera-t-elle $600$ loups~?
     \par
     Toute trace de recherche sera valorisée dans cette question.
\end{enumerate}

\end{document}
µ
\documentclass[a4paper]{article}

%================================================================================================================================
%
% Packages
%
%================================================================================================================================

\usepackage[T1]{fontenc} 	% pour caractères accentués
\usepackage[utf8]{inputenc}  % encodage utf8
\usepackage[french]{babel}	% langue : français
\usepackage{fourier}			% caractères plus lisibles
\usepackage[dvipsnames]{xcolor} % couleurs
\usepackage{fancyhdr}		% réglage header footer
\usepackage{needspace}		% empêcher sauts de page mal placés
\usepackage{graphicx}		% pour inclure des graphiques
\usepackage{enumitem,cprotect}		% personnalise les listes d'items (nécessaire pour ol, al ...)
\usepackage{hyperref}		% Liens hypertexte
\usepackage{pstricks,pst-all,pst-node,pstricks-add,pst-math,pst-plot,pst-tree,pst-eucl} % pstricks
\usepackage[a4paper,includeheadfoot,top=2cm,left=3cm, bottom=2cm,right=3cm]{geometry} % marges etc.
\usepackage{comment}			% commentaires multilignes
\usepackage{amsmath,environ} % maths (matrices, etc.)
\usepackage{amssymb,makeidx}
\usepackage{bm}				% bold maths
\usepackage{tabularx}		% tableaux
\usepackage{colortbl}		% tableaux en couleur
\usepackage{fontawesome}		% Fontawesome
\usepackage{environ}			% environment with command
\usepackage{fp}				% calculs pour ps-tricks
\usepackage{multido}			% pour ps tricks
\usepackage[np]{numprint}	% formattage nombre
\usepackage{tikz,tkz-tab} 			% package principal TikZ
\usepackage{pgfplots}   % axes
\usepackage{mathrsfs}    % cursives
\usepackage{calc}			% calcul taille boites
\usepackage[scaled=0.875]{helvet} % font sans serif
\usepackage{svg} % svg
\usepackage{scrextend} % local margin
\usepackage{scratch} %scratch
\usepackage{multicol} % colonnes
%\usepackage{infix-RPN,pst-func} % formule en notation polanaise inversée
\usepackage{listings}

%================================================================================================================================
%
% Réglages de base
%
%================================================================================================================================

\lstset{
language=Python,   % R code
literate=
{á}{{\'a}}1
{à}{{\`a}}1
{ã}{{\~a}}1
{é}{{\'e}}1
{è}{{\`e}}1
{ê}{{\^e}}1
{í}{{\'i}}1
{ó}{{\'o}}1
{õ}{{\~o}}1
{ú}{{\'u}}1
{ü}{{\"u}}1
{ç}{{\c{c}}}1
{~}{{ }}1
}


\definecolor{codegreen}{rgb}{0,0.6,0}
\definecolor{codegray}{rgb}{0.5,0.5,0.5}
\definecolor{codepurple}{rgb}{0.58,0,0.82}
\definecolor{backcolour}{rgb}{0.95,0.95,0.92}

\lstdefinestyle{mystyle}{
    backgroundcolor=\color{backcolour},   
    commentstyle=\color{codegreen},
    keywordstyle=\color{magenta},
    numberstyle=\tiny\color{codegray},
    stringstyle=\color{codepurple},
    basicstyle=\ttfamily\footnotesize,
    breakatwhitespace=false,         
    breaklines=true,                 
    captionpos=b,                    
    keepspaces=true,                 
    numbers=left,                    
xleftmargin=2em,
framexleftmargin=2em,            
    showspaces=false,                
    showstringspaces=false,
    showtabs=false,                  
    tabsize=2,
    upquote=true
}

\lstset{style=mystyle}


\lstset{style=mystyle}
\newcommand{\imgdir}{C:/laragon/www/newmc/assets/imgsvg/}
\newcommand{\imgsvgdir}{C:/laragon/www/newmc/assets/imgsvg/}

\definecolor{mcgris}{RGB}{220, 220, 220}% ancien~; pour compatibilité
\definecolor{mcbleu}{RGB}{52, 152, 219}
\definecolor{mcvert}{RGB}{125, 194, 70}
\definecolor{mcmauve}{RGB}{154, 0, 215}
\definecolor{mcorange}{RGB}{255, 96, 0}
\definecolor{mcturquoise}{RGB}{0, 153, 153}
\definecolor{mcrouge}{RGB}{255, 0, 0}
\definecolor{mclightvert}{RGB}{205, 234, 190}

\definecolor{gris}{RGB}{220, 220, 220}
\definecolor{bleu}{RGB}{52, 152, 219}
\definecolor{vert}{RGB}{125, 194, 70}
\definecolor{mauve}{RGB}{154, 0, 215}
\definecolor{orange}{RGB}{255, 96, 0}
\definecolor{turquoise}{RGB}{0, 153, 153}
\definecolor{rouge}{RGB}{255, 0, 0}
\definecolor{lightvert}{RGB}{205, 234, 190}
\setitemize[0]{label=\color{lightvert}  $\bullet$}

\pagestyle{fancy}
\renewcommand{\headrulewidth}{0.2pt}
\fancyhead[L]{maths-cours.fr}
\fancyhead[R]{\thepage}
\renewcommand{\footrulewidth}{0.2pt}
\fancyfoot[C]{}

\newcolumntype{C}{>{\centering\arraybackslash}X}
\newcolumntype{s}{>{\hsize=.35\hsize\arraybackslash}X}

\setlength{\parindent}{0pt}		 
\setlength{\parskip}{3mm}
\setlength{\headheight}{1cm}

\def\ebook{ebook}
\def\book{book}
\def\web{web}
\def\type{web}

\newcommand{\vect}[1]{\overrightarrow{\,\mathstrut#1\,}}

\def\Oij{$\left(\text{O}~;~\vect{\imath},~\vect{\jmath}\right)$}
\def\Oijk{$\left(\text{O}~;~\vect{\imath},~\vect{\jmath},~\vect{k}\right)$}
\def\Ouv{$\left(\text{O}~;~\vect{u},~\vect{v}\right)$}

\hypersetup{breaklinks=true, colorlinks = true, linkcolor = OliveGreen, urlcolor = OliveGreen, citecolor = OliveGreen, pdfauthor={Didier BONNEL - https://www.maths-cours.fr} } % supprime les bordures autour des liens

\renewcommand{\arg}[0]{\text{arg}}

\everymath{\displaystyle}

%================================================================================================================================
%
% Macros - Commandes
%
%================================================================================================================================

\newcommand\meta[2]{    			% Utilisé pour créer le post HTML.
	\def\titre{titre}
	\def\url{url}
	\def\arg{#1}
	\ifx\titre\arg
		\newcommand\maintitle{#2}
		\fancyhead[L]{#2}
		{\Large\sffamily \MakeUppercase{#2}}
		\vspace{1mm}\textcolor{mcvert}{\hrule}
	\fi 
	\ifx\url\arg
		\fancyfoot[L]{\href{https://www.maths-cours.fr#2}{\black \footnotesize{https://www.maths-cours.fr#2}}}
	\fi 
}


\newcommand\TitreC[1]{    		% Titre centré
     \needspace{3\baselineskip}
     \begin{center}\textbf{#1}\end{center}
}

\newcommand\newpar{    		% paragraphe
     \par
}

\newcommand\nosp {    		% commande vide (pas d'espace)
}
\newcommand{\id}[1]{} %ignore

\newcommand\boite[2]{				% Boite simple sans titre
	\vspace{5mm}
	\setlength{\fboxrule}{0.2mm}
	\setlength{\fboxsep}{5mm}	
	\fcolorbox{#1}{#1!3}{\makebox[\linewidth-2\fboxrule-2\fboxsep]{
  		\begin{minipage}[t]{\linewidth-2\fboxrule-4\fboxsep}\setlength{\parskip}{3mm}
  			 #2
  		\end{minipage}
	}}
	\vspace{5mm}
}

\newcommand\CBox[4]{				% Boites
	\vspace{5mm}
	\setlength{\fboxrule}{0.2mm}
	\setlength{\fboxsep}{5mm}
	
	\fcolorbox{#1}{#1!3}{\makebox[\linewidth-2\fboxrule-2\fboxsep]{
		\begin{minipage}[t]{1cm}\setlength{\parskip}{3mm}
	  		\textcolor{#1}{\LARGE{#2}}    
 	 	\end{minipage}  
  		\begin{minipage}[t]{\linewidth-2\fboxrule-4\fboxsep}\setlength{\parskip}{3mm}
			\raisebox{1.2mm}{\normalsize\sffamily{\textcolor{#1}{#3}}}						
  			 #4
  		\end{minipage}
	}}
	\vspace{5mm}
}

\newcommand\cadre[3]{				% Boites convertible html
	\par
	\vspace{2mm}
	\setlength{\fboxrule}{0.1mm}
	\setlength{\fboxsep}{5mm}
	\fcolorbox{#1}{white}{\makebox[\linewidth-2\fboxrule-2\fboxsep]{
  		\begin{minipage}[t]{\linewidth-2\fboxrule-4\fboxsep}\setlength{\parskip}{3mm}
			\raisebox{-2.5mm}{\sffamily \small{\textcolor{#1}{\MakeUppercase{#2}}}}		
			\par		
  			 #3
 	 		\end{minipage}
	}}
		\vspace{2mm}
	\par
}

\newcommand\bloc[3]{				% Boites convertible html sans bordure
     \needspace{2\baselineskip}
     {\sffamily \small{\textcolor{#1}{\MakeUppercase{#2}}}}    
		\par		
  			 #3
		\par
}

\newcommand\CHelp[1]{
     \CBox{Plum}{\faInfoCircle}{À RETENIR}{#1}
}

\newcommand\CUp[1]{
     \CBox{NavyBlue}{\faThumbsOUp}{EN PRATIQUE}{#1}
}

\newcommand\CInfo[1]{
     \CBox{Sepia}{\faArrowCircleRight}{REMARQUE}{#1}
}

\newcommand\CRedac[1]{
     \CBox{PineGreen}{\faEdit}{BIEN R\'EDIGER}{#1}
}

\newcommand\CError[1]{
     \CBox{Red}{\faExclamationTriangle}{ATTENTION}{#1}
}

\newcommand\TitreExo[2]{
\needspace{4\baselineskip}
 {\sffamily\large EXERCICE #1\ (\emph{#2 points})}
\vspace{5mm}
}

\newcommand\img[2]{
          \includegraphics[width=#2\paperwidth]{\imgdir#1}
}

\newcommand\imgsvg[2]{
       \begin{center}   \includegraphics[width=#2\paperwidth]{\imgsvgdir#1} \end{center}
}


\newcommand\Lien[2]{
     \href{#1}{#2 \tiny \faExternalLink}
}
\newcommand\mcLien[2]{
     \href{https~://www.maths-cours.fr/#1}{#2 \tiny \faExternalLink}
}

\newcommand{\euro}{\eurologo{}}

%================================================================================================================================
%
% Macros - Environement
%
%================================================================================================================================

\newenvironment{tex}{ %
}
{%
}

\newenvironment{indente}{ %
	\setlength\parindent{10mm}
}

{
	\setlength\parindent{0mm}
}

\newenvironment{corrige}{%
     \needspace{3\baselineskip}
     \medskip
     \textbf{\textsc{Corrigé}}
     \medskip
}
{
}

\newenvironment{extern}{%
     \begin{center}
     }
     {
     \end{center}
}

\NewEnviron{code}{%
	\par
     \boite{gray}{\texttt{%
     \BODY
     }}
     \par
}

\newenvironment{vbloc}{% boite sans cadre empeche saut de page
     \begin{minipage}[t]{\linewidth}
     }
     {
     \end{minipage}
}
\NewEnviron{h2}{%
    \needspace{3\baselineskip}
    \vspace{0.6cm}
	\noindent \MakeUppercase{\sffamily \large \BODY}
	\vspace{1mm}\textcolor{mcgris}{\hrule}\vspace{0.4cm}
	\par
}{}

\NewEnviron{h3}{%
    \needspace{3\baselineskip}
	\vspace{5mm}
	\textsc{\BODY}
	\par
}

\NewEnviron{margeneg}{ %
\begin{addmargin}[-1cm]{0cm}
\BODY
\end{addmargin}
}

\NewEnviron{html}{%
}

\begin{document}
\meta{url}{/exercices/matrices-de-transition-bac-es-asie-2018-spe/}
\meta{pid}{9377}
\meta{titre}{Matrices de transition – Bac ES Asie 2018 (spé)}
\meta{type}{exercices}
%
\begin{h2}Exercice 3 (5 points)\end{h2}
\textbf{Candidats de la série ES ayant choisi la spécialité \og mathématiques \fg{}}
\bigbreak
Pour la nouvelle année, Lisa prend la bonne résolution d'aller au travail tous les matins à
vélo. Le premier jour, très motivée, Lisa se rend au travail à vélo. Par la suite, elle se rend toujours au travail à vélo ou en voiture.
\par
Elle se rend compte que~:
\begin{indent}
     \begin{itemize}
          \item si elle a pris son vélo un jour, cela renforce sa motivation et elle reprend le vélo le lendemain avec une probabilité de $0,7$~;
          \item si elle a pris sa voiture un jour, la probabilité qu'elle reprenne la voiture le lendemain est de $0,5$.
     \end{itemize}
\end{indent}
Cette situation peut être modélisée par un graphe probabiliste de sommets A et B où~:
\begin{indent}
     \begin{itemize}
          \item $A$ est l'événement \og Lisa prend le vélo \fg{}~;
          \item B est l'événement \og Lisa prend la voiture \fg.
     \end{itemize}
\end{indent}
On note, pour tout entier naturel $n$ non nul~:
\begin{indent}
     \begin{itemize}
          \item $a_n$ la probabilité que Lisa aille au travail à vélo le jour $n$~;
          \item $b_n$ la probabilité que Lisa aille au travail en voiture le jour $n$.
     \end{itemize}
\end{indent}
\medbreak
\begin{enumerate}
     \item
     \begin{enumerate}[label=\alph*.]
          \item Traduire les données par un graphe probabiliste.
          \item En déduire la matrice de transition $M$.
     \end{enumerate}
     \item
     \begin{enumerate}[label=\alph*.]
          \item Donner les valeurs de $a_1$ et $b_1$ correspondant à l'état initial.
          \item Calculer la probabilité arrondie au centième que Lisa prenne le vélo le 8$^{e}$ jour.
     \end{enumerate}
     \item Déterminer l'état stable du graphe puis interpréter le résultat obtenu.
     \item
     \begin{enumerate}[label=\alph*.]
          \item Montrer que, pour tout nombre entier naturel $n$ non nul~: $a_{n+1} = 0,7a_n + 0,5b_n$.
          \item En déduire que pour tout entier naturel non nul $n$~: $a_{n+1} = 0,2a_n + 0,5$.
     \end{enumerate}
     \item
     \begin{enumerate}[label=\alph*.]
          \item Recopier et compléter l'algorithme suivant permettant de déterminer le plus petit entier $n$ tel que $a_n < 0,626$.
          \begin{center}
               \begin{extern}%width="220" alt="algorithme Bac ES Asie 2018"
                    \begin{tabularx}{0.3\linewidth}{|X|}\hline
                         $N \gets 1$\\
                         $A \gets 1$\\
                         Tant que \ldots faire\\
                         \hspace{1cm}$A \gets \ldots$\\
                         \hspace{1cm}$N \gets \ldots $\\
                         Fin Tant que\\ \hline
                    \end{tabularx}
               \end{extern}
          \end{center}
          \item Quelle est la valeur de $N$ après exécution de l'algorithme~? Interpréter ce
          résultat.
     \end{enumerate}
\end{enumerate}
\medbreak

\end{document}
µ
\documentclass[a4paper]{article}

%================================================================================================================================
%
% Packages
%
%================================================================================================================================

\usepackage[T1]{fontenc} 	% pour caractères accentués
\usepackage[utf8]{inputenc}  % encodage utf8
\usepackage[french]{babel}	% langue : français
\usepackage{fourier}			% caractères plus lisibles
\usepackage[dvipsnames]{xcolor} % couleurs
\usepackage{fancyhdr}		% réglage header footer
\usepackage{needspace}		% empêcher sauts de page mal placés
\usepackage{graphicx}		% pour inclure des graphiques
\usepackage{enumitem,cprotect}		% personnalise les listes d'items (nécessaire pour ol, al ...)
\usepackage{hyperref}		% Liens hypertexte
\usepackage{pstricks,pst-all,pst-node,pstricks-add,pst-math,pst-plot,pst-tree,pst-eucl} % pstricks
\usepackage[a4paper,includeheadfoot,top=2cm,left=3cm, bottom=2cm,right=3cm]{geometry} % marges etc.
\usepackage{comment}			% commentaires multilignes
\usepackage{amsmath,environ} % maths (matrices, etc.)
\usepackage{amssymb,makeidx}
\usepackage{bm}				% bold maths
\usepackage{tabularx}		% tableaux
\usepackage{colortbl}		% tableaux en couleur
\usepackage{fontawesome}		% Fontawesome
\usepackage{environ}			% environment with command
\usepackage{fp}				% calculs pour ps-tricks
\usepackage{multido}			% pour ps tricks
\usepackage[np]{numprint}	% formattage nombre
\usepackage{tikz,tkz-tab} 			% package principal TikZ
\usepackage{pgfplots}   % axes
\usepackage{mathrsfs}    % cursives
\usepackage{calc}			% calcul taille boites
\usepackage[scaled=0.875]{helvet} % font sans serif
\usepackage{svg} % svg
\usepackage{scrextend} % local margin
\usepackage{scratch} %scratch
\usepackage{multicol} % colonnes
%\usepackage{infix-RPN,pst-func} % formule en notation polanaise inversée
\usepackage{listings}

%================================================================================================================================
%
% Réglages de base
%
%================================================================================================================================

\lstset{
language=Python,   % R code
literate=
{á}{{\'a}}1
{à}{{\`a}}1
{ã}{{\~a}}1
{é}{{\'e}}1
{è}{{\`e}}1
{ê}{{\^e}}1
{í}{{\'i}}1
{ó}{{\'o}}1
{õ}{{\~o}}1
{ú}{{\'u}}1
{ü}{{\"u}}1
{ç}{{\c{c}}}1
{~}{{ }}1
}


\definecolor{codegreen}{rgb}{0,0.6,0}
\definecolor{codegray}{rgb}{0.5,0.5,0.5}
\definecolor{codepurple}{rgb}{0.58,0,0.82}
\definecolor{backcolour}{rgb}{0.95,0.95,0.92}

\lstdefinestyle{mystyle}{
    backgroundcolor=\color{backcolour},   
    commentstyle=\color{codegreen},
    keywordstyle=\color{magenta},
    numberstyle=\tiny\color{codegray},
    stringstyle=\color{codepurple},
    basicstyle=\ttfamily\footnotesize,
    breakatwhitespace=false,         
    breaklines=true,                 
    captionpos=b,                    
    keepspaces=true,                 
    numbers=left,                    
xleftmargin=2em,
framexleftmargin=2em,            
    showspaces=false,                
    showstringspaces=false,
    showtabs=false,                  
    tabsize=2,
    upquote=true
}

\lstset{style=mystyle}


\lstset{style=mystyle}
\newcommand{\imgdir}{C:/laragon/www/newmc/assets/imgsvg/}
\newcommand{\imgsvgdir}{C:/laragon/www/newmc/assets/imgsvg/}

\definecolor{mcgris}{RGB}{220, 220, 220}% ancien~; pour compatibilité
\definecolor{mcbleu}{RGB}{52, 152, 219}
\definecolor{mcvert}{RGB}{125, 194, 70}
\definecolor{mcmauve}{RGB}{154, 0, 215}
\definecolor{mcorange}{RGB}{255, 96, 0}
\definecolor{mcturquoise}{RGB}{0, 153, 153}
\definecolor{mcrouge}{RGB}{255, 0, 0}
\definecolor{mclightvert}{RGB}{205, 234, 190}

\definecolor{gris}{RGB}{220, 220, 220}
\definecolor{bleu}{RGB}{52, 152, 219}
\definecolor{vert}{RGB}{125, 194, 70}
\definecolor{mauve}{RGB}{154, 0, 215}
\definecolor{orange}{RGB}{255, 96, 0}
\definecolor{turquoise}{RGB}{0, 153, 153}
\definecolor{rouge}{RGB}{255, 0, 0}
\definecolor{lightvert}{RGB}{205, 234, 190}
\setitemize[0]{label=\color{lightvert}  $\bullet$}

\pagestyle{fancy}
\renewcommand{\headrulewidth}{0.2pt}
\fancyhead[L]{maths-cours.fr}
\fancyhead[R]{\thepage}
\renewcommand{\footrulewidth}{0.2pt}
\fancyfoot[C]{}

\newcolumntype{C}{>{\centering\arraybackslash}X}
\newcolumntype{s}{>{\hsize=.35\hsize\arraybackslash}X}

\setlength{\parindent}{0pt}		 
\setlength{\parskip}{3mm}
\setlength{\headheight}{1cm}

\def\ebook{ebook}
\def\book{book}
\def\web{web}
\def\type{web}

\newcommand{\vect}[1]{\overrightarrow{\,\mathstrut#1\,}}

\def\Oij{$\left(\text{O}~;~\vect{\imath},~\vect{\jmath}\right)$}
\def\Oijk{$\left(\text{O}~;~\vect{\imath},~\vect{\jmath},~\vect{k}\right)$}
\def\Ouv{$\left(\text{O}~;~\vect{u},~\vect{v}\right)$}

\hypersetup{breaklinks=true, colorlinks = true, linkcolor = OliveGreen, urlcolor = OliveGreen, citecolor = OliveGreen, pdfauthor={Didier BONNEL - https://www.maths-cours.fr} } % supprime les bordures autour des liens

\renewcommand{\arg}[0]{\text{arg}}

\everymath{\displaystyle}

%================================================================================================================================
%
% Macros - Commandes
%
%================================================================================================================================

\newcommand\meta[2]{    			% Utilisé pour créer le post HTML.
	\def\titre{titre}
	\def\url{url}
	\def\arg{#1}
	\ifx\titre\arg
		\newcommand\maintitle{#2}
		\fancyhead[L]{#2}
		{\Large\sffamily \MakeUppercase{#2}}
		\vspace{1mm}\textcolor{mcvert}{\hrule}
	\fi 
	\ifx\url\arg
		\fancyfoot[L]{\href{https://www.maths-cours.fr#2}{\black \footnotesize{https://www.maths-cours.fr#2}}}
	\fi 
}


\newcommand\TitreC[1]{    		% Titre centré
     \needspace{3\baselineskip}
     \begin{center}\textbf{#1}\end{center}
}

\newcommand\newpar{    		% paragraphe
     \par
}

\newcommand\nosp {    		% commande vide (pas d'espace)
}
\newcommand{\id}[1]{} %ignore

\newcommand\boite[2]{				% Boite simple sans titre
	\vspace{5mm}
	\setlength{\fboxrule}{0.2mm}
	\setlength{\fboxsep}{5mm}	
	\fcolorbox{#1}{#1!3}{\makebox[\linewidth-2\fboxrule-2\fboxsep]{
  		\begin{minipage}[t]{\linewidth-2\fboxrule-4\fboxsep}\setlength{\parskip}{3mm}
  			 #2
  		\end{minipage}
	}}
	\vspace{5mm}
}

\newcommand\CBox[4]{				% Boites
	\vspace{5mm}
	\setlength{\fboxrule}{0.2mm}
	\setlength{\fboxsep}{5mm}
	
	\fcolorbox{#1}{#1!3}{\makebox[\linewidth-2\fboxrule-2\fboxsep]{
		\begin{minipage}[t]{1cm}\setlength{\parskip}{3mm}
	  		\textcolor{#1}{\LARGE{#2}}    
 	 	\end{minipage}  
  		\begin{minipage}[t]{\linewidth-2\fboxrule-4\fboxsep}\setlength{\parskip}{3mm}
			\raisebox{1.2mm}{\normalsize\sffamily{\textcolor{#1}{#3}}}						
  			 #4
  		\end{minipage}
	}}
	\vspace{5mm}
}

\newcommand\cadre[3]{				% Boites convertible html
	\par
	\vspace{2mm}
	\setlength{\fboxrule}{0.1mm}
	\setlength{\fboxsep}{5mm}
	\fcolorbox{#1}{white}{\makebox[\linewidth-2\fboxrule-2\fboxsep]{
  		\begin{minipage}[t]{\linewidth-2\fboxrule-4\fboxsep}\setlength{\parskip}{3mm}
			\raisebox{-2.5mm}{\sffamily \small{\textcolor{#1}{\MakeUppercase{#2}}}}		
			\par		
  			 #3
 	 		\end{minipage}
	}}
		\vspace{2mm}
	\par
}

\newcommand\bloc[3]{				% Boites convertible html sans bordure
     \needspace{2\baselineskip}
     {\sffamily \small{\textcolor{#1}{\MakeUppercase{#2}}}}    
		\par		
  			 #3
		\par
}

\newcommand\CHelp[1]{
     \CBox{Plum}{\faInfoCircle}{À RETENIR}{#1}
}

\newcommand\CUp[1]{
     \CBox{NavyBlue}{\faThumbsOUp}{EN PRATIQUE}{#1}
}

\newcommand\CInfo[1]{
     \CBox{Sepia}{\faArrowCircleRight}{REMARQUE}{#1}
}

\newcommand\CRedac[1]{
     \CBox{PineGreen}{\faEdit}{BIEN R\'EDIGER}{#1}
}

\newcommand\CError[1]{
     \CBox{Red}{\faExclamationTriangle}{ATTENTION}{#1}
}

\newcommand\TitreExo[2]{
\needspace{4\baselineskip}
 {\sffamily\large EXERCICE #1\ (\emph{#2 points})}
\vspace{5mm}
}

\newcommand\img[2]{
          \includegraphics[width=#2\paperwidth]{\imgdir#1}
}

\newcommand\imgsvg[2]{
       \begin{center}   \includegraphics[width=#2\paperwidth]{\imgsvgdir#1} \end{center}
}


\newcommand\Lien[2]{
     \href{#1}{#2 \tiny \faExternalLink}
}
\newcommand\mcLien[2]{
     \href{https~://www.maths-cours.fr/#1}{#2 \tiny \faExternalLink}
}

\newcommand{\euro}{\eurologo{}}

%================================================================================================================================
%
% Macros - Environement
%
%================================================================================================================================

\newenvironment{tex}{ %
}
{%
}

\newenvironment{indente}{ %
	\setlength\parindent{10mm}
}

{
	\setlength\parindent{0mm}
}

\newenvironment{corrige}{%
     \needspace{3\baselineskip}
     \medskip
     \textbf{\textsc{Corrigé}}
     \medskip
}
{
}

\newenvironment{extern}{%
     \begin{center}
     }
     {
     \end{center}
}

\NewEnviron{code}{%
	\par
     \boite{gray}{\texttt{%
     \BODY
     }}
     \par
}

\newenvironment{vbloc}{% boite sans cadre empeche saut de page
     \begin{minipage}[t]{\linewidth}
     }
     {
     \end{minipage}
}
\NewEnviron{h2}{%
    \needspace{3\baselineskip}
    \vspace{0.6cm}
	\noindent \MakeUppercase{\sffamily \large \BODY}
	\vspace{1mm}\textcolor{mcgris}{\hrule}\vspace{0.4cm}
	\par
}{}

\NewEnviron{h3}{%
    \needspace{3\baselineskip}
	\vspace{5mm}
	\textsc{\BODY}
	\par
}

\NewEnviron{margeneg}{ %
\begin{addmargin}[-1cm]{0cm}
\BODY
\end{addmargin}
}

\NewEnviron{html}{%
}

\begin{document}
\meta{url}{/exercices/fonctions-bac-es-l-asie-2018/}
\meta{pid}{9381}
\meta{titre}{Fonctions – Bac ES/L Asie 2018}
\meta{type}{exercices}
%
\begin{h2}Exercice 4 (6 points)\end{h2}
\par
\textbf{Commun à tous les candidats}
\bigbreak
\TitreC{Partie A}
\medbreak
On a tracé sur le graphique ci-dessous la courbe représentative $\mathscr{C}_f$ d'une fonction $f$ définie sur [0~;~25] par~:
\par
\[f(x) = (ax + b)\text{e}^{- 0,2x}\]
\par
où $a$ et $b$ sont deux nombres réels.
\par
On a représenté également sa tangente $T$ au point A(0~;~ -7). $T$ passe par le point B(2~;~14,2).
\begin{center}
     \begin{extern}%width="700" alt="Courbe représentative fonction Bac ES/L Asie 2018"
          \psset{unit=0.6cm}
          \begin{pspicture*}(-3.5,-1)(25.5,15.5)
               \psgrid[gridlabels=0pt,subgriddiv=1,gridwidth=0.2pt,gridcolor=lightgray](0,0)(-3.5,0)(25.5,15.5)
               \psaxes[linewidth=0.5pt]{->}(0,0)(-3.5,0)(25.5,15.5)
               \psplot[plotpoints=3000,linewidth=1.25pt,linecolor=blue]{0}{25}{5 x mul 7 add 2.71828 0.2 x mul exp div}
               \psplot[plotpoints=3000,linewidth=1pt]{-2}{2.5}{3.5 x mul 7 add}
               \uput[dr](0,7){A} \uput[r](2,14.2){B} \uput[dr](-2.7,1){$T$} \uput[ur](7,10.4){\blue $\mathcal{C}_f$}
               \psdots(0,7)(2,14.2)
          \end{pspicture*}
     \end{extern}
\end{center}
\medbreak
\begin{enumerate}
     \item Résoudre graphiquement l'équation $f(x) = 6$.
     \item
     \begin{enumerate}[label=\alph*.]
          \item Déterminer, par un calcul. le coefficient directeur de la droite $T$.
          \item Exprimer, pour tout $x \in [0~;~25]$,\: $f'(x)$ en fonction de $a$ et $b$.
          \item Montrer que $a$ et $b$ sont solutions du système
          \par
          \[\left\{\begin{array}{r c r}
                    a - 0,2b&=&3,6 \\
                    b&=&7
          \end{array}\right.\]
          En déduire la valeur de $a$.
     \end{enumerate}
\end{enumerate}
\bigbreak
\TitreC{Partie B}
\medbreak
\begin{enumerate}
     \item Étudier les variations de la fonction $f$ définie sur [0~;~25] par
     \par
     \[f(x) = (5x + 7)\text{e}^{- 0,2x}.\]
     \par
     Justifier.
     \item  Montrer que l'équation $f(x) = 6$ admet une unique solution $\alpha$ sur l'intervalle [0~;~25].
     \par
     Donner une valeur approchée au dixième de $\alpha$.
     \item  Un logiciel de calcul formel donne le résultat suivant.
     \par
     \begin{center}
          \begin{extern}%width="320" alt="Calcul formel Bac ES/L Asie 2018"
               \renewcommand\arraystretch{1.8}
               \begin{tabularx}{0.5\linewidth}{|X|}\hline%
                    \texttt{Dériver} $((- 25x - 160)\text{e}^{- 0,2x}$\\
                    \multicolumn{1}{|r|}{$(5x + 7)\text{e}^{- 0,2x}$}\\ \hline
               \end{tabularx}
          \end{extern}
     \end{center}
     Exploiter ce résultat pour donner la valeur exacte puis la valeur arrondie au millième de
     $\displaystyle\int_0^{25} f(x)\:\text{d}x$.
\end{enumerate}
\bigbreak
\TitreC{Partie C}
\medbreak
Un organisme de vacances souhaite ouvrir un nouveau centre avec une piscine bordée de
sable. Il dispose d'un espace rectangulaire de $25$ mètres de longueur sur $14$ mètres de
largeur et souhaite que la piscine et la \og plage \fg se partagent l'espace comme indiqué sur le schéma ci-dessous.
\par
La bordure est modélisée par la fonction $f$ étudiée dans la partie précédente.
\medbreak
\begin{enumerate}
     \item Quelle est l'aire en m$^2$ de la zone hachurée représentant la piscine~?
     \item L'organisme décide de remplacer cette piscine par une piscine rectangulaire de 25
     mètres de longueur et de même superficie.
     \par
     Quelle en sera la largeur arrondie au dixième de mètre~?
\end{enumerate}
\medbreak
\begin{center}
     \begin{extern}%width="600" alt="Plan piscine  Bac ES/L Asie 2018"
          \psset{unit=0.7cm}
          \begin{pspicture}(0,0)(25,14)
               \psgrid[gridlabels=0pt,subgriddiv=1,gridwidth=0.2pt](0,0)(25,14)
               \psplot[plotpoints=3000,linewidth=1.25pt,linecolor=blue]{0}{25}{5 x mul 7 add 2.71828 0.2 x mul exp div}
               \pscustom[fillstyle=hlines,hatchcolor=blue]{
                    \psplot[plotpoints=3000,linewidth=1.25pt,linecolor=blue]{0}{25}{5 x mul 7 add 2.71828 0.2 x mul exp div}
                    \psline(25,0)(0,0)
                    \closepath
               }
               \pscustom[fillstyle=solid,fillcolor=lightgray,opacity=0.5]{
                    \psline(0,14)(25,14)
                    \psplot[plotpoints=3000,linewidth=1.25pt,linecolor=blue]{25}{0}{5 x mul 7 add 2.71828 0.2 x mul exp div}
                    \closepath
               }
               \psplot[plotpoints=3000,linewidth=1.75pt,linecolor=blue]{0}{25}{5 x mul 7 add 2.71828 0.2 x mul exp div}
               \psline[linecolor=blue,linewidth=1.75pt](25,1)(25,0)(0,0)(0,7)
               \rput(17.5,10.5){PLAGE}\rput(7.5,4.5){PISCINE}\uput[ur](7,10.4){\blue $\mathcal{C}_f$}
          \end{pspicture}
     \end{extern}
\end{center}

\end{document}
µ
\documentclass[a4paper]{article}

%================================================================================================================================
%
% Packages
%
%================================================================================================================================

\usepackage[T1]{fontenc} 	% pour caractères accentués
\usepackage[utf8]{inputenc}  % encodage utf8
\usepackage[french]{babel}	% langue : français
\usepackage{fourier}			% caractères plus lisibles
\usepackage[dvipsnames]{xcolor} % couleurs
\usepackage{fancyhdr}		% réglage header footer
\usepackage{needspace}		% empêcher sauts de page mal placés
\usepackage{graphicx}		% pour inclure des graphiques
\usepackage{enumitem,cprotect}		% personnalise les listes d'items (nécessaire pour ol, al ...)
\usepackage{hyperref}		% Liens hypertexte
\usepackage{pstricks,pst-all,pst-node,pstricks-add,pst-math,pst-plot,pst-tree,pst-eucl} % pstricks
\usepackage[a4paper,includeheadfoot,top=2cm,left=3cm, bottom=2cm,right=3cm]{geometry} % marges etc.
\usepackage{comment}			% commentaires multilignes
\usepackage{amsmath,environ} % maths (matrices, etc.)
\usepackage{amssymb,makeidx}
\usepackage{bm}				% bold maths
\usepackage{tabularx}		% tableaux
\usepackage{colortbl}		% tableaux en couleur
\usepackage{fontawesome}		% Fontawesome
\usepackage{environ}			% environment with command
\usepackage{fp}				% calculs pour ps-tricks
\usepackage{multido}			% pour ps tricks
\usepackage[np]{numprint}	% formattage nombre
\usepackage{tikz,tkz-tab} 			% package principal TikZ
\usepackage{pgfplots}   % axes
\usepackage{mathrsfs}    % cursives
\usepackage{calc}			% calcul taille boites
\usepackage[scaled=0.875]{helvet} % font sans serif
\usepackage{svg} % svg
\usepackage{scrextend} % local margin
\usepackage{scratch} %scratch
\usepackage{multicol} % colonnes
%\usepackage{infix-RPN,pst-func} % formule en notation polanaise inversée
\usepackage{listings}

%================================================================================================================================
%
% Réglages de base
%
%================================================================================================================================

\lstset{
language=Python,   % R code
literate=
{á}{{\'a}}1
{à}{{\`a}}1
{ã}{{\~a}}1
{é}{{\'e}}1
{è}{{\`e}}1
{ê}{{\^e}}1
{í}{{\'i}}1
{ó}{{\'o}}1
{õ}{{\~o}}1
{ú}{{\'u}}1
{ü}{{\"u}}1
{ç}{{\c{c}}}1
{~}{{ }}1
}


\definecolor{codegreen}{rgb}{0,0.6,0}
\definecolor{codegray}{rgb}{0.5,0.5,0.5}
\definecolor{codepurple}{rgb}{0.58,0,0.82}
\definecolor{backcolour}{rgb}{0.95,0.95,0.92}

\lstdefinestyle{mystyle}{
    backgroundcolor=\color{backcolour},   
    commentstyle=\color{codegreen},
    keywordstyle=\color{magenta},
    numberstyle=\tiny\color{codegray},
    stringstyle=\color{codepurple},
    basicstyle=\ttfamily\footnotesize,
    breakatwhitespace=false,         
    breaklines=true,                 
    captionpos=b,                    
    keepspaces=true,                 
    numbers=left,                    
xleftmargin=2em,
framexleftmargin=2em,            
    showspaces=false,                
    showstringspaces=false,
    showtabs=false,                  
    tabsize=2,
    upquote=true
}

\lstset{style=mystyle}


\lstset{style=mystyle}
\newcommand{\imgdir}{C:/laragon/www/newmc/assets/imgsvg/}
\newcommand{\imgsvgdir}{C:/laragon/www/newmc/assets/imgsvg/}

\definecolor{mcgris}{RGB}{220, 220, 220}% ancien~; pour compatibilité
\definecolor{mcbleu}{RGB}{52, 152, 219}
\definecolor{mcvert}{RGB}{125, 194, 70}
\definecolor{mcmauve}{RGB}{154, 0, 215}
\definecolor{mcorange}{RGB}{255, 96, 0}
\definecolor{mcturquoise}{RGB}{0, 153, 153}
\definecolor{mcrouge}{RGB}{255, 0, 0}
\definecolor{mclightvert}{RGB}{205, 234, 190}

\definecolor{gris}{RGB}{220, 220, 220}
\definecolor{bleu}{RGB}{52, 152, 219}
\definecolor{vert}{RGB}{125, 194, 70}
\definecolor{mauve}{RGB}{154, 0, 215}
\definecolor{orange}{RGB}{255, 96, 0}
\definecolor{turquoise}{RGB}{0, 153, 153}
\definecolor{rouge}{RGB}{255, 0, 0}
\definecolor{lightvert}{RGB}{205, 234, 190}
\setitemize[0]{label=\color{lightvert}  $\bullet$}

\pagestyle{fancy}
\renewcommand{\headrulewidth}{0.2pt}
\fancyhead[L]{maths-cours.fr}
\fancyhead[R]{\thepage}
\renewcommand{\footrulewidth}{0.2pt}
\fancyfoot[C]{}

\newcolumntype{C}{>{\centering\arraybackslash}X}
\newcolumntype{s}{>{\hsize=.35\hsize\arraybackslash}X}

\setlength{\parindent}{0pt}		 
\setlength{\parskip}{3mm}
\setlength{\headheight}{1cm}

\def\ebook{ebook}
\def\book{book}
\def\web{web}
\def\type{web}

\newcommand{\vect}[1]{\overrightarrow{\,\mathstrut#1\,}}

\def\Oij{$\left(\text{O}~;~\vect{\imath},~\vect{\jmath}\right)$}
\def\Oijk{$\left(\text{O}~;~\vect{\imath},~\vect{\jmath},~\vect{k}\right)$}
\def\Ouv{$\left(\text{O}~;~\vect{u},~\vect{v}\right)$}

\hypersetup{breaklinks=true, colorlinks = true, linkcolor = OliveGreen, urlcolor = OliveGreen, citecolor = OliveGreen, pdfauthor={Didier BONNEL - https://www.maths-cours.fr} } % supprime les bordures autour des liens

\renewcommand{\arg}[0]{\text{arg}}

\everymath{\displaystyle}

%================================================================================================================================
%
% Macros - Commandes
%
%================================================================================================================================

\newcommand\meta[2]{    			% Utilisé pour créer le post HTML.
	\def\titre{titre}
	\def\url{url}
	\def\arg{#1}
	\ifx\titre\arg
		\newcommand\maintitle{#2}
		\fancyhead[L]{#2}
		{\Large\sffamily \MakeUppercase{#2}}
		\vspace{1mm}\textcolor{mcvert}{\hrule}
	\fi 
	\ifx\url\arg
		\fancyfoot[L]{\href{https://www.maths-cours.fr#2}{\black \footnotesize{https://www.maths-cours.fr#2}}}
	\fi 
}


\newcommand\TitreC[1]{    		% Titre centré
     \needspace{3\baselineskip}
     \begin{center}\textbf{#1}\end{center}
}

\newcommand\newpar{    		% paragraphe
     \par
}

\newcommand\nosp {    		% commande vide (pas d'espace)
}
\newcommand{\id}[1]{} %ignore

\newcommand\boite[2]{				% Boite simple sans titre
	\vspace{5mm}
	\setlength{\fboxrule}{0.2mm}
	\setlength{\fboxsep}{5mm}	
	\fcolorbox{#1}{#1!3}{\makebox[\linewidth-2\fboxrule-2\fboxsep]{
  		\begin{minipage}[t]{\linewidth-2\fboxrule-4\fboxsep}\setlength{\parskip}{3mm}
  			 #2
  		\end{minipage}
	}}
	\vspace{5mm}
}

\newcommand\CBox[4]{				% Boites
	\vspace{5mm}
	\setlength{\fboxrule}{0.2mm}
	\setlength{\fboxsep}{5mm}
	
	\fcolorbox{#1}{#1!3}{\makebox[\linewidth-2\fboxrule-2\fboxsep]{
		\begin{minipage}[t]{1cm}\setlength{\parskip}{3mm}
	  		\textcolor{#1}{\LARGE{#2}}    
 	 	\end{minipage}  
  		\begin{minipage}[t]{\linewidth-2\fboxrule-4\fboxsep}\setlength{\parskip}{3mm}
			\raisebox{1.2mm}{\normalsize\sffamily{\textcolor{#1}{#3}}}						
  			 #4
  		\end{minipage}
	}}
	\vspace{5mm}
}

\newcommand\cadre[3]{				% Boites convertible html
	\par
	\vspace{2mm}
	\setlength{\fboxrule}{0.1mm}
	\setlength{\fboxsep}{5mm}
	\fcolorbox{#1}{white}{\makebox[\linewidth-2\fboxrule-2\fboxsep]{
  		\begin{minipage}[t]{\linewidth-2\fboxrule-4\fboxsep}\setlength{\parskip}{3mm}
			\raisebox{-2.5mm}{\sffamily \small{\textcolor{#1}{\MakeUppercase{#2}}}}		
			\par		
  			 #3
 	 		\end{minipage}
	}}
		\vspace{2mm}
	\par
}

\newcommand\bloc[3]{				% Boites convertible html sans bordure
     \needspace{2\baselineskip}
     {\sffamily \small{\textcolor{#1}{\MakeUppercase{#2}}}}    
		\par		
  			 #3
		\par
}

\newcommand\CHelp[1]{
     \CBox{Plum}{\faInfoCircle}{À RETENIR}{#1}
}

\newcommand\CUp[1]{
     \CBox{NavyBlue}{\faThumbsOUp}{EN PRATIQUE}{#1}
}

\newcommand\CInfo[1]{
     \CBox{Sepia}{\faArrowCircleRight}{REMARQUE}{#1}
}

\newcommand\CRedac[1]{
     \CBox{PineGreen}{\faEdit}{BIEN R\'EDIGER}{#1}
}

\newcommand\CError[1]{
     \CBox{Red}{\faExclamationTriangle}{ATTENTION}{#1}
}

\newcommand\TitreExo[2]{
\needspace{4\baselineskip}
 {\sffamily\large EXERCICE #1\ (\emph{#2 points})}
\vspace{5mm}
}

\newcommand\img[2]{
          \includegraphics[width=#2\paperwidth]{\imgdir#1}
}

\newcommand\imgsvg[2]{
       \begin{center}   \includegraphics[width=#2\paperwidth]{\imgsvgdir#1} \end{center}
}


\newcommand\Lien[2]{
     \href{#1}{#2 \tiny \faExternalLink}
}
\newcommand\mcLien[2]{
     \href{https~://www.maths-cours.fr/#1}{#2 \tiny \faExternalLink}
}

\newcommand{\euro}{\eurologo{}}

%================================================================================================================================
%
% Macros - Environement
%
%================================================================================================================================

\newenvironment{tex}{ %
}
{%
}

\newenvironment{indente}{ %
	\setlength\parindent{10mm}
}

{
	\setlength\parindent{0mm}
}

\newenvironment{corrige}{%
     \needspace{3\baselineskip}
     \medskip
     \textbf{\textsc{Corrigé}}
     \medskip
}
{
}

\newenvironment{extern}{%
     \begin{center}
     }
     {
     \end{center}
}

\NewEnviron{code}{%
	\par
     \boite{gray}{\texttt{%
     \BODY
     }}
     \par
}

\newenvironment{vbloc}{% boite sans cadre empeche saut de page
     \begin{minipage}[t]{\linewidth}
     }
     {
     \end{minipage}
}
\NewEnviron{h2}{%
    \needspace{3\baselineskip}
    \vspace{0.6cm}
	\noindent \MakeUppercase{\sffamily \large \BODY}
	\vspace{1mm}\textcolor{mcgris}{\hrule}\vspace{0.4cm}
	\par
}{}

\NewEnviron{h3}{%
    \needspace{3\baselineskip}
	\vspace{5mm}
	\textsc{\BODY}
	\par
}

\NewEnviron{margeneg}{ %
\begin{addmargin}[-1cm]{0cm}
\BODY
\end{addmargin}
}

\NewEnviron{html}{%
}

\begin{document}
\meta{url}{/exercices/fonctions-bac-s-asie-2018/}
\meta{pid}{9440}
\meta{titre}{Fonctions – Bac S Asie 2018}
\meta{type}{exercices}
%
\begin{h2}Exercice 1 (5 points)\end{h2}
\textbf{Commun  à tous les candidats}
\bigbreak
Une ferme aquatique exploite une population de crevettes qui évolue en fonction de la reproduction naturelle et des prélèvements effectués.
\par
La masse initiale de celte population de crevettes est estimée à $100$ tonnes.
\par
Compte tenu des conditions de reproduction et de prélèvement, on modélise la masse de la
population de crevettes, exprimée en tonne, en fonction du temps, exprimé en semaine, par la fonction $f_p$, définie sur l'intervalle $[0~;~ +\infty[$ par~:
\par
\[f_p(t) = \dfrac{100p}{1 - (1 - p)\text{e}^{- pt}}\]
\par
où $p$ est un paramètre strictement compris entre $0$ et $1$ et qui dépend des différentes conditions de vie et d'exploitation des crevettes.
\medbreak
\begin{enumerate}
     \item Cohérence du modèle
     \begin{enumerate}[label=\alph*.]
          \item Calculer $f_p(0)$.
          \item On rappelle que $0 < p < 1$.
          \par
          Démontrer que pour tout nombre réel $t \geqslant 0$,\: $1 - (1 - p)\text{e}^{- pt} \geqslant p$.
          \item En déduire que pour tout nombre réel $t \geqslant 0$,\: $0 < f_p(t) \leqslant  100$.
     \end{enumerate}
     \item  Étude de l'évolution lorsque $p = 0,9$
     \par
     Dans cette question, on prend $p = 0,9$ et on étudie la fonction $f_{0,9}$ définie sur $[0~;~ +\infty[$ par~:
     \par
     \par
     \[f_{0,9}(t) = \dfrac{90}{1 - 0,1 \text{e}^{- 0,9t}}.\]
     \begin{enumerate}[label=\alph*.]
          \item Déterminer les variations de la fonction $f_{0,9}$.
          \item Démontrer pour tout nombre réel $t \geqslant 0$,\: $f_{0,9}(t) \geqslant 90$.
          \item Interpréter les résultats des questions 2. a. et 2. b. dans le contexte.
     \end{enumerate}
     \item  Retour au cas général
     \par
     On rappelle que $0 < p < 1$.
     \par
     Exprimer en fonction de $p$ la limite de $f_p$ lorsque $t$ tend vers $+ \infty$.
     \item  Dans cette question, on prend $p = \dfrac{1}{2}$.
     \begin{enumerate}[label=\alph*.]
          \item Montrer que la fonction $H$ définie sur l'intervalle $[0~;~ +\infty[$ par~:
          \par
          \[H(t) = 100\ln \left(2 - \text{e}^{- \frac{t}{2}}\right) + 50t\]
          \par
          est une primitive de la fonction $f_{\frac{1}{2}}$ sur cet intervalle.
          \item En déduire la masse moyenne de crevettes lors des 5 premières semaines d'exploitation, c'est-à-dire la valeur moyenne de la fonction $f_{\frac{1}{2}}$ sur l'intervalle $[0~;~5].$
          \par
          En donner une valeur approchée arrondie à la tonne.
     \end{enumerate}
\end{enumerate}

\end{document}
µ
\documentclass[a4paper]{article}

%================================================================================================================================
%
% Packages
%
%================================================================================================================================

\usepackage[T1]{fontenc} 	% pour caractères accentués
\usepackage[utf8]{inputenc}  % encodage utf8
\usepackage[french]{babel}	% langue : français
\usepackage{fourier}			% caractères plus lisibles
\usepackage[dvipsnames]{xcolor} % couleurs
\usepackage{fancyhdr}		% réglage header footer
\usepackage{needspace}		% empêcher sauts de page mal placés
\usepackage{graphicx}		% pour inclure des graphiques
\usepackage{enumitem,cprotect}		% personnalise les listes d'items (nécessaire pour ol, al ...)
\usepackage{hyperref}		% Liens hypertexte
\usepackage{pstricks,pst-all,pst-node,pstricks-add,pst-math,pst-plot,pst-tree,pst-eucl} % pstricks
\usepackage[a4paper,includeheadfoot,top=2cm,left=3cm, bottom=2cm,right=3cm]{geometry} % marges etc.
\usepackage{comment}			% commentaires multilignes
\usepackage{amsmath,environ} % maths (matrices, etc.)
\usepackage{amssymb,makeidx}
\usepackage{bm}				% bold maths
\usepackage{tabularx}		% tableaux
\usepackage{colortbl}		% tableaux en couleur
\usepackage{fontawesome}		% Fontawesome
\usepackage{environ}			% environment with command
\usepackage{fp}				% calculs pour ps-tricks
\usepackage{multido}			% pour ps tricks
\usepackage[np]{numprint}	% formattage nombre
\usepackage{tikz,tkz-tab} 			% package principal TikZ
\usepackage{pgfplots}   % axes
\usepackage{mathrsfs}    % cursives
\usepackage{calc}			% calcul taille boites
\usepackage[scaled=0.875]{helvet} % font sans serif
\usepackage{svg} % svg
\usepackage{scrextend} % local margin
\usepackage{scratch} %scratch
\usepackage{multicol} % colonnes
%\usepackage{infix-RPN,pst-func} % formule en notation polanaise inversée
\usepackage{listings}

%================================================================================================================================
%
% Réglages de base
%
%================================================================================================================================

\lstset{
language=Python,   % R code
literate=
{á}{{\'a}}1
{à}{{\`a}}1
{ã}{{\~a}}1
{é}{{\'e}}1
{è}{{\`e}}1
{ê}{{\^e}}1
{í}{{\'i}}1
{ó}{{\'o}}1
{õ}{{\~o}}1
{ú}{{\'u}}1
{ü}{{\"u}}1
{ç}{{\c{c}}}1
{~}{{ }}1
}


\definecolor{codegreen}{rgb}{0,0.6,0}
\definecolor{codegray}{rgb}{0.5,0.5,0.5}
\definecolor{codepurple}{rgb}{0.58,0,0.82}
\definecolor{backcolour}{rgb}{0.95,0.95,0.92}

\lstdefinestyle{mystyle}{
    backgroundcolor=\color{backcolour},   
    commentstyle=\color{codegreen},
    keywordstyle=\color{magenta},
    numberstyle=\tiny\color{codegray},
    stringstyle=\color{codepurple},
    basicstyle=\ttfamily\footnotesize,
    breakatwhitespace=false,         
    breaklines=true,                 
    captionpos=b,                    
    keepspaces=true,                 
    numbers=left,                    
xleftmargin=2em,
framexleftmargin=2em,            
    showspaces=false,                
    showstringspaces=false,
    showtabs=false,                  
    tabsize=2,
    upquote=true
}

\lstset{style=mystyle}


\lstset{style=mystyle}
\newcommand{\imgdir}{C:/laragon/www/newmc/assets/imgsvg/}
\newcommand{\imgsvgdir}{C:/laragon/www/newmc/assets/imgsvg/}

\definecolor{mcgris}{RGB}{220, 220, 220}% ancien~; pour compatibilité
\definecolor{mcbleu}{RGB}{52, 152, 219}
\definecolor{mcvert}{RGB}{125, 194, 70}
\definecolor{mcmauve}{RGB}{154, 0, 215}
\definecolor{mcorange}{RGB}{255, 96, 0}
\definecolor{mcturquoise}{RGB}{0, 153, 153}
\definecolor{mcrouge}{RGB}{255, 0, 0}
\definecolor{mclightvert}{RGB}{205, 234, 190}

\definecolor{gris}{RGB}{220, 220, 220}
\definecolor{bleu}{RGB}{52, 152, 219}
\definecolor{vert}{RGB}{125, 194, 70}
\definecolor{mauve}{RGB}{154, 0, 215}
\definecolor{orange}{RGB}{255, 96, 0}
\definecolor{turquoise}{RGB}{0, 153, 153}
\definecolor{rouge}{RGB}{255, 0, 0}
\definecolor{lightvert}{RGB}{205, 234, 190}
\setitemize[0]{label=\color{lightvert}  $\bullet$}

\pagestyle{fancy}
\renewcommand{\headrulewidth}{0.2pt}
\fancyhead[L]{maths-cours.fr}
\fancyhead[R]{\thepage}
\renewcommand{\footrulewidth}{0.2pt}
\fancyfoot[C]{}

\newcolumntype{C}{>{\centering\arraybackslash}X}
\newcolumntype{s}{>{\hsize=.35\hsize\arraybackslash}X}

\setlength{\parindent}{0pt}		 
\setlength{\parskip}{3mm}
\setlength{\headheight}{1cm}

\def\ebook{ebook}
\def\book{book}
\def\web{web}
\def\type{web}

\newcommand{\vect}[1]{\overrightarrow{\,\mathstrut#1\,}}

\def\Oij{$\left(\text{O}~;~\vect{\imath},~\vect{\jmath}\right)$}
\def\Oijk{$\left(\text{O}~;~\vect{\imath},~\vect{\jmath},~\vect{k}\right)$}
\def\Ouv{$\left(\text{O}~;~\vect{u},~\vect{v}\right)$}

\hypersetup{breaklinks=true, colorlinks = true, linkcolor = OliveGreen, urlcolor = OliveGreen, citecolor = OliveGreen, pdfauthor={Didier BONNEL - https://www.maths-cours.fr} } % supprime les bordures autour des liens

\renewcommand{\arg}[0]{\text{arg}}

\everymath{\displaystyle}

%================================================================================================================================
%
% Macros - Commandes
%
%================================================================================================================================

\newcommand\meta[2]{    			% Utilisé pour créer le post HTML.
	\def\titre{titre}
	\def\url{url}
	\def\arg{#1}
	\ifx\titre\arg
		\newcommand\maintitle{#2}
		\fancyhead[L]{#2}
		{\Large\sffamily \MakeUppercase{#2}}
		\vspace{1mm}\textcolor{mcvert}{\hrule}
	\fi 
	\ifx\url\arg
		\fancyfoot[L]{\href{https://www.maths-cours.fr#2}{\black \footnotesize{https://www.maths-cours.fr#2}}}
	\fi 
}


\newcommand\TitreC[1]{    		% Titre centré
     \needspace{3\baselineskip}
     \begin{center}\textbf{#1}\end{center}
}

\newcommand\newpar{    		% paragraphe
     \par
}

\newcommand\nosp {    		% commande vide (pas d'espace)
}
\newcommand{\id}[1]{} %ignore

\newcommand\boite[2]{				% Boite simple sans titre
	\vspace{5mm}
	\setlength{\fboxrule}{0.2mm}
	\setlength{\fboxsep}{5mm}	
	\fcolorbox{#1}{#1!3}{\makebox[\linewidth-2\fboxrule-2\fboxsep]{
  		\begin{minipage}[t]{\linewidth-2\fboxrule-4\fboxsep}\setlength{\parskip}{3mm}
  			 #2
  		\end{minipage}
	}}
	\vspace{5mm}
}

\newcommand\CBox[4]{				% Boites
	\vspace{5mm}
	\setlength{\fboxrule}{0.2mm}
	\setlength{\fboxsep}{5mm}
	
	\fcolorbox{#1}{#1!3}{\makebox[\linewidth-2\fboxrule-2\fboxsep]{
		\begin{minipage}[t]{1cm}\setlength{\parskip}{3mm}
	  		\textcolor{#1}{\LARGE{#2}}    
 	 	\end{minipage}  
  		\begin{minipage}[t]{\linewidth-2\fboxrule-4\fboxsep}\setlength{\parskip}{3mm}
			\raisebox{1.2mm}{\normalsize\sffamily{\textcolor{#1}{#3}}}						
  			 #4
  		\end{minipage}
	}}
	\vspace{5mm}
}

\newcommand\cadre[3]{				% Boites convertible html
	\par
	\vspace{2mm}
	\setlength{\fboxrule}{0.1mm}
	\setlength{\fboxsep}{5mm}
	\fcolorbox{#1}{white}{\makebox[\linewidth-2\fboxrule-2\fboxsep]{
  		\begin{minipage}[t]{\linewidth-2\fboxrule-4\fboxsep}\setlength{\parskip}{3mm}
			\raisebox{-2.5mm}{\sffamily \small{\textcolor{#1}{\MakeUppercase{#2}}}}		
			\par		
  			 #3
 	 		\end{minipage}
	}}
		\vspace{2mm}
	\par
}

\newcommand\bloc[3]{				% Boites convertible html sans bordure
     \needspace{2\baselineskip}
     {\sffamily \small{\textcolor{#1}{\MakeUppercase{#2}}}}    
		\par		
  			 #3
		\par
}

\newcommand\CHelp[1]{
     \CBox{Plum}{\faInfoCircle}{À RETENIR}{#1}
}

\newcommand\CUp[1]{
     \CBox{NavyBlue}{\faThumbsOUp}{EN PRATIQUE}{#1}
}

\newcommand\CInfo[1]{
     \CBox{Sepia}{\faArrowCircleRight}{REMARQUE}{#1}
}

\newcommand\CRedac[1]{
     \CBox{PineGreen}{\faEdit}{BIEN R\'EDIGER}{#1}
}

\newcommand\CError[1]{
     \CBox{Red}{\faExclamationTriangle}{ATTENTION}{#1}
}

\newcommand\TitreExo[2]{
\needspace{4\baselineskip}
 {\sffamily\large EXERCICE #1\ (\emph{#2 points})}
\vspace{5mm}
}

\newcommand\img[2]{
          \includegraphics[width=#2\paperwidth]{\imgdir#1}
}

\newcommand\imgsvg[2]{
       \begin{center}   \includegraphics[width=#2\paperwidth]{\imgsvgdir#1} \end{center}
}


\newcommand\Lien[2]{
     \href{#1}{#2 \tiny \faExternalLink}
}
\newcommand\mcLien[2]{
     \href{https~://www.maths-cours.fr/#1}{#2 \tiny \faExternalLink}
}

\newcommand{\euro}{\eurologo{}}

%================================================================================================================================
%
% Macros - Environement
%
%================================================================================================================================

\newenvironment{tex}{ %
}
{%
}

\newenvironment{indente}{ %
	\setlength\parindent{10mm}
}

{
	\setlength\parindent{0mm}
}

\newenvironment{corrige}{%
     \needspace{3\baselineskip}
     \medskip
     \textbf{\textsc{Corrigé}}
     \medskip
}
{
}

\newenvironment{extern}{%
     \begin{center}
     }
     {
     \end{center}
}

\NewEnviron{code}{%
	\par
     \boite{gray}{\texttt{%
     \BODY
     }}
     \par
}

\newenvironment{vbloc}{% boite sans cadre empeche saut de page
     \begin{minipage}[t]{\linewidth}
     }
     {
     \end{minipage}
}
\NewEnviron{h2}{%
    \needspace{3\baselineskip}
    \vspace{0.6cm}
	\noindent \MakeUppercase{\sffamily \large \BODY}
	\vspace{1mm}\textcolor{mcgris}{\hrule}\vspace{0.4cm}
	\par
}{}

\NewEnviron{h3}{%
    \needspace{3\baselineskip}
	\vspace{5mm}
	\textsc{\BODY}
	\par
}

\NewEnviron{margeneg}{ %
\begin{addmargin}[-1cm]{0cm}
\BODY
\end{addmargin}
}

\NewEnviron{html}{%
}

\begin{document}
\meta{url}{/exercices/probabilites-bac-s-asie-2018/}
\meta{pid}{9442}
\meta{titre}{Probabilités – Bac S Asie 2018}
\meta{type}{exercices}
%
\begin{h2}Exercice 2 (5 points)\end{h2}
\textbf{Commun  à tous les candidats}
\bigbreak
Dans les parties A et B de cet exercice, on considère une maladie~; tout individu a une probabilité égale à $0,15$ d'être touché par cette maladie.
\bigbreak
\TitreC{Partie A }
\medbreak
\emph{Cette partie est un questionnaire à choix multiples (Q. C. M.). Pour chacune des questions, une seule des quatre réponses est exacte. Le candidat indiquera sur sa copie le numéro de la question et la lettre correspondant à la réponse exacte. \\Aucune justification n'est demandée. \\Une réponse exacte rapporte un point, une réponse fausse ou une absence de réponse ne rapporte ni n'enlève aucun point.}
\medbreak
Un test de dépistage de cette maladie a été mis au point. Si l'individu est malade, dans 94\,\% des cas le test est positif. Pour un individu choisi au hasard dans cette population, la probabilité que le test soit positif vaut $0,158$.
\medbreak
\begin{enumerate}
     \item On teste un individu choisi au hasard dans la population~: le test est positif. Une valeur arrondie au centième de la probabilité que la personne soit malade est égale à~:
     \medbreak
     \begin{tabularx}{\linewidth}{*{2}{X}}%class="cel50 noborder"
          \textbf{A~:~~} 0,94 &\textbf{B~:~~} 1 \\
          \textbf{C~:~~} 0,89 &\textbf{D~:~~} on ne peut pas savoir
     \end{tabularx}
     \medbreak
     \item  On prélève un échantillon aléatoire dans la population, et on fait passer le test aux individus de cet échantillon. On souhaite que la probabilité qu'au moins un individu soit testé positivement soit supérieure ou égale à $0,99$. La taille minimum de l'échantillon doit être égale à~:
     \medbreak
     \begin{tabularx}{\linewidth}{*{2}{X}}%class="cel50 noborder"
          \textbf{A~:~~}26 personnes &\textbf{B~:~~} 27 personnes \\
          \textbf{C~:~~} 3 personnes &\textbf{D~:~~} 7 personnes
     \end{tabularx}
     \medbreak
     \item  Un vaccin pour lutter contre cette maladie a été mis au point. II est fabriqué par une entreprise sous forme de dose injectable par seringue. Le volume $V$ (exprimé en millilitre) d'une dose suit une loi normale d'espérance $\mu = 2$ et d'écart-type $\sigma$. La probabilité que le volume d'une dose, exprimé en millilitre, soit compris entre $1,99$ et $2,01$ millilitres est égale à $0,997$.
     \par
     La valeur de $\sigma$ doit vérifier~:
     \medbreak
     \begin{tabularx}{\linewidth}{*{2}{X}}%class="cel50 noborder"
          \textbf{A~:~~}$\sigma = 0,02$ &\textbf{B~:~~} $\sigma < 0,003$ \\
          \textbf{C~:~~} $\sigma > 0,003$ &\textbf{D~:~~} $\sigma = 0,003$
     \end{tabularx}
     \medbreak
\end{enumerate}
\bigbreak
\TitreC{Partie B}
\medbreak
\begin{enumerate}
     \item Une boîte d'un certain médicament permet de soigner un malade.
     \par
     La durée d'efficacité (exprimée en mois) de ce médicament est modélisée de la manière
     suivante~:
     \begin{itemize}
          \item durant les $12$ premiers mois après fabrication, on est certain qu'il demeure efficace~;
          \item au-delà, sa durée d'efficacité restante suit une loi exponentielle de paramètre
          $\lambda$.
     \end{itemize}
     La probabilité que l'une des boîtes prise au hasard dans un stock ait une durée d'efficacité totale supérieure à $18$ mois est égale à $0,887$.
     \par
     Quelle est la valeur moyenne de la durée d'efficacité totale de ce médicament~?
     \item Une ville de 100~000 habitants veut constituer un stock de ces boîtes afin de soigner les personnes malades.
     \par
     Quelle doit être la taille minimale de ce stock pour que la probabilité qu'il suffise à soigner tous les malades de cette ville soit supérieure à $95$\,\%~?
\end{enumerate}

\end{document}
µ
\documentclass[a4paper]{article}

%================================================================================================================================
%
% Packages
%
%================================================================================================================================

\usepackage[T1]{fontenc} 	% pour caractères accentués
\usepackage[utf8]{inputenc}  % encodage utf8
\usepackage[french]{babel}	% langue : français
\usepackage{fourier}			% caractères plus lisibles
\usepackage[dvipsnames]{xcolor} % couleurs
\usepackage{fancyhdr}		% réglage header footer
\usepackage{needspace}		% empêcher sauts de page mal placés
\usepackage{graphicx}		% pour inclure des graphiques
\usepackage{enumitem,cprotect}		% personnalise les listes d'items (nécessaire pour ol, al ...)
\usepackage{hyperref}		% Liens hypertexte
\usepackage{pstricks,pst-all,pst-node,pstricks-add,pst-math,pst-plot,pst-tree,pst-eucl} % pstricks
\usepackage[a4paper,includeheadfoot,top=2cm,left=3cm, bottom=2cm,right=3cm]{geometry} % marges etc.
\usepackage{comment}			% commentaires multilignes
\usepackage{amsmath,environ} % maths (matrices, etc.)
\usepackage{amssymb,makeidx}
\usepackage{bm}				% bold maths
\usepackage{tabularx}		% tableaux
\usepackage{colortbl}		% tableaux en couleur
\usepackage{fontawesome}		% Fontawesome
\usepackage{environ}			% environment with command
\usepackage{fp}				% calculs pour ps-tricks
\usepackage{multido}			% pour ps tricks
\usepackage[np]{numprint}	% formattage nombre
\usepackage{tikz,tkz-tab} 			% package principal TikZ
\usepackage{pgfplots}   % axes
\usepackage{mathrsfs}    % cursives
\usepackage{calc}			% calcul taille boites
\usepackage[scaled=0.875]{helvet} % font sans serif
\usepackage{svg} % svg
\usepackage{scrextend} % local margin
\usepackage{scratch} %scratch
\usepackage{multicol} % colonnes
%\usepackage{infix-RPN,pst-func} % formule en notation polanaise inversée
\usepackage{listings}

%================================================================================================================================
%
% Réglages de base
%
%================================================================================================================================

\lstset{
language=Python,   % R code
literate=
{á}{{\'a}}1
{à}{{\`a}}1
{ã}{{\~a}}1
{é}{{\'e}}1
{è}{{\`e}}1
{ê}{{\^e}}1
{í}{{\'i}}1
{ó}{{\'o}}1
{õ}{{\~o}}1
{ú}{{\'u}}1
{ü}{{\"u}}1
{ç}{{\c{c}}}1
{~}{{ }}1
}


\definecolor{codegreen}{rgb}{0,0.6,0}
\definecolor{codegray}{rgb}{0.5,0.5,0.5}
\definecolor{codepurple}{rgb}{0.58,0,0.82}
\definecolor{backcolour}{rgb}{0.95,0.95,0.92}

\lstdefinestyle{mystyle}{
    backgroundcolor=\color{backcolour},   
    commentstyle=\color{codegreen},
    keywordstyle=\color{magenta},
    numberstyle=\tiny\color{codegray},
    stringstyle=\color{codepurple},
    basicstyle=\ttfamily\footnotesize,
    breakatwhitespace=false,         
    breaklines=true,                 
    captionpos=b,                    
    keepspaces=true,                 
    numbers=left,                    
xleftmargin=2em,
framexleftmargin=2em,            
    showspaces=false,                
    showstringspaces=false,
    showtabs=false,                  
    tabsize=2,
    upquote=true
}

\lstset{style=mystyle}


\lstset{style=mystyle}
\newcommand{\imgdir}{C:/laragon/www/newmc/assets/imgsvg/}
\newcommand{\imgsvgdir}{C:/laragon/www/newmc/assets/imgsvg/}

\definecolor{mcgris}{RGB}{220, 220, 220}% ancien~; pour compatibilité
\definecolor{mcbleu}{RGB}{52, 152, 219}
\definecolor{mcvert}{RGB}{125, 194, 70}
\definecolor{mcmauve}{RGB}{154, 0, 215}
\definecolor{mcorange}{RGB}{255, 96, 0}
\definecolor{mcturquoise}{RGB}{0, 153, 153}
\definecolor{mcrouge}{RGB}{255, 0, 0}
\definecolor{mclightvert}{RGB}{205, 234, 190}

\definecolor{gris}{RGB}{220, 220, 220}
\definecolor{bleu}{RGB}{52, 152, 219}
\definecolor{vert}{RGB}{125, 194, 70}
\definecolor{mauve}{RGB}{154, 0, 215}
\definecolor{orange}{RGB}{255, 96, 0}
\definecolor{turquoise}{RGB}{0, 153, 153}
\definecolor{rouge}{RGB}{255, 0, 0}
\definecolor{lightvert}{RGB}{205, 234, 190}
\setitemize[0]{label=\color{lightvert}  $\bullet$}

\pagestyle{fancy}
\renewcommand{\headrulewidth}{0.2pt}
\fancyhead[L]{maths-cours.fr}
\fancyhead[R]{\thepage}
\renewcommand{\footrulewidth}{0.2pt}
\fancyfoot[C]{}

\newcolumntype{C}{>{\centering\arraybackslash}X}
\newcolumntype{s}{>{\hsize=.35\hsize\arraybackslash}X}

\setlength{\parindent}{0pt}		 
\setlength{\parskip}{3mm}
\setlength{\headheight}{1cm}

\def\ebook{ebook}
\def\book{book}
\def\web{web}
\def\type{web}

\newcommand{\vect}[1]{\overrightarrow{\,\mathstrut#1\,}}

\def\Oij{$\left(\text{O}~;~\vect{\imath},~\vect{\jmath}\right)$}
\def\Oijk{$\left(\text{O}~;~\vect{\imath},~\vect{\jmath},~\vect{k}\right)$}
\def\Ouv{$\left(\text{O}~;~\vect{u},~\vect{v}\right)$}

\hypersetup{breaklinks=true, colorlinks = true, linkcolor = OliveGreen, urlcolor = OliveGreen, citecolor = OliveGreen, pdfauthor={Didier BONNEL - https://www.maths-cours.fr} } % supprime les bordures autour des liens

\renewcommand{\arg}[0]{\text{arg}}

\everymath{\displaystyle}

%================================================================================================================================
%
% Macros - Commandes
%
%================================================================================================================================

\newcommand\meta[2]{    			% Utilisé pour créer le post HTML.
	\def\titre{titre}
	\def\url{url}
	\def\arg{#1}
	\ifx\titre\arg
		\newcommand\maintitle{#2}
		\fancyhead[L]{#2}
		{\Large\sffamily \MakeUppercase{#2}}
		\vspace{1mm}\textcolor{mcvert}{\hrule}
	\fi 
	\ifx\url\arg
		\fancyfoot[L]{\href{https://www.maths-cours.fr#2}{\black \footnotesize{https://www.maths-cours.fr#2}}}
	\fi 
}


\newcommand\TitreC[1]{    		% Titre centré
     \needspace{3\baselineskip}
     \begin{center}\textbf{#1}\end{center}
}

\newcommand\newpar{    		% paragraphe
     \par
}

\newcommand\nosp {    		% commande vide (pas d'espace)
}
\newcommand{\id}[1]{} %ignore

\newcommand\boite[2]{				% Boite simple sans titre
	\vspace{5mm}
	\setlength{\fboxrule}{0.2mm}
	\setlength{\fboxsep}{5mm}	
	\fcolorbox{#1}{#1!3}{\makebox[\linewidth-2\fboxrule-2\fboxsep]{
  		\begin{minipage}[t]{\linewidth-2\fboxrule-4\fboxsep}\setlength{\parskip}{3mm}
  			 #2
  		\end{minipage}
	}}
	\vspace{5mm}
}

\newcommand\CBox[4]{				% Boites
	\vspace{5mm}
	\setlength{\fboxrule}{0.2mm}
	\setlength{\fboxsep}{5mm}
	
	\fcolorbox{#1}{#1!3}{\makebox[\linewidth-2\fboxrule-2\fboxsep]{
		\begin{minipage}[t]{1cm}\setlength{\parskip}{3mm}
	  		\textcolor{#1}{\LARGE{#2}}    
 	 	\end{minipage}  
  		\begin{minipage}[t]{\linewidth-2\fboxrule-4\fboxsep}\setlength{\parskip}{3mm}
			\raisebox{1.2mm}{\normalsize\sffamily{\textcolor{#1}{#3}}}						
  			 #4
  		\end{minipage}
	}}
	\vspace{5mm}
}

\newcommand\cadre[3]{				% Boites convertible html
	\par
	\vspace{2mm}
	\setlength{\fboxrule}{0.1mm}
	\setlength{\fboxsep}{5mm}
	\fcolorbox{#1}{white}{\makebox[\linewidth-2\fboxrule-2\fboxsep]{
  		\begin{minipage}[t]{\linewidth-2\fboxrule-4\fboxsep}\setlength{\parskip}{3mm}
			\raisebox{-2.5mm}{\sffamily \small{\textcolor{#1}{\MakeUppercase{#2}}}}		
			\par		
  			 #3
 	 		\end{minipage}
	}}
		\vspace{2mm}
	\par
}

\newcommand\bloc[3]{				% Boites convertible html sans bordure
     \needspace{2\baselineskip}
     {\sffamily \small{\textcolor{#1}{\MakeUppercase{#2}}}}    
		\par		
  			 #3
		\par
}

\newcommand\CHelp[1]{
     \CBox{Plum}{\faInfoCircle}{À RETENIR}{#1}
}

\newcommand\CUp[1]{
     \CBox{NavyBlue}{\faThumbsOUp}{EN PRATIQUE}{#1}
}

\newcommand\CInfo[1]{
     \CBox{Sepia}{\faArrowCircleRight}{REMARQUE}{#1}
}

\newcommand\CRedac[1]{
     \CBox{PineGreen}{\faEdit}{BIEN R\'EDIGER}{#1}
}

\newcommand\CError[1]{
     \CBox{Red}{\faExclamationTriangle}{ATTENTION}{#1}
}

\newcommand\TitreExo[2]{
\needspace{4\baselineskip}
 {\sffamily\large EXERCICE #1\ (\emph{#2 points})}
\vspace{5mm}
}

\newcommand\img[2]{
          \includegraphics[width=#2\paperwidth]{\imgdir#1}
}

\newcommand\imgsvg[2]{
       \begin{center}   \includegraphics[width=#2\paperwidth]{\imgsvgdir#1} \end{center}
}


\newcommand\Lien[2]{
     \href{#1}{#2 \tiny \faExternalLink}
}
\newcommand\mcLien[2]{
     \href{https~://www.maths-cours.fr/#1}{#2 \tiny \faExternalLink}
}

\newcommand{\euro}{\eurologo{}}

%================================================================================================================================
%
% Macros - Environement
%
%================================================================================================================================

\newenvironment{tex}{ %
}
{%
}

\newenvironment{indente}{ %
	\setlength\parindent{10mm}
}

{
	\setlength\parindent{0mm}
}

\newenvironment{corrige}{%
     \needspace{3\baselineskip}
     \medskip
     \textbf{\textsc{Corrigé}}
     \medskip
}
{
}

\newenvironment{extern}{%
     \begin{center}
     }
     {
     \end{center}
}

\NewEnviron{code}{%
	\par
     \boite{gray}{\texttt{%
     \BODY
     }}
     \par
}

\newenvironment{vbloc}{% boite sans cadre empeche saut de page
     \begin{minipage}[t]{\linewidth}
     }
     {
     \end{minipage}
}
\NewEnviron{h2}{%
    \needspace{3\baselineskip}
    \vspace{0.6cm}
	\noindent \MakeUppercase{\sffamily \large \BODY}
	\vspace{1mm}\textcolor{mcgris}{\hrule}\vspace{0.4cm}
	\par
}{}

\NewEnviron{h3}{%
    \needspace{3\baselineskip}
	\vspace{5mm}
	\textsc{\BODY}
	\par
}

\NewEnviron{margeneg}{ %
\begin{addmargin}[-1cm]{0cm}
\BODY
\end{addmargin}
}

\NewEnviron{html}{%
}

\begin{document}
\meta{url}{/exercices/geometrie-dans-lespace-bac-s-asie-2018/}
\meta{pid}{9444}
\meta{titre}{Géométrie dans l'espace – Bac S Asie 2018}
\meta{type}{exercices}
%
\begin{h2}Exercice 3 (5 points)\end{h2}
\textbf{Commun  à tous les candidats}
\bigbreak
On se place dans un repère orthonormé d'origine O et d'axes (O$x$), (O$y$) et (O$z$).
\par
Dans ce repère, on donne les points A$(- 3~;~0~;~0)$, B(3~;~0~;~0) , C$\left(0~;~3\sqrt{3}~;~0\right)$ et D$\left(0~;~\sqrt{3}~;~2\sqrt{6}\right)$.
\par
On note H le milieu du segment [CD] et I le milieu du segment [BC].
\begin{center}
     \begin{extern}%width="500" alt="Tétraèdre Bac S Asie 2018"
          \resizebox{10cm}{!}{
               \psset{unit=1cm,arrowsize=2pt 4}
               \begin{pspicture}(13.5,11)
                    \psline[linestyle=solid,linecolor=lightgray]{->}(0,3)(13.5,3)
                    \psline[linestyle=solid,linecolor=lightgray]{->}(3.2,0)(3.2,9.7)
                    \psline[linestyle=solid,linecolor=lightgray]{->}(0,4.9)(7,0.7)
                    \pspolygon(1.5,4)(5,1.9)(10.8,3)(5.6,10.3)%ABCDA
                    \psline(5,1.92)(5.6,10.3)
                    \psline(1.5,4)(10.8,3)
                    \uput[dl](1.5,4){A} \uput[d](5,1.9){B} \uput[dr](10.8,3){C}
                    \uput[u](5.6,10.3){D} \uput[ur](8.2,6.65){H} \uput[dr](7.9,2.45){I}
                    \psdots[dotsize=2pt](8.2,6.65)(7.9,2.45)
                    \uput[dl](3.2,3){O}\uput[dr](13.5,3){\lightgray $y$} \uput[r](7,0.3){\lightgray $x$}
                    \uput[ul](3.2,9.7){\lightgray $z$}
               \end{pspicture}
          }
     \end{extern}
\end{center}
\medbreak
\begin{enumerate}
     \item Calculer les longueurs AB et AD.
     \par
     \begin{margeneg}
          On admet pour la suite que toutes les arêtes du solide ABCD ont la même longueur, c'est-à-dire que le tétraèdre ABCD est un tétraèdre régulier.\\
          On appelle $\mathscr{P}$ le plan de vecteur normal $\overrightarrow{\text{OH}}$ et passant par le point I.
     \end{margeneg}
     \par
     \item  Étude de la section du tétraèdre ABCD par le plan $\mathscr{P}$
     \begin{enumerate}[label=\alph*.]
          \item Montrer qu'une équation cartésienne du plan $\mathscr{P}$ est~: $2y\sqrt{3} + z\sqrt{6} - 9 = 0$.
          \item Démontrer que le milieu J de [BD] est le point d'intersection de la droite (BD) et du plan $\mathscr{P}$.
          \item  Donner une représentation paramétrique de la droite (AD), puis démontrer que le plans $\mathscr{P}$ et la droite (AD) sont sécants en un point K dont on déterminera les coordonnées.
          \item  Démontrer que les droites (IJ) et (JK) sont perpendiculaires.
          \item  Déterminer précisément la nature de la section du tétraèdre ABCD par le plan $\mathscr{P}$.
     \end{enumerate}
     \item  Peut-on placer un point M sur l'arête [BD] tel que le triangle OIM soit rectangle en M~?
\end{enumerate}

\end{document}
µ
\documentclass[a4paper]{article}

%================================================================================================================================
%
% Packages
%
%================================================================================================================================

\usepackage[T1]{fontenc} 	% pour caractères accentués
\usepackage[utf8]{inputenc}  % encodage utf8
\usepackage[french]{babel}	% langue : français
\usepackage{fourier}			% caractères plus lisibles
\usepackage[dvipsnames]{xcolor} % couleurs
\usepackage{fancyhdr}		% réglage header footer
\usepackage{needspace}		% empêcher sauts de page mal placés
\usepackage{graphicx}		% pour inclure des graphiques
\usepackage{enumitem,cprotect}		% personnalise les listes d'items (nécessaire pour ol, al ...)
\usepackage{hyperref}		% Liens hypertexte
\usepackage{pstricks,pst-all,pst-node,pstricks-add,pst-math,pst-plot,pst-tree,pst-eucl} % pstricks
\usepackage[a4paper,includeheadfoot,top=2cm,left=3cm, bottom=2cm,right=3cm]{geometry} % marges etc.
\usepackage{comment}			% commentaires multilignes
\usepackage{amsmath,environ} % maths (matrices, etc.)
\usepackage{amssymb,makeidx}
\usepackage{bm}				% bold maths
\usepackage{tabularx}		% tableaux
\usepackage{colortbl}		% tableaux en couleur
\usepackage{fontawesome}		% Fontawesome
\usepackage{environ}			% environment with command
\usepackage{fp}				% calculs pour ps-tricks
\usepackage{multido}			% pour ps tricks
\usepackage[np]{numprint}	% formattage nombre
\usepackage{tikz,tkz-tab} 			% package principal TikZ
\usepackage{pgfplots}   % axes
\usepackage{mathrsfs}    % cursives
\usepackage{calc}			% calcul taille boites
\usepackage[scaled=0.875]{helvet} % font sans serif
\usepackage{svg} % svg
\usepackage{scrextend} % local margin
\usepackage{scratch} %scratch
\usepackage{multicol} % colonnes
%\usepackage{infix-RPN,pst-func} % formule en notation polanaise inversée
\usepackage{listings}

%================================================================================================================================
%
% Réglages de base
%
%================================================================================================================================

\lstset{
language=Python,   % R code
literate=
{á}{{\'a}}1
{à}{{\`a}}1
{ã}{{\~a}}1
{é}{{\'e}}1
{è}{{\`e}}1
{ê}{{\^e}}1
{í}{{\'i}}1
{ó}{{\'o}}1
{õ}{{\~o}}1
{ú}{{\'u}}1
{ü}{{\"u}}1
{ç}{{\c{c}}}1
{~}{{ }}1
}


\definecolor{codegreen}{rgb}{0,0.6,0}
\definecolor{codegray}{rgb}{0.5,0.5,0.5}
\definecolor{codepurple}{rgb}{0.58,0,0.82}
\definecolor{backcolour}{rgb}{0.95,0.95,0.92}

\lstdefinestyle{mystyle}{
    backgroundcolor=\color{backcolour},   
    commentstyle=\color{codegreen},
    keywordstyle=\color{magenta},
    numberstyle=\tiny\color{codegray},
    stringstyle=\color{codepurple},
    basicstyle=\ttfamily\footnotesize,
    breakatwhitespace=false,         
    breaklines=true,                 
    captionpos=b,                    
    keepspaces=true,                 
    numbers=left,                    
xleftmargin=2em,
framexleftmargin=2em,            
    showspaces=false,                
    showstringspaces=false,
    showtabs=false,                  
    tabsize=2,
    upquote=true
}

\lstset{style=mystyle}


\lstset{style=mystyle}
\newcommand{\imgdir}{C:/laragon/www/newmc/assets/imgsvg/}
\newcommand{\imgsvgdir}{C:/laragon/www/newmc/assets/imgsvg/}

\definecolor{mcgris}{RGB}{220, 220, 220}% ancien~; pour compatibilité
\definecolor{mcbleu}{RGB}{52, 152, 219}
\definecolor{mcvert}{RGB}{125, 194, 70}
\definecolor{mcmauve}{RGB}{154, 0, 215}
\definecolor{mcorange}{RGB}{255, 96, 0}
\definecolor{mcturquoise}{RGB}{0, 153, 153}
\definecolor{mcrouge}{RGB}{255, 0, 0}
\definecolor{mclightvert}{RGB}{205, 234, 190}

\definecolor{gris}{RGB}{220, 220, 220}
\definecolor{bleu}{RGB}{52, 152, 219}
\definecolor{vert}{RGB}{125, 194, 70}
\definecolor{mauve}{RGB}{154, 0, 215}
\definecolor{orange}{RGB}{255, 96, 0}
\definecolor{turquoise}{RGB}{0, 153, 153}
\definecolor{rouge}{RGB}{255, 0, 0}
\definecolor{lightvert}{RGB}{205, 234, 190}
\setitemize[0]{label=\color{lightvert}  $\bullet$}

\pagestyle{fancy}
\renewcommand{\headrulewidth}{0.2pt}
\fancyhead[L]{maths-cours.fr}
\fancyhead[R]{\thepage}
\renewcommand{\footrulewidth}{0.2pt}
\fancyfoot[C]{}

\newcolumntype{C}{>{\centering\arraybackslash}X}
\newcolumntype{s}{>{\hsize=.35\hsize\arraybackslash}X}

\setlength{\parindent}{0pt}		 
\setlength{\parskip}{3mm}
\setlength{\headheight}{1cm}

\def\ebook{ebook}
\def\book{book}
\def\web{web}
\def\type{web}

\newcommand{\vect}[1]{\overrightarrow{\,\mathstrut#1\,}}

\def\Oij{$\left(\text{O}~;~\vect{\imath},~\vect{\jmath}\right)$}
\def\Oijk{$\left(\text{O}~;~\vect{\imath},~\vect{\jmath},~\vect{k}\right)$}
\def\Ouv{$\left(\text{O}~;~\vect{u},~\vect{v}\right)$}

\hypersetup{breaklinks=true, colorlinks = true, linkcolor = OliveGreen, urlcolor = OliveGreen, citecolor = OliveGreen, pdfauthor={Didier BONNEL - https://www.maths-cours.fr} } % supprime les bordures autour des liens

\renewcommand{\arg}[0]{\text{arg}}

\everymath{\displaystyle}

%================================================================================================================================
%
% Macros - Commandes
%
%================================================================================================================================

\newcommand\meta[2]{    			% Utilisé pour créer le post HTML.
	\def\titre{titre}
	\def\url{url}
	\def\arg{#1}
	\ifx\titre\arg
		\newcommand\maintitle{#2}
		\fancyhead[L]{#2}
		{\Large\sffamily \MakeUppercase{#2}}
		\vspace{1mm}\textcolor{mcvert}{\hrule}
	\fi 
	\ifx\url\arg
		\fancyfoot[L]{\href{https://www.maths-cours.fr#2}{\black \footnotesize{https://www.maths-cours.fr#2}}}
	\fi 
}


\newcommand\TitreC[1]{    		% Titre centré
     \needspace{3\baselineskip}
     \begin{center}\textbf{#1}\end{center}
}

\newcommand\newpar{    		% paragraphe
     \par
}

\newcommand\nosp {    		% commande vide (pas d'espace)
}
\newcommand{\id}[1]{} %ignore

\newcommand\boite[2]{				% Boite simple sans titre
	\vspace{5mm}
	\setlength{\fboxrule}{0.2mm}
	\setlength{\fboxsep}{5mm}	
	\fcolorbox{#1}{#1!3}{\makebox[\linewidth-2\fboxrule-2\fboxsep]{
  		\begin{minipage}[t]{\linewidth-2\fboxrule-4\fboxsep}\setlength{\parskip}{3mm}
  			 #2
  		\end{minipage}
	}}
	\vspace{5mm}
}

\newcommand\CBox[4]{				% Boites
	\vspace{5mm}
	\setlength{\fboxrule}{0.2mm}
	\setlength{\fboxsep}{5mm}
	
	\fcolorbox{#1}{#1!3}{\makebox[\linewidth-2\fboxrule-2\fboxsep]{
		\begin{minipage}[t]{1cm}\setlength{\parskip}{3mm}
	  		\textcolor{#1}{\LARGE{#2}}    
 	 	\end{minipage}  
  		\begin{minipage}[t]{\linewidth-2\fboxrule-4\fboxsep}\setlength{\parskip}{3mm}
			\raisebox{1.2mm}{\normalsize\sffamily{\textcolor{#1}{#3}}}						
  			 #4
  		\end{minipage}
	}}
	\vspace{5mm}
}

\newcommand\cadre[3]{				% Boites convertible html
	\par
	\vspace{2mm}
	\setlength{\fboxrule}{0.1mm}
	\setlength{\fboxsep}{5mm}
	\fcolorbox{#1}{white}{\makebox[\linewidth-2\fboxrule-2\fboxsep]{
  		\begin{minipage}[t]{\linewidth-2\fboxrule-4\fboxsep}\setlength{\parskip}{3mm}
			\raisebox{-2.5mm}{\sffamily \small{\textcolor{#1}{\MakeUppercase{#2}}}}		
			\par		
  			 #3
 	 		\end{minipage}
	}}
		\vspace{2mm}
	\par
}

\newcommand\bloc[3]{				% Boites convertible html sans bordure
     \needspace{2\baselineskip}
     {\sffamily \small{\textcolor{#1}{\MakeUppercase{#2}}}}    
		\par		
  			 #3
		\par
}

\newcommand\CHelp[1]{
     \CBox{Plum}{\faInfoCircle}{À RETENIR}{#1}
}

\newcommand\CUp[1]{
     \CBox{NavyBlue}{\faThumbsOUp}{EN PRATIQUE}{#1}
}

\newcommand\CInfo[1]{
     \CBox{Sepia}{\faArrowCircleRight}{REMARQUE}{#1}
}

\newcommand\CRedac[1]{
     \CBox{PineGreen}{\faEdit}{BIEN R\'EDIGER}{#1}
}

\newcommand\CError[1]{
     \CBox{Red}{\faExclamationTriangle}{ATTENTION}{#1}
}

\newcommand\TitreExo[2]{
\needspace{4\baselineskip}
 {\sffamily\large EXERCICE #1\ (\emph{#2 points})}
\vspace{5mm}
}

\newcommand\img[2]{
          \includegraphics[width=#2\paperwidth]{\imgdir#1}
}

\newcommand\imgsvg[2]{
       \begin{center}   \includegraphics[width=#2\paperwidth]{\imgsvgdir#1} \end{center}
}


\newcommand\Lien[2]{
     \href{#1}{#2 \tiny \faExternalLink}
}
\newcommand\mcLien[2]{
     \href{https~://www.maths-cours.fr/#1}{#2 \tiny \faExternalLink}
}

\newcommand{\euro}{\eurologo{}}

%================================================================================================================================
%
% Macros - Environement
%
%================================================================================================================================

\newenvironment{tex}{ %
}
{%
}

\newenvironment{indente}{ %
	\setlength\parindent{10mm}
}

{
	\setlength\parindent{0mm}
}

\newenvironment{corrige}{%
     \needspace{3\baselineskip}
     \medskip
     \textbf{\textsc{Corrigé}}
     \medskip
}
{
}

\newenvironment{extern}{%
     \begin{center}
     }
     {
     \end{center}
}

\NewEnviron{code}{%
	\par
     \boite{gray}{\texttt{%
     \BODY
     }}
     \par
}

\newenvironment{vbloc}{% boite sans cadre empeche saut de page
     \begin{minipage}[t]{\linewidth}
     }
     {
     \end{minipage}
}
\NewEnviron{h2}{%
    \needspace{3\baselineskip}
    \vspace{0.6cm}
	\noindent \MakeUppercase{\sffamily \large \BODY}
	\vspace{1mm}\textcolor{mcgris}{\hrule}\vspace{0.4cm}
	\par
}{}

\NewEnviron{h3}{%
    \needspace{3\baselineskip}
	\vspace{5mm}
	\textsc{\BODY}
	\par
}

\NewEnviron{margeneg}{ %
\begin{addmargin}[-1cm]{0cm}
\BODY
\end{addmargin}
}

\NewEnviron{html}{%
}

\begin{document}
\meta{url}{/exercices/nombres-complexes-bac-s-asie-2018/}
\meta{pid}{9446}
\meta{titre}{Nombres complexes – Bac S Asie 2018}
\meta{type}{exercices}
%
\begin{h2}Exercice 4 (5 points)\end{h2}
\textbf{Candidats n'ayant pas choisi la spécialité \og mathématiques \fg{}}
\bigbreak
Dans cet exercice, $x$ et $y$ sont des nombres réels supérieurs à 1.
\medbreak
Dans le plan complexe muni d'un repère orthonormé direct $(O~;~\overrightarrow{i},~\overrightarrow{j}~,~\overrightarrow{k})$, on considère les points A, B
et C d'affixes respectives
\[z_{\text{A}} = 1 + \text{i}, \:  z_{\text{B}} = x + \text{i}\: \text{ et }\: z_{\text{C}} = y + \text{i}.\]
\begin{center}
     \begin{extern}%width="550" alt="Nombres complexes Bac S Asie 2018"
          \psset{unit=4cm}
          \begin{pspicture}(-0.2,-0.2)(3,1.1)
               \psaxes[linewidth=1pt]{->}(0,0)(1,1)
               \psframe[linecolor=lightgray](0,0)(3,1)
               \psline[linecolor=lightgray](1,0)(1,1)
               \psline[linecolor=lightgray](1.6,0)(1.6,1)
               \uput[d](0.5,0){$\overrightarrow{u}$}
               \uput[l](0,0.5){$\overrightarrow{v}$}
               \uput[u](1,1){A$(1 + \text{i})$}
               \uput[u](1.6,1){B$(x + \text{i})$}
               \uput[u](3,1){C$(y + \text{i})$}
               \uput[l](0,0){O}
               \psline(3,1)(0,0)(1.6,1)
               \psline(1,1)
          \end{pspicture}
     \end{extern}
\end{center}
\medbreak
\textbf{Problème~:} on cherche les valeurs éventuelles des réels $x$ et $y$, supérieures à 1, pour lesquelles~:
$\text{OC} = \text{OA} \times \text{OB} $ et $\left(\overrightarrow{u},~\overrightarrow{\text{OB}}\right) + \left(\overrightarrow{u},~\overrightarrow{\text{OC}}\right) = \left(\overrightarrow{u},~\overrightarrow{\text{OA}}\right).$
\medbreak
\begin{enumerate}
     \item Démontrer que si $\text{OC} = \text{OA} \times \text{OB}$, alors $y^2 = 2x^2 + 1$.
     \item Reproduire sur la copie et compléter l'algorithme ci-après pour qu'il affiche tous les couples $(x,~y)$ tels que~:
     \begin{center}
          $\left\{\begin{array}{l}
                    y^2 = 2x^2 + 1\\
                    x~\text{et}~y~\text{sont des nombres entiers} \\
                    1  \leqslant x \leqslant 10 ~\text{et}~ 1 \leqslant y \leqslant 10
          \end{array}\right.$
     \end{center}
     \begin{center}
          \begin{extern}%width="250" alt="algorithme Bac S Asie 2018"
               \begin{tabular}{|l|}\hline
                    Pour $x$ allant de 1 à \ldots faire\\
                    \hspace{0.5cm}Pour \ldots\\
                    \hspace{1cm}Si \ldots\\
                    \hspace{1.5cm}Afficher $x$ et $y$\\
                    \hspace{1cm}Fin Si\\
                    \hspace{0.5cm}Fin Pour\\
                    Fin Pour\\ \hline
               \end{tabular}
          \end{extern}
     \end{center}
     \emph{Lorsque l'on exécute cet algorithme, il affiche la valeur $2$ pour la variable $x$ et la valeur $3$ pour la variable $y$.}
     \smallbreak
     \item Étude d'un cas particulier~: dans cette question seulement, on prend $x = 2$ et $y = 3$.
     \begin{enumerate}[label=\alph*.]
          \item Donner le module et un argument de $z_{\text{A}}$.
          \item Montrer que $\text{OC} = \text{OA} \times \text{OB}$.
          \item Montrer que $z_{\text{B}}z_{\text{C}} = 5 z_{\text{A}}$ et en déduire que :
          $\left(\overrightarrow{u},~\overrightarrow{\text{OB}}\right) + \left(\overrightarrow{u},~\overrightarrow{\text{OC}}\right)$\nosp$ = \left(\overrightarrow{u},~\overrightarrow{\text{OA}}\right)$.
     \end{enumerate}
     \item On revient au cas général, et on cherche s'il existe d'autres valeurs des réels $x$ et $y$ telles que les points A, B et C vérifient les deux conditions~:
     \par
     $\text{OC} = \text{OA} \times \text{OB} \quad$ et $\:  \left(\overrightarrow{u},~\overrightarrow{\text{OB}}\right) + \left(\overrightarrow{u},~\overrightarrow{\text{OC}}\right)$\nosp$ = \left(\overrightarrow{u},~\overrightarrow{\text{OA}}\right)$.
     \par
     On rappelle que si $\text{OC} = \text{OA} \times \text{OB}$, alors $y^2 = 2x^2 + 1$ (question 1.).
     \begin{enumerate}[label=\alph*.]
          \item Démontrer que si $\left(\overrightarrow{u},~\overrightarrow{\text{OB}}\right) + \left(\overrightarrow{u},~\overrightarrow{\text{OC}}\right)$\nosp$ = \left(\overrightarrow{u},~\overrightarrow{\text{OA}}\right)$, alors arg$\left[\dfrac{(x + \text{i})(y + \text{i})}{1 + \text{i}}\right] = 0$\nosp$ \:\text{mod }\: 2\pi$.
          \par
          En déduire que sous cette condition~: $x + y - xy + 1 = 0$.
          \item Démontrer que si les deux conditions sont vérifiées et que de plus $x \neq 1$, alors~:
          \begin{center}
               $y= \sqrt{2x^2 + 1}\quad $ et $\: y = \dfrac{x + 1}{x - 1}.$
          \end{center}
     \end{enumerate}
     \item On définit les fonctions $f$ et $g$ sur l'intervalle $]1~;~+ \infty[$ par~:
     \begin{center}
          $f(x) = \sqrt{2x^2 + 1}\quad$ et $\: g(x) = \dfrac{x + 1}{x - 1}.$
     \end{center}
     Déterminer le nombre de solutions du problème initial.
     \par
     On pourra utiliser la fonction $h$ définie sur l'intervalle $]1~;~+ \infty[$ par $h(x) = f(x) - g(x)$ et s'appuyer sur la copie d'écran d'un logiciel de calcul formel donnée ci-dessous.
     \begin{center}
          \begin{extern}%width="230" alt="Calcul formel Bac S Asie 2018"
               \renewcommand{\arraystretch}{1.5}
               \begin{tabular}{|l|}
                    \hline
                    $f(x)~:= \text{sqrt}(2*x\verb+^+2+1)$	\\
                    \hspace{1.5cm}$x \to  \sqrt{2*x^2+1}$	\\ \hline
                    deriver$(f)$							\\
                    \hspace{1.5cm}$x \to \dfrac{2*x}{\sqrt{2*x^2+ 1}}$ \\[0.3cm] \hline
                    $g(x)~:=(x+1)/(x-1)$					\\
                    \hspace{1.5cm}$x \to \dfrac{x + 1}{x - 1}$ \\[0.2cm] \hline
                    deriver$(g)$							\\
                    \hspace{1.5cm}$x \to  - \dfrac{2}{(x - 1)^2}$ \\[0.2cm] \hline
               \end{tabular}
          \end{extern}
     \end{center}
\end{enumerate}

\end{document}
µ
\documentclass[a4paper]{article}

%================================================================================================================================
%
% Packages
%
%================================================================================================================================

\usepackage[T1]{fontenc} 	% pour caractères accentués
\usepackage[utf8]{inputenc}  % encodage utf8
\usepackage[french]{babel}	% langue : français
\usepackage{fourier}			% caractères plus lisibles
\usepackage[dvipsnames]{xcolor} % couleurs
\usepackage{fancyhdr}		% réglage header footer
\usepackage{needspace}		% empêcher sauts de page mal placés
\usepackage{graphicx}		% pour inclure des graphiques
\usepackage{enumitem,cprotect}		% personnalise les listes d'items (nécessaire pour ol, al ...)
\usepackage{hyperref}		% Liens hypertexte
\usepackage{pstricks,pst-all,pst-node,pstricks-add,pst-math,pst-plot,pst-tree,pst-eucl} % pstricks
\usepackage[a4paper,includeheadfoot,top=2cm,left=3cm, bottom=2cm,right=3cm]{geometry} % marges etc.
\usepackage{comment}			% commentaires multilignes
\usepackage{amsmath,environ} % maths (matrices, etc.)
\usepackage{amssymb,makeidx}
\usepackage{bm}				% bold maths
\usepackage{tabularx}		% tableaux
\usepackage{colortbl}		% tableaux en couleur
\usepackage{fontawesome}		% Fontawesome
\usepackage{environ}			% environment with command
\usepackage{fp}				% calculs pour ps-tricks
\usepackage{multido}			% pour ps tricks
\usepackage[np]{numprint}	% formattage nombre
\usepackage{tikz,tkz-tab} 			% package principal TikZ
\usepackage{pgfplots}   % axes
\usepackage{mathrsfs}    % cursives
\usepackage{calc}			% calcul taille boites
\usepackage[scaled=0.875]{helvet} % font sans serif
\usepackage{svg} % svg
\usepackage{scrextend} % local margin
\usepackage{scratch} %scratch
\usepackage{multicol} % colonnes
%\usepackage{infix-RPN,pst-func} % formule en notation polanaise inversée
\usepackage{listings}

%================================================================================================================================
%
% Réglages de base
%
%================================================================================================================================

\lstset{
language=Python,   % R code
literate=
{á}{{\'a}}1
{à}{{\`a}}1
{ã}{{\~a}}1
{é}{{\'e}}1
{è}{{\`e}}1
{ê}{{\^e}}1
{í}{{\'i}}1
{ó}{{\'o}}1
{õ}{{\~o}}1
{ú}{{\'u}}1
{ü}{{\"u}}1
{ç}{{\c{c}}}1
{~}{{ }}1
}


\definecolor{codegreen}{rgb}{0,0.6,0}
\definecolor{codegray}{rgb}{0.5,0.5,0.5}
\definecolor{codepurple}{rgb}{0.58,0,0.82}
\definecolor{backcolour}{rgb}{0.95,0.95,0.92}

\lstdefinestyle{mystyle}{
    backgroundcolor=\color{backcolour},   
    commentstyle=\color{codegreen},
    keywordstyle=\color{magenta},
    numberstyle=\tiny\color{codegray},
    stringstyle=\color{codepurple},
    basicstyle=\ttfamily\footnotesize,
    breakatwhitespace=false,         
    breaklines=true,                 
    captionpos=b,                    
    keepspaces=true,                 
    numbers=left,                    
xleftmargin=2em,
framexleftmargin=2em,            
    showspaces=false,                
    showstringspaces=false,
    showtabs=false,                  
    tabsize=2,
    upquote=true
}

\lstset{style=mystyle}


\lstset{style=mystyle}
\newcommand{\imgdir}{C:/laragon/www/newmc/assets/imgsvg/}
\newcommand{\imgsvgdir}{C:/laragon/www/newmc/assets/imgsvg/}

\definecolor{mcgris}{RGB}{220, 220, 220}% ancien~; pour compatibilité
\definecolor{mcbleu}{RGB}{52, 152, 219}
\definecolor{mcvert}{RGB}{125, 194, 70}
\definecolor{mcmauve}{RGB}{154, 0, 215}
\definecolor{mcorange}{RGB}{255, 96, 0}
\definecolor{mcturquoise}{RGB}{0, 153, 153}
\definecolor{mcrouge}{RGB}{255, 0, 0}
\definecolor{mclightvert}{RGB}{205, 234, 190}

\definecolor{gris}{RGB}{220, 220, 220}
\definecolor{bleu}{RGB}{52, 152, 219}
\definecolor{vert}{RGB}{125, 194, 70}
\definecolor{mauve}{RGB}{154, 0, 215}
\definecolor{orange}{RGB}{255, 96, 0}
\definecolor{turquoise}{RGB}{0, 153, 153}
\definecolor{rouge}{RGB}{255, 0, 0}
\definecolor{lightvert}{RGB}{205, 234, 190}
\setitemize[0]{label=\color{lightvert}  $\bullet$}

\pagestyle{fancy}
\renewcommand{\headrulewidth}{0.2pt}
\fancyhead[L]{maths-cours.fr}
\fancyhead[R]{\thepage}
\renewcommand{\footrulewidth}{0.2pt}
\fancyfoot[C]{}

\newcolumntype{C}{>{\centering\arraybackslash}X}
\newcolumntype{s}{>{\hsize=.35\hsize\arraybackslash}X}

\setlength{\parindent}{0pt}		 
\setlength{\parskip}{3mm}
\setlength{\headheight}{1cm}

\def\ebook{ebook}
\def\book{book}
\def\web{web}
\def\type{web}

\newcommand{\vect}[1]{\overrightarrow{\,\mathstrut#1\,}}

\def\Oij{$\left(\text{O}~;~\vect{\imath},~\vect{\jmath}\right)$}
\def\Oijk{$\left(\text{O}~;~\vect{\imath},~\vect{\jmath},~\vect{k}\right)$}
\def\Ouv{$\left(\text{O}~;~\vect{u},~\vect{v}\right)$}

\hypersetup{breaklinks=true, colorlinks = true, linkcolor = OliveGreen, urlcolor = OliveGreen, citecolor = OliveGreen, pdfauthor={Didier BONNEL - https://www.maths-cours.fr} } % supprime les bordures autour des liens

\renewcommand{\arg}[0]{\text{arg}}

\everymath{\displaystyle}

%================================================================================================================================
%
% Macros - Commandes
%
%================================================================================================================================

\newcommand\meta[2]{    			% Utilisé pour créer le post HTML.
	\def\titre{titre}
	\def\url{url}
	\def\arg{#1}
	\ifx\titre\arg
		\newcommand\maintitle{#2}
		\fancyhead[L]{#2}
		{\Large\sffamily \MakeUppercase{#2}}
		\vspace{1mm}\textcolor{mcvert}{\hrule}
	\fi 
	\ifx\url\arg
		\fancyfoot[L]{\href{https://www.maths-cours.fr#2}{\black \footnotesize{https://www.maths-cours.fr#2}}}
	\fi 
}


\newcommand\TitreC[1]{    		% Titre centré
     \needspace{3\baselineskip}
     \begin{center}\textbf{#1}\end{center}
}

\newcommand\newpar{    		% paragraphe
     \par
}

\newcommand\nosp {    		% commande vide (pas d'espace)
}
\newcommand{\id}[1]{} %ignore

\newcommand\boite[2]{				% Boite simple sans titre
	\vspace{5mm}
	\setlength{\fboxrule}{0.2mm}
	\setlength{\fboxsep}{5mm}	
	\fcolorbox{#1}{#1!3}{\makebox[\linewidth-2\fboxrule-2\fboxsep]{
  		\begin{minipage}[t]{\linewidth-2\fboxrule-4\fboxsep}\setlength{\parskip}{3mm}
  			 #2
  		\end{minipage}
	}}
	\vspace{5mm}
}

\newcommand\CBox[4]{				% Boites
	\vspace{5mm}
	\setlength{\fboxrule}{0.2mm}
	\setlength{\fboxsep}{5mm}
	
	\fcolorbox{#1}{#1!3}{\makebox[\linewidth-2\fboxrule-2\fboxsep]{
		\begin{minipage}[t]{1cm}\setlength{\parskip}{3mm}
	  		\textcolor{#1}{\LARGE{#2}}    
 	 	\end{minipage}  
  		\begin{minipage}[t]{\linewidth-2\fboxrule-4\fboxsep}\setlength{\parskip}{3mm}
			\raisebox{1.2mm}{\normalsize\sffamily{\textcolor{#1}{#3}}}						
  			 #4
  		\end{minipage}
	}}
	\vspace{5mm}
}

\newcommand\cadre[3]{				% Boites convertible html
	\par
	\vspace{2mm}
	\setlength{\fboxrule}{0.1mm}
	\setlength{\fboxsep}{5mm}
	\fcolorbox{#1}{white}{\makebox[\linewidth-2\fboxrule-2\fboxsep]{
  		\begin{minipage}[t]{\linewidth-2\fboxrule-4\fboxsep}\setlength{\parskip}{3mm}
			\raisebox{-2.5mm}{\sffamily \small{\textcolor{#1}{\MakeUppercase{#2}}}}		
			\par		
  			 #3
 	 		\end{minipage}
	}}
		\vspace{2mm}
	\par
}

\newcommand\bloc[3]{				% Boites convertible html sans bordure
     \needspace{2\baselineskip}
     {\sffamily \small{\textcolor{#1}{\MakeUppercase{#2}}}}    
		\par		
  			 #3
		\par
}

\newcommand\CHelp[1]{
     \CBox{Plum}{\faInfoCircle}{À RETENIR}{#1}
}

\newcommand\CUp[1]{
     \CBox{NavyBlue}{\faThumbsOUp}{EN PRATIQUE}{#1}
}

\newcommand\CInfo[1]{
     \CBox{Sepia}{\faArrowCircleRight}{REMARQUE}{#1}
}

\newcommand\CRedac[1]{
     \CBox{PineGreen}{\faEdit}{BIEN R\'EDIGER}{#1}
}

\newcommand\CError[1]{
     \CBox{Red}{\faExclamationTriangle}{ATTENTION}{#1}
}

\newcommand\TitreExo[2]{
\needspace{4\baselineskip}
 {\sffamily\large EXERCICE #1\ (\emph{#2 points})}
\vspace{5mm}
}

\newcommand\img[2]{
          \includegraphics[width=#2\paperwidth]{\imgdir#1}
}

\newcommand\imgsvg[2]{
       \begin{center}   \includegraphics[width=#2\paperwidth]{\imgsvgdir#1} \end{center}
}


\newcommand\Lien[2]{
     \href{#1}{#2 \tiny \faExternalLink}
}
\newcommand\mcLien[2]{
     \href{https~://www.maths-cours.fr/#1}{#2 \tiny \faExternalLink}
}

\newcommand{\euro}{\eurologo{}}

%================================================================================================================================
%
% Macros - Environement
%
%================================================================================================================================

\newenvironment{tex}{ %
}
{%
}

\newenvironment{indente}{ %
	\setlength\parindent{10mm}
}

{
	\setlength\parindent{0mm}
}

\newenvironment{corrige}{%
     \needspace{3\baselineskip}
     \medskip
     \textbf{\textsc{Corrigé}}
     \medskip
}
{
}

\newenvironment{extern}{%
     \begin{center}
     }
     {
     \end{center}
}

\NewEnviron{code}{%
	\par
     \boite{gray}{\texttt{%
     \BODY
     }}
     \par
}

\newenvironment{vbloc}{% boite sans cadre empeche saut de page
     \begin{minipage}[t]{\linewidth}
     }
     {
     \end{minipage}
}
\NewEnviron{h2}{%
    \needspace{3\baselineskip}
    \vspace{0.6cm}
	\noindent \MakeUppercase{\sffamily \large \BODY}
	\vspace{1mm}\textcolor{mcgris}{\hrule}\vspace{0.4cm}
	\par
}{}

\NewEnviron{h3}{%
    \needspace{3\baselineskip}
	\vspace{5mm}
	\textsc{\BODY}
	\par
}

\NewEnviron{margeneg}{ %
\begin{addmargin}[-1cm]{0cm}
\BODY
\end{addmargin}
}

\NewEnviron{html}{%
}

\begin{document}
\meta{url}{/exercices/arithmetique-bac-s-asie-2018-spe/}
\meta{pid}{9448}
\meta{titre}{Arithmétique – Bac S Asie 2018 (spé)}
\meta{type}{exercices}
%
\begin{h2}Exercice 4 (5 points)\end{h2}
\textbf{Candidats ayant  choisi la spécialité \og mathématiques \fg{}}
\bigbreak
On s'intéresse à la figure suivante, dans laquelle $a$, $b$ et $c$ désignent les longueurs des hypoténuses des trois triangles rectangles en O dessinés ci-dessous.
\begin{center}
     \begin{extern}%width="400" alt="Arithmétique – Bac S Asie 2018 (spé)"
          \psset{unit=3cm}
          \begin{pspicture}(-0.2,-0.8)(3.4,1)
               %\psgrid
               \pspolygon(0,0)(1,0)(0,1)
               \psline(1,0)(1.3,0)\psline(1.7,0)(2.1,0)\psline(2.1,0)(2.4,0)
               \psline(2.7,0)(3.2,0)
               \psline(0,1)(2.1,0)\psline(0,1)(3.2,0)
               \psframe(0.1,0.1)
               \psline[linestyle=dashed](1,0)(1,-0.2)\psline{<->}(0,-0.2)(1,-0.2)\uput[u](0.5,-0.2){1}
               \psline[linestyle=dashed](2.1,0)(2.1,-0.4)\psline{<->}(0,-0.4)(2.1,-0.4)\uput[u](1.05,-0.4){$u$}
               \psline[linestyle=dashed](3.2,0)(3.2,-0.6)\psline{<->}(0,-0.6)(3.2,-0.6)\uput[u](1.6,-0.6){$v$}
               \uput[ur](0.57,0.4){$a$}\uput[ur](1.12,0.45){$b$}\uput[ur](1.5,0.55){$c$}
               \psline[linestyle=dashed](0,0)(-0.2,0)\psline[linestyle=dashed](0,1)(-0.2,1)
               \psline{<->}(-0.2,0)(-0.2,1)\uput[l](-0.2,0.5){1}
               \psline[linestyle=dotted](1.2,0)(1.7,0)
               \psline[linestyle=dotted](2.2,0)(2.7,0)
               \psline[linestyle=dashed](0,0)(0,-0.6)
          \end{pspicture}
     \end{extern}
\end{center}
\textbf{Problème~:} on cherche les couples de \textbf{nombres entiers naturels non nuls} $(u,~v)$ tels que $ab = c$.
\medbreak
\begin{enumerate}
     \item \textbf{Modélisation}
     \par
     Démontrer que les solutions du problème sont des solutions de l'équation~:
     \begin{center}
          $(E)~:\quad  v^2 - 2u^2 = 1\quad$  ($v$  et $u$ étant des entiers naturels non nuls).
     \end{center}
     \item  \textbf{Recherche systématique de solutions de l'équation $(E)$}
     \par
     Recopier et compléter l'algorithme suivant pour qu'il affiche au cours de son exécution tous les couples solutions de l'équation pour lesquels $1 \leqslant u \leqslant 1~000$ et $1 \leqslant v \leqslant 1~000$.
     \begin{center}
          \begin{extern}%width="230" alt="Algorithme Bac S Asie 2018 (spé)"
               \begin{tabular}{|l|}\hline
                    Pour $u$ allant de 1 à \ldots faire\\
                    \hspace{0.5cm}Pour \ldots\\
                    \hspace{1cm}Si \ldots\\
                    \hspace{1.5cm}Afficher $u$ et $v$\\
                    \hspace{1cm}Fin Si\\
                    \hspace{0.5cm}Fin Pour\\
                    Fin Pour\\ \hline
               \end{tabular}
          \end{extern}
     \end{center}
     Au cours de son exécution, l'algorithme affiche~:\\
     \begin{indent}
          2 \quad 3\\
          12 \quad 17\\
          70 \quad 99\\
          408 \quad 577\\
     \end{indent}
     \item \textbf{Analyse des solutions éventuelles de l'équation }$(E)$
     \par
     On suppose que le couple $(u,~v)$ est une solution de l'équation $(E)$.
     \begin{enumerate}[label=\alph*.]
          \item Établir que $u < v$.
          \item  Démontrer que $n$ et $n^2$ ont la même parité pour tout entier naturel $n$.
          \item  Démontrer que $v$ est un nombre impair.
          \item  Établir que $2u^2 =(v-1)(v+1)$.
          \par
          En déduire que $u$ est un nombre pair.
     \end{enumerate}
     \item \textbf{ Une famille de solutions}
     \par
     On assimile un couple de nombres entiers $(u,~v)$ à la matrice colonne $X = \begin{pmatrix}u\\v\end{pmatrix}.$
     On définit également la matrice $A = \begin{pmatrix}3&2\\4&3\end{pmatrix}$.
     \begin{enumerate}[label=\alph*.]
          \item Démontrer que si une matrice colonne $X$ est une solution de l'équation $(E)$, alors $AX$ est aussi une solution de l'équation $(E)$.
          \item Démontrer que si une matrice colonne $X$ est une solution de l'équation $(E)$, alors pour tout entier naturel $n$,\: $A^n X$ est aussi une solution de l'équation $(E)$.
          \item À l'aide de la calculatrice, donner un couple $(u,~v)$ solution de l'équation $(E)$ tel que $v > 10~000$.
     \end{enumerate}
\end{enumerate}

\end{document}
µ
\documentclass[a4paper]{article}

%================================================================================================================================
%
% Packages
%
%================================================================================================================================

\usepackage[T1]{fontenc} 	% pour caractères accentués
\usepackage[utf8]{inputenc}  % encodage utf8
\usepackage[french]{babel}	% langue : français
\usepackage{fourier}			% caractères plus lisibles
\usepackage[dvipsnames]{xcolor} % couleurs
\usepackage{fancyhdr}		% réglage header footer
\usepackage{needspace}		% empêcher sauts de page mal placés
\usepackage{graphicx}		% pour inclure des graphiques
\usepackage{enumitem,cprotect}		% personnalise les listes d'items (nécessaire pour ol, al ...)
\usepackage{hyperref}		% Liens hypertexte
\usepackage{pstricks,pst-all,pst-node,pstricks-add,pst-math,pst-plot,pst-tree,pst-eucl} % pstricks
\usepackage[a4paper,includeheadfoot,top=2cm,left=3cm, bottom=2cm,right=3cm]{geometry} % marges etc.
\usepackage{comment}			% commentaires multilignes
\usepackage{amsmath,environ} % maths (matrices, etc.)
\usepackage{amssymb,makeidx}
\usepackage{bm}				% bold maths
\usepackage{tabularx}		% tableaux
\usepackage{colortbl}		% tableaux en couleur
\usepackage{fontawesome}		% Fontawesome
\usepackage{environ}			% environment with command
\usepackage{fp}				% calculs pour ps-tricks
\usepackage{multido}			% pour ps tricks
\usepackage[np]{numprint}	% formattage nombre
\usepackage{tikz,tkz-tab} 			% package principal TikZ
\usepackage{pgfplots}   % axes
\usepackage{mathrsfs}    % cursives
\usepackage{calc}			% calcul taille boites
\usepackage[scaled=0.875]{helvet} % font sans serif
\usepackage{svg} % svg
\usepackage{scrextend} % local margin
\usepackage{scratch} %scratch
\usepackage{multicol} % colonnes
%\usepackage{infix-RPN,pst-func} % formule en notation polanaise inversée
\usepackage{listings}

%================================================================================================================================
%
% Réglages de base
%
%================================================================================================================================

\lstset{
language=Python,   % R code
literate=
{á}{{\'a}}1
{à}{{\`a}}1
{ã}{{\~a}}1
{é}{{\'e}}1
{è}{{\`e}}1
{ê}{{\^e}}1
{í}{{\'i}}1
{ó}{{\'o}}1
{õ}{{\~o}}1
{ú}{{\'u}}1
{ü}{{\"u}}1
{ç}{{\c{c}}}1
{~}{{ }}1
}


\definecolor{codegreen}{rgb}{0,0.6,0}
\definecolor{codegray}{rgb}{0.5,0.5,0.5}
\definecolor{codepurple}{rgb}{0.58,0,0.82}
\definecolor{backcolour}{rgb}{0.95,0.95,0.92}

\lstdefinestyle{mystyle}{
    backgroundcolor=\color{backcolour},   
    commentstyle=\color{codegreen},
    keywordstyle=\color{magenta},
    numberstyle=\tiny\color{codegray},
    stringstyle=\color{codepurple},
    basicstyle=\ttfamily\footnotesize,
    breakatwhitespace=false,         
    breaklines=true,                 
    captionpos=b,                    
    keepspaces=true,                 
    numbers=left,                    
xleftmargin=2em,
framexleftmargin=2em,            
    showspaces=false,                
    showstringspaces=false,
    showtabs=false,                  
    tabsize=2,
    upquote=true
}

\lstset{style=mystyle}


\lstset{style=mystyle}
\newcommand{\imgdir}{C:/laragon/www/newmc/assets/imgsvg/}
\newcommand{\imgsvgdir}{C:/laragon/www/newmc/assets/imgsvg/}

\definecolor{mcgris}{RGB}{220, 220, 220}% ancien~; pour compatibilité
\definecolor{mcbleu}{RGB}{52, 152, 219}
\definecolor{mcvert}{RGB}{125, 194, 70}
\definecolor{mcmauve}{RGB}{154, 0, 215}
\definecolor{mcorange}{RGB}{255, 96, 0}
\definecolor{mcturquoise}{RGB}{0, 153, 153}
\definecolor{mcrouge}{RGB}{255, 0, 0}
\definecolor{mclightvert}{RGB}{205, 234, 190}

\definecolor{gris}{RGB}{220, 220, 220}
\definecolor{bleu}{RGB}{52, 152, 219}
\definecolor{vert}{RGB}{125, 194, 70}
\definecolor{mauve}{RGB}{154, 0, 215}
\definecolor{orange}{RGB}{255, 96, 0}
\definecolor{turquoise}{RGB}{0, 153, 153}
\definecolor{rouge}{RGB}{255, 0, 0}
\definecolor{lightvert}{RGB}{205, 234, 190}
\setitemize[0]{label=\color{lightvert}  $\bullet$}

\pagestyle{fancy}
\renewcommand{\headrulewidth}{0.2pt}
\fancyhead[L]{maths-cours.fr}
\fancyhead[R]{\thepage}
\renewcommand{\footrulewidth}{0.2pt}
\fancyfoot[C]{}

\newcolumntype{C}{>{\centering\arraybackslash}X}
\newcolumntype{s}{>{\hsize=.35\hsize\arraybackslash}X}

\setlength{\parindent}{0pt}		 
\setlength{\parskip}{3mm}
\setlength{\headheight}{1cm}

\def\ebook{ebook}
\def\book{book}
\def\web{web}
\def\type{web}

\newcommand{\vect}[1]{\overrightarrow{\,\mathstrut#1\,}}

\def\Oij{$\left(\text{O}~;~\vect{\imath},~\vect{\jmath}\right)$}
\def\Oijk{$\left(\text{O}~;~\vect{\imath},~\vect{\jmath},~\vect{k}\right)$}
\def\Ouv{$\left(\text{O}~;~\vect{u},~\vect{v}\right)$}

\hypersetup{breaklinks=true, colorlinks = true, linkcolor = OliveGreen, urlcolor = OliveGreen, citecolor = OliveGreen, pdfauthor={Didier BONNEL - https://www.maths-cours.fr} } % supprime les bordures autour des liens

\renewcommand{\arg}[0]{\text{arg}}

\everymath{\displaystyle}

%================================================================================================================================
%
% Macros - Commandes
%
%================================================================================================================================

\newcommand\meta[2]{    			% Utilisé pour créer le post HTML.
	\def\titre{titre}
	\def\url{url}
	\def\arg{#1}
	\ifx\titre\arg
		\newcommand\maintitle{#2}
		\fancyhead[L]{#2}
		{\Large\sffamily \MakeUppercase{#2}}
		\vspace{1mm}\textcolor{mcvert}{\hrule}
	\fi 
	\ifx\url\arg
		\fancyfoot[L]{\href{https://www.maths-cours.fr#2}{\black \footnotesize{https://www.maths-cours.fr#2}}}
	\fi 
}


\newcommand\TitreC[1]{    		% Titre centré
     \needspace{3\baselineskip}
     \begin{center}\textbf{#1}\end{center}
}

\newcommand\newpar{    		% paragraphe
     \par
}

\newcommand\nosp {    		% commande vide (pas d'espace)
}
\newcommand{\id}[1]{} %ignore

\newcommand\boite[2]{				% Boite simple sans titre
	\vspace{5mm}
	\setlength{\fboxrule}{0.2mm}
	\setlength{\fboxsep}{5mm}	
	\fcolorbox{#1}{#1!3}{\makebox[\linewidth-2\fboxrule-2\fboxsep]{
  		\begin{minipage}[t]{\linewidth-2\fboxrule-4\fboxsep}\setlength{\parskip}{3mm}
  			 #2
  		\end{minipage}
	}}
	\vspace{5mm}
}

\newcommand\CBox[4]{				% Boites
	\vspace{5mm}
	\setlength{\fboxrule}{0.2mm}
	\setlength{\fboxsep}{5mm}
	
	\fcolorbox{#1}{#1!3}{\makebox[\linewidth-2\fboxrule-2\fboxsep]{
		\begin{minipage}[t]{1cm}\setlength{\parskip}{3mm}
	  		\textcolor{#1}{\LARGE{#2}}    
 	 	\end{minipage}  
  		\begin{minipage}[t]{\linewidth-2\fboxrule-4\fboxsep}\setlength{\parskip}{3mm}
			\raisebox{1.2mm}{\normalsize\sffamily{\textcolor{#1}{#3}}}						
  			 #4
  		\end{minipage}
	}}
	\vspace{5mm}
}

\newcommand\cadre[3]{				% Boites convertible html
	\par
	\vspace{2mm}
	\setlength{\fboxrule}{0.1mm}
	\setlength{\fboxsep}{5mm}
	\fcolorbox{#1}{white}{\makebox[\linewidth-2\fboxrule-2\fboxsep]{
  		\begin{minipage}[t]{\linewidth-2\fboxrule-4\fboxsep}\setlength{\parskip}{3mm}
			\raisebox{-2.5mm}{\sffamily \small{\textcolor{#1}{\MakeUppercase{#2}}}}		
			\par		
  			 #3
 	 		\end{minipage}
	}}
		\vspace{2mm}
	\par
}

\newcommand\bloc[3]{				% Boites convertible html sans bordure
     \needspace{2\baselineskip}
     {\sffamily \small{\textcolor{#1}{\MakeUppercase{#2}}}}    
		\par		
  			 #3
		\par
}

\newcommand\CHelp[1]{
     \CBox{Plum}{\faInfoCircle}{À RETENIR}{#1}
}

\newcommand\CUp[1]{
     \CBox{NavyBlue}{\faThumbsOUp}{EN PRATIQUE}{#1}
}

\newcommand\CInfo[1]{
     \CBox{Sepia}{\faArrowCircleRight}{REMARQUE}{#1}
}

\newcommand\CRedac[1]{
     \CBox{PineGreen}{\faEdit}{BIEN R\'EDIGER}{#1}
}

\newcommand\CError[1]{
     \CBox{Red}{\faExclamationTriangle}{ATTENTION}{#1}
}

\newcommand\TitreExo[2]{
\needspace{4\baselineskip}
 {\sffamily\large EXERCICE #1\ (\emph{#2 points})}
\vspace{5mm}
}

\newcommand\img[2]{
          \includegraphics[width=#2\paperwidth]{\imgdir#1}
}

\newcommand\imgsvg[2]{
       \begin{center}   \includegraphics[width=#2\paperwidth]{\imgsvgdir#1} \end{center}
}


\newcommand\Lien[2]{
     \href{#1}{#2 \tiny \faExternalLink}
}
\newcommand\mcLien[2]{
     \href{https~://www.maths-cours.fr/#1}{#2 \tiny \faExternalLink}
}

\newcommand{\euro}{\eurologo{}}

%================================================================================================================================
%
% Macros - Environement
%
%================================================================================================================================

\newenvironment{tex}{ %
}
{%
}

\newenvironment{indente}{ %
	\setlength\parindent{10mm}
}

{
	\setlength\parindent{0mm}
}

\newenvironment{corrige}{%
     \needspace{3\baselineskip}
     \medskip
     \textbf{\textsc{Corrigé}}
     \medskip
}
{
}

\newenvironment{extern}{%
     \begin{center}
     }
     {
     \end{center}
}

\NewEnviron{code}{%
	\par
     \boite{gray}{\texttt{%
     \BODY
     }}
     \par
}

\newenvironment{vbloc}{% boite sans cadre empeche saut de page
     \begin{minipage}[t]{\linewidth}
     }
     {
     \end{minipage}
}
\NewEnviron{h2}{%
    \needspace{3\baselineskip}
    \vspace{0.6cm}
	\noindent \MakeUppercase{\sffamily \large \BODY}
	\vspace{1mm}\textcolor{mcgris}{\hrule}\vspace{0.4cm}
	\par
}{}

\NewEnviron{h3}{%
    \needspace{3\baselineskip}
	\vspace{5mm}
	\textsc{\BODY}
	\par
}

\NewEnviron{margeneg}{ %
\begin{addmargin}[-1cm]{0cm}
\BODY
\end{addmargin}
}

\NewEnviron{html}{%
}

\begin{document}
\meta{url}{/exercices/volumes-brevet-metropole-2018/}
\meta{pid}{9484}
\meta{titre}{Volumes - Brevet Métropole 2018}
\meta{type}{exercices}
%
\begin{h2}Exercice 1 (11 points)\end{h2}
\medbreak
Le gros globe de cristal est un trophée attribué au vainqueur de la coupe du monde de ski.
\par
Ce trophée pèse 9 kg et mesure 46 cm de hauteur.
\begin{center}
     \img{globe}{0.1}%width="130" alt="trophée Brevet Métropole 2018"
\end{center}
\begin{enumerate}
     \item Le biathlète français Martin Fourcade a remporté le sixième gros globe de cristal de
     sa carrière en 2017 à Pyeongchang en Corée du Sud.
     \par
     Donner approximativement la latitude et la longitude de ce lieu repéré sur la carte
     ci-dessous.
     \medbreak
     \begin{center}
          \img{mappemonde}{0.6}%width="700" alt="Mappemonde Brevet Métropole 2018"
     \end{center}
     \item On considère que ce globe est composé d'un cylindre en cristal de
     diamètre 6cm, surmonté d'une boule de cristal. Voir schéma ci -dessous.
     \begin{center}
          \begin{extern}%width="180" alt="globe Brevet Métropole 2018"
               \begin{pspicture}(-0.5,0)(5.3,7.5)
                    \psset{unit=0.6cm,arrowsize=2pt 4}
                    %\psgrid
                    \psframe(2.8,0.8)(3.7,4.2)
                    \pscircle(3.25,5.85){1.65}
                    \psline{<->}(1.3,4.2)(1.3,7.45)\uput[l](1.3,5.8125){23~cm}
                    \psline{<->}(1.3,0.8)(1.3,4.2)\uput[l](1.3,2.5){23~cm}
                    \psline[linestyle=dashed](1.2,0.8)(5.3,0.8)
                    \psline[linestyle=dashed](1.2,4.2)(5.3,4.2)
                    \psline[linestyle=dashed](1.2,7.45)(5.3,7.5)
               \end{pspicture}
          \end{extern}
     \end{center}
     Montrer qu'une valeur approchée du volume de la boule de ce trophée
     est de $6~371~\text{cm}^3$.
     \item Marie affirme que le volume de la boule de cristal représente environ
     90\,\% du volume total du trophée.
     \par
     A-t-elle raison~?
\end{enumerate}
\medbreak
Rappels~:
\par
Volume d'une boule de rayon $R$~: $V = \frac{4}{3}\pi R^3$
\par
Volume d'un cylindre de rayon $r$ et de hauteur $h~: V =\pi r^2h$.

\end{document}
µ
\documentclass[a4paper]{article}

%================================================================================================================================
%
% Packages
%
%================================================================================================================================

\usepackage[T1]{fontenc} 	% pour caractères accentués
\usepackage[utf8]{inputenc}  % encodage utf8
\usepackage[french]{babel}	% langue : français
\usepackage{fourier}			% caractères plus lisibles
\usepackage[dvipsnames]{xcolor} % couleurs
\usepackage{fancyhdr}		% réglage header footer
\usepackage{needspace}		% empêcher sauts de page mal placés
\usepackage{graphicx}		% pour inclure des graphiques
\usepackage{enumitem,cprotect}		% personnalise les listes d'items (nécessaire pour ol, al ...)
\usepackage{hyperref}		% Liens hypertexte
\usepackage{pstricks,pst-all,pst-node,pstricks-add,pst-math,pst-plot,pst-tree,pst-eucl} % pstricks
\usepackage[a4paper,includeheadfoot,top=2cm,left=3cm, bottom=2cm,right=3cm]{geometry} % marges etc.
\usepackage{comment}			% commentaires multilignes
\usepackage{amsmath,environ} % maths (matrices, etc.)
\usepackage{amssymb,makeidx}
\usepackage{bm}				% bold maths
\usepackage{tabularx}		% tableaux
\usepackage{colortbl}		% tableaux en couleur
\usepackage{fontawesome}		% Fontawesome
\usepackage{environ}			% environment with command
\usepackage{fp}				% calculs pour ps-tricks
\usepackage{multido}			% pour ps tricks
\usepackage[np]{numprint}	% formattage nombre
\usepackage{tikz,tkz-tab} 			% package principal TikZ
\usepackage{pgfplots}   % axes
\usepackage{mathrsfs}    % cursives
\usepackage{calc}			% calcul taille boites
\usepackage[scaled=0.875]{helvet} % font sans serif
\usepackage{svg} % svg
\usepackage{scrextend} % local margin
\usepackage{scratch} %scratch
\usepackage{multicol} % colonnes
%\usepackage{infix-RPN,pst-func} % formule en notation polanaise inversée
\usepackage{listings}

%================================================================================================================================
%
% Réglages de base
%
%================================================================================================================================

\lstset{
language=Python,   % R code
literate=
{á}{{\'a}}1
{à}{{\`a}}1
{ã}{{\~a}}1
{é}{{\'e}}1
{è}{{\`e}}1
{ê}{{\^e}}1
{í}{{\'i}}1
{ó}{{\'o}}1
{õ}{{\~o}}1
{ú}{{\'u}}1
{ü}{{\"u}}1
{ç}{{\c{c}}}1
{~}{{ }}1
}


\definecolor{codegreen}{rgb}{0,0.6,0}
\definecolor{codegray}{rgb}{0.5,0.5,0.5}
\definecolor{codepurple}{rgb}{0.58,0,0.82}
\definecolor{backcolour}{rgb}{0.95,0.95,0.92}

\lstdefinestyle{mystyle}{
    backgroundcolor=\color{backcolour},   
    commentstyle=\color{codegreen},
    keywordstyle=\color{magenta},
    numberstyle=\tiny\color{codegray},
    stringstyle=\color{codepurple},
    basicstyle=\ttfamily\footnotesize,
    breakatwhitespace=false,         
    breaklines=true,                 
    captionpos=b,                    
    keepspaces=true,                 
    numbers=left,                    
xleftmargin=2em,
framexleftmargin=2em,            
    showspaces=false,                
    showstringspaces=false,
    showtabs=false,                  
    tabsize=2,
    upquote=true
}

\lstset{style=mystyle}


\lstset{style=mystyle}
\newcommand{\imgdir}{C:/laragon/www/newmc/assets/imgsvg/}
\newcommand{\imgsvgdir}{C:/laragon/www/newmc/assets/imgsvg/}

\definecolor{mcgris}{RGB}{220, 220, 220}% ancien~; pour compatibilité
\definecolor{mcbleu}{RGB}{52, 152, 219}
\definecolor{mcvert}{RGB}{125, 194, 70}
\definecolor{mcmauve}{RGB}{154, 0, 215}
\definecolor{mcorange}{RGB}{255, 96, 0}
\definecolor{mcturquoise}{RGB}{0, 153, 153}
\definecolor{mcrouge}{RGB}{255, 0, 0}
\definecolor{mclightvert}{RGB}{205, 234, 190}

\definecolor{gris}{RGB}{220, 220, 220}
\definecolor{bleu}{RGB}{52, 152, 219}
\definecolor{vert}{RGB}{125, 194, 70}
\definecolor{mauve}{RGB}{154, 0, 215}
\definecolor{orange}{RGB}{255, 96, 0}
\definecolor{turquoise}{RGB}{0, 153, 153}
\definecolor{rouge}{RGB}{255, 0, 0}
\definecolor{lightvert}{RGB}{205, 234, 190}
\setitemize[0]{label=\color{lightvert}  $\bullet$}

\pagestyle{fancy}
\renewcommand{\headrulewidth}{0.2pt}
\fancyhead[L]{maths-cours.fr}
\fancyhead[R]{\thepage}
\renewcommand{\footrulewidth}{0.2pt}
\fancyfoot[C]{}

\newcolumntype{C}{>{\centering\arraybackslash}X}
\newcolumntype{s}{>{\hsize=.35\hsize\arraybackslash}X}

\setlength{\parindent}{0pt}		 
\setlength{\parskip}{3mm}
\setlength{\headheight}{1cm}

\def\ebook{ebook}
\def\book{book}
\def\web{web}
\def\type{web}

\newcommand{\vect}[1]{\overrightarrow{\,\mathstrut#1\,}}

\def\Oij{$\left(\text{O}~;~\vect{\imath},~\vect{\jmath}\right)$}
\def\Oijk{$\left(\text{O}~;~\vect{\imath},~\vect{\jmath},~\vect{k}\right)$}
\def\Ouv{$\left(\text{O}~;~\vect{u},~\vect{v}\right)$}

\hypersetup{breaklinks=true, colorlinks = true, linkcolor = OliveGreen, urlcolor = OliveGreen, citecolor = OliveGreen, pdfauthor={Didier BONNEL - https://www.maths-cours.fr} } % supprime les bordures autour des liens

\renewcommand{\arg}[0]{\text{arg}}

\everymath{\displaystyle}

%================================================================================================================================
%
% Macros - Commandes
%
%================================================================================================================================

\newcommand\meta[2]{    			% Utilisé pour créer le post HTML.
	\def\titre{titre}
	\def\url{url}
	\def\arg{#1}
	\ifx\titre\arg
		\newcommand\maintitle{#2}
		\fancyhead[L]{#2}
		{\Large\sffamily \MakeUppercase{#2}}
		\vspace{1mm}\textcolor{mcvert}{\hrule}
	\fi 
	\ifx\url\arg
		\fancyfoot[L]{\href{https://www.maths-cours.fr#2}{\black \footnotesize{https://www.maths-cours.fr#2}}}
	\fi 
}


\newcommand\TitreC[1]{    		% Titre centré
     \needspace{3\baselineskip}
     \begin{center}\textbf{#1}\end{center}
}

\newcommand\newpar{    		% paragraphe
     \par
}

\newcommand\nosp {    		% commande vide (pas d'espace)
}
\newcommand{\id}[1]{} %ignore

\newcommand\boite[2]{				% Boite simple sans titre
	\vspace{5mm}
	\setlength{\fboxrule}{0.2mm}
	\setlength{\fboxsep}{5mm}	
	\fcolorbox{#1}{#1!3}{\makebox[\linewidth-2\fboxrule-2\fboxsep]{
  		\begin{minipage}[t]{\linewidth-2\fboxrule-4\fboxsep}\setlength{\parskip}{3mm}
  			 #2
  		\end{minipage}
	}}
	\vspace{5mm}
}

\newcommand\CBox[4]{				% Boites
	\vspace{5mm}
	\setlength{\fboxrule}{0.2mm}
	\setlength{\fboxsep}{5mm}
	
	\fcolorbox{#1}{#1!3}{\makebox[\linewidth-2\fboxrule-2\fboxsep]{
		\begin{minipage}[t]{1cm}\setlength{\parskip}{3mm}
	  		\textcolor{#1}{\LARGE{#2}}    
 	 	\end{minipage}  
  		\begin{minipage}[t]{\linewidth-2\fboxrule-4\fboxsep}\setlength{\parskip}{3mm}
			\raisebox{1.2mm}{\normalsize\sffamily{\textcolor{#1}{#3}}}						
  			 #4
  		\end{minipage}
	}}
	\vspace{5mm}
}

\newcommand\cadre[3]{				% Boites convertible html
	\par
	\vspace{2mm}
	\setlength{\fboxrule}{0.1mm}
	\setlength{\fboxsep}{5mm}
	\fcolorbox{#1}{white}{\makebox[\linewidth-2\fboxrule-2\fboxsep]{
  		\begin{minipage}[t]{\linewidth-2\fboxrule-4\fboxsep}\setlength{\parskip}{3mm}
			\raisebox{-2.5mm}{\sffamily \small{\textcolor{#1}{\MakeUppercase{#2}}}}		
			\par		
  			 #3
 	 		\end{minipage}
	}}
		\vspace{2mm}
	\par
}

\newcommand\bloc[3]{				% Boites convertible html sans bordure
     \needspace{2\baselineskip}
     {\sffamily \small{\textcolor{#1}{\MakeUppercase{#2}}}}    
		\par		
  			 #3
		\par
}

\newcommand\CHelp[1]{
     \CBox{Plum}{\faInfoCircle}{À RETENIR}{#1}
}

\newcommand\CUp[1]{
     \CBox{NavyBlue}{\faThumbsOUp}{EN PRATIQUE}{#1}
}

\newcommand\CInfo[1]{
     \CBox{Sepia}{\faArrowCircleRight}{REMARQUE}{#1}
}

\newcommand\CRedac[1]{
     \CBox{PineGreen}{\faEdit}{BIEN R\'EDIGER}{#1}
}

\newcommand\CError[1]{
     \CBox{Red}{\faExclamationTriangle}{ATTENTION}{#1}
}

\newcommand\TitreExo[2]{
\needspace{4\baselineskip}
 {\sffamily\large EXERCICE #1\ (\emph{#2 points})}
\vspace{5mm}
}

\newcommand\img[2]{
          \includegraphics[width=#2\paperwidth]{\imgdir#1}
}

\newcommand\imgsvg[2]{
       \begin{center}   \includegraphics[width=#2\paperwidth]{\imgsvgdir#1} \end{center}
}


\newcommand\Lien[2]{
     \href{#1}{#2 \tiny \faExternalLink}
}
\newcommand\mcLien[2]{
     \href{https~://www.maths-cours.fr/#1}{#2 \tiny \faExternalLink}
}

\newcommand{\euro}{\eurologo{}}

%================================================================================================================================
%
% Macros - Environement
%
%================================================================================================================================

\newenvironment{tex}{ %
}
{%
}

\newenvironment{indente}{ %
	\setlength\parindent{10mm}
}

{
	\setlength\parindent{0mm}
}

\newenvironment{corrige}{%
     \needspace{3\baselineskip}
     \medskip
     \textbf{\textsc{Corrigé}}
     \medskip
}
{
}

\newenvironment{extern}{%
     \begin{center}
     }
     {
     \end{center}
}

\NewEnviron{code}{%
	\par
     \boite{gray}{\texttt{%
     \BODY
     }}
     \par
}

\newenvironment{vbloc}{% boite sans cadre empeche saut de page
     \begin{minipage}[t]{\linewidth}
     }
     {
     \end{minipage}
}
\NewEnviron{h2}{%
    \needspace{3\baselineskip}
    \vspace{0.6cm}
	\noindent \MakeUppercase{\sffamily \large \BODY}
	\vspace{1mm}\textcolor{mcgris}{\hrule}\vspace{0.4cm}
	\par
}{}

\NewEnviron{h3}{%
    \needspace{3\baselineskip}
	\vspace{5mm}
	\textsc{\BODY}
	\par
}

\NewEnviron{margeneg}{ %
\begin{addmargin}[-1cm]{0cm}
\BODY
\end{addmargin}
}

\NewEnviron{html}{%
}

\begin{document}
\meta{url}{/exercices/statistiques-brevet-metropole-2018/}
\meta{pid}{9500}
\meta{titre}{Statistiques - Brevet Métropole 2018}
\meta{type}{exercices}
%
\begin{h2}Exercice 2 (14 points)\end{h2}
\medbreak
Parmi les nombreux polluants de l'air, les particules fines sont régulièrement surveillées.
\par
Les PM10 sont des particules fines dont le diamètre est inférieur à 0,01 mm.
\par
En janvier 2017, les villes de Lyon et Grenoble ont connu un épisode de pollution aux particules fines. Voici des données concernant la période du 16 au 25 janvier 2017~:
\medbreak
\textbf{Données statistiques sur les concentrations journalières en PM10 du
16 au 25 janvier 2017 à Lyon~:}
\par
\begin{indent}
     Moyenne~: 72,5 $\mu$g/m$^3$\\
     Médiane~: 83,5 $\mu$g/m$^3$\\
     Concentration minimale~: 22 $\mu$g/m$^3$\\
     Concentration maximale~: 107 $\mu$g/m$^3$\\
\end{indent}
\emph{(Source~: http~://www.air-rhonealpesfr)}
\bigbreak
\textbf{Relevés des concentrations journalières en PM10 du 16 au 25 janvier 2017 à
Grenoble~:}\\
\begin{center}
     \begin{tabular}{|l|c|}\hline%class="compact"
          Date 		&~~Concentration~~PM10  en $\mu\text{g/m}^3$\\ \hline
          16 janvier 	&32\\ \hline
          17 janvier 	&39\\ \hline
          18 janvier 	&52\\ \hline
          19 janvier 	&57\\ \hline
          20 janvier 	&78\\ \hline
          21 janvier 	&63\\ \hline
          22 janvier 	&60\\ \hline
          23 janvier 	&82\\ \hline
          24 janvier 	&82\\ \hline
          25 janvier 	&89\\ \hline
     \end{tabular}
\end{center}
\medbreak
\begin{enumerate}
     \item Laquelle de ces deux villes a eu la plus forte concentration moyenne en PM10 entre le 16 et le 25 janvier~?
     \item Calculer l'étendue des séries des relevés en PM10 à Lyon et à Grenoble. Laquelle de ces deux villes a eu l'étendue la plus importante~?
     \par
     Interpréter ce dernier résultat.
     \item L'affirmation suivante est-elle exacte~? Justifier votre réponse.
     \par
     \og Du 16 au 25 janvier, le seuil d'alerte de 80$\mu\text{g/m}^3$ par jour a été dépassé au moins 5 fois à Lyon \fg.
\end{enumerate}

\end{document}
µ
\documentclass[a4paper]{article}

%================================================================================================================================
%
% Packages
%
%================================================================================================================================

\usepackage[T1]{fontenc} 	% pour caractères accentués
\usepackage[utf8]{inputenc}  % encodage utf8
\usepackage[french]{babel}	% langue : français
\usepackage{fourier}			% caractères plus lisibles
\usepackage[dvipsnames]{xcolor} % couleurs
\usepackage{fancyhdr}		% réglage header footer
\usepackage{needspace}		% empêcher sauts de page mal placés
\usepackage{graphicx}		% pour inclure des graphiques
\usepackage{enumitem,cprotect}		% personnalise les listes d'items (nécessaire pour ol, al ...)
\usepackage{hyperref}		% Liens hypertexte
\usepackage{pstricks,pst-all,pst-node,pstricks-add,pst-math,pst-plot,pst-tree,pst-eucl} % pstricks
\usepackage[a4paper,includeheadfoot,top=2cm,left=3cm, bottom=2cm,right=3cm]{geometry} % marges etc.
\usepackage{comment}			% commentaires multilignes
\usepackage{amsmath,environ} % maths (matrices, etc.)
\usepackage{amssymb,makeidx}
\usepackage{bm}				% bold maths
\usepackage{tabularx}		% tableaux
\usepackage{colortbl}		% tableaux en couleur
\usepackage{fontawesome}		% Fontawesome
\usepackage{environ}			% environment with command
\usepackage{fp}				% calculs pour ps-tricks
\usepackage{multido}			% pour ps tricks
\usepackage[np]{numprint}	% formattage nombre
\usepackage{tikz,tkz-tab} 			% package principal TikZ
\usepackage{pgfplots}   % axes
\usepackage{mathrsfs}    % cursives
\usepackage{calc}			% calcul taille boites
\usepackage[scaled=0.875]{helvet} % font sans serif
\usepackage{svg} % svg
\usepackage{scrextend} % local margin
\usepackage{scratch} %scratch
\usepackage{multicol} % colonnes
%\usepackage{infix-RPN,pst-func} % formule en notation polanaise inversée
\usepackage{listings}

%================================================================================================================================
%
% Réglages de base
%
%================================================================================================================================

\lstset{
language=Python,   % R code
literate=
{á}{{\'a}}1
{à}{{\`a}}1
{ã}{{\~a}}1
{é}{{\'e}}1
{è}{{\`e}}1
{ê}{{\^e}}1
{í}{{\'i}}1
{ó}{{\'o}}1
{õ}{{\~o}}1
{ú}{{\'u}}1
{ü}{{\"u}}1
{ç}{{\c{c}}}1
{~}{{ }}1
}


\definecolor{codegreen}{rgb}{0,0.6,0}
\definecolor{codegray}{rgb}{0.5,0.5,0.5}
\definecolor{codepurple}{rgb}{0.58,0,0.82}
\definecolor{backcolour}{rgb}{0.95,0.95,0.92}

\lstdefinestyle{mystyle}{
    backgroundcolor=\color{backcolour},   
    commentstyle=\color{codegreen},
    keywordstyle=\color{magenta},
    numberstyle=\tiny\color{codegray},
    stringstyle=\color{codepurple},
    basicstyle=\ttfamily\footnotesize,
    breakatwhitespace=false,         
    breaklines=true,                 
    captionpos=b,                    
    keepspaces=true,                 
    numbers=left,                    
xleftmargin=2em,
framexleftmargin=2em,            
    showspaces=false,                
    showstringspaces=false,
    showtabs=false,                  
    tabsize=2,
    upquote=true
}

\lstset{style=mystyle}


\lstset{style=mystyle}
\newcommand{\imgdir}{C:/laragon/www/newmc/assets/imgsvg/}
\newcommand{\imgsvgdir}{C:/laragon/www/newmc/assets/imgsvg/}

\definecolor{mcgris}{RGB}{220, 220, 220}% ancien~; pour compatibilité
\definecolor{mcbleu}{RGB}{52, 152, 219}
\definecolor{mcvert}{RGB}{125, 194, 70}
\definecolor{mcmauve}{RGB}{154, 0, 215}
\definecolor{mcorange}{RGB}{255, 96, 0}
\definecolor{mcturquoise}{RGB}{0, 153, 153}
\definecolor{mcrouge}{RGB}{255, 0, 0}
\definecolor{mclightvert}{RGB}{205, 234, 190}

\definecolor{gris}{RGB}{220, 220, 220}
\definecolor{bleu}{RGB}{52, 152, 219}
\definecolor{vert}{RGB}{125, 194, 70}
\definecolor{mauve}{RGB}{154, 0, 215}
\definecolor{orange}{RGB}{255, 96, 0}
\definecolor{turquoise}{RGB}{0, 153, 153}
\definecolor{rouge}{RGB}{255, 0, 0}
\definecolor{lightvert}{RGB}{205, 234, 190}
\setitemize[0]{label=\color{lightvert}  $\bullet$}

\pagestyle{fancy}
\renewcommand{\headrulewidth}{0.2pt}
\fancyhead[L]{maths-cours.fr}
\fancyhead[R]{\thepage}
\renewcommand{\footrulewidth}{0.2pt}
\fancyfoot[C]{}

\newcolumntype{C}{>{\centering\arraybackslash}X}
\newcolumntype{s}{>{\hsize=.35\hsize\arraybackslash}X}

\setlength{\parindent}{0pt}		 
\setlength{\parskip}{3mm}
\setlength{\headheight}{1cm}

\def\ebook{ebook}
\def\book{book}
\def\web{web}
\def\type{web}

\newcommand{\vect}[1]{\overrightarrow{\,\mathstrut#1\,}}

\def\Oij{$\left(\text{O}~;~\vect{\imath},~\vect{\jmath}\right)$}
\def\Oijk{$\left(\text{O}~;~\vect{\imath},~\vect{\jmath},~\vect{k}\right)$}
\def\Ouv{$\left(\text{O}~;~\vect{u},~\vect{v}\right)$}

\hypersetup{breaklinks=true, colorlinks = true, linkcolor = OliveGreen, urlcolor = OliveGreen, citecolor = OliveGreen, pdfauthor={Didier BONNEL - https://www.maths-cours.fr} } % supprime les bordures autour des liens

\renewcommand{\arg}[0]{\text{arg}}

\everymath{\displaystyle}

%================================================================================================================================
%
% Macros - Commandes
%
%================================================================================================================================

\newcommand\meta[2]{    			% Utilisé pour créer le post HTML.
	\def\titre{titre}
	\def\url{url}
	\def\arg{#1}
	\ifx\titre\arg
		\newcommand\maintitle{#2}
		\fancyhead[L]{#2}
		{\Large\sffamily \MakeUppercase{#2}}
		\vspace{1mm}\textcolor{mcvert}{\hrule}
	\fi 
	\ifx\url\arg
		\fancyfoot[L]{\href{https://www.maths-cours.fr#2}{\black \footnotesize{https://www.maths-cours.fr#2}}}
	\fi 
}


\newcommand\TitreC[1]{    		% Titre centré
     \needspace{3\baselineskip}
     \begin{center}\textbf{#1}\end{center}
}

\newcommand\newpar{    		% paragraphe
     \par
}

\newcommand\nosp {    		% commande vide (pas d'espace)
}
\newcommand{\id}[1]{} %ignore

\newcommand\boite[2]{				% Boite simple sans titre
	\vspace{5mm}
	\setlength{\fboxrule}{0.2mm}
	\setlength{\fboxsep}{5mm}	
	\fcolorbox{#1}{#1!3}{\makebox[\linewidth-2\fboxrule-2\fboxsep]{
  		\begin{minipage}[t]{\linewidth-2\fboxrule-4\fboxsep}\setlength{\parskip}{3mm}
  			 #2
  		\end{minipage}
	}}
	\vspace{5mm}
}

\newcommand\CBox[4]{				% Boites
	\vspace{5mm}
	\setlength{\fboxrule}{0.2mm}
	\setlength{\fboxsep}{5mm}
	
	\fcolorbox{#1}{#1!3}{\makebox[\linewidth-2\fboxrule-2\fboxsep]{
		\begin{minipage}[t]{1cm}\setlength{\parskip}{3mm}
	  		\textcolor{#1}{\LARGE{#2}}    
 	 	\end{minipage}  
  		\begin{minipage}[t]{\linewidth-2\fboxrule-4\fboxsep}\setlength{\parskip}{3mm}
			\raisebox{1.2mm}{\normalsize\sffamily{\textcolor{#1}{#3}}}						
  			 #4
  		\end{minipage}
	}}
	\vspace{5mm}
}

\newcommand\cadre[3]{				% Boites convertible html
	\par
	\vspace{2mm}
	\setlength{\fboxrule}{0.1mm}
	\setlength{\fboxsep}{5mm}
	\fcolorbox{#1}{white}{\makebox[\linewidth-2\fboxrule-2\fboxsep]{
  		\begin{minipage}[t]{\linewidth-2\fboxrule-4\fboxsep}\setlength{\parskip}{3mm}
			\raisebox{-2.5mm}{\sffamily \small{\textcolor{#1}{\MakeUppercase{#2}}}}		
			\par		
  			 #3
 	 		\end{minipage}
	}}
		\vspace{2mm}
	\par
}

\newcommand\bloc[3]{				% Boites convertible html sans bordure
     \needspace{2\baselineskip}
     {\sffamily \small{\textcolor{#1}{\MakeUppercase{#2}}}}    
		\par		
  			 #3
		\par
}

\newcommand\CHelp[1]{
     \CBox{Plum}{\faInfoCircle}{À RETENIR}{#1}
}

\newcommand\CUp[1]{
     \CBox{NavyBlue}{\faThumbsOUp}{EN PRATIQUE}{#1}
}

\newcommand\CInfo[1]{
     \CBox{Sepia}{\faArrowCircleRight}{REMARQUE}{#1}
}

\newcommand\CRedac[1]{
     \CBox{PineGreen}{\faEdit}{BIEN R\'EDIGER}{#1}
}

\newcommand\CError[1]{
     \CBox{Red}{\faExclamationTriangle}{ATTENTION}{#1}
}

\newcommand\TitreExo[2]{
\needspace{4\baselineskip}
 {\sffamily\large EXERCICE #1\ (\emph{#2 points})}
\vspace{5mm}
}

\newcommand\img[2]{
          \includegraphics[width=#2\paperwidth]{\imgdir#1}
}

\newcommand\imgsvg[2]{
       \begin{center}   \includegraphics[width=#2\paperwidth]{\imgsvgdir#1} \end{center}
}


\newcommand\Lien[2]{
     \href{#1}{#2 \tiny \faExternalLink}
}
\newcommand\mcLien[2]{
     \href{https~://www.maths-cours.fr/#1}{#2 \tiny \faExternalLink}
}

\newcommand{\euro}{\eurologo{}}

%================================================================================================================================
%
% Macros - Environement
%
%================================================================================================================================

\newenvironment{tex}{ %
}
{%
}

\newenvironment{indente}{ %
	\setlength\parindent{10mm}
}

{
	\setlength\parindent{0mm}
}

\newenvironment{corrige}{%
     \needspace{3\baselineskip}
     \medskip
     \textbf{\textsc{Corrigé}}
     \medskip
}
{
}

\newenvironment{extern}{%
     \begin{center}
     }
     {
     \end{center}
}

\NewEnviron{code}{%
	\par
     \boite{gray}{\texttt{%
     \BODY
     }}
     \par
}

\newenvironment{vbloc}{% boite sans cadre empeche saut de page
     \begin{minipage}[t]{\linewidth}
     }
     {
     \end{minipage}
}
\NewEnviron{h2}{%
    \needspace{3\baselineskip}
    \vspace{0.6cm}
	\noindent \MakeUppercase{\sffamily \large \BODY}
	\vspace{1mm}\textcolor{mcgris}{\hrule}\vspace{0.4cm}
	\par
}{}

\NewEnviron{h3}{%
    \needspace{3\baselineskip}
	\vspace{5mm}
	\textsc{\BODY}
	\par
}

\NewEnviron{margeneg}{ %
\begin{addmargin}[-1cm]{0cm}
\BODY
\end{addmargin}
}

\NewEnviron{html}{%
}

\begin{document}
\meta{url}{/exercices/probabilites-brevet-metropole-2018/}
\meta{pid}{9508}
\meta{titre}{Probabilités - Brevet Métropole 2018}
\meta{type}{exercices}
%
\begin{h2}Exercice 3 (12 points)\end{h2}
\medbreak
Dans son lecteur audio, Théo a téléchargé 375 morceaux de musique. Parmi eux, il y a 125 morceaux de rap. Il appuie sur la touche \og lecture aléatoire\fg{} qui lui permet d'écouter un morceau choisi au hasard parmi tous les morceaux disponibles.
\medbreak
\begin{enumerate}
     \item Quelle est la probabilité qu'il écoute du rap~?
     \item La probabilité qu'il écoute du rock est égale à $\frac{7}{15}$.
     \par
     Combien Théo a-t-il de morceaux de rock dans son lecteur audio~?
     \item  Alice possède 40\,\% de morceaux de rock dans son lecteur audio.
     \par
     Si Théo et Alice appuient tous les deux sur la touche \og lecture aléatoire\fg{} de leur lecteur audio, lequel a le plus de chances d'écouter un morceau de rock~?
\end{enumerate}

\end{document}
µ
\documentclass[a4paper]{article}

%================================================================================================================================
%
% Packages
%
%================================================================================================================================

\usepackage[T1]{fontenc} 	% pour caractères accentués
\usepackage[utf8]{inputenc}  % encodage utf8
\usepackage[french]{babel}	% langue : français
\usepackage{fourier}			% caractères plus lisibles
\usepackage[dvipsnames]{xcolor} % couleurs
\usepackage{fancyhdr}		% réglage header footer
\usepackage{needspace}		% empêcher sauts de page mal placés
\usepackage{graphicx}		% pour inclure des graphiques
\usepackage{enumitem,cprotect}		% personnalise les listes d'items (nécessaire pour ol, al ...)
\usepackage{hyperref}		% Liens hypertexte
\usepackage{pstricks,pst-all,pst-node,pstricks-add,pst-math,pst-plot,pst-tree,pst-eucl} % pstricks
\usepackage[a4paper,includeheadfoot,top=2cm,left=3cm, bottom=2cm,right=3cm]{geometry} % marges etc.
\usepackage{comment}			% commentaires multilignes
\usepackage{amsmath,environ} % maths (matrices, etc.)
\usepackage{amssymb,makeidx}
\usepackage{bm}				% bold maths
\usepackage{tabularx}		% tableaux
\usepackage{colortbl}		% tableaux en couleur
\usepackage{fontawesome}		% Fontawesome
\usepackage{environ}			% environment with command
\usepackage{fp}				% calculs pour ps-tricks
\usepackage{multido}			% pour ps tricks
\usepackage[np]{numprint}	% formattage nombre
\usepackage{tikz,tkz-tab} 			% package principal TikZ
\usepackage{pgfplots}   % axes
\usepackage{mathrsfs}    % cursives
\usepackage{calc}			% calcul taille boites
\usepackage[scaled=0.875]{helvet} % font sans serif
\usepackage{svg} % svg
\usepackage{scrextend} % local margin
\usepackage{scratch} %scratch
\usepackage{multicol} % colonnes
%\usepackage{infix-RPN,pst-func} % formule en notation polanaise inversée
\usepackage{listings}

%================================================================================================================================
%
% Réglages de base
%
%================================================================================================================================

\lstset{
language=Python,   % R code
literate=
{á}{{\'a}}1
{à}{{\`a}}1
{ã}{{\~a}}1
{é}{{\'e}}1
{è}{{\`e}}1
{ê}{{\^e}}1
{í}{{\'i}}1
{ó}{{\'o}}1
{õ}{{\~o}}1
{ú}{{\'u}}1
{ü}{{\"u}}1
{ç}{{\c{c}}}1
{~}{{ }}1
}


\definecolor{codegreen}{rgb}{0,0.6,0}
\definecolor{codegray}{rgb}{0.5,0.5,0.5}
\definecolor{codepurple}{rgb}{0.58,0,0.82}
\definecolor{backcolour}{rgb}{0.95,0.95,0.92}

\lstdefinestyle{mystyle}{
    backgroundcolor=\color{backcolour},   
    commentstyle=\color{codegreen},
    keywordstyle=\color{magenta},
    numberstyle=\tiny\color{codegray},
    stringstyle=\color{codepurple},
    basicstyle=\ttfamily\footnotesize,
    breakatwhitespace=false,         
    breaklines=true,                 
    captionpos=b,                    
    keepspaces=true,                 
    numbers=left,                    
xleftmargin=2em,
framexleftmargin=2em,            
    showspaces=false,                
    showstringspaces=false,
    showtabs=false,                  
    tabsize=2,
    upquote=true
}

\lstset{style=mystyle}


\lstset{style=mystyle}
\newcommand{\imgdir}{C:/laragon/www/newmc/assets/imgsvg/}
\newcommand{\imgsvgdir}{C:/laragon/www/newmc/assets/imgsvg/}

\definecolor{mcgris}{RGB}{220, 220, 220}% ancien~; pour compatibilité
\definecolor{mcbleu}{RGB}{52, 152, 219}
\definecolor{mcvert}{RGB}{125, 194, 70}
\definecolor{mcmauve}{RGB}{154, 0, 215}
\definecolor{mcorange}{RGB}{255, 96, 0}
\definecolor{mcturquoise}{RGB}{0, 153, 153}
\definecolor{mcrouge}{RGB}{255, 0, 0}
\definecolor{mclightvert}{RGB}{205, 234, 190}

\definecolor{gris}{RGB}{220, 220, 220}
\definecolor{bleu}{RGB}{52, 152, 219}
\definecolor{vert}{RGB}{125, 194, 70}
\definecolor{mauve}{RGB}{154, 0, 215}
\definecolor{orange}{RGB}{255, 96, 0}
\definecolor{turquoise}{RGB}{0, 153, 153}
\definecolor{rouge}{RGB}{255, 0, 0}
\definecolor{lightvert}{RGB}{205, 234, 190}
\setitemize[0]{label=\color{lightvert}  $\bullet$}

\pagestyle{fancy}
\renewcommand{\headrulewidth}{0.2pt}
\fancyhead[L]{maths-cours.fr}
\fancyhead[R]{\thepage}
\renewcommand{\footrulewidth}{0.2pt}
\fancyfoot[C]{}

\newcolumntype{C}{>{\centering\arraybackslash}X}
\newcolumntype{s}{>{\hsize=.35\hsize\arraybackslash}X}

\setlength{\parindent}{0pt}		 
\setlength{\parskip}{3mm}
\setlength{\headheight}{1cm}

\def\ebook{ebook}
\def\book{book}
\def\web{web}
\def\type{web}

\newcommand{\vect}[1]{\overrightarrow{\,\mathstrut#1\,}}

\def\Oij{$\left(\text{O}~;~\vect{\imath},~\vect{\jmath}\right)$}
\def\Oijk{$\left(\text{O}~;~\vect{\imath},~\vect{\jmath},~\vect{k}\right)$}
\def\Ouv{$\left(\text{O}~;~\vect{u},~\vect{v}\right)$}

\hypersetup{breaklinks=true, colorlinks = true, linkcolor = OliveGreen, urlcolor = OliveGreen, citecolor = OliveGreen, pdfauthor={Didier BONNEL - https://www.maths-cours.fr} } % supprime les bordures autour des liens

\renewcommand{\arg}[0]{\text{arg}}

\everymath{\displaystyle}

%================================================================================================================================
%
% Macros - Commandes
%
%================================================================================================================================

\newcommand\meta[2]{    			% Utilisé pour créer le post HTML.
	\def\titre{titre}
	\def\url{url}
	\def\arg{#1}
	\ifx\titre\arg
		\newcommand\maintitle{#2}
		\fancyhead[L]{#2}
		{\Large\sffamily \MakeUppercase{#2}}
		\vspace{1mm}\textcolor{mcvert}{\hrule}
	\fi 
	\ifx\url\arg
		\fancyfoot[L]{\href{https://www.maths-cours.fr#2}{\black \footnotesize{https://www.maths-cours.fr#2}}}
	\fi 
}


\newcommand\TitreC[1]{    		% Titre centré
     \needspace{3\baselineskip}
     \begin{center}\textbf{#1}\end{center}
}

\newcommand\newpar{    		% paragraphe
     \par
}

\newcommand\nosp {    		% commande vide (pas d'espace)
}
\newcommand{\id}[1]{} %ignore

\newcommand\boite[2]{				% Boite simple sans titre
	\vspace{5mm}
	\setlength{\fboxrule}{0.2mm}
	\setlength{\fboxsep}{5mm}	
	\fcolorbox{#1}{#1!3}{\makebox[\linewidth-2\fboxrule-2\fboxsep]{
  		\begin{minipage}[t]{\linewidth-2\fboxrule-4\fboxsep}\setlength{\parskip}{3mm}
  			 #2
  		\end{minipage}
	}}
	\vspace{5mm}
}

\newcommand\CBox[4]{				% Boites
	\vspace{5mm}
	\setlength{\fboxrule}{0.2mm}
	\setlength{\fboxsep}{5mm}
	
	\fcolorbox{#1}{#1!3}{\makebox[\linewidth-2\fboxrule-2\fboxsep]{
		\begin{minipage}[t]{1cm}\setlength{\parskip}{3mm}
	  		\textcolor{#1}{\LARGE{#2}}    
 	 	\end{minipage}  
  		\begin{minipage}[t]{\linewidth-2\fboxrule-4\fboxsep}\setlength{\parskip}{3mm}
			\raisebox{1.2mm}{\normalsize\sffamily{\textcolor{#1}{#3}}}						
  			 #4
  		\end{minipage}
	}}
	\vspace{5mm}
}

\newcommand\cadre[3]{				% Boites convertible html
	\par
	\vspace{2mm}
	\setlength{\fboxrule}{0.1mm}
	\setlength{\fboxsep}{5mm}
	\fcolorbox{#1}{white}{\makebox[\linewidth-2\fboxrule-2\fboxsep]{
  		\begin{minipage}[t]{\linewidth-2\fboxrule-4\fboxsep}\setlength{\parskip}{3mm}
			\raisebox{-2.5mm}{\sffamily \small{\textcolor{#1}{\MakeUppercase{#2}}}}		
			\par		
  			 #3
 	 		\end{minipage}
	}}
		\vspace{2mm}
	\par
}

\newcommand\bloc[3]{				% Boites convertible html sans bordure
     \needspace{2\baselineskip}
     {\sffamily \small{\textcolor{#1}{\MakeUppercase{#2}}}}    
		\par		
  			 #3
		\par
}

\newcommand\CHelp[1]{
     \CBox{Plum}{\faInfoCircle}{À RETENIR}{#1}
}

\newcommand\CUp[1]{
     \CBox{NavyBlue}{\faThumbsOUp}{EN PRATIQUE}{#1}
}

\newcommand\CInfo[1]{
     \CBox{Sepia}{\faArrowCircleRight}{REMARQUE}{#1}
}

\newcommand\CRedac[1]{
     \CBox{PineGreen}{\faEdit}{BIEN R\'EDIGER}{#1}
}

\newcommand\CError[1]{
     \CBox{Red}{\faExclamationTriangle}{ATTENTION}{#1}
}

\newcommand\TitreExo[2]{
\needspace{4\baselineskip}
 {\sffamily\large EXERCICE #1\ (\emph{#2 points})}
\vspace{5mm}
}

\newcommand\img[2]{
          \includegraphics[width=#2\paperwidth]{\imgdir#1}
}

\newcommand\imgsvg[2]{
       \begin{center}   \includegraphics[width=#2\paperwidth]{\imgsvgdir#1} \end{center}
}


\newcommand\Lien[2]{
     \href{#1}{#2 \tiny \faExternalLink}
}
\newcommand\mcLien[2]{
     \href{https~://www.maths-cours.fr/#1}{#2 \tiny \faExternalLink}
}

\newcommand{\euro}{\eurologo{}}

%================================================================================================================================
%
% Macros - Environement
%
%================================================================================================================================

\newenvironment{tex}{ %
}
{%
}

\newenvironment{indente}{ %
	\setlength\parindent{10mm}
}

{
	\setlength\parindent{0mm}
}

\newenvironment{corrige}{%
     \needspace{3\baselineskip}
     \medskip
     \textbf{\textsc{Corrigé}}
     \medskip
}
{
}

\newenvironment{extern}{%
     \begin{center}
     }
     {
     \end{center}
}

\NewEnviron{code}{%
	\par
     \boite{gray}{\texttt{%
     \BODY
     }}
     \par
}

\newenvironment{vbloc}{% boite sans cadre empeche saut de page
     \begin{minipage}[t]{\linewidth}
     }
     {
     \end{minipage}
}
\NewEnviron{h2}{%
    \needspace{3\baselineskip}
    \vspace{0.6cm}
	\noindent \MakeUppercase{\sffamily \large \BODY}
	\vspace{1mm}\textcolor{mcgris}{\hrule}\vspace{0.4cm}
	\par
}{}

\NewEnviron{h3}{%
    \needspace{3\baselineskip}
	\vspace{5mm}
	\textsc{\BODY}
	\par
}

\NewEnviron{margeneg}{ %
\begin{addmargin}[-1cm]{0cm}
\BODY
\end{addmargin}
}

\NewEnviron{html}{%
}

\begin{document}
\meta{url}{/exercices/geometrie-plane-brevet-metropole-2018/}
\meta{pid}{9512}
\meta{titre}{Géométrie plane - Brevet Métropole 2018}
\meta{type}{exercices}
%
\begin{h2}Exercice 4 (14 points)\end{h2}
\medbreak
La figure ci-dessous n'est pas représentée en vraie
grandeur.
\par
Les points C, B et E sont alignés.
\par
Le triangle ABC est rectangle en A.
\par
Le triangle BDC est rectangle en B.
\begin{center}
     \begin{extern}%width="450" alt="Géométrie plane Brevet Métropole 2018"
          \psset{unit=1.2cm}
          \begin{pspicture}(10,5)
               %\psgrid
               \pspolygon(4.3,0.5)(4.3,2.3)(0.5,2.3)%DBC
               \pspolygon(9.3,2.3)(4.3,2.3)(8.2,4.7)%EBF
               \psline(0.5,2.3)(1.4,3.8)(4.3,2.3)%CAB
               \uput[u](4.3,2.3){B} \uput[l](0.5,2.3){C} \uput[d](4.3,0.5){D}
               \uput[r](9.3,2.3){E} \uput[u](1.4,3.8){A} \uput[u](8.2,4.7){F}
               \psarc(0.5,2.3){0.5}{0}{61}
               \uput[u](2.4,2.2){7,5~cm}\uput[u](6.8,2.2){6,8~cm}
               \rput{32}(6.1,3.6){6~cm}\rput{-65}(9,3.5){3,2~cm}
               \rput{-28}(2.4,1.2){8,5~cm}
               \psframe[fillstyle=solid](4.3,2.3)(4.1,2.1)
               \rput{-26}(1.4,3.8){\psframe[fillstyle=solid](0,0)(0.2,-0.2)}
               \rput(1.25,2.7){61\degres}
          \end{pspicture}
     \end{extern}
\end{center}
\medbreak
\begin{enumerate}
     \item Montrer que la longueur BD est égale à $4$~cm.
     \item Montrer que les triangles CBD et BFE sont semblables.
     \item Sophie affirme que l'angle $\widehat{\text{BFE}}$ est un angle droit. A-t-elle raison~?
     \item Max affirme que l'angle $\widehat{\text{ACD}}$ est un angle droit. A-t-il raison~?
\end{enumerate}

\end{document}
µ
\documentclass[a4paper]{article}

%================================================================================================================================
%
% Packages
%
%================================================================================================================================

\usepackage[T1]{fontenc} 	% pour caractères accentués
\usepackage[utf8]{inputenc}  % encodage utf8
\usepackage[french]{babel}	% langue : français
\usepackage{fourier}			% caractères plus lisibles
\usepackage[dvipsnames]{xcolor} % couleurs
\usepackage{fancyhdr}		% réglage header footer
\usepackage{needspace}		% empêcher sauts de page mal placés
\usepackage{graphicx}		% pour inclure des graphiques
\usepackage{enumitem,cprotect}		% personnalise les listes d'items (nécessaire pour ol, al ...)
\usepackage{hyperref}		% Liens hypertexte
\usepackage{pstricks,pst-all,pst-node,pstricks-add,pst-math,pst-plot,pst-tree,pst-eucl} % pstricks
\usepackage[a4paper,includeheadfoot,top=2cm,left=3cm, bottom=2cm,right=3cm]{geometry} % marges etc.
\usepackage{comment}			% commentaires multilignes
\usepackage{amsmath,environ} % maths (matrices, etc.)
\usepackage{amssymb,makeidx}
\usepackage{bm}				% bold maths
\usepackage{tabularx}		% tableaux
\usepackage{colortbl}		% tableaux en couleur
\usepackage{fontawesome}		% Fontawesome
\usepackage{environ}			% environment with command
\usepackage{fp}				% calculs pour ps-tricks
\usepackage{multido}			% pour ps tricks
\usepackage[np]{numprint}	% formattage nombre
\usepackage{tikz,tkz-tab} 			% package principal TikZ
\usepackage{pgfplots}   % axes
\usepackage{mathrsfs}    % cursives
\usepackage{calc}			% calcul taille boites
\usepackage[scaled=0.875]{helvet} % font sans serif
\usepackage{svg} % svg
\usepackage{scrextend} % local margin
\usepackage{scratch} %scratch
\usepackage{multicol} % colonnes
%\usepackage{infix-RPN,pst-func} % formule en notation polanaise inversée
\usepackage{listings}

%================================================================================================================================
%
% Réglages de base
%
%================================================================================================================================

\lstset{
language=Python,   % R code
literate=
{á}{{\'a}}1
{à}{{\`a}}1
{ã}{{\~a}}1
{é}{{\'e}}1
{è}{{\`e}}1
{ê}{{\^e}}1
{í}{{\'i}}1
{ó}{{\'o}}1
{õ}{{\~o}}1
{ú}{{\'u}}1
{ü}{{\"u}}1
{ç}{{\c{c}}}1
{~}{{ }}1
}


\definecolor{codegreen}{rgb}{0,0.6,0}
\definecolor{codegray}{rgb}{0.5,0.5,0.5}
\definecolor{codepurple}{rgb}{0.58,0,0.82}
\definecolor{backcolour}{rgb}{0.95,0.95,0.92}

\lstdefinestyle{mystyle}{
    backgroundcolor=\color{backcolour},   
    commentstyle=\color{codegreen},
    keywordstyle=\color{magenta},
    numberstyle=\tiny\color{codegray},
    stringstyle=\color{codepurple},
    basicstyle=\ttfamily\footnotesize,
    breakatwhitespace=false,         
    breaklines=true,                 
    captionpos=b,                    
    keepspaces=true,                 
    numbers=left,                    
xleftmargin=2em,
framexleftmargin=2em,            
    showspaces=false,                
    showstringspaces=false,
    showtabs=false,                  
    tabsize=2,
    upquote=true
}

\lstset{style=mystyle}


\lstset{style=mystyle}
\newcommand{\imgdir}{C:/laragon/www/newmc/assets/imgsvg/}
\newcommand{\imgsvgdir}{C:/laragon/www/newmc/assets/imgsvg/}

\definecolor{mcgris}{RGB}{220, 220, 220}% ancien~; pour compatibilité
\definecolor{mcbleu}{RGB}{52, 152, 219}
\definecolor{mcvert}{RGB}{125, 194, 70}
\definecolor{mcmauve}{RGB}{154, 0, 215}
\definecolor{mcorange}{RGB}{255, 96, 0}
\definecolor{mcturquoise}{RGB}{0, 153, 153}
\definecolor{mcrouge}{RGB}{255, 0, 0}
\definecolor{mclightvert}{RGB}{205, 234, 190}

\definecolor{gris}{RGB}{220, 220, 220}
\definecolor{bleu}{RGB}{52, 152, 219}
\definecolor{vert}{RGB}{125, 194, 70}
\definecolor{mauve}{RGB}{154, 0, 215}
\definecolor{orange}{RGB}{255, 96, 0}
\definecolor{turquoise}{RGB}{0, 153, 153}
\definecolor{rouge}{RGB}{255, 0, 0}
\definecolor{lightvert}{RGB}{205, 234, 190}
\setitemize[0]{label=\color{lightvert}  $\bullet$}

\pagestyle{fancy}
\renewcommand{\headrulewidth}{0.2pt}
\fancyhead[L]{maths-cours.fr}
\fancyhead[R]{\thepage}
\renewcommand{\footrulewidth}{0.2pt}
\fancyfoot[C]{}

\newcolumntype{C}{>{\centering\arraybackslash}X}
\newcolumntype{s}{>{\hsize=.35\hsize\arraybackslash}X}

\setlength{\parindent}{0pt}		 
\setlength{\parskip}{3mm}
\setlength{\headheight}{1cm}

\def\ebook{ebook}
\def\book{book}
\def\web{web}
\def\type{web}

\newcommand{\vect}[1]{\overrightarrow{\,\mathstrut#1\,}}

\def\Oij{$\left(\text{O}~;~\vect{\imath},~\vect{\jmath}\right)$}
\def\Oijk{$\left(\text{O}~;~\vect{\imath},~\vect{\jmath},~\vect{k}\right)$}
\def\Ouv{$\left(\text{O}~;~\vect{u},~\vect{v}\right)$}

\hypersetup{breaklinks=true, colorlinks = true, linkcolor = OliveGreen, urlcolor = OliveGreen, citecolor = OliveGreen, pdfauthor={Didier BONNEL - https://www.maths-cours.fr} } % supprime les bordures autour des liens

\renewcommand{\arg}[0]{\text{arg}}

\everymath{\displaystyle}

%================================================================================================================================
%
% Macros - Commandes
%
%================================================================================================================================

\newcommand\meta[2]{    			% Utilisé pour créer le post HTML.
	\def\titre{titre}
	\def\url{url}
	\def\arg{#1}
	\ifx\titre\arg
		\newcommand\maintitle{#2}
		\fancyhead[L]{#2}
		{\Large\sffamily \MakeUppercase{#2}}
		\vspace{1mm}\textcolor{mcvert}{\hrule}
	\fi 
	\ifx\url\arg
		\fancyfoot[L]{\href{https://www.maths-cours.fr#2}{\black \footnotesize{https://www.maths-cours.fr#2}}}
	\fi 
}


\newcommand\TitreC[1]{    		% Titre centré
     \needspace{3\baselineskip}
     \begin{center}\textbf{#1}\end{center}
}

\newcommand\newpar{    		% paragraphe
     \par
}

\newcommand\nosp {    		% commande vide (pas d'espace)
}
\newcommand{\id}[1]{} %ignore

\newcommand\boite[2]{				% Boite simple sans titre
	\vspace{5mm}
	\setlength{\fboxrule}{0.2mm}
	\setlength{\fboxsep}{5mm}	
	\fcolorbox{#1}{#1!3}{\makebox[\linewidth-2\fboxrule-2\fboxsep]{
  		\begin{minipage}[t]{\linewidth-2\fboxrule-4\fboxsep}\setlength{\parskip}{3mm}
  			 #2
  		\end{minipage}
	}}
	\vspace{5mm}
}

\newcommand\CBox[4]{				% Boites
	\vspace{5mm}
	\setlength{\fboxrule}{0.2mm}
	\setlength{\fboxsep}{5mm}
	
	\fcolorbox{#1}{#1!3}{\makebox[\linewidth-2\fboxrule-2\fboxsep]{
		\begin{minipage}[t]{1cm}\setlength{\parskip}{3mm}
	  		\textcolor{#1}{\LARGE{#2}}    
 	 	\end{minipage}  
  		\begin{minipage}[t]{\linewidth-2\fboxrule-4\fboxsep}\setlength{\parskip}{3mm}
			\raisebox{1.2mm}{\normalsize\sffamily{\textcolor{#1}{#3}}}						
  			 #4
  		\end{minipage}
	}}
	\vspace{5mm}
}

\newcommand\cadre[3]{				% Boites convertible html
	\par
	\vspace{2mm}
	\setlength{\fboxrule}{0.1mm}
	\setlength{\fboxsep}{5mm}
	\fcolorbox{#1}{white}{\makebox[\linewidth-2\fboxrule-2\fboxsep]{
  		\begin{minipage}[t]{\linewidth-2\fboxrule-4\fboxsep}\setlength{\parskip}{3mm}
			\raisebox{-2.5mm}{\sffamily \small{\textcolor{#1}{\MakeUppercase{#2}}}}		
			\par		
  			 #3
 	 		\end{minipage}
	}}
		\vspace{2mm}
	\par
}

\newcommand\bloc[3]{				% Boites convertible html sans bordure
     \needspace{2\baselineskip}
     {\sffamily \small{\textcolor{#1}{\MakeUppercase{#2}}}}    
		\par		
  			 #3
		\par
}

\newcommand\CHelp[1]{
     \CBox{Plum}{\faInfoCircle}{À RETENIR}{#1}
}

\newcommand\CUp[1]{
     \CBox{NavyBlue}{\faThumbsOUp}{EN PRATIQUE}{#1}
}

\newcommand\CInfo[1]{
     \CBox{Sepia}{\faArrowCircleRight}{REMARQUE}{#1}
}

\newcommand\CRedac[1]{
     \CBox{PineGreen}{\faEdit}{BIEN R\'EDIGER}{#1}
}

\newcommand\CError[1]{
     \CBox{Red}{\faExclamationTriangle}{ATTENTION}{#1}
}

\newcommand\TitreExo[2]{
\needspace{4\baselineskip}
 {\sffamily\large EXERCICE #1\ (\emph{#2 points})}
\vspace{5mm}
}

\newcommand\img[2]{
          \includegraphics[width=#2\paperwidth]{\imgdir#1}
}

\newcommand\imgsvg[2]{
       \begin{center}   \includegraphics[width=#2\paperwidth]{\imgsvgdir#1} \end{center}
}


\newcommand\Lien[2]{
     \href{#1}{#2 \tiny \faExternalLink}
}
\newcommand\mcLien[2]{
     \href{https~://www.maths-cours.fr/#1}{#2 \tiny \faExternalLink}
}

\newcommand{\euro}{\eurologo{}}

%================================================================================================================================
%
% Macros - Environement
%
%================================================================================================================================

\newenvironment{tex}{ %
}
{%
}

\newenvironment{indente}{ %
	\setlength\parindent{10mm}
}

{
	\setlength\parindent{0mm}
}

\newenvironment{corrige}{%
     \needspace{3\baselineskip}
     \medskip
     \textbf{\textsc{Corrigé}}
     \medskip
}
{
}

\newenvironment{extern}{%
     \begin{center}
     }
     {
     \end{center}
}

\NewEnviron{code}{%
	\par
     \boite{gray}{\texttt{%
     \BODY
     }}
     \par
}

\newenvironment{vbloc}{% boite sans cadre empeche saut de page
     \begin{minipage}[t]{\linewidth}
     }
     {
     \end{minipage}
}
\NewEnviron{h2}{%
    \needspace{3\baselineskip}
    \vspace{0.6cm}
	\noindent \MakeUppercase{\sffamily \large \BODY}
	\vspace{1mm}\textcolor{mcgris}{\hrule}\vspace{0.4cm}
	\par
}{}

\NewEnviron{h3}{%
    \needspace{3\baselineskip}
	\vspace{5mm}
	\textsc{\BODY}
	\par
}

\NewEnviron{margeneg}{ %
\begin{addmargin}[-1cm]{0cm}
\BODY
\end{addmargin}
}

\NewEnviron{html}{%
}

\begin{document}
\meta{url}{/exercices/algorithme-de-calcul-brevet-metropole-2018/}
\meta{pid}{9518}
\meta{titre}{Algorithme de calcul - Brevet Métropole 2018}
\meta{type}{exercices}
%
\begin{h2}Exercice 5 (16 points)\end{h2}
\medbreak
Voici un programme de calcul~:
\begin{center}
     \begin{extern}%width="250" alt="Programme de calcul"
          \begin{tabular}{|l|}\hline
               $\bullet~~$Choisir un nombre\\
               $\bullet~~$Multiplier ce nombre par 4\\
               $\bullet~~$Ajouter 8\\
               $\bullet~~$Multiplier le résultat par 2\\ \hline
          \end{tabular}
     \end{extern}
\end{center}
\medbreak
\begin{enumerate}
     \item Vérifier que si on choisit le nombre $- 1$, ce programme donne 8 comme résultat final.
     \item Le programme donne 30 comme résultat final, quel est le nombre choisi au départ~?
     \bigbreak
     \begin{margeneg}
          Dans la suite de l'exercice, on nomme $x$ le nombre choisi au départ.
     \end{margeneg}
     \bigbreak
     \item L'expression $A = 2(4x + 8)$ donne le résultat du programme de calcul précédent pour un nombre $x$ donné.
     \par
     On pose $B = (4 + x)^2 - x^2$.
     \par
     Prouver que les expressions $A$ et $B$ sont égales pour toutes les valeurs de $x$.
     \item Pour chacune des affirmations suivantes, indiquer si elle est vraie ou fausse. On rappelle que les réponses doivent être justifiées.
     \smallbreak
     \begin{itemize}
          \item %
          Affirmation 1~: Ce programme donne un résultat positif pour toutes les valeurs de $x.$
          \item %
          Affirmation 2~: Si le nombre $x$ choisi est un nombre entier, le résultat obtenu est un multiple de $8.$
     \end{itemize}
\end{enumerate}

\end{document}
µ
\documentclass[a4paper]{article}

%================================================================================================================================
%
% Packages
%
%================================================================================================================================

\usepackage[T1]{fontenc} 	% pour caractères accentués
\usepackage[utf8]{inputenc}  % encodage utf8
\usepackage[french]{babel}	% langue : français
\usepackage{fourier}			% caractères plus lisibles
\usepackage[dvipsnames]{xcolor} % couleurs
\usepackage{fancyhdr}		% réglage header footer
\usepackage{needspace}		% empêcher sauts de page mal placés
\usepackage{graphicx}		% pour inclure des graphiques
\usepackage{enumitem,cprotect}		% personnalise les listes d'items (nécessaire pour ol, al ...)
\usepackage{hyperref}		% Liens hypertexte
\usepackage{pstricks,pst-all,pst-node,pstricks-add,pst-math,pst-plot,pst-tree,pst-eucl} % pstricks
\usepackage[a4paper,includeheadfoot,top=2cm,left=3cm, bottom=2cm,right=3cm]{geometry} % marges etc.
\usepackage{comment}			% commentaires multilignes
\usepackage{amsmath,environ} % maths (matrices, etc.)
\usepackage{amssymb,makeidx}
\usepackage{bm}				% bold maths
\usepackage{tabularx}		% tableaux
\usepackage{colortbl}		% tableaux en couleur
\usepackage{fontawesome}		% Fontawesome
\usepackage{environ}			% environment with command
\usepackage{fp}				% calculs pour ps-tricks
\usepackage{multido}			% pour ps tricks
\usepackage[np]{numprint}	% formattage nombre
\usepackage{tikz,tkz-tab} 			% package principal TikZ
\usepackage{pgfplots}   % axes
\usepackage{mathrsfs}    % cursives
\usepackage{calc}			% calcul taille boites
\usepackage[scaled=0.875]{helvet} % font sans serif
\usepackage{svg} % svg
\usepackage{scrextend} % local margin
\usepackage{scratch} %scratch
\usepackage{multicol} % colonnes
%\usepackage{infix-RPN,pst-func} % formule en notation polanaise inversée
\usepackage{listings}

%================================================================================================================================
%
% Réglages de base
%
%================================================================================================================================

\lstset{
language=Python,   % R code
literate=
{á}{{\'a}}1
{à}{{\`a}}1
{ã}{{\~a}}1
{é}{{\'e}}1
{è}{{\`e}}1
{ê}{{\^e}}1
{í}{{\'i}}1
{ó}{{\'o}}1
{õ}{{\~o}}1
{ú}{{\'u}}1
{ü}{{\"u}}1
{ç}{{\c{c}}}1
{~}{{ }}1
}


\definecolor{codegreen}{rgb}{0,0.6,0}
\definecolor{codegray}{rgb}{0.5,0.5,0.5}
\definecolor{codepurple}{rgb}{0.58,0,0.82}
\definecolor{backcolour}{rgb}{0.95,0.95,0.92}

\lstdefinestyle{mystyle}{
    backgroundcolor=\color{backcolour},   
    commentstyle=\color{codegreen},
    keywordstyle=\color{magenta},
    numberstyle=\tiny\color{codegray},
    stringstyle=\color{codepurple},
    basicstyle=\ttfamily\footnotesize,
    breakatwhitespace=false,         
    breaklines=true,                 
    captionpos=b,                    
    keepspaces=true,                 
    numbers=left,                    
xleftmargin=2em,
framexleftmargin=2em,            
    showspaces=false,                
    showstringspaces=false,
    showtabs=false,                  
    tabsize=2,
    upquote=true
}

\lstset{style=mystyle}


\lstset{style=mystyle}
\newcommand{\imgdir}{C:/laragon/www/newmc/assets/imgsvg/}
\newcommand{\imgsvgdir}{C:/laragon/www/newmc/assets/imgsvg/}

\definecolor{mcgris}{RGB}{220, 220, 220}% ancien~; pour compatibilité
\definecolor{mcbleu}{RGB}{52, 152, 219}
\definecolor{mcvert}{RGB}{125, 194, 70}
\definecolor{mcmauve}{RGB}{154, 0, 215}
\definecolor{mcorange}{RGB}{255, 96, 0}
\definecolor{mcturquoise}{RGB}{0, 153, 153}
\definecolor{mcrouge}{RGB}{255, 0, 0}
\definecolor{mclightvert}{RGB}{205, 234, 190}

\definecolor{gris}{RGB}{220, 220, 220}
\definecolor{bleu}{RGB}{52, 152, 219}
\definecolor{vert}{RGB}{125, 194, 70}
\definecolor{mauve}{RGB}{154, 0, 215}
\definecolor{orange}{RGB}{255, 96, 0}
\definecolor{turquoise}{RGB}{0, 153, 153}
\definecolor{rouge}{RGB}{255, 0, 0}
\definecolor{lightvert}{RGB}{205, 234, 190}
\setitemize[0]{label=\color{lightvert}  $\bullet$}

\pagestyle{fancy}
\renewcommand{\headrulewidth}{0.2pt}
\fancyhead[L]{maths-cours.fr}
\fancyhead[R]{\thepage}
\renewcommand{\footrulewidth}{0.2pt}
\fancyfoot[C]{}

\newcolumntype{C}{>{\centering\arraybackslash}X}
\newcolumntype{s}{>{\hsize=.35\hsize\arraybackslash}X}

\setlength{\parindent}{0pt}		 
\setlength{\parskip}{3mm}
\setlength{\headheight}{1cm}

\def\ebook{ebook}
\def\book{book}
\def\web{web}
\def\type{web}

\newcommand{\vect}[1]{\overrightarrow{\,\mathstrut#1\,}}

\def\Oij{$\left(\text{O}~;~\vect{\imath},~\vect{\jmath}\right)$}
\def\Oijk{$\left(\text{O}~;~\vect{\imath},~\vect{\jmath},~\vect{k}\right)$}
\def\Ouv{$\left(\text{O}~;~\vect{u},~\vect{v}\right)$}

\hypersetup{breaklinks=true, colorlinks = true, linkcolor = OliveGreen, urlcolor = OliveGreen, citecolor = OliveGreen, pdfauthor={Didier BONNEL - https://www.maths-cours.fr} } % supprime les bordures autour des liens

\renewcommand{\arg}[0]{\text{arg}}

\everymath{\displaystyle}

%================================================================================================================================
%
% Macros - Commandes
%
%================================================================================================================================

\newcommand\meta[2]{    			% Utilisé pour créer le post HTML.
	\def\titre{titre}
	\def\url{url}
	\def\arg{#1}
	\ifx\titre\arg
		\newcommand\maintitle{#2}
		\fancyhead[L]{#2}
		{\Large\sffamily \MakeUppercase{#2}}
		\vspace{1mm}\textcolor{mcvert}{\hrule}
	\fi 
	\ifx\url\arg
		\fancyfoot[L]{\href{https://www.maths-cours.fr#2}{\black \footnotesize{https://www.maths-cours.fr#2}}}
	\fi 
}


\newcommand\TitreC[1]{    		% Titre centré
     \needspace{3\baselineskip}
     \begin{center}\textbf{#1}\end{center}
}

\newcommand\newpar{    		% paragraphe
     \par
}

\newcommand\nosp {    		% commande vide (pas d'espace)
}
\newcommand{\id}[1]{} %ignore

\newcommand\boite[2]{				% Boite simple sans titre
	\vspace{5mm}
	\setlength{\fboxrule}{0.2mm}
	\setlength{\fboxsep}{5mm}	
	\fcolorbox{#1}{#1!3}{\makebox[\linewidth-2\fboxrule-2\fboxsep]{
  		\begin{minipage}[t]{\linewidth-2\fboxrule-4\fboxsep}\setlength{\parskip}{3mm}
  			 #2
  		\end{minipage}
	}}
	\vspace{5mm}
}

\newcommand\CBox[4]{				% Boites
	\vspace{5mm}
	\setlength{\fboxrule}{0.2mm}
	\setlength{\fboxsep}{5mm}
	
	\fcolorbox{#1}{#1!3}{\makebox[\linewidth-2\fboxrule-2\fboxsep]{
		\begin{minipage}[t]{1cm}\setlength{\parskip}{3mm}
	  		\textcolor{#1}{\LARGE{#2}}    
 	 	\end{minipage}  
  		\begin{minipage}[t]{\linewidth-2\fboxrule-4\fboxsep}\setlength{\parskip}{3mm}
			\raisebox{1.2mm}{\normalsize\sffamily{\textcolor{#1}{#3}}}						
  			 #4
  		\end{minipage}
	}}
	\vspace{5mm}
}

\newcommand\cadre[3]{				% Boites convertible html
	\par
	\vspace{2mm}
	\setlength{\fboxrule}{0.1mm}
	\setlength{\fboxsep}{5mm}
	\fcolorbox{#1}{white}{\makebox[\linewidth-2\fboxrule-2\fboxsep]{
  		\begin{minipage}[t]{\linewidth-2\fboxrule-4\fboxsep}\setlength{\parskip}{3mm}
			\raisebox{-2.5mm}{\sffamily \small{\textcolor{#1}{\MakeUppercase{#2}}}}		
			\par		
  			 #3
 	 		\end{minipage}
	}}
		\vspace{2mm}
	\par
}

\newcommand\bloc[3]{				% Boites convertible html sans bordure
     \needspace{2\baselineskip}
     {\sffamily \small{\textcolor{#1}{\MakeUppercase{#2}}}}    
		\par		
  			 #3
		\par
}

\newcommand\CHelp[1]{
     \CBox{Plum}{\faInfoCircle}{À RETENIR}{#1}
}

\newcommand\CUp[1]{
     \CBox{NavyBlue}{\faThumbsOUp}{EN PRATIQUE}{#1}
}

\newcommand\CInfo[1]{
     \CBox{Sepia}{\faArrowCircleRight}{REMARQUE}{#1}
}

\newcommand\CRedac[1]{
     \CBox{PineGreen}{\faEdit}{BIEN R\'EDIGER}{#1}
}

\newcommand\CError[1]{
     \CBox{Red}{\faExclamationTriangle}{ATTENTION}{#1}
}

\newcommand\TitreExo[2]{
\needspace{4\baselineskip}
 {\sffamily\large EXERCICE #1\ (\emph{#2 points})}
\vspace{5mm}
}

\newcommand\img[2]{
          \includegraphics[width=#2\paperwidth]{\imgdir#1}
}

\newcommand\imgsvg[2]{
       \begin{center}   \includegraphics[width=#2\paperwidth]{\imgsvgdir#1} \end{center}
}


\newcommand\Lien[2]{
     \href{#1}{#2 \tiny \faExternalLink}
}
\newcommand\mcLien[2]{
     \href{https~://www.maths-cours.fr/#1}{#2 \tiny \faExternalLink}
}

\newcommand{\euro}{\eurologo{}}

%================================================================================================================================
%
% Macros - Environement
%
%================================================================================================================================

\newenvironment{tex}{ %
}
{%
}

\newenvironment{indente}{ %
	\setlength\parindent{10mm}
}

{
	\setlength\parindent{0mm}
}

\newenvironment{corrige}{%
     \needspace{3\baselineskip}
     \medskip
     \textbf{\textsc{Corrigé}}
     \medskip
}
{
}

\newenvironment{extern}{%
     \begin{center}
     }
     {
     \end{center}
}

\NewEnviron{code}{%
	\par
     \boite{gray}{\texttt{%
     \BODY
     }}
     \par
}

\newenvironment{vbloc}{% boite sans cadre empeche saut de page
     \begin{minipage}[t]{\linewidth}
     }
     {
     \end{minipage}
}
\NewEnviron{h2}{%
    \needspace{3\baselineskip}
    \vspace{0.6cm}
	\noindent \MakeUppercase{\sffamily \large \BODY}
	\vspace{1mm}\textcolor{mcgris}{\hrule}\vspace{0.4cm}
	\par
}{}

\NewEnviron{h3}{%
    \needspace{3\baselineskip}
	\vspace{5mm}
	\textsc{\BODY}
	\par
}

\NewEnviron{margeneg}{ %
\begin{addmargin}[-1cm]{0cm}
\BODY
\end{addmargin}
}

\NewEnviron{html}{%
}

\begin{document}
\meta{url}{/exercices/programme-scratch-brevet-metropole-2018/}
\meta{pid}{9525}
\meta{titre}{Programme scratch - Brevet Métropole 2018}
\meta{type}{exercices}
%
\begin{h2}Exercice 6 (16 points)\end{h2}
\medbreak
Les longueurs sont en pixels.
\par
L'expression \og s'orienter à 90 \fg{} signifie que l'on s'oriente vers la droite.
\smallbreak
On donne le programme suivant~:
\begin{center}
     \begin{extern}%width="500" alt="Programme scratch Brevet Métropole 2018"
          \begin{tabular}{m{6cm} m{6cm}}
               \begin{scratch}
                    \blockinit{quand \greenflag est cliqué}
                    \blockmove{aller à x~: \ovalnum0 y~: \ovalnum0}
                    \blockpen{stylo en position d'écriture}
                    \blockmove{s'orienter à \ovalnum{90\selectarrownum} degrés}
                    \blockvariable{mettre \ovalvariable{Longueur} à 300}
                    \blockevent{Carré}
                    \blockevent{Triangle}
                    \blockmove{avancer de \ovalnum{Longueur}/\ovalnum{6}}
                    {\blockvariable{mettre \ovalvariable{Longueur} à \ovaloperator{}}
                         \blockevent{Carré}
                    \blockevent{Triangle}}
                    \end{scratch}&\begin{scratch}
                    \initmoreblocks{définir \namemoreblocks{Carré}}
                    \blockrepeat{répéter \ovalnum{4} fois}
                    {\blockmove{avancer de \ovalnum{Longueur}}
                         \blockmove{tourner \turnleft{} de \ovalnum{90} degrés}
                    }
               \end{scratch}
               \begin{scratch}
                    \initmoreblocks{définir \namemoreblocks{Triangle}}
                    \blockrepeat{répéter \ovalnum{3} fois}
                    {\blockmove{avancer de \ovalnum{Longueur}}
                         \blockmove{tourner \turnleft{} de \ovalnum{120} degrés}
                    }
               \end{scratch}\\
          \end{tabular}
     \end{extern}
\end{center}
\medbreak
\begin{enumerate}
     \item On prend comme échelle 1~cm pour $50$ pixels.
     \begin{enumerate}[label=\alph*.]
          \item Représenter sur votre copie la figure obtenue si le programme est exécuté jusqu'à la ligne 7 comprise.
          \item Quelles sont les coordonnées du stylo après l'exécution de la ligne 8~?
     \end{enumerate}
     \item  On exécute le programme complet et on obtient la figure ci-dessous qui possède un axe de symétrie vertical.
     \begin{center}
          \begin{extern}%width="250" alt="Transformations Brevet Métropole 2018"
               \psset{unit=1cm}
               \begin{pspicture}(4.6,4.6)
                    \psframe(4.6,4.6)
                    \psframe(0.75,0)(3.85,3.1)
                    \psline(0,0)(2.3,4)(4.6,0)
                    \psline(0.75,0)(2.3,2.65)(3.85,0)
               \end{pspicture}
          \end{extern}
     \end{center}
     Recopier et compléter la ligne 9 du programme pour obtenir cette figure.
     \item
     \begin{enumerate}[label=\alph*.]
          \item Parmi les transformations suivantes, translation, homothétie, rotation, symétrie axiale, quelle est la transformation géométrique qui permet d'obtenir le petit carré à partir du grand carré~? Préciser le rapport de réduction.
          \item Quel est le rapport des aires entre les deux carrés dessinés~?
     \end{enumerate}
\end{enumerate}

\end{document}
µ
\documentclass[a4paper]{article}

%================================================================================================================================
%
% Packages
%
%================================================================================================================================

\usepackage[T1]{fontenc} 	% pour caractères accentués
\usepackage[utf8]{inputenc}  % encodage utf8
\usepackage[french]{babel}	% langue : français
\usepackage{fourier}			% caractères plus lisibles
\usepackage[dvipsnames]{xcolor} % couleurs
\usepackage{fancyhdr}		% réglage header footer
\usepackage{needspace}		% empêcher sauts de page mal placés
\usepackage{graphicx}		% pour inclure des graphiques
\usepackage{enumitem,cprotect}		% personnalise les listes d'items (nécessaire pour ol, al ...)
\usepackage{hyperref}		% Liens hypertexte
\usepackage{pstricks,pst-all,pst-node,pstricks-add,pst-math,pst-plot,pst-tree,pst-eucl} % pstricks
\usepackage[a4paper,includeheadfoot,top=2cm,left=3cm, bottom=2cm,right=3cm]{geometry} % marges etc.
\usepackage{comment}			% commentaires multilignes
\usepackage{amsmath,environ} % maths (matrices, etc.)
\usepackage{amssymb,makeidx}
\usepackage{bm}				% bold maths
\usepackage{tabularx}		% tableaux
\usepackage{colortbl}		% tableaux en couleur
\usepackage{fontawesome}		% Fontawesome
\usepackage{environ}			% environment with command
\usepackage{fp}				% calculs pour ps-tricks
\usepackage{multido}			% pour ps tricks
\usepackage[np]{numprint}	% formattage nombre
\usepackage{tikz,tkz-tab} 			% package principal TikZ
\usepackage{pgfplots}   % axes
\usepackage{mathrsfs}    % cursives
\usepackage{calc}			% calcul taille boites
\usepackage[scaled=0.875]{helvet} % font sans serif
\usepackage{svg} % svg
\usepackage{scrextend} % local margin
\usepackage{scratch} %scratch
\usepackage{multicol} % colonnes
%\usepackage{infix-RPN,pst-func} % formule en notation polanaise inversée
\usepackage{listings}

%================================================================================================================================
%
% Réglages de base
%
%================================================================================================================================

\lstset{
language=Python,   % R code
literate=
{á}{{\'a}}1
{à}{{\`a}}1
{ã}{{\~a}}1
{é}{{\'e}}1
{è}{{\`e}}1
{ê}{{\^e}}1
{í}{{\'i}}1
{ó}{{\'o}}1
{õ}{{\~o}}1
{ú}{{\'u}}1
{ü}{{\"u}}1
{ç}{{\c{c}}}1
{~}{{ }}1
}


\definecolor{codegreen}{rgb}{0,0.6,0}
\definecolor{codegray}{rgb}{0.5,0.5,0.5}
\definecolor{codepurple}{rgb}{0.58,0,0.82}
\definecolor{backcolour}{rgb}{0.95,0.95,0.92}

\lstdefinestyle{mystyle}{
    backgroundcolor=\color{backcolour},   
    commentstyle=\color{codegreen},
    keywordstyle=\color{magenta},
    numberstyle=\tiny\color{codegray},
    stringstyle=\color{codepurple},
    basicstyle=\ttfamily\footnotesize,
    breakatwhitespace=false,         
    breaklines=true,                 
    captionpos=b,                    
    keepspaces=true,                 
    numbers=left,                    
xleftmargin=2em,
framexleftmargin=2em,            
    showspaces=false,                
    showstringspaces=false,
    showtabs=false,                  
    tabsize=2,
    upquote=true
}

\lstset{style=mystyle}


\lstset{style=mystyle}
\newcommand{\imgdir}{C:/laragon/www/newmc/assets/imgsvg/}
\newcommand{\imgsvgdir}{C:/laragon/www/newmc/assets/imgsvg/}

\definecolor{mcgris}{RGB}{220, 220, 220}% ancien~; pour compatibilité
\definecolor{mcbleu}{RGB}{52, 152, 219}
\definecolor{mcvert}{RGB}{125, 194, 70}
\definecolor{mcmauve}{RGB}{154, 0, 215}
\definecolor{mcorange}{RGB}{255, 96, 0}
\definecolor{mcturquoise}{RGB}{0, 153, 153}
\definecolor{mcrouge}{RGB}{255, 0, 0}
\definecolor{mclightvert}{RGB}{205, 234, 190}

\definecolor{gris}{RGB}{220, 220, 220}
\definecolor{bleu}{RGB}{52, 152, 219}
\definecolor{vert}{RGB}{125, 194, 70}
\definecolor{mauve}{RGB}{154, 0, 215}
\definecolor{orange}{RGB}{255, 96, 0}
\definecolor{turquoise}{RGB}{0, 153, 153}
\definecolor{rouge}{RGB}{255, 0, 0}
\definecolor{lightvert}{RGB}{205, 234, 190}
\setitemize[0]{label=\color{lightvert}  $\bullet$}

\pagestyle{fancy}
\renewcommand{\headrulewidth}{0.2pt}
\fancyhead[L]{maths-cours.fr}
\fancyhead[R]{\thepage}
\renewcommand{\footrulewidth}{0.2pt}
\fancyfoot[C]{}

\newcolumntype{C}{>{\centering\arraybackslash}X}
\newcolumntype{s}{>{\hsize=.35\hsize\arraybackslash}X}

\setlength{\parindent}{0pt}		 
\setlength{\parskip}{3mm}
\setlength{\headheight}{1cm}

\def\ebook{ebook}
\def\book{book}
\def\web{web}
\def\type{web}

\newcommand{\vect}[1]{\overrightarrow{\,\mathstrut#1\,}}

\def\Oij{$\left(\text{O}~;~\vect{\imath},~\vect{\jmath}\right)$}
\def\Oijk{$\left(\text{O}~;~\vect{\imath},~\vect{\jmath},~\vect{k}\right)$}
\def\Ouv{$\left(\text{O}~;~\vect{u},~\vect{v}\right)$}

\hypersetup{breaklinks=true, colorlinks = true, linkcolor = OliveGreen, urlcolor = OliveGreen, citecolor = OliveGreen, pdfauthor={Didier BONNEL - https://www.maths-cours.fr} } % supprime les bordures autour des liens

\renewcommand{\arg}[0]{\text{arg}}

\everymath{\displaystyle}

%================================================================================================================================
%
% Macros - Commandes
%
%================================================================================================================================

\newcommand\meta[2]{    			% Utilisé pour créer le post HTML.
	\def\titre{titre}
	\def\url{url}
	\def\arg{#1}
	\ifx\titre\arg
		\newcommand\maintitle{#2}
		\fancyhead[L]{#2}
		{\Large\sffamily \MakeUppercase{#2}}
		\vspace{1mm}\textcolor{mcvert}{\hrule}
	\fi 
	\ifx\url\arg
		\fancyfoot[L]{\href{https://www.maths-cours.fr#2}{\black \footnotesize{https://www.maths-cours.fr#2}}}
	\fi 
}


\newcommand\TitreC[1]{    		% Titre centré
     \needspace{3\baselineskip}
     \begin{center}\textbf{#1}\end{center}
}

\newcommand\newpar{    		% paragraphe
     \par
}

\newcommand\nosp {    		% commande vide (pas d'espace)
}
\newcommand{\id}[1]{} %ignore

\newcommand\boite[2]{				% Boite simple sans titre
	\vspace{5mm}
	\setlength{\fboxrule}{0.2mm}
	\setlength{\fboxsep}{5mm}	
	\fcolorbox{#1}{#1!3}{\makebox[\linewidth-2\fboxrule-2\fboxsep]{
  		\begin{minipage}[t]{\linewidth-2\fboxrule-4\fboxsep}\setlength{\parskip}{3mm}
  			 #2
  		\end{minipage}
	}}
	\vspace{5mm}
}

\newcommand\CBox[4]{				% Boites
	\vspace{5mm}
	\setlength{\fboxrule}{0.2mm}
	\setlength{\fboxsep}{5mm}
	
	\fcolorbox{#1}{#1!3}{\makebox[\linewidth-2\fboxrule-2\fboxsep]{
		\begin{minipage}[t]{1cm}\setlength{\parskip}{3mm}
	  		\textcolor{#1}{\LARGE{#2}}    
 	 	\end{minipage}  
  		\begin{minipage}[t]{\linewidth-2\fboxrule-4\fboxsep}\setlength{\parskip}{3mm}
			\raisebox{1.2mm}{\normalsize\sffamily{\textcolor{#1}{#3}}}						
  			 #4
  		\end{minipage}
	}}
	\vspace{5mm}
}

\newcommand\cadre[3]{				% Boites convertible html
	\par
	\vspace{2mm}
	\setlength{\fboxrule}{0.1mm}
	\setlength{\fboxsep}{5mm}
	\fcolorbox{#1}{white}{\makebox[\linewidth-2\fboxrule-2\fboxsep]{
  		\begin{minipage}[t]{\linewidth-2\fboxrule-4\fboxsep}\setlength{\parskip}{3mm}
			\raisebox{-2.5mm}{\sffamily \small{\textcolor{#1}{\MakeUppercase{#2}}}}		
			\par		
  			 #3
 	 		\end{minipage}
	}}
		\vspace{2mm}
	\par
}

\newcommand\bloc[3]{				% Boites convertible html sans bordure
     \needspace{2\baselineskip}
     {\sffamily \small{\textcolor{#1}{\MakeUppercase{#2}}}}    
		\par		
  			 #3
		\par
}

\newcommand\CHelp[1]{
     \CBox{Plum}{\faInfoCircle}{À RETENIR}{#1}
}

\newcommand\CUp[1]{
     \CBox{NavyBlue}{\faThumbsOUp}{EN PRATIQUE}{#1}
}

\newcommand\CInfo[1]{
     \CBox{Sepia}{\faArrowCircleRight}{REMARQUE}{#1}
}

\newcommand\CRedac[1]{
     \CBox{PineGreen}{\faEdit}{BIEN R\'EDIGER}{#1}
}

\newcommand\CError[1]{
     \CBox{Red}{\faExclamationTriangle}{ATTENTION}{#1}
}

\newcommand\TitreExo[2]{
\needspace{4\baselineskip}
 {\sffamily\large EXERCICE #1\ (\emph{#2 points})}
\vspace{5mm}
}

\newcommand\img[2]{
          \includegraphics[width=#2\paperwidth]{\imgdir#1}
}

\newcommand\imgsvg[2]{
       \begin{center}   \includegraphics[width=#2\paperwidth]{\imgsvgdir#1} \end{center}
}


\newcommand\Lien[2]{
     \href{#1}{#2 \tiny \faExternalLink}
}
\newcommand\mcLien[2]{
     \href{https~://www.maths-cours.fr/#1}{#2 \tiny \faExternalLink}
}

\newcommand{\euro}{\eurologo{}}

%================================================================================================================================
%
% Macros - Environement
%
%================================================================================================================================

\newenvironment{tex}{ %
}
{%
}

\newenvironment{indente}{ %
	\setlength\parindent{10mm}
}

{
	\setlength\parindent{0mm}
}

\newenvironment{corrige}{%
     \needspace{3\baselineskip}
     \medskip
     \textbf{\textsc{Corrigé}}
     \medskip
}
{
}

\newenvironment{extern}{%
     \begin{center}
     }
     {
     \end{center}
}

\NewEnviron{code}{%
	\par
     \boite{gray}{\texttt{%
     \BODY
     }}
     \par
}

\newenvironment{vbloc}{% boite sans cadre empeche saut de page
     \begin{minipage}[t]{\linewidth}
     }
     {
     \end{minipage}
}
\NewEnviron{h2}{%
    \needspace{3\baselineskip}
    \vspace{0.6cm}
	\noindent \MakeUppercase{\sffamily \large \BODY}
	\vspace{1mm}\textcolor{mcgris}{\hrule}\vspace{0.4cm}
	\par
}{}

\NewEnviron{h3}{%
    \needspace{3\baselineskip}
	\vspace{5mm}
	\textsc{\BODY}
	\par
}

\NewEnviron{margeneg}{ %
\begin{addmargin}[-1cm]{0cm}
\BODY
\end{addmargin}
}

\NewEnviron{html}{%
}

\begin{document}
\meta{url}{/exercices/hand-spinner-brevet-metropole-2018/}
\meta{pid}{9532}
\meta{titre}{Hand-spinner - Brevet Métropole 2018}
\meta{type}{exercices}
%
\begin{h2}Exercice 7 (17 points)\end{h2}
\medbreak
Le \og hand-spinner \fg{} est une sorte de toupie plate qui tourne sur elle-même.
\begin{center}
     \img{handspinner}{0.2}%width="150" alt="hand-spinner"
\end{center}
On donne au \og hand-spinner \fg{} une vitesse de rotation initiale au temps $t = 0$, puis, au cours du temps, sa vitesse de rotation diminue jusqu'à l'arrêt complet du \og hand-spinner \fg{}.
\par
Sa vitesse de rotation est alors égale à $0$.
\par
Grâce à un appareil de mesure, on a relevé la vitesse de rotation exprimée en
nombre de tours par seconde.
\medbreak
Sur le graphique ci-dessous, on a représenté cette vitesse en fonction du temps exprimé en seconde~:
\begin{center}
     \begin{extern}%width="560" alt="Fonction vitesse Brevet Métropole 2018 "
          \psset{xunit=0.11cm,yunit=0.44cm}
          \begin{pspicture}(-20,-1)(100,25)
               \rput{90}(-8,12.5){\footnotesize Vitesse de rotation (en nombre de tours par seconde) }
               \uput[d](50,-1){\footnotesize Temps (en s)}
               \multido{\n=0+4}{26}{\psline[linewidth=0.6pt,linecolor=lightgray](\n,0)(\n,25)}
               \multido{\n=0+1}{26}{\psline[linewidth=0.6pt,linecolor=lightgray](0,\n)(100,\n)}
               \psaxes[linewidth=1pt,Dx=20,Dy=5,labelFontSize=\scriptstyle](0,0)(0,0)(100,25)
               \psline[linewidth=1.2pt,linecolor=blue](0,20)(94,0)
          \end{pspicture}
     \end{extern}
\end{center}
\emph{Inspiré de~: \Lien{https://www.sciencesetavenir.fr/fondamental/combien-de-temps-peut-tourner-votre-hand-spinner_112808}{Sciences et avenir : Combien de temps peut tourner votre Hand Spinner ?}}
\medbreak
\begin{enumerate}
     \item Le temps et la vitesse de rotation du \og hand-spinner \fg{} sont-ils proportionnels~? Justifier.
     \item Par \textbf{lecture graphique}, répondre aux questions suivantes~:
     \begin{enumerate}[label=\alph*.]
          \item Quelle est la vitesse de rotation initiale du \og hand-spinner \fg{} (en nombre de tours par seconde)~?
          \item Quelle est la vitesse de rotation du \og hand-spinner \fg{} (en nombre de tours par seconde) au bout d'une minute et vingt secondes~?
          \item Au bout de combien de temps, le \og hand-spinner \fg{} va-t-il s'arrêter~?
     \end{enumerate}
     \item  Pour calculer la vitesse de rotation du \og hand-spinner \fg{} en fonction du temps $t$, notée $V(t)$, on utilise la fonction suivante~:
     \par
     \[V(t) = - 0,214 \times t + V_{\text{initiale}}.\]
     \par
     $\bullet~~$ $t$ est le temps (exprimé en s) qui s'est écoulé depuis le début de rotation du \og hand-spinner \fg{}~;
     \par
     $\bullet~~$ $V_{\text{initiale}}$ est la vitesse de rotation à laquelle on a lancé le \og hand-spinner \fg{} au départ.
     \begin{enumerate}[label=\alph*.]
          \item On lance le \og hand-spinner \fg{} à une vitesse initiale de $20$ tours par seconde. Sa vitesse de rotation est donc donnée par la formule~:
          \par
          \[V(t) = - 0,214 \times t + 20.\]
          \par
          Calculer sa vitesse de rotation au bout de $30$~s.
          \item Au bout de combien de temps le hand-spinner va-t-il s'arrêter~? Justifier par un calcul.
          \item Est-il vrai que, d'une manière générale, si l'on fait tourner le hand-spinner deux fois plus vite au départ, il tournera deux fois plus longtemps~? Justifier.
     \end{enumerate}
\end{enumerate}

\end{document}
µ
\documentclass[a4paper]{article}

%================================================================================================================================
%
% Packages
%
%================================================================================================================================

\usepackage[T1]{fontenc} 	% pour caractères accentués
\usepackage[utf8]{inputenc}  % encodage utf8
\usepackage[french]{babel}	% langue : français
\usepackage{fourier}			% caractères plus lisibles
\usepackage[dvipsnames]{xcolor} % couleurs
\usepackage{fancyhdr}		% réglage header footer
\usepackage{needspace}		% empêcher sauts de page mal placés
\usepackage{graphicx}		% pour inclure des graphiques
\usepackage{enumitem,cprotect}		% personnalise les listes d'items (nécessaire pour ol, al ...)
\usepackage{hyperref}		% Liens hypertexte
\usepackage{pstricks,pst-all,pst-node,pstricks-add,pst-math,pst-plot,pst-tree,pst-eucl} % pstricks
\usepackage[a4paper,includeheadfoot,top=2cm,left=3cm, bottom=2cm,right=3cm]{geometry} % marges etc.
\usepackage{comment}			% commentaires multilignes
\usepackage{amsmath,environ} % maths (matrices, etc.)
\usepackage{amssymb,makeidx}
\usepackage{bm}				% bold maths
\usepackage{tabularx}		% tableaux
\usepackage{colortbl}		% tableaux en couleur
\usepackage{fontawesome}		% Fontawesome
\usepackage{environ}			% environment with command
\usepackage{fp}				% calculs pour ps-tricks
\usepackage{multido}			% pour ps tricks
\usepackage[np]{numprint}	% formattage nombre
\usepackage{tikz,tkz-tab} 			% package principal TikZ
\usepackage{pgfplots}   % axes
\usepackage{mathrsfs}    % cursives
\usepackage{calc}			% calcul taille boites
\usepackage[scaled=0.875]{helvet} % font sans serif
\usepackage{svg} % svg
\usepackage{scrextend} % local margin
\usepackage{scratch} %scratch
\usepackage{multicol} % colonnes
%\usepackage{infix-RPN,pst-func} % formule en notation polanaise inversée
\usepackage{listings}

%================================================================================================================================
%
% Réglages de base
%
%================================================================================================================================

\lstset{
language=Python,   % R code
literate=
{á}{{\'a}}1
{à}{{\`a}}1
{ã}{{\~a}}1
{é}{{\'e}}1
{è}{{\`e}}1
{ê}{{\^e}}1
{í}{{\'i}}1
{ó}{{\'o}}1
{õ}{{\~o}}1
{ú}{{\'u}}1
{ü}{{\"u}}1
{ç}{{\c{c}}}1
{~}{{ }}1
}


\definecolor{codegreen}{rgb}{0,0.6,0}
\definecolor{codegray}{rgb}{0.5,0.5,0.5}
\definecolor{codepurple}{rgb}{0.58,0,0.82}
\definecolor{backcolour}{rgb}{0.95,0.95,0.92}

\lstdefinestyle{mystyle}{
    backgroundcolor=\color{backcolour},   
    commentstyle=\color{codegreen},
    keywordstyle=\color{magenta},
    numberstyle=\tiny\color{codegray},
    stringstyle=\color{codepurple},
    basicstyle=\ttfamily\footnotesize,
    breakatwhitespace=false,         
    breaklines=true,                 
    captionpos=b,                    
    keepspaces=true,                 
    numbers=left,                    
xleftmargin=2em,
framexleftmargin=2em,            
    showspaces=false,                
    showstringspaces=false,
    showtabs=false,                  
    tabsize=2,
    upquote=true
}

\lstset{style=mystyle}


\lstset{style=mystyle}
\newcommand{\imgdir}{C:/laragon/www/newmc/assets/imgsvg/}
\newcommand{\imgsvgdir}{C:/laragon/www/newmc/assets/imgsvg/}

\definecolor{mcgris}{RGB}{220, 220, 220}% ancien~; pour compatibilité
\definecolor{mcbleu}{RGB}{52, 152, 219}
\definecolor{mcvert}{RGB}{125, 194, 70}
\definecolor{mcmauve}{RGB}{154, 0, 215}
\definecolor{mcorange}{RGB}{255, 96, 0}
\definecolor{mcturquoise}{RGB}{0, 153, 153}
\definecolor{mcrouge}{RGB}{255, 0, 0}
\definecolor{mclightvert}{RGB}{205, 234, 190}

\definecolor{gris}{RGB}{220, 220, 220}
\definecolor{bleu}{RGB}{52, 152, 219}
\definecolor{vert}{RGB}{125, 194, 70}
\definecolor{mauve}{RGB}{154, 0, 215}
\definecolor{orange}{RGB}{255, 96, 0}
\definecolor{turquoise}{RGB}{0, 153, 153}
\definecolor{rouge}{RGB}{255, 0, 0}
\definecolor{lightvert}{RGB}{205, 234, 190}
\setitemize[0]{label=\color{lightvert}  $\bullet$}

\pagestyle{fancy}
\renewcommand{\headrulewidth}{0.2pt}
\fancyhead[L]{maths-cours.fr}
\fancyhead[R]{\thepage}
\renewcommand{\footrulewidth}{0.2pt}
\fancyfoot[C]{}

\newcolumntype{C}{>{\centering\arraybackslash}X}
\newcolumntype{s}{>{\hsize=.35\hsize\arraybackslash}X}

\setlength{\parindent}{0pt}		 
\setlength{\parskip}{3mm}
\setlength{\headheight}{1cm}

\def\ebook{ebook}
\def\book{book}
\def\web{web}
\def\type{web}

\newcommand{\vect}[1]{\overrightarrow{\,\mathstrut#1\,}}

\def\Oij{$\left(\text{O}~;~\vect{\imath},~\vect{\jmath}\right)$}
\def\Oijk{$\left(\text{O}~;~\vect{\imath},~\vect{\jmath},~\vect{k}\right)$}
\def\Ouv{$\left(\text{O}~;~\vect{u},~\vect{v}\right)$}

\hypersetup{breaklinks=true, colorlinks = true, linkcolor = OliveGreen, urlcolor = OliveGreen, citecolor = OliveGreen, pdfauthor={Didier BONNEL - https://www.maths-cours.fr} } % supprime les bordures autour des liens

\renewcommand{\arg}[0]{\text{arg}}

\everymath{\displaystyle}

%================================================================================================================================
%
% Macros - Commandes
%
%================================================================================================================================

\newcommand\meta[2]{    			% Utilisé pour créer le post HTML.
	\def\titre{titre}
	\def\url{url}
	\def\arg{#1}
	\ifx\titre\arg
		\newcommand\maintitle{#2}
		\fancyhead[L]{#2}
		{\Large\sffamily \MakeUppercase{#2}}
		\vspace{1mm}\textcolor{mcvert}{\hrule}
	\fi 
	\ifx\url\arg
		\fancyfoot[L]{\href{https://www.maths-cours.fr#2}{\black \footnotesize{https://www.maths-cours.fr#2}}}
	\fi 
}


\newcommand\TitreC[1]{    		% Titre centré
     \needspace{3\baselineskip}
     \begin{center}\textbf{#1}\end{center}
}

\newcommand\newpar{    		% paragraphe
     \par
}

\newcommand\nosp {    		% commande vide (pas d'espace)
}
\newcommand{\id}[1]{} %ignore

\newcommand\boite[2]{				% Boite simple sans titre
	\vspace{5mm}
	\setlength{\fboxrule}{0.2mm}
	\setlength{\fboxsep}{5mm}	
	\fcolorbox{#1}{#1!3}{\makebox[\linewidth-2\fboxrule-2\fboxsep]{
  		\begin{minipage}[t]{\linewidth-2\fboxrule-4\fboxsep}\setlength{\parskip}{3mm}
  			 #2
  		\end{minipage}
	}}
	\vspace{5mm}
}

\newcommand\CBox[4]{				% Boites
	\vspace{5mm}
	\setlength{\fboxrule}{0.2mm}
	\setlength{\fboxsep}{5mm}
	
	\fcolorbox{#1}{#1!3}{\makebox[\linewidth-2\fboxrule-2\fboxsep]{
		\begin{minipage}[t]{1cm}\setlength{\parskip}{3mm}
	  		\textcolor{#1}{\LARGE{#2}}    
 	 	\end{minipage}  
  		\begin{minipage}[t]{\linewidth-2\fboxrule-4\fboxsep}\setlength{\parskip}{3mm}
			\raisebox{1.2mm}{\normalsize\sffamily{\textcolor{#1}{#3}}}						
  			 #4
  		\end{minipage}
	}}
	\vspace{5mm}
}

\newcommand\cadre[3]{				% Boites convertible html
	\par
	\vspace{2mm}
	\setlength{\fboxrule}{0.1mm}
	\setlength{\fboxsep}{5mm}
	\fcolorbox{#1}{white}{\makebox[\linewidth-2\fboxrule-2\fboxsep]{
  		\begin{minipage}[t]{\linewidth-2\fboxrule-4\fboxsep}\setlength{\parskip}{3mm}
			\raisebox{-2.5mm}{\sffamily \small{\textcolor{#1}{\MakeUppercase{#2}}}}		
			\par		
  			 #3
 	 		\end{minipage}
	}}
		\vspace{2mm}
	\par
}

\newcommand\bloc[3]{				% Boites convertible html sans bordure
     \needspace{2\baselineskip}
     {\sffamily \small{\textcolor{#1}{\MakeUppercase{#2}}}}    
		\par		
  			 #3
		\par
}

\newcommand\CHelp[1]{
     \CBox{Plum}{\faInfoCircle}{À RETENIR}{#1}
}

\newcommand\CUp[1]{
     \CBox{NavyBlue}{\faThumbsOUp}{EN PRATIQUE}{#1}
}

\newcommand\CInfo[1]{
     \CBox{Sepia}{\faArrowCircleRight}{REMARQUE}{#1}
}

\newcommand\CRedac[1]{
     \CBox{PineGreen}{\faEdit}{BIEN R\'EDIGER}{#1}
}

\newcommand\CError[1]{
     \CBox{Red}{\faExclamationTriangle}{ATTENTION}{#1}
}

\newcommand\TitreExo[2]{
\needspace{4\baselineskip}
 {\sffamily\large EXERCICE #1\ (\emph{#2 points})}
\vspace{5mm}
}

\newcommand\img[2]{
          \includegraphics[width=#2\paperwidth]{\imgdir#1}
}

\newcommand\imgsvg[2]{
       \begin{center}   \includegraphics[width=#2\paperwidth]{\imgsvgdir#1} \end{center}
}


\newcommand\Lien[2]{
     \href{#1}{#2 \tiny \faExternalLink}
}
\newcommand\mcLien[2]{
     \href{https~://www.maths-cours.fr/#1}{#2 \tiny \faExternalLink}
}

\newcommand{\euro}{\eurologo{}}

%================================================================================================================================
%
% Macros - Environement
%
%================================================================================================================================

\newenvironment{tex}{ %
}
{%
}

\newenvironment{indente}{ %
	\setlength\parindent{10mm}
}

{
	\setlength\parindent{0mm}
}

\newenvironment{corrige}{%
     \needspace{3\baselineskip}
     \medskip
     \textbf{\textsc{Corrigé}}
     \medskip
}
{
}

\newenvironment{extern}{%
     \begin{center}
     }
     {
     \end{center}
}

\NewEnviron{code}{%
	\par
     \boite{gray}{\texttt{%
     \BODY
     }}
     \par
}

\newenvironment{vbloc}{% boite sans cadre empeche saut de page
     \begin{minipage}[t]{\linewidth}
     }
     {
     \end{minipage}
}
\NewEnviron{h2}{%
    \needspace{3\baselineskip}
    \vspace{0.6cm}
	\noindent \MakeUppercase{\sffamily \large \BODY}
	\vspace{1mm}\textcolor{mcgris}{\hrule}\vspace{0.4cm}
	\par
}{}

\NewEnviron{h3}{%
    \needspace{3\baselineskip}
	\vspace{5mm}
	\textsc{\BODY}
	\par
}

\NewEnviron{margeneg}{ %
\begin{addmargin}[-1cm]{0cm}
\BODY
\end{addmargin}
}

\NewEnviron{html}{%
}

\begin{document}
\meta{url}{/exercices/probabilites-brevet-metropole-2017/}
\meta{pid}{9897}
\meta{titre}{Probabilités - Brevet Métropole 2017}
\meta{type}{exercices}
%
\begin{h2}Exercice 1 (4 points)\end{h2}
\medbreak
Dans une urne contenant des boules vertes et des boules bleues, on tire au hasard une boule et on regarde sa
couleur. On replace ensuite la boule dans l'urne et on mélange les boules.
\par
La probabilité d'obtenir une boule verte est $\dfrac{2}{5}$.
\medbreak
\begin{enumerate}
     \item Expliquer pourquoi la probabilité d'obtenir une boule bleue est égale à $\dfrac{3}{5}$.
     \item  Paul a effectué 6 tirages et a obtenu une boule verte à chaque fois.
     \par
     Au 7$^{e}$ tirage, aura-t-il plus de chances d'obtenir une boule bleue qu'une boule verte~?
     \item  Déterminer le nombre de boules bleues dans cette urne sachant qu'il y a 8 boules vertes.
\end{enumerate}
\medbreak

\end{document}
µ
\documentclass[a4paper]{article}

%================================================================================================================================
%
% Packages
%
%================================================================================================================================

\usepackage[T1]{fontenc} 	% pour caractères accentués
\usepackage[utf8]{inputenc}  % encodage utf8
\usepackage[french]{babel}	% langue : français
\usepackage{fourier}			% caractères plus lisibles
\usepackage[dvipsnames]{xcolor} % couleurs
\usepackage{fancyhdr}		% réglage header footer
\usepackage{needspace}		% empêcher sauts de page mal placés
\usepackage{graphicx}		% pour inclure des graphiques
\usepackage{enumitem,cprotect}		% personnalise les listes d'items (nécessaire pour ol, al ...)
\usepackage{hyperref}		% Liens hypertexte
\usepackage{pstricks,pst-all,pst-node,pstricks-add,pst-math,pst-plot,pst-tree,pst-eucl} % pstricks
\usepackage[a4paper,includeheadfoot,top=2cm,left=3cm, bottom=2cm,right=3cm]{geometry} % marges etc.
\usepackage{comment}			% commentaires multilignes
\usepackage{amsmath,environ} % maths (matrices, etc.)
\usepackage{amssymb,makeidx}
\usepackage{bm}				% bold maths
\usepackage{tabularx}		% tableaux
\usepackage{colortbl}		% tableaux en couleur
\usepackage{fontawesome}		% Fontawesome
\usepackage{environ}			% environment with command
\usepackage{fp}				% calculs pour ps-tricks
\usepackage{multido}			% pour ps tricks
\usepackage[np]{numprint}	% formattage nombre
\usepackage{tikz,tkz-tab} 			% package principal TikZ
\usepackage{pgfplots}   % axes
\usepackage{mathrsfs}    % cursives
\usepackage{calc}			% calcul taille boites
\usepackage[scaled=0.875]{helvet} % font sans serif
\usepackage{svg} % svg
\usepackage{scrextend} % local margin
\usepackage{scratch} %scratch
\usepackage{multicol} % colonnes
%\usepackage{infix-RPN,pst-func} % formule en notation polanaise inversée
\usepackage{listings}

%================================================================================================================================
%
% Réglages de base
%
%================================================================================================================================

\lstset{
language=Python,   % R code
literate=
{á}{{\'a}}1
{à}{{\`a}}1
{ã}{{\~a}}1
{é}{{\'e}}1
{è}{{\`e}}1
{ê}{{\^e}}1
{í}{{\'i}}1
{ó}{{\'o}}1
{õ}{{\~o}}1
{ú}{{\'u}}1
{ü}{{\"u}}1
{ç}{{\c{c}}}1
{~}{{ }}1
}


\definecolor{codegreen}{rgb}{0,0.6,0}
\definecolor{codegray}{rgb}{0.5,0.5,0.5}
\definecolor{codepurple}{rgb}{0.58,0,0.82}
\definecolor{backcolour}{rgb}{0.95,0.95,0.92}

\lstdefinestyle{mystyle}{
    backgroundcolor=\color{backcolour},   
    commentstyle=\color{codegreen},
    keywordstyle=\color{magenta},
    numberstyle=\tiny\color{codegray},
    stringstyle=\color{codepurple},
    basicstyle=\ttfamily\footnotesize,
    breakatwhitespace=false,         
    breaklines=true,                 
    captionpos=b,                    
    keepspaces=true,                 
    numbers=left,                    
xleftmargin=2em,
framexleftmargin=2em,            
    showspaces=false,                
    showstringspaces=false,
    showtabs=false,                  
    tabsize=2,
    upquote=true
}

\lstset{style=mystyle}


\lstset{style=mystyle}
\newcommand{\imgdir}{C:/laragon/www/newmc/assets/imgsvg/}
\newcommand{\imgsvgdir}{C:/laragon/www/newmc/assets/imgsvg/}

\definecolor{mcgris}{RGB}{220, 220, 220}% ancien~; pour compatibilité
\definecolor{mcbleu}{RGB}{52, 152, 219}
\definecolor{mcvert}{RGB}{125, 194, 70}
\definecolor{mcmauve}{RGB}{154, 0, 215}
\definecolor{mcorange}{RGB}{255, 96, 0}
\definecolor{mcturquoise}{RGB}{0, 153, 153}
\definecolor{mcrouge}{RGB}{255, 0, 0}
\definecolor{mclightvert}{RGB}{205, 234, 190}

\definecolor{gris}{RGB}{220, 220, 220}
\definecolor{bleu}{RGB}{52, 152, 219}
\definecolor{vert}{RGB}{125, 194, 70}
\definecolor{mauve}{RGB}{154, 0, 215}
\definecolor{orange}{RGB}{255, 96, 0}
\definecolor{turquoise}{RGB}{0, 153, 153}
\definecolor{rouge}{RGB}{255, 0, 0}
\definecolor{lightvert}{RGB}{205, 234, 190}
\setitemize[0]{label=\color{lightvert}  $\bullet$}

\pagestyle{fancy}
\renewcommand{\headrulewidth}{0.2pt}
\fancyhead[L]{maths-cours.fr}
\fancyhead[R]{\thepage}
\renewcommand{\footrulewidth}{0.2pt}
\fancyfoot[C]{}

\newcolumntype{C}{>{\centering\arraybackslash}X}
\newcolumntype{s}{>{\hsize=.35\hsize\arraybackslash}X}

\setlength{\parindent}{0pt}		 
\setlength{\parskip}{3mm}
\setlength{\headheight}{1cm}

\def\ebook{ebook}
\def\book{book}
\def\web{web}
\def\type{web}

\newcommand{\vect}[1]{\overrightarrow{\,\mathstrut#1\,}}

\def\Oij{$\left(\text{O}~;~\vect{\imath},~\vect{\jmath}\right)$}
\def\Oijk{$\left(\text{O}~;~\vect{\imath},~\vect{\jmath},~\vect{k}\right)$}
\def\Ouv{$\left(\text{O}~;~\vect{u},~\vect{v}\right)$}

\hypersetup{breaklinks=true, colorlinks = true, linkcolor = OliveGreen, urlcolor = OliveGreen, citecolor = OliveGreen, pdfauthor={Didier BONNEL - https://www.maths-cours.fr} } % supprime les bordures autour des liens

\renewcommand{\arg}[0]{\text{arg}}

\everymath{\displaystyle}

%================================================================================================================================
%
% Macros - Commandes
%
%================================================================================================================================

\newcommand\meta[2]{    			% Utilisé pour créer le post HTML.
	\def\titre{titre}
	\def\url{url}
	\def\arg{#1}
	\ifx\titre\arg
		\newcommand\maintitle{#2}
		\fancyhead[L]{#2}
		{\Large\sffamily \MakeUppercase{#2}}
		\vspace{1mm}\textcolor{mcvert}{\hrule}
	\fi 
	\ifx\url\arg
		\fancyfoot[L]{\href{https://www.maths-cours.fr#2}{\black \footnotesize{https://www.maths-cours.fr#2}}}
	\fi 
}


\newcommand\TitreC[1]{    		% Titre centré
     \needspace{3\baselineskip}
     \begin{center}\textbf{#1}\end{center}
}

\newcommand\newpar{    		% paragraphe
     \par
}

\newcommand\nosp {    		% commande vide (pas d'espace)
}
\newcommand{\id}[1]{} %ignore

\newcommand\boite[2]{				% Boite simple sans titre
	\vspace{5mm}
	\setlength{\fboxrule}{0.2mm}
	\setlength{\fboxsep}{5mm}	
	\fcolorbox{#1}{#1!3}{\makebox[\linewidth-2\fboxrule-2\fboxsep]{
  		\begin{minipage}[t]{\linewidth-2\fboxrule-4\fboxsep}\setlength{\parskip}{3mm}
  			 #2
  		\end{minipage}
	}}
	\vspace{5mm}
}

\newcommand\CBox[4]{				% Boites
	\vspace{5mm}
	\setlength{\fboxrule}{0.2mm}
	\setlength{\fboxsep}{5mm}
	
	\fcolorbox{#1}{#1!3}{\makebox[\linewidth-2\fboxrule-2\fboxsep]{
		\begin{minipage}[t]{1cm}\setlength{\parskip}{3mm}
	  		\textcolor{#1}{\LARGE{#2}}    
 	 	\end{minipage}  
  		\begin{minipage}[t]{\linewidth-2\fboxrule-4\fboxsep}\setlength{\parskip}{3mm}
			\raisebox{1.2mm}{\normalsize\sffamily{\textcolor{#1}{#3}}}						
  			 #4
  		\end{minipage}
	}}
	\vspace{5mm}
}

\newcommand\cadre[3]{				% Boites convertible html
	\par
	\vspace{2mm}
	\setlength{\fboxrule}{0.1mm}
	\setlength{\fboxsep}{5mm}
	\fcolorbox{#1}{white}{\makebox[\linewidth-2\fboxrule-2\fboxsep]{
  		\begin{minipage}[t]{\linewidth-2\fboxrule-4\fboxsep}\setlength{\parskip}{3mm}
			\raisebox{-2.5mm}{\sffamily \small{\textcolor{#1}{\MakeUppercase{#2}}}}		
			\par		
  			 #3
 	 		\end{minipage}
	}}
		\vspace{2mm}
	\par
}

\newcommand\bloc[3]{				% Boites convertible html sans bordure
     \needspace{2\baselineskip}
     {\sffamily \small{\textcolor{#1}{\MakeUppercase{#2}}}}    
		\par		
  			 #3
		\par
}

\newcommand\CHelp[1]{
     \CBox{Plum}{\faInfoCircle}{À RETENIR}{#1}
}

\newcommand\CUp[1]{
     \CBox{NavyBlue}{\faThumbsOUp}{EN PRATIQUE}{#1}
}

\newcommand\CInfo[1]{
     \CBox{Sepia}{\faArrowCircleRight}{REMARQUE}{#1}
}

\newcommand\CRedac[1]{
     \CBox{PineGreen}{\faEdit}{BIEN R\'EDIGER}{#1}
}

\newcommand\CError[1]{
     \CBox{Red}{\faExclamationTriangle}{ATTENTION}{#1}
}

\newcommand\TitreExo[2]{
\needspace{4\baselineskip}
 {\sffamily\large EXERCICE #1\ (\emph{#2 points})}
\vspace{5mm}
}

\newcommand\img[2]{
          \includegraphics[width=#2\paperwidth]{\imgdir#1}
}

\newcommand\imgsvg[2]{
       \begin{center}   \includegraphics[width=#2\paperwidth]{\imgsvgdir#1} \end{center}
}


\newcommand\Lien[2]{
     \href{#1}{#2 \tiny \faExternalLink}
}
\newcommand\mcLien[2]{
     \href{https~://www.maths-cours.fr/#1}{#2 \tiny \faExternalLink}
}

\newcommand{\euro}{\eurologo{}}

%================================================================================================================================
%
% Macros - Environement
%
%================================================================================================================================

\newenvironment{tex}{ %
}
{%
}

\newenvironment{indente}{ %
	\setlength\parindent{10mm}
}

{
	\setlength\parindent{0mm}
}

\newenvironment{corrige}{%
     \needspace{3\baselineskip}
     \medskip
     \textbf{\textsc{Corrigé}}
     \medskip
}
{
}

\newenvironment{extern}{%
     \begin{center}
     }
     {
     \end{center}
}

\NewEnviron{code}{%
	\par
     \boite{gray}{\texttt{%
     \BODY
     }}
     \par
}

\newenvironment{vbloc}{% boite sans cadre empeche saut de page
     \begin{minipage}[t]{\linewidth}
     }
     {
     \end{minipage}
}
\NewEnviron{h2}{%
    \needspace{3\baselineskip}
    \vspace{0.6cm}
	\noindent \MakeUppercase{\sffamily \large \BODY}
	\vspace{1mm}\textcolor{mcgris}{\hrule}\vspace{0.4cm}
	\par
}{}

\NewEnviron{h3}{%
    \needspace{3\baselineskip}
	\vspace{5mm}
	\textsc{\BODY}
	\par
}

\NewEnviron{margeneg}{ %
\begin{addmargin}[-1cm]{0cm}
\BODY
\end{addmargin}
}

\NewEnviron{html}{%
}

\begin{document}
\meta{url}{/exercices/programme-scratch-brevet-metropole-2017/}
\meta{pid}{9900}
\meta{titre}{Programme Scratch - Brevet Métropole 2017}
\meta{type}{exercices}
%
\begin{h2}Exercice 2 (6 points)\end{h2}
\medbreak
On donne le programme suivant qui permet de tracer plusieurs triangles équilatéraux de tailles différentes.
\par
Ce programme comporte une variable nommée \og \textbf{côté} \fg. Les longueurs sont données en pixels.
\par
On rappelle que l'instruction :
\begin{center}
     \begin{extern}%width="140" alt="Instruction scratch"
          \begin{scratch}
               \blockmove{s'orienter à  \ovalnum{90\selectarrownum}}
          \end{scratch}
     \end{extern}
\end{center}
signifie que l'on se dirige vers la droite.
\begin{center}
     \begin{extern}%width="300" alt="Programme scratch"
          \begin{tabular}[t]{cc}
               \begin{tabular}[t]{c} Numéros\\ d'instruction\\\end{tabular}&Script\\
               \raisebox{-\height}{
                    \renewcommand{\arraystretch}{1.82}\begin{tabular}{c}
                         \\[-.5cm]
                         1\\
                         2\\
                         3\\
                         4\\
                         5\\
                         6\\
                         7\\
                         8\\
                         9\\
                    \end{tabular}
               }
               & \raisebox{-\height}{
                    \begin{scratch}
                         \blockinit{Quand \greenflag est cliqué}
                         \blockpen{effacer tout}
                         \blockmove{aller à x~:\ovalnum{$-200$} y~: \ovalnum{$-100$}}
                         \blockmove{s'orienter à  \ovalnum{90\selectarrownum}}
                         \blockvariable{Mettre \selectmenu{côté} à \txtbox{100}}
                         \blockrepeat{répéter \ovalnum{$5$} fois}
                         {
                              \blockmoreblocks{triangle}
                              \blockmove{avancer de \ovalvariable{côté}  }
                              \blockvariable{Ajouter à  \selectmenu{côté} \ovalnum{$-20$}}
                         }
                    \end{scratch}
               }
          \end{tabular}
     \end{extern}
\end{center}
Le bloc \textbf{triangle} :
\begin{center}
     \begin{extern}%width="210" alt="bloc scratch"
          \begin{scratch}
               \initmoreblocks{définir \namemoreblocks{triangle}}
               \blockpen{stylo en position écriture}
               \blockrepeat{répéter \ovalnum{3} fois}
               {
                    \blockmove{avancer de \ovalvariable{côté} }
                    \blockmove{tourner \turnleft{} de \ovalnum{120} degrés}
               }
               \blockpen{relever le stylo}
          \end{scratch}
     \end{extern}
\end{center}
\begin{enumerate}
     \item Quelles sont les coordonnées du point de départ du tracé~?
     \item Combien de triangles sont dessinés par le script~?
     \item
     \begin{enumerate}[label=\alph*.]
          \item Quelle est la longueur (en pixels) du côté du deuxième triangle tracé~?
          \item Tracer à main levée l'allure de la figure obtenue quand on exécute ce script.
     \end{enumerate}
     \item  On modifie le script initial pour obtenir la figure ci-dessous.
     Indiquer le numéro d'une instruction du script \textbf{après laquelle} on peut placer l'instruction :
     \begin{center}
          \begin{extern}%width="190" alt="Instruction scratch"
               \begin{scratch}
                    \blockmove{tourner \turnleft{} de \ovalnum{60} degrés}
               \end{scratch}
          \end{extern}
     \end{center}
     pour obtenir cette nouvelle figure~:
\end{enumerate}
\begin{center}
     \begin{extern}%width="240" alt="Triangles"
          \psset{unit=1cm}
          \begin{pspicture}(5,5)
               %\psgrid
               \def\tria{\pspolygon(2;-30)(2;90)(2;210)}
               \def\trib{\pspolygon(1.6;-30)(1.6;90)(1.6;210)}
               \def\tric{\pspolygon(1.2;-30)(1.22;90)(1.22;210)}
               \def\trid{\pspolygon(0.8;-30)(0.8;90)(0.8;210)}
               \def\trie{\pspolygon(0.4;-30)(0.4;90)(0.4;210)}
               \rput(2,1.5){\tria}\rput(3.75,2.1){\rotatedown{\trib}}
               \rput(4.1,3.5){\tric}\rput(3.4,4.3){\rotatedown{\trid}}
               \rput(2.7,4.3){\trie}
          \end{pspicture}
     \end{extern}
\end{center}

\end{document}
µ
\documentclass[a4paper]{article}

%================================================================================================================================
%
% Packages
%
%================================================================================================================================

\usepackage[T1]{fontenc} 	% pour caractères accentués
\usepackage[utf8]{inputenc}  % encodage utf8
\usepackage[french]{babel}	% langue : français
\usepackage{fourier}			% caractères plus lisibles
\usepackage[dvipsnames]{xcolor} % couleurs
\usepackage{fancyhdr}		% réglage header footer
\usepackage{needspace}		% empêcher sauts de page mal placés
\usepackage{graphicx}		% pour inclure des graphiques
\usepackage{enumitem,cprotect}		% personnalise les listes d'items (nécessaire pour ol, al ...)
\usepackage{hyperref}		% Liens hypertexte
\usepackage{pstricks,pst-all,pst-node,pstricks-add,pst-math,pst-plot,pst-tree,pst-eucl} % pstricks
\usepackage[a4paper,includeheadfoot,top=2cm,left=3cm, bottom=2cm,right=3cm]{geometry} % marges etc.
\usepackage{comment}			% commentaires multilignes
\usepackage{amsmath,environ} % maths (matrices, etc.)
\usepackage{amssymb,makeidx}
\usepackage{bm}				% bold maths
\usepackage{tabularx}		% tableaux
\usepackage{colortbl}		% tableaux en couleur
\usepackage{fontawesome}		% Fontawesome
\usepackage{environ}			% environment with command
\usepackage{fp}				% calculs pour ps-tricks
\usepackage{multido}			% pour ps tricks
\usepackage[np]{numprint}	% formattage nombre
\usepackage{tikz,tkz-tab} 			% package principal TikZ
\usepackage{pgfplots}   % axes
\usepackage{mathrsfs}    % cursives
\usepackage{calc}			% calcul taille boites
\usepackage[scaled=0.875]{helvet} % font sans serif
\usepackage{svg} % svg
\usepackage{scrextend} % local margin
\usepackage{scratch} %scratch
\usepackage{multicol} % colonnes
%\usepackage{infix-RPN,pst-func} % formule en notation polanaise inversée
\usepackage{listings}

%================================================================================================================================
%
% Réglages de base
%
%================================================================================================================================

\lstset{
language=Python,   % R code
literate=
{á}{{\'a}}1
{à}{{\`a}}1
{ã}{{\~a}}1
{é}{{\'e}}1
{è}{{\`e}}1
{ê}{{\^e}}1
{í}{{\'i}}1
{ó}{{\'o}}1
{õ}{{\~o}}1
{ú}{{\'u}}1
{ü}{{\"u}}1
{ç}{{\c{c}}}1
{~}{{ }}1
}


\definecolor{codegreen}{rgb}{0,0.6,0}
\definecolor{codegray}{rgb}{0.5,0.5,0.5}
\definecolor{codepurple}{rgb}{0.58,0,0.82}
\definecolor{backcolour}{rgb}{0.95,0.95,0.92}

\lstdefinestyle{mystyle}{
    backgroundcolor=\color{backcolour},   
    commentstyle=\color{codegreen},
    keywordstyle=\color{magenta},
    numberstyle=\tiny\color{codegray},
    stringstyle=\color{codepurple},
    basicstyle=\ttfamily\footnotesize,
    breakatwhitespace=false,         
    breaklines=true,                 
    captionpos=b,                    
    keepspaces=true,                 
    numbers=left,                    
xleftmargin=2em,
framexleftmargin=2em,            
    showspaces=false,                
    showstringspaces=false,
    showtabs=false,                  
    tabsize=2,
    upquote=true
}

\lstset{style=mystyle}


\lstset{style=mystyle}
\newcommand{\imgdir}{C:/laragon/www/newmc/assets/imgsvg/}
\newcommand{\imgsvgdir}{C:/laragon/www/newmc/assets/imgsvg/}

\definecolor{mcgris}{RGB}{220, 220, 220}% ancien~; pour compatibilité
\definecolor{mcbleu}{RGB}{52, 152, 219}
\definecolor{mcvert}{RGB}{125, 194, 70}
\definecolor{mcmauve}{RGB}{154, 0, 215}
\definecolor{mcorange}{RGB}{255, 96, 0}
\definecolor{mcturquoise}{RGB}{0, 153, 153}
\definecolor{mcrouge}{RGB}{255, 0, 0}
\definecolor{mclightvert}{RGB}{205, 234, 190}

\definecolor{gris}{RGB}{220, 220, 220}
\definecolor{bleu}{RGB}{52, 152, 219}
\definecolor{vert}{RGB}{125, 194, 70}
\definecolor{mauve}{RGB}{154, 0, 215}
\definecolor{orange}{RGB}{255, 96, 0}
\definecolor{turquoise}{RGB}{0, 153, 153}
\definecolor{rouge}{RGB}{255, 0, 0}
\definecolor{lightvert}{RGB}{205, 234, 190}
\setitemize[0]{label=\color{lightvert}  $\bullet$}

\pagestyle{fancy}
\renewcommand{\headrulewidth}{0.2pt}
\fancyhead[L]{maths-cours.fr}
\fancyhead[R]{\thepage}
\renewcommand{\footrulewidth}{0.2pt}
\fancyfoot[C]{}

\newcolumntype{C}{>{\centering\arraybackslash}X}
\newcolumntype{s}{>{\hsize=.35\hsize\arraybackslash}X}

\setlength{\parindent}{0pt}		 
\setlength{\parskip}{3mm}
\setlength{\headheight}{1cm}

\def\ebook{ebook}
\def\book{book}
\def\web{web}
\def\type{web}

\newcommand{\vect}[1]{\overrightarrow{\,\mathstrut#1\,}}

\def\Oij{$\left(\text{O}~;~\vect{\imath},~\vect{\jmath}\right)$}
\def\Oijk{$\left(\text{O}~;~\vect{\imath},~\vect{\jmath},~\vect{k}\right)$}
\def\Ouv{$\left(\text{O}~;~\vect{u},~\vect{v}\right)$}

\hypersetup{breaklinks=true, colorlinks = true, linkcolor = OliveGreen, urlcolor = OliveGreen, citecolor = OliveGreen, pdfauthor={Didier BONNEL - https://www.maths-cours.fr} } % supprime les bordures autour des liens

\renewcommand{\arg}[0]{\text{arg}}

\everymath{\displaystyle}

%================================================================================================================================
%
% Macros - Commandes
%
%================================================================================================================================

\newcommand\meta[2]{    			% Utilisé pour créer le post HTML.
	\def\titre{titre}
	\def\url{url}
	\def\arg{#1}
	\ifx\titre\arg
		\newcommand\maintitle{#2}
		\fancyhead[L]{#2}
		{\Large\sffamily \MakeUppercase{#2}}
		\vspace{1mm}\textcolor{mcvert}{\hrule}
	\fi 
	\ifx\url\arg
		\fancyfoot[L]{\href{https://www.maths-cours.fr#2}{\black \footnotesize{https://www.maths-cours.fr#2}}}
	\fi 
}


\newcommand\TitreC[1]{    		% Titre centré
     \needspace{3\baselineskip}
     \begin{center}\textbf{#1}\end{center}
}

\newcommand\newpar{    		% paragraphe
     \par
}

\newcommand\nosp {    		% commande vide (pas d'espace)
}
\newcommand{\id}[1]{} %ignore

\newcommand\boite[2]{				% Boite simple sans titre
	\vspace{5mm}
	\setlength{\fboxrule}{0.2mm}
	\setlength{\fboxsep}{5mm}	
	\fcolorbox{#1}{#1!3}{\makebox[\linewidth-2\fboxrule-2\fboxsep]{
  		\begin{minipage}[t]{\linewidth-2\fboxrule-4\fboxsep}\setlength{\parskip}{3mm}
  			 #2
  		\end{minipage}
	}}
	\vspace{5mm}
}

\newcommand\CBox[4]{				% Boites
	\vspace{5mm}
	\setlength{\fboxrule}{0.2mm}
	\setlength{\fboxsep}{5mm}
	
	\fcolorbox{#1}{#1!3}{\makebox[\linewidth-2\fboxrule-2\fboxsep]{
		\begin{minipage}[t]{1cm}\setlength{\parskip}{3mm}
	  		\textcolor{#1}{\LARGE{#2}}    
 	 	\end{minipage}  
  		\begin{minipage}[t]{\linewidth-2\fboxrule-4\fboxsep}\setlength{\parskip}{3mm}
			\raisebox{1.2mm}{\normalsize\sffamily{\textcolor{#1}{#3}}}						
  			 #4
  		\end{minipage}
	}}
	\vspace{5mm}
}

\newcommand\cadre[3]{				% Boites convertible html
	\par
	\vspace{2mm}
	\setlength{\fboxrule}{0.1mm}
	\setlength{\fboxsep}{5mm}
	\fcolorbox{#1}{white}{\makebox[\linewidth-2\fboxrule-2\fboxsep]{
  		\begin{minipage}[t]{\linewidth-2\fboxrule-4\fboxsep}\setlength{\parskip}{3mm}
			\raisebox{-2.5mm}{\sffamily \small{\textcolor{#1}{\MakeUppercase{#2}}}}		
			\par		
  			 #3
 	 		\end{minipage}
	}}
		\vspace{2mm}
	\par
}

\newcommand\bloc[3]{				% Boites convertible html sans bordure
     \needspace{2\baselineskip}
     {\sffamily \small{\textcolor{#1}{\MakeUppercase{#2}}}}    
		\par		
  			 #3
		\par
}

\newcommand\CHelp[1]{
     \CBox{Plum}{\faInfoCircle}{À RETENIR}{#1}
}

\newcommand\CUp[1]{
     \CBox{NavyBlue}{\faThumbsOUp}{EN PRATIQUE}{#1}
}

\newcommand\CInfo[1]{
     \CBox{Sepia}{\faArrowCircleRight}{REMARQUE}{#1}
}

\newcommand\CRedac[1]{
     \CBox{PineGreen}{\faEdit}{BIEN R\'EDIGER}{#1}
}

\newcommand\CError[1]{
     \CBox{Red}{\faExclamationTriangle}{ATTENTION}{#1}
}

\newcommand\TitreExo[2]{
\needspace{4\baselineskip}
 {\sffamily\large EXERCICE #1\ (\emph{#2 points})}
\vspace{5mm}
}

\newcommand\img[2]{
          \includegraphics[width=#2\paperwidth]{\imgdir#1}
}

\newcommand\imgsvg[2]{
       \begin{center}   \includegraphics[width=#2\paperwidth]{\imgsvgdir#1} \end{center}
}


\newcommand\Lien[2]{
     \href{#1}{#2 \tiny \faExternalLink}
}
\newcommand\mcLien[2]{
     \href{https~://www.maths-cours.fr/#1}{#2 \tiny \faExternalLink}
}

\newcommand{\euro}{\eurologo{}}

%================================================================================================================================
%
% Macros - Environement
%
%================================================================================================================================

\newenvironment{tex}{ %
}
{%
}

\newenvironment{indente}{ %
	\setlength\parindent{10mm}
}

{
	\setlength\parindent{0mm}
}

\newenvironment{corrige}{%
     \needspace{3\baselineskip}
     \medskip
     \textbf{\textsc{Corrigé}}
     \medskip
}
{
}

\newenvironment{extern}{%
     \begin{center}
     }
     {
     \end{center}
}

\NewEnviron{code}{%
	\par
     \boite{gray}{\texttt{%
     \BODY
     }}
     \par
}

\newenvironment{vbloc}{% boite sans cadre empeche saut de page
     \begin{minipage}[t]{\linewidth}
     }
     {
     \end{minipage}
}
\NewEnviron{h2}{%
    \needspace{3\baselineskip}
    \vspace{0.6cm}
	\noindent \MakeUppercase{\sffamily \large \BODY}
	\vspace{1mm}\textcolor{mcgris}{\hrule}\vspace{0.4cm}
	\par
}{}

\NewEnviron{h3}{%
    \needspace{3\baselineskip}
	\vspace{5mm}
	\textsc{\BODY}
	\par
}

\NewEnviron{margeneg}{ %
\begin{addmargin}[-1cm]{0cm}
\BODY
\end{addmargin}
}

\NewEnviron{html}{%
}

\begin{document}
\meta{url}{/exercices/fonctions-brevet-metropole-2017/}
\meta{pid}{9906}
\meta{titre}{Fonctions - Brevet Métropole 2017}
\meta{type}{exercices}
%
\begin{h2}Exercice 3 (4 points)\end{h2}
\medbreak
Un condensateur est un composant électronique qui permet de stocker de l'énergie électrique pour la restituer plus tard.
\par
Le graphique suivant montre l'évolution de la tension mesurée aux bornes d'un condensateur en fonction du
temps lorsqu'il est en charge.
\begin{center}
     \begin{extern}%width="420" alt="Graphique fonction tension"
          \psset{xunit=14cm,yunit=0.8cm,comma=true}
          \begin{pspicture}(-0.1,-1)(0.6,6)
               \multido{\n=0.00+0.01}{61}{\psline[linewidth=0.2pt,linecolor=lightgray](\n,0)(\n,6)}
               \multido{\n=0.0+0.1}{7}{\psline[linewidth=0.5pt,linecolor=gray](\n,0)(\n,6)}
               \multido{\n=0.0+0.2}{31}{\psline[linewidth=0.2pt,linecolor=lightgray](0,\n)(0.6,\n)}
               \multido{\n=0+1}{7}{\psline[linewidth=0.5pt,linecolor=gray](0,\n)(0.6,\n)}
               \psaxes[linewidth=1pt,Dx=0.1]{->}(0,0)(0,0)(0.6,6)
               \psaxes[linewidth=1pt,Dx=0.1](0,0)(0,0)(0.61,6)
               \uput[d](0.55,-0.6){Temps (s)}
               \rput{90}(-0.055,5){Tension (V)}
               \psplot[plotpoints=3000,linewidth=1pt,linecolor=blue]{0}{0.6}{5 5 2.71828 x 10 mul  exp div sub}
          \end{pspicture}
     \end{extern}
\end{center}
\begin{enumerate}
     \item S'agit-il d'une situation de proportionnalité~? Justifier.
     \item Quelle est la tension mesurée au bout de $0,2$~s~?
     \item Au bout de combien de temps la tension aux bornes du condensateur aura-t-elle atteint 60\,\% de la tension maximale qui est estimée à 5 V~?
\end{enumerate}

\end{document}
µ
\documentclass[a4paper]{article}

%================================================================================================================================
%
% Packages
%
%================================================================================================================================

\usepackage[T1]{fontenc} 	% pour caractères accentués
\usepackage[utf8]{inputenc}  % encodage utf8
\usepackage[french]{babel}	% langue : français
\usepackage{fourier}			% caractères plus lisibles
\usepackage[dvipsnames]{xcolor} % couleurs
\usepackage{fancyhdr}		% réglage header footer
\usepackage{needspace}		% empêcher sauts de page mal placés
\usepackage{graphicx}		% pour inclure des graphiques
\usepackage{enumitem,cprotect}		% personnalise les listes d'items (nécessaire pour ol, al ...)
\usepackage{hyperref}		% Liens hypertexte
\usepackage{pstricks,pst-all,pst-node,pstricks-add,pst-math,pst-plot,pst-tree,pst-eucl} % pstricks
\usepackage[a4paper,includeheadfoot,top=2cm,left=3cm, bottom=2cm,right=3cm]{geometry} % marges etc.
\usepackage{comment}			% commentaires multilignes
\usepackage{amsmath,environ} % maths (matrices, etc.)
\usepackage{amssymb,makeidx}
\usepackage{bm}				% bold maths
\usepackage{tabularx}		% tableaux
\usepackage{colortbl}		% tableaux en couleur
\usepackage{fontawesome}		% Fontawesome
\usepackage{environ}			% environment with command
\usepackage{fp}				% calculs pour ps-tricks
\usepackage{multido}			% pour ps tricks
\usepackage[np]{numprint}	% formattage nombre
\usepackage{tikz,tkz-tab} 			% package principal TikZ
\usepackage{pgfplots}   % axes
\usepackage{mathrsfs}    % cursives
\usepackage{calc}			% calcul taille boites
\usepackage[scaled=0.875]{helvet} % font sans serif
\usepackage{svg} % svg
\usepackage{scrextend} % local margin
\usepackage{scratch} %scratch
\usepackage{multicol} % colonnes
%\usepackage{infix-RPN,pst-func} % formule en notation polanaise inversée
\usepackage{listings}

%================================================================================================================================
%
% Réglages de base
%
%================================================================================================================================

\lstset{
language=Python,   % R code
literate=
{á}{{\'a}}1
{à}{{\`a}}1
{ã}{{\~a}}1
{é}{{\'e}}1
{è}{{\`e}}1
{ê}{{\^e}}1
{í}{{\'i}}1
{ó}{{\'o}}1
{õ}{{\~o}}1
{ú}{{\'u}}1
{ü}{{\"u}}1
{ç}{{\c{c}}}1
{~}{{ }}1
}


\definecolor{codegreen}{rgb}{0,0.6,0}
\definecolor{codegray}{rgb}{0.5,0.5,0.5}
\definecolor{codepurple}{rgb}{0.58,0,0.82}
\definecolor{backcolour}{rgb}{0.95,0.95,0.92}

\lstdefinestyle{mystyle}{
    backgroundcolor=\color{backcolour},   
    commentstyle=\color{codegreen},
    keywordstyle=\color{magenta},
    numberstyle=\tiny\color{codegray},
    stringstyle=\color{codepurple},
    basicstyle=\ttfamily\footnotesize,
    breakatwhitespace=false,         
    breaklines=true,                 
    captionpos=b,                    
    keepspaces=true,                 
    numbers=left,                    
xleftmargin=2em,
framexleftmargin=2em,            
    showspaces=false,                
    showstringspaces=false,
    showtabs=false,                  
    tabsize=2,
    upquote=true
}

\lstset{style=mystyle}


\lstset{style=mystyle}
\newcommand{\imgdir}{C:/laragon/www/newmc/assets/imgsvg/}
\newcommand{\imgsvgdir}{C:/laragon/www/newmc/assets/imgsvg/}

\definecolor{mcgris}{RGB}{220, 220, 220}% ancien~; pour compatibilité
\definecolor{mcbleu}{RGB}{52, 152, 219}
\definecolor{mcvert}{RGB}{125, 194, 70}
\definecolor{mcmauve}{RGB}{154, 0, 215}
\definecolor{mcorange}{RGB}{255, 96, 0}
\definecolor{mcturquoise}{RGB}{0, 153, 153}
\definecolor{mcrouge}{RGB}{255, 0, 0}
\definecolor{mclightvert}{RGB}{205, 234, 190}

\definecolor{gris}{RGB}{220, 220, 220}
\definecolor{bleu}{RGB}{52, 152, 219}
\definecolor{vert}{RGB}{125, 194, 70}
\definecolor{mauve}{RGB}{154, 0, 215}
\definecolor{orange}{RGB}{255, 96, 0}
\definecolor{turquoise}{RGB}{0, 153, 153}
\definecolor{rouge}{RGB}{255, 0, 0}
\definecolor{lightvert}{RGB}{205, 234, 190}
\setitemize[0]{label=\color{lightvert}  $\bullet$}

\pagestyle{fancy}
\renewcommand{\headrulewidth}{0.2pt}
\fancyhead[L]{maths-cours.fr}
\fancyhead[R]{\thepage}
\renewcommand{\footrulewidth}{0.2pt}
\fancyfoot[C]{}

\newcolumntype{C}{>{\centering\arraybackslash}X}
\newcolumntype{s}{>{\hsize=.35\hsize\arraybackslash}X}

\setlength{\parindent}{0pt}		 
\setlength{\parskip}{3mm}
\setlength{\headheight}{1cm}

\def\ebook{ebook}
\def\book{book}
\def\web{web}
\def\type{web}

\newcommand{\vect}[1]{\overrightarrow{\,\mathstrut#1\,}}

\def\Oij{$\left(\text{O}~;~\vect{\imath},~\vect{\jmath}\right)$}
\def\Oijk{$\left(\text{O}~;~\vect{\imath},~\vect{\jmath},~\vect{k}\right)$}
\def\Ouv{$\left(\text{O}~;~\vect{u},~\vect{v}\right)$}

\hypersetup{breaklinks=true, colorlinks = true, linkcolor = OliveGreen, urlcolor = OliveGreen, citecolor = OliveGreen, pdfauthor={Didier BONNEL - https://www.maths-cours.fr} } % supprime les bordures autour des liens

\renewcommand{\arg}[0]{\text{arg}}

\everymath{\displaystyle}

%================================================================================================================================
%
% Macros - Commandes
%
%================================================================================================================================

\newcommand\meta[2]{    			% Utilisé pour créer le post HTML.
	\def\titre{titre}
	\def\url{url}
	\def\arg{#1}
	\ifx\titre\arg
		\newcommand\maintitle{#2}
		\fancyhead[L]{#2}
		{\Large\sffamily \MakeUppercase{#2}}
		\vspace{1mm}\textcolor{mcvert}{\hrule}
	\fi 
	\ifx\url\arg
		\fancyfoot[L]{\href{https://www.maths-cours.fr#2}{\black \footnotesize{https://www.maths-cours.fr#2}}}
	\fi 
}


\newcommand\TitreC[1]{    		% Titre centré
     \needspace{3\baselineskip}
     \begin{center}\textbf{#1}\end{center}
}

\newcommand\newpar{    		% paragraphe
     \par
}

\newcommand\nosp {    		% commande vide (pas d'espace)
}
\newcommand{\id}[1]{} %ignore

\newcommand\boite[2]{				% Boite simple sans titre
	\vspace{5mm}
	\setlength{\fboxrule}{0.2mm}
	\setlength{\fboxsep}{5mm}	
	\fcolorbox{#1}{#1!3}{\makebox[\linewidth-2\fboxrule-2\fboxsep]{
  		\begin{minipage}[t]{\linewidth-2\fboxrule-4\fboxsep}\setlength{\parskip}{3mm}
  			 #2
  		\end{minipage}
	}}
	\vspace{5mm}
}

\newcommand\CBox[4]{				% Boites
	\vspace{5mm}
	\setlength{\fboxrule}{0.2mm}
	\setlength{\fboxsep}{5mm}
	
	\fcolorbox{#1}{#1!3}{\makebox[\linewidth-2\fboxrule-2\fboxsep]{
		\begin{minipage}[t]{1cm}\setlength{\parskip}{3mm}
	  		\textcolor{#1}{\LARGE{#2}}    
 	 	\end{minipage}  
  		\begin{minipage}[t]{\linewidth-2\fboxrule-4\fboxsep}\setlength{\parskip}{3mm}
			\raisebox{1.2mm}{\normalsize\sffamily{\textcolor{#1}{#3}}}						
  			 #4
  		\end{minipage}
	}}
	\vspace{5mm}
}

\newcommand\cadre[3]{				% Boites convertible html
	\par
	\vspace{2mm}
	\setlength{\fboxrule}{0.1mm}
	\setlength{\fboxsep}{5mm}
	\fcolorbox{#1}{white}{\makebox[\linewidth-2\fboxrule-2\fboxsep]{
  		\begin{minipage}[t]{\linewidth-2\fboxrule-4\fboxsep}\setlength{\parskip}{3mm}
			\raisebox{-2.5mm}{\sffamily \small{\textcolor{#1}{\MakeUppercase{#2}}}}		
			\par		
  			 #3
 	 		\end{minipage}
	}}
		\vspace{2mm}
	\par
}

\newcommand\bloc[3]{				% Boites convertible html sans bordure
     \needspace{2\baselineskip}
     {\sffamily \small{\textcolor{#1}{\MakeUppercase{#2}}}}    
		\par		
  			 #3
		\par
}

\newcommand\CHelp[1]{
     \CBox{Plum}{\faInfoCircle}{À RETENIR}{#1}
}

\newcommand\CUp[1]{
     \CBox{NavyBlue}{\faThumbsOUp}{EN PRATIQUE}{#1}
}

\newcommand\CInfo[1]{
     \CBox{Sepia}{\faArrowCircleRight}{REMARQUE}{#1}
}

\newcommand\CRedac[1]{
     \CBox{PineGreen}{\faEdit}{BIEN R\'EDIGER}{#1}
}

\newcommand\CError[1]{
     \CBox{Red}{\faExclamationTriangle}{ATTENTION}{#1}
}

\newcommand\TitreExo[2]{
\needspace{4\baselineskip}
 {\sffamily\large EXERCICE #1\ (\emph{#2 points})}
\vspace{5mm}
}

\newcommand\img[2]{
          \includegraphics[width=#2\paperwidth]{\imgdir#1}
}

\newcommand\imgsvg[2]{
       \begin{center}   \includegraphics[width=#2\paperwidth]{\imgsvgdir#1} \end{center}
}


\newcommand\Lien[2]{
     \href{#1}{#2 \tiny \faExternalLink}
}
\newcommand\mcLien[2]{
     \href{https~://www.maths-cours.fr/#1}{#2 \tiny \faExternalLink}
}

\newcommand{\euro}{\eurologo{}}

%================================================================================================================================
%
% Macros - Environement
%
%================================================================================================================================

\newenvironment{tex}{ %
}
{%
}

\newenvironment{indente}{ %
	\setlength\parindent{10mm}
}

{
	\setlength\parindent{0mm}
}

\newenvironment{corrige}{%
     \needspace{3\baselineskip}
     \medskip
     \textbf{\textsc{Corrigé}}
     \medskip
}
{
}

\newenvironment{extern}{%
     \begin{center}
     }
     {
     \end{center}
}

\NewEnviron{code}{%
	\par
     \boite{gray}{\texttt{%
     \BODY
     }}
     \par
}

\newenvironment{vbloc}{% boite sans cadre empeche saut de page
     \begin{minipage}[t]{\linewidth}
     }
     {
     \end{minipage}
}
\NewEnviron{h2}{%
    \needspace{3\baselineskip}
    \vspace{0.6cm}
	\noindent \MakeUppercase{\sffamily \large \BODY}
	\vspace{1mm}\textcolor{mcgris}{\hrule}\vspace{0.4cm}
	\par
}{}

\NewEnviron{h3}{%
    \needspace{3\baselineskip}
	\vspace{5mm}
	\textsc{\BODY}
	\par
}

\NewEnviron{margeneg}{ %
\begin{addmargin}[-1cm]{0cm}
\BODY
\end{addmargin}
}

\NewEnviron{html}{%
}

\begin{document}
\meta{url}{/exercices/panneaux-solaires-brevet-metropole-2017/}
\meta{pid}{9910}
\meta{titre}{Panneaux solaires - Brevet Métropole 2017}
\meta{type}{exercices}
%
\begin{h2}Exercice 4 (8 points)\end{h2}
\medbreak
Les panneaux photovoltaïques permettent de produire de l'électricité à partir du rayonnement solaire.
\par
Une unité courante pour mesurer l'énergie électrique est le kilowatt-heure, abrégé en kWh.
\medbreak
\begin{enumerate}
     \item Le plus souvent, l'électricité produite n'est pas utilisée directement, mais vendue pour être distribuée dans le réseau électrique collectif. Le prix d'achat du kWh, donné en \textbf{centimes d'euro}, dépend du type d'installation et de sa puissance totale, ainsi que de la date d'installation des panneaux photovoltaïques.
     \par
     Ce prix d'achat du kWh est donné dans le tableau ci-dessous.
     \par
     \begin{center}
          \begin{extern}%width="700" alt="Tarifs d'un kWh"
               \begin{tabularx}{\linewidth}{|m{2.2cm}|*{5}{>{\centering \arraybackslash}X|}}\cline{3-6}
                    \multicolumn{2}{c}{~}&\multicolumn{4}{|c|}{Date d'installation}\\ \hline
                    Type d'installation		& Puissance totale&\scriptsize Du 01/01/15 au 31/03/15&\scriptsize du 01/04/15 au 30/06/15&\scriptsize du 01/07/15  au 30/09/15 &\scriptsize du 01/10/15 au 31/12/15\\ \hline
                    Type A 					& 0 à 9 kW 		&26,57 &26,17 &25,78 	&25,39\\ \hline
                    Type B	& 0 à 36 kW 	&13,46 &13,95 &14,7 	&14,4\\ \cline{2-6}
                    & 36 à 100 kW 	&12,79 &13,25 &13,96 	&13,68\\ \hline
               \end{tabularx}
          \end{extern}
     \end{center}
     \emph{Tarifs d'un kWh en \textbf{centimes d'euros} - Source~: http~://www.developpement-durable.gouv.fr}
     \par
     En mai 2015, on installe une centrale solaire du type B, d'une puissance de 28 kW.
     \par
     Vérifier que le prix d'achat de 31~420~kWh est d'environ 4~383~\euro.
     \item
     Une personne souhaite installer des panneaux photovoltaïques
     sur la partie du toit de sa maison orientée au sud. Cette partie est
     grisée sur la figure ci-dessous. Elle est appelée pan sud du toit.
     \par
     La production d'électricité des panneaux solaires dépend de
     l'inclinaison du toit.
     \par
     Déterminer, au degré près, l'angle $\widehat{\text{ABC}}$ que forme ce pan sud du
     toit avec l'horizontale.
     \begin{center}
          \begin{extern}%width="200" alt="Pan sud du toit"
               \psset{unit=0.8cm,arrowsize=2pt 5}
               \begin{pspicture}(5,5)
                    %\psgrid
                    \pspolygon(0,0.5)(3.6,0.5)(3.6,2.6)(1.8,3.7)(0,2.6)
                    \psline(3.6,0.5)(4.6,1.3)(4.6,3.4)(3.6,2.6)
                    \pspolygon[fillstyle=solid,fillcolor=lightgray](4.6,3.4)(3.6,2.6)(1.8,3.7)(2.8,4.5)
                    \pspolygon(1.8,3.7)(0,2.6)(1.4,3.8)(2.8,4.5)
                    \psline[linewidth=0.5pt]{<->}(1.8,0.3)(3.6,0.3)\uput[d](2.7,0.3){4,5 m}
                    \psline[linewidth=0.5pt]{<->}(3.7,0.3)(4.7,1.1)\uput[d](4.7,0.9){7,5 m}
                    \psline[linewidth=0.5pt]{<->}(3.7,0.5)(3.7,2.6)\uput[l](3.7,1.55){4,8 m}
                    \psline[linewidth=0.5pt]{<->}(1.7,0.5)(1.7,3.6)\uput[l](1.7,2.05){7 m}
                    \psline[linestyle=dashed](1.8,0.5)(1.8,3.6)
                    \psline[linestyle=dashed](3.6,2.6)(1.8,2.6)
                    \uput[ul](1.8,3.6){A}\uput[dl](3.6,2.6){B}\uput[dr](1.8,2.6){C}
                    \rput(3.5,3.6){\small pan sud}
                    \rput(3.5,3.3){\small du toit}
                    \psline(2.1,2.6)(2.1,2.9)(1.8,2.9)
               \end{pspicture}
          \end{extern}
     \end{center}
     \item
     \textbf{a.} Montrer que la longueur AB est environ égale à $5$~m.
     \par
     \textbf{b.} Les panneaux photovoltaïques ont la forme d'un carré de $1$~m de côté.
     \par
     Le propriétaire prévoit d'installer 20 panneaux.
     \par
     Quel pourcentage de la surface totale du pan sud du toit sera alors
     couvert par les panneaux solaires~? On donnera une valeur approchée du
     résultat à 1\,\% près.
     \par
     \textbf{c.}La notice d'installation indique que les panneaux doivent être accolés
     les uns aux autres et qu'une bordure d'au moins $30$~cm de large doit être
     laissée libre pour le système de fixation tout autour de l'ensemble des
     panneaux.
     \begin{center}
          \begin{extern}%width="200" alt="Surface du toit"
               \psset{unit=1cm}
               \begin{pspicture}(4,3.5)
                    %\psgrid
                    \psline(0,0)(0,2.8)(3,2.8)
                    \psline(0.4,0)(0.4,2.4)(3,2.4)
                    \psframe(0.4,0.4)(1,1)\psframe(0.4,1)(1,1.6)
                    \psframe(1,1)(1.6,1.6)
                    \psline[linestyle=dotted](0.85,0)(0.85,0.4)
                    \psline[linestyle=dotted](1.3,0.85)(2.2,0.85)
                    \psline[linestyle=dotted](2.3,1.85)(3,1.85)
                    \rput(2.4,3.1){Bordure}\rput(3.3,1){Panneau}
                    \psline(2.4,2.9)(1.4,2.6) \psline(2.6,1)(1.4,1.3)
               \end{pspicture}
          \end{extern}
     \end{center}
     Le propriétaire peut-il installer les $20$~panneaux prévus~?
     \par
\end{enumerate}

\end{document}
µ
\documentclass[a4paper]{article}

%================================================================================================================================
%
% Packages
%
%================================================================================================================================

\usepackage[T1]{fontenc} 	% pour caractères accentués
\usepackage[utf8]{inputenc}  % encodage utf8
\usepackage[french]{babel}	% langue : français
\usepackage{fourier}			% caractères plus lisibles
\usepackage[dvipsnames]{xcolor} % couleurs
\usepackage{fancyhdr}		% réglage header footer
\usepackage{needspace}		% empêcher sauts de page mal placés
\usepackage{graphicx}		% pour inclure des graphiques
\usepackage{enumitem,cprotect}		% personnalise les listes d'items (nécessaire pour ol, al ...)
\usepackage{hyperref}		% Liens hypertexte
\usepackage{pstricks,pst-all,pst-node,pstricks-add,pst-math,pst-plot,pst-tree,pst-eucl} % pstricks
\usepackage[a4paper,includeheadfoot,top=2cm,left=3cm, bottom=2cm,right=3cm]{geometry} % marges etc.
\usepackage{comment}			% commentaires multilignes
\usepackage{amsmath,environ} % maths (matrices, etc.)
\usepackage{amssymb,makeidx}
\usepackage{bm}				% bold maths
\usepackage{tabularx}		% tableaux
\usepackage{colortbl}		% tableaux en couleur
\usepackage{fontawesome}		% Fontawesome
\usepackage{environ}			% environment with command
\usepackage{fp}				% calculs pour ps-tricks
\usepackage{multido}			% pour ps tricks
\usepackage[np]{numprint}	% formattage nombre
\usepackage{tikz,tkz-tab} 			% package principal TikZ
\usepackage{pgfplots}   % axes
\usepackage{mathrsfs}    % cursives
\usepackage{calc}			% calcul taille boites
\usepackage[scaled=0.875]{helvet} % font sans serif
\usepackage{svg} % svg
\usepackage{scrextend} % local margin
\usepackage{scratch} %scratch
\usepackage{multicol} % colonnes
%\usepackage{infix-RPN,pst-func} % formule en notation polanaise inversée
\usepackage{listings}

%================================================================================================================================
%
% Réglages de base
%
%================================================================================================================================

\lstset{
language=Python,   % R code
literate=
{á}{{\'a}}1
{à}{{\`a}}1
{ã}{{\~a}}1
{é}{{\'e}}1
{è}{{\`e}}1
{ê}{{\^e}}1
{í}{{\'i}}1
{ó}{{\'o}}1
{õ}{{\~o}}1
{ú}{{\'u}}1
{ü}{{\"u}}1
{ç}{{\c{c}}}1
{~}{{ }}1
}


\definecolor{codegreen}{rgb}{0,0.6,0}
\definecolor{codegray}{rgb}{0.5,0.5,0.5}
\definecolor{codepurple}{rgb}{0.58,0,0.82}
\definecolor{backcolour}{rgb}{0.95,0.95,0.92}

\lstdefinestyle{mystyle}{
    backgroundcolor=\color{backcolour},   
    commentstyle=\color{codegreen},
    keywordstyle=\color{magenta},
    numberstyle=\tiny\color{codegray},
    stringstyle=\color{codepurple},
    basicstyle=\ttfamily\footnotesize,
    breakatwhitespace=false,         
    breaklines=true,                 
    captionpos=b,                    
    keepspaces=true,                 
    numbers=left,                    
xleftmargin=2em,
framexleftmargin=2em,            
    showspaces=false,                
    showstringspaces=false,
    showtabs=false,                  
    tabsize=2,
    upquote=true
}

\lstset{style=mystyle}


\lstset{style=mystyle}
\newcommand{\imgdir}{C:/laragon/www/newmc/assets/imgsvg/}
\newcommand{\imgsvgdir}{C:/laragon/www/newmc/assets/imgsvg/}

\definecolor{mcgris}{RGB}{220, 220, 220}% ancien~; pour compatibilité
\definecolor{mcbleu}{RGB}{52, 152, 219}
\definecolor{mcvert}{RGB}{125, 194, 70}
\definecolor{mcmauve}{RGB}{154, 0, 215}
\definecolor{mcorange}{RGB}{255, 96, 0}
\definecolor{mcturquoise}{RGB}{0, 153, 153}
\definecolor{mcrouge}{RGB}{255, 0, 0}
\definecolor{mclightvert}{RGB}{205, 234, 190}

\definecolor{gris}{RGB}{220, 220, 220}
\definecolor{bleu}{RGB}{52, 152, 219}
\definecolor{vert}{RGB}{125, 194, 70}
\definecolor{mauve}{RGB}{154, 0, 215}
\definecolor{orange}{RGB}{255, 96, 0}
\definecolor{turquoise}{RGB}{0, 153, 153}
\definecolor{rouge}{RGB}{255, 0, 0}
\definecolor{lightvert}{RGB}{205, 234, 190}
\setitemize[0]{label=\color{lightvert}  $\bullet$}

\pagestyle{fancy}
\renewcommand{\headrulewidth}{0.2pt}
\fancyhead[L]{maths-cours.fr}
\fancyhead[R]{\thepage}
\renewcommand{\footrulewidth}{0.2pt}
\fancyfoot[C]{}

\newcolumntype{C}{>{\centering\arraybackslash}X}
\newcolumntype{s}{>{\hsize=.35\hsize\arraybackslash}X}

\setlength{\parindent}{0pt}		 
\setlength{\parskip}{3mm}
\setlength{\headheight}{1cm}

\def\ebook{ebook}
\def\book{book}
\def\web{web}
\def\type{web}

\newcommand{\vect}[1]{\overrightarrow{\,\mathstrut#1\,}}

\def\Oij{$\left(\text{O}~;~\vect{\imath},~\vect{\jmath}\right)$}
\def\Oijk{$\left(\text{O}~;~\vect{\imath},~\vect{\jmath},~\vect{k}\right)$}
\def\Ouv{$\left(\text{O}~;~\vect{u},~\vect{v}\right)$}

\hypersetup{breaklinks=true, colorlinks = true, linkcolor = OliveGreen, urlcolor = OliveGreen, citecolor = OliveGreen, pdfauthor={Didier BONNEL - https://www.maths-cours.fr} } % supprime les bordures autour des liens

\renewcommand{\arg}[0]{\text{arg}}

\everymath{\displaystyle}

%================================================================================================================================
%
% Macros - Commandes
%
%================================================================================================================================

\newcommand\meta[2]{    			% Utilisé pour créer le post HTML.
	\def\titre{titre}
	\def\url{url}
	\def\arg{#1}
	\ifx\titre\arg
		\newcommand\maintitle{#2}
		\fancyhead[L]{#2}
		{\Large\sffamily \MakeUppercase{#2}}
		\vspace{1mm}\textcolor{mcvert}{\hrule}
	\fi 
	\ifx\url\arg
		\fancyfoot[L]{\href{https://www.maths-cours.fr#2}{\black \footnotesize{https://www.maths-cours.fr#2}}}
	\fi 
}


\newcommand\TitreC[1]{    		% Titre centré
     \needspace{3\baselineskip}
     \begin{center}\textbf{#1}\end{center}
}

\newcommand\newpar{    		% paragraphe
     \par
}

\newcommand\nosp {    		% commande vide (pas d'espace)
}
\newcommand{\id}[1]{} %ignore

\newcommand\boite[2]{				% Boite simple sans titre
	\vspace{5mm}
	\setlength{\fboxrule}{0.2mm}
	\setlength{\fboxsep}{5mm}	
	\fcolorbox{#1}{#1!3}{\makebox[\linewidth-2\fboxrule-2\fboxsep]{
  		\begin{minipage}[t]{\linewidth-2\fboxrule-4\fboxsep}\setlength{\parskip}{3mm}
  			 #2
  		\end{minipage}
	}}
	\vspace{5mm}
}

\newcommand\CBox[4]{				% Boites
	\vspace{5mm}
	\setlength{\fboxrule}{0.2mm}
	\setlength{\fboxsep}{5mm}
	
	\fcolorbox{#1}{#1!3}{\makebox[\linewidth-2\fboxrule-2\fboxsep]{
		\begin{minipage}[t]{1cm}\setlength{\parskip}{3mm}
	  		\textcolor{#1}{\LARGE{#2}}    
 	 	\end{minipage}  
  		\begin{minipage}[t]{\linewidth-2\fboxrule-4\fboxsep}\setlength{\parskip}{3mm}
			\raisebox{1.2mm}{\normalsize\sffamily{\textcolor{#1}{#3}}}						
  			 #4
  		\end{minipage}
	}}
	\vspace{5mm}
}

\newcommand\cadre[3]{				% Boites convertible html
	\par
	\vspace{2mm}
	\setlength{\fboxrule}{0.1mm}
	\setlength{\fboxsep}{5mm}
	\fcolorbox{#1}{white}{\makebox[\linewidth-2\fboxrule-2\fboxsep]{
  		\begin{minipage}[t]{\linewidth-2\fboxrule-4\fboxsep}\setlength{\parskip}{3mm}
			\raisebox{-2.5mm}{\sffamily \small{\textcolor{#1}{\MakeUppercase{#2}}}}		
			\par		
  			 #3
 	 		\end{minipage}
	}}
		\vspace{2mm}
	\par
}

\newcommand\bloc[3]{				% Boites convertible html sans bordure
     \needspace{2\baselineskip}
     {\sffamily \small{\textcolor{#1}{\MakeUppercase{#2}}}}    
		\par		
  			 #3
		\par
}

\newcommand\CHelp[1]{
     \CBox{Plum}{\faInfoCircle}{À RETENIR}{#1}
}

\newcommand\CUp[1]{
     \CBox{NavyBlue}{\faThumbsOUp}{EN PRATIQUE}{#1}
}

\newcommand\CInfo[1]{
     \CBox{Sepia}{\faArrowCircleRight}{REMARQUE}{#1}
}

\newcommand\CRedac[1]{
     \CBox{PineGreen}{\faEdit}{BIEN R\'EDIGER}{#1}
}

\newcommand\CError[1]{
     \CBox{Red}{\faExclamationTriangle}{ATTENTION}{#1}
}

\newcommand\TitreExo[2]{
\needspace{4\baselineskip}
 {\sffamily\large EXERCICE #1\ (\emph{#2 points})}
\vspace{5mm}
}

\newcommand\img[2]{
          \includegraphics[width=#2\paperwidth]{\imgdir#1}
}

\newcommand\imgsvg[2]{
       \begin{center}   \includegraphics[width=#2\paperwidth]{\imgsvgdir#1} \end{center}
}


\newcommand\Lien[2]{
     \href{#1}{#2 \tiny \faExternalLink}
}
\newcommand\mcLien[2]{
     \href{https~://www.maths-cours.fr/#1}{#2 \tiny \faExternalLink}
}

\newcommand{\euro}{\eurologo{}}

%================================================================================================================================
%
% Macros - Environement
%
%================================================================================================================================

\newenvironment{tex}{ %
}
{%
}

\newenvironment{indente}{ %
	\setlength\parindent{10mm}
}

{
	\setlength\parindent{0mm}
}

\newenvironment{corrige}{%
     \needspace{3\baselineskip}
     \medskip
     \textbf{\textsc{Corrigé}}
     \medskip
}
{
}

\newenvironment{extern}{%
     \begin{center}
     }
     {
     \end{center}
}

\NewEnviron{code}{%
	\par
     \boite{gray}{\texttt{%
     \BODY
     }}
     \par
}

\newenvironment{vbloc}{% boite sans cadre empeche saut de page
     \begin{minipage}[t]{\linewidth}
     }
     {
     \end{minipage}
}
\NewEnviron{h2}{%
    \needspace{3\baselineskip}
    \vspace{0.6cm}
	\noindent \MakeUppercase{\sffamily \large \BODY}
	\vspace{1mm}\textcolor{mcgris}{\hrule}\vspace{0.4cm}
	\par
}{}

\NewEnviron{h3}{%
    \needspace{3\baselineskip}
	\vspace{5mm}
	\textsc{\BODY}
	\par
}

\NewEnviron{margeneg}{ %
\begin{addmargin}[-1cm]{0cm}
\BODY
\end{addmargin}
}

\NewEnviron{html}{%
}

\begin{document}
\meta{url}{/exercices/calculs-de-vitesse-brevet-metropole-2017/}
\meta{pid}{9917}
\meta{titre}{Calculs de vitesse - Brevet Métropole 2017}
\meta{type}{exercices}
%
\begin{h2}Exercice 5 (8 points)\end{h2}
\medbreak
\begin{enumerate}
     \item Lors des Jeux Olympiques de Rio en 2016, la danoise Pernille Blume a remporté le $50$~m nage libre en $24,07$~secondes.
     \par
     A-t-elle nagé plus rapidement qu'une personne qui se déplace en marchant vite, c'est-à-dire à $6$~km/h~?
     \item  On donne l'expression $E = (3x + 8)^2 - 64$.
     \begin{enumerate}[label=\alph*.]
          \item Développer $E$.
          \item  Montrer que $E$ peut s'écrire sous forme factorisée~: $3x(3x + 16)$.
          \item  Résoudre l'équation $(3x + 8)^2 - 64 = 0$.
     \end{enumerate}
     \item  La distance $d$ de freinage d'un véhicule dépend de sa vitesse et de l'état de la route.
     \par
     On peut la calculer à l'aide de la formule suivante~:
     $d = k \times  V^2$ avec~:
     \begin{itemize}
          \item %
          $d$~: distance de freinage en m
          \item %
          $V$~: vitesse du véhicule en m/s
          \item %
          $k$~: coefficient dépendant de l'état de la route :
          \begin{indent}
               $  k = 0,14 $ sur route mouillée \\
               $ k = 0,08$ sur route sèche.
          \end{indent}
     \end{itemize}
     Quelle est la vitesse d'un véhicule dont la distance de freinage sur route mouillée est égale à $15$~m~?
\end{enumerate}

\end{document}
µ
\documentclass[a4paper]{article}

%================================================================================================================================
%
% Packages
%
%================================================================================================================================

\usepackage[T1]{fontenc} 	% pour caractères accentués
\usepackage[utf8]{inputenc}  % encodage utf8
\usepackage[french]{babel}	% langue : français
\usepackage{fourier}			% caractères plus lisibles
\usepackage[dvipsnames]{xcolor} % couleurs
\usepackage{fancyhdr}		% réglage header footer
\usepackage{needspace}		% empêcher sauts de page mal placés
\usepackage{graphicx}		% pour inclure des graphiques
\usepackage{enumitem,cprotect}		% personnalise les listes d'items (nécessaire pour ol, al ...)
\usepackage{hyperref}		% Liens hypertexte
\usepackage{pstricks,pst-all,pst-node,pstricks-add,pst-math,pst-plot,pst-tree,pst-eucl} % pstricks
\usepackage[a4paper,includeheadfoot,top=2cm,left=3cm, bottom=2cm,right=3cm]{geometry} % marges etc.
\usepackage{comment}			% commentaires multilignes
\usepackage{amsmath,environ} % maths (matrices, etc.)
\usepackage{amssymb,makeidx}
\usepackage{bm}				% bold maths
\usepackage{tabularx}		% tableaux
\usepackage{colortbl}		% tableaux en couleur
\usepackage{fontawesome}		% Fontawesome
\usepackage{environ}			% environment with command
\usepackage{fp}				% calculs pour ps-tricks
\usepackage{multido}			% pour ps tricks
\usepackage[np]{numprint}	% formattage nombre
\usepackage{tikz,tkz-tab} 			% package principal TikZ
\usepackage{pgfplots}   % axes
\usepackage{mathrsfs}    % cursives
\usepackage{calc}			% calcul taille boites
\usepackage[scaled=0.875]{helvet} % font sans serif
\usepackage{svg} % svg
\usepackage{scrextend} % local margin
\usepackage{scratch} %scratch
\usepackage{multicol} % colonnes
%\usepackage{infix-RPN,pst-func} % formule en notation polanaise inversée
\usepackage{listings}

%================================================================================================================================
%
% Réglages de base
%
%================================================================================================================================

\lstset{
language=Python,   % R code
literate=
{á}{{\'a}}1
{à}{{\`a}}1
{ã}{{\~a}}1
{é}{{\'e}}1
{è}{{\`e}}1
{ê}{{\^e}}1
{í}{{\'i}}1
{ó}{{\'o}}1
{õ}{{\~o}}1
{ú}{{\'u}}1
{ü}{{\"u}}1
{ç}{{\c{c}}}1
{~}{{ }}1
}


\definecolor{codegreen}{rgb}{0,0.6,0}
\definecolor{codegray}{rgb}{0.5,0.5,0.5}
\definecolor{codepurple}{rgb}{0.58,0,0.82}
\definecolor{backcolour}{rgb}{0.95,0.95,0.92}

\lstdefinestyle{mystyle}{
    backgroundcolor=\color{backcolour},   
    commentstyle=\color{codegreen},
    keywordstyle=\color{magenta},
    numberstyle=\tiny\color{codegray},
    stringstyle=\color{codepurple},
    basicstyle=\ttfamily\footnotesize,
    breakatwhitespace=false,         
    breaklines=true,                 
    captionpos=b,                    
    keepspaces=true,                 
    numbers=left,                    
xleftmargin=2em,
framexleftmargin=2em,            
    showspaces=false,                
    showstringspaces=false,
    showtabs=false,                  
    tabsize=2,
    upquote=true
}

\lstset{style=mystyle}


\lstset{style=mystyle}
\newcommand{\imgdir}{C:/laragon/www/newmc/assets/imgsvg/}
\newcommand{\imgsvgdir}{C:/laragon/www/newmc/assets/imgsvg/}

\definecolor{mcgris}{RGB}{220, 220, 220}% ancien~; pour compatibilité
\definecolor{mcbleu}{RGB}{52, 152, 219}
\definecolor{mcvert}{RGB}{125, 194, 70}
\definecolor{mcmauve}{RGB}{154, 0, 215}
\definecolor{mcorange}{RGB}{255, 96, 0}
\definecolor{mcturquoise}{RGB}{0, 153, 153}
\definecolor{mcrouge}{RGB}{255, 0, 0}
\definecolor{mclightvert}{RGB}{205, 234, 190}

\definecolor{gris}{RGB}{220, 220, 220}
\definecolor{bleu}{RGB}{52, 152, 219}
\definecolor{vert}{RGB}{125, 194, 70}
\definecolor{mauve}{RGB}{154, 0, 215}
\definecolor{orange}{RGB}{255, 96, 0}
\definecolor{turquoise}{RGB}{0, 153, 153}
\definecolor{rouge}{RGB}{255, 0, 0}
\definecolor{lightvert}{RGB}{205, 234, 190}
\setitemize[0]{label=\color{lightvert}  $\bullet$}

\pagestyle{fancy}
\renewcommand{\headrulewidth}{0.2pt}
\fancyhead[L]{maths-cours.fr}
\fancyhead[R]{\thepage}
\renewcommand{\footrulewidth}{0.2pt}
\fancyfoot[C]{}

\newcolumntype{C}{>{\centering\arraybackslash}X}
\newcolumntype{s}{>{\hsize=.35\hsize\arraybackslash}X}

\setlength{\parindent}{0pt}		 
\setlength{\parskip}{3mm}
\setlength{\headheight}{1cm}

\def\ebook{ebook}
\def\book{book}
\def\web{web}
\def\type{web}

\newcommand{\vect}[1]{\overrightarrow{\,\mathstrut#1\,}}

\def\Oij{$\left(\text{O}~;~\vect{\imath},~\vect{\jmath}\right)$}
\def\Oijk{$\left(\text{O}~;~\vect{\imath},~\vect{\jmath},~\vect{k}\right)$}
\def\Ouv{$\left(\text{O}~;~\vect{u},~\vect{v}\right)$}

\hypersetup{breaklinks=true, colorlinks = true, linkcolor = OliveGreen, urlcolor = OliveGreen, citecolor = OliveGreen, pdfauthor={Didier BONNEL - https://www.maths-cours.fr} } % supprime les bordures autour des liens

\renewcommand{\arg}[0]{\text{arg}}

\everymath{\displaystyle}

%================================================================================================================================
%
% Macros - Commandes
%
%================================================================================================================================

\newcommand\meta[2]{    			% Utilisé pour créer le post HTML.
	\def\titre{titre}
	\def\url{url}
	\def\arg{#1}
	\ifx\titre\arg
		\newcommand\maintitle{#2}
		\fancyhead[L]{#2}
		{\Large\sffamily \MakeUppercase{#2}}
		\vspace{1mm}\textcolor{mcvert}{\hrule}
	\fi 
	\ifx\url\arg
		\fancyfoot[L]{\href{https://www.maths-cours.fr#2}{\black \footnotesize{https://www.maths-cours.fr#2}}}
	\fi 
}


\newcommand\TitreC[1]{    		% Titre centré
     \needspace{3\baselineskip}
     \begin{center}\textbf{#1}\end{center}
}

\newcommand\newpar{    		% paragraphe
     \par
}

\newcommand\nosp {    		% commande vide (pas d'espace)
}
\newcommand{\id}[1]{} %ignore

\newcommand\boite[2]{				% Boite simple sans titre
	\vspace{5mm}
	\setlength{\fboxrule}{0.2mm}
	\setlength{\fboxsep}{5mm}	
	\fcolorbox{#1}{#1!3}{\makebox[\linewidth-2\fboxrule-2\fboxsep]{
  		\begin{minipage}[t]{\linewidth-2\fboxrule-4\fboxsep}\setlength{\parskip}{3mm}
  			 #2
  		\end{minipage}
	}}
	\vspace{5mm}
}

\newcommand\CBox[4]{				% Boites
	\vspace{5mm}
	\setlength{\fboxrule}{0.2mm}
	\setlength{\fboxsep}{5mm}
	
	\fcolorbox{#1}{#1!3}{\makebox[\linewidth-2\fboxrule-2\fboxsep]{
		\begin{minipage}[t]{1cm}\setlength{\parskip}{3mm}
	  		\textcolor{#1}{\LARGE{#2}}    
 	 	\end{minipage}  
  		\begin{minipage}[t]{\linewidth-2\fboxrule-4\fboxsep}\setlength{\parskip}{3mm}
			\raisebox{1.2mm}{\normalsize\sffamily{\textcolor{#1}{#3}}}						
  			 #4
  		\end{minipage}
	}}
	\vspace{5mm}
}

\newcommand\cadre[3]{				% Boites convertible html
	\par
	\vspace{2mm}
	\setlength{\fboxrule}{0.1mm}
	\setlength{\fboxsep}{5mm}
	\fcolorbox{#1}{white}{\makebox[\linewidth-2\fboxrule-2\fboxsep]{
  		\begin{minipage}[t]{\linewidth-2\fboxrule-4\fboxsep}\setlength{\parskip}{3mm}
			\raisebox{-2.5mm}{\sffamily \small{\textcolor{#1}{\MakeUppercase{#2}}}}		
			\par		
  			 #3
 	 		\end{minipage}
	}}
		\vspace{2mm}
	\par
}

\newcommand\bloc[3]{				% Boites convertible html sans bordure
     \needspace{2\baselineskip}
     {\sffamily \small{\textcolor{#1}{\MakeUppercase{#2}}}}    
		\par		
  			 #3
		\par
}

\newcommand\CHelp[1]{
     \CBox{Plum}{\faInfoCircle}{À RETENIR}{#1}
}

\newcommand\CUp[1]{
     \CBox{NavyBlue}{\faThumbsOUp}{EN PRATIQUE}{#1}
}

\newcommand\CInfo[1]{
     \CBox{Sepia}{\faArrowCircleRight}{REMARQUE}{#1}
}

\newcommand\CRedac[1]{
     \CBox{PineGreen}{\faEdit}{BIEN R\'EDIGER}{#1}
}

\newcommand\CError[1]{
     \CBox{Red}{\faExclamationTriangle}{ATTENTION}{#1}
}

\newcommand\TitreExo[2]{
\needspace{4\baselineskip}
 {\sffamily\large EXERCICE #1\ (\emph{#2 points})}
\vspace{5mm}
}

\newcommand\img[2]{
          \includegraphics[width=#2\paperwidth]{\imgdir#1}
}

\newcommand\imgsvg[2]{
       \begin{center}   \includegraphics[width=#2\paperwidth]{\imgsvgdir#1} \end{center}
}


\newcommand\Lien[2]{
     \href{#1}{#2 \tiny \faExternalLink}
}
\newcommand\mcLien[2]{
     \href{https~://www.maths-cours.fr/#1}{#2 \tiny \faExternalLink}
}

\newcommand{\euro}{\eurologo{}}

%================================================================================================================================
%
% Macros - Environement
%
%================================================================================================================================

\newenvironment{tex}{ %
}
{%
}

\newenvironment{indente}{ %
	\setlength\parindent{10mm}
}

{
	\setlength\parindent{0mm}
}

\newenvironment{corrige}{%
     \needspace{3\baselineskip}
     \medskip
     \textbf{\textsc{Corrigé}}
     \medskip
}
{
}

\newenvironment{extern}{%
     \begin{center}
     }
     {
     \end{center}
}

\NewEnviron{code}{%
	\par
     \boite{gray}{\texttt{%
     \BODY
     }}
     \par
}

\newenvironment{vbloc}{% boite sans cadre empeche saut de page
     \begin{minipage}[t]{\linewidth}
     }
     {
     \end{minipage}
}
\NewEnviron{h2}{%
    \needspace{3\baselineskip}
    \vspace{0.6cm}
	\noindent \MakeUppercase{\sffamily \large \BODY}
	\vspace{1mm}\textcolor{mcgris}{\hrule}\vspace{0.4cm}
	\par
}{}

\NewEnviron{h3}{%
    \needspace{3\baselineskip}
	\vspace{5mm}
	\textsc{\BODY}
	\par
}

\NewEnviron{margeneg}{ %
\begin{addmargin}[-1cm]{0cm}
\BODY
\end{addmargin}
}

\NewEnviron{html}{%
}

\begin{document}
\meta{url}{/exercices/statistiques-brevet-metropole-2017/}
\meta{pid}{9926}
\meta{titre}{Statistiques - Brevet Métropole 2017}
\meta{type}{exercices}
%
\begin{h2}Exercice 6 (8 points)\end{h2}
\medbreak
\textbf{Document n° 1}
\par
Le surpoids est devenu un problème majeur de santé, celui-ci prédispose à beaucoup de maladies et diminue l'espérance de vie.\\
L' indice le plus couramment utilisé est celui de masse corporelle (IMC).
\par
\textbf{Document n° 2}
\par
L'IMC est une grandeur internationale permettant de déterminer la corpulence d'une personne adulte entre
18 ans et 65 ans.\\
Il se calcule avec la formule suivante~: IMC $= \dfrac{\text{masse}}{\text{taille}^2}$
avec \og masse \fg{} en kg et \og taille \fg{} en m.
\par
Normes~: \\
\begin{indent}
     $18,5 \leqslant  \text{IMC} < 25$ : corpulence normale\\
     $25 \leqslant IMC < 30$  : surpoids\\
     $\text{IMC} > 30$ : obésité\\
\end{indent}
\begin{enumerate}
     \item  Dans une entreprise, lors d'une visite médicale, un médecin calcule l'IMC de six des employés.
     \par
     Il utilise pour cela une feuille de tableur dont voici un extrait~:
     \begin{center}
          \begin{tabular}{|c|c|c|c|c|c|c|c|}\hline%class="compact"
               &A				&B 		&C 		&D 		&E 		&F 		&G\\ \hline
               1	&Taille (en m) 	&1,69 	&1,72 	&1,75 	&1,78 	&1,86 	&1,88\\ \hline
               2	&Masse (en kg) 	&72 	&85 	&74 	&70 	&115 	&85\\ \hline
               3	&IMC (valeur approchée au dixième)		&25,2 	&28,7 	&24,2 	&22,1 	&33,2 	&24,0\\ \hline
          \end{tabular}\\
     \end{center}
     \begin{enumerate}[label=\alph*.]
          \item Combien d'employés sont en situation de surpoids ou d'obésité dans cette entreprise~?
          \item Laquelle de ces formules a-t-on écrite dans la cellule B3, puis recopiée à droite, pour calculer l'IMC~?
          \begin{itemize}
               \item %
               =72/1,69 $\hat{}$ 2
               \item %
               = B1/ (B2 * B2)
               \item %
               = B2/ (B1 * B1)
               \item %
               =\$B2/ (\$B1*\$B1)
          \end{itemize}
          Recopier la formule correcte sur la copie.
          \medbreak
     \end{enumerate}
     \item Le médecin a fait le bilan de l'IMC de chacun des 41 employés de cette entreprise. Il a reporté les
     informations recueillies dans le tableau suivant dans lequel les IMC ont été arrondis à l'unité près.
     \begin{center}
          \begin{tabular}{|c|c|c|c|c|c|c|c|c|c|}\hline%class="compact"
               IMC &20 &22 &23 &24 &25 &29 &30 &33 &Total\\ \hline
               Effectif &9 &12&6 &8 &2 &1 &1 &2 &41\\ \hline
          \end{tabular}
     \end{center}
     \begin{enumerate}[label=\alph*.]
          \item Calculer une valeur approchée, arrondie à l'entier près, de l'IMC moyen des employés de cette
          entreprise.
          \item Quel est l'IMC médian~? Interpréter ce résultat.
          \item On lit sur certains magazines~: \og On estime qu'au moins 5\,\% de la population mondiale est en surpoids ou est obèse \fg. Est-ce le cas pour les employés de cette entreprise~?
     \end{enumerate}
\end{enumerate}

\end{document}
µ
\documentclass[a4paper]{article}

%================================================================================================================================
%
% Packages
%
%================================================================================================================================

\usepackage[T1]{fontenc} 	% pour caractères accentués
\usepackage[utf8]{inputenc}  % encodage utf8
\usepackage[french]{babel}	% langue : français
\usepackage{fourier}			% caractères plus lisibles
\usepackage[dvipsnames]{xcolor} % couleurs
\usepackage{fancyhdr}		% réglage header footer
\usepackage{needspace}		% empêcher sauts de page mal placés
\usepackage{graphicx}		% pour inclure des graphiques
\usepackage{enumitem,cprotect}		% personnalise les listes d'items (nécessaire pour ol, al ...)
\usepackage{hyperref}		% Liens hypertexte
\usepackage{pstricks,pst-all,pst-node,pstricks-add,pst-math,pst-plot,pst-tree,pst-eucl} % pstricks
\usepackage[a4paper,includeheadfoot,top=2cm,left=3cm, bottom=2cm,right=3cm]{geometry} % marges etc.
\usepackage{comment}			% commentaires multilignes
\usepackage{amsmath,environ} % maths (matrices, etc.)
\usepackage{amssymb,makeidx}
\usepackage{bm}				% bold maths
\usepackage{tabularx}		% tableaux
\usepackage{colortbl}		% tableaux en couleur
\usepackage{fontawesome}		% Fontawesome
\usepackage{environ}			% environment with command
\usepackage{fp}				% calculs pour ps-tricks
\usepackage{multido}			% pour ps tricks
\usepackage[np]{numprint}	% formattage nombre
\usepackage{tikz,tkz-tab} 			% package principal TikZ
\usepackage{pgfplots}   % axes
\usepackage{mathrsfs}    % cursives
\usepackage{calc}			% calcul taille boites
\usepackage[scaled=0.875]{helvet} % font sans serif
\usepackage{svg} % svg
\usepackage{scrextend} % local margin
\usepackage{scratch} %scratch
\usepackage{multicol} % colonnes
%\usepackage{infix-RPN,pst-func} % formule en notation polanaise inversée
\usepackage{listings}

%================================================================================================================================
%
% Réglages de base
%
%================================================================================================================================

\lstset{
language=Python,   % R code
literate=
{á}{{\'a}}1
{à}{{\`a}}1
{ã}{{\~a}}1
{é}{{\'e}}1
{è}{{\`e}}1
{ê}{{\^e}}1
{í}{{\'i}}1
{ó}{{\'o}}1
{õ}{{\~o}}1
{ú}{{\'u}}1
{ü}{{\"u}}1
{ç}{{\c{c}}}1
{~}{{ }}1
}


\definecolor{codegreen}{rgb}{0,0.6,0}
\definecolor{codegray}{rgb}{0.5,0.5,0.5}
\definecolor{codepurple}{rgb}{0.58,0,0.82}
\definecolor{backcolour}{rgb}{0.95,0.95,0.92}

\lstdefinestyle{mystyle}{
    backgroundcolor=\color{backcolour},   
    commentstyle=\color{codegreen},
    keywordstyle=\color{magenta},
    numberstyle=\tiny\color{codegray},
    stringstyle=\color{codepurple},
    basicstyle=\ttfamily\footnotesize,
    breakatwhitespace=false,         
    breaklines=true,                 
    captionpos=b,                    
    keepspaces=true,                 
    numbers=left,                    
xleftmargin=2em,
framexleftmargin=2em,            
    showspaces=false,                
    showstringspaces=false,
    showtabs=false,                  
    tabsize=2,
    upquote=true
}

\lstset{style=mystyle}


\lstset{style=mystyle}
\newcommand{\imgdir}{C:/laragon/www/newmc/assets/imgsvg/}
\newcommand{\imgsvgdir}{C:/laragon/www/newmc/assets/imgsvg/}

\definecolor{mcgris}{RGB}{220, 220, 220}% ancien~; pour compatibilité
\definecolor{mcbleu}{RGB}{52, 152, 219}
\definecolor{mcvert}{RGB}{125, 194, 70}
\definecolor{mcmauve}{RGB}{154, 0, 215}
\definecolor{mcorange}{RGB}{255, 96, 0}
\definecolor{mcturquoise}{RGB}{0, 153, 153}
\definecolor{mcrouge}{RGB}{255, 0, 0}
\definecolor{mclightvert}{RGB}{205, 234, 190}

\definecolor{gris}{RGB}{220, 220, 220}
\definecolor{bleu}{RGB}{52, 152, 219}
\definecolor{vert}{RGB}{125, 194, 70}
\definecolor{mauve}{RGB}{154, 0, 215}
\definecolor{orange}{RGB}{255, 96, 0}
\definecolor{turquoise}{RGB}{0, 153, 153}
\definecolor{rouge}{RGB}{255, 0, 0}
\definecolor{lightvert}{RGB}{205, 234, 190}
\setitemize[0]{label=\color{lightvert}  $\bullet$}

\pagestyle{fancy}
\renewcommand{\headrulewidth}{0.2pt}
\fancyhead[L]{maths-cours.fr}
\fancyhead[R]{\thepage}
\renewcommand{\footrulewidth}{0.2pt}
\fancyfoot[C]{}

\newcolumntype{C}{>{\centering\arraybackslash}X}
\newcolumntype{s}{>{\hsize=.35\hsize\arraybackslash}X}

\setlength{\parindent}{0pt}		 
\setlength{\parskip}{3mm}
\setlength{\headheight}{1cm}

\def\ebook{ebook}
\def\book{book}
\def\web{web}
\def\type{web}

\newcommand{\vect}[1]{\overrightarrow{\,\mathstrut#1\,}}

\def\Oij{$\left(\text{O}~;~\vect{\imath},~\vect{\jmath}\right)$}
\def\Oijk{$\left(\text{O}~;~\vect{\imath},~\vect{\jmath},~\vect{k}\right)$}
\def\Ouv{$\left(\text{O}~;~\vect{u},~\vect{v}\right)$}

\hypersetup{breaklinks=true, colorlinks = true, linkcolor = OliveGreen, urlcolor = OliveGreen, citecolor = OliveGreen, pdfauthor={Didier BONNEL - https://www.maths-cours.fr} } % supprime les bordures autour des liens

\renewcommand{\arg}[0]{\text{arg}}

\everymath{\displaystyle}

%================================================================================================================================
%
% Macros - Commandes
%
%================================================================================================================================

\newcommand\meta[2]{    			% Utilisé pour créer le post HTML.
	\def\titre{titre}
	\def\url{url}
	\def\arg{#1}
	\ifx\titre\arg
		\newcommand\maintitle{#2}
		\fancyhead[L]{#2}
		{\Large\sffamily \MakeUppercase{#2}}
		\vspace{1mm}\textcolor{mcvert}{\hrule}
	\fi 
	\ifx\url\arg
		\fancyfoot[L]{\href{https://www.maths-cours.fr#2}{\black \footnotesize{https://www.maths-cours.fr#2}}}
	\fi 
}


\newcommand\TitreC[1]{    		% Titre centré
     \needspace{3\baselineskip}
     \begin{center}\textbf{#1}\end{center}
}

\newcommand\newpar{    		% paragraphe
     \par
}

\newcommand\nosp {    		% commande vide (pas d'espace)
}
\newcommand{\id}[1]{} %ignore

\newcommand\boite[2]{				% Boite simple sans titre
	\vspace{5mm}
	\setlength{\fboxrule}{0.2mm}
	\setlength{\fboxsep}{5mm}	
	\fcolorbox{#1}{#1!3}{\makebox[\linewidth-2\fboxrule-2\fboxsep]{
  		\begin{minipage}[t]{\linewidth-2\fboxrule-4\fboxsep}\setlength{\parskip}{3mm}
  			 #2
  		\end{minipage}
	}}
	\vspace{5mm}
}

\newcommand\CBox[4]{				% Boites
	\vspace{5mm}
	\setlength{\fboxrule}{0.2mm}
	\setlength{\fboxsep}{5mm}
	
	\fcolorbox{#1}{#1!3}{\makebox[\linewidth-2\fboxrule-2\fboxsep]{
		\begin{minipage}[t]{1cm}\setlength{\parskip}{3mm}
	  		\textcolor{#1}{\LARGE{#2}}    
 	 	\end{minipage}  
  		\begin{minipage}[t]{\linewidth-2\fboxrule-4\fboxsep}\setlength{\parskip}{3mm}
			\raisebox{1.2mm}{\normalsize\sffamily{\textcolor{#1}{#3}}}						
  			 #4
  		\end{minipage}
	}}
	\vspace{5mm}
}

\newcommand\cadre[3]{				% Boites convertible html
	\par
	\vspace{2mm}
	\setlength{\fboxrule}{0.1mm}
	\setlength{\fboxsep}{5mm}
	\fcolorbox{#1}{white}{\makebox[\linewidth-2\fboxrule-2\fboxsep]{
  		\begin{minipage}[t]{\linewidth-2\fboxrule-4\fboxsep}\setlength{\parskip}{3mm}
			\raisebox{-2.5mm}{\sffamily \small{\textcolor{#1}{\MakeUppercase{#2}}}}		
			\par		
  			 #3
 	 		\end{minipage}
	}}
		\vspace{2mm}
	\par
}

\newcommand\bloc[3]{				% Boites convertible html sans bordure
     \needspace{2\baselineskip}
     {\sffamily \small{\textcolor{#1}{\MakeUppercase{#2}}}}    
		\par		
  			 #3
		\par
}

\newcommand\CHelp[1]{
     \CBox{Plum}{\faInfoCircle}{À RETENIR}{#1}
}

\newcommand\CUp[1]{
     \CBox{NavyBlue}{\faThumbsOUp}{EN PRATIQUE}{#1}
}

\newcommand\CInfo[1]{
     \CBox{Sepia}{\faArrowCircleRight}{REMARQUE}{#1}
}

\newcommand\CRedac[1]{
     \CBox{PineGreen}{\faEdit}{BIEN R\'EDIGER}{#1}
}

\newcommand\CError[1]{
     \CBox{Red}{\faExclamationTriangle}{ATTENTION}{#1}
}

\newcommand\TitreExo[2]{
\needspace{4\baselineskip}
 {\sffamily\large EXERCICE #1\ (\emph{#2 points})}
\vspace{5mm}
}

\newcommand\img[2]{
          \includegraphics[width=#2\paperwidth]{\imgdir#1}
}

\newcommand\imgsvg[2]{
       \begin{center}   \includegraphics[width=#2\paperwidth]{\imgsvgdir#1} \end{center}
}


\newcommand\Lien[2]{
     \href{#1}{#2 \tiny \faExternalLink}
}
\newcommand\mcLien[2]{
     \href{https~://www.maths-cours.fr/#1}{#2 \tiny \faExternalLink}
}

\newcommand{\euro}{\eurologo{}}

%================================================================================================================================
%
% Macros - Environement
%
%================================================================================================================================

\newenvironment{tex}{ %
}
{%
}

\newenvironment{indente}{ %
	\setlength\parindent{10mm}
}

{
	\setlength\parindent{0mm}
}

\newenvironment{corrige}{%
     \needspace{3\baselineskip}
     \medskip
     \textbf{\textsc{Corrigé}}
     \medskip
}
{
}

\newenvironment{extern}{%
     \begin{center}
     }
     {
     \end{center}
}

\NewEnviron{code}{%
	\par
     \boite{gray}{\texttt{%
     \BODY
     }}
     \par
}

\newenvironment{vbloc}{% boite sans cadre empeche saut de page
     \begin{minipage}[t]{\linewidth}
     }
     {
     \end{minipage}
}
\NewEnviron{h2}{%
    \needspace{3\baselineskip}
    \vspace{0.6cm}
	\noindent \MakeUppercase{\sffamily \large \BODY}
	\vspace{1mm}\textcolor{mcgris}{\hrule}\vspace{0.4cm}
	\par
}{}

\NewEnviron{h3}{%
    \needspace{3\baselineskip}
	\vspace{5mm}
	\textsc{\BODY}
	\par
}

\NewEnviron{margeneg}{ %
\begin{addmargin}[-1cm]{0cm}
\BODY
\end{addmargin}
}

\NewEnviron{html}{%
}

\begin{document}
\meta{url}{/exercices/volumes-brevet-metropole-2017/}
\meta{pid}{9936}
\meta{titre}{Volumes - Brevet Métropole 2017}
\meta{type}{exercices}
%
\begin{h2}Exercice 7 (7 points)\end{h2}
\medbreak
Léo a ramassé des fraises pour faire de la confiture.
\medbreak
\begin{enumerate}
     \item Il utilise les proportions de sa grand-mère~: 700 g de sucre pour 1 kg de fraises.
     \par
     Il a ramassé 1,8 kg de fraises. De quelle quantité de sucre a-t-il besoin~?
     \item  Après cuisson, Léo a obtenu 2,7 litres de confiture.
     \par
     Il verse la confiture dans des pots cylindriques de 6 cm de diamètre et de
     12 cm de haut, qu'il remplit jusqu'à 1 cm du bord supérieur.
     \par
     Combien pourra-t-il remplir de pots~?
     \par
     \emph{Rappels}~:\\
     1 litre = 1000 cm$^3$ \\
     Volume d'un cylindre  $= \pi \times R^2 \times h$.
     \item  Il colle ensuite sur ses pots une étiquette rectangulaire de fond blanc qui recouvre toute la surface latérale du pot.
     \begin{enumerate}[label=\alph*.]
          \item Montrer que la longueur de l'étiquette est d'environ 18,8 cm.
          \item Dessiner l'étiquette à l' échelle $\dfrac{1}{3}$.
     \end{enumerate}
\end{enumerate}

\end{document}
µ
\documentclass[a4paper]{article}

%================================================================================================================================
%
% Packages
%
%================================================================================================================================

\usepackage[T1]{fontenc} 	% pour caractères accentués
\usepackage[utf8]{inputenc}  % encodage utf8
\usepackage[french]{babel}	% langue : français
\usepackage{fourier}			% caractères plus lisibles
\usepackage[dvipsnames]{xcolor} % couleurs
\usepackage{fancyhdr}		% réglage header footer
\usepackage{needspace}		% empêcher sauts de page mal placés
\usepackage{graphicx}		% pour inclure des graphiques
\usepackage{enumitem,cprotect}		% personnalise les listes d'items (nécessaire pour ol, al ...)
\usepackage{hyperref}		% Liens hypertexte
\usepackage{pstricks,pst-all,pst-node,pstricks-add,pst-math,pst-plot,pst-tree,pst-eucl} % pstricks
\usepackage[a4paper,includeheadfoot,top=2cm,left=3cm, bottom=2cm,right=3cm]{geometry} % marges etc.
\usepackage{comment}			% commentaires multilignes
\usepackage{amsmath,environ} % maths (matrices, etc.)
\usepackage{amssymb,makeidx}
\usepackage{bm}				% bold maths
\usepackage{tabularx}		% tableaux
\usepackage{colortbl}		% tableaux en couleur
\usepackage{fontawesome}		% Fontawesome
\usepackage{environ}			% environment with command
\usepackage{fp}				% calculs pour ps-tricks
\usepackage{multido}			% pour ps tricks
\usepackage[np]{numprint}	% formattage nombre
\usepackage{tikz,tkz-tab} 			% package principal TikZ
\usepackage{pgfplots}   % axes
\usepackage{mathrsfs}    % cursives
\usepackage{calc}			% calcul taille boites
\usepackage[scaled=0.875]{helvet} % font sans serif
\usepackage{svg} % svg
\usepackage{scrextend} % local margin
\usepackage{scratch} %scratch
\usepackage{multicol} % colonnes
%\usepackage{infix-RPN,pst-func} % formule en notation polanaise inversée
\usepackage{listings}

%================================================================================================================================
%
% Réglages de base
%
%================================================================================================================================

\lstset{
language=Python,   % R code
literate=
{á}{{\'a}}1
{à}{{\`a}}1
{ã}{{\~a}}1
{é}{{\'e}}1
{è}{{\`e}}1
{ê}{{\^e}}1
{í}{{\'i}}1
{ó}{{\'o}}1
{õ}{{\~o}}1
{ú}{{\'u}}1
{ü}{{\"u}}1
{ç}{{\c{c}}}1
{~}{{ }}1
}


\definecolor{codegreen}{rgb}{0,0.6,0}
\definecolor{codegray}{rgb}{0.5,0.5,0.5}
\definecolor{codepurple}{rgb}{0.58,0,0.82}
\definecolor{backcolour}{rgb}{0.95,0.95,0.92}

\lstdefinestyle{mystyle}{
    backgroundcolor=\color{backcolour},   
    commentstyle=\color{codegreen},
    keywordstyle=\color{magenta},
    numberstyle=\tiny\color{codegray},
    stringstyle=\color{codepurple},
    basicstyle=\ttfamily\footnotesize,
    breakatwhitespace=false,         
    breaklines=true,                 
    captionpos=b,                    
    keepspaces=true,                 
    numbers=left,                    
xleftmargin=2em,
framexleftmargin=2em,            
    showspaces=false,                
    showstringspaces=false,
    showtabs=false,                  
    tabsize=2,
    upquote=true
}

\lstset{style=mystyle}


\lstset{style=mystyle}
\newcommand{\imgdir}{C:/laragon/www/newmc/assets/imgsvg/}
\newcommand{\imgsvgdir}{C:/laragon/www/newmc/assets/imgsvg/}

\definecolor{mcgris}{RGB}{220, 220, 220}% ancien~; pour compatibilité
\definecolor{mcbleu}{RGB}{52, 152, 219}
\definecolor{mcvert}{RGB}{125, 194, 70}
\definecolor{mcmauve}{RGB}{154, 0, 215}
\definecolor{mcorange}{RGB}{255, 96, 0}
\definecolor{mcturquoise}{RGB}{0, 153, 153}
\definecolor{mcrouge}{RGB}{255, 0, 0}
\definecolor{mclightvert}{RGB}{205, 234, 190}

\definecolor{gris}{RGB}{220, 220, 220}
\definecolor{bleu}{RGB}{52, 152, 219}
\definecolor{vert}{RGB}{125, 194, 70}
\definecolor{mauve}{RGB}{154, 0, 215}
\definecolor{orange}{RGB}{255, 96, 0}
\definecolor{turquoise}{RGB}{0, 153, 153}
\definecolor{rouge}{RGB}{255, 0, 0}
\definecolor{lightvert}{RGB}{205, 234, 190}
\setitemize[0]{label=\color{lightvert}  $\bullet$}

\pagestyle{fancy}
\renewcommand{\headrulewidth}{0.2pt}
\fancyhead[L]{maths-cours.fr}
\fancyhead[R]{\thepage}
\renewcommand{\footrulewidth}{0.2pt}
\fancyfoot[C]{}

\newcolumntype{C}{>{\centering\arraybackslash}X}
\newcolumntype{s}{>{\hsize=.35\hsize\arraybackslash}X}

\setlength{\parindent}{0pt}		 
\setlength{\parskip}{3mm}
\setlength{\headheight}{1cm}

\def\ebook{ebook}
\def\book{book}
\def\web{web}
\def\type{web}

\newcommand{\vect}[1]{\overrightarrow{\,\mathstrut#1\,}}

\def\Oij{$\left(\text{O}~;~\vect{\imath},~\vect{\jmath}\right)$}
\def\Oijk{$\left(\text{O}~;~\vect{\imath},~\vect{\jmath},~\vect{k}\right)$}
\def\Ouv{$\left(\text{O}~;~\vect{u},~\vect{v}\right)$}

\hypersetup{breaklinks=true, colorlinks = true, linkcolor = OliveGreen, urlcolor = OliveGreen, citecolor = OliveGreen, pdfauthor={Didier BONNEL - https://www.maths-cours.fr} } % supprime les bordures autour des liens

\renewcommand{\arg}[0]{\text{arg}}

\everymath{\displaystyle}

%================================================================================================================================
%
% Macros - Commandes
%
%================================================================================================================================

\newcommand\meta[2]{    			% Utilisé pour créer le post HTML.
	\def\titre{titre}
	\def\url{url}
	\def\arg{#1}
	\ifx\titre\arg
		\newcommand\maintitle{#2}
		\fancyhead[L]{#2}
		{\Large\sffamily \MakeUppercase{#2}}
		\vspace{1mm}\textcolor{mcvert}{\hrule}
	\fi 
	\ifx\url\arg
		\fancyfoot[L]{\href{https://www.maths-cours.fr#2}{\black \footnotesize{https://www.maths-cours.fr#2}}}
	\fi 
}


\newcommand\TitreC[1]{    		% Titre centré
     \needspace{3\baselineskip}
     \begin{center}\textbf{#1}\end{center}
}

\newcommand\newpar{    		% paragraphe
     \par
}

\newcommand\nosp {    		% commande vide (pas d'espace)
}
\newcommand{\id}[1]{} %ignore

\newcommand\boite[2]{				% Boite simple sans titre
	\vspace{5mm}
	\setlength{\fboxrule}{0.2mm}
	\setlength{\fboxsep}{5mm}	
	\fcolorbox{#1}{#1!3}{\makebox[\linewidth-2\fboxrule-2\fboxsep]{
  		\begin{minipage}[t]{\linewidth-2\fboxrule-4\fboxsep}\setlength{\parskip}{3mm}
  			 #2
  		\end{minipage}
	}}
	\vspace{5mm}
}

\newcommand\CBox[4]{				% Boites
	\vspace{5mm}
	\setlength{\fboxrule}{0.2mm}
	\setlength{\fboxsep}{5mm}
	
	\fcolorbox{#1}{#1!3}{\makebox[\linewidth-2\fboxrule-2\fboxsep]{
		\begin{minipage}[t]{1cm}\setlength{\parskip}{3mm}
	  		\textcolor{#1}{\LARGE{#2}}    
 	 	\end{minipage}  
  		\begin{minipage}[t]{\linewidth-2\fboxrule-4\fboxsep}\setlength{\parskip}{3mm}
			\raisebox{1.2mm}{\normalsize\sffamily{\textcolor{#1}{#3}}}						
  			 #4
  		\end{minipage}
	}}
	\vspace{5mm}
}

\newcommand\cadre[3]{				% Boites convertible html
	\par
	\vspace{2mm}
	\setlength{\fboxrule}{0.1mm}
	\setlength{\fboxsep}{5mm}
	\fcolorbox{#1}{white}{\makebox[\linewidth-2\fboxrule-2\fboxsep]{
  		\begin{minipage}[t]{\linewidth-2\fboxrule-4\fboxsep}\setlength{\parskip}{3mm}
			\raisebox{-2.5mm}{\sffamily \small{\textcolor{#1}{\MakeUppercase{#2}}}}		
			\par		
  			 #3
 	 		\end{minipage}
	}}
		\vspace{2mm}
	\par
}

\newcommand\bloc[3]{				% Boites convertible html sans bordure
     \needspace{2\baselineskip}
     {\sffamily \small{\textcolor{#1}{\MakeUppercase{#2}}}}    
		\par		
  			 #3
		\par
}

\newcommand\CHelp[1]{
     \CBox{Plum}{\faInfoCircle}{À RETENIR}{#1}
}

\newcommand\CUp[1]{
     \CBox{NavyBlue}{\faThumbsOUp}{EN PRATIQUE}{#1}
}

\newcommand\CInfo[1]{
     \CBox{Sepia}{\faArrowCircleRight}{REMARQUE}{#1}
}

\newcommand\CRedac[1]{
     \CBox{PineGreen}{\faEdit}{BIEN R\'EDIGER}{#1}
}

\newcommand\CError[1]{
     \CBox{Red}{\faExclamationTriangle}{ATTENTION}{#1}
}

\newcommand\TitreExo[2]{
\needspace{4\baselineskip}
 {\sffamily\large EXERCICE #1\ (\emph{#2 points})}
\vspace{5mm}
}

\newcommand\img[2]{
          \includegraphics[width=#2\paperwidth]{\imgdir#1}
}

\newcommand\imgsvg[2]{
       \begin{center}   \includegraphics[width=#2\paperwidth]{\imgsvgdir#1} \end{center}
}


\newcommand\Lien[2]{
     \href{#1}{#2 \tiny \faExternalLink}
}
\newcommand\mcLien[2]{
     \href{https~://www.maths-cours.fr/#1}{#2 \tiny \faExternalLink}
}

\newcommand{\euro}{\eurologo{}}

%================================================================================================================================
%
% Macros - Environement
%
%================================================================================================================================

\newenvironment{tex}{ %
}
{%
}

\newenvironment{indente}{ %
	\setlength\parindent{10mm}
}

{
	\setlength\parindent{0mm}
}

\newenvironment{corrige}{%
     \needspace{3\baselineskip}
     \medskip
     \textbf{\textsc{Corrigé}}
     \medskip
}
{
}

\newenvironment{extern}{%
     \begin{center}
     }
     {
     \end{center}
}

\NewEnviron{code}{%
	\par
     \boite{gray}{\texttt{%
     \BODY
     }}
     \par
}

\newenvironment{vbloc}{% boite sans cadre empeche saut de page
     \begin{minipage}[t]{\linewidth}
     }
     {
     \end{minipage}
}
\NewEnviron{h2}{%
    \needspace{3\baselineskip}
    \vspace{0.6cm}
	\noindent \MakeUppercase{\sffamily \large \BODY}
	\vspace{1mm}\textcolor{mcgris}{\hrule}\vspace{0.4cm}
	\par
}{}

\NewEnviron{h3}{%
    \needspace{3\baselineskip}
	\vspace{5mm}
	\textsc{\BODY}
	\par
}

\NewEnviron{margeneg}{ %
\begin{addmargin}[-1cm]{0cm}
\BODY
\end{addmargin}
}

\NewEnviron{html}{%
}

\begin{document}
\meta{url}{/exercices/arbre-pondere-et-probabilites/}
\meta{pid}{9942}
\meta{titre}{Arbre pondéré et probabilités}
\meta{type}{exercices}
%
Dans un sachet opaque, on place 12 jetons indiscernables au toucher sur lesquels sont inscrites les 12 lettres du mot ANNIVERSAIRE :
\begin{center}
     \begin{extern}%width="450" alt="Lettres tirage probabilité"
          % Racine en Haut, développement vers le bas
          \begin{tikzpicture}[xscale=1,yscale=1]
               % Styles (MODIFIABLES)
               \tikzstyle{feuille}=[fill=white,circle,draw]
               \tikzstyle{etiquette}=[midway,fill=white,draw]
               % Dimensions (MODIFIABLES)
               \def\DistanceInterNiveaux{0}
               \def\DistanceInterFeuilles{0.8}
               % Dimensions calculées (NON MODIFIABLES)
               \def\NiveauA{(-0)*\DistanceInterNiveaux}
               \def\NiveauB{(-1)*\DistanceInterNiveaux}
               \def\InterFeuilles{(1)*\DistanceInterFeuilles}
               % Noeuds (MODIFIABLES : Styles et Coefficients d'InterFeuilles)
               \node[feuille] (Ra) at ({(0)*\InterFeuilles},{\NiveauB}) {A};
               \node[feuille] (Rb) at ({(1)*\InterFeuilles},{\NiveauB}) {N};
               \node[feuille] (Rc) at ({(2)*\InterFeuilles},{\NiveauB}) {N};
               \node[feuille] (Rd) at ({(3)*\InterFeuilles},{\NiveauB}) {I};
               \node[feuille] (Re) at ({(4)*\InterFeuilles},{\NiveauB}) {V};
               \node[feuille] (Rf) at ({(5)*\InterFeuilles},{\NiveauB}) {E};
               \node[feuille] (Rg) at ({(6)*\InterFeuilles},{\NiveauB}) {R};
               \node[feuille] (Rh) at ({(7)*\InterFeuilles},{\NiveauB}) {S};
               \node[feuille] (Ri) at ({(8)*\InterFeuilles},{\NiveauB}) {A};
               \node[feuille] (Rj) at ({(9)*\InterFeuilles},{\NiveauB}) {I};
               \node[feuille] (Rk) at ({(10)*\InterFeuilles},{\NiveauB}) {R};
               \node[feuille] (Rl) at ({(11)*\InterFeuilles},{\NiveauB}) {E};
          \end{tikzpicture}
     \end{extern}
\end{center}
%
On tire un jeton au hasard.
\begin{enumerate}
     \item %
     Déterminer la probabilité de l'événement : \og la lettre tirée  est un A\fg{}.
     \item %
     Compléter l'arbre pondéré ci-dessous :
     %
     \begin{center}
          % Racine en Haut, développement vers le bas
          \begin{extern}%width="500" alt=""
               \begin{tikzpicture}[xscale=1,yscale=1]
                    % Styles (MODIFIABLES)
                    \tikzstyle{fleche}=[>=latex,thick]
                    \tikzstyle{noeud}=[circle,draw]
                    \tikzstyle{feuille}=[circle,draw]
                    \tikzstyle{etiquette}=[midway,fill=white]
                    % Dimensions (MODIFIABLES)
                    \def\DistanceInterNiveaux{3}
                    \def\DistanceInterFeuilles{2}
                    % Dimensions calculées (NON MODIFIABLES)
                    \def\NiveauA{(-0)*\DistanceInterNiveaux}
                    \def\NiveauB{(-1)*\DistanceInterNiveaux}
                    \def\InterFeuilles{(1)*\DistanceInterFeuilles}
                    % Noeuds (MODIFIABLES : Styles et Coefficients d'InterFeuilles)
                    \node[noeud] (R) at ({(3)*\InterFeuilles},{\NiveauA}) {};
                    \node[feuille] (Ra) at ({(0)*\InterFeuilles},{\NiveauB}) {A};
                    \node[feuille] (Rb) at ({(1)*\InterFeuilles},{\NiveauB}) {E};
                    \node[feuille] (Rc) at ({(2)*\InterFeuilles},{\NiveauB}) {I};
                    \node[feuille] (Rd) at ({(3)*\InterFeuilles},{\NiveauB}) {N};
                    \node[feuille] (Re) at ({(4)*\InterFeuilles},{\NiveauB}) {R};
                    \node[feuille] (Rf) at ({(5)*\InterFeuilles},{\NiveauB}) {S};
                    \node[feuille] (Rg) at ({(6)*\InterFeuilles},{\NiveauB}) {...};
                    % Arcs (MODIFIABLES : Styles)
                    \draw[fleche] (R)--(Ra) node[etiquette] {...};
                    \draw[fleche] (R)--(Rb) node[etiquette] {...};
                    \draw[fleche] (R)--(Rc) node[etiquette] {...};
                    \draw[fleche] (R)--(Rd) node[etiquette] {...};
                    \draw[fleche] (R)--(Re) node[etiquette] {...};
                    \draw[fleche] (R)--(Rf) node[etiquette] {...};
                    \draw[fleche] (R)--(Rg) node[etiquette] {$\dfrac{1}{12}$};
               \end{tikzpicture}
          \end{extern}
     \end{center}
     \item %
     Kévin affirme qu'il y a 3 voyelles (A, E, I) et 7 lettres différentes au total (A, E, I, N, R, S, V) donc que la probabilité de tirer une voyelle est $\dfrac{3}{7}.$\\
     A-t-il raison~?
     \item %
     Après avoir tiré un jeton portant la lettre A, Kévin ne la remet pas dans le sac et tire ensuite un second jeton.\\
     Quelle est la probabilité que ce second jeton porte également la lettre A ?
\end{enumerate}
\begin{corrige}
     \begin{enumerate}
          \item %
          L'expression \og au hasard \fg{} indique que chaque \textbf{jeton} a la même probabilité d'être tiré.
          \par
          La probabilité de l'événement : \og la lettre tirée est un A\fg{}  est donc donné par la formule~:\\
          $p=\dfrac{\text{nombre d'issues favorables à l'événement}}{\text{nombre total d'issues possibles}}.$
          \par
          Ici il y a 2 jetons portant la lettre A sur un total de 12 jetons donc :\\
          $p=\dfrac{2}{12}=\dfrac{1}{6}.$
          \item %
          Le raisonnement précédent peut s'appliquer à chacune des lettres.
          \par
          On obtient alors l'arbre suivant :
          \begin{center}
               % Racine en Haut, développement vers le bas
               \begin{extern}%width="500" alt=""
                    \begin{tikzpicture}[xscale=1,yscale=1]
                         % Styles (MODIFIABLES)
                         \tikzstyle{fleche}=[>=latex,thick]
                         \tikzstyle{noeud}=[circle,draw]
                         \tikzstyle{feuille}=[circle,draw]
                         \tikzstyle{etiquette}=[midway,fill=white]
                         % Dimensions (MODIFIABLES)
                         \def\DistanceInterNiveaux{3}
                         \def\DistanceInterFeuilles{2}
                         % Dimensions calculées (NON MODIFIABLES)
                         \def\NiveauA{(-0)*\DistanceInterNiveaux}
                         \def\NiveauB{(-1)*\DistanceInterNiveaux}
                         \def\InterFeuilles{(1)*\DistanceInterFeuilles}
                         % Noeuds (MODIFIABLES : Styles et Coefficients d'InterFeuilles)
                         \node[noeud] (R) at ({(3)*\InterFeuilles},{\NiveauA}) {};
                         \node[feuille] (Ra) at ({(0)*\InterFeuilles},{\NiveauB}) {A};
                         \node[feuille] (Rb) at ({(1)*\InterFeuilles},{\NiveauB}) {E};
                         \node[feuille] (Rc) at ({(2)*\InterFeuilles},{\NiveauB}) {I};
                         \node[feuille] (Rd) at ({(3)*\InterFeuilles},{\NiveauB}) {N};
                         \node[feuille] (Re) at ({(4)*\InterFeuilles},{\NiveauB}) {R};
                         \node[feuille] (Rf) at ({(5)*\InterFeuilles},{\NiveauB}) {S};
                         \node[feuille] (Rg) at ({(6)*\InterFeuilles},{\NiveauB}) {\red V};
                         % Arcs (MODIFIABLES : Styles)
                         \draw[fleche] (R)--(Ra) node[etiquette] {$\red \dfrac{1}{6}$};
                         \draw[fleche] (R)--(Rb) node[etiquette] {$\red \dfrac{1}{6}$};
                         \draw[fleche] (R)--(Rc) node[etiquette] {$\red \dfrac{1}{6}$};
                         \draw[fleche] (R)--(Rd) node[etiquette] {$\red \dfrac{1}{6}$};
                         \draw[fleche] (R)--(Re) node[etiquette] {$\red \dfrac{1}{6}$};
                         \draw[fleche] (R)--(Rf) node[etiquette] {$\red \dfrac{1}{12}$};
                         \draw[fleche] (R)--(Rg) node[etiquette] {$\dfrac{1}{12}$};
                    \end{tikzpicture}
               \end{extern}
          \end{center}
          \item %
          Le raisonnement de Kévin est faux.
          \par
          En effet, comme le montre l'arbre ci-dessus, toutes les lettres n'ont pas la même probabilité d'être tirées.
          \par
          Il faut donc raisonner en terme de jetons et non en terme de lettres:\\
          6 jetons portent une voyelle sur un total de 12. \\
          La probabilité de tirer une voyelle est donc~:
          \par
          $p=\dfrac{6}{12}=\dfrac{1}{2}.$
          \item %
          Après avoir tiré un jeton portant la lettre A, il reste 11 jetons~dans le sachet dont un seul porte la lettre A.
          \par
          La probabilité de tirer à nouveau la lettre A est alors~:
          $p=\dfrac{1}{11}.$
     \end{enumerate}
\end{corrige}

\end{document}
µ
\documentclass[a4paper]{article}

%================================================================================================================================
%
% Packages
%
%================================================================================================================================

\usepackage[T1]{fontenc} 	% pour caractères accentués
\usepackage[utf8]{inputenc}  % encodage utf8
\usepackage[french]{babel}	% langue : français
\usepackage{fourier}			% caractères plus lisibles
\usepackage[dvipsnames]{xcolor} % couleurs
\usepackage{fancyhdr}		% réglage header footer
\usepackage{needspace}		% empêcher sauts de page mal placés
\usepackage{graphicx}		% pour inclure des graphiques
\usepackage{enumitem,cprotect}		% personnalise les listes d'items (nécessaire pour ol, al ...)
\usepackage{hyperref}		% Liens hypertexte
\usepackage{pstricks,pst-all,pst-node,pstricks-add,pst-math,pst-plot,pst-tree,pst-eucl} % pstricks
\usepackage[a4paper,includeheadfoot,top=2cm,left=3cm, bottom=2cm,right=3cm]{geometry} % marges etc.
\usepackage{comment}			% commentaires multilignes
\usepackage{amsmath,environ} % maths (matrices, etc.)
\usepackage{amssymb,makeidx}
\usepackage{bm}				% bold maths
\usepackage{tabularx}		% tableaux
\usepackage{colortbl}		% tableaux en couleur
\usepackage{fontawesome}		% Fontawesome
\usepackage{environ}			% environment with command
\usepackage{fp}				% calculs pour ps-tricks
\usepackage{multido}			% pour ps tricks
\usepackage[np]{numprint}	% formattage nombre
\usepackage{tikz,tkz-tab} 			% package principal TikZ
\usepackage{pgfplots}   % axes
\usepackage{mathrsfs}    % cursives
\usepackage{calc}			% calcul taille boites
\usepackage[scaled=0.875]{helvet} % font sans serif
\usepackage{svg} % svg
\usepackage{scrextend} % local margin
\usepackage{scratch} %scratch
\usepackage{multicol} % colonnes
%\usepackage{infix-RPN,pst-func} % formule en notation polanaise inversée
\usepackage{listings}

%================================================================================================================================
%
% Réglages de base
%
%================================================================================================================================

\lstset{
language=Python,   % R code
literate=
{á}{{\'a}}1
{à}{{\`a}}1
{ã}{{\~a}}1
{é}{{\'e}}1
{è}{{\`e}}1
{ê}{{\^e}}1
{í}{{\'i}}1
{ó}{{\'o}}1
{õ}{{\~o}}1
{ú}{{\'u}}1
{ü}{{\"u}}1
{ç}{{\c{c}}}1
{~}{{ }}1
}


\definecolor{codegreen}{rgb}{0,0.6,0}
\definecolor{codegray}{rgb}{0.5,0.5,0.5}
\definecolor{codepurple}{rgb}{0.58,0,0.82}
\definecolor{backcolour}{rgb}{0.95,0.95,0.92}

\lstdefinestyle{mystyle}{
    backgroundcolor=\color{backcolour},   
    commentstyle=\color{codegreen},
    keywordstyle=\color{magenta},
    numberstyle=\tiny\color{codegray},
    stringstyle=\color{codepurple},
    basicstyle=\ttfamily\footnotesize,
    breakatwhitespace=false,         
    breaklines=true,                 
    captionpos=b,                    
    keepspaces=true,                 
    numbers=left,                    
xleftmargin=2em,
framexleftmargin=2em,            
    showspaces=false,                
    showstringspaces=false,
    showtabs=false,                  
    tabsize=2,
    upquote=true
}

\lstset{style=mystyle}


\lstset{style=mystyle}
\newcommand{\imgdir}{C:/laragon/www/newmc/assets/imgsvg/}
\newcommand{\imgsvgdir}{C:/laragon/www/newmc/assets/imgsvg/}

\definecolor{mcgris}{RGB}{220, 220, 220}% ancien~; pour compatibilité
\definecolor{mcbleu}{RGB}{52, 152, 219}
\definecolor{mcvert}{RGB}{125, 194, 70}
\definecolor{mcmauve}{RGB}{154, 0, 215}
\definecolor{mcorange}{RGB}{255, 96, 0}
\definecolor{mcturquoise}{RGB}{0, 153, 153}
\definecolor{mcrouge}{RGB}{255, 0, 0}
\definecolor{mclightvert}{RGB}{205, 234, 190}

\definecolor{gris}{RGB}{220, 220, 220}
\definecolor{bleu}{RGB}{52, 152, 219}
\definecolor{vert}{RGB}{125, 194, 70}
\definecolor{mauve}{RGB}{154, 0, 215}
\definecolor{orange}{RGB}{255, 96, 0}
\definecolor{turquoise}{RGB}{0, 153, 153}
\definecolor{rouge}{RGB}{255, 0, 0}
\definecolor{lightvert}{RGB}{205, 234, 190}
\setitemize[0]{label=\color{lightvert}  $\bullet$}

\pagestyle{fancy}
\renewcommand{\headrulewidth}{0.2pt}
\fancyhead[L]{maths-cours.fr}
\fancyhead[R]{\thepage}
\renewcommand{\footrulewidth}{0.2pt}
\fancyfoot[C]{}

\newcolumntype{C}{>{\centering\arraybackslash}X}
\newcolumntype{s}{>{\hsize=.35\hsize\arraybackslash}X}

\setlength{\parindent}{0pt}		 
\setlength{\parskip}{3mm}
\setlength{\headheight}{1cm}

\def\ebook{ebook}
\def\book{book}
\def\web{web}
\def\type{web}

\newcommand{\vect}[1]{\overrightarrow{\,\mathstrut#1\,}}

\def\Oij{$\left(\text{O}~;~\vect{\imath},~\vect{\jmath}\right)$}
\def\Oijk{$\left(\text{O}~;~\vect{\imath},~\vect{\jmath},~\vect{k}\right)$}
\def\Ouv{$\left(\text{O}~;~\vect{u},~\vect{v}\right)$}

\hypersetup{breaklinks=true, colorlinks = true, linkcolor = OliveGreen, urlcolor = OliveGreen, citecolor = OliveGreen, pdfauthor={Didier BONNEL - https://www.maths-cours.fr} } % supprime les bordures autour des liens

\renewcommand{\arg}[0]{\text{arg}}

\everymath{\displaystyle}

%================================================================================================================================
%
% Macros - Commandes
%
%================================================================================================================================

\newcommand\meta[2]{    			% Utilisé pour créer le post HTML.
	\def\titre{titre}
	\def\url{url}
	\def\arg{#1}
	\ifx\titre\arg
		\newcommand\maintitle{#2}
		\fancyhead[L]{#2}
		{\Large\sffamily \MakeUppercase{#2}}
		\vspace{1mm}\textcolor{mcvert}{\hrule}
	\fi 
	\ifx\url\arg
		\fancyfoot[L]{\href{https://www.maths-cours.fr#2}{\black \footnotesize{https://www.maths-cours.fr#2}}}
	\fi 
}


\newcommand\TitreC[1]{    		% Titre centré
     \needspace{3\baselineskip}
     \begin{center}\textbf{#1}\end{center}
}

\newcommand\newpar{    		% paragraphe
     \par
}

\newcommand\nosp {    		% commande vide (pas d'espace)
}
\newcommand{\id}[1]{} %ignore

\newcommand\boite[2]{				% Boite simple sans titre
	\vspace{5mm}
	\setlength{\fboxrule}{0.2mm}
	\setlength{\fboxsep}{5mm}	
	\fcolorbox{#1}{#1!3}{\makebox[\linewidth-2\fboxrule-2\fboxsep]{
  		\begin{minipage}[t]{\linewidth-2\fboxrule-4\fboxsep}\setlength{\parskip}{3mm}
  			 #2
  		\end{minipage}
	}}
	\vspace{5mm}
}

\newcommand\CBox[4]{				% Boites
	\vspace{5mm}
	\setlength{\fboxrule}{0.2mm}
	\setlength{\fboxsep}{5mm}
	
	\fcolorbox{#1}{#1!3}{\makebox[\linewidth-2\fboxrule-2\fboxsep]{
		\begin{minipage}[t]{1cm}\setlength{\parskip}{3mm}
	  		\textcolor{#1}{\LARGE{#2}}    
 	 	\end{minipage}  
  		\begin{minipage}[t]{\linewidth-2\fboxrule-4\fboxsep}\setlength{\parskip}{3mm}
			\raisebox{1.2mm}{\normalsize\sffamily{\textcolor{#1}{#3}}}						
  			 #4
  		\end{minipage}
	}}
	\vspace{5mm}
}

\newcommand\cadre[3]{				% Boites convertible html
	\par
	\vspace{2mm}
	\setlength{\fboxrule}{0.1mm}
	\setlength{\fboxsep}{5mm}
	\fcolorbox{#1}{white}{\makebox[\linewidth-2\fboxrule-2\fboxsep]{
  		\begin{minipage}[t]{\linewidth-2\fboxrule-4\fboxsep}\setlength{\parskip}{3mm}
			\raisebox{-2.5mm}{\sffamily \small{\textcolor{#1}{\MakeUppercase{#2}}}}		
			\par		
  			 #3
 	 		\end{minipage}
	}}
		\vspace{2mm}
	\par
}

\newcommand\bloc[3]{				% Boites convertible html sans bordure
     \needspace{2\baselineskip}
     {\sffamily \small{\textcolor{#1}{\MakeUppercase{#2}}}}    
		\par		
  			 #3
		\par
}

\newcommand\CHelp[1]{
     \CBox{Plum}{\faInfoCircle}{À RETENIR}{#1}
}

\newcommand\CUp[1]{
     \CBox{NavyBlue}{\faThumbsOUp}{EN PRATIQUE}{#1}
}

\newcommand\CInfo[1]{
     \CBox{Sepia}{\faArrowCircleRight}{REMARQUE}{#1}
}

\newcommand\CRedac[1]{
     \CBox{PineGreen}{\faEdit}{BIEN R\'EDIGER}{#1}
}

\newcommand\CError[1]{
     \CBox{Red}{\faExclamationTriangle}{ATTENTION}{#1}
}

\newcommand\TitreExo[2]{
\needspace{4\baselineskip}
 {\sffamily\large EXERCICE #1\ (\emph{#2 points})}
\vspace{5mm}
}

\newcommand\img[2]{
          \includegraphics[width=#2\paperwidth]{\imgdir#1}
}

\newcommand\imgsvg[2]{
       \begin{center}   \includegraphics[width=#2\paperwidth]{\imgsvgdir#1} \end{center}
}


\newcommand\Lien[2]{
     \href{#1}{#2 \tiny \faExternalLink}
}
\newcommand\mcLien[2]{
     \href{https~://www.maths-cours.fr/#1}{#2 \tiny \faExternalLink}
}

\newcommand{\euro}{\eurologo{}}

%================================================================================================================================
%
% Macros - Environement
%
%================================================================================================================================

\newenvironment{tex}{ %
}
{%
}

\newenvironment{indente}{ %
	\setlength\parindent{10mm}
}

{
	\setlength\parindent{0mm}
}

\newenvironment{corrige}{%
     \needspace{3\baselineskip}
     \medskip
     \textbf{\textsc{Corrigé}}
     \medskip
}
{
}

\newenvironment{extern}{%
     \begin{center}
     }
     {
     \end{center}
}

\NewEnviron{code}{%
	\par
     \boite{gray}{\texttt{%
     \BODY
     }}
     \par
}

\newenvironment{vbloc}{% boite sans cadre empeche saut de page
     \begin{minipage}[t]{\linewidth}
     }
     {
     \end{minipage}
}
\NewEnviron{h2}{%
    \needspace{3\baselineskip}
    \vspace{0.6cm}
	\noindent \MakeUppercase{\sffamily \large \BODY}
	\vspace{1mm}\textcolor{mcgris}{\hrule}\vspace{0.4cm}
	\par
}{}

\NewEnviron{h3}{%
    \needspace{3\baselineskip}
	\vspace{5mm}
	\textsc{\BODY}
	\par
}

\NewEnviron{margeneg}{ %
\begin{addmargin}[-1cm]{0cm}
\BODY
\end{addmargin}
}

\NewEnviron{html}{%
}

\begin{document}
\meta{url}{exercices/tableau-a-double-entree-et-probabilites/}
\meta{pid}{9954}
\meta{titre}{Tableau à double entrée et probabilités}
\meta{type}{exercices}
%
Dans une classe de 24 élèves, chaque élève doit choisir une et une seule langue vivante parmi~: anglais, allemand et espagnol.
\par
Le tableau incomplet ci-dessous présente la répartition des langues choisie en fonction du sexe de l'élève~:
\begin{center}
     \begin{tabular}{|c|c|c|c|c|} %class="compact"
          \hline
          & Anglais & Allemand & Espagnol & Total \\
          \hline
          Garçons & 10 & 2 &   & 15 \\
          \hline
          Filles &   &   & 1 &   \\
          \hline
          Total & 16  &   &  &  24 \\
          \hline
     \end{tabular}
\end{center}
\begin{enumerate}
     \item %
     Recopier et compléter le tableau ci-dessus.
     \item %
     On choisit un élève au hasard.\\
     Quelle est la probabilité :
     \begin{enumerate}[label=\alph*.]
          \item %
          que l'élève soit un garçon ayant choisi l'anglais~?
          \item %
          que l'élève soit une fille~?
     \end{enumerate}
     \item %
     On interroge une fille choisie au hasard. \\
     Quelle est la probabilité qu'elle ait choisi l'allemand~?
\end{enumerate}
\begin{corrige}
     \begin{enumerate}
          \item %
          ~\\
          \begin{center}
               \begin{tabular}{|c|c|c|c|c|} %class="compact"
                    \hline
                    & Anglais & Allemand & Espagnol & Total \\
                    \hline
                    Garçons & 10 & 2 & $\red 3$   & 15 \\
                    \hline
                    Filles & $\red 6$   & $\red 2$   & 1 &  $\red 9$  \\
                    \hline
                    Total & 16  &  $\red 4$  & $\red 4$  &  24 \\
                    \hline
               \end{tabular}
          \end{center}
          \item %
          L'expression \og au hasard \fg{} indique que l'on est en situation d'\textbf{équiprobabilité}.
          \par
          Dans chacune des questions suivantes, on calculera donc les probabilités en utilisant la formule~:
          \begin{center}
               $p=\dfrac{\text{nombre d'issues favorables à l'événement}}{\text{nombre total d'issues possibles}}.$
          \end{center}
          \begin{enumerate}[label=\alph*.]
               \item %
               Il y a 10 garçons ayant choisi l'anglais sur un total de 24 élèves.
               \par
               La probabilité demandée est donc~:\\
               $p=\dfrac{10}{24}=\dfrac{5}{12}.$
               \item %
               Il y a 9 filles  sur un total de 24 élèves.
               \par
               La probabilité cherchée est alors~:\\
               $p=\dfrac{9}{24}.$
          \end{enumerate}
          \item %
          2 filles ont choisi l'allemand sur un total de 9 filles.
          \par
          La probabilité que la fille interrogée ait choisi l'allemand est donc~:\\
          $p=\dfrac{2}{9}.$
     \end{enumerate}
\end{corrige}

\end{document}
µ
\documentclass[a4paper]{article}

%================================================================================================================================
%
% Packages
%
%================================================================================================================================

\usepackage[T1]{fontenc} 	% pour caractères accentués
\usepackage[utf8]{inputenc}  % encodage utf8
\usepackage[french]{babel}	% langue : français
\usepackage{fourier}			% caractères plus lisibles
\usepackage[dvipsnames]{xcolor} % couleurs
\usepackage{fancyhdr}		% réglage header footer
\usepackage{needspace}		% empêcher sauts de page mal placés
\usepackage{graphicx}		% pour inclure des graphiques
\usepackage{enumitem,cprotect}		% personnalise les listes d'items (nécessaire pour ol, al ...)
\usepackage{hyperref}		% Liens hypertexte
\usepackage{pstricks,pst-all,pst-node,pstricks-add,pst-math,pst-plot,pst-tree,pst-eucl} % pstricks
\usepackage[a4paper,includeheadfoot,top=2cm,left=3cm, bottom=2cm,right=3cm]{geometry} % marges etc.
\usepackage{comment}			% commentaires multilignes
\usepackage{amsmath,environ} % maths (matrices, etc.)
\usepackage{amssymb,makeidx}
\usepackage{bm}				% bold maths
\usepackage{tabularx}		% tableaux
\usepackage{colortbl}		% tableaux en couleur
\usepackage{fontawesome}		% Fontawesome
\usepackage{environ}			% environment with command
\usepackage{fp}				% calculs pour ps-tricks
\usepackage{multido}			% pour ps tricks
\usepackage[np]{numprint}	% formattage nombre
\usepackage{tikz,tkz-tab} 			% package principal TikZ
\usepackage{pgfplots}   % axes
\usepackage{mathrsfs}    % cursives
\usepackage{calc}			% calcul taille boites
\usepackage[scaled=0.875]{helvet} % font sans serif
\usepackage{svg} % svg
\usepackage{scrextend} % local margin
\usepackage{scratch} %scratch
\usepackage{multicol} % colonnes
%\usepackage{infix-RPN,pst-func} % formule en notation polanaise inversée
\usepackage{listings}

%================================================================================================================================
%
% Réglages de base
%
%================================================================================================================================

\lstset{
language=Python,   % R code
literate=
{á}{{\'a}}1
{à}{{\`a}}1
{ã}{{\~a}}1
{é}{{\'e}}1
{è}{{\`e}}1
{ê}{{\^e}}1
{í}{{\'i}}1
{ó}{{\'o}}1
{õ}{{\~o}}1
{ú}{{\'u}}1
{ü}{{\"u}}1
{ç}{{\c{c}}}1
{~}{{ }}1
}


\definecolor{codegreen}{rgb}{0,0.6,0}
\definecolor{codegray}{rgb}{0.5,0.5,0.5}
\definecolor{codepurple}{rgb}{0.58,0,0.82}
\definecolor{backcolour}{rgb}{0.95,0.95,0.92}

\lstdefinestyle{mystyle}{
    backgroundcolor=\color{backcolour},   
    commentstyle=\color{codegreen},
    keywordstyle=\color{magenta},
    numberstyle=\tiny\color{codegray},
    stringstyle=\color{codepurple},
    basicstyle=\ttfamily\footnotesize,
    breakatwhitespace=false,         
    breaklines=true,                 
    captionpos=b,                    
    keepspaces=true,                 
    numbers=left,                    
xleftmargin=2em,
framexleftmargin=2em,            
    showspaces=false,                
    showstringspaces=false,
    showtabs=false,                  
    tabsize=2,
    upquote=true
}

\lstset{style=mystyle}


\lstset{style=mystyle}
\newcommand{\imgdir}{C:/laragon/www/newmc/assets/imgsvg/}
\newcommand{\imgsvgdir}{C:/laragon/www/newmc/assets/imgsvg/}

\definecolor{mcgris}{RGB}{220, 220, 220}% ancien~; pour compatibilité
\definecolor{mcbleu}{RGB}{52, 152, 219}
\definecolor{mcvert}{RGB}{125, 194, 70}
\definecolor{mcmauve}{RGB}{154, 0, 215}
\definecolor{mcorange}{RGB}{255, 96, 0}
\definecolor{mcturquoise}{RGB}{0, 153, 153}
\definecolor{mcrouge}{RGB}{255, 0, 0}
\definecolor{mclightvert}{RGB}{205, 234, 190}

\definecolor{gris}{RGB}{220, 220, 220}
\definecolor{bleu}{RGB}{52, 152, 219}
\definecolor{vert}{RGB}{125, 194, 70}
\definecolor{mauve}{RGB}{154, 0, 215}
\definecolor{orange}{RGB}{255, 96, 0}
\definecolor{turquoise}{RGB}{0, 153, 153}
\definecolor{rouge}{RGB}{255, 0, 0}
\definecolor{lightvert}{RGB}{205, 234, 190}
\setitemize[0]{label=\color{lightvert}  $\bullet$}

\pagestyle{fancy}
\renewcommand{\headrulewidth}{0.2pt}
\fancyhead[L]{maths-cours.fr}
\fancyhead[R]{\thepage}
\renewcommand{\footrulewidth}{0.2pt}
\fancyfoot[C]{}

\newcolumntype{C}{>{\centering\arraybackslash}X}
\newcolumntype{s}{>{\hsize=.35\hsize\arraybackslash}X}

\setlength{\parindent}{0pt}		 
\setlength{\parskip}{3mm}
\setlength{\headheight}{1cm}

\def\ebook{ebook}
\def\book{book}
\def\web{web}
\def\type{web}

\newcommand{\vect}[1]{\overrightarrow{\,\mathstrut#1\,}}

\def\Oij{$\left(\text{O}~;~\vect{\imath},~\vect{\jmath}\right)$}
\def\Oijk{$\left(\text{O}~;~\vect{\imath},~\vect{\jmath},~\vect{k}\right)$}
\def\Ouv{$\left(\text{O}~;~\vect{u},~\vect{v}\right)$}

\hypersetup{breaklinks=true, colorlinks = true, linkcolor = OliveGreen, urlcolor = OliveGreen, citecolor = OliveGreen, pdfauthor={Didier BONNEL - https://www.maths-cours.fr} } % supprime les bordures autour des liens

\renewcommand{\arg}[0]{\text{arg}}

\everymath{\displaystyle}

%================================================================================================================================
%
% Macros - Commandes
%
%================================================================================================================================

\newcommand\meta[2]{    			% Utilisé pour créer le post HTML.
	\def\titre{titre}
	\def\url{url}
	\def\arg{#1}
	\ifx\titre\arg
		\newcommand\maintitle{#2}
		\fancyhead[L]{#2}
		{\Large\sffamily \MakeUppercase{#2}}
		\vspace{1mm}\textcolor{mcvert}{\hrule}
	\fi 
	\ifx\url\arg
		\fancyfoot[L]{\href{https://www.maths-cours.fr#2}{\black \footnotesize{https://www.maths-cours.fr#2}}}
	\fi 
}


\newcommand\TitreC[1]{    		% Titre centré
     \needspace{3\baselineskip}
     \begin{center}\textbf{#1}\end{center}
}

\newcommand\newpar{    		% paragraphe
     \par
}

\newcommand\nosp {    		% commande vide (pas d'espace)
}
\newcommand{\id}[1]{} %ignore

\newcommand\boite[2]{				% Boite simple sans titre
	\vspace{5mm}
	\setlength{\fboxrule}{0.2mm}
	\setlength{\fboxsep}{5mm}	
	\fcolorbox{#1}{#1!3}{\makebox[\linewidth-2\fboxrule-2\fboxsep]{
  		\begin{minipage}[t]{\linewidth-2\fboxrule-4\fboxsep}\setlength{\parskip}{3mm}
  			 #2
  		\end{minipage}
	}}
	\vspace{5mm}
}

\newcommand\CBox[4]{				% Boites
	\vspace{5mm}
	\setlength{\fboxrule}{0.2mm}
	\setlength{\fboxsep}{5mm}
	
	\fcolorbox{#1}{#1!3}{\makebox[\linewidth-2\fboxrule-2\fboxsep]{
		\begin{minipage}[t]{1cm}\setlength{\parskip}{3mm}
	  		\textcolor{#1}{\LARGE{#2}}    
 	 	\end{minipage}  
  		\begin{minipage}[t]{\linewidth-2\fboxrule-4\fboxsep}\setlength{\parskip}{3mm}
			\raisebox{1.2mm}{\normalsize\sffamily{\textcolor{#1}{#3}}}						
  			 #4
  		\end{minipage}
	}}
	\vspace{5mm}
}

\newcommand\cadre[3]{				% Boites convertible html
	\par
	\vspace{2mm}
	\setlength{\fboxrule}{0.1mm}
	\setlength{\fboxsep}{5mm}
	\fcolorbox{#1}{white}{\makebox[\linewidth-2\fboxrule-2\fboxsep]{
  		\begin{minipage}[t]{\linewidth-2\fboxrule-4\fboxsep}\setlength{\parskip}{3mm}
			\raisebox{-2.5mm}{\sffamily \small{\textcolor{#1}{\MakeUppercase{#2}}}}		
			\par		
  			 #3
 	 		\end{minipage}
	}}
		\vspace{2mm}
	\par
}

\newcommand\bloc[3]{				% Boites convertible html sans bordure
     \needspace{2\baselineskip}
     {\sffamily \small{\textcolor{#1}{\MakeUppercase{#2}}}}    
		\par		
  			 #3
		\par
}

\newcommand\CHelp[1]{
     \CBox{Plum}{\faInfoCircle}{À RETENIR}{#1}
}

\newcommand\CUp[1]{
     \CBox{NavyBlue}{\faThumbsOUp}{EN PRATIQUE}{#1}
}

\newcommand\CInfo[1]{
     \CBox{Sepia}{\faArrowCircleRight}{REMARQUE}{#1}
}

\newcommand\CRedac[1]{
     \CBox{PineGreen}{\faEdit}{BIEN R\'EDIGER}{#1}
}

\newcommand\CError[1]{
     \CBox{Red}{\faExclamationTriangle}{ATTENTION}{#1}
}

\newcommand\TitreExo[2]{
\needspace{4\baselineskip}
 {\sffamily\large EXERCICE #1\ (\emph{#2 points})}
\vspace{5mm}
}

\newcommand\img[2]{
          \includegraphics[width=#2\paperwidth]{\imgdir#1}
}

\newcommand\imgsvg[2]{
       \begin{center}   \includegraphics[width=#2\paperwidth]{\imgsvgdir#1} \end{center}
}


\newcommand\Lien[2]{
     \href{#1}{#2 \tiny \faExternalLink}
}
\newcommand\mcLien[2]{
     \href{https~://www.maths-cours.fr/#1}{#2 \tiny \faExternalLink}
}

\newcommand{\euro}{\eurologo{}}

%================================================================================================================================
%
% Macros - Environement
%
%================================================================================================================================

\newenvironment{tex}{ %
}
{%
}

\newenvironment{indente}{ %
	\setlength\parindent{10mm}
}

{
	\setlength\parindent{0mm}
}

\newenvironment{corrige}{%
     \needspace{3\baselineskip}
     \medskip
     \textbf{\textsc{Corrigé}}
     \medskip
}
{
}

\newenvironment{extern}{%
     \begin{center}
     }
     {
     \end{center}
}

\NewEnviron{code}{%
	\par
     \boite{gray}{\texttt{%
     \BODY
     }}
     \par
}

\newenvironment{vbloc}{% boite sans cadre empeche saut de page
     \begin{minipage}[t]{\linewidth}
     }
     {
     \end{minipage}
}
\NewEnviron{h2}{%
    \needspace{3\baselineskip}
    \vspace{0.6cm}
	\noindent \MakeUppercase{\sffamily \large \BODY}
	\vspace{1mm}\textcolor{mcgris}{\hrule}\vspace{0.4cm}
	\par
}{}

\NewEnviron{h3}{%
    \needspace{3\baselineskip}
	\vspace{5mm}
	\textsc{\BODY}
	\par
}

\NewEnviron{margeneg}{ %
\begin{addmargin}[-1cm]{0cm}
\BODY
\end{addmargin}
}

\NewEnviron{html}{%
}

\begin{document}
\meta{url}{/methode/relation-de-chasles-et-calculs-vectoriels/}
\meta{pid}{10320}
\meta{titre}{Relation de Chasles et Calculs vectoriels}
\meta{type}{methode}
%
\boite{gray}{
     Michel Chasles est un mathématicien français du 19ème siècle. Il n'a pas découvert la relation qui porte son nom et qui était déjà connue depuis plusieurs années mais il a contribué, par ses travaux,  à populariser cette relation dans les pays francophones.
}
\par
Différentes méthodes peuvent être utilisées pour simplifier des expressions vectorielles. La plupart d'entre elles sont basées sur la relation de Chasles.
\par
\textit{Dans toute la suite, les lettres A, B, C, etc. désignent des points du plan.}
%
\begin{h2}1. Utilisation de la relation de Chasles\end{h2}
%
La relation de Chasles indique que pour trois points $A, B, C$ quelconques du plan :
\begin{center}
     $\overrightarrow{AB}+\overrightarrow{BC}=\overrightarrow{AC}$
\end{center}
\begin{center}
     \begin{extern}%width="210" alt="relation de Chasles"
          \resizebox{5.5cm}{!}{
               \psset{xunit=1.0cm,yunit=1.0cm,algebraic=true,dimen=middle,dotstyle=*,dotsize=3pt 0,linewidth=1pt,arrowsize=3pt 2,arrowinset=0.25}
               \begin{pspicture*}(0.,1.5)(8.2,6.)
                    \begin{Large}
                         \psline{->}(1.,3.)(4.,5.)
                         \psline{->}(4,5)(7.,2.)
                         \psline[linecolor=blue]{->}(1.,3.)(7.,2.)
                         \psdots[dotstyle=*](1.,3.)
                         \rput[bl](0.38,2.94){{$A$}}
                         \psdots[dotstyle=*](4.,5.)
                         \rput[bl](4.08,5.2){{$B$}}
                         \psdots[dotstyle=*](7.,2.)
                         \rput[bl](7.08,2.2){{$C$}}
                    \end{Large}
               \end{pspicture*}
          }
     \end{extern}
\end{center}
\begin{itemize}
     \item %
     Il faut remarquer que l'extrémité du premier vecteur est identique à l'origine du second~; ce point situé \og à l'intérieur \fg{} (ici $B$) disparaît dans le résultat (ici $\overrightarrow{AC}$) tandis que restent les extrémités (ici $A$ et $C$) dans le même ordre : $\overrightarrow{{\color{blue} A}{\color{red} B}}+\overrightarrow{{\color{red} B} {\color{blue}C}}=\overrightarrow{{\color{blue} AC}}$
     \item %
     Dans un vecteur, l'ordre des points a de l'importance ! Les vecteurs $\overrightarrow{AB}$ et $\overrightarrow{BA}$ ne sont pas égaux mais ils sont opposés (voir \textbf{2.}).\\
     Par contre, dans une somme, l'ordre des vecteurs n'a pas d'importance : $\overrightarrow{AB}+\overrightarrow{BC}$ est égal à $\overrightarrow{BC}+\overrightarrow{AB}$.
\end{itemize}
\bloc{orange}{Exemple 1}{ % id="e10"
     Simplifier $\overrightarrow{AB}+\overrightarrow{BC}+\overrightarrow{CA}.$
     \par
     \textbf{Solution :}
     \par
     On utilise deux fois la relation de Chasles :
     \par
     $\overrightarrow{AB}+\overrightarrow{BC}+\overrightarrow{CA} = \overrightarrow{AC}+\overrightarrow{CA} $~ ~(car $\overrightarrow{AB}+\overrightarrow{BC}=\overrightarrow{AC}$)\\
     $\phantom{{AB}+{BC}+{CA} }= \overrightarrow{AA} .$
     \par
     Le vecteur $\overrightarrow{AA}$ a son origine égale à son extrémité : c'est le vecteur nul $\overrightarrow{0}.$
     \par
     Finalement :
     \begin{center}
          $\overrightarrow{AB}+\overrightarrow{BC}+\overrightarrow{CA} =\overrightarrow{0}$
     \end{center}
} % fin ex
\bloc{orange}{Exemple 2}{ % id="e20"
     Simplifier $\overrightarrow{AB}+\overrightarrow{CA}+\overrightarrow{CB}.$
     \par
     \textbf{Solution :}
     \par
     Il suffit de modifier l'ordre des deux premiers termes pour pouvoir appliquer la relation de Chasles~:
     \par
     $\overrightarrow{AB}+\overrightarrow{CA}+\overrightarrow{CB}=\overrightarrow{CA}+\overrightarrow{AB}+\overrightarrow{CB}$\\
     $\phantom{{AB}+{CA}+{CB}}=\overrightarrow{CB}+\overrightarrow{CB}$~ ~(car $\overrightarrow{CA}+\overrightarrow{AB}=\overrightarrow{CB}$)\\
     $\phantom{{AB}+{CA}+{CB}}=2\overrightarrow{CB}.$\\
} % fin ex
\begin{h2}2. Utiliser les règles de calcul sur les vecteurs\end{h2}
Les règles suivantes, en particulier, sont fréquemment utilisées~:
\cadre{vert}{Règles de calcul }{ % id="p40"
     \textbf{Opposés~:} $\overrightarrow{BA}= -\overrightarrow{AB}$
     \par
     \textbf{Distributivité~:} $k \left(\overrightarrow{AB}+\overrightarrow{AC} \right)=k \overrightarrow{AB}+k \overrightarrow{AC} $
     \par
     \textbf{Milieu~:} Si $M$ est le milieu de$[AB]$ :
     \begin{center}
          \begin{extern}%width="250" alt="vecteurs et milieu"
               \psset{xunit=1.0cm,yunit=1.0cm,algebraic=true,dimen=middle,dotstyle=*,dotsize=3pt 0,linewidth=1.pt,arrowsize=3pt 2,arrowinset=0.25}
               \begin{pspicture*}(0.,1.)(7.,4.)
                    \psline(1.,2.)(6.,3.)
                    \psline[linecolor=gray](2.17,2.35)(2.22,2.12)
                    \psline[linecolor=gray](2.27,2.37)(2.32,2.14)
                    \psline[linecolor=gray](4.67,2.85)(4.72,2.62)
                    \psline[linecolor=gray](4.77,2.87)(4.82,2.64)
                    \rput[br](0.95,2.05){$A$}
                    \rput[bl](6.05,3.05){$B$}
                    \psdots(1.,2.)(6.,3.)
                    \psdot[linecolor=red](3.5,2.5)
                    \rput[b](3.5,2.7){\red{$M$}}
               \end{pspicture*}
          \end{extern}
     \end{center}
     \begin{itemize}
          \item %
          $\overrightarrow{AM}=\overrightarrow{MB}$
          \item %
          $\overrightarrow{MA}+\overrightarrow{MB}=\overrightarrow{0}$
          \item %
          $\overrightarrow{AM}=\dfrac{1}{2}\overrightarrow{AB}.$
     \end{itemize}
} % fin pr
\par
\textbf{Remarque~: }La distributivité peut être aussi bien utilisée pour développer que pour \og factoriser \fg{} une expression.
\bloc{orange}{Exemple 1}{ % id="e70"
     Simplifier $2(\overrightarrow{AB}+\overrightarrow{AC})-\overrightarrow{BA}+\overrightarrow{CA}.$
     \par
     \textbf{Solution :}
     \par
     $2(\overrightarrow{AB}+\overrightarrow{AC})-\overrightarrow{BA}+\overrightarrow{CA}=2\overrightarrow{AB}+2\overrightarrow{AC}-\overrightarrow{BA}+\overrightarrow{CA}$ (distributivité)\\
     $\phantom{2({AB}+{AC})-{BA}+{CA}}=2\overrightarrow{AB}+2\overrightarrow{AC}+\overrightarrow{AB}-\overrightarrow{AC}$ (opposés)\\
     $\phantom{2({AB}+{AC})-{BA}+{CA}}=3\overrightarrow{AB}+\overrightarrow{AC}.$
} % fin ex
\bloc{orange}{Exemple 2}{ % id="e80"
     $ABCD$ est un parallélogramme de centre $O$.
     \begin{center}
          \begin{extern}%width="250" alt="Parallélogramme - Calcul vectoriel"
               \psset{xunit=1.0cm,yunit=1.0cm,algebraic=true,dimen=middle,dotstyle=o,dotsize=5pt 0,linewidth=1.pt,arrowsize=3pt 2,arrowinset=0.25}
               \begin{pspicture*}(0.,0.)(7.,5.)
                    \psline(1.,1.)(2.,3.)
                    \psline(2.,3.)(6.,4.)
                    \psline(6.,4.)(5.,2.)
                    \psline(5.,2.)(1.,1.)
                    \psline[linecolor=red](1.,1.)(6.,4.)
                    \psline[linecolor=red](2.,3.)(5.,2.)
                    \begin{scriptsize}
                         \rput[tr](0.95,0.95){$A$}
                         \rput[br](1.95,3.05){$B$}
                         \rput[bl](6.05,4.05){$C$}
                         \rput[tl](5.05,1.95){$D$}
                         \rput[b](3.5,2.6){\red{$O$}}
                    \end{scriptsize}
               \end{pspicture*}
          \end{extern}
     \end{center}
     Calculer $\overrightarrow{OA}+\overrightarrow{OB}+\overrightarrow{OC}+\overrightarrow{OD}.$
} % fin ex
\par
\textbf{Solution :}
\par
On regroupe les termes de la façon suivante :
\par
$\overrightarrow{OA}+\overrightarrow{OB}+\overrightarrow{OC}+\overrightarrow{OD}=\left(\overrightarrow{OA}+\overrightarrow{OC}\right)+\left(\overrightarrow{OB}+\overrightarrow{OD}\right).$
\par
Dans un parallélogramme les diagonales se coupent en leur milieu.\\ Par conséquent, $O$ est le milieu de $[AC]$ et de $[BD]$.
\par
Donc $\overrightarrow{OA}+\overrightarrow{OC}=\overrightarrow{0}$ et $\overrightarrow{OB}+\overrightarrow{OD}=\overrightarrow{0}$ (propriété du milieu).
\par
On en déduit que :
\begin{center}
     $\overrightarrow{OA}+\overrightarrow{OB}+\overrightarrow{OC}+\overrightarrow{OD}=\overrightarrow{0}+\overrightarrow{0}=\overrightarrow{0}.$
\end{center}
\begin{h2}3. Utiliser la relation de Chasles \og en sens inverse \fg{} \end{h2}
La relation de Chasles $\overrightarrow{AB}+\overrightarrow{BC}=\overrightarrow{AC}$, pour tous points $A, B, C$, peut également s'écrire dans l'autre sens :
\begin{center}
     $\overrightarrow{AC}=\overrightarrow{AB}+\overrightarrow{BC}.$
\end{center}
Cela revient à \og insérer  \fg{}  un point $B$ quelconque entre $A$ et $C$:
\begin{center}
     $\overrightarrow{AC}=\overrightarrow{AB}+\overrightarrow{BC}$\alt{\\}{<hr style="visibility:hidden;margin: -0.9em;padding: 0;">}
     $\phantom{A}\red \uparrow\phantom{C={AB}+{BC}}$\alt{\\}{<hr style="visibility:hidden;margin: -0.5em;padding: 0;">}
     $\phantom{A}\red B\phantom{C={AB}+{BC}}$
\end{center}
$B$ peut être \textbf{n'importe quel point du plan}. Cette méthode est particulièrement puissante mais la difficulté consiste parfois à trouver le point qu'il sera intéressant d'introduire.
%
\bloc{orange}{Exemple 1}{ % id="e90"
     Soit $ABC$ un triangle quelconque et $I$ le milieu de $[BC].$
     \par
     Montrer que $\overrightarrow{AB}+\overrightarrow{AC}=2\overrightarrow{AI}.$
     \par
     \textbf{Solution :}
     \par
     On introduit le point $I$ dans $\overrightarrow{AB}$ ($\overrightarrow{AB}= \overrightarrow{AI}+\overrightarrow{IB}$) et dans $\overrightarrow{AC}$ ($\overrightarrow{AC}= \overrightarrow{AI}+\overrightarrow{IC}$)~:
     \par
     $\overrightarrow{AB}+\overrightarrow{AC}=\overrightarrow{AI}+\overrightarrow{IB}+\overrightarrow{AI}+\overrightarrow{IC}$\\
     $\phantom{{AB}+{AC}}=2\overrightarrow{AI}+\overrightarrow{IB}+\overrightarrow{IC}$
     \par
     Or $I$ étant le milieu de $[BC]$ : $\overrightarrow{IB}+\overrightarrow{IC}=\overrightarrow{0}.$ (voir \textbf{2.}).
     \par
     Finalement on obtient bien :
     \begin{center}
          $\overrightarrow{AB}+\overrightarrow{AC}=2\overrightarrow{AI}.$
     \end{center}
} % fin ex
%
\bloc{orange}{Exemple 2}{ % id="e100"
     Soient trois points quelconques $A, B, C$ et le point $P$ tel que $\overrightarrow{BP}=\dfrac{1}{3}\overrightarrow{BC}.$
     \begin{center}
          \begin{extern}%width="250" alt="relation de Chasles : Exemple"
               \psset{xunit=1.0cm,yunit=1.0cm,algebraic=true,dimen=middle,dotstyle=*,dotsize=3pt 0,linewidth=1.pt,arrowsize=3pt 2,arrowinset=0.25}
               \begin{pspicture*}(0.,0.)(8.,5.)
                    \psline(1.,1.)(3.,4.)
                    \psline(3.,4.)(6.,3.)
                    \psline(6.,3.)(1.,1.)
                    \psdots[linecolor=blue](4.,3.67)
                    \rput[tr](1.0,1.){$A$}
                    \rput[b](3.0,4.1){$B$}
                    \rput[b](6.0,3.1){$C$}
                    \rput[b](4,3.8){$\blue P$}
               \end{pspicture*}
          \end{extern}
     \end{center}
     \begin{enumerate}[label=\alph*.]
          \item %
          Montrer que $\overrightarrow{PC}=2 \overrightarrow{BP}.$
          \item %
          Montrer que $2\overrightarrow{AB}+\overrightarrow{AC}=3 \overrightarrow{AP}.$
     \end{enumerate}
     \par
     \textbf{Solution :}
     \begin{enumerate}[label=\alph*.]
          \item %
          On introduit le point $B$ dans le vecteur $\overrightarrow{PC}$~:
          \par
          $\overrightarrow{PC}=\overrightarrow{PB}+\overrightarrow{BC}.$
          \par
          D'après l'énoncé $\overrightarrow{BP}=\dfrac{1}{3}\overrightarrow{BC}$, donc (en multipliant chaque membre par 3) $\overrightarrow{BC}=3 \overrightarrow{BP}$~; et, en remplaçant $\overrightarrow{BC}$ par $3 \overrightarrow{BP}$, on obtient~:
          \par
          $\overrightarrow{PC}=\overrightarrow{PB}+3 \overrightarrow{BP}$\\
          $\phantom{{PC}}=-\overrightarrow{BP}+3 \overrightarrow{BP}$\\
          $\phantom{{PC}}=2 \overrightarrow{BP}.$
          \item %
          On introduit le point $P$ dans $\overrightarrow{AB}$ et $\overrightarrow{AC}$~:
          \par
          $2\overrightarrow{AB}+\overrightarrow{AC}=2 \left(\overrightarrow{AP}+\overrightarrow{PB} \right)+\overrightarrow{AP}+\overrightarrow{PC}$ (Bien penser aux parenthèses !)\\
          $\phantom{2{AB}+{AC}}=2 \overrightarrow{AP}+2\overrightarrow{PB}+\overrightarrow{AP}+\overrightarrow{PC}$ (Distributivité)\\
          $\phantom{2{AB}+{AC}}=3 \overrightarrow{AP}+2\overrightarrow{PB}+\overrightarrow{PC}$
          \par
          Or, d'après la question \textbf{a.} $\overrightarrow{PC}=2 \overrightarrow{BP}$, donc :
          \par
          $2\overrightarrow{AB}+\overrightarrow{AC}=3 \overrightarrow{AP}+2\overrightarrow{PB}+2\overrightarrow{BP}$ \\
          $\phantom{2{AB}+{AC}}=3 \overrightarrow{AP}$
          \par
          car les vecteurs  $2\overrightarrow{PB}$ et $2\overrightarrow{BP}$ sont opposés.
     \end{enumerate}
} % fin ex
%
\bloc{orange}{Exemple 3}{ % id="e110"
     Soit $ABCD$ un quadrilatère quelconque et $I, J, K, L$ les milieux de $[AB], [BC], [CD], [DA].$
     \begin{center}
          \begin{extern}%width="250" alt="milieu des côtés d'un parallélogramme"
               \psset{xunit=1.0cm,yunit=1.0cm,algebraic=true,dimen=middle,dotstyle=o,dotsize=4pt 0,linewidth=1.pt,arrowsize=3pt 2,arrowinset=0.25}
               \begin{pspicture*}(0.,0.)(8.,5.)
                    \psline[linecolor=red]{->}(2.,2.5)(4.5,3.5)
                    \psline[linecolor=red]{->}(4.,1.)(6.5,2.)
                    \psline(1.,1.)(3.,4.)
                    \psline(3.,4.)(6.,3.)
                    \psline(6.,3.)(7.,1.)
                    \psline(7.,1.)(1.,1.)
                    \rput[tr](1.0,1.){$A$}
                    \rput[b](3.0,4.1){$B$}
                    \rput[b](6.0,3.1){$C$}
                    \rput[tl](7.0,1.){$D$}
                    \rput[r](1.9,2.5){$I$}
                    \rput[b](4.5,3.66){$J$}
                    \rput[l](6.6,2.){$K$}
                    \rput[t](4.0,0.9){$L$}
               \end{pspicture*}
          \end{extern}
     \end{center}
     Montrer que $\overrightarrow{IJ}=\overrightarrow{LK}.$
     \par
     \textbf{Solution :}
     \par
     Voici une solution (parmi d'autres...) :
     \par
     $\overrightarrow{IJ}=\overrightarrow{IB}+\overrightarrow{BJ}$ (introduction du point $B$)\\
     $\phantom{{IJ}}=\dfrac{1}{2}\overrightarrow{AB}+\dfrac{1}{2}\overrightarrow{BC}$ (propriété des milieux)\\
     $\phantom{{IJ}}=\dfrac{1}{2} \left(\overrightarrow{AB}+\overrightarrow{BC} \right)$ (\og factorisation \fg{} )\\
     $\phantom{{IJ}}=\dfrac{1}{2} \overrightarrow{AC}$ (relation de Chasles).
     \par
     On procède ensuite de façon analogue pour $\overrightarrow{LK}$ :
     \par
     $\overrightarrow{LK}=\overrightarrow{LD}+\overrightarrow{DK}$ (introduction du point $D$)\\
     $\phantom{LK}=\dfrac{1}{2}\overrightarrow{AD}+\dfrac{1}{2}\overrightarrow{DC}$ (propriété des milieux)\\
     $\phantom{{LK}}=\dfrac{1}{2} \left(\overrightarrow{AD}+\overrightarrow{DC} \right)$ (\og factorisation \fg{} )\\
     $\phantom{{LK}}=\dfrac{1}{2} \overrightarrow{AC}$ (relation de Chasles).
     \par
     Par conséquent :      $\overrightarrow{IJ}=\overrightarrow{LK}.$
} % fin ex

\end{document}
µ
\documentclass[a4paper]{article}

%================================================================================================================================
%
% Packages
%
%================================================================================================================================

\usepackage[T1]{fontenc} 	% pour caractères accentués
\usepackage[utf8]{inputenc}  % encodage utf8
\usepackage[french]{babel}	% langue : français
\usepackage{fourier}			% caractères plus lisibles
\usepackage[dvipsnames]{xcolor} % couleurs
\usepackage{fancyhdr}		% réglage header footer
\usepackage{needspace}		% empêcher sauts de page mal placés
\usepackage{graphicx}		% pour inclure des graphiques
\usepackage{enumitem,cprotect}		% personnalise les listes d'items (nécessaire pour ol, al ...)
\usepackage{hyperref}		% Liens hypertexte
\usepackage{pstricks,pst-all,pst-node,pstricks-add,pst-math,pst-plot,pst-tree,pst-eucl} % pstricks
\usepackage[a4paper,includeheadfoot,top=2cm,left=3cm, bottom=2cm,right=3cm]{geometry} % marges etc.
\usepackage{comment}			% commentaires multilignes
\usepackage{amsmath,environ} % maths (matrices, etc.)
\usepackage{amssymb,makeidx}
\usepackage{bm}				% bold maths
\usepackage{tabularx}		% tableaux
\usepackage{colortbl}		% tableaux en couleur
\usepackage{fontawesome}		% Fontawesome
\usepackage{environ}			% environment with command
\usepackage{fp}				% calculs pour ps-tricks
\usepackage{multido}			% pour ps tricks
\usepackage[np]{numprint}	% formattage nombre
\usepackage{tikz,tkz-tab} 			% package principal TikZ
\usepackage{pgfplots}   % axes
\usepackage{mathrsfs}    % cursives
\usepackage{calc}			% calcul taille boites
\usepackage[scaled=0.875]{helvet} % font sans serif
\usepackage{svg} % svg
\usepackage{scrextend} % local margin
\usepackage{scratch} %scratch
\usepackage{multicol} % colonnes
%\usepackage{infix-RPN,pst-func} % formule en notation polanaise inversée
\usepackage{listings}

%================================================================================================================================
%
% Réglages de base
%
%================================================================================================================================

\lstset{
language=Python,   % R code
literate=
{á}{{\'a}}1
{à}{{\`a}}1
{ã}{{\~a}}1
{é}{{\'e}}1
{è}{{\`e}}1
{ê}{{\^e}}1
{í}{{\'i}}1
{ó}{{\'o}}1
{õ}{{\~o}}1
{ú}{{\'u}}1
{ü}{{\"u}}1
{ç}{{\c{c}}}1
{~}{{ }}1
}


\definecolor{codegreen}{rgb}{0,0.6,0}
\definecolor{codegray}{rgb}{0.5,0.5,0.5}
\definecolor{codepurple}{rgb}{0.58,0,0.82}
\definecolor{backcolour}{rgb}{0.95,0.95,0.92}

\lstdefinestyle{mystyle}{
    backgroundcolor=\color{backcolour},   
    commentstyle=\color{codegreen},
    keywordstyle=\color{magenta},
    numberstyle=\tiny\color{codegray},
    stringstyle=\color{codepurple},
    basicstyle=\ttfamily\footnotesize,
    breakatwhitespace=false,         
    breaklines=true,                 
    captionpos=b,                    
    keepspaces=true,                 
    numbers=left,                    
xleftmargin=2em,
framexleftmargin=2em,            
    showspaces=false,                
    showstringspaces=false,
    showtabs=false,                  
    tabsize=2,
    upquote=true
}

\lstset{style=mystyle}


\lstset{style=mystyle}
\newcommand{\imgdir}{C:/laragon/www/newmc/assets/imgsvg/}
\newcommand{\imgsvgdir}{C:/laragon/www/newmc/assets/imgsvg/}

\definecolor{mcgris}{RGB}{220, 220, 220}% ancien~; pour compatibilité
\definecolor{mcbleu}{RGB}{52, 152, 219}
\definecolor{mcvert}{RGB}{125, 194, 70}
\definecolor{mcmauve}{RGB}{154, 0, 215}
\definecolor{mcorange}{RGB}{255, 96, 0}
\definecolor{mcturquoise}{RGB}{0, 153, 153}
\definecolor{mcrouge}{RGB}{255, 0, 0}
\definecolor{mclightvert}{RGB}{205, 234, 190}

\definecolor{gris}{RGB}{220, 220, 220}
\definecolor{bleu}{RGB}{52, 152, 219}
\definecolor{vert}{RGB}{125, 194, 70}
\definecolor{mauve}{RGB}{154, 0, 215}
\definecolor{orange}{RGB}{255, 96, 0}
\definecolor{turquoise}{RGB}{0, 153, 153}
\definecolor{rouge}{RGB}{255, 0, 0}
\definecolor{lightvert}{RGB}{205, 234, 190}
\setitemize[0]{label=\color{lightvert}  $\bullet$}

\pagestyle{fancy}
\renewcommand{\headrulewidth}{0.2pt}
\fancyhead[L]{maths-cours.fr}
\fancyhead[R]{\thepage}
\renewcommand{\footrulewidth}{0.2pt}
\fancyfoot[C]{}

\newcolumntype{C}{>{\centering\arraybackslash}X}
\newcolumntype{s}{>{\hsize=.35\hsize\arraybackslash}X}

\setlength{\parindent}{0pt}		 
\setlength{\parskip}{3mm}
\setlength{\headheight}{1cm}

\def\ebook{ebook}
\def\book{book}
\def\web{web}
\def\type{web}

\newcommand{\vect}[1]{\overrightarrow{\,\mathstrut#1\,}}

\def\Oij{$\left(\text{O}~;~\vect{\imath},~\vect{\jmath}\right)$}
\def\Oijk{$\left(\text{O}~;~\vect{\imath},~\vect{\jmath},~\vect{k}\right)$}
\def\Ouv{$\left(\text{O}~;~\vect{u},~\vect{v}\right)$}

\hypersetup{breaklinks=true, colorlinks = true, linkcolor = OliveGreen, urlcolor = OliveGreen, citecolor = OliveGreen, pdfauthor={Didier BONNEL - https://www.maths-cours.fr} } % supprime les bordures autour des liens

\renewcommand{\arg}[0]{\text{arg}}

\everymath{\displaystyle}

%================================================================================================================================
%
% Macros - Commandes
%
%================================================================================================================================

\newcommand\meta[2]{    			% Utilisé pour créer le post HTML.
	\def\titre{titre}
	\def\url{url}
	\def\arg{#1}
	\ifx\titre\arg
		\newcommand\maintitle{#2}
		\fancyhead[L]{#2}
		{\Large\sffamily \MakeUppercase{#2}}
		\vspace{1mm}\textcolor{mcvert}{\hrule}
	\fi 
	\ifx\url\arg
		\fancyfoot[L]{\href{https://www.maths-cours.fr#2}{\black \footnotesize{https://www.maths-cours.fr#2}}}
	\fi 
}


\newcommand\TitreC[1]{    		% Titre centré
     \needspace{3\baselineskip}
     \begin{center}\textbf{#1}\end{center}
}

\newcommand\newpar{    		% paragraphe
     \par
}

\newcommand\nosp {    		% commande vide (pas d'espace)
}
\newcommand{\id}[1]{} %ignore

\newcommand\boite[2]{				% Boite simple sans titre
	\vspace{5mm}
	\setlength{\fboxrule}{0.2mm}
	\setlength{\fboxsep}{5mm}	
	\fcolorbox{#1}{#1!3}{\makebox[\linewidth-2\fboxrule-2\fboxsep]{
  		\begin{minipage}[t]{\linewidth-2\fboxrule-4\fboxsep}\setlength{\parskip}{3mm}
  			 #2
  		\end{minipage}
	}}
	\vspace{5mm}
}

\newcommand\CBox[4]{				% Boites
	\vspace{5mm}
	\setlength{\fboxrule}{0.2mm}
	\setlength{\fboxsep}{5mm}
	
	\fcolorbox{#1}{#1!3}{\makebox[\linewidth-2\fboxrule-2\fboxsep]{
		\begin{minipage}[t]{1cm}\setlength{\parskip}{3mm}
	  		\textcolor{#1}{\LARGE{#2}}    
 	 	\end{minipage}  
  		\begin{minipage}[t]{\linewidth-2\fboxrule-4\fboxsep}\setlength{\parskip}{3mm}
			\raisebox{1.2mm}{\normalsize\sffamily{\textcolor{#1}{#3}}}						
  			 #4
  		\end{minipage}
	}}
	\vspace{5mm}
}

\newcommand\cadre[3]{				% Boites convertible html
	\par
	\vspace{2mm}
	\setlength{\fboxrule}{0.1mm}
	\setlength{\fboxsep}{5mm}
	\fcolorbox{#1}{white}{\makebox[\linewidth-2\fboxrule-2\fboxsep]{
  		\begin{minipage}[t]{\linewidth-2\fboxrule-4\fboxsep}\setlength{\parskip}{3mm}
			\raisebox{-2.5mm}{\sffamily \small{\textcolor{#1}{\MakeUppercase{#2}}}}		
			\par		
  			 #3
 	 		\end{minipage}
	}}
		\vspace{2mm}
	\par
}

\newcommand\bloc[3]{				% Boites convertible html sans bordure
     \needspace{2\baselineskip}
     {\sffamily \small{\textcolor{#1}{\MakeUppercase{#2}}}}    
		\par		
  			 #3
		\par
}

\newcommand\CHelp[1]{
     \CBox{Plum}{\faInfoCircle}{À RETENIR}{#1}
}

\newcommand\CUp[1]{
     \CBox{NavyBlue}{\faThumbsOUp}{EN PRATIQUE}{#1}
}

\newcommand\CInfo[1]{
     \CBox{Sepia}{\faArrowCircleRight}{REMARQUE}{#1}
}

\newcommand\CRedac[1]{
     \CBox{PineGreen}{\faEdit}{BIEN R\'EDIGER}{#1}
}

\newcommand\CError[1]{
     \CBox{Red}{\faExclamationTriangle}{ATTENTION}{#1}
}

\newcommand\TitreExo[2]{
\needspace{4\baselineskip}
 {\sffamily\large EXERCICE #1\ (\emph{#2 points})}
\vspace{5mm}
}

\newcommand\img[2]{
          \includegraphics[width=#2\paperwidth]{\imgdir#1}
}

\newcommand\imgsvg[2]{
       \begin{center}   \includegraphics[width=#2\paperwidth]{\imgsvgdir#1} \end{center}
}


\newcommand\Lien[2]{
     \href{#1}{#2 \tiny \faExternalLink}
}
\newcommand\mcLien[2]{
     \href{https~://www.maths-cours.fr/#1}{#2 \tiny \faExternalLink}
}

\newcommand{\euro}{\eurologo{}}

%================================================================================================================================
%
% Macros - Environement
%
%================================================================================================================================

\newenvironment{tex}{ %
}
{%
}

\newenvironment{indente}{ %
	\setlength\parindent{10mm}
}

{
	\setlength\parindent{0mm}
}

\newenvironment{corrige}{%
     \needspace{3\baselineskip}
     \medskip
     \textbf{\textsc{Corrigé}}
     \medskip
}
{
}

\newenvironment{extern}{%
     \begin{center}
     }
     {
     \end{center}
}

\NewEnviron{code}{%
	\par
     \boite{gray}{\texttt{%
     \BODY
     }}
     \par
}

\newenvironment{vbloc}{% boite sans cadre empeche saut de page
     \begin{minipage}[t]{\linewidth}
     }
     {
     \end{minipage}
}
\NewEnviron{h2}{%
    \needspace{3\baselineskip}
    \vspace{0.6cm}
	\noindent \MakeUppercase{\sffamily \large \BODY}
	\vspace{1mm}\textcolor{mcgris}{\hrule}\vspace{0.4cm}
	\par
}{}

\NewEnviron{h3}{%
    \needspace{3\baselineskip}
	\vspace{5mm}
	\textsc{\BODY}
	\par
}

\NewEnviron{margeneg}{ %
\begin{addmargin}[-1cm]{0cm}
\BODY
\end{addmargin}
}

\NewEnviron{html}{%
}

\begin{document}
\meta{url}{/exercices/trapeze-et-vecteurs/}
\meta{pid}{10357}
\meta{titre}{Trapèze et vecteurs}
\meta{type}{exercices}
%
Le plan est muni d'un repère orthonormé $(O~,~\vec{i},~\vec{j})$.
On considère les points $A(2~;~4), B(5~;~5), C(1~;~1)$ et $D(7~;~3).$
\begin{enumerate}
     \item %
     Faire une figure.
     \item %
     Montrer que le quadrilatère $ABDC$ est un trapèze.
     \item %
     On note $E$ le symétrique de $C$ par rapport à $A$.\\
     Déterminer, par le calcul les coordonnées de $E$.
     \item %
     Montrer que $B$ est le milieu du segment $[ED]$.
     \item %
     Soient $M$ et $N$ les milieux respectifs des segments $[AB]$ et $[CD]$.\\
     Déterminer les coordonnées de $M$ et de $N$.\\
     En déduire que les ponts $E$, $M$ et $N$ sont alignés.
\end{enumerate}
\begin{corrige}
     \begin{enumerate}
          \item %
          ~\\
          \begin{center}
               \begin{extern}%width="400" alt="trapèze"
                    \psset{xunit=1.0cm,yunit=1.0cm,algebraic=true,dimen=middle,dotstyle=o,dotsize=5pt 0,linewidth=1pt,arrowsize=3pt 2,arrowinset=0.25}
                    \begin{pspicture*}(-1.,-1.)(8.,8.)
                         \multips(0,-1)(0,1.0){10}{\psline[linewidth=0.4pt,linecolor=lightgray]{c-c}(-1.,0)(8.,0)}
                         \multips(-1,0)(1.0,0){10}{\psline[linewidth=0.4pt,linecolor=lightgray]{c-c}(0,-1.)(0,8.)}
                         \psaxes[labelFontSize=\scriptstyle,linewidth=0.6pt,xAxis=true,yAxis=true,Dx=1.,Dy=1.,ticksize=-2pt 0,subticks=2]{->}(0,0)(-1.,-1.)(8.,8.)
                         \psline[linecolor=blue](2.,4.)(5.,5.)
                         \psline[linecolor=blue](5.,5.)(7.,3.)
                         \psline[linecolor=blue](7.,3.)(1.,1.)
                         \psline[linecolor=blue](1.,1.)(2.,4.)
                         \begin{scriptsize}
                              \rput[br](1.95,4.05){$A$}
                              \rput[bl](5.05,5.05){$B$}
                              \rput[tr](0.95,0.95){$C$}
                              \rput[tl](7.05,2.95){$D$}
                              %\rput[bl](3.05,7.05){$E$}
                              \rput[t](0.5,-0.15){$\vec{i}$}
                              \rput[r](-0.15,0.5){$\vec{j}$}
                              \rput[bl](0.15,0.15){$O$}
                              %\rput[bl](3.55,4.55){$M$}
                              %\rput[tl](4.05,2.05){$N$}
                         \end{scriptsize}
                    \end{pspicture*}
               \end{extern}
          \end{center}
          \item %
          Les coordonnées du vecteur 	$\overrightarrow{AB}$ sont :
          \par
          $\overrightarrow{AB} \begin{pmatrix}  x_B-x_A \\ y_B-y_A  \end{pmatrix} $\\
          $\overrightarrow{AB} \begin{pmatrix}  5-2 \\ 5-4  \end{pmatrix} $\\
          $\overrightarrow{AB} \begin{pmatrix}  3 \\ 1  \end{pmatrix} $
          \par
          De même,  les coordonnées du vecteur $\overrightarrow{CD}$ sont :
          \par
          $\overrightarrow{CD} \begin{pmatrix}  x_D-x_C \\ y_D-y_C  \end{pmatrix} $\\
          $\overrightarrow{CD} \begin{pmatrix}  7-1 \\ 3-1  \end{pmatrix} $\\
          $\overrightarrow{CD} \begin{pmatrix}  6 \\ 2 \end{pmatrix} $
          \par
          On remarque que $\overrightarrow{CD}=2\overrightarrow{AB}$~;~les vecteurs  $\overrightarrow{CD}$ et $\overrightarrow{AB}$ sont colinéaires donc les droites $(AB)$ et $(CD)$ sont parallèles.
          \par
          Par conséquent, $ABDC$ est un trapèze.
          \item %
          Notons $(x_E~;~y_E)$ les coordonnées du point $E$.
          \par
          $E$ le symétrique de $C$ par rapport à $A$, par conséquent $A$ est le milieu de $[EC]$.
          \par
          Les coordonnées du milieu de $[EC]$ sont $\left(\dfrac{x_E+x_C}{2}~;~\dfrac{y_E+y_C}{2}\right)$, c'est à dire  $\left(\dfrac{x_E+1}{2}~;~\dfrac{y_E+1}{2}\right)$.
          \par
          Les coordonnées de $A$ sont $(2~;~4)$~; $A$ est donc le milieu de $[EC]$ si et seulement si :
          \par
          $ \begin{cases} \dfrac{x_E+1}{2}=2 \\~\\  \dfrac{y_E+1}{2}=4\end{cases} \Leftrightarrow  \begin{cases} x_E+1=4 \\ y_E+1=8\end{cases} $\\
          $ \phantom{\begin{cases} \dfrac{x_E+1}{2}=2 \\~\\  \dfrac{y_E+1}{2}=4\end{cases}} \Leftrightarrow  \begin{cases} x_E=3 \\ y_E=7\end{cases} $
          \par
          Le point $E$ a donc pour coordonnées  $(3~;~7)$.
          \begin{center}
               \begin{extern}%width="400" alt="Position de E"
                    \psset{xunit=1.0cm,yunit=1.0cm,algebraic=true,dimen=middle,dotstyle=*,dotsize=3pt 0,linewidth=1pt,arrowsize=3pt 2,arrowinset=0.25}
                    \begin{pspicture*}(-1.,-1.)(8.,8.)
                         \multips(0,-1)(0,1.0){10}{\psline[linewidth=0.4pt,linecolor=lightgray]{c-c}(-1.,0)(8.,0)}
                         \multips(-1,0)(1.0,0){10}{\psline[linewidth=0.4pt,linecolor=lightgray]{c-c}(0,-1.)(0,8.)}
                         \psaxes[labelFontSize=\scriptstyle,linewidth=0.6pt,xAxis=true,yAxis=true,Dx=1.,Dy=1.,ticksize=-2pt 0,subticks=2]{->}(0,0)(-1.,-1.)(8.,8.)
                         \psline[linecolor=blue](2.,4.)(5.,5.)
                         \psline[linecolor=blue](5.,5.)(7.,3.)
                         \psline[linecolor=blue](7.,3.)(1.,1.)
                         \psline[linecolor=blue](1.,1.)(2.,4.)
                         \psdot[linecolor=red](3,7)
                         \begin{scriptsize}
                              \rput[br](1.95,4.05){$A$}
                              \rput[bl](5.05,5.05){$B$}
                              \rput[tr](0.95,0.95){$C$}
                              \rput[tl](7.05,2.95){$D$}
                              \rput[bl](3.05,7.05){$\red E$}
                              \rput[t](0.5,-0.15){$\vec{i}$}
                              \rput[r](-0.15,0.5){$\vec{j}$}
                              \rput[bl](0.15,0.15){$O$}
                              %\rput[bl](3.55,4.55){$M$}
                              %\rput[tl](4.05,2.05){$N$}
                         \end{scriptsize}
                    \end{pspicture*}
               \end{extern}
          \end{center}
          \item %
          Le milieu de $[ED]$ a pour coordonnées~:
          \par
          $\left(\dfrac{x_E+x_D}{2}~;~\dfrac{y_E+y_D}{2}\right)$\nosp$=\left(\dfrac{3+7}{2}~;~\dfrac{7+3}{2}\right)$\nosp$=(5~;~5).$
          \par
          Le milieu de $[ED]$ est donc le point $B.$
          \item %
          ~\\
          \begin{center}
               \begin{extern}%width="400" alt="Points alignés"
                    \psset{xunit=1.0cm,yunit=1.0cm,algebraic=true,dimen=middle,dotstyle=*,dotsize=3pt 0,linewidth=1pt,arrowsize=3pt 2,arrowinset=0.25}
                    \begin{pspicture*}(-1.,-1.)(8.,8.)
                         \multips(0,-1)(0,1.0){10}{\psline[linewidth=0.4pt,linecolor=lightgray]{c-c}(-1.,0)(8.,0)}
                         \multips(-1,0)(1.0,0){10}{\psline[linewidth=0.4pt,linecolor=lightgray]{c-c}(0,-1.)(0,8.)}
                         \psaxes[labelFontSize=\scriptstyle,linewidth=0.6pt,xAxis=true,yAxis=true,Dx=1.,Dy=1.,ticksize=-2pt 0,subticks=2]{->}(0,0)(-1.,-1.)(8.,8.)
                         \psline[linecolor=blue](2.,4.)(5.,5.)
                         \psline[linecolor=blue](5.,5.)(7.,3.)
                         \psline[linecolor=blue](7.,3.)(1.,1.)
                         \psline[linecolor=blue](1.,1.)(2.,4.)
                         \psdot[linecolor=red](3,7)
                         \psdot[linecolor=red](3.5,4.5)
                         \psdot[linecolor=red](4,2)
                         \begin{scriptsize}
                              \rput[br](1.95,4.05){$A$}
                              \rput[bl](5.05,5.05){$B$}
                              \rput[tr](0.95,0.95){$C$}
                              \rput[tl](7.05,2.95){$D$}
                              \rput[bl](3.05,7.05){$\red E$}
                              \rput[t](0.5,-0.15){$\vec{i}$}
                              \rput[r](-0.15,0.5){$\vec{j}$}
                              \rput[bl](0.15,0.15){$O$}
                              \rput[bl](3.55,4.65){$\red M$}
                              \rput[tl](4.05,1.95){$\red N$}
                         \end{scriptsize}
                    \end{pspicture*}
               \end{extern}
          \end{center}
          Les coordonnées du milieu $M$ de $[AB]$ sont $\left(\dfrac{x_A+x_B}{2}~;~\dfrac{y_A+y_B}{2}\right)$\nosp$=\left(\dfrac{7}{2}~;~\dfrac{9}{2}\right).$
          \par
          Les coordonnées du milieu $N$ de $[CD]$ sont $\left(\dfrac{x_C+x_D}{2}~;~\dfrac{y_C+y_D}{2}\right)$\nosp$=\left(4~;~2\right).$
          \par
          Les coordonnées du vecteur $\overrightarrow{EM}$ sont alors~:
          \par
          $\begin{pmatrix}  x_M-x_E\\y_M-y_E  \end{pmatrix}$\nosp$ =\begin{pmatrix} \dfrac{7}{2}-3\\ \\ \dfrac{9}{2}-7 \end{pmatrix} $\nosp$=\begin{pmatrix} \dfrac{1}{2}\\ \\ -\dfrac{5}{2} \end{pmatrix} $
          \par
          et les coordonnées du vecteur $\overrightarrow{EN}$~: $\begin{pmatrix}  x_N-x_E\\y_N-y_E  \end{pmatrix}$\nosp$ =\begin{pmatrix} 4-3 \\ 2-7 \end{pmatrix} $\nosp$=\begin{pmatrix} 1 \\ -5 \end{pmatrix} .$
          \par
          On remarque que $\overrightarrow{EN}=2\overrightarrow{EM}$, donc les vecteurs  $\overrightarrow{EN}$ et $\overrightarrow{EM}$ sont colinéaires et les points $E, M$ et $N$ sont alignés. ( La relation $\overrightarrow{EN}=2\overrightarrow{EM}$ montre également que $M$ est le milieu de $[EN]$.)
     \end{enumerate}
\end{corrige}

\end{document}
µ
\documentclass[a4paper]{article}

%================================================================================================================================
%
% Packages
%
%================================================================================================================================

\usepackage[T1]{fontenc} 	% pour caractères accentués
\usepackage[utf8]{inputenc}  % encodage utf8
\usepackage[french]{babel}	% langue : français
\usepackage{fourier}			% caractères plus lisibles
\usepackage[dvipsnames]{xcolor} % couleurs
\usepackage{fancyhdr}		% réglage header footer
\usepackage{needspace}		% empêcher sauts de page mal placés
\usepackage{graphicx}		% pour inclure des graphiques
\usepackage{enumitem,cprotect}		% personnalise les listes d'items (nécessaire pour ol, al ...)
\usepackage{hyperref}		% Liens hypertexte
\usepackage{pstricks,pst-all,pst-node,pstricks-add,pst-math,pst-plot,pst-tree,pst-eucl} % pstricks
\usepackage[a4paper,includeheadfoot,top=2cm,left=3cm, bottom=2cm,right=3cm]{geometry} % marges etc.
\usepackage{comment}			% commentaires multilignes
\usepackage{amsmath,environ} % maths (matrices, etc.)
\usepackage{amssymb,makeidx}
\usepackage{bm}				% bold maths
\usepackage{tabularx}		% tableaux
\usepackage{colortbl}		% tableaux en couleur
\usepackage{fontawesome}		% Fontawesome
\usepackage{environ}			% environment with command
\usepackage{fp}				% calculs pour ps-tricks
\usepackage{multido}			% pour ps tricks
\usepackage[np]{numprint}	% formattage nombre
\usepackage{tikz,tkz-tab} 			% package principal TikZ
\usepackage{pgfplots}   % axes
\usepackage{mathrsfs}    % cursives
\usepackage{calc}			% calcul taille boites
\usepackage[scaled=0.875]{helvet} % font sans serif
\usepackage{svg} % svg
\usepackage{scrextend} % local margin
\usepackage{scratch} %scratch
\usepackage{multicol} % colonnes
%\usepackage{infix-RPN,pst-func} % formule en notation polanaise inversée
\usepackage{listings}

%================================================================================================================================
%
% Réglages de base
%
%================================================================================================================================

\lstset{
language=Python,   % R code
literate=
{á}{{\'a}}1
{à}{{\`a}}1
{ã}{{\~a}}1
{é}{{\'e}}1
{è}{{\`e}}1
{ê}{{\^e}}1
{í}{{\'i}}1
{ó}{{\'o}}1
{õ}{{\~o}}1
{ú}{{\'u}}1
{ü}{{\"u}}1
{ç}{{\c{c}}}1
{~}{{ }}1
}


\definecolor{codegreen}{rgb}{0,0.6,0}
\definecolor{codegray}{rgb}{0.5,0.5,0.5}
\definecolor{codepurple}{rgb}{0.58,0,0.82}
\definecolor{backcolour}{rgb}{0.95,0.95,0.92}

\lstdefinestyle{mystyle}{
    backgroundcolor=\color{backcolour},   
    commentstyle=\color{codegreen},
    keywordstyle=\color{magenta},
    numberstyle=\tiny\color{codegray},
    stringstyle=\color{codepurple},
    basicstyle=\ttfamily\footnotesize,
    breakatwhitespace=false,         
    breaklines=true,                 
    captionpos=b,                    
    keepspaces=true,                 
    numbers=left,                    
xleftmargin=2em,
framexleftmargin=2em,            
    showspaces=false,                
    showstringspaces=false,
    showtabs=false,                  
    tabsize=2,
    upquote=true
}

\lstset{style=mystyle}


\lstset{style=mystyle}
\newcommand{\imgdir}{C:/laragon/www/newmc/assets/imgsvg/}
\newcommand{\imgsvgdir}{C:/laragon/www/newmc/assets/imgsvg/}

\definecolor{mcgris}{RGB}{220, 220, 220}% ancien~; pour compatibilité
\definecolor{mcbleu}{RGB}{52, 152, 219}
\definecolor{mcvert}{RGB}{125, 194, 70}
\definecolor{mcmauve}{RGB}{154, 0, 215}
\definecolor{mcorange}{RGB}{255, 96, 0}
\definecolor{mcturquoise}{RGB}{0, 153, 153}
\definecolor{mcrouge}{RGB}{255, 0, 0}
\definecolor{mclightvert}{RGB}{205, 234, 190}

\definecolor{gris}{RGB}{220, 220, 220}
\definecolor{bleu}{RGB}{52, 152, 219}
\definecolor{vert}{RGB}{125, 194, 70}
\definecolor{mauve}{RGB}{154, 0, 215}
\definecolor{orange}{RGB}{255, 96, 0}
\definecolor{turquoise}{RGB}{0, 153, 153}
\definecolor{rouge}{RGB}{255, 0, 0}
\definecolor{lightvert}{RGB}{205, 234, 190}
\setitemize[0]{label=\color{lightvert}  $\bullet$}

\pagestyle{fancy}
\renewcommand{\headrulewidth}{0.2pt}
\fancyhead[L]{maths-cours.fr}
\fancyhead[R]{\thepage}
\renewcommand{\footrulewidth}{0.2pt}
\fancyfoot[C]{}

\newcolumntype{C}{>{\centering\arraybackslash}X}
\newcolumntype{s}{>{\hsize=.35\hsize\arraybackslash}X}

\setlength{\parindent}{0pt}		 
\setlength{\parskip}{3mm}
\setlength{\headheight}{1cm}

\def\ebook{ebook}
\def\book{book}
\def\web{web}
\def\type{web}

\newcommand{\vect}[1]{\overrightarrow{\,\mathstrut#1\,}}

\def\Oij{$\left(\text{O}~;~\vect{\imath},~\vect{\jmath}\right)$}
\def\Oijk{$\left(\text{O}~;~\vect{\imath},~\vect{\jmath},~\vect{k}\right)$}
\def\Ouv{$\left(\text{O}~;~\vect{u},~\vect{v}\right)$}

\hypersetup{breaklinks=true, colorlinks = true, linkcolor = OliveGreen, urlcolor = OliveGreen, citecolor = OliveGreen, pdfauthor={Didier BONNEL - https://www.maths-cours.fr} } % supprime les bordures autour des liens

\renewcommand{\arg}[0]{\text{arg}}

\everymath{\displaystyle}

%================================================================================================================================
%
% Macros - Commandes
%
%================================================================================================================================

\newcommand\meta[2]{    			% Utilisé pour créer le post HTML.
	\def\titre{titre}
	\def\url{url}
	\def\arg{#1}
	\ifx\titre\arg
		\newcommand\maintitle{#2}
		\fancyhead[L]{#2}
		{\Large\sffamily \MakeUppercase{#2}}
		\vspace{1mm}\textcolor{mcvert}{\hrule}
	\fi 
	\ifx\url\arg
		\fancyfoot[L]{\href{https://www.maths-cours.fr#2}{\black \footnotesize{https://www.maths-cours.fr#2}}}
	\fi 
}


\newcommand\TitreC[1]{    		% Titre centré
     \needspace{3\baselineskip}
     \begin{center}\textbf{#1}\end{center}
}

\newcommand\newpar{    		% paragraphe
     \par
}

\newcommand\nosp {    		% commande vide (pas d'espace)
}
\newcommand{\id}[1]{} %ignore

\newcommand\boite[2]{				% Boite simple sans titre
	\vspace{5mm}
	\setlength{\fboxrule}{0.2mm}
	\setlength{\fboxsep}{5mm}	
	\fcolorbox{#1}{#1!3}{\makebox[\linewidth-2\fboxrule-2\fboxsep]{
  		\begin{minipage}[t]{\linewidth-2\fboxrule-4\fboxsep}\setlength{\parskip}{3mm}
  			 #2
  		\end{minipage}
	}}
	\vspace{5mm}
}

\newcommand\CBox[4]{				% Boites
	\vspace{5mm}
	\setlength{\fboxrule}{0.2mm}
	\setlength{\fboxsep}{5mm}
	
	\fcolorbox{#1}{#1!3}{\makebox[\linewidth-2\fboxrule-2\fboxsep]{
		\begin{minipage}[t]{1cm}\setlength{\parskip}{3mm}
	  		\textcolor{#1}{\LARGE{#2}}    
 	 	\end{minipage}  
  		\begin{minipage}[t]{\linewidth-2\fboxrule-4\fboxsep}\setlength{\parskip}{3mm}
			\raisebox{1.2mm}{\normalsize\sffamily{\textcolor{#1}{#3}}}						
  			 #4
  		\end{minipage}
	}}
	\vspace{5mm}
}

\newcommand\cadre[3]{				% Boites convertible html
	\par
	\vspace{2mm}
	\setlength{\fboxrule}{0.1mm}
	\setlength{\fboxsep}{5mm}
	\fcolorbox{#1}{white}{\makebox[\linewidth-2\fboxrule-2\fboxsep]{
  		\begin{minipage}[t]{\linewidth-2\fboxrule-4\fboxsep}\setlength{\parskip}{3mm}
			\raisebox{-2.5mm}{\sffamily \small{\textcolor{#1}{\MakeUppercase{#2}}}}		
			\par		
  			 #3
 	 		\end{minipage}
	}}
		\vspace{2mm}
	\par
}

\newcommand\bloc[3]{				% Boites convertible html sans bordure
     \needspace{2\baselineskip}
     {\sffamily \small{\textcolor{#1}{\MakeUppercase{#2}}}}    
		\par		
  			 #3
		\par
}

\newcommand\CHelp[1]{
     \CBox{Plum}{\faInfoCircle}{À RETENIR}{#1}
}

\newcommand\CUp[1]{
     \CBox{NavyBlue}{\faThumbsOUp}{EN PRATIQUE}{#1}
}

\newcommand\CInfo[1]{
     \CBox{Sepia}{\faArrowCircleRight}{REMARQUE}{#1}
}

\newcommand\CRedac[1]{
     \CBox{PineGreen}{\faEdit}{BIEN R\'EDIGER}{#1}
}

\newcommand\CError[1]{
     \CBox{Red}{\faExclamationTriangle}{ATTENTION}{#1}
}

\newcommand\TitreExo[2]{
\needspace{4\baselineskip}
 {\sffamily\large EXERCICE #1\ (\emph{#2 points})}
\vspace{5mm}
}

\newcommand\img[2]{
          \includegraphics[width=#2\paperwidth]{\imgdir#1}
}

\newcommand\imgsvg[2]{
       \begin{center}   \includegraphics[width=#2\paperwidth]{\imgsvgdir#1} \end{center}
}


\newcommand\Lien[2]{
     \href{#1}{#2 \tiny \faExternalLink}
}
\newcommand\mcLien[2]{
     \href{https~://www.maths-cours.fr/#1}{#2 \tiny \faExternalLink}
}

\newcommand{\euro}{\eurologo{}}

%================================================================================================================================
%
% Macros - Environement
%
%================================================================================================================================

\newenvironment{tex}{ %
}
{%
}

\newenvironment{indente}{ %
	\setlength\parindent{10mm}
}

{
	\setlength\parindent{0mm}
}

\newenvironment{corrige}{%
     \needspace{3\baselineskip}
     \medskip
     \textbf{\textsc{Corrigé}}
     \medskip
}
{
}

\newenvironment{extern}{%
     \begin{center}
     }
     {
     \end{center}
}

\NewEnviron{code}{%
	\par
     \boite{gray}{\texttt{%
     \BODY
     }}
     \par
}

\newenvironment{vbloc}{% boite sans cadre empeche saut de page
     \begin{minipage}[t]{\linewidth}
     }
     {
     \end{minipage}
}
\NewEnviron{h2}{%
    \needspace{3\baselineskip}
    \vspace{0.6cm}
	\noindent \MakeUppercase{\sffamily \large \BODY}
	\vspace{1mm}\textcolor{mcgris}{\hrule}\vspace{0.4cm}
	\par
}{}

\NewEnviron{h3}{%
    \needspace{3\baselineskip}
	\vspace{5mm}
	\textsc{\BODY}
	\par
}

\NewEnviron{margeneg}{ %
\begin{addmargin}[-1cm]{0cm}
\BODY
\end{addmargin}
}

\NewEnviron{html}{%
}

\begin{document}
\meta{url}{/exercices/qcm-bac-blanc-es-l-sujet-1-maths-cours-2017/}
\meta{pid}{10381}
\meta{titre}{QCM - Bac blanc ES/L Sujet 1 - Maths-cours 2017}
\meta{type}{exercices}
%
\begin{h2}Exercice 1 (5 points)\end{h2}
\par
\emph{Cet exercice est un QCM (questionnaire à choix multiples). Pour chacune des questions suivantes, une seule des réponses proposées est exacte. \\Indiquer sur la copie le numéro de la question et la réponse choisie en justifiant le choix effectué. }
\par
\medskip
\par
\emph{\textbf{Une réponse non justifiée ne sera pas prise en compte.}}
\par
Pour les questions \textbf{1.}, \textbf{2.} et \textbf{3.}, $f$ est une fonction définie et dérivable sur l'intervalle $[-4~;~2]$ dont la courbe représentative $\mathscr{C}_{f}$, dans un repère orthogonal, est tracée ci-après.
\par
\begin{center}
     \begin{extern}%width="600" alt=""
          \includegraphics[width=0.9\textwidth]{images/BBESL-s1-1-1}% gbb 1 unite=2cm
     \end{extern}
\end{center}
\par
La courbe $\mathscr{C}_{f}$ passe par l'origine $O$ du repère et par les points $A(-3~;~9)$ et $B\left(1~;~-\dfrac{5}{3}\right)$.
\par
Les tangentes à la courbe $\mathscr{C}_{f}$ aux points $A$ et $B$ sont parallèles à l'axe des abscisses.
\par
La tangente à la courbe $\mathscr{C}_{f}$ au point $O$ passe par le point $A$.
\par
\medskip
\par
On note $f^{\prime}$  la fonction dérivée de la fonction $f$.
\par
\begin{itemize}
     \needspace{4\baselineskip}
     \item \textbf{Question 1 :}
     \par
     La valeur exacte de $f(0)$ est :
     \par
     \textbf{a.~~} $0$ \\
     \textbf{b.~~} $1$ \\
     \textbf{c.~~} $-\dfrac{5}{3}$ \\
     \textbf{d.~~} autre réponse \\
     \par
     \needspace{4\baselineskip}
     \item \textbf{Question 2 :}
     \par
     La valeur exacte de $f'(0)$ est :
     \par
     \textbf{a.~~} $0$ \\
     \textbf{b.~~} $-3$ \\
     \textbf{c.~~} $3$ \\
     \textbf{d.~~} autre réponse \\
     \par
     \needspace{4\baselineskip}
     \item \textbf{Question 3 :}
     \par
     L'ensemble $S$ des solutions de l'équation $f'(x)=0$ est :
     \par
     \textbf{a.~~} $S=\emptyset$ \\
     \textbf{b.~~} $S=\left\{0\right\}$  \\
     \textbf{c.~~} $S=\left\{-3~;~1\right\}$ \\
     \textbf{d.~~} $S=\left\{-\dfrac{5}{3}~;~9\right\}$ \\
     \par
     \needspace{4\baselineskip}
     \item \textbf{Question 4 :}
     \par
     Lors des soldes d'hiver, le prix d'un article est passé de 150~euros à 120~euros.
     \par
     Quel est le taux de la remise accordée par le vendeur ?
     \par
     \textbf{a.~~} 15\% \\
     \textbf{b.~~} 20\%  \\
     \textbf{c.~~} 25\% \\
     \textbf{d.~~} 30\% \\
     \par
     \needspace{4\baselineskip}
     \item \textbf{Question 5 :}
     \par
     De 2005 à 2010, la population d'une ville a augmenté de 5\% puis, de 2010 à 2015, a diminué de 3\%.
     \par
     Le taux d'évolution global de cette population entre 2005 et 2015 est :
     \par
     \textbf{a.~~} 2\% \\
     \textbf{b.~~} 8,15\%  \\
     \textbf{c.~~} 1,85\% \\
     \textbf{d.~~} 0,2\% \\
     \par
\end{itemize}
\begin{corrige}
     \begin{itemize}
          \item \textbf{Question 1 :}
          \par
          Réponse correcte :\quad\textbf{ a.}
          \par
          La courbe $\mathscr{C}_f$ passe par l'origine donc $f(0)=0$.
          \par
          \cadre{rouge}{À retenir}{
               \par
               Soient $\mathscr{C}_f$ la courbe représentative d'une fonction $f$ et $M$ un point de coordonnées $\left(x_M~;~y_M\right)$.
               \par
               Le point $M$ appartient à la courbe $\mathscr{C}_f$ si et seulement si ${f\left(x_M\right)=y_M}$.
               \par
               \textit{Ici, le point $O(0~;~0)$ appartient à la courbe $\mathscr{C}_f$ donc ${f(0)=0}$.}
          }
          \par
          \item \textbf{Question 2 :}
          \par
          Réponse correcte :\quad\textbf{ b.}
          \par
          $f'(0)$ est le coefficient directeur de la tangente en $O$ à la courbe $\mathscr{C}_f$.
          \par
          Cette tangente est la droite $(OA)$ donc :
          \[ f'(0)=\dfrac{y_A-y_O}{x_A-x_O}=\dfrac{9}{-3}=-3.\]
          \par
          \cadre{rouge}{À retenir}{
               $f'(a)$ est le \textbf{coefficient directeur de la tangente} au point de la courbe $\mathscr{C}_f$ d'\textbf{abscisse} $a$.
          }
          \par
          \vspace{-8mm}
          \par
          \cadre{rouge}{À retenir}{
               Soient $A$ et $B$ deux points de coordonnées respectives $\left(x_A~;~y_A\right)$ et $\left(x_B~;~y_B\right)$.
               \par
               Le \textbf{coefficient directeur} de la droite $(AB)$ est :
               \[ a = \dfrac{y_B-y_A}{x_B-x_A}.\]
               \par
          }
          \par
          \item \textbf{Question 3 :}
          \par
          Réponse correcte :\quad\textbf{ c.}
          \par
          $f'(x)=0$ si et seulement si la tangente à la courbe $\mathscr{C}_f$ au point d'abscisse $x$ est parallèle à l'axe des abscisses.
          \par
          D'après l'énoncé, ceci se produit aux points $A$ et $B$ d'abscisses respectives $-3$ et $1$.
          \par
          \cadre{rouge}{À retenir}{
               \par
               $f'(a)=0$ si et seulement si la tangente à la courbe représentative de $f$ au point d'abscisse $a$ est \textbf{parallèle à l'axe des abscisses}.
          }
          \par
          \item \textbf{Question 4 :}
          \par
          Réponse correcte :\quad\textbf{ b.}
          \par
          Le taux d'évolution faisant passer de 150 à 120 est :
          \par
          $t=\dfrac{120-150}{150}=-\dfrac{30}{150}=-0,2=-\dfrac{20}{100}=-20\%$.
          \par
          Le taux de la remise effectuée par le vendeur est 20\%.
          \par
          \cadre{rouge}{À retenir}{
               \par
               Lorsqu'une valeur passe de $V_0$ à $V_1$, le taux d'évolution est :
               \[ t = \dfrac{V_1-V_0}{V_0}. \]
          }
          \par
          \item \textbf{Question 5 :}
          \par
          Réponse correcte :\quad\textbf{ c.}
          \par
          Une augmentation de $5\%$ correspond à un coefficient multiplicateur :
          \[ CM_1=1+\dfrac{5}{100} = 1,05.\]
          \par
          Une diminution de $3\%$ correspond à un coefficient multiplicateur :
          \[ CM_2=1-\dfrac{3}{100} =0,97. \]
          \par
          Le coefficient multiplicateur global $CM_g$ est :
          \par
          $ CM_g=CM_1\times CM_2 = 1,05\times 0,97 = 1,0185 = 1+\dfrac{1,85}{100}. $
          \par
          Le taux d'évolution global de la population entre 2005 et 2015 est $1,85\%$.
          \par
          \cadre{rouge}{À retenir}{
               \par
               Lorsqu'une valeur subit plusieurs variations successives, \textbf{le coefficient multiplicateur global est égal au produit des coefficients multiplicateurs} associés à chaque évolution.
          }
          \par
     \end{itemize}
     \par
\end{corrige}

\end{document}
µ
\documentclass[a4paper]{article}

%================================================================================================================================
%
% Packages
%
%================================================================================================================================

\usepackage[T1]{fontenc} 	% pour caractères accentués
\usepackage[utf8]{inputenc}  % encodage utf8
\usepackage[french]{babel}	% langue : français
\usepackage{fourier}			% caractères plus lisibles
\usepackage[dvipsnames]{xcolor} % couleurs
\usepackage{fancyhdr}		% réglage header footer
\usepackage{needspace}		% empêcher sauts de page mal placés
\usepackage{graphicx}		% pour inclure des graphiques
\usepackage{enumitem,cprotect}		% personnalise les listes d'items (nécessaire pour ol, al ...)
\usepackage{hyperref}		% Liens hypertexte
\usepackage{pstricks,pst-all,pst-node,pstricks-add,pst-math,pst-plot,pst-tree,pst-eucl} % pstricks
\usepackage[a4paper,includeheadfoot,top=2cm,left=3cm, bottom=2cm,right=3cm]{geometry} % marges etc.
\usepackage{comment}			% commentaires multilignes
\usepackage{amsmath,environ} % maths (matrices, etc.)
\usepackage{amssymb,makeidx}
\usepackage{bm}				% bold maths
\usepackage{tabularx}		% tableaux
\usepackage{colortbl}		% tableaux en couleur
\usepackage{fontawesome}		% Fontawesome
\usepackage{environ}			% environment with command
\usepackage{fp}				% calculs pour ps-tricks
\usepackage{multido}			% pour ps tricks
\usepackage[np]{numprint}	% formattage nombre
\usepackage{tikz,tkz-tab} 			% package principal TikZ
\usepackage{pgfplots}   % axes
\usepackage{mathrsfs}    % cursives
\usepackage{calc}			% calcul taille boites
\usepackage[scaled=0.875]{helvet} % font sans serif
\usepackage{svg} % svg
\usepackage{scrextend} % local margin
\usepackage{scratch} %scratch
\usepackage{multicol} % colonnes
%\usepackage{infix-RPN,pst-func} % formule en notation polanaise inversée
\usepackage{listings}

%================================================================================================================================
%
% Réglages de base
%
%================================================================================================================================

\lstset{
language=Python,   % R code
literate=
{á}{{\'a}}1
{à}{{\`a}}1
{ã}{{\~a}}1
{é}{{\'e}}1
{è}{{\`e}}1
{ê}{{\^e}}1
{í}{{\'i}}1
{ó}{{\'o}}1
{õ}{{\~o}}1
{ú}{{\'u}}1
{ü}{{\"u}}1
{ç}{{\c{c}}}1
{~}{{ }}1
}


\definecolor{codegreen}{rgb}{0,0.6,0}
\definecolor{codegray}{rgb}{0.5,0.5,0.5}
\definecolor{codepurple}{rgb}{0.58,0,0.82}
\definecolor{backcolour}{rgb}{0.95,0.95,0.92}

\lstdefinestyle{mystyle}{
    backgroundcolor=\color{backcolour},   
    commentstyle=\color{codegreen},
    keywordstyle=\color{magenta},
    numberstyle=\tiny\color{codegray},
    stringstyle=\color{codepurple},
    basicstyle=\ttfamily\footnotesize,
    breakatwhitespace=false,         
    breaklines=true,                 
    captionpos=b,                    
    keepspaces=true,                 
    numbers=left,                    
xleftmargin=2em,
framexleftmargin=2em,            
    showspaces=false,                
    showstringspaces=false,
    showtabs=false,                  
    tabsize=2,
    upquote=true
}

\lstset{style=mystyle}


\lstset{style=mystyle}
\newcommand{\imgdir}{C:/laragon/www/newmc/assets/imgsvg/}
\newcommand{\imgsvgdir}{C:/laragon/www/newmc/assets/imgsvg/}

\definecolor{mcgris}{RGB}{220, 220, 220}% ancien~; pour compatibilité
\definecolor{mcbleu}{RGB}{52, 152, 219}
\definecolor{mcvert}{RGB}{125, 194, 70}
\definecolor{mcmauve}{RGB}{154, 0, 215}
\definecolor{mcorange}{RGB}{255, 96, 0}
\definecolor{mcturquoise}{RGB}{0, 153, 153}
\definecolor{mcrouge}{RGB}{255, 0, 0}
\definecolor{mclightvert}{RGB}{205, 234, 190}

\definecolor{gris}{RGB}{220, 220, 220}
\definecolor{bleu}{RGB}{52, 152, 219}
\definecolor{vert}{RGB}{125, 194, 70}
\definecolor{mauve}{RGB}{154, 0, 215}
\definecolor{orange}{RGB}{255, 96, 0}
\definecolor{turquoise}{RGB}{0, 153, 153}
\definecolor{rouge}{RGB}{255, 0, 0}
\definecolor{lightvert}{RGB}{205, 234, 190}
\setitemize[0]{label=\color{lightvert}  $\bullet$}

\pagestyle{fancy}
\renewcommand{\headrulewidth}{0.2pt}
\fancyhead[L]{maths-cours.fr}
\fancyhead[R]{\thepage}
\renewcommand{\footrulewidth}{0.2pt}
\fancyfoot[C]{}

\newcolumntype{C}{>{\centering\arraybackslash}X}
\newcolumntype{s}{>{\hsize=.35\hsize\arraybackslash}X}

\setlength{\parindent}{0pt}		 
\setlength{\parskip}{3mm}
\setlength{\headheight}{1cm}

\def\ebook{ebook}
\def\book{book}
\def\web{web}
\def\type{web}

\newcommand{\vect}[1]{\overrightarrow{\,\mathstrut#1\,}}

\def\Oij{$\left(\text{O}~;~\vect{\imath},~\vect{\jmath}\right)$}
\def\Oijk{$\left(\text{O}~;~\vect{\imath},~\vect{\jmath},~\vect{k}\right)$}
\def\Ouv{$\left(\text{O}~;~\vect{u},~\vect{v}\right)$}

\hypersetup{breaklinks=true, colorlinks = true, linkcolor = OliveGreen, urlcolor = OliveGreen, citecolor = OliveGreen, pdfauthor={Didier BONNEL - https://www.maths-cours.fr} } % supprime les bordures autour des liens

\renewcommand{\arg}[0]{\text{arg}}

\everymath{\displaystyle}

%================================================================================================================================
%
% Macros - Commandes
%
%================================================================================================================================

\newcommand\meta[2]{    			% Utilisé pour créer le post HTML.
	\def\titre{titre}
	\def\url{url}
	\def\arg{#1}
	\ifx\titre\arg
		\newcommand\maintitle{#2}
		\fancyhead[L]{#2}
		{\Large\sffamily \MakeUppercase{#2}}
		\vspace{1mm}\textcolor{mcvert}{\hrule}
	\fi 
	\ifx\url\arg
		\fancyfoot[L]{\href{https://www.maths-cours.fr#2}{\black \footnotesize{https://www.maths-cours.fr#2}}}
	\fi 
}


\newcommand\TitreC[1]{    		% Titre centré
     \needspace{3\baselineskip}
     \begin{center}\textbf{#1}\end{center}
}

\newcommand\newpar{    		% paragraphe
     \par
}

\newcommand\nosp {    		% commande vide (pas d'espace)
}
\newcommand{\id}[1]{} %ignore

\newcommand\boite[2]{				% Boite simple sans titre
	\vspace{5mm}
	\setlength{\fboxrule}{0.2mm}
	\setlength{\fboxsep}{5mm}	
	\fcolorbox{#1}{#1!3}{\makebox[\linewidth-2\fboxrule-2\fboxsep]{
  		\begin{minipage}[t]{\linewidth-2\fboxrule-4\fboxsep}\setlength{\parskip}{3mm}
  			 #2
  		\end{minipage}
	}}
	\vspace{5mm}
}

\newcommand\CBox[4]{				% Boites
	\vspace{5mm}
	\setlength{\fboxrule}{0.2mm}
	\setlength{\fboxsep}{5mm}
	
	\fcolorbox{#1}{#1!3}{\makebox[\linewidth-2\fboxrule-2\fboxsep]{
		\begin{minipage}[t]{1cm}\setlength{\parskip}{3mm}
	  		\textcolor{#1}{\LARGE{#2}}    
 	 	\end{minipage}  
  		\begin{minipage}[t]{\linewidth-2\fboxrule-4\fboxsep}\setlength{\parskip}{3mm}
			\raisebox{1.2mm}{\normalsize\sffamily{\textcolor{#1}{#3}}}						
  			 #4
  		\end{minipage}
	}}
	\vspace{5mm}
}

\newcommand\cadre[3]{				% Boites convertible html
	\par
	\vspace{2mm}
	\setlength{\fboxrule}{0.1mm}
	\setlength{\fboxsep}{5mm}
	\fcolorbox{#1}{white}{\makebox[\linewidth-2\fboxrule-2\fboxsep]{
  		\begin{minipage}[t]{\linewidth-2\fboxrule-4\fboxsep}\setlength{\parskip}{3mm}
			\raisebox{-2.5mm}{\sffamily \small{\textcolor{#1}{\MakeUppercase{#2}}}}		
			\par		
  			 #3
 	 		\end{minipage}
	}}
		\vspace{2mm}
	\par
}

\newcommand\bloc[3]{				% Boites convertible html sans bordure
     \needspace{2\baselineskip}
     {\sffamily \small{\textcolor{#1}{\MakeUppercase{#2}}}}    
		\par		
  			 #3
		\par
}

\newcommand\CHelp[1]{
     \CBox{Plum}{\faInfoCircle}{À RETENIR}{#1}
}

\newcommand\CUp[1]{
     \CBox{NavyBlue}{\faThumbsOUp}{EN PRATIQUE}{#1}
}

\newcommand\CInfo[1]{
     \CBox{Sepia}{\faArrowCircleRight}{REMARQUE}{#1}
}

\newcommand\CRedac[1]{
     \CBox{PineGreen}{\faEdit}{BIEN R\'EDIGER}{#1}
}

\newcommand\CError[1]{
     \CBox{Red}{\faExclamationTriangle}{ATTENTION}{#1}
}

\newcommand\TitreExo[2]{
\needspace{4\baselineskip}
 {\sffamily\large EXERCICE #1\ (\emph{#2 points})}
\vspace{5mm}
}

\newcommand\img[2]{
          \includegraphics[width=#2\paperwidth]{\imgdir#1}
}

\newcommand\imgsvg[2]{
       \begin{center}   \includegraphics[width=#2\paperwidth]{\imgsvgdir#1} \end{center}
}


\newcommand\Lien[2]{
     \href{#1}{#2 \tiny \faExternalLink}
}
\newcommand\mcLien[2]{
     \href{https~://www.maths-cours.fr/#1}{#2 \tiny \faExternalLink}
}

\newcommand{\euro}{\eurologo{}}

%================================================================================================================================
%
% Macros - Environement
%
%================================================================================================================================

\newenvironment{tex}{ %
}
{%
}

\newenvironment{indente}{ %
	\setlength\parindent{10mm}
}

{
	\setlength\parindent{0mm}
}

\newenvironment{corrige}{%
     \needspace{3\baselineskip}
     \medskip
     \textbf{\textsc{Corrigé}}
     \medskip
}
{
}

\newenvironment{extern}{%
     \begin{center}
     }
     {
     \end{center}
}

\NewEnviron{code}{%
	\par
     \boite{gray}{\texttt{%
     \BODY
     }}
     \par
}

\newenvironment{vbloc}{% boite sans cadre empeche saut de page
     \begin{minipage}[t]{\linewidth}
     }
     {
     \end{minipage}
}
\NewEnviron{h2}{%
    \needspace{3\baselineskip}
    \vspace{0.6cm}
	\noindent \MakeUppercase{\sffamily \large \BODY}
	\vspace{1mm}\textcolor{mcgris}{\hrule}\vspace{0.4cm}
	\par
}{}

\NewEnviron{h3}{%
    \needspace{3\baselineskip}
	\vspace{5mm}
	\textsc{\BODY}
	\par
}

\NewEnviron{margeneg}{ %
\begin{addmargin}[-1cm]{0cm}
\BODY
\end{addmargin}
}

\NewEnviron{html}{%
}

\begin{document}
\meta{url}{/exercices/fonctions-bac-blanc-es-l-sujet-1-maths-cours-2017/}
\meta{pid}{10394}
\meta{titre}{Fonctions - Bac blanc ES/L Sujet 1 - Maths-cours 2017}
\meta{type}{exercices}
%
\begin{h2}Exercice 2 (5 points)\end{h2}
\par
\medskip
\par
Une entreprise produit et commercialise des granulés de céréales destinés à l'alimentation des volailles.
\par
Elle produit, chaque jour, entre 0 et 5 tonnes de granulés.
\par
On note $x$ le nombre de tonnes de granulés produits quotidiennement par cette entreprise.
\par
Le coût de fabrication quotidien, en centaines d'euros peut être modélisé par une fonction $C$, définie sur l'intervalle $[0~;~5]$  dont la représentation graphique $\mathscr{C}$ est fournie en \textbf{Annexe} (voir à la fin du sujet).
\par
L'entreprise vend la totalité des granulés produits au prix de 5 400 euros la tonne.
\par
%============================================================================================================================
%
\TitreC{Partie A}
%
%============================================================================================================================
\par
\begin{enumerate}
     \item Expliquer pourquoi la recette quotidienne de l'entreprise peut être modélisée par la fonction $R$ définie sur l'intervalle $[0~;~5]$ par~:
     \[ R(x)= 54x \]
     Tracer la représentation $\mathscr{R}$ de la fonction $R$ sur le graphique fourni en annexe 1 (à rendre avec la copie).
     \par
     \medskip
     \par
     \item Estimer, à l'aide du graphique, le coût de fabrication quotidien et la recette quotidienne pour une production de 2 tonnes puis de 5 tonnes.
     \par
     Indiquer, dans chacun de ces deux cas, si l'entreprise est bénéficiaire.
     \par
     \medskip
     \par
     \item Par lecture graphique, estimer l'intervalle auquel doit appartenir $x$ pour que l'entreprise réalise un bénéfice.
\end{enumerate}
\par
%============================================================================================================================
%
\TitreC{Partie B}
%
%============================================================================================================================
\par
Dans cette partie, on suppose que le coût de fabrication de $x$ tonnes de granulés, en centaines d'euros, peut être modélisé sur l'intervalle $[0~;~5]$ par~:
\[ C(x)=8x^3-60x^2+150x \]
\begin{enumerate}
     \item Calculer $C'(x)$.
     \par
     En déduire que la fonction $C$ est croissante sur $[0~;~5]$.
     \par
     \medskip
     \par
     \item On note $B(x)$ le résultat net quotidien (en centaines d'euros) de l'entreprise, c'est à dire la différence entre la recette et le coût de fabrication quotidien.
     \par
     Exprimer $B(x)$ puis $B'(x)$ en fonction de $x$.
     \par
     \medskip
     \par
     \item Tracer le tableau de variations de la fonction $B$ sur l'intervalle $[0~;~5]$.
     \par
     \medskip
     \par
     \item Pour quelle production le résultat net de l'entreprise est-t-il maximal ?
     \par
     Quel est alors ce maximum ?
     \par
     \medskip
     \par
\end{enumerate}
%
%============================================================================================================================
%
%			Annexe
%
%============================================================================================================================
%
\TitreC{ANNEXE}
%
\medskip
%
\begin{center}
     \emph{\`A rendre avec la copie}
\end{center}
%
\bigskip
%
\begin{center}
     \begin{extern}%width="600" alt="Courbe résultats nets-bénéfices"
          \includegraphics[width=10cm]{images/BBESL-s1-2-1.eps}% gbb 1 unite=3cm
     \end{extern}
\end{center}
%
\begin{corrige}
     %============================================================================================================================
     %
     \TitreC{Partie A}
     %
     %============================================================================================================================
     \par
     \begin{enumerate}
          \item Pour une tonne de granulés, l'entreprise encaisse $5\ 400$~euros soit {$54$ centaines} d'euros.
          \par
          Comme l'entreprise produit et vend quotidiennement $x$ tonnes de granulés (avec $0 \leqslant x \leqslant 5$), la recette quotidienne de l'entreprise, en centaines d'euros, peut être modélisée par la fonction $R$ définie sur l'intervalle $[0~;~5]$ par~:
          \[ R(x)= 54x.\]
          \par
          Cette fonction est la restriction à l'intervalle $[0~;~5]$ d'une \textbf{fonction linéaire}.
          \par
          \cadre{rouge}{Bien rédiger}{
               Lorsque vous rencontrez une fonction étudiée en cours (fonction linéaire, affine, carré, inverse, polynôme du second degré, ...), indiquez-le clairement sur votre copie.
               \par
               Cela vous permettra notamment de justifier l'allure de la courbe représentative (droite, hyperbole, parabole, ...).
          }
          \par
          Sa représentation graphique $\mathscr{R}$ est un segment de \textbf{droite passant par l'origine}.
          \par
          Pour $x=5$, la recette est $R(5)=270$ ; donc la droite passe par le point de coordonnées $(5~;~270)$ (voir graphique).
          \begin{center}
               \begin{extern}%width="600" alt="Lecture graphique coût-recette"
                    \includegraphics[width=0.9\textwidth]{images/BBESL-s1-2-1c.eps}% gbb 1 unite=3cm
               \end{extern}
          \end{center}
          \cadre{vert}{En pratique}{
               Pour tracer une droite, il suffit de connaître deux points de la droite.
               \par
               Pour obtenir un tracé précis, il est préférable de choisir deux points espacés.
          }
          \par
          \item \`A l'aide du graphique , on voit que~:
          \begin{itemize}
               \item pour une production de 2 tonnes (construction en \textit{rouge})~:
               \begin{itemize}[label=---]
                    \item le coût de fabrication quotidien avoisine \textbf{124~euros} ;
                    \item la recette quotidienne avoisine \textbf{108~euros}.
               \end{itemize}
               \item pour une production de 5 tonnes (construction en \textit{violet})~:
               \begin{itemize}[label=---]
                    \item le coût de fabrication quotidien avoisine \textbf{250~euros} ;
                    \item la recette quotidienne avoisine \textbf{270~euros}.
               \end{itemize}
               \par
          \end{itemize}
          \par
          Dans le premier cas, la recette est inférieure au coût donc l'entreprise est \textbf{déficitaire}.
          \par
          Dans le second cas, la recette est supérieure au coût donc l'entreprise est \textbf{bénéficiaire}.
          \par
          \cadre{rouge}{Bien rédiger}{
               Il n'est pas toujours facile d'expliquer clairement une lecture graphique.
               \par
               Si, comme ici, vous rendez le graphique avec votre copie, \textbf{laissez apparents vos traits de construction} afin que le correcteur puisse comprendre votre démarche.
          }
          \par
          \item L'entreprise réalise un bénéfice lorsque la recette est supérieure au coût de production, c'est à dire lorsque le segment $\mathscr{R}$ est situé au-dessus de la courbe $\mathscr{C}$.
          D'après le graphique, on peut estimer que ceci se produit lorsque $x$ appartient à l'intervalle $[2,3~;~5]$ (tracé en \textit{brun}), c'est à dire pour \textbf{une production supérieure à $\bm{2,3}$ tonnes}.
          \par
     \end{enumerate}
     \par
     %============================================================================================================================
     %
     \TitreC{Partie B}
     %
     %============================================================================================================================
     \par
     \begin{enumerate}
          \item La fonction $C$ est une fonction polynôme sur $[0~;~5]$ donc $C$ est dérivable sur $[0~;~5]$ et~:
          \par
          $C'(x)= 8 \times 3x^2 - 60 \times 2x + 150$\\
          $\phantom{C'(x)}= 24x^2 - 120x + 150.$
          \par
          On peut mettre $6$ en facteur~:
          \par
          $C'(x)= 6(4x^2-20x+25)$.
          \par
          On applique l'identité remarquable $(a-b)^2=a^2-2ab+b^2$~:
          \par
          $C'(x)= 6(2x-5)^2$.
          \par
          Un carré est toujours positif ou nul donc $C'(x) \geqslant 0$ sur $[0~;~5]$.
          \par
          Par conséquent, la fonction $C$ est croissante sur l'intervalle $[0~;~5]$.
          \par
          \cadre{bleu}{Remarque}{
               Il serait tout à fait correct de calculer le discriminant $\Delta$ de la fonction polynôme $C^{\prime}$ pour déterminer son signe.
               \par
               Toutefois, ici, il est préférable et plus rapide de factoriser $C'(x)$ grâce à une identité remarquable.
          }
          \par
          \item
          Le résultat net est la différence entre la recette et le coût de fabrication, donc~:
          \par
          $B(x)=R(x)-C(x)$\\
          $\phantom{B(x)}=54x-(8x^3-60x^2+150x)$\\
          $\phantom{B(x)}=-8x^3+60x^2-96x$
          \par
          $B$ est une fonction polynôme du troisième degré sur $[0~;~5]$ ; elle est donc dérivable et~:
          \par
          $B'(x)=-8\times 3x^2+60\times 2x-96$\\
          $\phantom{B'(x)}=-24x^2+120x-96$.
          \par
          \item \'Etude du signe du trinôme $-24x^2+120x-96$~:
          \par
          $\Delta=120^2-4 \times (-24)\times (-96)=5184=72^2$.
          \par
          Le discriminant est strictement positif donc le trinôme possède deux racines distinctes~:
          \par
          \[ x_1=\dfrac{-120+72}{-2\times (-24)}=1 \quad \text{et} \quad\ x_2=\dfrac{-120-72}{-2\times (-24)}=4. \]
          \par
          Le coefficient de $x^2$ est -24 ; il est négatif, donc $B'(x)$ est négatif à l'extérieur des racines, c'est à dire sur l'ensemble $[0~;~1] \cup [4~;~5]$.
          \par
          De plus~: $B(0)=0 \ ;\ B(1)=-44\ ;\ B(4)=64\ ;\ B(5)=20$.
          \par
          \cadre{vert}{En pratique}{
               Pour obtenir rapidement différentes valeurs de $B(x)$ le plus simple est de saisir la fonction $B$ dans la calculatrice et de faire afficher un tableau de valeurs.
               \par
               Par ailleurs, il est aussi conseillé de tracer le graphique sur la calculatrice pour vérifier le tableau de variations.
          }
          \par
          On en déduit le tableau de signes de $B^{\prime}$ et le tableau de variations de $B$~:
          %:-+-+-+-+- Engendré par~: http://math.et.info.free.fr/TikZ/TableauxVariations/
          \begin{center}
               \begin{extern}%width="450" alt="tableau de variations de la fonction bénéfice"
                    \begin{tikzpicture}[scale=0.875]
                         % Styles
                         \tikzstyle{cadre}=[thin]
                         \tikzstyle{fleche}=[->,>=latex,thin]
                         \tikzstyle{nondefini}=[lightgray]
                         % Dimensions Modifiables
                         \def\Lrg{1.5}
                         \def\HtX{1}
                         \def\HtY{0.5}
                         % Dimensions Calculées
                         \def\lignex{-0.5*\HtX}
                         \def\lignef{-1.5*\HtX}
                         \def\separateur{-0.5*\Lrg}
                         % Largeur du tableau
                         \def\gauche{-1.5*\Lrg}
                         \def\droite{6.5*\Lrg}
                         % Hauteur du tableau
                         \def\haut{0.5*\HtX}
                         \def\bas{-2.5*\HtX-2*\HtY}
                         % Pointillés
                         \draw[dotted,black] (0*\Lrg,\lignex-0.1*\HtX) -- (0*\Lrg,\lignef+0.1*\HtX);
                         \draw[dotted,black] (0*\Lrg,\lignef-0.1*\HtX) -- (0*\Lrg,\bas+0.1*\HtX);
                         \draw[dotted,black] (2*\Lrg,\lignex-0.1*\HtX) -- (2*\Lrg,\lignef+0.1*\HtX);
                         \draw[dotted,black] (2*\Lrg,\lignef-0.1*\HtX) -- (2*\Lrg,\bas+0.1*\HtX);
                         \draw[dotted,black] (4*\Lrg,\lignex-0.1*\HtX) -- (4*\Lrg,\lignef+0.1*\HtX);
                         \draw[dotted,black] (4*\Lrg,\lignef-0.1*\HtX) -- (4*\Lrg,\bas+0.1*\HtX);
                         \draw[dotted,black] (6*\Lrg,\lignex-0.1*\HtX) -- (6*\Lrg,\lignef+0.1*\HtX);
                         \draw[dotted,black] (6*\Lrg,\lignef-0.1*\HtX) -- (6*\Lrg,\bas+0.1*\HtX);
                         % Ligne de l'abscisse~: x
                         \node at (-1*\Lrg,0) {$x$};
                         \node at (0*\Lrg,0) {$0$};
                         \node at (2*\Lrg,0) {$1$};
                         \node at (4*\Lrg,0) {$4$};
                         \node at (6*\Lrg,0) {$5$};
                         % Ligne de la dérivée~: f'(x)
                         \node at (-1*\Lrg,-1*\HtX) {$B'(x)$};
                         \node at (0*\Lrg,-1*\HtX) {$$};
                         \node at (1*\Lrg,-1*\HtX) {$-$};
                         \node at (2*\Lrg,-1*\HtX) {$0$};
                         \node at (3*\Lrg,-1*\HtX) {$+$};
                         \node at (4*\Lrg,-1*\HtX) {$0$};
                         \node at (5*\Lrg,-1*\HtX) {$-$};
                         \node at (6*\Lrg,-1*\HtX) {$$};
                         % Ligne de la fonction~: f(x)
                         \node  at (-1*\Lrg,{-2*\HtX+(-1)*\HtY}) {$B(x)$};
                         \node (f1) at (0*\Lrg,{-2*\HtX+(0)*\HtY}) {$0$};
                         \node (f2) at (2*\Lrg,{-2*\HtX+(-2)*\HtY}) {$-44$};
                         \node (f3) at (4*\Lrg,{-2*\HtX+(0)*\HtY}) {$64$};
                         \node (f4) at (6*\Lrg,{-2*\HtX+(-2)*\HtY}) {$20$};
                         % Flèches
                         \draw[fleche] (f1) -- (f2);
                         \draw[fleche] (f2) -- (f3);
                         \draw[fleche] (f3) -- (f4);
                         % Encadrement
                         \draw[cadre] (\separateur,\haut) -- (\separateur,\bas);
                         \draw[cadre] (\gauche,\haut) rectangle  (\droite,\bas);
                         \draw[cadre] (\gauche,\lignex) -- (\droite,\lignex);
                         \draw[cadre] (\gauche,\lignef) -- (\droite,\lignef);
                    \end{tikzpicture}
               \end{extern}
          \end{center}
          \item Le tableau de variations précédent montre que $B$ atteint un maximum de 64 pour $x=4$.
          \par
          Le résultat net de l'entreprise est donc maximal pour une production de 4 tonnes de granulés.
          \par
          Ce maximum vaut alors 64 centaines d'euros soit $6\ 400$~euros.
          \par
     \end{enumerate}
\end{corrige}

\end{document}
µ
\documentclass[a4paper]{article}

%================================================================================================================================
%
% Packages
%
%================================================================================================================================

\usepackage[T1]{fontenc} 	% pour caractères accentués
\usepackage[utf8]{inputenc}  % encodage utf8
\usepackage[french]{babel}	% langue : français
\usepackage{fourier}			% caractères plus lisibles
\usepackage[dvipsnames]{xcolor} % couleurs
\usepackage{fancyhdr}		% réglage header footer
\usepackage{needspace}		% empêcher sauts de page mal placés
\usepackage{graphicx}		% pour inclure des graphiques
\usepackage{enumitem,cprotect}		% personnalise les listes d'items (nécessaire pour ol, al ...)
\usepackage{hyperref}		% Liens hypertexte
\usepackage{pstricks,pst-all,pst-node,pstricks-add,pst-math,pst-plot,pst-tree,pst-eucl} % pstricks
\usepackage[a4paper,includeheadfoot,top=2cm,left=3cm, bottom=2cm,right=3cm]{geometry} % marges etc.
\usepackage{comment}			% commentaires multilignes
\usepackage{amsmath,environ} % maths (matrices, etc.)
\usepackage{amssymb,makeidx}
\usepackage{bm}				% bold maths
\usepackage{tabularx}		% tableaux
\usepackage{colortbl}		% tableaux en couleur
\usepackage{fontawesome}		% Fontawesome
\usepackage{environ}			% environment with command
\usepackage{fp}				% calculs pour ps-tricks
\usepackage{multido}			% pour ps tricks
\usepackage[np]{numprint}	% formattage nombre
\usepackage{tikz,tkz-tab} 			% package principal TikZ
\usepackage{pgfplots}   % axes
\usepackage{mathrsfs}    % cursives
\usepackage{calc}			% calcul taille boites
\usepackage[scaled=0.875]{helvet} % font sans serif
\usepackage{svg} % svg
\usepackage{scrextend} % local margin
\usepackage{scratch} %scratch
\usepackage{multicol} % colonnes
%\usepackage{infix-RPN,pst-func} % formule en notation polanaise inversée
\usepackage{listings}

%================================================================================================================================
%
% Réglages de base
%
%================================================================================================================================

\lstset{
language=Python,   % R code
literate=
{á}{{\'a}}1
{à}{{\`a}}1
{ã}{{\~a}}1
{é}{{\'e}}1
{è}{{\`e}}1
{ê}{{\^e}}1
{í}{{\'i}}1
{ó}{{\'o}}1
{õ}{{\~o}}1
{ú}{{\'u}}1
{ü}{{\"u}}1
{ç}{{\c{c}}}1
{~}{{ }}1
}


\definecolor{codegreen}{rgb}{0,0.6,0}
\definecolor{codegray}{rgb}{0.5,0.5,0.5}
\definecolor{codepurple}{rgb}{0.58,0,0.82}
\definecolor{backcolour}{rgb}{0.95,0.95,0.92}

\lstdefinestyle{mystyle}{
    backgroundcolor=\color{backcolour},   
    commentstyle=\color{codegreen},
    keywordstyle=\color{magenta},
    numberstyle=\tiny\color{codegray},
    stringstyle=\color{codepurple},
    basicstyle=\ttfamily\footnotesize,
    breakatwhitespace=false,         
    breaklines=true,                 
    captionpos=b,                    
    keepspaces=true,                 
    numbers=left,                    
xleftmargin=2em,
framexleftmargin=2em,            
    showspaces=false,                
    showstringspaces=false,
    showtabs=false,                  
    tabsize=2,
    upquote=true
}

\lstset{style=mystyle}


\lstset{style=mystyle}
\newcommand{\imgdir}{C:/laragon/www/newmc/assets/imgsvg/}
\newcommand{\imgsvgdir}{C:/laragon/www/newmc/assets/imgsvg/}

\definecolor{mcgris}{RGB}{220, 220, 220}% ancien~; pour compatibilité
\definecolor{mcbleu}{RGB}{52, 152, 219}
\definecolor{mcvert}{RGB}{125, 194, 70}
\definecolor{mcmauve}{RGB}{154, 0, 215}
\definecolor{mcorange}{RGB}{255, 96, 0}
\definecolor{mcturquoise}{RGB}{0, 153, 153}
\definecolor{mcrouge}{RGB}{255, 0, 0}
\definecolor{mclightvert}{RGB}{205, 234, 190}

\definecolor{gris}{RGB}{220, 220, 220}
\definecolor{bleu}{RGB}{52, 152, 219}
\definecolor{vert}{RGB}{125, 194, 70}
\definecolor{mauve}{RGB}{154, 0, 215}
\definecolor{orange}{RGB}{255, 96, 0}
\definecolor{turquoise}{RGB}{0, 153, 153}
\definecolor{rouge}{RGB}{255, 0, 0}
\definecolor{lightvert}{RGB}{205, 234, 190}
\setitemize[0]{label=\color{lightvert}  $\bullet$}

\pagestyle{fancy}
\renewcommand{\headrulewidth}{0.2pt}
\fancyhead[L]{maths-cours.fr}
\fancyhead[R]{\thepage}
\renewcommand{\footrulewidth}{0.2pt}
\fancyfoot[C]{}

\newcolumntype{C}{>{\centering\arraybackslash}X}
\newcolumntype{s}{>{\hsize=.35\hsize\arraybackslash}X}

\setlength{\parindent}{0pt}		 
\setlength{\parskip}{3mm}
\setlength{\headheight}{1cm}

\def\ebook{ebook}
\def\book{book}
\def\web{web}
\def\type{web}

\newcommand{\vect}[1]{\overrightarrow{\,\mathstrut#1\,}}

\def\Oij{$\left(\text{O}~;~\vect{\imath},~\vect{\jmath}\right)$}
\def\Oijk{$\left(\text{O}~;~\vect{\imath},~\vect{\jmath},~\vect{k}\right)$}
\def\Ouv{$\left(\text{O}~;~\vect{u},~\vect{v}\right)$}

\hypersetup{breaklinks=true, colorlinks = true, linkcolor = OliveGreen, urlcolor = OliveGreen, citecolor = OliveGreen, pdfauthor={Didier BONNEL - https://www.maths-cours.fr} } % supprime les bordures autour des liens

\renewcommand{\arg}[0]{\text{arg}}

\everymath{\displaystyle}

%================================================================================================================================
%
% Macros - Commandes
%
%================================================================================================================================

\newcommand\meta[2]{    			% Utilisé pour créer le post HTML.
	\def\titre{titre}
	\def\url{url}
	\def\arg{#1}
	\ifx\titre\arg
		\newcommand\maintitle{#2}
		\fancyhead[L]{#2}
		{\Large\sffamily \MakeUppercase{#2}}
		\vspace{1mm}\textcolor{mcvert}{\hrule}
	\fi 
	\ifx\url\arg
		\fancyfoot[L]{\href{https://www.maths-cours.fr#2}{\black \footnotesize{https://www.maths-cours.fr#2}}}
	\fi 
}


\newcommand\TitreC[1]{    		% Titre centré
     \needspace{3\baselineskip}
     \begin{center}\textbf{#1}\end{center}
}

\newcommand\newpar{    		% paragraphe
     \par
}

\newcommand\nosp {    		% commande vide (pas d'espace)
}
\newcommand{\id}[1]{} %ignore

\newcommand\boite[2]{				% Boite simple sans titre
	\vspace{5mm}
	\setlength{\fboxrule}{0.2mm}
	\setlength{\fboxsep}{5mm}	
	\fcolorbox{#1}{#1!3}{\makebox[\linewidth-2\fboxrule-2\fboxsep]{
  		\begin{minipage}[t]{\linewidth-2\fboxrule-4\fboxsep}\setlength{\parskip}{3mm}
  			 #2
  		\end{minipage}
	}}
	\vspace{5mm}
}

\newcommand\CBox[4]{				% Boites
	\vspace{5mm}
	\setlength{\fboxrule}{0.2mm}
	\setlength{\fboxsep}{5mm}
	
	\fcolorbox{#1}{#1!3}{\makebox[\linewidth-2\fboxrule-2\fboxsep]{
		\begin{minipage}[t]{1cm}\setlength{\parskip}{3mm}
	  		\textcolor{#1}{\LARGE{#2}}    
 	 	\end{minipage}  
  		\begin{minipage}[t]{\linewidth-2\fboxrule-4\fboxsep}\setlength{\parskip}{3mm}
			\raisebox{1.2mm}{\normalsize\sffamily{\textcolor{#1}{#3}}}						
  			 #4
  		\end{minipage}
	}}
	\vspace{5mm}
}

\newcommand\cadre[3]{				% Boites convertible html
	\par
	\vspace{2mm}
	\setlength{\fboxrule}{0.1mm}
	\setlength{\fboxsep}{5mm}
	\fcolorbox{#1}{white}{\makebox[\linewidth-2\fboxrule-2\fboxsep]{
  		\begin{minipage}[t]{\linewidth-2\fboxrule-4\fboxsep}\setlength{\parskip}{3mm}
			\raisebox{-2.5mm}{\sffamily \small{\textcolor{#1}{\MakeUppercase{#2}}}}		
			\par		
  			 #3
 	 		\end{minipage}
	}}
		\vspace{2mm}
	\par
}

\newcommand\bloc[3]{				% Boites convertible html sans bordure
     \needspace{2\baselineskip}
     {\sffamily \small{\textcolor{#1}{\MakeUppercase{#2}}}}    
		\par		
  			 #3
		\par
}

\newcommand\CHelp[1]{
     \CBox{Plum}{\faInfoCircle}{À RETENIR}{#1}
}

\newcommand\CUp[1]{
     \CBox{NavyBlue}{\faThumbsOUp}{EN PRATIQUE}{#1}
}

\newcommand\CInfo[1]{
     \CBox{Sepia}{\faArrowCircleRight}{REMARQUE}{#1}
}

\newcommand\CRedac[1]{
     \CBox{PineGreen}{\faEdit}{BIEN R\'EDIGER}{#1}
}

\newcommand\CError[1]{
     \CBox{Red}{\faExclamationTriangle}{ATTENTION}{#1}
}

\newcommand\TitreExo[2]{
\needspace{4\baselineskip}
 {\sffamily\large EXERCICE #1\ (\emph{#2 points})}
\vspace{5mm}
}

\newcommand\img[2]{
          \includegraphics[width=#2\paperwidth]{\imgdir#1}
}

\newcommand\imgsvg[2]{
       \begin{center}   \includegraphics[width=#2\paperwidth]{\imgsvgdir#1} \end{center}
}


\newcommand\Lien[2]{
     \href{#1}{#2 \tiny \faExternalLink}
}
\newcommand\mcLien[2]{
     \href{https~://www.maths-cours.fr/#1}{#2 \tiny \faExternalLink}
}

\newcommand{\euro}{\eurologo{}}

%================================================================================================================================
%
% Macros - Environement
%
%================================================================================================================================

\newenvironment{tex}{ %
}
{%
}

\newenvironment{indente}{ %
	\setlength\parindent{10mm}
}

{
	\setlength\parindent{0mm}
}

\newenvironment{corrige}{%
     \needspace{3\baselineskip}
     \medskip
     \textbf{\textsc{Corrigé}}
     \medskip
}
{
}

\newenvironment{extern}{%
     \begin{center}
     }
     {
     \end{center}
}

\NewEnviron{code}{%
	\par
     \boite{gray}{\texttt{%
     \BODY
     }}
     \par
}

\newenvironment{vbloc}{% boite sans cadre empeche saut de page
     \begin{minipage}[t]{\linewidth}
     }
     {
     \end{minipage}
}
\NewEnviron{h2}{%
    \needspace{3\baselineskip}
    \vspace{0.6cm}
	\noindent \MakeUppercase{\sffamily \large \BODY}
	\vspace{1mm}\textcolor{mcgris}{\hrule}\vspace{0.4cm}
	\par
}{}

\NewEnviron{h3}{%
    \needspace{3\baselineskip}
	\vspace{5mm}
	\textsc{\BODY}
	\par
}

\NewEnviron{margeneg}{ %
\begin{addmargin}[-1cm]{0cm}
\BODY
\end{addmargin}
}

\NewEnviron{html}{%
}

\begin{document}
\meta{url}{/exercices/suites-bac-blanc-es-l-sujet-1-maths-cours-2017/}
\meta{pid}{10405}
\meta{titre}{Suites - Bac blanc ES/L Sujet 1 - Maths-cours 2017}
\meta{type}{exercices}
%
\begin{h2}Exercice 3 (5 points)\end{h2}
\par
Antoine et Bruno travaillent dans deux entreprises différentes depuis le premier janvier 2015.
\par
En 2015, leurs salaires annuels s'élevaient à  $19\ 500$~euros pour Antoine et à $21\ 000$~euros pour Bruno.
\par
Chaque année, leurs salaires sont réévalués de la façon suivante :\nopagebreak
\par
\begin{itemize}
     \item le salaire d'Antoine augmente de 3\% par an ;
     \item le salaire de Bruno augmente de 500~euros par an.
\end{itemize}
\par
On note $a_n$ et $b_n$ les salaires respectifs d'Antoine et de Bruno (en~euros) pour l'année $(2015+n)$.
\par
On a donc $a_0=19\ 500$ et $b_0=21\ 000$.
\par
%============================================================================================================================
%
\TitreC{Partie A}
%
%============================================================================================================================
\par
\begin{enumerate}
     \par
     \item %1
     \par
     \begin{enumerate}[label=\alph*.]
          \par
          \item %1a
          Calculer $a_1$.
          \par
          \item %1b
          \'Etablir une relation entre $a_{n+1}$ et $a_n$.
          \par
          \item %1c
          Quelle est la nature de la suite $(a_n)$ ?
          \par
     \end{enumerate}
     \par
     \item %2
     \par
     \begin{enumerate}[label=\alph*.]
          \par
          \item %2a
          Exprimer $a_n$ en fonction de $n$.
          \par
          \item %2b
          Quel sera le salaire d'Antoine en 2030 ?
          \par
     \end{enumerate}
     \par
\end{enumerate}
\par
%============================================================================================================================
%
\TitreC{Partie B}
%
%============================================================================================================================
\par
\begin{enumerate}
     \par
     \item %1
     \par
     \begin{enumerate}[label=\alph*.]
          \item %1a
          Calculer $b_1$.
          \par
          \item %1b
          \'Etablir une relation entre $b_{n+1}$ et $b_n$.
          \par
          \item %1c
          Quelle est la nature de la suite $(b_n)$ ?
          \par
     \end{enumerate}
     \par
     \item %2
     \par
     \begin{enumerate}[label=\alph*.]
          \item %2a
          Exprimer $b_n$ en fonction de $n$.
          \par
          \item %2b
          D'Antoine ou de Bruno, qui percevra le salaire le plus élevé en 2030 ? Justifier la réponse.
          \par
     \end{enumerate}
     \par
\end{enumerate}
\par
%============================================================================================================================
%
\TitreC{Partie C}
%
%============================================================================================================================
\par
On considère l'algorithme suivant :
\par
\begin{center}
     \begin{extern}%width="400" alt="algorithme suites"
          \begin{tabularx}{0.60\linewidth}{|l|X|}\hline
               Variables :	& $n$ est un entier naturel\\
               &$a$ et $b$ sont des nombres réels\\
               & \\
               Initialisation: &Affecter à $n$ la valeur 0\\
               &Affecter à $a$ la valeur $19\ 500$\\
               &Affecter à $b$ la valeur $21\ 000$\\
               & \\
               Traitement: &Tant que $a \leqslant  b$ faire\\
               &\qquad$n$ prend la valeur $n + 1$\\
               &\qquad$a$ prend la valeur $1,03a$\\
               &\qquad$b$ prend la valeur $b+500$\\
               &Fin Tant que\\
               & \\
               Sortie :	&Afficher 2015+n \\
               \hline
          \end{tabularx}
     \end{extern}
\end{center}
\par
\begin{enumerate}
     \item Recopier et compléter le tableau ci-après, en ajoutant autant de colonnes que nécessaire. On arrondira les résultats à l'unité près.
     \begin{center}
          \begin{tabular}{|l|c|c|c|}\hline %class="compact"
               Valeur de $n$	&0	&	1 &	 $\quad \cdots \quad$  \\ \hline
               Valeur de $a$	&$19\ 500$	& $\quad \cdots \quad$ & $\quad \cdots \quad$ 	 \\ \hline
               Valeur de $b$	&$21\ 000$	& $\quad \cdots \quad$ & $\quad \cdots \quad$  \\ \hline
               Condition $a \leqslant  b$	&vraie	& $\quad \cdots \quad$ & $\quad \cdots \quad$ 	\\ \hline
          \end{tabular}
     \end{center}
     \item Quelle valeur affichera cet algorithme en sortie ?\\
     Interpréter cette valeur dans le contexte de l'exercice.
\end{enumerate}
\begin{corrige}
     %============================================================================================================================
     %
     \TitreC{Partie A}
     %
     %============================================================================================================================
     \par
     \begin{enumerate}
          \par
          \item %1
          \par
          \begin{enumerate}[label=\alph*.]
               \par
               \item %1a
               \par
               \`A une augmentation de $3\%$ correspond un coefficient multiplicateur :
               \[CM=1+\frac{3}{100}=1,03. \]
               \par
               $a_1$ représente le salaire d'Antoine en 2016. On a donc :
               \par
               $a_1=a_0 \times 1,03=19\ 500 \times 1,03 = 20\ 085.$
               \par
               \cadre{rouge}{À retenir}{
                    Pour augmenter une valeur de $t\%$, on multiplie cette valeur par le coefficient multiplicateur :
                    \[CM=1+\frac{t}{100} \]
                    \par
                    Pour diminuer une valeur de $t\%$, on multiplie cette valeur par le coefficient multiplicateur :
                    \[CM=1-\frac{t}{100} \]
               }
               \par
               \item %1b
               \par
               Le salaire d'Antoine pour l'année $n+1$ est égal à son salaire de l'année $n$ augmenté de $3\%$ ; par conséquent :
               \[a_{n+1}=1,03a_n.\]
               \par
               \item %1c
               \par
               La suite $(a_n)$ est donc une suite géométrique de premier terme $a_0=19\ 500$ et de raison $q=1,03$.
               \cadre{rouge}{Attention}{
                    \textbf{Ne pas écrire} \og la suite $a_n$ \fg{} (sans parenthèses) mais : \og la suite $(a_n)$ \fg{} (entre parenthèses).
                    \par
                    $(a_n)$ représente la \textbf{suite}, tandis que $a_n$ représente un \textbf{terme} de la suite, c'est à dire un \textbf{nombre} réel.
               }
               \par
               \cadre{rouge}{À retenir}{
                    Une \textbf{suite géométrique} $(u_n)$ est définie par une relation de récurrence de la forme :
                    \[u_{n+1}=q \times u_n\]
                    où $q$ s'appelle la \textbf{raison} de la suite.
               }
               \par
          \end{enumerate}
          \par
          \item %2
          \par
          \begin{enumerate}[label=\alph*.]
               \par
               \item %2a
               La suite $(a_n)$ étant une suite géométrique de premier terme $a_0=19\ 500$ et de raison $q=1,03$ :
               \par
               $a_n = a_0q^n=19\ 500 \times 1,03^n$.
               \cadre{rouge}{À retenir}{
                    Pour une suite \textbf{géométrique} $(u_n)$ de premier terme $u_0$ et de raison $q$, le $n$-ième terme vaut :
                    \[u_{n}=u_0 \times q^n.\]
               }
               \item %2b
               ${2030=2015+15}$. Le salaire d'Antoine en 2030 sera donc égal à $a_{15}$ :
               \par
               $a_{15}=19\ 500 \times 1,03^{15}\approx 30\ 380$ (arrondi à l'euro).
               \par
               Le salaire d'Antoine en 2030 sera 30~380~euros.
               \par
          \end{enumerate}
          \par
     \end{enumerate}
     \par
     %============================================================================================================================
     %
     \TitreC{Partie B}
     %
     %============================================================================================================================
     \par
     \begin{enumerate}
          \par
          \item %1
          \par
          \begin{enumerate}[label=\alph*.]
               \par
               \item %1a
               \par
               Le salaire de Bruno augmente de 500~euros par an, donc :
               \par
               $b_1=b_0+500=21\ 000+500=21\ 500$.
               \par
               \item %1b
               \par
               Pour la même raison :
               \par
               \[b_{n+1}=b_{n}+500.\]
               \par
               \item %1c
               \par
               La suite $(b_n)$ est une suite arithmétique de premier terme ${b_0=21\ 000}$ et de raison ${r=500}$.
               \cadre{rouge}{À retenir}{
                    Une \textbf{suite arithmétique} $(u_n)$ est définie par une relation de récurrence de la forme :
                    \[u_{n+1}= u_n + r\]
                    où $r$ s'appelle la \textbf{raison} de la suite.
               }
          \end{enumerate}
          \par
          \item %2
          \par
          \begin{enumerate}[label=\alph*.]
               \par
               \item %2a
               La suite $(b_n)$ étant une suite arithmétique de premier terme ${b_0=21\ 000}$ et de raison ${r=500}$ :
               \par
               $b_n=b_0+nr=21\ 000+500n$.
               \cadre{rouge}{À retenir}{
                    Pour une suite \textbf{arithmétique} $(u_n)$ de premier terme $u_0$ et de raison $r$, le $n$-ième terme vaut :
                    \[u_{n}=u_0 +nr \]
               }
               \item %2b
               En 2030 le salaire de Bruno sera :
               \par
               $b_{15}=21~000 + 500 \times 15 = 28~500$.
               \par
               Par conséquent, en 2030, \textbf{le salaire d'Antoine sera supérieur au salaire de Bruno}.
               \par
          \end{enumerate}
          \par
     \end{enumerate}
     \par
     %============================================================================================================================
     %
     \TitreC{Partie C}
     %
     %============================================================================================================================
     \par
     \begin{enumerate}
          \item \`A l'aide de la calculatrice on obtient le tableau suivant :
          \par
          \begin{tabular}{|c|c|c|c|c|c|c|}\hline  %class="compact"
               $n$	& 0	&	1 &	2 &	3 &	4 &	5  \\ \hline
               $a$	& $19\ 500$	& $20\ 085$  	& $20\ 688$  	&$21\ 308$  	& $21\ 947$  	& $22\ 606$   \\ \hline
               $b$	& $21\ 000$	& $21\ 500$  	& $22\ 000$  	& $22\ 500$  	& $23\ 000$  	& $23\ 500$ 	\\ \hline
               $a \leqslant  b$	& vraie	& 	vraie	& vraie	& vraie	& vraie	& vraie	 \\ \hline
          \end{tabular}
          \par
          \begin{tabular}{|c|c|c|c|c|c|}\hline %class="compact"
               $n$	& 6 &	7 &	8 &	9 &	10 \\ \hline
               $a$	& $23\ 284$  	& $23\ 983$  	& $24\ 702$  	& $25\ 443$  	& $26\ 206$  	\\ \hline
               $b$	& $24\ 000$  	& $24\ 500$  	& $25\ 000$  	& $25\ 500$  	& $26\ 000$  	\\ \hline
               $a \leqslant  b$	& vraie	& vraie	& vraie	& vraie	& fausse	  \\ \hline
          \end{tabular}
          \par
          \cadre{vert}{En pratique}{
               Pour calculer rapidement les différentes valeurs de $a$ et de $b$, on peut utiliser l'écran \og suite \fg{} ou l'écran \og fonction \fg{} de la calculatrice.
               \par
               Par contre, le jour du bac, saisir l'algorithme complet dans la calculatrice est long et peu utile.
          }
          \par
          \item Le tableau précédent montre que lorsque l'algorithme se termine $n$ vaut $10$.
          \par
          L'algorithme affiche donc la valeur ${2015+10}$ soit $2025$. Cette valeur correspond à l'année à partir de laquelle le salaire d'Antoine dépassera celui de Bruno.
     \end{enumerate}
\end{corrige}

\end{document}
µ
\documentclass[a4paper]{article}

%================================================================================================================================
%
% Packages
%
%================================================================================================================================

\usepackage[T1]{fontenc} 	% pour caractères accentués
\usepackage[utf8]{inputenc}  % encodage utf8
\usepackage[french]{babel}	% langue : français
\usepackage{fourier}			% caractères plus lisibles
\usepackage[dvipsnames]{xcolor} % couleurs
\usepackage{fancyhdr}		% réglage header footer
\usepackage{needspace}		% empêcher sauts de page mal placés
\usepackage{graphicx}		% pour inclure des graphiques
\usepackage{enumitem,cprotect}		% personnalise les listes d'items (nécessaire pour ol, al ...)
\usepackage{hyperref}		% Liens hypertexte
\usepackage{pstricks,pst-all,pst-node,pstricks-add,pst-math,pst-plot,pst-tree,pst-eucl} % pstricks
\usepackage[a4paper,includeheadfoot,top=2cm,left=3cm, bottom=2cm,right=3cm]{geometry} % marges etc.
\usepackage{comment}			% commentaires multilignes
\usepackage{amsmath,environ} % maths (matrices, etc.)
\usepackage{amssymb,makeidx}
\usepackage{bm}				% bold maths
\usepackage{tabularx}		% tableaux
\usepackage{colortbl}		% tableaux en couleur
\usepackage{fontawesome}		% Fontawesome
\usepackage{environ}			% environment with command
\usepackage{fp}				% calculs pour ps-tricks
\usepackage{multido}			% pour ps tricks
\usepackage[np]{numprint}	% formattage nombre
\usepackage{tikz,tkz-tab} 			% package principal TikZ
\usepackage{pgfplots}   % axes
\usepackage{mathrsfs}    % cursives
\usepackage{calc}			% calcul taille boites
\usepackage[scaled=0.875]{helvet} % font sans serif
\usepackage{svg} % svg
\usepackage{scrextend} % local margin
\usepackage{scratch} %scratch
\usepackage{multicol} % colonnes
%\usepackage{infix-RPN,pst-func} % formule en notation polanaise inversée
\usepackage{listings}

%================================================================================================================================
%
% Réglages de base
%
%================================================================================================================================

\lstset{
language=Python,   % R code
literate=
{á}{{\'a}}1
{à}{{\`a}}1
{ã}{{\~a}}1
{é}{{\'e}}1
{è}{{\`e}}1
{ê}{{\^e}}1
{í}{{\'i}}1
{ó}{{\'o}}1
{õ}{{\~o}}1
{ú}{{\'u}}1
{ü}{{\"u}}1
{ç}{{\c{c}}}1
{~}{{ }}1
}


\definecolor{codegreen}{rgb}{0,0.6,0}
\definecolor{codegray}{rgb}{0.5,0.5,0.5}
\definecolor{codepurple}{rgb}{0.58,0,0.82}
\definecolor{backcolour}{rgb}{0.95,0.95,0.92}

\lstdefinestyle{mystyle}{
    backgroundcolor=\color{backcolour},   
    commentstyle=\color{codegreen},
    keywordstyle=\color{magenta},
    numberstyle=\tiny\color{codegray},
    stringstyle=\color{codepurple},
    basicstyle=\ttfamily\footnotesize,
    breakatwhitespace=false,         
    breaklines=true,                 
    captionpos=b,                    
    keepspaces=true,                 
    numbers=left,                    
xleftmargin=2em,
framexleftmargin=2em,            
    showspaces=false,                
    showstringspaces=false,
    showtabs=false,                  
    tabsize=2,
    upquote=true
}

\lstset{style=mystyle}


\lstset{style=mystyle}
\newcommand{\imgdir}{C:/laragon/www/newmc/assets/imgsvg/}
\newcommand{\imgsvgdir}{C:/laragon/www/newmc/assets/imgsvg/}

\definecolor{mcgris}{RGB}{220, 220, 220}% ancien~; pour compatibilité
\definecolor{mcbleu}{RGB}{52, 152, 219}
\definecolor{mcvert}{RGB}{125, 194, 70}
\definecolor{mcmauve}{RGB}{154, 0, 215}
\definecolor{mcorange}{RGB}{255, 96, 0}
\definecolor{mcturquoise}{RGB}{0, 153, 153}
\definecolor{mcrouge}{RGB}{255, 0, 0}
\definecolor{mclightvert}{RGB}{205, 234, 190}

\definecolor{gris}{RGB}{220, 220, 220}
\definecolor{bleu}{RGB}{52, 152, 219}
\definecolor{vert}{RGB}{125, 194, 70}
\definecolor{mauve}{RGB}{154, 0, 215}
\definecolor{orange}{RGB}{255, 96, 0}
\definecolor{turquoise}{RGB}{0, 153, 153}
\definecolor{rouge}{RGB}{255, 0, 0}
\definecolor{lightvert}{RGB}{205, 234, 190}
\setitemize[0]{label=\color{lightvert}  $\bullet$}

\pagestyle{fancy}
\renewcommand{\headrulewidth}{0.2pt}
\fancyhead[L]{maths-cours.fr}
\fancyhead[R]{\thepage}
\renewcommand{\footrulewidth}{0.2pt}
\fancyfoot[C]{}

\newcolumntype{C}{>{\centering\arraybackslash}X}
\newcolumntype{s}{>{\hsize=.35\hsize\arraybackslash}X}

\setlength{\parindent}{0pt}		 
\setlength{\parskip}{3mm}
\setlength{\headheight}{1cm}

\def\ebook{ebook}
\def\book{book}
\def\web{web}
\def\type{web}

\newcommand{\vect}[1]{\overrightarrow{\,\mathstrut#1\,}}

\def\Oij{$\left(\text{O}~;~\vect{\imath},~\vect{\jmath}\right)$}
\def\Oijk{$\left(\text{O}~;~\vect{\imath},~\vect{\jmath},~\vect{k}\right)$}
\def\Ouv{$\left(\text{O}~;~\vect{u},~\vect{v}\right)$}

\hypersetup{breaklinks=true, colorlinks = true, linkcolor = OliveGreen, urlcolor = OliveGreen, citecolor = OliveGreen, pdfauthor={Didier BONNEL - https://www.maths-cours.fr} } % supprime les bordures autour des liens

\renewcommand{\arg}[0]{\text{arg}}

\everymath{\displaystyle}

%================================================================================================================================
%
% Macros - Commandes
%
%================================================================================================================================

\newcommand\meta[2]{    			% Utilisé pour créer le post HTML.
	\def\titre{titre}
	\def\url{url}
	\def\arg{#1}
	\ifx\titre\arg
		\newcommand\maintitle{#2}
		\fancyhead[L]{#2}
		{\Large\sffamily \MakeUppercase{#2}}
		\vspace{1mm}\textcolor{mcvert}{\hrule}
	\fi 
	\ifx\url\arg
		\fancyfoot[L]{\href{https://www.maths-cours.fr#2}{\black \footnotesize{https://www.maths-cours.fr#2}}}
	\fi 
}


\newcommand\TitreC[1]{    		% Titre centré
     \needspace{3\baselineskip}
     \begin{center}\textbf{#1}\end{center}
}

\newcommand\newpar{    		% paragraphe
     \par
}

\newcommand\nosp {    		% commande vide (pas d'espace)
}
\newcommand{\id}[1]{} %ignore

\newcommand\boite[2]{				% Boite simple sans titre
	\vspace{5mm}
	\setlength{\fboxrule}{0.2mm}
	\setlength{\fboxsep}{5mm}	
	\fcolorbox{#1}{#1!3}{\makebox[\linewidth-2\fboxrule-2\fboxsep]{
  		\begin{minipage}[t]{\linewidth-2\fboxrule-4\fboxsep}\setlength{\parskip}{3mm}
  			 #2
  		\end{minipage}
	}}
	\vspace{5mm}
}

\newcommand\CBox[4]{				% Boites
	\vspace{5mm}
	\setlength{\fboxrule}{0.2mm}
	\setlength{\fboxsep}{5mm}
	
	\fcolorbox{#1}{#1!3}{\makebox[\linewidth-2\fboxrule-2\fboxsep]{
		\begin{minipage}[t]{1cm}\setlength{\parskip}{3mm}
	  		\textcolor{#1}{\LARGE{#2}}    
 	 	\end{minipage}  
  		\begin{minipage}[t]{\linewidth-2\fboxrule-4\fboxsep}\setlength{\parskip}{3mm}
			\raisebox{1.2mm}{\normalsize\sffamily{\textcolor{#1}{#3}}}						
  			 #4
  		\end{minipage}
	}}
	\vspace{5mm}
}

\newcommand\cadre[3]{				% Boites convertible html
	\par
	\vspace{2mm}
	\setlength{\fboxrule}{0.1mm}
	\setlength{\fboxsep}{5mm}
	\fcolorbox{#1}{white}{\makebox[\linewidth-2\fboxrule-2\fboxsep]{
  		\begin{minipage}[t]{\linewidth-2\fboxrule-4\fboxsep}\setlength{\parskip}{3mm}
			\raisebox{-2.5mm}{\sffamily \small{\textcolor{#1}{\MakeUppercase{#2}}}}		
			\par		
  			 #3
 	 		\end{minipage}
	}}
		\vspace{2mm}
	\par
}

\newcommand\bloc[3]{				% Boites convertible html sans bordure
     \needspace{2\baselineskip}
     {\sffamily \small{\textcolor{#1}{\MakeUppercase{#2}}}}    
		\par		
  			 #3
		\par
}

\newcommand\CHelp[1]{
     \CBox{Plum}{\faInfoCircle}{À RETENIR}{#1}
}

\newcommand\CUp[1]{
     \CBox{NavyBlue}{\faThumbsOUp}{EN PRATIQUE}{#1}
}

\newcommand\CInfo[1]{
     \CBox{Sepia}{\faArrowCircleRight}{REMARQUE}{#1}
}

\newcommand\CRedac[1]{
     \CBox{PineGreen}{\faEdit}{BIEN R\'EDIGER}{#1}
}

\newcommand\CError[1]{
     \CBox{Red}{\faExclamationTriangle}{ATTENTION}{#1}
}

\newcommand\TitreExo[2]{
\needspace{4\baselineskip}
 {\sffamily\large EXERCICE #1\ (\emph{#2 points})}
\vspace{5mm}
}

\newcommand\img[2]{
          \includegraphics[width=#2\paperwidth]{\imgdir#1}
}

\newcommand\imgsvg[2]{
       \begin{center}   \includegraphics[width=#2\paperwidth]{\imgsvgdir#1} \end{center}
}


\newcommand\Lien[2]{
     \href{#1}{#2 \tiny \faExternalLink}
}
\newcommand\mcLien[2]{
     \href{https~://www.maths-cours.fr/#1}{#2 \tiny \faExternalLink}
}

\newcommand{\euro}{\eurologo{}}

%================================================================================================================================
%
% Macros - Environement
%
%================================================================================================================================

\newenvironment{tex}{ %
}
{%
}

\newenvironment{indente}{ %
	\setlength\parindent{10mm}
}

{
	\setlength\parindent{0mm}
}

\newenvironment{corrige}{%
     \needspace{3\baselineskip}
     \medskip
     \textbf{\textsc{Corrigé}}
     \medskip
}
{
}

\newenvironment{extern}{%
     \begin{center}
     }
     {
     \end{center}
}

\NewEnviron{code}{%
	\par
     \boite{gray}{\texttt{%
     \BODY
     }}
     \par
}

\newenvironment{vbloc}{% boite sans cadre empeche saut de page
     \begin{minipage}[t]{\linewidth}
     }
     {
     \end{minipage}
}
\NewEnviron{h2}{%
    \needspace{3\baselineskip}
    \vspace{0.6cm}
	\noindent \MakeUppercase{\sffamily \large \BODY}
	\vspace{1mm}\textcolor{mcgris}{\hrule}\vspace{0.4cm}
	\par
}{}

\NewEnviron{h3}{%
    \needspace{3\baselineskip}
	\vspace{5mm}
	\textsc{\BODY}
	\par
}

\NewEnviron{margeneg}{ %
\begin{addmargin}[-1cm]{0cm}
\BODY
\end{addmargin}
}

\NewEnviron{html}{%
}

\begin{document}
\meta{url}{/exercices/probabilites-bac-blanc-es-l-sujet-1-maths-cours-2017/}
\meta{pid}{10412}
\meta{titre}{Probabilités - Bac blanc ES/L Sujet 1 - Maths-cours 2017}
\meta{type}{exercices}
%
\begin{h2}Exercice 4 (5 points)\end{h2}
\par
Un constructeur fabrique des tablettes informatiques. Le coût de production est 250~euros par unité.
\par
Les tablettes sont garanties contre un défaut de fonctionnement de l'écran ou du disque dur.
\par
Cette garantie permet à l'acheteur, en cas de panne, d'effectuer les réparations suivantes aux frais du constructeur~:
\begin{itemize}
     \item réparation de l'écran (coût pour le constructeur ~: 50~euros) ;
     \item réparation du disque dur (coût pour le constructeur ~: 30~euros).
\end{itemize}
\par
Une étude statistique a montré que ~:
\begin{itemize}
     \item 3\% des tablettes présentent un défaut de disque dur ;
     \item 4\% des tablettes présentent un défaut d'écran ;
     \item 95\% des tablettes ne présentent aucun des deux défauts.
\end{itemize}
\par
%============================================================================================================================
%
\TitreC{Partie A}
%
%============================================================================================================================
\par
\begin{enumerate}
     \par
     \item %1
     Recopier et compléter le tableau ci-après à l'aide des données de l'énoncé.
     \begin{center}
          \begin{tabular}{|c|p{2cm}|p{2cm}|c|}%class="compact"
               \hline
               $\ $ & Disque dur OK & Disque dur défectueux & Total \\
               \hline
               \'Ecran OK &  $\cdots$ & $\cdots$ & $\cdots$ \\
               \hline
               \'Ecran défectueux &  $\cdots$ & $\cdots$ & $\cdots$ \\
               \hline
               Total & $\cdots$ & 3\% & 100 \% \\
               \hline
          \end{tabular}
     \end{center}
     \item %2
     Le prix de revient d'une tablette est égal à son coût de production augmenté du coût de réparation éventuel.
     On note $X$ la variable aléatoire correspondant au prix de revient d'une tablette.\\
     \'Etablir la loi de probabilité de $X$.
     \par
     \item %3
     Calculer l'espérance mathématique de $X$. Interpréter ce résultat dans le contexte de l'énoncé.
     \par
     \item %4
     L'entreprise vend chaque tablette 400~euros. Quel sera son bénéfice mensuel moyen si elle vend 750 tablettes par mois ?
     \par
\end{enumerate}
\par
%============================================================================================================================
%
\TitreC{Partie B}
%
%============================================================================================================================
\par
Un établissement scolaire achète 50 tablettes à ce constructeur.
\par
On suppose que l'on peut assimiler cet achat à un tirage aléatoire de 50 tablettes avec remise, les tirages étant supposés indépendants.
\par
On rappelle que 95\% des tablettes ne présentent aucun défaut couvert par la garantie constructeur.
\par
On note $Y$ la variable aléatoire égale au nombre de tablettes achetées par l'établissement présentant un défaut couvert par la garantie constructeur.
\par
\begin{enumerate}
     \par
     \item %1
     Justifier que $Y$ suit une loi binomiale dont on précisera les paramètres.
     \par
     \item %2
     Quelle est la probabilité qu'aucune des tablettes achetées par l'établissement ne présente de défaut couvert par la garantie constructeur ?
     \par
     \item %3
     \par
     Quelle est l'espérance mathématique de $Y$ ? Interpréter ce résultat.
     \par
\end{enumerate}
\begin{corrige}
     %============================================================================================================================
     %
     \TitreC{Partie A}
     %
     %============================================================================================================================
     \par
     \begin{enumerate}
          \par
          \item %1
          \par
          On place dans le tableau les données fournies par l'énoncé ~:
          \par
          \begin{itemize}
               \item %
               3\% des tablettes présentent un défaut de disque dur ;
               \item %
               4\% des tablettes présentent un défaut d'écran ;
               \item %
               95\% des tablettes ne présentent aucun des deux défauts.
          \end{itemize}
          \begin{center}
               \begin{tabular}{|c|p{2cm}|p{2cm}|c|}%class="compact"
                    \hline
                    $\ $ & Disque dur OK & Disque dur défectueux & Total \\
                    \hline
                    \'Ecran OK &  95\% & $\cdots$ & $\cdots$ \\
                    \hline
                    \'Ecran défectueux & $\cdots$ & $\cdots$ & 4\% \\
                    \hline
                    Total & $\cdots$ & 3\% & 100 \% \\
                    \hline
               \end{tabular}
          \end{center}
          On complète ensuite les totaux partiels afin que le total global soit égal à 100\% ~:
          \begin{center}
               \begin{tabular}{|c|p{2cm}|p{2cm}|c|}%class="compact"
                    \hline
                    $\ $ & Disque dur OK & Disque dur défectueux & Total \\
                    \hline
                    \'Ecran OK &  95\% & $\cdots$ & 96\% \\
                    \hline
                    \'Ecran défectueux & $\cdots$ & $\cdots$ & 4\% \\
                    \hline
                    Total & 97\% & 3\% & 100 \% \\
                    \hline
               \end{tabular}
          \end{center}
          Les données restantes peuvent être calculées simplement à partir des totaux ~:
          \begin{center}
               \begin{tabular}{|c|p{2cm}|p{2cm}|c|}%class="compact"
                    \hline
                    $\ $ & Disque dur OK & Disque dur défectueux & Total \\
                    \hline
                    \'Ecran OK &  95\% & 1\% & 96\% \\
                    \hline
                    \'Ecran défectueux & 2\% & 2\%  & 4\% \\
                    \hline
                    Total & 97\% & 3\% & 100 \% \\
                    \hline
               \end{tabular}
          \end{center}
          \item %2
          \par
          La variable aléatoire $X$ peut prendre quatre valeurs distinctes ; le tableau de la question précédente fournit la probabilité de chacune d'elle ~:
          \par
          \begin{itemize}
               \item si la tablette ne présente aucun défaut ~: ${X=250}$ \textit{(probabilité ~: 0,95)} ;
               \par
               \item si la tablette présente \textbf{uniquement} un défaut de disque dur ~: ${X=250+30=280}$ \textit{(probabilité ~: 0,01)} ;
               \par
               \item si la tablette présente \textbf{uniquement} un défaut d'écran ~:${X=250+50=300}$ \textit{(probabilité ~: 0,02)} ;
               \par
               \item si la tablette présente \textbf{à la fois un défaut de disque dur et un défaut d'écran} ~: ${X=250+50+30=330}$ \textit{(probabilité ~: 0,02)}.
               \par
          \end{itemize}
          \par
          On peut regrouper ces résultats dans un tableau ~:
          \par
          \begin{center}
               \begin{tabular}{|c|c|c|c|c|}%class="compact"
                    \hline
                    $x_i$ & 250 & 280 & 300 & 330 \\
                    \hline
                    $p(X=x_i)$ & 0,95 & 0,01 & 0,02 & 0,02 \\
                    \hline
               \end{tabular}
          \end{center}
          \par
          \cadre{rouge}{À retenir}{
               La \textbf{loi de probabilité} d'une variable aléatoire X est un tableau qui recense les différentes valeurs $x_1, x_2, \cdots, x_n$ prises par X et les probabilités des événements ${(X=x_1), (X=x_2), \cdots, (X=x_n)}$
          }
          \par
          \item %3
          L'espérance mathématique de $X$ est ~:
          \par
          $E(X)=250 \times 0,95 + 280 \times 0,01 + 300 \times 0,02 + 330 \times 0,02 = 252,9$.
          \par
          \cadre{rouge}{À retenir}{
               Si X est une variable aléatoire qui prend valeurs $x_1, x_2, \cdots, x_n$ avec les probabilités respectives $p_1, p_2, \cdots, p_n$, l'\textbf{espérance mathématique} de X est ~:
               \[ E(X)=p_1x_1+p_2x_2+ \cdots +p_nx_n \]
               \par
          }
          \par
          \item %4
          D'après la question précédente, le prix de revient moyen d'une tablette est de 252,9~euros.
          \par
          Si chaque tablette est vendu 400~euros, le bénéfice moyen par tablette vendue sera de $400 - 252,9 = 147,1$~euros.
          \par
          Pour une vente mensuelle de 750 tablettes, l'entreprise fera un bénéfice mensuel moyen de $750 \times 147,1 =\bm{110\ 325}$ euros.
          \par
     \end{enumerate}
     \par
     %============================================================================================================================
     %
     \TitreC{Partie B}
     %
     %============================================================================================================================
     \par
     \begin{enumerate}
          \par
          \item %1
          \par
          La variable aléatoire $Y$ suit une loi binomiale de paramètres $n=50$ et $p=0,05$ puisque ~:
          \par
          \begin{itemize}
               \par
               \item on assimile l'expérience à la répétition de 50 tirages aléatoires identiques et indépendants ;
               \par
               \item chaque tirage possède deux issues ~:
               \par
               \begin{itemize}
                    \par
                    \item \textit{succès}, correspondant au tirage d'une tablette défectueuse (probabilité $p=0,05$) ;
                    \item \textit{échec}, correspondant au tirage d'une tablette fonctionnant correctement ;
                    \par
               \end{itemize}
               \par
               \item la variable aléatoire $Y$ comptabilise le nombre de succès.
               \par
          \end{itemize}
          \par
          \cadre{rouge}{Bien rédiger}{
               Pour montrer qu'une variable aléatoire suit une \textbf{loi binomiale} $\mathscr{B}(n~;~p)$ de paramètres $n$ et $p$, on précise que ~:
               \par
               \begin{itemize}
                    \par
                    \item l'expérience aléatoire est la \textbf{répétition} de $n$ épreuves de Bernoulli \textbf{identiques et indépendantes} ;
                    \par
                    \item chaque épreuve de Bernoulli possède \textbf{deux issues} ~:
                    \begin{itemize}[label=---]
                         \item %
                         \textit{succès}, de probabilité $p$;
                         \item %
                         \textit{échec}, de probabilité $1-p$ ;
                    \end{itemize}
                    \item la variable aléatoire $X$ \textbf{comptabilise le nombre de succès}.
                    \par
               \end{itemize}
          }
          \item %2
          La probabilité qu'aucune des tablettes achetées par l'établissement ne présente de défaut est ~:
          \par
          $P(Y=0)=\begin{pmatrix} 50 \\ 0 \end{pmatrix} \times 0,05^0 \times 0,95^{50} = 0,95^{50}$
          \par
          $P(Y=0) \approx 0,077$ (arrondi au millième).
          \par
          \cadre{rouge}{À retenir}{
               Si la variable aléatoire $X$ suit une \textbf{loi binomiale} $\mathscr B \left(n ; p\right)$, pour tout entier naturel $k$ compris entre $0$ et $n$, la probabilité que $X$ prenne la valeur $k$ est ~:
               \[ P\left(X=k\right)=\begin{pmatrix} n \\ k \end{pmatrix}p^{k} \left(1-p\right)^{n-k} \]
          }
          \par
          \item %3
          \par
          L'espérance mathématique de $Y$ est ~:
          \par
          $E(Y)=np=50 \times 0,05=2,5$.
          \par
          En moyenne, parmi les 50 tablettes achetées par l'école, 2,5~tablettes présenteront un défaut.
          \par
          \cadre{rouge}{À retenir}{
               Pour une variable aléatoire $X$ qui une \textbf{loi binomiale} $\mathscr B \left(n ; p\right)$, l'\textbf{espérance mathématique} vaut ~:
               \[ E(X)=np \]
          }
          \par
     \end{enumerate}
\end{corrige}

\end{document}
µ
\documentclass[a4paper]{article}

%================================================================================================================================
%
% Packages
%
%================================================================================================================================

\usepackage[T1]{fontenc} 	% pour caractères accentués
\usepackage[utf8]{inputenc}  % encodage utf8
\usepackage[french]{babel}	% langue : français
\usepackage{fourier}			% caractères plus lisibles
\usepackage[dvipsnames]{xcolor} % couleurs
\usepackage{fancyhdr}		% réglage header footer
\usepackage{needspace}		% empêcher sauts de page mal placés
\usepackage{graphicx}		% pour inclure des graphiques
\usepackage{enumitem,cprotect}		% personnalise les listes d'items (nécessaire pour ol, al ...)
\usepackage{hyperref}		% Liens hypertexte
\usepackage{pstricks,pst-all,pst-node,pstricks-add,pst-math,pst-plot,pst-tree,pst-eucl} % pstricks
\usepackage[a4paper,includeheadfoot,top=2cm,left=3cm, bottom=2cm,right=3cm]{geometry} % marges etc.
\usepackage{comment}			% commentaires multilignes
\usepackage{amsmath,environ} % maths (matrices, etc.)
\usepackage{amssymb,makeidx}
\usepackage{bm}				% bold maths
\usepackage{tabularx}		% tableaux
\usepackage{colortbl}		% tableaux en couleur
\usepackage{fontawesome}		% Fontawesome
\usepackage{environ}			% environment with command
\usepackage{fp}				% calculs pour ps-tricks
\usepackage{multido}			% pour ps tricks
\usepackage[np]{numprint}	% formattage nombre
\usepackage{tikz,tkz-tab} 			% package principal TikZ
\usepackage{pgfplots}   % axes
\usepackage{mathrsfs}    % cursives
\usepackage{calc}			% calcul taille boites
\usepackage[scaled=0.875]{helvet} % font sans serif
\usepackage{svg} % svg
\usepackage{scrextend} % local margin
\usepackage{scratch} %scratch
\usepackage{multicol} % colonnes
%\usepackage{infix-RPN,pst-func} % formule en notation polanaise inversée
\usepackage{listings}

%================================================================================================================================
%
% Réglages de base
%
%================================================================================================================================

\lstset{
language=Python,   % R code
literate=
{á}{{\'a}}1
{à}{{\`a}}1
{ã}{{\~a}}1
{é}{{\'e}}1
{è}{{\`e}}1
{ê}{{\^e}}1
{í}{{\'i}}1
{ó}{{\'o}}1
{õ}{{\~o}}1
{ú}{{\'u}}1
{ü}{{\"u}}1
{ç}{{\c{c}}}1
{~}{{ }}1
}


\definecolor{codegreen}{rgb}{0,0.6,0}
\definecolor{codegray}{rgb}{0.5,0.5,0.5}
\definecolor{codepurple}{rgb}{0.58,0,0.82}
\definecolor{backcolour}{rgb}{0.95,0.95,0.92}

\lstdefinestyle{mystyle}{
    backgroundcolor=\color{backcolour},   
    commentstyle=\color{codegreen},
    keywordstyle=\color{magenta},
    numberstyle=\tiny\color{codegray},
    stringstyle=\color{codepurple},
    basicstyle=\ttfamily\footnotesize,
    breakatwhitespace=false,         
    breaklines=true,                 
    captionpos=b,                    
    keepspaces=true,                 
    numbers=left,                    
xleftmargin=2em,
framexleftmargin=2em,            
    showspaces=false,                
    showstringspaces=false,
    showtabs=false,                  
    tabsize=2,
    upquote=true
}

\lstset{style=mystyle}


\lstset{style=mystyle}
\newcommand{\imgdir}{C:/laragon/www/newmc/assets/imgsvg/}
\newcommand{\imgsvgdir}{C:/laragon/www/newmc/assets/imgsvg/}

\definecolor{mcgris}{RGB}{220, 220, 220}% ancien~; pour compatibilité
\definecolor{mcbleu}{RGB}{52, 152, 219}
\definecolor{mcvert}{RGB}{125, 194, 70}
\definecolor{mcmauve}{RGB}{154, 0, 215}
\definecolor{mcorange}{RGB}{255, 96, 0}
\definecolor{mcturquoise}{RGB}{0, 153, 153}
\definecolor{mcrouge}{RGB}{255, 0, 0}
\definecolor{mclightvert}{RGB}{205, 234, 190}

\definecolor{gris}{RGB}{220, 220, 220}
\definecolor{bleu}{RGB}{52, 152, 219}
\definecolor{vert}{RGB}{125, 194, 70}
\definecolor{mauve}{RGB}{154, 0, 215}
\definecolor{orange}{RGB}{255, 96, 0}
\definecolor{turquoise}{RGB}{0, 153, 153}
\definecolor{rouge}{RGB}{255, 0, 0}
\definecolor{lightvert}{RGB}{205, 234, 190}
\setitemize[0]{label=\color{lightvert}  $\bullet$}

\pagestyle{fancy}
\renewcommand{\headrulewidth}{0.2pt}
\fancyhead[L]{maths-cours.fr}
\fancyhead[R]{\thepage}
\renewcommand{\footrulewidth}{0.2pt}
\fancyfoot[C]{}

\newcolumntype{C}{>{\centering\arraybackslash}X}
\newcolumntype{s}{>{\hsize=.35\hsize\arraybackslash}X}

\setlength{\parindent}{0pt}		 
\setlength{\parskip}{3mm}
\setlength{\headheight}{1cm}

\def\ebook{ebook}
\def\book{book}
\def\web{web}
\def\type{web}

\newcommand{\vect}[1]{\overrightarrow{\,\mathstrut#1\,}}

\def\Oij{$\left(\text{O}~;~\vect{\imath},~\vect{\jmath}\right)$}
\def\Oijk{$\left(\text{O}~;~\vect{\imath},~\vect{\jmath},~\vect{k}\right)$}
\def\Ouv{$\left(\text{O}~;~\vect{u},~\vect{v}\right)$}

\hypersetup{breaklinks=true, colorlinks = true, linkcolor = OliveGreen, urlcolor = OliveGreen, citecolor = OliveGreen, pdfauthor={Didier BONNEL - https://www.maths-cours.fr} } % supprime les bordures autour des liens

\renewcommand{\arg}[0]{\text{arg}}

\everymath{\displaystyle}

%================================================================================================================================
%
% Macros - Commandes
%
%================================================================================================================================

\newcommand\meta[2]{    			% Utilisé pour créer le post HTML.
	\def\titre{titre}
	\def\url{url}
	\def\arg{#1}
	\ifx\titre\arg
		\newcommand\maintitle{#2}
		\fancyhead[L]{#2}
		{\Large\sffamily \MakeUppercase{#2}}
		\vspace{1mm}\textcolor{mcvert}{\hrule}
	\fi 
	\ifx\url\arg
		\fancyfoot[L]{\href{https://www.maths-cours.fr#2}{\black \footnotesize{https://www.maths-cours.fr#2}}}
	\fi 
}


\newcommand\TitreC[1]{    		% Titre centré
     \needspace{3\baselineskip}
     \begin{center}\textbf{#1}\end{center}
}

\newcommand\newpar{    		% paragraphe
     \par
}

\newcommand\nosp {    		% commande vide (pas d'espace)
}
\newcommand{\id}[1]{} %ignore

\newcommand\boite[2]{				% Boite simple sans titre
	\vspace{5mm}
	\setlength{\fboxrule}{0.2mm}
	\setlength{\fboxsep}{5mm}	
	\fcolorbox{#1}{#1!3}{\makebox[\linewidth-2\fboxrule-2\fboxsep]{
  		\begin{minipage}[t]{\linewidth-2\fboxrule-4\fboxsep}\setlength{\parskip}{3mm}
  			 #2
  		\end{minipage}
	}}
	\vspace{5mm}
}

\newcommand\CBox[4]{				% Boites
	\vspace{5mm}
	\setlength{\fboxrule}{0.2mm}
	\setlength{\fboxsep}{5mm}
	
	\fcolorbox{#1}{#1!3}{\makebox[\linewidth-2\fboxrule-2\fboxsep]{
		\begin{minipage}[t]{1cm}\setlength{\parskip}{3mm}
	  		\textcolor{#1}{\LARGE{#2}}    
 	 	\end{minipage}  
  		\begin{minipage}[t]{\linewidth-2\fboxrule-4\fboxsep}\setlength{\parskip}{3mm}
			\raisebox{1.2mm}{\normalsize\sffamily{\textcolor{#1}{#3}}}						
  			 #4
  		\end{minipage}
	}}
	\vspace{5mm}
}

\newcommand\cadre[3]{				% Boites convertible html
	\par
	\vspace{2mm}
	\setlength{\fboxrule}{0.1mm}
	\setlength{\fboxsep}{5mm}
	\fcolorbox{#1}{white}{\makebox[\linewidth-2\fboxrule-2\fboxsep]{
  		\begin{minipage}[t]{\linewidth-2\fboxrule-4\fboxsep}\setlength{\parskip}{3mm}
			\raisebox{-2.5mm}{\sffamily \small{\textcolor{#1}{\MakeUppercase{#2}}}}		
			\par		
  			 #3
 	 		\end{minipage}
	}}
		\vspace{2mm}
	\par
}

\newcommand\bloc[3]{				% Boites convertible html sans bordure
     \needspace{2\baselineskip}
     {\sffamily \small{\textcolor{#1}{\MakeUppercase{#2}}}}    
		\par		
  			 #3
		\par
}

\newcommand\CHelp[1]{
     \CBox{Plum}{\faInfoCircle}{À RETENIR}{#1}
}

\newcommand\CUp[1]{
     \CBox{NavyBlue}{\faThumbsOUp}{EN PRATIQUE}{#1}
}

\newcommand\CInfo[1]{
     \CBox{Sepia}{\faArrowCircleRight}{REMARQUE}{#1}
}

\newcommand\CRedac[1]{
     \CBox{PineGreen}{\faEdit}{BIEN R\'EDIGER}{#1}
}

\newcommand\CError[1]{
     \CBox{Red}{\faExclamationTriangle}{ATTENTION}{#1}
}

\newcommand\TitreExo[2]{
\needspace{4\baselineskip}
 {\sffamily\large EXERCICE #1\ (\emph{#2 points})}
\vspace{5mm}
}

\newcommand\img[2]{
          \includegraphics[width=#2\paperwidth]{\imgdir#1}
}

\newcommand\imgsvg[2]{
       \begin{center}   \includegraphics[width=#2\paperwidth]{\imgsvgdir#1} \end{center}
}


\newcommand\Lien[2]{
     \href{#1}{#2 \tiny \faExternalLink}
}
\newcommand\mcLien[2]{
     \href{https~://www.maths-cours.fr/#1}{#2 \tiny \faExternalLink}
}

\newcommand{\euro}{\eurologo{}}

%================================================================================================================================
%
% Macros - Environement
%
%================================================================================================================================

\newenvironment{tex}{ %
}
{%
}

\newenvironment{indente}{ %
	\setlength\parindent{10mm}
}

{
	\setlength\parindent{0mm}
}

\newenvironment{corrige}{%
     \needspace{3\baselineskip}
     \medskip
     \textbf{\textsc{Corrigé}}
     \medskip
}
{
}

\newenvironment{extern}{%
     \begin{center}
     }
     {
     \end{center}
}

\NewEnviron{code}{%
	\par
     \boite{gray}{\texttt{%
     \BODY
     }}
     \par
}

\newenvironment{vbloc}{% boite sans cadre empeche saut de page
     \begin{minipage}[t]{\linewidth}
     }
     {
     \end{minipage}
}
\NewEnviron{h2}{%
    \needspace{3\baselineskip}
    \vspace{0.6cm}
	\noindent \MakeUppercase{\sffamily \large \BODY}
	\vspace{1mm}\textcolor{mcgris}{\hrule}\vspace{0.4cm}
	\par
}{}

\NewEnviron{h3}{%
    \needspace{3\baselineskip}
	\vspace{5mm}
	\textsc{\BODY}
	\par
}

\NewEnviron{margeneg}{ %
\begin{addmargin}[-1cm]{0cm}
\BODY
\end{addmargin}
}

\NewEnviron{html}{%
}

\begin{document}
\meta{url}{/exercices/graphes-bac-blanc-es-sujet-1-maths-cours-2018-spe/}
\meta{pid}{10428}
\meta{titre}{Graphes - Bac blanc ES Sujet 1 - Maths-cours 2018 (spé)}
\meta{type}{exercices}
%
\begin{h2}Exercice 4 (5 points)\end{h2}
\par
\textbf{Candidats ayant suivi l'enseignement de spécialité}
\par
Un appartement comporte 6 pièces notées A, B, C, D, E et F.
\par
Le plan ci-après présente la disposition des pièces ainsi que les portes de communication entre ces pièces.
\par
\begin{center}
     \begin{extern}%width="500" alt="Plan et graphes"
          \includegraphics[width=0.9\textwidth]{images/BBESL-spe-1-1}% gbb 1 unite=1cm
     \end{extern}
\end{center}
\par
Par exemple, il y a une porte de communication entre les pièces A et B mais il n'y en a pas entre les pièces B et E.
\par
La porte donnant accès à l'appartement est sans importance dans le cadre de l'exercice et n'a pas été représentée.
\par
\textit{Toutes les réponses aux questions posées devront être justifiées.}
\par
\begin{enumerate}
     \item %1
     \par
     \begin{enumerate}[label=\alph*.]
          \item %1a
          Traduire la situation à l'aide d'un graphe (G) dont les sommets représentent les pièces et dont les arêtes représentent les portes de communication.
          \item %1b
          Le graphe (G) est-il connexe ? complet ?
     \end{enumerate}
     \par
     \item %2
     \par
     \begin{enumerate}[label=\alph*.]
          \item %2a
          Est-il possible de parcourir l'appartement en empruntant chaque porte une fois et une seule ?\\
          Si oui, donner un exemple d'un tel chemin.
          \item %2b
          Est-il possible de parcourir l'appartement en empruntant chaque porte une fois et une seule et \textit{en partant et en arrivant dans la même pièce} ?\\
          Si oui, donner un exemple d'un tel chemin.
     \end{enumerate}
     \par
     \item %3
     Déterminer la matrice de transition $M$ associée au graphe précédent en prenant les sommets par ordre alphabétique.
     \par
     \item %4
     \par
     \`A l'aide d'une calculatrice on trouve~:
     \[ M^3 = \begin{pmatrix}
          2 &5 &2 &3 &7 &4 \\
          5 &0 &5 &2 &2 &2 \\
          2 &5 &2 &4 &7 &3 \\
          3 &2 &4 &2 &5 &2 \\
          7 &2 &7 &5 &4 &5\\
     4 &2 &3 &2 &5 &2  \end{pmatrix} \]
     \begin{enumerate}[label=\alph*.]
          \item %4a
          \par
          Combien existe-t-il de chemins permettant d'aller de la pièce A à la pièce D en empruntant exactement trois portes ?\\
          Donner la liste de ces chemins.
          \par
          \item %4b
          \par
          Est-il toujours possible de relier deux pièces différentes en empruntant exactement trois portes ?
          \par
     \end{enumerate}
     \par
     \item %5
     \par
     \begin{enumerate}[label=\alph*.]
          \item %5a
          \par
          Montrer qu'il existe au moins un sous-graphe \textit{complet} de (G) d'ordre 3.
          \par
          \item %4b
          \par
          Le propriétaire souhaite repeindre l'appartement en respectant les règles suivantes~:
          \par
          \begin{itemize}
               \item %
               chaque pièce sera repeinte avec une couleur unique~;
               \item %
               deux pièces adjacentes, c'est à dire reliées par une porte, seront repeintes avec des couleurs différentes.
          \end{itemize}
          \par
          Pourra-t-il réaliser ces objectifs en utilisant seulement trois couleurs ?
     \end{enumerate}
\end{enumerate}
\begin{corrige}
     \begin{enumerate}
          \item %1
          \par
          \begin{enumerate}[label=\alph*.]
               \item %1a
               On place d'abord les sommets A, B, C, D, E et F qui représentent les pièces et on relie, par des arêtes, les pièces qui communiquent~: A-B, A-E, A-F, B-C, C-D, C-E, D-E, E-F.
               \par
               \begin{center}
                    \begin{extern}%width="2_0" alt="Graphe communications entre pièces"
                         \psset{unit=0.7cm}
                         \begin{pspicture}(10,7)
                              \rput(2,6){\circlenode{A}{A}}
                              \rput(6,6){\circlenode{B}{B}}
                              \rput(9,4){\circlenode{C}{C}}
                              \rput(8,1){\circlenode{D}{D}}
                              \rput(5,4){\circlenode{E}{E}}
                              \rput(1,2){\circlenode{F}{F}}
                              \ncline{A}{B}
                              \ncline{A}{E}
                              \ncline{B}{C}
                              \ncline{C}{D}
                              \ncline{D}{E}
                              \ncline{A}{F}
                              \ncline{E}{F}
                              \ncline{E}{C}
                         \end{pspicture}
                    \end{extern}
               \end{center}
               \par
               \item %1b
               Le graphe est \textbf{connexe}. En effet, un graphe est connexe si deux sommets quelconques peuvent être reliés par une chaîne ce qui est le cas ici.
               \par
               Le graphe \textbf{n'est pas complet}. Un graphe non orienté est complet si et seulement si tous ses sommets sont reliés par une arête. Ce n'est pas le cas ici pour A et C par exemple.
               \cadre{rouge}{À retenir}{
                    Un graphe est \textbf{complet} si et seulement si tous ses sommets sont deux à deux adjacents (c'est à dire reliés par une arête).
                    \par
                    Un graphe est \textbf{connexe} si et seulement si deux sommets quelconques peuvent être reliés par une chaîne (intuitivement cela signifie que le graphe est en \og un seul morceau \fg{}).
               }
          \end{enumerate}
          \par
          \item %2
          \par
          \begin{enumerate}[label=\alph*.]
               \item %2a
               On recherche s'il existe une \textbf{chaîne eulérienne}, c'est à dire une chaîne qui contient une fois et une seule chacune des arêtes du graphe.
               \par
               D'après le théorème d'Euler, un graphe connexe contient une chaîne eulérienne si et seulement s'il possède \textbf{0 ou 2 sommets de degré impair}.
               \par
               Le degré de chacun des sommets est donné par le tableau ci-après~:
               \par
               \begin{center}
                    \begin{tabular}{|l|c|c|c|c|c|c|c|c|}%class="compact"
                         \hline
                         Sommet & A & B & C & D & E & F  \\
                         \hline
                         Degré & 3 & 2 & 3 & 2 & 4 & 2 \\
                         \hline
                    \end{tabular}
               \end{center}
               \par
               Le graphe (G) possède \textbf{deux} sommets de degré impair~: A et C.
               \par
               Il est donc \textbf{possible} de parcourir l'appartement en empruntant chacune des 8 portes une fois et une seule, par exemple en suivant le trajet~: A-B-C-D-E-F-A-E-C.
               \par
               \cadre{rouge}{À retenir}{
                    Une \textbf{chaîne eulérienne} est une chaîne qui contient une fois et une seule chacune des arêtes du graphe.
                    \par
                    Trouver un chemin qui emprunte chaque arête une fois et une seule revient à trouver une chaîne eulérienne.
                    \par
                    Un graphe connexe admet une \textbf{chaîne eulérienne} si et seulement s'il possède \textbf{0 ou 2 sommet(s) de degré impair}.
               }
               \par
               \item %2b
               Dans cette question, on recherche l'existence d'un \textbf{cycle eulérien} (un cycle est une chaîne fermée).
               \par
               Or, d'après le théorème d'Euler, un graphe connexe contient un cycle eulérien si et seulement s'il ne possède \textbf{aucun sommet de degré impair}.
               \par
               Ici, A et C sont de degré impair. Toute chaîne eulérienne aura pour extrémités A et C et ne sera donc pas un cycle.
               \par
               Par conséquent, \textbf{il n'est pas possible} de parcourir l'appartement en empruntant chaque porte une fois et une seule et en partant et en arrivant dans la même pièce.
               \par
               \cadre{rouge}{À retenir}{
                    Un \textbf{cycle eulérien} est une chaîne \textbf{fermée} qui contient une fois et une seule chacune des arêtes du graphe.
                    \par
                    Trouver un chemin qui emprunte chaque arête une fois et une seule et dont \textit{les sommets de départ et d'arrivée sont identiques}  revient à trouver un cycle eulérien.
                    \par
                    Un graphe connexe admet un \textbf{cycle eulérien} si et seulement s'il ne possède \textbf{aucun sommet de degré impair}.
               }
               \par
          \end{enumerate}
          \par
          \item %3
          Numérotons les pièces A: 1, B: 2, C: 3, D: 4, E: 5, F: 6.\\
          La matrice de transition $M$ associée au graphe précédent s'obtient en plaçant à la $i$-ième ligne et à la $j$-ième colonne~:
          \begin{itemize}
               \item %
               un \og 1 \fg{} si les pièces numérotées $i$ et $j$ sont reliés par une arête~;
               \item %
               un \og 0 \fg{} sinon.
          \end{itemize}
          \par
          On obtient alors la matrice~:
          \par
          \[ M = \begin{pmatrix}
               0 &1 &0 &0 &1 &1 \\
               1 &0 &1 &0 &0 &0 \\
               0 &1 &0 &1 &1 &0 \\
               0 &0 &1 &0 &1 &0 \\
               1 &0 &1 &1 &0 &1\\
          1 &0 &0 &0 &1 &0  \end{pmatrix} \]
          \par
          \item %4
          \par
          \begin{enumerate}[label=\alph*.]
               \item %4a
               \par
               Le coefficient de $M^3$ situé à la $i$-ième ligne et à la $j$-ième colonne indique le nombre de chemins de trois arêtes menant du sommet numéro $i$ au sommet numéro $j$.
               \par
               Ici, le coefficient situé à la première ligne et à la quatrième colonne est 3. Il y a donc 3 chemins permettant d'aller de la pièce A à la pièce D en empruntant exactement trois portes.
               \par
               \`A l'aide du graphe, on trouve les chemins~: A-B-C-D, A-E-C-D et A-F-E-D.
               \par
               \cadre{rouge}{À retenir}{
                    Le coefficient de la matrice $M^n$ situé à la $i$-ième ligne et à la $j$-ième colonne correspond au nombre de chemins \textbf{de longueur} $\bm{n}$ menant du sommet numéro $i$ au sommet numéro $j$.
               }
               \par
               \item %4b
               \par
               La matrice $M^3$ comporte un unique coefficient nul situé en ligne 2 et en colonne 2. Cela signifie qu'il n'est pas possible de partir de la pièce B pour revenir à la pièce B en empruntant exactement 3 portes mais que, mis à part ce cas, il est toujours possible de joindre deux pièces en empruntant exactement 3 portes.
               \par
               Comme l'énoncé précise deux pièces \textbf{différentes}, il est effectivement \textbf{toujours possible} de joindre deux pièces différentes en empruntant exactement trois portes.
          \end{enumerate}
          \par
          \item %5
          \par
          \begin{enumerate}[label=\alph*.]
               \item %5a
               Considérons le sous-graphe constitué des sommets A, E et F. Chacun de ces trois sommets est relié aux deux autres, donc \textbf{ce sous-graphe est complet}. Le sous-graphe constitué de C, E et D est lui-aussi complet.
               \par
               \item %5b
               \par
               Il est possible de repeindre les pièces en respectant les consignes de l'énoncé et avec seulement trois couleurs.
               \par
               Le sous-graphe A, E et F étant complet, il faudra nécessairement trois couleurs différentes pour peindre ces trois pièces.\\
               Il en est de même pour les pièces C, E et D.
               \par
               En respectant ces contraintes, il est facile de trouver une solution au problème posé~; par exemple (mais il y a d'autres solutions ...)~:
               \par
               Couleur 1~: E, B\\
               Couleur 2~: A, C\\
               Couleur 3~: F, D.\\
               \par
          \end{enumerate}
     \end{enumerate}
\end{corrige}

\end{document}
µ
\documentclass[a4paper]{article}

%================================================================================================================================
%
% Packages
%
%================================================================================================================================

\usepackage[T1]{fontenc} 	% pour caractères accentués
\usepackage[utf8]{inputenc}  % encodage utf8
\usepackage[french]{babel}	% langue : français
\usepackage{fourier}			% caractères plus lisibles
\usepackage[dvipsnames]{xcolor} % couleurs
\usepackage{fancyhdr}		% réglage header footer
\usepackage{needspace}		% empêcher sauts de page mal placés
\usepackage{graphicx}		% pour inclure des graphiques
\usepackage{enumitem,cprotect}		% personnalise les listes d'items (nécessaire pour ol, al ...)
\usepackage{hyperref}		% Liens hypertexte
\usepackage{pstricks,pst-all,pst-node,pstricks-add,pst-math,pst-plot,pst-tree,pst-eucl} % pstricks
\usepackage[a4paper,includeheadfoot,top=2cm,left=3cm, bottom=2cm,right=3cm]{geometry} % marges etc.
\usepackage{comment}			% commentaires multilignes
\usepackage{amsmath,environ} % maths (matrices, etc.)
\usepackage{amssymb,makeidx}
\usepackage{bm}				% bold maths
\usepackage{tabularx}		% tableaux
\usepackage{colortbl}		% tableaux en couleur
\usepackage{fontawesome}		% Fontawesome
\usepackage{environ}			% environment with command
\usepackage{fp}				% calculs pour ps-tricks
\usepackage{multido}			% pour ps tricks
\usepackage[np]{numprint}	% formattage nombre
\usepackage{tikz,tkz-tab} 			% package principal TikZ
\usepackage{pgfplots}   % axes
\usepackage{mathrsfs}    % cursives
\usepackage{calc}			% calcul taille boites
\usepackage[scaled=0.875]{helvet} % font sans serif
\usepackage{svg} % svg
\usepackage{scrextend} % local margin
\usepackage{scratch} %scratch
\usepackage{multicol} % colonnes
%\usepackage{infix-RPN,pst-func} % formule en notation polanaise inversée
\usepackage{listings}

%================================================================================================================================
%
% Réglages de base
%
%================================================================================================================================

\lstset{
language=Python,   % R code
literate=
{á}{{\'a}}1
{à}{{\`a}}1
{ã}{{\~a}}1
{é}{{\'e}}1
{è}{{\`e}}1
{ê}{{\^e}}1
{í}{{\'i}}1
{ó}{{\'o}}1
{õ}{{\~o}}1
{ú}{{\'u}}1
{ü}{{\"u}}1
{ç}{{\c{c}}}1
{~}{{ }}1
}


\definecolor{codegreen}{rgb}{0,0.6,0}
\definecolor{codegray}{rgb}{0.5,0.5,0.5}
\definecolor{codepurple}{rgb}{0.58,0,0.82}
\definecolor{backcolour}{rgb}{0.95,0.95,0.92}

\lstdefinestyle{mystyle}{
    backgroundcolor=\color{backcolour},   
    commentstyle=\color{codegreen},
    keywordstyle=\color{magenta},
    numberstyle=\tiny\color{codegray},
    stringstyle=\color{codepurple},
    basicstyle=\ttfamily\footnotesize,
    breakatwhitespace=false,         
    breaklines=true,                 
    captionpos=b,                    
    keepspaces=true,                 
    numbers=left,                    
xleftmargin=2em,
framexleftmargin=2em,            
    showspaces=false,                
    showstringspaces=false,
    showtabs=false,                  
    tabsize=2,
    upquote=true
}

\lstset{style=mystyle}


\lstset{style=mystyle}
\newcommand{\imgdir}{C:/laragon/www/newmc/assets/imgsvg/}
\newcommand{\imgsvgdir}{C:/laragon/www/newmc/assets/imgsvg/}

\definecolor{mcgris}{RGB}{220, 220, 220}% ancien~; pour compatibilité
\definecolor{mcbleu}{RGB}{52, 152, 219}
\definecolor{mcvert}{RGB}{125, 194, 70}
\definecolor{mcmauve}{RGB}{154, 0, 215}
\definecolor{mcorange}{RGB}{255, 96, 0}
\definecolor{mcturquoise}{RGB}{0, 153, 153}
\definecolor{mcrouge}{RGB}{255, 0, 0}
\definecolor{mclightvert}{RGB}{205, 234, 190}

\definecolor{gris}{RGB}{220, 220, 220}
\definecolor{bleu}{RGB}{52, 152, 219}
\definecolor{vert}{RGB}{125, 194, 70}
\definecolor{mauve}{RGB}{154, 0, 215}
\definecolor{orange}{RGB}{255, 96, 0}
\definecolor{turquoise}{RGB}{0, 153, 153}
\definecolor{rouge}{RGB}{255, 0, 0}
\definecolor{lightvert}{RGB}{205, 234, 190}
\setitemize[0]{label=\color{lightvert}  $\bullet$}

\pagestyle{fancy}
\renewcommand{\headrulewidth}{0.2pt}
\fancyhead[L]{maths-cours.fr}
\fancyhead[R]{\thepage}
\renewcommand{\footrulewidth}{0.2pt}
\fancyfoot[C]{}

\newcolumntype{C}{>{\centering\arraybackslash}X}
\newcolumntype{s}{>{\hsize=.35\hsize\arraybackslash}X}

\setlength{\parindent}{0pt}		 
\setlength{\parskip}{3mm}
\setlength{\headheight}{1cm}

\def\ebook{ebook}
\def\book{book}
\def\web{web}
\def\type{web}

\newcommand{\vect}[1]{\overrightarrow{\,\mathstrut#1\,}}

\def\Oij{$\left(\text{O}~;~\vect{\imath},~\vect{\jmath}\right)$}
\def\Oijk{$\left(\text{O}~;~\vect{\imath},~\vect{\jmath},~\vect{k}\right)$}
\def\Ouv{$\left(\text{O}~;~\vect{u},~\vect{v}\right)$}

\hypersetup{breaklinks=true, colorlinks = true, linkcolor = OliveGreen, urlcolor = OliveGreen, citecolor = OliveGreen, pdfauthor={Didier BONNEL - https://www.maths-cours.fr} } % supprime les bordures autour des liens

\renewcommand{\arg}[0]{\text{arg}}

\everymath{\displaystyle}

%================================================================================================================================
%
% Macros - Commandes
%
%================================================================================================================================

\newcommand\meta[2]{    			% Utilisé pour créer le post HTML.
	\def\titre{titre}
	\def\url{url}
	\def\arg{#1}
	\ifx\titre\arg
		\newcommand\maintitle{#2}
		\fancyhead[L]{#2}
		{\Large\sffamily \MakeUppercase{#2}}
		\vspace{1mm}\textcolor{mcvert}{\hrule}
	\fi 
	\ifx\url\arg
		\fancyfoot[L]{\href{https://www.maths-cours.fr#2}{\black \footnotesize{https://www.maths-cours.fr#2}}}
	\fi 
}


\newcommand\TitreC[1]{    		% Titre centré
     \needspace{3\baselineskip}
     \begin{center}\textbf{#1}\end{center}
}

\newcommand\newpar{    		% paragraphe
     \par
}

\newcommand\nosp {    		% commande vide (pas d'espace)
}
\newcommand{\id}[1]{} %ignore

\newcommand\boite[2]{				% Boite simple sans titre
	\vspace{5mm}
	\setlength{\fboxrule}{0.2mm}
	\setlength{\fboxsep}{5mm}	
	\fcolorbox{#1}{#1!3}{\makebox[\linewidth-2\fboxrule-2\fboxsep]{
  		\begin{minipage}[t]{\linewidth-2\fboxrule-4\fboxsep}\setlength{\parskip}{3mm}
  			 #2
  		\end{minipage}
	}}
	\vspace{5mm}
}

\newcommand\CBox[4]{				% Boites
	\vspace{5mm}
	\setlength{\fboxrule}{0.2mm}
	\setlength{\fboxsep}{5mm}
	
	\fcolorbox{#1}{#1!3}{\makebox[\linewidth-2\fboxrule-2\fboxsep]{
		\begin{minipage}[t]{1cm}\setlength{\parskip}{3mm}
	  		\textcolor{#1}{\LARGE{#2}}    
 	 	\end{minipage}  
  		\begin{minipage}[t]{\linewidth-2\fboxrule-4\fboxsep}\setlength{\parskip}{3mm}
			\raisebox{1.2mm}{\normalsize\sffamily{\textcolor{#1}{#3}}}						
  			 #4
  		\end{minipage}
	}}
	\vspace{5mm}
}

\newcommand\cadre[3]{				% Boites convertible html
	\par
	\vspace{2mm}
	\setlength{\fboxrule}{0.1mm}
	\setlength{\fboxsep}{5mm}
	\fcolorbox{#1}{white}{\makebox[\linewidth-2\fboxrule-2\fboxsep]{
  		\begin{minipage}[t]{\linewidth-2\fboxrule-4\fboxsep}\setlength{\parskip}{3mm}
			\raisebox{-2.5mm}{\sffamily \small{\textcolor{#1}{\MakeUppercase{#2}}}}		
			\par		
  			 #3
 	 		\end{minipage}
	}}
		\vspace{2mm}
	\par
}

\newcommand\bloc[3]{				% Boites convertible html sans bordure
     \needspace{2\baselineskip}
     {\sffamily \small{\textcolor{#1}{\MakeUppercase{#2}}}}    
		\par		
  			 #3
		\par
}

\newcommand\CHelp[1]{
     \CBox{Plum}{\faInfoCircle}{À RETENIR}{#1}
}

\newcommand\CUp[1]{
     \CBox{NavyBlue}{\faThumbsOUp}{EN PRATIQUE}{#1}
}

\newcommand\CInfo[1]{
     \CBox{Sepia}{\faArrowCircleRight}{REMARQUE}{#1}
}

\newcommand\CRedac[1]{
     \CBox{PineGreen}{\faEdit}{BIEN R\'EDIGER}{#1}
}

\newcommand\CError[1]{
     \CBox{Red}{\faExclamationTriangle}{ATTENTION}{#1}
}

\newcommand\TitreExo[2]{
\needspace{4\baselineskip}
 {\sffamily\large EXERCICE #1\ (\emph{#2 points})}
\vspace{5mm}
}

\newcommand\img[2]{
          \includegraphics[width=#2\paperwidth]{\imgdir#1}
}

\newcommand\imgsvg[2]{
       \begin{center}   \includegraphics[width=#2\paperwidth]{\imgsvgdir#1} \end{center}
}


\newcommand\Lien[2]{
     \href{#1}{#2 \tiny \faExternalLink}
}
\newcommand\mcLien[2]{
     \href{https~://www.maths-cours.fr/#1}{#2 \tiny \faExternalLink}
}

\newcommand{\euro}{\eurologo{}}

%================================================================================================================================
%
% Macros - Environement
%
%================================================================================================================================

\newenvironment{tex}{ %
}
{%
}

\newenvironment{indente}{ %
	\setlength\parindent{10mm}
}

{
	\setlength\parindent{0mm}
}

\newenvironment{corrige}{%
     \needspace{3\baselineskip}
     \medskip
     \textbf{\textsc{Corrigé}}
     \medskip
}
{
}

\newenvironment{extern}{%
     \begin{center}
     }
     {
     \end{center}
}

\NewEnviron{code}{%
	\par
     \boite{gray}{\texttt{%
     \BODY
     }}
     \par
}

\newenvironment{vbloc}{% boite sans cadre empeche saut de page
     \begin{minipage}[t]{\linewidth}
     }
     {
     \end{minipage}
}
\NewEnviron{h2}{%
    \needspace{3\baselineskip}
    \vspace{0.6cm}
	\noindent \MakeUppercase{\sffamily \large \BODY}
	\vspace{1mm}\textcolor{mcgris}{\hrule}\vspace{0.4cm}
	\par
}{}

\NewEnviron{h3}{%
    \needspace{3\baselineskip}
	\vspace{5mm}
	\textsc{\BODY}
	\par
}

\NewEnviron{margeneg}{ %
\begin{addmargin}[-1cm]{0cm}
\BODY
\end{addmargin}
}

\NewEnviron{html}{%
}

\begin{document}
\meta{url}{/exercices/qcm-bac-blanc-es-l-sujet-2-maths-cours-2018/}
\meta{pid}{10437}
\meta{titre}{QCM - Bac blanc ES/L Sujet 2 - Maths-cours 2018}
\meta{type}{exercices}
%
\begin{h2}Exercice 1 (5 points)\end{h2}
\par
\emph{Cet exercice est un questionnaire à choix multiples (QCM). Les questions sont indépendantes les unes des autres. Pour chacune des questions suivantes, une seule des trois réponses proposées est exacte.  \\Indiquer sur la copie le numéro de la question et la réponse exacte \textbf{en justifiant le choix effectué}. }
\par
\emph{\textbf{Toute réponse non justifiée ne sera pas prise en compte.}}
\par
\begin{itemize}
     \item \textbf{Question 1 :}
     \par
     Soient A et B deux événements d'une expérience aléatoire tels que $p(A)=0,7$, $p(B)=0,5$ et $p(A \cap B)=0,4$.
     \par
     Alors :
     \par
     \textbf{a.~~} $p(A \cup B)=0,9$ \\
     \textbf{b.~~} $p_A(B)=0,7$ \\
     \textbf{c.~~} $p_B(A)=0,8$ \\
     \par
     \item \textbf{Question 2 :}
     \par
     Soit la fonction $f$ définie sur $\mathbb{R}$ par $f(x)=x^3+2x^2-x+1$.
     \par
     Une équation de la tangente à la courbe représentative de $f$ au point $A(0~;~1)$ est :
     \par
     \textbf{a.~~} $y=-x+1$ \\
     \textbf{b.~~} $y=x-1$ \\
     \textbf{c.~~} $y=3x^2+4x-1$ \\
     \par
     \item \textbf{Question 3 :}
     \par
     On lance trois dés équilibrés à six faces. La probabilité $p$ d'obtenir au moins un \og 6 \fg{} (arrondie à $10^{-2}$) est :
     \par
     \textbf{a.~~} $p \approx 0,17$ \\
     \textbf{b.~~} $p \approx 0,42$  \\
     \textbf{c.~~} $p \approx 0,84$ \\
     \par
     \item \textbf{Question 4 :}
     \par
     $f$ est une fonction définie sur l'intervalle $[0~;~10]$ dont le tableau de variations est donné ci-après :
     \begin{center}
          \begin{extern}%width="400" alt="Tableau de Variations"
               \begin{tikzpicture}[scale=0.875]
                    % Styles
                    \tikzstyle{cadre}=[thin]
                    \tikzstyle{fleche}=[->,>=latex,thin]
                    \tikzstyle{nondefini}=[lightgray]
                    % Dimensions Modifiables
                    \def\Lrg{1.5}
                    \def\HtX{1}
                    \def\HtY{0.5}
                    % Dimensions Calculées
                    \def\lignex{-0.5*\HtX}
                    \def\lignef{-1.5*\HtX}
                    \def\separateur{-0.5*\Lrg}
                    % Largeur du tableau
                    \def\gauche{-1.5*\Lrg}
                    \def\droite{4.5*\Lrg}
                    % Hauteur du tableau
                    \def\haut{0.5*\HtX}
                    \def\bas{-1.5*\HtX-2*\HtY}
                    % Ligne de l'abscisse : x
                    \node at (-1*\Lrg,0) {$x$};
                    \node at (0*\Lrg,0) {$0$};
                    \node at (2*\Lrg,0) {$3$};
                    \node at (4*\Lrg,0) {$10$};
                    % Ligne de la fonction : f(x)
                    \node  at (-1*\Lrg,{-1*\HtX+(-1)*\HtY}) {$f(x)$};
                    \node (f1) at (0*\Lrg,{-1*\HtX+(0)*\HtY}) {$5$};
                    \node (f2) at (2*\Lrg,{-1*\HtX+(-2)*\HtY}) {$-2$};
                    \node (f3) at (4*\Lrg,{-1*\HtX+(0)*\HtY}) {$1$};
                    % Flèches
                    \draw[fleche] (f1) -- (f2);
                    \draw[fleche] (f2) -- (f3);
                    % Encadrement
                    \draw[cadre] (\separateur,\haut) -- (\separateur,\bas);
                    \draw[cadre] (\gauche,\haut) rectangle  (\droite,\bas);
                    \draw[cadre] (\gauche,\lignex) -- (\droite,\lignex);
               \end{tikzpicture}
          \end{extern}
     \end{center}
     L'équation $f(x)=3$ :
     \par
     \textbf{a.~~} n'admet aucune solution sur l'intervalle $[0~;~10]$ \\
     \textbf{b.~~} admet une unique solution sur l'intervalle $[0~;~10]$  \\
     \textbf{c.~~} admet deux solutions sur l'intervalle $[0~;~10]$ \\
     \par
     \item \textbf{Question 5 :}
     \par
     On considère la suite $(u_n)$ définie par $u_0=1$ et pour tout entier naturel $n$ :
     \[u_{n+1}=2u_n.\]
     \par
     La somme $S=u_0+u_1+u_2+\ \cdots\ +u_{10}$ vaut :
     \par
     \textbf{a.~~} $S=1\ 023$ \\
     \textbf{b.~~} $S=2\ 047$  \\
     \textbf{c.~~} $S=4\ 095$ \\
     \par
\end{itemize}
\begin{corrige}
     \begin{itemize}
          % =============================================================================================================================
          \item \textbf{Question 1 :}
          \par
          Réponse correcte :\quad\textbf{ c.}
          \par
          $p_B(A)=\dfrac{p(A \cap B)}{p(B)}=\dfrac{0,4}{0,5}=0,8$.
          \cadre{bleu}{Remarque}{
               Les réponses \textbf{a.} et \textbf{b.} sont incorrectes. En effet :
               $p(A \cup B) = p(A) + p(B) - p(A \cap B)$\\
               $\phantom{p(A \cup B)}=  0,7 + 0,5 - 0,4 = 0,8.$
               \par
               $ p_A(B)=\dfrac{p(A \cap B)}{p(A)}=\dfrac{0,4}{0,7}=\dfrac{4}{7}$.
          }
          \cadre{rouge}{À retenir}{
               Quels que soient les événements $A$ et $B$ :
               \begin{itemize}
                    \item
                    $p(A \cup B) = p(A) + p(B) - p(A \cap B)$
                    \item
                    $p_A(B)=\dfrac{p(A \cap B)}{p(A)}$.
               \end{itemize}
          }
          \par
          % =============================================================================================================================
          \item \textbf{Question 2 :}
          \par
          Réponse correcte :\quad\textbf{ a.}
          \par
          $f$ est une fonction polynôme donc $f$ est dérivable sur $\mathbb{R}$ et :
          \par
          $f'(x)=3x^2-4x-1$.
          \par
          L'équation réduite de la tangente à la courbe représentative de $f$ au point $A$ d'abscisse $0$ est :
          \par
          $y=f'(0)(x-0)+f(0)$.
          \par
          Or:
          \par
          $f(0)=0^3+2 \times 0^2 - 0 + 1 =1$
          \par
          $f'(0)=3 \times 0^2 - 4 \times 0 - 1 = -1$.
          \par
          L'équation cherchée est donc :
          \par
          $y=- 1(x-0)+1$
          \par
          $y=- x+1$.
          \par
          \cadre{rouge}{À retenir}{
               L'équation réduite de la tangente à la courbe représentative de $f$ au point d'\textbf{abscisse} $\bm{a}$ est :
               \[ y=f'(a)(x-a)+f(a). \]
          }
          \par
          % =============================================================================================================================
          \item \textbf{Question 3 :}
          \par
          Réponse correcte :\quad\textbf{ b.}
          \par
          Soit $X$ la variable aléatoire comptabilisant le nombre de \og 6 \fg{} obtenus.
          \par
          $X$ suit une loi binomiale de paramètres $n=3$ (nombre de dés) et $p=\dfrac{1}{6}$ (probabilité d'obtenir un \og 6 \fg{})
          \par
          La probabilité demandée est la probabilité de l'événement ${(X \geqslant 1)}$. L'événement contraire de ${(X \geqslant 1)}$ est ${(X < 1)}$ qui équivaut à ${(X = 0)}$.
          \par
          Par conséquent :
          \par
          $p=p(X \geqslant 1)=1 - p(X=0)$.
          \par
          Or:
          \par
          $p(X=0) = \begin{pmatrix} 3 \\ 0 \end{pmatrix} \times \left(\dfrac{1}{6}\right)^0 \times \left(\dfrac{5}{6}\right)^3$\nosp$ = \left(\dfrac{5}{6}\right)^3  \approx  0,58 $ (à $10^{-2}$ près).
          \cadre{bleu}{Remarque}{
               On peut également utiliser la calculatrice pour calculer $p(X=0)$ (par exemple BinomFdP(3, 1/6, 0) sur TI ou BinomialPD(0, 3, 1/6) sur Casio).
          }
          Par conséquent :
          \par
          $p \approx 0,42$ (à $10^{-2}$ près)
          \par
          \cadre{rouge}{À retenir}{
               \par
               L'événement \textbf{contraire} de l'événement \og obtenir \textbf{au moins un} six \fg{} est \og n'obtenir \textbf{aucun} six \fg{}.
               \par
          }
          \par
          %=============================================================================================================================
          \item \textbf{Question 4 :}
          \par
          Réponse correcte :\quad\textbf{ b.}
          \par
          Sur l'intervalle $[0~;~3]$, $f$ est \textbf{continue} et \textbf{strictement décroissante}. 3 appartient à l'intervalle $[-2~;~5]$ donc l'équation ${f(x)=3}$ admet une unique solution sur l'intervalle $[0~;~3]$ (\textit{théorème de la bijection} aussi appelé \textit{corollaire du théorème des valeurs intermédiaires}).
          \cadre{rouge}{Bien rédiger}{
               Pour prouver l'\textbf{existence} et l'\textbf{unicité} d'une solution il est important de préciser que :
               \begin{itemize}
                    \item la fonction $f$ est \textbf{continue},
                    \item la fonction $f$ est \textbf{strictement monotone}.
               \end{itemize}
          }
          Sur l'intervalle $[3~;~10]$, le maximum de $f$ est 1 donc l'équation ${f(x)=3}$ n'a pas de solution sur cet intervalle.
          \cadre{rouge}{Bien rédiger}{
               Pour montrer que l'équation $f(x)=k$ \textbf{n'admet pas de solution} sur un intervalle $I$, il suffit d'indiquer que le maximum de $f$ sur $I$ est strictement inférieur à $k$ ou que le minimum de $f$ sur $I$ est strictement supérieur à $k$.
               \par
               On n'utilise pas, dans ce cas, le théorème des valeurs intermédiaires (que l'on emploie, au contraire, lorsque l'on souhaite prouver qu'il y a une ou plusieurs solution(s) sur un intervalle).
          }
          Par conséquent, l'équation $f(x)=3$ admet une unique solution sur l'intervalle $[0~;~10]$.
          \par
          %=============================================================================================================================
          \item \textbf{Question 5 :}
          \par
          Réponse correcte :\quad\textbf{ b.}
          \par
          La relation $u_{n+1}=2u_n$, pour tout entier naturel $n$, montre que la suite $(u_n)$ est une suite géométrique de raison $q=2$.
          \par
          On a donc, pour tout entier naturel $n$ :
          \[ u_n=u_0q^n=2^n \]
          La somme $S$ vaut alors :
          \par
          $S=1+2+2^2+\cdots+2^{10}=\dfrac{1-2^{11}}{1-2}$\nosp$=2^{11}-1=2\ 047$.
          \cadre{rouge}{À retenir}{
               La formule suivante permet de calculer la somme des premiers termes d'une suite géométrique :
               \[ 1+q+q^2+\cdots+q^{n}=\dfrac{1-q^{n+1}}{1-q}. \]
          }
     \end{itemize}
\end{corrige}

\end{document}
µ
\documentclass[a4paper]{article}

%================================================================================================================================
%
% Packages
%
%================================================================================================================================

\usepackage[T1]{fontenc} 	% pour caractères accentués
\usepackage[utf8]{inputenc}  % encodage utf8
\usepackage[french]{babel}	% langue : français
\usepackage{fourier}			% caractères plus lisibles
\usepackage[dvipsnames]{xcolor} % couleurs
\usepackage{fancyhdr}		% réglage header footer
\usepackage{needspace}		% empêcher sauts de page mal placés
\usepackage{graphicx}		% pour inclure des graphiques
\usepackage{enumitem,cprotect}		% personnalise les listes d'items (nécessaire pour ol, al ...)
\usepackage{hyperref}		% Liens hypertexte
\usepackage{pstricks,pst-all,pst-node,pstricks-add,pst-math,pst-plot,pst-tree,pst-eucl} % pstricks
\usepackage[a4paper,includeheadfoot,top=2cm,left=3cm, bottom=2cm,right=3cm]{geometry} % marges etc.
\usepackage{comment}			% commentaires multilignes
\usepackage{amsmath,environ} % maths (matrices, etc.)
\usepackage{amssymb,makeidx}
\usepackage{bm}				% bold maths
\usepackage{tabularx}		% tableaux
\usepackage{colortbl}		% tableaux en couleur
\usepackage{fontawesome}		% Fontawesome
\usepackage{environ}			% environment with command
\usepackage{fp}				% calculs pour ps-tricks
\usepackage{multido}			% pour ps tricks
\usepackage[np]{numprint}	% formattage nombre
\usepackage{tikz,tkz-tab} 			% package principal TikZ
\usepackage{pgfplots}   % axes
\usepackage{mathrsfs}    % cursives
\usepackage{calc}			% calcul taille boites
\usepackage[scaled=0.875]{helvet} % font sans serif
\usepackage{svg} % svg
\usepackage{scrextend} % local margin
\usepackage{scratch} %scratch
\usepackage{multicol} % colonnes
%\usepackage{infix-RPN,pst-func} % formule en notation polanaise inversée
\usepackage{listings}

%================================================================================================================================
%
% Réglages de base
%
%================================================================================================================================

\lstset{
language=Python,   % R code
literate=
{á}{{\'a}}1
{à}{{\`a}}1
{ã}{{\~a}}1
{é}{{\'e}}1
{è}{{\`e}}1
{ê}{{\^e}}1
{í}{{\'i}}1
{ó}{{\'o}}1
{õ}{{\~o}}1
{ú}{{\'u}}1
{ü}{{\"u}}1
{ç}{{\c{c}}}1
{~}{{ }}1
}


\definecolor{codegreen}{rgb}{0,0.6,0}
\definecolor{codegray}{rgb}{0.5,0.5,0.5}
\definecolor{codepurple}{rgb}{0.58,0,0.82}
\definecolor{backcolour}{rgb}{0.95,0.95,0.92}

\lstdefinestyle{mystyle}{
    backgroundcolor=\color{backcolour},   
    commentstyle=\color{codegreen},
    keywordstyle=\color{magenta},
    numberstyle=\tiny\color{codegray},
    stringstyle=\color{codepurple},
    basicstyle=\ttfamily\footnotesize,
    breakatwhitespace=false,         
    breaklines=true,                 
    captionpos=b,                    
    keepspaces=true,                 
    numbers=left,                    
xleftmargin=2em,
framexleftmargin=2em,            
    showspaces=false,                
    showstringspaces=false,
    showtabs=false,                  
    tabsize=2,
    upquote=true
}

\lstset{style=mystyle}


\lstset{style=mystyle}
\newcommand{\imgdir}{C:/laragon/www/newmc/assets/imgsvg/}
\newcommand{\imgsvgdir}{C:/laragon/www/newmc/assets/imgsvg/}

\definecolor{mcgris}{RGB}{220, 220, 220}% ancien~; pour compatibilité
\definecolor{mcbleu}{RGB}{52, 152, 219}
\definecolor{mcvert}{RGB}{125, 194, 70}
\definecolor{mcmauve}{RGB}{154, 0, 215}
\definecolor{mcorange}{RGB}{255, 96, 0}
\definecolor{mcturquoise}{RGB}{0, 153, 153}
\definecolor{mcrouge}{RGB}{255, 0, 0}
\definecolor{mclightvert}{RGB}{205, 234, 190}

\definecolor{gris}{RGB}{220, 220, 220}
\definecolor{bleu}{RGB}{52, 152, 219}
\definecolor{vert}{RGB}{125, 194, 70}
\definecolor{mauve}{RGB}{154, 0, 215}
\definecolor{orange}{RGB}{255, 96, 0}
\definecolor{turquoise}{RGB}{0, 153, 153}
\definecolor{rouge}{RGB}{255, 0, 0}
\definecolor{lightvert}{RGB}{205, 234, 190}
\setitemize[0]{label=\color{lightvert}  $\bullet$}

\pagestyle{fancy}
\renewcommand{\headrulewidth}{0.2pt}
\fancyhead[L]{maths-cours.fr}
\fancyhead[R]{\thepage}
\renewcommand{\footrulewidth}{0.2pt}
\fancyfoot[C]{}

\newcolumntype{C}{>{\centering\arraybackslash}X}
\newcolumntype{s}{>{\hsize=.35\hsize\arraybackslash}X}

\setlength{\parindent}{0pt}		 
\setlength{\parskip}{3mm}
\setlength{\headheight}{1cm}

\def\ebook{ebook}
\def\book{book}
\def\web{web}
\def\type{web}

\newcommand{\vect}[1]{\overrightarrow{\,\mathstrut#1\,}}

\def\Oij{$\left(\text{O}~;~\vect{\imath},~\vect{\jmath}\right)$}
\def\Oijk{$\left(\text{O}~;~\vect{\imath},~\vect{\jmath},~\vect{k}\right)$}
\def\Ouv{$\left(\text{O}~;~\vect{u},~\vect{v}\right)$}

\hypersetup{breaklinks=true, colorlinks = true, linkcolor = OliveGreen, urlcolor = OliveGreen, citecolor = OliveGreen, pdfauthor={Didier BONNEL - https://www.maths-cours.fr} } % supprime les bordures autour des liens

\renewcommand{\arg}[0]{\text{arg}}

\everymath{\displaystyle}

%================================================================================================================================
%
% Macros - Commandes
%
%================================================================================================================================

\newcommand\meta[2]{    			% Utilisé pour créer le post HTML.
	\def\titre{titre}
	\def\url{url}
	\def\arg{#1}
	\ifx\titre\arg
		\newcommand\maintitle{#2}
		\fancyhead[L]{#2}
		{\Large\sffamily \MakeUppercase{#2}}
		\vspace{1mm}\textcolor{mcvert}{\hrule}
	\fi 
	\ifx\url\arg
		\fancyfoot[L]{\href{https://www.maths-cours.fr#2}{\black \footnotesize{https://www.maths-cours.fr#2}}}
	\fi 
}


\newcommand\TitreC[1]{    		% Titre centré
     \needspace{3\baselineskip}
     \begin{center}\textbf{#1}\end{center}
}

\newcommand\newpar{    		% paragraphe
     \par
}

\newcommand\nosp {    		% commande vide (pas d'espace)
}
\newcommand{\id}[1]{} %ignore

\newcommand\boite[2]{				% Boite simple sans titre
	\vspace{5mm}
	\setlength{\fboxrule}{0.2mm}
	\setlength{\fboxsep}{5mm}	
	\fcolorbox{#1}{#1!3}{\makebox[\linewidth-2\fboxrule-2\fboxsep]{
  		\begin{minipage}[t]{\linewidth-2\fboxrule-4\fboxsep}\setlength{\parskip}{3mm}
  			 #2
  		\end{minipage}
	}}
	\vspace{5mm}
}

\newcommand\CBox[4]{				% Boites
	\vspace{5mm}
	\setlength{\fboxrule}{0.2mm}
	\setlength{\fboxsep}{5mm}
	
	\fcolorbox{#1}{#1!3}{\makebox[\linewidth-2\fboxrule-2\fboxsep]{
		\begin{minipage}[t]{1cm}\setlength{\parskip}{3mm}
	  		\textcolor{#1}{\LARGE{#2}}    
 	 	\end{minipage}  
  		\begin{minipage}[t]{\linewidth-2\fboxrule-4\fboxsep}\setlength{\parskip}{3mm}
			\raisebox{1.2mm}{\normalsize\sffamily{\textcolor{#1}{#3}}}						
  			 #4
  		\end{minipage}
	}}
	\vspace{5mm}
}

\newcommand\cadre[3]{				% Boites convertible html
	\par
	\vspace{2mm}
	\setlength{\fboxrule}{0.1mm}
	\setlength{\fboxsep}{5mm}
	\fcolorbox{#1}{white}{\makebox[\linewidth-2\fboxrule-2\fboxsep]{
  		\begin{minipage}[t]{\linewidth-2\fboxrule-4\fboxsep}\setlength{\parskip}{3mm}
			\raisebox{-2.5mm}{\sffamily \small{\textcolor{#1}{\MakeUppercase{#2}}}}		
			\par		
  			 #3
 	 		\end{minipage}
	}}
		\vspace{2mm}
	\par
}

\newcommand\bloc[3]{				% Boites convertible html sans bordure
     \needspace{2\baselineskip}
     {\sffamily \small{\textcolor{#1}{\MakeUppercase{#2}}}}    
		\par		
  			 #3
		\par
}

\newcommand\CHelp[1]{
     \CBox{Plum}{\faInfoCircle}{À RETENIR}{#1}
}

\newcommand\CUp[1]{
     \CBox{NavyBlue}{\faThumbsOUp}{EN PRATIQUE}{#1}
}

\newcommand\CInfo[1]{
     \CBox{Sepia}{\faArrowCircleRight}{REMARQUE}{#1}
}

\newcommand\CRedac[1]{
     \CBox{PineGreen}{\faEdit}{BIEN R\'EDIGER}{#1}
}

\newcommand\CError[1]{
     \CBox{Red}{\faExclamationTriangle}{ATTENTION}{#1}
}

\newcommand\TitreExo[2]{
\needspace{4\baselineskip}
 {\sffamily\large EXERCICE #1\ (\emph{#2 points})}
\vspace{5mm}
}

\newcommand\img[2]{
          \includegraphics[width=#2\paperwidth]{\imgdir#1}
}

\newcommand\imgsvg[2]{
       \begin{center}   \includegraphics[width=#2\paperwidth]{\imgsvgdir#1} \end{center}
}


\newcommand\Lien[2]{
     \href{#1}{#2 \tiny \faExternalLink}
}
\newcommand\mcLien[2]{
     \href{https~://www.maths-cours.fr/#1}{#2 \tiny \faExternalLink}
}

\newcommand{\euro}{\eurologo{}}

%================================================================================================================================
%
% Macros - Environement
%
%================================================================================================================================

\newenvironment{tex}{ %
}
{%
}

\newenvironment{indente}{ %
	\setlength\parindent{10mm}
}

{
	\setlength\parindent{0mm}
}

\newenvironment{corrige}{%
     \needspace{3\baselineskip}
     \medskip
     \textbf{\textsc{Corrigé}}
     \medskip
}
{
}

\newenvironment{extern}{%
     \begin{center}
     }
     {
     \end{center}
}

\NewEnviron{code}{%
	\par
     \boite{gray}{\texttt{%
     \BODY
     }}
     \par
}

\newenvironment{vbloc}{% boite sans cadre empeche saut de page
     \begin{minipage}[t]{\linewidth}
     }
     {
     \end{minipage}
}
\NewEnviron{h2}{%
    \needspace{3\baselineskip}
    \vspace{0.6cm}
	\noindent \MakeUppercase{\sffamily \large \BODY}
	\vspace{1mm}\textcolor{mcgris}{\hrule}\vspace{0.4cm}
	\par
}{}

\NewEnviron{h3}{%
    \needspace{3\baselineskip}
	\vspace{5mm}
	\textsc{\BODY}
	\par
}

\NewEnviron{margeneg}{ %
\begin{addmargin}[-1cm]{0cm}
\BODY
\end{addmargin}
}

\NewEnviron{html}{%
}

\begin{document}
\meta{url}{/exercices/pourcentages-fonctions-bac-blanc-es-l-sujet-2-maths-cours-2018/}
\meta{pid}{10442}
\meta{titre}{Pourcentages - Fonctions - Bac blanc ES/L Sujet 2 - Maths-cours 2018}
\meta{type}{exercices}
%
\begin{h2}Exercice 2 (6 points)\end{h2}
\par
Le propriétaire d'un restaurant a constaté que, lorsque le prix de son menu était fixé à 25~euros, il accueillait 20~clients, et qu'à chaque baisse de 1~euro, il attirait 2~clients supplémentaires.
%============================================================================================================================
%
\TitreC{Partie A}
%
%============================================================================================================================
\begin{enumerate}
     \item De quel pourcentage le prix a-t-il baissé lorsqu'il est passé de 25 à 24~euros ?
     \item Quel est le pourcentage d'augmentation du nombre de clients lorsque celui-ci passe de 20 à 22~clients ?
     \item L'\textbf{élasticité} de la demande par rapport au prix est le rapport :
     \[ e=\dfrac{V_c}{V_p} \]
     où $V_c$ désigne la variation en pourcentage du nombre de clients et $V_p$ la variation en pourcentage du prix du menu. \\
     Le nombre de clients variant en sens contraire du prix, ce rapport sera \textbf{négatif}.
     \par
     \begin{enumerate}[label=\alph*.]
          \item
          Montrer que l'élasticité de la demande par rapport au prix lorsque le prix du menu passe de 25 à 24 euros est égale à -2,5.
          \item
          Déterminer le nombre de clients pour un menu à 20 euros puis pour un menu à 19~euros.\\
          En déduire l'élasticité de la demande par rapport au prix lorsque le prix du menu passe de 20 à 19~euros.
     \end{enumerate}
     \item La recette est égale au produit du nombre de clients par le prix du menu.
     Calculer le montant de la recette lorsque le prix du menu est 25~euros puis lorsque le prix du menu est 20~euros.
\end{enumerate}
%============================================================================================================================
%
\TitreC{Partie B}
%
%============================================================================================================================
\begin{enumerate}
     \item On admet que la fonction $f$ donnant le nombre de clients en fonction du prix du menu $x$ est définie sur l'intervalle $[10~;~25]$ par :
     \[f(x)=70-2x.\]
     \par
     Exprimer la recette du restaurant en fonction de $x$.
     \par
     Pour quel prix $x_0$ la recette est-elle maximale ?
     \item Pour $x \in [10~;~25]$, on note $e(x)$ l'élasticité de la demande par rapport au prix du menu lorsque ce dernier passe de $x$ à $x-1$ euros.
     Montrer que $e(x)=x\ \dfrac{f'(x)}{f(x)}$.
     \item Calculer $e'(x)$ et en déduire le sens de variation de la fonction $e$ sur l'intervalle $[10~;~25]$.
     Montrer que $e(x_0)=-1$.
     \par
\end{enumerate}
\begin{corrige}
     %============================================================================================================================
     %
     \TitreC{Partie A}
     %
     %============================================================================================================================
     \par
     \begin{enumerate}
          \item %1
          \textbf{Lorsque le prix passe de 25 à 24 euros}, le pourcentage de variation est :
          \par
          $V_p=\dfrac{24-25}{25}=-\dfrac{1}{25}=-0,04=-4\%$.
          \par
          Le prix a \textbf{baissé de 4\%} lorsqu'il est passé de 25 à 24 euros.
          \cadre{rouge}{À retenir}{
               Lorsqu'une valeur passe de $V_0$ à $V_1$, le pourcentage de variation est :
               \[ t = \dfrac{V_1-V_0}{V_0}. \]
          }
          \item %2
          Lorsque le nombre de clients passe de 20 à 22, le pourcentage de variation est :
          \par
          $ V_c=\dfrac{22-20}{20}=\dfrac{2}{20}=0,1=10\%. $
          \par
          Le nombre de clients a \textbf{augmenté de 10\%} lorsqu'il est passé de 20 à 22.
          \item %3
          \begin{enumerate}[label=\alph*.]
               \item %3a
               D'après la définition de l'énoncé, l'élasticité $e$ de la demande par rapport au prix est le quotient de la variation en pourcentage du nombre de clients par la variation en pourcentage du prix du menu.
               \par
               C'est à dire :
               \par
               $e=\dfrac{V_c}{V_p}=\dfrac{0,1}{-0,04}=-\dfrac{10}{4}=-2,5$.
               \par
               L'élasticité de la demande par rapport au prix, lorsque le prix du menu passe de 25 à 24~euros, est de $-2,5$.
               \item %3b
               \`A chaque baisse de 1 euro, le restaurant attire 2 clients supplémentaires. Le tableau ci-après indique le nombre de clients en fonction du prix :
               \par
               \begin{center}
                    \begin{tabular}{|l|c|c|c|c|c|c|c|c|}%class="compact"
                         \hline
                         Prix & 25 & 24 & 23 & 22 & 21 & 20 & 19 & $\cdots$ \\
                         \hline
                         Nb clients & 20 & 22 & 24 & 26 & 28 & 30 & 32 & $\cdots$ \\
                         \hline
                    \end{tabular}
               \end{center}
               \par
               Pour un menu à 20 euros, le nombre de clients est \textbf{30}, et, pour un menu à 19 euros, le nombre de clients est \textbf{32}.
               \par
               \textbf{Lorsque le prix passe de 20 à 19 euros}, le pourcentage de variation du prix est :
               \par
               $ V_p=\dfrac{19-20}{20}=-\dfrac{1}{20} $.
               \par
               Le nombre de clients passe alors de 30 à 32 ; le pourcentage de variation du nombre de clients est donc :
               \par
               $V_c=\dfrac{32-30}{30}=\dfrac{2}{30}=\dfrac{1}{15}$.
               \par
               L'élasticité $e$ de la demande par rapport au prix vaut alors :
               \par
               $e=\dfrac{V_c}{V_p}=\dfrac{\dfrac{1}{15}}{-\dfrac{1}{20}}=-\dfrac{20}{15}=-\dfrac{4}{3}$.
               \par
               L'élasticité de la demande par rapport au prix lorsque le prix du menu passe de 20 à 19~euros est de $-\dfrac{4}{3}$.
               \par
          \end{enumerate}
          \item %4
          Lorsque le prix du menu est égal à 25~euros, le restaurant accueille {20~clients}.
          \par
          La recette vaut :
          \par
          $R_{25}=25 \times 20 = 500$ euros.
          \par
          La recette est de \textbf{500 euros} lorsque le prix du menu est 25~euros.
          \par
          Lorsque le prix du menu est égal à 20 euros, le restaurant accueille 30~clients.
          \par
          La recette vaut alors :
          \par
          $R_{20}=20 \times 30 = 600$ euros.
          \par
          La recette est de \textbf{600 euros} lorsque le prix du menu est 20~euros.
          \par
     \end{enumerate}
     \par
     %============================================================================================================================
     %
     \TitreC{Partie B}
     %
     %============================================================================================================================
     \par
     \begin{enumerate}
          \item %1
          La recette est égal au produit du nombre de clients par le prix du menu, par conséquent :
          \par
          $R(x)=xf(x)=x(70-2x)$\nosp$=-2x^2+70x$.
          \par
          La fonction $R$ est une fonction polynôme du second degré. Le coefficient de $x^2$ est -2 ; il est négatif.
          \par
          $R$ admet un maximum pour :
          \par
          $x_0=-\dfrac{b}{2a}=-\dfrac{70}{-4}=17,5$.
          \par
          \vspace{2mm}
          La recette est maximale lorsque le prix du menu est égal à \textbf{17,50~euros}.
          \cadre{rouge}{À retenir}{
               Une fonction polynôme du second degré ${x \longmapsto ax^2+bx=c}$ ($a \neq 0$) admet un extremum pour :
               \[ x_0=-\dfrac{b}{2a}. \]
               \par
               Cet extremum est :
               \begin{itemize}
                    \item
                    un \textbf{minimum} si $a>0$,
                    \item
                    un \textbf{maximum} si $a<0$.
               \end{itemize}
          }
          \cadre{bleu}{Remarque}{
               Il est aussi possible de calculer $R'(x)$, d'étudier son signe et de dresser le tableau de variations de $R$.
               \par
               Toutefois, la propriété employée ici (et vue en classe de Seconde) permet d'aboutir au résultat plus rapidement.
          }
          \item %2
          La variation en pourcentage du prix du menu lorsqu'il passe de $x$ à $x-1$ euros est :
          \par
          $V_p(x)=\dfrac{(x-1)-x}{x}=-\dfrac{1}{x}$.
          \par
          Lorsque le prix du menu passe de $x$ à $x-1$ euros, le nombre de clients passe de $f(x)$ à $f(x-1)$.
          \par
          La variation en pourcentage du nombre de clients est alors :
          \par
          $V_c(x)=\dfrac{f(x-1)-f(x)}{f(x)}$\\
          $\phantom{V_c(x)}=\dfrac{70-2(x-1)-(70-2x)}{70-2x}$\\
          $\phantom{V_c(x)}=\dfrac{70-2x+2-70+2x}{70-2x}$\\
          $\phantom{V_c(x)}=\dfrac{2}{70-2x}.$
          \par
          L'élasticité vaut donc :
          \par
          $e(x)=\dfrac{V_c(x)}{V_p(x)}$//
          $\phantom{e(x)}=\dfrac{\dfrac{2}{70-2x}}{-\dfrac{1}{x}}$\\
          $\phantom{e(x)}=\dfrac{2}{70-2x} \times \dfrac{-x}{1}$\\
          $\phantom{e(x)}=\dfrac{-2x}{70-2x}.$
          \par
          \vspace{2mm}
          Par ailleurs :
          \par
          $f(x)=70-2x$ donc $f'(x)=-2$ ;
          \par
          $x\ \dfrac{f'(x)}{f(x)} = x \times \dfrac{-2}{70-2x}=\dfrac{-2x}{70-2x}$.
          \par
          On a donc bien pour tout réel $x$ de l'intervalle $[10~;~25]$ :
          \[ e(x)=x\ \dfrac{f'(x)}{f(x)}. \]
          \cadre{rouge}{Bien rédiger}{
               Pour démontrer l'égalité $e(x)=x\ \dfrac{f'(x)}{f(x)}$, il est incorrect de présenter les calculs en partant de cette égalité (puisque l'on n'a pas encore démontré qu'elle était vraie...).
               \par
               Ici, on calcule \textbf{séparément} $e(x)$ et $x\ \dfrac{f'(x)}{f(x)}$ et on montre que l'on aboutit au même résultat.
          }
          \item
          D'après la question précédente, $e$ est une fonction rationnelle définie et dérivable sur l'intervalle $[10~;~25]$ telle que:
          \[ e(x)=\dfrac{-2x}{70-2x} \]
          Posons :
          \par
          $u(x)=-2x$ ;
          \par
          $v(x)=70-2x$ ;
          \par
          alors :
          \par
          $u'(x)=-2$ ;
          \par
          $v'(x)=-2$.
          \par
          Par conséquent :
          \par
          $e'(x)= \dfrac{u'(x)v(x)-u(x)v'(x)}{v(x)^2}$\\
          $\phantom{e'(x)}=\dfrac{-2(70-2x)-(-2x) \times (-2)}{(70-2x)^2}$\\
          $\phantom{e'(x)}= \dfrac{-140+4x-4x}{(70-2x)^2}$\\
          $\phantom{e'(x)}= \dfrac{-140}{(70-2x)^2}.$
          \par
          Le numérateur est strictement négatif et le dénominateur est strictement positif sur l'intervalle $[10~;~25]$ donc $e'$ est strictement négative et la fonction $e$ est strictement \textbf{décroissante} sur l'intervalle $[10~;~25]$.
          \par
          Enfin :
          \par
          $e(x_0)=e(17,5)=\dfrac{-2 \times 17,5}{70-2 \times 17,5}$\nosp$=\dfrac{-35}{35}=-1$.
          \par
     \end{enumerate}
\end{corrige}

\end{document}
µ
\documentclass[a4paper]{article}

%================================================================================================================================
%
% Packages
%
%================================================================================================================================

\usepackage[T1]{fontenc} 	% pour caractères accentués
\usepackage[utf8]{inputenc}  % encodage utf8
\usepackage[french]{babel}	% langue : français
\usepackage{fourier}			% caractères plus lisibles
\usepackage[dvipsnames]{xcolor} % couleurs
\usepackage{fancyhdr}		% réglage header footer
\usepackage{needspace}		% empêcher sauts de page mal placés
\usepackage{graphicx}		% pour inclure des graphiques
\usepackage{enumitem,cprotect}		% personnalise les listes d'items (nécessaire pour ol, al ...)
\usepackage{hyperref}		% Liens hypertexte
\usepackage{pstricks,pst-all,pst-node,pstricks-add,pst-math,pst-plot,pst-tree,pst-eucl} % pstricks
\usepackage[a4paper,includeheadfoot,top=2cm,left=3cm, bottom=2cm,right=3cm]{geometry} % marges etc.
\usepackage{comment}			% commentaires multilignes
\usepackage{amsmath,environ} % maths (matrices, etc.)
\usepackage{amssymb,makeidx}
\usepackage{bm}				% bold maths
\usepackage{tabularx}		% tableaux
\usepackage{colortbl}		% tableaux en couleur
\usepackage{fontawesome}		% Fontawesome
\usepackage{environ}			% environment with command
\usepackage{fp}				% calculs pour ps-tricks
\usepackage{multido}			% pour ps tricks
\usepackage[np]{numprint}	% formattage nombre
\usepackage{tikz,tkz-tab} 			% package principal TikZ
\usepackage{pgfplots}   % axes
\usepackage{mathrsfs}    % cursives
\usepackage{calc}			% calcul taille boites
\usepackage[scaled=0.875]{helvet} % font sans serif
\usepackage{svg} % svg
\usepackage{scrextend} % local margin
\usepackage{scratch} %scratch
\usepackage{multicol} % colonnes
%\usepackage{infix-RPN,pst-func} % formule en notation polanaise inversée
\usepackage{listings}

%================================================================================================================================
%
% Réglages de base
%
%================================================================================================================================

\lstset{
language=Python,   % R code
literate=
{á}{{\'a}}1
{à}{{\`a}}1
{ã}{{\~a}}1
{é}{{\'e}}1
{è}{{\`e}}1
{ê}{{\^e}}1
{í}{{\'i}}1
{ó}{{\'o}}1
{õ}{{\~o}}1
{ú}{{\'u}}1
{ü}{{\"u}}1
{ç}{{\c{c}}}1
{~}{{ }}1
}


\definecolor{codegreen}{rgb}{0,0.6,0}
\definecolor{codegray}{rgb}{0.5,0.5,0.5}
\definecolor{codepurple}{rgb}{0.58,0,0.82}
\definecolor{backcolour}{rgb}{0.95,0.95,0.92}

\lstdefinestyle{mystyle}{
    backgroundcolor=\color{backcolour},   
    commentstyle=\color{codegreen},
    keywordstyle=\color{magenta},
    numberstyle=\tiny\color{codegray},
    stringstyle=\color{codepurple},
    basicstyle=\ttfamily\footnotesize,
    breakatwhitespace=false,         
    breaklines=true,                 
    captionpos=b,                    
    keepspaces=true,                 
    numbers=left,                    
xleftmargin=2em,
framexleftmargin=2em,            
    showspaces=false,                
    showstringspaces=false,
    showtabs=false,                  
    tabsize=2,
    upquote=true
}

\lstset{style=mystyle}


\lstset{style=mystyle}
\newcommand{\imgdir}{C:/laragon/www/newmc/assets/imgsvg/}
\newcommand{\imgsvgdir}{C:/laragon/www/newmc/assets/imgsvg/}

\definecolor{mcgris}{RGB}{220, 220, 220}% ancien~; pour compatibilité
\definecolor{mcbleu}{RGB}{52, 152, 219}
\definecolor{mcvert}{RGB}{125, 194, 70}
\definecolor{mcmauve}{RGB}{154, 0, 215}
\definecolor{mcorange}{RGB}{255, 96, 0}
\definecolor{mcturquoise}{RGB}{0, 153, 153}
\definecolor{mcrouge}{RGB}{255, 0, 0}
\definecolor{mclightvert}{RGB}{205, 234, 190}

\definecolor{gris}{RGB}{220, 220, 220}
\definecolor{bleu}{RGB}{52, 152, 219}
\definecolor{vert}{RGB}{125, 194, 70}
\definecolor{mauve}{RGB}{154, 0, 215}
\definecolor{orange}{RGB}{255, 96, 0}
\definecolor{turquoise}{RGB}{0, 153, 153}
\definecolor{rouge}{RGB}{255, 0, 0}
\definecolor{lightvert}{RGB}{205, 234, 190}
\setitemize[0]{label=\color{lightvert}  $\bullet$}

\pagestyle{fancy}
\renewcommand{\headrulewidth}{0.2pt}
\fancyhead[L]{maths-cours.fr}
\fancyhead[R]{\thepage}
\renewcommand{\footrulewidth}{0.2pt}
\fancyfoot[C]{}

\newcolumntype{C}{>{\centering\arraybackslash}X}
\newcolumntype{s}{>{\hsize=.35\hsize\arraybackslash}X}

\setlength{\parindent}{0pt}		 
\setlength{\parskip}{3mm}
\setlength{\headheight}{1cm}

\def\ebook{ebook}
\def\book{book}
\def\web{web}
\def\type{web}

\newcommand{\vect}[1]{\overrightarrow{\,\mathstrut#1\,}}

\def\Oij{$\left(\text{O}~;~\vect{\imath},~\vect{\jmath}\right)$}
\def\Oijk{$\left(\text{O}~;~\vect{\imath},~\vect{\jmath},~\vect{k}\right)$}
\def\Ouv{$\left(\text{O}~;~\vect{u},~\vect{v}\right)$}

\hypersetup{breaklinks=true, colorlinks = true, linkcolor = OliveGreen, urlcolor = OliveGreen, citecolor = OliveGreen, pdfauthor={Didier BONNEL - https://www.maths-cours.fr} } % supprime les bordures autour des liens

\renewcommand{\arg}[0]{\text{arg}}

\everymath{\displaystyle}

%================================================================================================================================
%
% Macros - Commandes
%
%================================================================================================================================

\newcommand\meta[2]{    			% Utilisé pour créer le post HTML.
	\def\titre{titre}
	\def\url{url}
	\def\arg{#1}
	\ifx\titre\arg
		\newcommand\maintitle{#2}
		\fancyhead[L]{#2}
		{\Large\sffamily \MakeUppercase{#2}}
		\vspace{1mm}\textcolor{mcvert}{\hrule}
	\fi 
	\ifx\url\arg
		\fancyfoot[L]{\href{https://www.maths-cours.fr#2}{\black \footnotesize{https://www.maths-cours.fr#2}}}
	\fi 
}


\newcommand\TitreC[1]{    		% Titre centré
     \needspace{3\baselineskip}
     \begin{center}\textbf{#1}\end{center}
}

\newcommand\newpar{    		% paragraphe
     \par
}

\newcommand\nosp {    		% commande vide (pas d'espace)
}
\newcommand{\id}[1]{} %ignore

\newcommand\boite[2]{				% Boite simple sans titre
	\vspace{5mm}
	\setlength{\fboxrule}{0.2mm}
	\setlength{\fboxsep}{5mm}	
	\fcolorbox{#1}{#1!3}{\makebox[\linewidth-2\fboxrule-2\fboxsep]{
  		\begin{minipage}[t]{\linewidth-2\fboxrule-4\fboxsep}\setlength{\parskip}{3mm}
  			 #2
  		\end{minipage}
	}}
	\vspace{5mm}
}

\newcommand\CBox[4]{				% Boites
	\vspace{5mm}
	\setlength{\fboxrule}{0.2mm}
	\setlength{\fboxsep}{5mm}
	
	\fcolorbox{#1}{#1!3}{\makebox[\linewidth-2\fboxrule-2\fboxsep]{
		\begin{minipage}[t]{1cm}\setlength{\parskip}{3mm}
	  		\textcolor{#1}{\LARGE{#2}}    
 	 	\end{minipage}  
  		\begin{minipage}[t]{\linewidth-2\fboxrule-4\fboxsep}\setlength{\parskip}{3mm}
			\raisebox{1.2mm}{\normalsize\sffamily{\textcolor{#1}{#3}}}						
  			 #4
  		\end{minipage}
	}}
	\vspace{5mm}
}

\newcommand\cadre[3]{				% Boites convertible html
	\par
	\vspace{2mm}
	\setlength{\fboxrule}{0.1mm}
	\setlength{\fboxsep}{5mm}
	\fcolorbox{#1}{white}{\makebox[\linewidth-2\fboxrule-2\fboxsep]{
  		\begin{minipage}[t]{\linewidth-2\fboxrule-4\fboxsep}\setlength{\parskip}{3mm}
			\raisebox{-2.5mm}{\sffamily \small{\textcolor{#1}{\MakeUppercase{#2}}}}		
			\par		
  			 #3
 	 		\end{minipage}
	}}
		\vspace{2mm}
	\par
}

\newcommand\bloc[3]{				% Boites convertible html sans bordure
     \needspace{2\baselineskip}
     {\sffamily \small{\textcolor{#1}{\MakeUppercase{#2}}}}    
		\par		
  			 #3
		\par
}

\newcommand\CHelp[1]{
     \CBox{Plum}{\faInfoCircle}{À RETENIR}{#1}
}

\newcommand\CUp[1]{
     \CBox{NavyBlue}{\faThumbsOUp}{EN PRATIQUE}{#1}
}

\newcommand\CInfo[1]{
     \CBox{Sepia}{\faArrowCircleRight}{REMARQUE}{#1}
}

\newcommand\CRedac[1]{
     \CBox{PineGreen}{\faEdit}{BIEN R\'EDIGER}{#1}
}

\newcommand\CError[1]{
     \CBox{Red}{\faExclamationTriangle}{ATTENTION}{#1}
}

\newcommand\TitreExo[2]{
\needspace{4\baselineskip}
 {\sffamily\large EXERCICE #1\ (\emph{#2 points})}
\vspace{5mm}
}

\newcommand\img[2]{
          \includegraphics[width=#2\paperwidth]{\imgdir#1}
}

\newcommand\imgsvg[2]{
       \begin{center}   \includegraphics[width=#2\paperwidth]{\imgsvgdir#1} \end{center}
}


\newcommand\Lien[2]{
     \href{#1}{#2 \tiny \faExternalLink}
}
\newcommand\mcLien[2]{
     \href{https~://www.maths-cours.fr/#1}{#2 \tiny \faExternalLink}
}

\newcommand{\euro}{\eurologo{}}

%================================================================================================================================
%
% Macros - Environement
%
%================================================================================================================================

\newenvironment{tex}{ %
}
{%
}

\newenvironment{indente}{ %
	\setlength\parindent{10mm}
}

{
	\setlength\parindent{0mm}
}

\newenvironment{corrige}{%
     \needspace{3\baselineskip}
     \medskip
     \textbf{\textsc{Corrigé}}
     \medskip
}
{
}

\newenvironment{extern}{%
     \begin{center}
     }
     {
     \end{center}
}

\NewEnviron{code}{%
	\par
     \boite{gray}{\texttt{%
     \BODY
     }}
     \par
}

\newenvironment{vbloc}{% boite sans cadre empeche saut de page
     \begin{minipage}[t]{\linewidth}
     }
     {
     \end{minipage}
}
\NewEnviron{h2}{%
    \needspace{3\baselineskip}
    \vspace{0.6cm}
	\noindent \MakeUppercase{\sffamily \large \BODY}
	\vspace{1mm}\textcolor{mcgris}{\hrule}\vspace{0.4cm}
	\par
}{}

\NewEnviron{h3}{%
    \needspace{3\baselineskip}
	\vspace{5mm}
	\textsc{\BODY}
	\par
}

\NewEnviron{margeneg}{ %
\begin{addmargin}[-1cm]{0cm}
\BODY
\end{addmargin}
}

\NewEnviron{html}{%
}

\begin{document}
\meta{url}{/exercices/suites-bac-blanc-es-l-sujet-2-maths-cours-2018/}
\meta{pid}{10448}
\meta{titre}{Suites - Bac blanc ES/L Sujet 2 - Maths-cours 2018}
\meta{type}{exercices}
%
\begin{h2}Exercice 3 (4 points)\end{h2}
\par
Un cinéma de trois salles propose le choix entre les films \textbf{A}, \textbf{B} ou \textbf{C}. Suivant leur âge, les spectateurs payent leur place plein tarif ou bénéficient d'un tarif réduit.
\par
Le directeur de la salle a constaté que :
\par
\begin{itemize}
     \item 30\% des spectateurs bénéficient du tarif réduit (les 70\% restant payant plein tarif) ;
     \item 45\% des spectateurs payant plein tarif et 40\% des spectateurs bénéficiant du tarif réduit ont été voir le film \textbf{A} ;
     \item 30\% des spectateurs payant plein tarif et 37\% des spectateurs bénéficiant du tarif réduit ont été voir le film \textbf{B} ;
     \item 25\% des spectateurs payant plein tarif et 23\% des spectateurs bénéficiant du tarif réduit ont été voir le film \textbf{C}.
\end{itemize}
\par
On choisit au hasard un spectateur à la sortie du cinéma. On note :
\par
\begin{itemize}
     \item $R$ : l'événement \og le spectateur bénéficie du tarif réduit \fg{} ;
     \item $A$ : l'événement \og le spectateur a été voir le film \textbf{A} \fg{} ;
     \item $B$ : l'événement \og le spectateur a été voir le film \textbf{B} \fg{} ;
     \item $C$ : l'événement \og le spectateur a été voir le film \textbf{C} \fg{}.
\end{itemize}
\par
\begin{enumerate}
     \item %1
     Représenter la situation à l'aide d'un arbre pondéré.
     \item %2
     Montrer que la probabilité que le spectateur choisi vienne d'aller voir le film \textbf{A} est égale à $0,435$.
     \item %3
     On sait que le spectateur vient de voir le film \textbf{A}. Quelle est la probabilité qu'il bénéficie du tarif réduit ?
     \item %4
     On choisit maintenant au hasard et de façon indépendante, trois spectateurs. On suppose que ces choix peuvent être assimilés à des tirages successifs avec remise.
     \par
     On note $X$ la variable aléatoire correspondant au nombre de ces spectateurs qui viennent de voir le film \textbf{A}.
     \par
     \begin{enumerate}
          \item %4a
          Quelle est la loi de probabilité suivie par $X$ ? Préciser ses paramètres.
          \item %4b
          Calculer la probabilité $p(X \geqslant 1)$. Interpréter cette probabilité dans le cadre de l'énoncé.
          \par
     \end{enumerate}
     \par
\end{enumerate}
\begin{corrige}
     \begin{enumerate}
          \item %1
          La situation peut être modélisée par l'arbre pondéré ci-après :
          \par
          %:-+-+-+- Engendré par : http://math.et.info.free.fr/TikZ/Arbre/
          \begin{center}
               \begin{extern}%width="320" alt="Arbre pondéré de probabilité"
                    % Racine à Gauche, développement vers la droite
                    \begin{tikzpicture}[xscale=1,yscale=1]
                         % Styles (MODIFIABLES)
                         \tikzstyle{fleche}=[-,>=latex,thick]
                         \tikzstyle{noeud}=[fill=white,circle,draw]
                         \tikzstyle{feuille}=[fill=white,circle,draw]
                         \tikzstyle{etiquette}=[midway,fill=white]
                         % Dimensions (MODIFIABLES)
                         \def\DistanceInterNiveaux{2.5}
                         \def\DistanceInterFeuilles{1.2}
                         % Dimensions calculées (NON MODIFIABLES)
                         \def\NiveauA{(0)*\DistanceInterNiveaux}
                         \def\NiveauB{(1.5)*\DistanceInterNiveaux}
                         \def\NiveauC{(2.5)*\DistanceInterNiveaux}
                         \def\InterFeuilles{(-1)*\DistanceInterFeuilles}
                         % Noeuds (MODIFIABLES : Styles et Coefficients d'InterFeuilles)
                         \node[noeud] (R) at ({\NiveauA},{(2.5)*\InterFeuilles}) {$\ $};
                         \node[noeud] (Ra) at ({\NiveauB},{(1)*\InterFeuilles}) {$R$};
                         \node[feuille] (Raa) at ({\NiveauC},{(0)*\InterFeuilles}) {$A$};
                         \node[feuille] (Rab) at ({\NiveauC},{(1)*\InterFeuilles}) {$B$};
                         \node[feuille] (Rac) at ({\NiveauC},{(2)*\InterFeuilles}) {$C$};
                         \node[noeud] (Rb) at ({\NiveauB},{(4)*\InterFeuilles}) {$\overline{R}$};
                         \node[feuille] (Rba) at ({\NiveauC},{(3)*\InterFeuilles}) {$A$};
                         \node[feuille] (Rbb) at ({\NiveauC},{(4)*\InterFeuilles}) {$B$};
                         \node[feuille] (Rbc) at ({\NiveauC},{(5)*\InterFeuilles}) {$C$};
                         % Arcs (MODIFIABLES : Styles)
                         \draw[fleche] (R)--(Ra) node[etiquette] {$0,3$};
                         \draw[fleche] (Ra)--(Raa) node[etiquette] {$0,4$};
                         \draw[fleche] (Ra)--(Rab) node[etiquette] {$0,37$};
                         \draw[fleche] (Ra)--(Rac) node[etiquette] {$0,23$};
                         \draw[fleche] (R)--(Rb) node[etiquette] {$0,7$};
                         \draw[fleche] (Rb)--(Rba) node[etiquette] {$0,45$};
                         \draw[fleche] (Rb)--(Rbb) node[etiquette] {$0,3$};
                         \draw[fleche] (Rb)--(Rbc) node[etiquette] {$0,25$};
                    \end{tikzpicture}
               \end{extern}
          \end{center}
          \cadre{rouge}{À retenir}{
               Le total des probabilités figurant sur l'ensemble des branches partant d'un même nœud est toujours égal à 1.
          }
          \item %2
          La probabilité que le spectateur ait été voir le film \textbf{A} est $p(A)$.
          \par
          D'après la formule des probabilités totales :
          \par
          $p(A)=p(A\cap R)+p(A\cap \overline{R})$\\
          $\phantom{p(A)}=p(R) \times p_R(A)+ p({\overline{R}}) \times p_{\overline{R}}(A)$\\
          $\phantom{p(A)}=0,3 \times 0,4 + 0,7 \times 0,45 = 0,435.$
          \par
          \cadre{rouge}{À retenir}{
               \textbf{Formule des probabilités totales :}
               \par
               Si les événements $B_1, B_2, \cdots , B_n$ forment une partition de l'univers (c'est à dire regroupent toutes les éventualités) alors, pour tout événement $A$ :
               \begin{center}
                    $p(A)= p(A\cap B_1)+p(A\cap B_2)$\nosp$+\cdots+p(A\cap B_n).$
               \end{center}
               \par
               Un cas particulier très fréquent, dû au fait que $B$ et $\overline{B}$ forment une partition de l'univers, donne :
               \[ p(A)= p(A\cap B)+p(A\cap \overline{B}). \]
          }
          \item %3
          La probabilité demandée est $p_A(R)$.
          \cadre{vert}{En pratique}{
               Très souvent, en probabilités, la première étape consiste à traduire la probabilité cherchée en utilisant les notations de l'énoncé.
               \par
               Dans le cas présent, on sait que l'événement $A$ est vérifié et on souhaite déterminer la probabilité de l'événement $R$. On recherche donc $p_A(R)$.
          }
          \cadre{rouge}{Attention}{
               Ne pas confondre  :
               \begin{itemize}
                    \item
                    $\bm{p(A\cap R)}$ : probabilité que $A$ \textbf{et} $R$ se réalisent (alors que l'on n'a, \textit{a priori}, aucune information concernant la réalisation de $A$ ou de $R$) ;
                    \item
                    $\bm{p_A(R)}$ : probabilité que $R$ se réalise alors que l'\textbf{on sait que $\bm{A}$ est réalisé}.
               \end{itemize}
          }
          \par
          D'après la formule des probabilités conditionnelles :
          \par
          $p_A(R)=\dfrac{p(A\cap R)}{p(A)}=\dfrac{0,3 \times 0,4}{0,435}$\nosp$=\dfrac{0,12}{0,435} \approx 0,276\ $ (à $10^{-3}$ près).
          \item %4
          \begin{enumerate}
               \item %4a
               La variable aléatoire $X$ suit une loi binomiale de paramètres ${n=3}$ et ${p=0,435}$.
               \par
               En effet :
               \par
               \begin{itemize}
                    \item on assimile l'expérience aux tirages successifs et avec remise de 3 spectateurs ;
                    \item pour chaque spectateur, deux issues sont possibles :
                    \begin{itemize}
                         \item \textit{succès} : le spectateur vient d'aller voir le film \textbf{A} (probabilité $p=0,435$) ;
                         \item \textit{échec} : le spectateur ne vient pas d'aller voir le film \textbf{A}.
                    \end{itemize}
                    \item la variable aléatoire $X$ comptabilise le nombre de succès.
               \end{itemize}
               \item
               L'événement contraire de $(X \geqslant 1)$ est $(X<1)$ c'est à dire $(X=0)$.
               \cadre{rouge}{Attention}{
                    L'événement contraire de ($X \geqslant a$) est ($X < a$) et non ($X \leqslant a$).
               }
               Comme $X$ suit une loi binomiale :
               \par
               $p(X=0)=\begin{pmatrix} 3 \\ 0 \end{pmatrix} \times 0,435^0 \times 0,565^{3}$\nosp$ = 0,565^{3}$.
               \par
               Par conséquent :
               \par
               $p(X \geqslant 1)=1-p(X=0)$\nosp$=1-0,565^{3} \approx 0,820\ $ (à $10^{-3}$ près).
               \par
          \end{enumerate}
          \par
     \end{enumerate}
\end{corrige}

\end{document}
µ
\documentclass[a4paper]{article}

%================================================================================================================================
%
% Packages
%
%================================================================================================================================

\usepackage[T1]{fontenc} 	% pour caractères accentués
\usepackage[utf8]{inputenc}  % encodage utf8
\usepackage[french]{babel}	% langue : français
\usepackage{fourier}			% caractères plus lisibles
\usepackage[dvipsnames]{xcolor} % couleurs
\usepackage{fancyhdr}		% réglage header footer
\usepackage{needspace}		% empêcher sauts de page mal placés
\usepackage{graphicx}		% pour inclure des graphiques
\usepackage{enumitem,cprotect}		% personnalise les listes d'items (nécessaire pour ol, al ...)
\usepackage{hyperref}		% Liens hypertexte
\usepackage{pstricks,pst-all,pst-node,pstricks-add,pst-math,pst-plot,pst-tree,pst-eucl} % pstricks
\usepackage[a4paper,includeheadfoot,top=2cm,left=3cm, bottom=2cm,right=3cm]{geometry} % marges etc.
\usepackage{comment}			% commentaires multilignes
\usepackage{amsmath,environ} % maths (matrices, etc.)
\usepackage{amssymb,makeidx}
\usepackage{bm}				% bold maths
\usepackage{tabularx}		% tableaux
\usepackage{colortbl}		% tableaux en couleur
\usepackage{fontawesome}		% Fontawesome
\usepackage{environ}			% environment with command
\usepackage{fp}				% calculs pour ps-tricks
\usepackage{multido}			% pour ps tricks
\usepackage[np]{numprint}	% formattage nombre
\usepackage{tikz,tkz-tab} 			% package principal TikZ
\usepackage{pgfplots}   % axes
\usepackage{mathrsfs}    % cursives
\usepackage{calc}			% calcul taille boites
\usepackage[scaled=0.875]{helvet} % font sans serif
\usepackage{svg} % svg
\usepackage{scrextend} % local margin
\usepackage{scratch} %scratch
\usepackage{multicol} % colonnes
%\usepackage{infix-RPN,pst-func} % formule en notation polanaise inversée
\usepackage{listings}

%================================================================================================================================
%
% Réglages de base
%
%================================================================================================================================

\lstset{
language=Python,   % R code
literate=
{á}{{\'a}}1
{à}{{\`a}}1
{ã}{{\~a}}1
{é}{{\'e}}1
{è}{{\`e}}1
{ê}{{\^e}}1
{í}{{\'i}}1
{ó}{{\'o}}1
{õ}{{\~o}}1
{ú}{{\'u}}1
{ü}{{\"u}}1
{ç}{{\c{c}}}1
{~}{{ }}1
}


\definecolor{codegreen}{rgb}{0,0.6,0}
\definecolor{codegray}{rgb}{0.5,0.5,0.5}
\definecolor{codepurple}{rgb}{0.58,0,0.82}
\definecolor{backcolour}{rgb}{0.95,0.95,0.92}

\lstdefinestyle{mystyle}{
    backgroundcolor=\color{backcolour},   
    commentstyle=\color{codegreen},
    keywordstyle=\color{magenta},
    numberstyle=\tiny\color{codegray},
    stringstyle=\color{codepurple},
    basicstyle=\ttfamily\footnotesize,
    breakatwhitespace=false,         
    breaklines=true,                 
    captionpos=b,                    
    keepspaces=true,                 
    numbers=left,                    
xleftmargin=2em,
framexleftmargin=2em,            
    showspaces=false,                
    showstringspaces=false,
    showtabs=false,                  
    tabsize=2,
    upquote=true
}

\lstset{style=mystyle}


\lstset{style=mystyle}
\newcommand{\imgdir}{C:/laragon/www/newmc/assets/imgsvg/}
\newcommand{\imgsvgdir}{C:/laragon/www/newmc/assets/imgsvg/}

\definecolor{mcgris}{RGB}{220, 220, 220}% ancien~; pour compatibilité
\definecolor{mcbleu}{RGB}{52, 152, 219}
\definecolor{mcvert}{RGB}{125, 194, 70}
\definecolor{mcmauve}{RGB}{154, 0, 215}
\definecolor{mcorange}{RGB}{255, 96, 0}
\definecolor{mcturquoise}{RGB}{0, 153, 153}
\definecolor{mcrouge}{RGB}{255, 0, 0}
\definecolor{mclightvert}{RGB}{205, 234, 190}

\definecolor{gris}{RGB}{220, 220, 220}
\definecolor{bleu}{RGB}{52, 152, 219}
\definecolor{vert}{RGB}{125, 194, 70}
\definecolor{mauve}{RGB}{154, 0, 215}
\definecolor{orange}{RGB}{255, 96, 0}
\definecolor{turquoise}{RGB}{0, 153, 153}
\definecolor{rouge}{RGB}{255, 0, 0}
\definecolor{lightvert}{RGB}{205, 234, 190}
\setitemize[0]{label=\color{lightvert}  $\bullet$}

\pagestyle{fancy}
\renewcommand{\headrulewidth}{0.2pt}
\fancyhead[L]{maths-cours.fr}
\fancyhead[R]{\thepage}
\renewcommand{\footrulewidth}{0.2pt}
\fancyfoot[C]{}

\newcolumntype{C}{>{\centering\arraybackslash}X}
\newcolumntype{s}{>{\hsize=.35\hsize\arraybackslash}X}

\setlength{\parindent}{0pt}		 
\setlength{\parskip}{3mm}
\setlength{\headheight}{1cm}

\def\ebook{ebook}
\def\book{book}
\def\web{web}
\def\type{web}

\newcommand{\vect}[1]{\overrightarrow{\,\mathstrut#1\,}}

\def\Oij{$\left(\text{O}~;~\vect{\imath},~\vect{\jmath}\right)$}
\def\Oijk{$\left(\text{O}~;~\vect{\imath},~\vect{\jmath},~\vect{k}\right)$}
\def\Ouv{$\left(\text{O}~;~\vect{u},~\vect{v}\right)$}

\hypersetup{breaklinks=true, colorlinks = true, linkcolor = OliveGreen, urlcolor = OliveGreen, citecolor = OliveGreen, pdfauthor={Didier BONNEL - https://www.maths-cours.fr} } % supprime les bordures autour des liens

\renewcommand{\arg}[0]{\text{arg}}

\everymath{\displaystyle}

%================================================================================================================================
%
% Macros - Commandes
%
%================================================================================================================================

\newcommand\meta[2]{    			% Utilisé pour créer le post HTML.
	\def\titre{titre}
	\def\url{url}
	\def\arg{#1}
	\ifx\titre\arg
		\newcommand\maintitle{#2}
		\fancyhead[L]{#2}
		{\Large\sffamily \MakeUppercase{#2}}
		\vspace{1mm}\textcolor{mcvert}{\hrule}
	\fi 
	\ifx\url\arg
		\fancyfoot[L]{\href{https://www.maths-cours.fr#2}{\black \footnotesize{https://www.maths-cours.fr#2}}}
	\fi 
}


\newcommand\TitreC[1]{    		% Titre centré
     \needspace{3\baselineskip}
     \begin{center}\textbf{#1}\end{center}
}

\newcommand\newpar{    		% paragraphe
     \par
}

\newcommand\nosp {    		% commande vide (pas d'espace)
}
\newcommand{\id}[1]{} %ignore

\newcommand\boite[2]{				% Boite simple sans titre
	\vspace{5mm}
	\setlength{\fboxrule}{0.2mm}
	\setlength{\fboxsep}{5mm}	
	\fcolorbox{#1}{#1!3}{\makebox[\linewidth-2\fboxrule-2\fboxsep]{
  		\begin{minipage}[t]{\linewidth-2\fboxrule-4\fboxsep}\setlength{\parskip}{3mm}
  			 #2
  		\end{minipage}
	}}
	\vspace{5mm}
}

\newcommand\CBox[4]{				% Boites
	\vspace{5mm}
	\setlength{\fboxrule}{0.2mm}
	\setlength{\fboxsep}{5mm}
	
	\fcolorbox{#1}{#1!3}{\makebox[\linewidth-2\fboxrule-2\fboxsep]{
		\begin{minipage}[t]{1cm}\setlength{\parskip}{3mm}
	  		\textcolor{#1}{\LARGE{#2}}    
 	 	\end{minipage}  
  		\begin{minipage}[t]{\linewidth-2\fboxrule-4\fboxsep}\setlength{\parskip}{3mm}
			\raisebox{1.2mm}{\normalsize\sffamily{\textcolor{#1}{#3}}}						
  			 #4
  		\end{minipage}
	}}
	\vspace{5mm}
}

\newcommand\cadre[3]{				% Boites convertible html
	\par
	\vspace{2mm}
	\setlength{\fboxrule}{0.1mm}
	\setlength{\fboxsep}{5mm}
	\fcolorbox{#1}{white}{\makebox[\linewidth-2\fboxrule-2\fboxsep]{
  		\begin{minipage}[t]{\linewidth-2\fboxrule-4\fboxsep}\setlength{\parskip}{3mm}
			\raisebox{-2.5mm}{\sffamily \small{\textcolor{#1}{\MakeUppercase{#2}}}}		
			\par		
  			 #3
 	 		\end{minipage}
	}}
		\vspace{2mm}
	\par
}

\newcommand\bloc[3]{				% Boites convertible html sans bordure
     \needspace{2\baselineskip}
     {\sffamily \small{\textcolor{#1}{\MakeUppercase{#2}}}}    
		\par		
  			 #3
		\par
}

\newcommand\CHelp[1]{
     \CBox{Plum}{\faInfoCircle}{À RETENIR}{#1}
}

\newcommand\CUp[1]{
     \CBox{NavyBlue}{\faThumbsOUp}{EN PRATIQUE}{#1}
}

\newcommand\CInfo[1]{
     \CBox{Sepia}{\faArrowCircleRight}{REMARQUE}{#1}
}

\newcommand\CRedac[1]{
     \CBox{PineGreen}{\faEdit}{BIEN R\'EDIGER}{#1}
}

\newcommand\CError[1]{
     \CBox{Red}{\faExclamationTriangle}{ATTENTION}{#1}
}

\newcommand\TitreExo[2]{
\needspace{4\baselineskip}
 {\sffamily\large EXERCICE #1\ (\emph{#2 points})}
\vspace{5mm}
}

\newcommand\img[2]{
          \includegraphics[width=#2\paperwidth]{\imgdir#1}
}

\newcommand\imgsvg[2]{
       \begin{center}   \includegraphics[width=#2\paperwidth]{\imgsvgdir#1} \end{center}
}


\newcommand\Lien[2]{
     \href{#1}{#2 \tiny \faExternalLink}
}
\newcommand\mcLien[2]{
     \href{https~://www.maths-cours.fr/#1}{#2 \tiny \faExternalLink}
}

\newcommand{\euro}{\eurologo{}}

%================================================================================================================================
%
% Macros - Environement
%
%================================================================================================================================

\newenvironment{tex}{ %
}
{%
}

\newenvironment{indente}{ %
	\setlength\parindent{10mm}
}

{
	\setlength\parindent{0mm}
}

\newenvironment{corrige}{%
     \needspace{3\baselineskip}
     \medskip
     \textbf{\textsc{Corrigé}}
     \medskip
}
{
}

\newenvironment{extern}{%
     \begin{center}
     }
     {
     \end{center}
}

\NewEnviron{code}{%
	\par
     \boite{gray}{\texttt{%
     \BODY
     }}
     \par
}

\newenvironment{vbloc}{% boite sans cadre empeche saut de page
     \begin{minipage}[t]{\linewidth}
     }
     {
     \end{minipage}
}
\NewEnviron{h2}{%
    \needspace{3\baselineskip}
    \vspace{0.6cm}
	\noindent \MakeUppercase{\sffamily \large \BODY}
	\vspace{1mm}\textcolor{mcgris}{\hrule}\vspace{0.4cm}
	\par
}{}

\NewEnviron{h3}{%
    \needspace{3\baselineskip}
	\vspace{5mm}
	\textsc{\BODY}
	\par
}

\NewEnviron{margeneg}{ %
\begin{addmargin}[-1cm]{0cm}
\BODY
\end{addmargin}
}

\NewEnviron{html}{%
}

\begin{document}
\meta{url}{/exercices/suites-bac-blanc-es-l-sujet-2-maths-cours-2018/}
\meta{pid}{10455}
\meta{titre}{Suites - Bac blanc ES/L Sujet 2 - Maths-cours 2018}
\meta{type}{exercices}
%
\begin{h2}Exercice 4 (5 points)\end{h2}
\par
On considère la suite $(u_n)$ définie par $u_0=250$ et, pour tout entier naturel $n$ :
\[u_{n+1}=0,8u_n+60.\]
\begin{enumerate}
     \item Calculer $u_1$ et $u_2$.
     \item
     Compléter l'algorithme ci-après afin qu'il affiche le plus petit entier naturel $n$ tel que $u_n \geqslant 290$.
     \begin{center}
          \begin{extern}%width="340" alt="Algorithme de calcul de la somme S10"
               \begin{tabular}{|l l|}\hline
                    Variables :	& $N$ est un entier naturel\\
                    & $U$ est un nombre réel\\
                    & \\
                    Initialisation: &$U$ prend la valeur $250$ \\
                    &$N$ prend la valeur $0$ \\
                    & \\
                    Traitement: &Tant que ... faire\\
                    &\qquad$U$ prend la valeur ...\\
                    &\qquad$N$ prend la valeur ...\\
                    &Fin Tant que\\
                    & \\
                    Sortie :	&Afficher ... \\
                    \hline
               \end{tabular}
          \end{extern}
     \end{center}
     \item %3
     Soit la suite $(v_n)$ définie, pour tout entier naturel $n$, par :
     \[ v_n=u_n-300.\]
     \begin{enumerate}[label=\alph*.]
          \item %3a
          Montrer que la suite $(v_n)$ est une suite géométrique dont on précisera le premier terme et la raison.
          \item %3b
          Exprimer $v_n$ en fonction de $n$.
          \item %3c
          En déduire que pour tout entier naturel $n$ :
          \[ u_n=300-50 \times 0,8^n. \]
     \end{enumerate}
     \item %4
     \`A l'aide de la calculatrice, déterminer la valeur affichée par l'algorithme de la question \textbf{2.}
     \item %5
     Une ville organise chaque année un tournoi d'\'Echecs.
     En 2016, $200$ joueurs ont participé à ce tournoi.
     Les organisateurs font l'hypothèse que, d'une année sur l'autre :
     \begin{itemize}
          \item 20\% des joueurs ne reviennent pas l'année suivante,
          \item 60 nouveaux joueurs s'inscrivent au tournoi.
     \end{itemize}
     La taille de la salle dans laquelle se déroule le tournoi limite le nombre de joueurs à 320.
     Les organisateurs vont-ils devoir refuser des inscriptions par manque de places dans les années à venir ?
     Justifier la réponse.
\end{enumerate}
\begin{corrige}
     \begin{enumerate}
          \item %1
          Pour tout entier naturel $n$, ${u_{n+1}=0,8u_n+60}$ ; par conséquent :
          \par
          $u_{1}=0,8u_0+60=0,8 \times 250+60=260$.
          \par
          $u_{2}=0,8u_1+60=0,8 \times 260+60=268$.
          \item %2
          L'algorithme peut être complété de la façon suivante :
          \begin{center}
               \begin{extern}%width="380" alt="Algorithme de calcul de la somme S10"
                    \begin{tabular}{|l l|}\hline
                         Variables :	& $N$ est un entier naturel\\
                         & $U$ est un nombre réel\\
                         & \\
                         Initialisation: &$U$ prend la valeur $250$ \\
                         &$N$ prend la valeur $0$ \\
                         & \\
                         Traitement: &Tant que \textcolor{red}{\ $U<290$\ } faire\\
                         &\quad$U$ prend la valeur \textcolor{red}{\ $0,8U+60$}\\
                         &\quad$N$ prend la valeur \textcolor{red}{\ $N+1$\ }\\
                         &Fin Tant que\\
                         & \\
                         Sortie :	&Afficher \textcolor{red}{\ $N$\ } \\
                         \hline
                    \end{tabular}
               \end{extern}
          \end{center}
          (\textbf{Attention au sens de la condition} \og Tant que ${U<290}$ \fg{}. On veut que la boucle \og Tant que \fg{} \textbf{se termine} lorsque  $\bm{U \geqslant 290}$ ; on souhaite donc qu'elle \textbf{continue} à s'effectuer dans le cas contraire, c'est à dire tant que $\bm{U<290}$.)
          \item %3
          \begin{enumerate}[label=\alph*.]
               \item %3a
               Pour tout entier naturel $n$, $v_{n}= u_{n}-300$ ; par conséquent :
               \par
               $v_{n+1}= u_{n+1}-300$.
               \par
               Comme $u_{n+1}=0,8u_n + 60$ :
               \par
               $v_{n+1} = 0,8u_n+60-300$\\
               $\phantom{v_{n+1}} = 0,8u_n-240.$
               \par
               Puisque $v_{n}= u_{n}-300$, alors $u_{n}= v_{n}+300$. On en déduit :
               \par
               $v_{n+1} = 0,8(v_n+300)-240$\\
               $\phantom{v_{n+1}} = 0,8v_n+240-240$\\
               $\phantom{v_{n+1}} = 0,8v_n.$
               \par
               Par ailleurs :
               \par
               $v_{0}= u_{0}-300=250-300=-50$.
               \par
               La suite $(v_n)$ est une suite géométrique de premier terme ${v_0=-50}$ et de raison $0,8$.
               \item %3b
               La suite $(v_n)$ étant une suite géométrique :
               \par
               $v_n=v_0q^n=-50 \times 0,8^n$.
               \item %3c
               D'après les questions précédentes :
               \par
               $u_{n}= v_{n}+300 = 300 -50 \times 0,8^n$.
          \end{enumerate}
          \item %4
          \`A la calculatrice, on affiche un tableau de valeurs de la fonction $x \longmapsto 300 -50 \times 0,8^x$.
          \par
          On trouve alors :
          \par
          $u_7 \approx 289,51 \quad $ et $\quad u_8 \approx 291,61 $
          \par
          L'algorithme affiche le plus petit entier naturel $n$ tel que $u_n \geqslant 290$. L'algorithme affichera donc la valeur 8.
          \item %5
          Notons $a_n$ le nombre de joueurs inscrits au tournoi l'année $2016+n$.
          \par
          En 2016, 250 joueurs ont participé au tournoi donc $a_0=250$.
          \par
          Une diminution de 20\% correspond à un coefficient multiplicateur de ${1-\dfrac{20}{100}=0,8}$ ; on ajoute ensuite les 60 nouveaux inscrits.
          \par
          On a donc :
          \par
          \[ a_{n+1}=0,8a_n+60. \]
          \par
          Les suites $(u_n)$ et $(a_n)$ sont définies par la même relation de récurrence et le même premier terme ; elles sont donc identiques.
          \par
          Par conséquent, d'après la question \textbf{3.c.} :
          \par
          $a_{n}= 300 -50 \times 0,8^n$.
          \par
          Comme $50 \times 0,8^n$ est strictement positif pour tout entier $n$, le nombre $300 -50 \times 0,8^n$ est strictement inférieur à 300.
          \par
          Quelle que soit l'année, \textbf{le nombre d'inscrits sera inférieur à 300}. Les organisateurs \textbf{n'auront donc pas à refuser des inscriptions} par manque de places dans les années à venir.
     \end{enumerate}
\end{corrige}

\end{document}
µ
\documentclass[a4paper]{article}

%================================================================================================================================
%
% Packages
%
%================================================================================================================================

\usepackage[T1]{fontenc} 	% pour caractères accentués
\usepackage[utf8]{inputenc}  % encodage utf8
\usepackage[french]{babel}	% langue : français
\usepackage{fourier}			% caractères plus lisibles
\usepackage[dvipsnames]{xcolor} % couleurs
\usepackage{fancyhdr}		% réglage header footer
\usepackage{needspace}		% empêcher sauts de page mal placés
\usepackage{graphicx}		% pour inclure des graphiques
\usepackage{enumitem,cprotect}		% personnalise les listes d'items (nécessaire pour ol, al ...)
\usepackage{hyperref}		% Liens hypertexte
\usepackage{pstricks,pst-all,pst-node,pstricks-add,pst-math,pst-plot,pst-tree,pst-eucl} % pstricks
\usepackage[a4paper,includeheadfoot,top=2cm,left=3cm, bottom=2cm,right=3cm]{geometry} % marges etc.
\usepackage{comment}			% commentaires multilignes
\usepackage{amsmath,environ} % maths (matrices, etc.)
\usepackage{amssymb,makeidx}
\usepackage{bm}				% bold maths
\usepackage{tabularx}		% tableaux
\usepackage{colortbl}		% tableaux en couleur
\usepackage{fontawesome}		% Fontawesome
\usepackage{environ}			% environment with command
\usepackage{fp}				% calculs pour ps-tricks
\usepackage{multido}			% pour ps tricks
\usepackage[np]{numprint}	% formattage nombre
\usepackage{tikz,tkz-tab} 			% package principal TikZ
\usepackage{pgfplots}   % axes
\usepackage{mathrsfs}    % cursives
\usepackage{calc}			% calcul taille boites
\usepackage[scaled=0.875]{helvet} % font sans serif
\usepackage{svg} % svg
\usepackage{scrextend} % local margin
\usepackage{scratch} %scratch
\usepackage{multicol} % colonnes
%\usepackage{infix-RPN,pst-func} % formule en notation polanaise inversée
\usepackage{listings}

%================================================================================================================================
%
% Réglages de base
%
%================================================================================================================================

\lstset{
language=Python,   % R code
literate=
{á}{{\'a}}1
{à}{{\`a}}1
{ã}{{\~a}}1
{é}{{\'e}}1
{è}{{\`e}}1
{ê}{{\^e}}1
{í}{{\'i}}1
{ó}{{\'o}}1
{õ}{{\~o}}1
{ú}{{\'u}}1
{ü}{{\"u}}1
{ç}{{\c{c}}}1
{~}{{ }}1
}


\definecolor{codegreen}{rgb}{0,0.6,0}
\definecolor{codegray}{rgb}{0.5,0.5,0.5}
\definecolor{codepurple}{rgb}{0.58,0,0.82}
\definecolor{backcolour}{rgb}{0.95,0.95,0.92}

\lstdefinestyle{mystyle}{
    backgroundcolor=\color{backcolour},   
    commentstyle=\color{codegreen},
    keywordstyle=\color{magenta},
    numberstyle=\tiny\color{codegray},
    stringstyle=\color{codepurple},
    basicstyle=\ttfamily\footnotesize,
    breakatwhitespace=false,         
    breaklines=true,                 
    captionpos=b,                    
    keepspaces=true,                 
    numbers=left,                    
xleftmargin=2em,
framexleftmargin=2em,            
    showspaces=false,                
    showstringspaces=false,
    showtabs=false,                  
    tabsize=2,
    upquote=true
}

\lstset{style=mystyle}


\lstset{style=mystyle}
\newcommand{\imgdir}{C:/laragon/www/newmc/assets/imgsvg/}
\newcommand{\imgsvgdir}{C:/laragon/www/newmc/assets/imgsvg/}

\definecolor{mcgris}{RGB}{220, 220, 220}% ancien~; pour compatibilité
\definecolor{mcbleu}{RGB}{52, 152, 219}
\definecolor{mcvert}{RGB}{125, 194, 70}
\definecolor{mcmauve}{RGB}{154, 0, 215}
\definecolor{mcorange}{RGB}{255, 96, 0}
\definecolor{mcturquoise}{RGB}{0, 153, 153}
\definecolor{mcrouge}{RGB}{255, 0, 0}
\definecolor{mclightvert}{RGB}{205, 234, 190}

\definecolor{gris}{RGB}{220, 220, 220}
\definecolor{bleu}{RGB}{52, 152, 219}
\definecolor{vert}{RGB}{125, 194, 70}
\definecolor{mauve}{RGB}{154, 0, 215}
\definecolor{orange}{RGB}{255, 96, 0}
\definecolor{turquoise}{RGB}{0, 153, 153}
\definecolor{rouge}{RGB}{255, 0, 0}
\definecolor{lightvert}{RGB}{205, 234, 190}
\setitemize[0]{label=\color{lightvert}  $\bullet$}

\pagestyle{fancy}
\renewcommand{\headrulewidth}{0.2pt}
\fancyhead[L]{maths-cours.fr}
\fancyhead[R]{\thepage}
\renewcommand{\footrulewidth}{0.2pt}
\fancyfoot[C]{}

\newcolumntype{C}{>{\centering\arraybackslash}X}
\newcolumntype{s}{>{\hsize=.35\hsize\arraybackslash}X}

\setlength{\parindent}{0pt}		 
\setlength{\parskip}{3mm}
\setlength{\headheight}{1cm}

\def\ebook{ebook}
\def\book{book}
\def\web{web}
\def\type{web}

\newcommand{\vect}[1]{\overrightarrow{\,\mathstrut#1\,}}

\def\Oij{$\left(\text{O}~;~\vect{\imath},~\vect{\jmath}\right)$}
\def\Oijk{$\left(\text{O}~;~\vect{\imath},~\vect{\jmath},~\vect{k}\right)$}
\def\Ouv{$\left(\text{O}~;~\vect{u},~\vect{v}\right)$}

\hypersetup{breaklinks=true, colorlinks = true, linkcolor = OliveGreen, urlcolor = OliveGreen, citecolor = OliveGreen, pdfauthor={Didier BONNEL - https://www.maths-cours.fr} } % supprime les bordures autour des liens

\renewcommand{\arg}[0]{\text{arg}}

\everymath{\displaystyle}

%================================================================================================================================
%
% Macros - Commandes
%
%================================================================================================================================

\newcommand\meta[2]{    			% Utilisé pour créer le post HTML.
	\def\titre{titre}
	\def\url{url}
	\def\arg{#1}
	\ifx\titre\arg
		\newcommand\maintitle{#2}
		\fancyhead[L]{#2}
		{\Large\sffamily \MakeUppercase{#2}}
		\vspace{1mm}\textcolor{mcvert}{\hrule}
	\fi 
	\ifx\url\arg
		\fancyfoot[L]{\href{https://www.maths-cours.fr#2}{\black \footnotesize{https://www.maths-cours.fr#2}}}
	\fi 
}


\newcommand\TitreC[1]{    		% Titre centré
     \needspace{3\baselineskip}
     \begin{center}\textbf{#1}\end{center}
}

\newcommand\newpar{    		% paragraphe
     \par
}

\newcommand\nosp {    		% commande vide (pas d'espace)
}
\newcommand{\id}[1]{} %ignore

\newcommand\boite[2]{				% Boite simple sans titre
	\vspace{5mm}
	\setlength{\fboxrule}{0.2mm}
	\setlength{\fboxsep}{5mm}	
	\fcolorbox{#1}{#1!3}{\makebox[\linewidth-2\fboxrule-2\fboxsep]{
  		\begin{minipage}[t]{\linewidth-2\fboxrule-4\fboxsep}\setlength{\parskip}{3mm}
  			 #2
  		\end{minipage}
	}}
	\vspace{5mm}
}

\newcommand\CBox[4]{				% Boites
	\vspace{5mm}
	\setlength{\fboxrule}{0.2mm}
	\setlength{\fboxsep}{5mm}
	
	\fcolorbox{#1}{#1!3}{\makebox[\linewidth-2\fboxrule-2\fboxsep]{
		\begin{minipage}[t]{1cm}\setlength{\parskip}{3mm}
	  		\textcolor{#1}{\LARGE{#2}}    
 	 	\end{minipage}  
  		\begin{minipage}[t]{\linewidth-2\fboxrule-4\fboxsep}\setlength{\parskip}{3mm}
			\raisebox{1.2mm}{\normalsize\sffamily{\textcolor{#1}{#3}}}						
  			 #4
  		\end{minipage}
	}}
	\vspace{5mm}
}

\newcommand\cadre[3]{				% Boites convertible html
	\par
	\vspace{2mm}
	\setlength{\fboxrule}{0.1mm}
	\setlength{\fboxsep}{5mm}
	\fcolorbox{#1}{white}{\makebox[\linewidth-2\fboxrule-2\fboxsep]{
  		\begin{minipage}[t]{\linewidth-2\fboxrule-4\fboxsep}\setlength{\parskip}{3mm}
			\raisebox{-2.5mm}{\sffamily \small{\textcolor{#1}{\MakeUppercase{#2}}}}		
			\par		
  			 #3
 	 		\end{minipage}
	}}
		\vspace{2mm}
	\par
}

\newcommand\bloc[3]{				% Boites convertible html sans bordure
     \needspace{2\baselineskip}
     {\sffamily \small{\textcolor{#1}{\MakeUppercase{#2}}}}    
		\par		
  			 #3
		\par
}

\newcommand\CHelp[1]{
     \CBox{Plum}{\faInfoCircle}{À RETENIR}{#1}
}

\newcommand\CUp[1]{
     \CBox{NavyBlue}{\faThumbsOUp}{EN PRATIQUE}{#1}
}

\newcommand\CInfo[1]{
     \CBox{Sepia}{\faArrowCircleRight}{REMARQUE}{#1}
}

\newcommand\CRedac[1]{
     \CBox{PineGreen}{\faEdit}{BIEN R\'EDIGER}{#1}
}

\newcommand\CError[1]{
     \CBox{Red}{\faExclamationTriangle}{ATTENTION}{#1}
}

\newcommand\TitreExo[2]{
\needspace{4\baselineskip}
 {\sffamily\large EXERCICE #1\ (\emph{#2 points})}
\vspace{5mm}
}

\newcommand\img[2]{
          \includegraphics[width=#2\paperwidth]{\imgdir#1}
}

\newcommand\imgsvg[2]{
       \begin{center}   \includegraphics[width=#2\paperwidth]{\imgsvgdir#1} \end{center}
}


\newcommand\Lien[2]{
     \href{#1}{#2 \tiny \faExternalLink}
}
\newcommand\mcLien[2]{
     \href{https~://www.maths-cours.fr/#1}{#2 \tiny \faExternalLink}
}

\newcommand{\euro}{\eurologo{}}

%================================================================================================================================
%
% Macros - Environement
%
%================================================================================================================================

\newenvironment{tex}{ %
}
{%
}

\newenvironment{indente}{ %
	\setlength\parindent{10mm}
}

{
	\setlength\parindent{0mm}
}

\newenvironment{corrige}{%
     \needspace{3\baselineskip}
     \medskip
     \textbf{\textsc{Corrigé}}
     \medskip
}
{
}

\newenvironment{extern}{%
     \begin{center}
     }
     {
     \end{center}
}

\NewEnviron{code}{%
	\par
     \boite{gray}{\texttt{%
     \BODY
     }}
     \par
}

\newenvironment{vbloc}{% boite sans cadre empeche saut de page
     \begin{minipage}[t]{\linewidth}
     }
     {
     \end{minipage}
}
\NewEnviron{h2}{%
    \needspace{3\baselineskip}
    \vspace{0.6cm}
	\noindent \MakeUppercase{\sffamily \large \BODY}
	\vspace{1mm}\textcolor{mcgris}{\hrule}\vspace{0.4cm}
	\par
}{}

\NewEnviron{h3}{%
    \needspace{3\baselineskip}
	\vspace{5mm}
	\textsc{\BODY}
	\par
}

\NewEnviron{margeneg}{ %
\begin{addmargin}[-1cm]{0cm}
\BODY
\end{addmargin}
}

\NewEnviron{html}{%
}

\begin{document}
\meta{url}{/exercices/graphes-bac-blanc-es-sujet-2-maths-cours-2018-spe/}
\meta{pid}{10457}
\meta{titre}{Graphes - Bac blanc ES Sujet 2 - Maths-cours 2018 (spé)}
\meta{type}{exercices}
%
\begin{h2}Exercice 4 (5 points)\end{h2}
\par
\textbf{Candidats ayant suivi l'enseignement de spécialité}
\par
Une agence de tourisme propose la visite de certains monuments parisiens.
\par
Chacun de ces monuments est désigné par une lettre comme suit :
\par
\begin{itemize}
     \item %
     E : Tour Eiffel
     \item %
     L : Musée du Louvre
     \item %
     M : Tour Montparnasse
     \item %
     N : Cathédrale Notre-Dame de Paris
     \item %
     S : Basilique du Sacré-Cœur de Montmartre
     \item %
     T : Arc de triomphe
\end{itemize}
\par
Cette agence fait appel à une société de transport par autocar qui propose les liaisons suivantes (chacune de ces liaisons pouvant s'effectuer dans les deux sens de circulation) :
\begin{center}
     \begin{extern}%width="300"
          \psset{unit=0.7cm}
          \begin{pspicture}(8,18)(21,5)
               %\psgrid
               \rput(10,10){\circlenode{E}{E}}
               \rput(16,11){\circlenode{L}{L}}
               \rput(16,6){\circlenode{M}{M}}
               \rput(19,9){\circlenode{N}{N}}
               \rput(17,17){\circlenode{S}{S}}
               \rput(10,14){\circlenode{T}{T}}
               \ncarc[arcangle=20]{E}{L}
               \ncarc[arcangle=-30]{M}{L}
               \ncarc[arcangle=-30]{M}{N}
               \ncarc[arcangle=20]{L}{N}
               \ncarc[arcangle=20]{E}{S}
               \ncarc[arcangle=20]{L}{S}
               \ncarc[arcangle=-30]{T}{E}
               \ncarc[arcangle=20]{T}{S}
               \ncarc[arcangle=-50]{E}{M}
               \ncarc[arcangle=20]{S}{N}
          \end{pspicture}
     \end{extern}
\end{center}
\begin{enumerate}
     \item %1
     Expliquer pourquoi il est possible de trouver un trajet empruntant une fois et une seule chacune des dix liaisons indiquées sur le graphe.\\
     Donner un exemple d'un tel trajet.
     \item %2
     \begin{enumerate}[label=\alph*.]
          \item %2a
          Donner la matrice d'adjacence $M$ associée à ce graphe en classant les sommets par ordre alphabétique.
          \item %2b
          On donne :
          \par
          \[ M^2 = \begin{pmatrix}
               4 &2 &1 &3 &2 &1 \\
               2 &4 &2 &2 &2 &2 \\
               1 &2 &3 &1 &3 &1 \\
               3 &2 &1 &3 &1 &1 \\
               2 &2 &3 &1 &4 &1\\
          1 &2 &1 &1 &1 &2  \end{pmatrix}\]
          \[ M^3 = \begin{pmatrix}
               6 &10 &9 &5 &10 &6 \\
               10 &8 &8 &8 &10 &4 \\
               9 &8 &4 &8 &5 &4 \\
               5 &8 &8 &4 &9 &4 \\
               10 &10 &5 &9 &6 &6\\
          6 &4 &4 &4 &6 &2  \end{pmatrix}\]
          \par
          Combien y a-t-il de trajets permettant de relier la cathédrale Notre-Dame de Paris et la tour Eiffel en utilisant au maximum trois liaisons.\\
          Justifier votre réponse.
     \end{enumerate}
     \item %3
     On complète le graphe précédent en indiquant, sur chacune des branches, la durée du trajet, en minutes, entre deux monuments.
     \begin{center}
          \begin{extern}%width="300"
               \psset{unit=0.7cm}
               \begin{pspicture}(8,18)(21,5)
                    \rput(10,10){\circlenode{E}{E}}
                    \rput(16,11){\circlenode{L}{L}}
                    \rput(16,6){\circlenode{M}{M}}
                    \rput(19,9){\circlenode{N}{N}}
                    \rput(17,17){\circlenode{S}{S}}
                    \rput(10,14){\circlenode{T}{T}}
                    \ncarc[arcangle=20]{E}{L}\ncput*[nrot=:U]{8}
                    \ncarc[arcangle=-30]{M}{L}\ncput*[nrot=:U]{7}
                    \ncarc[arcangle=-30]{M}{N}\ncput*[nrot=:U]{4}
                    \ncarc[arcangle=20]{L}{N}\ncput*[nrot=:U]{2}
                    \ncarc[arcangle=20]{E}{S}\ncput*[nrot=:U]{10}
                    \ncarc[arcangle=20]{L}{S}\ncput*[nrot=:U]{5}
                    \ncarc[arcangle=-30]{T}{E}\ncput*[nrot=:U]{4}
                    \ncarc[arcangle=20]{T}{S}\ncput*[nrot=:U]{8}
                    \ncarc[arcangle=-50]{E}{M}\ncput*[nrot=:U]{10}
                    \ncarc[arcangle=20]{S}{N}\ncput*[nrot=:U]{8}
               \end{pspicture}
          \end{extern}
     \end{center}
     On souhaite aller de la tour Montparnasse à la Basilique du Sacré-Cœur de Montmartre.
     \par
     En utilisant un algorithme, déterminer le trajet le plus rapide ainsi que la durée de ce trajet.
\end{enumerate}
\begin{corrige}
     \begin{enumerate}
          \item %1
          Trouver un trajet qui emprunte une fois et une seule chacune des liaisons indiquées sur le graphe revient à déterminer une chaîne \textbf{eulérienne}.
          \par
          Le théorème d'Euler indique qu'un graphe connexe contient une chaîne eulérienne si et seulement s'il ne possède que 0 ou 2 sommets de degré impair.
          \par
          Les degrés des sommets sont indiqués dans le tableau ci-après :
          \par
          \begin{center}
               \begin{tabular}{|l|c|c|c|c|c|c|c|c|} %class="compact"
                    \hline
                    Sommet & E & L & M & N & S & T  \\
                    \hline
                    Degré & 4 & 4 & 3 & 3 & 4 & 2 \\
                    \hline
               \end{tabular}
          \end{center}
          \par
          Le graphe comporte deux sommets de degré impair : M et N. Il est donc possible de relier M (Tour Montparnasse) à N (Cathédrale Notre-Dame de Paris) en empruntant une fois et une seule chacune des liaisons; par exemple : M-E-T-S-E-L-S-N-L-M-N.
          \item %2
          \begin{enumerate}[label=\alph*.]
               \item %2a
               En classant les sommets par ordre alphabétique, on obtient la matrice d'adjacence suivante :
               \par
               \[ M = \begin{pmatrix}
                    0 &1 &1 &0 &1 &1 \\
                    1 &0 &1 &1 &1 &0 \\
                    1 &1 &0 &1 &0 &0 \\
                    0 &1 &1 &0 &1 &0 \\
                    1 &1 &0 &1 &0 &1\\
               1 &0 &0 &0 &1 &0  \end{pmatrix}
               \]
               \item %2b
               Le coefficient situé sur la $4^{\text{e}}$ ligne (correspondant à la cathédrale Notre-Dame de Paris) et la $1^{\text{ère}}$ colonne (correspondant à la tour Eiffel) de la matrice $M$ est égal à 0.\\
               Il n'y a donc \textbf{aucun} trajet reliant la cathédrale Notre-Dame de Paris et la tour Eiffel en utilisant \textbf{une} et une seule liaison.
               \par
               Le coefficient situé sur la $4^{\text{e}}$ ligne et la $1^{\text{ère}}$ colonne de la matrice $M^2$ est égal à 3.\\
               Il y a donc \textbf{3} trajets reliant la cathédrale Notre-Dame de Paris et la tour Eiffel en utilisant exactement \textbf{deux} liaisons.
               \par
               Le coefficient situé sur la $4^{\text{e}}$ ligne et la $1^{\text{ère}}$ colonne de la matrice $M^3$ est égal à 5.\\
               Il y a donc \textbf{5} trajets reliant la cathédrale Notre-Dame de Paris et la tour Eiffel en utilisant exactement \textbf{trois} liaisons.
               \par
               Au total, on trouve qu'il existe exactement \textbf{huit trajets} permettant de relier la cathédrale Notre-Dame de Paris et la tour Eiffel en utilisant au maximum trois liaisons.
               \par
          \end{enumerate}
          \item %3
          On utilise l'algorithme de Dijkstra :
          \begin{center}
               \begin{extern}
                    \begin{tabularx}{0.9\linewidth}{|c|C|C|C|C|C|C|}
                         \hline
                         \			&  E 						& L							& M							& N 							& S								& T  						\\ \hline
                         Départ			&  $\infty$	 				& $\infty$					& $\color{red}0_{\text{M}}$	& $\infty$					& $\infty$						& $\infty$	  				\\ \hline
                         M (0) 			&  $10_{\text{M}}$	 		& $7_{\text{M}}$	 			& \cellcolor{black!20}		& $\color{red}4_{\text{M}}$	& $\infty$						& $\infty$ 					\\ \hline
                         N (4)			&  $10_{\text{M}}$	 		& $\color{red}6_{\text{N}}$	& \cellcolor{black!20}		& \cellcolor{black!20}		& $12_{\text{N}}$				& $\infty$ 					\\ \hline
                         L (6)			&  $\color{red}10_{\text{M}}$	& \cellcolor{black!20}		& \cellcolor{black!20}		& \cellcolor{black!20}		& $11_{\text{L}}$				& $\infty$ 					\\ \hline
                         E (10)			&  \cellcolor{black!20}		& \cellcolor{black!20}		& \cellcolor{black!20}		& \cellcolor{black!20}		& $\color{red}11_{\text{L}}$		& $14_{\text{E}}$	 		\\ \hline
                    \end{tabularx}
               \end{extern}
          \end{center}
          \textit{Reportez-vous à la page \og méthode \fg{}~: \mcLien{/methode/algorithme-de-dijkstra-etape-par-etape/}{ l'algorithme de Dijkstra étape par étape} pour obtenir la méthode de construction détaillée de ce tableau.}
          \par
          Le trajet le plus rapide pour aller de la tour Montparnasse à la Basilique du Sacré-Cœur de Montmartre est le trajet M-N-L-S, c'est à dire Montparnasse - Notre-Dame de Paris - Louvre - Sacré-Cœur.
          \par
          Sa durée est de \textbf{11 minutes}.
          \par
     \end{enumerate}
     \par
\end{corrige}

\end{document}
µ
\documentclass[a4paper]{article}

%================================================================================================================================
%
% Packages
%
%================================================================================================================================

\usepackage[T1]{fontenc} 	% pour caractères accentués
\usepackage[utf8]{inputenc}  % encodage utf8
\usepackage[french]{babel}	% langue : français
\usepackage{fourier}			% caractères plus lisibles
\usepackage[dvipsnames]{xcolor} % couleurs
\usepackage{fancyhdr}		% réglage header footer
\usepackage{needspace}		% empêcher sauts de page mal placés
\usepackage{graphicx}		% pour inclure des graphiques
\usepackage{enumitem,cprotect}		% personnalise les listes d'items (nécessaire pour ol, al ...)
\usepackage{hyperref}		% Liens hypertexte
\usepackage{pstricks,pst-all,pst-node,pstricks-add,pst-math,pst-plot,pst-tree,pst-eucl} % pstricks
\usepackage[a4paper,includeheadfoot,top=2cm,left=3cm, bottom=2cm,right=3cm]{geometry} % marges etc.
\usepackage{comment}			% commentaires multilignes
\usepackage{amsmath,environ} % maths (matrices, etc.)
\usepackage{amssymb,makeidx}
\usepackage{bm}				% bold maths
\usepackage{tabularx}		% tableaux
\usepackage{colortbl}		% tableaux en couleur
\usepackage{fontawesome}		% Fontawesome
\usepackage{environ}			% environment with command
\usepackage{fp}				% calculs pour ps-tricks
\usepackage{multido}			% pour ps tricks
\usepackage[np]{numprint}	% formattage nombre
\usepackage{tikz,tkz-tab} 			% package principal TikZ
\usepackage{pgfplots}   % axes
\usepackage{mathrsfs}    % cursives
\usepackage{calc}			% calcul taille boites
\usepackage[scaled=0.875]{helvet} % font sans serif
\usepackage{svg} % svg
\usepackage{scrextend} % local margin
\usepackage{scratch} %scratch
\usepackage{multicol} % colonnes
%\usepackage{infix-RPN,pst-func} % formule en notation polanaise inversée
\usepackage{listings}

%================================================================================================================================
%
% Réglages de base
%
%================================================================================================================================

\lstset{
language=Python,   % R code
literate=
{á}{{\'a}}1
{à}{{\`a}}1
{ã}{{\~a}}1
{é}{{\'e}}1
{è}{{\`e}}1
{ê}{{\^e}}1
{í}{{\'i}}1
{ó}{{\'o}}1
{õ}{{\~o}}1
{ú}{{\'u}}1
{ü}{{\"u}}1
{ç}{{\c{c}}}1
{~}{{ }}1
}


\definecolor{codegreen}{rgb}{0,0.6,0}
\definecolor{codegray}{rgb}{0.5,0.5,0.5}
\definecolor{codepurple}{rgb}{0.58,0,0.82}
\definecolor{backcolour}{rgb}{0.95,0.95,0.92}

\lstdefinestyle{mystyle}{
    backgroundcolor=\color{backcolour},   
    commentstyle=\color{codegreen},
    keywordstyle=\color{magenta},
    numberstyle=\tiny\color{codegray},
    stringstyle=\color{codepurple},
    basicstyle=\ttfamily\footnotesize,
    breakatwhitespace=false,         
    breaklines=true,                 
    captionpos=b,                    
    keepspaces=true,                 
    numbers=left,                    
xleftmargin=2em,
framexleftmargin=2em,            
    showspaces=false,                
    showstringspaces=false,
    showtabs=false,                  
    tabsize=2,
    upquote=true
}

\lstset{style=mystyle}


\lstset{style=mystyle}
\newcommand{\imgdir}{C:/laragon/www/newmc/assets/imgsvg/}
\newcommand{\imgsvgdir}{C:/laragon/www/newmc/assets/imgsvg/}

\definecolor{mcgris}{RGB}{220, 220, 220}% ancien~; pour compatibilité
\definecolor{mcbleu}{RGB}{52, 152, 219}
\definecolor{mcvert}{RGB}{125, 194, 70}
\definecolor{mcmauve}{RGB}{154, 0, 215}
\definecolor{mcorange}{RGB}{255, 96, 0}
\definecolor{mcturquoise}{RGB}{0, 153, 153}
\definecolor{mcrouge}{RGB}{255, 0, 0}
\definecolor{mclightvert}{RGB}{205, 234, 190}

\definecolor{gris}{RGB}{220, 220, 220}
\definecolor{bleu}{RGB}{52, 152, 219}
\definecolor{vert}{RGB}{125, 194, 70}
\definecolor{mauve}{RGB}{154, 0, 215}
\definecolor{orange}{RGB}{255, 96, 0}
\definecolor{turquoise}{RGB}{0, 153, 153}
\definecolor{rouge}{RGB}{255, 0, 0}
\definecolor{lightvert}{RGB}{205, 234, 190}
\setitemize[0]{label=\color{lightvert}  $\bullet$}

\pagestyle{fancy}
\renewcommand{\headrulewidth}{0.2pt}
\fancyhead[L]{maths-cours.fr}
\fancyhead[R]{\thepage}
\renewcommand{\footrulewidth}{0.2pt}
\fancyfoot[C]{}

\newcolumntype{C}{>{\centering\arraybackslash}X}
\newcolumntype{s}{>{\hsize=.35\hsize\arraybackslash}X}

\setlength{\parindent}{0pt}		 
\setlength{\parskip}{3mm}
\setlength{\headheight}{1cm}

\def\ebook{ebook}
\def\book{book}
\def\web{web}
\def\type{web}

\newcommand{\vect}[1]{\overrightarrow{\,\mathstrut#1\,}}

\def\Oij{$\left(\text{O}~;~\vect{\imath},~\vect{\jmath}\right)$}
\def\Oijk{$\left(\text{O}~;~\vect{\imath},~\vect{\jmath},~\vect{k}\right)$}
\def\Ouv{$\left(\text{O}~;~\vect{u},~\vect{v}\right)$}

\hypersetup{breaklinks=true, colorlinks = true, linkcolor = OliveGreen, urlcolor = OliveGreen, citecolor = OliveGreen, pdfauthor={Didier BONNEL - https://www.maths-cours.fr} } % supprime les bordures autour des liens

\renewcommand{\arg}[0]{\text{arg}}

\everymath{\displaystyle}

%================================================================================================================================
%
% Macros - Commandes
%
%================================================================================================================================

\newcommand\meta[2]{    			% Utilisé pour créer le post HTML.
	\def\titre{titre}
	\def\url{url}
	\def\arg{#1}
	\ifx\titre\arg
		\newcommand\maintitle{#2}
		\fancyhead[L]{#2}
		{\Large\sffamily \MakeUppercase{#2}}
		\vspace{1mm}\textcolor{mcvert}{\hrule}
	\fi 
	\ifx\url\arg
		\fancyfoot[L]{\href{https://www.maths-cours.fr#2}{\black \footnotesize{https://www.maths-cours.fr#2}}}
	\fi 
}


\newcommand\TitreC[1]{    		% Titre centré
     \needspace{3\baselineskip}
     \begin{center}\textbf{#1}\end{center}
}

\newcommand\newpar{    		% paragraphe
     \par
}

\newcommand\nosp {    		% commande vide (pas d'espace)
}
\newcommand{\id}[1]{} %ignore

\newcommand\boite[2]{				% Boite simple sans titre
	\vspace{5mm}
	\setlength{\fboxrule}{0.2mm}
	\setlength{\fboxsep}{5mm}	
	\fcolorbox{#1}{#1!3}{\makebox[\linewidth-2\fboxrule-2\fboxsep]{
  		\begin{minipage}[t]{\linewidth-2\fboxrule-4\fboxsep}\setlength{\parskip}{3mm}
  			 #2
  		\end{minipage}
	}}
	\vspace{5mm}
}

\newcommand\CBox[4]{				% Boites
	\vspace{5mm}
	\setlength{\fboxrule}{0.2mm}
	\setlength{\fboxsep}{5mm}
	
	\fcolorbox{#1}{#1!3}{\makebox[\linewidth-2\fboxrule-2\fboxsep]{
		\begin{minipage}[t]{1cm}\setlength{\parskip}{3mm}
	  		\textcolor{#1}{\LARGE{#2}}    
 	 	\end{minipage}  
  		\begin{minipage}[t]{\linewidth-2\fboxrule-4\fboxsep}\setlength{\parskip}{3mm}
			\raisebox{1.2mm}{\normalsize\sffamily{\textcolor{#1}{#3}}}						
  			 #4
  		\end{minipage}
	}}
	\vspace{5mm}
}

\newcommand\cadre[3]{				% Boites convertible html
	\par
	\vspace{2mm}
	\setlength{\fboxrule}{0.1mm}
	\setlength{\fboxsep}{5mm}
	\fcolorbox{#1}{white}{\makebox[\linewidth-2\fboxrule-2\fboxsep]{
  		\begin{minipage}[t]{\linewidth-2\fboxrule-4\fboxsep}\setlength{\parskip}{3mm}
			\raisebox{-2.5mm}{\sffamily \small{\textcolor{#1}{\MakeUppercase{#2}}}}		
			\par		
  			 #3
 	 		\end{minipage}
	}}
		\vspace{2mm}
	\par
}

\newcommand\bloc[3]{				% Boites convertible html sans bordure
     \needspace{2\baselineskip}
     {\sffamily \small{\textcolor{#1}{\MakeUppercase{#2}}}}    
		\par		
  			 #3
		\par
}

\newcommand\CHelp[1]{
     \CBox{Plum}{\faInfoCircle}{À RETENIR}{#1}
}

\newcommand\CUp[1]{
     \CBox{NavyBlue}{\faThumbsOUp}{EN PRATIQUE}{#1}
}

\newcommand\CInfo[1]{
     \CBox{Sepia}{\faArrowCircleRight}{REMARQUE}{#1}
}

\newcommand\CRedac[1]{
     \CBox{PineGreen}{\faEdit}{BIEN R\'EDIGER}{#1}
}

\newcommand\CError[1]{
     \CBox{Red}{\faExclamationTriangle}{ATTENTION}{#1}
}

\newcommand\TitreExo[2]{
\needspace{4\baselineskip}
 {\sffamily\large EXERCICE #1\ (\emph{#2 points})}
\vspace{5mm}
}

\newcommand\img[2]{
          \includegraphics[width=#2\paperwidth]{\imgdir#1}
}

\newcommand\imgsvg[2]{
       \begin{center}   \includegraphics[width=#2\paperwidth]{\imgsvgdir#1} \end{center}
}


\newcommand\Lien[2]{
     \href{#1}{#2 \tiny \faExternalLink}
}
\newcommand\mcLien[2]{
     \href{https~://www.maths-cours.fr/#1}{#2 \tiny \faExternalLink}
}

\newcommand{\euro}{\eurologo{}}

%================================================================================================================================
%
% Macros - Environement
%
%================================================================================================================================

\newenvironment{tex}{ %
}
{%
}

\newenvironment{indente}{ %
	\setlength\parindent{10mm}
}

{
	\setlength\parindent{0mm}
}

\newenvironment{corrige}{%
     \needspace{3\baselineskip}
     \medskip
     \textbf{\textsc{Corrigé}}
     \medskip
}
{
}

\newenvironment{extern}{%
     \begin{center}
     }
     {
     \end{center}
}

\NewEnviron{code}{%
	\par
     \boite{gray}{\texttt{%
     \BODY
     }}
     \par
}

\newenvironment{vbloc}{% boite sans cadre empeche saut de page
     \begin{minipage}[t]{\linewidth}
     }
     {
     \end{minipage}
}
\NewEnviron{h2}{%
    \needspace{3\baselineskip}
    \vspace{0.6cm}
	\noindent \MakeUppercase{\sffamily \large \BODY}
	\vspace{1mm}\textcolor{mcgris}{\hrule}\vspace{0.4cm}
	\par
}{}

\NewEnviron{h3}{%
    \needspace{3\baselineskip}
	\vspace{5mm}
	\textsc{\BODY}
	\par
}

\NewEnviron{margeneg}{ %
\begin{addmargin}[-1cm]{0cm}
\BODY
\end{addmargin}
}

\NewEnviron{html}{%
}

\begin{document}
\meta{url}{/methode/resoudre-un-systeme-de-2-equations-a-2-inconnues-par-substitution/}
\meta{pid}{10465}
\meta{titre}{Résoudre un système de 2 équations à 2 inconnues (par substitution)}
\meta{type}{methode}
%
\cadre{rouge}{Méthode}{ % id="m010"
     \begin{itemize}
          \item %
          \textbf{1ère étape~:} (facultative mais permet de simplifier les calculs)~:\\
          Rechercher l'équation dans laquelle il sera facile d'exprimer $y$ en fonction de $x$, ou $x$ en fonction de $y$.\\
          \textit{On supposera, dans l'explication qui suit, que l'on a choisi d'exprimer $y$ en fonction de $x$ dans la première équation.}
          \item %
          \textbf{2ème étape~:} \\
          Dans la première équation, exprimer $y$ en fonction de $x$.\\
          Ne pas modifier la seconde équation.
          \item %
          \textbf{3ème étape~:} \\
          Remplacer $y$ par l'expression trouvée précédemment dans la seconde équation.\\
          La seconde équation n'a alors plus qu'une seule inconnue $x.$
          \item %
          \textbf{4ème étape~:} \\
          Résoudre la seconde équation pour trouver $x.$
          \item %
          \textbf{5ème étape~:} \\
          Calculer $y$ en remplaçant $x$, dans la première équation,  par la valeur trouvée à l'étape précédente.
          \item %
          \textbf{6ème étape~:} \\
          Conclure en précisant la ou les couple(s) de solution(s).
     \end{itemize}
} % fin méth
\bloc{cyan}{Remarques}{ % id="r020"
     \begin{itemize}
          \item %
          Pour présenter les calculs, il est préférable de recopier à chaque étape un système équivalent au système de départ en réécrivant les deux équations, y compris celle que l'on n'a pas modifiée.
          \item %
          Un système admet souvent un unique couple solution mais peut aussi n'avoir aucune solution ou admettre une infinité de solutions (voir exemple 3 et 4).
     \end{itemize}
} % fin rem
\bloc{orange}{Exemple 1}{ % id="e030"
     Résoudre le système :
     \par
     $(S_1)~~\begin{cases} 3x+y=2 \\  5x+2y=3\end{cases}$
     \par
     \textbf{Solution~:}
     \begin{itemize}
          \item %
          \textbf{1ère étape~: Recherche de la méthode la plus rapide.}
          \par
          On remarque qu'ici,  il sera particulièrement simple d'exprimer $y$ en fonction de $x$ dans la première équation.
          \item %
          \textbf{2ème étape~: Expression de  $y$ en fonction de $x$.}
          \par
          Il suffit de \og faire passer \fg{} $3x$ dans l'autre membre dans la première équation~;
          \\on recopie la seconde équation sans y toucher.
          \par
          $(S_1)~\Leftrightarrow~\begin{cases} y=2-3x \\  5x+2y=3\end{cases}$
          \item %
          \textbf{3ème étape~: Remplacement de $y$.}
          \par
          On remplace $y$ par $(2-3x)$ dans la seconde équation (ne pas oublier la parenthèse~!).\\
          On ne touche pas à la première équation.
          \par
          $(S_1)~\Leftrightarrow~\begin{cases} y=2-3x \\  5x+2(2-3x)=3\end{cases}$
          \item %
          \textbf{4ème étape~: Calcul de $x.$}
          \par
          On résout la seconde équation (en recopiant à chaque fois la première à l'identique).
          \par
          $(S_1)~\Leftrightarrow~\begin{cases} y=2-3x \\  5x+4-6x=3\end{cases}$\\
          $(S_1)~\Leftrightarrow~\begin{cases} y=2-3x \\  -x=3-4\end{cases}$\\
          $(S_1)~\Leftrightarrow~\begin{cases} y=2-3x \\  x=1\end{cases}$\\
          \item %
          \textbf{5ème étape~: Calcul de $y.$}
          \par
          On  remplace $x$ par $1$ dans la première équation~:
          \par
          $(S_1)~\Leftrightarrow~\begin{cases} y=2-3\times 1\\  x=1\end{cases}$\\
          $(S_1)~\Leftrightarrow~\begin{cases} y=-1\\  x=1\end{cases}$\\
          \item %
          \textbf{6ème étape~: Conclusion. }
          \par
          Le couple $(1~;~-1)$ est l'unique couple solution du système $(S_1)$.
     \end{itemize}
} % fin ex
\bloc{orange}{Exemple 2}{ % id="e040"
     Résoudre le système :
     \par
     $(S_1)~~\begin{cases} 5x-2y=1 \\  x+3y=7\end{cases}$
     \par
     \textbf{Solution~:}
     \begin{itemize}
          \item %
          \textbf{1ère étape~: Recherche de la méthode la plus rapide.}
          \par
          On pourrait, comme dans l'exemple précédent, exprimer $y$ en fonction de $x$ dans la première équation. Toutefois, à cause du coefficient $-2$, cela entraînerait des calculs plus longs comportant des fractions (on trouverait $y=\dfrac{-1+5x}{2}$).
          \par
          Il est plus simple, ici, d'exprimer $x$ en fonction de $y$ dans la deuxième équation.
          \item %
          \textbf{2ème étape~: Expression de  $x$ en fonction de $y$.}
          \par
          $(S_2)~\Leftrightarrow~\begin{cases} 5x-2y=1 \\  x=7-3y\end{cases}$
          \item %
          \textbf{3ème étape~: Remplacement de $x$.}
          \par
          On remplace $x$ par $(7-3y)$ dans la première équation.
          \par
          $(S_2)~\Leftrightarrow~\begin{cases} 5(7-3y)-2y=1 \\  x=7-3y\end{cases}$
          \item %
          \textbf{4ème étape~: Calcul de $y.$}
          \par
          On résout la première équation.
          \par
          $(S_2)~\Leftrightarrow~\begin{cases} 35-15y-2y=1 \\  x=7-3y\end{cases}$\\
          $(S_2)~\Leftrightarrow~\begin{cases} -17y=-34 \\  x=7-3y\end{cases}$\\
          $(S_2)~\Leftrightarrow~\begin{cases} y=\dfrac{-34}{-17} \\  x=7-3y\end{cases}$\\
          $(S_2)~\Leftrightarrow~\begin{cases} y=2\\  x=7-3y\end{cases}$\\
          \item %
          \textbf{5ème étape~: Calcul de $x.$}
          \par
          On  remplace $y$ par $2$ dans la seconde équation~:
          \par
          $(S_2)~\Leftrightarrow~\begin{cases}  y=2\\  x=7-3 \times 2\end{cases}$\\
          $(S_2)~\Leftrightarrow~\begin{cases} y=2\\  x=1\end{cases}$\\
          \item %
          \textbf{6ème étape~: Conclusion. }
          \par
          Le couple $(1~;~2)$ est l'unique solution du système $(S_2)$.
     \end{itemize}
} % fin ex
\bloc{orange}{Exemple 3}{ % id="e050"
     Résoudre le système :
     \par
     $(S_3)~~\begin{cases} 6x-2y=3 \\  -3x+y=5\end{cases}$
     \par
     \textbf{Solution~:}
     \begin{itemize}
          \item %
          \textbf{1ère étape~: Recherche de la méthode la plus rapide.}
          \par
          Ici, il est facile d'exprimer $y$ en fonction de $x$ dans la seconde équation.
          \item %
          \textbf{2ème étape~: Expression de  $y$ en fonction de $x$.}
          \par
          $(S_3)~\Leftrightarrow~\begin{cases} 6x-2y=3 \\ y=5+3x\end{cases}$
          \item %
          \textbf{3ème étape~: Remplacement de $y$.}
          \par
          $(S_3)~\Leftrightarrow~\begin{cases}  6x-2(5+3x)=3 \\ y=5+3x\end{cases}$
          \item %
          \textbf{4ème étape~: Calcul de $x.$}
          \par
          $(S_3)~\Leftrightarrow~\begin{cases}  6x-10-6x=3 \\ y=5+3x\end{cases}$\\
          $(S_3)~\Leftrightarrow~\begin{cases}  -10=3 \\ y=5+3x\end{cases}$\\
          \par
          La première équation n'a pas de solution, donc le système n'en a pas non plus.
          \par
          On peut donc passer directement à la conclusion~:
          \item %
          \textbf{6ème étape~: Conclusion. }
          \par
          Le système $(S_3)$ n'admet aucune solution dans $\mathbb{R}.$
     \end{itemize}
} % fin ex
\bloc{orange}{Exemple 4}{ % id="e060"
     Résoudre le système :
     \par
     $(S_4)~~\begin{cases} 4x-2y=6 \\  -6x+3y=-9\end{cases}$
     \par
     \textbf{Solution~:}
     \begin{itemize}
          \item %
          \textbf{1ère étape~: Recherche de la méthode la plus rapide.}
          \par
          On choisit d'exprimer $y$ en fonction de $x$ dans la première équation.
          \item %
          \textbf{2ème étape~: Expression de  $y$ en fonction de $x$.}
          \par
          $(S_4)~\Leftrightarrow~\begin{cases}-2y=6 -4x\\  -6x+3y=-9\end{cases}$\\
          $(S_4)~\Leftrightarrow~\begin{cases}2y=-6 +4x\\  -6x+3y=-9\end{cases}$\\
          $(S_4)~\Leftrightarrow~\begin{cases}y=\dfrac{-6 +4x}{2}\\  -6x+3y=-9\end{cases}$\\
          $(S_4)~\Leftrightarrow~\begin{cases}y=\dfrac{-6}{2} +\dfrac{4x}{2}\\  -6x+3y=-9\end{cases}$\\
          $(S_4)~\Leftrightarrow~\begin{cases}y=-3+2x\\  -6x+3y=-9\end{cases}$\\
          \item %
          \textbf{3ème étape~: Remplacement de $y$.}
          \par
          $(S_4)~\Leftrightarrow~\begin{cases}  y=-3+2x\\  -6x+3(-3+2x)=-9\end{cases}$\\
          \item %
          \textbf{4ème étape~: Calcul de $x.$}
          \par
          $(S_4)~\Leftrightarrow~\begin{cases}  y=-3+2x\\  -6x-9+6x=-9\end{cases}$\\
          $(S_4)~\Leftrightarrow~\begin{cases}  y=-3+2x\\  -9=-9\end{cases}$\\
          \par
          La deuxième équation est toujours vérifiée. Il suffit donc qu'un couple soit solution de la première équation $y=-3+2x$ pour qu'il soit solution du système.
          \par
          Or, cette équation possède une infinité de solutions (par exemple $(0~;~-3)$, $(1~;~-1)$, etc.).
          \par
          On peut donc sauter la cinquième étape et passer à la conclusion.
          \item %
          \textbf{6ème étape~: Conclusion. }
          \par
          Le système $(S_4)$ admet une infinité de solutions dans $\mathbb{R}.$
     \end{itemize}
} % fin ex

\end{document}
µ
\documentclass[a4paper]{article}

%================================================================================================================================
%
% Packages
%
%================================================================================================================================

\usepackage[T1]{fontenc} 	% pour caractères accentués
\usepackage[utf8]{inputenc}  % encodage utf8
\usepackage[french]{babel}	% langue : français
\usepackage{fourier}			% caractères plus lisibles
\usepackage[dvipsnames]{xcolor} % couleurs
\usepackage{fancyhdr}		% réglage header footer
\usepackage{needspace}		% empêcher sauts de page mal placés
\usepackage{graphicx}		% pour inclure des graphiques
\usepackage{enumitem,cprotect}		% personnalise les listes d'items (nécessaire pour ol, al ...)
\usepackage{hyperref}		% Liens hypertexte
\usepackage{pstricks,pst-all,pst-node,pstricks-add,pst-math,pst-plot,pst-tree,pst-eucl} % pstricks
\usepackage[a4paper,includeheadfoot,top=2cm,left=3cm, bottom=2cm,right=3cm]{geometry} % marges etc.
\usepackage{comment}			% commentaires multilignes
\usepackage{amsmath,environ} % maths (matrices, etc.)
\usepackage{amssymb,makeidx}
\usepackage{bm}				% bold maths
\usepackage{tabularx}		% tableaux
\usepackage{colortbl}		% tableaux en couleur
\usepackage{fontawesome}		% Fontawesome
\usepackage{environ}			% environment with command
\usepackage{fp}				% calculs pour ps-tricks
\usepackage{multido}			% pour ps tricks
\usepackage[np]{numprint}	% formattage nombre
\usepackage{tikz,tkz-tab} 			% package principal TikZ
\usepackage{pgfplots}   % axes
\usepackage{mathrsfs}    % cursives
\usepackage{calc}			% calcul taille boites
\usepackage[scaled=0.875]{helvet} % font sans serif
\usepackage{svg} % svg
\usepackage{scrextend} % local margin
\usepackage{scratch} %scratch
\usepackage{multicol} % colonnes
%\usepackage{infix-RPN,pst-func} % formule en notation polanaise inversée
\usepackage{listings}

%================================================================================================================================
%
% Réglages de base
%
%================================================================================================================================

\lstset{
language=Python,   % R code
literate=
{á}{{\'a}}1
{à}{{\`a}}1
{ã}{{\~a}}1
{é}{{\'e}}1
{è}{{\`e}}1
{ê}{{\^e}}1
{í}{{\'i}}1
{ó}{{\'o}}1
{õ}{{\~o}}1
{ú}{{\'u}}1
{ü}{{\"u}}1
{ç}{{\c{c}}}1
{~}{{ }}1
}


\definecolor{codegreen}{rgb}{0,0.6,0}
\definecolor{codegray}{rgb}{0.5,0.5,0.5}
\definecolor{codepurple}{rgb}{0.58,0,0.82}
\definecolor{backcolour}{rgb}{0.95,0.95,0.92}

\lstdefinestyle{mystyle}{
    backgroundcolor=\color{backcolour},   
    commentstyle=\color{codegreen},
    keywordstyle=\color{magenta},
    numberstyle=\tiny\color{codegray},
    stringstyle=\color{codepurple},
    basicstyle=\ttfamily\footnotesize,
    breakatwhitespace=false,         
    breaklines=true,                 
    captionpos=b,                    
    keepspaces=true,                 
    numbers=left,                    
xleftmargin=2em,
framexleftmargin=2em,            
    showspaces=false,                
    showstringspaces=false,
    showtabs=false,                  
    tabsize=2,
    upquote=true
}

\lstset{style=mystyle}


\lstset{style=mystyle}
\newcommand{\imgdir}{C:/laragon/www/newmc/assets/imgsvg/}
\newcommand{\imgsvgdir}{C:/laragon/www/newmc/assets/imgsvg/}

\definecolor{mcgris}{RGB}{220, 220, 220}% ancien~; pour compatibilité
\definecolor{mcbleu}{RGB}{52, 152, 219}
\definecolor{mcvert}{RGB}{125, 194, 70}
\definecolor{mcmauve}{RGB}{154, 0, 215}
\definecolor{mcorange}{RGB}{255, 96, 0}
\definecolor{mcturquoise}{RGB}{0, 153, 153}
\definecolor{mcrouge}{RGB}{255, 0, 0}
\definecolor{mclightvert}{RGB}{205, 234, 190}

\definecolor{gris}{RGB}{220, 220, 220}
\definecolor{bleu}{RGB}{52, 152, 219}
\definecolor{vert}{RGB}{125, 194, 70}
\definecolor{mauve}{RGB}{154, 0, 215}
\definecolor{orange}{RGB}{255, 96, 0}
\definecolor{turquoise}{RGB}{0, 153, 153}
\definecolor{rouge}{RGB}{255, 0, 0}
\definecolor{lightvert}{RGB}{205, 234, 190}
\setitemize[0]{label=\color{lightvert}  $\bullet$}

\pagestyle{fancy}
\renewcommand{\headrulewidth}{0.2pt}
\fancyhead[L]{maths-cours.fr}
\fancyhead[R]{\thepage}
\renewcommand{\footrulewidth}{0.2pt}
\fancyfoot[C]{}

\newcolumntype{C}{>{\centering\arraybackslash}X}
\newcolumntype{s}{>{\hsize=.35\hsize\arraybackslash}X}

\setlength{\parindent}{0pt}		 
\setlength{\parskip}{3mm}
\setlength{\headheight}{1cm}

\def\ebook{ebook}
\def\book{book}
\def\web{web}
\def\type{web}

\newcommand{\vect}[1]{\overrightarrow{\,\mathstrut#1\,}}

\def\Oij{$\left(\text{O}~;~\vect{\imath},~\vect{\jmath}\right)$}
\def\Oijk{$\left(\text{O}~;~\vect{\imath},~\vect{\jmath},~\vect{k}\right)$}
\def\Ouv{$\left(\text{O}~;~\vect{u},~\vect{v}\right)$}

\hypersetup{breaklinks=true, colorlinks = true, linkcolor = OliveGreen, urlcolor = OliveGreen, citecolor = OliveGreen, pdfauthor={Didier BONNEL - https://www.maths-cours.fr} } % supprime les bordures autour des liens

\renewcommand{\arg}[0]{\text{arg}}

\everymath{\displaystyle}

%================================================================================================================================
%
% Macros - Commandes
%
%================================================================================================================================

\newcommand\meta[2]{    			% Utilisé pour créer le post HTML.
	\def\titre{titre}
	\def\url{url}
	\def\arg{#1}
	\ifx\titre\arg
		\newcommand\maintitle{#2}
		\fancyhead[L]{#2}
		{\Large\sffamily \MakeUppercase{#2}}
		\vspace{1mm}\textcolor{mcvert}{\hrule}
	\fi 
	\ifx\url\arg
		\fancyfoot[L]{\href{https://www.maths-cours.fr#2}{\black \footnotesize{https://www.maths-cours.fr#2}}}
	\fi 
}


\newcommand\TitreC[1]{    		% Titre centré
     \needspace{3\baselineskip}
     \begin{center}\textbf{#1}\end{center}
}

\newcommand\newpar{    		% paragraphe
     \par
}

\newcommand\nosp {    		% commande vide (pas d'espace)
}
\newcommand{\id}[1]{} %ignore

\newcommand\boite[2]{				% Boite simple sans titre
	\vspace{5mm}
	\setlength{\fboxrule}{0.2mm}
	\setlength{\fboxsep}{5mm}	
	\fcolorbox{#1}{#1!3}{\makebox[\linewidth-2\fboxrule-2\fboxsep]{
  		\begin{minipage}[t]{\linewidth-2\fboxrule-4\fboxsep}\setlength{\parskip}{3mm}
  			 #2
  		\end{minipage}
	}}
	\vspace{5mm}
}

\newcommand\CBox[4]{				% Boites
	\vspace{5mm}
	\setlength{\fboxrule}{0.2mm}
	\setlength{\fboxsep}{5mm}
	
	\fcolorbox{#1}{#1!3}{\makebox[\linewidth-2\fboxrule-2\fboxsep]{
		\begin{minipage}[t]{1cm}\setlength{\parskip}{3mm}
	  		\textcolor{#1}{\LARGE{#2}}    
 	 	\end{minipage}  
  		\begin{minipage}[t]{\linewidth-2\fboxrule-4\fboxsep}\setlength{\parskip}{3mm}
			\raisebox{1.2mm}{\normalsize\sffamily{\textcolor{#1}{#3}}}						
  			 #4
  		\end{minipage}
	}}
	\vspace{5mm}
}

\newcommand\cadre[3]{				% Boites convertible html
	\par
	\vspace{2mm}
	\setlength{\fboxrule}{0.1mm}
	\setlength{\fboxsep}{5mm}
	\fcolorbox{#1}{white}{\makebox[\linewidth-2\fboxrule-2\fboxsep]{
  		\begin{minipage}[t]{\linewidth-2\fboxrule-4\fboxsep}\setlength{\parskip}{3mm}
			\raisebox{-2.5mm}{\sffamily \small{\textcolor{#1}{\MakeUppercase{#2}}}}		
			\par		
  			 #3
 	 		\end{minipage}
	}}
		\vspace{2mm}
	\par
}

\newcommand\bloc[3]{				% Boites convertible html sans bordure
     \needspace{2\baselineskip}
     {\sffamily \small{\textcolor{#1}{\MakeUppercase{#2}}}}    
		\par		
  			 #3
		\par
}

\newcommand\CHelp[1]{
     \CBox{Plum}{\faInfoCircle}{À RETENIR}{#1}
}

\newcommand\CUp[1]{
     \CBox{NavyBlue}{\faThumbsOUp}{EN PRATIQUE}{#1}
}

\newcommand\CInfo[1]{
     \CBox{Sepia}{\faArrowCircleRight}{REMARQUE}{#1}
}

\newcommand\CRedac[1]{
     \CBox{PineGreen}{\faEdit}{BIEN R\'EDIGER}{#1}
}

\newcommand\CError[1]{
     \CBox{Red}{\faExclamationTriangle}{ATTENTION}{#1}
}

\newcommand\TitreExo[2]{
\needspace{4\baselineskip}
 {\sffamily\large EXERCICE #1\ (\emph{#2 points})}
\vspace{5mm}
}

\newcommand\img[2]{
          \includegraphics[width=#2\paperwidth]{\imgdir#1}
}

\newcommand\imgsvg[2]{
       \begin{center}   \includegraphics[width=#2\paperwidth]{\imgsvgdir#1} \end{center}
}


\newcommand\Lien[2]{
     \href{#1}{#2 \tiny \faExternalLink}
}
\newcommand\mcLien[2]{
     \href{https~://www.maths-cours.fr/#1}{#2 \tiny \faExternalLink}
}

\newcommand{\euro}{\eurologo{}}

%================================================================================================================================
%
% Macros - Environement
%
%================================================================================================================================

\newenvironment{tex}{ %
}
{%
}

\newenvironment{indente}{ %
	\setlength\parindent{10mm}
}

{
	\setlength\parindent{0mm}
}

\newenvironment{corrige}{%
     \needspace{3\baselineskip}
     \medskip
     \textbf{\textsc{Corrigé}}
     \medskip
}
{
}

\newenvironment{extern}{%
     \begin{center}
     }
     {
     \end{center}
}

\NewEnviron{code}{%
	\par
     \boite{gray}{\texttt{%
     \BODY
     }}
     \par
}

\newenvironment{vbloc}{% boite sans cadre empeche saut de page
     \begin{minipage}[t]{\linewidth}
     }
     {
     \end{minipage}
}
\NewEnviron{h2}{%
    \needspace{3\baselineskip}
    \vspace{0.6cm}
	\noindent \MakeUppercase{\sffamily \large \BODY}
	\vspace{1mm}\textcolor{mcgris}{\hrule}\vspace{0.4cm}
	\par
}{}

\NewEnviron{h3}{%
    \needspace{3\baselineskip}
	\vspace{5mm}
	\textsc{\BODY}
	\par
}

\NewEnviron{margeneg}{ %
\begin{addmargin}[-1cm]{0cm}
\BODY
\end{addmargin}
}

\NewEnviron{html}{%
}

\begin{document}
\meta{url}{/exercices/suites-bac-blanc-es-l-sujet-3-maths-cours-2018/}
\meta{pid}{10469}
\meta{titre}{Suites - Bac blanc ES/L Sujet 3 - Maths-cours 2018}
\meta{type}{exercices}
%
\begin{h2}Exercice 2 (6 points)\end{h2}
\par
L'objectif de ce problème est d'étudier la convergence de la suite $(u_n)$ définie par $u_0=2$ et pour tout entier naturel $n$ :
\[ u_{n+1} = 0,9u_n+2.\]
\par
%============================================================================================================================
\par
\TitreC{Partie A\\ \'Etude graphique}
\par
Sur le graphique fourni en Annexe (voir ci-dessous), on a représenté les droites $D$ et $\Delta$ d'équations respectives $y=0,9x+2$ et $y=x$.
\par
Ces deux droites se coupent en un point $M$.
\par
\begin{enumerate}
     \item
     Déterminer, par le calcul, les coordonnées exactes du point $M$.
     \item
     $A_0$ est le point de la droite $D$ d'abscisse $u_0=2$.
     \par
     Expliquer pourquoi l'ordonnée de $A_0$ est égale à $u_1$.
     \item
     $B_1$ est le point de la droite $\Delta$ tel que la droite $(A_0B_1)$ est parallèle à l'axe des abscisses.
     \par
     Exprimer, en fonction de $u_1$, les coordonnées de $B_1$.
     \item
     Compléter le graphique de l'annexe de manière à faire apparaître, sur l'axe des abscisses, les valeurs de $u_1,\ u_2,\ u_3,\ u_4,\ u_5$ et $u_6$.
     \item
     \`A l'aide du graphique, conjecturer la limite de la suite $(u_n)$.
     \par
\end{enumerate}
\par
%============================================================================================================================
\par
\TitreC{Partie B\\ Utilisation d'une suite annexe}
\par
Pour tout entier naturel $n$, on pose $v_n=u_n-20$.
\par
\begin{enumerate}
     \item
     Montrer que la suite $(v_n)$ est une suite géométrique dont on précisera le premier terme et la raison.
     \item
     Exprimer $v_n$ en fonction de $n$.
     \item
     Montrer que pour tout entier naturel $n$ :
     \par
     \[ u_n=20-18 \times 0,9^n. \]
     \item
     En déduire la limite de la suite $(u_n)$.
     \par
\end{enumerate}
\newpage
\begin{center}
     \textbf{ANNEXE}
\end{center}
\begin{center}
     \emph{\`A rendre avec la copie}
\end{center}
\vspace{0.1\paperheight}
\begin{center}
     \begin{extern}%width="600" alt="Suite récurrente - Bac blanc"
          \includegraphics[width=0.9\textwidth]{images/BBESL-s3-2-1.eps}% gbb 1 unite=3cm
     \end{extern}
\end{center}
\newpage
\begin{corrige}
     %============================================================================================================================
     %
     \TitreC{Partie A}
     %
     %============================================================================================================================
     \par
     \begin{enumerate}
          \item %1
          Le point $M$ est le point d'intersection des droites $D$ et $\Delta$ d'équations $y=0,9x+2$ et $y=x$.
          \par
          Son abscisse $x_M$ est donc solution de l'équation $0,9x_M+2 = x_M$.
          \par
          $0,9x_M+2 = x_M\ \Leftrightarrow \ 2=x_M-0,9x_M$
          \par
          $\phantom{0,9x_M+2 = x_M}\ \Leftrightarrow \ 2=0,1x_M$
          \par
          $\phantom{0,9x_M+2 = x_M}\ \Leftrightarrow \ \dfrac{2}{0,1}=x_M$
          \par
          $\phantom{0,9x_M+2 = x_M}\ \Leftrightarrow \ x_M=20$.
          \par
          Comme le point $M$ est situé sur la droite $\Delta$ d'équation $y=x$ son ordonnée est $y_M=x_M=20$.
          \par
          Les coordonnées de $M$ sont donc $(20~;~20)$.
          \item %2
          Le point $A_0$ est situé sur la droite $D$ d'équation $y=0,9x+2$.
          \par
          Son abscisse est $u_0$ ; son ordonnée est donc :
          \par
          $y_{A_0}=0,9u_0+2$
          \par
          Or, d'après la définition de la suite $(u_n)$ : $u_1=0,9u_0+2$ ; par conséquent $y_{A_0}=u_1$.
          \par
          L'ordonnée de $A_0$ est donc $u_1$.
          \item %3
          La droite $(A_0B_1)$ est parallèle à l'axe des abscisses donc l'ordonnée de $B_1$ est égale à l'ordonnée de $A_0$ c'est à dire $u_1$.
          \par
          Comme le point $B_1$ appartient à la droite $\Delta$ d'équation $y=x$ :
          \par
          $y_{B_1}=x_{B_1}=u_1$
          \par
          Les coordonnées du point $B$ sont $(u_1~;~u_1)$.
          \par
          On réitère la procédure de la manière suivante :
          \par
          \begin{itemize}
               \item
               on trace la droite parallèle à l'axe des ordonnées passant par le point $B_1$ ; cette droite coupe $D$ en un point $A_1(u_1~;~u_2)$
               \item
               on trace la droite parallèle à l'axe des abscisses passant par le point  $A_1$ ; cette droite coupe $D$ en un point $B_2(u_2~;~u_2)$
               \par
          \end{itemize}
          \par
          et ainsi de suite...
          \par
          On obtient ainsi le graphique ci-après :
          \begin{center}
               \begin{extern}%width="600" alt="Construction des termes d'une suite récurrente"
                    \includegraphics[width=0.9\textwidth]{images/BBESL-s3-2-2.eps}% gbb 1 unite=3cm
               \end{extern}
          \end{center}
          \begin{center}
               {\footnotesize \textit{(Les ordonnées des points n'ont pas été indiquées pour ne pas surcharger la figure)}}
          \end{center}
          \item
          On conjecture que lorsque $n$ augmente, les points $A_n$ et $B_n$ se rapprochent du point $M$ et donc que :
          \[ \lim_{n \rightarrow +\infty} u_n =20. \]
          \par
     \end{enumerate}
     \par
     %============================================================================================================================
     %
     \TitreC{Partie B}
     %
     %============================================================================================================================
     \par
     \textit{Reportez-vous à la \hyperlink{suite-ag-pap}{page \pageref*{suite-ag-pap}}  \og \'Etude d'une suite arithmético-géométrique étape par étape  \fg{} si vous souhaitez plus d'informations sur la méthode utilisée dans cette partie.
     }
     \par
     \begin{enumerate}
          \item %1
          Pour tout entier naturel $n$ :
          \par
          $v_{n+1}=u_{n+1}-20$
          \par
          $\phantom{v_{n+1}}=0,9u_n+2-20$
          \par
          $\phantom{v_{n+1}}=0,9u_n-18$.
          \par
          Or $v_n=u_n-20$ donc $u_n=v_n+20$ ; alors :
          \par
          $v_{n+1}=0,9(v_n+20)-18$
          \par
          $\phantom{v_{n+1}}=0,9v_n+18-18$
          \par
          $\phantom{v_{n+1}}=0,9v_n$.
          \par
          De plus ${v_0=u_0-20=2-20=-18}$ ; par conséquent, la suite $(v_n)$ est une suite géométrique de premier terme ${v_0=-18}$ et de raison ${q=0,9}$.
          \item %2
          On en déduit que :
          \par
          $v_n=v_0q^n=-18 \times 0,9^n$.
          \item %3
          En utilisant la question précédente et la relation $u_n=v_n+20$ on obtient, pour tout entier naturel $n$ :
          \par
          $u_n=v_n+20=20-18 \times 0,9^n$.
          \item %4
          ${0 \leqslant 0,9 < 1}\ $ donc $\ \lim\limits_{n \rightarrow +\infty } 0,9^n = 0$.
          \par
          Alors :
          \par
          $\lim\limits_{n \rightarrow +\infty}18 \times 0,9^n = 0\ $ et $\ \lim\limits_{n \rightarrow +\infty}20-18 \times 0,9^n = 20$.
          \par
          La suite $(u_n)$ converge vers 20.
          \par
          \cadre{rouge}{À retenir}{
               Soit $q$ un nombre réel positif ou nul.
               \par
               \begin{itemize}
                    \item %
                    Si $\bm{0 \leqslant q < 1}$, alors $\lim\limits_{n \rightarrow +\infty}q^n=\bm{0}$.
                    \item %
                    Si $\bm{q > 1}$, alors $\lim\limits_{n \rightarrow +\infty}q^n=\bm{+\infty}$.
               \end{itemize}
               (Remarque : si $q=1$ alors $q^n=1$ pour tout entier naturel $n$, donc $\lim\limits_{n \rightarrow +\infty}q^n=1)$.
               \par
          }
          \par
     \end{enumerate}
\end{corrige}

\end{document}
µ
\documentclass[a4paper]{article}

%================================================================================================================================
%
% Packages
%
%================================================================================================================================

\usepackage[T1]{fontenc} 	% pour caractères accentués
\usepackage[utf8]{inputenc}  % encodage utf8
\usepackage[french]{babel}	% langue : français
\usepackage{fourier}			% caractères plus lisibles
\usepackage[dvipsnames]{xcolor} % couleurs
\usepackage{fancyhdr}		% réglage header footer
\usepackage{needspace}		% empêcher sauts de page mal placés
\usepackage{graphicx}		% pour inclure des graphiques
\usepackage{enumitem,cprotect}		% personnalise les listes d'items (nécessaire pour ol, al ...)
\usepackage{hyperref}		% Liens hypertexte
\usepackage{pstricks,pst-all,pst-node,pstricks-add,pst-math,pst-plot,pst-tree,pst-eucl} % pstricks
\usepackage[a4paper,includeheadfoot,top=2cm,left=3cm, bottom=2cm,right=3cm]{geometry} % marges etc.
\usepackage{comment}			% commentaires multilignes
\usepackage{amsmath,environ} % maths (matrices, etc.)
\usepackage{amssymb,makeidx}
\usepackage{bm}				% bold maths
\usepackage{tabularx}		% tableaux
\usepackage{colortbl}		% tableaux en couleur
\usepackage{fontawesome}		% Fontawesome
\usepackage{environ}			% environment with command
\usepackage{fp}				% calculs pour ps-tricks
\usepackage{multido}			% pour ps tricks
\usepackage[np]{numprint}	% formattage nombre
\usepackage{tikz,tkz-tab} 			% package principal TikZ
\usepackage{pgfplots}   % axes
\usepackage{mathrsfs}    % cursives
\usepackage{calc}			% calcul taille boites
\usepackage[scaled=0.875]{helvet} % font sans serif
\usepackage{svg} % svg
\usepackage{scrextend} % local margin
\usepackage{scratch} %scratch
\usepackage{multicol} % colonnes
%\usepackage{infix-RPN,pst-func} % formule en notation polanaise inversée
\usepackage{listings}

%================================================================================================================================
%
% Réglages de base
%
%================================================================================================================================

\lstset{
language=Python,   % R code
literate=
{á}{{\'a}}1
{à}{{\`a}}1
{ã}{{\~a}}1
{é}{{\'e}}1
{è}{{\`e}}1
{ê}{{\^e}}1
{í}{{\'i}}1
{ó}{{\'o}}1
{õ}{{\~o}}1
{ú}{{\'u}}1
{ü}{{\"u}}1
{ç}{{\c{c}}}1
{~}{{ }}1
}


\definecolor{codegreen}{rgb}{0,0.6,0}
\definecolor{codegray}{rgb}{0.5,0.5,0.5}
\definecolor{codepurple}{rgb}{0.58,0,0.82}
\definecolor{backcolour}{rgb}{0.95,0.95,0.92}

\lstdefinestyle{mystyle}{
    backgroundcolor=\color{backcolour},   
    commentstyle=\color{codegreen},
    keywordstyle=\color{magenta},
    numberstyle=\tiny\color{codegray},
    stringstyle=\color{codepurple},
    basicstyle=\ttfamily\footnotesize,
    breakatwhitespace=false,         
    breaklines=true,                 
    captionpos=b,                    
    keepspaces=true,                 
    numbers=left,                    
xleftmargin=2em,
framexleftmargin=2em,            
    showspaces=false,                
    showstringspaces=false,
    showtabs=false,                  
    tabsize=2,
    upquote=true
}

\lstset{style=mystyle}


\lstset{style=mystyle}
\newcommand{\imgdir}{C:/laragon/www/newmc/assets/imgsvg/}
\newcommand{\imgsvgdir}{C:/laragon/www/newmc/assets/imgsvg/}

\definecolor{mcgris}{RGB}{220, 220, 220}% ancien~; pour compatibilité
\definecolor{mcbleu}{RGB}{52, 152, 219}
\definecolor{mcvert}{RGB}{125, 194, 70}
\definecolor{mcmauve}{RGB}{154, 0, 215}
\definecolor{mcorange}{RGB}{255, 96, 0}
\definecolor{mcturquoise}{RGB}{0, 153, 153}
\definecolor{mcrouge}{RGB}{255, 0, 0}
\definecolor{mclightvert}{RGB}{205, 234, 190}

\definecolor{gris}{RGB}{220, 220, 220}
\definecolor{bleu}{RGB}{52, 152, 219}
\definecolor{vert}{RGB}{125, 194, 70}
\definecolor{mauve}{RGB}{154, 0, 215}
\definecolor{orange}{RGB}{255, 96, 0}
\definecolor{turquoise}{RGB}{0, 153, 153}
\definecolor{rouge}{RGB}{255, 0, 0}
\definecolor{lightvert}{RGB}{205, 234, 190}
\setitemize[0]{label=\color{lightvert}  $\bullet$}

\pagestyle{fancy}
\renewcommand{\headrulewidth}{0.2pt}
\fancyhead[L]{maths-cours.fr}
\fancyhead[R]{\thepage}
\renewcommand{\footrulewidth}{0.2pt}
\fancyfoot[C]{}

\newcolumntype{C}{>{\centering\arraybackslash}X}
\newcolumntype{s}{>{\hsize=.35\hsize\arraybackslash}X}

\setlength{\parindent}{0pt}		 
\setlength{\parskip}{3mm}
\setlength{\headheight}{1cm}

\def\ebook{ebook}
\def\book{book}
\def\web{web}
\def\type{web}

\newcommand{\vect}[1]{\overrightarrow{\,\mathstrut#1\,}}

\def\Oij{$\left(\text{O}~;~\vect{\imath},~\vect{\jmath}\right)$}
\def\Oijk{$\left(\text{O}~;~\vect{\imath},~\vect{\jmath},~\vect{k}\right)$}
\def\Ouv{$\left(\text{O}~;~\vect{u},~\vect{v}\right)$}

\hypersetup{breaklinks=true, colorlinks = true, linkcolor = OliveGreen, urlcolor = OliveGreen, citecolor = OliveGreen, pdfauthor={Didier BONNEL - https://www.maths-cours.fr} } % supprime les bordures autour des liens

\renewcommand{\arg}[0]{\text{arg}}

\everymath{\displaystyle}

%================================================================================================================================
%
% Macros - Commandes
%
%================================================================================================================================

\newcommand\meta[2]{    			% Utilisé pour créer le post HTML.
	\def\titre{titre}
	\def\url{url}
	\def\arg{#1}
	\ifx\titre\arg
		\newcommand\maintitle{#2}
		\fancyhead[L]{#2}
		{\Large\sffamily \MakeUppercase{#2}}
		\vspace{1mm}\textcolor{mcvert}{\hrule}
	\fi 
	\ifx\url\arg
		\fancyfoot[L]{\href{https://www.maths-cours.fr#2}{\black \footnotesize{https://www.maths-cours.fr#2}}}
	\fi 
}


\newcommand\TitreC[1]{    		% Titre centré
     \needspace{3\baselineskip}
     \begin{center}\textbf{#1}\end{center}
}

\newcommand\newpar{    		% paragraphe
     \par
}

\newcommand\nosp {    		% commande vide (pas d'espace)
}
\newcommand{\id}[1]{} %ignore

\newcommand\boite[2]{				% Boite simple sans titre
	\vspace{5mm}
	\setlength{\fboxrule}{0.2mm}
	\setlength{\fboxsep}{5mm}	
	\fcolorbox{#1}{#1!3}{\makebox[\linewidth-2\fboxrule-2\fboxsep]{
  		\begin{minipage}[t]{\linewidth-2\fboxrule-4\fboxsep}\setlength{\parskip}{3mm}
  			 #2
  		\end{minipage}
	}}
	\vspace{5mm}
}

\newcommand\CBox[4]{				% Boites
	\vspace{5mm}
	\setlength{\fboxrule}{0.2mm}
	\setlength{\fboxsep}{5mm}
	
	\fcolorbox{#1}{#1!3}{\makebox[\linewidth-2\fboxrule-2\fboxsep]{
		\begin{minipage}[t]{1cm}\setlength{\parskip}{3mm}
	  		\textcolor{#1}{\LARGE{#2}}    
 	 	\end{minipage}  
  		\begin{minipage}[t]{\linewidth-2\fboxrule-4\fboxsep}\setlength{\parskip}{3mm}
			\raisebox{1.2mm}{\normalsize\sffamily{\textcolor{#1}{#3}}}						
  			 #4
  		\end{minipage}
	}}
	\vspace{5mm}
}

\newcommand\cadre[3]{				% Boites convertible html
	\par
	\vspace{2mm}
	\setlength{\fboxrule}{0.1mm}
	\setlength{\fboxsep}{5mm}
	\fcolorbox{#1}{white}{\makebox[\linewidth-2\fboxrule-2\fboxsep]{
  		\begin{minipage}[t]{\linewidth-2\fboxrule-4\fboxsep}\setlength{\parskip}{3mm}
			\raisebox{-2.5mm}{\sffamily \small{\textcolor{#1}{\MakeUppercase{#2}}}}		
			\par		
  			 #3
 	 		\end{minipage}
	}}
		\vspace{2mm}
	\par
}

\newcommand\bloc[3]{				% Boites convertible html sans bordure
     \needspace{2\baselineskip}
     {\sffamily \small{\textcolor{#1}{\MakeUppercase{#2}}}}    
		\par		
  			 #3
		\par
}

\newcommand\CHelp[1]{
     \CBox{Plum}{\faInfoCircle}{À RETENIR}{#1}
}

\newcommand\CUp[1]{
     \CBox{NavyBlue}{\faThumbsOUp}{EN PRATIQUE}{#1}
}

\newcommand\CInfo[1]{
     \CBox{Sepia}{\faArrowCircleRight}{REMARQUE}{#1}
}

\newcommand\CRedac[1]{
     \CBox{PineGreen}{\faEdit}{BIEN R\'EDIGER}{#1}
}

\newcommand\CError[1]{
     \CBox{Red}{\faExclamationTriangle}{ATTENTION}{#1}
}

\newcommand\TitreExo[2]{
\needspace{4\baselineskip}
 {\sffamily\large EXERCICE #1\ (\emph{#2 points})}
\vspace{5mm}
}

\newcommand\img[2]{
          \includegraphics[width=#2\paperwidth]{\imgdir#1}
}

\newcommand\imgsvg[2]{
       \begin{center}   \includegraphics[width=#2\paperwidth]{\imgsvgdir#1} \end{center}
}


\newcommand\Lien[2]{
     \href{#1}{#2 \tiny \faExternalLink}
}
\newcommand\mcLien[2]{
     \href{https~://www.maths-cours.fr/#1}{#2 \tiny \faExternalLink}
}

\newcommand{\euro}{\eurologo{}}

%================================================================================================================================
%
% Macros - Environement
%
%================================================================================================================================

\newenvironment{tex}{ %
}
{%
}

\newenvironment{indente}{ %
	\setlength\parindent{10mm}
}

{
	\setlength\parindent{0mm}
}

\newenvironment{corrige}{%
     \needspace{3\baselineskip}
     \medskip
     \textbf{\textsc{Corrigé}}
     \medskip
}
{
}

\newenvironment{extern}{%
     \begin{center}
     }
     {
     \end{center}
}

\NewEnviron{code}{%
	\par
     \boite{gray}{\texttt{%
     \BODY
     }}
     \par
}

\newenvironment{vbloc}{% boite sans cadre empeche saut de page
     \begin{minipage}[t]{\linewidth}
     }
     {
     \end{minipage}
}
\NewEnviron{h2}{%
    \needspace{3\baselineskip}
    \vspace{0.6cm}
	\noindent \MakeUppercase{\sffamily \large \BODY}
	\vspace{1mm}\textcolor{mcgris}{\hrule}\vspace{0.4cm}
	\par
}{}

\NewEnviron{h3}{%
    \needspace{3\baselineskip}
	\vspace{5mm}
	\textsc{\BODY}
	\par
}

\NewEnviron{margeneg}{ %
\begin{addmargin}[-1cm]{0cm}
\BODY
\end{addmargin}
}

\NewEnviron{html}{%
}

\begin{document}
\meta{url}{/exercices/cout-marginal-bac-blanc-es-l-sujet-3-maths-cours-2018/}
\meta{pid}{10473}
\meta{titre}{Coût marginal - Bac blanc ES/L Sujet 3 - Maths-cours 2018}
\meta{type}{exercices}
%
\begin{h2}Exercice 1 (5 points)\end{h2}
\par
Une entreprise fabrique une boisson conditionnée en bouteille d'un litre.
\par
Le coût total, exprimé en euros est donné par la fonction $C_t$ :
\[ C_t(x)=4x^3-20x^2+80x+100. \]
où $x$ représente le volume exprimé en centaines de litres, $x$ variant dans l'intervalle $[0~;~5]$.
\par
Le graphique ci-après affiche la représentation graphique $\mathscr{C}$ de la fonction $C_t$ dans un repère orthogonal.
\par
Le point $A$ est le point de la courbe $\mathscr{C}$ d'abscisse 5 et $B$ un point d'inflexion de cette courbe.
\par
$T_1$ et $T_2$ sont les tangentes à $\mathscr{C}$ respectivement aux points $A$ et $B$.
\par
\begin{center}
     \begin{extern}%width="550" alt="Graphique coût marginal"
          \includegraphics[width=0.9\textwidth]{images/BBESL-s3-1-1}% gbb 1 unite=3cm
     \end{extern}
\end{center}
\par
%============================================================================================================================
%
\TitreC{Partie A}
%
%============================================================================================================================
\par
\begin{enumerate}
     \item
     Les coûts fixes sont les coûts que supporte l'entreprise même lorsque la production est nulle.
     \par
     \`A l'aide du graphique ou de la formule définissant $C_t$, déterminer les coûts fixes puis le coût pour une production de 500 litres.
     \item
     Le coût marginal est égal au coût de fabrication d'une unité supplémentaire.
     \par
     On rappelle que l'on peut assimiler le coût marginal à la dérivée du coût total.
     \par
     Par lecture graphique, donner une estimation du coefficient directeur à la courbe $\mathscr{C}$ au point $A$ d'abscisse 5.
     \par
     En déduire une estimation du coût marginal pour une production de 500 litres.
     \item
     Donner, par lecture graphique, une estimation de l'intervalle sur lequel la fonction $C_t$ est convexe et une estimation de l'intervalle sur lequel la fonction $C_t$ est concave.
     \item
     \`A l'aide du graphique, estimer la valeur minimum du coût marginal.
     \par
\end{enumerate}
\par
%============================================================================================================================
%
\TitreC{Partie B}
%
%============================================================================================================================
\par
\begin{enumerate}
     \item
     Pour $x$ appartenant à l'intervalle $[0~;~5]$, exprimer le coût marginal $C_m(x)$ en fonction de $x$.
     \item
     Déterminer les coordonnées exactes du point $B$.
     \par
     Retrouver, par le calcul, la valeur minimum du coût marginal.
     \par
\end{enumerate}
\begin{corrige}
     %============================================================================================================================
     %
     \TitreC{Partie A}
     %
     %============================================================================================================================
     \par
     \begin{enumerate}
          \item %1
          Les coûts fixes sont égaux à $C_t(0)$ :
          \par
          $C_t(0)=4 \times 0^3 -20 \times 0^2 + 80 \times 0 + 100 = 100$
          \par
          Les coûts fixes sont de \textbf{100 euros}.
          \par
          Pour une production de 500 litres, soit 5 centaines de litres, le coût total est égal à :
          \par
          $C_t(5)=4 \times 5^3 -20 \times 5^2 + 80 \times 5 + 100 = 500$
          \par
          Le coût total pour une production de 500 litres est égal à \textbf{500 euros}.
          \item %2
          La tangente en $A$ à la courbe $\mathscr{C}$ est la droite $T_1$. Cette droite passe par le point $A(5~;~500)$ et passe par un point $M$ de coordonnées proches de $(4~;~320)$.
          \par
          Le coefficient directeur $a$ de cette tangente est donc approximativement :
          \par
          $a=\dfrac{y_M-y_A}{x_M-x_A}=\dfrac{500-320}{5-4}=180$
          \par
          Le coût marginal $C_m(5)$ pour une production de 500 litres est égal au nombre dérivé $C'_t(5)$. Or, ce nombre est le coefficient directeur de la tangente au point $A$.
          \par
          \cadre{rouge}{À retenir}{
               Le \textbf{coefficient directeur de la tangente} à la courbe représentative de $f$ au point d'\textbf{abscisse} $\alpha$ est égal à $f'(\alpha)$.
          }
          \par
          Le coût marginal pour une production de 500 litres est donc approximativement égal à \textbf{180 euros}.
          \item %3
          Notons $x_B$ l'abscisse du point $B$.
          \par
          Par lecture graphique, on voit que la fonction $C_t$ est concave sur l'intervalle $[0~;~x_B]$ et convexe sur l'intervalle $[x_B~;~5]$.
          \par
          On constate également que les coordonnées du point d'inflexion $B$ sont approximativement $(1,7~;~195)$.
          \par
          On peut donc estimer que la fonction $C_t$ est \textbf{concave sur l'intervalle} $\bm{[0~;~1,7]}$ et \textbf{convexe sur l'intervalle} $\bm{[1,7~;~5]}$.
          \par
          \cadre{rouge}{À retenir}{
               Une fonction est \textbf{convexe} si et seulement si sa courbe représentative est située \textbf{au-dessus de ses tangentes} (courbe en \og $\cup$ \fg{}).
               \par
               Une fonction est \textbf{concave} si et seulement si sa courbe représentative est située \textbf{au-dessous de ses tangentes} (courbe en \og $\cap$ \fg{}).
               \par
               Un \textbf{point d'inflexion} est un point où la fonction \textbf{change de convexité}. En ce point, la tangente \og traverse \fg{} la courbe.
          }
          \item
          La fonction $C_t$ est convexe si et seulement si sa fonction dérivée $C'_t$ (identique à la fonction $C_m$) est croissante.
          \par
          \cadre{rouge}{À retenir}{
               Si $f$ est une fonction deux fois dérivable sur un intervalle $I$, les propositions suivantes sont équivalentes :
               \par
               \begin{itemize}
                    \item
                    la fonction $f$ est \textbf{connexe} sur $I$;
                    \item
                    la fonction dérivée $f'$ est \textbf{croissante} sur $I$;
                    \item
                    la fonction dérivée seconde $f''$ est \textbf{positive} sur $I$;
               \end{itemize}
               \vspace{3mm}
               \par
               De même, les propositions suivantes sont équivalentes :
               \par
               \begin{itemize}
                    \item
                    la fonction $f$ est \textbf{concave} sur $I$;
                    \item
                    la fonction dérivée $f'$ est \textbf{décroissante} sur $I$;
                    \item
                    la fonction dérivée seconde $f''$ est \textbf{négative} sur $I$;
               \end{itemize}
          }
          \par
          D'après la question précédente on peut tracer le tableau ci-après :
          \par
          %:-+-+-+-+- Engendré par : http://math.et.info.free.fr/TikZ/TableauxVariations/
          \begin{center}
               \begin{extern}%width="400" alt="tableau de variations coût marginal"
                    \begin{tikzpicture}[scale=0.875]
                         % Styles
                         \tikzstyle{cadre}=[thin]
                         \tikzstyle{fleche}=[->,>=latex,thin]
                         \tikzstyle{nondefini}=[lightgray]
                         % Dimensions Modifiables
                         \def\Lrg{1.5}
                         \def\HtX{1}
                         \def\HtY{0.5}
                         % Dimensions Calculées
                         \def\lignex{-0.5*\HtX}
                         \def\lignef{-1.5*\HtX}
                         \def\separateur{-0.5*\Lrg}
                         % Largeur du tableau
                         \def\gauche{-2*\Lrg}
                         \def\droite{4.5*\Lrg}
                         % Hauteur du tableau
                         \def\haut{0.5*\HtX}
                         \def\bas{-2.5*\HtX-2*\HtY}
                         % Ligne de l'abscisse : x
                         \node at (-1.3*\Lrg,0) {$x$};
                         \node at (0*\Lrg,0) {$0$};
                         \node at (2*\Lrg,0) {$x_B$};
                         \node at (4*\Lrg,0) {$5$};
                         % Ligne de la dérivée : f'(x)
                         \node at (-1.3*\Lrg,-1*\HtX) {$C_t$};
                         \node at (0*\Lrg,-1*\HtX) {$\ $};
                         \node at (1*\Lrg,-1*\HtX) {$\text{concave}$};
                         \node at (2*\Lrg,-1*\HtX) {$\ $};
                         \node at (3*\Lrg,-1*\HtX) {$\text{convexe}$};
                         \node at (4*\Lrg,-1*\HtX) {$\ $};
                         % Ligne de la fonction : f(x)
                         \node  at (-1.3*\Lrg,{-2*\HtX+(-1)*\HtY}) {$C'_t=C_m$};
                         \node (f1) at (0*\Lrg,{-2*\HtX+(0)*\HtY}) {$\ $};
                         \node (f2) at (2*\Lrg,{-2*\HtX+(-2)*\HtY}) {$\ $};
                         \node (f3) at (4*\Lrg,{-2*\HtX+(0)*\HtY}) {$\ $};
                         % Flèches
                         \draw[fleche] (f1) -- (f2);
                         \draw[fleche] (f2) -- (f3);
                         % Encadrement
                         \draw[cadre] (\separateur,\haut) -- (\separateur,\bas);
                         \draw[cadre] (\gauche,\haut) rectangle  (\droite,\bas);
                         \draw[cadre] (\gauche,\lignex) -- (\droite,\lignex);
                         \draw[cadre] (\gauche,\lignef) -- (\droite,\lignef);
                    \end{tikzpicture}
               \end{extern}
          \end{center}
          Le coût marginal est minimal pour $x=x_B \approx 1,7$.
          \par
          Ce minimum vaut $C_m(x_B)=C'_t(x_B)$.
          \par
          Pour déterminer la valeur de $C'_t(x_B)$, on procède comme à la question \textbf{2.}
          \par
          $C'_t(x_B)$ est le coefficient directeur de la tangente $T_2$ à la courbe $\mathscr{C}$ au point $B$.
          \par
          Cette tangente passe approximativement par les points de coordonnées $(0~;~120)$ et $(1,7~;~195)$.
          \par
          On a donc :
          \par
          $C'_t(x_B) \approx \dfrac{195-120}{1,7-0} \approx 44$
          \par
          Le coût marginal minimal peut être estimé à \textbf{44 euros}.
          \par
          \textit{Remarque : il ne s'agit que d'une estimation. Vous pouvez tout à fait trouver un résultat légèrement différent. La valeur exacte, calculée dans la partie B, est comprise entre 46 et 47 euros.}
          \par
     \end{enumerate}
     \par
     %============================================================================================================================
     %
     \TitreC{Partie B}
     %
     %============================================================================================================================
     \par
     \begin{enumerate}
          \item
          Pour $x$ appartenant à l'intervalle $[0~;~5]$ :
          \par
          $C_m(x)=C'_t(x)=4 \times 3x^2-20 \times 2x + 80 = 12 x^2-40x+ 80$
          \item
          Le point $B$ est le point d'inflexion de la courbe $\mathscr{C}$.
          Son abscisse correspond à la valeur de $x$ pour laquelle $C''_t(x)$ s'annule et change de signe.
          Or :
          \par
          $C''_t(x)=C'_m(x)=12 \times 2x-40=24x-40$
          \par
          $C''_t$ est une fonction affine qui s'annule et change de signe pour ${x=\dfrac{40}{24}=\dfrac{5}{3}}$.
          \par
          Le point $B$ a donc pour abscisse $\dfrac{5}{3}$.
          \par
          L'ordonnée de $B$ est :
          \par
          $C_t\left(\dfrac{5}{3}\right)=4 \times \left(\dfrac{5}{3}\right)^3 - 20 \times \left(\dfrac{5}{3}\right)^2 + 80 \times \dfrac{5}{3} +100 = \dfrac{5300}{27}$
          \par
          Les coordonnées de $B$ sont donc $ \left(\dfrac{5}{3}~;~\dfrac{5300}{27}\right)$.
          \vspace{3mm}
          \par
          D'après le tableau de la question \textbf{4.} de la partie \textbf{A.}, le coût marginal minimal correspond à $C_m\left(\dfrac{5}{3}\right)$ :
          \par
          $C_m\left(\dfrac{5}{3}\right)=12 \times \left(\dfrac{5}{3}\right)^2 - 40 \times \dfrac{5}{3} +80 = \dfrac{140}{3}$
          \par
          Le coût marginal minimum est donc $\dfrac{140}{3}\ $($\approx 46,7$ euros).
          \par
     \end{enumerate}
\end{corrige}

\end{document}
µ
\documentclass[a4paper]{article}

%================================================================================================================================
%
% Packages
%
%================================================================================================================================

\usepackage[T1]{fontenc} 	% pour caractères accentués
\usepackage[utf8]{inputenc}  % encodage utf8
\usepackage[french]{babel}	% langue : français
\usepackage{fourier}			% caractères plus lisibles
\usepackage[dvipsnames]{xcolor} % couleurs
\usepackage{fancyhdr}		% réglage header footer
\usepackage{needspace}		% empêcher sauts de page mal placés
\usepackage{graphicx}		% pour inclure des graphiques
\usepackage{enumitem,cprotect}		% personnalise les listes d'items (nécessaire pour ol, al ...)
\usepackage{hyperref}		% Liens hypertexte
\usepackage{pstricks,pst-all,pst-node,pstricks-add,pst-math,pst-plot,pst-tree,pst-eucl} % pstricks
\usepackage[a4paper,includeheadfoot,top=2cm,left=3cm, bottom=2cm,right=3cm]{geometry} % marges etc.
\usepackage{comment}			% commentaires multilignes
\usepackage{amsmath,environ} % maths (matrices, etc.)
\usepackage{amssymb,makeidx}
\usepackage{bm}				% bold maths
\usepackage{tabularx}		% tableaux
\usepackage{colortbl}		% tableaux en couleur
\usepackage{fontawesome}		% Fontawesome
\usepackage{environ}			% environment with command
\usepackage{fp}				% calculs pour ps-tricks
\usepackage{multido}			% pour ps tricks
\usepackage[np]{numprint}	% formattage nombre
\usepackage{tikz,tkz-tab} 			% package principal TikZ
\usepackage{pgfplots}   % axes
\usepackage{mathrsfs}    % cursives
\usepackage{calc}			% calcul taille boites
\usepackage[scaled=0.875]{helvet} % font sans serif
\usepackage{svg} % svg
\usepackage{scrextend} % local margin
\usepackage{scratch} %scratch
\usepackage{multicol} % colonnes
%\usepackage{infix-RPN,pst-func} % formule en notation polanaise inversée
\usepackage{listings}

%================================================================================================================================
%
% Réglages de base
%
%================================================================================================================================

\lstset{
language=Python,   % R code
literate=
{á}{{\'a}}1
{à}{{\`a}}1
{ã}{{\~a}}1
{é}{{\'e}}1
{è}{{\`e}}1
{ê}{{\^e}}1
{í}{{\'i}}1
{ó}{{\'o}}1
{õ}{{\~o}}1
{ú}{{\'u}}1
{ü}{{\"u}}1
{ç}{{\c{c}}}1
{~}{{ }}1
}


\definecolor{codegreen}{rgb}{0,0.6,0}
\definecolor{codegray}{rgb}{0.5,0.5,0.5}
\definecolor{codepurple}{rgb}{0.58,0,0.82}
\definecolor{backcolour}{rgb}{0.95,0.95,0.92}

\lstdefinestyle{mystyle}{
    backgroundcolor=\color{backcolour},   
    commentstyle=\color{codegreen},
    keywordstyle=\color{magenta},
    numberstyle=\tiny\color{codegray},
    stringstyle=\color{codepurple},
    basicstyle=\ttfamily\footnotesize,
    breakatwhitespace=false,         
    breaklines=true,                 
    captionpos=b,                    
    keepspaces=true,                 
    numbers=left,                    
xleftmargin=2em,
framexleftmargin=2em,            
    showspaces=false,                
    showstringspaces=false,
    showtabs=false,                  
    tabsize=2,
    upquote=true
}

\lstset{style=mystyle}


\lstset{style=mystyle}
\newcommand{\imgdir}{C:/laragon/www/newmc/assets/imgsvg/}
\newcommand{\imgsvgdir}{C:/laragon/www/newmc/assets/imgsvg/}

\definecolor{mcgris}{RGB}{220, 220, 220}% ancien~; pour compatibilité
\definecolor{mcbleu}{RGB}{52, 152, 219}
\definecolor{mcvert}{RGB}{125, 194, 70}
\definecolor{mcmauve}{RGB}{154, 0, 215}
\definecolor{mcorange}{RGB}{255, 96, 0}
\definecolor{mcturquoise}{RGB}{0, 153, 153}
\definecolor{mcrouge}{RGB}{255, 0, 0}
\definecolor{mclightvert}{RGB}{205, 234, 190}

\definecolor{gris}{RGB}{220, 220, 220}
\definecolor{bleu}{RGB}{52, 152, 219}
\definecolor{vert}{RGB}{125, 194, 70}
\definecolor{mauve}{RGB}{154, 0, 215}
\definecolor{orange}{RGB}{255, 96, 0}
\definecolor{turquoise}{RGB}{0, 153, 153}
\definecolor{rouge}{RGB}{255, 0, 0}
\definecolor{lightvert}{RGB}{205, 234, 190}
\setitemize[0]{label=\color{lightvert}  $\bullet$}

\pagestyle{fancy}
\renewcommand{\headrulewidth}{0.2pt}
\fancyhead[L]{maths-cours.fr}
\fancyhead[R]{\thepage}
\renewcommand{\footrulewidth}{0.2pt}
\fancyfoot[C]{}

\newcolumntype{C}{>{\centering\arraybackslash}X}
\newcolumntype{s}{>{\hsize=.35\hsize\arraybackslash}X}

\setlength{\parindent}{0pt}		 
\setlength{\parskip}{3mm}
\setlength{\headheight}{1cm}

\def\ebook{ebook}
\def\book{book}
\def\web{web}
\def\type{web}

\newcommand{\vect}[1]{\overrightarrow{\,\mathstrut#1\,}}

\def\Oij{$\left(\text{O}~;~\vect{\imath},~\vect{\jmath}\right)$}
\def\Oijk{$\left(\text{O}~;~\vect{\imath},~\vect{\jmath},~\vect{k}\right)$}
\def\Ouv{$\left(\text{O}~;~\vect{u},~\vect{v}\right)$}

\hypersetup{breaklinks=true, colorlinks = true, linkcolor = OliveGreen, urlcolor = OliveGreen, citecolor = OliveGreen, pdfauthor={Didier BONNEL - https://www.maths-cours.fr} } % supprime les bordures autour des liens

\renewcommand{\arg}[0]{\text{arg}}

\everymath{\displaystyle}

%================================================================================================================================
%
% Macros - Commandes
%
%================================================================================================================================

\newcommand\meta[2]{    			% Utilisé pour créer le post HTML.
	\def\titre{titre}
	\def\url{url}
	\def\arg{#1}
	\ifx\titre\arg
		\newcommand\maintitle{#2}
		\fancyhead[L]{#2}
		{\Large\sffamily \MakeUppercase{#2}}
		\vspace{1mm}\textcolor{mcvert}{\hrule}
	\fi 
	\ifx\url\arg
		\fancyfoot[L]{\href{https://www.maths-cours.fr#2}{\black \footnotesize{https://www.maths-cours.fr#2}}}
	\fi 
}


\newcommand\TitreC[1]{    		% Titre centré
     \needspace{3\baselineskip}
     \begin{center}\textbf{#1}\end{center}
}

\newcommand\newpar{    		% paragraphe
     \par
}

\newcommand\nosp {    		% commande vide (pas d'espace)
}
\newcommand{\id}[1]{} %ignore

\newcommand\boite[2]{				% Boite simple sans titre
	\vspace{5mm}
	\setlength{\fboxrule}{0.2mm}
	\setlength{\fboxsep}{5mm}	
	\fcolorbox{#1}{#1!3}{\makebox[\linewidth-2\fboxrule-2\fboxsep]{
  		\begin{minipage}[t]{\linewidth-2\fboxrule-4\fboxsep}\setlength{\parskip}{3mm}
  			 #2
  		\end{minipage}
	}}
	\vspace{5mm}
}

\newcommand\CBox[4]{				% Boites
	\vspace{5mm}
	\setlength{\fboxrule}{0.2mm}
	\setlength{\fboxsep}{5mm}
	
	\fcolorbox{#1}{#1!3}{\makebox[\linewidth-2\fboxrule-2\fboxsep]{
		\begin{minipage}[t]{1cm}\setlength{\parskip}{3mm}
	  		\textcolor{#1}{\LARGE{#2}}    
 	 	\end{minipage}  
  		\begin{minipage}[t]{\linewidth-2\fboxrule-4\fboxsep}\setlength{\parskip}{3mm}
			\raisebox{1.2mm}{\normalsize\sffamily{\textcolor{#1}{#3}}}						
  			 #4
  		\end{minipage}
	}}
	\vspace{5mm}
}

\newcommand\cadre[3]{				% Boites convertible html
	\par
	\vspace{2mm}
	\setlength{\fboxrule}{0.1mm}
	\setlength{\fboxsep}{5mm}
	\fcolorbox{#1}{white}{\makebox[\linewidth-2\fboxrule-2\fboxsep]{
  		\begin{minipage}[t]{\linewidth-2\fboxrule-4\fboxsep}\setlength{\parskip}{3mm}
			\raisebox{-2.5mm}{\sffamily \small{\textcolor{#1}{\MakeUppercase{#2}}}}		
			\par		
  			 #3
 	 		\end{minipage}
	}}
		\vspace{2mm}
	\par
}

\newcommand\bloc[3]{				% Boites convertible html sans bordure
     \needspace{2\baselineskip}
     {\sffamily \small{\textcolor{#1}{\MakeUppercase{#2}}}}    
		\par		
  			 #3
		\par
}

\newcommand\CHelp[1]{
     \CBox{Plum}{\faInfoCircle}{À RETENIR}{#1}
}

\newcommand\CUp[1]{
     \CBox{NavyBlue}{\faThumbsOUp}{EN PRATIQUE}{#1}
}

\newcommand\CInfo[1]{
     \CBox{Sepia}{\faArrowCircleRight}{REMARQUE}{#1}
}

\newcommand\CRedac[1]{
     \CBox{PineGreen}{\faEdit}{BIEN R\'EDIGER}{#1}
}

\newcommand\CError[1]{
     \CBox{Red}{\faExclamationTriangle}{ATTENTION}{#1}
}

\newcommand\TitreExo[2]{
\needspace{4\baselineskip}
 {\sffamily\large EXERCICE #1\ (\emph{#2 points})}
\vspace{5mm}
}

\newcommand\img[2]{
          \includegraphics[width=#2\paperwidth]{\imgdir#1}
}

\newcommand\imgsvg[2]{
       \begin{center}   \includegraphics[width=#2\paperwidth]{\imgsvgdir#1} \end{center}
}


\newcommand\Lien[2]{
     \href{#1}{#2 \tiny \faExternalLink}
}
\newcommand\mcLien[2]{
     \href{https~://www.maths-cours.fr/#1}{#2 \tiny \faExternalLink}
}

\newcommand{\euro}{\eurologo{}}

%================================================================================================================================
%
% Macros - Environement
%
%================================================================================================================================

\newenvironment{tex}{ %
}
{%
}

\newenvironment{indente}{ %
	\setlength\parindent{10mm}
}

{
	\setlength\parindent{0mm}
}

\newenvironment{corrige}{%
     \needspace{3\baselineskip}
     \medskip
     \textbf{\textsc{Corrigé}}
     \medskip
}
{
}

\newenvironment{extern}{%
     \begin{center}
     }
     {
     \end{center}
}

\NewEnviron{code}{%
	\par
     \boite{gray}{\texttt{%
     \BODY
     }}
     \par
}

\newenvironment{vbloc}{% boite sans cadre empeche saut de page
     \begin{minipage}[t]{\linewidth}
     }
     {
     \end{minipage}
}
\NewEnviron{h2}{%
    \needspace{3\baselineskip}
    \vspace{0.6cm}
	\noindent \MakeUppercase{\sffamily \large \BODY}
	\vspace{1mm}\textcolor{mcgris}{\hrule}\vspace{0.4cm}
	\par
}{}

\NewEnviron{h3}{%
    \needspace{3\baselineskip}
	\vspace{5mm}
	\textsc{\BODY}
	\par
}

\NewEnviron{margeneg}{ %
\begin{addmargin}[-1cm]{0cm}
\BODY
\end{addmargin}
}

\NewEnviron{html}{%
}

\begin{document}
\meta{url}{/exercices/fonction-exponentielle-bac-blanc-es-l-sujet-3-maths-cours-2018/}
\meta{pid}{10475}
\meta{titre}{Fonction exponentielle - Bac blanc ES/L Sujet 3 - Maths-cours 2018}
\meta{type}{exercices}
%
\begin{h2}Exercice 3 (5 points)\end{h2}
\par
On a représenté, ci-après, la courbe $\mathscr{C}$ d'une fonction définie et dérivable sur l'intervalle $[0~;~5]$ ainsi que la tangente $T$ à cette courbe au point $O$, origine du repère.
\par
\begin{center}
     \begin{extern}%width="550" alt="Courbe représentative de f"
          \includegraphics[width=0.9\textwidth]{images/BBESL-s3-3-1}% gbb 1 unite=3cm
     \end{extern}
\end{center}
\par
On note $f'$ la fonction dérivée de la fonction $f$.
\par
%============================================================================================================================
%
\TitreC{Partie A}
%
%============================================================================================================================
\par
\begin{enumerate}
     \item
     Préciser la valeur de $f(0)$.
     \item
     La tangente $T$ passe par le point $A(1~;~3)$.
     \par
     Déterminer la valeur de $f'(0)$.
     \item
     On admet que la fonction $f$ est définie sur l'intervalle $[0~;~5]$ par une expression de la forme :
     \[ f(x)=(ax+b)\text{e}^{-x}+2 \]
     \par
     où $a$ et $b$ sont deux nombres réels.
     \par
     \begin{enumerate}
          \item
          Montrer que pour tout réel $x$ de l'intervalle $[0~;~5]$ :
          \[ f'(x)=(-ax+a-b)\text{e}^{-x}. \]
          \item
          \`A l'aide des questions \textbf{1.} et \textbf{2.}, déterminer les valeurs de $a$ et $b$.
          \par
     \end{enumerate}
     \par
\end{enumerate}
\par
%============================================================================================================================
%
\TitreC{Partie B}
%
\par
Par la suite, on considèrera que la fonction $f$ est définie sur l'intervalle $[0~;~5]$ par :
\[ f(x)=(x-2)\text{e}^{-x}+2. \]
\par
\begin{enumerate}
     \item
     Calculer $f'(x)$ et tracer le tableau de variations de $f$ sur l'intervalle $[0~;~5]$.
     \par
     On placera, dans le tableau, les valeurs exactes de $f(0)$, de $f(5)$ et du maximum de $f$ sur l'intervalle $[0~;~5]$.
     \item
     Montrer que l'équation $f(x)=1$ admet une unique solution $\alpha$ sur l'intervalle $[0~;~5]$.
     \item
     Donner un encadrement de $\alpha$ d'amplitude $10^{-3}$.
     \item
     Montrer que la courbe $\mathscr{C}$ possède un unique point d'inflexion dont on déterminera les coordonnées.
     \par
\end{enumerate}
\begin{corrige}
     %============================================================================================================================
     %
     \TitreC{Partie A}
     %
     %============================================================================================================================
     \par
     \begin{enumerate}
          \item %1
          La courbe $\mathscr{C}$ passe par le point $O(0~;~0)$. Par conséquent :
          \[ f(0)=0. \]
          \item %2
          $f'(0)$ est le coefficient directeur de la tangente $T$ au point $O$. Cette droite passe par les points $O(0~;~0)$ et $A(1~;~3)$ donc :
          \par
          $f'(0)=\dfrac{y_A-y_O}{x_A-x_0}=\dfrac{3-0}{1-0}=3$.
          \item %3
          \begin{enumerate}
               \item %3a
               La fonction $f$ est définie et dérivable sur l'intervalle $[0~;~5]$ et ${f(x)=(ax+b)\text{e}^{-x}+2}$.
               \par
               Posons $u(x)=ax+b$ et $v(x)=\text{e}^{-x}$.
               \par
               Alors :
               \par
               $u'(x)=a$ et $v'(x)=-\text{e}^{-x}$.
               \par
               Par ailleurs, 2 étant une constante, la dérivée de la fonction ${x \longmapsto 2}$ est nulle ; par conséquent :
               \par
               $f'(x)=u'(x)v(x)+u(x)v'(x)$
               \par
               $\phantom{f'(x)}=a \text{e}^{-x}+(ax+b)(-\text{e}^{-x})$
               \par
               $\phantom{f'(x)}=a \text{e}^{-x}-(ax+b)\text{e}^{-x}$
               \par
               $\phantom{f'(x)}=a \text{e}^{-x}-ax\text{e}^{-x} - b\text{e}^{-x}$.
               \par
               On factorise $\text{e}^{-x}$ :
               \par
               $f'(x)=(-ax+a-b)\text{e}^{-x}$.
               \par
               \cadre{rouge}{Attention}{
                    La dérivée du produit $uv$ n'est pas $u'v'$ mais $u'v+uv'$ !
               }
               \item %3b
               Comme $f(x)=(ax+b)\text{e}^{-x}+2$, alors :
               \par
               $f(0)=b\text{e}^{0}+2=b+2$.
               \par
               Par ailleurs, $f'(x)=(-ax+a-b)\text{e}^{-x}$ donc :
               \par
               $f'(0)=(a-b)\text{e}^{0}=a-b$.
               \par
               Or, $f(0)=0$ donc $b+2=0$ et $b=-2$.
               \par
               De plus $f'(0)=3$ donc $a-b=3$ soit ${a=b+3=-2+3=1}$.
               \par
               \cadre{vert}{En pratique}{
                    Pour déterminer $a$ et $b$, \textbf{pensez à utiliser les résultats des questions précédentes} (ici, c'est même indiqué dans l'énoncé !).
                    \par
                    Les égalités $f(0)=0$ et $f'(0)=3$ nous donnent deux équations qui nous permettent de déterminer $a$ et $b$.
               }
               \par
               $f$ est donc définie sur $[0~;~5]$ par :
               \[ f(x)=(x-2)\text{e}^{-x}+2. \]
               \par
          \end{enumerate}
          \par
     \end{enumerate}
     \par
     %============================================================================================================================
     %
     \TitreC{Partie B}
     %
     %============================================================================================================================
     \par
     \begin{enumerate}
          \item %1
          La fonction $f : x \longmapsto (x-2)\text{e}^{-x}+2$ est définie et dérivable sur l'intervalle $[0~;~5]$.
          \par
          Posons $u(x)=x-2$ et $v(x)=\text{e}^{-x}$.
          \par
          Alors :
          \par
          $u'(x)=1$ et $v'(x)=-\text{e}^{-x}$.
          \par
          $f'(x)=u'(x)v(x)+u(x)v'(x) + 0$
          \par
          $\phantom{f'(x)}= \text{e}^{-x}+(x-2)(-\text{e}^{-x})$
          \par
          $\phantom{f'(x)}= \text{e}^{-x}-(x-2)\text{e}^{-x}$
          \par
          $\phantom{f'(x)}= \text{e}^{-x}-x\text{e}^{-x} + 2\text{e}^{-x}$.
          \par
          On factorise $\text{e}^{-x}$ :
          \par
          $f'(x)=(3-x)\text{e}^{-x}$.
          \par
          \cadre{bleu}{Remarque}{
               Pour calculer $f'(x)$ on pouvait également utiliser le résultat de la question \textbf{3.a.} et remplacer $a$ par $1$ et $b$ par $-2$.
          }
          \par
          La fonction exponentielle prend ses valeurs dans l'intervalle $]0~;+~\infty[$ donc, pour tout réel $x$, ${\text{e}^{-x} > 0}$.
          \par
          $f'(x)$ est donc du signe de $3-x$.
          \par
          La fonction $x \longmapsto 3-x$ est une fonction affine qui s'annule pour $x=3$ et est strictement positive si et seulement si $x < 3$.
          \par
          De plus :
          \par
          $f(3)=(3-2)\text{e}^{-3}+2=\text{e}^{-3}+2\ $ et $f(5)=(5-2)\text{e}^{-5}+2=3\text{e}^{-5}+2$.
          \par
          On en déduit le tableau de variations de $f$ :
          \par
          %:-+-+-+-+- Engendré par : http://math.et.info.free.fr/TikZ/TableauxVariations/
          \begin{center}
               \begin{extern}%width="360" alt="Tableau de variations de f"
                    \begin{tikzpicture}[scale=0.875]
                         % Styles
                         \tikzstyle{cadre}=[thin]
                         \tikzstyle{fleche}=[->,>=latex,thin]
                         \tikzstyle{nondefini}=[lightgray]
                         % Dimensions Modifiables
                         \def\Lrg{1.5}
                         \def\HtX{1}
                         \def\HtY{0.5}
                         % Dimensions Calculées
                         \def\lignex{-0.5*\HtX}
                         \def\lignef{-1.5*\HtX}
                         \def\separateur{-0.5*\Lrg}
                         % Largeur du tableau
                         \def\gauche{-1.5*\Lrg}
                         \def\droite{4.6*\Lrg}
                         % Hauteur du tableau
                         \def\haut{0.5*\HtX}
                         \def\bas{-2.5*\HtX-2*\HtY}
                         % Ligne de l'abscisse : x
                         \node at (-1*\Lrg,0) {$x$};
                         \node at (0*\Lrg,0) {$0$};
                         \node at (2*\Lrg,0) {$3$};
                         \node at (4*\Lrg,0) {$5$};
                         % Ligne de la dérivée : f'(x)
                         \node at (-1*\Lrg,-1*\HtX) {$f'(x)$};
                         \node at (0*\Lrg,-1*\HtX) {$\ $};
                         \node at (1*\Lrg,-1*\HtX) {$+$};
                         \node at (2*\Lrg,-1*\HtX) {$0$};
                         \node at (3*\Lrg,-1*\HtX) {$-$};
                         \node at (4*\Lrg,-1*\HtX) {$\ $};
                         % Ligne de la fonction : f(x)
                         \node  at (-1*\Lrg,{-2*\HtX+(-1)*\HtY}) {$f(x)$};
                         \node (f1) at (0*\Lrg,{-2*\HtX+(-2)*\HtY}) {$0$};
                         \node (f2) at (2*\Lrg,{-2*\HtX+(0)*\HtY}) {$\text{e}^{-3}+2$};
                         \node (f3) at (3.9*\Lrg,{-2*\HtX+(-2)*\HtY}) {$3\text{e}^{-5}+2$};
                         % Flèches
                         \draw[fleche] (f1) -- (f2);
                         \draw[fleche] (f2) -- (f3);
                         % Encadrement
                         \draw[cadre] (\separateur,\haut) -- (\separateur,\bas);
                         \draw[cadre] (\gauche,\haut) rectangle  (\droite,\bas);
                         \draw[cadre] (\gauche,\lignex) -- (\droite,\lignex);
                         \draw[cadre] (\gauche,\lignef) -- (\droite,\lignef);
                    \end{tikzpicture}
               \end{extern}
          \end{center}
          \cadre{bleu}{Remarque}{
               Sauf indication contraire de l'énoncé, il est préférable de \textbf{conserver les valeurs exactes} (ici, c'est même impératif car précisé dans la question) dans le tableau de variations, quitte à calculer une valeur approchée par la suite si nécessaire.
          }
          \item %2
          $\text{e}^{-3}+2 \approx 2,05$\\
          $3\text{e}^{-5}+2 \approx 2,02$
          \par
          Sur l'intervalle $[0~;~3]$, $f$ est \textbf{continue} et \textbf{strictement croissante}. 1 appartient à l'intervalle $[0~;\text{e}^{-3}+2 ]$ donc l'équation $f(x)=1$ admet une unique solution sur l'intervalle $[0~;~3]$.
          \par
          Sur l'intervalle $[3~;~5]$, le minimum de $f$ est supérieur à 2 donc l'équation ${f(x)=1}$ n'a pas de solution sur cet intervalle.
          \par
          Par conséquent, l'équation $f(x)=1$ admet une unique solution sur l'intervalle $[0~;~5]$.
          \item %3
          \`A la calculatrice \textit{(voir détails  \hyperlink{tvi-pap}{ci-après})}, on trouve :
          \par
          $f(0,442) \approx 0,9986 < 1$ ;
          \par
          $f(0,443) \approx 1,0002 > 1$.
          \par
          Par conséquent : $0,442 < \alpha < 0,443$.
          \par
          \cadre{rouge}{Bien rédiger}{
               Pour justifier un encadrement du type ${\alpha_1 < \alpha < \alpha_2}$, vous pouvez indiquer sur votre copie les valeurs de $f(\alpha_1)$ et de $f(\alpha_2)$ que vous avez obtenues à la calculatrice.
          }
          \item %4
          La fonction $f' : x \longmapsto (3-x)\text{e}^{-x}$ est dérivable sur l'intervalle $[0~;~5]$.
          \par
          Posons $u(x)=3-x$ et $v(x)=\text{e}^{-x}$.
          \par
          Alors :
          \par
          $u'(x)=-1$ et $v'(x)=-\text{e}^{-x}$.
          \par
          $f''(x)=u'(x)v(x)+u(x)v'(x) + 0$
          \par
          $\phantom{f''(x)}= -1 \times \text{e}^{-x}+(3-x)(-\text{e}^{-x})$
          \par
          $\phantom{f''(x)}= -\text{e}^{-x}-(3-x)\text{e}^{-x}$
          \par
          $\phantom{f''(x)}=(-1-3+x)\text{e}^{-x}$
          \par
          $\phantom{f''(x)}=(x-4)\text{e}^{-x}$.
          \par
          Pour tout réel $x$, ${\text{e}^{-x} > 0}$, donc $f''(x)$ est donc du signe de $x-4$.
          \par
          La fonction $x \longmapsto x-4$ est une fonction affine qui s'annule pour et \textbf{change de signe} pour $x=4$.
          \par
          La courbe $\mathscr{C}$ possède donc un unique point d'inflexion d'abscisse $4$ et d'ordonnée $f(4)=2 \text{e}^{-4}+2$.
          \par
     \end{enumerate}
\end{corrige}

\end{document}
µ
\documentclass[a4paper]{article}

%================================================================================================================================
%
% Packages
%
%================================================================================================================================

\usepackage[T1]{fontenc} 	% pour caractères accentués
\usepackage[utf8]{inputenc}  % encodage utf8
\usepackage[french]{babel}	% langue : français
\usepackage{fourier}			% caractères plus lisibles
\usepackage[dvipsnames]{xcolor} % couleurs
\usepackage{fancyhdr}		% réglage header footer
\usepackage{needspace}		% empêcher sauts de page mal placés
\usepackage{graphicx}		% pour inclure des graphiques
\usepackage{enumitem,cprotect}		% personnalise les listes d'items (nécessaire pour ol, al ...)
\usepackage{hyperref}		% Liens hypertexte
\usepackage{pstricks,pst-all,pst-node,pstricks-add,pst-math,pst-plot,pst-tree,pst-eucl} % pstricks
\usepackage[a4paper,includeheadfoot,top=2cm,left=3cm, bottom=2cm,right=3cm]{geometry} % marges etc.
\usepackage{comment}			% commentaires multilignes
\usepackage{amsmath,environ} % maths (matrices, etc.)
\usepackage{amssymb,makeidx}
\usepackage{bm}				% bold maths
\usepackage{tabularx}		% tableaux
\usepackage{colortbl}		% tableaux en couleur
\usepackage{fontawesome}		% Fontawesome
\usepackage{environ}			% environment with command
\usepackage{fp}				% calculs pour ps-tricks
\usepackage{multido}			% pour ps tricks
\usepackage[np]{numprint}	% formattage nombre
\usepackage{tikz,tkz-tab} 			% package principal TikZ
\usepackage{pgfplots}   % axes
\usepackage{mathrsfs}    % cursives
\usepackage{calc}			% calcul taille boites
\usepackage[scaled=0.875]{helvet} % font sans serif
\usepackage{svg} % svg
\usepackage{scrextend} % local margin
\usepackage{scratch} %scratch
\usepackage{multicol} % colonnes
%\usepackage{infix-RPN,pst-func} % formule en notation polanaise inversée
\usepackage{listings}

%================================================================================================================================
%
% Réglages de base
%
%================================================================================================================================

\lstset{
language=Python,   % R code
literate=
{á}{{\'a}}1
{à}{{\`a}}1
{ã}{{\~a}}1
{é}{{\'e}}1
{è}{{\`e}}1
{ê}{{\^e}}1
{í}{{\'i}}1
{ó}{{\'o}}1
{õ}{{\~o}}1
{ú}{{\'u}}1
{ü}{{\"u}}1
{ç}{{\c{c}}}1
{~}{{ }}1
}


\definecolor{codegreen}{rgb}{0,0.6,0}
\definecolor{codegray}{rgb}{0.5,0.5,0.5}
\definecolor{codepurple}{rgb}{0.58,0,0.82}
\definecolor{backcolour}{rgb}{0.95,0.95,0.92}

\lstdefinestyle{mystyle}{
    backgroundcolor=\color{backcolour},   
    commentstyle=\color{codegreen},
    keywordstyle=\color{magenta},
    numberstyle=\tiny\color{codegray},
    stringstyle=\color{codepurple},
    basicstyle=\ttfamily\footnotesize,
    breakatwhitespace=false,         
    breaklines=true,                 
    captionpos=b,                    
    keepspaces=true,                 
    numbers=left,                    
xleftmargin=2em,
framexleftmargin=2em,            
    showspaces=false,                
    showstringspaces=false,
    showtabs=false,                  
    tabsize=2,
    upquote=true
}

\lstset{style=mystyle}


\lstset{style=mystyle}
\newcommand{\imgdir}{C:/laragon/www/newmc/assets/imgsvg/}
\newcommand{\imgsvgdir}{C:/laragon/www/newmc/assets/imgsvg/}

\definecolor{mcgris}{RGB}{220, 220, 220}% ancien~; pour compatibilité
\definecolor{mcbleu}{RGB}{52, 152, 219}
\definecolor{mcvert}{RGB}{125, 194, 70}
\definecolor{mcmauve}{RGB}{154, 0, 215}
\definecolor{mcorange}{RGB}{255, 96, 0}
\definecolor{mcturquoise}{RGB}{0, 153, 153}
\definecolor{mcrouge}{RGB}{255, 0, 0}
\definecolor{mclightvert}{RGB}{205, 234, 190}

\definecolor{gris}{RGB}{220, 220, 220}
\definecolor{bleu}{RGB}{52, 152, 219}
\definecolor{vert}{RGB}{125, 194, 70}
\definecolor{mauve}{RGB}{154, 0, 215}
\definecolor{orange}{RGB}{255, 96, 0}
\definecolor{turquoise}{RGB}{0, 153, 153}
\definecolor{rouge}{RGB}{255, 0, 0}
\definecolor{lightvert}{RGB}{205, 234, 190}
\setitemize[0]{label=\color{lightvert}  $\bullet$}

\pagestyle{fancy}
\renewcommand{\headrulewidth}{0.2pt}
\fancyhead[L]{maths-cours.fr}
\fancyhead[R]{\thepage}
\renewcommand{\footrulewidth}{0.2pt}
\fancyfoot[C]{}

\newcolumntype{C}{>{\centering\arraybackslash}X}
\newcolumntype{s}{>{\hsize=.35\hsize\arraybackslash}X}

\setlength{\parindent}{0pt}		 
\setlength{\parskip}{3mm}
\setlength{\headheight}{1cm}

\def\ebook{ebook}
\def\book{book}
\def\web{web}
\def\type{web}

\newcommand{\vect}[1]{\overrightarrow{\,\mathstrut#1\,}}

\def\Oij{$\left(\text{O}~;~\vect{\imath},~\vect{\jmath}\right)$}
\def\Oijk{$\left(\text{O}~;~\vect{\imath},~\vect{\jmath},~\vect{k}\right)$}
\def\Ouv{$\left(\text{O}~;~\vect{u},~\vect{v}\right)$}

\hypersetup{breaklinks=true, colorlinks = true, linkcolor = OliveGreen, urlcolor = OliveGreen, citecolor = OliveGreen, pdfauthor={Didier BONNEL - https://www.maths-cours.fr} } % supprime les bordures autour des liens

\renewcommand{\arg}[0]{\text{arg}}

\everymath{\displaystyle}

%================================================================================================================================
%
% Macros - Commandes
%
%================================================================================================================================

\newcommand\meta[2]{    			% Utilisé pour créer le post HTML.
	\def\titre{titre}
	\def\url{url}
	\def\arg{#1}
	\ifx\titre\arg
		\newcommand\maintitle{#2}
		\fancyhead[L]{#2}
		{\Large\sffamily \MakeUppercase{#2}}
		\vspace{1mm}\textcolor{mcvert}{\hrule}
	\fi 
	\ifx\url\arg
		\fancyfoot[L]{\href{https://www.maths-cours.fr#2}{\black \footnotesize{https://www.maths-cours.fr#2}}}
	\fi 
}


\newcommand\TitreC[1]{    		% Titre centré
     \needspace{3\baselineskip}
     \begin{center}\textbf{#1}\end{center}
}

\newcommand\newpar{    		% paragraphe
     \par
}

\newcommand\nosp {    		% commande vide (pas d'espace)
}
\newcommand{\id}[1]{} %ignore

\newcommand\boite[2]{				% Boite simple sans titre
	\vspace{5mm}
	\setlength{\fboxrule}{0.2mm}
	\setlength{\fboxsep}{5mm}	
	\fcolorbox{#1}{#1!3}{\makebox[\linewidth-2\fboxrule-2\fboxsep]{
  		\begin{minipage}[t]{\linewidth-2\fboxrule-4\fboxsep}\setlength{\parskip}{3mm}
  			 #2
  		\end{minipage}
	}}
	\vspace{5mm}
}

\newcommand\CBox[4]{				% Boites
	\vspace{5mm}
	\setlength{\fboxrule}{0.2mm}
	\setlength{\fboxsep}{5mm}
	
	\fcolorbox{#1}{#1!3}{\makebox[\linewidth-2\fboxrule-2\fboxsep]{
		\begin{minipage}[t]{1cm}\setlength{\parskip}{3mm}
	  		\textcolor{#1}{\LARGE{#2}}    
 	 	\end{minipage}  
  		\begin{minipage}[t]{\linewidth-2\fboxrule-4\fboxsep}\setlength{\parskip}{3mm}
			\raisebox{1.2mm}{\normalsize\sffamily{\textcolor{#1}{#3}}}						
  			 #4
  		\end{minipage}
	}}
	\vspace{5mm}
}

\newcommand\cadre[3]{				% Boites convertible html
	\par
	\vspace{2mm}
	\setlength{\fboxrule}{0.1mm}
	\setlength{\fboxsep}{5mm}
	\fcolorbox{#1}{white}{\makebox[\linewidth-2\fboxrule-2\fboxsep]{
  		\begin{minipage}[t]{\linewidth-2\fboxrule-4\fboxsep}\setlength{\parskip}{3mm}
			\raisebox{-2.5mm}{\sffamily \small{\textcolor{#1}{\MakeUppercase{#2}}}}		
			\par		
  			 #3
 	 		\end{minipage}
	}}
		\vspace{2mm}
	\par
}

\newcommand\bloc[3]{				% Boites convertible html sans bordure
     \needspace{2\baselineskip}
     {\sffamily \small{\textcolor{#1}{\MakeUppercase{#2}}}}    
		\par		
  			 #3
		\par
}

\newcommand\CHelp[1]{
     \CBox{Plum}{\faInfoCircle}{À RETENIR}{#1}
}

\newcommand\CUp[1]{
     \CBox{NavyBlue}{\faThumbsOUp}{EN PRATIQUE}{#1}
}

\newcommand\CInfo[1]{
     \CBox{Sepia}{\faArrowCircleRight}{REMARQUE}{#1}
}

\newcommand\CRedac[1]{
     \CBox{PineGreen}{\faEdit}{BIEN R\'EDIGER}{#1}
}

\newcommand\CError[1]{
     \CBox{Red}{\faExclamationTriangle}{ATTENTION}{#1}
}

\newcommand\TitreExo[2]{
\needspace{4\baselineskip}
 {\sffamily\large EXERCICE #1\ (\emph{#2 points})}
\vspace{5mm}
}

\newcommand\img[2]{
          \includegraphics[width=#2\paperwidth]{\imgdir#1}
}

\newcommand\imgsvg[2]{
       \begin{center}   \includegraphics[width=#2\paperwidth]{\imgsvgdir#1} \end{center}
}


\newcommand\Lien[2]{
     \href{#1}{#2 \tiny \faExternalLink}
}
\newcommand\mcLien[2]{
     \href{https~://www.maths-cours.fr/#1}{#2 \tiny \faExternalLink}
}

\newcommand{\euro}{\eurologo{}}

%================================================================================================================================
%
% Macros - Environement
%
%================================================================================================================================

\newenvironment{tex}{ %
}
{%
}

\newenvironment{indente}{ %
	\setlength\parindent{10mm}
}

{
	\setlength\parindent{0mm}
}

\newenvironment{corrige}{%
     \needspace{3\baselineskip}
     \medskip
     \textbf{\textsc{Corrigé}}
     \medskip
}
{
}

\newenvironment{extern}{%
     \begin{center}
     }
     {
     \end{center}
}

\NewEnviron{code}{%
	\par
     \boite{gray}{\texttt{%
     \BODY
     }}
     \par
}

\newenvironment{vbloc}{% boite sans cadre empeche saut de page
     \begin{minipage}[t]{\linewidth}
     }
     {
     \end{minipage}
}
\NewEnviron{h2}{%
    \needspace{3\baselineskip}
    \vspace{0.6cm}
	\noindent \MakeUppercase{\sffamily \large \BODY}
	\vspace{1mm}\textcolor{mcgris}{\hrule}\vspace{0.4cm}
	\par
}{}

\NewEnviron{h3}{%
    \needspace{3\baselineskip}
	\vspace{5mm}
	\textsc{\BODY}
	\par
}

\NewEnviron{margeneg}{ %
\begin{addmargin}[-1cm]{0cm}
\BODY
\end{addmargin}
}

\NewEnviron{html}{%
}

\begin{document}
\meta{url}{/exercices/probabilites-bac-blanc-es-l-sujet-3-maths-cours-2018/}
\meta{pid}{10477}
\meta{titre}{Probabilités - Bac blanc ES/L Sujet 3 - Maths-cours 2018}
\meta{type}{exercices}
%
\begin{h2}Exercice 4 (3 points)\end{h2}
\par
\textit{Dans cet exercice, toute trace de recherche, même incomplète, ou d'initiative, même infructueuse, sera prise en compte dans l'évaluation.}
\par
\vspace{0.5cm}
\par
Dans le cadre d'essais cliniques, on souhaite tester l'efficacité d'un nouveau médicament destiné à lutter contre l'excès de cholestérol.
\par
L'expérimentation s'effectue sur un échantillon de patients présentant un excès de cholestérol dans le sang.
\par
Lors de cet essai clinique, 70\% des patients reçoivent le médicament tandis que les 30\% restant reçoivent un placebo (comprimé sans principe actif).
\par
\`A la fin de la période de test, le taux de cholestérol de chaque patient est mesuré et comparé au taux initial.
\par
On observe une baisse significative du taux de cholestérol chez 85\% des personnes ayant pris le médicament tandis que chez les personnes ayant pris le placebo, cette baisse n'est constatée que dans 20\% des cas.
\par
Le laboratoire pharmaceutique ayant réalisé cette étude affirme que \og plus de 90\% des patients chez qui une baisse significative a été constatée avaient pris le médicament \fg{}.
\par
Que pensez-vous de cette affirmation ? \\
Justifier votre réponse.
\begin{corrige}
     \par
     Choisissons un patient au hasard et notons :
     \par
     \begin{itemize}
          \item
          $M$ : l'événement \og le patient a pris le médicament \fg{} ;
          \item
          $\overline{M}$ : l'événement \og le patient a pris le placebo \fg{} ;
          \item
          $B$ : l'événement \og le taux de cholestérol du patient a baissé \fg{} ;
          \item
          $\overline{B}$ : l'événement \og le taux de cholestérol du patient n'a pas baissé \fg{}.
          \par
     \end{itemize}
     \par
     Les données de l'énoncé permettent de construire l'arbre suivant :
     \par
     %:-+-+-+- Engendré par : http://math.et.info.free.fr/TikZ/Arbre/
     \begin{center}
          % Racine à Gauche, développement vers la droite
          \begin{extern}%width="360" alt="Arbre bac blanc"
               \begin{tikzpicture}[xscale=1,yscale=1]
                    % Styles (MODIFIABLES)
                    \tikzstyle{fleche}=[-,>=latex,thick]
                    \tikzstyle{noeud}=[fill=white,circle,draw]
                    \tikzstyle{feuille}=[fill=white,circle,draw]
                    \tikzstyle{etiquette}=[midway,fill=white]
                    % Dimensions (MODIFIABLES)
                    \def\DistanceInterNiveaux{3}
                    \def\DistanceInterFeuilles{2}
                    % Dimensions calculées (NON MODIFIABLES)
                    \def\NiveauA{(0)*\DistanceInterNiveaux}
                    \def\NiveauB{(1.5)*\DistanceInterNiveaux}
                    \def\NiveauC{(2.5)*\DistanceInterNiveaux}
                    \def\InterFeuilles{(-1)*\DistanceInterFeuilles}
                    % Noeuds (MODIFIABLES : Styles et Coefficients d'InterFeuilles)
                    \node[noeud] (R) at ({\NiveauA},{(1.5)*\InterFeuilles}) {$\ $};
                    \node[noeud] (Ra) at ({\NiveauB},{(0.5)*\InterFeuilles}) {$M$};
                    \node[feuille] (Raa) at ({\NiveauC},{(0)*\InterFeuilles}) {$B$};
                    \node[feuille] (Rab) at ({\NiveauC},{(1)*\InterFeuilles}) {$\overline{B}$};
                    \node[noeud] (Rb) at ({\NiveauB},{(2.5)*\InterFeuilles}) {$\overline{M}$};
                    \node[feuille] (Rba) at ({\NiveauC},{(2)*\InterFeuilles}) {$B$};
                    \node[feuille] (Rbb) at ({\NiveauC},{(3)*\InterFeuilles}) {$\overline{B}$};
                    % Arcs (MODIFIABLES : Styles)
                    \draw[fleche] (R)--(Ra) node[etiquette] {$0,7$};
                    \draw[fleche] (Ra)--(Raa) node[etiquette] {$0,85$};
                    \draw[fleche] (Ra)--(Rab) node[etiquette] {$0,15$};
                    \draw[fleche] (R)--(Rb) node[etiquette] {$0,3$};
                    \draw[fleche] (Rb)--(Rba) node[etiquette] {$0,2$};
                    \draw[fleche] (Rb)--(Rbb) node[etiquette] {$0,8$};
               \end{tikzpicture}
          \end{extern}
     \end{center}
     Pour juger la validité de l'affirmation du laboratoire, il faut évaluer la probabilité qu'un patient ait pris le médicament, sachant que son taux de cholestérol a diminué.
     \par
     Il faut calculer $p_B(M)$.
     \par
     D'après la formule des probabilités conditionnelles :
     \par
     $p_B(M)=\dfrac{p(B \cap M)}{p(B)}$.
     \par
     Or :
     \par
     $p(B \cap M) = p(M) \times p_M(B)=0,7 \times 0,85 = 0,595$ ;
     \par
     et, d'après la formule des probabilités totales :
     \par
     $p(B)=p(M) \times p_M(B) + p(\overline{M})  p_{\overline{M}}(B) = 0,7 \times 0,85 +0,3 \times 0,2=0,655$.
     \par
     Par conséquent :
     \par
     $p_B(M)=\dfrac{0,595}{0,655} \approx 0,91 = 91\%$.
     \par
     Cette probabilité est supérieure à 90\% donc \textbf{l'affirmation du laboratoire pharmaceutique est exacte}.
\end{corrige}

\end{document}
µ
\documentclass[a4paper]{article}

%================================================================================================================================
%
% Packages
%
%================================================================================================================================

\usepackage[T1]{fontenc} 	% pour caractères accentués
\usepackage[utf8]{inputenc}  % encodage utf8
\usepackage[french]{babel}	% langue : français
\usepackage{fourier}			% caractères plus lisibles
\usepackage[dvipsnames]{xcolor} % couleurs
\usepackage{fancyhdr}		% réglage header footer
\usepackage{needspace}		% empêcher sauts de page mal placés
\usepackage{graphicx}		% pour inclure des graphiques
\usepackage{enumitem,cprotect}		% personnalise les listes d'items (nécessaire pour ol, al ...)
\usepackage{hyperref}		% Liens hypertexte
\usepackage{pstricks,pst-all,pst-node,pstricks-add,pst-math,pst-plot,pst-tree,pst-eucl} % pstricks
\usepackage[a4paper,includeheadfoot,top=2cm,left=3cm, bottom=2cm,right=3cm]{geometry} % marges etc.
\usepackage{comment}			% commentaires multilignes
\usepackage{amsmath,environ} % maths (matrices, etc.)
\usepackage{amssymb,makeidx}
\usepackage{bm}				% bold maths
\usepackage{tabularx}		% tableaux
\usepackage{colortbl}		% tableaux en couleur
\usepackage{fontawesome}		% Fontawesome
\usepackage{environ}			% environment with command
\usepackage{fp}				% calculs pour ps-tricks
\usepackage{multido}			% pour ps tricks
\usepackage[np]{numprint}	% formattage nombre
\usepackage{tikz,tkz-tab} 			% package principal TikZ
\usepackage{pgfplots}   % axes
\usepackage{mathrsfs}    % cursives
\usepackage{calc}			% calcul taille boites
\usepackage[scaled=0.875]{helvet} % font sans serif
\usepackage{svg} % svg
\usepackage{scrextend} % local margin
\usepackage{scratch} %scratch
\usepackage{multicol} % colonnes
%\usepackage{infix-RPN,pst-func} % formule en notation polanaise inversée
\usepackage{listings}

%================================================================================================================================
%
% Réglages de base
%
%================================================================================================================================

\lstset{
language=Python,   % R code
literate=
{á}{{\'a}}1
{à}{{\`a}}1
{ã}{{\~a}}1
{é}{{\'e}}1
{è}{{\`e}}1
{ê}{{\^e}}1
{í}{{\'i}}1
{ó}{{\'o}}1
{õ}{{\~o}}1
{ú}{{\'u}}1
{ü}{{\"u}}1
{ç}{{\c{c}}}1
{~}{{ }}1
}


\definecolor{codegreen}{rgb}{0,0.6,0}
\definecolor{codegray}{rgb}{0.5,0.5,0.5}
\definecolor{codepurple}{rgb}{0.58,0,0.82}
\definecolor{backcolour}{rgb}{0.95,0.95,0.92}

\lstdefinestyle{mystyle}{
    backgroundcolor=\color{backcolour},   
    commentstyle=\color{codegreen},
    keywordstyle=\color{magenta},
    numberstyle=\tiny\color{codegray},
    stringstyle=\color{codepurple},
    basicstyle=\ttfamily\footnotesize,
    breakatwhitespace=false,         
    breaklines=true,                 
    captionpos=b,                    
    keepspaces=true,                 
    numbers=left,                    
xleftmargin=2em,
framexleftmargin=2em,            
    showspaces=false,                
    showstringspaces=false,
    showtabs=false,                  
    tabsize=2,
    upquote=true
}

\lstset{style=mystyle}


\lstset{style=mystyle}
\newcommand{\imgdir}{C:/laragon/www/newmc/assets/imgsvg/}
\newcommand{\imgsvgdir}{C:/laragon/www/newmc/assets/imgsvg/}

\definecolor{mcgris}{RGB}{220, 220, 220}% ancien~; pour compatibilité
\definecolor{mcbleu}{RGB}{52, 152, 219}
\definecolor{mcvert}{RGB}{125, 194, 70}
\definecolor{mcmauve}{RGB}{154, 0, 215}
\definecolor{mcorange}{RGB}{255, 96, 0}
\definecolor{mcturquoise}{RGB}{0, 153, 153}
\definecolor{mcrouge}{RGB}{255, 0, 0}
\definecolor{mclightvert}{RGB}{205, 234, 190}

\definecolor{gris}{RGB}{220, 220, 220}
\definecolor{bleu}{RGB}{52, 152, 219}
\definecolor{vert}{RGB}{125, 194, 70}
\definecolor{mauve}{RGB}{154, 0, 215}
\definecolor{orange}{RGB}{255, 96, 0}
\definecolor{turquoise}{RGB}{0, 153, 153}
\definecolor{rouge}{RGB}{255, 0, 0}
\definecolor{lightvert}{RGB}{205, 234, 190}
\setitemize[0]{label=\color{lightvert}  $\bullet$}

\pagestyle{fancy}
\renewcommand{\headrulewidth}{0.2pt}
\fancyhead[L]{maths-cours.fr}
\fancyhead[R]{\thepage}
\renewcommand{\footrulewidth}{0.2pt}
\fancyfoot[C]{}

\newcolumntype{C}{>{\centering\arraybackslash}X}
\newcolumntype{s}{>{\hsize=.35\hsize\arraybackslash}X}

\setlength{\parindent}{0pt}		 
\setlength{\parskip}{3mm}
\setlength{\headheight}{1cm}

\def\ebook{ebook}
\def\book{book}
\def\web{web}
\def\type{web}

\newcommand{\vect}[1]{\overrightarrow{\,\mathstrut#1\,}}

\def\Oij{$\left(\text{O}~;~\vect{\imath},~\vect{\jmath}\right)$}
\def\Oijk{$\left(\text{O}~;~\vect{\imath},~\vect{\jmath},~\vect{k}\right)$}
\def\Ouv{$\left(\text{O}~;~\vect{u},~\vect{v}\right)$}

\hypersetup{breaklinks=true, colorlinks = true, linkcolor = OliveGreen, urlcolor = OliveGreen, citecolor = OliveGreen, pdfauthor={Didier BONNEL - https://www.maths-cours.fr} } % supprime les bordures autour des liens

\renewcommand{\arg}[0]{\text{arg}}

\everymath{\displaystyle}

%================================================================================================================================
%
% Macros - Commandes
%
%================================================================================================================================

\newcommand\meta[2]{    			% Utilisé pour créer le post HTML.
	\def\titre{titre}
	\def\url{url}
	\def\arg{#1}
	\ifx\titre\arg
		\newcommand\maintitle{#2}
		\fancyhead[L]{#2}
		{\Large\sffamily \MakeUppercase{#2}}
		\vspace{1mm}\textcolor{mcvert}{\hrule}
	\fi 
	\ifx\url\arg
		\fancyfoot[L]{\href{https://www.maths-cours.fr#2}{\black \footnotesize{https://www.maths-cours.fr#2}}}
	\fi 
}


\newcommand\TitreC[1]{    		% Titre centré
     \needspace{3\baselineskip}
     \begin{center}\textbf{#1}\end{center}
}

\newcommand\newpar{    		% paragraphe
     \par
}

\newcommand\nosp {    		% commande vide (pas d'espace)
}
\newcommand{\id}[1]{} %ignore

\newcommand\boite[2]{				% Boite simple sans titre
	\vspace{5mm}
	\setlength{\fboxrule}{0.2mm}
	\setlength{\fboxsep}{5mm}	
	\fcolorbox{#1}{#1!3}{\makebox[\linewidth-2\fboxrule-2\fboxsep]{
  		\begin{minipage}[t]{\linewidth-2\fboxrule-4\fboxsep}\setlength{\parskip}{3mm}
  			 #2
  		\end{minipage}
	}}
	\vspace{5mm}
}

\newcommand\CBox[4]{				% Boites
	\vspace{5mm}
	\setlength{\fboxrule}{0.2mm}
	\setlength{\fboxsep}{5mm}
	
	\fcolorbox{#1}{#1!3}{\makebox[\linewidth-2\fboxrule-2\fboxsep]{
		\begin{minipage}[t]{1cm}\setlength{\parskip}{3mm}
	  		\textcolor{#1}{\LARGE{#2}}    
 	 	\end{minipage}  
  		\begin{minipage}[t]{\linewidth-2\fboxrule-4\fboxsep}\setlength{\parskip}{3mm}
			\raisebox{1.2mm}{\normalsize\sffamily{\textcolor{#1}{#3}}}						
  			 #4
  		\end{minipage}
	}}
	\vspace{5mm}
}

\newcommand\cadre[3]{				% Boites convertible html
	\par
	\vspace{2mm}
	\setlength{\fboxrule}{0.1mm}
	\setlength{\fboxsep}{5mm}
	\fcolorbox{#1}{white}{\makebox[\linewidth-2\fboxrule-2\fboxsep]{
  		\begin{minipage}[t]{\linewidth-2\fboxrule-4\fboxsep}\setlength{\parskip}{3mm}
			\raisebox{-2.5mm}{\sffamily \small{\textcolor{#1}{\MakeUppercase{#2}}}}		
			\par		
  			 #3
 	 		\end{minipage}
	}}
		\vspace{2mm}
	\par
}

\newcommand\bloc[3]{				% Boites convertible html sans bordure
     \needspace{2\baselineskip}
     {\sffamily \small{\textcolor{#1}{\MakeUppercase{#2}}}}    
		\par		
  			 #3
		\par
}

\newcommand\CHelp[1]{
     \CBox{Plum}{\faInfoCircle}{À RETENIR}{#1}
}

\newcommand\CUp[1]{
     \CBox{NavyBlue}{\faThumbsOUp}{EN PRATIQUE}{#1}
}

\newcommand\CInfo[1]{
     \CBox{Sepia}{\faArrowCircleRight}{REMARQUE}{#1}
}

\newcommand\CRedac[1]{
     \CBox{PineGreen}{\faEdit}{BIEN R\'EDIGER}{#1}
}

\newcommand\CError[1]{
     \CBox{Red}{\faExclamationTriangle}{ATTENTION}{#1}
}

\newcommand\TitreExo[2]{
\needspace{4\baselineskip}
 {\sffamily\large EXERCICE #1\ (\emph{#2 points})}
\vspace{5mm}
}

\newcommand\img[2]{
          \includegraphics[width=#2\paperwidth]{\imgdir#1}
}

\newcommand\imgsvg[2]{
       \begin{center}   \includegraphics[width=#2\paperwidth]{\imgsvgdir#1} \end{center}
}


\newcommand\Lien[2]{
     \href{#1}{#2 \tiny \faExternalLink}
}
\newcommand\mcLien[2]{
     \href{https~://www.maths-cours.fr/#1}{#2 \tiny \faExternalLink}
}

\newcommand{\euro}{\eurologo{}}

%================================================================================================================================
%
% Macros - Environement
%
%================================================================================================================================

\newenvironment{tex}{ %
}
{%
}

\newenvironment{indente}{ %
	\setlength\parindent{10mm}
}

{
	\setlength\parindent{0mm}
}

\newenvironment{corrige}{%
     \needspace{3\baselineskip}
     \medskip
     \textbf{\textsc{Corrigé}}
     \medskip
}
{
}

\newenvironment{extern}{%
     \begin{center}
     }
     {
     \end{center}
}

\NewEnviron{code}{%
	\par
     \boite{gray}{\texttt{%
     \BODY
     }}
     \par
}

\newenvironment{vbloc}{% boite sans cadre empeche saut de page
     \begin{minipage}[t]{\linewidth}
     }
     {
     \end{minipage}
}
\NewEnviron{h2}{%
    \needspace{3\baselineskip}
    \vspace{0.6cm}
	\noindent \MakeUppercase{\sffamily \large \BODY}
	\vspace{1mm}\textcolor{mcgris}{\hrule}\vspace{0.4cm}
	\par
}{}

\NewEnviron{h3}{%
    \needspace{3\baselineskip}
	\vspace{5mm}
	\textsc{\BODY}
	\par
}

\NewEnviron{margeneg}{ %
\begin{addmargin}[-1cm]{0cm}
\BODY
\end{addmargin}
}

\NewEnviron{html}{%
}

\begin{document}
\meta{url}{/exercices/graphes-bac-blanc-es-l-sujet-3-maths-cours-2018-spe/}
\meta{pid}{10479}
\meta{titre}{Graphes - Bac blanc ES/L Sujet 3 - Maths-cours 2018 (spé)}
\meta{type}{exercices}
%
\begin{h2}Exercice 3 (5 points)\end{h2}
\par
\textbf{Candidats ayant suivi l'enseignement de spécialité}
\par
\emph{Pour chacune des cinq affirmations suivantes, indiquer si elle est vraie ou fausse en justifiant la réponse.\\ Il est attribué un point par réponse exacte correctement justifiée.\\ \textbf{Une réponse non justifiée n'est pas prise en compte.}}\index{Vrai--Faux}
\par
On modélise le plan d'un village à l'aide du graphe (G) ci-dessous :
\par
\begin{center}
     \begin{extern}%width="350" alt="modélisation à l'aide d'un graphe"
          \psset{unit=0.7cm}
          \begin{pspicture}(12,8)
               \rput(0.75,7.5){\circlenode{A}{A}}
               \rput(7.2,7.5){\circlenode{B}{B}}
               \rput(11.3,5.5){\circlenode{C}{C}}
               \rput(10.3,2){\circlenode{D}{D}}
               \rput(0.3,2.7){\circlenode{E}{E}}
               \rput(6.7,4.3){\circlenode{F}{F}}
               \ncline{A}{B}
               \ncline{A}{E}
               \ncline{B}{E}
               \ncline{B}{C}
               \ncline{B}{D}
               \ncline{B}{F}
               \ncline{C}{D}
               \ncline{D}{E}
               \ncline{D}{F}
               \ncline{E}{F}
          \end{pspicture}
     \end{extern}
\end{center}
\par
Les sommets du graphe (G) représentent les carrefours et les arêtes du graphe schématisent les routes reliant ces carrefours.
\par
\begin{itemize}
     \item %1
     \textbf{Affirmation 1 :}\quad Le graphe (G) est connexe.
     \item %2
     \textbf{Affirmation 2 :}\quad Le graphe (G) contient un sous-graphe complet d'ordre 4.
     \item %3
     \textbf{Affirmation 3 :}\quad Une personne peut parcourir toutes les routes du village sans emprunter plusieurs fois la même route.
     \item %4
     \textbf{Affirmation 4 :}\quad Il y a exactement 5 trajets de trois routes reliant les carrefours C et E.\\
     \textit{On pourra utiliser la calculatrice pour justifier la réponse à l'aide d'un calcul matriciel.}
     \par
\end{itemize}
On pondère le graphe (G) par les longueurs, en centaines de mètres, de chacune des routes :
\par
\begin{center}
     \begin{extern}%width="350" alt=" graphe pondéré"
          \psset{unit=0.7cm}
          \begin{pspicture}(12,8)
               \rput(0.75,7.5){\circlenode{A}{A}}
               \rput(7.2,7.5){\circlenode{B}{B}}
               \rput(11.3,5.5){\circlenode{C}{C}}
               \rput(10.3,2){\circlenode{D}{D}}
               \rput(0.3,2.7){\circlenode{E}{E}}
               \rput(6.7,4.3){\circlenode{F}{F}}
               \ncline{A}{B}\ncput*[nrot=:U]{3}
               \ncline{A}{E}\ncput*[nrot=:U]{2}
               \ncline{E}{B}\ncput*[nrot=:U]{5}
               \ncline{B}{C}\ncput*[nrot=:U]{2}
               \ncline{B}{D}\ncput*[nrot=:U]{4}
               \ncline{B}{F}\ncput*[nrot=:U]{1}
               \ncline{C}{D}\ncput*[nrot=:U]{2}
               \ncline{E}{D}\ncput*[nrot=:U]{5}
               \ncline{F}{D}\ncput*[nrot=:U]{2}
               \ncline{E}{F}\ncput*[nrot=:U]{5}
          \end{pspicture}
     \end{extern}
\end{center}
\par
\begin{itemize}
     \item %5
     \textbf{Affirmation 5 :}\quad Le plus court chemin menant de A à D mesure 700 mètres .\\
     \textit{On justifiera la réponse à l'aide d'un algorithme.}
\end{itemize}
\begin{corrige}
     \begin{itemize}
          \item %1
          \textbf{Affirmation 1 :}\quad Le graphe (G) est connexe : \textbf{EXACT}.
          \par
          Un graphe est connexe si et seulement si on peut relier deux quelconques de ses sommets par une chaîne.
          \par
          C'est bien le cas ici donc le graphe (G) est connexe.
          \item %2
          \textbf{Affirmation 2 :}\quad Le graphe (G) contient un sous-graphe complet d'ordre 4 : \textbf{EXACT}.
          \par
          Considérons le sous-graphe d'ordre 4 composé des sommets B, D, E et F. Chacun de ces sommets est relié aux trois autres.\\
          Ce sous-graphe est donc complet.
          \item %3
          \textbf{Affirmation 3 :}\quad Une personne peut parcourir toutes les routes du village sans emprunter plusieurs fois la même route : \textbf{EXACT}.
          \par
          La question revient à déterminer s'il existe une chaîne qui contient une fois et une seule chacune des arêtes du graphe c'est à dire une \textbf{chaîne eulérienne}.
          \par
          Or, d'après le théorème d'Euler, un graphe connexe contient une chaîne eulérienne si et seulement s'il possède \textbf{0 ou 2 sommets de degré impair}.
          \par
          Le tableau ci-après recense le degré de chacun des sommets :
          \par
          \begin{center}
               \begin{tabular}{|l|c|c|c|c|c|c|c|c|}%class="compact"
                    \hline
                    Sommet & A & B & C & D & E & F  \\
                    \hline
                    Degré & 2 & 5 & 2 & 4 & 4 & 3 \\
                    \hline
               \end{tabular}
          \end{center}
          \par
          Le graphe (G) possède deux sommets de degré impair : B et F. Il existe donc au moins un trajet qui emprunte une fois et une seule chacune des routes du graphe.\\
          Ces trajets ont nécessairement comme extrémités B et F ; par exemple : B-C-D-E-A-B-E-F-D-B-F.
          \item %4
          \textbf{Affirmation 4 :}\quad Il y a exactement 5 trajets de trois routes reliant les carrefours C et E: \textbf{EXACT}.
          \par
          La matrice de transition du graphe (G) obtenue en classant les sommets par ordre alphabétique est :
          \[ M = \begin{pmatrix}
               0 &1 &0 &0 &1 &0 \\
               1 &0 &1 &1 &1 &1 \\
               0 &1 &0 &1 &0 &0 \\
               0 &1 &1 &0 &1 &1 \\
               1 &1 &0 &1 &0 &1\\
          0 &1 &0 &1 &1 &0  \end{pmatrix}
          \]
          \par
          Pour obtenir le nombre de chemins de trois routes reliant deux sommets on calcule $M^3$.
          \par
          \`A la calculatrice, on trouve :
          \par
          \[ M^3 = \begin{pmatrix}
               2 &8 &3 &5 &7 &4 \\
               8 &10 &8 &11 &11 &11 \\
               3 &8 &2 &7 &5 &4 \\
               5 &11 &7 &8 &11 &9 \\
               7 &11 &5 &11 &8 &9\\
          4 &11 &4 &9 &9 &6  \end{pmatrix}
          \]
          \par
          Le nombre de chemins de trois routes, reliant C à E, est le coefficient de $M^3$ situé sur la troisième ligne (correspondant au sommet de départ C) et la cinquième colonne (correspondant au sommet d'arrivé E).
          \par
          Il y a donc bien \textbf{5 trajets} de trois routes reliant les carrefours C et E.
          \par
          \cadre{vert}{En pratique}{
               Soit $M$ la matrice de transition d'un graphe G. Pour déterminer \textbf{le nombre de chemins de longueur $\bm{n}$} reliant deux sommets du graphe on calcule $\bm{M^n}$.
               \par
               Le coefficient de la matrice $M^n$ situé à la $i$-ème ligne et à la $j$-ème colonne indique le nombre de chemins de longueur $n$ menant du sommet numéro $i$ au sommet numéro $j$.
          }
          \item %5
          \textbf{Affirmation 5 :}\quad Le plus court chemin menant de A à D mesure 700 mètres : \textbf{FAUX}.
          \par
          On utilise l'algorithme de Dijkstra en partant de A :
          \par
          \begin{center}
               \begin{extern}%width="600" alt="algorithme de Dijkstra"
                    \begin{tabularx}{0.9\linewidth}{|c|C|C|C|C|C|C|}
                         \hline
                         Sommets			&   A 						& B 							& C							& D 							& E								& F   						\\ \hline
                         Départ			&  $\color{red}0_{\text{A}}$ 	& $\infty$					& $\infty$					& $\infty$					& $\infty$						& $\infty$	  				\\ \hline
                         A (0) 			&  \cellcolor{black!20}		& $3_{\text{A}}$	 			& $\infty$ 					& $\infty$ 					& $\color{red}2_{\text{A}}$		& $\infty$ 					\\ \hline
                         E (2)			&  \cellcolor{black!20}	 	& $\color{red}3_{\text{A}}$	& $\infty$ 					& $7_{\text{E}}$   			& \cellcolor{black!20} 			& $7_{\text{E}}$				\\ \hline
                         B (3)			&  \cellcolor{black!20}		& \cellcolor{black!20}	 	& $5_{\text{B}}$ 			& $7_{\text{E}}$ 			& \cellcolor{black!20}			& $\color{red}4_{\text{B}}$ 	\\ \hline
                         F (4)			&  \cellcolor{black!20}		& \cellcolor{black!20}		& $\color{red}5_{\text{B}}$ 	& $6_{\text{F}}$  			& \cellcolor{black!20}			& \cellcolor{black!20}  		\\ \hline
                         C (5)			&  \cellcolor{black!20} 		& \cellcolor{black!20} 		& \cellcolor{black!20} 		& $\color{red}6_{\text{F}}$ 	& \cellcolor{black!20} 			& \cellcolor{black!20} 		\\ \hline
                    \end{tabularx}
               \end{extern}
          \end{center}
          Le trajet le plus court menant de A à D mesure 6 centaines de mètres soit 600 mètres.
          \par
          Il s'agit du trajet A-B-F-D.
          \par
          \textit{Pour plus de détails sur la méthode employée dans cette question se reporter à la fiche  consacrée à \mcLien{/methode/algorithme-de-dijkstra-etape-par-etape/}{ l'algorithme de Dijkstra}.}
          \par
     \end{itemize}
\end{corrige}

\end{document}
µ
\documentclass[a4paper]{article}

%================================================================================================================================
%
% Packages
%
%================================================================================================================================

\usepackage[T1]{fontenc} 	% pour caractères accentués
\usepackage[utf8]{inputenc}  % encodage utf8
\usepackage[french]{babel}	% langue : français
\usepackage{fourier}			% caractères plus lisibles
\usepackage[dvipsnames]{xcolor} % couleurs
\usepackage{fancyhdr}		% réglage header footer
\usepackage{needspace}		% empêcher sauts de page mal placés
\usepackage{graphicx}		% pour inclure des graphiques
\usepackage{enumitem,cprotect}		% personnalise les listes d'items (nécessaire pour ol, al ...)
\usepackage{hyperref}		% Liens hypertexte
\usepackage{pstricks,pst-all,pst-node,pstricks-add,pst-math,pst-plot,pst-tree,pst-eucl} % pstricks
\usepackage[a4paper,includeheadfoot,top=2cm,left=3cm, bottom=2cm,right=3cm]{geometry} % marges etc.
\usepackage{comment}			% commentaires multilignes
\usepackage{amsmath,environ} % maths (matrices, etc.)
\usepackage{amssymb,makeidx}
\usepackage{bm}				% bold maths
\usepackage{tabularx}		% tableaux
\usepackage{colortbl}		% tableaux en couleur
\usepackage{fontawesome}		% Fontawesome
\usepackage{environ}			% environment with command
\usepackage{fp}				% calculs pour ps-tricks
\usepackage{multido}			% pour ps tricks
\usepackage[np]{numprint}	% formattage nombre
\usepackage{tikz,tkz-tab} 			% package principal TikZ
\usepackage{pgfplots}   % axes
\usepackage{mathrsfs}    % cursives
\usepackage{calc}			% calcul taille boites
\usepackage[scaled=0.875]{helvet} % font sans serif
\usepackage{svg} % svg
\usepackage{scrextend} % local margin
\usepackage{scratch} %scratch
\usepackage{multicol} % colonnes
%\usepackage{infix-RPN,pst-func} % formule en notation polanaise inversée
\usepackage{listings}

%================================================================================================================================
%
% Réglages de base
%
%================================================================================================================================

\lstset{
language=Python,   % R code
literate=
{á}{{\'a}}1
{à}{{\`a}}1
{ã}{{\~a}}1
{é}{{\'e}}1
{è}{{\`e}}1
{ê}{{\^e}}1
{í}{{\'i}}1
{ó}{{\'o}}1
{õ}{{\~o}}1
{ú}{{\'u}}1
{ü}{{\"u}}1
{ç}{{\c{c}}}1
{~}{{ }}1
}


\definecolor{codegreen}{rgb}{0,0.6,0}
\definecolor{codegray}{rgb}{0.5,0.5,0.5}
\definecolor{codepurple}{rgb}{0.58,0,0.82}
\definecolor{backcolour}{rgb}{0.95,0.95,0.92}

\lstdefinestyle{mystyle}{
    backgroundcolor=\color{backcolour},   
    commentstyle=\color{codegreen},
    keywordstyle=\color{magenta},
    numberstyle=\tiny\color{codegray},
    stringstyle=\color{codepurple},
    basicstyle=\ttfamily\footnotesize,
    breakatwhitespace=false,         
    breaklines=true,                 
    captionpos=b,                    
    keepspaces=true,                 
    numbers=left,                    
xleftmargin=2em,
framexleftmargin=2em,            
    showspaces=false,                
    showstringspaces=false,
    showtabs=false,                  
    tabsize=2,
    upquote=true
}

\lstset{style=mystyle}


\lstset{style=mystyle}
\newcommand{\imgdir}{C:/laragon/www/newmc/assets/imgsvg/}
\newcommand{\imgsvgdir}{C:/laragon/www/newmc/assets/imgsvg/}

\definecolor{mcgris}{RGB}{220, 220, 220}% ancien~; pour compatibilité
\definecolor{mcbleu}{RGB}{52, 152, 219}
\definecolor{mcvert}{RGB}{125, 194, 70}
\definecolor{mcmauve}{RGB}{154, 0, 215}
\definecolor{mcorange}{RGB}{255, 96, 0}
\definecolor{mcturquoise}{RGB}{0, 153, 153}
\definecolor{mcrouge}{RGB}{255, 0, 0}
\definecolor{mclightvert}{RGB}{205, 234, 190}

\definecolor{gris}{RGB}{220, 220, 220}
\definecolor{bleu}{RGB}{52, 152, 219}
\definecolor{vert}{RGB}{125, 194, 70}
\definecolor{mauve}{RGB}{154, 0, 215}
\definecolor{orange}{RGB}{255, 96, 0}
\definecolor{turquoise}{RGB}{0, 153, 153}
\definecolor{rouge}{RGB}{255, 0, 0}
\definecolor{lightvert}{RGB}{205, 234, 190}
\setitemize[0]{label=\color{lightvert}  $\bullet$}

\pagestyle{fancy}
\renewcommand{\headrulewidth}{0.2pt}
\fancyhead[L]{maths-cours.fr}
\fancyhead[R]{\thepage}
\renewcommand{\footrulewidth}{0.2pt}
\fancyfoot[C]{}

\newcolumntype{C}{>{\centering\arraybackslash}X}
\newcolumntype{s}{>{\hsize=.35\hsize\arraybackslash}X}

\setlength{\parindent}{0pt}		 
\setlength{\parskip}{3mm}
\setlength{\headheight}{1cm}

\def\ebook{ebook}
\def\book{book}
\def\web{web}
\def\type{web}

\newcommand{\vect}[1]{\overrightarrow{\,\mathstrut#1\,}}

\def\Oij{$\left(\text{O}~;~\vect{\imath},~\vect{\jmath}\right)$}
\def\Oijk{$\left(\text{O}~;~\vect{\imath},~\vect{\jmath},~\vect{k}\right)$}
\def\Ouv{$\left(\text{O}~;~\vect{u},~\vect{v}\right)$}

\hypersetup{breaklinks=true, colorlinks = true, linkcolor = OliveGreen, urlcolor = OliveGreen, citecolor = OliveGreen, pdfauthor={Didier BONNEL - https://www.maths-cours.fr} } % supprime les bordures autour des liens

\renewcommand{\arg}[0]{\text{arg}}

\everymath{\displaystyle}

%================================================================================================================================
%
% Macros - Commandes
%
%================================================================================================================================

\newcommand\meta[2]{    			% Utilisé pour créer le post HTML.
	\def\titre{titre}
	\def\url{url}
	\def\arg{#1}
	\ifx\titre\arg
		\newcommand\maintitle{#2}
		\fancyhead[L]{#2}
		{\Large\sffamily \MakeUppercase{#2}}
		\vspace{1mm}\textcolor{mcvert}{\hrule}
	\fi 
	\ifx\url\arg
		\fancyfoot[L]{\href{https://www.maths-cours.fr#2}{\black \footnotesize{https://www.maths-cours.fr#2}}}
	\fi 
}


\newcommand\TitreC[1]{    		% Titre centré
     \needspace{3\baselineskip}
     \begin{center}\textbf{#1}\end{center}
}

\newcommand\newpar{    		% paragraphe
     \par
}

\newcommand\nosp {    		% commande vide (pas d'espace)
}
\newcommand{\id}[1]{} %ignore

\newcommand\boite[2]{				% Boite simple sans titre
	\vspace{5mm}
	\setlength{\fboxrule}{0.2mm}
	\setlength{\fboxsep}{5mm}	
	\fcolorbox{#1}{#1!3}{\makebox[\linewidth-2\fboxrule-2\fboxsep]{
  		\begin{minipage}[t]{\linewidth-2\fboxrule-4\fboxsep}\setlength{\parskip}{3mm}
  			 #2
  		\end{minipage}
	}}
	\vspace{5mm}
}

\newcommand\CBox[4]{				% Boites
	\vspace{5mm}
	\setlength{\fboxrule}{0.2mm}
	\setlength{\fboxsep}{5mm}
	
	\fcolorbox{#1}{#1!3}{\makebox[\linewidth-2\fboxrule-2\fboxsep]{
		\begin{minipage}[t]{1cm}\setlength{\parskip}{3mm}
	  		\textcolor{#1}{\LARGE{#2}}    
 	 	\end{minipage}  
  		\begin{minipage}[t]{\linewidth-2\fboxrule-4\fboxsep}\setlength{\parskip}{3mm}
			\raisebox{1.2mm}{\normalsize\sffamily{\textcolor{#1}{#3}}}						
  			 #4
  		\end{minipage}
	}}
	\vspace{5mm}
}

\newcommand\cadre[3]{				% Boites convertible html
	\par
	\vspace{2mm}
	\setlength{\fboxrule}{0.1mm}
	\setlength{\fboxsep}{5mm}
	\fcolorbox{#1}{white}{\makebox[\linewidth-2\fboxrule-2\fboxsep]{
  		\begin{minipage}[t]{\linewidth-2\fboxrule-4\fboxsep}\setlength{\parskip}{3mm}
			\raisebox{-2.5mm}{\sffamily \small{\textcolor{#1}{\MakeUppercase{#2}}}}		
			\par		
  			 #3
 	 		\end{minipage}
	}}
		\vspace{2mm}
	\par
}

\newcommand\bloc[3]{				% Boites convertible html sans bordure
     \needspace{2\baselineskip}
     {\sffamily \small{\textcolor{#1}{\MakeUppercase{#2}}}}    
		\par		
  			 #3
		\par
}

\newcommand\CHelp[1]{
     \CBox{Plum}{\faInfoCircle}{À RETENIR}{#1}
}

\newcommand\CUp[1]{
     \CBox{NavyBlue}{\faThumbsOUp}{EN PRATIQUE}{#1}
}

\newcommand\CInfo[1]{
     \CBox{Sepia}{\faArrowCircleRight}{REMARQUE}{#1}
}

\newcommand\CRedac[1]{
     \CBox{PineGreen}{\faEdit}{BIEN R\'EDIGER}{#1}
}

\newcommand\CError[1]{
     \CBox{Red}{\faExclamationTriangle}{ATTENTION}{#1}
}

\newcommand\TitreExo[2]{
\needspace{4\baselineskip}
 {\sffamily\large EXERCICE #1\ (\emph{#2 points})}
\vspace{5mm}
}

\newcommand\img[2]{
          \includegraphics[width=#2\paperwidth]{\imgdir#1}
}

\newcommand\imgsvg[2]{
       \begin{center}   \includegraphics[width=#2\paperwidth]{\imgsvgdir#1} \end{center}
}


\newcommand\Lien[2]{
     \href{#1}{#2 \tiny \faExternalLink}
}
\newcommand\mcLien[2]{
     \href{https~://www.maths-cours.fr/#1}{#2 \tiny \faExternalLink}
}

\newcommand{\euro}{\eurologo{}}

%================================================================================================================================
%
% Macros - Environement
%
%================================================================================================================================

\newenvironment{tex}{ %
}
{%
}

\newenvironment{indente}{ %
	\setlength\parindent{10mm}
}

{
	\setlength\parindent{0mm}
}

\newenvironment{corrige}{%
     \needspace{3\baselineskip}
     \medskip
     \textbf{\textsc{Corrigé}}
     \medskip
}
{
}

\newenvironment{extern}{%
     \begin{center}
     }
     {
     \end{center}
}

\NewEnviron{code}{%
	\par
     \boite{gray}{\texttt{%
     \BODY
     }}
     \par
}

\newenvironment{vbloc}{% boite sans cadre empeche saut de page
     \begin{minipage}[t]{\linewidth}
     }
     {
     \end{minipage}
}
\NewEnviron{h2}{%
    \needspace{3\baselineskip}
    \vspace{0.6cm}
	\noindent \MakeUppercase{\sffamily \large \BODY}
	\vspace{1mm}\textcolor{mcgris}{\hrule}\vspace{0.4cm}
	\par
}{}

\NewEnviron{h3}{%
    \needspace{3\baselineskip}
	\vspace{5mm}
	\textsc{\BODY}
	\par
}

\NewEnviron{margeneg}{ %
\begin{addmargin}[-1cm]{0cm}
\BODY
\end{addmargin}
}

\NewEnviron{html}{%
}

\begin{document}
\meta{url}{/exercices/qcm-bac-blanc-es-l-sujet-4-maths-cours-2018/}
\meta{pid}{10499}
\meta{titre}{QCM - Bac blanc ES/L Sujet 4 - Maths-cours 2018}
\meta{type}{exercices}
%
\begin{h2}Exercice 1 (4 points)\end{h2}
\par
\emph{Cet exercice est un questionnaire à choix multiples (QCM). Les questions sont indépendantes les unes des autres. Pour chacune des questions suivantes, une seule des trois réponses proposées est exacte.  \\Indiquer sur la copie le numéro de la question et la réponse exacte \textbf{en justifiant le choix effectué}. }
\par
\emph{\textbf{Toute réponse non justifiée ne sera pas prise en compte.}}
\par
\begin{itemize}
     \item \textbf{Question 1 :}
     \par
     Soit la fonction $f$ définie sur $\mathbb{R}$ par :
     \[ f(x)=\dfrac{\text{e}^{2x}+1}{\text{e}^{-2x}+1}. \]
     Alors, pour tout réel $x$ :
     \par
     \textbf{a.~~} $f(x)=\text{e}^{2x}$
     \par
     \textbf{b.~~}  $f(x)=\dfrac{1}{\text{e}^{x}}$
     \par
     \textbf{c.~~}  $f(x)=\dfrac{1}{\text{e}^{2x}}$
     \par
     \vspace{5mm}
     \item \textbf{Question 2 :}
     \par
     La courbe $\mathscr{C}_f$ ci-après est la représentation graphique d'une fonction $f$ dans un repère orthonormé.
     \begin{center}
          \begin{extern}%width="400" alt="Courbe représentative Cf"
               \includegraphics[width=0.7\textwidth]{images/BBESL-s4-1-1}% gbb 1 unite=4cm
          \end{extern}
     \end{center}
     On pose $I= \displaystyle\int_{0}^{2}f(x)\text{d}x$.
     \par
     Alors :
     \par
     \textbf{a.~~} $0 < I < 2$ \\
     \textbf{b.~~} $2 < I < 4$ \\
     \textbf{c.~~} $8 < I < 16$ \\
     \item \textbf{Question 3 :}
     \par
     Soit $g$ la fonction définie sur $\mathbb{R}$ par :
     \par
     \[ g(x)=x \text{e}^{x}+1. \]
     \par
     Alors, pour tout réel $x$ :
     \par
     \textbf{a.~~} $g'(x)=\text{e}^{x}+1$ \\
     \textbf{b.~~} $g'(x)=(1+x)\text{e}^{x}$  \\
     \textbf{c.~~} $g'(x)=\text{e}^{x}$ \\
     \item \textbf{Question 4 :}
     \par
     $f$ est la fonction définie sur l'intervalle $[0~;~5]$ par :
     \[ f(x)=x^3+6x+1. \]
     \par
     et $\mathscr{C}_f$ sa courbe représentative dans un repère orthogonal du plan.
     \par
     Alors, sur l'intervalle $[0~;~5]$  :
     \par
     \textbf{a.~~} La fonction $f$ est concave \\
     \textbf{b.~~} La fonction $f$ est convexe \\
     \textbf{c.~~} La courbe $\mathscr{C}_f$ possède un point d'inflexion \\
     \par
\end{itemize}
\begin{corrige}
     \begin{itemize}
          % =============================================================================================================================
          \item \textbf{Question 1 :}
          \par
          Réponse correcte :\quad\textbf{ a.}
          \par
          On utilise le fait que pour tout réel $x$ : ${\text{e}^{-2x}=\dfrac{1}{\text{e}^{2x}}}$.
          \par
          Alors :
          \par
          $\text{e}^{-2x}+1=\dfrac{1}{\text{e}^{2x}}+1=\dfrac{1}{\text{e}^{2x}}+\dfrac{\text{e}^{2x}}{\text{e}^{2x}}=\dfrac{1+\text{e}^{2x}}{\text{e}^{2x}}=\dfrac{\text{e}^{2x}+1}{\text{e}^{2x}}$.
          \par
          \vspace{2mm}
          Par conséquent :
          \par
          $f(x)=\dfrac{\text{e}^{2x}+1}{\dfrac{\text{e}^{2x}+1}{\text{e}^{2x}}}=(\text{e}^{2x}+1) \times {\dfrac{\text{e}^{2x}}{\text{e}^{2x}+1}}=\text{e}^{2x}$,
          \par
          après simplification par ${\text{e}^{2x}+1}$.
          \par
          \cadre{rouge}{À retenir}{
               Pour tout réel $u$ :
               \[ \text{e}^{-u}=\dfrac{1}{\text{e}^{u}}. \]
          }
          \par
          % =============================================================================================================================
          \item \textbf{Question 2 :}
          \par
          Réponse correcte :\quad\textbf{ b.}
          \par
          La fonction $f$ est positive sur l'intervalle $[0~;~2]$.
          \par
          L'intégrale $I$ est donc égale à l'aire, exprimée en unités d'aire, du domaine délimité par la courbe $\mathscr{C}_f$, l'axe des abscisses, l'axe des ordonnées et la droite d'équation $x=2$.
          \par
          Ce domaine est coloré en vert sur le graphique ci-après ; l'unité d'aire est l'aire du carré tracé en rouge.
          \par
          \begin{center}
               \begin{extern}%width="400" alt="Aire et intégrale"
                    \includegraphics[width=0.7\textwidth]{images/BBESL-s4-1-2}% gbb 1 unite=4cm
               \end{extern}
          \end{center}
          \par
          On voit facilement que l'aire $I$ est comprise entre 2 et 4 unités d'aire.
          \par
          \cadre{rouge}{À retenir}{
               Soit $f$ une fonction définie, continue et \textbf{positive} sur l'intervalle $[a~;~b]$ de courbe représentative $\mathscr{C}_f$.
               \par
               L'intégrale $\displaystyle\int_{a}^{b}f(x)\text{d}x$ est égale à l'aire, en unité d'aire, du domaine délimité par la courbe $\mathscr{C}_f$, l'axe des abscisses, et les droites d'équation $x=a$ et $x=b$.
          }
          \par
          % =============================================================================================================================
          \item \textbf{Question 3 :}
          \par
          Réponse correcte :\quad\textbf{ b.}
          \par
          La dérivée de la fonction constante $x \longmapsto 1$ est nulle.
          \par
          Pour dériver la fonction $x \longmapsto x \text{e}^{x}$ on pose :
          \[ u(x)=x  \qquad \text{et} \qquad v(x)=\text{e}^{x}. \]
          Alors :
          \[ u'(x)=1 \qquad \text{et} \qquad  v'(x)=\text{e}^{x}. \]
          Donc :
          \par
          $g'(x)=u'(x)v(x)+u(x)v'(x)=\text{e}^{x}+x \text{e}^{x}=(1+x)\text{e}^{x}$,
          \par
          après factorisation de $\text{e}^{x}$.
          \par
          %=============================================================================================================================
          \item \textbf{Question 4 :}
          \par
          Réponse correcte :\quad\textbf{ b.}
          \par
          $f$ est une fonction polynôme sur l'intervalle $[0~;~5]$.
          \par
          $f$ et $f'$ sont donc dérivables sur l'intervalle $[0~;~5]$ et :
          \par
          $f'(x)=3x^2+6$ ;
          \par
          $f''(x)=6x+1$.
          \par
          $f''$ est une fonction affine qui est strictement positive sur l'intervalle $[0~;~5]$ (en effet, $6x+1$ est strictement positif dès lors que $x>-\dfrac{1}{6}$).
          \par
          La fonction $f$ est donc \textbf{convexe} sur l'intervalle $[0~;~5]$.
          \cadre{vert}{En pratique}{
               Pour montrer qu'une fonction $f$, deux fois dérivable sur un intervalle $I$ est \textbf{convexe}, on montre que sa dérivée seconde est \textbf{positive} sur $I$.
          }
          \par
     \end{itemize}
\end{corrige}

\end{document}
µ
\documentclass[a4paper]{article}

%================================================================================================================================
%
% Packages
%
%================================================================================================================================

\usepackage[T1]{fontenc} 	% pour caractères accentués
\usepackage[utf8]{inputenc}  % encodage utf8
\usepackage[french]{babel}	% langue : français
\usepackage{fourier}			% caractères plus lisibles
\usepackage[dvipsnames]{xcolor} % couleurs
\usepackage{fancyhdr}		% réglage header footer
\usepackage{needspace}		% empêcher sauts de page mal placés
\usepackage{graphicx}		% pour inclure des graphiques
\usepackage{enumitem,cprotect}		% personnalise les listes d'items (nécessaire pour ol, al ...)
\usepackage{hyperref}		% Liens hypertexte
\usepackage{pstricks,pst-all,pst-node,pstricks-add,pst-math,pst-plot,pst-tree,pst-eucl} % pstricks
\usepackage[a4paper,includeheadfoot,top=2cm,left=3cm, bottom=2cm,right=3cm]{geometry} % marges etc.
\usepackage{comment}			% commentaires multilignes
\usepackage{amsmath,environ} % maths (matrices, etc.)
\usepackage{amssymb,makeidx}
\usepackage{bm}				% bold maths
\usepackage{tabularx}		% tableaux
\usepackage{colortbl}		% tableaux en couleur
\usepackage{fontawesome}		% Fontawesome
\usepackage{environ}			% environment with command
\usepackage{fp}				% calculs pour ps-tricks
\usepackage{multido}			% pour ps tricks
\usepackage[np]{numprint}	% formattage nombre
\usepackage{tikz,tkz-tab} 			% package principal TikZ
\usepackage{pgfplots}   % axes
\usepackage{mathrsfs}    % cursives
\usepackage{calc}			% calcul taille boites
\usepackage[scaled=0.875]{helvet} % font sans serif
\usepackage{svg} % svg
\usepackage{scrextend} % local margin
\usepackage{scratch} %scratch
\usepackage{multicol} % colonnes
%\usepackage{infix-RPN,pst-func} % formule en notation polanaise inversée
\usepackage{listings}

%================================================================================================================================
%
% Réglages de base
%
%================================================================================================================================

\lstset{
language=Python,   % R code
literate=
{á}{{\'a}}1
{à}{{\`a}}1
{ã}{{\~a}}1
{é}{{\'e}}1
{è}{{\`e}}1
{ê}{{\^e}}1
{í}{{\'i}}1
{ó}{{\'o}}1
{õ}{{\~o}}1
{ú}{{\'u}}1
{ü}{{\"u}}1
{ç}{{\c{c}}}1
{~}{{ }}1
}


\definecolor{codegreen}{rgb}{0,0.6,0}
\definecolor{codegray}{rgb}{0.5,0.5,0.5}
\definecolor{codepurple}{rgb}{0.58,0,0.82}
\definecolor{backcolour}{rgb}{0.95,0.95,0.92}

\lstdefinestyle{mystyle}{
    backgroundcolor=\color{backcolour},   
    commentstyle=\color{codegreen},
    keywordstyle=\color{magenta},
    numberstyle=\tiny\color{codegray},
    stringstyle=\color{codepurple},
    basicstyle=\ttfamily\footnotesize,
    breakatwhitespace=false,         
    breaklines=true,                 
    captionpos=b,                    
    keepspaces=true,                 
    numbers=left,                    
xleftmargin=2em,
framexleftmargin=2em,            
    showspaces=false,                
    showstringspaces=false,
    showtabs=false,                  
    tabsize=2,
    upquote=true
}

\lstset{style=mystyle}


\lstset{style=mystyle}
\newcommand{\imgdir}{C:/laragon/www/newmc/assets/imgsvg/}
\newcommand{\imgsvgdir}{C:/laragon/www/newmc/assets/imgsvg/}

\definecolor{mcgris}{RGB}{220, 220, 220}% ancien~; pour compatibilité
\definecolor{mcbleu}{RGB}{52, 152, 219}
\definecolor{mcvert}{RGB}{125, 194, 70}
\definecolor{mcmauve}{RGB}{154, 0, 215}
\definecolor{mcorange}{RGB}{255, 96, 0}
\definecolor{mcturquoise}{RGB}{0, 153, 153}
\definecolor{mcrouge}{RGB}{255, 0, 0}
\definecolor{mclightvert}{RGB}{205, 234, 190}

\definecolor{gris}{RGB}{220, 220, 220}
\definecolor{bleu}{RGB}{52, 152, 219}
\definecolor{vert}{RGB}{125, 194, 70}
\definecolor{mauve}{RGB}{154, 0, 215}
\definecolor{orange}{RGB}{255, 96, 0}
\definecolor{turquoise}{RGB}{0, 153, 153}
\definecolor{rouge}{RGB}{255, 0, 0}
\definecolor{lightvert}{RGB}{205, 234, 190}
\setitemize[0]{label=\color{lightvert}  $\bullet$}

\pagestyle{fancy}
\renewcommand{\headrulewidth}{0.2pt}
\fancyhead[L]{maths-cours.fr}
\fancyhead[R]{\thepage}
\renewcommand{\footrulewidth}{0.2pt}
\fancyfoot[C]{}

\newcolumntype{C}{>{\centering\arraybackslash}X}
\newcolumntype{s}{>{\hsize=.35\hsize\arraybackslash}X}

\setlength{\parindent}{0pt}		 
\setlength{\parskip}{3mm}
\setlength{\headheight}{1cm}

\def\ebook{ebook}
\def\book{book}
\def\web{web}
\def\type{web}

\newcommand{\vect}[1]{\overrightarrow{\,\mathstrut#1\,}}

\def\Oij{$\left(\text{O}~;~\vect{\imath},~\vect{\jmath}\right)$}
\def\Oijk{$\left(\text{O}~;~\vect{\imath},~\vect{\jmath},~\vect{k}\right)$}
\def\Ouv{$\left(\text{O}~;~\vect{u},~\vect{v}\right)$}

\hypersetup{breaklinks=true, colorlinks = true, linkcolor = OliveGreen, urlcolor = OliveGreen, citecolor = OliveGreen, pdfauthor={Didier BONNEL - https://www.maths-cours.fr} } % supprime les bordures autour des liens

\renewcommand{\arg}[0]{\text{arg}}

\everymath{\displaystyle}

%================================================================================================================================
%
% Macros - Commandes
%
%================================================================================================================================

\newcommand\meta[2]{    			% Utilisé pour créer le post HTML.
	\def\titre{titre}
	\def\url{url}
	\def\arg{#1}
	\ifx\titre\arg
		\newcommand\maintitle{#2}
		\fancyhead[L]{#2}
		{\Large\sffamily \MakeUppercase{#2}}
		\vspace{1mm}\textcolor{mcvert}{\hrule}
	\fi 
	\ifx\url\arg
		\fancyfoot[L]{\href{https://www.maths-cours.fr#2}{\black \footnotesize{https://www.maths-cours.fr#2}}}
	\fi 
}


\newcommand\TitreC[1]{    		% Titre centré
     \needspace{3\baselineskip}
     \begin{center}\textbf{#1}\end{center}
}

\newcommand\newpar{    		% paragraphe
     \par
}

\newcommand\nosp {    		% commande vide (pas d'espace)
}
\newcommand{\id}[1]{} %ignore

\newcommand\boite[2]{				% Boite simple sans titre
	\vspace{5mm}
	\setlength{\fboxrule}{0.2mm}
	\setlength{\fboxsep}{5mm}	
	\fcolorbox{#1}{#1!3}{\makebox[\linewidth-2\fboxrule-2\fboxsep]{
  		\begin{minipage}[t]{\linewidth-2\fboxrule-4\fboxsep}\setlength{\parskip}{3mm}
  			 #2
  		\end{minipage}
	}}
	\vspace{5mm}
}

\newcommand\CBox[4]{				% Boites
	\vspace{5mm}
	\setlength{\fboxrule}{0.2mm}
	\setlength{\fboxsep}{5mm}
	
	\fcolorbox{#1}{#1!3}{\makebox[\linewidth-2\fboxrule-2\fboxsep]{
		\begin{minipage}[t]{1cm}\setlength{\parskip}{3mm}
	  		\textcolor{#1}{\LARGE{#2}}    
 	 	\end{minipage}  
  		\begin{minipage}[t]{\linewidth-2\fboxrule-4\fboxsep}\setlength{\parskip}{3mm}
			\raisebox{1.2mm}{\normalsize\sffamily{\textcolor{#1}{#3}}}						
  			 #4
  		\end{minipage}
	}}
	\vspace{5mm}
}

\newcommand\cadre[3]{				% Boites convertible html
	\par
	\vspace{2mm}
	\setlength{\fboxrule}{0.1mm}
	\setlength{\fboxsep}{5mm}
	\fcolorbox{#1}{white}{\makebox[\linewidth-2\fboxrule-2\fboxsep]{
  		\begin{minipage}[t]{\linewidth-2\fboxrule-4\fboxsep}\setlength{\parskip}{3mm}
			\raisebox{-2.5mm}{\sffamily \small{\textcolor{#1}{\MakeUppercase{#2}}}}		
			\par		
  			 #3
 	 		\end{minipage}
	}}
		\vspace{2mm}
	\par
}

\newcommand\bloc[3]{				% Boites convertible html sans bordure
     \needspace{2\baselineskip}
     {\sffamily \small{\textcolor{#1}{\MakeUppercase{#2}}}}    
		\par		
  			 #3
		\par
}

\newcommand\CHelp[1]{
     \CBox{Plum}{\faInfoCircle}{À RETENIR}{#1}
}

\newcommand\CUp[1]{
     \CBox{NavyBlue}{\faThumbsOUp}{EN PRATIQUE}{#1}
}

\newcommand\CInfo[1]{
     \CBox{Sepia}{\faArrowCircleRight}{REMARQUE}{#1}
}

\newcommand\CRedac[1]{
     \CBox{PineGreen}{\faEdit}{BIEN R\'EDIGER}{#1}
}

\newcommand\CError[1]{
     \CBox{Red}{\faExclamationTriangle}{ATTENTION}{#1}
}

\newcommand\TitreExo[2]{
\needspace{4\baselineskip}
 {\sffamily\large EXERCICE #1\ (\emph{#2 points})}
\vspace{5mm}
}

\newcommand\img[2]{
          \includegraphics[width=#2\paperwidth]{\imgdir#1}
}

\newcommand\imgsvg[2]{
       \begin{center}   \includegraphics[width=#2\paperwidth]{\imgsvgdir#1} \end{center}
}


\newcommand\Lien[2]{
     \href{#1}{#2 \tiny \faExternalLink}
}
\newcommand\mcLien[2]{
     \href{https~://www.maths-cours.fr/#1}{#2 \tiny \faExternalLink}
}

\newcommand{\euro}{\eurologo{}}

%================================================================================================================================
%
% Macros - Environement
%
%================================================================================================================================

\newenvironment{tex}{ %
}
{%
}

\newenvironment{indente}{ %
	\setlength\parindent{10mm}
}

{
	\setlength\parindent{0mm}
}

\newenvironment{corrige}{%
     \needspace{3\baselineskip}
     \medskip
     \textbf{\textsc{Corrigé}}
     \medskip
}
{
}

\newenvironment{extern}{%
     \begin{center}
     }
     {
     \end{center}
}

\NewEnviron{code}{%
	\par
     \boite{gray}{\texttt{%
     \BODY
     }}
     \par
}

\newenvironment{vbloc}{% boite sans cadre empeche saut de page
     \begin{minipage}[t]{\linewidth}
     }
     {
     \end{minipage}
}
\NewEnviron{h2}{%
    \needspace{3\baselineskip}
    \vspace{0.6cm}
	\noindent \MakeUppercase{\sffamily \large \BODY}
	\vspace{1mm}\textcolor{mcgris}{\hrule}\vspace{0.4cm}
	\par
}{}

\NewEnviron{h3}{%
    \needspace{3\baselineskip}
	\vspace{5mm}
	\textsc{\BODY}
	\par
}

\NewEnviron{margeneg}{ %
\begin{addmargin}[-1cm]{0cm}
\BODY
\end{addmargin}
}

\NewEnviron{html}{%
}

\begin{document}
\meta{url}{/exercices/probabilites-bac-blanc-es-l-sujet-4-maths-cours-2018/}
\meta{pid}{10501}
\meta{titre}{Probabilités - Bac blanc ES/L Sujet 4 - Maths-cours 2018}
\meta{type}{exercices}
%
\begin{h2}Exercice 2 (5 points)\end{h2}
\par
\textit{Les parties A et B sont indépendantes.}
\par
\textit{Les probabilités demandées seront arrondies au dix-millième.}
\par
%============================================================================================================================
%
\TitreC{Partie A}
%
%============================================================================================================================
\par
Dans un lycée parisien, on a dénombré 52\% de filles et 48\% de garçons.
\par
Une étude a révélé que, dans ce lycée, 59\% des filles et 68\% des garçons pratiquaient un sport en dehors de l'établissement.
\par
On choisit au hasard un élève dans ce lycée et on considère les événements suivants :
\par
\begin{itemize}
     \item $F$ : \og l'élève choisi est une fille \fg{} ;
     \item $G$ : \og l'élève choisi est un garçon \fg{} ;
     \item $S$ : \og l'élève choisi pratique un sport en dehors de l'établissement\fg{} ;
     \item $\overline{S}$ : l'événement contraire de $S$.
\end{itemize}
\par
\begin{enumerate}
     \item Recopier et compléter l'arbre de probabilité ci-après :
     %:-+-+-+- Engendré par : http://math.et.info.free.fr/TikZ/Arbre/
     \begin{center}
          % Racine à Gauche, développement vers la droite
          \begin{extern}%width="400" alt="Arbre de probabilité à compléter "
               \begin{tikzpicture}[xscale=1,yscale=1]
                    % Styles (MODIFIABLES)
                    \tikzstyle{fleche}=[-,>=latex,thick]
                    \tikzstyle{noeud}=[fill=white,circle,draw]
                    \tikzstyle{feuille}=[fill=white,circle,draw]
                    \tikzstyle{etiquette}=[midway,fill=white]
                    % Dimensions (MODIFIABLES)
                    \def\DistanceInterNiveaux{3}
                    \def\DistanceInterFeuilles{2}
                    % Dimensions calculées (NON MODIFIABLES)
                    \def\NiveauA{(0)*\DistanceInterNiveaux}
                    \def\NiveauB{(1.5)*\DistanceInterNiveaux}
                    \def\NiveauC{(2.5)*\DistanceInterNiveaux}
                    \def\InterFeuilles{(-1)*\DistanceInterFeuilles}
                    % Noeuds (MODIFIABLES : Styles et Coefficients d'InterFeuilles)
                    \node[noeud] (R) at ({\NiveauA},{(1.5)*\InterFeuilles}) {$\ $};
                    \node[noeud] (Ra) at ({\NiveauB},{(0.5)*\InterFeuilles}) {$F$};
                    \node[feuille] (Raa) at ({\NiveauC},{(0)*\InterFeuilles}) {$S$};
                    \node[feuille] (Rab) at ({\NiveauC},{(1)*\InterFeuilles}) {$\overline{S}$};
                    \node[noeud] (Rb) at ({\NiveauB},{(2.5)*\InterFeuilles}) {$G$};
                    \node[feuille] (Rba) at ({\NiveauC},{(2)*\InterFeuilles}) {$S$};
                    \node[feuille] (Rbb) at ({\NiveauC},{(3)*\InterFeuilles}) {$\overline{S}$};
                    % Arcs (MODIFIABLES : Styles)
                    \draw[fleche] (R)--(Ra) node[etiquette] {$\cdots$};
                    \draw[fleche] (Ra)--(Raa) node[etiquette] {$\cdots$};
                    \draw[fleche] (Ra)--(Rab) node[etiquette] {$\cdots$};
                    \draw[fleche] (R)--(Rb) node[etiquette] {$\cdots$};
                    \draw[fleche] (Rb)--(Rba) node[etiquette] {$\cdots$};
                    \draw[fleche] (Rb)--(Rbb) node[etiquette] {$\cdots$};
               \end{tikzpicture}
          \end{extern}
     \end{center}
     %:-+-+-+-+- Fin
     \item Quel est la probabilité que l'élève choisi soit un garçon pratiquant un sport en dehors du lycée ?
     \item Quel est la probabilité que l'élève choisi pratique un sport en dehors du lycée ?
     \item On sait que l'élève choisi pratique un sport en dehors de l'établissement. Quel est la probabilité que ce soit un garçon ?
\end{enumerate}
\par
%============================================================================================================================
%
\TitreC{Partie B}
%
%============================================================================================================================
\par
Luc doit se rendre, par les transports en commun,  à un cours de natation qui débute à 10h. En fonction de la circulation, il arrive entre 9h30 et 10h15.
\par
On suppose que son heure d'arrivée peut être modélisée par une variable aléatoire $T$ qui suit la loi uniforme sur l'intervalle ${[9,5~;~10,25]}$.
\par
\begin{enumerate}
     \item Quelle est la probabilité que Luc arrive à l'heure à son cours ?
     \item Quelle est la probabilité que Luc arrive avec plus d'un quart d'heure d'avance à son cours ?
     \item Quelle est l'espérance mathématique de la variable aléatoire $T$ ?
     Interpréter cette valeur dans le cadre de l'exercice.
     \par
\end{enumerate}
\begin{corrige}
     %============================================================================================================================
     %
     \TitreC{Partie A}
     %
     %============================================================================================================================
     \par
     \begin{enumerate}
          \item %1
          D'après les données de l'énoncé :
          \par
          \begin{itemize}
               \item $p(F)=0,52$ ;
               \item $p(G)=0,48$ ;
               \item $p_F(S)=0,59$ ;
               \item $p_G(S)=0,68$.
          \end{itemize}
          \par
          On obtient alors l'arbre ci-après :
          %:-+-+-+- Engendré par : http://math.et.info.free.fr/TikZ/Arbre/
          \begin{center}
               \begin{extern}%width="400" alt="Arbre de probabilité complété"
                    % Racine à Gauche, développement vers la droite
                    \begin{tikzpicture}[xscale=1,yscale=1]
                         % Styles (MODIFIABLES)
                         \tikzstyle{fleche}=[-,>=latex,thick]
                         \tikzstyle{noeud}=[fill=white,circle,draw]
                         \tikzstyle{feuille}=[fill=white,circle,draw]
                         \tikzstyle{etiquette}=[midway,fill=white]
                         % Dimensions (MODIFIABLES)
                         \def\DistanceInterNiveaux{3}
                         \def\DistanceInterFeuilles{2}
                         % Dimensions calculées (NON MODIFIABLES)
                         \def\NiveauA{(0)*\DistanceInterNiveaux}
                         \def\NiveauB{(1.5)*\DistanceInterNiveaux}
                         \def\NiveauC{(2.5)*\DistanceInterNiveaux}
                         \def\InterFeuilles{(-1)*\DistanceInterFeuilles}
                         % Noeuds (MODIFIABLES : Styles et Coefficients d'InterFeuilles)
                         \node[noeud] (R) at ({\NiveauA},{(1.5)*\InterFeuilles}) {$\ $};
                         \node[noeud] (Ra) at ({\NiveauB},{(0.5)*\InterFeuilles}) {$F$};
                         \node[feuille] (Raa) at ({\NiveauC},{(0)*\InterFeuilles}) {$S$};
                         \node[feuille] (Rab) at ({\NiveauC},{(1)*\InterFeuilles}) {$\overline{S}$};
                         \node[noeud] (Rb) at ({\NiveauB},{(2.5)*\InterFeuilles}) {$G$};
                         \node[feuille] (Rba) at ({\NiveauC},{(2)*\InterFeuilles}) {$S$};
                         \node[feuille] (Rbb) at ({\NiveauC},{(3)*\InterFeuilles}) {$\overline{S}$};
                         % Arcs (MODIFIABLES : Styles)
                         \draw[fleche] (R)--(Ra) node[etiquette] {$0,52$};
                         \draw[fleche] (Ra)--(Raa) node[etiquette] {$0,59$};
                         \draw[fleche] (Ra)--(Rab) node[etiquette] {$0,41$};
                         \draw[fleche] (R)--(Rb) node[etiquette] {$0,48$};
                         \draw[fleche] (Rb)--(Rba) node[etiquette] {$0,68$};
                         \draw[fleche] (Rb)--(Rbb) node[etiquette] {$0,32$};
                    \end{tikzpicture}
               \end{extern}
          \end{center}
          %:-+-+-+-+- Fin
          \item %2
          La probabilité demandée est $p(G \cap S)$ :
          \par
          $p(G \cap S)= p(G) \times p_S(G)=0,48 \times 0,68 = 0,3264$.
          \par
          \cadre{vert}{En pratique}{
               L'événement $G \cap S$ correspond à : \og les événements $G$ \textbf{et} $S$ sont \textbf{tous les deux} réalisés \fg{}.
               \par
               La probabilité de  $G \cap S$ peut se calculer à l'aide de la formule :
               \[ p(G \cap S)= p(G \times p_G(S). \]
               \par
               \`A partir de l'arbre pondéré, cela revient à multiplier les probabilités situées sur :
               \begin{itemize}
                    \item %
                    la branche qui aboutit à $G$,
                    \item %
                    La branche qui relie $G$ à $S$.
               \end{itemize}
          }
          \item %3
          La probabilité cherchée est $p(S)$.
          \par
          D'après la formule des probabilités totales :
          \par
          $p(S)=p(F\cap S) + p(G\cap S)$\\
          $\phantom{p(S)}=p(F) \times p_F(S) + p(G) \times p_{G}(S)$\\
          $\phantom{p(S)} = 0,52 \times 0,59 +0,48 \times 0,68=0,6332$.
          \item %4
          La probabilité demandée est $p_S(G)$.
          \par
          D'après la formule des probabilités conditionnelles :
          \par
          $p_S(G)=\dfrac{p(G\cap S)}{p(S)}=\dfrac{0,3264}{0,6332} \approx 0,5155\ $ (à $10^{-4}$ près).
          \par
     \end{enumerate}
     \par
     %============================================================================================================================
     %
     \TitreC{Partie B}
     %
     %============================================================================================================================
     \par
     \begin{enumerate}
          \item %1
          Luc est à l'heure à son cours s'il arrive entre 9h30 et 10h, c'est à dire si $9,5 \leqslant T \leqslant 10$.
          \par
          $T$ suivant la loi uniforme sur l'intervalle $[9,5~;~10,25]$ :
          \par
          $p(9,5 \leqslant T \leqslant 10)=\dfrac{10-9,5}{10,25-9,5}=\dfrac{0,5}{0,75}=\dfrac{2}{3} \approx 0,6667\ $ (à $10^{-4}$ près).
          \par
          \cadre{rouge}{À retenir}{
               Si $X$ suit la \textbf{loi uniforme} sur l'intervalle $[a~;~b]$, alors pour tous réels $c$ et $d$ de l'intervalle $[a~;~b]$ avec $c \leqslant d$ :
               \[ p(c \leqslant X \leqslant d) = \dfrac{d-c}{b-a}. \]
          }
          \item %2
          Luc arrive à son cours avec plus d'un quart d'heure d'avance s'il arrive entre 9h30 et 9h45, c'est à dire si ${9+\dfrac{1}{2} \leqslant T \leqslant 9+\dfrac{3}{4}}$ ou encore ${9,5 \leqslant T \leqslant 9,75}$.
          \par
          La probabilité de cet événement est :
          \par
          $p(9,5 \leqslant T \leqslant 9,75)=\dfrac{9,75-9,5}{10,25-9,5}=\dfrac{0,25}{0,75}=\dfrac{1}{3} \approx 0,3333\ $ (à $10^{-4}$ près).
          \item %3
          Comme $T$ suit la loi uniforme sur l'intervalle $[9,5~;~10,25]$ :
          \par
          $E(T)=\dfrac{9,5+10,25}{2}=\dfrac{19,75}{2}=9,875$.
          \par
          L'espérance mathématique de $T$ représente l'heure d'arrivée \textbf{moyenne} de Luc.
          \par
          $0,875$ heure correspond à $0,875 \times 60 = 52,5$ minutes.
          \par
          En moyenne, Luc arrivera à son cours à 9h 52min 30s.
          \par
          \cadre{rouge}{À retenir}{
               L'espérance mathématique de la \textbf{loi uniforme} sur l'intervalle $[a~;~b]$ est :
               \[ E(X) = \dfrac{a+b}{2}. \]
          }
          \par
     \end{enumerate}
\end{corrige}

\end{document}
µ
\documentclass[a4paper]{article}

%================================================================================================================================
%
% Packages
%
%================================================================================================================================

\usepackage[T1]{fontenc} 	% pour caractères accentués
\usepackage[utf8]{inputenc}  % encodage utf8
\usepackage[french]{babel}	% langue : français
\usepackage{fourier}			% caractères plus lisibles
\usepackage[dvipsnames]{xcolor} % couleurs
\usepackage{fancyhdr}		% réglage header footer
\usepackage{needspace}		% empêcher sauts de page mal placés
\usepackage{graphicx}		% pour inclure des graphiques
\usepackage{enumitem,cprotect}		% personnalise les listes d'items (nécessaire pour ol, al ...)
\usepackage{hyperref}		% Liens hypertexte
\usepackage{pstricks,pst-all,pst-node,pstricks-add,pst-math,pst-plot,pst-tree,pst-eucl} % pstricks
\usepackage[a4paper,includeheadfoot,top=2cm,left=3cm, bottom=2cm,right=3cm]{geometry} % marges etc.
\usepackage{comment}			% commentaires multilignes
\usepackage{amsmath,environ} % maths (matrices, etc.)
\usepackage{amssymb,makeidx}
\usepackage{bm}				% bold maths
\usepackage{tabularx}		% tableaux
\usepackage{colortbl}		% tableaux en couleur
\usepackage{fontawesome}		% Fontawesome
\usepackage{environ}			% environment with command
\usepackage{fp}				% calculs pour ps-tricks
\usepackage{multido}			% pour ps tricks
\usepackage[np]{numprint}	% formattage nombre
\usepackage{tikz,tkz-tab} 			% package principal TikZ
\usepackage{pgfplots}   % axes
\usepackage{mathrsfs}    % cursives
\usepackage{calc}			% calcul taille boites
\usepackage[scaled=0.875]{helvet} % font sans serif
\usepackage{svg} % svg
\usepackage{scrextend} % local margin
\usepackage{scratch} %scratch
\usepackage{multicol} % colonnes
%\usepackage{infix-RPN,pst-func} % formule en notation polanaise inversée
\usepackage{listings}

%================================================================================================================================
%
% Réglages de base
%
%================================================================================================================================

\lstset{
language=Python,   % R code
literate=
{á}{{\'a}}1
{à}{{\`a}}1
{ã}{{\~a}}1
{é}{{\'e}}1
{è}{{\`e}}1
{ê}{{\^e}}1
{í}{{\'i}}1
{ó}{{\'o}}1
{õ}{{\~o}}1
{ú}{{\'u}}1
{ü}{{\"u}}1
{ç}{{\c{c}}}1
{~}{{ }}1
}


\definecolor{codegreen}{rgb}{0,0.6,0}
\definecolor{codegray}{rgb}{0.5,0.5,0.5}
\definecolor{codepurple}{rgb}{0.58,0,0.82}
\definecolor{backcolour}{rgb}{0.95,0.95,0.92}

\lstdefinestyle{mystyle}{
    backgroundcolor=\color{backcolour},   
    commentstyle=\color{codegreen},
    keywordstyle=\color{magenta},
    numberstyle=\tiny\color{codegray},
    stringstyle=\color{codepurple},
    basicstyle=\ttfamily\footnotesize,
    breakatwhitespace=false,         
    breaklines=true,                 
    captionpos=b,                    
    keepspaces=true,                 
    numbers=left,                    
xleftmargin=2em,
framexleftmargin=2em,            
    showspaces=false,                
    showstringspaces=false,
    showtabs=false,                  
    tabsize=2,
    upquote=true
}

\lstset{style=mystyle}


\lstset{style=mystyle}
\newcommand{\imgdir}{C:/laragon/www/newmc/assets/imgsvg/}
\newcommand{\imgsvgdir}{C:/laragon/www/newmc/assets/imgsvg/}

\definecolor{mcgris}{RGB}{220, 220, 220}% ancien~; pour compatibilité
\definecolor{mcbleu}{RGB}{52, 152, 219}
\definecolor{mcvert}{RGB}{125, 194, 70}
\definecolor{mcmauve}{RGB}{154, 0, 215}
\definecolor{mcorange}{RGB}{255, 96, 0}
\definecolor{mcturquoise}{RGB}{0, 153, 153}
\definecolor{mcrouge}{RGB}{255, 0, 0}
\definecolor{mclightvert}{RGB}{205, 234, 190}

\definecolor{gris}{RGB}{220, 220, 220}
\definecolor{bleu}{RGB}{52, 152, 219}
\definecolor{vert}{RGB}{125, 194, 70}
\definecolor{mauve}{RGB}{154, 0, 215}
\definecolor{orange}{RGB}{255, 96, 0}
\definecolor{turquoise}{RGB}{0, 153, 153}
\definecolor{rouge}{RGB}{255, 0, 0}
\definecolor{lightvert}{RGB}{205, 234, 190}
\setitemize[0]{label=\color{lightvert}  $\bullet$}

\pagestyle{fancy}
\renewcommand{\headrulewidth}{0.2pt}
\fancyhead[L]{maths-cours.fr}
\fancyhead[R]{\thepage}
\renewcommand{\footrulewidth}{0.2pt}
\fancyfoot[C]{}

\newcolumntype{C}{>{\centering\arraybackslash}X}
\newcolumntype{s}{>{\hsize=.35\hsize\arraybackslash}X}

\setlength{\parindent}{0pt}		 
\setlength{\parskip}{3mm}
\setlength{\headheight}{1cm}

\def\ebook{ebook}
\def\book{book}
\def\web{web}
\def\type{web}

\newcommand{\vect}[1]{\overrightarrow{\,\mathstrut#1\,}}

\def\Oij{$\left(\text{O}~;~\vect{\imath},~\vect{\jmath}\right)$}
\def\Oijk{$\left(\text{O}~;~\vect{\imath},~\vect{\jmath},~\vect{k}\right)$}
\def\Ouv{$\left(\text{O}~;~\vect{u},~\vect{v}\right)$}

\hypersetup{breaklinks=true, colorlinks = true, linkcolor = OliveGreen, urlcolor = OliveGreen, citecolor = OliveGreen, pdfauthor={Didier BONNEL - https://www.maths-cours.fr} } % supprime les bordures autour des liens

\renewcommand{\arg}[0]{\text{arg}}

\everymath{\displaystyle}

%================================================================================================================================
%
% Macros - Commandes
%
%================================================================================================================================

\newcommand\meta[2]{    			% Utilisé pour créer le post HTML.
	\def\titre{titre}
	\def\url{url}
	\def\arg{#1}
	\ifx\titre\arg
		\newcommand\maintitle{#2}
		\fancyhead[L]{#2}
		{\Large\sffamily \MakeUppercase{#2}}
		\vspace{1mm}\textcolor{mcvert}{\hrule}
	\fi 
	\ifx\url\arg
		\fancyfoot[L]{\href{https://www.maths-cours.fr#2}{\black \footnotesize{https://www.maths-cours.fr#2}}}
	\fi 
}


\newcommand\TitreC[1]{    		% Titre centré
     \needspace{3\baselineskip}
     \begin{center}\textbf{#1}\end{center}
}

\newcommand\newpar{    		% paragraphe
     \par
}

\newcommand\nosp {    		% commande vide (pas d'espace)
}
\newcommand{\id}[1]{} %ignore

\newcommand\boite[2]{				% Boite simple sans titre
	\vspace{5mm}
	\setlength{\fboxrule}{0.2mm}
	\setlength{\fboxsep}{5mm}	
	\fcolorbox{#1}{#1!3}{\makebox[\linewidth-2\fboxrule-2\fboxsep]{
  		\begin{minipage}[t]{\linewidth-2\fboxrule-4\fboxsep}\setlength{\parskip}{3mm}
  			 #2
  		\end{minipage}
	}}
	\vspace{5mm}
}

\newcommand\CBox[4]{				% Boites
	\vspace{5mm}
	\setlength{\fboxrule}{0.2mm}
	\setlength{\fboxsep}{5mm}
	
	\fcolorbox{#1}{#1!3}{\makebox[\linewidth-2\fboxrule-2\fboxsep]{
		\begin{minipage}[t]{1cm}\setlength{\parskip}{3mm}
	  		\textcolor{#1}{\LARGE{#2}}    
 	 	\end{minipage}  
  		\begin{minipage}[t]{\linewidth-2\fboxrule-4\fboxsep}\setlength{\parskip}{3mm}
			\raisebox{1.2mm}{\normalsize\sffamily{\textcolor{#1}{#3}}}						
  			 #4
  		\end{minipage}
	}}
	\vspace{5mm}
}

\newcommand\cadre[3]{				% Boites convertible html
	\par
	\vspace{2mm}
	\setlength{\fboxrule}{0.1mm}
	\setlength{\fboxsep}{5mm}
	\fcolorbox{#1}{white}{\makebox[\linewidth-2\fboxrule-2\fboxsep]{
  		\begin{minipage}[t]{\linewidth-2\fboxrule-4\fboxsep}\setlength{\parskip}{3mm}
			\raisebox{-2.5mm}{\sffamily \small{\textcolor{#1}{\MakeUppercase{#2}}}}		
			\par		
  			 #3
 	 		\end{minipage}
	}}
		\vspace{2mm}
	\par
}

\newcommand\bloc[3]{				% Boites convertible html sans bordure
     \needspace{2\baselineskip}
     {\sffamily \small{\textcolor{#1}{\MakeUppercase{#2}}}}    
		\par		
  			 #3
		\par
}

\newcommand\CHelp[1]{
     \CBox{Plum}{\faInfoCircle}{À RETENIR}{#1}
}

\newcommand\CUp[1]{
     \CBox{NavyBlue}{\faThumbsOUp}{EN PRATIQUE}{#1}
}

\newcommand\CInfo[1]{
     \CBox{Sepia}{\faArrowCircleRight}{REMARQUE}{#1}
}

\newcommand\CRedac[1]{
     \CBox{PineGreen}{\faEdit}{BIEN R\'EDIGER}{#1}
}

\newcommand\CError[1]{
     \CBox{Red}{\faExclamationTriangle}{ATTENTION}{#1}
}

\newcommand\TitreExo[2]{
\needspace{4\baselineskip}
 {\sffamily\large EXERCICE #1\ (\emph{#2 points})}
\vspace{5mm}
}

\newcommand\img[2]{
          \includegraphics[width=#2\paperwidth]{\imgdir#1}
}

\newcommand\imgsvg[2]{
       \begin{center}   \includegraphics[width=#2\paperwidth]{\imgsvgdir#1} \end{center}
}


\newcommand\Lien[2]{
     \href{#1}{#2 \tiny \faExternalLink}
}
\newcommand\mcLien[2]{
     \href{https~://www.maths-cours.fr/#1}{#2 \tiny \faExternalLink}
}

\newcommand{\euro}{\eurologo{}}

%================================================================================================================================
%
% Macros - Environement
%
%================================================================================================================================

\newenvironment{tex}{ %
}
{%
}

\newenvironment{indente}{ %
	\setlength\parindent{10mm}
}

{
	\setlength\parindent{0mm}
}

\newenvironment{corrige}{%
     \needspace{3\baselineskip}
     \medskip
     \textbf{\textsc{Corrigé}}
     \medskip
}
{
}

\newenvironment{extern}{%
     \begin{center}
     }
     {
     \end{center}
}

\NewEnviron{code}{%
	\par
     \boite{gray}{\texttt{%
     \BODY
     }}
     \par
}

\newenvironment{vbloc}{% boite sans cadre empeche saut de page
     \begin{minipage}[t]{\linewidth}
     }
     {
     \end{minipage}
}
\NewEnviron{h2}{%
    \needspace{3\baselineskip}
    \vspace{0.6cm}
	\noindent \MakeUppercase{\sffamily \large \BODY}
	\vspace{1mm}\textcolor{mcgris}{\hrule}\vspace{0.4cm}
	\par
}{}

\NewEnviron{h3}{%
    \needspace{3\baselineskip}
	\vspace{5mm}
	\textsc{\BODY}
	\par
}

\NewEnviron{margeneg}{ %
\begin{addmargin}[-1cm]{0cm}
\BODY
\end{addmargin}
}

\NewEnviron{html}{%
}

\begin{document}
\meta{url}{/exercices/fonctions-et-integrales-bac-blanc-es-l-sujet-4-maths-cours-2018/}
\meta{pid}{10503}
\meta{titre}{Fonctions et intégrales - Bac blanc ES/L Sujet 4 - Maths-cours 2018}
\meta{type}{exercices}
%
\begin{h2}Exercice 3 (5 points)\end{h2}
\par
On considère la fonction $f$ définie sur l'intervalle $[0,5~;~10]$ par :
\[ f(x)=x-2-2\ln x. \]
où $\ln$ désigne la fonction logarithme népérien.
\par
On note $\mathscr{C}_f$ la courbe représentative de $f$ dans un repère orthonormé. Cette courbe est tracée ci-après :
\par
\begin{center}
     \begin{extern}%width="500" alt="fonction à base de logarithme népérien"
          \includegraphics[width=0.7\textwidth]{images/BBESL-s4-3-1.eps}% gbb 1 unite=1cm
     \end{extern}
\end{center}
\par
\begin{enumerate}
     \item %1
     Montrer que pour tout réel $x$ appartenant à l'intervalle $[0,5~;~10]$ :
     \[ f'(x) =\dfrac{x-2}{x}. \]
     \item %2
     Dresser le tableau de variations de $f$ sur l'intervalle $[0,5~;~10]$.
     \item %3
     Déterminer l'équation réduite de la tangente $T$ à la courbe $\mathscr{C}_f$ au point $A(1~;~-1)$.
     \item %4
     \'Etudiez la convexité de $f$ sur l'intervalle $[0,5~;~10]$.
     \item %5
     Montrer que l'équation $f(x)=0$ admet une et une seule solution $\alpha$ sur l'intervalle $[0,5~;~10]$.
     \par
     Donner un encadrement de $\alpha$ d'amplitude $10^{-2}$.
     \item %6
     Montrer que la fonction $F$ définie par :
     \[ F(x)=\dfrac{x^2}{2} - 2x\ln x \]
     est une primitive de la fonction $f$ sur l'intervalle $[0,5~;~10]$.
     \item %7
     Donner la valeur exacte, puis la valeur arrondie à $10^{-2}$, de l'intégrale :
     \[ I=\displaystyle\int_{6}^{10} f(t)dt. \]
     Interpréter graphiquement la valeur de cette intégrale.
     \par
\end{enumerate}
\begin{corrige}
     \begin{enumerate}
          \item %1
          Sur l'intervalle $[0,5~;~10]$, la fonction $f$ est dérivable comme somme de fonctions dérivables et :
          \par
          $f'(x)=1-2 \times \dfrac{1}{x}=\dfrac{x}{x}-\dfrac{2}{x}=\dfrac{x-2}{x}$.
          \par
          \cadre{rouge}{À retenir}{
               La fonction logarithme népérien est définie et dérivable sur l'intervalle $]0~;~+\infty[$ et a pour dérivée la fonction $x \longmapsto \dfrac{1}{x}$.
          }
          \item %2
          $x$ est strictement positif sur l'intervalle $[0,5~;~10]$ ; la fonction $f'$ est donc du signe de $x-2$, c'est à dire qu'elle s'annule pour $x=2$ et est strictement positive pour $x>2$.
          \par
          De plus :
          \par
          $f(2)=2-2-2\ln2=-2\ln2$ ;
          \par
          $f(0,5)=0,5-2-2\ln(0,5)=0,5-2-2\ln\left(\dfrac{1}{2}\right)=-1,5+2\ln2$ ;
          \par
          $f(2)=10-2-2\ln10=8-2\ln10$.
          \par
          On obtient le tableau de variations suivant :
          \par
          %:-+-+-+-+- Engendré par : http://math.et.info.free.fr/TikZ/TableauxVariations/
          \begin{center}
               \begin{extern}%width="400" alt="tableau de variation de la fonction f"
                    \begin{tikzpicture}[scale=0.875]
                         % Styles
                         \tikzstyle{cadre}=[thin]
                         \tikzstyle{fleche}=[->,>=latex,thin]
                         \tikzstyle{nondefini}=[lightgray]
                         % Dimensions Modifiables
                         \def\Lrg{1.5}
                         \def\HtX{1}
                         \def\HtY{0.5}
                         % Dimensions Calculées
                         \def\lignex{-0.5*\HtX}
                         \def\lignef{-1.5*\HtX}
                         \def\separateur{-0.5*\Lrg}
                         % Largeur du tableau
                         \def\gauche{-1.5*\Lrg}
                         \def\droite{5.5*\Lrg}
                         % Hauteur du tableau
                         \def\haut{0.5*\HtX}
                         \def\bas{-2.5*\HtX-2*\HtY}
                         % Ligne de l'abscisse : x
                         \node at (-1*\Lrg,0) {$x$};
                         \node at (0.5*\Lrg,0) {$0,5$};
                         \node at (2.5*\Lrg,0) {$2$};
                         \node at (4.5*\Lrg,0) {$10$};
                         % Ligne de la dérivée : f'(x)
                         \node at (-1*\Lrg,-1*\HtX) {$f'(x)$};
                         \node at (0.5*\Lrg,-1*\HtX) {$\ $};
                         \node at (1.5*\Lrg,-1*\HtX) {$-$};
                         \node at (2.5*\Lrg,-1*\HtX) {$0$};
                         \node at (3.5*\Lrg,-1*\HtX) {$+$};
                         \node at (4.5*\Lrg,-1*\HtX) {$$};
                         % Ligne de la fonction : f(x)
                         \node  at (-1*\Lrg,{-2*\HtX+(-1)*\HtY}) {$f(x)$};
                         \node (f1) at (0.5*\Lrg,{-2*\HtX+(0)*\HtY}) {$2\ln2-1,5$};
                         \node (f2) at (2.5*\Lrg,{-2*\HtX+(-2)*\HtY}) {$-2\ln2$};
                         \node (f3) at (4.5*\Lrg,{-2*\HtX+(0)*\HtY}) {$8-2\ln10$};
                         % Flèches
                         \draw[fleche] (f1) -- (f2);
                         \draw[fleche] (f2) -- (f3);
                         % Encadrement
                         \draw[cadre] (\separateur,\haut) -- (\separateur,\bas);
                         \draw[cadre] (\gauche,\haut) rectangle  (\droite,\bas);
                         \draw[cadre] (\gauche,\lignex) -- (\droite,\lignex);
                         \draw[cadre] (\gauche,\lignef) -- (\droite,\lignef);
                    \end{tikzpicture}
               \end{extern}
          \end{center}
          \item %3
          L'équation réduite de la tangente $T$ à la courbe $\mathscr{C}_f$ au point $A$ d'abscisse $1$ est :
          \par
          $y=f'(1)(x-1)+f(1).$
          \par
          Or :
          \par
          $f(1)=1-2-2\ln(1)=-1\ $ et $f'(1)=\dfrac{1-2}{1}=-1.$
          \par
          L'équation réduite de $T$ est donc :
          \par
          $y=-1(x-1)-1$
          \par
          $y=-x.$
          \par
          \textit{(N.B. : Cette droite passe par le point $A$ et par l'origine du repère.)}
          \par
          \cadre{rouge}{À retenir}{
               L'équation réduite de la tangente à la courbe représentative de $f$ au point d'\textbf{abscisse} $\bm{a}$ est :
               \[ y=f'(a)(x-a)+f(a). \]
          }
          \item %4
          La fonction $f'$ est dérivable sur l'intervalle $[0,5~;~10]$ ; posons :
          \par
          $u(x)=x-2\ $ et $\ v(x)=x.$
          \par
          Alors :
          \par
          $u'(x)=1\ $ et $\ v('x)=1$.
          \par
          Par conséquent :
          \par
          $f''(x)=\dfrac{u'(x)v(x)-u(x)v'(x)}{v(x)^2}$
          \par
          $\phantom{f''(x)}=\dfrac{x-(x-2)}{x^2}$
          \par
          $\phantom{f''(x)}=\dfrac{2}{x^2}$.
          \par
          $f''(x)$ est strictement positive  sur l'intervalle $[0,5~;~10]$ donc la fonction $f$ est \textbf{convexe} sur cet intervalle.
          \item %5
          $f(0)=2\ln2-1,5 \approx -0,11 < 0$ ;
          \par
          $f(2)=2\ln2 \approx -1,39 < 0$ ;
          \par
          $f(10)= 8-2\ln10 \approx 3,39 >0$.
          \par
          D'après le tableau de variations de la question \textbf{2.}, on voit que :
          \par
          \begin{itemize}
               \item
               Pour $x \in [0,5~;~2]$, $f(x)$ est strictement négatif (car inférieur à $f(0)$ qui est négatif).
               \par
               L'équation $f(x)=0$ n'a donc pas de solution sur cet intervalle.
               \item
               Sur l'intervalle $[2~;~10]$, $f$ est \textbf{continue}, \textbf{strictement croissante} et \textbf{change de signe} entre 2 et 10. Donc l'équation $f(x)=0$ admet une unique solution sur l'intervalle $[2~;~10]$.
               \par
               Par conséquent, l'équation $f(x)=0$ admet une unique solution sur l'intervalle $[0,5~;~10]$.
               \par
               \`A la calculatrice, on trouve :
               \par
               $f(5,35) \approx -0,004 < 0$ ;
               \par
               $f(5,36) \approx 20,002 > 0$.
               \par
               Par conséquent :
               \[ 5,35 < \alpha < 5,36. \]
          \end{itemize}
          \item %6
          Pour montrer que $F$ est une primitive de $f$  sur l'intervalle $[0,5~;~10]$, il suffit de montrer que $F'=f$.
          \par
          La dérivée de la fonction $x \longmapsto \dfrac{x^2}{2}$ est la fonction ${x \longmapsto \dfrac{2x}{2}=x}$.
          \par
          Pour calculer la dérivée de la fonction $x \longmapsto -2x\ln x$ on pose :
          \par
          $u(x)=-2x\ $ et $\ v(x)=\ln x$.
          \par
          Alors :
          \par
          $u'(x)=-2\ $ et $\ v('x)=\dfrac{1}{x}$ ;
          \par
          et :
          \par
          $u'(x)v(x)+u(x)v'(x)=-2\ln x - 2x \times \dfrac{1}{x}=-2\ln x - 2$.
          \par
          Par conséquent :
          \par
          $F'(x) = x -2\ln x - 2 = f(x)$.
          \par
          La fonction $F$ est donc une primitive de la fonction $f$ sur l'intervalle $[0,5~;~10]$.
          \par
          \cadre{vert}{En pratique}{
               Pour montrer qu'une fonction $F$ est une primitive de la fonction $f$ sur un intervalle $I$, on calcule la dérivée $F'$ de $F$ et on montre que $F'=f$.
          }
          \item %7
          La fonction $F$ étant une primitive de la fonction $f$ sur l'intervalle $[0,5~;~10]$, on a :
          \par
          $I=\displaystyle\int_{6}^{10}f(t)\text{d}t=\left[F(t)\right]_6^{10}=F(10)-F(6)$
          \par
          $\phantom{I}=\dfrac{10^2}{2} - 20\ln 10 - \left[\dfrac{6^2}{2} - 12\ln 6\right]$
          \par
          $\phantom{I}=50 - 20\ln 10 - 18 + 12\ln 6$
          \par
          $\phantom{I}=32 - 20\ln 10 + 12\ln 6$
          \par
          $I \approx 7,45$ (arrondi au centième).
          \par
          La fonction $f$ étant positive sur l'intervalle $[6~;~10]$, l'intégrale $I$ est égale à l'aire, exprimée en unités d'aire, du domaine délimité par la courbe $\mathscr{C}_f$, l'axe des abscisses et les droites d'équations respectives $x=6$ et $x=10$.
          \par
          \cadre{rouge}{À retenir}{
               Pour calculer l'intégrale $\displaystyle\int_{a}^{b}f(x)\text{d}x$ alors que l'on connaît une primitive $F$ de $f$ sur l'intervalle $[a~;~b]$, on utilise la formule :
               \[  \displaystyle\int_{a}^{b}f(x)\text{d}x = [F(x)]_a^b = F(b)-F(a). \]
               \par
          }
          \cadre{bleu}{Remarque}{
               La variable $x$ dans l'expression $\displaystyle\int_{a}^{b}f(x)\text{d}x$ est une variable \og muette \fg{}.
               \par
               Cela signifie qu'elle n'apparaît pas dans le résultat du calcul et que l'on peut lui substituer n'importe quelle autre lettre ; par exemple il est équivalent d'écrire $\displaystyle\int_{a}^{b}f(x)\text{d}x$ ou $\displaystyle\int_{a}^{b}f(t)\text{d}t$.
          }
     \end{enumerate}
\end{corrige}

\end{document}
µ
\documentclass[a4paper]{article}

%================================================================================================================================
%
% Packages
%
%================================================================================================================================

\usepackage[T1]{fontenc} 	% pour caractères accentués
\usepackage[utf8]{inputenc}  % encodage utf8
\usepackage[french]{babel}	% langue : français
\usepackage{fourier}			% caractères plus lisibles
\usepackage[dvipsnames]{xcolor} % couleurs
\usepackage{fancyhdr}		% réglage header footer
\usepackage{needspace}		% empêcher sauts de page mal placés
\usepackage{graphicx}		% pour inclure des graphiques
\usepackage{enumitem,cprotect}		% personnalise les listes d'items (nécessaire pour ol, al ...)
\usepackage{hyperref}		% Liens hypertexte
\usepackage{pstricks,pst-all,pst-node,pstricks-add,pst-math,pst-plot,pst-tree,pst-eucl} % pstricks
\usepackage[a4paper,includeheadfoot,top=2cm,left=3cm, bottom=2cm,right=3cm]{geometry} % marges etc.
\usepackage{comment}			% commentaires multilignes
\usepackage{amsmath,environ} % maths (matrices, etc.)
\usepackage{amssymb,makeidx}
\usepackage{bm}				% bold maths
\usepackage{tabularx}		% tableaux
\usepackage{colortbl}		% tableaux en couleur
\usepackage{fontawesome}		% Fontawesome
\usepackage{environ}			% environment with command
\usepackage{fp}				% calculs pour ps-tricks
\usepackage{multido}			% pour ps tricks
\usepackage[np]{numprint}	% formattage nombre
\usepackage{tikz,tkz-tab} 			% package principal TikZ
\usepackage{pgfplots}   % axes
\usepackage{mathrsfs}    % cursives
\usepackage{calc}			% calcul taille boites
\usepackage[scaled=0.875]{helvet} % font sans serif
\usepackage{svg} % svg
\usepackage{scrextend} % local margin
\usepackage{scratch} %scratch
\usepackage{multicol} % colonnes
%\usepackage{infix-RPN,pst-func} % formule en notation polanaise inversée
\usepackage{listings}

%================================================================================================================================
%
% Réglages de base
%
%================================================================================================================================

\lstset{
language=Python,   % R code
literate=
{á}{{\'a}}1
{à}{{\`a}}1
{ã}{{\~a}}1
{é}{{\'e}}1
{è}{{\`e}}1
{ê}{{\^e}}1
{í}{{\'i}}1
{ó}{{\'o}}1
{õ}{{\~o}}1
{ú}{{\'u}}1
{ü}{{\"u}}1
{ç}{{\c{c}}}1
{~}{{ }}1
}


\definecolor{codegreen}{rgb}{0,0.6,0}
\definecolor{codegray}{rgb}{0.5,0.5,0.5}
\definecolor{codepurple}{rgb}{0.58,0,0.82}
\definecolor{backcolour}{rgb}{0.95,0.95,0.92}

\lstdefinestyle{mystyle}{
    backgroundcolor=\color{backcolour},   
    commentstyle=\color{codegreen},
    keywordstyle=\color{magenta},
    numberstyle=\tiny\color{codegray},
    stringstyle=\color{codepurple},
    basicstyle=\ttfamily\footnotesize,
    breakatwhitespace=false,         
    breaklines=true,                 
    captionpos=b,                    
    keepspaces=true,                 
    numbers=left,                    
xleftmargin=2em,
framexleftmargin=2em,            
    showspaces=false,                
    showstringspaces=false,
    showtabs=false,                  
    tabsize=2,
    upquote=true
}

\lstset{style=mystyle}


\lstset{style=mystyle}
\newcommand{\imgdir}{C:/laragon/www/newmc/assets/imgsvg/}
\newcommand{\imgsvgdir}{C:/laragon/www/newmc/assets/imgsvg/}

\definecolor{mcgris}{RGB}{220, 220, 220}% ancien~; pour compatibilité
\definecolor{mcbleu}{RGB}{52, 152, 219}
\definecolor{mcvert}{RGB}{125, 194, 70}
\definecolor{mcmauve}{RGB}{154, 0, 215}
\definecolor{mcorange}{RGB}{255, 96, 0}
\definecolor{mcturquoise}{RGB}{0, 153, 153}
\definecolor{mcrouge}{RGB}{255, 0, 0}
\definecolor{mclightvert}{RGB}{205, 234, 190}

\definecolor{gris}{RGB}{220, 220, 220}
\definecolor{bleu}{RGB}{52, 152, 219}
\definecolor{vert}{RGB}{125, 194, 70}
\definecolor{mauve}{RGB}{154, 0, 215}
\definecolor{orange}{RGB}{255, 96, 0}
\definecolor{turquoise}{RGB}{0, 153, 153}
\definecolor{rouge}{RGB}{255, 0, 0}
\definecolor{lightvert}{RGB}{205, 234, 190}
\setitemize[0]{label=\color{lightvert}  $\bullet$}

\pagestyle{fancy}
\renewcommand{\headrulewidth}{0.2pt}
\fancyhead[L]{maths-cours.fr}
\fancyhead[R]{\thepage}
\renewcommand{\footrulewidth}{0.2pt}
\fancyfoot[C]{}

\newcolumntype{C}{>{\centering\arraybackslash}X}
\newcolumntype{s}{>{\hsize=.35\hsize\arraybackslash}X}

\setlength{\parindent}{0pt}		 
\setlength{\parskip}{3mm}
\setlength{\headheight}{1cm}

\def\ebook{ebook}
\def\book{book}
\def\web{web}
\def\type{web}

\newcommand{\vect}[1]{\overrightarrow{\,\mathstrut#1\,}}

\def\Oij{$\left(\text{O}~;~\vect{\imath},~\vect{\jmath}\right)$}
\def\Oijk{$\left(\text{O}~;~\vect{\imath},~\vect{\jmath},~\vect{k}\right)$}
\def\Ouv{$\left(\text{O}~;~\vect{u},~\vect{v}\right)$}

\hypersetup{breaklinks=true, colorlinks = true, linkcolor = OliveGreen, urlcolor = OliveGreen, citecolor = OliveGreen, pdfauthor={Didier BONNEL - https://www.maths-cours.fr} } % supprime les bordures autour des liens

\renewcommand{\arg}[0]{\text{arg}}

\everymath{\displaystyle}

%================================================================================================================================
%
% Macros - Commandes
%
%================================================================================================================================

\newcommand\meta[2]{    			% Utilisé pour créer le post HTML.
	\def\titre{titre}
	\def\url{url}
	\def\arg{#1}
	\ifx\titre\arg
		\newcommand\maintitle{#2}
		\fancyhead[L]{#2}
		{\Large\sffamily \MakeUppercase{#2}}
		\vspace{1mm}\textcolor{mcvert}{\hrule}
	\fi 
	\ifx\url\arg
		\fancyfoot[L]{\href{https://www.maths-cours.fr#2}{\black \footnotesize{https://www.maths-cours.fr#2}}}
	\fi 
}


\newcommand\TitreC[1]{    		% Titre centré
     \needspace{3\baselineskip}
     \begin{center}\textbf{#1}\end{center}
}

\newcommand\newpar{    		% paragraphe
     \par
}

\newcommand\nosp {    		% commande vide (pas d'espace)
}
\newcommand{\id}[1]{} %ignore

\newcommand\boite[2]{				% Boite simple sans titre
	\vspace{5mm}
	\setlength{\fboxrule}{0.2mm}
	\setlength{\fboxsep}{5mm}	
	\fcolorbox{#1}{#1!3}{\makebox[\linewidth-2\fboxrule-2\fboxsep]{
  		\begin{minipage}[t]{\linewidth-2\fboxrule-4\fboxsep}\setlength{\parskip}{3mm}
  			 #2
  		\end{minipage}
	}}
	\vspace{5mm}
}

\newcommand\CBox[4]{				% Boites
	\vspace{5mm}
	\setlength{\fboxrule}{0.2mm}
	\setlength{\fboxsep}{5mm}
	
	\fcolorbox{#1}{#1!3}{\makebox[\linewidth-2\fboxrule-2\fboxsep]{
		\begin{minipage}[t]{1cm}\setlength{\parskip}{3mm}
	  		\textcolor{#1}{\LARGE{#2}}    
 	 	\end{minipage}  
  		\begin{minipage}[t]{\linewidth-2\fboxrule-4\fboxsep}\setlength{\parskip}{3mm}
			\raisebox{1.2mm}{\normalsize\sffamily{\textcolor{#1}{#3}}}						
  			 #4
  		\end{minipage}
	}}
	\vspace{5mm}
}

\newcommand\cadre[3]{				% Boites convertible html
	\par
	\vspace{2mm}
	\setlength{\fboxrule}{0.1mm}
	\setlength{\fboxsep}{5mm}
	\fcolorbox{#1}{white}{\makebox[\linewidth-2\fboxrule-2\fboxsep]{
  		\begin{minipage}[t]{\linewidth-2\fboxrule-4\fboxsep}\setlength{\parskip}{3mm}
			\raisebox{-2.5mm}{\sffamily \small{\textcolor{#1}{\MakeUppercase{#2}}}}		
			\par		
  			 #3
 	 		\end{minipage}
	}}
		\vspace{2mm}
	\par
}

\newcommand\bloc[3]{				% Boites convertible html sans bordure
     \needspace{2\baselineskip}
     {\sffamily \small{\textcolor{#1}{\MakeUppercase{#2}}}}    
		\par		
  			 #3
		\par
}

\newcommand\CHelp[1]{
     \CBox{Plum}{\faInfoCircle}{À RETENIR}{#1}
}

\newcommand\CUp[1]{
     \CBox{NavyBlue}{\faThumbsOUp}{EN PRATIQUE}{#1}
}

\newcommand\CInfo[1]{
     \CBox{Sepia}{\faArrowCircleRight}{REMARQUE}{#1}
}

\newcommand\CRedac[1]{
     \CBox{PineGreen}{\faEdit}{BIEN R\'EDIGER}{#1}
}

\newcommand\CError[1]{
     \CBox{Red}{\faExclamationTriangle}{ATTENTION}{#1}
}

\newcommand\TitreExo[2]{
\needspace{4\baselineskip}
 {\sffamily\large EXERCICE #1\ (\emph{#2 points})}
\vspace{5mm}
}

\newcommand\img[2]{
          \includegraphics[width=#2\paperwidth]{\imgdir#1}
}

\newcommand\imgsvg[2]{
       \begin{center}   \includegraphics[width=#2\paperwidth]{\imgsvgdir#1} \end{center}
}


\newcommand\Lien[2]{
     \href{#1}{#2 \tiny \faExternalLink}
}
\newcommand\mcLien[2]{
     \href{https~://www.maths-cours.fr/#1}{#2 \tiny \faExternalLink}
}

\newcommand{\euro}{\eurologo{}}

%================================================================================================================================
%
% Macros - Environement
%
%================================================================================================================================

\newenvironment{tex}{ %
}
{%
}

\newenvironment{indente}{ %
	\setlength\parindent{10mm}
}

{
	\setlength\parindent{0mm}
}

\newenvironment{corrige}{%
     \needspace{3\baselineskip}
     \medskip
     \textbf{\textsc{Corrigé}}
     \medskip
}
{
}

\newenvironment{extern}{%
     \begin{center}
     }
     {
     \end{center}
}

\NewEnviron{code}{%
	\par
     \boite{gray}{\texttt{%
     \BODY
     }}
     \par
}

\newenvironment{vbloc}{% boite sans cadre empeche saut de page
     \begin{minipage}[t]{\linewidth}
     }
     {
     \end{minipage}
}
\NewEnviron{h2}{%
    \needspace{3\baselineskip}
    \vspace{0.6cm}
	\noindent \MakeUppercase{\sffamily \large \BODY}
	\vspace{1mm}\textcolor{mcgris}{\hrule}\vspace{0.4cm}
	\par
}{}

\NewEnviron{h3}{%
    \needspace{3\baselineskip}
	\vspace{5mm}
	\textsc{\BODY}
	\par
}

\NewEnviron{margeneg}{ %
\begin{addmargin}[-1cm]{0cm}
\BODY
\end{addmargin}
}

\NewEnviron{html}{%
}

\begin{document}
\meta{url}{/exercices/suites-bac-blanc-es-l-sujet-4-maths-cours-2018/}
\meta{pid}{10505}
\meta{titre}{Suites - Bac blanc ES/L Sujet 4 - Maths-cours 2018}
\meta{type}{exercices}
%
\begin{h2}Exercice 4 (6 points)\end{h2}
\par
Un fournisseur d'accès internet propose deux formules, nommées \og Start \fg{} et \og Plus \fg{}, à ses abonnés.
\par
On suppose que le nombre global d'abonnés à ce fournisseur d'accès est stable d'une année sur l'autre et égal à 2 millions.
\par
En 2010, 1,5 million de personnes étaient abonnés à la formule \og Start \fg{} et 500 000 personnes étaient abonnés à la formule \og Plus \fg{}.
\par
Chaque année :
\par
\begin{itemize}
     \item 30\% des abonnés à la formule \og Start \fg{} choisissent de passer à la formule \og Plus  \fg{} l'année suivante (les 70\% restant conservant la formule \og Start \fg{}) ;
     \item 10\% des abonnés à la formule \og Plus \fg{} décident de migrer vers la formule \og Start  \fg{} l'année suivante (les 90\% restant conservant la formule \og Plus \fg{}).
\end{itemize}
\par
Pour tout entier naturel $n$, on note $a_n$ (respectivement $b_n$) le nombre d'abonnés, en milliers, à la formule \og Start \fg{} (respectivement à la formule \og Plus \fg{} ) l'année $2010+n$.
\par
On a donc ${a_0=1~500}$ et ${b_0=500}$.
\par
\begin{enumerate}
     \item %1
     Expliquer pourquoi, pour tout entier naturel $n$, ${a_n+b_n=2~000}$.
     \item %2
     Montrer que, pour tout entier naturel $n$ :
     \[ a_{n+1} =0,7a_n+0,1b_n. \]
     \item %3
     En déduire que, pour tout entier naturel $n$ :
     \[ a_{n+1} =0,6a_n+200. \]
     \item %4
     Soit la suite $(u_n)$ définie, pour tout entier naturel $n$, par :
     \[ u_n=a_n-500. \]
     \par
     \begin{enumerate}[label=\alph*.]
          \item %a
          Montrer que la suite $(u_n)$ est une suite géométrique dont on précisera le premier terme et la raison.
          \item %b
          Exprimer $u_n$ en fonction de $n$.
          \item %c
          En déduire que, pour tout entier naturel $n$ :
          \[ a_n=500+1000 \times 0,6^n. \]
     \end{enumerate}
     \item %5
     Montrer que la suite $(a_n)$ est décroissante et converge vers une limite que l'on déterminera.
     Que peut-on en déduire concernant le nombre d'abonnés à la formule \og Start \fg{} ?
     \item %6
     On souhaite utiliser un tableur pour calculer les termes $a_n$ et $b_n$.
     \par
     \begin{center}
          \img{BBESL-s4-4-1}{0.4}%width="450" alt="utilisation d'un tableur""
     \end{center}
     \par
     \begin{enumerate}
          \item %a
          Proposer une formule à saisir dans la cellule \textbf{C2} pour calculer $a_1$.
          \par
          \textit{Cette formule devra permettre de calculer les valeurs successives de la suite $(a_n)$ en la \og tirant vers la droite \fg{}. }
          \item %b
          Proposer une formule à saisir dans la cellule \textbf{B3} pour calculer $b_0$.
          \par
          \textit{Cette formule devra permettre de calculer les valeurs successives de la suite $(b_n)$ en la \og tirant vers la droite \fg{}. }
          \par
     \end{enumerate}
     \par
\end{enumerate}
\begin{corrige}
     \begin{enumerate}
          \item %1
          Pour tout entier naturel $n$, la somme ${a_n+b_n}$ représente le nombre total d'abonnés (en milliers) chez ce fournisseur d'accès internet.
          \par
          D'après l'énoncé ce nombre est stable et correspond à 2 millions, soit 2~000 milliers d'abonnés.
          \par
          Par conséquent, pour tout entier naturel $n$ :
          \[ a_n+b_n = 2~000. \]
          \item %2
          $a_{n+1}$ représente le nombre d'abonnés, en milliers, à la formule \og Start \fg{} l'année $2010+n+1$.
          \par
          Ce nombre comporte :
          \par
          \begin{itemize}
               \item
               les abonnés à la formule \og Start \fg{} de l'année précédente qui se réinscrivent à cette même formule, c'est à dire 70\% de $a_n$ soit $0,7a_n$ ;
               \item
               les abonnés à la formule \og Plus \fg{} de l'année précédente qui décident de migrer vers la formule \og Start  \fg{}, c'est à dire 10\% de $b_n$ soit $0,1b_n$ ;
               \par
          \end{itemize}
          \par
          Au total, on obtient :
          \par
          \[ a_{n+1} =0,7a_n+0,1b_n. \]
          \item %3
          D'après la question \textbf{1.}, $a_n+b_n = 2~000$ donc $b_n = 2~000 - a_n$.
          \par
          Comme $a_{n+1} =0,7a_n+0,1b_n$, alors :
          \par
          $a_{n+1} =0,7a_n+0,1(2~000 - a_n)$\\
          $\phantom{a_{n+1}} =0,7a_n+200 - 0,1a_n$\\
          $\phantom{a_{n+1}} =0,6a_n+200.$
          \item %4
          \begin{enumerate}[label=\alph*.]
               \item %a
               Pour tout entier naturel $n$ :
               \par
               $u_{n+1}=a_{n+1}-500$
               \par
               $\phantom{u_{n+1}}=0,6a_n+200-500$
               \par
               $\phantom{u_{n+1}}=0,6a_n-300$.
               \par
               Or $u_n=a_n-500$ donc $a_n=u_n+500$ ; alors :
               \par
               $u_{n+1}=0,6(u_n+500)-300$
               \par
               $\phantom{u_{n+1}}=0,6u_n+500-500$
               \par
               $\phantom{u_{n+1}}=0,6u_n$.
               \par
               De plus, comme ${u_0=a_0-500=1~500-500=1~000}$, la suite $(u_n)$ est une suite géométrique de premier terme ${u_0=1~000}$ et de raison ${q=0,6}$.
               \item %b
               Par conséquent :
               \par
               $u_n=u_0q^n=1~000 \times 0,6^n$.
               \item %c
               En utilisant la question précédente et la relation ${a_n=u_n+500}$, on en déduit que pour tout entier naturel $n$ :
               \par
               $a_n=u_n+500=500+1~000 \times 0,6^n$.
               \par
          \end{enumerate}
          \item %5
          D'après la question précédente, pour tout entier naturel $n$ :
          \par
          $a_{n+1}-a_n=500+1000 \times 0,6^{n+1}-\left[500+1000 \times 0,6^n\right]$
          \par
          $\phantom{a_{n+1}-a_n}=500+1000 \times 0,6^{n+1}-500-1000 \times 0,6^n$
          \par
          $\phantom{a_{n+1}-a_n}=1000 \times 0,6^{n+1}-1000 \times 0,6^n$.
          \par
          \vspace{2mm}
          Or $0,6^{n+1}=0,6^n \times 0,6$, donc :
          \par
          $a_{n+1}-a_n=1000 \times 0,6^n \times 0,6-1000 \times 0,6^n$
          \par
          $\phantom{a_{n+1}-a_n}=1000 \times 0,6^{n}[0,6-1)$
          \par
          $\phantom{a_{n+1}-a_n}=-0,4 \times 1000 \times 0,6^{n}$.
          \par
          $a_{n+1}-a_n$ est strictement négatif pour tout entier naturel $n$, donc la suite $(a_n)$ est strictement \textbf{décroissante}.
          \par
          \cadre{rouge}{À retenir}{
               Pour démontrer qu'une suite $(u_n)$ est \textbf{croissante}, on montre que pour tout entier naturel $n$ :
               \[ u_{n+1}-u_n \geqslant 0. \]
               \par
               Pour démontrer qu'une suite $(u_n)$ est \textbf{décroissante}, on montre que pour tout entier naturel $n$ :
               \[ u_{n+1}-u_n \leqslant 0. \]
          }
          \par
          \cadre{vert}{En pratique}{
               La formule :
               \[ a^{n+1} = a^n \times a. \]
               qui est un cas particulier de la formule ${a^{n+m} = a^n \times a^m}$ est très souvent utilisée dans les calculs concernant les suites géométriques.
          }
          Comme $0 < 0,6 < 1$, ${\lim\limits_{n \rightarrow +\infty}0,6^n=0}$ et ${\lim\limits_{n \rightarrow +\infty}1000 \times 0,6^n=0}$. Par conséquent :
          \par
          $\lim\limits_{n \rightarrow +\infty}a_n = \lim\limits_{n \rightarrow +\infty}500+1000 \times 0,6^n =500.$
          \par
          On en déduit que le nombre d'abonnés à la formule \og Start \fg{} va \textbf{décroître et se rapprocher de} $\bm{500~000}$.
          \item %6
          \begin{enumerate}[label=\alph*.]
               \item %a
               \textbf{Solution n°1}
               \par
               On sait que pour tout entier naturel $n$ : $a_{n+1}=0,6a_n+200$.
               \par
               En particulier $a_1=0,6a_0+200$.
               \par
               $a_0$ est situé dans la cellule \textbf{B2}. On peut donc saisir dans la cellule \textbf{C2} :
               \par
               \begin{center}
                    \mintxtbox{=0,6*B2+200}
               \end{center}
               \par
               \textbf{Solution n°2}
               \par
               On sait que pour tout entier naturel $n$ : $a_{n}=500+1000 \times 0,6^n$.
               \par
               En particulier $a_1=500+1000 \times 0,6^1$.
               \par
               Les indices sont situés sur la ligne n°1 ; l'indice 1 est situé dans la cellule \textbf{C1}. On peut donc saisir dans la cellule \textbf{C2} :
               \par
               \begin{center}
                    \mintxtbox{=500+1000*PUISSANCE(0,6~;~C1)}
               \end{center}
               \item %b
               Pour tout entier naturel $n$, $b_n=2~000-a_n$.\\
               En particulier : ${b_0=2~000-a_0}$.
               \par
               $a_0$ est situé dans la cellule \textbf{B2}. On peut donc saisir dans la cellule \textbf{B3} :
               \par
               \begin{center}
                    \mintxtbox{=2000-B2}
               \end{center}
               \par
               \textit{Remarque : D'autres solutions sont également possibles.}
               \par
               \cadre{rouge}{À retenir}{
                    Dans un tableur, une formule mathématique doit débuter par le signe \textbf{=} pour être exécutée.
               }
               \par
          \end{enumerate}
          \par
     \end{enumerate}
\end{corrige}

\end{document}
µ
\documentclass[a4paper]{article}

%================================================================================================================================
%
% Packages
%
%================================================================================================================================

\usepackage[T1]{fontenc} 	% pour caractères accentués
\usepackage[utf8]{inputenc}  % encodage utf8
\usepackage[french]{babel}	% langue : français
\usepackage{fourier}			% caractères plus lisibles
\usepackage[dvipsnames]{xcolor} % couleurs
\usepackage{fancyhdr}		% réglage header footer
\usepackage{needspace}		% empêcher sauts de page mal placés
\usepackage{graphicx}		% pour inclure des graphiques
\usepackage{enumitem,cprotect}		% personnalise les listes d'items (nécessaire pour ol, al ...)
\usepackage{hyperref}		% Liens hypertexte
\usepackage{pstricks,pst-all,pst-node,pstricks-add,pst-math,pst-plot,pst-tree,pst-eucl} % pstricks
\usepackage[a4paper,includeheadfoot,top=2cm,left=3cm, bottom=2cm,right=3cm]{geometry} % marges etc.
\usepackage{comment}			% commentaires multilignes
\usepackage{amsmath,environ} % maths (matrices, etc.)
\usepackage{amssymb,makeidx}
\usepackage{bm}				% bold maths
\usepackage{tabularx}		% tableaux
\usepackage{colortbl}		% tableaux en couleur
\usepackage{fontawesome}		% Fontawesome
\usepackage{environ}			% environment with command
\usepackage{fp}				% calculs pour ps-tricks
\usepackage{multido}			% pour ps tricks
\usepackage[np]{numprint}	% formattage nombre
\usepackage{tikz,tkz-tab} 			% package principal TikZ
\usepackage{pgfplots}   % axes
\usepackage{mathrsfs}    % cursives
\usepackage{calc}			% calcul taille boites
\usepackage[scaled=0.875]{helvet} % font sans serif
\usepackage{svg} % svg
\usepackage{scrextend} % local margin
\usepackage{scratch} %scratch
\usepackage{multicol} % colonnes
%\usepackage{infix-RPN,pst-func} % formule en notation polanaise inversée
\usepackage{listings}

%================================================================================================================================
%
% Réglages de base
%
%================================================================================================================================

\lstset{
language=Python,   % R code
literate=
{á}{{\'a}}1
{à}{{\`a}}1
{ã}{{\~a}}1
{é}{{\'e}}1
{è}{{\`e}}1
{ê}{{\^e}}1
{í}{{\'i}}1
{ó}{{\'o}}1
{õ}{{\~o}}1
{ú}{{\'u}}1
{ü}{{\"u}}1
{ç}{{\c{c}}}1
{~}{{ }}1
}


\definecolor{codegreen}{rgb}{0,0.6,0}
\definecolor{codegray}{rgb}{0.5,0.5,0.5}
\definecolor{codepurple}{rgb}{0.58,0,0.82}
\definecolor{backcolour}{rgb}{0.95,0.95,0.92}

\lstdefinestyle{mystyle}{
    backgroundcolor=\color{backcolour},   
    commentstyle=\color{codegreen},
    keywordstyle=\color{magenta},
    numberstyle=\tiny\color{codegray},
    stringstyle=\color{codepurple},
    basicstyle=\ttfamily\footnotesize,
    breakatwhitespace=false,         
    breaklines=true,                 
    captionpos=b,                    
    keepspaces=true,                 
    numbers=left,                    
xleftmargin=2em,
framexleftmargin=2em,            
    showspaces=false,                
    showstringspaces=false,
    showtabs=false,                  
    tabsize=2,
    upquote=true
}

\lstset{style=mystyle}


\lstset{style=mystyle}
\newcommand{\imgdir}{C:/laragon/www/newmc/assets/imgsvg/}
\newcommand{\imgsvgdir}{C:/laragon/www/newmc/assets/imgsvg/}

\definecolor{mcgris}{RGB}{220, 220, 220}% ancien~; pour compatibilité
\definecolor{mcbleu}{RGB}{52, 152, 219}
\definecolor{mcvert}{RGB}{125, 194, 70}
\definecolor{mcmauve}{RGB}{154, 0, 215}
\definecolor{mcorange}{RGB}{255, 96, 0}
\definecolor{mcturquoise}{RGB}{0, 153, 153}
\definecolor{mcrouge}{RGB}{255, 0, 0}
\definecolor{mclightvert}{RGB}{205, 234, 190}

\definecolor{gris}{RGB}{220, 220, 220}
\definecolor{bleu}{RGB}{52, 152, 219}
\definecolor{vert}{RGB}{125, 194, 70}
\definecolor{mauve}{RGB}{154, 0, 215}
\definecolor{orange}{RGB}{255, 96, 0}
\definecolor{turquoise}{RGB}{0, 153, 153}
\definecolor{rouge}{RGB}{255, 0, 0}
\definecolor{lightvert}{RGB}{205, 234, 190}
\setitemize[0]{label=\color{lightvert}  $\bullet$}

\pagestyle{fancy}
\renewcommand{\headrulewidth}{0.2pt}
\fancyhead[L]{maths-cours.fr}
\fancyhead[R]{\thepage}
\renewcommand{\footrulewidth}{0.2pt}
\fancyfoot[C]{}

\newcolumntype{C}{>{\centering\arraybackslash}X}
\newcolumntype{s}{>{\hsize=.35\hsize\arraybackslash}X}

\setlength{\parindent}{0pt}		 
\setlength{\parskip}{3mm}
\setlength{\headheight}{1cm}

\def\ebook{ebook}
\def\book{book}
\def\web{web}
\def\type{web}

\newcommand{\vect}[1]{\overrightarrow{\,\mathstrut#1\,}}

\def\Oij{$\left(\text{O}~;~\vect{\imath},~\vect{\jmath}\right)$}
\def\Oijk{$\left(\text{O}~;~\vect{\imath},~\vect{\jmath},~\vect{k}\right)$}
\def\Ouv{$\left(\text{O}~;~\vect{u},~\vect{v}\right)$}

\hypersetup{breaklinks=true, colorlinks = true, linkcolor = OliveGreen, urlcolor = OliveGreen, citecolor = OliveGreen, pdfauthor={Didier BONNEL - https://www.maths-cours.fr} } % supprime les bordures autour des liens

\renewcommand{\arg}[0]{\text{arg}}

\everymath{\displaystyle}

%================================================================================================================================
%
% Macros - Commandes
%
%================================================================================================================================

\newcommand\meta[2]{    			% Utilisé pour créer le post HTML.
	\def\titre{titre}
	\def\url{url}
	\def\arg{#1}
	\ifx\titre\arg
		\newcommand\maintitle{#2}
		\fancyhead[L]{#2}
		{\Large\sffamily \MakeUppercase{#2}}
		\vspace{1mm}\textcolor{mcvert}{\hrule}
	\fi 
	\ifx\url\arg
		\fancyfoot[L]{\href{https://www.maths-cours.fr#2}{\black \footnotesize{https://www.maths-cours.fr#2}}}
	\fi 
}


\newcommand\TitreC[1]{    		% Titre centré
     \needspace{3\baselineskip}
     \begin{center}\textbf{#1}\end{center}
}

\newcommand\newpar{    		% paragraphe
     \par
}

\newcommand\nosp {    		% commande vide (pas d'espace)
}
\newcommand{\id}[1]{} %ignore

\newcommand\boite[2]{				% Boite simple sans titre
	\vspace{5mm}
	\setlength{\fboxrule}{0.2mm}
	\setlength{\fboxsep}{5mm}	
	\fcolorbox{#1}{#1!3}{\makebox[\linewidth-2\fboxrule-2\fboxsep]{
  		\begin{minipage}[t]{\linewidth-2\fboxrule-4\fboxsep}\setlength{\parskip}{3mm}
  			 #2
  		\end{minipage}
	}}
	\vspace{5mm}
}

\newcommand\CBox[4]{				% Boites
	\vspace{5mm}
	\setlength{\fboxrule}{0.2mm}
	\setlength{\fboxsep}{5mm}
	
	\fcolorbox{#1}{#1!3}{\makebox[\linewidth-2\fboxrule-2\fboxsep]{
		\begin{minipage}[t]{1cm}\setlength{\parskip}{3mm}
	  		\textcolor{#1}{\LARGE{#2}}    
 	 	\end{minipage}  
  		\begin{minipage}[t]{\linewidth-2\fboxrule-4\fboxsep}\setlength{\parskip}{3mm}
			\raisebox{1.2mm}{\normalsize\sffamily{\textcolor{#1}{#3}}}						
  			 #4
  		\end{minipage}
	}}
	\vspace{5mm}
}

\newcommand\cadre[3]{				% Boites convertible html
	\par
	\vspace{2mm}
	\setlength{\fboxrule}{0.1mm}
	\setlength{\fboxsep}{5mm}
	\fcolorbox{#1}{white}{\makebox[\linewidth-2\fboxrule-2\fboxsep]{
  		\begin{minipage}[t]{\linewidth-2\fboxrule-4\fboxsep}\setlength{\parskip}{3mm}
			\raisebox{-2.5mm}{\sffamily \small{\textcolor{#1}{\MakeUppercase{#2}}}}		
			\par		
  			 #3
 	 		\end{minipage}
	}}
		\vspace{2mm}
	\par
}

\newcommand\bloc[3]{				% Boites convertible html sans bordure
     \needspace{2\baselineskip}
     {\sffamily \small{\textcolor{#1}{\MakeUppercase{#2}}}}    
		\par		
  			 #3
		\par
}

\newcommand\CHelp[1]{
     \CBox{Plum}{\faInfoCircle}{À RETENIR}{#1}
}

\newcommand\CUp[1]{
     \CBox{NavyBlue}{\faThumbsOUp}{EN PRATIQUE}{#1}
}

\newcommand\CInfo[1]{
     \CBox{Sepia}{\faArrowCircleRight}{REMARQUE}{#1}
}

\newcommand\CRedac[1]{
     \CBox{PineGreen}{\faEdit}{BIEN R\'EDIGER}{#1}
}

\newcommand\CError[1]{
     \CBox{Red}{\faExclamationTriangle}{ATTENTION}{#1}
}

\newcommand\TitreExo[2]{
\needspace{4\baselineskip}
 {\sffamily\large EXERCICE #1\ (\emph{#2 points})}
\vspace{5mm}
}

\newcommand\img[2]{
          \includegraphics[width=#2\paperwidth]{\imgdir#1}
}

\newcommand\imgsvg[2]{
       \begin{center}   \includegraphics[width=#2\paperwidth]{\imgsvgdir#1} \end{center}
}


\newcommand\Lien[2]{
     \href{#1}{#2 \tiny \faExternalLink}
}
\newcommand\mcLien[2]{
     \href{https~://www.maths-cours.fr/#1}{#2 \tiny \faExternalLink}
}

\newcommand{\euro}{\eurologo{}}

%================================================================================================================================
%
% Macros - Environement
%
%================================================================================================================================

\newenvironment{tex}{ %
}
{%
}

\newenvironment{indente}{ %
	\setlength\parindent{10mm}
}

{
	\setlength\parindent{0mm}
}

\newenvironment{corrige}{%
     \needspace{3\baselineskip}
     \medskip
     \textbf{\textsc{Corrigé}}
     \medskip
}
{
}

\newenvironment{extern}{%
     \begin{center}
     }
     {
     \end{center}
}

\NewEnviron{code}{%
	\par
     \boite{gray}{\texttt{%
     \BODY
     }}
     \par
}

\newenvironment{vbloc}{% boite sans cadre empeche saut de page
     \begin{minipage}[t]{\linewidth}
     }
     {
     \end{minipage}
}
\NewEnviron{h2}{%
    \needspace{3\baselineskip}
    \vspace{0.6cm}
	\noindent \MakeUppercase{\sffamily \large \BODY}
	\vspace{1mm}\textcolor{mcgris}{\hrule}\vspace{0.4cm}
	\par
}{}

\NewEnviron{h3}{%
    \needspace{3\baselineskip}
	\vspace{5mm}
	\textsc{\BODY}
	\par
}

\NewEnviron{margeneg}{ %
\begin{addmargin}[-1cm]{0cm}
\BODY
\end{addmargin}
}

\NewEnviron{html}{%
}

\begin{document}
\meta{url}{/exercices/matrices-bac-blanc-es-l-sujet-4-maths-cours-2018-spe/}
\meta{pid}{10508}
\meta{titre}{Matrices - Bac blanc ES/L Sujet 4 - Maths-cours 2018 (spé)}
\meta{type}{exercices}
%
\begin{h2}Exercice 3 (5 points)\end{h2}
\par
\textbf{Candidats ayant suivi l'enseignement de spécialité}
\par
%============================================================================================================================
%
\TitreC{Partie A}
%
%============================================================================================================================
\par
Un service de garde d'enfants dispose d'un toboggan dans son espace de jeux.
\par
Le profil de ce toboggan peut être représenté, dans un repère orthonormé d'unité 1 mètre, par la courbe $\mathscr{C}$ d'une fonction $f$ définie sur l'intervalle $[0~;~3]$ à l'aide d'une formule du type :
\[ f(x)=ax^3+bx^2+cx+d \]
où $a, b, c$ et $d$ sont quatre réels.
\par
\begin{center}
     \begin{extern}%width="550" alt="Courbe fonction toboggan"
          \includegraphics[width=0.9\textwidth]{images/BBESL-spe-4-1}% gbb 1 unite=5cm
     \end{extern}
\end{center}
\par
La courbe $\mathscr{C}$ passe par les points $A(0~;~2)$, $B(1~;~1,49)$, $C(2~;~0,66)$ et $D(3~;~0,23)$.
\par
\begin{enumerate}
     \item %1
     Montrer que les réels $a, b, c$ et $d$ sont les solutions d'un système (S) de quatre équations que l'on déterminera.
     \item %2
     On pose :
     \begin{center}
          $M = \begin{pmatrix}
               0 &0 &0 &1 \\
               1 &1 &1 &1 \\
               8 &4 &2 &1 \\
          27 &9 &3 &1  \end{pmatrix}$,
          $     X = \begin{pmatrix}
               a \\
               b \\
               c \\
          d  \end{pmatrix} $
          et
          $   Y = \begin{pmatrix}
               2 \\
               1,49 \\
               0,66 \\
          0,23 \end{pmatrix}$
     \end{center}
     Donner une écriture matricielle du système (S) utilisant les matrices $M, X$ et $Y$
     \item %3
     \`A l'aide d'une calculatrice, vérifier que la matrice $M$ est inversible et déterminer $M^{-1}$.
     \item %4
     Calculer $a, b, c$ et $d$ et en déduire l'expression de $f(x)$.
     \par
\end{enumerate}
\par
%============================================================================================================================
%
\TitreC{Partie B}
%
%============================================================================================================================
\par
Cette garderie propose des déjeuners pour les enfants le mercredi après-midi.
\par
Les enfants ont le choix entre deux menus : le menu \textit{steak haché - frites} et le menu \textit{plat du jour}.
\par
On a remarqué que :
\par
\begin{itemize}
     \item
     si un enfant a choisi le menu \textit{steak haché - frites} un mercredi, la probabilité qu'il choisisse à nouveau ce menu le mercredi suivant est de 0,5 ;
     \item
     si un enfant a choisi le menu \textit{plat du jour} un mercredi, la probabilité qu'il choisisse à nouveau ce menu le mercredi suivant est de 0,7.
\end{itemize}
\par
On sélectionne un enfant au hasard et on note $A$ l'état \og l'enfant choisit le menu \textit{steak haché - frites} \fg{} et $B$ l'état \og l'enfant choisit le menu \textit{plat du jour} \fg{}.
\par
\begin{enumerate}
     \item Traduire les données de l'énoncé par un graphe probabiliste de sommets $A$ et $B$.
     \item Écrire la matrice de transition $M$ de ce graphe en respectant l'ordre alphabétique des sommets.
     \item Montrer que ce graphe admet un état stable que l'on déterminera.\\
     Interpréter ce résultat.
\end{enumerate}
\begin{corrige}
     %============================================================================================================================
     %
     \TitreC{Partie A}
     %
     %============================================================================================================================
     \par
     \begin{enumerate}
          \item %1
          Comme la courbe $\mathscr{C}$ passe par les points $A(0~;~2)$, ${B(1~;~1,49)}$, ${C(2~;~0,66)}$ et ${D(3~;~0,23)}$, on a
          ${f(0)=2}$, ${f(1)=1,49}$, ${f(2)=0,66}$ et ${f(3)=0,23}$.
          \par
          Or :
          \par
          $f(0)=a \times 0^3 +b \times 0^2 + c \times 0 + d $\nosp$= d$ ;\\
          $f(1)=a \times 1^3 +b \times 1^2 + c \times 1 + d $\nosp$= a+b+c+d$ ;\\
          $f(2)=a \times 2^3 +b \times 2^2 + c \times 2 + d $\nosp$= 8a+4b+2c+d$ ;\\
          $f(3)=a \times 3^3 +b \times 3^2 + c \times 3 + d $\nosp$= 27a+9b+3c+d$.\\
          \par
          Donc $a, b, c$ et $d$  sont les solutions du système (S) :
          \par
          \[\left\{\begin{array}{c c c c c c c c c}
                    &&&&&&d& = &2\\
                    a&+&b&+&c&+&d& = &1,49\\
                    8a&+&4b&+&2c&+&d& = &0,66\\
                    27a&+&9b&+&3c&+&d& = &0,23
          \end{array}\right.\]
          \item %2
          Le produit de matrices $ M \times X $ est égal à :
          \par
          $ M \times X = \begin{pmatrix}
               0 &0 &0 &1 \\
               1 &1 &1 &1 \\
               8 &4 &2 &1 \\
          27 &9 &3 &1  \end{pmatrix} \times
          \begin{pmatrix}
               a \\
               b \\
               c \\
               d  \end{pmatrix}  = \begin{pmatrix}
               d \\
               a+b+c+d \\
               8a+4b+2c+d \\
          27a+9b+3c+d \end{pmatrix}
          $
          \par
          Le système $(S)$ peut donc s'écrire sous forme matricielle :
          \[ M \times X = Y. \]
          \item %3
          \`A la calculatrice, on constate que la matrice $M$ est inversible et que :
          \[ M^{-1}=  \begin{pmatrix}
               -1/6 &1/2 &-1/2 &1/6 \\
               1 &-5/2 &2 &-1/2 \\
               -11/6 &3 &-3/2 &1/3 \\
          1 &0 &0 &0  \end{pmatrix} \]
          \item %4
          $ MX=Y \Leftrightarrow X=M^{-1}Y. $
          \par
          \cadre{rouge}{Attention}{
               \textbf{Attention à l'ordre des matrices !}
               \par
               $M^{-1}Y$ n'est pas égal à $YM^{-1}$ !
               \par
               Dans le cas présent, $YM^{-1}$ n'est même pas calculable car le nombre de colonnes de $Y$ n'est pas égal au nombre de lignes de $M^{-1}$.
          }
          En utilisant le résultat de la question précédente, on obtient :
          \par
          $ MX=Y  \Leftrightarrow X=$\nosp$\begin{pmatrix}
               -1/6 &1/2 &-1/2 &1/6 \\
               1 &-5/2 &2 &-1/2 \\
               -11/6 &3 &-3/2 &1/3 \\
          1 &0 &0 &0  \end{pmatrix}
          \begin{pmatrix}
               2 \\
               1,49 \\
               0,66 \\
          0,23 \end{pmatrix}$
          \par
          $\phantom{ MX=Y }\Leftrightarrow X=$\nosp$
          \begin{pmatrix}
               0,12 \\
               -0,52 \\
               -0,11 \\
          2 \end{pmatrix}.$
          \par
          Par conséquent $a=0,12$, $b=-0,52$, $c=-0,11$ et $d=2$.
          \par
          $f$ est donc la fonction définie sur $[0~;~3]$ par :
          \[ f(x)=0,12x^3-0,52x^2-0,11x+2. \]
          \par
     \end{enumerate}
     \par
     %============================================================================================================================
     %
     \TitreC{Partie B}
     %
     %============================================================================================================================
     \par
     \begin{enumerate}
          \item On traduit les données de l'énoncé par un graphe probabiliste de sommets $A$ et $B$:
          \begin{center}
               %\hspace*{1cm}
               \begin{extern}%width="420" alt="Graphe probabiliste"
                    \begin{pspicture}(-1,-0.5)(4,0.5)
                         \circlenode{A}{$A$} \hskip 4cm \circlenode{B}{$B$}% définition des sommets
                         \psset{arcangle=15,arrowsize=2pt 3}%  différents paramètres
                         \ncarc{->}{A}{B} \Aput{0,5}%              arc pondéré partant de A
                         \ncarc{->}{B}{A} \Aput{0,3}%              arc pondéré arrivant à B
                         \nccircle[angleA=90]{->}{A}{4mm}   \Bput{0,5}%    boucle autour de A
                         \nccircle[angleA=-90]{->}{B}{.4cm} \Bput{0,7}%    boucle autour de B
                    \end{pspicture}
               \end{extern}
          \end{center}
          \item  La matrice de transition de ce graphe en considérant les sommets dans l'ordre $A$, $B$ est:
          \begin{center}
               $M=
               \begin{pmatrix}
                    0,5 & 0,5\\
                    0,3 & 0,7
               \end{pmatrix}.$
          \end{center}
          \cadre{rouge}{À retenir}{
               La \textbf{matrice de transition} $M$ d'un graphe $G$ d'ordre $n$ est une \textbf{matrice carrée} d'ordre $n$.
               \par
               Le coefficient de $M$ situé sur la $i$-ième ligne et la $j$-ième colonne est la probabilité inscrite sur l'arc reliant le sommet $i$ au sommet $j$ (ou 0 s'il cet arc n'existe pas).
               \par
               La \textbf{somme} des coefficients de chacune des \textbf{lignes} de $M$ est égale à \textbf{1}.
          }
          \item
          Pour tous les états $P = (a\quad b)$ du graphe : $a + b = 1$.
          \par
          Pour que $P$ soit un état stable, il faut de plus que $PM = P$ :
          \par
          $PM=P \Leftrightarrow \begin{pmatrix} a&b\end{pmatrix}
          \times \begin{pmatrix} 0,5 & 0,5\\ 0,3 & 0,7 \end{pmatrix}
          =\begin{pmatrix} a&b\end{pmatrix}$
          \par
          $\phantom{PM=P} \Leftrightarrow \begin{pmatrix} 0,5a+0,3b & 0,5a+0,7b\end{pmatrix}
          =$\nosp$\begin{pmatrix} a&b\end{pmatrix}$
          \par
          $\phantom{PM=P} \Leftrightarrow
          \left\lbrace
          \begin{array}{r c l}
               0,5a+0,3b  &= &a\\
               0,5a+0,7b &=&b
          \end{array}
     \right.$
     \par
     $\phantom{PM=P}
     \Leftrightarrow
     \left\lbrace
     \begin{array}{r c l}
          -0,5a+0,3b &= &0\\
          0,5a-0,3b &=& 0
     \end{array}
\right.$
\par
$\phantom{PM=P}
\Leftrightarrow
0,5a-0,3b = 0$.
\par
\vspace{0.5cm}
\par
Comme $a+b=1$, l'état stable est solution du système $(S)$ :
\par
$(S) :\ \left\lbrace
\begin{array}{r c l}
     0,5a-0,3b & = &0\\
     a+b  & = & 1
\end{array}
\right.$
\par
$(S)\ \Leftrightarrow \ \left\lbrace
\begin{array}{r c l}
     0,5a - 0,3(1-a)  & = & 0\\
     b  & = & 1-a
\end{array}
\right.$
\par
$(S)\ \Leftrightarrow \ \left\lbrace
\begin{array}{r c l}
     0,8a   & = & 0,3\\
     b  & = & 1-a
\end{array}
\right.$
\par
$(S)\ \Leftrightarrow \ \left\lbrace
\begin{aligned}
     a  & = & \dfrac{3}{8}\\
     ~//
     b & = & \dfrac{5}{8}
\end{aligned}
\right.
$
\par
L'état stable est donc $P=\begin{pmatrix} \dfrac{3}{8} & \dfrac{5}{8} \end{pmatrix}$.
\par
\cadre{rouge}{À retenir}{
     Un état probabiliste $P$ est \textbf{stable} si $\bm{PM = P}$ où $M$ est la matrice de transition associée au graphe.
     \par
     Pour tout graphe probabiliste dont \textbf{la matrice de transition ne comporte pas de 0}, il existe \textbf{un unique état stable} $P$ indépendant de l'état initial.
     \par
     Les états $P_n$ (états probabilistes à l'étape $n$) \textbf{convergent vers cet état stable} lorsque $n$ tend vers l'infini.
}
\par
\cadre{vert}{En pratique}{
     Pour trouver l'état stable $P = (a\quad b)$ d'un graphe d'ordre 2, on résout le système :
     \par
     $(a\quad b) \times M = (a\quad b)$\: et \: $a + b = 1$.
     \par
     Pour trouver l'état stable $P = (a\quad b\quad c)$ d'un graphe d'ordre 3, on résout le système :
     \par
     $(a\quad b\quad c) \times M = (a\quad b\quad c)$\: et \:$a + b + c = 1$.
}
\par
\vspace{0.5cm}
\par
Ce résultat peut s'interpréter de la manière suivante : \og \`A long terme, les $\dfrac{3}{8}$-ièmes des enfants choisiront le menu \textit{steak haché - frites} et les $\dfrac{5}{8}$-ièmes restants, le menu \textit{plat du jour}\fg{}.
\end{enumerate}
\end{corrige}

\end{document}
µ
\documentclass[a4paper]{article}

%================================================================================================================================
%
% Packages
%
%================================================================================================================================

\usepackage[T1]{fontenc} 	% pour caractères accentués
\usepackage[utf8]{inputenc}  % encodage utf8
\usepackage[french]{babel}	% langue : français
\usepackage{fourier}			% caractères plus lisibles
\usepackage[dvipsnames]{xcolor} % couleurs
\usepackage{fancyhdr}		% réglage header footer
\usepackage{needspace}		% empêcher sauts de page mal placés
\usepackage{graphicx}		% pour inclure des graphiques
\usepackage{enumitem,cprotect}		% personnalise les listes d'items (nécessaire pour ol, al ...)
\usepackage{hyperref}		% Liens hypertexte
\usepackage{pstricks,pst-all,pst-node,pstricks-add,pst-math,pst-plot,pst-tree,pst-eucl} % pstricks
\usepackage[a4paper,includeheadfoot,top=2cm,left=3cm, bottom=2cm,right=3cm]{geometry} % marges etc.
\usepackage{comment}			% commentaires multilignes
\usepackage{amsmath,environ} % maths (matrices, etc.)
\usepackage{amssymb,makeidx}
\usepackage{bm}				% bold maths
\usepackage{tabularx}		% tableaux
\usepackage{colortbl}		% tableaux en couleur
\usepackage{fontawesome}		% Fontawesome
\usepackage{environ}			% environment with command
\usepackage{fp}				% calculs pour ps-tricks
\usepackage{multido}			% pour ps tricks
\usepackage[np]{numprint}	% formattage nombre
\usepackage{tikz,tkz-tab} 			% package principal TikZ
\usepackage{pgfplots}   % axes
\usepackage{mathrsfs}    % cursives
\usepackage{calc}			% calcul taille boites
\usepackage[scaled=0.875]{helvet} % font sans serif
\usepackage{svg} % svg
\usepackage{scrextend} % local margin
\usepackage{scratch} %scratch
\usepackage{multicol} % colonnes
%\usepackage{infix-RPN,pst-func} % formule en notation polanaise inversée
\usepackage{listings}

%================================================================================================================================
%
% Réglages de base
%
%================================================================================================================================

\lstset{
language=Python,   % R code
literate=
{á}{{\'a}}1
{à}{{\`a}}1
{ã}{{\~a}}1
{é}{{\'e}}1
{è}{{\`e}}1
{ê}{{\^e}}1
{í}{{\'i}}1
{ó}{{\'o}}1
{õ}{{\~o}}1
{ú}{{\'u}}1
{ü}{{\"u}}1
{ç}{{\c{c}}}1
{~}{{ }}1
}


\definecolor{codegreen}{rgb}{0,0.6,0}
\definecolor{codegray}{rgb}{0.5,0.5,0.5}
\definecolor{codepurple}{rgb}{0.58,0,0.82}
\definecolor{backcolour}{rgb}{0.95,0.95,0.92}

\lstdefinestyle{mystyle}{
    backgroundcolor=\color{backcolour},   
    commentstyle=\color{codegreen},
    keywordstyle=\color{magenta},
    numberstyle=\tiny\color{codegray},
    stringstyle=\color{codepurple},
    basicstyle=\ttfamily\footnotesize,
    breakatwhitespace=false,         
    breaklines=true,                 
    captionpos=b,                    
    keepspaces=true,                 
    numbers=left,                    
xleftmargin=2em,
framexleftmargin=2em,            
    showspaces=false,                
    showstringspaces=false,
    showtabs=false,                  
    tabsize=2,
    upquote=true
}

\lstset{style=mystyle}


\lstset{style=mystyle}
\newcommand{\imgdir}{C:/laragon/www/newmc/assets/imgsvg/}
\newcommand{\imgsvgdir}{C:/laragon/www/newmc/assets/imgsvg/}

\definecolor{mcgris}{RGB}{220, 220, 220}% ancien~; pour compatibilité
\definecolor{mcbleu}{RGB}{52, 152, 219}
\definecolor{mcvert}{RGB}{125, 194, 70}
\definecolor{mcmauve}{RGB}{154, 0, 215}
\definecolor{mcorange}{RGB}{255, 96, 0}
\definecolor{mcturquoise}{RGB}{0, 153, 153}
\definecolor{mcrouge}{RGB}{255, 0, 0}
\definecolor{mclightvert}{RGB}{205, 234, 190}

\definecolor{gris}{RGB}{220, 220, 220}
\definecolor{bleu}{RGB}{52, 152, 219}
\definecolor{vert}{RGB}{125, 194, 70}
\definecolor{mauve}{RGB}{154, 0, 215}
\definecolor{orange}{RGB}{255, 96, 0}
\definecolor{turquoise}{RGB}{0, 153, 153}
\definecolor{rouge}{RGB}{255, 0, 0}
\definecolor{lightvert}{RGB}{205, 234, 190}
\setitemize[0]{label=\color{lightvert}  $\bullet$}

\pagestyle{fancy}
\renewcommand{\headrulewidth}{0.2pt}
\fancyhead[L]{maths-cours.fr}
\fancyhead[R]{\thepage}
\renewcommand{\footrulewidth}{0.2pt}
\fancyfoot[C]{}

\newcolumntype{C}{>{\centering\arraybackslash}X}
\newcolumntype{s}{>{\hsize=.35\hsize\arraybackslash}X}

\setlength{\parindent}{0pt}		 
\setlength{\parskip}{3mm}
\setlength{\headheight}{1cm}

\def\ebook{ebook}
\def\book{book}
\def\web{web}
\def\type{web}

\newcommand{\vect}[1]{\overrightarrow{\,\mathstrut#1\,}}

\def\Oij{$\left(\text{O}~;~\vect{\imath},~\vect{\jmath}\right)$}
\def\Oijk{$\left(\text{O}~;~\vect{\imath},~\vect{\jmath},~\vect{k}\right)$}
\def\Ouv{$\left(\text{O}~;~\vect{u},~\vect{v}\right)$}

\hypersetup{breaklinks=true, colorlinks = true, linkcolor = OliveGreen, urlcolor = OliveGreen, citecolor = OliveGreen, pdfauthor={Didier BONNEL - https://www.maths-cours.fr} } % supprime les bordures autour des liens

\renewcommand{\arg}[0]{\text{arg}}

\everymath{\displaystyle}

%================================================================================================================================
%
% Macros - Commandes
%
%================================================================================================================================

\newcommand\meta[2]{    			% Utilisé pour créer le post HTML.
	\def\titre{titre}
	\def\url{url}
	\def\arg{#1}
	\ifx\titre\arg
		\newcommand\maintitle{#2}
		\fancyhead[L]{#2}
		{\Large\sffamily \MakeUppercase{#2}}
		\vspace{1mm}\textcolor{mcvert}{\hrule}
	\fi 
	\ifx\url\arg
		\fancyfoot[L]{\href{https://www.maths-cours.fr#2}{\black \footnotesize{https://www.maths-cours.fr#2}}}
	\fi 
}


\newcommand\TitreC[1]{    		% Titre centré
     \needspace{3\baselineskip}
     \begin{center}\textbf{#1}\end{center}
}

\newcommand\newpar{    		% paragraphe
     \par
}

\newcommand\nosp {    		% commande vide (pas d'espace)
}
\newcommand{\id}[1]{} %ignore

\newcommand\boite[2]{				% Boite simple sans titre
	\vspace{5mm}
	\setlength{\fboxrule}{0.2mm}
	\setlength{\fboxsep}{5mm}	
	\fcolorbox{#1}{#1!3}{\makebox[\linewidth-2\fboxrule-2\fboxsep]{
  		\begin{minipage}[t]{\linewidth-2\fboxrule-4\fboxsep}\setlength{\parskip}{3mm}
  			 #2
  		\end{minipage}
	}}
	\vspace{5mm}
}

\newcommand\CBox[4]{				% Boites
	\vspace{5mm}
	\setlength{\fboxrule}{0.2mm}
	\setlength{\fboxsep}{5mm}
	
	\fcolorbox{#1}{#1!3}{\makebox[\linewidth-2\fboxrule-2\fboxsep]{
		\begin{minipage}[t]{1cm}\setlength{\parskip}{3mm}
	  		\textcolor{#1}{\LARGE{#2}}    
 	 	\end{minipage}  
  		\begin{minipage}[t]{\linewidth-2\fboxrule-4\fboxsep}\setlength{\parskip}{3mm}
			\raisebox{1.2mm}{\normalsize\sffamily{\textcolor{#1}{#3}}}						
  			 #4
  		\end{minipage}
	}}
	\vspace{5mm}
}

\newcommand\cadre[3]{				% Boites convertible html
	\par
	\vspace{2mm}
	\setlength{\fboxrule}{0.1mm}
	\setlength{\fboxsep}{5mm}
	\fcolorbox{#1}{white}{\makebox[\linewidth-2\fboxrule-2\fboxsep]{
  		\begin{minipage}[t]{\linewidth-2\fboxrule-4\fboxsep}\setlength{\parskip}{3mm}
			\raisebox{-2.5mm}{\sffamily \small{\textcolor{#1}{\MakeUppercase{#2}}}}		
			\par		
  			 #3
 	 		\end{minipage}
	}}
		\vspace{2mm}
	\par
}

\newcommand\bloc[3]{				% Boites convertible html sans bordure
     \needspace{2\baselineskip}
     {\sffamily \small{\textcolor{#1}{\MakeUppercase{#2}}}}    
		\par		
  			 #3
		\par
}

\newcommand\CHelp[1]{
     \CBox{Plum}{\faInfoCircle}{À RETENIR}{#1}
}

\newcommand\CUp[1]{
     \CBox{NavyBlue}{\faThumbsOUp}{EN PRATIQUE}{#1}
}

\newcommand\CInfo[1]{
     \CBox{Sepia}{\faArrowCircleRight}{REMARQUE}{#1}
}

\newcommand\CRedac[1]{
     \CBox{PineGreen}{\faEdit}{BIEN R\'EDIGER}{#1}
}

\newcommand\CError[1]{
     \CBox{Red}{\faExclamationTriangle}{ATTENTION}{#1}
}

\newcommand\TitreExo[2]{
\needspace{4\baselineskip}
 {\sffamily\large EXERCICE #1\ (\emph{#2 points})}
\vspace{5mm}
}

\newcommand\img[2]{
          \includegraphics[width=#2\paperwidth]{\imgdir#1}
}

\newcommand\imgsvg[2]{
       \begin{center}   \includegraphics[width=#2\paperwidth]{\imgsvgdir#1} \end{center}
}


\newcommand\Lien[2]{
     \href{#1}{#2 \tiny \faExternalLink}
}
\newcommand\mcLien[2]{
     \href{https~://www.maths-cours.fr/#1}{#2 \tiny \faExternalLink}
}

\newcommand{\euro}{\eurologo{}}

%================================================================================================================================
%
% Macros - Environement
%
%================================================================================================================================

\newenvironment{tex}{ %
}
{%
}

\newenvironment{indente}{ %
	\setlength\parindent{10mm}
}

{
	\setlength\parindent{0mm}
}

\newenvironment{corrige}{%
     \needspace{3\baselineskip}
     \medskip
     \textbf{\textsc{Corrigé}}
     \medskip
}
{
}

\newenvironment{extern}{%
     \begin{center}
     }
     {
     \end{center}
}

\NewEnviron{code}{%
	\par
     \boite{gray}{\texttt{%
     \BODY
     }}
     \par
}

\newenvironment{vbloc}{% boite sans cadre empeche saut de page
     \begin{minipage}[t]{\linewidth}
     }
     {
     \end{minipage}
}
\NewEnviron{h2}{%
    \needspace{3\baselineskip}
    \vspace{0.6cm}
	\noindent \MakeUppercase{\sffamily \large \BODY}
	\vspace{1mm}\textcolor{mcgris}{\hrule}\vspace{0.4cm}
	\par
}{}

\NewEnviron{h3}{%
    \needspace{3\baselineskip}
	\vspace{5mm}
	\textsc{\BODY}
	\par
}

\NewEnviron{margeneg}{ %
\begin{addmargin}[-1cm]{0cm}
\BODY
\end{addmargin}
}

\NewEnviron{html}{%
}

\begin{document}
\meta{url}{/exercices/loi-normale-estimation-bac-blanc-es-l-sujet-5-maths-cours-2018/}
\meta{pid}{10549}
\meta{titre}{Loi normale - Estimation - Bac blanc ES/L Sujet 5 - Maths-cours 2018}
\meta{type}{exercices}
%
\begin{h2}Exercice 1 (5 points)\end{h2}
\par
\textit{Les parties A et B sont indépendantes.}
\par
%============================================================================================================================
%
\TitreC{Partie A}
%
%============================================================================================================================
\par
La durée de vie, en heures, d'une lampe à incandescence peut être modélisée par une variable aléatoire $X$ qui suit la loi normale de moyenne $\mu =1\ 000$ et d'écart-type $\sigma = 200$.
\par
\begin{enumerate}
     \item
     Déterminer la probabilité que la durée de vie d'une lampe à incandescence soit supérieure à $1\ 400$ heures. On donnera une valeur approchée à $10^{-3}$ près.
     \item
     Déterminer la valeur du réel $m$, arrondie à la centaine, telle que :
     \[ p(X \geqslant m) = 0,16. \]
     \par
     Interpréter ce résultat dans le cadre de l'exercice.
     \item
     Parmi les quatre graphiques proposés ci-après, l'un d'eux représente la fonction de densité de probabilité associée à la loi normale de moyenne $\mu =1\ 000$ et d'écart-type $\sigma = 200$.
     \par
     \begin{center}
          \begin{extern}%width="600" alt="Graphique loi normale Proposition 1"
               \includegraphics[width=0.9\textwidth]{images/BBESL-s5-1-1}% gbb 1 unite=1/100cm
          \end{extern}
     \end{center}
     \begin{center}
          Graphique 1.
     \end{center}
     \par
     \begin{center}
          \begin{extern}%width="600" alt="Graphique loi normale Proposition 2"
               \includegraphics[width=0.9\textwidth]{images/BBESL-s5-1-2}% gbb 1 unite=1/100cm
          \end{extern}
     \end{center}
     \begin{center}
          Graphique 2.
     \end{center}
     \begin{center}
          \begin{extern}%width="600" alt="Graphique loi normale Proposition 3"
               \includegraphics[width=0.9\textwidth]{images/BBESL-s5-1-3}% gbb 1 unite=1/100cm
          \end{extern}
     \end{center}
     \begin{center}
          Graphique 3.
     \end{center}
     \par
     \begin{center}
          \begin{extern}%width="600" alt="Graphique loi normale Proposition 4"
               \includegraphics[width=0.9\textwidth]{images/BBESL-s5-1-4}% gbb 1 unite=1/100cm
          \end{extern}
     \end{center}
     \begin{center}
          Graphique 4.
     \end{center}
     \par
     Lequel ?
     \par
     Justifier votre réponse.
     \par
\end{enumerate}
\par
%============================================================================================================================
%
\TitreC{Partie B}
%
%============================================================================================================================
\par
Un fabriquant de lampes halogènes affirme que $75\%$ des lampes qu'il produit ont une durée de vie supérieure à 3\ 000 heures.
\par
Afin de vérifier cette affirmation, un laboratoire indépendant effectue un test sur 200 lampes choisies au hasard chez ce fabriquant.
\par
Il s'avère que seulement 140 d'entre elles ont une  durée de vie supérieure à 3\ 000 heures.
\par
\begin{enumerate}
     \item
     Déterminer, pour ce fabriquant, un intervalle de fluctuation asymptotique au seuil de $95\%$ de la proportion de lampes ayant une durée de vie supérieure à 3\ 000 heures pour un échantillon de taille 200.
     \item
     Les résultats du test mené par le laboratoire remettent-ils en cause l'affirmation du fabriquant ?
     \par
\end{enumerate}
\begin{corrige}
     %============================================================================================================================
     %
     \TitreC{Partie A}
     %
     %============================================================================================================================
     \par
     \begin{enumerate}
          \item %A1
          On cherche la probabilité de l'événement $(X>1~400)$.
          \par
          \`A la calculatrice, on trouve :
          \[ p(X>1~400) \approx 0,023\ \text{(au millième près)}. \]
          \par
          \item %A2
          On recherche la valeur du réel $m$ tel que :
          \[ p(X \geqslant m) = 0,16. \]
          \par
          \`A la calculatrice, on trouve :
          \par
          $m \approx 1200\ $ (arrondi à la centaine).
          \item %A3
          Le graphique correct est le \textbf{graphique 1.}
          \par
          La moyenne $\mu$ d'une loi binomiale correspond à l'abscisse du sommet de la courbe de Gauss.
          \par
          Cela permet d'éliminer immédiatement les graphiques \textbf{2.} et \textbf{4.} pour lesquels $\mu =200$.
          \par
          On sait que, pour une loi normale de moyenne $\mu$ et d'écart-type $\sigma$ :
          \[ p(\mu -\sigma \leqslant X \leqslant\mu +\sigma ) \approx 0,68. \]
          \par
          C'est à dire ici :
          \[ p(800 \leqslant X \leqslant 1\ 200 ) \approx 0,68 = 68\%. \]
          \par
          Or, $p(800 \leqslant X \leqslant 1\ 200$ représente l'aire, en unité d'aire, du domaine compris entre la courbe, l'axe des abscisses et les droites d'équations $x=800$ et $x=1\ 200$.
          \par
          Par ailleurs, l'aire totale du domaine compris entre la courbe et l'axe des abscisses vaut 1 unité d'aire.
          \par
          Pour le graphique \textbf{1.}, il est tout à fait raisonnable de penser que l'aire comprise entre les abscisses 800 et 1\ 200 représente 68\% de l'aire totale (\textit{voir figure ci-après}).
          \begin{center}
               \begin{extern}%width="600" alt="Graphique loi normale Solution 1"
                    \includegraphics[width=0.9\textwidth]{images/BBESL-s5-1-5}% gbb 1 unite=1/100cm
               \end{extern}
          \end{center}
          \begin{center}
               Graphique 1.
          \end{center}
          \par
          Par contre, pour le graphique \textbf{3.}, l'aire comprise entre les abscisses 800 et 1\ 200 représente pratiquement 100\% de l'aire totale (\textit{voir figure ci-après}) ; ce graphique ne convient donc pas.
          \begin{center}
               \begin{extern}%width="600" alt="Graphique loi normale Solution 3"
                    \includegraphics[width=0.9\textwidth]{images/BBESL-s5-1-6}% gbb 1 unite=1/100cm
               \end{extern}
          \end{center}
          \begin{center}
               Graphique 3.
          \end{center}
          \cadre{rouge}{À retenir}{
               Si la variable aléatoire $X$ suit une loi normale d'espérance $\mu$ et d'écart-type $\sigma$, alors :
               \par
               \begin{itemize}
                    \item  $p\left(\mu -\sigma \leqslant X\leqslant \mu + \sigma \right)\approx 0,68$ (à $10^{-2}$ près) ;
                    \item  $p\left(\mu -2\sigma \leqslant X\leqslant \mu + 2\sigma \right)\approx 0,95$ (à $10^{-2}$ près) ;
                    \item  $p\left(\mu -3\sigma \leqslant X\leqslant \mu + 3\sigma \right)\approx 0,997$ (à $10^{-3}$ près).
               \end{itemize}
          }
          \cadre{vert}{En pratique}{
               Pour une variable aléatoire $X$ qui suit une loi normale d'espérance $\mu$ et d'écart-type $\sigma$ :
               \par
               \begin{itemize}
                    \item  $\mu$ correspond à l'\textbf{abscisse du sommet} de la courbe ;
                    \item  plus l'écart-type $\sigma$ est grand, plus la \og cloche \fg{} est évasée ;
                    \item  environ \textbf{68\%} de l'aire située entre la courbe et l'axe des abscisses est comprise entre les abscisses $\mu - \sigma $ et $\mu + \sigma $.
               \end{itemize}
          }
     \end{enumerate}
     \par
     %============================================================================================================================
     %
     \TitreC{Partie B}
     %
     %============================================================================================================================
     \par
     \begin{enumerate}
          \item %B1
          D'après le fabriquant, la proportion théorique de lampes ayant une durée de vie supérieure à 3\ 000 heures est $p=0,75$.
          \par
          La taille  de l'échantillon est $n=200$.
          \par
          On vérifie que :
          \par
          \begin{itemize}
               \item $n=200 \geqslant 30$ ;
               \item $np=200 \times 0,75=150 \geqslant 5$ ;
               \item $n(1-p)=200 \times 0,25=50 \geqslant 5$.
          \end{itemize}
          \par
          Les conditions de validité étant remplies, l'intervalle cherché est :
          \par
          \[ I=\left[p-1,96\dfrac{\sqrt{p(1-p)}}{\sqrt{n}}~;~p+1,96\dfrac{\sqrt{p(1-p)}}{\sqrt{n}}\right]. \]
          \par
          \vspace{1cm}
          \par
          $p-1,96\dfrac{\sqrt{p(1-p)}}{\sqrt{n}}=0,75-1,96\dfrac{\sqrt{0,75(1-0,75)}}{\sqrt{200}}$
          \par
          $\phantom{p-1,96\dfrac{\sqrt{p(1-p)}}{\sqrt{n}}} \approx 0,690 $ (arrondi au millième).
          \par
          $p+1,96\dfrac{\sqrt{p(1-p)}}{\sqrt{n}}=0,75+1,96\dfrac{\sqrt{0,75(1-0,75)}}{\sqrt{200}}$
          \par
          $\phantom{p+1,96\dfrac{\sqrt{p(1-p)}}{\sqrt{n}}} \approx 0,810 $ (arrondi au millième).
          \par
          L'intervalle de fluctuation asymptotique au seuil de $95\%$ de la proportion de lampes ayant une durée de vie supérieure à 3\ 000 heures pour un échantillon de taille 200 est donc :
          \[ I=[0,69~;~0,81]. \]
          \cadre{rouge}{À retenir}{
               On note :
               \par
               \begin{itemize}
                    \item %
                    $n$ : la taille  de l'\textbf{échantillon},
                    \item %
                    $f$ : la fréquence du caractère dans l'\textbf{échan\-til\-lon},
                    \item %
                    $p$ : la proportion (connue ou supposée) du caractère dans la \textbf{population}.
               \end{itemize}
               \par
               Si les conditions $\bm{n\geqslant 30}$, $\bm{np\geqslant 5}$ et $\bm{n\left(1-p\right)\geqslant 5}$ sont vérifiées, l'intervalle de fluctuation asymptotique au seuil de 95\% est :
               \[  I=\left[ p-1,96\frac{\sqrt{p\left(1-p\right)}}{\sqrt{n}}~ ; ~p+1,96 \frac{\sqrt{p\left(1-p\right)}}{\sqrt{n}} \right]. \]
          }
          \item %B2
          Sur l'échantillon de 200 lampes, 140 ont une  durée de vie supérieure à 3\ 000 heures.
          \par
          Cela correspond à une fréquence observée de :
          \par
          $f=\dfrac{140}{200}=0,7$.
          \par
          Or $f=0,7$ appartient à l'intervalle de fluctuation $I$ trouvé à la question précédente.
          \par
          Au seuil de 95\%, les résultats du test \textbf{ne remettent pas en cause} l'affirmation du fabriquant.
          \cadre{vert}{En pratique}{
               En pratique, \textbf{pour valider ou rejeter une hypothèse} à l'aide d'un intervalle de fluctuation asymptotique :
               \begin{itemize}
                    \item %
                    On détermine l'intervalle de fluctuation asymptotique $I$ au seuil de 95\% en prenant pour $p$ la proportion \textbf{supposée} du caractère dans l'ensemble de la \textbf{population}.
                    \item %
                    On calcule la fréquence \textbf{observée} $f$ du caractère dans l'\textbf{échantillon}.
                    \item %
                    Si $f$ appartient à l'intervalle $I$ on \textbf{valide} l'hypothèse.
                    \item %
                    Si $f$ n'appartient pas à l'intervalle $I$ on \textbf{rejette} l'hypothèse. Le risque d'erreur en rejetant l'hypothèse est alors inférieur 5\%.
               \end{itemize}
          }
     \end{enumerate}
\end{corrige}

\end{document}
µ
\documentclass[a4paper]{article}

%================================================================================================================================
%
% Packages
%
%================================================================================================================================

\usepackage[T1]{fontenc} 	% pour caractères accentués
\usepackage[utf8]{inputenc}  % encodage utf8
\usepackage[french]{babel}	% langue : français
\usepackage{fourier}			% caractères plus lisibles
\usepackage[dvipsnames]{xcolor} % couleurs
\usepackage{fancyhdr}		% réglage header footer
\usepackage{needspace}		% empêcher sauts de page mal placés
\usepackage{graphicx}		% pour inclure des graphiques
\usepackage{enumitem,cprotect}		% personnalise les listes d'items (nécessaire pour ol, al ...)
\usepackage{hyperref}		% Liens hypertexte
\usepackage{pstricks,pst-all,pst-node,pstricks-add,pst-math,pst-plot,pst-tree,pst-eucl} % pstricks
\usepackage[a4paper,includeheadfoot,top=2cm,left=3cm, bottom=2cm,right=3cm]{geometry} % marges etc.
\usepackage{comment}			% commentaires multilignes
\usepackage{amsmath,environ} % maths (matrices, etc.)
\usepackage{amssymb,makeidx}
\usepackage{bm}				% bold maths
\usepackage{tabularx}		% tableaux
\usepackage{colortbl}		% tableaux en couleur
\usepackage{fontawesome}		% Fontawesome
\usepackage{environ}			% environment with command
\usepackage{fp}				% calculs pour ps-tricks
\usepackage{multido}			% pour ps tricks
\usepackage[np]{numprint}	% formattage nombre
\usepackage{tikz,tkz-tab} 			% package principal TikZ
\usepackage{pgfplots}   % axes
\usepackage{mathrsfs}    % cursives
\usepackage{calc}			% calcul taille boites
\usepackage[scaled=0.875]{helvet} % font sans serif
\usepackage{svg} % svg
\usepackage{scrextend} % local margin
\usepackage{scratch} %scratch
\usepackage{multicol} % colonnes
%\usepackage{infix-RPN,pst-func} % formule en notation polanaise inversée
\usepackage{listings}

%================================================================================================================================
%
% Réglages de base
%
%================================================================================================================================

\lstset{
language=Python,   % R code
literate=
{á}{{\'a}}1
{à}{{\`a}}1
{ã}{{\~a}}1
{é}{{\'e}}1
{è}{{\`e}}1
{ê}{{\^e}}1
{í}{{\'i}}1
{ó}{{\'o}}1
{õ}{{\~o}}1
{ú}{{\'u}}1
{ü}{{\"u}}1
{ç}{{\c{c}}}1
{~}{{ }}1
}


\definecolor{codegreen}{rgb}{0,0.6,0}
\definecolor{codegray}{rgb}{0.5,0.5,0.5}
\definecolor{codepurple}{rgb}{0.58,0,0.82}
\definecolor{backcolour}{rgb}{0.95,0.95,0.92}

\lstdefinestyle{mystyle}{
    backgroundcolor=\color{backcolour},   
    commentstyle=\color{codegreen},
    keywordstyle=\color{magenta},
    numberstyle=\tiny\color{codegray},
    stringstyle=\color{codepurple},
    basicstyle=\ttfamily\footnotesize,
    breakatwhitespace=false,         
    breaklines=true,                 
    captionpos=b,                    
    keepspaces=true,                 
    numbers=left,                    
xleftmargin=2em,
framexleftmargin=2em,            
    showspaces=false,                
    showstringspaces=false,
    showtabs=false,                  
    tabsize=2,
    upquote=true
}

\lstset{style=mystyle}


\lstset{style=mystyle}
\newcommand{\imgdir}{C:/laragon/www/newmc/assets/imgsvg/}
\newcommand{\imgsvgdir}{C:/laragon/www/newmc/assets/imgsvg/}

\definecolor{mcgris}{RGB}{220, 220, 220}% ancien~; pour compatibilité
\definecolor{mcbleu}{RGB}{52, 152, 219}
\definecolor{mcvert}{RGB}{125, 194, 70}
\definecolor{mcmauve}{RGB}{154, 0, 215}
\definecolor{mcorange}{RGB}{255, 96, 0}
\definecolor{mcturquoise}{RGB}{0, 153, 153}
\definecolor{mcrouge}{RGB}{255, 0, 0}
\definecolor{mclightvert}{RGB}{205, 234, 190}

\definecolor{gris}{RGB}{220, 220, 220}
\definecolor{bleu}{RGB}{52, 152, 219}
\definecolor{vert}{RGB}{125, 194, 70}
\definecolor{mauve}{RGB}{154, 0, 215}
\definecolor{orange}{RGB}{255, 96, 0}
\definecolor{turquoise}{RGB}{0, 153, 153}
\definecolor{rouge}{RGB}{255, 0, 0}
\definecolor{lightvert}{RGB}{205, 234, 190}
\setitemize[0]{label=\color{lightvert}  $\bullet$}

\pagestyle{fancy}
\renewcommand{\headrulewidth}{0.2pt}
\fancyhead[L]{maths-cours.fr}
\fancyhead[R]{\thepage}
\renewcommand{\footrulewidth}{0.2pt}
\fancyfoot[C]{}

\newcolumntype{C}{>{\centering\arraybackslash}X}
\newcolumntype{s}{>{\hsize=.35\hsize\arraybackslash}X}

\setlength{\parindent}{0pt}		 
\setlength{\parskip}{3mm}
\setlength{\headheight}{1cm}

\def\ebook{ebook}
\def\book{book}
\def\web{web}
\def\type{web}

\newcommand{\vect}[1]{\overrightarrow{\,\mathstrut#1\,}}

\def\Oij{$\left(\text{O}~;~\vect{\imath},~\vect{\jmath}\right)$}
\def\Oijk{$\left(\text{O}~;~\vect{\imath},~\vect{\jmath},~\vect{k}\right)$}
\def\Ouv{$\left(\text{O}~;~\vect{u},~\vect{v}\right)$}

\hypersetup{breaklinks=true, colorlinks = true, linkcolor = OliveGreen, urlcolor = OliveGreen, citecolor = OliveGreen, pdfauthor={Didier BONNEL - https://www.maths-cours.fr} } % supprime les bordures autour des liens

\renewcommand{\arg}[0]{\text{arg}}

\everymath{\displaystyle}

%================================================================================================================================
%
% Macros - Commandes
%
%================================================================================================================================

\newcommand\meta[2]{    			% Utilisé pour créer le post HTML.
	\def\titre{titre}
	\def\url{url}
	\def\arg{#1}
	\ifx\titre\arg
		\newcommand\maintitle{#2}
		\fancyhead[L]{#2}
		{\Large\sffamily \MakeUppercase{#2}}
		\vspace{1mm}\textcolor{mcvert}{\hrule}
	\fi 
	\ifx\url\arg
		\fancyfoot[L]{\href{https://www.maths-cours.fr#2}{\black \footnotesize{https://www.maths-cours.fr#2}}}
	\fi 
}


\newcommand\TitreC[1]{    		% Titre centré
     \needspace{3\baselineskip}
     \begin{center}\textbf{#1}\end{center}
}

\newcommand\newpar{    		% paragraphe
     \par
}

\newcommand\nosp {    		% commande vide (pas d'espace)
}
\newcommand{\id}[1]{} %ignore

\newcommand\boite[2]{				% Boite simple sans titre
	\vspace{5mm}
	\setlength{\fboxrule}{0.2mm}
	\setlength{\fboxsep}{5mm}	
	\fcolorbox{#1}{#1!3}{\makebox[\linewidth-2\fboxrule-2\fboxsep]{
  		\begin{minipage}[t]{\linewidth-2\fboxrule-4\fboxsep}\setlength{\parskip}{3mm}
  			 #2
  		\end{minipage}
	}}
	\vspace{5mm}
}

\newcommand\CBox[4]{				% Boites
	\vspace{5mm}
	\setlength{\fboxrule}{0.2mm}
	\setlength{\fboxsep}{5mm}
	
	\fcolorbox{#1}{#1!3}{\makebox[\linewidth-2\fboxrule-2\fboxsep]{
		\begin{minipage}[t]{1cm}\setlength{\parskip}{3mm}
	  		\textcolor{#1}{\LARGE{#2}}    
 	 	\end{minipage}  
  		\begin{minipage}[t]{\linewidth-2\fboxrule-4\fboxsep}\setlength{\parskip}{3mm}
			\raisebox{1.2mm}{\normalsize\sffamily{\textcolor{#1}{#3}}}						
  			 #4
  		\end{minipage}
	}}
	\vspace{5mm}
}

\newcommand\cadre[3]{				% Boites convertible html
	\par
	\vspace{2mm}
	\setlength{\fboxrule}{0.1mm}
	\setlength{\fboxsep}{5mm}
	\fcolorbox{#1}{white}{\makebox[\linewidth-2\fboxrule-2\fboxsep]{
  		\begin{minipage}[t]{\linewidth-2\fboxrule-4\fboxsep}\setlength{\parskip}{3mm}
			\raisebox{-2.5mm}{\sffamily \small{\textcolor{#1}{\MakeUppercase{#2}}}}		
			\par		
  			 #3
 	 		\end{minipage}
	}}
		\vspace{2mm}
	\par
}

\newcommand\bloc[3]{				% Boites convertible html sans bordure
     \needspace{2\baselineskip}
     {\sffamily \small{\textcolor{#1}{\MakeUppercase{#2}}}}    
		\par		
  			 #3
		\par
}

\newcommand\CHelp[1]{
     \CBox{Plum}{\faInfoCircle}{À RETENIR}{#1}
}

\newcommand\CUp[1]{
     \CBox{NavyBlue}{\faThumbsOUp}{EN PRATIQUE}{#1}
}

\newcommand\CInfo[1]{
     \CBox{Sepia}{\faArrowCircleRight}{REMARQUE}{#1}
}

\newcommand\CRedac[1]{
     \CBox{PineGreen}{\faEdit}{BIEN R\'EDIGER}{#1}
}

\newcommand\CError[1]{
     \CBox{Red}{\faExclamationTriangle}{ATTENTION}{#1}
}

\newcommand\TitreExo[2]{
\needspace{4\baselineskip}
 {\sffamily\large EXERCICE #1\ (\emph{#2 points})}
\vspace{5mm}
}

\newcommand\img[2]{
          \includegraphics[width=#2\paperwidth]{\imgdir#1}
}

\newcommand\imgsvg[2]{
       \begin{center}   \includegraphics[width=#2\paperwidth]{\imgsvgdir#1} \end{center}
}


\newcommand\Lien[2]{
     \href{#1}{#2 \tiny \faExternalLink}
}
\newcommand\mcLien[2]{
     \href{https~://www.maths-cours.fr/#1}{#2 \tiny \faExternalLink}
}

\newcommand{\euro}{\eurologo{}}

%================================================================================================================================
%
% Macros - Environement
%
%================================================================================================================================

\newenvironment{tex}{ %
}
{%
}

\newenvironment{indente}{ %
	\setlength\parindent{10mm}
}

{
	\setlength\parindent{0mm}
}

\newenvironment{corrige}{%
     \needspace{3\baselineskip}
     \medskip
     \textbf{\textsc{Corrigé}}
     \medskip
}
{
}

\newenvironment{extern}{%
     \begin{center}
     }
     {
     \end{center}
}

\NewEnviron{code}{%
	\par
     \boite{gray}{\texttt{%
     \BODY
     }}
     \par
}

\newenvironment{vbloc}{% boite sans cadre empeche saut de page
     \begin{minipage}[t]{\linewidth}
     }
     {
     \end{minipage}
}
\NewEnviron{h2}{%
    \needspace{3\baselineskip}
    \vspace{0.6cm}
	\noindent \MakeUppercase{\sffamily \large \BODY}
	\vspace{1mm}\textcolor{mcgris}{\hrule}\vspace{0.4cm}
	\par
}{}

\NewEnviron{h3}{%
    \needspace{3\baselineskip}
	\vspace{5mm}
	\textsc{\BODY}
	\par
}

\NewEnviron{margeneg}{ %
\begin{addmargin}[-1cm]{0cm}
\BODY
\end{addmargin}
}

\NewEnviron{html}{%
}

\begin{document}
\meta{url}{/exercices/qcm-fonctions-bac-blanc-es-l-sujet-5-maths-cours-2018/}
\meta{pid}{10552}
\meta{titre}{QCM Fonctions - Bac blanc ES/L Sujet 5 - Maths-cours 2018}
\meta{type}{exercices}
%
\begin{h2}Exercice 2 (5 points)\end{h2}
\par
\emph{Pour chacune des cinq affirmations suivantes, indiquer si elle est vraie ou fausse en justifiant la réponse, soit à l'aide du graphique, soit par un calcul.\\ Il est attribué un point par réponse exacte correctement justifiée.\\ \textbf{Une réponse non justifiée n'est pas prise en compte.}}\index{Vrai--Faux}
\par
Soit la fonction $f$ définie sur l'intervalle $]0~;~5]$ par :
\par
\[f(x) =  \ln x - x + 3.  \]
\par
On a tracé ci-après la courbe représentative $\mathscr{C}_f$ de la fonction $f$, ainsi que $\mathscr{D}$, la tangente à la courbe $\mathscr{C}_f$ au point $A$ de coordonnées $(1~;~2)$.\\ Cette tangente est parallèle à l'axe des abscisses.
\par
On note respectivement $f'$ et $f''$ la dérivée et la dérivée seconde de la fonction $f$.
\par
\begin{center}
     \begin{extern}%width="500" alt="Courbe fonction et tangente"
          \includegraphics[width=0.9\textwidth]{images/BBESL-s5-2-1}% gbb 1 unite=3cm
     \end{extern}
\end{center}
\par
\begin{itemize}
     \item %1
     \textbf{Affirmation 1 :}\quad $f'(1)=0$.
     \item %2
     \textbf{Affirmation 2 :}\quad Pour tout réel $x$ de l'intervalle $]0~;~5]$, $f''(x)>0$.
     \item %3
     \textbf{Affirmation 3 :}\quad La courbe $\mathscr{C}_f$ possède un et un seul point d'inflexion.
     \item %4
     \textbf{Affirmation 4 :}\quad $1 \leqslant \displaystyle\int_{1}^{2}f(x)\text{d}x \leqslant 2$.
     \item %5
     \textbf{Affirmation 5 :}\quad La fonction $F$ définie sur l'intervalle $]0~;~5]$ par
     \par
     \[F(x) =  x\ln x - \dfrac{x^2}{2} + 3x  \]
     \par
     est une primitive de la fonction $f$ sur l'intervalle $]0~;~5]$.
     \par
\end{itemize}
\begin{corrige}
     \begin{itemize}
          \item %1
          \textbf{Affirmation 1 :}\quad $f'(1)=0$ : \textbf{VRAI}.
          \par
          \begin{itemize}
               \item
               \textbf{Méthode 1 : \`A l'aide du graphique}
               \par
               La tangente $\mathscr{D}$ à la courbe $\mathscr{C_f}$ au point $A$ est parallèle à l'axe des abscisses.
               \par
               $f'(1)$ est le coefficient directeur de $\mathscr{D}$ donc $f'(1)=0$.
               \par
               \cadre{rouge}{À retenir}{
                    Le \textbf{coefficient directeur de la tangente} à la courbe représentative de $f$ au point d'\textbf{abscisse} $\alpha$ est égal à $f'(\alpha)$.
                    \par
                    Si cette tangente est parallèle à l'axe des abscisses, $f'(\alpha) = 0$.
               }
               \vspace{0.5cm}
               \item
               \textbf{Méthode 2 : Par le calcul}
               \par
               $f$ est dérivable sur l'intervalle $]0~;~5]$ comme somme de fonctions dérivables et :
               \par
               $f'(x)=\dfrac{1}{x}-1$.
               \par
               Par conséquent :
               \par
               $f'(1)=\dfrac{1}{1}-1=0$.
               \par
          \end{itemize}
          \item %2
          \textbf{Affirmation 2 :}\quad Pour tout réel $x$ de l'intervalle $]0~;~5]$, $f''(x)>0$ : \textbf{FAUX}.
          \par
          Pour tout réel $x$ de l'intervalle $]0~;~5]$, $f'(x)=\dfrac{1}{x}-1$ (voir question précédente). $f'$ est dérivable sur $]0~;~5]$ et :
          \par
          $f''(x)=-\dfrac{1}{x^2}.$
          \par
          $f''$ est strictement négative sur l'intervalle $]0~;~5]$, donc la proposition est fausse.
          \par
          \cadre{bleu}{Remarque}{
               Une autre possibilité consiste à dire que la courbe $\mathscr{C_f}$ est située \textbf{au-dessous} de ses tangentes donc que la fonction $f$ est \textbf{concave}.
               \par
               Cette méthode est toutefois moins rigoureuse ici, notamment, parce qu'au voisinage de 0, la courbe sort de l'image (par le bas) et qu'il est alors difficile de visualiser la position de la courbe et de ses tangentes...
          }
          \item %3
          \textbf{Affirmation 3 :}\quad La courbe $\mathscr{C}_f$ possède un et un seul point d'inflexion : \textbf{FAUX}.
          \par
          $f$ étant deux fois dérivable, la courbe $\mathscr{C_f}$ possède un point d'inflexion sur l'intervalle $]0~;~5]$ si et seulement $f''$ \textbf{s'annule et change de signe} en ce point. Or, d'après la question précédente, pour tout réel $x$ de l'intervalle $]0~;~5]$, $f''(x)<0$ donc la courbe $\mathscr{C_f}$ ne possède pas de point d'inflexion.
          \item %4
          \textbf{Affirmation 4 :}\quad $1 \leqslant \displaystyle\int_{1}^{2}f(x)\text{d}x \leqslant 2$ : \textbf{VRAI}.
          \par
          La fonction $f$ étant positive sur l'intervalle $]0~;~5]$, l'intégrale $\displaystyle\int_{1}^{2}f(x)\text{d}x$ est égale à l'aire, exprimée en unités d'aire, du domaine délimité par la courbe $\mathscr{C}_f$, l'axe des abscisses et les droites d'équations $x=1$ et $x=2$.
          \par
          Or, sur la figure, l'unité d'aire correspond à un carré du quadrillage. On voit donc facilement que :
          \[ 1 \leqslant \displaystyle\int_{1}^{2}f(x)\text{d}x \leqslant 2. \]
          \item %5
          \textbf{Affirmation 5 :}\quad La fonction $F$ définie sur l'intervalle $]0~;~5]$ par $F(x) =  x\ln x - \dfrac{x^2}{2} + 3x $ est une primitive de la fonction $f$ sur l'intervalle $]0~;~5]$ : \textbf{FAUX}.
          \par
          Pour tout réel $x$ strictement positif, posons :
          \[ u(x)=x \qquad \text{et} \qquad v(x)=\ln x. \]
          On en déduit :
          \[ u'(x)=1 \qquad \text{et} \qquad  v'(x)=\dfrac{1}{x}.\]
          Par conséquent :
          \par
          $(uv)'(x)=u'(x)v(x)+u(x)v'(x)=\ln x+x \times \dfrac{1}{x}=\ln x + 1$
          \par
          et :
          \par
          $F'(x)=\ln x + 1 - \dfrac{2x}{2} + 3 = \ln x - x + 4$.
          \par
          La fonction $F'$ est différente de la fonction $f$ donc $F$ \textbf{n'est pas} une primitive de la fonction $f$ sur l'intervalle $]0~;~5]$.
          \par
     \end{itemize}
\end{corrige}

\end{document}
µ
\documentclass[a4paper]{article}

%================================================================================================================================
%
% Packages
%
%================================================================================================================================

\usepackage[T1]{fontenc} 	% pour caractères accentués
\usepackage[utf8]{inputenc}  % encodage utf8
\usepackage[french]{babel}	% langue : français
\usepackage{fourier}			% caractères plus lisibles
\usepackage[dvipsnames]{xcolor} % couleurs
\usepackage{fancyhdr}		% réglage header footer
\usepackage{needspace}		% empêcher sauts de page mal placés
\usepackage{graphicx}		% pour inclure des graphiques
\usepackage{enumitem,cprotect}		% personnalise les listes d'items (nécessaire pour ol, al ...)
\usepackage{hyperref}		% Liens hypertexte
\usepackage{pstricks,pst-all,pst-node,pstricks-add,pst-math,pst-plot,pst-tree,pst-eucl} % pstricks
\usepackage[a4paper,includeheadfoot,top=2cm,left=3cm, bottom=2cm,right=3cm]{geometry} % marges etc.
\usepackage{comment}			% commentaires multilignes
\usepackage{amsmath,environ} % maths (matrices, etc.)
\usepackage{amssymb,makeidx}
\usepackage{bm}				% bold maths
\usepackage{tabularx}		% tableaux
\usepackage{colortbl}		% tableaux en couleur
\usepackage{fontawesome}		% Fontawesome
\usepackage{environ}			% environment with command
\usepackage{fp}				% calculs pour ps-tricks
\usepackage{multido}			% pour ps tricks
\usepackage[np]{numprint}	% formattage nombre
\usepackage{tikz,tkz-tab} 			% package principal TikZ
\usepackage{pgfplots}   % axes
\usepackage{mathrsfs}    % cursives
\usepackage{calc}			% calcul taille boites
\usepackage[scaled=0.875]{helvet} % font sans serif
\usepackage{svg} % svg
\usepackage{scrextend} % local margin
\usepackage{scratch} %scratch
\usepackage{multicol} % colonnes
%\usepackage{infix-RPN,pst-func} % formule en notation polanaise inversée
\usepackage{listings}

%================================================================================================================================
%
% Réglages de base
%
%================================================================================================================================

\lstset{
language=Python,   % R code
literate=
{á}{{\'a}}1
{à}{{\`a}}1
{ã}{{\~a}}1
{é}{{\'e}}1
{è}{{\`e}}1
{ê}{{\^e}}1
{í}{{\'i}}1
{ó}{{\'o}}1
{õ}{{\~o}}1
{ú}{{\'u}}1
{ü}{{\"u}}1
{ç}{{\c{c}}}1
{~}{{ }}1
}


\definecolor{codegreen}{rgb}{0,0.6,0}
\definecolor{codegray}{rgb}{0.5,0.5,0.5}
\definecolor{codepurple}{rgb}{0.58,0,0.82}
\definecolor{backcolour}{rgb}{0.95,0.95,0.92}

\lstdefinestyle{mystyle}{
    backgroundcolor=\color{backcolour},   
    commentstyle=\color{codegreen},
    keywordstyle=\color{magenta},
    numberstyle=\tiny\color{codegray},
    stringstyle=\color{codepurple},
    basicstyle=\ttfamily\footnotesize,
    breakatwhitespace=false,         
    breaklines=true,                 
    captionpos=b,                    
    keepspaces=true,                 
    numbers=left,                    
xleftmargin=2em,
framexleftmargin=2em,            
    showspaces=false,                
    showstringspaces=false,
    showtabs=false,                  
    tabsize=2,
    upquote=true
}

\lstset{style=mystyle}


\lstset{style=mystyle}
\newcommand{\imgdir}{C:/laragon/www/newmc/assets/imgsvg/}
\newcommand{\imgsvgdir}{C:/laragon/www/newmc/assets/imgsvg/}

\definecolor{mcgris}{RGB}{220, 220, 220}% ancien~; pour compatibilité
\definecolor{mcbleu}{RGB}{52, 152, 219}
\definecolor{mcvert}{RGB}{125, 194, 70}
\definecolor{mcmauve}{RGB}{154, 0, 215}
\definecolor{mcorange}{RGB}{255, 96, 0}
\definecolor{mcturquoise}{RGB}{0, 153, 153}
\definecolor{mcrouge}{RGB}{255, 0, 0}
\definecolor{mclightvert}{RGB}{205, 234, 190}

\definecolor{gris}{RGB}{220, 220, 220}
\definecolor{bleu}{RGB}{52, 152, 219}
\definecolor{vert}{RGB}{125, 194, 70}
\definecolor{mauve}{RGB}{154, 0, 215}
\definecolor{orange}{RGB}{255, 96, 0}
\definecolor{turquoise}{RGB}{0, 153, 153}
\definecolor{rouge}{RGB}{255, 0, 0}
\definecolor{lightvert}{RGB}{205, 234, 190}
\setitemize[0]{label=\color{lightvert}  $\bullet$}

\pagestyle{fancy}
\renewcommand{\headrulewidth}{0.2pt}
\fancyhead[L]{maths-cours.fr}
\fancyhead[R]{\thepage}
\renewcommand{\footrulewidth}{0.2pt}
\fancyfoot[C]{}

\newcolumntype{C}{>{\centering\arraybackslash}X}
\newcolumntype{s}{>{\hsize=.35\hsize\arraybackslash}X}

\setlength{\parindent}{0pt}		 
\setlength{\parskip}{3mm}
\setlength{\headheight}{1cm}

\def\ebook{ebook}
\def\book{book}
\def\web{web}
\def\type{web}

\newcommand{\vect}[1]{\overrightarrow{\,\mathstrut#1\,}}

\def\Oij{$\left(\text{O}~;~\vect{\imath},~\vect{\jmath}\right)$}
\def\Oijk{$\left(\text{O}~;~\vect{\imath},~\vect{\jmath},~\vect{k}\right)$}
\def\Ouv{$\left(\text{O}~;~\vect{u},~\vect{v}\right)$}

\hypersetup{breaklinks=true, colorlinks = true, linkcolor = OliveGreen, urlcolor = OliveGreen, citecolor = OliveGreen, pdfauthor={Didier BONNEL - https://www.maths-cours.fr} } % supprime les bordures autour des liens

\renewcommand{\arg}[0]{\text{arg}}

\everymath{\displaystyle}

%================================================================================================================================
%
% Macros - Commandes
%
%================================================================================================================================

\newcommand\meta[2]{    			% Utilisé pour créer le post HTML.
	\def\titre{titre}
	\def\url{url}
	\def\arg{#1}
	\ifx\titre\arg
		\newcommand\maintitle{#2}
		\fancyhead[L]{#2}
		{\Large\sffamily \MakeUppercase{#2}}
		\vspace{1mm}\textcolor{mcvert}{\hrule}
	\fi 
	\ifx\url\arg
		\fancyfoot[L]{\href{https://www.maths-cours.fr#2}{\black \footnotesize{https://www.maths-cours.fr#2}}}
	\fi 
}


\newcommand\TitreC[1]{    		% Titre centré
     \needspace{3\baselineskip}
     \begin{center}\textbf{#1}\end{center}
}

\newcommand\newpar{    		% paragraphe
     \par
}

\newcommand\nosp {    		% commande vide (pas d'espace)
}
\newcommand{\id}[1]{} %ignore

\newcommand\boite[2]{				% Boite simple sans titre
	\vspace{5mm}
	\setlength{\fboxrule}{0.2mm}
	\setlength{\fboxsep}{5mm}	
	\fcolorbox{#1}{#1!3}{\makebox[\linewidth-2\fboxrule-2\fboxsep]{
  		\begin{minipage}[t]{\linewidth-2\fboxrule-4\fboxsep}\setlength{\parskip}{3mm}
  			 #2
  		\end{minipage}
	}}
	\vspace{5mm}
}

\newcommand\CBox[4]{				% Boites
	\vspace{5mm}
	\setlength{\fboxrule}{0.2mm}
	\setlength{\fboxsep}{5mm}
	
	\fcolorbox{#1}{#1!3}{\makebox[\linewidth-2\fboxrule-2\fboxsep]{
		\begin{minipage}[t]{1cm}\setlength{\parskip}{3mm}
	  		\textcolor{#1}{\LARGE{#2}}    
 	 	\end{minipage}  
  		\begin{minipage}[t]{\linewidth-2\fboxrule-4\fboxsep}\setlength{\parskip}{3mm}
			\raisebox{1.2mm}{\normalsize\sffamily{\textcolor{#1}{#3}}}						
  			 #4
  		\end{minipage}
	}}
	\vspace{5mm}
}

\newcommand\cadre[3]{				% Boites convertible html
	\par
	\vspace{2mm}
	\setlength{\fboxrule}{0.1mm}
	\setlength{\fboxsep}{5mm}
	\fcolorbox{#1}{white}{\makebox[\linewidth-2\fboxrule-2\fboxsep]{
  		\begin{minipage}[t]{\linewidth-2\fboxrule-4\fboxsep}\setlength{\parskip}{3mm}
			\raisebox{-2.5mm}{\sffamily \small{\textcolor{#1}{\MakeUppercase{#2}}}}		
			\par		
  			 #3
 	 		\end{minipage}
	}}
		\vspace{2mm}
	\par
}

\newcommand\bloc[3]{				% Boites convertible html sans bordure
     \needspace{2\baselineskip}
     {\sffamily \small{\textcolor{#1}{\MakeUppercase{#2}}}}    
		\par		
  			 #3
		\par
}

\newcommand\CHelp[1]{
     \CBox{Plum}{\faInfoCircle}{À RETENIR}{#1}
}

\newcommand\CUp[1]{
     \CBox{NavyBlue}{\faThumbsOUp}{EN PRATIQUE}{#1}
}

\newcommand\CInfo[1]{
     \CBox{Sepia}{\faArrowCircleRight}{REMARQUE}{#1}
}

\newcommand\CRedac[1]{
     \CBox{PineGreen}{\faEdit}{BIEN R\'EDIGER}{#1}
}

\newcommand\CError[1]{
     \CBox{Red}{\faExclamationTriangle}{ATTENTION}{#1}
}

\newcommand\TitreExo[2]{
\needspace{4\baselineskip}
 {\sffamily\large EXERCICE #1\ (\emph{#2 points})}
\vspace{5mm}
}

\newcommand\img[2]{
          \includegraphics[width=#2\paperwidth]{\imgdir#1}
}

\newcommand\imgsvg[2]{
       \begin{center}   \includegraphics[width=#2\paperwidth]{\imgsvgdir#1} \end{center}
}


\newcommand\Lien[2]{
     \href{#1}{#2 \tiny \faExternalLink}
}
\newcommand\mcLien[2]{
     \href{https~://www.maths-cours.fr/#1}{#2 \tiny \faExternalLink}
}

\newcommand{\euro}{\eurologo{}}

%================================================================================================================================
%
% Macros - Environement
%
%================================================================================================================================

\newenvironment{tex}{ %
}
{%
}

\newenvironment{indente}{ %
	\setlength\parindent{10mm}
}

{
	\setlength\parindent{0mm}
}

\newenvironment{corrige}{%
     \needspace{3\baselineskip}
     \medskip
     \textbf{\textsc{Corrigé}}
     \medskip
}
{
}

\newenvironment{extern}{%
     \begin{center}
     }
     {
     \end{center}
}

\NewEnviron{code}{%
	\par
     \boite{gray}{\texttt{%
     \BODY
     }}
     \par
}

\newenvironment{vbloc}{% boite sans cadre empeche saut de page
     \begin{minipage}[t]{\linewidth}
     }
     {
     \end{minipage}
}
\NewEnviron{h2}{%
    \needspace{3\baselineskip}
    \vspace{0.6cm}
	\noindent \MakeUppercase{\sffamily \large \BODY}
	\vspace{1mm}\textcolor{mcgris}{\hrule}\vspace{0.4cm}
	\par
}{}

\NewEnviron{h3}{%
    \needspace{3\baselineskip}
	\vspace{5mm}
	\textsc{\BODY}
	\par
}

\NewEnviron{margeneg}{ %
\begin{addmargin}[-1cm]{0cm}
\BODY
\end{addmargin}
}

\NewEnviron{html}{%
}

\begin{document}
\meta{url}{/exercices/fonctions-et-integrales-bac-blanc-es-l-sujet-5-maths-cours-2018/}
\meta{pid}{10554}
\meta{titre}{Fonctions et intégrales - Bac blanc ES/L Sujet 5 - Maths-cours 2018}
\meta{type}{exercices}
%
\begin{h2}Exercice 3 (6 points)\end{h2}
\par
Une entreprise fabrique et commercialise des VTT (vélos tout terrain).
\par
Les bénéfices (ou les pertes) mensuels de cette entreprise, en centaines d'euros,  peuvent être modélisés par la fonction $f$ définie sur l'intervalle $[0~;~5]$ par :
\[ f(x)=x \text{e}^{x}-2 \text{e}^{x} -6. \]
où $x$ représente le nombre de vélos, en milliers, vendus mensuellement.
\par
Si $f(x)$ est positif, l'entreprise réalise un bénéfice et dans le cas contraire, elle subit une perte.
\par
\begin{enumerate}
     \item
     On note $f'$ la fonction dérivée de la fonction $f$ sur l'intervalle $[0~;~5]$.
     \par
     Montrer que pour tout réel $x \in [0~;~5] $ :
     \[ f'(x)=(x-1) \text{e}^{x}. \]
     \item %2
     Dresser le tableau de variations de $f$ sur l'intervalle $[0~;~5]$.
     \item %3
     \begin{enumerate}[label=\alph*.]
          \item %3a
          Montrer l'équation $f(x)=0$ admet une unique solution $x_0$ sur l'intervalle $[0~;~5]$.
          \item %3b
          Donner une valeur approchée de $x_0$ à $10^{-3}$ près.
          \item %3c
          \'Etudier le signe de la fonction $f$ sur l'intervalle $[0~;~5]$.
          \item %3d
          Combien de vélos, au minimum, l'entreprise doit-elle vendre pour réaliser des bénéfices ?
          \par
     \end{enumerate}
     \item %4
     \begin{enumerate}[label=\alph*.]
          \item %4a
          Pour tout réel $x$ appartenant à l'intervalle $[0~;~5]$, on pose :
          \[ g(x)=(x-3)\text{e}^{x}. \]
          \par
          Montrer que, sur l'intervalle $[0~;~5]$, $g'(x)=x\text{e}^{x}-2\text{e}^{x}$.
          \item %4b
          En déduire une primitive $F$ de la fonction $f$ sur l'intervalle $[0~;~5]$.
          \item %4c
          Donner une valeur approchée, à l'unité près, de l'intégrale :
          \[ I=\displaystyle\int_{4}^{5} f(t)\text{d}t. \]
          \item %4d
          L'entreprise vend régulièrement entre 4\ 000 et 5\ 000 vélos par mois.
          \par
          Estimer la valeur moyenne du bénéfice mensuel arrondie à la centaine d'euros.
          \par
     \end{enumerate}
     \par
\end{enumerate}
\begin{corrige}
     \begin{enumerate}
          \item %1
          Sur l'intervalle $[0~;~5]$, $f$ est la somme des fonctions $h$ et $k$ définies par :
          \[ h(x)=x \text{e}^{x} \quad \text{et} \quad k(x)=-2 \text{e}^{x} -6. \]
          \par
          Pour dériver $h$, posons :
          \begin{center}
               $u(x)=x \quad  $\nosp$\quad v(x)=\text{e}^{x}. $ \\
               $u'(x)=1 \quad$\nosp$\quad v'(x)=\text{e}^{x}.$
          \end{center}
          Alors :
          \par
          $h'(x)=u'(x)v(x)+u(x)v'(x)$ \\
          $	\phantom{h'(x)}=1 \times \text{e}^{x} + x \times \text{e}^{x} $\\
          $		\phantom{h'(x)}=\text{e}^{x} + x \text{e}^{x}.$
          \par
          La dérivée de $k$ est définie sur l'intervalle $[0~;~5]$ par :
          \par
          $k'(x)=-2 \text{e}^{x}.$
          \par
          Donc, pour tout réel $x$ appartenant à l'intervalle $[0~;~5]$ :
          \par
          $f'(x)=\text{e}^{x} + x \text{e}^{x}-2 \text{e}^{x} $\\
          $\phantom{	f'(x)} =x \text{e}^{x} - \text{e}^{x} $\\
          $	\phantom{f'(x)} =\text{e}^{x} (x-1) $ après factorisation de $ \text{e}^{x}.$
          \par
          \cadre{vert}{En pratique}{
               Il est, en général, préférable de factoriser $f'(x)$. Cela facilite ensuite l'étude du signe de la dérivée.
          }
          \item %2
          $\text{e}^{x}$ est strictement positif pour tout réel $x$, donc la fonction $f'$ est du signe de $x-1$, c'est à dire nulle pour ${x=1}$, strictement positive pour ${x > 1}$ et strictement négative pour ${x < 1}$.
          \par
          On calcule :
          \par
          $f(0)=-8$ ;\\
          $f(1)=-\text{e}-6$ ;\\
          $f(5)=3 \text{e}^{5}-6$.
          \par
          On obtient alors le tableau de variations suivant :
          \par
          %:-+-+-+-+- Engendré par : http://math.et.info.free.fr/TikZ/TableauxVariations/
          \begin{center}
               \begin{extern}%width="360" alt="Tableau de variations de f"
                    \begin{tikzpicture}[scale=0.875]
                         % Styles
                         \tikzstyle{cadre}=[thin]
                         \tikzstyle{fleche}=[->,>=latex,thin]
                         \tikzstyle{nondefini}=[lightgray]
                         % Dimensions Modifiables
                         \def\Lrg{1.5}
                         \def\HtX{1}
                         \def\HtY{0.5}
                         % Dimensions Calculées
                         \def\lignex{-0.5*\HtX}
                         \def\lignef{-1.5*\HtX}
                         \def\separateur{-0.5*\Lrg}
                         % Largeur du tableau
                         \def\gauche{-1.5*\Lrg}
                         \def\droite{4.7*\Lrg}
                         % Hauteur du tableau
                         \def\haut{0.5*\HtX}
                         \def\bas{-2.5*\HtX-2*\HtY}
                         % Ligne de l'abscisse : x
                         \node at (-1*\Lrg,0) {$x$};
                         \node at (0*\Lrg,0) {$0$};
                         \node at (2*\Lrg,0) {$1$};
                         \node at (4*\Lrg,0) {$5$};
                         % Ligne de la dérivée : f'(x)
                         \node at (-1*\Lrg,-1*\HtX) {$f'(x)$};
                         \node at (0*\Lrg,-1*\HtX) {$\ $};
                         \node at (1*\Lrg,-1*\HtX) {$-$};
                         \node at (2*\Lrg,-1*\HtX) {$0$};
                         \node at (3*\Lrg,-1*\HtX) {$+$};
                         \node at (4*\Lrg,-1*\HtX) {$\ $};
                         % Ligne de la fonction : f(x)
                         \node  at (-1*\Lrg,{-2*\HtX+(-1)*\HtY}) {$f(x)$};
                         \node (f1) at (0*\Lrg,{-2*\HtX+(0)*\HtY}) {$-8$};
                         \node (f2) at (2*\Lrg,{-2*\HtX+(-2)*\HtY}) {$-\text{e}-6$};
                         \node (f3) at (4*\Lrg,{-2*\HtX+(0)*\HtY}) {$3\text{e}^5-6$};
                         % Flèches
                         \draw[fleche] (f1) -- (f2);
                         \draw[fleche] (f2) -- (f3);
                         % Encadrement
                         \draw[cadre] (\separateur,\haut) -- (\separateur,\bas);
                         \draw[cadre] (\gauche,\haut) rectangle  (\droite,\bas);
                         \draw[cadre] (\gauche,\lignex) -- (\droite,\lignex);
                         \draw[cadre] (\gauche,\lignef) -- (\droite,\lignef);
                    \end{tikzpicture}
               \end{extern}
          \end{center}
          %:-+-+-+-+- Fin
          \par
          %:>>>>> code du tableau à ré-injecter
          %[
          %	["x", "f'(x)", "f(x)"],
          %	["0", "", "-", "-8"],
          %	["1", "0", "+", "-\\text{e}-6"],
          %	["5", "", "?", "3\\text{e}^5-6"]
          %]
          \item %3
          \begin{enumerate}
               \item %3a
               $f(1)=-\text{e}-6 \approx -8,72 $ ; \\
               $f(5)=3 \text{e}^{5}-6 \approx 439,24. $
               \par
               \begin{itemize}
                    \item
                    Sur l'intervalle $[0~;~1]$, $f$ est inférieure à -8 donc strictement négative. \\
                    L'équation $f(x)=0$ n'a donc pas de solution sur cet intervalle.
                    \item %3b
                    Sur l'intervalle $[1~;~5]$, $f$ est \textbf{continue}, \textbf{strictement croissante} et \textbf{change de signe}. Donc l'équation ${f(x)=0}$ admet une unique solution sur cet intervalle.
               \end{itemize}
               \par
               En résumé, l'équation $f(x)=0$ admet une \textbf{unique solution} $x_0$ sur l'intervalle $[0~;~5]$. Cette solution est comprise entre 1 et 5.
               \item %3c
               \`A la calculatrice, on trouve :
               \par
               $f(2,494) \approx -0,018 < 0$ ;
               \par
               $f(2,495) \approx 2,6 \times 10^{-4} > 0$.
               \par
               Par conséquent :
               \[ 2,494 < x_0 < 2,495. \]
               \par
               Une valeur approchée à $10^{-3}$ près par défaut de $x_0$ est 2,494.
               \par
               \item %3d
               \begin{itemize}
                    \item
                    Sur l'intervalle $[0~;~1]$, $f$ est strictement négative.
                    \item %3b
                    Sur l'intervalle $[1~;~5]$, $f$ est strictement croissante et s'annule pour $x=x_0$.\\
                    Donc $f$ est strictement négative sur l'intervalle $[1~;~x_0[$, s'annule en $x_0$ et est strictement positive sur l'intervalle $]x_0~;~5]$.
                    \par
               \end{itemize}
               \par
               En résumé, $f$ est strictement négative sur l'intervalle $[0~;~x_0[$, s'annule en $x_0$ et est strictement positive sur l'intervalle $]x_0~;~5]$ :
               \par
               %:-+-+-+-+- Engendré par : http://math.et.info.free.fr/TikZ/TableauxVariations/
               \begin{center}
                    \begin{extern}%width="360" alt="Tableau de signes et TVI"
                         \begin{tikzpicture}[scale=0.875]
                              % Styles
                              \tikzstyle{cadre}=[thin]
                              \tikzstyle{fleche}=[->,>=latex,thin]
                              \tikzstyle{nondefini}=[lightgray]
                              % Dimensions Modifiables
                              \def\Lrg{1.5}
                              \def\HtX{1}
                              \def\HtY{0.5}
                              % Dimensions Calculées
                              \def\lignex{-0.5*\HtX}
                              \def\lignef{-1.5*\HtX}
                              \def\separateur{-0.5*\Lrg}
                              % Largeur du tableau
                              \def\gauche{-1.5*\Lrg}
                              \def\droite{4.7*\Lrg}
                              % Hauteur du tableau
                              \def\haut{0.5*\HtX}
                              \def\bas{-0.5*\HtX-2*\HtY}
                              % Ligne de l'abscisse : x
                              \node at (-1*\Lrg,0) {$x$};
                              \node at (0*\Lrg,0) {$0$};
                              \node at (2*\Lrg,0) {$x_0$};
                              \node at (4*\Lrg,0) {$5$};
                              % Ligne de f(x)
                              \node at (-1*\Lrg,-1*\HtX) {$f(x)$};
                              \node at (0*\Lrg,-1*\HtX) {$\ $};
                              \node at (1*\Lrg,-1*\HtX) {$-$};
                              \node at (2*\Lrg,-1*\HtX) {$0$};
                              \node at (3*\Lrg,-1*\HtX) {$+$};
                              \node at (4*\Lrg,-1*\HtX) {$\ $};
                              % Encadrement
                              \draw[cadre] (\separateur,\haut) -- (\separateur,\bas);
                              \draw[cadre] (\gauche,\haut) rectangle  (\droite,\bas);
                              \draw[cadre] (\gauche,\lignex) -- (\droite,\lignex);
                         \end{tikzpicture}
                    \end{extern}
               \end{center}
               %:-+-+-+-+- Fin
               \item %3d
               L'entreprise réalise des bénéfices lorsque la fonction $f$ est positive c'est à dire lorsque $x > x_0$.
               \par
               Comme $x$ représente le nombre de vélos vendus, en milliers, et comme $2,494 < x_0 < 2,495$, l'entreprise doit vendre au minimum 2\ 495 vélos pour réaliser des bénéfices.
               \par
          \end{enumerate}
          \item %4
          \begin{enumerate}
               \item %4a
               Posons :
               \begin{center}
                    $u(x)=x-3 \quad $\nosp$\quad v(x)=\text{e}^{x}. $\\
                    $u'(x)=1 \quad $\nosp$\quad v'(x)=\text{e}^{x}.$
               \end{center}
               Alors :
               \par
               $g'(x)=u'(x)v(x)+u(x)v'(x)$ \\
               $\phantom{g'(x)}=1 \times \text{e}^{x} + (x-3) \times \text{e}^{x} $\\
               $\phantom{g'(x)}=\text{e}^{x} + x\text{e}^{x}-3\text{e}^{x} $\\
               $\phantom{g'(x)}=x\text{e}^{x}-2\text{e}^{x}.$
               \par
               \item %4b
               D'après la question précédente, la fonction $g$ est une primitive de la fonction ${x \longmapsto x\text{e}^{x}-2\text{e}^{x}}$.
               \par
               Par ailleurs, la fonction $x \longmapsto 6x$ est une primitive de la fonction constante ${x \longmapsto 6}$.
               \par
               On en déduit que la fonction $F$ définie sur l'intervalle $[0~;~5]$ par :
               \[ F(x)=(x-3)\text{e}^{x} -6x \]
               est une primitive de la fonction $f$.
               \par
               \cadre{vert}{En pratique}{
                    Il est toujours très important de trouver \textbf{les liens} pouvant exister \textbf{entre les différentes questions}.
                    \par
                    Ici, il faut remarquer que le résultat de la question précédent $g'(x)=x \text{e}^{x} - 2 \text{e}^{x}$ coïncide avec le début de l'expression de $f(x)$. Il reste le terme constant $-6$ pour lequel il est facile de trouver une primitive.
               }
               \item %4c
               Comme la fonction $F$ est une primitive de la fonction $f$ sur l'intervalle $[0~;~5]$ :
               $I=\displaystyle\int_{4}^{5} f(t)\text{d}t = \left[F(t)\right]_4^5$ \\
               $\phantom{l}=(5-3)\text{e}^{5} -6 \times 5 - \left( (4-3)\text{e}^{4} -6 \times 4 \right)$ \\
               $\phantom{l}=2\text{e}^{5} - \text{e}^{4} -6  \approx 236 $ (à l'unité près).
               \par
               \item %4d
               Si l'entreprise vend entre 4 et 5 milliers de vélos par mois de manière régulière, son bénéfice mensuel moyen en centaines d'euros pourra être estimé à :
               \par
               $m=\dfrac{1}{5-4}\displaystyle\int_{4}^{5}f(t)\text{d}t=I  \approx 236 $.
               \par
               Le bénéfice mensuel moyen peut donc être estimé à 23~600 euros (arrondi à la centaine d'euros).
               \par
               \cadre{rouge}{À retenir}{
                    La valeur moyenne de la fonction $f$ sur l'intervalle $[a~;~b]$ est :
                    \[ m=\dfrac{1}{b-a}\displaystyle\int_{a}^{b}f(t)\text{d}t. \]
               }
               \par
          \end{enumerate}
          \par
     \end{enumerate}
\end{corrige}

\end{document}
µ
\documentclass[a4paper]{article}

%================================================================================================================================
%
% Packages
%
%================================================================================================================================

\usepackage[T1]{fontenc} 	% pour caractères accentués
\usepackage[utf8]{inputenc}  % encodage utf8
\usepackage[french]{babel}	% langue : français
\usepackage{fourier}			% caractères plus lisibles
\usepackage[dvipsnames]{xcolor} % couleurs
\usepackage{fancyhdr}		% réglage header footer
\usepackage{needspace}		% empêcher sauts de page mal placés
\usepackage{graphicx}		% pour inclure des graphiques
\usepackage{enumitem,cprotect}		% personnalise les listes d'items (nécessaire pour ol, al ...)
\usepackage{hyperref}		% Liens hypertexte
\usepackage{pstricks,pst-all,pst-node,pstricks-add,pst-math,pst-plot,pst-tree,pst-eucl} % pstricks
\usepackage[a4paper,includeheadfoot,top=2cm,left=3cm, bottom=2cm,right=3cm]{geometry} % marges etc.
\usepackage{comment}			% commentaires multilignes
\usepackage{amsmath,environ} % maths (matrices, etc.)
\usepackage{amssymb,makeidx}
\usepackage{bm}				% bold maths
\usepackage{tabularx}		% tableaux
\usepackage{colortbl}		% tableaux en couleur
\usepackage{fontawesome}		% Fontawesome
\usepackage{environ}			% environment with command
\usepackage{fp}				% calculs pour ps-tricks
\usepackage{multido}			% pour ps tricks
\usepackage[np]{numprint}	% formattage nombre
\usepackage{tikz,tkz-tab} 			% package principal TikZ
\usepackage{pgfplots}   % axes
\usepackage{mathrsfs}    % cursives
\usepackage{calc}			% calcul taille boites
\usepackage[scaled=0.875]{helvet} % font sans serif
\usepackage{svg} % svg
\usepackage{scrextend} % local margin
\usepackage{scratch} %scratch
\usepackage{multicol} % colonnes
%\usepackage{infix-RPN,pst-func} % formule en notation polanaise inversée
\usepackage{listings}

%================================================================================================================================
%
% Réglages de base
%
%================================================================================================================================

\lstset{
language=Python,   % R code
literate=
{á}{{\'a}}1
{à}{{\`a}}1
{ã}{{\~a}}1
{é}{{\'e}}1
{è}{{\`e}}1
{ê}{{\^e}}1
{í}{{\'i}}1
{ó}{{\'o}}1
{õ}{{\~o}}1
{ú}{{\'u}}1
{ü}{{\"u}}1
{ç}{{\c{c}}}1
{~}{{ }}1
}


\definecolor{codegreen}{rgb}{0,0.6,0}
\definecolor{codegray}{rgb}{0.5,0.5,0.5}
\definecolor{codepurple}{rgb}{0.58,0,0.82}
\definecolor{backcolour}{rgb}{0.95,0.95,0.92}

\lstdefinestyle{mystyle}{
    backgroundcolor=\color{backcolour},   
    commentstyle=\color{codegreen},
    keywordstyle=\color{magenta},
    numberstyle=\tiny\color{codegray},
    stringstyle=\color{codepurple},
    basicstyle=\ttfamily\footnotesize,
    breakatwhitespace=false,         
    breaklines=true,                 
    captionpos=b,                    
    keepspaces=true,                 
    numbers=left,                    
xleftmargin=2em,
framexleftmargin=2em,            
    showspaces=false,                
    showstringspaces=false,
    showtabs=false,                  
    tabsize=2,
    upquote=true
}

\lstset{style=mystyle}


\lstset{style=mystyle}
\newcommand{\imgdir}{C:/laragon/www/newmc/assets/imgsvg/}
\newcommand{\imgsvgdir}{C:/laragon/www/newmc/assets/imgsvg/}

\definecolor{mcgris}{RGB}{220, 220, 220}% ancien~; pour compatibilité
\definecolor{mcbleu}{RGB}{52, 152, 219}
\definecolor{mcvert}{RGB}{125, 194, 70}
\definecolor{mcmauve}{RGB}{154, 0, 215}
\definecolor{mcorange}{RGB}{255, 96, 0}
\definecolor{mcturquoise}{RGB}{0, 153, 153}
\definecolor{mcrouge}{RGB}{255, 0, 0}
\definecolor{mclightvert}{RGB}{205, 234, 190}

\definecolor{gris}{RGB}{220, 220, 220}
\definecolor{bleu}{RGB}{52, 152, 219}
\definecolor{vert}{RGB}{125, 194, 70}
\definecolor{mauve}{RGB}{154, 0, 215}
\definecolor{orange}{RGB}{255, 96, 0}
\definecolor{turquoise}{RGB}{0, 153, 153}
\definecolor{rouge}{RGB}{255, 0, 0}
\definecolor{lightvert}{RGB}{205, 234, 190}
\setitemize[0]{label=\color{lightvert}  $\bullet$}

\pagestyle{fancy}
\renewcommand{\headrulewidth}{0.2pt}
\fancyhead[L]{maths-cours.fr}
\fancyhead[R]{\thepage}
\renewcommand{\footrulewidth}{0.2pt}
\fancyfoot[C]{}

\newcolumntype{C}{>{\centering\arraybackslash}X}
\newcolumntype{s}{>{\hsize=.35\hsize\arraybackslash}X}

\setlength{\parindent}{0pt}		 
\setlength{\parskip}{3mm}
\setlength{\headheight}{1cm}

\def\ebook{ebook}
\def\book{book}
\def\web{web}
\def\type{web}

\newcommand{\vect}[1]{\overrightarrow{\,\mathstrut#1\,}}

\def\Oij{$\left(\text{O}~;~\vect{\imath},~\vect{\jmath}\right)$}
\def\Oijk{$\left(\text{O}~;~\vect{\imath},~\vect{\jmath},~\vect{k}\right)$}
\def\Ouv{$\left(\text{O}~;~\vect{u},~\vect{v}\right)$}

\hypersetup{breaklinks=true, colorlinks = true, linkcolor = OliveGreen, urlcolor = OliveGreen, citecolor = OliveGreen, pdfauthor={Didier BONNEL - https://www.maths-cours.fr} } % supprime les bordures autour des liens

\renewcommand{\arg}[0]{\text{arg}}

\everymath{\displaystyle}

%================================================================================================================================
%
% Macros - Commandes
%
%================================================================================================================================

\newcommand\meta[2]{    			% Utilisé pour créer le post HTML.
	\def\titre{titre}
	\def\url{url}
	\def\arg{#1}
	\ifx\titre\arg
		\newcommand\maintitle{#2}
		\fancyhead[L]{#2}
		{\Large\sffamily \MakeUppercase{#2}}
		\vspace{1mm}\textcolor{mcvert}{\hrule}
	\fi 
	\ifx\url\arg
		\fancyfoot[L]{\href{https://www.maths-cours.fr#2}{\black \footnotesize{https://www.maths-cours.fr#2}}}
	\fi 
}


\newcommand\TitreC[1]{    		% Titre centré
     \needspace{3\baselineskip}
     \begin{center}\textbf{#1}\end{center}
}

\newcommand\newpar{    		% paragraphe
     \par
}

\newcommand\nosp {    		% commande vide (pas d'espace)
}
\newcommand{\id}[1]{} %ignore

\newcommand\boite[2]{				% Boite simple sans titre
	\vspace{5mm}
	\setlength{\fboxrule}{0.2mm}
	\setlength{\fboxsep}{5mm}	
	\fcolorbox{#1}{#1!3}{\makebox[\linewidth-2\fboxrule-2\fboxsep]{
  		\begin{minipage}[t]{\linewidth-2\fboxrule-4\fboxsep}\setlength{\parskip}{3mm}
  			 #2
  		\end{minipage}
	}}
	\vspace{5mm}
}

\newcommand\CBox[4]{				% Boites
	\vspace{5mm}
	\setlength{\fboxrule}{0.2mm}
	\setlength{\fboxsep}{5mm}
	
	\fcolorbox{#1}{#1!3}{\makebox[\linewidth-2\fboxrule-2\fboxsep]{
		\begin{minipage}[t]{1cm}\setlength{\parskip}{3mm}
	  		\textcolor{#1}{\LARGE{#2}}    
 	 	\end{minipage}  
  		\begin{minipage}[t]{\linewidth-2\fboxrule-4\fboxsep}\setlength{\parskip}{3mm}
			\raisebox{1.2mm}{\normalsize\sffamily{\textcolor{#1}{#3}}}						
  			 #4
  		\end{minipage}
	}}
	\vspace{5mm}
}

\newcommand\cadre[3]{				% Boites convertible html
	\par
	\vspace{2mm}
	\setlength{\fboxrule}{0.1mm}
	\setlength{\fboxsep}{5mm}
	\fcolorbox{#1}{white}{\makebox[\linewidth-2\fboxrule-2\fboxsep]{
  		\begin{minipage}[t]{\linewidth-2\fboxrule-4\fboxsep}\setlength{\parskip}{3mm}
			\raisebox{-2.5mm}{\sffamily \small{\textcolor{#1}{\MakeUppercase{#2}}}}		
			\par		
  			 #3
 	 		\end{minipage}
	}}
		\vspace{2mm}
	\par
}

\newcommand\bloc[3]{				% Boites convertible html sans bordure
     \needspace{2\baselineskip}
     {\sffamily \small{\textcolor{#1}{\MakeUppercase{#2}}}}    
		\par		
  			 #3
		\par
}

\newcommand\CHelp[1]{
     \CBox{Plum}{\faInfoCircle}{À RETENIR}{#1}
}

\newcommand\CUp[1]{
     \CBox{NavyBlue}{\faThumbsOUp}{EN PRATIQUE}{#1}
}

\newcommand\CInfo[1]{
     \CBox{Sepia}{\faArrowCircleRight}{REMARQUE}{#1}
}

\newcommand\CRedac[1]{
     \CBox{PineGreen}{\faEdit}{BIEN R\'EDIGER}{#1}
}

\newcommand\CError[1]{
     \CBox{Red}{\faExclamationTriangle}{ATTENTION}{#1}
}

\newcommand\TitreExo[2]{
\needspace{4\baselineskip}
 {\sffamily\large EXERCICE #1\ (\emph{#2 points})}
\vspace{5mm}
}

\newcommand\img[2]{
          \includegraphics[width=#2\paperwidth]{\imgdir#1}
}

\newcommand\imgsvg[2]{
       \begin{center}   \includegraphics[width=#2\paperwidth]{\imgsvgdir#1} \end{center}
}


\newcommand\Lien[2]{
     \href{#1}{#2 \tiny \faExternalLink}
}
\newcommand\mcLien[2]{
     \href{https~://www.maths-cours.fr/#1}{#2 \tiny \faExternalLink}
}

\newcommand{\euro}{\eurologo{}}

%================================================================================================================================
%
% Macros - Environement
%
%================================================================================================================================

\newenvironment{tex}{ %
}
{%
}

\newenvironment{indente}{ %
	\setlength\parindent{10mm}
}

{
	\setlength\parindent{0mm}
}

\newenvironment{corrige}{%
     \needspace{3\baselineskip}
     \medskip
     \textbf{\textsc{Corrigé}}
     \medskip
}
{
}

\newenvironment{extern}{%
     \begin{center}
     }
     {
     \end{center}
}

\NewEnviron{code}{%
	\par
     \boite{gray}{\texttt{%
     \BODY
     }}
     \par
}

\newenvironment{vbloc}{% boite sans cadre empeche saut de page
     \begin{minipage}[t]{\linewidth}
     }
     {
     \end{minipage}
}
\NewEnviron{h2}{%
    \needspace{3\baselineskip}
    \vspace{0.6cm}
	\noindent \MakeUppercase{\sffamily \large \BODY}
	\vspace{1mm}\textcolor{mcgris}{\hrule}\vspace{0.4cm}
	\par
}{}

\NewEnviron{h3}{%
    \needspace{3\baselineskip}
	\vspace{5mm}
	\textsc{\BODY}
	\par
}

\NewEnviron{margeneg}{ %
\begin{addmargin}[-1cm]{0cm}
\BODY
\end{addmargin}
}

\NewEnviron{html}{%
}

\begin{document}
\meta{url}{/exercices/suites-et-algorithmes-bac-blanc-es-l-sujet-5-maths-cours-2018/}
\meta{pid}{10556}
\meta{titre}{Suites et algorithmes - Bac blanc ES/L Sujet 5 - Maths-cours 2018}
\meta{type}{exercices}
%
\begin{h2}Exercice 4 (4 points)\end{h2}
\par
On considère la suite géométrique $(u_n)$ de premier terme ${u_0= 1~000}$ et de raison ${q=0,9}$.
\par
\begin{enumerate}
     \item
     Exprimer $u_n$ en fonction de $n$.
     \item
     On pose $S_n=u_0+u_1+u_2+ \cdots +u_n$.
     \par
     Compléter l'algorithme ci-après afin qu'il calcule et affiche la valeur de $S_{10}$.
     \par
     \begin{center}
          \begin{extern}%width="340" alt="Algorithme de calcul de la somme S10"
               \begin{tabular}{|l l|}\hline
                    Variables :	&$I$ est un entier naturel\\
                    &$S$ est un nombre réel\\
                    & \\
                    Initialisation: &$S$ prend la valeur 0\\
                    & \\
                    Traitement: &Pour $I=0$ à $\cdots$ faire :\\
                    &\qquad$S$ prend la valeur $\cdots$\\
                    &Fin Pour\\
                    & \\
                    Sortie :	&Afficher $\cdots$ \\
                    \hline
               \end{tabular}
          \end{extern}
     \end{center}
     \item
     Montrer que, pour tout entier naturel $n$ :
     \[ S_n=10~000 \left( 1-0,9^{n+1} \right). \]
     \par
     En déduire la valeur affichée en sortie de l'algorithme précédent.
     \par
     On arrondira le résultat à l'unité.
     \item
     Quelle est la limite de la somme $S_n$ lorsque $n$ tend vers $+\infty$ ?
     \item
     Déterminer, par la méthode de votre choix, la plus petite valeur de l'entier $n$ telle que :
     \[ S_n > 9~000. \]
     \par
\end{enumerate}
\begin{corrige}
     \begin{enumerate}
          \item
          La suite $(u_n)$ étant une suite géométrique, pour tout entier naturel $n$ :
          \par
          $u_n=u_0 \times q^n=1~000 \times 0,9^n$.
          \item
          L'algorithme calculant et affichant la valeur de $S_{10}$ peut être complété comme suit :
          \par
          \begin{center}
               \begin{extern}%width="420" alt="Algorithme de calcul de S10 complété"
                    \begin{tabular}{|l l|}\hline
                         Variables :	&$I$ est un entier naturel\\
                         &$S$ est un nombre réel\\
                         & \\
                         Initialisation: &$S$ prend la valeur 0\\
                         & \\
                         Traitement: &Pour $I=0$ à \textcolor{red}{10} faire :\\
                         &\qquad$S$ prend la valeur $\textcolor{red}{S+1~000 \times 0,9^I}$\\
                         &Fin Pour\\
                         & \\
                         Sortie :	&Afficher \textcolor{red}{$S$} \\
                         \hline
                    \end{tabular}
               \end{extern}
          \end{center}
          \par
          \cadre{vert}{En pratique}{
               Pour calculer la somme $S=u_0+u_1+ \cdots + u_n$ à l'aide d'un algorithme :
               \par
               \begin{itemize}
                    \item %
                    On \textbf{initialise} $S$ à \textbf{0}.
                    \item %
                    On \textbf{cumule} les termes de la suite $(u_n)$ dans la variable $S$ grâce à une instruction du type :
                    \par
                    Pour $i=0$ à $n$ faire :\\
                    $\phantom{-}S$ prend la valeur $S+$\textit{<formule donnant $u_i$>}\\
                    Fin Pour\\
                    \par
               \end{itemize}
          }
          \item
          D'après la question \textbf{1.} :
          \par
          $u_0=1~000$\\
          $u_1=1~000 \times 0,9 $\\
          $u_2=1~000 \times 0,9^2 $\\
          $\qquad \cdots $\\
          $u_n=1~000 \times 0,9^n $.\\
          \par
          On a alors :
          \par
          $S_n=1~000 + 1~000 \times 0,9 + 1~000 \times 0,9^2 +$\nosp$ \cdots + 1~000 \times 0,9^n$.
          \par
          En mettant 1~000 en facteur, on obtient :
          \par
          $S_n=1~000~\left( 1 + 0,9 + 0,9^2 + \cdots + 0,9^n \right)$.
          \par
          Or :
          \par
          $1+0,9++0,9^{2}+\cdots+0,9^{n}=\dfrac{1-0,9^{n+1}}{1-0,9}$\nosp$=\dfrac{1-0,9^{n+1}}{0,1}=10 \left(1-0,9^{n+1} \right)$.
          \par
          Donc :
          \par
          $S_n=1~000 \times 10 \left(1-0,9^{n+1} \right)$\\
          $\phantom{S_n}=10~000 \left( 1-0,9^{n+1} \right)$.
          \par
          \cadre{rouge}{À retenir}{
               \par
               La formule suivante permet de calculer la somme des premiers termes d'une suite géométrique :
               \par
               \[ 1+q+q^2+\cdots+q^{n}=\dfrac{1-q^{n+1}}{1-q}. \]
               \par
          }
          \par
          L'algorithme précédent affiche la valeur $S_{10}$ c'est à dire :
          \par
          $S_{10}=10~000 \left( 1-0,9^{11} \right) \approx 6~862~$ (arrondi à l'unité).
          \item
          $0 \leqslant 0,9 < 1$ donc $\lim\limits_{n \rightarrow +\infty}0,9 ^n = 0$.
          \par
          Comme $0,9^{n+1} = 0,9 \times 0,9 ^n$, alors ${\lim\limits_{n \rightarrow +\infty}0,9 ^{n+1} = 0}$.
          \par
          Par conséquent :
          \par
          ${\lim\limits_{n \rightarrow +\infty}\left(1-0,9 ^{n+1}\right) = 1}$\quad et \quad${\lim\limits_{n \rightarrow +\infty}\left(10~000 \left( 1-0,9^{n+1} \right)\right) = 10~000}$.
          \par
          La somme $S_n$ tend vers 10~000 lorsque $n$ tend vers $+\infty$.
          \item
          \textbf{Méthode 1 : \`A la calculatrice}
          \par
          La suite $S_n$ est croissante. \`A l'aide d'un tableau de valeurs pour la fonction $x \longmapsto 10~000 \left( 1-0,9^{x+1} \right)$, on trouve :
          \[ S_{20} \approx 8~905 \quad \text{et} \quad S_{21} \approx 9~015. \]
          La plus petite valeur de l'entier $n$ telle que $S_n > 9~000$ est donc $21$.
          \vspace{0.5cm}
          \par
          \textbf{Méthode 2 : Par le calcul}
          \par
          $S_n > 9~000 ~ \Leftrightarrow ~10~000 \left( 1-0,9^{n+1}\right) > 9~000 $\\
          $	\phantom{S_n > 9~000 ~} \Leftrightarrow ~ 1-0,9^{n+1} > 0,9$ \\
          $	\phantom{S_n > 9~000 ~}  \Leftrightarrow ~ -0,9^{n+1} > - 0,1$ \\
          $	\phantom{S_n > 9~000 ~}  \Leftrightarrow ~ 0,9^{n+1} < 0,1 $
          \par
          La fonction $\ln$ étant strictement croissante sur $]0~;~+\infty[$ :
          \par
          $S_n > 9~000 ~ \Leftrightarrow ~ \ln \left(0,9^{n+1} \right) < \ln (0,1) $ \\
          $\phantom{S_n > 9~000 ~ } \Leftrightarrow ~ (n+1)\ln (0,9) < \ln (0,1)$
          \par
          $0,9 < 1$ donc $\ln (0,9) < 0$ ; par conséquent :
          \par
          $S_n > 9~000  ~\Leftrightarrow ~n+1 > \dfrac{\ln (0,1)}{\ln (0,9)}$ \\
          $\phantom{S_n > 9~000  ~}   \Leftrightarrow ~ n > \dfrac{\ln (0,1)}{\ln (0,9)} - 1$
          \par
          $\dfrac{\ln (0,1)}{\ln (0,9)} - 1 \approx 20,9$ (arrondi au dixième)
          \par
          La plus petite valeur de l'entier $n$ telle que $S_n > 9~000$ est donc $21$.
          \cadre{rouge}{Attention}{
               Lorsque $0 < \alpha < 1$, $\ln(\alpha)$ est strictement \textbf{négatif}.
               \par
               Il faut donc penser à \textbf{changer le sens de l'inégalité} lorsque l'on divise par $\ln(\alpha)$.
          }
          \par
     \end{enumerate}
\end{corrige}

\end{document}
µ
\documentclass[a4paper]{article}

%================================================================================================================================
%
% Packages
%
%================================================================================================================================

\usepackage[T1]{fontenc} 	% pour caractères accentués
\usepackage[utf8]{inputenc}  % encodage utf8
\usepackage[french]{babel}	% langue : français
\usepackage{fourier}			% caractères plus lisibles
\usepackage[dvipsnames]{xcolor} % couleurs
\usepackage{fancyhdr}		% réglage header footer
\usepackage{needspace}		% empêcher sauts de page mal placés
\usepackage{graphicx}		% pour inclure des graphiques
\usepackage{enumitem,cprotect}		% personnalise les listes d'items (nécessaire pour ol, al ...)
\usepackage{hyperref}		% Liens hypertexte
\usepackage{pstricks,pst-all,pst-node,pstricks-add,pst-math,pst-plot,pst-tree,pst-eucl} % pstricks
\usepackage[a4paper,includeheadfoot,top=2cm,left=3cm, bottom=2cm,right=3cm]{geometry} % marges etc.
\usepackage{comment}			% commentaires multilignes
\usepackage{amsmath,environ} % maths (matrices, etc.)
\usepackage{amssymb,makeidx}
\usepackage{bm}				% bold maths
\usepackage{tabularx}		% tableaux
\usepackage{colortbl}		% tableaux en couleur
\usepackage{fontawesome}		% Fontawesome
\usepackage{environ}			% environment with command
\usepackage{fp}				% calculs pour ps-tricks
\usepackage{multido}			% pour ps tricks
\usepackage[np]{numprint}	% formattage nombre
\usepackage{tikz,tkz-tab} 			% package principal TikZ
\usepackage{pgfplots}   % axes
\usepackage{mathrsfs}    % cursives
\usepackage{calc}			% calcul taille boites
\usepackage[scaled=0.875]{helvet} % font sans serif
\usepackage{svg} % svg
\usepackage{scrextend} % local margin
\usepackage{scratch} %scratch
\usepackage{multicol} % colonnes
%\usepackage{infix-RPN,pst-func} % formule en notation polanaise inversée
\usepackage{listings}

%================================================================================================================================
%
% Réglages de base
%
%================================================================================================================================

\lstset{
language=Python,   % R code
literate=
{á}{{\'a}}1
{à}{{\`a}}1
{ã}{{\~a}}1
{é}{{\'e}}1
{è}{{\`e}}1
{ê}{{\^e}}1
{í}{{\'i}}1
{ó}{{\'o}}1
{õ}{{\~o}}1
{ú}{{\'u}}1
{ü}{{\"u}}1
{ç}{{\c{c}}}1
{~}{{ }}1
}


\definecolor{codegreen}{rgb}{0,0.6,0}
\definecolor{codegray}{rgb}{0.5,0.5,0.5}
\definecolor{codepurple}{rgb}{0.58,0,0.82}
\definecolor{backcolour}{rgb}{0.95,0.95,0.92}

\lstdefinestyle{mystyle}{
    backgroundcolor=\color{backcolour},   
    commentstyle=\color{codegreen},
    keywordstyle=\color{magenta},
    numberstyle=\tiny\color{codegray},
    stringstyle=\color{codepurple},
    basicstyle=\ttfamily\footnotesize,
    breakatwhitespace=false,         
    breaklines=true,                 
    captionpos=b,                    
    keepspaces=true,                 
    numbers=left,                    
xleftmargin=2em,
framexleftmargin=2em,            
    showspaces=false,                
    showstringspaces=false,
    showtabs=false,                  
    tabsize=2,
    upquote=true
}

\lstset{style=mystyle}


\lstset{style=mystyle}
\newcommand{\imgdir}{C:/laragon/www/newmc/assets/imgsvg/}
\newcommand{\imgsvgdir}{C:/laragon/www/newmc/assets/imgsvg/}

\definecolor{mcgris}{RGB}{220, 220, 220}% ancien~; pour compatibilité
\definecolor{mcbleu}{RGB}{52, 152, 219}
\definecolor{mcvert}{RGB}{125, 194, 70}
\definecolor{mcmauve}{RGB}{154, 0, 215}
\definecolor{mcorange}{RGB}{255, 96, 0}
\definecolor{mcturquoise}{RGB}{0, 153, 153}
\definecolor{mcrouge}{RGB}{255, 0, 0}
\definecolor{mclightvert}{RGB}{205, 234, 190}

\definecolor{gris}{RGB}{220, 220, 220}
\definecolor{bleu}{RGB}{52, 152, 219}
\definecolor{vert}{RGB}{125, 194, 70}
\definecolor{mauve}{RGB}{154, 0, 215}
\definecolor{orange}{RGB}{255, 96, 0}
\definecolor{turquoise}{RGB}{0, 153, 153}
\definecolor{rouge}{RGB}{255, 0, 0}
\definecolor{lightvert}{RGB}{205, 234, 190}
\setitemize[0]{label=\color{lightvert}  $\bullet$}

\pagestyle{fancy}
\renewcommand{\headrulewidth}{0.2pt}
\fancyhead[L]{maths-cours.fr}
\fancyhead[R]{\thepage}
\renewcommand{\footrulewidth}{0.2pt}
\fancyfoot[C]{}

\newcolumntype{C}{>{\centering\arraybackslash}X}
\newcolumntype{s}{>{\hsize=.35\hsize\arraybackslash}X}

\setlength{\parindent}{0pt}		 
\setlength{\parskip}{3mm}
\setlength{\headheight}{1cm}

\def\ebook{ebook}
\def\book{book}
\def\web{web}
\def\type{web}

\newcommand{\vect}[1]{\overrightarrow{\,\mathstrut#1\,}}

\def\Oij{$\left(\text{O}~;~\vect{\imath},~\vect{\jmath}\right)$}
\def\Oijk{$\left(\text{O}~;~\vect{\imath},~\vect{\jmath},~\vect{k}\right)$}
\def\Ouv{$\left(\text{O}~;~\vect{u},~\vect{v}\right)$}

\hypersetup{breaklinks=true, colorlinks = true, linkcolor = OliveGreen, urlcolor = OliveGreen, citecolor = OliveGreen, pdfauthor={Didier BONNEL - https://www.maths-cours.fr} } % supprime les bordures autour des liens

\renewcommand{\arg}[0]{\text{arg}}

\everymath{\displaystyle}

%================================================================================================================================
%
% Macros - Commandes
%
%================================================================================================================================

\newcommand\meta[2]{    			% Utilisé pour créer le post HTML.
	\def\titre{titre}
	\def\url{url}
	\def\arg{#1}
	\ifx\titre\arg
		\newcommand\maintitle{#2}
		\fancyhead[L]{#2}
		{\Large\sffamily \MakeUppercase{#2}}
		\vspace{1mm}\textcolor{mcvert}{\hrule}
	\fi 
	\ifx\url\arg
		\fancyfoot[L]{\href{https://www.maths-cours.fr#2}{\black \footnotesize{https://www.maths-cours.fr#2}}}
	\fi 
}


\newcommand\TitreC[1]{    		% Titre centré
     \needspace{3\baselineskip}
     \begin{center}\textbf{#1}\end{center}
}

\newcommand\newpar{    		% paragraphe
     \par
}

\newcommand\nosp {    		% commande vide (pas d'espace)
}
\newcommand{\id}[1]{} %ignore

\newcommand\boite[2]{				% Boite simple sans titre
	\vspace{5mm}
	\setlength{\fboxrule}{0.2mm}
	\setlength{\fboxsep}{5mm}	
	\fcolorbox{#1}{#1!3}{\makebox[\linewidth-2\fboxrule-2\fboxsep]{
  		\begin{minipage}[t]{\linewidth-2\fboxrule-4\fboxsep}\setlength{\parskip}{3mm}
  			 #2
  		\end{minipage}
	}}
	\vspace{5mm}
}

\newcommand\CBox[4]{				% Boites
	\vspace{5mm}
	\setlength{\fboxrule}{0.2mm}
	\setlength{\fboxsep}{5mm}
	
	\fcolorbox{#1}{#1!3}{\makebox[\linewidth-2\fboxrule-2\fboxsep]{
		\begin{minipage}[t]{1cm}\setlength{\parskip}{3mm}
	  		\textcolor{#1}{\LARGE{#2}}    
 	 	\end{minipage}  
  		\begin{minipage}[t]{\linewidth-2\fboxrule-4\fboxsep}\setlength{\parskip}{3mm}
			\raisebox{1.2mm}{\normalsize\sffamily{\textcolor{#1}{#3}}}						
  			 #4
  		\end{minipage}
	}}
	\vspace{5mm}
}

\newcommand\cadre[3]{				% Boites convertible html
	\par
	\vspace{2mm}
	\setlength{\fboxrule}{0.1mm}
	\setlength{\fboxsep}{5mm}
	\fcolorbox{#1}{white}{\makebox[\linewidth-2\fboxrule-2\fboxsep]{
  		\begin{minipage}[t]{\linewidth-2\fboxrule-4\fboxsep}\setlength{\parskip}{3mm}
			\raisebox{-2.5mm}{\sffamily \small{\textcolor{#1}{\MakeUppercase{#2}}}}		
			\par		
  			 #3
 	 		\end{minipage}
	}}
		\vspace{2mm}
	\par
}

\newcommand\bloc[3]{				% Boites convertible html sans bordure
     \needspace{2\baselineskip}
     {\sffamily \small{\textcolor{#1}{\MakeUppercase{#2}}}}    
		\par		
  			 #3
		\par
}

\newcommand\CHelp[1]{
     \CBox{Plum}{\faInfoCircle}{À RETENIR}{#1}
}

\newcommand\CUp[1]{
     \CBox{NavyBlue}{\faThumbsOUp}{EN PRATIQUE}{#1}
}

\newcommand\CInfo[1]{
     \CBox{Sepia}{\faArrowCircleRight}{REMARQUE}{#1}
}

\newcommand\CRedac[1]{
     \CBox{PineGreen}{\faEdit}{BIEN R\'EDIGER}{#1}
}

\newcommand\CError[1]{
     \CBox{Red}{\faExclamationTriangle}{ATTENTION}{#1}
}

\newcommand\TitreExo[2]{
\needspace{4\baselineskip}
 {\sffamily\large EXERCICE #1\ (\emph{#2 points})}
\vspace{5mm}
}

\newcommand\img[2]{
          \includegraphics[width=#2\paperwidth]{\imgdir#1}
}

\newcommand\imgsvg[2]{
       \begin{center}   \includegraphics[width=#2\paperwidth]{\imgsvgdir#1} \end{center}
}


\newcommand\Lien[2]{
     \href{#1}{#2 \tiny \faExternalLink}
}
\newcommand\mcLien[2]{
     \href{https~://www.maths-cours.fr/#1}{#2 \tiny \faExternalLink}
}

\newcommand{\euro}{\eurologo{}}

%================================================================================================================================
%
% Macros - Environement
%
%================================================================================================================================

\newenvironment{tex}{ %
}
{%
}

\newenvironment{indente}{ %
	\setlength\parindent{10mm}
}

{
	\setlength\parindent{0mm}
}

\newenvironment{corrige}{%
     \needspace{3\baselineskip}
     \medskip
     \textbf{\textsc{Corrigé}}
     \medskip
}
{
}

\newenvironment{extern}{%
     \begin{center}
     }
     {
     \end{center}
}

\NewEnviron{code}{%
	\par
     \boite{gray}{\texttt{%
     \BODY
     }}
     \par
}

\newenvironment{vbloc}{% boite sans cadre empeche saut de page
     \begin{minipage}[t]{\linewidth}
     }
     {
     \end{minipage}
}
\NewEnviron{h2}{%
    \needspace{3\baselineskip}
    \vspace{0.6cm}
	\noindent \MakeUppercase{\sffamily \large \BODY}
	\vspace{1mm}\textcolor{mcgris}{\hrule}\vspace{0.4cm}
	\par
}{}

\NewEnviron{h3}{%
    \needspace{3\baselineskip}
	\vspace{5mm}
	\textsc{\BODY}
	\par
}

\NewEnviron{margeneg}{ %
\begin{addmargin}[-1cm]{0cm}
\BODY
\end{addmargin}
}

\NewEnviron{html}{%
}

\begin{document}
\meta{url}{/exercices/graphes-probabilistes-bac-blanc-es-l-sujet-5-maths-cours-2018-spe/}
\meta{pid}{10559}
\meta{titre}{Graphes probabilistes - Bac blanc ES/L Sujet 5 - Maths-cours 2018 (spé)}
\meta{type}{exercices}
%
\begin{h2}Exercice 2 (5 points)\end{h2}
\par
\textbf{Candidats ayant suivi l'enseignement de spécialité}
\par
Depuis le début de l'année 2015, une agence bancaire propose à ses clients, titulaires d'une carte de crédit, une assurance \og Tranquillité \fg{} qui leur permet d'être mieux indemnisé en cas de perte ou de vol de leur carte.
\par
De 2015 à 2017, on a constaté que :
\par
\begin{itemize}
     \item
     20\% des titulaires d'une carte de crédit qui ne bénéficient pas de l'assurance \og Tranquillité \fg{} souscrivent à cette assurance l'année suivante ;
     \item
     5\% des titulaires d'une carte de crédit qui ont souscrit à l'assurance \og Tranquillité \fg{} résilient cette assurance l'année suivante  ;
\end{itemize}
\par
On suppose que cette évolution se poursuivra de manière identique durant les années à venir.
\par
On sélectionne au hasard un client titulaire d'une carte de crédit et, pour tout entier naturel $n$, on note :
\par
\begin{itemize}
     \item
     $a_{n}$, la probabilité que le client ait souscrit à l'assurance \og Tranquillité \fg{} au début de l'année $2015 + n$ ;
     \item
     $b_{n}$, la probabilité que le client choisi n'ait pas souscrit à l'assurance \og Tranquillité \fg{} au début de l'année $2015 + n$ ;
     \item
     $P_{n}$, la matrice ligne $\left(a_{n} \quad b_{n}\right)$ donnant l'état probabiliste au début de l'année $2015 + n$.
\end{itemize}
\par
Au début de l'année 2015, aucun client n'a encore souscrit à l'assurance \og Tranquillité \fg{}. \\
On a donc $P_0=\left(0 \quad 1\right)$.
\par
%============================================================================================================================
%
\TitreC{Partie A}
%
%============================================================================================================================
\par
\begin{enumerate}
     \item %1
     Représenter la situation par un graphe probabiliste de sommets $T$ et $\overline{T}$ où $T$ correspond à l'état " le client a souscrit à l'assurance \og Tranquillité \fg{} " et $\overline{T}$ correspond à l'état contraire.
     \item %2
     Déterminer la matrice de transition $M$ associée à ce graphe, les sommets $T$ et $\overline{T}$ étant classés dans cet ordre.
     \item %3
     Déterminer l'état stable de ce graphe probabiliste. \\
     Que peut-on en conclure concernant le pourcentage de clients qui souscriront à l'assurance \og Tranquillité \fg{} dans les années futures ?
     \par
\end{enumerate}
\par
%============================================================================================================================
%
\TitreC{Partie B}
%
%============================================================================================================================
\par
Le directeur de l'agence cherche à déterminer au début de quelle année plus de 70\% des titulaires d'une carte de crédit auront souscrit à l'assurance \og Tranquillité \fg{}.
\par
\begin{enumerate}
     \item
     Recopier et compléter l'algorithme ci-après afin qu'il réponde à l'interrogation du directeur.
     \par
     \begin{center}
          \begin{extern}%width="400" alt="Algorithme"
               \begin{tabular}{|l l|}\hline
                    \textbf{Variables :}	& 	$n$ un nombre entier naturel \\
                    &$a$ et $b$ sont des nombres réels\\
                    \textbf{Initialisation :}	& Affecter à $n$ la valeur ...\\
                    & Affecter à $a$ la valeur ...\\
                    & Affecter à $b$ la valeur ...\\
                    \textbf{Traitement :} & Tant que ...\\
                    &\qquad Affecter à $a$ la valeur \\
                    &\qquad \phantom{Affecter }$0,95 \times a + 0,2 \times b$\\
                    &\qquad Affecter à $b$ la valeur $1-a$ \\
                    &\qquad Affecter à $n$ la valeur ...\\
                    &Fin Tant que\\
                    \textbf{Sortie :}		&Afficher $2015+n$ \\ \hline
               \end{tabular}
          \end{extern}
     \end{center}
     \item
     On admet que pour tout entier naturel $n$ :
     \[ a_{n}=0,8(1-0,75^n). \]
     Déterminer la valeur affichée par l'algorithme de la question précédente.
     \par
\end{enumerate}
\begin{corrige}
     %============================================================================================================================
     %
     \TitreC{Partie A}
     %
     %============================================================================================================================
     \par
     \begin{enumerate}
          \item On traduit les données de l'énoncé par un graphe probabiliste de sommets $T$ et $\overline{T}$:
          \begin{center}
               %\hspace*{1cm}
               \begin{extern}%width="400" alt="Graphe probabiliste"
                    \begin{pspicture}(-2,-0.5)(4,1)
                         \circlenode{T}{$T$} \hskip 4cm \circlenode{B}{$\overline{T}$}% définition des sommets
                         \psset{arcangle=15,arrowsize=2pt 3}%  différents paramètres
                         \ncarc{->}{T}{B} \Aput{0,05}%              arc pondéré partant de T
                         \ncarc{->}{B}{T} \Aput{0,2}%              arc pondéré arrivant à B
                         \nccircle[angleA=90]{->}{T}{4mm}   \Bput{0,95}%    boucle autour de T
                         \nccircle[angleA=-90]{->}{B}{.4cm} \Bput{0,8}%    boucle autour de B
                    \end{pspicture}
               \end{extern}
          \end{center}
          \item  La matrice de transition de ce graphe en considérant les sommets dans l'ordre $T$, $\overline{T}$ est:
          \[ M=
          \begin{pmatrix}
               0,95 & 0,05\\
               0,2 & 0,8
          \end{pmatrix}. \]
          \item
          Les états stables sont les matrices-ligne $P = (a\quad b)$ telles que ${a + b = 1}$ et ${PM = P}$.
          \par
          $PM=P \Leftrightarrow \begin{pmatrix} a&b\end{pmatrix}
          \times \begin{pmatrix} 0,95 & 0,05 \\ 0,2 & 0,8 \end{pmatrix}
          $\nosp$=\begin{pmatrix} a&b\end{pmatrix}$
          \par
          $\phantom{PM=P} \Leftrightarrow \begin{pmatrix} 0,95a+0,2b & 0,05a+0,8b\end{pmatrix}
          $\nosp$=\begin{pmatrix} a&b\end{pmatrix}$
          \par
          $\phantom{PM=P} \Leftrightarrow
          \left\lbrace
          \begin{array}{r c l}
               0,95a+0,2b &=& a\\
               0,05a+0,8b &=& b
          \end{array}
     \right.$
     \par
     $\phantom{PM=P}
     \Leftrightarrow
     \left\lbrace
     \begin{array}{r c l}
          -0,05a+0,2b &=& 0\\
          0,05a-0,2b &=& 0
     \end{array}
\right.$
\par
$\phantom{PM=P}
\Leftrightarrow
0,05a-0,2b = 0$.
\par
Or $a+b=1$ ; donc  $b=1-a$ et:
\par
$0,05a-0,2(1-a) = 0$
\par
$0,25a-0,2 = 0$
\par
$a = \dfrac{0,2}{0.25}=0,8$.
\par
Et $b=1-a=1-0,8=0,2$.
\par
L'état stable du graphe est donc
$P=\begin{pmatrix} 0,8 & 0,2 \end{pmatrix}$.
\par
Au fil du temps, le pourcentage de clients qui choisiront l'assurance \og Tranquillité \fg{} se rapprochera de 80\% ($=0,8$).
\end{enumerate}
\par
%============================================================================================================================
%
\TitreC{Partie B}
%
%============================================================================================================================
\par
\begin{enumerate}
     \item
     On complète l'algorithme comme suit :
     \begin{center}
          \begin{extern}%width="400" alt="Algorithme"
               \begin{tabular}{|l l|}\hline
                    \textbf{Variables :}	& 	$n$ un nombre entier naturel \\
                    &$a$ et $b$ sont des nombres réels\\
                    \textbf{Initialisation :}	& Affecter à $n$ la valeur $\color{red}{0}$\\
                    & Affecter à $a$ la valeur $\color{red}{0}$\\
                    & Affecter à $b$ la valeur $\color{red}{1}$\\
                    \textbf{Traitement :} & Tant que $\color{red}{a \leqslant 0,7}$ \\
                    &\qquad Affecter à $a$ la valeur \\
                    &\qquad \phantom{Affecter }$0,95 \times a + 0,2 \times b$\\
                    &\qquad Affecter à $b$ la valeur $1-a$ \\
                    &\qquad Affecter à $n$ la valeur $\color{red}{n+1}$ \\
                    &Fin Tant que\\
                    \textbf{Sortie :}		&Afficher $2015+n$ \\ \hline
               \end{tabular}
          \end{extern}
     \end{center}
     \item
     L'algorithme affiche l'année à partir de laquelle plus de 70\% des titulaires d'une carte de crédit auront souscrit à l'assurance \og Tranquillité \fg{}.
     \par
     Pour trouver cette année, il nous faut donc résoudre l'inéquation $a_n > 0,7$.
     \par
     D'après l'énoncé $a_{n}=0,8(1-0,75^n)$ donc :
     \par
     $a_n > 0,7  \Leftrightarrow 0,8(1-0,75^n) > 0,7$\\
     $\phantom{ a_n > 0,7 } \Leftrightarrow \  1-0,75^n > \dfrac{0,7}{0,8}$ \\
     $\phantom{ a_n > 0,7 } \Leftrightarrow \  1-0,75^n > \dfrac{7}{8}$ \\
     $\phantom{ a_n > 0,7 } \Leftrightarrow \  -0,75^n  > \dfrac{7}{8}-1 $\\
     $\phantom{ a_n > 0,7 } \Leftrightarrow \  -0,75^n  > -\dfrac{1}{8}$\\
     $\phantom{ a_n > 0,7 } \Leftrightarrow \  0,75^n < \dfrac{1}{8}$
     \par
     La fonction $\ln$ étant strictement croissante sur $]0~;~+\infty[$ :
     \par
     $a_n > 0,7  \Leftrightarrow \  \ln \left(0,75^n \right) < \ln \left(\dfrac{1}{8}\right) $ \\
     $\phantom{ a_n > 0,7 } \Leftrightarrow \ n\ln (0,75) < -\ln (8)$
     \par
     Or $0,9 < 1$ donc $\ln (0,75)$ est strictement négatif ; par conséquent :
     \par
     $a_n > 0,7 \Leftrightarrow \ n > \dfrac{-\ln (8)}{\ln (0,75)}$
     \par
     \`A la calculatrice : $\dfrac{-\ln (8)}{\ln (0,75)} - 1 \approx 7,2$ (arrondi au dixième).
     \par
     La plus petite valeur de l'entier $n$ telle que $a_n > 0,7$, est donc $8$.
     \par
     Par conséquent, l'algorithme affichera l'année 2015+8=\textbf{2023}.
     \par
\end{enumerate}
\end{corrige}

\end{document}
µ
\documentclass[a4paper]{article}

%================================================================================================================================
%
% Packages
%
%================================================================================================================================

\usepackage[T1]{fontenc} 	% pour caractères accentués
\usepackage[utf8]{inputenc}  % encodage utf8
\usepackage[french]{babel}	% langue : français
\usepackage{fourier}			% caractères plus lisibles
\usepackage[dvipsnames]{xcolor} % couleurs
\usepackage{fancyhdr}		% réglage header footer
\usepackage{needspace}		% empêcher sauts de page mal placés
\usepackage{graphicx}		% pour inclure des graphiques
\usepackage{enumitem,cprotect}		% personnalise les listes d'items (nécessaire pour ol, al ...)
\usepackage{hyperref}		% Liens hypertexte
\usepackage{pstricks,pst-all,pst-node,pstricks-add,pst-math,pst-plot,pst-tree,pst-eucl} % pstricks
\usepackage[a4paper,includeheadfoot,top=2cm,left=3cm, bottom=2cm,right=3cm]{geometry} % marges etc.
\usepackage{comment}			% commentaires multilignes
\usepackage{amsmath,environ} % maths (matrices, etc.)
\usepackage{amssymb,makeidx}
\usepackage{bm}				% bold maths
\usepackage{tabularx}		% tableaux
\usepackage{colortbl}		% tableaux en couleur
\usepackage{fontawesome}		% Fontawesome
\usepackage{environ}			% environment with command
\usepackage{fp}				% calculs pour ps-tricks
\usepackage{multido}			% pour ps tricks
\usepackage[np]{numprint}	% formattage nombre
\usepackage{tikz,tkz-tab} 			% package principal TikZ
\usepackage{pgfplots}   % axes
\usepackage{mathrsfs}    % cursives
\usepackage{calc}			% calcul taille boites
\usepackage[scaled=0.875]{helvet} % font sans serif
\usepackage{svg} % svg
\usepackage{scrextend} % local margin
\usepackage{scratch} %scratch
\usepackage{multicol} % colonnes
%\usepackage{infix-RPN,pst-func} % formule en notation polanaise inversée
\usepackage{listings}

%================================================================================================================================
%
% Réglages de base
%
%================================================================================================================================

\lstset{
language=Python,   % R code
literate=
{á}{{\'a}}1
{à}{{\`a}}1
{ã}{{\~a}}1
{é}{{\'e}}1
{è}{{\`e}}1
{ê}{{\^e}}1
{í}{{\'i}}1
{ó}{{\'o}}1
{õ}{{\~o}}1
{ú}{{\'u}}1
{ü}{{\"u}}1
{ç}{{\c{c}}}1
{~}{{ }}1
}


\definecolor{codegreen}{rgb}{0,0.6,0}
\definecolor{codegray}{rgb}{0.5,0.5,0.5}
\definecolor{codepurple}{rgb}{0.58,0,0.82}
\definecolor{backcolour}{rgb}{0.95,0.95,0.92}

\lstdefinestyle{mystyle}{
    backgroundcolor=\color{backcolour},   
    commentstyle=\color{codegreen},
    keywordstyle=\color{magenta},
    numberstyle=\tiny\color{codegray},
    stringstyle=\color{codepurple},
    basicstyle=\ttfamily\footnotesize,
    breakatwhitespace=false,         
    breaklines=true,                 
    captionpos=b,                    
    keepspaces=true,                 
    numbers=left,                    
xleftmargin=2em,
framexleftmargin=2em,            
    showspaces=false,                
    showstringspaces=false,
    showtabs=false,                  
    tabsize=2,
    upquote=true
}

\lstset{style=mystyle}


\lstset{style=mystyle}
\newcommand{\imgdir}{C:/laragon/www/newmc/assets/imgsvg/}
\newcommand{\imgsvgdir}{C:/laragon/www/newmc/assets/imgsvg/}

\definecolor{mcgris}{RGB}{220, 220, 220}% ancien~; pour compatibilité
\definecolor{mcbleu}{RGB}{52, 152, 219}
\definecolor{mcvert}{RGB}{125, 194, 70}
\definecolor{mcmauve}{RGB}{154, 0, 215}
\definecolor{mcorange}{RGB}{255, 96, 0}
\definecolor{mcturquoise}{RGB}{0, 153, 153}
\definecolor{mcrouge}{RGB}{255, 0, 0}
\definecolor{mclightvert}{RGB}{205, 234, 190}

\definecolor{gris}{RGB}{220, 220, 220}
\definecolor{bleu}{RGB}{52, 152, 219}
\definecolor{vert}{RGB}{125, 194, 70}
\definecolor{mauve}{RGB}{154, 0, 215}
\definecolor{orange}{RGB}{255, 96, 0}
\definecolor{turquoise}{RGB}{0, 153, 153}
\definecolor{rouge}{RGB}{255, 0, 0}
\definecolor{lightvert}{RGB}{205, 234, 190}
\setitemize[0]{label=\color{lightvert}  $\bullet$}

\pagestyle{fancy}
\renewcommand{\headrulewidth}{0.2pt}
\fancyhead[L]{maths-cours.fr}
\fancyhead[R]{\thepage}
\renewcommand{\footrulewidth}{0.2pt}
\fancyfoot[C]{}

\newcolumntype{C}{>{\centering\arraybackslash}X}
\newcolumntype{s}{>{\hsize=.35\hsize\arraybackslash}X}

\setlength{\parindent}{0pt}		 
\setlength{\parskip}{3mm}
\setlength{\headheight}{1cm}

\def\ebook{ebook}
\def\book{book}
\def\web{web}
\def\type{web}

\newcommand{\vect}[1]{\overrightarrow{\,\mathstrut#1\,}}

\def\Oij{$\left(\text{O}~;~\vect{\imath},~\vect{\jmath}\right)$}
\def\Oijk{$\left(\text{O}~;~\vect{\imath},~\vect{\jmath},~\vect{k}\right)$}
\def\Ouv{$\left(\text{O}~;~\vect{u},~\vect{v}\right)$}

\hypersetup{breaklinks=true, colorlinks = true, linkcolor = OliveGreen, urlcolor = OliveGreen, citecolor = OliveGreen, pdfauthor={Didier BONNEL - https://www.maths-cours.fr} } % supprime les bordures autour des liens

\renewcommand{\arg}[0]{\text{arg}}

\everymath{\displaystyle}

%================================================================================================================================
%
% Macros - Commandes
%
%================================================================================================================================

\newcommand\meta[2]{    			% Utilisé pour créer le post HTML.
	\def\titre{titre}
	\def\url{url}
	\def\arg{#1}
	\ifx\titre\arg
		\newcommand\maintitle{#2}
		\fancyhead[L]{#2}
		{\Large\sffamily \MakeUppercase{#2}}
		\vspace{1mm}\textcolor{mcvert}{\hrule}
	\fi 
	\ifx\url\arg
		\fancyfoot[L]{\href{https://www.maths-cours.fr#2}{\black \footnotesize{https://www.maths-cours.fr#2}}}
	\fi 
}


\newcommand\TitreC[1]{    		% Titre centré
     \needspace{3\baselineskip}
     \begin{center}\textbf{#1}\end{center}
}

\newcommand\newpar{    		% paragraphe
     \par
}

\newcommand\nosp {    		% commande vide (pas d'espace)
}
\newcommand{\id}[1]{} %ignore

\newcommand\boite[2]{				% Boite simple sans titre
	\vspace{5mm}
	\setlength{\fboxrule}{0.2mm}
	\setlength{\fboxsep}{5mm}	
	\fcolorbox{#1}{#1!3}{\makebox[\linewidth-2\fboxrule-2\fboxsep]{
  		\begin{minipage}[t]{\linewidth-2\fboxrule-4\fboxsep}\setlength{\parskip}{3mm}
  			 #2
  		\end{minipage}
	}}
	\vspace{5mm}
}

\newcommand\CBox[4]{				% Boites
	\vspace{5mm}
	\setlength{\fboxrule}{0.2mm}
	\setlength{\fboxsep}{5mm}
	
	\fcolorbox{#1}{#1!3}{\makebox[\linewidth-2\fboxrule-2\fboxsep]{
		\begin{minipage}[t]{1cm}\setlength{\parskip}{3mm}
	  		\textcolor{#1}{\LARGE{#2}}    
 	 	\end{minipage}  
  		\begin{minipage}[t]{\linewidth-2\fboxrule-4\fboxsep}\setlength{\parskip}{3mm}
			\raisebox{1.2mm}{\normalsize\sffamily{\textcolor{#1}{#3}}}						
  			 #4
  		\end{minipage}
	}}
	\vspace{5mm}
}

\newcommand\cadre[3]{				% Boites convertible html
	\par
	\vspace{2mm}
	\setlength{\fboxrule}{0.1mm}
	\setlength{\fboxsep}{5mm}
	\fcolorbox{#1}{white}{\makebox[\linewidth-2\fboxrule-2\fboxsep]{
  		\begin{minipage}[t]{\linewidth-2\fboxrule-4\fboxsep}\setlength{\parskip}{3mm}
			\raisebox{-2.5mm}{\sffamily \small{\textcolor{#1}{\MakeUppercase{#2}}}}		
			\par		
  			 #3
 	 		\end{minipage}
	}}
		\vspace{2mm}
	\par
}

\newcommand\bloc[3]{				% Boites convertible html sans bordure
     \needspace{2\baselineskip}
     {\sffamily \small{\textcolor{#1}{\MakeUppercase{#2}}}}    
		\par		
  			 #3
		\par
}

\newcommand\CHelp[1]{
     \CBox{Plum}{\faInfoCircle}{À RETENIR}{#1}
}

\newcommand\CUp[1]{
     \CBox{NavyBlue}{\faThumbsOUp}{EN PRATIQUE}{#1}
}

\newcommand\CInfo[1]{
     \CBox{Sepia}{\faArrowCircleRight}{REMARQUE}{#1}
}

\newcommand\CRedac[1]{
     \CBox{PineGreen}{\faEdit}{BIEN R\'EDIGER}{#1}
}

\newcommand\CError[1]{
     \CBox{Red}{\faExclamationTriangle}{ATTENTION}{#1}
}

\newcommand\TitreExo[2]{
\needspace{4\baselineskip}
 {\sffamily\large EXERCICE #1\ (\emph{#2 points})}
\vspace{5mm}
}

\newcommand\img[2]{
          \includegraphics[width=#2\paperwidth]{\imgdir#1}
}

\newcommand\imgsvg[2]{
       \begin{center}   \includegraphics[width=#2\paperwidth]{\imgsvgdir#1} \end{center}
}


\newcommand\Lien[2]{
     \href{#1}{#2 \tiny \faExternalLink}
}
\newcommand\mcLien[2]{
     \href{https~://www.maths-cours.fr/#1}{#2 \tiny \faExternalLink}
}

\newcommand{\euro}{\eurologo{}}

%================================================================================================================================
%
% Macros - Environement
%
%================================================================================================================================

\newenvironment{tex}{ %
}
{%
}

\newenvironment{indente}{ %
	\setlength\parindent{10mm}
}

{
	\setlength\parindent{0mm}
}

\newenvironment{corrige}{%
     \needspace{3\baselineskip}
     \medskip
     \textbf{\textsc{Corrigé}}
     \medskip
}
{
}

\newenvironment{extern}{%
     \begin{center}
     }
     {
     \end{center}
}

\NewEnviron{code}{%
	\par
     \boite{gray}{\texttt{%
     \BODY
     }}
     \par
}

\newenvironment{vbloc}{% boite sans cadre empeche saut de page
     \begin{minipage}[t]{\linewidth}
     }
     {
     \end{minipage}
}
\NewEnviron{h2}{%
    \needspace{3\baselineskip}
    \vspace{0.6cm}
	\noindent \MakeUppercase{\sffamily \large \BODY}
	\vspace{1mm}\textcolor{mcgris}{\hrule}\vspace{0.4cm}
	\par
}{}

\NewEnviron{h3}{%
    \needspace{3\baselineskip}
	\vspace{5mm}
	\textsc{\BODY}
	\par
}

\NewEnviron{margeneg}{ %
\begin{addmargin}[-1cm]{0cm}
\BODY
\end{addmargin}
}

\NewEnviron{html}{%
}

\begin{document}
\meta{url}{/exercices/qcm-bac-blanc-es-l-sujet-6-maths-cours-2018/}
\meta{pid}{10590}
\meta{titre}{QCM - Bac blanc ES/L Sujet 6 - Maths-cours 2018}
\meta{type}{exercices}
%\begin{h2}Exercice 1 (5 points)\end{h2}
\par
\emph{Cet exercice est un questionnaire à choix multiples (QCM). Les questions sont indépendantes les unes des autres. Pour chacune des questions suivantes, une seule des quatre réponses proposées est exacte.  \\Indiquer sur la copie le numéro de la question et la réponse exacte \textbf{en justifiant le choix effectué}. }
\par
\emph{\textbf{Toute réponse non justifiée ne sera pas prise en compte.}}
\par
\begin{itemize}
     \item \textbf{Question 1 :}
     La valeur exacte de $A=\ln(2 \text{e}^{2})$ est :
     \par
     \textbf{a.~~} $A=\ln2$ \\
     \textbf{b.~~}  $A=2 \text{e}\ln 2$ \\
     \textbf{c.~~}  $A=2 \text{e}+\ln2$ \\
     \textbf{d.~~}  $A=2 +\ln2$ \\
     \item \textbf{Question 2 :}
     On considère la fonction $f$ définie sur l'intervalle $]0~;~+\infty[$ par :
     \[ f(x)=x\ln(x) \]
     \par
     La fonction $f$ est dérivable sur l'intervalle $]0~;~+\infty[$ et sa dérivée $f'$ est donnée par :
     \par
     \textbf{a.~~} $f'(x)=1$ \\
     \textbf{b.~~}  $f'(x)=1+\ln(x)$ \\
     \textbf{c.~~}  $f'(x)=x+\ln(x)$
     \par
     \textbf{d.~~}  $f'(x)=\dfrac{1}{x}$ \\
     \item \textbf{Question 3 :}
     Le plus petit entier naturel $n$ tel que :
     \[ 0,8^n \leqslant 0,01 \]
     est :
     \par
     \textbf{a.~~} $n=19$ \\
     \textbf{b.~~} $n=20$ \\
     \textbf{c.~~} $n=21$ \\
     \textbf{d.~~} $n=22$ \\
     \item \textbf{Question 4 :}
     Soit $g$ la fonction définie sur l'intervalle $]0~;~+\infty[$ par :
     \[ g(x)=\dfrac{1}{x} \]
     On note $\mathscr{C}$ sa courbe représentative dans un repère orthogonal du plan.
     \par
     L'aire $\mathscr{A}$, exprimée en unités d'aire, du domaine délimité par la courbe $\mathscr{C}$, l'axe des abscisses et les droites d'équations respectives $x=1$ et $x=2$ est :
     \par
     \textbf{a.~~} $ \mathscr{A} = \dfrac{1}{2}$
     \par
     \textbf{b.~~} $ \mathscr{A} = \dfrac{1}{4}$
     \par
     \textbf{c.~~} $ \mathscr{A} =\ln 2$ \\
     \textbf{d.~~} $ \mathscr{A} = \text{e}^{2}$ \\
     \item \textbf{Question 5 :}
     Une variable aléatoire $X$ suit la loi uniforme sur l'intervalle $[1~;~5]$.
     \par
     Alors :
     \par
     \textbf{a.~~} $p(1 < X < 5) = p(2 \leqslant X \leqslant 4)$\\
     \textbf{b.~~} $p( X < 2) = p(X > 2)$
     \par
     \textbf{c.~~} $p(X=3) = \dfrac{1}{2}$
     \par
     \textbf{d.~~} $p(2 \leqslant X \leqslant 3) = \dfrac{1}{4}$ \\
\end{itemize}
\begin{corrige}
     \begin{itemize}
          % =============================================================================================================================
          \item \textbf{Question 1 :}
          Réponse correcte :\quad\textbf{ d.}
          \par
          Pour tous réels $x$ et $y$ strictement positifs : $\ln(x y)=\ln(x)+\ln(y)$. Donc :
          \par
          $\ln(2 \text{e}^{2})=\ln2+\ln(\text{e}^{2})$.
          \par
          Or, pour tout réel $x$: $\ln(\text{e}^{x})=x$. Par conséquent :
          \par
          $\ln(2 \text{e}^{2})=\ln2+2$.
          \par
          % =============================================================================================================================
          \item \textbf{Question 2 :}
          Réponse correcte :\quad\textbf{ b.}
          \par
          Posons, pour tout réel $x$ strictement positif :
          \[u(x)=x\  \:\text{et}\: \ v(x)=\ln x.\]
          Alors :
          \[u'(x)=1\ \:\text{et}\: \ v'(x)=\dfrac{1}{x}.\]
          Donc :
          \par
          $f'(x)=u'(x)v(x)+u(x)v'(x)=\ln x+x \times \dfrac{1}{x}=\ln x + 1$.
          \par
          % =============================================================================================================================
          \item \textbf{Question 3 :}
          Réponse correcte :\quad\textbf{ c.}
          \par
          La fonction $\ln$ étant strictement croissante sur $]0~;~+\infty[$ :
          \par
          $0,8^n \leqslant 0,01 	 \Leftrightarrow \ln(0,8^n) \leqslant \ln(0,01) $ \\
          $\phantom{0,8^n \leqslant 0,01 }	 \Leftrightarrow n\ln(0,8) \leqslant \ln(0,01). $
          \par
          $0 < 0,8 < 1$ donc $\ln(0,8)$ est strictement négatif. Par conséquent :
          \par
          $0,8^n \leqslant 0,01  \Leftrightarrow n \geqslant \dfrac{\ln(0,01)}{\ln(0,8)} $.
          \par
          \`A la calculatrice $\dfrac{\ln(0,01)}{\ln(0,8)} \approx 20,63$ (au centième près). Le plus petit entier naturel supérieur ou égal à 20,63 est 21.
          \par
          Le plus petit entier naturel $n$ tel que $0,8^n \leqslant 0,01$ est donc 21.
          \par
          \cadre{rouge}{Attention}{
               Lorsque $0 < \alpha < 1$, $\ln(\alpha)$ est strictement \textbf{négatif}.
               \par
               Il faut donc penser à \textbf{changer le sens de l'inégalité} lorsque l'on divise par $\ln(\alpha)$.
          }
          \par
          %=============================================================================================================================
          \item \textbf{Question 4 :}
          Réponse correcte :\quad\textbf{ c.}
          \par
          L'aire $\mathscr{A}$, exprimée en unités d'aire, du domaine délimité par la courbe représentative de la fonction $x \longmapsto \dfrac{1}{x}$, l'axe des abscisses et les droites d'équations respectives $x=1$ et $x=2$ est égale à :
          \par
          $\mathscr{A}=\displaystyle\int_{1}^{2}\dfrac{1}{t}\text{d}t$.
          \par
          Une primitive de la fonction $x \longmapsto \dfrac{1}{x}$ sur l'intervalle $]0~;~+\infty[$ est la fonction $\ln$, donc :
          \par
          $\mathscr{A}=\left[\ln x \right]_1^2=\ln 2 - \ln 1=\ln 2$.
          \par
          %=============================================================================================================================
          \item \textbf{Question 5 :}
          Réponse correcte :\quad\textbf{ d.}
          \par
          $p(2 \leqslant X \leqslant 3)=\dfrac{3-2}{5-1}=\dfrac{1}{4}$.
          \par
          \cadre{rouge}{À retenir}{
               Si la variable aléatoire $X$ suit une \textbf{loi uniforme} sur l'intervalle $\left[a;b\right]$, alors pour tous réels $c$ et $d$ compris entre $a$ et $b$ avec $c \leqslant d$ :
               \[p\left(c\leqslant X\leqslant d\right) = \frac{d-c}{b-a}. \]
               \par
          }
          \par
          Les autres réponses sont fausses. En effet :
          \par
          \textbf{a.}
          $p(1<X<5)=\dfrac{5-1}{5-1}=1$
          \par
          tandis que :
          \par
          $p(2 \leqslant X \leqslant 4)=\dfrac{4-2}{5-1}=\dfrac{1}{2}$.
          \par
          \textbf{b.}
          $p(X < 2)=p(1\leqslant X<2)=\dfrac{2-1}{5-1}=\dfrac{1}{4}$
          \par
          alors que :
          \par
          $p(X > 2)=p(2 < X\leqslant 5)=\dfrac{5-2}{5-1}=\dfrac{3}{4}$.
          \par
          \textbf{c.}
          $p(X=3)=0$.
          \par
          \cadre{rouge}{À retenir}{
               Pour une \textbf{loi continue} (loi uniforme, loi normale...), les événements du type $(X=k)$ ont une probabilité nulle.
          }
          \par
          \cadre{vert}{En pratique}{
               Le jour du bac, il suffit de justifier la réponse correcte. Il n'est pas nécessaire de prouver que les autres réponses sont erronées.
          }
          \par
     \end{itemize}
\end{corrige}

\end{document}
µ
\documentclass[a4paper]{article}

%================================================================================================================================
%
% Packages
%
%================================================================================================================================

\usepackage[T1]{fontenc} 	% pour caractères accentués
\usepackage[utf8]{inputenc}  % encodage utf8
\usepackage[french]{babel}	% langue : français
\usepackage{fourier}			% caractères plus lisibles
\usepackage[dvipsnames]{xcolor} % couleurs
\usepackage{fancyhdr}		% réglage header footer
\usepackage{needspace}		% empêcher sauts de page mal placés
\usepackage{graphicx}		% pour inclure des graphiques
\usepackage{enumitem,cprotect}		% personnalise les listes d'items (nécessaire pour ol, al ...)
\usepackage{hyperref}		% Liens hypertexte
\usepackage{pstricks,pst-all,pst-node,pstricks-add,pst-math,pst-plot,pst-tree,pst-eucl} % pstricks
\usepackage[a4paper,includeheadfoot,top=2cm,left=3cm, bottom=2cm,right=3cm]{geometry} % marges etc.
\usepackage{comment}			% commentaires multilignes
\usepackage{amsmath,environ} % maths (matrices, etc.)
\usepackage{amssymb,makeidx}
\usepackage{bm}				% bold maths
\usepackage{tabularx}		% tableaux
\usepackage{colortbl}		% tableaux en couleur
\usepackage{fontawesome}		% Fontawesome
\usepackage{environ}			% environment with command
\usepackage{fp}				% calculs pour ps-tricks
\usepackage{multido}			% pour ps tricks
\usepackage[np]{numprint}	% formattage nombre
\usepackage{tikz,tkz-tab} 			% package principal TikZ
\usepackage{pgfplots}   % axes
\usepackage{mathrsfs}    % cursives
\usepackage{calc}			% calcul taille boites
\usepackage[scaled=0.875]{helvet} % font sans serif
\usepackage{svg} % svg
\usepackage{scrextend} % local margin
\usepackage{scratch} %scratch
\usepackage{multicol} % colonnes
%\usepackage{infix-RPN,pst-func} % formule en notation polanaise inversée
\usepackage{listings}

%================================================================================================================================
%
% Réglages de base
%
%================================================================================================================================

\lstset{
language=Python,   % R code
literate=
{á}{{\'a}}1
{à}{{\`a}}1
{ã}{{\~a}}1
{é}{{\'e}}1
{è}{{\`e}}1
{ê}{{\^e}}1
{í}{{\'i}}1
{ó}{{\'o}}1
{õ}{{\~o}}1
{ú}{{\'u}}1
{ü}{{\"u}}1
{ç}{{\c{c}}}1
{~}{{ }}1
}


\definecolor{codegreen}{rgb}{0,0.6,0}
\definecolor{codegray}{rgb}{0.5,0.5,0.5}
\definecolor{codepurple}{rgb}{0.58,0,0.82}
\definecolor{backcolour}{rgb}{0.95,0.95,0.92}

\lstdefinestyle{mystyle}{
    backgroundcolor=\color{backcolour},   
    commentstyle=\color{codegreen},
    keywordstyle=\color{magenta},
    numberstyle=\tiny\color{codegray},
    stringstyle=\color{codepurple},
    basicstyle=\ttfamily\footnotesize,
    breakatwhitespace=false,         
    breaklines=true,                 
    captionpos=b,                    
    keepspaces=true,                 
    numbers=left,                    
xleftmargin=2em,
framexleftmargin=2em,            
    showspaces=false,                
    showstringspaces=false,
    showtabs=false,                  
    tabsize=2,
    upquote=true
}

\lstset{style=mystyle}


\lstset{style=mystyle}
\newcommand{\imgdir}{C:/laragon/www/newmc/assets/imgsvg/}
\newcommand{\imgsvgdir}{C:/laragon/www/newmc/assets/imgsvg/}

\definecolor{mcgris}{RGB}{220, 220, 220}% ancien~; pour compatibilité
\definecolor{mcbleu}{RGB}{52, 152, 219}
\definecolor{mcvert}{RGB}{125, 194, 70}
\definecolor{mcmauve}{RGB}{154, 0, 215}
\definecolor{mcorange}{RGB}{255, 96, 0}
\definecolor{mcturquoise}{RGB}{0, 153, 153}
\definecolor{mcrouge}{RGB}{255, 0, 0}
\definecolor{mclightvert}{RGB}{205, 234, 190}

\definecolor{gris}{RGB}{220, 220, 220}
\definecolor{bleu}{RGB}{52, 152, 219}
\definecolor{vert}{RGB}{125, 194, 70}
\definecolor{mauve}{RGB}{154, 0, 215}
\definecolor{orange}{RGB}{255, 96, 0}
\definecolor{turquoise}{RGB}{0, 153, 153}
\definecolor{rouge}{RGB}{255, 0, 0}
\definecolor{lightvert}{RGB}{205, 234, 190}
\setitemize[0]{label=\color{lightvert}  $\bullet$}

\pagestyle{fancy}
\renewcommand{\headrulewidth}{0.2pt}
\fancyhead[L]{maths-cours.fr}
\fancyhead[R]{\thepage}
\renewcommand{\footrulewidth}{0.2pt}
\fancyfoot[C]{}

\newcolumntype{C}{>{\centering\arraybackslash}X}
\newcolumntype{s}{>{\hsize=.35\hsize\arraybackslash}X}

\setlength{\parindent}{0pt}		 
\setlength{\parskip}{3mm}
\setlength{\headheight}{1cm}

\def\ebook{ebook}
\def\book{book}
\def\web{web}
\def\type{web}

\newcommand{\vect}[1]{\overrightarrow{\,\mathstrut#1\,}}

\def\Oij{$\left(\text{O}~;~\vect{\imath},~\vect{\jmath}\right)$}
\def\Oijk{$\left(\text{O}~;~\vect{\imath},~\vect{\jmath},~\vect{k}\right)$}
\def\Ouv{$\left(\text{O}~;~\vect{u},~\vect{v}\right)$}

\hypersetup{breaklinks=true, colorlinks = true, linkcolor = OliveGreen, urlcolor = OliveGreen, citecolor = OliveGreen, pdfauthor={Didier BONNEL - https://www.maths-cours.fr} } % supprime les bordures autour des liens

\renewcommand{\arg}[0]{\text{arg}}

\everymath{\displaystyle}

%================================================================================================================================
%
% Macros - Commandes
%
%================================================================================================================================

\newcommand\meta[2]{    			% Utilisé pour créer le post HTML.
	\def\titre{titre}
	\def\url{url}
	\def\arg{#1}
	\ifx\titre\arg
		\newcommand\maintitle{#2}
		\fancyhead[L]{#2}
		{\Large\sffamily \MakeUppercase{#2}}
		\vspace{1mm}\textcolor{mcvert}{\hrule}
	\fi 
	\ifx\url\arg
		\fancyfoot[L]{\href{https://www.maths-cours.fr#2}{\black \footnotesize{https://www.maths-cours.fr#2}}}
	\fi 
}


\newcommand\TitreC[1]{    		% Titre centré
     \needspace{3\baselineskip}
     \begin{center}\textbf{#1}\end{center}
}

\newcommand\newpar{    		% paragraphe
     \par
}

\newcommand\nosp {    		% commande vide (pas d'espace)
}
\newcommand{\id}[1]{} %ignore

\newcommand\boite[2]{				% Boite simple sans titre
	\vspace{5mm}
	\setlength{\fboxrule}{0.2mm}
	\setlength{\fboxsep}{5mm}	
	\fcolorbox{#1}{#1!3}{\makebox[\linewidth-2\fboxrule-2\fboxsep]{
  		\begin{minipage}[t]{\linewidth-2\fboxrule-4\fboxsep}\setlength{\parskip}{3mm}
  			 #2
  		\end{minipage}
	}}
	\vspace{5mm}
}

\newcommand\CBox[4]{				% Boites
	\vspace{5mm}
	\setlength{\fboxrule}{0.2mm}
	\setlength{\fboxsep}{5mm}
	
	\fcolorbox{#1}{#1!3}{\makebox[\linewidth-2\fboxrule-2\fboxsep]{
		\begin{minipage}[t]{1cm}\setlength{\parskip}{3mm}
	  		\textcolor{#1}{\LARGE{#2}}    
 	 	\end{minipage}  
  		\begin{minipage}[t]{\linewidth-2\fboxrule-4\fboxsep}\setlength{\parskip}{3mm}
			\raisebox{1.2mm}{\normalsize\sffamily{\textcolor{#1}{#3}}}						
  			 #4
  		\end{minipage}
	}}
	\vspace{5mm}
}

\newcommand\cadre[3]{				% Boites convertible html
	\par
	\vspace{2mm}
	\setlength{\fboxrule}{0.1mm}
	\setlength{\fboxsep}{5mm}
	\fcolorbox{#1}{white}{\makebox[\linewidth-2\fboxrule-2\fboxsep]{
  		\begin{minipage}[t]{\linewidth-2\fboxrule-4\fboxsep}\setlength{\parskip}{3mm}
			\raisebox{-2.5mm}{\sffamily \small{\textcolor{#1}{\MakeUppercase{#2}}}}		
			\par		
  			 #3
 	 		\end{minipage}
	}}
		\vspace{2mm}
	\par
}

\newcommand\bloc[3]{				% Boites convertible html sans bordure
     \needspace{2\baselineskip}
     {\sffamily \small{\textcolor{#1}{\MakeUppercase{#2}}}}    
		\par		
  			 #3
		\par
}

\newcommand\CHelp[1]{
     \CBox{Plum}{\faInfoCircle}{À RETENIR}{#1}
}

\newcommand\CUp[1]{
     \CBox{NavyBlue}{\faThumbsOUp}{EN PRATIQUE}{#1}
}

\newcommand\CInfo[1]{
     \CBox{Sepia}{\faArrowCircleRight}{REMARQUE}{#1}
}

\newcommand\CRedac[1]{
     \CBox{PineGreen}{\faEdit}{BIEN R\'EDIGER}{#1}
}

\newcommand\CError[1]{
     \CBox{Red}{\faExclamationTriangle}{ATTENTION}{#1}
}

\newcommand\TitreExo[2]{
\needspace{4\baselineskip}
 {\sffamily\large EXERCICE #1\ (\emph{#2 points})}
\vspace{5mm}
}

\newcommand\img[2]{
          \includegraphics[width=#2\paperwidth]{\imgdir#1}
}

\newcommand\imgsvg[2]{
       \begin{center}   \includegraphics[width=#2\paperwidth]{\imgsvgdir#1} \end{center}
}


\newcommand\Lien[2]{
     \href{#1}{#2 \tiny \faExternalLink}
}
\newcommand\mcLien[2]{
     \href{https~://www.maths-cours.fr/#1}{#2 \tiny \faExternalLink}
}

\newcommand{\euro}{\eurologo{}}

%================================================================================================================================
%
% Macros - Environement
%
%================================================================================================================================

\newenvironment{tex}{ %
}
{%
}

\newenvironment{indente}{ %
	\setlength\parindent{10mm}
}

{
	\setlength\parindent{0mm}
}

\newenvironment{corrige}{%
     \needspace{3\baselineskip}
     \medskip
     \textbf{\textsc{Corrigé}}
     \medskip
}
{
}

\newenvironment{extern}{%
     \begin{center}
     }
     {
     \end{center}
}

\NewEnviron{code}{%
	\par
     \boite{gray}{\texttt{%
     \BODY
     }}
     \par
}

\newenvironment{vbloc}{% boite sans cadre empeche saut de page
     \begin{minipage}[t]{\linewidth}
     }
     {
     \end{minipage}
}
\NewEnviron{h2}{%
    \needspace{3\baselineskip}
    \vspace{0.6cm}
	\noindent \MakeUppercase{\sffamily \large \BODY}
	\vspace{1mm}\textcolor{mcgris}{\hrule}\vspace{0.4cm}
	\par
}{}

\NewEnviron{h3}{%
    \needspace{3\baselineskip}
	\vspace{5mm}
	\textsc{\BODY}
	\par
}

\NewEnviron{margeneg}{ %
\begin{addmargin}[-1cm]{0cm}
\BODY
\end{addmargin}
}

\NewEnviron{html}{%
}

\begin{document}
\meta{url}{/exercices/probabilites-et-suites-bac-blanc-es-l-sujet-6-maths-cours-2018/}
\meta{pid}{10592}
\meta{titre}{Probabilités et suites - Bac blanc ES/L Sujet 6 - Maths-cours 2018}
\meta{type}{exercices}
%
\begin{h2}Exercice 2 (5 points)\end{h2}
\par
Lors d'un tournoi de jeux vidéo, Loïc dispute plusieurs parties d'affilée.
\par
La probabilité qu'il gagne la première partie est 0,5.
\par
Lorsqu'il gagne une partie, la probabilité qu'il gagne la suivante est 0,7.
\par
Si, par contre, il perd une partie, la probabilité qu'il gagne la suivante est seulement 0,4.
\par
Pour tout entier naturel $n$ supérieur ou égal à 1, on note $G_n$ l'événement \og Loïc a gagné la $n$-ième partie \fg{}, $\overline{G_n}$ l'événement contraire et $p_n$ la probabilité de l'événement $G_n$. On a donc $p_1=0,5$.
\par
\begin{enumerate}
     \item
     Recopier et compléter l'arbre ci-après :
     \par
     %:-+-+-+- Engendré par : http://math.et.info.free.fr/TikZ/Arbre/
     \begin{center}
          \begin{extern}%width="340" alt="Arbre de probabilités à compléter"
               % Racine à Gauche, développement vers la droite
               \begin{tikzpicture}[xscale=1,yscale=1]
                    % Styles (MODIFIABLES)
                    \tikzstyle{fleche}=[-,>=latex,thick]
                    \tikzstyle{noeud}=[fill=white,circle,draw]
                    \tikzstyle{feuille}=[fill=white,circle,draw]
                    \tikzstyle{etiquette}=[midway,fill=white]
                    % Dimensions (MODIFIABLES)
                    \def\DistanceInterNiveaux{3}
                    \def\DistanceInterFeuilles{2}
                    % Dimensions calculées (NON MODIFIABLES)
                    \def\NiveauA{(0)*\DistanceInterNiveaux}
                    \def\NiveauB{(1.5)*\DistanceInterNiveaux}
                    \def\NiveauC{(2.5)*\DistanceInterNiveaux}
                    \def\InterFeuilles{(-1)*\DistanceInterFeuilles}
                    % Noeuds (MODIFIABLES : Styles et Coefficients d'InterFeuilles)
                    \node[noeud] (R) at ({\NiveauA},{(1.5)*\InterFeuilles}) {$\ $};
                    \node[noeud] (Ra) at ({\NiveauB},{(0.5)*\InterFeuilles}) {$G_n$};
                    \node[feuille] (Raa) at ({\NiveauC},{(0)*\InterFeuilles}) {$G_{n+1}$};
                    \node[feuille] (Rab) at ({\NiveauC},{(1)*\InterFeuilles}) {$\overline{G_{n+1}}$};
                    \node[noeud] (Rb) at ({\NiveauB},{(2.5)*\InterFeuilles}) {$\overline{G_n}$};
                    \node[feuille] (Rba) at ({\NiveauC},{(2)*\InterFeuilles}) {$G_{n+1}$};
                    \node[feuille] (Rbb) at ({\NiveauC},{(3)*\InterFeuilles}) {$\overline{G_{n+1}}$};
                    % Arcs (MODIFIABLES : Styles)
                    \draw[fleche] (R)--(Ra) node[etiquette] {$p_n$};
                    \draw[fleche] (Ra)--(Raa) node[etiquette] {$\cdots$};
                    \draw[fleche] (Ra)--(Rab) node[etiquette] {$\cdots$};
                    \draw[fleche] (R)--(Rb) node[etiquette] {$1-p_n$};
                    \draw[fleche] (Rb)--(Rba) node[etiquette] {$\cdots$};
                    \draw[fleche] (Rb)--(Rbb) node[etiquette] {$\cdots$};
               \end{tikzpicture}
          \end{extern}
     \end{center}
     %:-+-+-+-+- Fin
     \item
     Montrer que, pour tout entier naturel $n \geqslant 1$ :
     \[ p_{n+1}=0,3p_n+0,4. \]
     \item
     On considère la suite $(u_n)$ définie, pour tout entier naturel $n$ supérieur ou égal à $1$, par :
     \[ u_n=p_n-\dfrac{4}{7}. \]
     \par
     \begin{enumerate}[label=\alph*.]
          \item
          Montrer que la suite $(u_n)$ est une suite géométrique dont on précisera la raison et le premier terme $u_1$.
          \item
          En déduire que, pour tout entier naturel $n$ supérieur ou égal à $1$ :
          \[ p_n=\dfrac{4}{7} - \dfrac{1}{14} \times (0,3)^{n-1}. \]
          \par
     \end{enumerate}
     \item
     Déterminer la limite de la suite $(p_n)$.
     \par
     Interpréter ce résultat dans le contexte de l'exercice.
     \par
\end{enumerate}
\begin{corrige}
     \begin{enumerate}
          \item
          D'après l'énoncé :
          \par
          \begin{itemize}
               \item
               \og Lorsqu'il gagne une partie, la probabilité qu'il gagne la suivante est 0,7 \fg{} : donc $\ {p_{G_n}(G_{n+1})=0,7}$.
               \item
               \og Si, par contre, il perd une partie, la probabilité qu'il gagne la suivante est seulement 0,4 \fg{} ; donc $\ {p_{\overline{G_n}}(G_{n+1})=0,4}$.
               \par
          \end{itemize}
          \par
          La somme des probabilités inscrites sur les branches partant d'un même nœud est égale à 1 ; on peut donc compléter l'arbre comme suit :
          \par
          %:-+-+-+- Engendré par : http://math.et.info.free.fr/TikZ/Arbre/
          \begin{center}
               \begin{extern}%width="340" alt="Arbre de probabilités complété"
                    % Racine à Gauche, développement vers la droite
                    \begin{tikzpicture}[xscale=1,yscale=1]
                         % Styles (MODIFIABLES)
                         \tikzstyle{fleche}=[-,>=latex,thick]
                         \tikzstyle{noeud}=[fill=white,circle,draw]
                         \tikzstyle{feuille}=[fill=white,circle,draw]
                         \tikzstyle{etiquette}=[midway,fill=white]
                         % Dimensions (MODIFIABLES)
                         \def\DistanceInterNiveaux{3}
                         \def\DistanceInterFeuilles{2}
                         % Dimensions calculées (NON MODIFIABLES)
                         \def\NiveauA{(0)*\DistanceInterNiveaux}
                         \def\NiveauB{(1.5)*\DistanceInterNiveaux}
                         \def\NiveauC{(2.5)*\DistanceInterNiveaux}
                         \def\InterFeuilles{(-1)*\DistanceInterFeuilles}
                         % Noeuds (MODIFIABLES : Styles et Coefficients d'InterFeuilles)
                         \node[noeud] (R) at ({\NiveauA},{(1.5)*\InterFeuilles}) {$\ $};
                         \node[noeud] (Ra) at ({\NiveauB},{(0.5)*\InterFeuilles}) {$G_n$};
                         \node[feuille] (Raa) at ({\NiveauC},{(0)*\InterFeuilles}) {$G_{n+1}$};
                         \node[feuille] (Rab) at ({\NiveauC},{(1)*\InterFeuilles}) {$\overline{G_{n+1}}$};
                         \node[noeud] (Rb) at ({\NiveauB},{(2.5)*\InterFeuilles}) {$\overline{G_n}$};
                         \node[feuille] (Rba) at ({\NiveauC},{(2)*\InterFeuilles}) {$G_{n+1}$};
                         \node[feuille] (Rbb) at ({\NiveauC},{(3)*\InterFeuilles}) {$\overline{G_{n+1}}$};
                         % Arcs (MODIFIABLES : Styles)
                         \draw[fleche] (R)--(Ra) node[etiquette] {$p_n$};
                         \draw[fleche] (Ra)--(Raa) node[etiquette] {\textcolor{red}{$0,7$}};
                         \draw[fleche] (Ra)--(Rab) node[etiquette] {\textcolor{red}{$0,3$}};
                         \draw[fleche] (R)--(Rb) node[etiquette] {$1-p_n$};
                         \draw[fleche] (Rb)--(Rba) node[etiquette] {\textcolor{red}{$0,4$}};
                         \draw[fleche] (Rb)--(Rbb) node[etiquette] {\textcolor{red}{$0,6$}};
                    \end{tikzpicture}
               \end{extern}
          \end{center}
          %:-+-+-+-+- Fin
          \item
          $p_{n+1}$ représente $p(G_{n+1})$. D'après la formule des probabilités totales :
          \par
          $p_{n+1} = p(G_n) \times p_{G_n}(G_{n+1}) + p(\overline{G_n}) \times  p_{\overline{G_n}}(G_{n+1})$ \\
          $\phantom{p_{n+1}} =  p_n \times 0,7 + (1-p_n) \times  0,4 $\\
          $\phantom{p_{n+1}} = 0,7p_n + 0,4-0,4p_n $\\
          $\phantom{p_{n+1}} = 0,3p_n+0,4$
          \par
          \cadre{bleu}{Remarque}{
               La suite $(p_n)$ vérifie une relation de récurrence de la forme : $p_{n+1}=ap_n+b$ ; c'est donc une suite arithmético-géométrique.
          }
          \item %3
          \textit{Pour plus de détails sur la méthode employée dans cette question se reporter à la \hyperlink{suite-ag-pap}{page \pageref*{suite-ag-pap}} : \og \'Etude d'une suite arithmético-géométrique étape par étape \fg{}.}
          \begin{enumerate}[label=\alph*.]
               \item %a
               Pour tout entier naturel $n$ strictement positif : $u_n=p_n-\dfrac{4}{7}$ ; par conséquent :
               \par
               $u_{n+1} = p_{n+1}-\dfrac{4}{7}$ \\
               $\phantom{u_{n+1}} = (0,3p_n+0,4)-\dfrac{4}{7}$ \\
               $\phantom{u_{n+1}}= 0,3p_n+\dfrac{2,8}{7}-\dfrac{4}{7}$ \\
               $\phantom{u_{n+1}}= 0,3p_n-\dfrac{1,2}{7}$.
               \par
               Or $u_n=p_n-\dfrac{4}{7}$ donc $p_n=u_n+\dfrac{4}{7}$ ; on obtient donc :
               \par
               $u_{n+1}  = 0,3\left(u_n+\dfrac{4}{7}\right)-\dfrac{1,2}{7}$\\
               $\phantom{u_{n+1}}= 0,3u_n+\dfrac{1,2}{7}-\dfrac{1,2}{7}$\\
               $\phantom{u_{n+1}} = 0,3u_n.$
               \par
               Par ailleurs, ${u_1=p_1-\dfrac{4}{7}=\dfrac{1}{2}-\dfrac{4}{7}=-\dfrac{1}{14}}$.
               \par
               La suite $(u_n)$ est donc une suite géométrique de premier terme ${u_1=-\dfrac{1}{14}}$ et de raison ${q=0,3}$.
               \item %b
               On en déduit que pour tout entier naturel $n$ strictement positif:
               \par
               $u_n=u_1q^{n-1}=-\dfrac{1}{14} \times 0,3^{n-1}$.
               \par
               \cadre{rouge}{À retenir}{
                    Pour une suite \textbf{géométrique} $(u_n)$ de premier terme $u_0$ et de raison $q$, le $n$-ième terme vaut :
                    \[u_{n}=u_0 \times q^n.\]
                    \par
                    Pour une suite \textbf{géométrique} $(u_n)$ de premier terme $u_1$ et de raison $q$, le $n$-ième terme vaut :
                    \[u_{n}=u_1 \times q^{n-1}.\]
                    \par
                    Plus généralement, pour une suite \textbf{géométrique} $(u_n)$ de raison $q$ dont on connait le terme $u_p$, le $n$-ième terme vaut :
                    \[u_{n}=u_p \times q^{n-p}.\]
               }
               \par
               En utilisant la relation $p_n=u_n+\dfrac{4}{7}$, on obtient :
               \[ p_n=\dfrac{4}{7} - \dfrac{1}{14} \times 0,3^{n-1}. \]
               \vspace{2mm}
          \end{enumerate}
          \item
          Comme $0 \leqslant 0,3 < 1$, alors $\lim\limits_{n \rightarrow +\infty}0,3^n=0$.
          \par
          $0,3^{n-1} = \dfrac{0,3^n}{0,3}$ donc on a également  $\lim\limits_{n \rightarrow +\infty}0,3^{n-1}=0$.
          \par
          Par conséquent $\lim\limits_{n \rightarrow +\infty} \left(\dfrac{1}{14} \times 0,3^{n-1}\right)=0$ et :
          \[ \lim\limits_{n \rightarrow +\infty} \left(\dfrac{4}{7} - \dfrac{1}{14} \times 0,3^{n-1}\right)=\dfrac{4}{7}. \]
          \par
          La suite $(p_n)$ converge donc vers $\dfrac{4}{7}$.
          \par
          Lorsque Loïc a joué beaucoup de parties, sa probabilité de gagner une partie est proche de $\dfrac{4}{7}$.
          \par
          \cadre{rouge}{À retenir}{
               Soit $q$ un nombre réel positif ou nul.
               \par
               \begin{itemize}
                    \item %
                    Si $\bm{0 \leqslant q < 1}$, alors $\lim\limits_{n \rightarrow +\infty}q^n=\bm{0}$.
                    \item %
                    Si $\bm{q > 1}$, alors $\lim\limits_{n \rightarrow +\infty}q^n=\bm{+\infty}$.
               \end{itemize}
          }
     \end{enumerate}
\end{corrige}

\end{document}
µ
\documentclass[a4paper]{article}

%================================================================================================================================
%
% Packages
%
%================================================================================================================================

\usepackage[T1]{fontenc} 	% pour caractères accentués
\usepackage[utf8]{inputenc}  % encodage utf8
\usepackage[french]{babel}	% langue : français
\usepackage{fourier}			% caractères plus lisibles
\usepackage[dvipsnames]{xcolor} % couleurs
\usepackage{fancyhdr}		% réglage header footer
\usepackage{needspace}		% empêcher sauts de page mal placés
\usepackage{graphicx}		% pour inclure des graphiques
\usepackage{enumitem,cprotect}		% personnalise les listes d'items (nécessaire pour ol, al ...)
\usepackage{hyperref}		% Liens hypertexte
\usepackage{pstricks,pst-all,pst-node,pstricks-add,pst-math,pst-plot,pst-tree,pst-eucl} % pstricks
\usepackage[a4paper,includeheadfoot,top=2cm,left=3cm, bottom=2cm,right=3cm]{geometry} % marges etc.
\usepackage{comment}			% commentaires multilignes
\usepackage{amsmath,environ} % maths (matrices, etc.)
\usepackage{amssymb,makeidx}
\usepackage{bm}				% bold maths
\usepackage{tabularx}		% tableaux
\usepackage{colortbl}		% tableaux en couleur
\usepackage{fontawesome}		% Fontawesome
\usepackage{environ}			% environment with command
\usepackage{fp}				% calculs pour ps-tricks
\usepackage{multido}			% pour ps tricks
\usepackage[np]{numprint}	% formattage nombre
\usepackage{tikz,tkz-tab} 			% package principal TikZ
\usepackage{pgfplots}   % axes
\usepackage{mathrsfs}    % cursives
\usepackage{calc}			% calcul taille boites
\usepackage[scaled=0.875]{helvet} % font sans serif
\usepackage{svg} % svg
\usepackage{scrextend} % local margin
\usepackage{scratch} %scratch
\usepackage{multicol} % colonnes
%\usepackage{infix-RPN,pst-func} % formule en notation polanaise inversée
\usepackage{listings}

%================================================================================================================================
%
% Réglages de base
%
%================================================================================================================================

\lstset{
language=Python,   % R code
literate=
{á}{{\'a}}1
{à}{{\`a}}1
{ã}{{\~a}}1
{é}{{\'e}}1
{è}{{\`e}}1
{ê}{{\^e}}1
{í}{{\'i}}1
{ó}{{\'o}}1
{õ}{{\~o}}1
{ú}{{\'u}}1
{ü}{{\"u}}1
{ç}{{\c{c}}}1
{~}{{ }}1
}


\definecolor{codegreen}{rgb}{0,0.6,0}
\definecolor{codegray}{rgb}{0.5,0.5,0.5}
\definecolor{codepurple}{rgb}{0.58,0,0.82}
\definecolor{backcolour}{rgb}{0.95,0.95,0.92}

\lstdefinestyle{mystyle}{
    backgroundcolor=\color{backcolour},   
    commentstyle=\color{codegreen},
    keywordstyle=\color{magenta},
    numberstyle=\tiny\color{codegray},
    stringstyle=\color{codepurple},
    basicstyle=\ttfamily\footnotesize,
    breakatwhitespace=false,         
    breaklines=true,                 
    captionpos=b,                    
    keepspaces=true,                 
    numbers=left,                    
xleftmargin=2em,
framexleftmargin=2em,            
    showspaces=false,                
    showstringspaces=false,
    showtabs=false,                  
    tabsize=2,
    upquote=true
}

\lstset{style=mystyle}


\lstset{style=mystyle}
\newcommand{\imgdir}{C:/laragon/www/newmc/assets/imgsvg/}
\newcommand{\imgsvgdir}{C:/laragon/www/newmc/assets/imgsvg/}

\definecolor{mcgris}{RGB}{220, 220, 220}% ancien~; pour compatibilité
\definecolor{mcbleu}{RGB}{52, 152, 219}
\definecolor{mcvert}{RGB}{125, 194, 70}
\definecolor{mcmauve}{RGB}{154, 0, 215}
\definecolor{mcorange}{RGB}{255, 96, 0}
\definecolor{mcturquoise}{RGB}{0, 153, 153}
\definecolor{mcrouge}{RGB}{255, 0, 0}
\definecolor{mclightvert}{RGB}{205, 234, 190}

\definecolor{gris}{RGB}{220, 220, 220}
\definecolor{bleu}{RGB}{52, 152, 219}
\definecolor{vert}{RGB}{125, 194, 70}
\definecolor{mauve}{RGB}{154, 0, 215}
\definecolor{orange}{RGB}{255, 96, 0}
\definecolor{turquoise}{RGB}{0, 153, 153}
\definecolor{rouge}{RGB}{255, 0, 0}
\definecolor{lightvert}{RGB}{205, 234, 190}
\setitemize[0]{label=\color{lightvert}  $\bullet$}

\pagestyle{fancy}
\renewcommand{\headrulewidth}{0.2pt}
\fancyhead[L]{maths-cours.fr}
\fancyhead[R]{\thepage}
\renewcommand{\footrulewidth}{0.2pt}
\fancyfoot[C]{}

\newcolumntype{C}{>{\centering\arraybackslash}X}
\newcolumntype{s}{>{\hsize=.35\hsize\arraybackslash}X}

\setlength{\parindent}{0pt}		 
\setlength{\parskip}{3mm}
\setlength{\headheight}{1cm}

\def\ebook{ebook}
\def\book{book}
\def\web{web}
\def\type{web}

\newcommand{\vect}[1]{\overrightarrow{\,\mathstrut#1\,}}

\def\Oij{$\left(\text{O}~;~\vect{\imath},~\vect{\jmath}\right)$}
\def\Oijk{$\left(\text{O}~;~\vect{\imath},~\vect{\jmath},~\vect{k}\right)$}
\def\Ouv{$\left(\text{O}~;~\vect{u},~\vect{v}\right)$}

\hypersetup{breaklinks=true, colorlinks = true, linkcolor = OliveGreen, urlcolor = OliveGreen, citecolor = OliveGreen, pdfauthor={Didier BONNEL - https://www.maths-cours.fr} } % supprime les bordures autour des liens

\renewcommand{\arg}[0]{\text{arg}}

\everymath{\displaystyle}

%================================================================================================================================
%
% Macros - Commandes
%
%================================================================================================================================

\newcommand\meta[2]{    			% Utilisé pour créer le post HTML.
	\def\titre{titre}
	\def\url{url}
	\def\arg{#1}
	\ifx\titre\arg
		\newcommand\maintitle{#2}
		\fancyhead[L]{#2}
		{\Large\sffamily \MakeUppercase{#2}}
		\vspace{1mm}\textcolor{mcvert}{\hrule}
	\fi 
	\ifx\url\arg
		\fancyfoot[L]{\href{https://www.maths-cours.fr#2}{\black \footnotesize{https://www.maths-cours.fr#2}}}
	\fi 
}


\newcommand\TitreC[1]{    		% Titre centré
     \needspace{3\baselineskip}
     \begin{center}\textbf{#1}\end{center}
}

\newcommand\newpar{    		% paragraphe
     \par
}

\newcommand\nosp {    		% commande vide (pas d'espace)
}
\newcommand{\id}[1]{} %ignore

\newcommand\boite[2]{				% Boite simple sans titre
	\vspace{5mm}
	\setlength{\fboxrule}{0.2mm}
	\setlength{\fboxsep}{5mm}	
	\fcolorbox{#1}{#1!3}{\makebox[\linewidth-2\fboxrule-2\fboxsep]{
  		\begin{minipage}[t]{\linewidth-2\fboxrule-4\fboxsep}\setlength{\parskip}{3mm}
  			 #2
  		\end{minipage}
	}}
	\vspace{5mm}
}

\newcommand\CBox[4]{				% Boites
	\vspace{5mm}
	\setlength{\fboxrule}{0.2mm}
	\setlength{\fboxsep}{5mm}
	
	\fcolorbox{#1}{#1!3}{\makebox[\linewidth-2\fboxrule-2\fboxsep]{
		\begin{minipage}[t]{1cm}\setlength{\parskip}{3mm}
	  		\textcolor{#1}{\LARGE{#2}}    
 	 	\end{minipage}  
  		\begin{minipage}[t]{\linewidth-2\fboxrule-4\fboxsep}\setlength{\parskip}{3mm}
			\raisebox{1.2mm}{\normalsize\sffamily{\textcolor{#1}{#3}}}						
  			 #4
  		\end{minipage}
	}}
	\vspace{5mm}
}

\newcommand\cadre[3]{				% Boites convertible html
	\par
	\vspace{2mm}
	\setlength{\fboxrule}{0.1mm}
	\setlength{\fboxsep}{5mm}
	\fcolorbox{#1}{white}{\makebox[\linewidth-2\fboxrule-2\fboxsep]{
  		\begin{minipage}[t]{\linewidth-2\fboxrule-4\fboxsep}\setlength{\parskip}{3mm}
			\raisebox{-2.5mm}{\sffamily \small{\textcolor{#1}{\MakeUppercase{#2}}}}		
			\par		
  			 #3
 	 		\end{minipage}
	}}
		\vspace{2mm}
	\par
}

\newcommand\bloc[3]{				% Boites convertible html sans bordure
     \needspace{2\baselineskip}
     {\sffamily \small{\textcolor{#1}{\MakeUppercase{#2}}}}    
		\par		
  			 #3
		\par
}

\newcommand\CHelp[1]{
     \CBox{Plum}{\faInfoCircle}{À RETENIR}{#1}
}

\newcommand\CUp[1]{
     \CBox{NavyBlue}{\faThumbsOUp}{EN PRATIQUE}{#1}
}

\newcommand\CInfo[1]{
     \CBox{Sepia}{\faArrowCircleRight}{REMARQUE}{#1}
}

\newcommand\CRedac[1]{
     \CBox{PineGreen}{\faEdit}{BIEN R\'EDIGER}{#1}
}

\newcommand\CError[1]{
     \CBox{Red}{\faExclamationTriangle}{ATTENTION}{#1}
}

\newcommand\TitreExo[2]{
\needspace{4\baselineskip}
 {\sffamily\large EXERCICE #1\ (\emph{#2 points})}
\vspace{5mm}
}

\newcommand\img[2]{
          \includegraphics[width=#2\paperwidth]{\imgdir#1}
}

\newcommand\imgsvg[2]{
       \begin{center}   \includegraphics[width=#2\paperwidth]{\imgsvgdir#1} \end{center}
}


\newcommand\Lien[2]{
     \href{#1}{#2 \tiny \faExternalLink}
}
\newcommand\mcLien[2]{
     \href{https~://www.maths-cours.fr/#1}{#2 \tiny \faExternalLink}
}

\newcommand{\euro}{\eurologo{}}

%================================================================================================================================
%
% Macros - Environement
%
%================================================================================================================================

\newenvironment{tex}{ %
}
{%
}

\newenvironment{indente}{ %
	\setlength\parindent{10mm}
}

{
	\setlength\parindent{0mm}
}

\newenvironment{corrige}{%
     \needspace{3\baselineskip}
     \medskip
     \textbf{\textsc{Corrigé}}
     \medskip
}
{
}

\newenvironment{extern}{%
     \begin{center}
     }
     {
     \end{center}
}

\NewEnviron{code}{%
	\par
     \boite{gray}{\texttt{%
     \BODY
     }}
     \par
}

\newenvironment{vbloc}{% boite sans cadre empeche saut de page
     \begin{minipage}[t]{\linewidth}
     }
     {
     \end{minipage}
}
\NewEnviron{h2}{%
    \needspace{3\baselineskip}
    \vspace{0.6cm}
	\noindent \MakeUppercase{\sffamily \large \BODY}
	\vspace{1mm}\textcolor{mcgris}{\hrule}\vspace{0.4cm}
	\par
}{}

\NewEnviron{h3}{%
    \needspace{3\baselineskip}
	\vspace{5mm}
	\textsc{\BODY}
	\par
}

\NewEnviron{margeneg}{ %
\begin{addmargin}[-1cm]{0cm}
\BODY
\end{addmargin}
}

\NewEnviron{html}{%
}

\begin{document}
\meta{url}{/exercices/loi-normales-intervalle-de-confiance-bac-blanc-es-l-sujet-6-maths-cours-2018/}
\meta{pid}{10594}
\meta{titre}{Loi normales - Intervalle de confiance - Bac blanc ES/L Sujet 6 - Maths-cours 2018}
\meta{type}{exercices}
%
\begin{h2}Exercice 3 (5 points)\end{h2}
\par
Un producteur de pommes \og bio \fg{} calibre ses fruits à l'aide d'une machine qui les trie en fonction de leur diamètre.
\par
Les pommes dont le diamètre est compris entre 6~cm et 8~cm sont jugées \og conformes \fg{} et vendues en l'état. Les autres, dites \og hors calibre \fg{}, sont utilisées pour confectionner de la compote.
\par
On admet que la variable aléatoire $X$ qui donne le diamètre, en cm, d'une pomme prélevée au hasard chez ce producteur suit une loi normale d'espérance mathématique $\mu =7$ et d'écart-type $\sigma =1$.
\par
%============================================================================================================================
%
\TitreC{Partie A}
%
%============================================================================================================================
\par
\begin{enumerate}
     \item %1
     Montrer que la probabilité, arrondie au centième, qu'une pomme prélevée au hasard soit jugée \og conforme \fg{} est égale à 0,68.
     \item %2
     On sélectionne 500 pommes au hasard chez ce producteur.
     \par
     On note $Y$ la variable aléatoire égale au nombre de pommes jugées \og conformes \fg{} parmi les 500 fruits sélectionnés.
     \par
     La production est suffisamment importante pour assimiler cette sélection à un tirage aléatoire avec remise.
     \par
     \begin{enumerate}[label=\alph*.]
          \item %2a
          Justifier que la variable aléatoire $Y$ suit une loi binomiale dont on précisera les paramètres.
          \item %2b
          Quelle est la probabilité que, sur cet échantillon de 500~pommes, 350 au moins soient jugées \og conformes \fg{}.
     \end{enumerate}
     \item %3
     \`A l'aide de la calculatrice, déterminer, au centième près, le réel $k$ tel que :
     \[ p(X \geqslant k)=0,95. \]
     Interpréter ce résultat.
     \par
\end{enumerate}
\par
%============================================================================================================================
%
\TitreC{Partie B}
%
%============================================================================================================================
\par
Un distributeur souhaite estimer la proportion de consommateurs satisfaits par la qualité de ces pommes \og bio \fg{}.
\par
Il réalise pour cela un sondage auprès de 200 consommateurs. Sur ces 200 personnes interrogées, 170 se déclarent satisfaites.
\par
\begin{enumerate}
     \item %1
     Déterminer un intervalle de confiance, au niveau de confiance 0,95, de la proportion $p$ de consommateurs satisfaits par la qualité de ces fruits.\\
     Les bornes de l'intervalle seront arrondies au millième.
     \item %2
     Combien de personnes, au minimum, le distributeur aurait-il dû interroger pour obtenir un intervalle de confiance, au niveau de confiance 0,95, d'amplitude inférieure ou égale à 4\% ?
     \par
\end{enumerate}
\begin{corrige}
     %============================================================================================================================
     %
     \TitreC{Partie A}
     %
     %============================================================================================================================
     \par
     \begin{enumerate}
          \item %1
          D'après le cours :
          \[ p( \mu - \sigma \leqslant X \leqslant \mu + \sigma ) \approx 0,68 \ \text{au centième près} \]
          ce qui donne ici pour $ \mu =7$ et $\sigma =1$ :
          \[ p( 6 \leqslant X \leqslant 8 ) \approx 0,68 \ \text{au centième près}. \]
          Or, d'après l'énoncé, l'événement $( 6 \leqslant X \leqslant 8 )$ est identique à l'événement \og la pomme est jugée conforme \fg{}.
          \par
          \cadre{rouge}{À retenir}{
               Si la variable aléatoire $X$ suit une loi normale d'espérance $\mu$ et d'écart-type $\sigma$, alors :
               \par
               \begin{itemize}
                    \item  $p\left(\mu -\sigma \leqslant X\leqslant \mu + \sigma \right)\approx 0,68$ (à $10^{-2}$ près) ;
                    \item  $p\left(\mu -2\sigma \leqslant X\leqslant \mu + 2\sigma \right)\approx 0,95$ (à $10^{-2}$ près) ;
                    \item  $p\left(\mu -3\sigma \leqslant X\leqslant \mu + 3\sigma \right)\approx 0,997$ (à $10^{-3}$ près).
               \end{itemize}
          }
          \par
          \cadre{bleu}{Remarque}{
               Le jour du bac, on ne vous retirera pas de points si vous utilisez la calculatrice pour répondre à cette question. Toutefois, il est toujours préférable de montrer que vous connaissez le cours...
          }
          \item %2
          \begin{enumerate}[label=\alph*.]
               \item %2a
               La variable aléatoire $Y$ suit une loi binomiale de paramètres ${n=500}$ et ${p=0,68}$. En effet :
               \par
               \begin{itemize}
                    \item on assimile la sélection à un tirage aléatoire avec remise ;
                    \item chaque tirage possède deux issues :
                    \begin{itemize}
                         \item \textit{succès}, correspondant à la sélection d'une pomme \og con\-forme \fg{} (probabilité ${p=0,68}$),
                         \item \textit{échec}, correspondant à la sélection d'une pomme \og non conforme \fg{} ;
                    \end{itemize}
                    \item la variable aléatoire $Y$ comptabilise le nombre de pommes \og conformes \fg{}.
               \end{itemize}
               \par
               \cadre{rouge}{Attention}{
                    Ne confondez pas, dans cet exercice, les variables aléatoires $X$ (qui suit une loi normale) et $Y$ (qui suit une loi binomiale)~!
               }
               \item %2b
               On cherche la probabilité de l'événement $(Y \geqslant 350)$. L'événement contraire de ${(Y \geqslant 350)}$ est ${(Y < 350)}$ qui est identique à ${(Y \leqslant 349)}$.
               \par
               \cadre{rouge}{Attention}{
                    L'événement contraire de ({$Y \geqslant a$}) est ({$Y < a$}) et non (${Y \leqslant a}$).
                    \par
                    Pour une loi \textbf{continue}, cela n'a aucun impact sur le résultat car alors ${p(Y < a)=p(Y \leqslant a)}$. Par contre, pour une loi \textbf{discrète} (comme une loi binomiale, par exemple), les probabilités {${p(Y < a)}$} et {${p(Y \leqslant a)}$} sont différentes.
               }
               \par
               On a donc :
               \par
               $p(Y \geqslant 350) = 1 - p(Y < 350) $\\
               $ \phantom{p(Y \geqslant 350) }    =1 - p(Y \leqslant 349).$
               \par
               \`A la calculatrice on trouve : $p(Y \leqslant 349) \approx 0,82$ (au centième près). Par conséquent :
               \par
               $p(Y \geqslant 350) \approx 1 - 0,82 = 0,18\ $ (au centième près).
               \par
          \end{enumerate}
          \par
          Ce résultat s'interprète de la façon suivante : \og la probabilité que le diamètre d'une pomme soit supérieur à 5,36 cm est égale à 0,95 \fg{}.
          \par
     \end{enumerate}
     \par
     %============================================================================================================================
     %
     \TitreC{Partie B}
     %
     %============================================================================================================================
     \par
     \begin{enumerate}
          \item %1
          La taille de l'échantillon est $n= 500$. \\La fréquence observée des personnes satisfaites dans cet échantillon est $f=\dfrac{17}{20}=0,85$.
          \par
          On vérifie que :
          \par
          \begin{itemize}
               \item $n=500 \geqslant 30$ ;
               \item $nf=500 \times 0,85=425 \geqslant 5$ ;
               \item $n(1-f)=500 \times 0,15=75 \geqslant 5$.
          \end{itemize}
          \par
          Les conditions de validité sont bien remplies ; l'intervalle de confiance, au niveau de confiance 0,95 est alors :
          \[ I=\left[f-\dfrac{1}{\sqrt{n}}~;~f+\dfrac{1)}{\sqrt{n}}\right] \]
          soit:
          \par
          $I =\left[0,85-\dfrac{1}{\sqrt{200}}~;~0,85+\dfrac{1}{\sqrt{200}}\right]$
          \par
          $I \approx \left[0,779~;~0,921\right] \quad \text{(extrémités arrondies au millième)}$.
          \par
          \cadre{rouge}{À retenir}{
               On note :
               \begin{itemize}
                    \item %
                    $n$ : la taille  de l'\textbf{échantillon},
                    \item %
                    $f$ : la fréquence du caractère dans l'\textbf{échan\-til\-lon},
                    \item %
                    $p$ : la proportion (connue ou supposée) du caractère dans la \textbf{population}.
               \end{itemize}
               \par
               Si les conditions $\bm{f\geqslant 30}$, $\bm{nf\geqslant 5}$ et $\bm{n\left(1-f\right)\geqslant 5}$ sont vérifiées, l'intervalle de confiance, au niveau de confiance de 95\%, est :
               \[  I=\left[ f-\frac{1}{\sqrt{n}}~ ; ~f+\frac{1}{\sqrt{n}} \right] \]
          }
          \item %2
          L'amplitude de l'intervalle de confiance, au niveau de confiance 0,95, pour un échantillon de taille $n$ est $\dfrac{2}{\sqrt{n}}$.
          \par
          Cette amplitude est inférieure ou égale à 4\% si et seulement si :
          \par
          $\dfrac{2}{\sqrt{n}} \leqslant 0,04 \Leftrightarrow 2 \leqslant 0,04\sqrt{n}$
          \par
          $\phantom{\dfrac{2}{\sqrt{n}} \leqslant 0,04} \Leftrightarrow 0,04\sqrt{n} \geqslant 2$
          \par
          $\phantom{\dfrac{2}{\sqrt{n}} \leqslant 0,04} \Leftrightarrow \sqrt{n} \geqslant \dfrac{2}{0,04}$
          \par
          $\phantom{\dfrac{2}{\sqrt{n}} \leqslant 0,04} \Leftrightarrow \sqrt{n} \geqslant 50$
          \par
          $\phantom{\dfrac{2}{\sqrt{n}} \leqslant 0,04} \Leftrightarrow n \geqslant 50^2$
          \par
          $\phantom{\dfrac{2}{\sqrt{n}} \leqslant 0,04} \Leftrightarrow n \geqslant 2\ 500$.
          \par
          Le distributeur aurait dû interroger, au minimum, 2\ 500 personnes pour obtenir un intervalle de confiance, au niveau de confiance 0,95, d'amplitude inférieure ou égale à 4\%.
          \par
          \cadre{rouge}{À retenir}{
               L'amplitude de l'intervalle de confiance, au niveau de confiance 95\%, pour un échantillon de taille $n$ est $\dfrac{2}{\sqrt{n}}$.
          }
          \par
     \end{enumerate}
\end{corrige}

\end{document}
µ
\documentclass[a4paper]{article}

%================================================================================================================================
%
% Packages
%
%================================================================================================================================

\usepackage[T1]{fontenc} 	% pour caractères accentués
\usepackage[utf8]{inputenc}  % encodage utf8
\usepackage[french]{babel}	% langue : français
\usepackage{fourier}			% caractères plus lisibles
\usepackage[dvipsnames]{xcolor} % couleurs
\usepackage{fancyhdr}		% réglage header footer
\usepackage{needspace}		% empêcher sauts de page mal placés
\usepackage{graphicx}		% pour inclure des graphiques
\usepackage{enumitem,cprotect}		% personnalise les listes d'items (nécessaire pour ol, al ...)
\usepackage{hyperref}		% Liens hypertexte
\usepackage{pstricks,pst-all,pst-node,pstricks-add,pst-math,pst-plot,pst-tree,pst-eucl} % pstricks
\usepackage[a4paper,includeheadfoot,top=2cm,left=3cm, bottom=2cm,right=3cm]{geometry} % marges etc.
\usepackage{comment}			% commentaires multilignes
\usepackage{amsmath,environ} % maths (matrices, etc.)
\usepackage{amssymb,makeidx}
\usepackage{bm}				% bold maths
\usepackage{tabularx}		% tableaux
\usepackage{colortbl}		% tableaux en couleur
\usepackage{fontawesome}		% Fontawesome
\usepackage{environ}			% environment with command
\usepackage{fp}				% calculs pour ps-tricks
\usepackage{multido}			% pour ps tricks
\usepackage[np]{numprint}	% formattage nombre
\usepackage{tikz,tkz-tab} 			% package principal TikZ
\usepackage{pgfplots}   % axes
\usepackage{mathrsfs}    % cursives
\usepackage{calc}			% calcul taille boites
\usepackage[scaled=0.875]{helvet} % font sans serif
\usepackage{svg} % svg
\usepackage{scrextend} % local margin
\usepackage{scratch} %scratch
\usepackage{multicol} % colonnes
%\usepackage{infix-RPN,pst-func} % formule en notation polanaise inversée
\usepackage{listings}

%================================================================================================================================
%
% Réglages de base
%
%================================================================================================================================

\lstset{
language=Python,   % R code
literate=
{á}{{\'a}}1
{à}{{\`a}}1
{ã}{{\~a}}1
{é}{{\'e}}1
{è}{{\`e}}1
{ê}{{\^e}}1
{í}{{\'i}}1
{ó}{{\'o}}1
{õ}{{\~o}}1
{ú}{{\'u}}1
{ü}{{\"u}}1
{ç}{{\c{c}}}1
{~}{{ }}1
}


\definecolor{codegreen}{rgb}{0,0.6,0}
\definecolor{codegray}{rgb}{0.5,0.5,0.5}
\definecolor{codepurple}{rgb}{0.58,0,0.82}
\definecolor{backcolour}{rgb}{0.95,0.95,0.92}

\lstdefinestyle{mystyle}{
    backgroundcolor=\color{backcolour},   
    commentstyle=\color{codegreen},
    keywordstyle=\color{magenta},
    numberstyle=\tiny\color{codegray},
    stringstyle=\color{codepurple},
    basicstyle=\ttfamily\footnotesize,
    breakatwhitespace=false,         
    breaklines=true,                 
    captionpos=b,                    
    keepspaces=true,                 
    numbers=left,                    
xleftmargin=2em,
framexleftmargin=2em,            
    showspaces=false,                
    showstringspaces=false,
    showtabs=false,                  
    tabsize=2,
    upquote=true
}

\lstset{style=mystyle}


\lstset{style=mystyle}
\newcommand{\imgdir}{C:/laragon/www/newmc/assets/imgsvg/}
\newcommand{\imgsvgdir}{C:/laragon/www/newmc/assets/imgsvg/}

\definecolor{mcgris}{RGB}{220, 220, 220}% ancien~; pour compatibilité
\definecolor{mcbleu}{RGB}{52, 152, 219}
\definecolor{mcvert}{RGB}{125, 194, 70}
\definecolor{mcmauve}{RGB}{154, 0, 215}
\definecolor{mcorange}{RGB}{255, 96, 0}
\definecolor{mcturquoise}{RGB}{0, 153, 153}
\definecolor{mcrouge}{RGB}{255, 0, 0}
\definecolor{mclightvert}{RGB}{205, 234, 190}

\definecolor{gris}{RGB}{220, 220, 220}
\definecolor{bleu}{RGB}{52, 152, 219}
\definecolor{vert}{RGB}{125, 194, 70}
\definecolor{mauve}{RGB}{154, 0, 215}
\definecolor{orange}{RGB}{255, 96, 0}
\definecolor{turquoise}{RGB}{0, 153, 153}
\definecolor{rouge}{RGB}{255, 0, 0}
\definecolor{lightvert}{RGB}{205, 234, 190}
\setitemize[0]{label=\color{lightvert}  $\bullet$}

\pagestyle{fancy}
\renewcommand{\headrulewidth}{0.2pt}
\fancyhead[L]{maths-cours.fr}
\fancyhead[R]{\thepage}
\renewcommand{\footrulewidth}{0.2pt}
\fancyfoot[C]{}

\newcolumntype{C}{>{\centering\arraybackslash}X}
\newcolumntype{s}{>{\hsize=.35\hsize\arraybackslash}X}

\setlength{\parindent}{0pt}		 
\setlength{\parskip}{3mm}
\setlength{\headheight}{1cm}

\def\ebook{ebook}
\def\book{book}
\def\web{web}
\def\type{web}

\newcommand{\vect}[1]{\overrightarrow{\,\mathstrut#1\,}}

\def\Oij{$\left(\text{O}~;~\vect{\imath},~\vect{\jmath}\right)$}
\def\Oijk{$\left(\text{O}~;~\vect{\imath},~\vect{\jmath},~\vect{k}\right)$}
\def\Ouv{$\left(\text{O}~;~\vect{u},~\vect{v}\right)$}

\hypersetup{breaklinks=true, colorlinks = true, linkcolor = OliveGreen, urlcolor = OliveGreen, citecolor = OliveGreen, pdfauthor={Didier BONNEL - https://www.maths-cours.fr} } % supprime les bordures autour des liens

\renewcommand{\arg}[0]{\text{arg}}

\everymath{\displaystyle}

%================================================================================================================================
%
% Macros - Commandes
%
%================================================================================================================================

\newcommand\meta[2]{    			% Utilisé pour créer le post HTML.
	\def\titre{titre}
	\def\url{url}
	\def\arg{#1}
	\ifx\titre\arg
		\newcommand\maintitle{#2}
		\fancyhead[L]{#2}
		{\Large\sffamily \MakeUppercase{#2}}
		\vspace{1mm}\textcolor{mcvert}{\hrule}
	\fi 
	\ifx\url\arg
		\fancyfoot[L]{\href{https://www.maths-cours.fr#2}{\black \footnotesize{https://www.maths-cours.fr#2}}}
	\fi 
}


\newcommand\TitreC[1]{    		% Titre centré
     \needspace{3\baselineskip}
     \begin{center}\textbf{#1}\end{center}
}

\newcommand\newpar{    		% paragraphe
     \par
}

\newcommand\nosp {    		% commande vide (pas d'espace)
}
\newcommand{\id}[1]{} %ignore

\newcommand\boite[2]{				% Boite simple sans titre
	\vspace{5mm}
	\setlength{\fboxrule}{0.2mm}
	\setlength{\fboxsep}{5mm}	
	\fcolorbox{#1}{#1!3}{\makebox[\linewidth-2\fboxrule-2\fboxsep]{
  		\begin{minipage}[t]{\linewidth-2\fboxrule-4\fboxsep}\setlength{\parskip}{3mm}
  			 #2
  		\end{minipage}
	}}
	\vspace{5mm}
}

\newcommand\CBox[4]{				% Boites
	\vspace{5mm}
	\setlength{\fboxrule}{0.2mm}
	\setlength{\fboxsep}{5mm}
	
	\fcolorbox{#1}{#1!3}{\makebox[\linewidth-2\fboxrule-2\fboxsep]{
		\begin{minipage}[t]{1cm}\setlength{\parskip}{3mm}
	  		\textcolor{#1}{\LARGE{#2}}    
 	 	\end{minipage}  
  		\begin{minipage}[t]{\linewidth-2\fboxrule-4\fboxsep}\setlength{\parskip}{3mm}
			\raisebox{1.2mm}{\normalsize\sffamily{\textcolor{#1}{#3}}}						
  			 #4
  		\end{minipage}
	}}
	\vspace{5mm}
}

\newcommand\cadre[3]{				% Boites convertible html
	\par
	\vspace{2mm}
	\setlength{\fboxrule}{0.1mm}
	\setlength{\fboxsep}{5mm}
	\fcolorbox{#1}{white}{\makebox[\linewidth-2\fboxrule-2\fboxsep]{
  		\begin{minipage}[t]{\linewidth-2\fboxrule-4\fboxsep}\setlength{\parskip}{3mm}
			\raisebox{-2.5mm}{\sffamily \small{\textcolor{#1}{\MakeUppercase{#2}}}}		
			\par		
  			 #3
 	 		\end{minipage}
	}}
		\vspace{2mm}
	\par
}

\newcommand\bloc[3]{				% Boites convertible html sans bordure
     \needspace{2\baselineskip}
     {\sffamily \small{\textcolor{#1}{\MakeUppercase{#2}}}}    
		\par		
  			 #3
		\par
}

\newcommand\CHelp[1]{
     \CBox{Plum}{\faInfoCircle}{À RETENIR}{#1}
}

\newcommand\CUp[1]{
     \CBox{NavyBlue}{\faThumbsOUp}{EN PRATIQUE}{#1}
}

\newcommand\CInfo[1]{
     \CBox{Sepia}{\faArrowCircleRight}{REMARQUE}{#1}
}

\newcommand\CRedac[1]{
     \CBox{PineGreen}{\faEdit}{BIEN R\'EDIGER}{#1}
}

\newcommand\CError[1]{
     \CBox{Red}{\faExclamationTriangle}{ATTENTION}{#1}
}

\newcommand\TitreExo[2]{
\needspace{4\baselineskip}
 {\sffamily\large EXERCICE #1\ (\emph{#2 points})}
\vspace{5mm}
}

\newcommand\img[2]{
          \includegraphics[width=#2\paperwidth]{\imgdir#1}
}

\newcommand\imgsvg[2]{
       \begin{center}   \includegraphics[width=#2\paperwidth]{\imgsvgdir#1} \end{center}
}


\newcommand\Lien[2]{
     \href{#1}{#2 \tiny \faExternalLink}
}
\newcommand\mcLien[2]{
     \href{https~://www.maths-cours.fr/#1}{#2 \tiny \faExternalLink}
}

\newcommand{\euro}{\eurologo{}}

%================================================================================================================================
%
% Macros - Environement
%
%================================================================================================================================

\newenvironment{tex}{ %
}
{%
}

\newenvironment{indente}{ %
	\setlength\parindent{10mm}
}

{
	\setlength\parindent{0mm}
}

\newenvironment{corrige}{%
     \needspace{3\baselineskip}
     \medskip
     \textbf{\textsc{Corrigé}}
     \medskip
}
{
}

\newenvironment{extern}{%
     \begin{center}
     }
     {
     \end{center}
}

\NewEnviron{code}{%
	\par
     \boite{gray}{\texttt{%
     \BODY
     }}
     \par
}

\newenvironment{vbloc}{% boite sans cadre empeche saut de page
     \begin{minipage}[t]{\linewidth}
     }
     {
     \end{minipage}
}
\NewEnviron{h2}{%
    \needspace{3\baselineskip}
    \vspace{0.6cm}
	\noindent \MakeUppercase{\sffamily \large \BODY}
	\vspace{1mm}\textcolor{mcgris}{\hrule}\vspace{0.4cm}
	\par
}{}

\NewEnviron{h3}{%
    \needspace{3\baselineskip}
	\vspace{5mm}
	\textsc{\BODY}
	\par
}

\NewEnviron{margeneg}{ %
\begin{addmargin}[-1cm]{0cm}
\BODY
\end{addmargin}
}

\NewEnviron{html}{%
}

\begin{document}
\meta{url}{/exercices/fonctions-bac-blanc-es-l-sujet-6-maths-cours-2018/}
\meta{pid}{10597}
\meta{titre}{Fonctions - Bac blanc ES/L Sujet 6 - Maths-cours 2018}
\meta{type}{exercices}
%
\begin{h2}Exercice 4 (5 points)\end{h2}
\par
On considère la fonction $f$ définie sur l'intervalle [0~;~5] par:
\par
\[ f(x) = 2\ln(x+1)-x+1. \]
\par
On a utilisé un logiciel de calcul formel pour déterminer la fonction dérivée $f'$, la fonction dérivée seconde $f''$ et une primitive $F$ de $f$ sur l'intervalle [0~;~5].
\par
On a obtenu les résultats suivants :
\par
\begin{center}
     \begin{extern}%width="300" alt="calcul formel "
          \begin{tabular}{|l|l|}\hline
               1&\textit{dériver} ( 2ln(x+1)-x+1 )\\
               & \\
               &$\textcolor{blue}{\rightarrow \quad \dfrac{-x+1}{x+1}}$ \\
               & \\ \hline
               2&\textit{dériver} ( (-x+1)/(x+1) )\\
               & \\
               &$\textcolor{blue}{\rightarrow \quad \dfrac{-2}{(x+1)^2}}$ \\
               & \\ \hline
               3&\textit{intégrer} ( 2ln(x+1)-x+1 )\\
               & \\
               &$\textcolor{blue}{\rightarrow \quad 2(x+1)\ln(x+1)-\dfrac{1}{2}x^2-x}$ \\
               & \\ \hline
          \end{tabular}
     \end{extern}
\end{center}
\par
\textbf{\textit{Dans les questions suivantes, on pourra utiliser tous les résultats fournis par le logiciel sans les avoir justifiés.}}
\par
\begin{enumerate}
     \item %1
     Dresser le tableau de variations de la fonction $f$ sur l'intervalle [0~;~5].
     \item %2
     Montrer que la fonction $f$ est concave sur l'intervalle [0~;~5].
     \item %3
     On rappelle que la valeur moyenne $m$ d'une fonction $f$ sur un intervalle [a~;~b] est donnée par la formule :
     \[ m=\dfrac{1}{b-a}\displaystyle\int_{a}^{b}f(t)\text{d}t. \]
     \par
     Déterminer la valeur exacte puis une valeur approchée à $10^{-3}$ de la valeur moyenne de la fonction $f$ sur l'intervalle [0~;~5].
     \item %4
     \begin{enumerate}[label=\alph*.]
          \item %4a
          Montrer que l'équation $f(x)=0$ admet une unique solution $\alpha$ sur l'intervalle [0~;~5] et que $4 < \alpha < 5$.
          \item %4b
          On a écrit l'algorithme suivant :
          \vspace{0.1cm}
          \par
          \begin{center}
               \begin{extern}%width="440" alt="algorithme"
                    \begin{tabular}{|l|l|}\hline
                         Variables :	&$x$ est un nombre réel\\
                         &$y$ est un nombre réel\\
                         & \\
                         Initialisation: &$x$ prend la valeur $4$ \\
                         &$y$ prend la valeur $2\ln(x+1)-x+1$ \\
                         & \\
                         Traitement: &Tant que $y > 0$ faire : \\
                         &\quad$x$ prend la valeur $x+0,1$\\
                         &\quad$y$ prend la valeur $2\ln(x+1)-x+1$\\
                         &Fin Tant que\\
                         & \\
                         Sortie :	&Afficher $x$ \\
                         \hline
                    \end{tabular}
               \end{extern}
          \end{center}
          \vspace{0.2cm}
          \par
          Recopier et compléter le tableau suivant, en ajoutant autant de colonnes que nécessaire. On arrondira les résultats au millième.
          \vspace{0.1cm}
          \par
          \begin{center}
               \begin{tabular}{|l|c|c|c|}\hline %class="compact"
                    Valeur de $x$	&$4$	&	 $4,1$  &	 $\quad \cdots \quad$ \\ \hline
                    Valeur de $y$	&$0,219$	& $\quad \cdots \quad$ & $\quad \cdots \quad$ 	 \\ \hline
                    Condition $y > 0$	&vraie	& $\quad \cdots \quad$ & $\quad \cdots \quad$ 	\\ \hline
               \end{tabular}
          \end{center}
          \vspace{0.2cm}
          \item Quel est le résultat affiché par cet algorithme ? \\
          Interpréter ce résultat dans le cadre de l'exercice.
     \end{enumerate}
\end{enumerate}
\begin{corrige}
     \begin{enumerate}
          \item %1
          D'après les résultats fournis par le logiciel, la fonction $f$ est dérivable sur l'intervalle $[0~;~5]$ et :
          \[ f'(x)=\dfrac{-x+1}{x+1}. \]
          \par
          Le dénominateur est strictement positif sur l'intervalle $[0~;~5]$ (puisqu'alors $x+1 \geqslant 1$). $f'(x)$ est donc du signe de $-x+1$, c'est à dire nul pour ${x=1}$, strictement positif pour ${x < 1}$ et strictement négatif pour ${x > 1}$.
          \par
          Par ailleurs :
          \par
          $f(0)=2\ln1-0+1=1$ ;
          \par
          $f(1)=2\ln(1+1)-1+1=2\ln2$ ;
          \par
          $f(5)=2\ln(5+1)-5+1=2\ln6 - 4$.
          \par
          On obtient alors le tableau de variations suivant :
          \par
          %:-+-+-+-+- Engendré par : http://math.et.info.free.fr/TikZ/TableauxVariations/
          \begin{center}
               \begin{extern}%width="340" alt="tableau de variations"
                    \begin{tikzpicture}[scale=0.875]
                         % Styles
                         \tikzstyle{cadre}=[thin]
                         \tikzstyle{fleche}=[->,>=latex,thin]
                         \tikzstyle{nondefini}=[lightgray]
                         % Dimensions Modifiables
                         \def\Lrg{1.5}
                         \def\HtX{1}
                         \def\HtY{0.5}
                         % Dimensions Calculées
                         \def\lignex{-0.5*\HtX}
                         \def\lignef{-1.5*\HtX}
                         \def\separateur{-0.5*\Lrg}
                         % Largeur du tableau
                         \def\gauche{-1.5*\Lrg}
                         \def\droite{4.6*\Lrg}
                         % Hauteur du tableau
                         \def\haut{0.5*\HtX}
                         \def\bas{-2.5*\HtX-2*\HtY}
                         % Ligne de l'abscisse : x
                         \node at (-1*\Lrg,0) {$x$};
                         \node at (0*\Lrg,0) {$0$};
                         \node at (2*\Lrg,0) {$1$};
                         \node at (4*\Lrg,0) {$5$};
                         % Ligne de la dérivée : f'(x)
                         \node at (-1*\Lrg,-1*\HtX) {$f'(x)$};
                         \node at (0*\Lrg,-1*\HtX) {$\ $};
                         \node at (1*\Lrg,-1*\HtX) {$+$};
                         \node at (2*\Lrg,-1*\HtX) {$0$};
                         \node at (3*\Lrg,-1*\HtX) {$-$};
                         \node at (4*\Lrg,-1*\HtX) {$\ $};
                         % Ligne de la fonction : f(x)
                         \node  at (-1*\Lrg,{-2*\HtX+(-1)*\HtY}) {$f(x)$};
                         \node (f1) at (0*\Lrg,{-2*\HtX+(-2)*\HtY}) {$1$};
                         \node (f2) at (2*\Lrg,{-2*\HtX+(0)*\HtY}) {$2\ln2$};
                         \node (f3) at (3.9*\Lrg,{-2*\HtX+(-2)*\HtY}) {$2\ln6-4$};
                         % Flèches
                         \draw[fleche] (f1) -- (f2);
                         \draw[fleche] (f2) -- (f3);
                         % Encadrement
                         \draw[cadre] (\separateur,\haut) -- (\separateur,\bas);
                         \draw[cadre] (\gauche,\haut) rectangle  (\droite,\bas);
                         \draw[cadre] (\gauche,\lignex) -- (\droite,\lignex);
                         \draw[cadre] (\gauche,\lignef) -- (\droite,\lignef);
                    \end{tikzpicture}
               \end{extern}
          \end{center}
          %:-+-+-+-+- Fin
          \item %2
          Le logiciel de calcul formel indique que la fonction $f$ est deux fois dérivable et que :
          \[ f''(x)=\dfrac{-2}{(x+1)^2}. \]
          \par
          Le dénominateur est strictement positif sur l'intervalle $[0~;~5]$ et le numérateur strictement négatif.
          \par
          $f''$ est donc strictement négative sur l'intervalle $[0~;~5]$ et, par conséquent, la fonction $f$ est concave sur cet intervalle.
          \item %3
          D'après la formule rappelée dans l'énoncé, la valeur moyenne $m$ de la fonction $f$ sur l'intervalle [0~;~5] est :
          \[ m=\dfrac{1}{5}\displaystyle\int_{0}^{5}f(t)\text{d}t. \]
          \par
          Le logiciel de calcul formel indique que la fonction $F$ définie sur $[0~;~5]$ par :
          \[ F(x)=2(x+1)\ln(x+1)-\dfrac{1}{2}x^2-x \]
          est une primitive de la fonction $f$.
          \par
          Par conséquent :
          \par
          $m=\dfrac{1}{5}\left(F(5)-F(0)\right)$.
          \par
          Or :
          \par
          $F(5)=12\ln6-\dfrac{1}{2} \times 25 -5 =12\ln6-\dfrac{35}{2}$,
          \par
          $F(0)=2\ln1=0$.
          \par
          Donc :
          \par
          $m=\dfrac{1}{5}\left(12\ln6-\dfrac{35}{2}\right)=\dfrac{12}{5}\ln6-\dfrac{7}{2}$.
          \par
          Une valeur approchée de $m$ à $10^{-3}$ près est 0,8.
          \item %4
          \begin{enumerate}[label=\alph*.]
               \item %4a
               Remarquons d'abord que $f(1) \approx 1,386$ et $f(4) \approx 0,219$ sont strictement positifs alors que $f(5) \approx -0,416$ est strictement négatif.
               \par
               Sur l'intervalle $[0~;~1]$, $f(x)$ est supérieur ou égal à 1 donc strictement positif.
               L'équation $f(x)=0$ n'a donc aucune solution sur cet intervalle.
               \par
               Sur l'intervalle $[1~;~5]$, la fonction $f$ est \textbf{continue}, \textbf{strictement décroissante} et \textbf{change de signe}. Donc l'équation $f(x)=0$ admet une unique solution $\alpha$  sur cet intervalle.
               \par
               Comme $f(x)$ change de signe entre $x=4$ et $x=5$ : ${4 < \alpha < 5}$.
               \item %4b
               En faisant fonctionner l'algorithme on obtient le tableau suivant :
               \par
               \begin{center}
                    \begin{tabular}{|l|c|c|c|c|c|}\hline %class="compact"
                         $x$	&$4$	&	 $4,1$  &	 $4,2$ &	 $4,3$ &	 $4,4$ \\ \hline
                         $y$	&$0,219$ &$0,158$	& $0,097$ &$0,035$	&$-0,027$\\ \hline
                         $y > 0$	&vraie	&vraie	&vraie	&vraie	&fausse		\\ \hline
                    \end{tabular}
               \end{center}
               \item %4c
               D'après la question précédente, l'algorithme affiche la valeur 4,4.
               \par
               Dans cet algorithme, $x$ est incrémenté par pas de 0,1 et $y$ représente la valeur de $f(x)$.
               \par
               On sort de la boucle \og Tant que \fg{} quand $y \leqslant 0$, c'est à dire quand $f(x) \leqslant 0$.
               \par
               Le tableau de la question précédente montre que $f(4,3)$ est strictement positif tandis que $f(4,4)$ est strictement négatif. $f$ change donc de signe entre 4,3 et 4,4 ; par conséquent : ${4,3 < \alpha < 4,4}$.
               \par
               L'algorithme affiche donc une \textbf{valeur approchée à 0,1 près par excès de $\alpha$.}
               \par
          \end{enumerate}
          \par
     \end{enumerate}
\end{corrige}

\end{document}
µ
\documentclass[a4paper]{article}

%================================================================================================================================
%
% Packages
%
%================================================================================================================================

\usepackage[T1]{fontenc} 	% pour caractères accentués
\usepackage[utf8]{inputenc}  % encodage utf8
\usepackage[french]{babel}	% langue : français
\usepackage{fourier}			% caractères plus lisibles
\usepackage[dvipsnames]{xcolor} % couleurs
\usepackage{fancyhdr}		% réglage header footer
\usepackage{needspace}		% empêcher sauts de page mal placés
\usepackage{graphicx}		% pour inclure des graphiques
\usepackage{enumitem,cprotect}		% personnalise les listes d'items (nécessaire pour ol, al ...)
\usepackage{hyperref}		% Liens hypertexte
\usepackage{pstricks,pst-all,pst-node,pstricks-add,pst-math,pst-plot,pst-tree,pst-eucl} % pstricks
\usepackage[a4paper,includeheadfoot,top=2cm,left=3cm, bottom=2cm,right=3cm]{geometry} % marges etc.
\usepackage{comment}			% commentaires multilignes
\usepackage{amsmath,environ} % maths (matrices, etc.)
\usepackage{amssymb,makeidx}
\usepackage{bm}				% bold maths
\usepackage{tabularx}		% tableaux
\usepackage{colortbl}		% tableaux en couleur
\usepackage{fontawesome}		% Fontawesome
\usepackage{environ}			% environment with command
\usepackage{fp}				% calculs pour ps-tricks
\usepackage{multido}			% pour ps tricks
\usepackage[np]{numprint}	% formattage nombre
\usepackage{tikz,tkz-tab} 			% package principal TikZ
\usepackage{pgfplots}   % axes
\usepackage{mathrsfs}    % cursives
\usepackage{calc}			% calcul taille boites
\usepackage[scaled=0.875]{helvet} % font sans serif
\usepackage{svg} % svg
\usepackage{scrextend} % local margin
\usepackage{scratch} %scratch
\usepackage{multicol} % colonnes
%\usepackage{infix-RPN,pst-func} % formule en notation polanaise inversée
\usepackage{listings}

%================================================================================================================================
%
% Réglages de base
%
%================================================================================================================================

\lstset{
language=Python,   % R code
literate=
{á}{{\'a}}1
{à}{{\`a}}1
{ã}{{\~a}}1
{é}{{\'e}}1
{è}{{\`e}}1
{ê}{{\^e}}1
{í}{{\'i}}1
{ó}{{\'o}}1
{õ}{{\~o}}1
{ú}{{\'u}}1
{ü}{{\"u}}1
{ç}{{\c{c}}}1
{~}{{ }}1
}


\definecolor{codegreen}{rgb}{0,0.6,0}
\definecolor{codegray}{rgb}{0.5,0.5,0.5}
\definecolor{codepurple}{rgb}{0.58,0,0.82}
\definecolor{backcolour}{rgb}{0.95,0.95,0.92}

\lstdefinestyle{mystyle}{
    backgroundcolor=\color{backcolour},   
    commentstyle=\color{codegreen},
    keywordstyle=\color{magenta},
    numberstyle=\tiny\color{codegray},
    stringstyle=\color{codepurple},
    basicstyle=\ttfamily\footnotesize,
    breakatwhitespace=false,         
    breaklines=true,                 
    captionpos=b,                    
    keepspaces=true,                 
    numbers=left,                    
xleftmargin=2em,
framexleftmargin=2em,            
    showspaces=false,                
    showstringspaces=false,
    showtabs=false,                  
    tabsize=2,
    upquote=true
}

\lstset{style=mystyle}


\lstset{style=mystyle}
\newcommand{\imgdir}{C:/laragon/www/newmc/assets/imgsvg/}
\newcommand{\imgsvgdir}{C:/laragon/www/newmc/assets/imgsvg/}

\definecolor{mcgris}{RGB}{220, 220, 220}% ancien~; pour compatibilité
\definecolor{mcbleu}{RGB}{52, 152, 219}
\definecolor{mcvert}{RGB}{125, 194, 70}
\definecolor{mcmauve}{RGB}{154, 0, 215}
\definecolor{mcorange}{RGB}{255, 96, 0}
\definecolor{mcturquoise}{RGB}{0, 153, 153}
\definecolor{mcrouge}{RGB}{255, 0, 0}
\definecolor{mclightvert}{RGB}{205, 234, 190}

\definecolor{gris}{RGB}{220, 220, 220}
\definecolor{bleu}{RGB}{52, 152, 219}
\definecolor{vert}{RGB}{125, 194, 70}
\definecolor{mauve}{RGB}{154, 0, 215}
\definecolor{orange}{RGB}{255, 96, 0}
\definecolor{turquoise}{RGB}{0, 153, 153}
\definecolor{rouge}{RGB}{255, 0, 0}
\definecolor{lightvert}{RGB}{205, 234, 190}
\setitemize[0]{label=\color{lightvert}  $\bullet$}

\pagestyle{fancy}
\renewcommand{\headrulewidth}{0.2pt}
\fancyhead[L]{maths-cours.fr}
\fancyhead[R]{\thepage}
\renewcommand{\footrulewidth}{0.2pt}
\fancyfoot[C]{}

\newcolumntype{C}{>{\centering\arraybackslash}X}
\newcolumntype{s}{>{\hsize=.35\hsize\arraybackslash}X}

\setlength{\parindent}{0pt}		 
\setlength{\parskip}{3mm}
\setlength{\headheight}{1cm}

\def\ebook{ebook}
\def\book{book}
\def\web{web}
\def\type{web}

\newcommand{\vect}[1]{\overrightarrow{\,\mathstrut#1\,}}

\def\Oij{$\left(\text{O}~;~\vect{\imath},~\vect{\jmath}\right)$}
\def\Oijk{$\left(\text{O}~;~\vect{\imath},~\vect{\jmath},~\vect{k}\right)$}
\def\Ouv{$\left(\text{O}~;~\vect{u},~\vect{v}\right)$}

\hypersetup{breaklinks=true, colorlinks = true, linkcolor = OliveGreen, urlcolor = OliveGreen, citecolor = OliveGreen, pdfauthor={Didier BONNEL - https://www.maths-cours.fr} } % supprime les bordures autour des liens

\renewcommand{\arg}[0]{\text{arg}}

\everymath{\displaystyle}

%================================================================================================================================
%
% Macros - Commandes
%
%================================================================================================================================

\newcommand\meta[2]{    			% Utilisé pour créer le post HTML.
	\def\titre{titre}
	\def\url{url}
	\def\arg{#1}
	\ifx\titre\arg
		\newcommand\maintitle{#2}
		\fancyhead[L]{#2}
		{\Large\sffamily \MakeUppercase{#2}}
		\vspace{1mm}\textcolor{mcvert}{\hrule}
	\fi 
	\ifx\url\arg
		\fancyfoot[L]{\href{https://www.maths-cours.fr#2}{\black \footnotesize{https://www.maths-cours.fr#2}}}
	\fi 
}


\newcommand\TitreC[1]{    		% Titre centré
     \needspace{3\baselineskip}
     \begin{center}\textbf{#1}\end{center}
}

\newcommand\newpar{    		% paragraphe
     \par
}

\newcommand\nosp {    		% commande vide (pas d'espace)
}
\newcommand{\id}[1]{} %ignore

\newcommand\boite[2]{				% Boite simple sans titre
	\vspace{5mm}
	\setlength{\fboxrule}{0.2mm}
	\setlength{\fboxsep}{5mm}	
	\fcolorbox{#1}{#1!3}{\makebox[\linewidth-2\fboxrule-2\fboxsep]{
  		\begin{minipage}[t]{\linewidth-2\fboxrule-4\fboxsep}\setlength{\parskip}{3mm}
  			 #2
  		\end{minipage}
	}}
	\vspace{5mm}
}

\newcommand\CBox[4]{				% Boites
	\vspace{5mm}
	\setlength{\fboxrule}{0.2mm}
	\setlength{\fboxsep}{5mm}
	
	\fcolorbox{#1}{#1!3}{\makebox[\linewidth-2\fboxrule-2\fboxsep]{
		\begin{minipage}[t]{1cm}\setlength{\parskip}{3mm}
	  		\textcolor{#1}{\LARGE{#2}}    
 	 	\end{minipage}  
  		\begin{minipage}[t]{\linewidth-2\fboxrule-4\fboxsep}\setlength{\parskip}{3mm}
			\raisebox{1.2mm}{\normalsize\sffamily{\textcolor{#1}{#3}}}						
  			 #4
  		\end{minipage}
	}}
	\vspace{5mm}
}

\newcommand\cadre[3]{				% Boites convertible html
	\par
	\vspace{2mm}
	\setlength{\fboxrule}{0.1mm}
	\setlength{\fboxsep}{5mm}
	\fcolorbox{#1}{white}{\makebox[\linewidth-2\fboxrule-2\fboxsep]{
  		\begin{minipage}[t]{\linewidth-2\fboxrule-4\fboxsep}\setlength{\parskip}{3mm}
			\raisebox{-2.5mm}{\sffamily \small{\textcolor{#1}{\MakeUppercase{#2}}}}		
			\par		
  			 #3
 	 		\end{minipage}
	}}
		\vspace{2mm}
	\par
}

\newcommand\bloc[3]{				% Boites convertible html sans bordure
     \needspace{2\baselineskip}
     {\sffamily \small{\textcolor{#1}{\MakeUppercase{#2}}}}    
		\par		
  			 #3
		\par
}

\newcommand\CHelp[1]{
     \CBox{Plum}{\faInfoCircle}{À RETENIR}{#1}
}

\newcommand\CUp[1]{
     \CBox{NavyBlue}{\faThumbsOUp}{EN PRATIQUE}{#1}
}

\newcommand\CInfo[1]{
     \CBox{Sepia}{\faArrowCircleRight}{REMARQUE}{#1}
}

\newcommand\CRedac[1]{
     \CBox{PineGreen}{\faEdit}{BIEN R\'EDIGER}{#1}
}

\newcommand\CError[1]{
     \CBox{Red}{\faExclamationTriangle}{ATTENTION}{#1}
}

\newcommand\TitreExo[2]{
\needspace{4\baselineskip}
 {\sffamily\large EXERCICE #1\ (\emph{#2 points})}
\vspace{5mm}
}

\newcommand\img[2]{
          \includegraphics[width=#2\paperwidth]{\imgdir#1}
}

\newcommand\imgsvg[2]{
       \begin{center}   \includegraphics[width=#2\paperwidth]{\imgsvgdir#1} \end{center}
}


\newcommand\Lien[2]{
     \href{#1}{#2 \tiny \faExternalLink}
}
\newcommand\mcLien[2]{
     \href{https~://www.maths-cours.fr/#1}{#2 \tiny \faExternalLink}
}

\newcommand{\euro}{\eurologo{}}

%================================================================================================================================
%
% Macros - Environement
%
%================================================================================================================================

\newenvironment{tex}{ %
}
{%
}

\newenvironment{indente}{ %
	\setlength\parindent{10mm}
}

{
	\setlength\parindent{0mm}
}

\newenvironment{corrige}{%
     \needspace{3\baselineskip}
     \medskip
     \textbf{\textsc{Corrigé}}
     \medskip
}
{
}

\newenvironment{extern}{%
     \begin{center}
     }
     {
     \end{center}
}

\NewEnviron{code}{%
	\par
     \boite{gray}{\texttt{%
     \BODY
     }}
     \par
}

\newenvironment{vbloc}{% boite sans cadre empeche saut de page
     \begin{minipage}[t]{\linewidth}
     }
     {
     \end{minipage}
}
\NewEnviron{h2}{%
    \needspace{3\baselineskip}
    \vspace{0.6cm}
	\noindent \MakeUppercase{\sffamily \large \BODY}
	\vspace{1mm}\textcolor{mcgris}{\hrule}\vspace{0.4cm}
	\par
}{}

\NewEnviron{h3}{%
    \needspace{3\baselineskip}
	\vspace{5mm}
	\textsc{\BODY}
	\par
}

\NewEnviron{margeneg}{ %
\begin{addmargin}[-1cm]{0cm}
\BODY
\end{addmargin}
}

\NewEnviron{html}{%
}

\begin{document}
\meta{url}{/exercices/graphes-probabilistes-bac-blanc-es-sujet-6-maths-cours-2018-spe/}
\meta{pid}{10600}
\meta{titre}{Graphes probabilistes - Bac blanc ES Sujet 6 - Maths-cours 2018 (spé)}
\meta{type}{exercices}
%
\begin{h2}Exercice 5 (5 points)\end{h2}
\par
\textbf{Candidats ayant suivi l'enseignement de spécialité}
\par
Pour le paiement des cotisations, une société d'assurance propose à ces clients le choix entre deux types de règlement :
\par
\begin{itemize}
     \item
     le prélèvement automatique mensuel ;
     \item
     le règlement annuel par chèque.
\end{itemize}
\par
En 2016, 50\% des clients avaient opté pour le prélèvement mensuel.
\par
Chaque année :
\begin{itemize}
     \item
     90\% des clients payant par prélèvement mensuel conservent ce mode de paiement l'année suivante ;
     \item
     25\% des clients réglant leur cotisation par chèque annuel choisissent le prélèvement mensuel l'année suivante ;
\end{itemize}
\vspace{0.2cm}
\par
Pour tout entier naturel $n$, on note :
\begin{itemize}
     \item
     $m_{n}$, la proportion de clients ayant choisi le prélèvement mensuel pour l'année $2016 + n$ ;
     \item
     $a_{n}$, la proportion de clients ayant choisi le règlement annuel pour l'année $2016 + n$ ;
     \item
     $P_{n}$, la matrice-ligne $\left(m_{n} \quad a_{n}\right)$ donnant l'état probabiliste de l'année $2016 + n$.
\end{itemize}
\par
On choisit au hasard un client de cet assureur.
\par
On note :
\begin{itemize}
     \item %
     $M$ l'état \og le client a opté pour le prélèvement mensuel \fg{} ;
     \item %
     $A$ l'état \og le client a opté pour le règlement annuel \fg{} .
\end{itemize}
\par
%============================================================================================================================
%
\TitreC{Partie A}
%
%============================================================================================================================
\par
\begin{enumerate}
     \item %1
     Traduire la situation par un graphe probabiliste.
     \item %2
     Déterminer la matrice de transition $T$ associée à ce graphe, les sommets étant classés dans l'ordre $M, A$.
     \item %3
     Déterminer les matrices-ligne $P_0, P_1$ et $P_2$ (\textit{Si nécessaire, on arrondira les résultats au millième}).\\
     Quel est le pourcentage de clients ayant choisi le prélèvement mensuel en 2018 ?
     \item %4
     Déterminer l'état stable de ce graphe. \\
     Interpréter ce résultat.
     \par
\end{enumerate}
\par
%============================================================================================================================
%
\TitreC{Partie B}
%
%============================================================================================================================
\par
\begin{enumerate}
     \item %1
     Montrer que pour tout entier naturel $n$ :
     \[ m_{n+1}=0,65m_n + 0,25. \]
     \item %2
     Le directeur d'agence souhaite connaître la proportion des clients qui optera pour le prélèvement mensuel pour l'année 2016+n où $n$ est un entier naturel non nul.
     \par
     On lui a proposé les trois algorithmes suivants :
     \par
     \begin{center}
          \begin{extern}%width="400" alt="algorithme n°1"
               \begin{tabular}{|l l|}\hline
                    \textbf{Entrée :}	& 	Saisir $n$\\
                    \textbf{Traitement :}	& Affecter à $i$ la valeur 1 \\
                    & Affecter à $m$ la valeur 0.5\\
                    & Tant que $i < n$\\
                    &\qquad Affecter à $m$ la valeur $0,65m + 0,25$ \\
                    &Fin Tant que\\
                    \textbf{Sortie :}		&Afficher $m$ \\ \hline
               \end{tabular}
          \end{extern}
          \par
          \textbf{Algorithme 1}
     \end{center}
     \begin{center}
          \begin{extern}%width="400" alt="algorithme n°2"
               \begin{tabular}{|l l|}\hline
                    \textbf{Entrée :}	& 	Saisir $n$\\
                    \textbf{Traitement :} & Affecter à $m$ la valeur 0,5\\
                    & Pour $i$ allant de $1$ à $n$\\
                    &\qquad Affecter à $m$ la valeur $0,65m + 0,25$ \\
                    &Fin Pour\\
                    \textbf{Sortie :}		&Afficher $m$ \\ \hline
               \end{tabular}
          \end{extern}
          \par
          \textbf{Algorithme 2}
     \end{center}
     \begin{center}
          \begin{extern}%width="400" alt="algorithme n°3"
               \begin{tabular}{|l l|}\hline
                    \textbf{Entrée :}	& 	Saisir $n$\\
                    \textbf{Traitement :} & Affecter à $m$ la valeur 0,5\\
                    & Pour $i$ allant de $1$ à $n$\\
                    &\qquad Affecter à $m$ la valeur $0,65m + 0,25$ \\
                    &Fin Pour\\
                    \textbf{Sortie :}		&Afficher $n$ \\ \hline
               \end{tabular}
          \end{extern}
          \par
          \textbf{Algorithme 3}
     \end{center}
     \par
     Parmi ces trois algorithmes, un seul répond correctement à la demande du directeur. Lequel ?\\
     Justifier votre réponse en indiquant les erreurs présentes dans les deux autres algorithmes.
     \item %3
     Pour tout entier naturel $n$, on pose $u_n=m_n-\dfrac{5}{7}$.
     \par
     \begin{enumerate}[label=\alph*.]
          \item %3a
          Montrer que la suite $(u_n)$ est une suite géométrique dont on précisera le premier terme et la raison.
          \item %3b
          En déduire que pour tout entier naturel $n$ :
          \[ m_n=\dfrac{5}{7}-\dfrac{3}{14} \times 0,65^n. \]
     \end{enumerate}
     \item %4
     Quelle est la limite de la suite $(m_n)$ ? \\
     Interpréter et comparer ce résultat à celui de la question \textbf{4.} de la \textbf{Partie A}.
     \par
\end{enumerate}
\begin{corrige}
     %============================================================================================================================
     %
     \TitreC{Partie A}
     %
     %============================================================================================================================
     \begin{enumerate}
          \item On traduit les données de l'énoncé par un graphe probabiliste de sommets $M$ et $A$:
          \begin{center}
               \begin{extern}%width="400" alt="Graphe probabiliste à deux états"
                    \begin{pspicture}(-2,-0.5)(4,1)
                         \circlenode{M}{$M$} \hskip 4cm \circlenode{A}{$A$}% définition des sommets
                         \psset{arcangle=15,arrowsize=2pt 3}%  différents paramètres
                         \ncarc{->}{M}{A} \Aput{0,1}%
                         \ncarc{->}{A}{M} \Aput{0,25}%
                         \nccircle[angleA=90]{->}{M}{4mm}   \Bput{0,9}%    boucle autour de T
                         \nccircle[angleA=-90]{->}{A}{.4cm} \Bput{0,75}%    boucle autour de B
                    \end{pspicture}
               \end{extern}
          \end{center}
          \item  La matrice de transition de ce graphe en plaçant les sommets dans l'ordre $M$, $A$ est :
          \[ T=
          \begin{pmatrix}
               0,9 & 0,1\\
               0,25 & 0,75
          \end{pmatrix}. \]
          \item
          D'après l'énoncé, en 2016, 50\% des clients avaient opté pour le prélèvement mensuel donc $m_0=0,5$ et $p_0=1-m_0=0,5$.
          \par
          Par conséquent $P_0=(0,5 \quad 0,5)$.
          \par
          On en déduit :
          \par
          $P_1 = P_0 \times T = (0,5 \quad 0,5) \times \begin{pmatrix}
               0,9 & 0,1\\
               0,25 & 0,75
          \end{pmatrix} = (0,575 \quad 0,425)$ ;
          \par
          $P_2 = P_1 \times T = (0,575 \quad 0,425) \times \begin{pmatrix}
               0,9 & 0,1\\
               0,25 & 0,75
          \end{pmatrix} = (0,624 \quad 0,376)$ \\(arrondi au millième).
          \item
          Une matrice $P = (a\quad b)$ est un état stable si et seulement si ${a + b = 1}$ et $PT = P$.
          \par
          $PT=P \Leftrightarrow \begin{pmatrix} a&b\end{pmatrix}
          \times \begin{pmatrix} 0,9 & 0,1\\0,25 & 0,75 \end{pmatrix}
          =\begin{pmatrix} a&b\end{pmatrix}$
          \par
          $\phantom{PT=P} \Leftrightarrow \begin{pmatrix} 0,9a+0,25b & 0,1a+0,75b\end{pmatrix}
          =\begin{pmatrix} a&b\end{pmatrix}$
          \par
          $\phantom{PT=P} \Leftrightarrow
          \left\lbrace
          \begin{array}{r c l}
               0,9a+0,25b  &=& a\\
               0,1a+0,75b  &=& b
          \end{array}
     \right.$
     \par
     $\phantom{PM=P} \Leftrightarrow
     \left\lbrace
     \begin{array}{r c l}
          -0,1a+0,25b  &=& 0\\
          0,1a-0,25b  &=& 0
     \end{array}
\right.$
\par
$\phantom{PM=P}
\Leftrightarrow
0,1a-0,25b = 0$
\par
Comme $a+b=1$ : $b=1-a$. Par conséquent :
\par
$0,1a-0,25(1-a) = 0$
\par
$0,35a-0,25 = 0$
\par
$a = \dfrac{0,25}{0.35}=\dfrac{5}{7}$
\par
et $b=1-a=1-\dfrac{5}{7}=\dfrac{2}{7}$.
\par
On peut donc en déduire que
\[ P=\begin{pmatrix} \dfrac{5}{7} & \dfrac{2}{7} \end{pmatrix} \]
est l'état stable du graphe.
\par
Au cours du temps, la proportion des clients qui opteront pour le prélèvement mensuel se rapprochera de cinq septièmes.
\end{enumerate}
\par
%============================================================================================================================
%
\TitreC{Partie B}
%
%============================================================================================================================
\par
\begin{enumerate}
     \item %B1
     Pour tout entier naturel $n$ :
     \par
     Par conséquent $P_0=(0,5 \quad 0,5)$.
     \par
     $T$ étant la matrice de transition du graphe, pour tout entier naturel $n$ :
     \par
     $P_{n+1} = P_n \times T$.
     \par
     Donc :
     \par
     $(m_{n+1} \quad a_{n+1}) = (m_n \quad a_n) \times \begin{pmatrix}
          0,9 & 0,1\\
          0,25 & 0,75
     \end{pmatrix} = (0,575 \quad 0,425)$
     \par
     $\phantom{(m_{n+1} \quad a_{n+1})} = \begin{pmatrix} 0,9m_n+0,25a_n & 0,1m_n+0,75a_n\end{pmatrix} $.
     \par
     En comparant les termes situés en première colonne, on obtient :
     \par
     $m_{n+1} = 0,9m_n+0,25a_n$.
     \par
     Or, pour tout entier naturel $n$, $m_n$ et $a_n$ sont les probabilités de deux événements contraires donc ${a_n=1-m_n}$.
     \par
     Par conséquent :
     \par
     \par
     $m_{n+1}  = 0,9m_n+0,25(1-m_n)$\\
     $\phantom{m_{n+1}	}	 = 0,9m_n+0,25-0,25m_n$\\
     $\phantom{m_{n+1}	}	 		 = 0,65m_n+0,25.$
     \par
     \item %B2
     L'algorithme correct est l'\textbf{algorithme 2}.
     \par
     L'algorithme 1 est incorrect car il manque l'instruction \og Affecter à $i$ la valeur $i+1$ \fg{} (incrémentation de $i$) à l'intérieur de la boucle \og Tant que \fg{}.
     \par
     L'algorithme 3 est incorrect car il affiche, en sortie, la valeur de $n$ (qui correspond au nombre saisi par l'utilisateur) et non le résultat souhaité qui est stocké dans la variable $m$.
     \item %B3
     \textit{Reportez-vous à la \hyperlink{suite-ag-pap}{page \pageref*{suite-ag-pap}}  \og \'Etude d'une suite arithmético-géométrique étape par étape  \fg{} si vous souhaitez plus d'informations sur la méthode utilisée dans cette question.
     }
     \begin{enumerate}[label=\alph*.]
          \item %3a
          Pour tout entier naturel $n$, $u_{n}= m_{n}-\dfrac{5}{7}$, donc :
          \par
          $u_{n+1}= m_{n+1}-\dfrac{5}{7}$.
          \par
          Or, pour tout entier naturel $n$, $m_{n+1}=0,65m_n+0,25$ ; par conséquent :
          \par
          $u_{n+1}= 0,65m_n+0,25-\dfrac{5}{7}$\\
          $\phantom{u_{n+1}}=0,65m_n+\dfrac{1}{4}-\dfrac{5}{7}$\\
          $\phantom{u_{n+1}}=0,65m_n+\dfrac{1}{4}-\dfrac{13}{28}.$
          \par
          \par
          Or, puisque $u_{n}= m_{n}-\dfrac{5}{7}$ : $m_{n}= u_{n}+\dfrac{5}{7}$.
          \par
          On en déduit :
          \par
          $u_{n+1}= 0,65\left(u_{n}+\dfrac{5}{7}\right)-\dfrac{13}{28}$\\
          $\phantom{u_{n+1}}= 0,65u_{n}+\dfrac{3,25}{7}-\dfrac{13}{28}$\\
          $\phantom{u_{n+1}}= 0,65u_{n}+\dfrac{13}{28}-\dfrac{13}{28}$\\
          $\phantom{u_{n+1}}= 0,65u_{n}.$
          \par
          \par
          Comme $u_{0}= m_{0}-\dfrac{5}{7}=\dfrac{1}{2}-\dfrac{5}{7}=-\dfrac{3}{14}$, la suite $(u_n)$ est une suite géométrique de premier terme ${u_0=-\dfrac{3}{14}}$ et de raison $0,65$.
          \item %3b
          La suite $(u_n)$ étant une suite géométrique de premier terme ${u_0=-\dfrac{3}{14}}$ et de raison $0,65$, pour tout entier naturel $n$ :
          \par
          $u_n=u_0q^n=-\dfrac{3}{14} \times 0,65^n$.
          \par
          Par conséquent :
          \par
          $m_{n}= u_n + \dfrac{5}{7}=\dfrac{5}{7}-\dfrac{3}{14} \times 0,65^n$.
          \par
     \end{enumerate}
     \item %B4
     ${0 \leqslant 0,65 < 1}\ $ donc $\ \lim\limits_{n \rightarrow +\infty } 0,65^n = 0$.
     \par
     On en déduit que :
     \par
     $\lim\limits_{n \rightarrow +\infty}\dfrac{3}{14} \times 0,65^n = 0\ $ et $\ \lim\limits_{n \rightarrow +\infty}\dfrac{5}{7}-\dfrac{3}{14} \times 0,65^n = \dfrac{5}{7}$.
     \par
     La suite $(m_n)$ converge donc vers $\dfrac{5}{7}$.
     \par
     On retrouve le résultat de la \textbf{partie A}, à savoir que la proportion de clients choisissant le prélèvement mensuel tendra vers $\dfrac{5}{7}$.
\end{enumerate}
\end{corrige}

\end{document}
µ
\documentclass[a4paper]{article}

%================================================================================================================================
%
% Packages
%
%================================================================================================================================

\usepackage[T1]{fontenc} 	% pour caractères accentués
\usepackage[utf8]{inputenc}  % encodage utf8
\usepackage[french]{babel}	% langue : français
\usepackage{fourier}			% caractères plus lisibles
\usepackage[dvipsnames]{xcolor} % couleurs
\usepackage{fancyhdr}		% réglage header footer
\usepackage{needspace}		% empêcher sauts de page mal placés
\usepackage{graphicx}		% pour inclure des graphiques
\usepackage{enumitem,cprotect}		% personnalise les listes d'items (nécessaire pour ol, al ...)
\usepackage{hyperref}		% Liens hypertexte
\usepackage{pstricks,pst-all,pst-node,pstricks-add,pst-math,pst-plot,pst-tree,pst-eucl} % pstricks
\usepackage[a4paper,includeheadfoot,top=2cm,left=3cm, bottom=2cm,right=3cm]{geometry} % marges etc.
\usepackage{comment}			% commentaires multilignes
\usepackage{amsmath,environ} % maths (matrices, etc.)
\usepackage{amssymb,makeidx}
\usepackage{bm}				% bold maths
\usepackage{tabularx}		% tableaux
\usepackage{colortbl}		% tableaux en couleur
\usepackage{fontawesome}		% Fontawesome
\usepackage{environ}			% environment with command
\usepackage{fp}				% calculs pour ps-tricks
\usepackage{multido}			% pour ps tricks
\usepackage[np]{numprint}	% formattage nombre
\usepackage{tikz,tkz-tab} 			% package principal TikZ
\usepackage{pgfplots}   % axes
\usepackage{mathrsfs}    % cursives
\usepackage{calc}			% calcul taille boites
\usepackage[scaled=0.875]{helvet} % font sans serif
\usepackage{svg} % svg
\usepackage{scrextend} % local margin
\usepackage{scratch} %scratch
\usepackage{multicol} % colonnes
%\usepackage{infix-RPN,pst-func} % formule en notation polanaise inversée
\usepackage{listings}

%================================================================================================================================
%
% Réglages de base
%
%================================================================================================================================

\lstset{
language=Python,   % R code
literate=
{á}{{\'a}}1
{à}{{\`a}}1
{ã}{{\~a}}1
{é}{{\'e}}1
{è}{{\`e}}1
{ê}{{\^e}}1
{í}{{\'i}}1
{ó}{{\'o}}1
{õ}{{\~o}}1
{ú}{{\'u}}1
{ü}{{\"u}}1
{ç}{{\c{c}}}1
{~}{{ }}1
}


\definecolor{codegreen}{rgb}{0,0.6,0}
\definecolor{codegray}{rgb}{0.5,0.5,0.5}
\definecolor{codepurple}{rgb}{0.58,0,0.82}
\definecolor{backcolour}{rgb}{0.95,0.95,0.92}

\lstdefinestyle{mystyle}{
    backgroundcolor=\color{backcolour},   
    commentstyle=\color{codegreen},
    keywordstyle=\color{magenta},
    numberstyle=\tiny\color{codegray},
    stringstyle=\color{codepurple},
    basicstyle=\ttfamily\footnotesize,
    breakatwhitespace=false,         
    breaklines=true,                 
    captionpos=b,                    
    keepspaces=true,                 
    numbers=left,                    
xleftmargin=2em,
framexleftmargin=2em,            
    showspaces=false,                
    showstringspaces=false,
    showtabs=false,                  
    tabsize=2,
    upquote=true
}

\lstset{style=mystyle}


\lstset{style=mystyle}
\newcommand{\imgdir}{C:/laragon/www/newmc/assets/imgsvg/}
\newcommand{\imgsvgdir}{C:/laragon/www/newmc/assets/imgsvg/}

\definecolor{mcgris}{RGB}{220, 220, 220}% ancien~; pour compatibilité
\definecolor{mcbleu}{RGB}{52, 152, 219}
\definecolor{mcvert}{RGB}{125, 194, 70}
\definecolor{mcmauve}{RGB}{154, 0, 215}
\definecolor{mcorange}{RGB}{255, 96, 0}
\definecolor{mcturquoise}{RGB}{0, 153, 153}
\definecolor{mcrouge}{RGB}{255, 0, 0}
\definecolor{mclightvert}{RGB}{205, 234, 190}

\definecolor{gris}{RGB}{220, 220, 220}
\definecolor{bleu}{RGB}{52, 152, 219}
\definecolor{vert}{RGB}{125, 194, 70}
\definecolor{mauve}{RGB}{154, 0, 215}
\definecolor{orange}{RGB}{255, 96, 0}
\definecolor{turquoise}{RGB}{0, 153, 153}
\definecolor{rouge}{RGB}{255, 0, 0}
\definecolor{lightvert}{RGB}{205, 234, 190}
\setitemize[0]{label=\color{lightvert}  $\bullet$}

\pagestyle{fancy}
\renewcommand{\headrulewidth}{0.2pt}
\fancyhead[L]{maths-cours.fr}
\fancyhead[R]{\thepage}
\renewcommand{\footrulewidth}{0.2pt}
\fancyfoot[C]{}

\newcolumntype{C}{>{\centering\arraybackslash}X}
\newcolumntype{s}{>{\hsize=.35\hsize\arraybackslash}X}

\setlength{\parindent}{0pt}		 
\setlength{\parskip}{3mm}
\setlength{\headheight}{1cm}

\def\ebook{ebook}
\def\book{book}
\def\web{web}
\def\type{web}

\newcommand{\vect}[1]{\overrightarrow{\,\mathstrut#1\,}}

\def\Oij{$\left(\text{O}~;~\vect{\imath},~\vect{\jmath}\right)$}
\def\Oijk{$\left(\text{O}~;~\vect{\imath},~\vect{\jmath},~\vect{k}\right)$}
\def\Ouv{$\left(\text{O}~;~\vect{u},~\vect{v}\right)$}

\hypersetup{breaklinks=true, colorlinks = true, linkcolor = OliveGreen, urlcolor = OliveGreen, citecolor = OliveGreen, pdfauthor={Didier BONNEL - https://www.maths-cours.fr} } % supprime les bordures autour des liens

\renewcommand{\arg}[0]{\text{arg}}

\everymath{\displaystyle}

%================================================================================================================================
%
% Macros - Commandes
%
%================================================================================================================================

\newcommand\meta[2]{    			% Utilisé pour créer le post HTML.
	\def\titre{titre}
	\def\url{url}
	\def\arg{#1}
	\ifx\titre\arg
		\newcommand\maintitle{#2}
		\fancyhead[L]{#2}
		{\Large\sffamily \MakeUppercase{#2}}
		\vspace{1mm}\textcolor{mcvert}{\hrule}
	\fi 
	\ifx\url\arg
		\fancyfoot[L]{\href{https://www.maths-cours.fr#2}{\black \footnotesize{https://www.maths-cours.fr#2}}}
	\fi 
}


\newcommand\TitreC[1]{    		% Titre centré
     \needspace{3\baselineskip}
     \begin{center}\textbf{#1}\end{center}
}

\newcommand\newpar{    		% paragraphe
     \par
}

\newcommand\nosp {    		% commande vide (pas d'espace)
}
\newcommand{\id}[1]{} %ignore

\newcommand\boite[2]{				% Boite simple sans titre
	\vspace{5mm}
	\setlength{\fboxrule}{0.2mm}
	\setlength{\fboxsep}{5mm}	
	\fcolorbox{#1}{#1!3}{\makebox[\linewidth-2\fboxrule-2\fboxsep]{
  		\begin{minipage}[t]{\linewidth-2\fboxrule-4\fboxsep}\setlength{\parskip}{3mm}
  			 #2
  		\end{minipage}
	}}
	\vspace{5mm}
}

\newcommand\CBox[4]{				% Boites
	\vspace{5mm}
	\setlength{\fboxrule}{0.2mm}
	\setlength{\fboxsep}{5mm}
	
	\fcolorbox{#1}{#1!3}{\makebox[\linewidth-2\fboxrule-2\fboxsep]{
		\begin{minipage}[t]{1cm}\setlength{\parskip}{3mm}
	  		\textcolor{#1}{\LARGE{#2}}    
 	 	\end{minipage}  
  		\begin{minipage}[t]{\linewidth-2\fboxrule-4\fboxsep}\setlength{\parskip}{3mm}
			\raisebox{1.2mm}{\normalsize\sffamily{\textcolor{#1}{#3}}}						
  			 #4
  		\end{minipage}
	}}
	\vspace{5mm}
}

\newcommand\cadre[3]{				% Boites convertible html
	\par
	\vspace{2mm}
	\setlength{\fboxrule}{0.1mm}
	\setlength{\fboxsep}{5mm}
	\fcolorbox{#1}{white}{\makebox[\linewidth-2\fboxrule-2\fboxsep]{
  		\begin{minipage}[t]{\linewidth-2\fboxrule-4\fboxsep}\setlength{\parskip}{3mm}
			\raisebox{-2.5mm}{\sffamily \small{\textcolor{#1}{\MakeUppercase{#2}}}}		
			\par		
  			 #3
 	 		\end{minipage}
	}}
		\vspace{2mm}
	\par
}

\newcommand\bloc[3]{				% Boites convertible html sans bordure
     \needspace{2\baselineskip}
     {\sffamily \small{\textcolor{#1}{\MakeUppercase{#2}}}}    
		\par		
  			 #3
		\par
}

\newcommand\CHelp[1]{
     \CBox{Plum}{\faInfoCircle}{À RETENIR}{#1}
}

\newcommand\CUp[1]{
     \CBox{NavyBlue}{\faThumbsOUp}{EN PRATIQUE}{#1}
}

\newcommand\CInfo[1]{
     \CBox{Sepia}{\faArrowCircleRight}{REMARQUE}{#1}
}

\newcommand\CRedac[1]{
     \CBox{PineGreen}{\faEdit}{BIEN R\'EDIGER}{#1}
}

\newcommand\CError[1]{
     \CBox{Red}{\faExclamationTriangle}{ATTENTION}{#1}
}

\newcommand\TitreExo[2]{
\needspace{4\baselineskip}
 {\sffamily\large EXERCICE #1\ (\emph{#2 points})}
\vspace{5mm}
}

\newcommand\img[2]{
          \includegraphics[width=#2\paperwidth]{\imgdir#1}
}

\newcommand\imgsvg[2]{
       \begin{center}   \includegraphics[width=#2\paperwidth]{\imgsvgdir#1} \end{center}
}


\newcommand\Lien[2]{
     \href{#1}{#2 \tiny \faExternalLink}
}
\newcommand\mcLien[2]{
     \href{https~://www.maths-cours.fr/#1}{#2 \tiny \faExternalLink}
}

\newcommand{\euro}{\eurologo{}}

%================================================================================================================================
%
% Macros - Environement
%
%================================================================================================================================

\newenvironment{tex}{ %
}
{%
}

\newenvironment{indente}{ %
	\setlength\parindent{10mm}
}

{
	\setlength\parindent{0mm}
}

\newenvironment{corrige}{%
     \needspace{3\baselineskip}
     \medskip
     \textbf{\textsc{Corrigé}}
     \medskip
}
{
}

\newenvironment{extern}{%
     \begin{center}
     }
     {
     \end{center}
}

\NewEnviron{code}{%
	\par
     \boite{gray}{\texttt{%
     \BODY
     }}
     \par
}

\newenvironment{vbloc}{% boite sans cadre empeche saut de page
     \begin{minipage}[t]{\linewidth}
     }
     {
     \end{minipage}
}
\NewEnviron{h2}{%
    \needspace{3\baselineskip}
    \vspace{0.6cm}
	\noindent \MakeUppercase{\sffamily \large \BODY}
	\vspace{1mm}\textcolor{mcgris}{\hrule}\vspace{0.4cm}
	\par
}{}

\NewEnviron{h3}{%
    \needspace{3\baselineskip}
	\vspace{5mm}
	\textsc{\BODY}
	\par
}

\NewEnviron{margeneg}{ %
\begin{addmargin}[-1cm]{0cm}
\BODY
\end{addmargin}
}

\NewEnviron{html}{%
}

\begin{document}
\meta{url}{/exercices/tableau-des-frequences-et-diagramme-circulaire/}
\meta{pid}{10660}
\meta{titre}{Tableau des fréquences et diagramme circulaire}
\meta{type}{exercices}
%
Une salle de spectacle propose 5 tarifs différents~:
\begin{itemize}
     \item %
     Tarif normal~: 30~€ / place
     \item %
     Plus de 60 ans~: 25~€ / place
     \item %
     Moins de 15 ans~: 20~€ / place
     \item %
     Groupe~: 18~€ / place
     \item %
     Place \og Dernière minute \fg{} ~: 15~€ / place
\end{itemize}
Le tableau (incomplet) ci-dessous indique le pourcentage de spectateurs pour chacun des tarifs lors de l'une des représentations~:
\begin{center}
     \begin{tabular}{|c|c|c|c|c|c|c|c|} %class="compact" width="600"
          \hline
          Tarif   &  30~€  &  25~€ &  20~€  &  18~€  &  15~€
          \\ \hline
          Spectateurs       & 60\%   &  10\%    & 15\%    &  10\%  & $\cdots$
          \\ \hline
     \end{tabular}
\end{center}
\begin{enumerate}
     \item %
     Quelle est l'étendue de cette série statistique~? À quoi correspond cette étendue~?
     \item %
     Compléter le tableau ci-dessus, en indiquant le pourcentage de spectateurs ayant bénéficié d'une place \og Dernière minute  \fg{}.
     \item %
     Au total, cette représentation a attiré 320 spectateurs.\\
     Dresser le tableau des effectifs donnant le nombre de spectateurs pour chacun des cinq tarifs.
     \item %
     Quelle a été la recette totale pour cette représentation~?
     \item %
     Représenter cette série statistique à l'aide d'un diagramme circulaire.
\end{enumerate}
\begin{corrige}
     \begin{enumerate}
          \item %
          L'étendue de cette série statistique est~:
          \begin{center}
               $30-15=15$~euros.
          \end{center}
          Cette étendue représente la différence entre le tarif le plus élevé et le tarif le moins élevé.
          \item %
          Le pourcentage total doit être égal à 100\%. \\
          Le pourcentage de spectateurs ayant bénéficié d'une place \og Dernière minute  \fg{} est donc :
          \begin{center}
               $\dfrac{100}{100}- $\nosp$\left(\dfrac{60}{100}+\dfrac{10}{100}+\dfrac{15}{100}+\dfrac{10}{100}\right)$\nosp$=\dfrac{5}{100}$
          \end{center}
          On peut alors compléter le tableau comme suit~:
          \begin{center}
               \begin{tabular}{|c|c|c|c|c|c|c|c|} %class="compact" width="600"
                    \hline
                    Tarif   &  30~€  &  25~€ &  20~€  &  18~€  &  15~€
                    \\ \hline
                    Spectateurs       & 60\%   &  10\%    & 15\%    &  10\%  &     \textcolor{red}{5\%}
                    \\ \hline
               \end{tabular}
          \end{center}
          \item %
          Pour obtenir le nombre de spectateurs pour chacun des cinq tarifs, il suffit de multiplier chacun des pourcentages par 320.
          \par
          Par exemple, pour trouver le nombre de spectateurs ayant payé 30~€, on calcule ~:
          \begin{center}
               $\dfrac{60}{100} \times 320 = 192$.
          \end{center}
          On obtient alors le tableau suivant~:
          \begin{center}
               \begin{tabular}{|c|c|c|c|c|c|c|c|} %class="compact" width="600"
                    \hline
                    Tarif   &  30~€  &  25~€ &  20~€  &  18~€  &  15~€
                    \\ \hline
                    Nb. de spect.       & 192   &  32   & 48   &  32 &     16
                    \\ \hline
               \end{tabular}
          \end{center}
          \item %
          Pour obtenir la recette totale, on calcule, pour chacun des tarifs, le produit du tarif par le nombre de spectateurs, puis on additionne le tout~:
          \begin{center}
               $R=30 \times 192 + 25 \times 32 + 20 \times 48$\nosp$ + 18 \times 32 + 15 \times 16 $\nosp$=8~336$.
          \end{center}
          La recette totale de la représentation s'élève à 8~336~€.
          \item %
          Pour construire un diagramme circulaire, on fait correspondre à chaque fréquence un angle de mesure \textbf{proportionnelle} sachant qu'une fréquence de 100\% correspond à un angle de 360°.
          On dresse donc le tableau ci-dessous~:
          \begin{center}
               \begin{tabular}{|c|c|c|c|c|c|c|c|c|} %class="compact" width="700"
                    \hline
                    Tarif   &  30~€  &  25~€ &  20~€  &  18~€  &  15~€ & \textit{Total}
                    \\ \hline
                    Spectateurs       & 60\%   &  10\%    & 15\%    &  10\%  &   5\% & \textit{100\%}
                    \\ \hline
                    Angle       & 216°   &  36°    & 54°    &  36°  & 18°  & \textit{360°}
                    \\ \hline
               \end{tabular}
          \end{center}
          On construit ensuite le diagramme circulaire à l'aide du rapporteur~:
          \begin{center}
               \begin{extern}%width="300" alt="diagramme circulaire"
                    \begin{pspicture}(-3,-3)(3,3)
                         \psChart[userColor={cyan!60,green!80,blue!40,yellow,magenta!50}]{60,10,15,10,5}{}{3}
                         \rput(psChartI1){30~€}\rput(psChartI2){25~€}\rput(psChartI3){20~€}
                         \rput(psChartI4){18~€}\rput(psChartI5){15~€}
                    \end{pspicture}
               \end{extern}
          \end{center}
     \end{enumerate}
\end{corrige}

\end{document}
µ
\documentclass[a4paper]{article}

%================================================================================================================================
%
% Packages
%
%================================================================================================================================

\usepackage[T1]{fontenc} 	% pour caractères accentués
\usepackage[utf8]{inputenc}  % encodage utf8
\usepackage[french]{babel}	% langue : français
\usepackage{fourier}			% caractères plus lisibles
\usepackage[dvipsnames]{xcolor} % couleurs
\usepackage{fancyhdr}		% réglage header footer
\usepackage{needspace}		% empêcher sauts de page mal placés
\usepackage{graphicx}		% pour inclure des graphiques
\usepackage{enumitem,cprotect}		% personnalise les listes d'items (nécessaire pour ol, al ...)
\usepackage{hyperref}		% Liens hypertexte
\usepackage{pstricks,pst-all,pst-node,pstricks-add,pst-math,pst-plot,pst-tree,pst-eucl} % pstricks
\usepackage[a4paper,includeheadfoot,top=2cm,left=3cm, bottom=2cm,right=3cm]{geometry} % marges etc.
\usepackage{comment}			% commentaires multilignes
\usepackage{amsmath,environ} % maths (matrices, etc.)
\usepackage{amssymb,makeidx}
\usepackage{bm}				% bold maths
\usepackage{tabularx}		% tableaux
\usepackage{colortbl}		% tableaux en couleur
\usepackage{fontawesome}		% Fontawesome
\usepackage{environ}			% environment with command
\usepackage{fp}				% calculs pour ps-tricks
\usepackage{multido}			% pour ps tricks
\usepackage[np]{numprint}	% formattage nombre
\usepackage{tikz,tkz-tab} 			% package principal TikZ
\usepackage{pgfplots}   % axes
\usepackage{mathrsfs}    % cursives
\usepackage{calc}			% calcul taille boites
\usepackage[scaled=0.875]{helvet} % font sans serif
\usepackage{svg} % svg
\usepackage{scrextend} % local margin
\usepackage{scratch} %scratch
\usepackage{multicol} % colonnes
%\usepackage{infix-RPN,pst-func} % formule en notation polanaise inversée
\usepackage{listings}

%================================================================================================================================
%
% Réglages de base
%
%================================================================================================================================

\lstset{
language=Python,   % R code
literate=
{á}{{\'a}}1
{à}{{\`a}}1
{ã}{{\~a}}1
{é}{{\'e}}1
{è}{{\`e}}1
{ê}{{\^e}}1
{í}{{\'i}}1
{ó}{{\'o}}1
{õ}{{\~o}}1
{ú}{{\'u}}1
{ü}{{\"u}}1
{ç}{{\c{c}}}1
{~}{{ }}1
}


\definecolor{codegreen}{rgb}{0,0.6,0}
\definecolor{codegray}{rgb}{0.5,0.5,0.5}
\definecolor{codepurple}{rgb}{0.58,0,0.82}
\definecolor{backcolour}{rgb}{0.95,0.95,0.92}

\lstdefinestyle{mystyle}{
    backgroundcolor=\color{backcolour},   
    commentstyle=\color{codegreen},
    keywordstyle=\color{magenta},
    numberstyle=\tiny\color{codegray},
    stringstyle=\color{codepurple},
    basicstyle=\ttfamily\footnotesize,
    breakatwhitespace=false,         
    breaklines=true,                 
    captionpos=b,                    
    keepspaces=true,                 
    numbers=left,                    
xleftmargin=2em,
framexleftmargin=2em,            
    showspaces=false,                
    showstringspaces=false,
    showtabs=false,                  
    tabsize=2,
    upquote=true
}

\lstset{style=mystyle}


\lstset{style=mystyle}
\newcommand{\imgdir}{C:/laragon/www/newmc/assets/imgsvg/}
\newcommand{\imgsvgdir}{C:/laragon/www/newmc/assets/imgsvg/}

\definecolor{mcgris}{RGB}{220, 220, 220}% ancien~; pour compatibilité
\definecolor{mcbleu}{RGB}{52, 152, 219}
\definecolor{mcvert}{RGB}{125, 194, 70}
\definecolor{mcmauve}{RGB}{154, 0, 215}
\definecolor{mcorange}{RGB}{255, 96, 0}
\definecolor{mcturquoise}{RGB}{0, 153, 153}
\definecolor{mcrouge}{RGB}{255, 0, 0}
\definecolor{mclightvert}{RGB}{205, 234, 190}

\definecolor{gris}{RGB}{220, 220, 220}
\definecolor{bleu}{RGB}{52, 152, 219}
\definecolor{vert}{RGB}{125, 194, 70}
\definecolor{mauve}{RGB}{154, 0, 215}
\definecolor{orange}{RGB}{255, 96, 0}
\definecolor{turquoise}{RGB}{0, 153, 153}
\definecolor{rouge}{RGB}{255, 0, 0}
\definecolor{lightvert}{RGB}{205, 234, 190}
\setitemize[0]{label=\color{lightvert}  $\bullet$}

\pagestyle{fancy}
\renewcommand{\headrulewidth}{0.2pt}
\fancyhead[L]{maths-cours.fr}
\fancyhead[R]{\thepage}
\renewcommand{\footrulewidth}{0.2pt}
\fancyfoot[C]{}

\newcolumntype{C}{>{\centering\arraybackslash}X}
\newcolumntype{s}{>{\hsize=.35\hsize\arraybackslash}X}

\setlength{\parindent}{0pt}		 
\setlength{\parskip}{3mm}
\setlength{\headheight}{1cm}

\def\ebook{ebook}
\def\book{book}
\def\web{web}
\def\type{web}

\newcommand{\vect}[1]{\overrightarrow{\,\mathstrut#1\,}}

\def\Oij{$\left(\text{O}~;~\vect{\imath},~\vect{\jmath}\right)$}
\def\Oijk{$\left(\text{O}~;~\vect{\imath},~\vect{\jmath},~\vect{k}\right)$}
\def\Ouv{$\left(\text{O}~;~\vect{u},~\vect{v}\right)$}

\hypersetup{breaklinks=true, colorlinks = true, linkcolor = OliveGreen, urlcolor = OliveGreen, citecolor = OliveGreen, pdfauthor={Didier BONNEL - https://www.maths-cours.fr} } % supprime les bordures autour des liens

\renewcommand{\arg}[0]{\text{arg}}

\everymath{\displaystyle}

%================================================================================================================================
%
% Macros - Commandes
%
%================================================================================================================================

\newcommand\meta[2]{    			% Utilisé pour créer le post HTML.
	\def\titre{titre}
	\def\url{url}
	\def\arg{#1}
	\ifx\titre\arg
		\newcommand\maintitle{#2}
		\fancyhead[L]{#2}
		{\Large\sffamily \MakeUppercase{#2}}
		\vspace{1mm}\textcolor{mcvert}{\hrule}
	\fi 
	\ifx\url\arg
		\fancyfoot[L]{\href{https://www.maths-cours.fr#2}{\black \footnotesize{https://www.maths-cours.fr#2}}}
	\fi 
}


\newcommand\TitreC[1]{    		% Titre centré
     \needspace{3\baselineskip}
     \begin{center}\textbf{#1}\end{center}
}

\newcommand\newpar{    		% paragraphe
     \par
}

\newcommand\nosp {    		% commande vide (pas d'espace)
}
\newcommand{\id}[1]{} %ignore

\newcommand\boite[2]{				% Boite simple sans titre
	\vspace{5mm}
	\setlength{\fboxrule}{0.2mm}
	\setlength{\fboxsep}{5mm}	
	\fcolorbox{#1}{#1!3}{\makebox[\linewidth-2\fboxrule-2\fboxsep]{
  		\begin{minipage}[t]{\linewidth-2\fboxrule-4\fboxsep}\setlength{\parskip}{3mm}
  			 #2
  		\end{minipage}
	}}
	\vspace{5mm}
}

\newcommand\CBox[4]{				% Boites
	\vspace{5mm}
	\setlength{\fboxrule}{0.2mm}
	\setlength{\fboxsep}{5mm}
	
	\fcolorbox{#1}{#1!3}{\makebox[\linewidth-2\fboxrule-2\fboxsep]{
		\begin{minipage}[t]{1cm}\setlength{\parskip}{3mm}
	  		\textcolor{#1}{\LARGE{#2}}    
 	 	\end{minipage}  
  		\begin{minipage}[t]{\linewidth-2\fboxrule-4\fboxsep}\setlength{\parskip}{3mm}
			\raisebox{1.2mm}{\normalsize\sffamily{\textcolor{#1}{#3}}}						
  			 #4
  		\end{minipage}
	}}
	\vspace{5mm}
}

\newcommand\cadre[3]{				% Boites convertible html
	\par
	\vspace{2mm}
	\setlength{\fboxrule}{0.1mm}
	\setlength{\fboxsep}{5mm}
	\fcolorbox{#1}{white}{\makebox[\linewidth-2\fboxrule-2\fboxsep]{
  		\begin{minipage}[t]{\linewidth-2\fboxrule-4\fboxsep}\setlength{\parskip}{3mm}
			\raisebox{-2.5mm}{\sffamily \small{\textcolor{#1}{\MakeUppercase{#2}}}}		
			\par		
  			 #3
 	 		\end{minipage}
	}}
		\vspace{2mm}
	\par
}

\newcommand\bloc[3]{				% Boites convertible html sans bordure
     \needspace{2\baselineskip}
     {\sffamily \small{\textcolor{#1}{\MakeUppercase{#2}}}}    
		\par		
  			 #3
		\par
}

\newcommand\CHelp[1]{
     \CBox{Plum}{\faInfoCircle}{À RETENIR}{#1}
}

\newcommand\CUp[1]{
     \CBox{NavyBlue}{\faThumbsOUp}{EN PRATIQUE}{#1}
}

\newcommand\CInfo[1]{
     \CBox{Sepia}{\faArrowCircleRight}{REMARQUE}{#1}
}

\newcommand\CRedac[1]{
     \CBox{PineGreen}{\faEdit}{BIEN R\'EDIGER}{#1}
}

\newcommand\CError[1]{
     \CBox{Red}{\faExclamationTriangle}{ATTENTION}{#1}
}

\newcommand\TitreExo[2]{
\needspace{4\baselineskip}
 {\sffamily\large EXERCICE #1\ (\emph{#2 points})}
\vspace{5mm}
}

\newcommand\img[2]{
          \includegraphics[width=#2\paperwidth]{\imgdir#1}
}

\newcommand\imgsvg[2]{
       \begin{center}   \includegraphics[width=#2\paperwidth]{\imgsvgdir#1} \end{center}
}


\newcommand\Lien[2]{
     \href{#1}{#2 \tiny \faExternalLink}
}
\newcommand\mcLien[2]{
     \href{https~://www.maths-cours.fr/#1}{#2 \tiny \faExternalLink}
}

\newcommand{\euro}{\eurologo{}}

%================================================================================================================================
%
% Macros - Environement
%
%================================================================================================================================

\newenvironment{tex}{ %
}
{%
}

\newenvironment{indente}{ %
	\setlength\parindent{10mm}
}

{
	\setlength\parindent{0mm}
}

\newenvironment{corrige}{%
     \needspace{3\baselineskip}
     \medskip
     \textbf{\textsc{Corrigé}}
     \medskip
}
{
}

\newenvironment{extern}{%
     \begin{center}
     }
     {
     \end{center}
}

\NewEnviron{code}{%
	\par
     \boite{gray}{\texttt{%
     \BODY
     }}
     \par
}

\newenvironment{vbloc}{% boite sans cadre empeche saut de page
     \begin{minipage}[t]{\linewidth}
     }
     {
     \end{minipage}
}
\NewEnviron{h2}{%
    \needspace{3\baselineskip}
    \vspace{0.6cm}
	\noindent \MakeUppercase{\sffamily \large \BODY}
	\vspace{1mm}\textcolor{mcgris}{\hrule}\vspace{0.4cm}
	\par
}{}

\NewEnviron{h3}{%
    \needspace{3\baselineskip}
	\vspace{5mm}
	\textsc{\BODY}
	\par
}

\NewEnviron{margeneg}{ %
\begin{addmargin}[-1cm]{0cm}
\BODY
\end{addmargin}
}

\NewEnviron{html}{%
}

\begin{document}
\meta{url}{/exercices/statistiques-nombre-de-buts-marques/}
\meta{pid}{10674}
\meta{titre}{Statistiques : nombre de buts marqués}
\meta{type}{exercices}
%
Le tableau ci-dessous recense le nombre de buts marqués par match lors d'un tournoi de football~:
\begin{center}
     \begin{tabular}{|c|c|c|c|c|c|c|c|c|c|c|} %class="compact" width="600"
          \hline
          Nb. buts   &  0 & 1 & 2 & 3 & 4 & 5 & 6 & 7 & 8
          \\ \hline
          Nb. matchs     &  6 & 8 & 6 & 4 & 2 & 2 & 2 & 1 & 2
          \\ \hline
     \end{tabular}
\end{center}
\begin{enumerate}
     \item %
     Combien de matchs, au total, ont été disputés lors de ce tournoi~?
     \item %
     Clara affirme~:
     \textit{\og Dans plus de 10\% des matchs, aucun but n'a été marqué \fg{}}.
     \par
     A-t-elle raison~? Justifier votre réponse.
     \item %
     Au total, combien de buts ont été marqués lors du tournoi~?
     \item %
     En moyenne, combien de buts par match ont été marqués~?
     \item %
     Quel est la médiane de cette série~?\\
     Comment peut-on interpréter cette médiane~?
\end{enumerate}
\begin{corrige}
     \begin{enumerate}
          \item %
          Pour déterminer le nombre de matchs disputés au cours de ce tournoi, il suffit d'effectuer la somme~:
          \par
          $6 + 8 + 6 + 4 + 2 + 2 + 2 + 1 + 2 $\nosp$=33.$
          \par
          33 matchs ont été disputés.
          \item %
          Aucun but n'a été marqué lors de 6 matchs sur 33. Pour vérifier l'affirmation de Clara, il suffit de convertir $\dfrac{6}{33}$ en pourcentage~:
          \par
          $\dfrac{6}{33} \approx 0,182=18,2$\%.
          \par
          Clara a donc raison lorsqu'elle affirme qu'aucun but n'a été marqué dans plus de 10\% des matchs.
          \item %
          Pour déterminer le nombre total de buts marqués, on effectue, pour chacune des colonnes du tableau, le produit du nombre de buts par le nombre de matchs, puis on additionne le tout~:
          \begin{center}
               $N=  0 \times 6 + 1 \times 8 $\nosp$ + 2 \times 6 + 3 \times 4 +  4 \times 2  $\nosp$ +  5 \times 2  +  6 \times 2  $\nosp$+  7 \times 1  + 8 \times 2$\nosp$ = 85.$
          \end{center}
          85 buts ont été marqués lors de ce tournoi.
          \item %
          Pour déterminer le nombre de buts marqués par match en moyenne, il suffit de diviser le résultat précédent par le nombre de matchs soit 33~:
          \par
          $m=\dfrac{85}{33} \approx 2,58.$
          \par
          En moyenne, il y a eu 2,58 buts marqués par match.
          \par
          \item %
          Si l'on détaille le nombre de buts marqués pour chacun des 33 matchs (trié par ordre croissant), on obtient~:
          \begin{center}
               0~; 0~; 0~; 0~; 0~; 0~; 1~; 1~; 1~; 1~; 1~; 1~; 1~; 1~; 2~; 2~; \textcolor{red}{2}~; 2~; 2~; 2~; 3~; 3~; 3~; 3~; 4~; 4~; 5~; 5~; 6~; 6~; 7~;  8~; 8.
          \end{center}
          La médiane correspond à la valeur située \og au milieu \fg{} de cette liste, c'est à dire en dix-septième position. \\
          La médiane est donc égale à 2.
          \par
          Cela signifie que, dans plus de la moitié des matchs, il y a eu au moins 2 buts marqués et, toujours dans plus de la moitié des matchs, il y a eu au plus 2 buts marqués.
     \end{enumerate}
\end{corrige}

\end{document}
µ
\documentclass[a4paper]{article}

%================================================================================================================================
%
% Packages
%
%================================================================================================================================

\usepackage[T1]{fontenc} 	% pour caractères accentués
\usepackage[utf8]{inputenc}  % encodage utf8
\usepackage[french]{babel}	% langue : français
\usepackage{fourier}			% caractères plus lisibles
\usepackage[dvipsnames]{xcolor} % couleurs
\usepackage{fancyhdr}		% réglage header footer
\usepackage{needspace}		% empêcher sauts de page mal placés
\usepackage{graphicx}		% pour inclure des graphiques
\usepackage{enumitem,cprotect}		% personnalise les listes d'items (nécessaire pour ol, al ...)
\usepackage{hyperref}		% Liens hypertexte
\usepackage{pstricks,pst-all,pst-node,pstricks-add,pst-math,pst-plot,pst-tree,pst-eucl} % pstricks
\usepackage[a4paper,includeheadfoot,top=2cm,left=3cm, bottom=2cm,right=3cm]{geometry} % marges etc.
\usepackage{comment}			% commentaires multilignes
\usepackage{amsmath,environ} % maths (matrices, etc.)
\usepackage{amssymb,makeidx}
\usepackage{bm}				% bold maths
\usepackage{tabularx}		% tableaux
\usepackage{colortbl}		% tableaux en couleur
\usepackage{fontawesome}		% Fontawesome
\usepackage{environ}			% environment with command
\usepackage{fp}				% calculs pour ps-tricks
\usepackage{multido}			% pour ps tricks
\usepackage[np]{numprint}	% formattage nombre
\usepackage{tikz,tkz-tab} 			% package principal TikZ
\usepackage{pgfplots}   % axes
\usepackage{mathrsfs}    % cursives
\usepackage{calc}			% calcul taille boites
\usepackage[scaled=0.875]{helvet} % font sans serif
\usepackage{svg} % svg
\usepackage{scrextend} % local margin
\usepackage{scratch} %scratch
\usepackage{multicol} % colonnes
%\usepackage{infix-RPN,pst-func} % formule en notation polanaise inversée
\usepackage{listings}

%================================================================================================================================
%
% Réglages de base
%
%================================================================================================================================

\lstset{
language=Python,   % R code
literate=
{á}{{\'a}}1
{à}{{\`a}}1
{ã}{{\~a}}1
{é}{{\'e}}1
{è}{{\`e}}1
{ê}{{\^e}}1
{í}{{\'i}}1
{ó}{{\'o}}1
{õ}{{\~o}}1
{ú}{{\'u}}1
{ü}{{\"u}}1
{ç}{{\c{c}}}1
{~}{{ }}1
}


\definecolor{codegreen}{rgb}{0,0.6,0}
\definecolor{codegray}{rgb}{0.5,0.5,0.5}
\definecolor{codepurple}{rgb}{0.58,0,0.82}
\definecolor{backcolour}{rgb}{0.95,0.95,0.92}

\lstdefinestyle{mystyle}{
    backgroundcolor=\color{backcolour},   
    commentstyle=\color{codegreen},
    keywordstyle=\color{magenta},
    numberstyle=\tiny\color{codegray},
    stringstyle=\color{codepurple},
    basicstyle=\ttfamily\footnotesize,
    breakatwhitespace=false,         
    breaklines=true,                 
    captionpos=b,                    
    keepspaces=true,                 
    numbers=left,                    
xleftmargin=2em,
framexleftmargin=2em,            
    showspaces=false,                
    showstringspaces=false,
    showtabs=false,                  
    tabsize=2,
    upquote=true
}

\lstset{style=mystyle}


\lstset{style=mystyle}
\newcommand{\imgdir}{C:/laragon/www/newmc/assets/imgsvg/}
\newcommand{\imgsvgdir}{C:/laragon/www/newmc/assets/imgsvg/}

\definecolor{mcgris}{RGB}{220, 220, 220}% ancien~; pour compatibilité
\definecolor{mcbleu}{RGB}{52, 152, 219}
\definecolor{mcvert}{RGB}{125, 194, 70}
\definecolor{mcmauve}{RGB}{154, 0, 215}
\definecolor{mcorange}{RGB}{255, 96, 0}
\definecolor{mcturquoise}{RGB}{0, 153, 153}
\definecolor{mcrouge}{RGB}{255, 0, 0}
\definecolor{mclightvert}{RGB}{205, 234, 190}

\definecolor{gris}{RGB}{220, 220, 220}
\definecolor{bleu}{RGB}{52, 152, 219}
\definecolor{vert}{RGB}{125, 194, 70}
\definecolor{mauve}{RGB}{154, 0, 215}
\definecolor{orange}{RGB}{255, 96, 0}
\definecolor{turquoise}{RGB}{0, 153, 153}
\definecolor{rouge}{RGB}{255, 0, 0}
\definecolor{lightvert}{RGB}{205, 234, 190}
\setitemize[0]{label=\color{lightvert}  $\bullet$}

\pagestyle{fancy}
\renewcommand{\headrulewidth}{0.2pt}
\fancyhead[L]{maths-cours.fr}
\fancyhead[R]{\thepage}
\renewcommand{\footrulewidth}{0.2pt}
\fancyfoot[C]{}

\newcolumntype{C}{>{\centering\arraybackslash}X}
\newcolumntype{s}{>{\hsize=.35\hsize\arraybackslash}X}

\setlength{\parindent}{0pt}		 
\setlength{\parskip}{3mm}
\setlength{\headheight}{1cm}

\def\ebook{ebook}
\def\book{book}
\def\web{web}
\def\type{web}

\newcommand{\vect}[1]{\overrightarrow{\,\mathstrut#1\,}}

\def\Oij{$\left(\text{O}~;~\vect{\imath},~\vect{\jmath}\right)$}
\def\Oijk{$\left(\text{O}~;~\vect{\imath},~\vect{\jmath},~\vect{k}\right)$}
\def\Ouv{$\left(\text{O}~;~\vect{u},~\vect{v}\right)$}

\hypersetup{breaklinks=true, colorlinks = true, linkcolor = OliveGreen, urlcolor = OliveGreen, citecolor = OliveGreen, pdfauthor={Didier BONNEL - https://www.maths-cours.fr} } % supprime les bordures autour des liens

\renewcommand{\arg}[0]{\text{arg}}

\everymath{\displaystyle}

%================================================================================================================================
%
% Macros - Commandes
%
%================================================================================================================================

\newcommand\meta[2]{    			% Utilisé pour créer le post HTML.
	\def\titre{titre}
	\def\url{url}
	\def\arg{#1}
	\ifx\titre\arg
		\newcommand\maintitle{#2}
		\fancyhead[L]{#2}
		{\Large\sffamily \MakeUppercase{#2}}
		\vspace{1mm}\textcolor{mcvert}{\hrule}
	\fi 
	\ifx\url\arg
		\fancyfoot[L]{\href{https://www.maths-cours.fr#2}{\black \footnotesize{https://www.maths-cours.fr#2}}}
	\fi 
}


\newcommand\TitreC[1]{    		% Titre centré
     \needspace{3\baselineskip}
     \begin{center}\textbf{#1}\end{center}
}

\newcommand\newpar{    		% paragraphe
     \par
}

\newcommand\nosp {    		% commande vide (pas d'espace)
}
\newcommand{\id}[1]{} %ignore

\newcommand\boite[2]{				% Boite simple sans titre
	\vspace{5mm}
	\setlength{\fboxrule}{0.2mm}
	\setlength{\fboxsep}{5mm}	
	\fcolorbox{#1}{#1!3}{\makebox[\linewidth-2\fboxrule-2\fboxsep]{
  		\begin{minipage}[t]{\linewidth-2\fboxrule-4\fboxsep}\setlength{\parskip}{3mm}
  			 #2
  		\end{minipage}
	}}
	\vspace{5mm}
}

\newcommand\CBox[4]{				% Boites
	\vspace{5mm}
	\setlength{\fboxrule}{0.2mm}
	\setlength{\fboxsep}{5mm}
	
	\fcolorbox{#1}{#1!3}{\makebox[\linewidth-2\fboxrule-2\fboxsep]{
		\begin{minipage}[t]{1cm}\setlength{\parskip}{3mm}
	  		\textcolor{#1}{\LARGE{#2}}    
 	 	\end{minipage}  
  		\begin{minipage}[t]{\linewidth-2\fboxrule-4\fboxsep}\setlength{\parskip}{3mm}
			\raisebox{1.2mm}{\normalsize\sffamily{\textcolor{#1}{#3}}}						
  			 #4
  		\end{minipage}
	}}
	\vspace{5mm}
}

\newcommand\cadre[3]{				% Boites convertible html
	\par
	\vspace{2mm}
	\setlength{\fboxrule}{0.1mm}
	\setlength{\fboxsep}{5mm}
	\fcolorbox{#1}{white}{\makebox[\linewidth-2\fboxrule-2\fboxsep]{
  		\begin{minipage}[t]{\linewidth-2\fboxrule-4\fboxsep}\setlength{\parskip}{3mm}
			\raisebox{-2.5mm}{\sffamily \small{\textcolor{#1}{\MakeUppercase{#2}}}}		
			\par		
  			 #3
 	 		\end{minipage}
	}}
		\vspace{2mm}
	\par
}

\newcommand\bloc[3]{				% Boites convertible html sans bordure
     \needspace{2\baselineskip}
     {\sffamily \small{\textcolor{#1}{\MakeUppercase{#2}}}}    
		\par		
  			 #3
		\par
}

\newcommand\CHelp[1]{
     \CBox{Plum}{\faInfoCircle}{À RETENIR}{#1}
}

\newcommand\CUp[1]{
     \CBox{NavyBlue}{\faThumbsOUp}{EN PRATIQUE}{#1}
}

\newcommand\CInfo[1]{
     \CBox{Sepia}{\faArrowCircleRight}{REMARQUE}{#1}
}

\newcommand\CRedac[1]{
     \CBox{PineGreen}{\faEdit}{BIEN R\'EDIGER}{#1}
}

\newcommand\CError[1]{
     \CBox{Red}{\faExclamationTriangle}{ATTENTION}{#1}
}

\newcommand\TitreExo[2]{
\needspace{4\baselineskip}
 {\sffamily\large EXERCICE #1\ (\emph{#2 points})}
\vspace{5mm}
}

\newcommand\img[2]{
          \includegraphics[width=#2\paperwidth]{\imgdir#1}
}

\newcommand\imgsvg[2]{
       \begin{center}   \includegraphics[width=#2\paperwidth]{\imgsvgdir#1} \end{center}
}


\newcommand\Lien[2]{
     \href{#1}{#2 \tiny \faExternalLink}
}
\newcommand\mcLien[2]{
     \href{https~://www.maths-cours.fr/#1}{#2 \tiny \faExternalLink}
}

\newcommand{\euro}{\eurologo{}}

%================================================================================================================================
%
% Macros - Environement
%
%================================================================================================================================

\newenvironment{tex}{ %
}
{%
}

\newenvironment{indente}{ %
	\setlength\parindent{10mm}
}

{
	\setlength\parindent{0mm}
}

\newenvironment{corrige}{%
     \needspace{3\baselineskip}
     \medskip
     \textbf{\textsc{Corrigé}}
     \medskip
}
{
}

\newenvironment{extern}{%
     \begin{center}
     }
     {
     \end{center}
}

\NewEnviron{code}{%
	\par
     \boite{gray}{\texttt{%
     \BODY
     }}
     \par
}

\newenvironment{vbloc}{% boite sans cadre empeche saut de page
     \begin{minipage}[t]{\linewidth}
     }
     {
     \end{minipage}
}
\NewEnviron{h2}{%
    \needspace{3\baselineskip}
    \vspace{0.6cm}
	\noindent \MakeUppercase{\sffamily \large \BODY}
	\vspace{1mm}\textcolor{mcgris}{\hrule}\vspace{0.4cm}
	\par
}{}

\NewEnviron{h3}{%
    \needspace{3\baselineskip}
	\vspace{5mm}
	\textsc{\BODY}
	\par
}

\NewEnviron{margeneg}{ %
\begin{addmargin}[-1cm]{0cm}
\BODY
\end{addmargin}
}

\NewEnviron{html}{%
}

\begin{document}
\meta{url}{/exercices/pourcentage-et-geometrie/}
\meta{pid}{10727}
\meta{titre}{Pourcentage et géométrie}
\meta{type}{exercices}
Sur l'image ci-dessous, quel pourcentage du rectangle est coloré en vert~?
\begin{center}
     \begin{extern} %width="300" alt="pourcentage aire"
          \newrgbcolor{zzccqq}{0.6 0.8 0.}
          \psset{xunit=1.0cm,yunit=1.0cm,algebraic=true,dimen=middle,dotstyle=o,dotsize=5pt 0,linewidth=1pt,arrowsize=3pt 2,arrowinset=0.25}
          \begin{pspicture*}(0.31248976630320724,-1.5868139302165756)(7.639721755514631,4.751466502168787)
               \psline[linewidth=0.5pt,linecolor=lightgray](2.,4.)(2.,-1.)
               \psline[linewidth=0.5pt,linecolor=lightgray](3.,4.)(3.,-1.)
               \psline[linewidth=0.5pt,linecolor=lightgray](4.,4.)(4.,-1.)
               \psline[linewidth=0.5pt,linecolor=lightgray](5.,4.)(5.,-1.)
               \psline[linewidth=0.5pt,linecolor=lightgray](6.,4.)(6.,-1.)
               \psline[linewidth=0.5pt,linecolor=lightgray](1.,3.)(7.,3.)
               \psline[linewidth=0.5pt,linecolor=lightgray](1.,2.)(7.,2.)
               \psline[linewidth=0.5pt,linecolor=lightgray](1.,1.)(7.,1.)
               \psline[linewidth=0.5pt,linecolor=lightgray](1.,0.)(7.,0.)
               \parametricplot[linewidth=0.5pt,linecolor=zzccqq,fillcolor=zzccqq,fillstyle=solid,opacity=1]{-0.7853981633974483}{2.356194490192345}{1.*1.4142135623730951*cos(t)+0.*1.4142135623730951*sin(t)+2.|0.*1.4142135623730951*cos(t)+1.*1.4142135623730951*sin(t)+0.}
               \parametricplot[linewidth=0.5pt,linecolor=zzccqq,fillcolor=zzccqq,fillstyle=solid,opacity=1]{-1.5707963267948966}{0.0}{1.*1.*cos(t)+0.*1.*sin(t)+1.|0.*1.*cos(t)+1.*1.*sin(t)+4.}
               \parametricplot[linewidth=0.5pt,linecolor=zzccqq,fillcolor=zzccqq,fillstyle=solid,opacity=0.95]{1.5707963267948966}{4.71238898038469}{1.*1.*cos(t)+0.*1.*sin(t)+7.|0.*1.*cos(t)+1.*1.*sin(t)+0.}
               \pscircle[linewidth=0.5pt,linecolor=zzccqq,fillcolor=zzccqq,fillstyle=solid,opacity=1](6.,3.){1.}
               \pspolygon[linewidth=0.5pt](1.,4.)(7.,4.)(7.,-1.)(1.,-1.)
               \pspolygon[linewidth=0.5pt,linecolor=zzccqq,fillcolor=zzccqq,fillstyle=solid,opacity=1](1.,1.)(3.,-1.)(1.,-1.)
               \pspolygon[linewidth=0.5pt,linecolor=zzccqq,fillcolor=zzccqq,fillstyle=solid,opacity=1](1.,4.)(1.,3.)(2.,4.)
               \psline[linewidth=0.5pt](1.,4.)(7.,4.)
               \psline[linewidth=0.5pt](7.,4.)(7.,-1.)
               \psline[linewidth=0.5pt](7.,-1.)(1.,-1.)
               \psline[linewidth=0.5pt](1.,-1.)(1.,4.)
               \psdots[dotsize=1pt 0,dotstyle=+](1.,4.)
               \rput[bl](0.6721085142399643,4.12213369327946){$A$}
               \psdots[dotsize=1pt 0,dotstyle=+](7.,4.)
               \rput[bl](7.0253730611226715,4.12213369327946){$B$}
               \psdots[dotsize=1pt 0,dotstyle=+](7.,-1.)
               \rput[bl](6.995404832127941,-1.3470680982587366){$C$}
               \psdots[dotsize=1pt 0,dotstyle=+](1.,-1.)
               \rput[bl](0.6721085142399643,-1.3470680982587366){$D$}
          \end{pspicture*}
     \end{extern}
\end{center}

\end{document}
µ
\documentclass[a4paper]{article}

%================================================================================================================================
%
% Packages
%
%================================================================================================================================

\usepackage[T1]{fontenc} 	% pour caractères accentués
\usepackage[utf8]{inputenc}  % encodage utf8
\usepackage[french]{babel}	% langue : français
\usepackage{fourier}			% caractères plus lisibles
\usepackage[dvipsnames]{xcolor} % couleurs
\usepackage{fancyhdr}		% réglage header footer
\usepackage{needspace}		% empêcher sauts de page mal placés
\usepackage{graphicx}		% pour inclure des graphiques
\usepackage{enumitem,cprotect}		% personnalise les listes d'items (nécessaire pour ol, al ...)
\usepackage{hyperref}		% Liens hypertexte
\usepackage{pstricks,pst-all,pst-node,pstricks-add,pst-math,pst-plot,pst-tree,pst-eucl} % pstricks
\usepackage[a4paper,includeheadfoot,top=2cm,left=3cm, bottom=2cm,right=3cm]{geometry} % marges etc.
\usepackage{comment}			% commentaires multilignes
\usepackage{amsmath,environ} % maths (matrices, etc.)
\usepackage{amssymb,makeidx}
\usepackage{bm}				% bold maths
\usepackage{tabularx}		% tableaux
\usepackage{colortbl}		% tableaux en couleur
\usepackage{fontawesome}		% Fontawesome
\usepackage{environ}			% environment with command
\usepackage{fp}				% calculs pour ps-tricks
\usepackage{multido}			% pour ps tricks
\usepackage[np]{numprint}	% formattage nombre
\usepackage{tikz,tkz-tab} 			% package principal TikZ
\usepackage{pgfplots}   % axes
\usepackage{mathrsfs}    % cursives
\usepackage{calc}			% calcul taille boites
\usepackage[scaled=0.875]{helvet} % font sans serif
\usepackage{svg} % svg
\usepackage{scrextend} % local margin
\usepackage{scratch} %scratch
\usepackage{multicol} % colonnes
%\usepackage{infix-RPN,pst-func} % formule en notation polanaise inversée
\usepackage{listings}

%================================================================================================================================
%
% Réglages de base
%
%================================================================================================================================

\lstset{
language=Python,   % R code
literate=
{á}{{\'a}}1
{à}{{\`a}}1
{ã}{{\~a}}1
{é}{{\'e}}1
{è}{{\`e}}1
{ê}{{\^e}}1
{í}{{\'i}}1
{ó}{{\'o}}1
{õ}{{\~o}}1
{ú}{{\'u}}1
{ü}{{\"u}}1
{ç}{{\c{c}}}1
{~}{{ }}1
}


\definecolor{codegreen}{rgb}{0,0.6,0}
\definecolor{codegray}{rgb}{0.5,0.5,0.5}
\definecolor{codepurple}{rgb}{0.58,0,0.82}
\definecolor{backcolour}{rgb}{0.95,0.95,0.92}

\lstdefinestyle{mystyle}{
    backgroundcolor=\color{backcolour},   
    commentstyle=\color{codegreen},
    keywordstyle=\color{magenta},
    numberstyle=\tiny\color{codegray},
    stringstyle=\color{codepurple},
    basicstyle=\ttfamily\footnotesize,
    breakatwhitespace=false,         
    breaklines=true,                 
    captionpos=b,                    
    keepspaces=true,                 
    numbers=left,                    
xleftmargin=2em,
framexleftmargin=2em,            
    showspaces=false,                
    showstringspaces=false,
    showtabs=false,                  
    tabsize=2,
    upquote=true
}

\lstset{style=mystyle}


\lstset{style=mystyle}
\newcommand{\imgdir}{C:/laragon/www/newmc/assets/imgsvg/}
\newcommand{\imgsvgdir}{C:/laragon/www/newmc/assets/imgsvg/}

\definecolor{mcgris}{RGB}{220, 220, 220}% ancien~; pour compatibilité
\definecolor{mcbleu}{RGB}{52, 152, 219}
\definecolor{mcvert}{RGB}{125, 194, 70}
\definecolor{mcmauve}{RGB}{154, 0, 215}
\definecolor{mcorange}{RGB}{255, 96, 0}
\definecolor{mcturquoise}{RGB}{0, 153, 153}
\definecolor{mcrouge}{RGB}{255, 0, 0}
\definecolor{mclightvert}{RGB}{205, 234, 190}

\definecolor{gris}{RGB}{220, 220, 220}
\definecolor{bleu}{RGB}{52, 152, 219}
\definecolor{vert}{RGB}{125, 194, 70}
\definecolor{mauve}{RGB}{154, 0, 215}
\definecolor{orange}{RGB}{255, 96, 0}
\definecolor{turquoise}{RGB}{0, 153, 153}
\definecolor{rouge}{RGB}{255, 0, 0}
\definecolor{lightvert}{RGB}{205, 234, 190}
\setitemize[0]{label=\color{lightvert}  $\bullet$}

\pagestyle{fancy}
\renewcommand{\headrulewidth}{0.2pt}
\fancyhead[L]{maths-cours.fr}
\fancyhead[R]{\thepage}
\renewcommand{\footrulewidth}{0.2pt}
\fancyfoot[C]{}

\newcolumntype{C}{>{\centering\arraybackslash}X}
\newcolumntype{s}{>{\hsize=.35\hsize\arraybackslash}X}

\setlength{\parindent}{0pt}		 
\setlength{\parskip}{3mm}
\setlength{\headheight}{1cm}

\def\ebook{ebook}
\def\book{book}
\def\web{web}
\def\type{web}

\newcommand{\vect}[1]{\overrightarrow{\,\mathstrut#1\,}}

\def\Oij{$\left(\text{O}~;~\vect{\imath},~\vect{\jmath}\right)$}
\def\Oijk{$\left(\text{O}~;~\vect{\imath},~\vect{\jmath},~\vect{k}\right)$}
\def\Ouv{$\left(\text{O}~;~\vect{u},~\vect{v}\right)$}

\hypersetup{breaklinks=true, colorlinks = true, linkcolor = OliveGreen, urlcolor = OliveGreen, citecolor = OliveGreen, pdfauthor={Didier BONNEL - https://www.maths-cours.fr} } % supprime les bordures autour des liens

\renewcommand{\arg}[0]{\text{arg}}

\everymath{\displaystyle}

%================================================================================================================================
%
% Macros - Commandes
%
%================================================================================================================================

\newcommand\meta[2]{    			% Utilisé pour créer le post HTML.
	\def\titre{titre}
	\def\url{url}
	\def\arg{#1}
	\ifx\titre\arg
		\newcommand\maintitle{#2}
		\fancyhead[L]{#2}
		{\Large\sffamily \MakeUppercase{#2}}
		\vspace{1mm}\textcolor{mcvert}{\hrule}
	\fi 
	\ifx\url\arg
		\fancyfoot[L]{\href{https://www.maths-cours.fr#2}{\black \footnotesize{https://www.maths-cours.fr#2}}}
	\fi 
}


\newcommand\TitreC[1]{    		% Titre centré
     \needspace{3\baselineskip}
     \begin{center}\textbf{#1}\end{center}
}

\newcommand\newpar{    		% paragraphe
     \par
}

\newcommand\nosp {    		% commande vide (pas d'espace)
}
\newcommand{\id}[1]{} %ignore

\newcommand\boite[2]{				% Boite simple sans titre
	\vspace{5mm}
	\setlength{\fboxrule}{0.2mm}
	\setlength{\fboxsep}{5mm}	
	\fcolorbox{#1}{#1!3}{\makebox[\linewidth-2\fboxrule-2\fboxsep]{
  		\begin{minipage}[t]{\linewidth-2\fboxrule-4\fboxsep}\setlength{\parskip}{3mm}
  			 #2
  		\end{minipage}
	}}
	\vspace{5mm}
}

\newcommand\CBox[4]{				% Boites
	\vspace{5mm}
	\setlength{\fboxrule}{0.2mm}
	\setlength{\fboxsep}{5mm}
	
	\fcolorbox{#1}{#1!3}{\makebox[\linewidth-2\fboxrule-2\fboxsep]{
		\begin{minipage}[t]{1cm}\setlength{\parskip}{3mm}
	  		\textcolor{#1}{\LARGE{#2}}    
 	 	\end{minipage}  
  		\begin{minipage}[t]{\linewidth-2\fboxrule-4\fboxsep}\setlength{\parskip}{3mm}
			\raisebox{1.2mm}{\normalsize\sffamily{\textcolor{#1}{#3}}}						
  			 #4
  		\end{minipage}
	}}
	\vspace{5mm}
}

\newcommand\cadre[3]{				% Boites convertible html
	\par
	\vspace{2mm}
	\setlength{\fboxrule}{0.1mm}
	\setlength{\fboxsep}{5mm}
	\fcolorbox{#1}{white}{\makebox[\linewidth-2\fboxrule-2\fboxsep]{
  		\begin{minipage}[t]{\linewidth-2\fboxrule-4\fboxsep}\setlength{\parskip}{3mm}
			\raisebox{-2.5mm}{\sffamily \small{\textcolor{#1}{\MakeUppercase{#2}}}}		
			\par		
  			 #3
 	 		\end{minipage}
	}}
		\vspace{2mm}
	\par
}

\newcommand\bloc[3]{				% Boites convertible html sans bordure
     \needspace{2\baselineskip}
     {\sffamily \small{\textcolor{#1}{\MakeUppercase{#2}}}}    
		\par		
  			 #3
		\par
}

\newcommand\CHelp[1]{
     \CBox{Plum}{\faInfoCircle}{À RETENIR}{#1}
}

\newcommand\CUp[1]{
     \CBox{NavyBlue}{\faThumbsOUp}{EN PRATIQUE}{#1}
}

\newcommand\CInfo[1]{
     \CBox{Sepia}{\faArrowCircleRight}{REMARQUE}{#1}
}

\newcommand\CRedac[1]{
     \CBox{PineGreen}{\faEdit}{BIEN R\'EDIGER}{#1}
}

\newcommand\CError[1]{
     \CBox{Red}{\faExclamationTriangle}{ATTENTION}{#1}
}

\newcommand\TitreExo[2]{
\needspace{4\baselineskip}
 {\sffamily\large EXERCICE #1\ (\emph{#2 points})}
\vspace{5mm}
}

\newcommand\img[2]{
          \includegraphics[width=#2\paperwidth]{\imgdir#1}
}

\newcommand\imgsvg[2]{
       \begin{center}   \includegraphics[width=#2\paperwidth]{\imgsvgdir#1} \end{center}
}


\newcommand\Lien[2]{
     \href{#1}{#2 \tiny \faExternalLink}
}
\newcommand\mcLien[2]{
     \href{https~://www.maths-cours.fr/#1}{#2 \tiny \faExternalLink}
}

\newcommand{\euro}{\eurologo{}}

%================================================================================================================================
%
% Macros - Environement
%
%================================================================================================================================

\newenvironment{tex}{ %
}
{%
}

\newenvironment{indente}{ %
	\setlength\parindent{10mm}
}

{
	\setlength\parindent{0mm}
}

\newenvironment{corrige}{%
     \needspace{3\baselineskip}
     \medskip
     \textbf{\textsc{Corrigé}}
     \medskip
}
{
}

\newenvironment{extern}{%
     \begin{center}
     }
     {
     \end{center}
}

\NewEnviron{code}{%
	\par
     \boite{gray}{\texttt{%
     \BODY
     }}
     \par
}

\newenvironment{vbloc}{% boite sans cadre empeche saut de page
     \begin{minipage}[t]{\linewidth}
     }
     {
     \end{minipage}
}
\NewEnviron{h2}{%
    \needspace{3\baselineskip}
    \vspace{0.6cm}
	\noindent \MakeUppercase{\sffamily \large \BODY}
	\vspace{1mm}\textcolor{mcgris}{\hrule}\vspace{0.4cm}
	\par
}{}

\NewEnviron{h3}{%
    \needspace{3\baselineskip}
	\vspace{5mm}
	\textsc{\BODY}
	\par
}

\NewEnviron{margeneg}{ %
\begin{addmargin}[-1cm]{0cm}
\BODY
\end{addmargin}
}

\NewEnviron{html}{%
}

\begin{document}
\meta{url}{/exercices/calcul-litteral-somme-de-fractions/}
\meta{pid}{10737}
\meta{titre}{Calcul littéral : Somme de fractions}
\meta{type}{exercices}
%
Écrire chacune des expressions suivantes sous forme d'une seule fraction~:
\begin{enumerate}
     \item %
     $A=\dfrac{1}{x+1} - \dfrac{1}{x} \quad $(pour $x \neq -1$ et $x \neq 0$)
     \item %
     $B=\dfrac{1+x}{1-x}+\dfrac{1-x}{1+x} \quad $(pour $x \neq -1$ et $x \neq 1$)
     \item %
     $C=\dfrac{1}{x(x+1)}-\dfrac{1}{(x+1)(x+2)} \quad $(pour $x \neq -2$ et $x \neq -1$ et $x \neq 0$)
\end{enumerate}
\begin{corrige}
     Pour chaque expression~:
     \begin{itemize}
          \item %
          on réduit les fractions au même dénominateur~;
          \item %
          puis, on calcule la somme algébrique au numérateur.
     \end{itemize}
     \textbf{Remarque~:} Il n'est pas nécessaire de développer les produits au dénominateur (mais ce n'est pas non plus une erreur de le faire...). \\En effet, dans la plupart des cas, la forme factorisée est plus utile que la forme développée.
     \begin{enumerate}
          \item %
          $A=\dfrac{1}{x+1} - \dfrac{1}{x}$
          \par
          L'expression est définie pour $x \neq -1$ et $x \neq 0$.
          \par
          Le dénominateur commun est $x(x+1)$.
          \par
          $A=\dfrac{\color{red}{x}\times 1}{\color{red}{x}(x+1)} - \dfrac{1\times \color{red}{(x+1)}}{x \color{red}{(x+1)}}$\\
          $A=\dfrac{x}{x(x+1)} - \dfrac{x+1}{x(x+1)}$\\
          $A=\dfrac{x- (x+1)}{x(x+1)}$
          \par
          \textbf{Attention à la parenthèse~:} le signe \og - \fg{}  est situé devant la fraction~; il s'applique donc à l'ensemble du numérateur.
          \par
          $A=\dfrac{x- x-1}{x(x+1)}=\dfrac{-1}{x(x+1)}.$
          \par
          \item %
          $B=\dfrac{1+x}{1-x}+\dfrac{1-x}{1+x}  $
          \par
          B est définie pour $x \neq -1$ et $x \neq 1$.
          \par
          Le dénominateur commun est $(1-x)(1+x)$.
          \par
          $B=\dfrac{(1+x) \color{red}{(1+x)}}{(1-x)\color{red}{(1+x)}}$\nosp$+\dfrac{\color{red}{(1-x)}(1-x)}{\color{red}{(1-x)}(1+x)}  $\\
          $B=\dfrac{(1+x)^2}{(1-x)(1+x)}$\nosp$+\dfrac{(1-x)^2}{(1-x)(1+x)}  $\\
          $B=\dfrac{1+2x+x^2}{(1-x)(1+x)}$\nosp$+\dfrac{1-2x+x^2}{(1-x)(1+x)}  $\\
          $B=\dfrac{1+2x+x^2+1-2x+x^2}{(1-x)(1+x)}$\\
          $B=\dfrac{2x^2+2}{(1-x)(1+x)}.$
          \item %
          $C=\dfrac{1}{x(x+1)}-\dfrac{1}{(x+1)(x+2)}  $
          \par
          La fraction est définie si et seulement si $x \neq -2$ et $x \neq -1$ et $x \neq 0$.
          \par
          Un dénominateur commun est $x(x+1)(x+2)$.\\
          \textbf{Remarque~:} Prendre $x(x+1)(x+1)(x+2)$ (c'est à dire $x(x+1)^2(x+2)$) comme dénominateur commun ne serait pas une faute mais mènerait à des calculs inutilement compliqués avec une fraction à simplifier à la fin. \\Essayez toujours de trouver un dénominateur commun aussi simple que possible~!
          \par
          $C=\dfrac{1\times \color{red}{(x+2)}}{x(x+1)\color{red}{(x+2)}}$\nosp$ -\dfrac{\color{red}{x}\times 1}{\color{red}{x}(x+1)(x+2)}  $\\
          $C=\dfrac{x+2}{x(x+1)(x+2)}$\nosp$ -\dfrac{x}{x(x+1)(x+2)}  $\\
          $C=\dfrac{x+2-x}{x(x+1)(x+2)}$ \\
          $C=\dfrac{2}{x(x+1)(x+2)}$ \\
     \end{enumerate}
\end{corrige}

\end{document}
µ
\documentclass[a4paper]{article}

%================================================================================================================================
%
% Packages
%
%================================================================================================================================

\usepackage[T1]{fontenc} 	% pour caractères accentués
\usepackage[utf8]{inputenc}  % encodage utf8
\usepackage[french]{babel}	% langue : français
\usepackage{fourier}			% caractères plus lisibles
\usepackage[dvipsnames]{xcolor} % couleurs
\usepackage{fancyhdr}		% réglage header footer
\usepackage{needspace}		% empêcher sauts de page mal placés
\usepackage{graphicx}		% pour inclure des graphiques
\usepackage{enumitem,cprotect}		% personnalise les listes d'items (nécessaire pour ol, al ...)
\usepackage{hyperref}		% Liens hypertexte
\usepackage{pstricks,pst-all,pst-node,pstricks-add,pst-math,pst-plot,pst-tree,pst-eucl} % pstricks
\usepackage[a4paper,includeheadfoot,top=2cm,left=3cm, bottom=2cm,right=3cm]{geometry} % marges etc.
\usepackage{comment}			% commentaires multilignes
\usepackage{amsmath,environ} % maths (matrices, etc.)
\usepackage{amssymb,makeidx}
\usepackage{bm}				% bold maths
\usepackage{tabularx}		% tableaux
\usepackage{colortbl}		% tableaux en couleur
\usepackage{fontawesome}		% Fontawesome
\usepackage{environ}			% environment with command
\usepackage{fp}				% calculs pour ps-tricks
\usepackage{multido}			% pour ps tricks
\usepackage[np]{numprint}	% formattage nombre
\usepackage{tikz,tkz-tab} 			% package principal TikZ
\usepackage{pgfplots}   % axes
\usepackage{mathrsfs}    % cursives
\usepackage{calc}			% calcul taille boites
\usepackage[scaled=0.875]{helvet} % font sans serif
\usepackage{svg} % svg
\usepackage{scrextend} % local margin
\usepackage{scratch} %scratch
\usepackage{multicol} % colonnes
%\usepackage{infix-RPN,pst-func} % formule en notation polanaise inversée
\usepackage{listings}

%================================================================================================================================
%
% Réglages de base
%
%================================================================================================================================

\lstset{
language=Python,   % R code
literate=
{á}{{\'a}}1
{à}{{\`a}}1
{ã}{{\~a}}1
{é}{{\'e}}1
{è}{{\`e}}1
{ê}{{\^e}}1
{í}{{\'i}}1
{ó}{{\'o}}1
{õ}{{\~o}}1
{ú}{{\'u}}1
{ü}{{\"u}}1
{ç}{{\c{c}}}1
{~}{{ }}1
}


\definecolor{codegreen}{rgb}{0,0.6,0}
\definecolor{codegray}{rgb}{0.5,0.5,0.5}
\definecolor{codepurple}{rgb}{0.58,0,0.82}
\definecolor{backcolour}{rgb}{0.95,0.95,0.92}

\lstdefinestyle{mystyle}{
    backgroundcolor=\color{backcolour},   
    commentstyle=\color{codegreen},
    keywordstyle=\color{magenta},
    numberstyle=\tiny\color{codegray},
    stringstyle=\color{codepurple},
    basicstyle=\ttfamily\footnotesize,
    breakatwhitespace=false,         
    breaklines=true,                 
    captionpos=b,                    
    keepspaces=true,                 
    numbers=left,                    
xleftmargin=2em,
framexleftmargin=2em,            
    showspaces=false,                
    showstringspaces=false,
    showtabs=false,                  
    tabsize=2,
    upquote=true
}

\lstset{style=mystyle}


\lstset{style=mystyle}
\newcommand{\imgdir}{C:/laragon/www/newmc/assets/imgsvg/}
\newcommand{\imgsvgdir}{C:/laragon/www/newmc/assets/imgsvg/}

\definecolor{mcgris}{RGB}{220, 220, 220}% ancien~; pour compatibilité
\definecolor{mcbleu}{RGB}{52, 152, 219}
\definecolor{mcvert}{RGB}{125, 194, 70}
\definecolor{mcmauve}{RGB}{154, 0, 215}
\definecolor{mcorange}{RGB}{255, 96, 0}
\definecolor{mcturquoise}{RGB}{0, 153, 153}
\definecolor{mcrouge}{RGB}{255, 0, 0}
\definecolor{mclightvert}{RGB}{205, 234, 190}

\definecolor{gris}{RGB}{220, 220, 220}
\definecolor{bleu}{RGB}{52, 152, 219}
\definecolor{vert}{RGB}{125, 194, 70}
\definecolor{mauve}{RGB}{154, 0, 215}
\definecolor{orange}{RGB}{255, 96, 0}
\definecolor{turquoise}{RGB}{0, 153, 153}
\definecolor{rouge}{RGB}{255, 0, 0}
\definecolor{lightvert}{RGB}{205, 234, 190}
\setitemize[0]{label=\color{lightvert}  $\bullet$}

\pagestyle{fancy}
\renewcommand{\headrulewidth}{0.2pt}
\fancyhead[L]{maths-cours.fr}
\fancyhead[R]{\thepage}
\renewcommand{\footrulewidth}{0.2pt}
\fancyfoot[C]{}

\newcolumntype{C}{>{\centering\arraybackslash}X}
\newcolumntype{s}{>{\hsize=.35\hsize\arraybackslash}X}

\setlength{\parindent}{0pt}		 
\setlength{\parskip}{3mm}
\setlength{\headheight}{1cm}

\def\ebook{ebook}
\def\book{book}
\def\web{web}
\def\type{web}

\newcommand{\vect}[1]{\overrightarrow{\,\mathstrut#1\,}}

\def\Oij{$\left(\text{O}~;~\vect{\imath},~\vect{\jmath}\right)$}
\def\Oijk{$\left(\text{O}~;~\vect{\imath},~\vect{\jmath},~\vect{k}\right)$}
\def\Ouv{$\left(\text{O}~;~\vect{u},~\vect{v}\right)$}

\hypersetup{breaklinks=true, colorlinks = true, linkcolor = OliveGreen, urlcolor = OliveGreen, citecolor = OliveGreen, pdfauthor={Didier BONNEL - https://www.maths-cours.fr} } % supprime les bordures autour des liens

\renewcommand{\arg}[0]{\text{arg}}

\everymath{\displaystyle}

%================================================================================================================================
%
% Macros - Commandes
%
%================================================================================================================================

\newcommand\meta[2]{    			% Utilisé pour créer le post HTML.
	\def\titre{titre}
	\def\url{url}
	\def\arg{#1}
	\ifx\titre\arg
		\newcommand\maintitle{#2}
		\fancyhead[L]{#2}
		{\Large\sffamily \MakeUppercase{#2}}
		\vspace{1mm}\textcolor{mcvert}{\hrule}
	\fi 
	\ifx\url\arg
		\fancyfoot[L]{\href{https://www.maths-cours.fr#2}{\black \footnotesize{https://www.maths-cours.fr#2}}}
	\fi 
}


\newcommand\TitreC[1]{    		% Titre centré
     \needspace{3\baselineskip}
     \begin{center}\textbf{#1}\end{center}
}

\newcommand\newpar{    		% paragraphe
     \par
}

\newcommand\nosp {    		% commande vide (pas d'espace)
}
\newcommand{\id}[1]{} %ignore

\newcommand\boite[2]{				% Boite simple sans titre
	\vspace{5mm}
	\setlength{\fboxrule}{0.2mm}
	\setlength{\fboxsep}{5mm}	
	\fcolorbox{#1}{#1!3}{\makebox[\linewidth-2\fboxrule-2\fboxsep]{
  		\begin{minipage}[t]{\linewidth-2\fboxrule-4\fboxsep}\setlength{\parskip}{3mm}
  			 #2
  		\end{minipage}
	}}
	\vspace{5mm}
}

\newcommand\CBox[4]{				% Boites
	\vspace{5mm}
	\setlength{\fboxrule}{0.2mm}
	\setlength{\fboxsep}{5mm}
	
	\fcolorbox{#1}{#1!3}{\makebox[\linewidth-2\fboxrule-2\fboxsep]{
		\begin{minipage}[t]{1cm}\setlength{\parskip}{3mm}
	  		\textcolor{#1}{\LARGE{#2}}    
 	 	\end{minipage}  
  		\begin{minipage}[t]{\linewidth-2\fboxrule-4\fboxsep}\setlength{\parskip}{3mm}
			\raisebox{1.2mm}{\normalsize\sffamily{\textcolor{#1}{#3}}}						
  			 #4
  		\end{minipage}
	}}
	\vspace{5mm}
}

\newcommand\cadre[3]{				% Boites convertible html
	\par
	\vspace{2mm}
	\setlength{\fboxrule}{0.1mm}
	\setlength{\fboxsep}{5mm}
	\fcolorbox{#1}{white}{\makebox[\linewidth-2\fboxrule-2\fboxsep]{
  		\begin{minipage}[t]{\linewidth-2\fboxrule-4\fboxsep}\setlength{\parskip}{3mm}
			\raisebox{-2.5mm}{\sffamily \small{\textcolor{#1}{\MakeUppercase{#2}}}}		
			\par		
  			 #3
 	 		\end{minipage}
	}}
		\vspace{2mm}
	\par
}

\newcommand\bloc[3]{				% Boites convertible html sans bordure
     \needspace{2\baselineskip}
     {\sffamily \small{\textcolor{#1}{\MakeUppercase{#2}}}}    
		\par		
  			 #3
		\par
}

\newcommand\CHelp[1]{
     \CBox{Plum}{\faInfoCircle}{À RETENIR}{#1}
}

\newcommand\CUp[1]{
     \CBox{NavyBlue}{\faThumbsOUp}{EN PRATIQUE}{#1}
}

\newcommand\CInfo[1]{
     \CBox{Sepia}{\faArrowCircleRight}{REMARQUE}{#1}
}

\newcommand\CRedac[1]{
     \CBox{PineGreen}{\faEdit}{BIEN R\'EDIGER}{#1}
}

\newcommand\CError[1]{
     \CBox{Red}{\faExclamationTriangle}{ATTENTION}{#1}
}

\newcommand\TitreExo[2]{
\needspace{4\baselineskip}
 {\sffamily\large EXERCICE #1\ (\emph{#2 points})}
\vspace{5mm}
}

\newcommand\img[2]{
          \includegraphics[width=#2\paperwidth]{\imgdir#1}
}

\newcommand\imgsvg[2]{
       \begin{center}   \includegraphics[width=#2\paperwidth]{\imgsvgdir#1} \end{center}
}


\newcommand\Lien[2]{
     \href{#1}{#2 \tiny \faExternalLink}
}
\newcommand\mcLien[2]{
     \href{https~://www.maths-cours.fr/#1}{#2 \tiny \faExternalLink}
}

\newcommand{\euro}{\eurologo{}}

%================================================================================================================================
%
% Macros - Environement
%
%================================================================================================================================

\newenvironment{tex}{ %
}
{%
}

\newenvironment{indente}{ %
	\setlength\parindent{10mm}
}

{
	\setlength\parindent{0mm}
}

\newenvironment{corrige}{%
     \needspace{3\baselineskip}
     \medskip
     \textbf{\textsc{Corrigé}}
     \medskip
}
{
}

\newenvironment{extern}{%
     \begin{center}
     }
     {
     \end{center}
}

\NewEnviron{code}{%
	\par
     \boite{gray}{\texttt{%
     \BODY
     }}
     \par
}

\newenvironment{vbloc}{% boite sans cadre empeche saut de page
     \begin{minipage}[t]{\linewidth}
     }
     {
     \end{minipage}
}
\NewEnviron{h2}{%
    \needspace{3\baselineskip}
    \vspace{0.6cm}
	\noindent \MakeUppercase{\sffamily \large \BODY}
	\vspace{1mm}\textcolor{mcgris}{\hrule}\vspace{0.4cm}
	\par
}{}

\NewEnviron{h3}{%
    \needspace{3\baselineskip}
	\vspace{5mm}
	\textsc{\BODY}
	\par
}

\NewEnviron{margeneg}{ %
\begin{addmargin}[-1cm]{0cm}
\BODY
\end{addmargin}
}

\NewEnviron{html}{%
}

\begin{document}
\meta{url}{/exercices/calcul-litteral-simplification-de-fractions/}
\meta{pid}{10740}
\meta{titre}{Calcul littéral~: Simplification de fractions}
\meta{type}{exercices}
%
Pour chacune des expressions suivantes~:
\begin{enumerate}[label=\alph*.]
     \item %
     préciser son ensemble de définition~;
     \item %
     simplifier la fraction~;
     \item %
     donner l'ensemble de définition de la fraction simplifiée.
\end{enumerate}
\par
\begin{enumerate}
     \item %
     $A=\dfrac{x^2-4}{(x-1)(x+2)}$
     \item %
     $B=\dfrac{(x+1)(x-5)+(x+1)^2}{x^2+2x+1}$
     \item %
     $C=\dfrac{x^4-1}{(x-1)(2x+1)}$
\end{enumerate}
\begin{corrige}
     Pour la question \textbf{b.} de ces exercices, la méthode consiste à factoriser le numérateur et le dénominateur et à simplifier par le(s) facteur(s) commun(s) au numérateur et au dénominateur.
     \begin{enumerate}
          \item %
          $A=\dfrac{x^2-4}{(x-1)(x+2)}$
          \begin{enumerate}[label=\alph*.]
               \item %
               La fraction $A$ est définie si et seulement si son dénominateur est non nul.
               \par
               Or~:
               \par
               $(x-1)(x+2)=0 \Leftrightarrow x-1 = 0 \text{ ou } x+2=0$
               \par
               $\phantom{(x-1)(x+2)=0} \Leftrightarrow x = 1 \text{ ou } x=-2.$
               \par
               Donc l'ensemble de définition de $A$ est $\mathscr{D}_A=\mathbb{R} \backslash \{-2~;~1\}.$
               \item %
               On factorise le numérateur à l'aide de l'identité remarquable $a^2-b^2=(a-b)(a+b)$~:
               \par
               $x^2-4=(x-2)(x+2)$
               \par
               Par conséquent pour tout réel $x \in \mathscr{D}_A~:$
               \par
               $A=\dfrac{(x-2)(x+2)}{(x-1)(x+2)}$\nosp$= \dfrac{x-2}{x-1}$
               \item %
               La fraction simplifiée est définie si et seulement si $x \neq 1$ donc sur $\mathbb{R} \backslash \{~1\}.$
          \end{enumerate}
          %
          \item %
          $B=\dfrac{(x+1)(x-5)+(x+1)^2}{x^2+2x+1}$
          \begin{enumerate}[label=\alph*.]
               \item %
               Le dénominateur se factorise grâce à l'identité remarquable $a^2+2ab+b^2=(a+b)^2~:$
               \par
               $x^2+2x+1=(x+1)^2$
               \par
               Le dénominateur est différent de zéro si et seulement si $x \neq -1$ donc $\mathscr{D}_B=\mathbb{R} \backslash \{-1\}.$
               \item %
               On peut mettre $(x+1)$ en facteur au numérateur~:
               \par
               $(x+1)(x-5)+(x+1)^2=(x+1)\left[(x-5)+(x+1)\right]$\\
               $\phantom{(x+1)(x-5)+(x+1)^2}=(x+1)(2x-4).$
               \par
               Par conséquent, pour tout réel $x \in \mathscr{D}_B$~:
               \par
               $B=\dfrac{(x+1)(x-5)+(x+1)^2}{(x+1)^2}$
               \par
               $\phantom{B}=\dfrac{(x+1)(2x-4)}{(x+1)^2}$
               \par
               $\phantom{B}=\dfrac{2x-4}{x+1}.$
               \item %
               L'ensemble de définition de la fraction simplifiée est encore $\mathbb{R} \backslash \{-1\}.$
          \end{enumerate}
          %
          \item %
          $C=\dfrac{x^4-1}{(x-1)(2x+1)}$
          \begin{enumerate}[label=\alph*.]
               \item %
               Le dénominateur est non nul si et seulement si $x \neq 1$ et $x \neq -\dfrac{1}{2}.$. Donc $\mathscr{D}_C=\mathbb{R} \backslash \left\{-\dfrac{1}{2}~;~1\right\}.$
               \item %
               $x^4-1$ se factorise de la manière suivante ~:
               \par
               $x^4-1=(x^2)^2-1^2=(x^2-1)(x^2+1)$\nosp$=(x-1)(x+1)(x^2+1).$
               \par
               \textbf{Remarque~: } $x^2+1$ ne peut pas être factorisé dans $\mathbb{R}.$
               \par
               On en déduit que~:
               \par
               $C=\dfrac{(x-1)(x+1)(x^2+1)}{(x-1)(2x+1)}$\nosp$=\dfrac{(x+1)(x^2+1)}{2x+1}$
               \item %
               La fraction simplifiée est définie sur l'ensemble $\mathbb{R} \backslash \left\{ -\dfrac{1}{2} \right\}.$
          \end{enumerate}
     \end{enumerate}
\end{corrige}

\end{document}
µ
\documentclass[a4paper]{article}

%================================================================================================================================
%
% Packages
%
%================================================================================================================================

\usepackage[T1]{fontenc} 	% pour caractères accentués
\usepackage[utf8]{inputenc}  % encodage utf8
\usepackage[french]{babel}	% langue : français
\usepackage{fourier}			% caractères plus lisibles
\usepackage[dvipsnames]{xcolor} % couleurs
\usepackage{fancyhdr}		% réglage header footer
\usepackage{needspace}		% empêcher sauts de page mal placés
\usepackage{graphicx}		% pour inclure des graphiques
\usepackage{enumitem,cprotect}		% personnalise les listes d'items (nécessaire pour ol, al ...)
\usepackage{hyperref}		% Liens hypertexte
\usepackage{pstricks,pst-all,pst-node,pstricks-add,pst-math,pst-plot,pst-tree,pst-eucl} % pstricks
\usepackage[a4paper,includeheadfoot,top=2cm,left=3cm, bottom=2cm,right=3cm]{geometry} % marges etc.
\usepackage{comment}			% commentaires multilignes
\usepackage{amsmath,environ} % maths (matrices, etc.)
\usepackage{amssymb,makeidx}
\usepackage{bm}				% bold maths
\usepackage{tabularx}		% tableaux
\usepackage{colortbl}		% tableaux en couleur
\usepackage{fontawesome}		% Fontawesome
\usepackage{environ}			% environment with command
\usepackage{fp}				% calculs pour ps-tricks
\usepackage{multido}			% pour ps tricks
\usepackage[np]{numprint}	% formattage nombre
\usepackage{tikz,tkz-tab} 			% package principal TikZ
\usepackage{pgfplots}   % axes
\usepackage{mathrsfs}    % cursives
\usepackage{calc}			% calcul taille boites
\usepackage[scaled=0.875]{helvet} % font sans serif
\usepackage{svg} % svg
\usepackage{scrextend} % local margin
\usepackage{scratch} %scratch
\usepackage{multicol} % colonnes
%\usepackage{infix-RPN,pst-func} % formule en notation polanaise inversée
\usepackage{listings}

%================================================================================================================================
%
% Réglages de base
%
%================================================================================================================================

\lstset{
language=Python,   % R code
literate=
{á}{{\'a}}1
{à}{{\`a}}1
{ã}{{\~a}}1
{é}{{\'e}}1
{è}{{\`e}}1
{ê}{{\^e}}1
{í}{{\'i}}1
{ó}{{\'o}}1
{õ}{{\~o}}1
{ú}{{\'u}}1
{ü}{{\"u}}1
{ç}{{\c{c}}}1
{~}{{ }}1
}


\definecolor{codegreen}{rgb}{0,0.6,0}
\definecolor{codegray}{rgb}{0.5,0.5,0.5}
\definecolor{codepurple}{rgb}{0.58,0,0.82}
\definecolor{backcolour}{rgb}{0.95,0.95,0.92}

\lstdefinestyle{mystyle}{
    backgroundcolor=\color{backcolour},   
    commentstyle=\color{codegreen},
    keywordstyle=\color{magenta},
    numberstyle=\tiny\color{codegray},
    stringstyle=\color{codepurple},
    basicstyle=\ttfamily\footnotesize,
    breakatwhitespace=false,         
    breaklines=true,                 
    captionpos=b,                    
    keepspaces=true,                 
    numbers=left,                    
xleftmargin=2em,
framexleftmargin=2em,            
    showspaces=false,                
    showstringspaces=false,
    showtabs=false,                  
    tabsize=2,
    upquote=true
}

\lstset{style=mystyle}


\lstset{style=mystyle}
\newcommand{\imgdir}{C:/laragon/www/newmc/assets/imgsvg/}
\newcommand{\imgsvgdir}{C:/laragon/www/newmc/assets/imgsvg/}

\definecolor{mcgris}{RGB}{220, 220, 220}% ancien~; pour compatibilité
\definecolor{mcbleu}{RGB}{52, 152, 219}
\definecolor{mcvert}{RGB}{125, 194, 70}
\definecolor{mcmauve}{RGB}{154, 0, 215}
\definecolor{mcorange}{RGB}{255, 96, 0}
\definecolor{mcturquoise}{RGB}{0, 153, 153}
\definecolor{mcrouge}{RGB}{255, 0, 0}
\definecolor{mclightvert}{RGB}{205, 234, 190}

\definecolor{gris}{RGB}{220, 220, 220}
\definecolor{bleu}{RGB}{52, 152, 219}
\definecolor{vert}{RGB}{125, 194, 70}
\definecolor{mauve}{RGB}{154, 0, 215}
\definecolor{orange}{RGB}{255, 96, 0}
\definecolor{turquoise}{RGB}{0, 153, 153}
\definecolor{rouge}{RGB}{255, 0, 0}
\definecolor{lightvert}{RGB}{205, 234, 190}
\setitemize[0]{label=\color{lightvert}  $\bullet$}

\pagestyle{fancy}
\renewcommand{\headrulewidth}{0.2pt}
\fancyhead[L]{maths-cours.fr}
\fancyhead[R]{\thepage}
\renewcommand{\footrulewidth}{0.2pt}
\fancyfoot[C]{}

\newcolumntype{C}{>{\centering\arraybackslash}X}
\newcolumntype{s}{>{\hsize=.35\hsize\arraybackslash}X}

\setlength{\parindent}{0pt}		 
\setlength{\parskip}{3mm}
\setlength{\headheight}{1cm}

\def\ebook{ebook}
\def\book{book}
\def\web{web}
\def\type{web}

\newcommand{\vect}[1]{\overrightarrow{\,\mathstrut#1\,}}

\def\Oij{$\left(\text{O}~;~\vect{\imath},~\vect{\jmath}\right)$}
\def\Oijk{$\left(\text{O}~;~\vect{\imath},~\vect{\jmath},~\vect{k}\right)$}
\def\Ouv{$\left(\text{O}~;~\vect{u},~\vect{v}\right)$}

\hypersetup{breaklinks=true, colorlinks = true, linkcolor = OliveGreen, urlcolor = OliveGreen, citecolor = OliveGreen, pdfauthor={Didier BONNEL - https://www.maths-cours.fr} } % supprime les bordures autour des liens

\renewcommand{\arg}[0]{\text{arg}}

\everymath{\displaystyle}

%================================================================================================================================
%
% Macros - Commandes
%
%================================================================================================================================

\newcommand\meta[2]{    			% Utilisé pour créer le post HTML.
	\def\titre{titre}
	\def\url{url}
	\def\arg{#1}
	\ifx\titre\arg
		\newcommand\maintitle{#2}
		\fancyhead[L]{#2}
		{\Large\sffamily \MakeUppercase{#2}}
		\vspace{1mm}\textcolor{mcvert}{\hrule}
	\fi 
	\ifx\url\arg
		\fancyfoot[L]{\href{https://www.maths-cours.fr#2}{\black \footnotesize{https://www.maths-cours.fr#2}}}
	\fi 
}


\newcommand\TitreC[1]{    		% Titre centré
     \needspace{3\baselineskip}
     \begin{center}\textbf{#1}\end{center}
}

\newcommand\newpar{    		% paragraphe
     \par
}

\newcommand\nosp {    		% commande vide (pas d'espace)
}
\newcommand{\id}[1]{} %ignore

\newcommand\boite[2]{				% Boite simple sans titre
	\vspace{5mm}
	\setlength{\fboxrule}{0.2mm}
	\setlength{\fboxsep}{5mm}	
	\fcolorbox{#1}{#1!3}{\makebox[\linewidth-2\fboxrule-2\fboxsep]{
  		\begin{minipage}[t]{\linewidth-2\fboxrule-4\fboxsep}\setlength{\parskip}{3mm}
  			 #2
  		\end{minipage}
	}}
	\vspace{5mm}
}

\newcommand\CBox[4]{				% Boites
	\vspace{5mm}
	\setlength{\fboxrule}{0.2mm}
	\setlength{\fboxsep}{5mm}
	
	\fcolorbox{#1}{#1!3}{\makebox[\linewidth-2\fboxrule-2\fboxsep]{
		\begin{minipage}[t]{1cm}\setlength{\parskip}{3mm}
	  		\textcolor{#1}{\LARGE{#2}}    
 	 	\end{minipage}  
  		\begin{minipage}[t]{\linewidth-2\fboxrule-4\fboxsep}\setlength{\parskip}{3mm}
			\raisebox{1.2mm}{\normalsize\sffamily{\textcolor{#1}{#3}}}						
  			 #4
  		\end{minipage}
	}}
	\vspace{5mm}
}

\newcommand\cadre[3]{				% Boites convertible html
	\par
	\vspace{2mm}
	\setlength{\fboxrule}{0.1mm}
	\setlength{\fboxsep}{5mm}
	\fcolorbox{#1}{white}{\makebox[\linewidth-2\fboxrule-2\fboxsep]{
  		\begin{minipage}[t]{\linewidth-2\fboxrule-4\fboxsep}\setlength{\parskip}{3mm}
			\raisebox{-2.5mm}{\sffamily \small{\textcolor{#1}{\MakeUppercase{#2}}}}		
			\par		
  			 #3
 	 		\end{minipage}
	}}
		\vspace{2mm}
	\par
}

\newcommand\bloc[3]{				% Boites convertible html sans bordure
     \needspace{2\baselineskip}
     {\sffamily \small{\textcolor{#1}{\MakeUppercase{#2}}}}    
		\par		
  			 #3
		\par
}

\newcommand\CHelp[1]{
     \CBox{Plum}{\faInfoCircle}{À RETENIR}{#1}
}

\newcommand\CUp[1]{
     \CBox{NavyBlue}{\faThumbsOUp}{EN PRATIQUE}{#1}
}

\newcommand\CInfo[1]{
     \CBox{Sepia}{\faArrowCircleRight}{REMARQUE}{#1}
}

\newcommand\CRedac[1]{
     \CBox{PineGreen}{\faEdit}{BIEN R\'EDIGER}{#1}
}

\newcommand\CError[1]{
     \CBox{Red}{\faExclamationTriangle}{ATTENTION}{#1}
}

\newcommand\TitreExo[2]{
\needspace{4\baselineskip}
 {\sffamily\large EXERCICE #1\ (\emph{#2 points})}
\vspace{5mm}
}

\newcommand\img[2]{
          \includegraphics[width=#2\paperwidth]{\imgdir#1}
}

\newcommand\imgsvg[2]{
       \begin{center}   \includegraphics[width=#2\paperwidth]{\imgsvgdir#1} \end{center}
}


\newcommand\Lien[2]{
     \href{#1}{#2 \tiny \faExternalLink}
}
\newcommand\mcLien[2]{
     \href{https~://www.maths-cours.fr/#1}{#2 \tiny \faExternalLink}
}

\newcommand{\euro}{\eurologo{}}

%================================================================================================================================
%
% Macros - Environement
%
%================================================================================================================================

\newenvironment{tex}{ %
}
{%
}

\newenvironment{indente}{ %
	\setlength\parindent{10mm}
}

{
	\setlength\parindent{0mm}
}

\newenvironment{corrige}{%
     \needspace{3\baselineskip}
     \medskip
     \textbf{\textsc{Corrigé}}
     \medskip
}
{
}

\newenvironment{extern}{%
     \begin{center}
     }
     {
     \end{center}
}

\NewEnviron{code}{%
	\par
     \boite{gray}{\texttt{%
     \BODY
     }}
     \par
}

\newenvironment{vbloc}{% boite sans cadre empeche saut de page
     \begin{minipage}[t]{\linewidth}
     }
     {
     \end{minipage}
}
\NewEnviron{h2}{%
    \needspace{3\baselineskip}
    \vspace{0.6cm}
	\noindent \MakeUppercase{\sffamily \large \BODY}
	\vspace{1mm}\textcolor{mcgris}{\hrule}\vspace{0.4cm}
	\par
}{}

\NewEnviron{h3}{%
    \needspace{3\baselineskip}
	\vspace{5mm}
	\textsc{\BODY}
	\par
}

\NewEnviron{margeneg}{ %
\begin{addmargin}[-1cm]{0cm}
\BODY
\end{addmargin}
}

\NewEnviron{html}{%
}

\begin{document}
\meta{url}{/exercices/inequation-produit-tableau-de-signes/}
\meta{pid}{10768}
\meta{titre}{Inéquation - Produit - Tableau de signes}
\meta{type}{exercices}
%
Résoudre, dans $\mathbb{R}$, l'inéquation~:
\begin{center}
     $(x-3)(4-3x) \geqslant 0$
\end{center}
\begin{corrige}
     \begin{itemize}
          \item %
          $x-3$ s'annule pour $x=3.$
          \par
          $4-3x$ s'annule si et seulement si~:
          \par
          $4-3x=0 \Leftrightarrow -3x=-4$\\
          $\phantom{4-3x=0} \Leftrightarrow x=\dfrac{-4}{-3}$\\
          $\phantom{4-3x=0} \Leftrightarrow x=\dfrac{4}{3}$\\
          \par
          \item %
          Par ailleurs, $x-3$ est positif si et seulement si~:
          \par
          $x-3 > 0 \Leftrightarrow x > 3$
          \par
          et  $4-3x$  est positif si et seulement si~:
          \par
          $4-3x>0 \Leftrightarrow -3x>-4$\\
          $\phantom{4-3x>0} \Leftrightarrow x<\dfrac{-4}{-3}$\\
          $\phantom{4-3x=0} \Leftrightarrow x<\dfrac{4}{3}$\\
          \par
     \end{itemize}
     On obtient alors le tableau de signes suivant~:
     \par
     \begin{center}
          \begin{extern}%width="500" alt="Exercice tableau de signes d'un produit"
               \resizebox{11cm}{!}{
                    \begin{tikzpicture}[scale=0.875]
                         % Styles
                         \tikzstyle{cadre}=[thin]
                         \tikzstyle{fleche}=[->,>=latex,thin]
                         \tikzstyle{nondefini}=[lightgray]
                         % Dimensions Modifiables
                         \def\Lrg{1.5}
                         \def\HtX{1.2}
                         \def\HtY{0.5}
                         % Dimensions Calculées
                         \def\lignex{-0.5*\HtX}
                         \def\lignea{-1.5*\HtX}
                         \def\ligneb{-2.5*\HtX}
                         \def\lignec{-3.5*\HtX}
                         \def\separateur{-0.5*\Lrg}
                         % Largeur du tableau
                         \def\gauche{-3.1*\Lrg}
                         \def\droite{6.5*\Lrg}
                         % Hauteur du tableau
                         \def\haut{0.5*\HtX}
                         \def\bas{-2.5*\HtX-2*\HtY}
                         % Pointillés
                         \draw[gray] (2*\Lrg,\lignex) -- (2*\Lrg,\lignec);
                         \draw[gray] (4*\Lrg,\lignex) -- (4*\Lrg,\lignec);
                         % Ligne de l'abscisse : x
                         \node at (-1.8*\Lrg,0) {$x$};
                         \node at (0*\Lrg,0) {$-\infty$};
                         \node at (2*\Lrg,0) {$\dfrac{4}{3}$};
                         \node at (4*\Lrg,0) {$3$};
                         \node at (6*\Lrg,0) {$+\infty$};
                         % Ligne a
                         \node at (-1.8*\Lrg,-1*\HtX) {$x-3$};
                         \node at (0*\Lrg,-1*\HtX) {$ $};
                         \node at (1*\Lrg,-1*\HtX) {$-$};
                         \node at (2*\Lrg,-1*\HtX) {$ $};
                         \node at (3*\Lrg,-1*\HtX) {$-$};
                         \node at (4*\Lrg,-1*\HtX) {$0$};
                         \node at (5*\Lrg,-1*\HtX) {$+$};
                         \node at (6*\Lrg,-1*\HtX) {$ $};
                         % Ligne b
                         \node at (-1.8*\Lrg,-2*\HtX) {$4-3x$};
                         \node at (2*\Lrg,-2*\HtX) {$ $};
                         \node at (1*\Lrg,-2*\HtX) {$+$};
                         \node at (2*\Lrg,-2*\HtX) {$0$};
                         \node at (3*\Lrg,-2*\HtX) {$-$};
                         \node at (4*\Lrg,-2*\HtX) {$ $};
                         \node at (5*\Lrg,-2*\HtX) {$-$};
                         \node at (6*\Lrg,-2*\HtX) {$ $};
                         % Ligne c
                         \node at (-1.8*\Lrg,-3*\HtX) {$(x-3)(4-3x)$};
                         \node at (0*\Lrg,-3*\HtX) {$ $};
                         \node at (1*\Lrg,-3*\HtX) {$-$};
                         \node at (2*\Lrg,-3*\HtX) {$0$};
                         \node at (3*\Lrg,-3*\HtX) {$+$};
                         \node at (4*\Lrg,-3*\HtX) {$0$};
                         \node at (5*\Lrg,-3*\HtX) {$-$};
                         \node at (6*\Lrg,-3*\HtX) {$ $};
                         % Encadrement
                         \draw[cadre] (\separateur,\haut) -- (\separateur, \lignec);
                         \draw[cadre] (\gauche,\haut) rectangle  (\droite, \lignec);
                         \draw[cadre] (\gauche,\lignex) -- (\droite,\lignex);
                         \draw[cadre] (\gauche,\lignea) -- (\droite,\lignea);
                         \draw[cadre] (\gauche,\ligneb) -- (\droite,\ligneb);
                    \end{tikzpicture}
               }
          \end{extern}
     \end{center}
     On recherche les valeurs de $x$ pour lesquelles $(x-3)(4-3x) \geqslant 0.$
     \par
     À partir du tableau, on obtient l'ensemble des solutions~:
     \begin{center}
          $ S=\left[ \dfrac{4}{3}~;~3 \right]$
     \end{center}
     L'intervalle est fermé car l'inégalité $\geqslant$ est large.
\end{corrige}

\end{document}
µ
\documentclass[a4paper]{article}

%================================================================================================================================
%
% Packages
%
%================================================================================================================================

\usepackage[T1]{fontenc} 	% pour caractères accentués
\usepackage[utf8]{inputenc}  % encodage utf8
\usepackage[french]{babel}	% langue : français
\usepackage{fourier}			% caractères plus lisibles
\usepackage[dvipsnames]{xcolor} % couleurs
\usepackage{fancyhdr}		% réglage header footer
\usepackage{needspace}		% empêcher sauts de page mal placés
\usepackage{graphicx}		% pour inclure des graphiques
\usepackage{enumitem,cprotect}		% personnalise les listes d'items (nécessaire pour ol, al ...)
\usepackage{hyperref}		% Liens hypertexte
\usepackage{pstricks,pst-all,pst-node,pstricks-add,pst-math,pst-plot,pst-tree,pst-eucl} % pstricks
\usepackage[a4paper,includeheadfoot,top=2cm,left=3cm, bottom=2cm,right=3cm]{geometry} % marges etc.
\usepackage{comment}			% commentaires multilignes
\usepackage{amsmath,environ} % maths (matrices, etc.)
\usepackage{amssymb,makeidx}
\usepackage{bm}				% bold maths
\usepackage{tabularx}		% tableaux
\usepackage{colortbl}		% tableaux en couleur
\usepackage{fontawesome}		% Fontawesome
\usepackage{environ}			% environment with command
\usepackage{fp}				% calculs pour ps-tricks
\usepackage{multido}			% pour ps tricks
\usepackage[np]{numprint}	% formattage nombre
\usepackage{tikz,tkz-tab} 			% package principal TikZ
\usepackage{pgfplots}   % axes
\usepackage{mathrsfs}    % cursives
\usepackage{calc}			% calcul taille boites
\usepackage[scaled=0.875]{helvet} % font sans serif
\usepackage{svg} % svg
\usepackage{scrextend} % local margin
\usepackage{scratch} %scratch
\usepackage{multicol} % colonnes
%\usepackage{infix-RPN,pst-func} % formule en notation polanaise inversée
\usepackage{listings}

%================================================================================================================================
%
% Réglages de base
%
%================================================================================================================================

\lstset{
language=Python,   % R code
literate=
{á}{{\'a}}1
{à}{{\`a}}1
{ã}{{\~a}}1
{é}{{\'e}}1
{è}{{\`e}}1
{ê}{{\^e}}1
{í}{{\'i}}1
{ó}{{\'o}}1
{õ}{{\~o}}1
{ú}{{\'u}}1
{ü}{{\"u}}1
{ç}{{\c{c}}}1
{~}{{ }}1
}


\definecolor{codegreen}{rgb}{0,0.6,0}
\definecolor{codegray}{rgb}{0.5,0.5,0.5}
\definecolor{codepurple}{rgb}{0.58,0,0.82}
\definecolor{backcolour}{rgb}{0.95,0.95,0.92}

\lstdefinestyle{mystyle}{
    backgroundcolor=\color{backcolour},   
    commentstyle=\color{codegreen},
    keywordstyle=\color{magenta},
    numberstyle=\tiny\color{codegray},
    stringstyle=\color{codepurple},
    basicstyle=\ttfamily\footnotesize,
    breakatwhitespace=false,         
    breaklines=true,                 
    captionpos=b,                    
    keepspaces=true,                 
    numbers=left,                    
xleftmargin=2em,
framexleftmargin=2em,            
    showspaces=false,                
    showstringspaces=false,
    showtabs=false,                  
    tabsize=2,
    upquote=true
}

\lstset{style=mystyle}


\lstset{style=mystyle}
\newcommand{\imgdir}{C:/laragon/www/newmc/assets/imgsvg/}
\newcommand{\imgsvgdir}{C:/laragon/www/newmc/assets/imgsvg/}

\definecolor{mcgris}{RGB}{220, 220, 220}% ancien~; pour compatibilité
\definecolor{mcbleu}{RGB}{52, 152, 219}
\definecolor{mcvert}{RGB}{125, 194, 70}
\definecolor{mcmauve}{RGB}{154, 0, 215}
\definecolor{mcorange}{RGB}{255, 96, 0}
\definecolor{mcturquoise}{RGB}{0, 153, 153}
\definecolor{mcrouge}{RGB}{255, 0, 0}
\definecolor{mclightvert}{RGB}{205, 234, 190}

\definecolor{gris}{RGB}{220, 220, 220}
\definecolor{bleu}{RGB}{52, 152, 219}
\definecolor{vert}{RGB}{125, 194, 70}
\definecolor{mauve}{RGB}{154, 0, 215}
\definecolor{orange}{RGB}{255, 96, 0}
\definecolor{turquoise}{RGB}{0, 153, 153}
\definecolor{rouge}{RGB}{255, 0, 0}
\definecolor{lightvert}{RGB}{205, 234, 190}
\setitemize[0]{label=\color{lightvert}  $\bullet$}

\pagestyle{fancy}
\renewcommand{\headrulewidth}{0.2pt}
\fancyhead[L]{maths-cours.fr}
\fancyhead[R]{\thepage}
\renewcommand{\footrulewidth}{0.2pt}
\fancyfoot[C]{}

\newcolumntype{C}{>{\centering\arraybackslash}X}
\newcolumntype{s}{>{\hsize=.35\hsize\arraybackslash}X}

\setlength{\parindent}{0pt}		 
\setlength{\parskip}{3mm}
\setlength{\headheight}{1cm}

\def\ebook{ebook}
\def\book{book}
\def\web{web}
\def\type{web}

\newcommand{\vect}[1]{\overrightarrow{\,\mathstrut#1\,}}

\def\Oij{$\left(\text{O}~;~\vect{\imath},~\vect{\jmath}\right)$}
\def\Oijk{$\left(\text{O}~;~\vect{\imath},~\vect{\jmath},~\vect{k}\right)$}
\def\Ouv{$\left(\text{O}~;~\vect{u},~\vect{v}\right)$}

\hypersetup{breaklinks=true, colorlinks = true, linkcolor = OliveGreen, urlcolor = OliveGreen, citecolor = OliveGreen, pdfauthor={Didier BONNEL - https://www.maths-cours.fr} } % supprime les bordures autour des liens

\renewcommand{\arg}[0]{\text{arg}}

\everymath{\displaystyle}

%================================================================================================================================
%
% Macros - Commandes
%
%================================================================================================================================

\newcommand\meta[2]{    			% Utilisé pour créer le post HTML.
	\def\titre{titre}
	\def\url{url}
	\def\arg{#1}
	\ifx\titre\arg
		\newcommand\maintitle{#2}
		\fancyhead[L]{#2}
		{\Large\sffamily \MakeUppercase{#2}}
		\vspace{1mm}\textcolor{mcvert}{\hrule}
	\fi 
	\ifx\url\arg
		\fancyfoot[L]{\href{https://www.maths-cours.fr#2}{\black \footnotesize{https://www.maths-cours.fr#2}}}
	\fi 
}


\newcommand\TitreC[1]{    		% Titre centré
     \needspace{3\baselineskip}
     \begin{center}\textbf{#1}\end{center}
}

\newcommand\newpar{    		% paragraphe
     \par
}

\newcommand\nosp {    		% commande vide (pas d'espace)
}
\newcommand{\id}[1]{} %ignore

\newcommand\boite[2]{				% Boite simple sans titre
	\vspace{5mm}
	\setlength{\fboxrule}{0.2mm}
	\setlength{\fboxsep}{5mm}	
	\fcolorbox{#1}{#1!3}{\makebox[\linewidth-2\fboxrule-2\fboxsep]{
  		\begin{minipage}[t]{\linewidth-2\fboxrule-4\fboxsep}\setlength{\parskip}{3mm}
  			 #2
  		\end{minipage}
	}}
	\vspace{5mm}
}

\newcommand\CBox[4]{				% Boites
	\vspace{5mm}
	\setlength{\fboxrule}{0.2mm}
	\setlength{\fboxsep}{5mm}
	
	\fcolorbox{#1}{#1!3}{\makebox[\linewidth-2\fboxrule-2\fboxsep]{
		\begin{minipage}[t]{1cm}\setlength{\parskip}{3mm}
	  		\textcolor{#1}{\LARGE{#2}}    
 	 	\end{minipage}  
  		\begin{minipage}[t]{\linewidth-2\fboxrule-4\fboxsep}\setlength{\parskip}{3mm}
			\raisebox{1.2mm}{\normalsize\sffamily{\textcolor{#1}{#3}}}						
  			 #4
  		\end{minipage}
	}}
	\vspace{5mm}
}

\newcommand\cadre[3]{				% Boites convertible html
	\par
	\vspace{2mm}
	\setlength{\fboxrule}{0.1mm}
	\setlength{\fboxsep}{5mm}
	\fcolorbox{#1}{white}{\makebox[\linewidth-2\fboxrule-2\fboxsep]{
  		\begin{minipage}[t]{\linewidth-2\fboxrule-4\fboxsep}\setlength{\parskip}{3mm}
			\raisebox{-2.5mm}{\sffamily \small{\textcolor{#1}{\MakeUppercase{#2}}}}		
			\par		
  			 #3
 	 		\end{minipage}
	}}
		\vspace{2mm}
	\par
}

\newcommand\bloc[3]{				% Boites convertible html sans bordure
     \needspace{2\baselineskip}
     {\sffamily \small{\textcolor{#1}{\MakeUppercase{#2}}}}    
		\par		
  			 #3
		\par
}

\newcommand\CHelp[1]{
     \CBox{Plum}{\faInfoCircle}{À RETENIR}{#1}
}

\newcommand\CUp[1]{
     \CBox{NavyBlue}{\faThumbsOUp}{EN PRATIQUE}{#1}
}

\newcommand\CInfo[1]{
     \CBox{Sepia}{\faArrowCircleRight}{REMARQUE}{#1}
}

\newcommand\CRedac[1]{
     \CBox{PineGreen}{\faEdit}{BIEN R\'EDIGER}{#1}
}

\newcommand\CError[1]{
     \CBox{Red}{\faExclamationTriangle}{ATTENTION}{#1}
}

\newcommand\TitreExo[2]{
\needspace{4\baselineskip}
 {\sffamily\large EXERCICE #1\ (\emph{#2 points})}
\vspace{5mm}
}

\newcommand\img[2]{
          \includegraphics[width=#2\paperwidth]{\imgdir#1}
}

\newcommand\imgsvg[2]{
       \begin{center}   \includegraphics[width=#2\paperwidth]{\imgsvgdir#1} \end{center}
}


\newcommand\Lien[2]{
     \href{#1}{#2 \tiny \faExternalLink}
}
\newcommand\mcLien[2]{
     \href{https~://www.maths-cours.fr/#1}{#2 \tiny \faExternalLink}
}

\newcommand{\euro}{\eurologo{}}

%================================================================================================================================
%
% Macros - Environement
%
%================================================================================================================================

\newenvironment{tex}{ %
}
{%
}

\newenvironment{indente}{ %
	\setlength\parindent{10mm}
}

{
	\setlength\parindent{0mm}
}

\newenvironment{corrige}{%
     \needspace{3\baselineskip}
     \medskip
     \textbf{\textsc{Corrigé}}
     \medskip
}
{
}

\newenvironment{extern}{%
     \begin{center}
     }
     {
     \end{center}
}

\NewEnviron{code}{%
	\par
     \boite{gray}{\texttt{%
     \BODY
     }}
     \par
}

\newenvironment{vbloc}{% boite sans cadre empeche saut de page
     \begin{minipage}[t]{\linewidth}
     }
     {
     \end{minipage}
}
\NewEnviron{h2}{%
    \needspace{3\baselineskip}
    \vspace{0.6cm}
	\noindent \MakeUppercase{\sffamily \large \BODY}
	\vspace{1mm}\textcolor{mcgris}{\hrule}\vspace{0.4cm}
	\par
}{}

\NewEnviron{h3}{%
    \needspace{3\baselineskip}
	\vspace{5mm}
	\textsc{\BODY}
	\par
}

\NewEnviron{margeneg}{ %
\begin{addmargin}[-1cm]{0cm}
\BODY
\end{addmargin}
}

\NewEnviron{html}{%
}

\begin{document}
\meta{url}{/supplement/fiche-de-revision-bac-les-suites/}
\meta{pid}{10791}
\meta{titre}{Fiche de révision BAC : les suites}
\meta{type}{supplement}
%
\begin{enumerate}
     \item %
     Comment peut-on montrer qu'une suite est croissante ? décroissante ? constante ?
     \item %
     Qu'est-ce qu'une suite majorée ? minorée ? bornée ?
     \item %
     Quelles méthodes peut-on utiliser pour montrer qu'une suite est convergente ?
     \item %
     Comment montre-t-on qu'une suite est arithmétique ?
     \item %
     Pour une suite arithmétique de raison $r$, quelle formule permet de calculer $u_n$ en fonction de $u_0 $ ? en fonction de $u_p$  $(p \in \mathbb{N})$ ?
     \item %
     Que vaut la somme : $1+2+3+\cdots+n$ ?
     \item %
     Comment montre-t-on qu'une suite est géométrique ?
     \item %
     Pour une suite géométrique de raison $q$, quelle formule permet de calculer $u_n$ en fonction de $u_0 $? en fonction de $u_p$  $(p \in \mathbb{N})$ ?
     \item %
     Que vaut la somme : $1+q+q^2+\cdots+q^n $?
     \item %
     Quelle est (en fonction de $q$) la limite de $q^n$ ?
     \item %
     Écrire un algorithme affichant les $n$ premiers termes d'une suite.
     \item %
     Quelles sont les étapes d'une démonstration par récurrence ?
\end{enumerate}
\begin{reponses}
     \begin{enumerate}
          \item %
          \textit{Comment peut-on montrer qu'une suite est croissante ? décroissante ? constante ?}
          \par
          Voici 3 des principales méthodes :
          \begin{enumerate}[label=\alph*.]
               \item %
               \textbf{Calcul de $u_{n+1}-u_n$.}
               \par
               Si cette différence est positive pour tout entier naturel $n$ la suite $(u_n)$ est croissante~;
               \par
               si cette différence est négative pour tout entier naturel $n$ la suite $(u_n)$ est décroissante~;
               \par
               enfin, si cette différence est nulle pour tout entier naturel $n$ la suite $(u_n)$ est constante.
               \item %
               \textbf{Par récurrence.}
               \par
               Dans ce cas, c'est la comparaison des deux premiers termes (e.g. $u_0$ et $u_1$) qui dira si la suite est croissante ou décroissante.
               \item %
               Si la suite $(u_n)$ est définie de façon explicite par une formule du type $u_n=f(n)$, on peut étudier les variations de $f$ sur $[0~;~+\infty[$ (calcul de la dérivée $f'$...).
          \end{enumerate}
          \item %
          \textit{Qu'est-ce qu'une suite majorée ? minorée ? bornée ?}
          \par
          Une suite $(u_n)$ est \textbf{majorée} s'il existe un réel $M$ tel que pour tout entier naturel $n$~: $u_n \leqslant M$.
          \par
          Une suite $(u_n)$ est \textbf{minorée} s'il existe un réel $m$ tel que pour tout entier naturel $n$~: $u_n \geqslant m$.
          \par
          Une suite est bornée si elle est à la fois majorée et minorée.
          \item %
          \textit{Quelles méthodes peut-on utiliser pour montrer qu'une suite est convergente ?}
          \par
          Voici 3 méthodes. La plus utilisée dans les sujets du bac est la première.
          \begin{enumerate}[label=\alph*.]
               \item %
               \textbf{Suite croissante majorée ou décroissante minorée.}
               Si une suite est croissante et majorée alors elle est convergente. De même, une suite décroissante et minorée est convergente.
               \item %
               \textbf{Théorème des gendarmes} (Voir \mcLien{cours/variations-convergence-suite\#t60}{cours}).
               \item %
               Si la suite $(u_n)$ est définie de façon explicite on peut calculer la limite en utilisant les règles de calculs des limites (similaires à celles utilisées pour les fonctions).
               \par
               Dans ce cas, gardez aussi à l'esprit la formule donnant la limite de $q^n$ (voir ci-dessous)
          \end{enumerate}
          \item %
          \textit{Comment montre-t-on qu'une suite est arithmétique ?}
          Pour montrer que la suite $(u_n)$ est arithmétique on calcule $u_{n+1}-u_n$ et on montre que le résultat est constant (indépendant de $n$). Ce résultat est la raison de la suite arithmétique.
          \item %
          \textit{Pour une suite arithmétique de raison $r$, quelle formule permet de calculer $u_n$ en fonction de $u_0 $~? en fonction de $u_p$  $(p \in \mathbb{N})$~?}
          \par
          En fonction de $u_0~:~u_n=u_0+nr$
          \par
          En fonction de $u_p~:~u_n=u_p+(n-p)r$
          \item %
          \textit{Que vaut la somme : $1+2+3+\cdots+n$ ?}
          \par
          $1+2+3+\cdots+n=\dfrac{n(n+1)}{2}$
          \item %
          \textit{Comment montre-t-on qu'une suite $(u_n)$ est géométrique ?}
          On montre qu'il existe un réel $q$, indépendant de $n$, tel que pour tout entier naturel $n$~: $u_{n+1}=qu_n$.
          (on peut également montrer que le rapport $\dfrac{u_{n+1}}{u_n}$ est constant si on sait que la suite $(u_n)$ ne s'annule pas.)
          \item %
          \textit{Pour une suite géométrique de raison $q$, quelle formule permet de calculer $u_n$ en fonction de $u_0 $? en fonction de $u_p$  $(p \in \mathbb{N})$ ?}
          En fonction de $u_0~:~u_n=u_0q^n$
          En fonction de $u_p~:~u_n=u_pq^{n-p}$
          \item %
          \textit{ Que vaut la somme : $1+q+q^2+\cdots+q^n $?}
          \par
          Pour tout réel $q \neq 1$~:
          \par
          $1+q+q^2+\cdots+q^n =\dfrac{1-q^{n+1}}{1-q}$
          \item %
          \textit{Quelle est (en fonction de $q$) la limite de $q^n$ ?}
          \begin{itemize}
               \item %
               si $q>1~:~\lim\limits_{n \rightarrow +\infty }q^n=+\infty$~;~la suite est divergente ;
               \item %
               si $-1<q<1~:~\lim\limits_{n \rightarrow +\infty }q^n=0$~;~la suite converge vers 0 ;
               \item %
               si $q \leqslant -1~:$~la suite est divergente (pas de limite) ;
               \item %
               pour $q=1$, la suite est constante.
          \end{itemize}
          \item %
          \textit{Écrire un algorithme affichant les $n$ premiers termes d'une suite.}
          \par
          Voir la fiche \mcLien{methode/algorithme-premiers-termes}{Algorithme de calcul des premiers termes d'une suite}.
          \item %
          \textit{ Quelles sont les étapes d'une démonstration par récurrence ?}
          \begin{itemize}
               \item %
               \textbf{Initialisation}~:~On montre que la propriété est vraie au premier rang (e.g. au rang 0).
               \item %
               \textbf{Hérédité}~:~On montre que si la propriété est vraie à un certain rang , alors elle est vraie au rang suivant.
               \item %
               \textbf{Conclusion}~:~On en déduit que la propriété est vraie pour tout entier naturel $n$ (ou pour tout entier $n \geqslant n_0$ si l'initialisation a été faite au rang $n_0$).
          \end{itemize}
     \end{enumerate}
\end{reponses}

\end{document}
µ
\documentclass[a4paper]{article}

%================================================================================================================================
%
% Packages
%
%================================================================================================================================

\usepackage[T1]{fontenc} 	% pour caractères accentués
\usepackage[utf8]{inputenc}  % encodage utf8
\usepackage[french]{babel}	% langue : français
\usepackage{fourier}			% caractères plus lisibles
\usepackage[dvipsnames]{xcolor} % couleurs
\usepackage{fancyhdr}		% réglage header footer
\usepackage{needspace}		% empêcher sauts de page mal placés
\usepackage{graphicx}		% pour inclure des graphiques
\usepackage{enumitem,cprotect}		% personnalise les listes d'items (nécessaire pour ol, al ...)
\usepackage{hyperref}		% Liens hypertexte
\usepackage{pstricks,pst-all,pst-node,pstricks-add,pst-math,pst-plot,pst-tree,pst-eucl} % pstricks
\usepackage[a4paper,includeheadfoot,top=2cm,left=3cm, bottom=2cm,right=3cm]{geometry} % marges etc.
\usepackage{comment}			% commentaires multilignes
\usepackage{amsmath,environ} % maths (matrices, etc.)
\usepackage{amssymb,makeidx}
\usepackage{bm}				% bold maths
\usepackage{tabularx}		% tableaux
\usepackage{colortbl}		% tableaux en couleur
\usepackage{fontawesome}		% Fontawesome
\usepackage{environ}			% environment with command
\usepackage{fp}				% calculs pour ps-tricks
\usepackage{multido}			% pour ps tricks
\usepackage[np]{numprint}	% formattage nombre
\usepackage{tikz,tkz-tab} 			% package principal TikZ
\usepackage{pgfplots}   % axes
\usepackage{mathrsfs}    % cursives
\usepackage{calc}			% calcul taille boites
\usepackage[scaled=0.875]{helvet} % font sans serif
\usepackage{svg} % svg
\usepackage{scrextend} % local margin
\usepackage{scratch} %scratch
\usepackage{multicol} % colonnes
%\usepackage{infix-RPN,pst-func} % formule en notation polanaise inversée
\usepackage{listings}

%================================================================================================================================
%
% Réglages de base
%
%================================================================================================================================

\lstset{
language=Python,   % R code
literate=
{á}{{\'a}}1
{à}{{\`a}}1
{ã}{{\~a}}1
{é}{{\'e}}1
{è}{{\`e}}1
{ê}{{\^e}}1
{í}{{\'i}}1
{ó}{{\'o}}1
{õ}{{\~o}}1
{ú}{{\'u}}1
{ü}{{\"u}}1
{ç}{{\c{c}}}1
{~}{{ }}1
}


\definecolor{codegreen}{rgb}{0,0.6,0}
\definecolor{codegray}{rgb}{0.5,0.5,0.5}
\definecolor{codepurple}{rgb}{0.58,0,0.82}
\definecolor{backcolour}{rgb}{0.95,0.95,0.92}

\lstdefinestyle{mystyle}{
    backgroundcolor=\color{backcolour},   
    commentstyle=\color{codegreen},
    keywordstyle=\color{magenta},
    numberstyle=\tiny\color{codegray},
    stringstyle=\color{codepurple},
    basicstyle=\ttfamily\footnotesize,
    breakatwhitespace=false,         
    breaklines=true,                 
    captionpos=b,                    
    keepspaces=true,                 
    numbers=left,                    
xleftmargin=2em,
framexleftmargin=2em,            
    showspaces=false,                
    showstringspaces=false,
    showtabs=false,                  
    tabsize=2,
    upquote=true
}

\lstset{style=mystyle}


\lstset{style=mystyle}
\newcommand{\imgdir}{C:/laragon/www/newmc/assets/imgsvg/}
\newcommand{\imgsvgdir}{C:/laragon/www/newmc/assets/imgsvg/}

\definecolor{mcgris}{RGB}{220, 220, 220}% ancien~; pour compatibilité
\definecolor{mcbleu}{RGB}{52, 152, 219}
\definecolor{mcvert}{RGB}{125, 194, 70}
\definecolor{mcmauve}{RGB}{154, 0, 215}
\definecolor{mcorange}{RGB}{255, 96, 0}
\definecolor{mcturquoise}{RGB}{0, 153, 153}
\definecolor{mcrouge}{RGB}{255, 0, 0}
\definecolor{mclightvert}{RGB}{205, 234, 190}

\definecolor{gris}{RGB}{220, 220, 220}
\definecolor{bleu}{RGB}{52, 152, 219}
\definecolor{vert}{RGB}{125, 194, 70}
\definecolor{mauve}{RGB}{154, 0, 215}
\definecolor{orange}{RGB}{255, 96, 0}
\definecolor{turquoise}{RGB}{0, 153, 153}
\definecolor{rouge}{RGB}{255, 0, 0}
\definecolor{lightvert}{RGB}{205, 234, 190}
\setitemize[0]{label=\color{lightvert}  $\bullet$}

\pagestyle{fancy}
\renewcommand{\headrulewidth}{0.2pt}
\fancyhead[L]{maths-cours.fr}
\fancyhead[R]{\thepage}
\renewcommand{\footrulewidth}{0.2pt}
\fancyfoot[C]{}

\newcolumntype{C}{>{\centering\arraybackslash}X}
\newcolumntype{s}{>{\hsize=.35\hsize\arraybackslash}X}

\setlength{\parindent}{0pt}		 
\setlength{\parskip}{3mm}
\setlength{\headheight}{1cm}

\def\ebook{ebook}
\def\book{book}
\def\web{web}
\def\type{web}

\newcommand{\vect}[1]{\overrightarrow{\,\mathstrut#1\,}}

\def\Oij{$\left(\text{O}~;~\vect{\imath},~\vect{\jmath}\right)$}
\def\Oijk{$\left(\text{O}~;~\vect{\imath},~\vect{\jmath},~\vect{k}\right)$}
\def\Ouv{$\left(\text{O}~;~\vect{u},~\vect{v}\right)$}

\hypersetup{breaklinks=true, colorlinks = true, linkcolor = OliveGreen, urlcolor = OliveGreen, citecolor = OliveGreen, pdfauthor={Didier BONNEL - https://www.maths-cours.fr} } % supprime les bordures autour des liens

\renewcommand{\arg}[0]{\text{arg}}

\everymath{\displaystyle}

%================================================================================================================================
%
% Macros - Commandes
%
%================================================================================================================================

\newcommand\meta[2]{    			% Utilisé pour créer le post HTML.
	\def\titre{titre}
	\def\url{url}
	\def\arg{#1}
	\ifx\titre\arg
		\newcommand\maintitle{#2}
		\fancyhead[L]{#2}
		{\Large\sffamily \MakeUppercase{#2}}
		\vspace{1mm}\textcolor{mcvert}{\hrule}
	\fi 
	\ifx\url\arg
		\fancyfoot[L]{\href{https://www.maths-cours.fr#2}{\black \footnotesize{https://www.maths-cours.fr#2}}}
	\fi 
}


\newcommand\TitreC[1]{    		% Titre centré
     \needspace{3\baselineskip}
     \begin{center}\textbf{#1}\end{center}
}

\newcommand\newpar{    		% paragraphe
     \par
}

\newcommand\nosp {    		% commande vide (pas d'espace)
}
\newcommand{\id}[1]{} %ignore

\newcommand\boite[2]{				% Boite simple sans titre
	\vspace{5mm}
	\setlength{\fboxrule}{0.2mm}
	\setlength{\fboxsep}{5mm}	
	\fcolorbox{#1}{#1!3}{\makebox[\linewidth-2\fboxrule-2\fboxsep]{
  		\begin{minipage}[t]{\linewidth-2\fboxrule-4\fboxsep}\setlength{\parskip}{3mm}
  			 #2
  		\end{minipage}
	}}
	\vspace{5mm}
}

\newcommand\CBox[4]{				% Boites
	\vspace{5mm}
	\setlength{\fboxrule}{0.2mm}
	\setlength{\fboxsep}{5mm}
	
	\fcolorbox{#1}{#1!3}{\makebox[\linewidth-2\fboxrule-2\fboxsep]{
		\begin{minipage}[t]{1cm}\setlength{\parskip}{3mm}
	  		\textcolor{#1}{\LARGE{#2}}    
 	 	\end{minipage}  
  		\begin{minipage}[t]{\linewidth-2\fboxrule-4\fboxsep}\setlength{\parskip}{3mm}
			\raisebox{1.2mm}{\normalsize\sffamily{\textcolor{#1}{#3}}}						
  			 #4
  		\end{minipage}
	}}
	\vspace{5mm}
}

\newcommand\cadre[3]{				% Boites convertible html
	\par
	\vspace{2mm}
	\setlength{\fboxrule}{0.1mm}
	\setlength{\fboxsep}{5mm}
	\fcolorbox{#1}{white}{\makebox[\linewidth-2\fboxrule-2\fboxsep]{
  		\begin{minipage}[t]{\linewidth-2\fboxrule-4\fboxsep}\setlength{\parskip}{3mm}
			\raisebox{-2.5mm}{\sffamily \small{\textcolor{#1}{\MakeUppercase{#2}}}}		
			\par		
  			 #3
 	 		\end{minipage}
	}}
		\vspace{2mm}
	\par
}

\newcommand\bloc[3]{				% Boites convertible html sans bordure
     \needspace{2\baselineskip}
     {\sffamily \small{\textcolor{#1}{\MakeUppercase{#2}}}}    
		\par		
  			 #3
		\par
}

\newcommand\CHelp[1]{
     \CBox{Plum}{\faInfoCircle}{À RETENIR}{#1}
}

\newcommand\CUp[1]{
     \CBox{NavyBlue}{\faThumbsOUp}{EN PRATIQUE}{#1}
}

\newcommand\CInfo[1]{
     \CBox{Sepia}{\faArrowCircleRight}{REMARQUE}{#1}
}

\newcommand\CRedac[1]{
     \CBox{PineGreen}{\faEdit}{BIEN R\'EDIGER}{#1}
}

\newcommand\CError[1]{
     \CBox{Red}{\faExclamationTriangle}{ATTENTION}{#1}
}

\newcommand\TitreExo[2]{
\needspace{4\baselineskip}
 {\sffamily\large EXERCICE #1\ (\emph{#2 points})}
\vspace{5mm}
}

\newcommand\img[2]{
          \includegraphics[width=#2\paperwidth]{\imgdir#1}
}

\newcommand\imgsvg[2]{
       \begin{center}   \includegraphics[width=#2\paperwidth]{\imgsvgdir#1} \end{center}
}


\newcommand\Lien[2]{
     \href{#1}{#2 \tiny \faExternalLink}
}
\newcommand\mcLien[2]{
     \href{https~://www.maths-cours.fr/#1}{#2 \tiny \faExternalLink}
}

\newcommand{\euro}{\eurologo{}}

%================================================================================================================================
%
% Macros - Environement
%
%================================================================================================================================

\newenvironment{tex}{ %
}
{%
}

\newenvironment{indente}{ %
	\setlength\parindent{10mm}
}

{
	\setlength\parindent{0mm}
}

\newenvironment{corrige}{%
     \needspace{3\baselineskip}
     \medskip
     \textbf{\textsc{Corrigé}}
     \medskip
}
{
}

\newenvironment{extern}{%
     \begin{center}
     }
     {
     \end{center}
}

\NewEnviron{code}{%
	\par
     \boite{gray}{\texttt{%
     \BODY
     }}
     \par
}

\newenvironment{vbloc}{% boite sans cadre empeche saut de page
     \begin{minipage}[t]{\linewidth}
     }
     {
     \end{minipage}
}
\NewEnviron{h2}{%
    \needspace{3\baselineskip}
    \vspace{0.6cm}
	\noindent \MakeUppercase{\sffamily \large \BODY}
	\vspace{1mm}\textcolor{mcgris}{\hrule}\vspace{0.4cm}
	\par
}{}

\NewEnviron{h3}{%
    \needspace{3\baselineskip}
	\vspace{5mm}
	\textsc{\BODY}
	\par
}

\NewEnviron{margeneg}{ %
\begin{addmargin}[-1cm]{0cm}
\BODY
\end{addmargin}
}

\NewEnviron{html}{%
}

\begin{document}
\meta{url}{/supplement/fiche-de-revision-bac-probabilites-discretes/}
\meta{pid}{10824}
\meta{titre}{Fiche de révision BAC : probabilités discrètes}
\meta{type}{supplement}
%
\begin{enumerate}
     \item %
     Quelle formule donne $p_B (A)$ ? Quelle est la différence entre $p_B (A)$ et $p(A \cap B)$ ?
     \item %
     Quand dit-on que deux événements sont indépendants ?
     \item %
     Quelle est la formule des probabilités totales ?
     \item %
     Qu'est ce que la « loi de probabilité » d'une variable aléatoire discrète ?
     \item %
     Comment calcule-t-on l'espérance mathématique d'une variable aléatoire discrète ? sa variance ? son écart-type ?
     \item %
     Quand dit-on qu'une variable aléatoire suit une loi binomiale $\mathscr{B}(n;p)$ ?
     \item %
     Quelle est l'espérance mathématique d'une loi binomiale ? sa variance ?
     \item %
     Quelle formule donne $p(X=k)$ lorsque $X$ suit une loi binomiale ?
\end{enumerate}
\begin{reponses}
     \begin{enumerate}
          \item %
          \textit{Quelle formule donne $p_B (A)$ ? Quelle est la différence entre $p_B (A)$ et $p(A \cap B)$ ?}
          \par
          $p_B(A)=\dfrac{p(A\cap B)}{p(B)}$ (formule des probabilités conditionnelles).
          \par
          ${p(A\cap B)}$  est la probabilité que $A$ \textbf{et} $B$ se réalisent (alors que l'on ne sait pas a priori si  $A$ ou si $B$ est réalisé) tandis que
          ${p_B(A)}$ est la probabilité que $A$ se réalise alors que l'\textbf{on sait que $B$ est réalisé}.
          \item %
          \textit{Quand dit-on que deux événements sont indépendants ?}
          \par
          $A$ et $B$ sont deux événements indépendants si et seulement si :
          \par
          $ p(A \cap B) = p(A) \times p(B)$.
          \par
          Si la probabilité de $B$ est non nulle cela équivaut à $P_B(A)=p(A)$.
          \par
          Intuitivement, cela revient à dire que la réalisation de $B$ n'a aucune influence sur la réalisation de $A$ (et réciproquement).
          \item %
          \textit{Quelle est la formule des probabilités totales ?}
          \par
          Pour deux événements $A$ et $B$~:
          \par
          $p(A)= p(A\cap B)+p(A\cap \overline{B})$.
          \par
          Plus généralement,  si les événements $B_1, B_2, \cdots , B_n$ forment une partition de l'univers alors, pour tout événement $A$ :
          \par
          $p(A)= p(A\cap B_1)+p(A\cap B_2)$\nosp$+\cdots+p(A\cap B_n).$
          \item %
          \textit{Qu'est ce que la « loi de probabilité » d'une variable aléatoire discrète ?}
          \par
          La loi de probabilité d'une variable aléatoire discrète $X$, généralement présentée sous forme d'un tableau, donne les probabilités de chacune des valeurs possibles $x_i$ de $X$.
          \item %
          \textit{Comment calcule-t-on l'espérance mathématique d'une variable aléatoire discrète ? sa variance ? son écart-type ?}
          \par
          Si $X$ prend les valeurs $x_i$ avec les probabilités $p_i$~ ;
          \par
          Espérance mathématique~:
          \par
          $E\left(X\right)= x_{1}\times p_{1}+x_{2}\times p_{2}+. . . +x_{n}\times p_{n} $\nosp$= \sum_{i=1}^{n}p_{i} x_{i}$
          \par
          Variance~:
          \par
          $V\left(X\right)=E\left(\left(X-\overline X\right)^{2}\right)$
          \par
          Ecart-type~:
          \par
          $\sigma \left(X\right)=\sqrt{V\left(X\right)}$
          \item %
          \textit{Quand dit-on qu'une variable aléatoire suit une loi binomiale $\mathscr{B}(n~;~p)$ ?}
          \par
          Une variable aléatoire $X$ suit une \textbf{loi binomiale} $\mathscr{B}(n~;~p)$ de paramètres $n$ et $p$, si ~:
          \par
          \begin{itemize}
               \par
               \item l'expérience est la \textbf{répétition} de $n$ épreuves de Bernoulli  identiques et indépendantes;
               \par
               \item chacune de ces épreuve de Bernoulli possède deux et uniquement issues~:
               \begin{itemize}[label=---]
                    \item %
                    \textit{succès}, de probabilité $p$;
                    \item %
                    \textit{échec}, de probabilité $1-p$ ;
               \end{itemize}
               \item la variable aléatoire $X$ est égal au nombre de succès.
               \par
          \end{itemize}
          \item %
          \textit{Quelle est l'espérance mathématique d'une loi binomiale ? sa variance ?}
          \par
          $E(X)=np $
          \par
          $V(X)=np(1-p) $
          \item %
          \par
          \textit{Quelle formule donne $p(X=k)$ lorsque $X$ suit une loi binomiale $\mathscr{B}(n~;~p)$ ?}
          $P\left(X=k\right)=\begin{pmatrix} n \\ k \end{pmatrix}p^{k} \left(1-p\right)^{n-k}$
     \end{enumerate}
\end{reponses}

\end{document}
µ
\documentclass[a4paper]{article}

%================================================================================================================================
%
% Packages
%
%================================================================================================================================

\usepackage[T1]{fontenc} 	% pour caractères accentués
\usepackage[utf8]{inputenc}  % encodage utf8
\usepackage[french]{babel}	% langue : français
\usepackage{fourier}			% caractères plus lisibles
\usepackage[dvipsnames]{xcolor} % couleurs
\usepackage{fancyhdr}		% réglage header footer
\usepackage{needspace}		% empêcher sauts de page mal placés
\usepackage{graphicx}		% pour inclure des graphiques
\usepackage{enumitem,cprotect}		% personnalise les listes d'items (nécessaire pour ol, al ...)
\usepackage{hyperref}		% Liens hypertexte
\usepackage{pstricks,pst-all,pst-node,pstricks-add,pst-math,pst-plot,pst-tree,pst-eucl} % pstricks
\usepackage[a4paper,includeheadfoot,top=2cm,left=3cm, bottom=2cm,right=3cm]{geometry} % marges etc.
\usepackage{comment}			% commentaires multilignes
\usepackage{amsmath,environ} % maths (matrices, etc.)
\usepackage{amssymb,makeidx}
\usepackage{bm}				% bold maths
\usepackage{tabularx}		% tableaux
\usepackage{colortbl}		% tableaux en couleur
\usepackage{fontawesome}		% Fontawesome
\usepackage{environ}			% environment with command
\usepackage{fp}				% calculs pour ps-tricks
\usepackage{multido}			% pour ps tricks
\usepackage[np]{numprint}	% formattage nombre
\usepackage{tikz,tkz-tab} 			% package principal TikZ
\usepackage{pgfplots}   % axes
\usepackage{mathrsfs}    % cursives
\usepackage{calc}			% calcul taille boites
\usepackage[scaled=0.875]{helvet} % font sans serif
\usepackage{svg} % svg
\usepackage{scrextend} % local margin
\usepackage{scratch} %scratch
\usepackage{multicol} % colonnes
%\usepackage{infix-RPN,pst-func} % formule en notation polanaise inversée
\usepackage{listings}

%================================================================================================================================
%
% Réglages de base
%
%================================================================================================================================

\lstset{
language=Python,   % R code
literate=
{á}{{\'a}}1
{à}{{\`a}}1
{ã}{{\~a}}1
{é}{{\'e}}1
{è}{{\`e}}1
{ê}{{\^e}}1
{í}{{\'i}}1
{ó}{{\'o}}1
{õ}{{\~o}}1
{ú}{{\'u}}1
{ü}{{\"u}}1
{ç}{{\c{c}}}1
{~}{{ }}1
}


\definecolor{codegreen}{rgb}{0,0.6,0}
\definecolor{codegray}{rgb}{0.5,0.5,0.5}
\definecolor{codepurple}{rgb}{0.58,0,0.82}
\definecolor{backcolour}{rgb}{0.95,0.95,0.92}

\lstdefinestyle{mystyle}{
    backgroundcolor=\color{backcolour},   
    commentstyle=\color{codegreen},
    keywordstyle=\color{magenta},
    numberstyle=\tiny\color{codegray},
    stringstyle=\color{codepurple},
    basicstyle=\ttfamily\footnotesize,
    breakatwhitespace=false,         
    breaklines=true,                 
    captionpos=b,                    
    keepspaces=true,                 
    numbers=left,                    
xleftmargin=2em,
framexleftmargin=2em,            
    showspaces=false,                
    showstringspaces=false,
    showtabs=false,                  
    tabsize=2,
    upquote=true
}

\lstset{style=mystyle}


\lstset{style=mystyle}
\newcommand{\imgdir}{C:/laragon/www/newmc/assets/imgsvg/}
\newcommand{\imgsvgdir}{C:/laragon/www/newmc/assets/imgsvg/}

\definecolor{mcgris}{RGB}{220, 220, 220}% ancien~; pour compatibilité
\definecolor{mcbleu}{RGB}{52, 152, 219}
\definecolor{mcvert}{RGB}{125, 194, 70}
\definecolor{mcmauve}{RGB}{154, 0, 215}
\definecolor{mcorange}{RGB}{255, 96, 0}
\definecolor{mcturquoise}{RGB}{0, 153, 153}
\definecolor{mcrouge}{RGB}{255, 0, 0}
\definecolor{mclightvert}{RGB}{205, 234, 190}

\definecolor{gris}{RGB}{220, 220, 220}
\definecolor{bleu}{RGB}{52, 152, 219}
\definecolor{vert}{RGB}{125, 194, 70}
\definecolor{mauve}{RGB}{154, 0, 215}
\definecolor{orange}{RGB}{255, 96, 0}
\definecolor{turquoise}{RGB}{0, 153, 153}
\definecolor{rouge}{RGB}{255, 0, 0}
\definecolor{lightvert}{RGB}{205, 234, 190}
\setitemize[0]{label=\color{lightvert}  $\bullet$}

\pagestyle{fancy}
\renewcommand{\headrulewidth}{0.2pt}
\fancyhead[L]{maths-cours.fr}
\fancyhead[R]{\thepage}
\renewcommand{\footrulewidth}{0.2pt}
\fancyfoot[C]{}

\newcolumntype{C}{>{\centering\arraybackslash}X}
\newcolumntype{s}{>{\hsize=.35\hsize\arraybackslash}X}

\setlength{\parindent}{0pt}		 
\setlength{\parskip}{3mm}
\setlength{\headheight}{1cm}

\def\ebook{ebook}
\def\book{book}
\def\web{web}
\def\type{web}

\newcommand{\vect}[1]{\overrightarrow{\,\mathstrut#1\,}}

\def\Oij{$\left(\text{O}~;~\vect{\imath},~\vect{\jmath}\right)$}
\def\Oijk{$\left(\text{O}~;~\vect{\imath},~\vect{\jmath},~\vect{k}\right)$}
\def\Ouv{$\left(\text{O}~;~\vect{u},~\vect{v}\right)$}

\hypersetup{breaklinks=true, colorlinks = true, linkcolor = OliveGreen, urlcolor = OliveGreen, citecolor = OliveGreen, pdfauthor={Didier BONNEL - https://www.maths-cours.fr} } % supprime les bordures autour des liens

\renewcommand{\arg}[0]{\text{arg}}

\everymath{\displaystyle}

%================================================================================================================================
%
% Macros - Commandes
%
%================================================================================================================================

\newcommand\meta[2]{    			% Utilisé pour créer le post HTML.
	\def\titre{titre}
	\def\url{url}
	\def\arg{#1}
	\ifx\titre\arg
		\newcommand\maintitle{#2}
		\fancyhead[L]{#2}
		{\Large\sffamily \MakeUppercase{#2}}
		\vspace{1mm}\textcolor{mcvert}{\hrule}
	\fi 
	\ifx\url\arg
		\fancyfoot[L]{\href{https://www.maths-cours.fr#2}{\black \footnotesize{https://www.maths-cours.fr#2}}}
	\fi 
}


\newcommand\TitreC[1]{    		% Titre centré
     \needspace{3\baselineskip}
     \begin{center}\textbf{#1}\end{center}
}

\newcommand\newpar{    		% paragraphe
     \par
}

\newcommand\nosp {    		% commande vide (pas d'espace)
}
\newcommand{\id}[1]{} %ignore

\newcommand\boite[2]{				% Boite simple sans titre
	\vspace{5mm}
	\setlength{\fboxrule}{0.2mm}
	\setlength{\fboxsep}{5mm}	
	\fcolorbox{#1}{#1!3}{\makebox[\linewidth-2\fboxrule-2\fboxsep]{
  		\begin{minipage}[t]{\linewidth-2\fboxrule-4\fboxsep}\setlength{\parskip}{3mm}
  			 #2
  		\end{minipage}
	}}
	\vspace{5mm}
}

\newcommand\CBox[4]{				% Boites
	\vspace{5mm}
	\setlength{\fboxrule}{0.2mm}
	\setlength{\fboxsep}{5mm}
	
	\fcolorbox{#1}{#1!3}{\makebox[\linewidth-2\fboxrule-2\fboxsep]{
		\begin{minipage}[t]{1cm}\setlength{\parskip}{3mm}
	  		\textcolor{#1}{\LARGE{#2}}    
 	 	\end{minipage}  
  		\begin{minipage}[t]{\linewidth-2\fboxrule-4\fboxsep}\setlength{\parskip}{3mm}
			\raisebox{1.2mm}{\normalsize\sffamily{\textcolor{#1}{#3}}}						
  			 #4
  		\end{minipage}
	}}
	\vspace{5mm}
}

\newcommand\cadre[3]{				% Boites convertible html
	\par
	\vspace{2mm}
	\setlength{\fboxrule}{0.1mm}
	\setlength{\fboxsep}{5mm}
	\fcolorbox{#1}{white}{\makebox[\linewidth-2\fboxrule-2\fboxsep]{
  		\begin{minipage}[t]{\linewidth-2\fboxrule-4\fboxsep}\setlength{\parskip}{3mm}
			\raisebox{-2.5mm}{\sffamily \small{\textcolor{#1}{\MakeUppercase{#2}}}}		
			\par		
  			 #3
 	 		\end{minipage}
	}}
		\vspace{2mm}
	\par
}

\newcommand\bloc[3]{				% Boites convertible html sans bordure
     \needspace{2\baselineskip}
     {\sffamily \small{\textcolor{#1}{\MakeUppercase{#2}}}}    
		\par		
  			 #3
		\par
}

\newcommand\CHelp[1]{
     \CBox{Plum}{\faInfoCircle}{À RETENIR}{#1}
}

\newcommand\CUp[1]{
     \CBox{NavyBlue}{\faThumbsOUp}{EN PRATIQUE}{#1}
}

\newcommand\CInfo[1]{
     \CBox{Sepia}{\faArrowCircleRight}{REMARQUE}{#1}
}

\newcommand\CRedac[1]{
     \CBox{PineGreen}{\faEdit}{BIEN R\'EDIGER}{#1}
}

\newcommand\CError[1]{
     \CBox{Red}{\faExclamationTriangle}{ATTENTION}{#1}
}

\newcommand\TitreExo[2]{
\needspace{4\baselineskip}
 {\sffamily\large EXERCICE #1\ (\emph{#2 points})}
\vspace{5mm}
}

\newcommand\img[2]{
          \includegraphics[width=#2\paperwidth]{\imgdir#1}
}

\newcommand\imgsvg[2]{
       \begin{center}   \includegraphics[width=#2\paperwidth]{\imgsvgdir#1} \end{center}
}


\newcommand\Lien[2]{
     \href{#1}{#2 \tiny \faExternalLink}
}
\newcommand\mcLien[2]{
     \href{https~://www.maths-cours.fr/#1}{#2 \tiny \faExternalLink}
}

\newcommand{\euro}{\eurologo{}}

%================================================================================================================================
%
% Macros - Environement
%
%================================================================================================================================

\newenvironment{tex}{ %
}
{%
}

\newenvironment{indente}{ %
	\setlength\parindent{10mm}
}

{
	\setlength\parindent{0mm}
}

\newenvironment{corrige}{%
     \needspace{3\baselineskip}
     \medskip
     \textbf{\textsc{Corrigé}}
     \medskip
}
{
}

\newenvironment{extern}{%
     \begin{center}
     }
     {
     \end{center}
}

\NewEnviron{code}{%
	\par
     \boite{gray}{\texttt{%
     \BODY
     }}
     \par
}

\newenvironment{vbloc}{% boite sans cadre empeche saut de page
     \begin{minipage}[t]{\linewidth}
     }
     {
     \end{minipage}
}
\NewEnviron{h2}{%
    \needspace{3\baselineskip}
    \vspace{0.6cm}
	\noindent \MakeUppercase{\sffamily \large \BODY}
	\vspace{1mm}\textcolor{mcgris}{\hrule}\vspace{0.4cm}
	\par
}{}

\NewEnviron{h3}{%
    \needspace{3\baselineskip}
	\vspace{5mm}
	\textsc{\BODY}
	\par
}

\NewEnviron{margeneg}{ %
\begin{addmargin}[-1cm]{0cm}
\BODY
\end{addmargin}
}

\NewEnviron{html}{%
}

\begin{document}
\meta{url}{/supplement/fiche-de-revision-bac-les-nombres-complexes/}
\meta{pid}{10842}
\meta{titre}{Fiche de révision BAC : les nombres complexes}
\meta{type}{supplement}
%
\begin{enumerate}
     \item %
     Quelle est la forme algébrique d'un nombre complexe~? Quelle est la partie réelle~? La partie imaginaire~?
     \item %
     Qu'est-ce que le conjugué d'un nombre complexe~?
     \item %
     Comment représente-t-on graphiquement un nombre complexe~?
     \item %
     Qu'est-ce que le module et un argument d'un nombre complexe~? Comment s'interprètent-ils graphiquement ~?
     \item %
     Quelles sont les propriétés des conjugués, des modules et des arguments (produit, etc…)~?
     \item %
     Comment obtient-on  la forme trigonométrique d'un nombre complexe~? La forme exponentielle~?
     \item %
     Comment s'obtient la distance $AB$ à partir des affixes des points $A$ et $B$~?
     \item %
     Quels sont les arguments possibles pour un nombre réel~? un nombre imaginaire pur~?
     \item %
     Quelles sont, dans $\mathbb{C}$,  les solutions de l'équation $az^2+bz+c=0$~?
     \bigskip
     \textbf{Rappels de collège utiles pour certains exercices portant sur les nombres complexes.}
     \par
     $A$ et $B$ désignent des points du plan.
     \item %
     Quel est l'ensemble des points $M$ tels que $AM=BM$~?
     \item %
     Quel est l'ensemble des points $M$ tels que $AM=k $ (où $k$ est un réel donné)~?
     \item %
     Quel est l'ensemble des points $M$ tels que $(\overrightarrow{MA}~;~\overrightarrow{MB})=\pm \dfrac{\pi}{2}~(\text{mod.}~2\pi)$~?
\end{enumerate}
\begin{reponses}
     \begin{enumerate}
          \item %
          \textit{Quelle est la forme algébrique d'un nombre complexe~? Quelle est la partie réelle~? La partie imaginaire~?}
          \par
          La forme algébrique d'un nombre complexe $z$ est $z=x+iy$ (ou $z=a+ib$...) où $x$ et $y$ sont deux réels.
          $x$ est la partie réelle de $z$ et $y$ sa partie imaginaire.
          \item %
          \textit{ Qu'est-ce que le conjugué d'un nombre complexe~?}
          \par
          Le conjugué de $z=x+iy$ est le nombre complexe $\overline{z}=x-iy$.
          \item %
          \textit{Comment représente-t-on graphiquement un nombre complexe~?}
          \par
          Dans un repère orthonormé, on représente ee nombre complexe $z=x+iy$ par le point $M(x~;~y)$.\\
          On dit que $M$ est l'image de $z$ et que $z$ est l'affixe de $M$.
          \item %
          \textit{Qu'est-ce que le module et un argument d'un nombre complexe~? Comment s'interprètent-ils graphiquement ~?}
          \par
          Si le plan est rapporté au repère $(O~;~\vec{u},~\vec{v})$, le module de $z$ d'image $M$ est la distance $OM$~:\\
          $|z|=OM=\sqrt{x^2+y^2}$
          \par
          Un argument $\theta$ de $z$ (pour $z$ non nul) est une mesure, en radians, de l'angle $( \vec{u}~;~\vec{OM})$.\\
          On a $\cos \theta = \dfrac{x}{|z|}$ et  $\sin \theta = \dfrac{y}{|z|}$
          \item %
          \textit{Quelles sont les propriétés des conjugués, des modules et des arguments (produit, etc…)~?}
          \par
          $z$, $z_1$, $z_2$ désignent des nombres complexes quelconques et $n$ un entier relatif.
          \textbf{Conjugués~:}
          \begin{itemize}
               \item $\overline{z_1+z_2} = \overline{z_1}+\overline{z_2}$
               \item $\overline{z_1z_2} = \overline{z_1}\times \overline{z_2}$
               \item $\overline{\left(\frac{z_1}{z_2}\right)} = \frac{\overline{z_1}}{\overline{z_2}}  $ (si $z_2\neq 0$)
               \item $\overline{\left(z^{n}\right)} = \left(\overline{z}\right)^{n}$.
          \end{itemize}
          \textbf{Modules~:}
          \begin{itemize}
               \item $|z_1z_2| = |z_1|\times |z_2|$
               \item $|\frac{z_1}{z_2}| = \frac{|z_1|}{|z_2|}  $ (si $z_2\neq 0$)
               \item $|z_1+z_2| \leqslant |z_1| + |z_2|$ (inégalité triangulaire)
          \end{itemize}
          \textbf{Arguments~:}
          \begin{itemize}
               \item $\text{arg}\left(\overline{z}\right)=-\text{arg}\left(z\right)$
               \item $\text{arg}\left(z_1z_2\right)=\text{arg}\left(z_1\right)+\text{arg}\left(z_2\right)$
               \item $\text{arg}\left(z^{n}\right)=n\times \text{arg}\left(z\right)$
               \item $\text{arg}\left(\frac{z_1}{z_2}\right)=\text{arg}\left(z_1\right)-\text{arg}\left(z_2\right)$
          \end{itemize}
          \par
          \item %
          \textit{Comment obtient-on  la forme trigonométrique d'un nombre complexe~? La forme exponentielle~?}
          \par
          La forme trigonométrique d'un nombre complexe $z$ de module $r$ et dont un argument est $\theta$ est~:\\
          $z=r(\cos \theta + i \sin \theta)$.
          \par
          La forme exponentielle est~:
          $z=r\text{e}^{i\theta}$
          \item %
          \textit{Comment s'obtient la distance $AB$ à partir des affixes des points $A$ et $B$~?}
          \par
          Si $A$ et $B$ ont pour affixes respectives $z_A$ et $z_B$~:
          $AB=\left|z_B-z_A\right|$
          \item %
          \textit{Quels sont les arguments possibles pour un nombre réel~? un nombre imaginaire pur~?}
          \par
          Un nombre réel non nul a pour argument $0~(\text{mod.}~2\pi)$ (s'il est positif) ou $\pi~(\text{mod.}~2\pi)$ (s'il est négatif).
          Un nombre imaginaire pur non nul a pour argument $\dfrac{\pi}{2}~(\text{mod.}~2\pi)$ (si sa partie imaginaire est positive) ou $-\dfrac{\pi}{2}~(\text{mod.}~2\pi)$ (si sa partie imaginaire est négative)
          \item %
          \textit{Quelles sont, dans $\mathbb{C}$,  les solutions de l'équation $az^2+bz+c=0$~?}
          \par
          Si $\Delta$ est positif ou nul, on retrouve les solutions réelles.\\
          Si $\Delta$ est strictement négatif, l'équation possède deux solutions conjuguées~:\\
          $z_{1}=\frac{-b-i\sqrt{-\Delta }}{2a}  $ \\
          $z_{2}=\frac{-b+i\sqrt{-\Delta }}{2a}$.
          \par
          \bigskip
          \textbf{Rappels de collège utiles pour certains exercices portant sur les nombres complexes.}
          \par
          $A$ et $B$ désignent des points du plan.
          \item %
          \textit{Quel est l'ensemble des points $M$ tels que $AM=BM$~?}
          \par
          L'ensemble des points $M$ tels que $AM=BM$ est la médiatrice du segment $[AB]$.
          \item %
          \textit{Quel est l'ensemble des points $M$ tels que $AM=k $ (où $k$ est un réel donné)~?}
          \par
          L'ensemble des points $M$ tels que $AM=k$ est :
          \begin{itemize}
               \item %
               le cercle de centre $A$ et de rayon $k$ si $k > 0$
               \item %
               le point $A$  si $k = 0$
               \item %
               l'ensemble vide  si $k < 0$
          \end{itemize}
          \item %
          \textit{Quel est l'ensemble des points $M$ tels que $(\overrightarrow{MA}~;~\overrightarrow{MB})=\pm \dfrac{\pi}{2}~(\text{mod.}~2\pi)$~?}
          \par
          l'ensemble des points $M$ tels que $(\overrightarrow{MA}~;~\overrightarrow{MB})=\pm \dfrac{\pi}{2}~(\text{mod.}~2\pi)$ est le cercle de diamètre $[AB]$ privé des points $A$ et $B$ (pour lesquels l'angle $(\overrightarrow{MA}~;~\overrightarrow{MB})$ n'est pas défini).
     \end{enumerate}
\end{reponses}

\end{document}
µ
\documentclass[a4paper]{article}

%================================================================================================================================
%
% Packages
%
%================================================================================================================================

\usepackage[T1]{fontenc} 	% pour caractères accentués
\usepackage[utf8]{inputenc}  % encodage utf8
\usepackage[french]{babel}	% langue : français
\usepackage{fourier}			% caractères plus lisibles
\usepackage[dvipsnames]{xcolor} % couleurs
\usepackage{fancyhdr}		% réglage header footer
\usepackage{needspace}		% empêcher sauts de page mal placés
\usepackage{graphicx}		% pour inclure des graphiques
\usepackage{enumitem,cprotect}		% personnalise les listes d'items (nécessaire pour ol, al ...)
\usepackage{hyperref}		% Liens hypertexte
\usepackage{pstricks,pst-all,pst-node,pstricks-add,pst-math,pst-plot,pst-tree,pst-eucl} % pstricks
\usepackage[a4paper,includeheadfoot,top=2cm,left=3cm, bottom=2cm,right=3cm]{geometry} % marges etc.
\usepackage{comment}			% commentaires multilignes
\usepackage{amsmath,environ} % maths (matrices, etc.)
\usepackage{amssymb,makeidx}
\usepackage{bm}				% bold maths
\usepackage{tabularx}		% tableaux
\usepackage{colortbl}		% tableaux en couleur
\usepackage{fontawesome}		% Fontawesome
\usepackage{environ}			% environment with command
\usepackage{fp}				% calculs pour ps-tricks
\usepackage{multido}			% pour ps tricks
\usepackage[np]{numprint}	% formattage nombre
\usepackage{tikz,tkz-tab} 			% package principal TikZ
\usepackage{pgfplots}   % axes
\usepackage{mathrsfs}    % cursives
\usepackage{calc}			% calcul taille boites
\usepackage[scaled=0.875]{helvet} % font sans serif
\usepackage{svg} % svg
\usepackage{scrextend} % local margin
\usepackage{scratch} %scratch
\usepackage{multicol} % colonnes
%\usepackage{infix-RPN,pst-func} % formule en notation polanaise inversée
\usepackage{listings}

%================================================================================================================================
%
% Réglages de base
%
%================================================================================================================================

\lstset{
language=Python,   % R code
literate=
{á}{{\'a}}1
{à}{{\`a}}1
{ã}{{\~a}}1
{é}{{\'e}}1
{è}{{\`e}}1
{ê}{{\^e}}1
{í}{{\'i}}1
{ó}{{\'o}}1
{õ}{{\~o}}1
{ú}{{\'u}}1
{ü}{{\"u}}1
{ç}{{\c{c}}}1
{~}{{ }}1
}


\definecolor{codegreen}{rgb}{0,0.6,0}
\definecolor{codegray}{rgb}{0.5,0.5,0.5}
\definecolor{codepurple}{rgb}{0.58,0,0.82}
\definecolor{backcolour}{rgb}{0.95,0.95,0.92}

\lstdefinestyle{mystyle}{
    backgroundcolor=\color{backcolour},   
    commentstyle=\color{codegreen},
    keywordstyle=\color{magenta},
    numberstyle=\tiny\color{codegray},
    stringstyle=\color{codepurple},
    basicstyle=\ttfamily\footnotesize,
    breakatwhitespace=false,         
    breaklines=true,                 
    captionpos=b,                    
    keepspaces=true,                 
    numbers=left,                    
xleftmargin=2em,
framexleftmargin=2em,            
    showspaces=false,                
    showstringspaces=false,
    showtabs=false,                  
    tabsize=2,
    upquote=true
}

\lstset{style=mystyle}


\lstset{style=mystyle}
\newcommand{\imgdir}{C:/laragon/www/newmc/assets/imgsvg/}
\newcommand{\imgsvgdir}{C:/laragon/www/newmc/assets/imgsvg/}

\definecolor{mcgris}{RGB}{220, 220, 220}% ancien~; pour compatibilité
\definecolor{mcbleu}{RGB}{52, 152, 219}
\definecolor{mcvert}{RGB}{125, 194, 70}
\definecolor{mcmauve}{RGB}{154, 0, 215}
\definecolor{mcorange}{RGB}{255, 96, 0}
\definecolor{mcturquoise}{RGB}{0, 153, 153}
\definecolor{mcrouge}{RGB}{255, 0, 0}
\definecolor{mclightvert}{RGB}{205, 234, 190}

\definecolor{gris}{RGB}{220, 220, 220}
\definecolor{bleu}{RGB}{52, 152, 219}
\definecolor{vert}{RGB}{125, 194, 70}
\definecolor{mauve}{RGB}{154, 0, 215}
\definecolor{orange}{RGB}{255, 96, 0}
\definecolor{turquoise}{RGB}{0, 153, 153}
\definecolor{rouge}{RGB}{255, 0, 0}
\definecolor{lightvert}{RGB}{205, 234, 190}
\setitemize[0]{label=\color{lightvert}  $\bullet$}

\pagestyle{fancy}
\renewcommand{\headrulewidth}{0.2pt}
\fancyhead[L]{maths-cours.fr}
\fancyhead[R]{\thepage}
\renewcommand{\footrulewidth}{0.2pt}
\fancyfoot[C]{}

\newcolumntype{C}{>{\centering\arraybackslash}X}
\newcolumntype{s}{>{\hsize=.35\hsize\arraybackslash}X}

\setlength{\parindent}{0pt}		 
\setlength{\parskip}{3mm}
\setlength{\headheight}{1cm}

\def\ebook{ebook}
\def\book{book}
\def\web{web}
\def\type{web}

\newcommand{\vect}[1]{\overrightarrow{\,\mathstrut#1\,}}

\def\Oij{$\left(\text{O}~;~\vect{\imath},~\vect{\jmath}\right)$}
\def\Oijk{$\left(\text{O}~;~\vect{\imath},~\vect{\jmath},~\vect{k}\right)$}
\def\Ouv{$\left(\text{O}~;~\vect{u},~\vect{v}\right)$}

\hypersetup{breaklinks=true, colorlinks = true, linkcolor = OliveGreen, urlcolor = OliveGreen, citecolor = OliveGreen, pdfauthor={Didier BONNEL - https://www.maths-cours.fr} } % supprime les bordures autour des liens

\renewcommand{\arg}[0]{\text{arg}}

\everymath{\displaystyle}

%================================================================================================================================
%
% Macros - Commandes
%
%================================================================================================================================

\newcommand\meta[2]{    			% Utilisé pour créer le post HTML.
	\def\titre{titre}
	\def\url{url}
	\def\arg{#1}
	\ifx\titre\arg
		\newcommand\maintitle{#2}
		\fancyhead[L]{#2}
		{\Large\sffamily \MakeUppercase{#2}}
		\vspace{1mm}\textcolor{mcvert}{\hrule}
	\fi 
	\ifx\url\arg
		\fancyfoot[L]{\href{https://www.maths-cours.fr#2}{\black \footnotesize{https://www.maths-cours.fr#2}}}
	\fi 
}


\newcommand\TitreC[1]{    		% Titre centré
     \needspace{3\baselineskip}
     \begin{center}\textbf{#1}\end{center}
}

\newcommand\newpar{    		% paragraphe
     \par
}

\newcommand\nosp {    		% commande vide (pas d'espace)
}
\newcommand{\id}[1]{} %ignore

\newcommand\boite[2]{				% Boite simple sans titre
	\vspace{5mm}
	\setlength{\fboxrule}{0.2mm}
	\setlength{\fboxsep}{5mm}	
	\fcolorbox{#1}{#1!3}{\makebox[\linewidth-2\fboxrule-2\fboxsep]{
  		\begin{minipage}[t]{\linewidth-2\fboxrule-4\fboxsep}\setlength{\parskip}{3mm}
  			 #2
  		\end{minipage}
	}}
	\vspace{5mm}
}

\newcommand\CBox[4]{				% Boites
	\vspace{5mm}
	\setlength{\fboxrule}{0.2mm}
	\setlength{\fboxsep}{5mm}
	
	\fcolorbox{#1}{#1!3}{\makebox[\linewidth-2\fboxrule-2\fboxsep]{
		\begin{minipage}[t]{1cm}\setlength{\parskip}{3mm}
	  		\textcolor{#1}{\LARGE{#2}}    
 	 	\end{minipage}  
  		\begin{minipage}[t]{\linewidth-2\fboxrule-4\fboxsep}\setlength{\parskip}{3mm}
			\raisebox{1.2mm}{\normalsize\sffamily{\textcolor{#1}{#3}}}						
  			 #4
  		\end{minipage}
	}}
	\vspace{5mm}
}

\newcommand\cadre[3]{				% Boites convertible html
	\par
	\vspace{2mm}
	\setlength{\fboxrule}{0.1mm}
	\setlength{\fboxsep}{5mm}
	\fcolorbox{#1}{white}{\makebox[\linewidth-2\fboxrule-2\fboxsep]{
  		\begin{minipage}[t]{\linewidth-2\fboxrule-4\fboxsep}\setlength{\parskip}{3mm}
			\raisebox{-2.5mm}{\sffamily \small{\textcolor{#1}{\MakeUppercase{#2}}}}		
			\par		
  			 #3
 	 		\end{minipage}
	}}
		\vspace{2mm}
	\par
}

\newcommand\bloc[3]{				% Boites convertible html sans bordure
     \needspace{2\baselineskip}
     {\sffamily \small{\textcolor{#1}{\MakeUppercase{#2}}}}    
		\par		
  			 #3
		\par
}

\newcommand\CHelp[1]{
     \CBox{Plum}{\faInfoCircle}{À RETENIR}{#1}
}

\newcommand\CUp[1]{
     \CBox{NavyBlue}{\faThumbsOUp}{EN PRATIQUE}{#1}
}

\newcommand\CInfo[1]{
     \CBox{Sepia}{\faArrowCircleRight}{REMARQUE}{#1}
}

\newcommand\CRedac[1]{
     \CBox{PineGreen}{\faEdit}{BIEN R\'EDIGER}{#1}
}

\newcommand\CError[1]{
     \CBox{Red}{\faExclamationTriangle}{ATTENTION}{#1}
}

\newcommand\TitreExo[2]{
\needspace{4\baselineskip}
 {\sffamily\large EXERCICE #1\ (\emph{#2 points})}
\vspace{5mm}
}

\newcommand\img[2]{
          \includegraphics[width=#2\paperwidth]{\imgdir#1}
}

\newcommand\imgsvg[2]{
       \begin{center}   \includegraphics[width=#2\paperwidth]{\imgsvgdir#1} \end{center}
}


\newcommand\Lien[2]{
     \href{#1}{#2 \tiny \faExternalLink}
}
\newcommand\mcLien[2]{
     \href{https~://www.maths-cours.fr/#1}{#2 \tiny \faExternalLink}
}

\newcommand{\euro}{\eurologo{}}

%================================================================================================================================
%
% Macros - Environement
%
%================================================================================================================================

\newenvironment{tex}{ %
}
{%
}

\newenvironment{indente}{ %
	\setlength\parindent{10mm}
}

{
	\setlength\parindent{0mm}
}

\newenvironment{corrige}{%
     \needspace{3\baselineskip}
     \medskip
     \textbf{\textsc{Corrigé}}
     \medskip
}
{
}

\newenvironment{extern}{%
     \begin{center}
     }
     {
     \end{center}
}

\NewEnviron{code}{%
	\par
     \boite{gray}{\texttt{%
     \BODY
     }}
     \par
}

\newenvironment{vbloc}{% boite sans cadre empeche saut de page
     \begin{minipage}[t]{\linewidth}
     }
     {
     \end{minipage}
}
\NewEnviron{h2}{%
    \needspace{3\baselineskip}
    \vspace{0.6cm}
	\noindent \MakeUppercase{\sffamily \large \BODY}
	\vspace{1mm}\textcolor{mcgris}{\hrule}\vspace{0.4cm}
	\par
}{}

\NewEnviron{h3}{%
    \needspace{3\baselineskip}
	\vspace{5mm}
	\textsc{\BODY}
	\par
}

\NewEnviron{margeneg}{ %
\begin{addmargin}[-1cm]{0cm}
\BODY
\end{addmargin}
}

\NewEnviron{html}{%
}

\begin{document}
\meta{url}{/cours/les-ensembles-de-nombres/}
\meta{pid}{10887}
\meta{titre}{Ensembles de nombres - Intervalles - Valeurs absolues}
\meta{type}{cours}
\begin{h2}I - Les ensembles de nombres \end{h2}

\cadre{bleu}{Définition}{%
     $\mathbb{N}=\left\{0; 1; 2; 3; 4; 5; \cdots \right\}$ est l'ensemble des \textbf{entiers naturels}.
}
\bloc{cyan}{Remarque}{% 
     On emploie le signe $ \in $ pour indiquer qu'un nombre appartient à un ensemble. On écrira par exemple: $2\in \mathbb{N}$ et $\frac{2}{3} \notin \mathbb{N}$. 
}
\cadre{bleu}{Définition}{%
     $\mathbb{Z}= \left\{\cdots;  -3; -2 ; -1; 0; 1; 2; 3; \cdots \right\}$ est l'ensemble des \textbf{entiers relatifs}. 
}
\cadre{bleu}{Définition}{%
     $\mathbb{D}$ est l'ensemble des \textbf{nombres décimaux}. Les nombres décimaux peuvent s'écrire sous la forme d'une fraction dont le dénominateur est une puissance de 10 (1; 10; 100; 1 000; ...). \\
     Ils peuvent aussi s'écrire sous forme décimale dont le nombre de chiffres après la virgule est finie.
}

\bloc{cyan}{Remarques}{%
     \begin{itemize}
          \item \textit{"fini"} signifie ici  \og qui n'est pas infini \fg{}. 
          \item Les calculatrices les plus simples ne manipulent que des nombres décimaux pour effectuer les calculs. Certaines permettent des opérations sur les fractions. Quelques modèles plus avancés (effectuant du "calcul formel") peuvent également effectuer des calculs avec des nombres irrationnels. 
     \end{itemize}
}
\cadre{bleu}{Définition}{%
     $\mathbb{Q}$ est l'ensemble des \textbf{nombres rationnels}. Les nombres rationnels peuvent s'écrire sous la forme d'une fraction dont le numérateur et le dénominateur sont des entiers relatifs.
}
\cadre{bleu}{Définition}{%
     $\mathbb{R}$ est l'ensemble des \textbf{nombres réels}. Les nombres réels sont tous les nombres connus (en Seconde...). 

}
\bloc{cyan}{Remarque}{%
     Les nombres réels qui ne sont pas rationnels (comme $\pi $ ou $\sqrt{2}$ ) sont appelés des nombres \textbf{irrationnels}.
}
\cadre{vert}{Propriété}{%
     $\mathbb{N} \subset \mathbb{Z} \subset \mathbb{D} \subset \mathbb{Q} \subset \mathbb{R}$.
}
\begin{center}
     \begin{extern} %width="500" alt=" représentation graphique d'une fonction"
          \resizebox{8cm}{!}{%
               \newrgbcolor{wwqqcc}{0.4 0. 0.8}
               \newrgbcolor{yqqqqq}{0.50 0. 0.}
               \newrgbcolor{qqwuqq}{0. 0.39 0.}
               \newrgbcolor{qqqqcc}{0. 0. 0.8}
               \newrgbcolor{ffqqtt}{1. 0. 0.2}
               \psset{xunit=1.0cm,yunit=1.0cm,algebraic=true,dimen=middle}
               \begin{pspicture*}(0.58,-17.18)(23.54,-3.1)
                    \fontsize{16pt}{16.1pt}\selectfont
                    \rput{0.}(10.,-10.){\psellipse[linewidth=2.pt,linecolor=wwqqcc](0,0)(3.,2.23)}
                    \rput{0.}(10.5,-10.){\psellipse[linewidth=2.pt,linecolor=yqqqqq](0,0)(4.54,2.90)}
                    \rput{0.}(10.97,-10.){\psellipse[linewidth=2.pt,linecolor=qqwuqq](0,0)(6.24,3.71)}
                    \rput{-0.17}(11.45,-10.02){\psellipse[linewidth=2.pt,linecolor=qqqqcc](0,0)(8.32,4.90)}
                    \rput{0}(12.064353510453511,-10.0){\psellipse[linewidth=2.pt,linecolor=ffqqtt](0,0)(11.03,6.46)}
                    \rput[tl](9.68,-8.8){$\wwqqcc{\mathbb{N}}$}
                    \rput[tl](13.64,-9.56){$\yqqqqq{\mathbb{Z}}$}
                    \rput[tl](15.76,-8.84){$\qqwuqq{\mathbb{D}}$}
                    \rput[tl](18.28,-9.52){$\qqqqcc{\mathbb{Q}}$}
                    \rput[tl](20.92,-9.52){$\ffqqtt{\mathbb{R}}$}
                    \rput[tl](8.28,-10.06){$0$}
                    \rput[tl](11.24,-10.7){$6$}
                    \rput[tl](12.36,-7.8){$-1$}
                    \rput[tl](13.2,-10.66){$-23$}
                    \rput[tl](15.32,-8.1){$1,5$}
                    \rput[tl](15,-10.86){$-\frac{3}{4}$}
                    \rput[tl](17,-11.82){$\frac{4}{3}$}
                    \rput[tl](16.88,-7.02){$-\frac{1}{7}$}
                    \rput[tl](20.04,-6.62){$\sqrt{2}$}
                    \rput[tl](19.2,-12.82){$\pi $}
               \end{pspicture*}
          }
     \end{extern}
\end{center}
\bloc{cyan}{Remarques}{%
     \begin{itemize}
          \item Le symbole $ \subset $ se lit \textit{"inclus dans"}.
          \item La proposition précédente signifie que tous les entiers naturels sont aussi des entiers relatifs qui sont eux-même des nombres décimaux qui sont des nombres rationnels qui sont des nombres réels.
          \item Un même nombre admet plusieurs écritures différentes. Par exemple le nombre 2 peut aussi s'écrire 2,0 (écriture décimale) $\frac{2}{1}$ ou $\frac{4}{2}$ etc. (écriture fractionnaire) $\sqrt{4}$ (écriture avec un radical) et même (aussi curieux que cela puisse vous paraitre) 1,999999.... (écriture décimale illimitée).
     \end{itemize}
}

\begin{h2}II - Intervalles \end{h2}

\begin{h3}Intervalles bornés\end{h3}
\cadre{bleu}{Définition}{ % id=d50 
Soient $a$ et $b$ deux nombres réels tels que $ a < b $.
\begin{itemize}
\item
L'intervalle  \textbf{fermé} $  \left[ a~;~b \right]  $ est l'ensemble des nombres réels $x$ tels que  $ a  \leqslant x  \leqslant b. $
\item
L'intervalle  \textbf{ouvert} $  \left] a~;~b \right[  $ est l'ensemble des nombres réels $x$ tels que  $ a  < x  < b. $
\item
L'intervalle  $  \left[ a~;~b \right[  $ (fermé en $a$, ouvert en $b$) est l'ensemble des nombres réels $x$ tels que  $ a   \leqslant  x  < b. $
\item
L'intervalle  $  \left] a~;~b \right]  $ (ouvert en $a$, fermé  en $b$) est l'ensemble des nombres réels $x$ tels que  $ a   < x  \leqslant  b.$
\end{itemize}
} % fin définition

\bloc{orange}{Exemple}{ % id=e55
Par exemple, l'intervalle $  \left[ -2~;~3 \right[  $ est constitué des nombres réels qui sont à la fois supérieur ou égal à $  -2$ et strictement inférieur à $3$.
\par
On pourra, par exemple, écrire~:

\begin{itemize}
\item
 $ -3  \notin \left[ -2~;~3 \right[  $
\item
 $ -2  \in \left[ -2~;~3 \right[  $
\item
 $ 0  \in \left[ -2~;~3 \right[  $
\item
 $ 3  \notin \left[ -2~;~3 \right[  $
\item
 $ 4  \notin \left[ -2~;~3 \right[  $
\end{itemize}
\medskip
On peut représenter l’intervalle  $  \left[ -2~;~3 \right[  $ de la façon suivante~:
\begin{center} 
\begin{extern}%alt="Intervalle ouvert-fermé" model="modeles-graphiques"
\psset{xunit=1.0cm,yunit=1.0cm,algebraic=true,dimen=middle,dotstyle=*,dotsize=4pt 0,linewidth=.5pt,arrowsize=3pt 2,arrowinset=0.25}
\begin{pspicture*}(-4.5,-0.75)(4.5,0.75)
\psaxes[labelFontSize=\small,xAxis=true,yAxis=false,Dx=1.,Dy=1.,ticksize=-2pt 0,subticks=2]{->}(0,0)(-4.5,-0.75)(4.5,0.75)
\rput[tl](-0.1,-0.25){$\small 0$}  
\psline[linewidth=1.2pt,linecolor=red](-2.,0.)(3.,0.)
\rput[tl](-2.07,0.2){$\red\Large\textbf{[}$}
\rput[tl](2.93,0.2){$\red\Large\textbf{[}$} 
\end{pspicture*}        
 \end{extern}
\end{center}
          
} % fin exemple

\begin{h3}Intervalles non bornés\end{h3}
\cadre{bleu}{Définition}{ % id=d60 
Soit $a$ un nombre réel.
\begin{itemize}
\item
L'intervalle  $\left[ a~;~+\infty   \right[ $ est l'ensemble des nombres réels $x$ tels que  $  x   \geqslant a. $
\item
L'intervalle $\left] a~;~+\infty   \right[ $  est l'ensemble des nombres réels $x$ tels que  $ x >  a. $
\item
L'intervalle $\left] -\infty~;~a \right] $  est l'ensemble des nombres réels $x$ tels que  $ x  \leqslant   a. $
\item
L'intervalle $\left] -\infty~;~a \right] $  est l'ensemble des nombres réels $x$ tels que  $ x  <   a. $
\end{itemize}
} % fin définition

\bloc{cyan}{Remarque}{ % id=r65 
En $  +\infty  $ et en $  -\infty  $, le crochet est toujours  \textbf{ouvert}.
} % fin remarque@
\bloc{orange}{Exemple}{ % id=e67

\begin{itemize}
\item
$ 0  \notin  \left[ 1~;~ +\infty  \right[  $
\item
$ 1 \in  \left[ 1~;~ +\infty  \right[  $
\item
$ 100  \in  \left[ 1~;~ +\infty  \right[  $
\end{itemize}
\medskip
On représente l’intervalle  $ \left[ 1~;~ +\infty  \right[  $ ainsi~:
\begin{center} 
\begin{extern}%alt="Intervalle non borné" model="modeles-graphiques"
\psset{xunit=1.0cm,yunit=1.0cm,algebraic=true,dimen=middle,dotstyle=*,dotsize=4pt 0,linewidth=.5pt,arrowsize=3pt 2,arrowinset=0.25}
\begin{pspicture*}(-4.5,-0.75)(4.5,0.75)
\psaxes[labelFontSize=\small,xAxis=true,yAxis=false,Dx=1.,Dy=1.,ticksize=-2pt 0,subticks=2]{->}(0,0)(-4.5,-0.75)(4.5,0.75)
\rput[tl](-0.1,-0.25){$\small 0$}  
\psline[linewidth=1.2pt,linecolor=red](1.,0.)(4.5,0.)
\rput[tl](0.95,0.2){$\red\Large\textbf{[}$}
\end{pspicture*}        
 \end{extern}
\end{center}
} % fin exemple

\begin{h3}Union et intersection\end{h3}

\cadre{bleu}{Définition}{ % id=d70
Soient $I$ et $J$ deux intervalles.

\begin{itemize}
\item
L'\textbf{intersection} de $I$ et de $J$ notée $ I  \cap J $ (lire  \og $I$ inter $J$  \fg{} ) est l'ensemble des nombres appartenant à la fois à $I$  \textbf{et} à $J$.

\item
L'\textbf{union} (ou la  \textbf{réunion}  de $I$ et de $J$ notée $ I  \cup J $ (lire  \og $I$ union $J$  \fg{} ) est l'ensemble des nombres appartenant à $I$  \textbf{ou} à $J$ \textbf{ou} aux deux intervalles.
\end{itemize} 
} % fin définition

\bloc{cyan}{Remarque}{ % id=r73
Retenir que l'\textbf{intersection} correspond au mot  \og  \textbf{et} \fg{} et que la  \textbf{réunion} correspond au mot  \og  \textbf{ou}\fg{}. 
} % fin remarque

\bloc{orange}{Exemple}{ % id=e75 
Si $ I =  \left[ -3~;~1 \right[  $ et $ J=  \left[ 0~;~3 \right] $, $I$ est représenté en bleu et $J$ en rouge sur la figure suivante~:
\\
\begin{center} 
\begin{extern}%alt="Intervalle union intersection" model="modeles-graphiques"
\psset{xunit=1.0cm,yunit=1.0cm,algebraic=true,dimen=middle,dotstyle=*,dotsize=4pt 0,linewidth=.5pt,arrowsize=3pt 2,arrowinset=0.25}
\begin{pspicture*}(-4.5,-0.75)(4.5,0.75)
\psaxes[labelFontSize=\small,xAxis=true,yAxis=false,Dx=1.,Dy=1.,ticksize=-2pt 0,subticks=2]{->}(0,0)(-4.5,-0.75)(4.5,0.75)
\rput[tl](-0.1,-0.25){$\small 0$}  
\psline[linewidth=1.2pt,linecolor=blue](-3.,-0.03)(1.,-0.03)
\rput[tl](-3.07,0.2){$\blue\Large\textbf{[}$}
\rput[tl](0.93,0.2){$\blue\Large\textbf{[}$} 
\psline[linewidth=1.2pt,linecolor=red](0.,0.03)(3.,0.03)
\rput[tl](-0.07,0.2){$\red\Large\textbf{[}$}
\rput[tl](2.93,0.2){$\red\Large\textbf{]}$} 
\end{pspicture*}        
 \end{extern}
\end{center}
} % fin exemple

\begin{center}
$ I  \cap J =  \left[ 0~;~1 \right[  $ et $ I  \cup J =  \left[ -3~;~3 \right] $ 
\end{center}

\begin{h2}III - Valeurs absolues \end{h2}
Intuitivement, la valeur absolue d'un nombre c'est  \og le nombre sans son signe. \fg{}. Par exemple, la valeur absolue de $ -5 $ est $5$ et la valeur absolue de $ 1,12 $ est $ 1,12 $. Toutefois, lors de calculs littéraux, le signe peut être  \og caché \fg{} à l'intérieur de la lettre~; par exemple, on ne peut pas dire que la valeur absolue de $ -x $ est égale à $ x $ car c'est faux si $x$ est négatif. D'où la définition suivante~:
\cadre{bleu}{Définition}{% id=d100 
Soit $x$ un nombre réel
     On appelle \textbf{valeur absolue} de $x$ et on note $ |x|$ le nombre réel positif ou nul défini par
     \begin{itemize}
          \item $|x|$ = $x$ si $x$ est positif ou nul,
          \item $|x|$ = $-x$ si $x$ est négatif ou nul.
     \end{itemize}
}

\bloc{orange}{Exemples}{ % id=e105 

\begin{itemize}
\item
$| -1 | = -(-1) = 1$ 
\item
$| \sqrt{ 2 } - 1 | = \sqrt{ 2 } - 1$ car $ \sqrt{ 2 } > 1 $ donc $  \sqrt{ 2 } - 1  $ est positif.
\end{itemize}
} % fin exemple
\cadre{vert}{Propriété}{% id=p110
     La distance entre les nombres réels $x$ et $y$ est égale à $|y-x|$ (ou aussi à $|x-y|$).
\par
En particulier, $  \left| x \right|  $ est la distance de $x$ à $0$.
}
\bloc{orange}{Exemple}{% id=e115 
Les nombres $ -3 $ et $2$ sont représentés sur l'axe ci-dessous~:
 \begin{center} 
\begin{extern}% alt="distance et valeur absolue"
\psset{xunit=1.0cm,yunit=1.0cm,algebraic=true,dimen=middle,dotstyle=*,dotsize=3pt 0,linewidth=.5pt,arrowsize=3pt 2,arrowinset=0.25}
\begin{pspicture*}(-4.5,-0.75)(4.5,0.75)
\psaxes[labelFontSize=\small,xAxis=true,yAxis=false,Dx=1.,Dy=1.,ticksize=-2pt 0,subticks=2]{->}(0,0)(-4.5,-0.75)(4.5,0.75)
\rput[tl](-0.1,-0.25){$\small 0$}  
\psline[linewidth=1.2pt,linecolor=blue](-3,0)(2.,0.)
\psdots[linecolor=blue](-3,0)(2.,0.)
\end{pspicture*}        
 \end{extern}
\end{center}

La distance entre $ -3 $ et $ 2 $ est égale à~:
\begin{center}
     $|-3-2|=|-5|=5$
\end{center}
\par
La distance entre $ -3 $ et $ 0 $ est égale à~:
\begin{center}
     $|-3|=3$
\end{center}
}

\end{document}

µ
\documentclass[a4paper]{article}

%================================================================================================================================
%
% Packages
%
%================================================================================================================================

\usepackage[T1]{fontenc} 	% pour caractères accentués
\usepackage[utf8]{inputenc}  % encodage utf8
\usepackage[french]{babel}	% langue : français
\usepackage{fourier}			% caractères plus lisibles
\usepackage[dvipsnames]{xcolor} % couleurs
\usepackage{fancyhdr}		% réglage header footer
\usepackage{needspace}		% empêcher sauts de page mal placés
\usepackage{graphicx}		% pour inclure des graphiques
\usepackage{enumitem,cprotect}		% personnalise les listes d'items (nécessaire pour ol, al ...)
\usepackage{hyperref}		% Liens hypertexte
\usepackage{pstricks,pst-all,pst-node,pstricks-add,pst-math,pst-plot,pst-tree,pst-eucl} % pstricks
\usepackage[a4paper,includeheadfoot,top=2cm,left=3cm, bottom=2cm,right=3cm]{geometry} % marges etc.
\usepackage{comment}			% commentaires multilignes
\usepackage{amsmath,environ} % maths (matrices, etc.)
\usepackage{amssymb,makeidx}
\usepackage{bm}				% bold maths
\usepackage{tabularx}		% tableaux
\usepackage{colortbl}		% tableaux en couleur
\usepackage{fontawesome}		% Fontawesome
\usepackage{environ}			% environment with command
\usepackage{fp}				% calculs pour ps-tricks
\usepackage{multido}			% pour ps tricks
\usepackage[np]{numprint}	% formattage nombre
\usepackage{tikz,tkz-tab} 			% package principal TikZ
\usepackage{pgfplots}   % axes
\usepackage{mathrsfs}    % cursives
\usepackage{calc}			% calcul taille boites
\usepackage[scaled=0.875]{helvet} % font sans serif
\usepackage{svg} % svg
\usepackage{scrextend} % local margin
\usepackage{scratch} %scratch
\usepackage{multicol} % colonnes
%\usepackage{infix-RPN,pst-func} % formule en notation polanaise inversée
\usepackage{listings}

%================================================================================================================================
%
% Réglages de base
%
%================================================================================================================================

\lstset{
language=Python,   % R code
literate=
{á}{{\'a}}1
{à}{{\`a}}1
{ã}{{\~a}}1
{é}{{\'e}}1
{è}{{\`e}}1
{ê}{{\^e}}1
{í}{{\'i}}1
{ó}{{\'o}}1
{õ}{{\~o}}1
{ú}{{\'u}}1
{ü}{{\"u}}1
{ç}{{\c{c}}}1
{~}{{ }}1
}


\definecolor{codegreen}{rgb}{0,0.6,0}
\definecolor{codegray}{rgb}{0.5,0.5,0.5}
\definecolor{codepurple}{rgb}{0.58,0,0.82}
\definecolor{backcolour}{rgb}{0.95,0.95,0.92}

\lstdefinestyle{mystyle}{
    backgroundcolor=\color{backcolour},   
    commentstyle=\color{codegreen},
    keywordstyle=\color{magenta},
    numberstyle=\tiny\color{codegray},
    stringstyle=\color{codepurple},
    basicstyle=\ttfamily\footnotesize,
    breakatwhitespace=false,         
    breaklines=true,                 
    captionpos=b,                    
    keepspaces=true,                 
    numbers=left,                    
xleftmargin=2em,
framexleftmargin=2em,            
    showspaces=false,                
    showstringspaces=false,
    showtabs=false,                  
    tabsize=2,
    upquote=true
}

\lstset{style=mystyle}


\lstset{style=mystyle}
\newcommand{\imgdir}{C:/laragon/www/newmc/assets/imgsvg/}
\newcommand{\imgsvgdir}{C:/laragon/www/newmc/assets/imgsvg/}

\definecolor{mcgris}{RGB}{220, 220, 220}% ancien~; pour compatibilité
\definecolor{mcbleu}{RGB}{52, 152, 219}
\definecolor{mcvert}{RGB}{125, 194, 70}
\definecolor{mcmauve}{RGB}{154, 0, 215}
\definecolor{mcorange}{RGB}{255, 96, 0}
\definecolor{mcturquoise}{RGB}{0, 153, 153}
\definecolor{mcrouge}{RGB}{255, 0, 0}
\definecolor{mclightvert}{RGB}{205, 234, 190}

\definecolor{gris}{RGB}{220, 220, 220}
\definecolor{bleu}{RGB}{52, 152, 219}
\definecolor{vert}{RGB}{125, 194, 70}
\definecolor{mauve}{RGB}{154, 0, 215}
\definecolor{orange}{RGB}{255, 96, 0}
\definecolor{turquoise}{RGB}{0, 153, 153}
\definecolor{rouge}{RGB}{255, 0, 0}
\definecolor{lightvert}{RGB}{205, 234, 190}
\setitemize[0]{label=\color{lightvert}  $\bullet$}

\pagestyle{fancy}
\renewcommand{\headrulewidth}{0.2pt}
\fancyhead[L]{maths-cours.fr}
\fancyhead[R]{\thepage}
\renewcommand{\footrulewidth}{0.2pt}
\fancyfoot[C]{}

\newcolumntype{C}{>{\centering\arraybackslash}X}
\newcolumntype{s}{>{\hsize=.35\hsize\arraybackslash}X}

\setlength{\parindent}{0pt}		 
\setlength{\parskip}{3mm}
\setlength{\headheight}{1cm}

\def\ebook{ebook}
\def\book{book}
\def\web{web}
\def\type{web}

\newcommand{\vect}[1]{\overrightarrow{\,\mathstrut#1\,}}

\def\Oij{$\left(\text{O}~;~\vect{\imath},~\vect{\jmath}\right)$}
\def\Oijk{$\left(\text{O}~;~\vect{\imath},~\vect{\jmath},~\vect{k}\right)$}
\def\Ouv{$\left(\text{O}~;~\vect{u},~\vect{v}\right)$}

\hypersetup{breaklinks=true, colorlinks = true, linkcolor = OliveGreen, urlcolor = OliveGreen, citecolor = OliveGreen, pdfauthor={Didier BONNEL - https://www.maths-cours.fr} } % supprime les bordures autour des liens

\renewcommand{\arg}[0]{\text{arg}}

\everymath{\displaystyle}

%================================================================================================================================
%
% Macros - Commandes
%
%================================================================================================================================

\newcommand\meta[2]{    			% Utilisé pour créer le post HTML.
	\def\titre{titre}
	\def\url{url}
	\def\arg{#1}
	\ifx\titre\arg
		\newcommand\maintitle{#2}
		\fancyhead[L]{#2}
		{\Large\sffamily \MakeUppercase{#2}}
		\vspace{1mm}\textcolor{mcvert}{\hrule}
	\fi 
	\ifx\url\arg
		\fancyfoot[L]{\href{https://www.maths-cours.fr#2}{\black \footnotesize{https://www.maths-cours.fr#2}}}
	\fi 
}


\newcommand\TitreC[1]{    		% Titre centré
     \needspace{3\baselineskip}
     \begin{center}\textbf{#1}\end{center}
}

\newcommand\newpar{    		% paragraphe
     \par
}

\newcommand\nosp {    		% commande vide (pas d'espace)
}
\newcommand{\id}[1]{} %ignore

\newcommand\boite[2]{				% Boite simple sans titre
	\vspace{5mm}
	\setlength{\fboxrule}{0.2mm}
	\setlength{\fboxsep}{5mm}	
	\fcolorbox{#1}{#1!3}{\makebox[\linewidth-2\fboxrule-2\fboxsep]{
  		\begin{minipage}[t]{\linewidth-2\fboxrule-4\fboxsep}\setlength{\parskip}{3mm}
  			 #2
  		\end{minipage}
	}}
	\vspace{5mm}
}

\newcommand\CBox[4]{				% Boites
	\vspace{5mm}
	\setlength{\fboxrule}{0.2mm}
	\setlength{\fboxsep}{5mm}
	
	\fcolorbox{#1}{#1!3}{\makebox[\linewidth-2\fboxrule-2\fboxsep]{
		\begin{minipage}[t]{1cm}\setlength{\parskip}{3mm}
	  		\textcolor{#1}{\LARGE{#2}}    
 	 	\end{minipage}  
  		\begin{minipage}[t]{\linewidth-2\fboxrule-4\fboxsep}\setlength{\parskip}{3mm}
			\raisebox{1.2mm}{\normalsize\sffamily{\textcolor{#1}{#3}}}						
  			 #4
  		\end{minipage}
	}}
	\vspace{5mm}
}

\newcommand\cadre[3]{				% Boites convertible html
	\par
	\vspace{2mm}
	\setlength{\fboxrule}{0.1mm}
	\setlength{\fboxsep}{5mm}
	\fcolorbox{#1}{white}{\makebox[\linewidth-2\fboxrule-2\fboxsep]{
  		\begin{minipage}[t]{\linewidth-2\fboxrule-4\fboxsep}\setlength{\parskip}{3mm}
			\raisebox{-2.5mm}{\sffamily \small{\textcolor{#1}{\MakeUppercase{#2}}}}		
			\par		
  			 #3
 	 		\end{minipage}
	}}
		\vspace{2mm}
	\par
}

\newcommand\bloc[3]{				% Boites convertible html sans bordure
     \needspace{2\baselineskip}
     {\sffamily \small{\textcolor{#1}{\MakeUppercase{#2}}}}    
		\par		
  			 #3
		\par
}

\newcommand\CHelp[1]{
     \CBox{Plum}{\faInfoCircle}{À RETENIR}{#1}
}

\newcommand\CUp[1]{
     \CBox{NavyBlue}{\faThumbsOUp}{EN PRATIQUE}{#1}
}

\newcommand\CInfo[1]{
     \CBox{Sepia}{\faArrowCircleRight}{REMARQUE}{#1}
}

\newcommand\CRedac[1]{
     \CBox{PineGreen}{\faEdit}{BIEN R\'EDIGER}{#1}
}

\newcommand\CError[1]{
     \CBox{Red}{\faExclamationTriangle}{ATTENTION}{#1}
}

\newcommand\TitreExo[2]{
\needspace{4\baselineskip}
 {\sffamily\large EXERCICE #1\ (\emph{#2 points})}
\vspace{5mm}
}

\newcommand\img[2]{
          \includegraphics[width=#2\paperwidth]{\imgdir#1}
}

\newcommand\imgsvg[2]{
       \begin{center}   \includegraphics[width=#2\paperwidth]{\imgsvgdir#1} \end{center}
}


\newcommand\Lien[2]{
     \href{#1}{#2 \tiny \faExternalLink}
}
\newcommand\mcLien[2]{
     \href{https~://www.maths-cours.fr/#1}{#2 \tiny \faExternalLink}
}

\newcommand{\euro}{\eurologo{}}

%================================================================================================================================
%
% Macros - Environement
%
%================================================================================================================================

\newenvironment{tex}{ %
}
{%
}

\newenvironment{indente}{ %
	\setlength\parindent{10mm}
}

{
	\setlength\parindent{0mm}
}

\newenvironment{corrige}{%
     \needspace{3\baselineskip}
     \medskip
     \textbf{\textsc{Corrigé}}
     \medskip
}
{
}

\newenvironment{extern}{%
     \begin{center}
     }
     {
     \end{center}
}

\NewEnviron{code}{%
	\par
     \boite{gray}{\texttt{%
     \BODY
     }}
     \par
}

\newenvironment{vbloc}{% boite sans cadre empeche saut de page
     \begin{minipage}[t]{\linewidth}
     }
     {
     \end{minipage}
}
\NewEnviron{h2}{%
    \needspace{3\baselineskip}
    \vspace{0.6cm}
	\noindent \MakeUppercase{\sffamily \large \BODY}
	\vspace{1mm}\textcolor{mcgris}{\hrule}\vspace{0.4cm}
	\par
}{}

\NewEnviron{h3}{%
    \needspace{3\baselineskip}
	\vspace{5mm}
	\textsc{\BODY}
	\par
}

\NewEnviron{margeneg}{ %
\begin{addmargin}[-1cm]{0cm}
\BODY
\end{addmargin}
}

\NewEnviron{html}{%
}

\begin{document}
\meta{url}{/cours/fonction-exponentielle-1ere/}
\meta{pid}{10903}
\meta{titre}{Fonction exponentielle}
\meta{type}{cours}
\begin{h2}1. Définition de la fonction exponentielle\end{h2}
\cadre{rouge}{Théorème et Définition}{% id="t10"
     Il existe une unique fonction $f$ dérivable sur $\mathbb{R}$ telle que $f^{\prime}=f$ et $f\left(0\right)=1$
     \par
     Cette fonction est appelée \textbf{fonction exponentielle} (de base e) et notée $\text{exp}$.
}
\bloc{cyan}{Remarque}{% id="r10"
     L'existence d'une telle fonction est admise.
     \par
     Son unicité est démontrée dans l'exercice~: \mcLien{/exercices/logarithme-exponentielle/roc-proprietes-fondamentales-exponentielle}{[ROC] Propriétés fondamentales de la fonction exponentielle}
}
\cadre{bleu}{Notation}{% id="n20"
     On note $\text{e}=\text{exp}\left(1\right)$.
     \par
     On démontre que pour tout entier relatif $n \in \mathbb{Z}$~: $\text{exp}\left(n\right)=\text{e}^{n}$
     \par
     Cette propriété conduit à noter $\text{e}^{x}$ l'exponentielle de $x$ pour tout $x \in \mathbb{R}$
}
\bloc{cyan}{Remarque}{% id="r20"
     On démontre (mais c'est hors programme) que $\text{e} \left(\approx 2,71828 . . . \right)$ est un nombre \textbf{irrationnel}, c'est à dire qu'il ne peut s'écrire sous forme de fraction.
}
\begin{h2}2. Etude de la fonction exponentielle\end{h2}
\cadre{vert}{Propriété}{% id="p30"
     La fonction exponentielle est \textbf{strictement positive} et \textbf{strictement croissante} sur $\mathbb{R}$.
}
\bloc{cyan}{Remarque}{% id="r30"
     Cette propriété très importante est démontrée dans l'exercice~: \mcLien{/exercices/logarithme-exponentielle/roc-proprietes-fondamentales-exponentielle}{[ROC] Propriétés fondamentales de la fonction exponentielle}
}
\cadre{vert}{Propriété}{% id="p35"
     Soit $u$ une fonction dérivable sur un intervalle $I$.
     \par
     Alors la fonction $ f~: x\mapsto \text{e}^{u\left(x\right)}$ est dérivable sur $I$ et~:
     \begin{center}$f^{\prime}=u^{\prime}\text{e}^{u}$\end{center}
}
\bloc{cyan}{Démonstration}{% id="r35"
     On utilise le \mcLien{/cours/terminale-s/fonctions-continues-2\#t100}{théorème de dérivation de fonctions composées}.
}
\bloc{orange}{Exemple}{% id="e35"
     Soit $f$ définie sur $\mathbb{R}$ par $f\left(x\right)=\text{e}^{-x}$
     \par
     $f$ est dérivable sur $\mathbb{R}$ et $f^{\prime}\left(x\right)=-\text{e}^{-x}$
}
\cadre{vert}{Limites}{% id="t40"
     \begin{itemize}
          \item $\lim\limits_{x\rightarrow -\infty }\text{e}^{x}=0$
          \item $\lim\limits_{x\rightarrow +\infty }\text{e}^{x}=+\infty $
     \end{itemize}
}
\bloc{cyan}{Remarques}{% id="r40"
     \begin{itemize}
          \item Ces résultats sont démontrés dans l'exercice~: \mcLien{/exercices/logarithme-exponentielle/roc-limites-exponentielle}{[ROC] Limites de la fonction exponentielle}
          \item On déduit des résultats précédents le tableau de variation et l'allure de la courbe de la fonction exponentielle:
     \end{itemize}
}
\begin{center}
     \begin{extern}%alt="Fonction exponentielle~: tableau de variation" style="width:50rem"
          \begin{tikzpicture}
               \tkzTabInit[lgt=3,espcl=10] {$x$ /1, $f'(x)=\text{e}^{x}$ /1.5,%
               $f(x)=\text{e}^{x}$/2} {$-\infty$ , $+\infty$}%
               \tkzTabLine{,+,}%
               \tkzTabVar{ - / $0$, + / $+\infty$ }
               \tkzTabVal[draw]{1}{2}{0.33}{0}{1}
               \tkzTabVal[draw]{1}{2}{0.66}{1}{e}
          \end{tikzpicture}
     \end{extern}
\end{center}
\begin{center}
     \textit{Tableau de variation de la fonction exponentielle }
\end{center}
\par
\begin{center}
     \begin{extern} %alt="Fonction exponentielle~: graphique" style="width:40rem"
          \resizebox{8cm}{!}{%
               % -+-+-+ variables modifiables
               \def\fonction{EXP(x) }
               \def\xmin{-3.5}
               \def\xmax{2.5}
               \def\ymin{-1.5}
               \def\ymax{5}
               \def\xunit{2} % unités en cm
               \def\yunit{2}
               \psset{xunit=\xunit,yunit=\yunit,algebraic=true}
               \fontsize{15pt}{15pt}\selectfont
               \begin{pspicture*}[linewidth=1pt](\xmin,\ymin)(\xmax,\ymax)
                    % \psgrid[gridcolor=mcgris, subgriddiv=5, gridlabels=0pt](\xmin,\ymin)(\xmax,\ymax)
                    \psaxes[linewidth=0.75pt]{->}(0,0)(\xmin,\ymin)(\xmax,\ymax)
                    \psplot[plotpoints=2000,linecolor=blue]{\xmin}{\xmax}{\fonction}
                    \rput[tr](-0.1,-0.1){$O$}
                    \rput[tl](1.5,4){$\color{blue} \mathcal{C}_{\text{exp}}$}
                    \psline(1,0)(1,2.7183)(0,2.7183)
                    \rput[r](-0.1,2.7183){$\text{e}$}
               \end{pspicture*}
          }
     \end{extern}
\end{center}
\begin{center}
     \textit{Graphique de la fonction exponentielle }
\end{center}
\cadre{rouge}{Théorème ( «Croissance comparée»)}{% id="t50"
     \begin{itemize}
          \item $\lim\limits_{x\rightarrow -\infty }x\text{e}^{x}=0$
          \item $\lim\limits_{x\rightarrow +\infty }\frac{\text{e}^{x}}{x}=+\infty $
          \item $\lim\limits_{x\rightarrow 0}\frac{\text{e}^{x}-1}{x}=1$
     \end{itemize}
}
\bloc{cyan}{Remarques}{% id="r50"
     \begin{itemize}
          \item Voir, à nouveau, l'exercice~: \mcLien{/exercices/logarithme-exponentielle/roc-limites-exponentielle}{[ROC] Limites de la fonction exponentielle} pour la démonstration des deux premières formules.
          \item Les deux premières formules peuvent se généraliser de la façon suivante~:
          \par
          Pour tout entier $n > 0$~:
          \par
          $ \lim\limits_{x\rightarrow -\infty }x^{n}\text{e}^{x}=0$
          \par
          $ \lim\limits_{x\rightarrow +\infty }\frac{\text{e}^{x}}{x^{n}}=+\infty $
          \item La troisième formule s'obtient en utilisant la définition du nombre dérivé pour x=0~: (voir \mcLien{/methodes/limites/calcul-limite-nombre-derive}{Calculer une limite à l'aide du nombre dérivé}).
          \par
          $\lim\limits_{x\rightarrow 0}\frac{\text{e}^{x}-1}{x}=\text{exp}^{\prime}\left(0\right)=\text{exp}\left(0\right)=1$
     \end{itemize}
}
\cadre{rouge}{Théorème}{% id="t60"
     La fonction exponentielle étant strictement \textbf{croissante}, si $a$ et $b$ sont deux réels~:
     \begin{itemize}
          \item $\text{e}^{a}=\text{e}^{b}$ si et seulement si $a=b$
          \item $\text{e}^{a} < \text{e}^{b}$ si et seulement si $ a < b $
     \end{itemize}
}
\bloc{cyan}{Remarque}{% id="r60"
     Ces résultats sont extrêmement utiles pour résoudre équations et inéquations.
}
\begin{h2}3. Propriétés algébriques de la fonction exponentielle\end{h2}
\cadre{vert}{Propriétés}{% id="p70"
     Pour tout réels $a$ et $b$ et tout entier $n \in \mathbb{Z}$~:
     \begin{itemize}
          \item $\text{e}^{a+b}=\text{e}^{a} \times \text{e}^{b}$
          \item $\text{e}^{-a}=\frac{1}{\text{e}^{a}}$
          \item $\text{e}^{a-b}=\frac{\text{e}^{a}}{\text{e}^{b}}$
          \item $\left(\text{e}^{a}\right)^{n}=\text{e}^{na}$
     \end{itemize}
}
\bloc{cyan}{Remarques}{% id="r70"
     \begin{itemize}
          \item Ces propriétés sont démontrées dans l'exercice~: \mcLien{/exercices/logarithme-exponentielle/roc-proprietes-algebriques-exponentielle}{[ROC] Propriétés algébriques de la fonction exponentielle}
          Elles sont similaires aux propriétés des puissances vues au collège (et justifient la notation $\text{e}^{x}$)
          \item Si l'on pose $a=\frac{1}{2}$ et $n=2$ dans la formule $\left(\text{e}^{a}\right)^{n}=\text{e}^{na}$ on obtient $\left(\text{e}^{^{\frac{1}{2}}}\right)^{2}=\text{e}^{1}=\text{e}$ donc comme $\text{e}^{^{\frac{1}{2}}} > 0$~: $\text{e}^{^{\frac{1}{2}}}=\sqrt{\text{e}}$
     \end{itemize}
}

\end{document}
µ
\documentclass[a4paper]{article}

%================================================================================================================================
%
% Packages
%
%================================================================================================================================

\usepackage[T1]{fontenc} 	% pour caractères accentués
\usepackage[utf8]{inputenc}  % encodage utf8
\usepackage[french]{babel}	% langue : français
\usepackage{fourier}			% caractères plus lisibles
\usepackage[dvipsnames]{xcolor} % couleurs
\usepackage{fancyhdr}		% réglage header footer
\usepackage{needspace}		% empêcher sauts de page mal placés
\usepackage{graphicx}		% pour inclure des graphiques
\usepackage{enumitem,cprotect}		% personnalise les listes d'items (nécessaire pour ol, al ...)
\usepackage{hyperref}		% Liens hypertexte
\usepackage{pstricks,pst-all,pst-node,pstricks-add,pst-math,pst-plot,pst-tree,pst-eucl} % pstricks
\usepackage[a4paper,includeheadfoot,top=2cm,left=3cm, bottom=2cm,right=3cm]{geometry} % marges etc.
\usepackage{comment}			% commentaires multilignes
\usepackage{amsmath,environ} % maths (matrices, etc.)
\usepackage{amssymb,makeidx}
\usepackage{bm}				% bold maths
\usepackage{tabularx}		% tableaux
\usepackage{colortbl}		% tableaux en couleur
\usepackage{fontawesome}		% Fontawesome
\usepackage{environ}			% environment with command
\usepackage{fp}				% calculs pour ps-tricks
\usepackage{multido}			% pour ps tricks
\usepackage[np]{numprint}	% formattage nombre
\usepackage{tikz,tkz-tab} 			% package principal TikZ
\usepackage{pgfplots}   % axes
\usepackage{mathrsfs}    % cursives
\usepackage{calc}			% calcul taille boites
\usepackage[scaled=0.875]{helvet} % font sans serif
\usepackage{svg} % svg
\usepackage{scrextend} % local margin
\usepackage{scratch} %scratch
\usepackage{multicol} % colonnes
%\usepackage{infix-RPN,pst-func} % formule en notation polanaise inversée
\usepackage{listings}

%================================================================================================================================
%
% Réglages de base
%
%================================================================================================================================

\lstset{
language=Python,   % R code
literate=
{á}{{\'a}}1
{à}{{\`a}}1
{ã}{{\~a}}1
{é}{{\'e}}1
{è}{{\`e}}1
{ê}{{\^e}}1
{í}{{\'i}}1
{ó}{{\'o}}1
{õ}{{\~o}}1
{ú}{{\'u}}1
{ü}{{\"u}}1
{ç}{{\c{c}}}1
{~}{{ }}1
}


\definecolor{codegreen}{rgb}{0,0.6,0}
\definecolor{codegray}{rgb}{0.5,0.5,0.5}
\definecolor{codepurple}{rgb}{0.58,0,0.82}
\definecolor{backcolour}{rgb}{0.95,0.95,0.92}

\lstdefinestyle{mystyle}{
    backgroundcolor=\color{backcolour},   
    commentstyle=\color{codegreen},
    keywordstyle=\color{magenta},
    numberstyle=\tiny\color{codegray},
    stringstyle=\color{codepurple},
    basicstyle=\ttfamily\footnotesize,
    breakatwhitespace=false,         
    breaklines=true,                 
    captionpos=b,                    
    keepspaces=true,                 
    numbers=left,                    
xleftmargin=2em,
framexleftmargin=2em,            
    showspaces=false,                
    showstringspaces=false,
    showtabs=false,                  
    tabsize=2,
    upquote=true
}

\lstset{style=mystyle}


\lstset{style=mystyle}
\newcommand{\imgdir}{C:/laragon/www/newmc/assets/imgsvg/}
\newcommand{\imgsvgdir}{C:/laragon/www/newmc/assets/imgsvg/}

\definecolor{mcgris}{RGB}{220, 220, 220}% ancien~; pour compatibilité
\definecolor{mcbleu}{RGB}{52, 152, 219}
\definecolor{mcvert}{RGB}{125, 194, 70}
\definecolor{mcmauve}{RGB}{154, 0, 215}
\definecolor{mcorange}{RGB}{255, 96, 0}
\definecolor{mcturquoise}{RGB}{0, 153, 153}
\definecolor{mcrouge}{RGB}{255, 0, 0}
\definecolor{mclightvert}{RGB}{205, 234, 190}

\definecolor{gris}{RGB}{220, 220, 220}
\definecolor{bleu}{RGB}{52, 152, 219}
\definecolor{vert}{RGB}{125, 194, 70}
\definecolor{mauve}{RGB}{154, 0, 215}
\definecolor{orange}{RGB}{255, 96, 0}
\definecolor{turquoise}{RGB}{0, 153, 153}
\definecolor{rouge}{RGB}{255, 0, 0}
\definecolor{lightvert}{RGB}{205, 234, 190}
\setitemize[0]{label=\color{lightvert}  $\bullet$}

\pagestyle{fancy}
\renewcommand{\headrulewidth}{0.2pt}
\fancyhead[L]{maths-cours.fr}
\fancyhead[R]{\thepage}
\renewcommand{\footrulewidth}{0.2pt}
\fancyfoot[C]{}

\newcolumntype{C}{>{\centering\arraybackslash}X}
\newcolumntype{s}{>{\hsize=.35\hsize\arraybackslash}X}

\setlength{\parindent}{0pt}		 
\setlength{\parskip}{3mm}
\setlength{\headheight}{1cm}

\def\ebook{ebook}
\def\book{book}
\def\web{web}
\def\type{web}

\newcommand{\vect}[1]{\overrightarrow{\,\mathstrut#1\,}}

\def\Oij{$\left(\text{O}~;~\vect{\imath},~\vect{\jmath}\right)$}
\def\Oijk{$\left(\text{O}~;~\vect{\imath},~\vect{\jmath},~\vect{k}\right)$}
\def\Ouv{$\left(\text{O}~;~\vect{u},~\vect{v}\right)$}

\hypersetup{breaklinks=true, colorlinks = true, linkcolor = OliveGreen, urlcolor = OliveGreen, citecolor = OliveGreen, pdfauthor={Didier BONNEL - https://www.maths-cours.fr} } % supprime les bordures autour des liens

\renewcommand{\arg}[0]{\text{arg}}

\everymath{\displaystyle}

%================================================================================================================================
%
% Macros - Commandes
%
%================================================================================================================================

\newcommand\meta[2]{    			% Utilisé pour créer le post HTML.
	\def\titre{titre}
	\def\url{url}
	\def\arg{#1}
	\ifx\titre\arg
		\newcommand\maintitle{#2}
		\fancyhead[L]{#2}
		{\Large\sffamily \MakeUppercase{#2}}
		\vspace{1mm}\textcolor{mcvert}{\hrule}
	\fi 
	\ifx\url\arg
		\fancyfoot[L]{\href{https://www.maths-cours.fr#2}{\black \footnotesize{https://www.maths-cours.fr#2}}}
	\fi 
}


\newcommand\TitreC[1]{    		% Titre centré
     \needspace{3\baselineskip}
     \begin{center}\textbf{#1}\end{center}
}

\newcommand\newpar{    		% paragraphe
     \par
}

\newcommand\nosp {    		% commande vide (pas d'espace)
}
\newcommand{\id}[1]{} %ignore

\newcommand\boite[2]{				% Boite simple sans titre
	\vspace{5mm}
	\setlength{\fboxrule}{0.2mm}
	\setlength{\fboxsep}{5mm}	
	\fcolorbox{#1}{#1!3}{\makebox[\linewidth-2\fboxrule-2\fboxsep]{
  		\begin{minipage}[t]{\linewidth-2\fboxrule-4\fboxsep}\setlength{\parskip}{3mm}
  			 #2
  		\end{minipage}
	}}
	\vspace{5mm}
}

\newcommand\CBox[4]{				% Boites
	\vspace{5mm}
	\setlength{\fboxrule}{0.2mm}
	\setlength{\fboxsep}{5mm}
	
	\fcolorbox{#1}{#1!3}{\makebox[\linewidth-2\fboxrule-2\fboxsep]{
		\begin{minipage}[t]{1cm}\setlength{\parskip}{3mm}
	  		\textcolor{#1}{\LARGE{#2}}    
 	 	\end{minipage}  
  		\begin{minipage}[t]{\linewidth-2\fboxrule-4\fboxsep}\setlength{\parskip}{3mm}
			\raisebox{1.2mm}{\normalsize\sffamily{\textcolor{#1}{#3}}}						
  			 #4
  		\end{minipage}
	}}
	\vspace{5mm}
}

\newcommand\cadre[3]{				% Boites convertible html
	\par
	\vspace{2mm}
	\setlength{\fboxrule}{0.1mm}
	\setlength{\fboxsep}{5mm}
	\fcolorbox{#1}{white}{\makebox[\linewidth-2\fboxrule-2\fboxsep]{
  		\begin{minipage}[t]{\linewidth-2\fboxrule-4\fboxsep}\setlength{\parskip}{3mm}
			\raisebox{-2.5mm}{\sffamily \small{\textcolor{#1}{\MakeUppercase{#2}}}}		
			\par		
  			 #3
 	 		\end{minipage}
	}}
		\vspace{2mm}
	\par
}

\newcommand\bloc[3]{				% Boites convertible html sans bordure
     \needspace{2\baselineskip}
     {\sffamily \small{\textcolor{#1}{\MakeUppercase{#2}}}}    
		\par		
  			 #3
		\par
}

\newcommand\CHelp[1]{
     \CBox{Plum}{\faInfoCircle}{À RETENIR}{#1}
}

\newcommand\CUp[1]{
     \CBox{NavyBlue}{\faThumbsOUp}{EN PRATIQUE}{#1}
}

\newcommand\CInfo[1]{
     \CBox{Sepia}{\faArrowCircleRight}{REMARQUE}{#1}
}

\newcommand\CRedac[1]{
     \CBox{PineGreen}{\faEdit}{BIEN R\'EDIGER}{#1}
}

\newcommand\CError[1]{
     \CBox{Red}{\faExclamationTriangle}{ATTENTION}{#1}
}

\newcommand\TitreExo[2]{
\needspace{4\baselineskip}
 {\sffamily\large EXERCICE #1\ (\emph{#2 points})}
\vspace{5mm}
}

\newcommand\img[2]{
          \includegraphics[width=#2\paperwidth]{\imgdir#1}
}

\newcommand\imgsvg[2]{
       \begin{center}   \includegraphics[width=#2\paperwidth]{\imgsvgdir#1} \end{center}
}


\newcommand\Lien[2]{
     \href{#1}{#2 \tiny \faExternalLink}
}
\newcommand\mcLien[2]{
     \href{https~://www.maths-cours.fr/#1}{#2 \tiny \faExternalLink}
}

\newcommand{\euro}{\eurologo{}}

%================================================================================================================================
%
% Macros - Environement
%
%================================================================================================================================

\newenvironment{tex}{ %
}
{%
}

\newenvironment{indente}{ %
	\setlength\parindent{10mm}
}

{
	\setlength\parindent{0mm}
}

\newenvironment{corrige}{%
     \needspace{3\baselineskip}
     \medskip
     \textbf{\textsc{Corrigé}}
     \medskip
}
{
}

\newenvironment{extern}{%
     \begin{center}
     }
     {
     \end{center}
}

\NewEnviron{code}{%
	\par
     \boite{gray}{\texttt{%
     \BODY
     }}
     \par
}

\newenvironment{vbloc}{% boite sans cadre empeche saut de page
     \begin{minipage}[t]{\linewidth}
     }
     {
     \end{minipage}
}
\NewEnviron{h2}{%
    \needspace{3\baselineskip}
    \vspace{0.6cm}
	\noindent \MakeUppercase{\sffamily \large \BODY}
	\vspace{1mm}\textcolor{mcgris}{\hrule}\vspace{0.4cm}
	\par
}{}

\NewEnviron{h3}{%
    \needspace{3\baselineskip}
	\vspace{5mm}
	\textsc{\BODY}
	\par
}

\NewEnviron{margeneg}{ %
\begin{addmargin}[-1cm]{0cm}
\BODY
\end{addmargin}
}

\NewEnviron{html}{%
}

\begin{document}
\meta{url}{/exercices/determiner-lexpression-dun-terme-dune-suite-en-fonction-de-n/}
\meta{pid}{10906}
\meta{titre}{Privé : Déterminer l'expression d'un terme d'une suite en fonction de n}
\meta{type}{exercices}
On considère la suite $(u_n)$ définie par $u_0=1$ et pour tout entier naturel $n$ :
\[ u_{n+1}=\dfrac{u_n}{u_n+1} \]
\par
Le but de cet exercice est de déterminer une formule donnant $u_n$ en fonction de $n$.
\par
On utilisera une méthode différente dans chacune des  parties.
\TitreC{Première méthode : Raisonnement par récurrence}
\begin{enumerate}
     \item %
     Calculer les valeurs de $u_1$, $u_2$, $u_3$ et $u_4$.
     \\
     Conjecturer l'expression de $u_n$ en fonction de $n$.
     \item %
     Démontrer, par récurrence, la conjecture faite à la question précédente.
\end{enumerate}
\TitreC{Deuxième méthode : utilisation d'une suite annexe}
Pour tout entier naturel $n$, on pose $v_n=\dfrac{1}{u_n}$.
\begin{enumerate}
     \item %
     Montrer que la suite $(v_n)$ est une suite arithmétique dont on déterminera le premier terme et la raison.
     \item %
     En déduire l'expression de $v_n$ puis celle de $u_n$ en fonction de $n$.
\end{enumerate}
\begin{corrige}
     \TitreC{Première méthode : Raisonnement par récurrence}
     \begin{enumerate}
          \item %
          $u_1 =\dfrac{u_0}{u_0+1}= \dfrac{1}{2}$\\
          $u_2 =\dfrac{u_1}{u_1+1}= \dfrac{1/2}{3/2}=\dfrac{1}{3}$\\
          $u_3 =\dfrac{u_2}{u_2+1}= \dfrac{1/3}{4/3}=\dfrac{1}{4}$\\
          $u_4 =\dfrac{u_3}{u_3+1}= \dfrac{1/4}{5/4}=\dfrac{1}{5}$
          \par
          Les résultats précédents laissent présager que pour tout entier naturel $n$ :
          \[ u_n=\dfrac{1}{n+1} \]
          \item %
          Montrons par récurrence que  pour tout entier naturel $n$ :
          \[ u_n=\dfrac{1}{n+1} \]
          \par
          \textbf{Initialisation :}
          \par
          $u_0=1=\dfrac{1}{0+1}$
          \par
          La propriété est donc vraie au rang $0$.
          \medbreak
          \textbf{Hérédité :}
          \par
          Supposons que, pour un certain entier $n$, $u_n=\dfrac{1}{n+1}$ et montrons que $u_{n+1}=\dfrac{1}{n+2}$~:
          \par
          $u_{n+1}=\dfrac{u_n}{u_n+1} $ (d'après l'énoncé)\par
          $\phantom{u_{n+1}}=\dfrac{1/(n+1)}{1+1/(n+1)} $ (hypothèse de récurrence)\par
          $\phantom{u_{n+1}}=\dfrac{1/(n+1)}{(n+1)/(n+1)+1/(n+1)} $\par
          $\phantom{u_{n+1}}=\dfrac{1/(n+1)}{(n+2)/(n+1)} $ \par
          $\phantom{u_{n+1}}=\dfrac{1}{n+2}.$
          \medbreak
          La propriété est donc héréditaire.
          \medbreak
          \textbf{Conclusion :}
          \par
          On en déduit, d'après le principe de récurrence, que pour tout entier naturel $n$~:
          \[ u_n=\dfrac{1}{n+1}. \]
     \end{enumerate}
     \TitreC{Deuxième méthode : utilisation d'une suite annexe}
     \begin{enumerate}
          \item %
          Pour montrer que la suite $(v_n)$ est arithmétique, montrons que $v_{n+1}-v_n$ est constant.
          \par
          D'après l'énoncé, pour tout entier naturel $n$~:
          \par
          $v_{n+1} -v_n = \dfrac{1}{u_{n+1}} - \dfrac{1}{u_n}$\par
          $\phantom{v_{n+1} -v_n} = \dfrac{1}{u_n/(u_n+1)} - \dfrac{1}{u_n}$\par
          $\phantom{v_{n+1} -v_n} = \dfrac{u_n+1}{u_n} - \dfrac{1}{u_n}$\par
          $\phantom{v_{n+1} -v_n} = \dfrac{u_n}{u_n} = 1.$
          \par
          La suite  $(v_n)$ est donc une suite arithmétique de raison $r=1$.
          \par
          Son premier terme est : \\
          $v_0=\dfrac{1}{u_0}=1.$
          \item %
          On en déduit donc que pour tout entier naturel $n$~:
          \par
          $v_n=v_0+nr=1+n.$
          \medbreak
          Par conséquent, pour tout entier naturel $n$~:
          $u_n=\dfrac{1}{v_n}=\dfrac{1}{n+1}.$
     \end{enumerate}
\end{corrige}

\end{document}
µ
\documentclass[a4paper]{article}

%================================================================================================================================
%
% Packages
%
%================================================================================================================================

\usepackage[T1]{fontenc} 	% pour caractères accentués
\usepackage[utf8]{inputenc}  % encodage utf8
\usepackage[french]{babel}	% langue : français
\usepackage{fourier}			% caractères plus lisibles
\usepackage[dvipsnames]{xcolor} % couleurs
\usepackage{fancyhdr}		% réglage header footer
\usepackage{needspace}		% empêcher sauts de page mal placés
\usepackage{graphicx}		% pour inclure des graphiques
\usepackage{enumitem,cprotect}		% personnalise les listes d'items (nécessaire pour ol, al ...)
\usepackage{hyperref}		% Liens hypertexte
\usepackage{pstricks,pst-all,pst-node,pstricks-add,pst-math,pst-plot,pst-tree,pst-eucl} % pstricks
\usepackage[a4paper,includeheadfoot,top=2cm,left=3cm, bottom=2cm,right=3cm]{geometry} % marges etc.
\usepackage{comment}			% commentaires multilignes
\usepackage{amsmath,environ} % maths (matrices, etc.)
\usepackage{amssymb,makeidx}
\usepackage{bm}				% bold maths
\usepackage{tabularx}		% tableaux
\usepackage{colortbl}		% tableaux en couleur
\usepackage{fontawesome}		% Fontawesome
\usepackage{environ}			% environment with command
\usepackage{fp}				% calculs pour ps-tricks
\usepackage{multido}			% pour ps tricks
\usepackage[np]{numprint}	% formattage nombre
\usepackage{tikz,tkz-tab} 			% package principal TikZ
\usepackage{pgfplots}   % axes
\usepackage{mathrsfs}    % cursives
\usepackage{calc}			% calcul taille boites
\usepackage[scaled=0.875]{helvet} % font sans serif
\usepackage{svg} % svg
\usepackage{scrextend} % local margin
\usepackage{scratch} %scratch
\usepackage{multicol} % colonnes
%\usepackage{infix-RPN,pst-func} % formule en notation polanaise inversée
\usepackage{listings}

%================================================================================================================================
%
% Réglages de base
%
%================================================================================================================================

\lstset{
language=Python,   % R code
literate=
{á}{{\'a}}1
{à}{{\`a}}1
{ã}{{\~a}}1
{é}{{\'e}}1
{è}{{\`e}}1
{ê}{{\^e}}1
{í}{{\'i}}1
{ó}{{\'o}}1
{õ}{{\~o}}1
{ú}{{\'u}}1
{ü}{{\"u}}1
{ç}{{\c{c}}}1
{~}{{ }}1
}


\definecolor{codegreen}{rgb}{0,0.6,0}
\definecolor{codegray}{rgb}{0.5,0.5,0.5}
\definecolor{codepurple}{rgb}{0.58,0,0.82}
\definecolor{backcolour}{rgb}{0.95,0.95,0.92}

\lstdefinestyle{mystyle}{
    backgroundcolor=\color{backcolour},   
    commentstyle=\color{codegreen},
    keywordstyle=\color{magenta},
    numberstyle=\tiny\color{codegray},
    stringstyle=\color{codepurple},
    basicstyle=\ttfamily\footnotesize,
    breakatwhitespace=false,         
    breaklines=true,                 
    captionpos=b,                    
    keepspaces=true,                 
    numbers=left,                    
xleftmargin=2em,
framexleftmargin=2em,            
    showspaces=false,                
    showstringspaces=false,
    showtabs=false,                  
    tabsize=2,
    upquote=true
}

\lstset{style=mystyle}


\lstset{style=mystyle}
\newcommand{\imgdir}{C:/laragon/www/newmc/assets/imgsvg/}
\newcommand{\imgsvgdir}{C:/laragon/www/newmc/assets/imgsvg/}

\definecolor{mcgris}{RGB}{220, 220, 220}% ancien~; pour compatibilité
\definecolor{mcbleu}{RGB}{52, 152, 219}
\definecolor{mcvert}{RGB}{125, 194, 70}
\definecolor{mcmauve}{RGB}{154, 0, 215}
\definecolor{mcorange}{RGB}{255, 96, 0}
\definecolor{mcturquoise}{RGB}{0, 153, 153}
\definecolor{mcrouge}{RGB}{255, 0, 0}
\definecolor{mclightvert}{RGB}{205, 234, 190}

\definecolor{gris}{RGB}{220, 220, 220}
\definecolor{bleu}{RGB}{52, 152, 219}
\definecolor{vert}{RGB}{125, 194, 70}
\definecolor{mauve}{RGB}{154, 0, 215}
\definecolor{orange}{RGB}{255, 96, 0}
\definecolor{turquoise}{RGB}{0, 153, 153}
\definecolor{rouge}{RGB}{255, 0, 0}
\definecolor{lightvert}{RGB}{205, 234, 190}
\setitemize[0]{label=\color{lightvert}  $\bullet$}

\pagestyle{fancy}
\renewcommand{\headrulewidth}{0.2pt}
\fancyhead[L]{maths-cours.fr}
\fancyhead[R]{\thepage}
\renewcommand{\footrulewidth}{0.2pt}
\fancyfoot[C]{}

\newcolumntype{C}{>{\centering\arraybackslash}X}
\newcolumntype{s}{>{\hsize=.35\hsize\arraybackslash}X}

\setlength{\parindent}{0pt}		 
\setlength{\parskip}{3mm}
\setlength{\headheight}{1cm}

\def\ebook{ebook}
\def\book{book}
\def\web{web}
\def\type{web}

\newcommand{\vect}[1]{\overrightarrow{\,\mathstrut#1\,}}

\def\Oij{$\left(\text{O}~;~\vect{\imath},~\vect{\jmath}\right)$}
\def\Oijk{$\left(\text{O}~;~\vect{\imath},~\vect{\jmath},~\vect{k}\right)$}
\def\Ouv{$\left(\text{O}~;~\vect{u},~\vect{v}\right)$}

\hypersetup{breaklinks=true, colorlinks = true, linkcolor = OliveGreen, urlcolor = OliveGreen, citecolor = OliveGreen, pdfauthor={Didier BONNEL - https://www.maths-cours.fr} } % supprime les bordures autour des liens

\renewcommand{\arg}[0]{\text{arg}}

\everymath{\displaystyle}

%================================================================================================================================
%
% Macros - Commandes
%
%================================================================================================================================

\newcommand\meta[2]{    			% Utilisé pour créer le post HTML.
	\def\titre{titre}
	\def\url{url}
	\def\arg{#1}
	\ifx\titre\arg
		\newcommand\maintitle{#2}
		\fancyhead[L]{#2}
		{\Large\sffamily \MakeUppercase{#2}}
		\vspace{1mm}\textcolor{mcvert}{\hrule}
	\fi 
	\ifx\url\arg
		\fancyfoot[L]{\href{https://www.maths-cours.fr#2}{\black \footnotesize{https://www.maths-cours.fr#2}}}
	\fi 
}


\newcommand\TitreC[1]{    		% Titre centré
     \needspace{3\baselineskip}
     \begin{center}\textbf{#1}\end{center}
}

\newcommand\newpar{    		% paragraphe
     \par
}

\newcommand\nosp {    		% commande vide (pas d'espace)
}
\newcommand{\id}[1]{} %ignore

\newcommand\boite[2]{				% Boite simple sans titre
	\vspace{5mm}
	\setlength{\fboxrule}{0.2mm}
	\setlength{\fboxsep}{5mm}	
	\fcolorbox{#1}{#1!3}{\makebox[\linewidth-2\fboxrule-2\fboxsep]{
  		\begin{minipage}[t]{\linewidth-2\fboxrule-4\fboxsep}\setlength{\parskip}{3mm}
  			 #2
  		\end{minipage}
	}}
	\vspace{5mm}
}

\newcommand\CBox[4]{				% Boites
	\vspace{5mm}
	\setlength{\fboxrule}{0.2mm}
	\setlength{\fboxsep}{5mm}
	
	\fcolorbox{#1}{#1!3}{\makebox[\linewidth-2\fboxrule-2\fboxsep]{
		\begin{minipage}[t]{1cm}\setlength{\parskip}{3mm}
	  		\textcolor{#1}{\LARGE{#2}}    
 	 	\end{minipage}  
  		\begin{minipage}[t]{\linewidth-2\fboxrule-4\fboxsep}\setlength{\parskip}{3mm}
			\raisebox{1.2mm}{\normalsize\sffamily{\textcolor{#1}{#3}}}						
  			 #4
  		\end{minipage}
	}}
	\vspace{5mm}
}

\newcommand\cadre[3]{				% Boites convertible html
	\par
	\vspace{2mm}
	\setlength{\fboxrule}{0.1mm}
	\setlength{\fboxsep}{5mm}
	\fcolorbox{#1}{white}{\makebox[\linewidth-2\fboxrule-2\fboxsep]{
  		\begin{minipage}[t]{\linewidth-2\fboxrule-4\fboxsep}\setlength{\parskip}{3mm}
			\raisebox{-2.5mm}{\sffamily \small{\textcolor{#1}{\MakeUppercase{#2}}}}		
			\par		
  			 #3
 	 		\end{minipage}
	}}
		\vspace{2mm}
	\par
}

\newcommand\bloc[3]{				% Boites convertible html sans bordure
     \needspace{2\baselineskip}
     {\sffamily \small{\textcolor{#1}{\MakeUppercase{#2}}}}    
		\par		
  			 #3
		\par
}

\newcommand\CHelp[1]{
     \CBox{Plum}{\faInfoCircle}{À RETENIR}{#1}
}

\newcommand\CUp[1]{
     \CBox{NavyBlue}{\faThumbsOUp}{EN PRATIQUE}{#1}
}

\newcommand\CInfo[1]{
     \CBox{Sepia}{\faArrowCircleRight}{REMARQUE}{#1}
}

\newcommand\CRedac[1]{
     \CBox{PineGreen}{\faEdit}{BIEN R\'EDIGER}{#1}
}

\newcommand\CError[1]{
     \CBox{Red}{\faExclamationTriangle}{ATTENTION}{#1}
}

\newcommand\TitreExo[2]{
\needspace{4\baselineskip}
 {\sffamily\large EXERCICE #1\ (\emph{#2 points})}
\vspace{5mm}
}

\newcommand\img[2]{
          \includegraphics[width=#2\paperwidth]{\imgdir#1}
}

\newcommand\imgsvg[2]{
       \begin{center}   \includegraphics[width=#2\paperwidth]{\imgsvgdir#1} \end{center}
}


\newcommand\Lien[2]{
     \href{#1}{#2 \tiny \faExternalLink}
}
\newcommand\mcLien[2]{
     \href{https~://www.maths-cours.fr/#1}{#2 \tiny \faExternalLink}
}

\newcommand{\euro}{\eurologo{}}

%================================================================================================================================
%
% Macros - Environement
%
%================================================================================================================================

\newenvironment{tex}{ %
}
{%
}

\newenvironment{indente}{ %
	\setlength\parindent{10mm}
}

{
	\setlength\parindent{0mm}
}

\newenvironment{corrige}{%
     \needspace{3\baselineskip}
     \medskip
     \textbf{\textsc{Corrigé}}
     \medskip
}
{
}

\newenvironment{extern}{%
     \begin{center}
     }
     {
     \end{center}
}

\NewEnviron{code}{%
	\par
     \boite{gray}{\texttt{%
     \BODY
     }}
     \par
}

\newenvironment{vbloc}{% boite sans cadre empeche saut de page
     \begin{minipage}[t]{\linewidth}
     }
     {
     \end{minipage}
}
\NewEnviron{h2}{%
    \needspace{3\baselineskip}
    \vspace{0.6cm}
	\noindent \MakeUppercase{\sffamily \large \BODY}
	\vspace{1mm}\textcolor{mcgris}{\hrule}\vspace{0.4cm}
	\par
}{}

\NewEnviron{h3}{%
    \needspace{3\baselineskip}
	\vspace{5mm}
	\textsc{\BODY}
	\par
}

\NewEnviron{margeneg}{ %
\begin{addmargin}[-1cm]{0cm}
\BODY
\end{addmargin}
}

\NewEnviron{html}{%
}

\begin{document}
\meta{url}{/exercices/theoreme-de-thales-et-cercles/}
\meta{pid}{10958}
\meta{titre}{Théorème de Thalès et cercles}
\meta{type}{exercices}
%
Dans la figure ci-dessous, les points $A, O, D$ sont alignés ainsi que les points $B, O, C$.\\
$B$ appartient au cercle de diamètre $[AO]$ et $C$ appartient au cercle de diamètre $[DO]$.
\begin{center}
     \begin{extern}
          \psset{xunit=1.0cm,yunit=1.0cm,algebraic=true,dimen=middle,dotstyle=o,dotsize=5pt 0,linewidth=1.6pt,arrowsize=3pt 2,arrowinset=0.25}
          \begin{pspicture*}(1.57,6.39)(17.93,20.35)
               \pscircle[linewidth=.8pt](8.,15.){4.24}
               \pscircle[linewidth=.8pt](13.,10.){2.82}
               \psline[linewidth=.8pt](5.,18.)(9.733,18.872)
               \psline[linewidth=.8pt](11.84,7.41)(15.,8.)
               \psline[linewidth=.8pt](5.,18.)(15.,8.)
               \psline[linewidth=.8pt](11.84,7.42)(9.73,18.87)
               \begin{scriptsize}
                    \fontsize{13pt}{13pt}\selectfont
                    \psdots[dotsize=2pt 0,dotstyle=*](11.,12.)
                    \rput[bl](10.31088,11.88928){$O$}
                    \psdots[dotsize=2pt 0,dotstyle=*](5.,18.)
                    \rput[bl](4.55088,18.08928){$A$}
                    \psdots[dotsize=2pt 0,dotstyle=*](15.,8.)
                    \rput[bl](15.23088,7.76928){$D$}
                    \psdots[dotsize=2pt 0,dotstyle=*](9.733,18.87)
                    \rput[bl](9.81088,18.94928){$B$}
                    \psdots[dotsize=2pt 0,dotstyle=*](11.844,7.418)
                    \rput[bl](11.45088,6.92928){$C$}
               \end{scriptsize}
          \end{pspicture*}
     \end{extern}
\end{center}
On donne~:
\par
$AO=8$cm\\
$OD=5$cm\\
$DC=3$cm.
\par
\begin{enumerate}
     \item %
     Montrer que le triangle $ABO$ est rectangle en $B$ et que le triangle $OCD$ est rectangle en $C$.
     \item %
     Justifier que les droites $(AB)$ et $(CD)$ sont parallèles.
     \item %
     Calculer la longueur $AB$.
\end{enumerate}
\begin{corrige}
     \begin{enumerate}
          \item %
          On utilise la propriété suivante~:
          \cadre{bleu}{Rappel}{
               Si  $[BC]$  est  un  diamètre  d'un  cercle  et $A$ un point de ce cercle (distinct de $B$ et de $C$), alors le triangle $ABC$ est rectangle en $A$.
          }
          Le côté $[AO]$ est un diamètre du cercle circonscrit au triangle $ABO$, donc le triangle $ABO$ est rectangle en $B$.\\
          De même, le côté $[DO]$ est un diamètre du cercle circonscrit au triangle $OCD$, donc le triangle $OCD$ est rectangle en $C$.
          \item %
          D'après la question précédente, les droites $(AB)$ et $(CD)$ sont toutes deux perpendiculaires à la droite $(AD)$~; elles sont donc parallèles entre elles.
          \item %
          Les droites $(AB)$ et $(CD)$ sont parallèles et les points $A, O, D$ sont alignés de même que les points $B, O, C$~; les triangles $ABO$ et $OCD$ forment alors une configuration de Thalès.\\
          On a donc, d'après le \mcLien{/cours/theoreme-thales\#d10}{théorème de Thalès}~:
          \par
          $\dfrac{AB}{CD}=\dfrac{AO}{OD}$
          \par
          $\dfrac{AB}{3}=\dfrac{8}{5}$
          \par
          Par conséquent~:
          \par
          $AB=\dfrac{3 \times 8}{5}=4,8.$
          \par
          Le côté $AB$ mesure $4,8$cm.
     \end{enumerate}
\end{corrige}

\end{document}

µ
\documentclass[a4paper]{article}

%================================================================================================================================
%
% Packages
%
%================================================================================================================================

\usepackage[T1]{fontenc} 	% pour caractères accentués
\usepackage[utf8]{inputenc}  % encodage utf8
\usepackage[french]{babel}	% langue : français
\usepackage{fourier}			% caractères plus lisibles
\usepackage[dvipsnames]{xcolor} % couleurs
\usepackage{fancyhdr}		% réglage header footer
\usepackage{needspace}		% empêcher sauts de page mal placés
\usepackage{graphicx}		% pour inclure des graphiques
\usepackage{enumitem,cprotect}		% personnalise les listes d'items (nécessaire pour ol, al ...)
\usepackage{hyperref}		% Liens hypertexte
\usepackage{pstricks,pst-all,pst-node,pstricks-add,pst-math,pst-plot,pst-tree,pst-eucl} % pstricks
\usepackage[a4paper,includeheadfoot,top=2cm,left=3cm, bottom=2cm,right=3cm]{geometry} % marges etc.
\usepackage{comment}			% commentaires multilignes
\usepackage{amsmath,environ} % maths (matrices, etc.)
\usepackage{amssymb,makeidx}
\usepackage{bm}				% bold maths
\usepackage{tabularx}		% tableaux
\usepackage{colortbl}		% tableaux en couleur
\usepackage{fontawesome}		% Fontawesome
\usepackage{environ}			% environment with command
\usepackage{fp}				% calculs pour ps-tricks
\usepackage{multido}			% pour ps tricks
\usepackage[np]{numprint}	% formattage nombre
\usepackage{tikz,tkz-tab} 			% package principal TikZ
\usepackage{pgfplots}   % axes
\usepackage{mathrsfs}    % cursives
\usepackage{calc}			% calcul taille boites
\usepackage[scaled=0.875]{helvet} % font sans serif
\usepackage{svg} % svg
\usepackage{scrextend} % local margin
\usepackage{scratch} %scratch
\usepackage{multicol} % colonnes
%\usepackage{infix-RPN,pst-func} % formule en notation polanaise inversée
\usepackage{listings}

%================================================================================================================================
%
% Réglages de base
%
%================================================================================================================================

\lstset{
language=Python,   % R code
literate=
{á}{{\'a}}1
{à}{{\`a}}1
{ã}{{\~a}}1
{é}{{\'e}}1
{è}{{\`e}}1
{ê}{{\^e}}1
{í}{{\'i}}1
{ó}{{\'o}}1
{õ}{{\~o}}1
{ú}{{\'u}}1
{ü}{{\"u}}1
{ç}{{\c{c}}}1
{~}{{ }}1
}


\definecolor{codegreen}{rgb}{0,0.6,0}
\definecolor{codegray}{rgb}{0.5,0.5,0.5}
\definecolor{codepurple}{rgb}{0.58,0,0.82}
\definecolor{backcolour}{rgb}{0.95,0.95,0.92}

\lstdefinestyle{mystyle}{
    backgroundcolor=\color{backcolour},   
    commentstyle=\color{codegreen},
    keywordstyle=\color{magenta},
    numberstyle=\tiny\color{codegray},
    stringstyle=\color{codepurple},
    basicstyle=\ttfamily\footnotesize,
    breakatwhitespace=false,         
    breaklines=true,                 
    captionpos=b,                    
    keepspaces=true,                 
    numbers=left,                    
xleftmargin=2em,
framexleftmargin=2em,            
    showspaces=false,                
    showstringspaces=false,
    showtabs=false,                  
    tabsize=2,
    upquote=true
}

\lstset{style=mystyle}


\lstset{style=mystyle}
\newcommand{\imgdir}{C:/laragon/www/newmc/assets/imgsvg/}
\newcommand{\imgsvgdir}{C:/laragon/www/newmc/assets/imgsvg/}

\definecolor{mcgris}{RGB}{220, 220, 220}% ancien~; pour compatibilité
\definecolor{mcbleu}{RGB}{52, 152, 219}
\definecolor{mcvert}{RGB}{125, 194, 70}
\definecolor{mcmauve}{RGB}{154, 0, 215}
\definecolor{mcorange}{RGB}{255, 96, 0}
\definecolor{mcturquoise}{RGB}{0, 153, 153}
\definecolor{mcrouge}{RGB}{255, 0, 0}
\definecolor{mclightvert}{RGB}{205, 234, 190}

\definecolor{gris}{RGB}{220, 220, 220}
\definecolor{bleu}{RGB}{52, 152, 219}
\definecolor{vert}{RGB}{125, 194, 70}
\definecolor{mauve}{RGB}{154, 0, 215}
\definecolor{orange}{RGB}{255, 96, 0}
\definecolor{turquoise}{RGB}{0, 153, 153}
\definecolor{rouge}{RGB}{255, 0, 0}
\definecolor{lightvert}{RGB}{205, 234, 190}
\setitemize[0]{label=\color{lightvert}  $\bullet$}

\pagestyle{fancy}
\renewcommand{\headrulewidth}{0.2pt}
\fancyhead[L]{maths-cours.fr}
\fancyhead[R]{\thepage}
\renewcommand{\footrulewidth}{0.2pt}
\fancyfoot[C]{}

\newcolumntype{C}{>{\centering\arraybackslash}X}
\newcolumntype{s}{>{\hsize=.35\hsize\arraybackslash}X}

\setlength{\parindent}{0pt}		 
\setlength{\parskip}{3mm}
\setlength{\headheight}{1cm}

\def\ebook{ebook}
\def\book{book}
\def\web{web}
\def\type{web}

\newcommand{\vect}[1]{\overrightarrow{\,\mathstrut#1\,}}

\def\Oij{$\left(\text{O}~;~\vect{\imath},~\vect{\jmath}\right)$}
\def\Oijk{$\left(\text{O}~;~\vect{\imath},~\vect{\jmath},~\vect{k}\right)$}
\def\Ouv{$\left(\text{O}~;~\vect{u},~\vect{v}\right)$}

\hypersetup{breaklinks=true, colorlinks = true, linkcolor = OliveGreen, urlcolor = OliveGreen, citecolor = OliveGreen, pdfauthor={Didier BONNEL - https://www.maths-cours.fr} } % supprime les bordures autour des liens

\renewcommand{\arg}[0]{\text{arg}}

\everymath{\displaystyle}

%================================================================================================================================
%
% Macros - Commandes
%
%================================================================================================================================

\newcommand\meta[2]{    			% Utilisé pour créer le post HTML.
	\def\titre{titre}
	\def\url{url}
	\def\arg{#1}
	\ifx\titre\arg
		\newcommand\maintitle{#2}
		\fancyhead[L]{#2}
		{\Large\sffamily \MakeUppercase{#2}}
		\vspace{1mm}\textcolor{mcvert}{\hrule}
	\fi 
	\ifx\url\arg
		\fancyfoot[L]{\href{https://www.maths-cours.fr#2}{\black \footnotesize{https://www.maths-cours.fr#2}}}
	\fi 
}


\newcommand\TitreC[1]{    		% Titre centré
     \needspace{3\baselineskip}
     \begin{center}\textbf{#1}\end{center}
}

\newcommand\newpar{    		% paragraphe
     \par
}

\newcommand\nosp {    		% commande vide (pas d'espace)
}
\newcommand{\id}[1]{} %ignore

\newcommand\boite[2]{				% Boite simple sans titre
	\vspace{5mm}
	\setlength{\fboxrule}{0.2mm}
	\setlength{\fboxsep}{5mm}	
	\fcolorbox{#1}{#1!3}{\makebox[\linewidth-2\fboxrule-2\fboxsep]{
  		\begin{minipage}[t]{\linewidth-2\fboxrule-4\fboxsep}\setlength{\parskip}{3mm}
  			 #2
  		\end{minipage}
	}}
	\vspace{5mm}
}

\newcommand\CBox[4]{				% Boites
	\vspace{5mm}
	\setlength{\fboxrule}{0.2mm}
	\setlength{\fboxsep}{5mm}
	
	\fcolorbox{#1}{#1!3}{\makebox[\linewidth-2\fboxrule-2\fboxsep]{
		\begin{minipage}[t]{1cm}\setlength{\parskip}{3mm}
	  		\textcolor{#1}{\LARGE{#2}}    
 	 	\end{minipage}  
  		\begin{minipage}[t]{\linewidth-2\fboxrule-4\fboxsep}\setlength{\parskip}{3mm}
			\raisebox{1.2mm}{\normalsize\sffamily{\textcolor{#1}{#3}}}						
  			 #4
  		\end{minipage}
	}}
	\vspace{5mm}
}

\newcommand\cadre[3]{				% Boites convertible html
	\par
	\vspace{2mm}
	\setlength{\fboxrule}{0.1mm}
	\setlength{\fboxsep}{5mm}
	\fcolorbox{#1}{white}{\makebox[\linewidth-2\fboxrule-2\fboxsep]{
  		\begin{minipage}[t]{\linewidth-2\fboxrule-4\fboxsep}\setlength{\parskip}{3mm}
			\raisebox{-2.5mm}{\sffamily \small{\textcolor{#1}{\MakeUppercase{#2}}}}		
			\par		
  			 #3
 	 		\end{minipage}
	}}
		\vspace{2mm}
	\par
}

\newcommand\bloc[3]{				% Boites convertible html sans bordure
     \needspace{2\baselineskip}
     {\sffamily \small{\textcolor{#1}{\MakeUppercase{#2}}}}    
		\par		
  			 #3
		\par
}

\newcommand\CHelp[1]{
     \CBox{Plum}{\faInfoCircle}{À RETENIR}{#1}
}

\newcommand\CUp[1]{
     \CBox{NavyBlue}{\faThumbsOUp}{EN PRATIQUE}{#1}
}

\newcommand\CInfo[1]{
     \CBox{Sepia}{\faArrowCircleRight}{REMARQUE}{#1}
}

\newcommand\CRedac[1]{
     \CBox{PineGreen}{\faEdit}{BIEN R\'EDIGER}{#1}
}

\newcommand\CError[1]{
     \CBox{Red}{\faExclamationTriangle}{ATTENTION}{#1}
}

\newcommand\TitreExo[2]{
\needspace{4\baselineskip}
 {\sffamily\large EXERCICE #1\ (\emph{#2 points})}
\vspace{5mm}
}

\newcommand\img[2]{
          \includegraphics[width=#2\paperwidth]{\imgdir#1}
}

\newcommand\imgsvg[2]{
       \begin{center}   \includegraphics[width=#2\paperwidth]{\imgsvgdir#1} \end{center}
}


\newcommand\Lien[2]{
     \href{#1}{#2 \tiny \faExternalLink}
}
\newcommand\mcLien[2]{
     \href{https~://www.maths-cours.fr/#1}{#2 \tiny \faExternalLink}
}

\newcommand{\euro}{\eurologo{}}

%================================================================================================================================
%
% Macros - Environement
%
%================================================================================================================================

\newenvironment{tex}{ %
}
{%
}

\newenvironment{indente}{ %
	\setlength\parindent{10mm}
}

{
	\setlength\parindent{0mm}
}

\newenvironment{corrige}{%
     \needspace{3\baselineskip}
     \medskip
     \textbf{\textsc{Corrigé}}
     \medskip
}
{
}

\newenvironment{extern}{%
     \begin{center}
     }
     {
     \end{center}
}

\NewEnviron{code}{%
	\par
     \boite{gray}{\texttt{%
     \BODY
     }}
     \par
}

\newenvironment{vbloc}{% boite sans cadre empeche saut de page
     \begin{minipage}[t]{\linewidth}
     }
     {
     \end{minipage}
}
\NewEnviron{h2}{%
    \needspace{3\baselineskip}
    \vspace{0.6cm}
	\noindent \MakeUppercase{\sffamily \large \BODY}
	\vspace{1mm}\textcolor{mcgris}{\hrule}\vspace{0.4cm}
	\par
}{}

\NewEnviron{h3}{%
    \needspace{3\baselineskip}
	\vspace{5mm}
	\textsc{\BODY}
	\par
}

\NewEnviron{margeneg}{ %
\begin{addmargin}[-1cm]{0cm}
\BODY
\end{addmargin}
}

\NewEnviron{html}{%
}

\begin{document}
\meta{url}{/cours/python-au-lycee-1/}
\meta{pid}{10980}
\meta{titre}{Privé~: Python au lycée (1)~: Les variables et opérateurs}
\meta{type}{cours}
\begin{h2}1. Types de variables\end{h2}
Dans le langage Python, chaque variable et chaque constante possède un \textbf{type}.
\par
Les principaux types de base sont~:
\begin{itemize}
     \item %
     \textbf{int} (integer)~: nombre entier comme 5 ou -12345678901234567890
     \item %
     \textbf{float} (floating-point number)~: nombre décimal comme 0.0001 ou 3.141592654
     \item %
     \textbf{str} (string)~: chaîne de caractères comme \og Bonjour toto123~! \fg{} ou \og A \fg{}
\end{itemize}
\par
D'autres types (booléens, nombres complexes, listes, ...) seront vus dans des chapitres ultérieurs.
\par
L'instruction \og \texttt{type} \fg{} renvoie le type d'une expression. Par exemple~:
\begin{lstlisting}[language=Python]
>>> type(1) # affiche <class 'int'>
>>> type(1.5) # affiche <class 'float'>
>>> type('A') # affiche <class 'str'>
>>> type(True) # affiche <class 'bool'>
\end{lstlisting}
\textbf{Remarque~:}\\
Dans l'exemple ci-dessus, les chevrons >>> indiquent que les commandes ont été saisies en mode interactif - sous IDLE par exemple.\\
Les \# indiquent le début d'un commentaire~; ici, les commentaires sont utilisés pour indiquer le résultat de l'instruction.
\begin{h3}Affectation d'une valeur à une variable\end{h3}
On affecte une valeur à une variable grâce au symbole \og = \fg{}. Par exemple, l'instruction~:
\begin{lstlisting}[language=Python]
a=2
\end{lstlisting}
\begin{itemize}
     \item %
     crée la variable a (si elle n'existait pas déjà)
     \item %
     affecte la valeur 2 à la variable a
     \item %
     définit le type de a (ici~: int)
\end{itemize}
En Python, il n'est pas nécessaire de déclarer préalablement les variables ou leur type. \\
Python détermine de façon dynamique le type d'une variable en fonction de sa valeur. \\
Il est ainsi possible de changer le type d'une variable à l'intérieur d'un programme.
\par
Par exemple~:
 \begin{lstlisting}[language=Python]
a=1 # a est de type 'int'
a='Bonjour' # ne provoque pas d'erreur. a est maintenant de type 'str'
\end{lstlisting}
Le nom d'une variable doit être composé uniquement de lettres, de chiffres et du symbole \og \_ \fg{} (underscore). Il ne doit pas commencer par un chiffre.
\par
Exemple~:
\begin{lstlisting}[language=Python]
ma_variable=5 # correct
ma-variable=5 # incorrect
variable1=5 # correct
1variable=5 # incorrect
\end{lstlisting}
Enfin, signalons que la casse (distinction majuscule/minuscule) est également prise en compte~: \\
\texttt{MaVariable} et \texttt{mavariable} représentent deux variables différentes.
\begin{h3}Instructions d'entrée/sortie\end{h3}
La fonction \textbf{print} affiche à l'écran la valeur d'une variable ou d'une constante. Il est possible d'afficher plusieurs valeurs en les séparant par une virgule.
\par
Par exemple~:
\begin{lstlisting}[language=Python]
x=12
print('La valeur de x est',x) # affiche : La valeur de x est 12
print('Le type de x est',type(x)) # affiche : Le type de x est <class 'int'>
\end{lstlisting}
La fonction \textbf{input} est utilisée pour permettre à l'utilisateur d'entrer des données. Elle suspend le programme jusqu'à ce que l'utilisateur saisisse un texte. Lorsque l'utilisateur valide sa saisie à l'aide de la touche <ENTER> la fonction renvoie le texte saisi et le programme reprend son exécution.
\par
Il est possible d'indiquer en paramètre de la fonction \textit{input} une chaine de caractères qui sera affichée pour guider l'utilisateur (\og prompt \fg{}).
\par
Par exemple~:
\begin{lstlisting}[language=Python]
a=input('Saisir une valeur') # a recevra la valeur saisie
print('La valeur de a est',a) # affiche : La valeur de a est <valeur saisie par l'utilisateur>
\end{lstlisting}
\par
La valeur saisie par l'utilisateur est toujours considérée par Python comme étant de type \textit{str}. Pour obtenir un autre type de données, il faut \textit{transtyper} la valeur saisie.
\par
Exemple~:
 \begin{lstlisting}[language=Python]
a=input('Saisir une valeur : ') # a sera du type string
a=int(input('Saisir une valeur entiere : ')) # a sera du type int
a=float(input('Saisir une valeur decimale : ')) # a sera du type float
\end{lstlisting}
\begin{h2}2. Type \og int \fg{} \end{h2}
À partir de la version 3 de Python, il n'y a pas de limite (autre que la mémoire de l'ordinateur) au nombre de chiffres que peut avoir un entier~:
 \begin{lstlisting}[language=Python]
print('2 puissance 100 =', 2^100) 
    # affiche : 2 puissance 100 = 1267650600228229401496703205376
\end{lstlisting}
Le tableau ci-dessous recense les différentes opérations que l'on peut effectuer sur des entiers~:
\begin{center}
     \begin{tabularx}{0.9\linewidth}{|*{3}{>{\centering \arraybackslash }X|}}%class="compact"
          \hline
          \textbf{Op.} & \textbf{Description} & \textbf{Exemple}
          \\ \hline
          + & Calcule la somme de deux entiers & b = a + 2
          \\ \hline
          - & Calcule la différence de deux entiers & b = 4 - a
          \\ \hline
          * & Calcule le produit de deux entiers & b = 5 * a
          \\ \hline
          / & Calcule le quotient décimal de deux entiers \newline (le résultat est de type \textit{float}) & b = 8/5 \newline (b vaut alors 1.6)
          \\ \hline
          // & Calcule le quotient entier de deux entiers \newline (division \og euclidienne \fg{} ) & b = 8/5 \newline (b vaut alors 1)
          \\ \hline
          \% & Calcule le reste de la division \og euclidienne \fg{} \newline de deux entiers & b = 8\%5 \newline (b vaut alors 3)
          \\ \hline
     \end{tabularx}
\end{center}
Python respecte l'ordre mathématique des calculs (parenthèses, puis puissances, puis multiplications et divisons, puis additions et soustractions). Par exemple~:
 \begin{lstlisting}[language=Python]
a=12+5*(3-1)**3 
print(a) # affiche 52
\end{lstlisting}
\begin{h2}3. Type \og float \fg{} \end{h2}
En Python, les nombres décimaux (\textit{float}) s'écrivent en utilisant un point comme séparateur décimal (par exemple 1.2 au lieu de 1,2).
Lorsqu'on veut définir un nombre entier comme nombre décimal, on peut le transtyper, mais il est plus simple d'ajouter \og .0 \fg{} à la fin de ce nombre~:
\begin{lstlisting}[language=Python]
>>> type(5) # affiche : <class 'int'>
>>> type(float(5)) # affiche : <class 'float'>
>>> type(5.0) # affiche : <class 'float'>
\end{lstlisting}
Les opérations que l'on peut effectuer sur les nombres décimaux sont les mêmes que pour les nombres entiers à la différence près qu'elles renvoient un nombre décimal.
\par
Par exemple~:
\begin{lstlisting}[language=Python]
>>> 2.5*1.2 # affiche 3.0
\end{lstlisting}
\begin{h2}4. Type \og str \fg{} \end{h2}
Pour définir une chaîne de caractères (type~: \textit{str} pour \textit{string}) en Python, il faut placer cette chaîne entre apostrophes (') ou entre guillemets ("). Par exemple~:
 \begin{lstlisting}[language=Python]
a="Bonjour les amis" # a est de type str
b='Comment allez-vous?' # b est de type str
\end{lstlisting}
Si la chaîne de caractères comporte déjà une apostrophe ou un guillemet, cela peut être problématique~:\\
par exemple, l'instruction~:
 \begin{lstlisting}[language=Python]
a='J'ai froid'
\end{lstlisting}
est incorrecte à cause de la présence d'une apostrophe dans le texte (Python croit alors que cette apostrophe indique la fin de la chaîne de caractères et il ne comprend pas la suite...)
On peut résoudre ce problème de 3 manières différentes~:
\begin{itemize}
     \item %
     alterner guillemets et apostrophes~: \texttt{a="J'ai froid"} est correct
     \item %
     \textit{échapper} l'apostrophe en la faisant précéder d'un antislash (\textbackslash)~: \textit{a='J\'ai froid'} est correct
     \item %
     placer le texte entre \textbf{trois} apostrophes~: \texttt{a='''J'ai froid'''} est correct
\end{itemize}
Une autre caractéristique de la notation avec les triples apostrophes est d'accepter plusieurs lignes de texte dans la chaîne.
\\Par exemple~:
 \begin{lstlisting}[language=Python]
a= '''Ligne 1
Ligne 2''' 
# a contient deux lignes de texte
\end{lstlisting}
Pour effectuer un saut de ligne, il est également possible d'utiliser le caractère spécial \og \textbackslash n \fg{}.
\\Par exemple~:
 \begin{lstlisting}[language=Python]
a='Ligne 1\nLigne 2'
# a contient aussi deux lignes de texte
\end{lstlisting}
L'opérateur + permet de concaténer (mettre bout à bout) deux chaînes de caractère~; l'opérateur * (suivi ou précédé d'un nombre entier) permet de répéter plusieurs fois une chaîne~:
\begin{lstlisting}[language=Python]
a='Bonjour '
b='Vincent'
print(a+b) # affiche : Bonjour Vincent
print(3*a) # affiche : Bonjour Bonjour Bonjour 
\end{lstlisting}

\end{document}
µ
\documentclass[a4paper]{article}

%================================================================================================================================
%
% Packages
%
%================================================================================================================================

\usepackage[T1]{fontenc} 	% pour caractères accentués
\usepackage[utf8]{inputenc}  % encodage utf8
\usepackage[french]{babel}	% langue : français
\usepackage{fourier}			% caractères plus lisibles
\usepackage[dvipsnames]{xcolor} % couleurs
\usepackage{fancyhdr}		% réglage header footer
\usepackage{needspace}		% empêcher sauts de page mal placés
\usepackage{graphicx}		% pour inclure des graphiques
\usepackage{enumitem,cprotect}		% personnalise les listes d'items (nécessaire pour ol, al ...)
\usepackage{hyperref}		% Liens hypertexte
\usepackage{pstricks,pst-all,pst-node,pstricks-add,pst-math,pst-plot,pst-tree,pst-eucl} % pstricks
\usepackage[a4paper,includeheadfoot,top=2cm,left=3cm, bottom=2cm,right=3cm]{geometry} % marges etc.
\usepackage{comment}			% commentaires multilignes
\usepackage{amsmath,environ} % maths (matrices, etc.)
\usepackage{amssymb,makeidx}
\usepackage{bm}				% bold maths
\usepackage{tabularx}		% tableaux
\usepackage{colortbl}		% tableaux en couleur
\usepackage{fontawesome}		% Fontawesome
\usepackage{environ}			% environment with command
\usepackage{fp}				% calculs pour ps-tricks
\usepackage{multido}			% pour ps tricks
\usepackage[np]{numprint}	% formattage nombre
\usepackage{tikz,tkz-tab} 			% package principal TikZ
\usepackage{pgfplots}   % axes
\usepackage{mathrsfs}    % cursives
\usepackage{calc}			% calcul taille boites
\usepackage[scaled=0.875]{helvet} % font sans serif
\usepackage{svg} % svg
\usepackage{scrextend} % local margin
\usepackage{scratch} %scratch
\usepackage{multicol} % colonnes
%\usepackage{infix-RPN,pst-func} % formule en notation polanaise inversée
\usepackage{listings}

%================================================================================================================================
%
% Réglages de base
%
%================================================================================================================================

\lstset{
language=Python,   % R code
literate=
{á}{{\'a}}1
{à}{{\`a}}1
{ã}{{\~a}}1
{é}{{\'e}}1
{è}{{\`e}}1
{ê}{{\^e}}1
{í}{{\'i}}1
{ó}{{\'o}}1
{õ}{{\~o}}1
{ú}{{\'u}}1
{ü}{{\"u}}1
{ç}{{\c{c}}}1
{~}{{ }}1
}


\definecolor{codegreen}{rgb}{0,0.6,0}
\definecolor{codegray}{rgb}{0.5,0.5,0.5}
\definecolor{codepurple}{rgb}{0.58,0,0.82}
\definecolor{backcolour}{rgb}{0.95,0.95,0.92}

\lstdefinestyle{mystyle}{
    backgroundcolor=\color{backcolour},   
    commentstyle=\color{codegreen},
    keywordstyle=\color{magenta},
    numberstyle=\tiny\color{codegray},
    stringstyle=\color{codepurple},
    basicstyle=\ttfamily\footnotesize,
    breakatwhitespace=false,         
    breaklines=true,                 
    captionpos=b,                    
    keepspaces=true,                 
    numbers=left,                    
xleftmargin=2em,
framexleftmargin=2em,            
    showspaces=false,                
    showstringspaces=false,
    showtabs=false,                  
    tabsize=2,
    upquote=true
}

\lstset{style=mystyle}


\lstset{style=mystyle}
\newcommand{\imgdir}{C:/laragon/www/newmc/assets/imgsvg/}
\newcommand{\imgsvgdir}{C:/laragon/www/newmc/assets/imgsvg/}

\definecolor{mcgris}{RGB}{220, 220, 220}% ancien~; pour compatibilité
\definecolor{mcbleu}{RGB}{52, 152, 219}
\definecolor{mcvert}{RGB}{125, 194, 70}
\definecolor{mcmauve}{RGB}{154, 0, 215}
\definecolor{mcorange}{RGB}{255, 96, 0}
\definecolor{mcturquoise}{RGB}{0, 153, 153}
\definecolor{mcrouge}{RGB}{255, 0, 0}
\definecolor{mclightvert}{RGB}{205, 234, 190}

\definecolor{gris}{RGB}{220, 220, 220}
\definecolor{bleu}{RGB}{52, 152, 219}
\definecolor{vert}{RGB}{125, 194, 70}
\definecolor{mauve}{RGB}{154, 0, 215}
\definecolor{orange}{RGB}{255, 96, 0}
\definecolor{turquoise}{RGB}{0, 153, 153}
\definecolor{rouge}{RGB}{255, 0, 0}
\definecolor{lightvert}{RGB}{205, 234, 190}
\setitemize[0]{label=\color{lightvert}  $\bullet$}

\pagestyle{fancy}
\renewcommand{\headrulewidth}{0.2pt}
\fancyhead[L]{maths-cours.fr}
\fancyhead[R]{\thepage}
\renewcommand{\footrulewidth}{0.2pt}
\fancyfoot[C]{}

\newcolumntype{C}{>{\centering\arraybackslash}X}
\newcolumntype{s}{>{\hsize=.35\hsize\arraybackslash}X}

\setlength{\parindent}{0pt}		 
\setlength{\parskip}{3mm}
\setlength{\headheight}{1cm}

\def\ebook{ebook}
\def\book{book}
\def\web{web}
\def\type{web}

\newcommand{\vect}[1]{\overrightarrow{\,\mathstrut#1\,}}

\def\Oij{$\left(\text{O}~;~\vect{\imath},~\vect{\jmath}\right)$}
\def\Oijk{$\left(\text{O}~;~\vect{\imath},~\vect{\jmath},~\vect{k}\right)$}
\def\Ouv{$\left(\text{O}~;~\vect{u},~\vect{v}\right)$}

\hypersetup{breaklinks=true, colorlinks = true, linkcolor = OliveGreen, urlcolor = OliveGreen, citecolor = OliveGreen, pdfauthor={Didier BONNEL - https://www.maths-cours.fr} } % supprime les bordures autour des liens

\renewcommand{\arg}[0]{\text{arg}}

\everymath{\displaystyle}

%================================================================================================================================
%
% Macros - Commandes
%
%================================================================================================================================

\newcommand\meta[2]{    			% Utilisé pour créer le post HTML.
	\def\titre{titre}
	\def\url{url}
	\def\arg{#1}
	\ifx\titre\arg
		\newcommand\maintitle{#2}
		\fancyhead[L]{#2}
		{\Large\sffamily \MakeUppercase{#2}}
		\vspace{1mm}\textcolor{mcvert}{\hrule}
	\fi 
	\ifx\url\arg
		\fancyfoot[L]{\href{https://www.maths-cours.fr#2}{\black \footnotesize{https://www.maths-cours.fr#2}}}
	\fi 
}


\newcommand\TitreC[1]{    		% Titre centré
     \needspace{3\baselineskip}
     \begin{center}\textbf{#1}\end{center}
}

\newcommand\newpar{    		% paragraphe
     \par
}

\newcommand\nosp {    		% commande vide (pas d'espace)
}
\newcommand{\id}[1]{} %ignore

\newcommand\boite[2]{				% Boite simple sans titre
	\vspace{5mm}
	\setlength{\fboxrule}{0.2mm}
	\setlength{\fboxsep}{5mm}	
	\fcolorbox{#1}{#1!3}{\makebox[\linewidth-2\fboxrule-2\fboxsep]{
  		\begin{minipage}[t]{\linewidth-2\fboxrule-4\fboxsep}\setlength{\parskip}{3mm}
  			 #2
  		\end{minipage}
	}}
	\vspace{5mm}
}

\newcommand\CBox[4]{				% Boites
	\vspace{5mm}
	\setlength{\fboxrule}{0.2mm}
	\setlength{\fboxsep}{5mm}
	
	\fcolorbox{#1}{#1!3}{\makebox[\linewidth-2\fboxrule-2\fboxsep]{
		\begin{minipage}[t]{1cm}\setlength{\parskip}{3mm}
	  		\textcolor{#1}{\LARGE{#2}}    
 	 	\end{minipage}  
  		\begin{minipage}[t]{\linewidth-2\fboxrule-4\fboxsep}\setlength{\parskip}{3mm}
			\raisebox{1.2mm}{\normalsize\sffamily{\textcolor{#1}{#3}}}						
  			 #4
  		\end{minipage}
	}}
	\vspace{5mm}
}

\newcommand\cadre[3]{				% Boites convertible html
	\par
	\vspace{2mm}
	\setlength{\fboxrule}{0.1mm}
	\setlength{\fboxsep}{5mm}
	\fcolorbox{#1}{white}{\makebox[\linewidth-2\fboxrule-2\fboxsep]{
  		\begin{minipage}[t]{\linewidth-2\fboxrule-4\fboxsep}\setlength{\parskip}{3mm}
			\raisebox{-2.5mm}{\sffamily \small{\textcolor{#1}{\MakeUppercase{#2}}}}		
			\par		
  			 #3
 	 		\end{minipage}
	}}
		\vspace{2mm}
	\par
}

\newcommand\bloc[3]{				% Boites convertible html sans bordure
     \needspace{2\baselineskip}
     {\sffamily \small{\textcolor{#1}{\MakeUppercase{#2}}}}    
		\par		
  			 #3
		\par
}

\newcommand\CHelp[1]{
     \CBox{Plum}{\faInfoCircle}{À RETENIR}{#1}
}

\newcommand\CUp[1]{
     \CBox{NavyBlue}{\faThumbsOUp}{EN PRATIQUE}{#1}
}

\newcommand\CInfo[1]{
     \CBox{Sepia}{\faArrowCircleRight}{REMARQUE}{#1}
}

\newcommand\CRedac[1]{
     \CBox{PineGreen}{\faEdit}{BIEN R\'EDIGER}{#1}
}

\newcommand\CError[1]{
     \CBox{Red}{\faExclamationTriangle}{ATTENTION}{#1}
}

\newcommand\TitreExo[2]{
\needspace{4\baselineskip}
 {\sffamily\large EXERCICE #1\ (\emph{#2 points})}
\vspace{5mm}
}

\newcommand\img[2]{
          \includegraphics[width=#2\paperwidth]{\imgdir#1}
}

\newcommand\imgsvg[2]{
       \begin{center}   \includegraphics[width=#2\paperwidth]{\imgsvgdir#1} \end{center}
}


\newcommand\Lien[2]{
     \href{#1}{#2 \tiny \faExternalLink}
}
\newcommand\mcLien[2]{
     \href{https~://www.maths-cours.fr/#1}{#2 \tiny \faExternalLink}
}

\newcommand{\euro}{\eurologo{}}

%================================================================================================================================
%
% Macros - Environement
%
%================================================================================================================================

\newenvironment{tex}{ %
}
{%
}

\newenvironment{indente}{ %
	\setlength\parindent{10mm}
}

{
	\setlength\parindent{0mm}
}

\newenvironment{corrige}{%
     \needspace{3\baselineskip}
     \medskip
     \textbf{\textsc{Corrigé}}
     \medskip
}
{
}

\newenvironment{extern}{%
     \begin{center}
     }
     {
     \end{center}
}

\NewEnviron{code}{%
	\par
     \boite{gray}{\texttt{%
     \BODY
     }}
     \par
}

\newenvironment{vbloc}{% boite sans cadre empeche saut de page
     \begin{minipage}[t]{\linewidth}
     }
     {
     \end{minipage}
}
\NewEnviron{h2}{%
    \needspace{3\baselineskip}
    \vspace{0.6cm}
	\noindent \MakeUppercase{\sffamily \large \BODY}
	\vspace{1mm}\textcolor{mcgris}{\hrule}\vspace{0.4cm}
	\par
}{}

\NewEnviron{h3}{%
    \needspace{3\baselineskip}
	\vspace{5mm}
	\textsc{\BODY}
	\par
}

\NewEnviron{margeneg}{ %
\begin{addmargin}[-1cm]{0cm}
\BODY
\end{addmargin}
}

\NewEnviron{html}{%
}

\begin{document}
\meta{url}{/exercices/python-calcul-du-volume-dune-sphere/}
\meta{pid}{11078}
\meta{titre}{Privé : Python : calcul du volume d'une sphère}
\meta{type}{exercices}
Écrire un programme Python qui demande, en entrée, le rayon d'une sphère en cm et affiche, en sortie, le volume de la sphère en cm$^3$.
\par
\textbf{Indication~:}
Pour utiliser le nombre $\pi$, on pourra faire appel au module \texttt{math} qui donne accès à différentes fonctions mathématiques comme cosinus (\texttt{cos}), sinus (\texttt{sin}), racine carrée (\texttt{sqrt}), et à certaine constantes mathématiques comme $\pi$ (\texttt{pi}).\\
Pour cela, il suffit d'inclure, en tête du programme, la ligne suivante~:
\begin{lstlisting}[language=Python]
from math import pi
\end{lstlisting}
Il suffira ensuite de faire appel à la constante \texttt{pi} à chaque fois que l'on souhaitera utiliser la valeur de $\pi$.
\begin{corrige}
     Le volume d'une sphère de rayon $r$ est donné par la formule~:
     \[V=\dfrac{4}{3}\pi r^3  \]
     Voici un programme possible~:
\begin{lstlisting}[language=Python]
from math import pi
r=float(input('Quel est le rayon de la sphere en cm : '))
v=4/3*pi*r**3
print('Le volume de la sphere est ', v, 'cm3')
\end{lstlisting}
\textbf{Remarques~:}
\begin{itemize}
     \item
     La ligne~:
\begin{lstlisting}[language=Python]
from math import pi
\end{lstlisting}
permet d'importer la valeur de $\pi$.\\
À la place, on peut également utiliser la ligne~:
\begin{lstlisting}[language=Python]
from math import *
\end{lstlisting}
pour importer toutes les constantes et toutes les fonctions du module \texttt{math.}
\item %
Il ne faut pas oublier de convertir le rayon en \texttt{float} sinon, il sera considéré comme étant une chaine de caractères et Python déclenchera  une erreur lors du calcul du volume.
\end{itemize}
\end{corrige}

\end{document}
µ
\documentclass[a4paper]{article}

%================================================================================================================================
%
% Packages
%
%================================================================================================================================

\usepackage[T1]{fontenc} 	% pour caractères accentués
\usepackage[utf8]{inputenc}  % encodage utf8
\usepackage[french]{babel}	% langue : français
\usepackage{fourier}			% caractères plus lisibles
\usepackage[dvipsnames]{xcolor} % couleurs
\usepackage{fancyhdr}		% réglage header footer
\usepackage{needspace}		% empêcher sauts de page mal placés
\usepackage{graphicx}		% pour inclure des graphiques
\usepackage{enumitem,cprotect}		% personnalise les listes d'items (nécessaire pour ol, al ...)
\usepackage{hyperref}		% Liens hypertexte
\usepackage{pstricks,pst-all,pst-node,pstricks-add,pst-math,pst-plot,pst-tree,pst-eucl} % pstricks
\usepackage[a4paper,includeheadfoot,top=2cm,left=3cm, bottom=2cm,right=3cm]{geometry} % marges etc.
\usepackage{comment}			% commentaires multilignes
\usepackage{amsmath,environ} % maths (matrices, etc.)
\usepackage{amssymb,makeidx}
\usepackage{bm}				% bold maths
\usepackage{tabularx}		% tableaux
\usepackage{colortbl}		% tableaux en couleur
\usepackage{fontawesome}		% Fontawesome
\usepackage{environ}			% environment with command
\usepackage{fp}				% calculs pour ps-tricks
\usepackage{multido}			% pour ps tricks
\usepackage[np]{numprint}	% formattage nombre
\usepackage{tikz,tkz-tab} 			% package principal TikZ
\usepackage{pgfplots}   % axes
\usepackage{mathrsfs}    % cursives
\usepackage{calc}			% calcul taille boites
\usepackage[scaled=0.875]{helvet} % font sans serif
\usepackage{svg} % svg
\usepackage{scrextend} % local margin
\usepackage{scratch} %scratch
\usepackage{multicol} % colonnes
%\usepackage{infix-RPN,pst-func} % formule en notation polanaise inversée
\usepackage{listings}

%================================================================================================================================
%
% Réglages de base
%
%================================================================================================================================

\lstset{
language=Python,   % R code
literate=
{á}{{\'a}}1
{à}{{\`a}}1
{ã}{{\~a}}1
{é}{{\'e}}1
{è}{{\`e}}1
{ê}{{\^e}}1
{í}{{\'i}}1
{ó}{{\'o}}1
{õ}{{\~o}}1
{ú}{{\'u}}1
{ü}{{\"u}}1
{ç}{{\c{c}}}1
{~}{{ }}1
}


\definecolor{codegreen}{rgb}{0,0.6,0}
\definecolor{codegray}{rgb}{0.5,0.5,0.5}
\definecolor{codepurple}{rgb}{0.58,0,0.82}
\definecolor{backcolour}{rgb}{0.95,0.95,0.92}

\lstdefinestyle{mystyle}{
    backgroundcolor=\color{backcolour},   
    commentstyle=\color{codegreen},
    keywordstyle=\color{magenta},
    numberstyle=\tiny\color{codegray},
    stringstyle=\color{codepurple},
    basicstyle=\ttfamily\footnotesize,
    breakatwhitespace=false,         
    breaklines=true,                 
    captionpos=b,                    
    keepspaces=true,                 
    numbers=left,                    
xleftmargin=2em,
framexleftmargin=2em,            
    showspaces=false,                
    showstringspaces=false,
    showtabs=false,                  
    tabsize=2,
    upquote=true
}

\lstset{style=mystyle}


\lstset{style=mystyle}
\newcommand{\imgdir}{C:/laragon/www/newmc/assets/imgsvg/}
\newcommand{\imgsvgdir}{C:/laragon/www/newmc/assets/imgsvg/}

\definecolor{mcgris}{RGB}{220, 220, 220}% ancien~; pour compatibilité
\definecolor{mcbleu}{RGB}{52, 152, 219}
\definecolor{mcvert}{RGB}{125, 194, 70}
\definecolor{mcmauve}{RGB}{154, 0, 215}
\definecolor{mcorange}{RGB}{255, 96, 0}
\definecolor{mcturquoise}{RGB}{0, 153, 153}
\definecolor{mcrouge}{RGB}{255, 0, 0}
\definecolor{mclightvert}{RGB}{205, 234, 190}

\definecolor{gris}{RGB}{220, 220, 220}
\definecolor{bleu}{RGB}{52, 152, 219}
\definecolor{vert}{RGB}{125, 194, 70}
\definecolor{mauve}{RGB}{154, 0, 215}
\definecolor{orange}{RGB}{255, 96, 0}
\definecolor{turquoise}{RGB}{0, 153, 153}
\definecolor{rouge}{RGB}{255, 0, 0}
\definecolor{lightvert}{RGB}{205, 234, 190}
\setitemize[0]{label=\color{lightvert}  $\bullet$}

\pagestyle{fancy}
\renewcommand{\headrulewidth}{0.2pt}
\fancyhead[L]{maths-cours.fr}
\fancyhead[R]{\thepage}
\renewcommand{\footrulewidth}{0.2pt}
\fancyfoot[C]{}

\newcolumntype{C}{>{\centering\arraybackslash}X}
\newcolumntype{s}{>{\hsize=.35\hsize\arraybackslash}X}

\setlength{\parindent}{0pt}		 
\setlength{\parskip}{3mm}
\setlength{\headheight}{1cm}

\def\ebook{ebook}
\def\book{book}
\def\web{web}
\def\type{web}

\newcommand{\vect}[1]{\overrightarrow{\,\mathstrut#1\,}}

\def\Oij{$\left(\text{O}~;~\vect{\imath},~\vect{\jmath}\right)$}
\def\Oijk{$\left(\text{O}~;~\vect{\imath},~\vect{\jmath},~\vect{k}\right)$}
\def\Ouv{$\left(\text{O}~;~\vect{u},~\vect{v}\right)$}

\hypersetup{breaklinks=true, colorlinks = true, linkcolor = OliveGreen, urlcolor = OliveGreen, citecolor = OliveGreen, pdfauthor={Didier BONNEL - https://www.maths-cours.fr} } % supprime les bordures autour des liens

\renewcommand{\arg}[0]{\text{arg}}

\everymath{\displaystyle}

%================================================================================================================================
%
% Macros - Commandes
%
%================================================================================================================================

\newcommand\meta[2]{    			% Utilisé pour créer le post HTML.
	\def\titre{titre}
	\def\url{url}
	\def\arg{#1}
	\ifx\titre\arg
		\newcommand\maintitle{#2}
		\fancyhead[L]{#2}
		{\Large\sffamily \MakeUppercase{#2}}
		\vspace{1mm}\textcolor{mcvert}{\hrule}
	\fi 
	\ifx\url\arg
		\fancyfoot[L]{\href{https://www.maths-cours.fr#2}{\black \footnotesize{https://www.maths-cours.fr#2}}}
	\fi 
}


\newcommand\TitreC[1]{    		% Titre centré
     \needspace{3\baselineskip}
     \begin{center}\textbf{#1}\end{center}
}

\newcommand\newpar{    		% paragraphe
     \par
}

\newcommand\nosp {    		% commande vide (pas d'espace)
}
\newcommand{\id}[1]{} %ignore

\newcommand\boite[2]{				% Boite simple sans titre
	\vspace{5mm}
	\setlength{\fboxrule}{0.2mm}
	\setlength{\fboxsep}{5mm}	
	\fcolorbox{#1}{#1!3}{\makebox[\linewidth-2\fboxrule-2\fboxsep]{
  		\begin{minipage}[t]{\linewidth-2\fboxrule-4\fboxsep}\setlength{\parskip}{3mm}
  			 #2
  		\end{minipage}
	}}
	\vspace{5mm}
}

\newcommand\CBox[4]{				% Boites
	\vspace{5mm}
	\setlength{\fboxrule}{0.2mm}
	\setlength{\fboxsep}{5mm}
	
	\fcolorbox{#1}{#1!3}{\makebox[\linewidth-2\fboxrule-2\fboxsep]{
		\begin{minipage}[t]{1cm}\setlength{\parskip}{3mm}
	  		\textcolor{#1}{\LARGE{#2}}    
 	 	\end{minipage}  
  		\begin{minipage}[t]{\linewidth-2\fboxrule-4\fboxsep}\setlength{\parskip}{3mm}
			\raisebox{1.2mm}{\normalsize\sffamily{\textcolor{#1}{#3}}}						
  			 #4
  		\end{minipage}
	}}
	\vspace{5mm}
}

\newcommand\cadre[3]{				% Boites convertible html
	\par
	\vspace{2mm}
	\setlength{\fboxrule}{0.1mm}
	\setlength{\fboxsep}{5mm}
	\fcolorbox{#1}{white}{\makebox[\linewidth-2\fboxrule-2\fboxsep]{
  		\begin{minipage}[t]{\linewidth-2\fboxrule-4\fboxsep}\setlength{\parskip}{3mm}
			\raisebox{-2.5mm}{\sffamily \small{\textcolor{#1}{\MakeUppercase{#2}}}}		
			\par		
  			 #3
 	 		\end{minipage}
	}}
		\vspace{2mm}
	\par
}

\newcommand\bloc[3]{				% Boites convertible html sans bordure
     \needspace{2\baselineskip}
     {\sffamily \small{\textcolor{#1}{\MakeUppercase{#2}}}}    
		\par		
  			 #3
		\par
}

\newcommand\CHelp[1]{
     \CBox{Plum}{\faInfoCircle}{À RETENIR}{#1}
}

\newcommand\CUp[1]{
     \CBox{NavyBlue}{\faThumbsOUp}{EN PRATIQUE}{#1}
}

\newcommand\CInfo[1]{
     \CBox{Sepia}{\faArrowCircleRight}{REMARQUE}{#1}
}

\newcommand\CRedac[1]{
     \CBox{PineGreen}{\faEdit}{BIEN R\'EDIGER}{#1}
}

\newcommand\CError[1]{
     \CBox{Red}{\faExclamationTriangle}{ATTENTION}{#1}
}

\newcommand\TitreExo[2]{
\needspace{4\baselineskip}
 {\sffamily\large EXERCICE #1\ (\emph{#2 points})}
\vspace{5mm}
}

\newcommand\img[2]{
          \includegraphics[width=#2\paperwidth]{\imgdir#1}
}

\newcommand\imgsvg[2]{
       \begin{center}   \includegraphics[width=#2\paperwidth]{\imgsvgdir#1} \end{center}
}


\newcommand\Lien[2]{
     \href{#1}{#2 \tiny \faExternalLink}
}
\newcommand\mcLien[2]{
     \href{https~://www.maths-cours.fr/#1}{#2 \tiny \faExternalLink}
}

\newcommand{\euro}{\eurologo{}}

%================================================================================================================================
%
% Macros - Environement
%
%================================================================================================================================

\newenvironment{tex}{ %
}
{%
}

\newenvironment{indente}{ %
	\setlength\parindent{10mm}
}

{
	\setlength\parindent{0mm}
}

\newenvironment{corrige}{%
     \needspace{3\baselineskip}
     \medskip
     \textbf{\textsc{Corrigé}}
     \medskip
}
{
}

\newenvironment{extern}{%
     \begin{center}
     }
     {
     \end{center}
}

\NewEnviron{code}{%
	\par
     \boite{gray}{\texttt{%
     \BODY
     }}
     \par
}

\newenvironment{vbloc}{% boite sans cadre empeche saut de page
     \begin{minipage}[t]{\linewidth}
     }
     {
     \end{minipage}
}
\NewEnviron{h2}{%
    \needspace{3\baselineskip}
    \vspace{0.6cm}
	\noindent \MakeUppercase{\sffamily \large \BODY}
	\vspace{1mm}\textcolor{mcgris}{\hrule}\vspace{0.4cm}
	\par
}{}

\NewEnviron{h3}{%
    \needspace{3\baselineskip}
	\vspace{5mm}
	\textsc{\BODY}
	\par
}

\NewEnviron{margeneg}{ %
\begin{addmargin}[-1cm]{0cm}
\BODY
\end{addmargin}
}

\NewEnviron{html}{%
}

\begin{document}
\meta{url}{/exercices/python-types-et-operateurs/}
\meta{pid}{11090}
\meta{titre}{Python : Types et opérateurs}
\meta{type}{exercices}
\par
En mode interactif, on entre chacune des instructions suivantes dans une console Python~:
\begin{lstlisting}[language=Python]
>>> 3+3
>>> 3.0+3
>>> "3"+3
>>> "3"+"3"
>>> 3*3
>>> 3.0*3
>>> "3"*3
>>> "3"*"3"
\end{lstlisting}
Pour chacune des lignes ci-dessus, indiquer le résultat renvoyé par l'interpréteur Python.
\begin{corrige}
     \begin{enumerate}
          \item %
\begin{lstlisting}[language=Python]
>>> 3+3
     \end{lstlisting}
     Le résultat de cette instruction est~: \texttt{6}.\\
     Python ajoute les deux entiers et renvoie un entier.
     \item %
\begin{lstlisting}[language=Python]
>>> 3.0+3
\end{lstlisting}
Le résultat de cette instruction est~: \texttt{6.0}.\\
Python ajoute les deux nombres. Comme l'un de ces nombres est un décimal (\texttt{float}), le résultat renvoyé est également un décimal (d'où la présence du \og .0 \fg{} ).
\item %
\begin{lstlisting}[language=Python]
>>> "3"+3
\end{lstlisting}
Cette instruction renvoie une erreur.\\
Python ne peut pas ajouter un entier et une chaine de caractères. Il faut noter que, contrairement à d'autres langages, Python n'effectue pas de conversion de types automatiquement. On dit que Python est \textbf{fortement typé}.
\item %
\begin{lstlisting}[language=Python]
>>> "3"+"3"
\end{lstlisting}
Le résultat de cette instruction est~: \texttt{'33'}.\\
Python concatène les chaînes de caractères. À cause des guillemets, \texttt{"3"} est considéré comme une chaîne de caractères.  L'interpréteur Python n'additionne donc pas les valeurs numériques mais effectue une concaténation.
\item %
\begin{lstlisting}[language=Python]
>>> 3*3
\end{lstlisting}
Le résultat de cette instruction est~: \texttt{9}.\\
Python effectue le produit des deux entiers et renvoie un entier.
\item %
\begin{lstlisting}[language=Python]
>>> 3.0*3
\end{lstlisting}
Le résultat de cette instruction est~: \texttt{9.0}.\\
Python multiplie les deux nombres. Comme l'un de ces nombres est un décimal (\texttt{float}), le résultat renvoyé est un décimal.
\item %
\begin{lstlisting}[language=Python]
>>> "3"*3
\end{lstlisting}
Le résultat de cette instruction est~: \texttt{'333'}.\\
Python répète trois fois la chaîne de caractères \texttt{"3"}. L'opérateur *,  suivi ou précédé d'un nombre entier, permet de répéter une chaîne de caractères.
\item %
\begin{lstlisting}[language=Python]
>>> "3"*"3"
\end{lstlisting}
Cette instruction renvoie une erreur.\\
Python ne peut pas multiplier une chaîne de caractères par une autre chaîne de caractères.
\end{enumerate}
\end{corrige}

\end{document}
µ
\documentclass[a4paper]{article}

%================================================================================================================================
%
% Packages
%
%================================================================================================================================

\usepackage[T1]{fontenc} 	% pour caractères accentués
\usepackage[utf8]{inputenc}  % encodage utf8
\usepackage[french]{babel}	% langue : français
\usepackage{fourier}			% caractères plus lisibles
\usepackage[dvipsnames]{xcolor} % couleurs
\usepackage{fancyhdr}		% réglage header footer
\usepackage{needspace}		% empêcher sauts de page mal placés
\usepackage{graphicx}		% pour inclure des graphiques
\usepackage{enumitem,cprotect}		% personnalise les listes d'items (nécessaire pour ol, al ...)
\usepackage{hyperref}		% Liens hypertexte
\usepackage{pstricks,pst-all,pst-node,pstricks-add,pst-math,pst-plot,pst-tree,pst-eucl} % pstricks
\usepackage[a4paper,includeheadfoot,top=2cm,left=3cm, bottom=2cm,right=3cm]{geometry} % marges etc.
\usepackage{comment}			% commentaires multilignes
\usepackage{amsmath,environ} % maths (matrices, etc.)
\usepackage{amssymb,makeidx}
\usepackage{bm}				% bold maths
\usepackage{tabularx}		% tableaux
\usepackage{colortbl}		% tableaux en couleur
\usepackage{fontawesome}		% Fontawesome
\usepackage{environ}			% environment with command
\usepackage{fp}				% calculs pour ps-tricks
\usepackage{multido}			% pour ps tricks
\usepackage[np]{numprint}	% formattage nombre
\usepackage{tikz,tkz-tab} 			% package principal TikZ
\usepackage{pgfplots}   % axes
\usepackage{mathrsfs}    % cursives
\usepackage{calc}			% calcul taille boites
\usepackage[scaled=0.875]{helvet} % font sans serif
\usepackage{svg} % svg
\usepackage{scrextend} % local margin
\usepackage{scratch} %scratch
\usepackage{multicol} % colonnes
%\usepackage{infix-RPN,pst-func} % formule en notation polanaise inversée
\usepackage{listings}

%================================================================================================================================
%
% Réglages de base
%
%================================================================================================================================

\lstset{
language=Python,   % R code
literate=
{á}{{\'a}}1
{à}{{\`a}}1
{ã}{{\~a}}1
{é}{{\'e}}1
{è}{{\`e}}1
{ê}{{\^e}}1
{í}{{\'i}}1
{ó}{{\'o}}1
{õ}{{\~o}}1
{ú}{{\'u}}1
{ü}{{\"u}}1
{ç}{{\c{c}}}1
{~}{{ }}1
}


\definecolor{codegreen}{rgb}{0,0.6,0}
\definecolor{codegray}{rgb}{0.5,0.5,0.5}
\definecolor{codepurple}{rgb}{0.58,0,0.82}
\definecolor{backcolour}{rgb}{0.95,0.95,0.92}

\lstdefinestyle{mystyle}{
    backgroundcolor=\color{backcolour},   
    commentstyle=\color{codegreen},
    keywordstyle=\color{magenta},
    numberstyle=\tiny\color{codegray},
    stringstyle=\color{codepurple},
    basicstyle=\ttfamily\footnotesize,
    breakatwhitespace=false,         
    breaklines=true,                 
    captionpos=b,                    
    keepspaces=true,                 
    numbers=left,                    
xleftmargin=2em,
framexleftmargin=2em,            
    showspaces=false,                
    showstringspaces=false,
    showtabs=false,                  
    tabsize=2,
    upquote=true
}

\lstset{style=mystyle}


\lstset{style=mystyle}
\newcommand{\imgdir}{C:/laragon/www/newmc/assets/imgsvg/}
\newcommand{\imgsvgdir}{C:/laragon/www/newmc/assets/imgsvg/}

\definecolor{mcgris}{RGB}{220, 220, 220}% ancien~; pour compatibilité
\definecolor{mcbleu}{RGB}{52, 152, 219}
\definecolor{mcvert}{RGB}{125, 194, 70}
\definecolor{mcmauve}{RGB}{154, 0, 215}
\definecolor{mcorange}{RGB}{255, 96, 0}
\definecolor{mcturquoise}{RGB}{0, 153, 153}
\definecolor{mcrouge}{RGB}{255, 0, 0}
\definecolor{mclightvert}{RGB}{205, 234, 190}

\definecolor{gris}{RGB}{220, 220, 220}
\definecolor{bleu}{RGB}{52, 152, 219}
\definecolor{vert}{RGB}{125, 194, 70}
\definecolor{mauve}{RGB}{154, 0, 215}
\definecolor{orange}{RGB}{255, 96, 0}
\definecolor{turquoise}{RGB}{0, 153, 153}
\definecolor{rouge}{RGB}{255, 0, 0}
\definecolor{lightvert}{RGB}{205, 234, 190}
\setitemize[0]{label=\color{lightvert}  $\bullet$}

\pagestyle{fancy}
\renewcommand{\headrulewidth}{0.2pt}
\fancyhead[L]{maths-cours.fr}
\fancyhead[R]{\thepage}
\renewcommand{\footrulewidth}{0.2pt}
\fancyfoot[C]{}

\newcolumntype{C}{>{\centering\arraybackslash}X}
\newcolumntype{s}{>{\hsize=.35\hsize\arraybackslash}X}

\setlength{\parindent}{0pt}		 
\setlength{\parskip}{3mm}
\setlength{\headheight}{1cm}

\def\ebook{ebook}
\def\book{book}
\def\web{web}
\def\type{web}

\newcommand{\vect}[1]{\overrightarrow{\,\mathstrut#1\,}}

\def\Oij{$\left(\text{O}~;~\vect{\imath},~\vect{\jmath}\right)$}
\def\Oijk{$\left(\text{O}~;~\vect{\imath},~\vect{\jmath},~\vect{k}\right)$}
\def\Ouv{$\left(\text{O}~;~\vect{u},~\vect{v}\right)$}

\hypersetup{breaklinks=true, colorlinks = true, linkcolor = OliveGreen, urlcolor = OliveGreen, citecolor = OliveGreen, pdfauthor={Didier BONNEL - https://www.maths-cours.fr} } % supprime les bordures autour des liens

\renewcommand{\arg}[0]{\text{arg}}

\everymath{\displaystyle}

%================================================================================================================================
%
% Macros - Commandes
%
%================================================================================================================================

\newcommand\meta[2]{    			% Utilisé pour créer le post HTML.
	\def\titre{titre}
	\def\url{url}
	\def\arg{#1}
	\ifx\titre\arg
		\newcommand\maintitle{#2}
		\fancyhead[L]{#2}
		{\Large\sffamily \MakeUppercase{#2}}
		\vspace{1mm}\textcolor{mcvert}{\hrule}
	\fi 
	\ifx\url\arg
		\fancyfoot[L]{\href{https://www.maths-cours.fr#2}{\black \footnotesize{https://www.maths-cours.fr#2}}}
	\fi 
}


\newcommand\TitreC[1]{    		% Titre centré
     \needspace{3\baselineskip}
     \begin{center}\textbf{#1}\end{center}
}

\newcommand\newpar{    		% paragraphe
     \par
}

\newcommand\nosp {    		% commande vide (pas d'espace)
}
\newcommand{\id}[1]{} %ignore

\newcommand\boite[2]{				% Boite simple sans titre
	\vspace{5mm}
	\setlength{\fboxrule}{0.2mm}
	\setlength{\fboxsep}{5mm}	
	\fcolorbox{#1}{#1!3}{\makebox[\linewidth-2\fboxrule-2\fboxsep]{
  		\begin{minipage}[t]{\linewidth-2\fboxrule-4\fboxsep}\setlength{\parskip}{3mm}
  			 #2
  		\end{minipage}
	}}
	\vspace{5mm}
}

\newcommand\CBox[4]{				% Boites
	\vspace{5mm}
	\setlength{\fboxrule}{0.2mm}
	\setlength{\fboxsep}{5mm}
	
	\fcolorbox{#1}{#1!3}{\makebox[\linewidth-2\fboxrule-2\fboxsep]{
		\begin{minipage}[t]{1cm}\setlength{\parskip}{3mm}
	  		\textcolor{#1}{\LARGE{#2}}    
 	 	\end{minipage}  
  		\begin{minipage}[t]{\linewidth-2\fboxrule-4\fboxsep}\setlength{\parskip}{3mm}
			\raisebox{1.2mm}{\normalsize\sffamily{\textcolor{#1}{#3}}}						
  			 #4
  		\end{minipage}
	}}
	\vspace{5mm}
}

\newcommand\cadre[3]{				% Boites convertible html
	\par
	\vspace{2mm}
	\setlength{\fboxrule}{0.1mm}
	\setlength{\fboxsep}{5mm}
	\fcolorbox{#1}{white}{\makebox[\linewidth-2\fboxrule-2\fboxsep]{
  		\begin{minipage}[t]{\linewidth-2\fboxrule-4\fboxsep}\setlength{\parskip}{3mm}
			\raisebox{-2.5mm}{\sffamily \small{\textcolor{#1}{\MakeUppercase{#2}}}}		
			\par		
  			 #3
 	 		\end{minipage}
	}}
		\vspace{2mm}
	\par
}

\newcommand\bloc[3]{				% Boites convertible html sans bordure
     \needspace{2\baselineskip}
     {\sffamily \small{\textcolor{#1}{\MakeUppercase{#2}}}}    
		\par		
  			 #3
		\par
}

\newcommand\CHelp[1]{
     \CBox{Plum}{\faInfoCircle}{À RETENIR}{#1}
}

\newcommand\CUp[1]{
     \CBox{NavyBlue}{\faThumbsOUp}{EN PRATIQUE}{#1}
}

\newcommand\CInfo[1]{
     \CBox{Sepia}{\faArrowCircleRight}{REMARQUE}{#1}
}

\newcommand\CRedac[1]{
     \CBox{PineGreen}{\faEdit}{BIEN R\'EDIGER}{#1}
}

\newcommand\CError[1]{
     \CBox{Red}{\faExclamationTriangle}{ATTENTION}{#1}
}

\newcommand\TitreExo[2]{
\needspace{4\baselineskip}
 {\sffamily\large EXERCICE #1\ (\emph{#2 points})}
\vspace{5mm}
}

\newcommand\img[2]{
          \includegraphics[width=#2\paperwidth]{\imgdir#1}
}

\newcommand\imgsvg[2]{
       \begin{center}   \includegraphics[width=#2\paperwidth]{\imgsvgdir#1} \end{center}
}


\newcommand\Lien[2]{
     \href{#1}{#2 \tiny \faExternalLink}
}
\newcommand\mcLien[2]{
     \href{https~://www.maths-cours.fr/#1}{#2 \tiny \faExternalLink}
}

\newcommand{\euro}{\eurologo{}}

%================================================================================================================================
%
% Macros - Environement
%
%================================================================================================================================

\newenvironment{tex}{ %
}
{%
}

\newenvironment{indente}{ %
	\setlength\parindent{10mm}
}

{
	\setlength\parindent{0mm}
}

\newenvironment{corrige}{%
     \needspace{3\baselineskip}
     \medskip
     \textbf{\textsc{Corrigé}}
     \medskip
}
{
}

\newenvironment{extern}{%
     \begin{center}
     }
     {
     \end{center}
}

\NewEnviron{code}{%
	\par
     \boite{gray}{\texttt{%
     \BODY
     }}
     \par
}

\newenvironment{vbloc}{% boite sans cadre empeche saut de page
     \begin{minipage}[t]{\linewidth}
     }
     {
     \end{minipage}
}
\NewEnviron{h2}{%
    \needspace{3\baselineskip}
    \vspace{0.6cm}
	\noindent \MakeUppercase{\sffamily \large \BODY}
	\vspace{1mm}\textcolor{mcgris}{\hrule}\vspace{0.4cm}
	\par
}{}

\NewEnviron{h3}{%
    \needspace{3\baselineskip}
	\vspace{5mm}
	\textsc{\BODY}
	\par
}

\NewEnviron{margeneg}{ %
\begin{addmargin}[-1cm]{0cm}
\BODY
\end{addmargin}
}

\NewEnviron{html}{%
}

\begin{document}
\meta{url}{/exercices/python-attention-aux-types-de-donnees/}
\meta{pid}{11105}
\meta{titre}{Python : Attention aux types de données !}
\meta{type}{exercices}
%
Clara a saisi le programme suivant dans son éditeur Python~:
\begin{lstlisting}[language=Python]
a=input('Entrez la valeur de a : ')
b=input('Entrez la valeur de b : ')
c=5*a+3*b
print(c)
\end{lstlisting}
Elle exécute ensuite ce programme en entrant \texttt{10} comme valeur pour \texttt{a} et \texttt{5} comme valeur pour \texttt{b}.
\par
Elle s'attend à obtenir \texttt{65} comme résultat.
\par
\begin{enumerate}
     \item %
     Quel est, en réalité, le résultat affiché par le programme~?\\
     Expliquez ce résultat.
     \item %
     Corrigez ce programme afin qu'il affiche le résultat \texttt{65} prévu par Clara.
\end{enumerate}
\begin{corrige}
     \begin{enumerate}
          \item %
          Lorsque l'on exécute le programme de Clara on obtient \texttt{'1010101010555'} comme résultat.
          \par
          En effet les valeurs entrées par Clara sont considérées comme étant des chaînes de caractères.
          \par
          On a donc \texttt{a='10'} et \texttt{5*a='1010101010'} (répétition de \texttt{10} cinq fois)~;\\
          \texttt{b='5'} et \texttt{3*b='555'} (répétition de \texttt{5} trois fois)~;\\
          et \texttt{5*a+3*b='1010101010555'} (concaténation).
          \item %
          Pour que le programme fonctionne comme Clara le souhaiterait, il faut convertir \texttt{a} et \texttt{b} en un type numérique~: \texttt{int} ou \texttt{float}.
          \par
          Par exemple~:
 \begin{lstlisting}[language=Python]
a=input('Entrez la valeur de a : ')
b=input('Entrez la valeur de b : ')
a=int(a)
b=int(b)
c=5*a+3*b
print(c)
     \end{lstlisting}
     ou plus simplement~:
\begin{lstlisting}[language=Python]
a=int(input('Entrez la valeur de a : '))
b=int(input('Entrez la valeur de b : '))
c=5*a+3*b
print(c)
\end{lstlisting}
\end{enumerate}
\end{corrige}

\end{document}
µ
\documentclass[a4paper]{article}

%================================================================================================================================
%
% Packages
%
%================================================================================================================================

\usepackage[T1]{fontenc} 	% pour caractères accentués
\usepackage[utf8]{inputenc}  % encodage utf8
\usepackage[french]{babel}	% langue : français
\usepackage{fourier}			% caractères plus lisibles
\usepackage[dvipsnames]{xcolor} % couleurs
\usepackage{fancyhdr}		% réglage header footer
\usepackage{needspace}		% empêcher sauts de page mal placés
\usepackage{graphicx}		% pour inclure des graphiques
\usepackage{enumitem,cprotect}		% personnalise les listes d'items (nécessaire pour ol, al ...)
\usepackage{hyperref}		% Liens hypertexte
\usepackage{pstricks,pst-all,pst-node,pstricks-add,pst-math,pst-plot,pst-tree,pst-eucl} % pstricks
\usepackage[a4paper,includeheadfoot,top=2cm,left=3cm, bottom=2cm,right=3cm]{geometry} % marges etc.
\usepackage{comment}			% commentaires multilignes
\usepackage{amsmath,environ} % maths (matrices, etc.)
\usepackage{amssymb,makeidx}
\usepackage{bm}				% bold maths
\usepackage{tabularx}		% tableaux
\usepackage{colortbl}		% tableaux en couleur
\usepackage{fontawesome}		% Fontawesome
\usepackage{environ}			% environment with command
\usepackage{fp}				% calculs pour ps-tricks
\usepackage{multido}			% pour ps tricks
\usepackage[np]{numprint}	% formattage nombre
\usepackage{tikz,tkz-tab} 			% package principal TikZ
\usepackage{pgfplots}   % axes
\usepackage{mathrsfs}    % cursives
\usepackage{calc}			% calcul taille boites
\usepackage[scaled=0.875]{helvet} % font sans serif
\usepackage{svg} % svg
\usepackage{scrextend} % local margin
\usepackage{scratch} %scratch
\usepackage{multicol} % colonnes
%\usepackage{infix-RPN,pst-func} % formule en notation polanaise inversée
\usepackage{listings}

%================================================================================================================================
%
% Réglages de base
%
%================================================================================================================================

\lstset{
language=Python,   % R code
literate=
{á}{{\'a}}1
{à}{{\`a}}1
{ã}{{\~a}}1
{é}{{\'e}}1
{è}{{\`e}}1
{ê}{{\^e}}1
{í}{{\'i}}1
{ó}{{\'o}}1
{õ}{{\~o}}1
{ú}{{\'u}}1
{ü}{{\"u}}1
{ç}{{\c{c}}}1
{~}{{ }}1
}


\definecolor{codegreen}{rgb}{0,0.6,0}
\definecolor{codegray}{rgb}{0.5,0.5,0.5}
\definecolor{codepurple}{rgb}{0.58,0,0.82}
\definecolor{backcolour}{rgb}{0.95,0.95,0.92}

\lstdefinestyle{mystyle}{
    backgroundcolor=\color{backcolour},   
    commentstyle=\color{codegreen},
    keywordstyle=\color{magenta},
    numberstyle=\tiny\color{codegray},
    stringstyle=\color{codepurple},
    basicstyle=\ttfamily\footnotesize,
    breakatwhitespace=false,         
    breaklines=true,                 
    captionpos=b,                    
    keepspaces=true,                 
    numbers=left,                    
xleftmargin=2em,
framexleftmargin=2em,            
    showspaces=false,                
    showstringspaces=false,
    showtabs=false,                  
    tabsize=2,
    upquote=true
}

\lstset{style=mystyle}


\lstset{style=mystyle}
\newcommand{\imgdir}{C:/laragon/www/newmc/assets/imgsvg/}
\newcommand{\imgsvgdir}{C:/laragon/www/newmc/assets/imgsvg/}

\definecolor{mcgris}{RGB}{220, 220, 220}% ancien~; pour compatibilité
\definecolor{mcbleu}{RGB}{52, 152, 219}
\definecolor{mcvert}{RGB}{125, 194, 70}
\definecolor{mcmauve}{RGB}{154, 0, 215}
\definecolor{mcorange}{RGB}{255, 96, 0}
\definecolor{mcturquoise}{RGB}{0, 153, 153}
\definecolor{mcrouge}{RGB}{255, 0, 0}
\definecolor{mclightvert}{RGB}{205, 234, 190}

\definecolor{gris}{RGB}{220, 220, 220}
\definecolor{bleu}{RGB}{52, 152, 219}
\definecolor{vert}{RGB}{125, 194, 70}
\definecolor{mauve}{RGB}{154, 0, 215}
\definecolor{orange}{RGB}{255, 96, 0}
\definecolor{turquoise}{RGB}{0, 153, 153}
\definecolor{rouge}{RGB}{255, 0, 0}
\definecolor{lightvert}{RGB}{205, 234, 190}
\setitemize[0]{label=\color{lightvert}  $\bullet$}

\pagestyle{fancy}
\renewcommand{\headrulewidth}{0.2pt}
\fancyhead[L]{maths-cours.fr}
\fancyhead[R]{\thepage}
\renewcommand{\footrulewidth}{0.2pt}
\fancyfoot[C]{}

\newcolumntype{C}{>{\centering\arraybackslash}X}
\newcolumntype{s}{>{\hsize=.35\hsize\arraybackslash}X}

\setlength{\parindent}{0pt}		 
\setlength{\parskip}{3mm}
\setlength{\headheight}{1cm}

\def\ebook{ebook}
\def\book{book}
\def\web{web}
\def\type{web}

\newcommand{\vect}[1]{\overrightarrow{\,\mathstrut#1\,}}

\def\Oij{$\left(\text{O}~;~\vect{\imath},~\vect{\jmath}\right)$}
\def\Oijk{$\left(\text{O}~;~\vect{\imath},~\vect{\jmath},~\vect{k}\right)$}
\def\Ouv{$\left(\text{O}~;~\vect{u},~\vect{v}\right)$}

\hypersetup{breaklinks=true, colorlinks = true, linkcolor = OliveGreen, urlcolor = OliveGreen, citecolor = OliveGreen, pdfauthor={Didier BONNEL - https://www.maths-cours.fr} } % supprime les bordures autour des liens

\renewcommand{\arg}[0]{\text{arg}}

\everymath{\displaystyle}

%================================================================================================================================
%
% Macros - Commandes
%
%================================================================================================================================

\newcommand\meta[2]{    			% Utilisé pour créer le post HTML.
	\def\titre{titre}
	\def\url{url}
	\def\arg{#1}
	\ifx\titre\arg
		\newcommand\maintitle{#2}
		\fancyhead[L]{#2}
		{\Large\sffamily \MakeUppercase{#2}}
		\vspace{1mm}\textcolor{mcvert}{\hrule}
	\fi 
	\ifx\url\arg
		\fancyfoot[L]{\href{https://www.maths-cours.fr#2}{\black \footnotesize{https://www.maths-cours.fr#2}}}
	\fi 
}


\newcommand\TitreC[1]{    		% Titre centré
     \needspace{3\baselineskip}
     \begin{center}\textbf{#1}\end{center}
}

\newcommand\newpar{    		% paragraphe
     \par
}

\newcommand\nosp {    		% commande vide (pas d'espace)
}
\newcommand{\id}[1]{} %ignore

\newcommand\boite[2]{				% Boite simple sans titre
	\vspace{5mm}
	\setlength{\fboxrule}{0.2mm}
	\setlength{\fboxsep}{5mm}	
	\fcolorbox{#1}{#1!3}{\makebox[\linewidth-2\fboxrule-2\fboxsep]{
  		\begin{minipage}[t]{\linewidth-2\fboxrule-4\fboxsep}\setlength{\parskip}{3mm}
  			 #2
  		\end{minipage}
	}}
	\vspace{5mm}
}

\newcommand\CBox[4]{				% Boites
	\vspace{5mm}
	\setlength{\fboxrule}{0.2mm}
	\setlength{\fboxsep}{5mm}
	
	\fcolorbox{#1}{#1!3}{\makebox[\linewidth-2\fboxrule-2\fboxsep]{
		\begin{minipage}[t]{1cm}\setlength{\parskip}{3mm}
	  		\textcolor{#1}{\LARGE{#2}}    
 	 	\end{minipage}  
  		\begin{minipage}[t]{\linewidth-2\fboxrule-4\fboxsep}\setlength{\parskip}{3mm}
			\raisebox{1.2mm}{\normalsize\sffamily{\textcolor{#1}{#3}}}						
  			 #4
  		\end{minipage}
	}}
	\vspace{5mm}
}

\newcommand\cadre[3]{				% Boites convertible html
	\par
	\vspace{2mm}
	\setlength{\fboxrule}{0.1mm}
	\setlength{\fboxsep}{5mm}
	\fcolorbox{#1}{white}{\makebox[\linewidth-2\fboxrule-2\fboxsep]{
  		\begin{minipage}[t]{\linewidth-2\fboxrule-4\fboxsep}\setlength{\parskip}{3mm}
			\raisebox{-2.5mm}{\sffamily \small{\textcolor{#1}{\MakeUppercase{#2}}}}		
			\par		
  			 #3
 	 		\end{minipage}
	}}
		\vspace{2mm}
	\par
}

\newcommand\bloc[3]{				% Boites convertible html sans bordure
     \needspace{2\baselineskip}
     {\sffamily \small{\textcolor{#1}{\MakeUppercase{#2}}}}    
		\par		
  			 #3
		\par
}

\newcommand\CHelp[1]{
     \CBox{Plum}{\faInfoCircle}{À RETENIR}{#1}
}

\newcommand\CUp[1]{
     \CBox{NavyBlue}{\faThumbsOUp}{EN PRATIQUE}{#1}
}

\newcommand\CInfo[1]{
     \CBox{Sepia}{\faArrowCircleRight}{REMARQUE}{#1}
}

\newcommand\CRedac[1]{
     \CBox{PineGreen}{\faEdit}{BIEN R\'EDIGER}{#1}
}

\newcommand\CError[1]{
     \CBox{Red}{\faExclamationTriangle}{ATTENTION}{#1}
}

\newcommand\TitreExo[2]{
\needspace{4\baselineskip}
 {\sffamily\large EXERCICE #1\ (\emph{#2 points})}
\vspace{5mm}
}

\newcommand\img[2]{
          \includegraphics[width=#2\paperwidth]{\imgdir#1}
}

\newcommand\imgsvg[2]{
       \begin{center}   \includegraphics[width=#2\paperwidth]{\imgsvgdir#1} \end{center}
}


\newcommand\Lien[2]{
     \href{#1}{#2 \tiny \faExternalLink}
}
\newcommand\mcLien[2]{
     \href{https~://www.maths-cours.fr/#1}{#2 \tiny \faExternalLink}
}

\newcommand{\euro}{\eurologo{}}

%================================================================================================================================
%
% Macros - Environement
%
%================================================================================================================================

\newenvironment{tex}{ %
}
{%
}

\newenvironment{indente}{ %
	\setlength\parindent{10mm}
}

{
	\setlength\parindent{0mm}
}

\newenvironment{corrige}{%
     \needspace{3\baselineskip}
     \medskip
     \textbf{\textsc{Corrigé}}
     \medskip
}
{
}

\newenvironment{extern}{%
     \begin{center}
     }
     {
     \end{center}
}

\NewEnviron{code}{%
	\par
     \boite{gray}{\texttt{%
     \BODY
     }}
     \par
}

\newenvironment{vbloc}{% boite sans cadre empeche saut de page
     \begin{minipage}[t]{\linewidth}
     }
     {
     \end{minipage}
}
\NewEnviron{h2}{%
    \needspace{3\baselineskip}
    \vspace{0.6cm}
	\noindent \MakeUppercase{\sffamily \large \BODY}
	\vspace{1mm}\textcolor{mcgris}{\hrule}\vspace{0.4cm}
	\par
}{}

\NewEnviron{h3}{%
    \needspace{3\baselineskip}
	\vspace{5mm}
	\textsc{\BODY}
	\par
}

\NewEnviron{margeneg}{ %
\begin{addmargin}[-1cm]{0cm}
\BODY
\end{addmargin}
}

\NewEnviron{html}{%
}

\begin{document}
\meta{url}{/exercices/differents-types-de-nombres/}
\meta{pid}{11111}
\meta{titre}{Différents types de nombres}
\meta{type}{exercices}
%
Compléter le tableau ci-dessous avec les symboles $\in$ ou $\notin$~:
\begin{center}
     \begin{tabular}{|c|c|c|c|c|c|c|} %class="compact" width="600"
          \hline
          &  $\mathbb{N}$ &   $\mathbb{Z}$  &  $\mathbb{D}$  &  $\mathbb{Q}$  &  $\mathbb{R}$
          \\ \hline
          $-2$  &  $\notin$ & $\in$   &   &   &
          \\ \hline
          $\dfrac{6}{3}$  &   &   &   &   &
          \\ \hline
          $ \sqrt{3}$  &   &   &   &   &
          \\ \hline
          $ -\dfrac{3}{5} $&   &  &   &   &
          \\ \hline
          $\dfrac{5}{7}$  &   &  &   &   &
          \\ \hline
          $\left(\sqrt{2}-1\right)^2$  &   &  &   &   &
          \\ \hline
          $\left(\sqrt{3}-\sqrt{2}\right)\left(\sqrt{3}+\sqrt{2}\right)$  &   &  &   &   &
          \\ \hline
     \end{tabular}
\end{center}
Par exemple : $-2 \notin \mathbb{N}$ et $-2 \in \mathbb{Z}.$
\begin{corrige}
     \begin{center}
          \begin{tabular}{|c|c|c|c|c|c|c|} %class="compact" width="600"
               \hline
               &  $\mathbb{N}$ &   $\mathbb{Z}$  &  $\mathbb{D}$  &  $\mathbb{Q}$  &  $\mathbb{R}$
               \\ \hline
               $-2$  &  $\notin$ & $\in$   & $\in$  & $\in$  &  $\in$
               \\ \hline
               $\dfrac{6}{3}$  & $\in$  & $\in$  &  $\in$ & $\in$  &  $\in$
               \\ \hline
               $ \sqrt{3}$  & $\notin$  & $\notin$  &  $\notin$ & $\notin$  &  $\in$
               \\ \hline
               $-\dfrac{3}{5}$  & $\notin$  & $\notin$  &  $\in$ & $\in$  &  $\in$
               \\ \hline
               $\dfrac{5}{7}$   & $\notin$  & $\notin$  &  $\notin$ & $\in$  &  $\in$
               \\ \hline
               $\left(\sqrt{2}-1\right)^2$   & $\notin$  & $\notin$  &  $\notin$ & $\notin$  &  $\in$
               \\ \hline
               $\left(\sqrt{3}-\sqrt{2}\right)\left(\sqrt{3}+\sqrt{2}\right)$  & $\in$  & $\in$  &  $\in$ & $\in$  &  $\in$
               \\ \hline
          \end{tabular}
     \end{center}
     \par
     \textbf{Explications~:}
     \par
     \begin{itemize}
          \item %
          Les ensembles de nombres sont placés en ordre croissant~; donc lorsqu'un nombre appartient à un ensemble, il appartient nécessairement aux ensembles suivants.
          \item %
          Tous les nombres présents sont des nombres réels.
          \item %
          $\dfrac{6}{3}=2 \in \mathbb{N}$
          \item %
          $ \sqrt{3}$ n'est pas un nombre rationnel donc  $ \sqrt{3} \notin \mathbb{Q}$~; par contre, comme tous les autres nombres de ce tableau,  $ \sqrt{3} \in \mathbb{R}.$
          \item %
          $ -\dfrac{3}{5}=-0,6 \in \mathbb{D}.$
          \item %
          $ \dfrac{5}{7}$ n'est pas un nombre décimal car son écriture décimale est illimitée.
          \item %
          $\left(\sqrt{2}-1\right)^2$ se développe grâce à l'identité remarquable~:
          \[ (a-b)^2 = a^2-2ab+b^2 \]
          \par
          On obtient alors~:\\
          $\left(\sqrt{2}-1\right)^2= \sqrt{2}^2-2\sqrt{2}+1^2$\nosp$=2-2\sqrt{2}+1=3-2\sqrt{2}$
          \par
          Ce n'est pas un nombre rationnel car $\sqrt{2}$ est irrationnel.
          \item %
          $\left(\sqrt{3}-\sqrt{2}\right)\left(\sqrt{3}+\sqrt{2}\right)$ se développe grâce à l'identité remarquable~:
          \[ (a-b)(a+b) = a^2-b^2 \]
          \par
          Cela donne ici~:\\
          $\left(\sqrt{3}-\sqrt{2}\right)\left(\sqrt{3}+\sqrt{2}\right)= \sqrt{3}^2-\sqrt{2}^2$\nosp$=3-2=1 \in \mathbb{N}$
     \end{itemize}
\end{corrige}

\end{document}

µ
\documentclass[a4paper]{article}

%================================================================================================================================
%
% Packages
%
%================================================================================================================================

\usepackage[T1]{fontenc} 	% pour caractères accentués
\usepackage[utf8]{inputenc}  % encodage utf8
\usepackage[french]{babel}	% langue : français
\usepackage{fourier}			% caractères plus lisibles
\usepackage[dvipsnames]{xcolor} % couleurs
\usepackage{fancyhdr}		% réglage header footer
\usepackage{needspace}		% empêcher sauts de page mal placés
\usepackage{graphicx}		% pour inclure des graphiques
\usepackage{enumitem,cprotect}		% personnalise les listes d'items (nécessaire pour ol, al ...)
\usepackage{hyperref}		% Liens hypertexte
\usepackage{pstricks,pst-all,pst-node,pstricks-add,pst-math,pst-plot,pst-tree,pst-eucl} % pstricks
\usepackage[a4paper,includeheadfoot,top=2cm,left=3cm, bottom=2cm,right=3cm]{geometry} % marges etc.
\usepackage{comment}			% commentaires multilignes
\usepackage{amsmath,environ} % maths (matrices, etc.)
\usepackage{amssymb,makeidx}
\usepackage{bm}				% bold maths
\usepackage{tabularx}		% tableaux
\usepackage{colortbl}		% tableaux en couleur
\usepackage{fontawesome}		% Fontawesome
\usepackage{environ}			% environment with command
\usepackage{fp}				% calculs pour ps-tricks
\usepackage{multido}			% pour ps tricks
\usepackage[np]{numprint}	% formattage nombre
\usepackage{tikz,tkz-tab} 			% package principal TikZ
\usepackage{pgfplots}   % axes
\usepackage{mathrsfs}    % cursives
\usepackage{calc}			% calcul taille boites
\usepackage[scaled=0.875]{helvet} % font sans serif
\usepackage{svg} % svg
\usepackage{scrextend} % local margin
\usepackage{scratch} %scratch
\usepackage{multicol} % colonnes
%\usepackage{infix-RPN,pst-func} % formule en notation polanaise inversée
\usepackage{listings}

%================================================================================================================================
%
% Réglages de base
%
%================================================================================================================================

\lstset{
language=Python,   % R code
literate=
{á}{{\'a}}1
{à}{{\`a}}1
{ã}{{\~a}}1
{é}{{\'e}}1
{è}{{\`e}}1
{ê}{{\^e}}1
{í}{{\'i}}1
{ó}{{\'o}}1
{õ}{{\~o}}1
{ú}{{\'u}}1
{ü}{{\"u}}1
{ç}{{\c{c}}}1
{~}{{ }}1
}


\definecolor{codegreen}{rgb}{0,0.6,0}
\definecolor{codegray}{rgb}{0.5,0.5,0.5}
\definecolor{codepurple}{rgb}{0.58,0,0.82}
\definecolor{backcolour}{rgb}{0.95,0.95,0.92}

\lstdefinestyle{mystyle}{
    backgroundcolor=\color{backcolour},   
    commentstyle=\color{codegreen},
    keywordstyle=\color{magenta},
    numberstyle=\tiny\color{codegray},
    stringstyle=\color{codepurple},
    basicstyle=\ttfamily\footnotesize,
    breakatwhitespace=false,         
    breaklines=true,                 
    captionpos=b,                    
    keepspaces=true,                 
    numbers=left,                    
xleftmargin=2em,
framexleftmargin=2em,            
    showspaces=false,                
    showstringspaces=false,
    showtabs=false,                  
    tabsize=2,
    upquote=true
}

\lstset{style=mystyle}


\lstset{style=mystyle}
\newcommand{\imgdir}{C:/laragon/www/newmc/assets/imgsvg/}
\newcommand{\imgsvgdir}{C:/laragon/www/newmc/assets/imgsvg/}

\definecolor{mcgris}{RGB}{220, 220, 220}% ancien~; pour compatibilité
\definecolor{mcbleu}{RGB}{52, 152, 219}
\definecolor{mcvert}{RGB}{125, 194, 70}
\definecolor{mcmauve}{RGB}{154, 0, 215}
\definecolor{mcorange}{RGB}{255, 96, 0}
\definecolor{mcturquoise}{RGB}{0, 153, 153}
\definecolor{mcrouge}{RGB}{255, 0, 0}
\definecolor{mclightvert}{RGB}{205, 234, 190}

\definecolor{gris}{RGB}{220, 220, 220}
\definecolor{bleu}{RGB}{52, 152, 219}
\definecolor{vert}{RGB}{125, 194, 70}
\definecolor{mauve}{RGB}{154, 0, 215}
\definecolor{orange}{RGB}{255, 96, 0}
\definecolor{turquoise}{RGB}{0, 153, 153}
\definecolor{rouge}{RGB}{255, 0, 0}
\definecolor{lightvert}{RGB}{205, 234, 190}
\setitemize[0]{label=\color{lightvert}  $\bullet$}

\pagestyle{fancy}
\renewcommand{\headrulewidth}{0.2pt}
\fancyhead[L]{maths-cours.fr}
\fancyhead[R]{\thepage}
\renewcommand{\footrulewidth}{0.2pt}
\fancyfoot[C]{}

\newcolumntype{C}{>{\centering\arraybackslash}X}
\newcolumntype{s}{>{\hsize=.35\hsize\arraybackslash}X}

\setlength{\parindent}{0pt}		 
\setlength{\parskip}{3mm}
\setlength{\headheight}{1cm}

\def\ebook{ebook}
\def\book{book}
\def\web{web}
\def\type{web}

\newcommand{\vect}[1]{\overrightarrow{\,\mathstrut#1\,}}

\def\Oij{$\left(\text{O}~;~\vect{\imath},~\vect{\jmath}\right)$}
\def\Oijk{$\left(\text{O}~;~\vect{\imath},~\vect{\jmath},~\vect{k}\right)$}
\def\Ouv{$\left(\text{O}~;~\vect{u},~\vect{v}\right)$}

\hypersetup{breaklinks=true, colorlinks = true, linkcolor = OliveGreen, urlcolor = OliveGreen, citecolor = OliveGreen, pdfauthor={Didier BONNEL - https://www.maths-cours.fr} } % supprime les bordures autour des liens

\renewcommand{\arg}[0]{\text{arg}}

\everymath{\displaystyle}

%================================================================================================================================
%
% Macros - Commandes
%
%================================================================================================================================

\newcommand\meta[2]{    			% Utilisé pour créer le post HTML.
	\def\titre{titre}
	\def\url{url}
	\def\arg{#1}
	\ifx\titre\arg
		\newcommand\maintitle{#2}
		\fancyhead[L]{#2}
		{\Large\sffamily \MakeUppercase{#2}}
		\vspace{1mm}\textcolor{mcvert}{\hrule}
	\fi 
	\ifx\url\arg
		\fancyfoot[L]{\href{https://www.maths-cours.fr#2}{\black \footnotesize{https://www.maths-cours.fr#2}}}
	\fi 
}


\newcommand\TitreC[1]{    		% Titre centré
     \needspace{3\baselineskip}
     \begin{center}\textbf{#1}\end{center}
}

\newcommand\newpar{    		% paragraphe
     \par
}

\newcommand\nosp {    		% commande vide (pas d'espace)
}
\newcommand{\id}[1]{} %ignore

\newcommand\boite[2]{				% Boite simple sans titre
	\vspace{5mm}
	\setlength{\fboxrule}{0.2mm}
	\setlength{\fboxsep}{5mm}	
	\fcolorbox{#1}{#1!3}{\makebox[\linewidth-2\fboxrule-2\fboxsep]{
  		\begin{minipage}[t]{\linewidth-2\fboxrule-4\fboxsep}\setlength{\parskip}{3mm}
  			 #2
  		\end{minipage}
	}}
	\vspace{5mm}
}

\newcommand\CBox[4]{				% Boites
	\vspace{5mm}
	\setlength{\fboxrule}{0.2mm}
	\setlength{\fboxsep}{5mm}
	
	\fcolorbox{#1}{#1!3}{\makebox[\linewidth-2\fboxrule-2\fboxsep]{
		\begin{minipage}[t]{1cm}\setlength{\parskip}{3mm}
	  		\textcolor{#1}{\LARGE{#2}}    
 	 	\end{minipage}  
  		\begin{minipage}[t]{\linewidth-2\fboxrule-4\fboxsep}\setlength{\parskip}{3mm}
			\raisebox{1.2mm}{\normalsize\sffamily{\textcolor{#1}{#3}}}						
  			 #4
  		\end{minipage}
	}}
	\vspace{5mm}
}

\newcommand\cadre[3]{				% Boites convertible html
	\par
	\vspace{2mm}
	\setlength{\fboxrule}{0.1mm}
	\setlength{\fboxsep}{5mm}
	\fcolorbox{#1}{white}{\makebox[\linewidth-2\fboxrule-2\fboxsep]{
  		\begin{minipage}[t]{\linewidth-2\fboxrule-4\fboxsep}\setlength{\parskip}{3mm}
			\raisebox{-2.5mm}{\sffamily \small{\textcolor{#1}{\MakeUppercase{#2}}}}		
			\par		
  			 #3
 	 		\end{minipage}
	}}
		\vspace{2mm}
	\par
}

\newcommand\bloc[3]{				% Boites convertible html sans bordure
     \needspace{2\baselineskip}
     {\sffamily \small{\textcolor{#1}{\MakeUppercase{#2}}}}    
		\par		
  			 #3
		\par
}

\newcommand\CHelp[1]{
     \CBox{Plum}{\faInfoCircle}{À RETENIR}{#1}
}

\newcommand\CUp[1]{
     \CBox{NavyBlue}{\faThumbsOUp}{EN PRATIQUE}{#1}
}

\newcommand\CInfo[1]{
     \CBox{Sepia}{\faArrowCircleRight}{REMARQUE}{#1}
}

\newcommand\CRedac[1]{
     \CBox{PineGreen}{\faEdit}{BIEN R\'EDIGER}{#1}
}

\newcommand\CError[1]{
     \CBox{Red}{\faExclamationTriangle}{ATTENTION}{#1}
}

\newcommand\TitreExo[2]{
\needspace{4\baselineskip}
 {\sffamily\large EXERCICE #1\ (\emph{#2 points})}
\vspace{5mm}
}

\newcommand\img[2]{
          \includegraphics[width=#2\paperwidth]{\imgdir#1}
}

\newcommand\imgsvg[2]{
       \begin{center}   \includegraphics[width=#2\paperwidth]{\imgsvgdir#1} \end{center}
}


\newcommand\Lien[2]{
     \href{#1}{#2 \tiny \faExternalLink}
}
\newcommand\mcLien[2]{
     \href{https~://www.maths-cours.fr/#1}{#2 \tiny \faExternalLink}
}

\newcommand{\euro}{\eurologo{}}

%================================================================================================================================
%
% Macros - Environement
%
%================================================================================================================================

\newenvironment{tex}{ %
}
{%
}

\newenvironment{indente}{ %
	\setlength\parindent{10mm}
}

{
	\setlength\parindent{0mm}
}

\newenvironment{corrige}{%
     \needspace{3\baselineskip}
     \medskip
     \textbf{\textsc{Corrigé}}
     \medskip
}
{
}

\newenvironment{extern}{%
     \begin{center}
     }
     {
     \end{center}
}

\NewEnviron{code}{%
	\par
     \boite{gray}{\texttt{%
     \BODY
     }}
     \par
}

\newenvironment{vbloc}{% boite sans cadre empeche saut de page
     \begin{minipage}[t]{\linewidth}
     }
     {
     \end{minipage}
}
\NewEnviron{h2}{%
    \needspace{3\baselineskip}
    \vspace{0.6cm}
	\noindent \MakeUppercase{\sffamily \large \BODY}
	\vspace{1mm}\textcolor{mcgris}{\hrule}\vspace{0.4cm}
	\par
}{}

\NewEnviron{h3}{%
    \needspace{3\baselineskip}
	\vspace{5mm}
	\textsc{\BODY}
	\par
}

\NewEnviron{margeneg}{ %
\begin{addmargin}[-1cm]{0cm}
\BODY
\end{addmargin}
}

\NewEnviron{html}{%
}

\begin{document}
\meta{url}{/exercices/ensembles-de-nombres-appartenance-et-inclusion/}
\meta{pid}{11116}
\meta{titre}{Ensembles de nombres : appartenance et inclusion}
\meta{type}{exercices}
%
Compléter chacune des lignes ci-dessous à l'aide d'un des symboles $\in$, $\subset$, $\notin$ ou $\cancel{\subset}$~:
\begin{enumerate}[label=\alph*.]
     \item %
     $\quad \pi \; \ldots \;  \mathbb{Q}$
     \item %
     $\quad \mathbb{Z} \; \ldots \;  \mathbb{Q}$
     \item %
     $\quad \mathbb{R} \; \ldots \;  \mathbb{Z}$
     \item %
     $\quad -\dfrac{6}{12} \; \ldots \;  \mathbb{D}$
     \item %
     $\quad -\dfrac{12}{6} \; \ldots \;  \mathbb{Z}$
     \item %
     $\quad \left\{\sqrt{3}\right\} \; \ldots \;  \mathbb{R}$
     \item %
     $\quad \left\{-1~;~0~;~1\right\} \; \ldots \;  \mathbb{N}$
     \item %
     $\quad \mathbb{R}^\star \; \ldots \;  \mathbb{R}$
\end{enumerate}
\begin{corrige}
     \begin{enumerate}[label=\alph*.]
          \item %
          $\pi \; \notin \;  \mathbb{Q}$
          \par
          $\pi$ est un nombre irrationnel (il ne peut pas s'écrire comme quotient de deux entiers). Donc il n'appartient pas à l'ensemble $\mathbb{Q}$ des nombres rationnels.
          \item %
          $\mathbb{Z} \; \subset \;  \mathbb{Q}$
          \par
          Tous les nombres entiers relatifs sont des nombres rationnels. L'ensemble $\mathbb{Z}$ est donc inclus dans l'ensemble $\mathbb{Q}$. On met le signe $\subset$ et non $\in$ car $\mathbb{Z}$ est un ensemble et non un nombre.
          \item %
          $ \mathbb{R} \; \cancel{\subset} \;  \mathbb{Z}$
          \par
          Tous les nombres réels ne sont pas des entiers relatifs (par exemple $\dfrac{1}{2}$, $\dfrac{1}{3}$, $\pi$,$ \ldots$ ne sont pas entiers). Donc l'ensemble $\mathbb{R}$ n'est pas inclus dans $\mathbb{Z}$.
          \item %
          $ -\dfrac{6}{12} \; \in \;  \mathbb{D}$
          \par
          $ - \dfrac{6}{12}=-\dfrac{1}{2}=-0,5$ est un nombre décimal. Il appartient donc à l'ensemble $\mathbb{D}$ des nombres décimaux.
          \item %
          $ - \dfrac{12}{6} \; \in \;  \mathbb{Z}$
          \par
          $ - \dfrac{12}{6}=-2$ est un nombre entier relatif. Il appartient donc à l'ensemble $\mathbb{Z}$ des nombres entiers relatifs.
          \item %
          $ \left\{\sqrt{3}\right\} \; \subset \;  \mathbb{R}$
          \par
          À cause des accolades, $\left\{\sqrt{3}\right\}$ représente un \textbf{ensemble}. C'est un ensemble qui contient uniquement le nombre réel $\sqrt{3}$. Cet ensemble est donc inclus dans $\mathbb{R}$ (\og inclus \fg{} et non \og appartient \fg{} car il s'agit d'un ensemble).
          \item %
          $ \left\{-1~;~0~;~1\right\} \; \cancel{\subset} \quad \mathbb{N}$
          \par
          L'ensemble $\left\{-1~;~0~;~1\right\} $ contient le nombre $-1$ qui n'est pas un entier naturel. Il n'est donc pas inclus dans l'ensemble $\mathbb{N}.$
          \item %
          $\mathbb{R}^\star \; \subset \;  \mathbb{R}$
          \par
          L'ensemble $\mathbb{R}^\star$ est l'ensemble de tous les nombres réels à l'exception de 0. Cet ensemble est donc inclus dans  $\mathbb{R}.$
     \end{enumerate}
\end{corrige}

\end{document}

µ
\documentclass[a4paper]{article}

%================================================================================================================================
%
% Packages
%
%================================================================================================================================

\usepackage[T1]{fontenc} 	% pour caractères accentués
\usepackage[utf8]{inputenc}  % encodage utf8
\usepackage[french]{babel}	% langue : français
\usepackage{fourier}			% caractères plus lisibles
\usepackage[dvipsnames]{xcolor} % couleurs
\usepackage{fancyhdr}		% réglage header footer
\usepackage{needspace}		% empêcher sauts de page mal placés
\usepackage{graphicx}		% pour inclure des graphiques
\usepackage{enumitem,cprotect}		% personnalise les listes d'items (nécessaire pour ol, al ...)
\usepackage{hyperref}		% Liens hypertexte
\usepackage{pstricks,pst-all,pst-node,pstricks-add,pst-math,pst-plot,pst-tree,pst-eucl} % pstricks
\usepackage[a4paper,includeheadfoot,top=2cm,left=3cm, bottom=2cm,right=3cm]{geometry} % marges etc.
\usepackage{comment}			% commentaires multilignes
\usepackage{amsmath,environ} % maths (matrices, etc.)
\usepackage{amssymb,makeidx}
\usepackage{bm}				% bold maths
\usepackage{tabularx}		% tableaux
\usepackage{colortbl}		% tableaux en couleur
\usepackage{fontawesome}		% Fontawesome
\usepackage{environ}			% environment with command
\usepackage{fp}				% calculs pour ps-tricks
\usepackage{multido}			% pour ps tricks
\usepackage[np]{numprint}	% formattage nombre
\usepackage{tikz,tkz-tab} 			% package principal TikZ
\usepackage{pgfplots}   % axes
\usepackage{mathrsfs}    % cursives
\usepackage{calc}			% calcul taille boites
\usepackage[scaled=0.875]{helvet} % font sans serif
\usepackage{svg} % svg
\usepackage{scrextend} % local margin
\usepackage{scratch} %scratch
\usepackage{multicol} % colonnes
%\usepackage{infix-RPN,pst-func} % formule en notation polanaise inversée
\usepackage{listings}

%================================================================================================================================
%
% Réglages de base
%
%================================================================================================================================

\lstset{
language=Python,   % R code
literate=
{á}{{\'a}}1
{à}{{\`a}}1
{ã}{{\~a}}1
{é}{{\'e}}1
{è}{{\`e}}1
{ê}{{\^e}}1
{í}{{\'i}}1
{ó}{{\'o}}1
{õ}{{\~o}}1
{ú}{{\'u}}1
{ü}{{\"u}}1
{ç}{{\c{c}}}1
{~}{{ }}1
}


\definecolor{codegreen}{rgb}{0,0.6,0}
\definecolor{codegray}{rgb}{0.5,0.5,0.5}
\definecolor{codepurple}{rgb}{0.58,0,0.82}
\definecolor{backcolour}{rgb}{0.95,0.95,0.92}

\lstdefinestyle{mystyle}{
    backgroundcolor=\color{backcolour},   
    commentstyle=\color{codegreen},
    keywordstyle=\color{magenta},
    numberstyle=\tiny\color{codegray},
    stringstyle=\color{codepurple},
    basicstyle=\ttfamily\footnotesize,
    breakatwhitespace=false,         
    breaklines=true,                 
    captionpos=b,                    
    keepspaces=true,                 
    numbers=left,                    
xleftmargin=2em,
framexleftmargin=2em,            
    showspaces=false,                
    showstringspaces=false,
    showtabs=false,                  
    tabsize=2,
    upquote=true
}

\lstset{style=mystyle}


\lstset{style=mystyle}
\newcommand{\imgdir}{C:/laragon/www/newmc/assets/imgsvg/}
\newcommand{\imgsvgdir}{C:/laragon/www/newmc/assets/imgsvg/}

\definecolor{mcgris}{RGB}{220, 220, 220}% ancien~; pour compatibilité
\definecolor{mcbleu}{RGB}{52, 152, 219}
\definecolor{mcvert}{RGB}{125, 194, 70}
\definecolor{mcmauve}{RGB}{154, 0, 215}
\definecolor{mcorange}{RGB}{255, 96, 0}
\definecolor{mcturquoise}{RGB}{0, 153, 153}
\definecolor{mcrouge}{RGB}{255, 0, 0}
\definecolor{mclightvert}{RGB}{205, 234, 190}

\definecolor{gris}{RGB}{220, 220, 220}
\definecolor{bleu}{RGB}{52, 152, 219}
\definecolor{vert}{RGB}{125, 194, 70}
\definecolor{mauve}{RGB}{154, 0, 215}
\definecolor{orange}{RGB}{255, 96, 0}
\definecolor{turquoise}{RGB}{0, 153, 153}
\definecolor{rouge}{RGB}{255, 0, 0}
\definecolor{lightvert}{RGB}{205, 234, 190}
\setitemize[0]{label=\color{lightvert}  $\bullet$}

\pagestyle{fancy}
\renewcommand{\headrulewidth}{0.2pt}
\fancyhead[L]{maths-cours.fr}
\fancyhead[R]{\thepage}
\renewcommand{\footrulewidth}{0.2pt}
\fancyfoot[C]{}

\newcolumntype{C}{>{\centering\arraybackslash}X}
\newcolumntype{s}{>{\hsize=.35\hsize\arraybackslash}X}

\setlength{\parindent}{0pt}		 
\setlength{\parskip}{3mm}
\setlength{\headheight}{1cm}

\def\ebook{ebook}
\def\book{book}
\def\web{web}
\def\type{web}

\newcommand{\vect}[1]{\overrightarrow{\,\mathstrut#1\,}}

\def\Oij{$\left(\text{O}~;~\vect{\imath},~\vect{\jmath}\right)$}
\def\Oijk{$\left(\text{O}~;~\vect{\imath},~\vect{\jmath},~\vect{k}\right)$}
\def\Ouv{$\left(\text{O}~;~\vect{u},~\vect{v}\right)$}

\hypersetup{breaklinks=true, colorlinks = true, linkcolor = OliveGreen, urlcolor = OliveGreen, citecolor = OliveGreen, pdfauthor={Didier BONNEL - https://www.maths-cours.fr} } % supprime les bordures autour des liens

\renewcommand{\arg}[0]{\text{arg}}

\everymath{\displaystyle}

%================================================================================================================================
%
% Macros - Commandes
%
%================================================================================================================================

\newcommand\meta[2]{    			% Utilisé pour créer le post HTML.
	\def\titre{titre}
	\def\url{url}
	\def\arg{#1}
	\ifx\titre\arg
		\newcommand\maintitle{#2}
		\fancyhead[L]{#2}
		{\Large\sffamily \MakeUppercase{#2}}
		\vspace{1mm}\textcolor{mcvert}{\hrule}
	\fi 
	\ifx\url\arg
		\fancyfoot[L]{\href{https://www.maths-cours.fr#2}{\black \footnotesize{https://www.maths-cours.fr#2}}}
	\fi 
}


\newcommand\TitreC[1]{    		% Titre centré
     \needspace{3\baselineskip}
     \begin{center}\textbf{#1}\end{center}
}

\newcommand\newpar{    		% paragraphe
     \par
}

\newcommand\nosp {    		% commande vide (pas d'espace)
}
\newcommand{\id}[1]{} %ignore

\newcommand\boite[2]{				% Boite simple sans titre
	\vspace{5mm}
	\setlength{\fboxrule}{0.2mm}
	\setlength{\fboxsep}{5mm}	
	\fcolorbox{#1}{#1!3}{\makebox[\linewidth-2\fboxrule-2\fboxsep]{
  		\begin{minipage}[t]{\linewidth-2\fboxrule-4\fboxsep}\setlength{\parskip}{3mm}
  			 #2
  		\end{minipage}
	}}
	\vspace{5mm}
}

\newcommand\CBox[4]{				% Boites
	\vspace{5mm}
	\setlength{\fboxrule}{0.2mm}
	\setlength{\fboxsep}{5mm}
	
	\fcolorbox{#1}{#1!3}{\makebox[\linewidth-2\fboxrule-2\fboxsep]{
		\begin{minipage}[t]{1cm}\setlength{\parskip}{3mm}
	  		\textcolor{#1}{\LARGE{#2}}    
 	 	\end{minipage}  
  		\begin{minipage}[t]{\linewidth-2\fboxrule-4\fboxsep}\setlength{\parskip}{3mm}
			\raisebox{1.2mm}{\normalsize\sffamily{\textcolor{#1}{#3}}}						
  			 #4
  		\end{minipage}
	}}
	\vspace{5mm}
}

\newcommand\cadre[3]{				% Boites convertible html
	\par
	\vspace{2mm}
	\setlength{\fboxrule}{0.1mm}
	\setlength{\fboxsep}{5mm}
	\fcolorbox{#1}{white}{\makebox[\linewidth-2\fboxrule-2\fboxsep]{
  		\begin{minipage}[t]{\linewidth-2\fboxrule-4\fboxsep}\setlength{\parskip}{3mm}
			\raisebox{-2.5mm}{\sffamily \small{\textcolor{#1}{\MakeUppercase{#2}}}}		
			\par		
  			 #3
 	 		\end{minipage}
	}}
		\vspace{2mm}
	\par
}

\newcommand\bloc[3]{				% Boites convertible html sans bordure
     \needspace{2\baselineskip}
     {\sffamily \small{\textcolor{#1}{\MakeUppercase{#2}}}}    
		\par		
  			 #3
		\par
}

\newcommand\CHelp[1]{
     \CBox{Plum}{\faInfoCircle}{À RETENIR}{#1}
}

\newcommand\CUp[1]{
     \CBox{NavyBlue}{\faThumbsOUp}{EN PRATIQUE}{#1}
}

\newcommand\CInfo[1]{
     \CBox{Sepia}{\faArrowCircleRight}{REMARQUE}{#1}
}

\newcommand\CRedac[1]{
     \CBox{PineGreen}{\faEdit}{BIEN R\'EDIGER}{#1}
}

\newcommand\CError[1]{
     \CBox{Red}{\faExclamationTriangle}{ATTENTION}{#1}
}

\newcommand\TitreExo[2]{
\needspace{4\baselineskip}
 {\sffamily\large EXERCICE #1\ (\emph{#2 points})}
\vspace{5mm}
}

\newcommand\img[2]{
          \includegraphics[width=#2\paperwidth]{\imgdir#1}
}

\newcommand\imgsvg[2]{
       \begin{center}   \includegraphics[width=#2\paperwidth]{\imgsvgdir#1} \end{center}
}


\newcommand\Lien[2]{
     \href{#1}{#2 \tiny \faExternalLink}
}
\newcommand\mcLien[2]{
     \href{https~://www.maths-cours.fr/#1}{#2 \tiny \faExternalLink}
}

\newcommand{\euro}{\eurologo{}}

%================================================================================================================================
%
% Macros - Environement
%
%================================================================================================================================

\newenvironment{tex}{ %
}
{%
}

\newenvironment{indente}{ %
	\setlength\parindent{10mm}
}

{
	\setlength\parindent{0mm}
}

\newenvironment{corrige}{%
     \needspace{3\baselineskip}
     \medskip
     \textbf{\textsc{Corrigé}}
     \medskip
}
{
}

\newenvironment{extern}{%
     \begin{center}
     }
     {
     \end{center}
}

\NewEnviron{code}{%
	\par
     \boite{gray}{\texttt{%
     \BODY
     }}
     \par
}

\newenvironment{vbloc}{% boite sans cadre empeche saut de page
     \begin{minipage}[t]{\linewidth}
     }
     {
     \end{minipage}
}
\NewEnviron{h2}{%
    \needspace{3\baselineskip}
    \vspace{0.6cm}
	\noindent \MakeUppercase{\sffamily \large \BODY}
	\vspace{1mm}\textcolor{mcgris}{\hrule}\vspace{0.4cm}
	\par
}{}

\NewEnviron{h3}{%
    \needspace{3\baselineskip}
	\vspace{5mm}
	\textsc{\BODY}
	\par
}

\NewEnviron{margeneg}{ %
\begin{addmargin}[-1cm]{0cm}
\BODY
\end{addmargin}
}

\NewEnviron{html}{%
}

\begin{document}
\meta{url}{/exercices/produit-scalaire-et-quadrillage-calcul-dangle/}
\meta{pid}{11160}
\meta{titre}{Produit scalaire et quadrillage. Calcul d'angle.}
\meta{type}{exercices}
%
Dans cet exercice, l'unité de longueur correspond au côté d'un carré du quadrillage.
\begin{center}
     \begin{extern}%width="500" alt="Produit Scalaire et quadrillage"
          \resizebox{8cm}{!}{
               % prod-scal-10.ggb
               \begin{tikzpicture}[line cap=round,line join=round,>=triangle 45,x=1.0cm,y=1.0cm]
                    \draw [color=cqcqcq,, xstep=1.0cm,ystep=1.0cm] (3.,4.) grid (11.,9.);
                    \clip(3.,4.) rectangle (11.,9.);
                    \draw [line width=0.4pt,color=tttttt] (4.,5.)-- (8.,8.);
                    \draw [line width=0.4pt,color=tttttt] (8.,8.)-- (10.,5.);
                    \draw [line width=0.4pt,color=tttttt] (10.,5.)-- (4.,5.);
                    \begin{scriptsize}
                         \draw [fill=tttttt] (8.,8.) circle (1.0pt);
                         \draw[color=tttttt] (8.015646625983543,8.306133358394499) node {$A$};
                         \draw [fill=tttttt] (10.,5.) circle (1.0pt);
                         \draw[color=tttttt] (10.08277841649217,4.688652725004405) node {$B$};
                         \draw [fill=tttttt] (4.,5.) circle (1.0pt);
                         \draw[color=tttttt] (3.7952525536950965,4.7132614367961745) node {$C$};
                    \end{scriptsize}
               \end{tikzpicture}
          }
     \end{extern}
\end{center}
\begin{enumerate}
     \item
     À l'aide du quadrillage, calculez le produit scalaire  $ \overrightarrow{CB}  \cdot \overrightarrow{CA} $ puis les normes
     $\left\Vert \overrightarrow{CB} \right\Vert $ et $\left\Vert \overrightarrow{CA} \right\Vert $ .
     \item
     En déduire la valeur exacte de $ \cos\left( \overrightarrow{CB} ;\overrightarrow{CA}  \right) $ .\\
     Donner la valeur arrondie au degré de l'angle $ \left( \overrightarrow{CB} ; \overrightarrow{CA}  \right)$ .
\end{enumerate}
\begin{corrige}
     \begin{enumerate}
          \item
          Tout d'abord, traçons le projeté orthogonal $H$ du point  $A$ sur la droite $(CB) $~:
          \begin{center}
               \begin{extern}%width="500" alt="Produit Scalaire et quadrillage"
                    \resizebox{8cm}{!}{
                         \begin{tikzpicture}[line cap=round,line join=round,>=triangle 45,x=1.0cm,y=1.0cm]
                              \draw [color=cqcqcq,, xstep=1.0cm,ystep=1.0cm] (3.,4.) grid (11.,9.);
                              \clip(3.,4.) rectangle (11.,9.);
                              \draw[line width=0.4pt,color=sqsqsq] (8.261014804763382,5.) -- (8.261014804763382,5.2610148047633825) -- (8.,5.2610148047633825) -- (8.,5.) -- cycle;
                              \draw [line width=0.4pt,color=tttttt] (4.,5.)-- (8.,8.);
                              \draw [line width=0.4pt,color=tttttt] (8.,8.)-- (10.,5.);
                              \draw [line width=0.4pt,color=tttttt] (10.,5.)-- (4.,5.);
                              \draw [line width=0.4pt,color=ffqqqq] (8.,8.)-- (8.,5.);
                              \begin{scriptsize}
                                   \draw [fill=tttttt] (8.,8.) circle (1.0pt);
                                   \draw[color=tttttt] (8.015646625983543,8.306133358394499) node {$A$};
                                   \draw [fill=tttttt] (10.,5.) circle (1.0pt);
                                   \draw[color=tttttt] (10.08277841649217,4.688652725004405) node {$B$};
                                   \draw [fill=tttttt] (4.,5.) circle (1.0pt);
                                   \draw[color=tttttt] (3.7952525536950965,4.7132614367961745) node {$C$};
                                   \draw [fill=ffqqqq] (8.,5.) circle (1.0pt);
                                   \draw[color=ffqqqq] (8.114081473150621,4.7132614367961745) node {$H$};
                              \end{scriptsize}
                         \end{tikzpicture}
                    }
               \end{extern}
          \end{center}
          Comme l'angle $\left( \overrightarrow{CB} ; \overrightarrow{CA}  \right) $ est un angle aigu~:
          \newpar
          $\overrightarrow{CB}  \cdot \overrightarrow{CA} =CB \times CH$\\
          $\overrightarrow{CB}  \cdot \overrightarrow{CA} =6 \times 4=24.$
          \medbreak
          Par ailleurs, on a immédiatement~:
          \newpar
          $\left\Vert \overrightarrow{CB} \right\Vert =CB=6.$
          \medbreak
          Pour calculer la longueur du segment $[CA] $, on utilise le théorème de Pythagore dans le triangle $AHC$ rectangle en $H$~:
          \newpar
          $AC{}^2 =CH{}^2 +HA{}^2 =4{}^2 +3{}^2 $\\
          $\phantom{AC{}^2 }=16+9=25$
          \newpar
          Donc :\\
          $ \left\Vert \overrightarrow{CA} \right\Vert =AC=\sqrt{25} = 5$
          \item
          Pour calculer la valeur de  $ \cos\left( \overrightarrow{CB} ;\overrightarrow{CA}  \right) $, on utilise la formule donnant le produit scalaire à l'aide du cosinus~:
          \newpar
          $\overrightarrow{CB}  \cdot \overrightarrow{CA} = \left\Vert \overrightarrow{CB} \right\Vert  \times \left\Vert \overrightarrow{CA} \right\Vert  \times \cos\left( \overrightarrow{CB} ;\overrightarrow{CA}  \right)$
          \newpar
          On en déduit :
          \newpar
          $\cos\left( \overrightarrow{CB} ;\overrightarrow{CA}  \right) = \frac{\overrightarrow{CB}  \cdot \overrightarrow{CA} }{ \left\Vert \overrightarrow{CB} \right\Vert  \times \left\Vert \overrightarrow{CA} \right\Vert } $
          \newpar
          $\cos\left( \overrightarrow{CB} ;\overrightarrow{CA}  \right) =\frac{24}{6 \times 5} =0,8.$
          \medbreak
          À la calculatrice (touche \og $\cos{}^{ - 1}$  \fg{}  ou \og Arccos \fg{} ), on trouve que l'angle $\left( \overrightarrow{CB} ;\overrightarrow{CA}  \right)$ vaut approximativement 37° au degré près.
          \medbreak
          Remarque : il était aussi possible et plus simple, ici, de calculer une valeur approchée de l'angle $\left( \overrightarrow{CB} ;\overrightarrow{CA}  \right)$ à l'aide des formules trigonométriques vues en classe de troisième dans le triangle rectangle $AHC$.
     \end{enumerate}
\end{corrige}

\end{document}
µ
\documentclass[a4paper]{article}

%================================================================================================================================
%
% Packages
%
%================================================================================================================================

\usepackage[T1]{fontenc} 	% pour caractères accentués
\usepackage[utf8]{inputenc}  % encodage utf8
\usepackage[french]{babel}	% langue : français
\usepackage{fourier}			% caractères plus lisibles
\usepackage[dvipsnames]{xcolor} % couleurs
\usepackage{fancyhdr}		% réglage header footer
\usepackage{needspace}		% empêcher sauts de page mal placés
\usepackage{graphicx}		% pour inclure des graphiques
\usepackage{enumitem,cprotect}		% personnalise les listes d'items (nécessaire pour ol, al ...)
\usepackage{hyperref}		% Liens hypertexte
\usepackage{pstricks,pst-all,pst-node,pstricks-add,pst-math,pst-plot,pst-tree,pst-eucl} % pstricks
\usepackage[a4paper,includeheadfoot,top=2cm,left=3cm, bottom=2cm,right=3cm]{geometry} % marges etc.
\usepackage{comment}			% commentaires multilignes
\usepackage{amsmath,environ} % maths (matrices, etc.)
\usepackage{amssymb,makeidx}
\usepackage{bm}				% bold maths
\usepackage{tabularx}		% tableaux
\usepackage{colortbl}		% tableaux en couleur
\usepackage{fontawesome}		% Fontawesome
\usepackage{environ}			% environment with command
\usepackage{fp}				% calculs pour ps-tricks
\usepackage{multido}			% pour ps tricks
\usepackage[np]{numprint}	% formattage nombre
\usepackage{tikz,tkz-tab} 			% package principal TikZ
\usepackage{pgfplots}   % axes
\usepackage{mathrsfs}    % cursives
\usepackage{calc}			% calcul taille boites
\usepackage[scaled=0.875]{helvet} % font sans serif
\usepackage{svg} % svg
\usepackage{scrextend} % local margin
\usepackage{scratch} %scratch
\usepackage{multicol} % colonnes
%\usepackage{infix-RPN,pst-func} % formule en notation polanaise inversée
\usepackage{listings}

%================================================================================================================================
%
% Réglages de base
%
%================================================================================================================================

\lstset{
language=Python,   % R code
literate=
{á}{{\'a}}1
{à}{{\`a}}1
{ã}{{\~a}}1
{é}{{\'e}}1
{è}{{\`e}}1
{ê}{{\^e}}1
{í}{{\'i}}1
{ó}{{\'o}}1
{õ}{{\~o}}1
{ú}{{\'u}}1
{ü}{{\"u}}1
{ç}{{\c{c}}}1
{~}{{ }}1
}


\definecolor{codegreen}{rgb}{0,0.6,0}
\definecolor{codegray}{rgb}{0.5,0.5,0.5}
\definecolor{codepurple}{rgb}{0.58,0,0.82}
\definecolor{backcolour}{rgb}{0.95,0.95,0.92}

\lstdefinestyle{mystyle}{
    backgroundcolor=\color{backcolour},   
    commentstyle=\color{codegreen},
    keywordstyle=\color{magenta},
    numberstyle=\tiny\color{codegray},
    stringstyle=\color{codepurple},
    basicstyle=\ttfamily\footnotesize,
    breakatwhitespace=false,         
    breaklines=true,                 
    captionpos=b,                    
    keepspaces=true,                 
    numbers=left,                    
xleftmargin=2em,
framexleftmargin=2em,            
    showspaces=false,                
    showstringspaces=false,
    showtabs=false,                  
    tabsize=2,
    upquote=true
}

\lstset{style=mystyle}


\lstset{style=mystyle}
\newcommand{\imgdir}{C:/laragon/www/newmc/assets/imgsvg/}
\newcommand{\imgsvgdir}{C:/laragon/www/newmc/assets/imgsvg/}

\definecolor{mcgris}{RGB}{220, 220, 220}% ancien~; pour compatibilité
\definecolor{mcbleu}{RGB}{52, 152, 219}
\definecolor{mcvert}{RGB}{125, 194, 70}
\definecolor{mcmauve}{RGB}{154, 0, 215}
\definecolor{mcorange}{RGB}{255, 96, 0}
\definecolor{mcturquoise}{RGB}{0, 153, 153}
\definecolor{mcrouge}{RGB}{255, 0, 0}
\definecolor{mclightvert}{RGB}{205, 234, 190}

\definecolor{gris}{RGB}{220, 220, 220}
\definecolor{bleu}{RGB}{52, 152, 219}
\definecolor{vert}{RGB}{125, 194, 70}
\definecolor{mauve}{RGB}{154, 0, 215}
\definecolor{orange}{RGB}{255, 96, 0}
\definecolor{turquoise}{RGB}{0, 153, 153}
\definecolor{rouge}{RGB}{255, 0, 0}
\definecolor{lightvert}{RGB}{205, 234, 190}
\setitemize[0]{label=\color{lightvert}  $\bullet$}

\pagestyle{fancy}
\renewcommand{\headrulewidth}{0.2pt}
\fancyhead[L]{maths-cours.fr}
\fancyhead[R]{\thepage}
\renewcommand{\footrulewidth}{0.2pt}
\fancyfoot[C]{}

\newcolumntype{C}{>{\centering\arraybackslash}X}
\newcolumntype{s}{>{\hsize=.35\hsize\arraybackslash}X}

\setlength{\parindent}{0pt}		 
\setlength{\parskip}{3mm}
\setlength{\headheight}{1cm}

\def\ebook{ebook}
\def\book{book}
\def\web{web}
\def\type{web}

\newcommand{\vect}[1]{\overrightarrow{\,\mathstrut#1\,}}

\def\Oij{$\left(\text{O}~;~\vect{\imath},~\vect{\jmath}\right)$}
\def\Oijk{$\left(\text{O}~;~\vect{\imath},~\vect{\jmath},~\vect{k}\right)$}
\def\Ouv{$\left(\text{O}~;~\vect{u},~\vect{v}\right)$}

\hypersetup{breaklinks=true, colorlinks = true, linkcolor = OliveGreen, urlcolor = OliveGreen, citecolor = OliveGreen, pdfauthor={Didier BONNEL - https://www.maths-cours.fr} } % supprime les bordures autour des liens

\renewcommand{\arg}[0]{\text{arg}}

\everymath{\displaystyle}

%================================================================================================================================
%
% Macros - Commandes
%
%================================================================================================================================

\newcommand\meta[2]{    			% Utilisé pour créer le post HTML.
	\def\titre{titre}
	\def\url{url}
	\def\arg{#1}
	\ifx\titre\arg
		\newcommand\maintitle{#2}
		\fancyhead[L]{#2}
		{\Large\sffamily \MakeUppercase{#2}}
		\vspace{1mm}\textcolor{mcvert}{\hrule}
	\fi 
	\ifx\url\arg
		\fancyfoot[L]{\href{https://www.maths-cours.fr#2}{\black \footnotesize{https://www.maths-cours.fr#2}}}
	\fi 
}


\newcommand\TitreC[1]{    		% Titre centré
     \needspace{3\baselineskip}
     \begin{center}\textbf{#1}\end{center}
}

\newcommand\newpar{    		% paragraphe
     \par
}

\newcommand\nosp {    		% commande vide (pas d'espace)
}
\newcommand{\id}[1]{} %ignore

\newcommand\boite[2]{				% Boite simple sans titre
	\vspace{5mm}
	\setlength{\fboxrule}{0.2mm}
	\setlength{\fboxsep}{5mm}	
	\fcolorbox{#1}{#1!3}{\makebox[\linewidth-2\fboxrule-2\fboxsep]{
  		\begin{minipage}[t]{\linewidth-2\fboxrule-4\fboxsep}\setlength{\parskip}{3mm}
  			 #2
  		\end{minipage}
	}}
	\vspace{5mm}
}

\newcommand\CBox[4]{				% Boites
	\vspace{5mm}
	\setlength{\fboxrule}{0.2mm}
	\setlength{\fboxsep}{5mm}
	
	\fcolorbox{#1}{#1!3}{\makebox[\linewidth-2\fboxrule-2\fboxsep]{
		\begin{minipage}[t]{1cm}\setlength{\parskip}{3mm}
	  		\textcolor{#1}{\LARGE{#2}}    
 	 	\end{minipage}  
  		\begin{minipage}[t]{\linewidth-2\fboxrule-4\fboxsep}\setlength{\parskip}{3mm}
			\raisebox{1.2mm}{\normalsize\sffamily{\textcolor{#1}{#3}}}						
  			 #4
  		\end{minipage}
	}}
	\vspace{5mm}
}

\newcommand\cadre[3]{				% Boites convertible html
	\par
	\vspace{2mm}
	\setlength{\fboxrule}{0.1mm}
	\setlength{\fboxsep}{5mm}
	\fcolorbox{#1}{white}{\makebox[\linewidth-2\fboxrule-2\fboxsep]{
  		\begin{minipage}[t]{\linewidth-2\fboxrule-4\fboxsep}\setlength{\parskip}{3mm}
			\raisebox{-2.5mm}{\sffamily \small{\textcolor{#1}{\MakeUppercase{#2}}}}		
			\par		
  			 #3
 	 		\end{minipage}
	}}
		\vspace{2mm}
	\par
}

\newcommand\bloc[3]{				% Boites convertible html sans bordure
     \needspace{2\baselineskip}
     {\sffamily \small{\textcolor{#1}{\MakeUppercase{#2}}}}    
		\par		
  			 #3
		\par
}

\newcommand\CHelp[1]{
     \CBox{Plum}{\faInfoCircle}{À RETENIR}{#1}
}

\newcommand\CUp[1]{
     \CBox{NavyBlue}{\faThumbsOUp}{EN PRATIQUE}{#1}
}

\newcommand\CInfo[1]{
     \CBox{Sepia}{\faArrowCircleRight}{REMARQUE}{#1}
}

\newcommand\CRedac[1]{
     \CBox{PineGreen}{\faEdit}{BIEN R\'EDIGER}{#1}
}

\newcommand\CError[1]{
     \CBox{Red}{\faExclamationTriangle}{ATTENTION}{#1}
}

\newcommand\TitreExo[2]{
\needspace{4\baselineskip}
 {\sffamily\large EXERCICE #1\ (\emph{#2 points})}
\vspace{5mm}
}

\newcommand\img[2]{
          \includegraphics[width=#2\paperwidth]{\imgdir#1}
}

\newcommand\imgsvg[2]{
       \begin{center}   \includegraphics[width=#2\paperwidth]{\imgsvgdir#1} \end{center}
}


\newcommand\Lien[2]{
     \href{#1}{#2 \tiny \faExternalLink}
}
\newcommand\mcLien[2]{
     \href{https~://www.maths-cours.fr/#1}{#2 \tiny \faExternalLink}
}

\newcommand{\euro}{\eurologo{}}

%================================================================================================================================
%
% Macros - Environement
%
%================================================================================================================================

\newenvironment{tex}{ %
}
{%
}

\newenvironment{indente}{ %
	\setlength\parindent{10mm}
}

{
	\setlength\parindent{0mm}
}

\newenvironment{corrige}{%
     \needspace{3\baselineskip}
     \medskip
     \textbf{\textsc{Corrigé}}
     \medskip
}
{
}

\newenvironment{extern}{%
     \begin{center}
     }
     {
     \end{center}
}

\NewEnviron{code}{%
	\par
     \boite{gray}{\texttt{%
     \BODY
     }}
     \par
}

\newenvironment{vbloc}{% boite sans cadre empeche saut de page
     \begin{minipage}[t]{\linewidth}
     }
     {
     \end{minipage}
}
\NewEnviron{h2}{%
    \needspace{3\baselineskip}
    \vspace{0.6cm}
	\noindent \MakeUppercase{\sffamily \large \BODY}
	\vspace{1mm}\textcolor{mcgris}{\hrule}\vspace{0.4cm}
	\par
}{}

\NewEnviron{h3}{%
    \needspace{3\baselineskip}
	\vspace{5mm}
	\textsc{\BODY}
	\par
}

\NewEnviron{margeneg}{ %
\begin{addmargin}[-1cm]{0cm}
\BODY
\end{addmargin}
}

\NewEnviron{html}{%
}

\begin{document}
\meta{url}{/methode/5-methodes-pour-calculer-un-produit-scalaire/}
\meta{pid}{11181}
\meta{titre}{5 méthodes pour calculer un produit scalaire}
\meta{type}{methode}
%
\cadre{vert}{Propriété}{ % id=100
     Il existe de nombreuses méthodes permettant de calculer un produit scalaire. C'est, en partie, ce qui fait la puissance de cet outil en mathématiques.
     \newpar
     Nous allons voir, dans ce chapitre, 5 des principales méthodes utilisées en classe de Première pour calculer un produit scalaire :
     \begin{enumerate}
          \item
          Utiliser une projection orthogonale,
          \item
          Appliquer une formule utilisant le cosinus d'un angle,
          \item
          Appliquer une formule utilisant les normes de 3 vecteurs,
          \item
          Se placer dans un repère orthonormé,
          \item
          Utiliser la relation de Chasles.
     \end{enumerate}
}% fin propriété
\begin{h2}1. Utiliser une projection orthogonale\end{h2}
Pour calculer le produit scalaire $ \overrightarrow{AB}  \cdot \overrightarrow{AC} $ , on projette orthogonalement le point $C$ sur la droite $(AB)$ .\\
Notons $H$ ce projeté orthogonal~:\\
\begin{center}
     \begin{extern} %width="450" alt="produit scalaire - projection orthogonale "
          \resizebox{8cm}{!}{
               % g11181-1
               \begin{tikzpicture}[line cap=round,line join=round,>=triangle 45,x=1.0cm,y=1.0cm]
                    \clip(0.,0.) rectangle (8.,6.);
                    \draw[line width=0.4pt,color=sqsqsq] (3.084751040430065,1.6525414932616558) -- (3.1027646208139212,1.7606229755647935) -- (2.9946831385107835,1.7786365559486499) -- (2.976669558126927,1.6705550736455121) -- cycle;
                    \draw [->,line width=0.4pt,color=tttttt] (1.,2.) -- (7.,1.);
                    \draw [->,line width=0.4pt,color=tttttt] (1.,2.) -- (3.532317271885081,5.0044413561944365);
                    \draw [line width=0.4pt,color=qqqqff] (3.532317271885081,5.0044413561944365)-- (2.976669558126927,1.6705550736455121);
                    \begin{scriptsize}
                         \draw [fill=tttttt] (1.,2.) circle (0.5pt);
                         \draw[color=tttttt] (0.9245867768595036,1.8248490077653157) node {$A$};
                         \draw [fill=tttttt] (7.,1.) circle (0.5pt);
                         \draw[color=tttttt] (7.055785123966939,0.8567730802415879) node {$B$};
                         \draw [fill=qqqqff] (2.976669558126927,1.6705550736455121) circle (0.5pt);
                         \draw[color=qqqqff] (2.9287190082644616,1.4676445211389135) node {$H$};
                         \draw [fill=black] (3.532317271885081,5.0044413561944365) circle (0.5pt);
                         \draw[color=black] (3.5485537190082628,5.194995685936154) node {$C$};
                    \end{scriptsize}
               \end{tikzpicture}
          }
     \end{extern}
\end{center}
On utilise alors le théorème suivant (voir \mcLien{https://www.maths-cours.fr/cours/produit-scalaire\#t50}{cours})~:\\
\cadre{rouge}{Théorème}{% id="t50"
     Soient $A, B, C$ trois points du plan et si $H$ est la projection orthogonale de $C$ sur la droite $\left(AB\right).$
     \par
     Alors :
     \begin{itemize}
          \item $\overrightarrow{AB} \cdot \overrightarrow{AC}=AB\times AH   $ si l'angle $\widehat{BAC}$ est aigu
          \item $\overrightarrow{AB} \cdot \overrightarrow{AC}=-AB\times AH   $ si l'angle $\widehat{BAC}$ est obtus
     \end{itemize}
}
\bloc{cyan}{Remarque}{% id="r60"
     \begin{itemize}
          \item
          Dire que l'angle $\widehat{BAC}$ est aigu revient à dire que les vecteurs  $\overrightarrow{AB}$ et  $\overrightarrow{AH}$ ont le même sens.
          \item
          Dire que l'angle $\widehat{BAC}$ est obtus revient à dire que les vecteurs  $\overrightarrow{AB}$ et  $\overrightarrow{AH}$ ont des sens opposés.
     \end{itemize}
}% fin remarque
\bloc{orange}{Exemple}{ % id="e70"
     Sur la figure ci-dessous,  $ABCD$  est un carré de côté 4 unités et  $I$  et le milieu du segment  $[AB]$.\\
     On cherche à calculer la valeur du produit scalaire $\overrightarrow{IB}  \cdot \overrightarrow{ID} $.
     \begin{center}
          \begin{extern} %width="300" alt="produit scalaire - exemple de projection orthogonale "
               \resizebox{8cm}{!}{
                    %
                    \begin{tikzpicture}[line cap=round,line join=round,>=triangle 45,x=1.0cm,y=1.0cm]
                         \clip(0.,0.) rectangle (8.,6.);
                         \draw [line width=0.4pt,color=tttttt] (2.,5.)-- (2.,1.);
                         \draw [line width=0.4pt,color=tttttt] (6.,1.)-- (6.,5.);
                         \draw [line width=0.4pt,color=tttttt] (6.,5.)-- (2.,5.);
                         \draw [line width=0.4pt,color=tttttt] (2.,1.)-- (4.,1.);
                         \draw [line width=0.4pt,color=tttttt] (4.,1.)-- (6.,1.);
                         \draw [->,line width=0.4pt,color=qqqqff] (4.,1.) -- (6.,1.);
                         \draw [->,line width=0.4pt,color=qqqqff] (4.,1.) -- (2.,5.);
                         \begin{scriptsize}
                              \draw [fill=tttttt] (2.,1.) circle (0.5pt);
                              \draw[color=tttttt] (1.9111570247933876,0.8050043140638484) node {$A$};
                              \draw [fill=tttttt] (6.,1.) circle (0.5pt);
                              \draw[color=tttttt] (6.038223140495865,0.8153580672993963) node {$B$};
                              \draw [fill=tttttt] (6.,5.) circle (0.5pt);
                              \draw[color=tttttt] (6.048553719008262,5.210526315789475) node {$C$};
                              \draw [fill=tttttt] (2.,5.) circle (0.5pt);
                              \draw[color=tttttt] (1.8801652892561975,5.179465056082832) node {$D$};
                              \draw [fill=tttttt] (4.,1.) circle (0.5pt);
                              \draw[color=tttttt] (4.003099173553717,0.8153580672993963) node {$I$};
                         \end{scriptsize}
                    \end{tikzpicture}
               }
          \end{extern}
     \end{center}
     La méthode utilisant la projection orthogonale est particulièrement bien adaptée ici puisque l'on connaît la projection orthogonale $A$ du point $D$ sur la droite $(IB).$
     \newpar
     L'angle $ \widehat{DIB}$ est ici un angle obtus. \\
     Les segments $IB$ et $AI$ mesure chacun 2 unités.\\
     On a donc~:
     \newpar
     $\overrightarrow{IB} \cdot \overrightarrow{ID}= - IB \times IA $\\
     $\overrightarrow{IB} \cdot \overrightarrow{ID}= - 2 \times 2= - 4$
}% fin exemple
\begin{h2}2. Appliquer une formule utilisant le cosinus d'un angle\end{h2}
\begin{center}
     \begin{extern} %width="400" alt="Produit scalaire à partir de la mesure d'un angle"
          \resizebox{8cm}{!}{
               %
               \begin{tikzpicture}[line cap=round,line join=round,>=triangle 45,x=1.0cm,y=1.0cm]
                    \clip(0.,0.) rectangle (8.,6.);
                    \draw [shift={(1.,2.)},line width=0.4pt,color=ffqqqq,fill=ffqqqq,fill opacity=0.10000000149011612] (0,0) -- (-9.462322208025617:0.3615702479338841) arc (-9.462322208025617:49.87386950127897:0.3615702479338841) -- cycle;
                    \draw [->,line width=0.4pt,color=tttttt] (1.,2.) -- (7.,1.);
                    \draw [->,line width=0.4pt,color=tttttt] (1.,2.) -- (3.532317271885081,5.0044413561944365);
                    \begin{scriptsize}
                         \draw [fill=tttttt] (1.,2.) circle (0.5pt);
                         \draw[color=tttttt] (0.9245867768595036,1.8248490077653157) node {$A$};
                         \draw [fill=tttttt] (7.,1.) circle (0.5pt);
                         \draw[color=tttttt] (7.055785123966939,0.8567730802415879) node {$B$};
                         \draw [fill=black] (3.532317271885081,5.0044413561944365) circle (0.5pt);
                         \draw[color=black] (3.5485537190082628,5.194995685936154) node {$C$};
                    \end{scriptsize}
               \end{tikzpicture}
          }
     \end{extern}
\end{center}
Si l'on connaît l'angle $ \widehat{BAC}$, on peut calculer le produit scalaire $ \overrightarrow{AB}  \cdot \overrightarrow{AC} $ en utilisant les longueurs $AB$ et  $AC$ ainsi que le cosinus de l'angle $ \widehat{BAC}$(Voir \mcLien{https://www.maths-cours.fr/cours/produit-scalaire\#d10}{Définition du produit scalaire.})
\cadre{bleu}{Définition}{% id="d100"
     Le \textbf{produit scalaire} de $ \overrightarrow{AB} $ et $ \overrightarrow{AC} $ est le \textbf{nombre réel} noté $ \overrightarrow{AB}  \cdot  \overrightarrow{AC} $ défini par :
     \begin{center}$\overrightarrow{AB}  \cdot  \overrightarrow{AC} =AB \times AC \times  \cos \left(\overrightarrow{AB} ; \overrightarrow{AC} \right) $\end{center}
}
\bloc{cyan}{Remarque}{% id="r110"
     Le sens de l'angle n'a pas d'importance dans cette formule puisque pour tout angle $\theta \ :$ $\cos \theta =\cos( -  \theta ).$\\
     On peut donc aussi bien utiliser des angles orientés ( comme $ \left(\overrightarrow{AB} ; \overrightarrow{AC} \right) $ )  que des angles géométriques ( comme $ \widehat{BAC}$ ).
} % fin remarque
\bloc{orange}{Exemple}{ % id="e120"
     Pour la figure ci-dessous, on souhaite déterminer une valeur approchée à  $10{}^{ - 2}$  près du produit scalaire $\overrightarrow{AB} \cdot \overrightarrow{AC}$ .
     \begin{center}
          \begin{extern} %width="400" alt="Calcul du produit scalaire à partir du cosinus"
               \resizebox{8cm}{!}{
                    %
                    \begin{tikzpicture}[line cap=round,line join=round,>=triangle 45,x=1.0cm,y=1.0cm]
                         \clip(0.,1.) rectangle (7.,4.6);
                         \draw [shift={(1.,2.)},line width=0.4pt,color=ffqqqq,fill=ffqqqq,fill opacity=0.10000000149011612] (0,0) -- (0.:0.36157024793388415) arc (0.:50.1826180001947:0.36157024793388415) -- cycle;
                         \draw [line width=0.4pt,color=tttttt] (2.6472246317724393,3.97584124245039)-- (6.,2.);
                         \draw [line width=0.4pt,color=tttttt] (2.6472246317724393,3.97584124245039)-- (1.,2.);
                         \draw [color=ffqqqq](1.6,2.2) node{\scriptsize 50°};
                         \draw [color=tttttt](3.4400826446280974,1.770491803278689) node {\scriptsize 12};
                         \draw [color=tttttt](1.6735537190082637,3.2510785159620372) node {\scriptsize 6};
                         \draw [line width=0.4pt,color=tttttt] (1.,2.)-- (6.,2.);
                         \begin{scriptsize}
                              \draw [fill=tttttt] (1.,2.) circle (0.5pt);
                              \draw[color=tttttt] (0.9245867768595036,1.8248490077653154) node {$A$};
                              \draw [fill=tttttt] (6.,2.) circle (0.5pt);
                              \draw[color=tttttt] (6.058884297520659,1.855910267471959) node {$B$};
                              \draw [fill=black] (2.6472246317724393,3.97584124245039) circle (0.5pt);
                              \draw[color=black] (2.665289256198346,4.164797238999138) node {$C$};
                         \end{scriptsize}
                    \end{tikzpicture}
               }
          \end{extern}
     \end{center}
     Bien sûr, on utilise la définition du produit scalaire à l'aide des angles puisqu'ici on connaît l'angle  $ \widehat{BAC}$ .
     \newpar
     $\overrightarrow{AB} \cdot \overrightarrow{AC}=AB  \times AC \times \cos \widehat{BAC}$\\
     $\overrightarrow{AB} \cdot \overrightarrow{AC}=12 \times 6 \times \cos(50 \degree)$\\
     $\overrightarrow{AB} \cdot \overrightarrow{AC} \approx 12 \times 6 \times 0,643 \approx 46,28.$
}% fin exemple
\begin{h2}3. Appliquer une formule utilisant les normes de 3 vecteurs\end{h2}
Lorsque l'on connaît trois distances, par exemple, les longueurs des trois côtés d'un triangle, On peut calculer un produit scalaire en utilisant l'une des égalités ci-dessous (Voir \mcLien{https://www.maths-cours.fr/cours/produit-scalaire\#t40}{propriété})~:
\cadre{rouge}{Théorème}{% id="t40"
     Pour tous vecteurs $\vec{u}$ et $\vec{v}$ :
     \begin{center}$\vec{u} \cdot \vec{v}=\frac{1}{2} \left(||\vec{u}+\vec{v}||^{2}-||\vec{u}||^{2}-||\vec{v}||^{2}\right)$\end{center}
     \begin{center} $\vec{u} \cdot \vec{v}=\frac{1}{2}\left(\left\Vert \vec{u}\right\Vert{}^2 +\left\Vert \vec{v}\right\Vert{}^2  - \left\Vert \vec{u} - \vec{v}\right\Vert{}^2 \right)$
     \end{center}
}
Cette formule est particulièrement utile lorsque l'on connaît les trois côtés d'un triangle ou lorsque l'on connaît 2 côtés et la médiane issus du même point~; on utilise alors souvent une des relations ci-dessous~:
\begin{itemize}
     \item
     $\overrightarrow{BC}=\overrightarrow{BA}+\overrightarrow{AC}=\overrightarrow{AC} - \overrightarrow{AB}$ (Relation de Chasles) \\
     \item
     Si $M$ et le milieu du segment $[BC]\ :$ \\
     $\overrightarrow{AB}+\overrightarrow{AC}=2\overrightarrow{AM}$ (Propriété de la médiane)
\end{itemize}
\bloc{orange}{Exemple}{ % id=150
     Pour la figure ci-dessous, on cherche, là encore, à calculer le produit scalaire $\overrightarrow{AB} \cdot \overrightarrow{AC}$ .
     \begin{center}
          \begin{extern} %width="360" alt="Produit scalaire connaissant trois longueurs"
               \resizebox{8cm}{!}{
                    %
                    \begin{tikzpicture}[line cap=round,line join=round,>=triangle 45,x=1.0cm,y=1.0cm]
                         \clip(0.,0.) rectangle (6.,4.6);
                         \draw [line width=0.4pt,color=tttttt] (1.7793181669564129,3.483291080200642)-- (5.,2.);
                         \draw [line width=0.4pt,color=tttttt] (1.7793181669564129,3.483291080200642)-- (1.,1.);
                         \draw [color=tttttt](3.034842223891809,1.2320966350302005) node {\scriptsize 9};
                         \draw [color=tttttt](1.1443463561232148,2.5470232959447823) node {\scriptsize 6};
                         \draw [line width=0.4pt,color=tttttt] (1.,1.)-- (5.,2.);
                         \draw [color=tttttt](3.396412471825693,3.0854184641932725) node {\scriptsize 8};
                         \begin{scriptsize}
                              \draw [fill=tttttt] (1.,1.) circle (0.5pt);
                              \draw[color=tttttt] (0.922238918106686,0.8257118205349459) node {$A$};
                              \draw [fill=tttttt] (5.,2.) circle (0.5pt);
                              \draw[color=tttttt] (5.188767843726519,1.8766177739430565) node {$B$};
                              \draw [fill=black] (1.7793181669564129,3.483291080200642) circle (0.5pt);
                              \draw[color=black] (1.6815364387678426,3.641932700603972) node {$C$};
                         \end{scriptsize}
                    \end{tikzpicture}
               }
          \end{extern}
     \end{center}
     Dans le triangle ci-dessus, d'après la relation de Chasles~:
     \newpar
     $\overrightarrow{BC}=\overrightarrow{BA}+\overrightarrow{AC}=\overrightarrow{AC} - \overrightarrow{AB}$
     \newpar
     On en déduit, d'après la seconde égalité du théorème précédent~:
     \newpar
     $\overrightarrow{AB} \cdot \overrightarrow{AC} =\frac{1}{2} \left( ||\overrightarrow{AB}||{}^2 +||\overrightarrow{AC}||{}^2  - ||\overrightarrow{BC}{}||^2  \right) $\\
     $\overrightarrow{AB} \cdot \overrightarrow{AC} =\frac{1}{2} \left( 9{}^2 +6{}^2  - 8{}^2  \right)$\\
     $\overrightarrow{AB} \cdot \overrightarrow{AC} =\frac{1}{2}  \times 53=26,5$
}% fin exemple
\begin{h2}4. Se placer dans un repère orthonormé\end{h2}
Dans un \textbf{repère orthonormé}, il est facile de calculer le produit scalaire des vecteurs $\overrightarrow{u}\begin{pmatrix} x \\ y \end{pmatrix} $ et $\overrightarrow{v}\begin{pmatrix} x'  \\ y'  \end{pmatrix} $ grâce à la \mcLien{https://www.maths-cours.fr/cours/produit-scalaire\#t60}{formule} suivante~:
\cadre{rouge}{Théorème}{% id="t200"
     Le plan étant rapporté à un repère orthonormé $\left(O; \vec{i}, \vec{j}\right)$, soient $\overrightarrow{u}\begin{pmatrix} x \\ y \end{pmatrix} $ et $\overrightarrow{v}\begin{pmatrix} x'  \\ y'  \end{pmatrix}$ deux vecteurs du plan; alors :
     \begin{center}$\vec{u} \cdot \vec{v}=xx^{\prime}+yy^{\prime}$\end{center}
}
\bloc{cyan}{Remarque}{ % id=r210
     Lorsque la figure ne comporte pas de repère orthonormé, il est toujours possible d'en choisir un soi-même. Attention toutefois, pour que la formule précédente soit valable, il est important que le repère soit \textbf{orthonormé}.
}% fin remarque
\bloc{orange}{Exemple}{ % id=e220
     Reprenons l'exemple étudié lors de la première méthode en nous plaçant, cette fois, dans le repère  $(A~;~\vec{i},~\vec{j})$ représenté ci-dessous~:
     \begin{center}
          \begin{extern} %width="500" alt="Produit scalaire dans un repère orthonormé"
               \resizebox{8cm}{!}{
                    %
                    \begin{tikzpicture}[line cap=round,line join=round,>=triangle 45,x=1.0cm,y=1.0cm]
                         \draw [color=cqcqcq,, xstep=1.0cm,ystep=1.0cm] (0.,0.) grid (8.,6.);
                         \clip(0.,0.) rectangle (8.,6.);
                         \draw [line width=0.4pt,color=tttttt] (2.,5.)-- (2.,1.);
                         \draw [line width=0.4pt,color=tttttt] (6.,1.)-- (6.,5.);
                         \draw [line width=0.4pt,color=tttttt] (6.,5.)-- (2.,5.);
                         \draw [line width=0.4pt,color=tttttt] (2.,1.)-- (4.,1.);
                         \draw [line width=0.4pt,color=tttttt] (4.,1.)-- (6.,1.);
                         \draw [->,line width=0.4pt,color=qqqqff] (4.,1.) -- (6.,1.);
                         \draw [->,line width=0.4pt,color=qqqqff] (4.,1.) -- (2.,5.);
                         \draw [->,line width=0.8pt,color=ffqqqq] (2.,1.) -- (3.,1.);
                         \draw [->,line width=0.8pt,color=ffqqqq] (2.,1.) -- (2.,2.);
                         \draw [color=ffqqqq](2.4,0.7) node {\scriptsize $\vec{i}$};
                         \draw [color=ffqqqq](1.7,1.5) node {\scriptsize $\vec{j}$};
                         \begin{scriptsize}
                              \draw [fill=tttttt] (2.,1.) circle (0.5pt);
                              \draw[color=ffqqqq] (1.9111570247933876,0.8050043140638496) node {$A$};
                              \draw [fill=tttttt] (6.,1.) circle (0.5pt);
                              \draw[color=tttttt] (6.038223140495865,0.8153580672993975) node {$B$};
                              \draw [fill=tttttt] (6.,5.) circle (0.5pt);
                              \draw[color=tttttt] (6.048553719008262,5.210526315789477) node {$C$};
                              \draw [fill=tttttt] (2.,5.) circle (0.5pt);
                              \draw[color=tttttt] (1.8801652892561975,5.179465056082833) node {$D$};
                              \draw [fill=tttttt] (4.,1.) circle (0.5pt);
                              \draw[color=tttttt] (4.003099173553717,0.8153580672993975) node {$I$};
                         \end{scriptsize}
                    \end{tikzpicture}
               }
          \end{extern}
     \end{center}
     Les coordonnées des points $A, B, C, D, I$ dans le repère orthonormé $(A~;~\vec{i},~\vec{j})$ sont~:\\
     $A(0~;~0)~; B(4~;~0)~;~C(4~;~4)~; D(0~;~4)~;~I(2~;~0) $
     \newpar
     On on déduit les coordonnées des vecteurs $\overrightarrow{IB}$ et $\overrightarrow{ID}~:$\\
     $\overrightarrow{IB}\begin{pmatrix} x_{B}  - x_{I} \\ y_{B} - y_{I} \end{pmatrix} $ donc $\overrightarrow{IB}\begin{pmatrix} 2 \\ 0 \end{pmatrix} $
     \newpar
     $\overrightarrow{ID}\begin{pmatrix} x_{D}  - x_{I} \\ y_{D} - y_{I} \end{pmatrix} $ donc $\overrightarrow{ID}\begin{pmatrix}  - 2 \\ 4 \end{pmatrix} $
     \medbreak
     Par conséquent~:
     \newpar
     $\overrightarrow{IB} \cdot \overrightarrow{ID}=2 \times ( - 2) +4 \times 0= - 4$
}% fin exemple
\begin{h2}5. Utiliser la relation de Chasles\end{h2}
Une autre façon de calculer le produit scalaire de 2 vecteurs consiste à décomposer ces vecteurs en utilisant la relation de Chasles puis à utiliser la \mcLien{https://www.maths-cours.fr/cours/produit-scalaire\#p30}{distributivité} du produit scalaire par rapport à l'addition ou à la soustraction de vecteurs.
\cadre{vert}{Propriété}{% id="p20"
     Pour tous vecteurs $\vec{u}, \vec{v}, \vec{w}~:$
     \begin{center}
          $\vec{u} \cdot \left(\vec{v}+\vec{w}\right)=\vec{u} \cdot \vec{v}+\vec{u} \cdot \vec{w}$
     \end{center}
}
\bloc{cyan}{Remarque}{ % id=r250
     Cette méthode est très générale et elle peut souvent remplacer les méthodes 1 ou 4~;~cependant, elle peut être parfois plus difficile à manier.
}% fin remarque
Sur la figure ci-dessous,   $ABCD$ est un losange dont les diagonales mesurent~: $AC=12$ et $BD=6. $\\
On souhaite calculer le produit scalaire  $\overrightarrow{AB} \cdot \overrightarrow{BC}. $
\bloc{orange}{Exemple}{ % id=e260
     \begin{center}
          \begin{extern}%width="300" alt="Produit scalaire et relation de Chasles"
               \resizebox{8cm}{!}{
                    %
                    \begin{tikzpicture}[line cap=round,line join=round,>=triangle 45,x=1.0cm,y=1.0cm]
                         \clip(0.,1.) rectangle (8.,6.);
                         \draw[line width=0.4pt,color=sqsqsq] (4.077479338842975,2.9225206611570247) -- (4.154958677685951,3.) -- (4.077479338842975,3.0774793388429753) -- (4.,3.) -- cycle;
                         \draw [line width=0.4pt,color=tttttt] (3.,4.)-- (2.,1.);
                         \draw [line width=0.4pt,color=tttttt] (5.,2.)-- (6.,5.);
                         \draw [line width=0.4pt,color=tttttt] (6.,5.)-- (3.,4.);
                         \draw [line width=0.4pt,color=tttttt] (2.,1.)-- (5.,2.);
                         \draw [line width=0.4pt,color=tttttt] (2.,1.)-- (6.,5.);
                         \draw [line width=0.4pt,color=tttttt] (3.,4.)-- (5.,2.);
                         \begin{scriptsize}
                              \draw [fill=tttttt] (2.,1.) circle (0.5pt);
                              \draw[color=tttttt] (1.9111570247933876,0.8050043140638496) node {$A$};
                              \draw [fill=tttttt] (5.,2.) circle (0.5pt);
                              \draw[color=tttttt] (5.036157024793386,1.814495254529769) node {$B$};
                              \draw [fill=tttttt] (6.,5.) circle (0.5pt);
                              \draw[color=tttttt] (6.048553719008262,5.210526315789477) node {$C$};
                              \draw [fill=tttttt] (3.,4.) circle (0.5pt);
                              \draw[color=tttttt] (2.8822314049586764,4.180327868852462) node {$D$};
                              \draw [fill=tttttt] (4.,3.) circle (0.5pt);
                              \draw[color=tttttt] (4.013429752066114,3.2070750647109603) node {$I$};
                         \end{scriptsize}
                    \end{tikzpicture}
               }
          \end{extern}
     \end{center}
     Pour trouver le résultat demandé, on peut se placer dans un repère de centre $I$ et employer la méthode précédente. Toutefois, Il est également possible ici de décomposer les vecteurs $\overrightarrow{AB}$ et $\overrightarrow{BC}$ en utilisant la relation de Chasles et en faisant intervenir le point $I$~:
     $\overrightarrow{AB}=\overrightarrow{AI}+\overrightarrow{IB}$ \\
     $\overrightarrow{BC}=\overrightarrow{BI}+\overrightarrow{IC}$
     \newpar
     On peut alors calculer le produit scalaire  $\overrightarrow{AB} \cdot \overrightarrow{BC}$ de la façon suivante~:
     \newpar
     $\overrightarrow{AB} \cdot \overrightarrow{BC}=\left( \overrightarrow{AI}+\overrightarrow{IB} \right)  \cdot \left( \overrightarrow{BI}+\overrightarrow{IC}  \right) $\\
     $\overrightarrow{AB} \cdot \overrightarrow{BC}=\overrightarrow{AI} \cdot \overrightarrow{BI}+\overrightarrow{AI} \cdot \overrightarrow{IC}+\overrightarrow{IB} \cdot \overrightarrow{BI}+\overrightarrow{IB} \cdot \overrightarrow{IC}$
     \newpar
     Comme les vecteurs   $\overrightarrow{AI}$ et $\overrightarrow{BI}$ sont orthogonaux le produit scalaire  $\overrightarrow{AI} \cdot \overrightarrow{BI}$ est nul~;~pour la même raison le produit scalaire $ \overrightarrow{IB}  \cdot \overrightarrow{IC}$ est lui aussi nul.\\
     De plus, $\overrightarrow{IC}=  \overrightarrow{AI}$, $IB=\frac{1}{2} DB=3$ et  $IC=AI=\frac{1}{2} AC=6.$
     \newpar
     Par conséquent~:\\
     $\overrightarrow{AB} \cdot \overrightarrow{BC}=\overrightarrow{AI}{}^2  - \overrightarrow{IB}{}^2  =AI{}^2  - IB{}^2 $\\
     $\overrightarrow{AB} \cdot \overrightarrow{BC}=6{}^2  - 3{}^2 =36 - 9=27.$
     \newpar
     (\textbf{Remarque}~: On peut montrer que ce résultat est encore correct si $ABCD$ est un parallélogramme quelconque et non nécessairement un losange)
}% fin exemple

\end{document}
µ
\documentclass[a4paper]{article}

%================================================================================================================================
%
% Packages
%
%================================================================================================================================

\usepackage[T1]{fontenc} 	% pour caractères accentués
\usepackage[utf8]{inputenc}  % encodage utf8
\usepackage[french]{babel}	% langue : français
\usepackage{fourier}			% caractères plus lisibles
\usepackage[dvipsnames]{xcolor} % couleurs
\usepackage{fancyhdr}		% réglage header footer
\usepackage{needspace}		% empêcher sauts de page mal placés
\usepackage{graphicx}		% pour inclure des graphiques
\usepackage{enumitem,cprotect}		% personnalise les listes d'items (nécessaire pour ol, al ...)
\usepackage{hyperref}		% Liens hypertexte
\usepackage{pstricks,pst-all,pst-node,pstricks-add,pst-math,pst-plot,pst-tree,pst-eucl} % pstricks
\usepackage[a4paper,includeheadfoot,top=2cm,left=3cm, bottom=2cm,right=3cm]{geometry} % marges etc.
\usepackage{comment}			% commentaires multilignes
\usepackage{amsmath,environ} % maths (matrices, etc.)
\usepackage{amssymb,makeidx}
\usepackage{bm}				% bold maths
\usepackage{tabularx}		% tableaux
\usepackage{colortbl}		% tableaux en couleur
\usepackage{fontawesome}		% Fontawesome
\usepackage{environ}			% environment with command
\usepackage{fp}				% calculs pour ps-tricks
\usepackage{multido}			% pour ps tricks
\usepackage[np]{numprint}	% formattage nombre
\usepackage{tikz,tkz-tab} 			% package principal TikZ
\usepackage{pgfplots}   % axes
\usepackage{mathrsfs}    % cursives
\usepackage{calc}			% calcul taille boites
\usepackage[scaled=0.875]{helvet} % font sans serif
\usepackage{svg} % svg
\usepackage{scrextend} % local margin
\usepackage{scratch} %scratch
\usepackage{multicol} % colonnes
%\usepackage{infix-RPN,pst-func} % formule en notation polanaise inversée
\usepackage{listings}

%================================================================================================================================
%
% Réglages de base
%
%================================================================================================================================

\lstset{
language=Python,   % R code
literate=
{á}{{\'a}}1
{à}{{\`a}}1
{ã}{{\~a}}1
{é}{{\'e}}1
{è}{{\`e}}1
{ê}{{\^e}}1
{í}{{\'i}}1
{ó}{{\'o}}1
{õ}{{\~o}}1
{ú}{{\'u}}1
{ü}{{\"u}}1
{ç}{{\c{c}}}1
{~}{{ }}1
}


\definecolor{codegreen}{rgb}{0,0.6,0}
\definecolor{codegray}{rgb}{0.5,0.5,0.5}
\definecolor{codepurple}{rgb}{0.58,0,0.82}
\definecolor{backcolour}{rgb}{0.95,0.95,0.92}

\lstdefinestyle{mystyle}{
    backgroundcolor=\color{backcolour},   
    commentstyle=\color{codegreen},
    keywordstyle=\color{magenta},
    numberstyle=\tiny\color{codegray},
    stringstyle=\color{codepurple},
    basicstyle=\ttfamily\footnotesize,
    breakatwhitespace=false,         
    breaklines=true,                 
    captionpos=b,                    
    keepspaces=true,                 
    numbers=left,                    
xleftmargin=2em,
framexleftmargin=2em,            
    showspaces=false,                
    showstringspaces=false,
    showtabs=false,                  
    tabsize=2,
    upquote=true
}

\lstset{style=mystyle}


\lstset{style=mystyle}
\newcommand{\imgdir}{C:/laragon/www/newmc/assets/imgsvg/}
\newcommand{\imgsvgdir}{C:/laragon/www/newmc/assets/imgsvg/}

\definecolor{mcgris}{RGB}{220, 220, 220}% ancien~; pour compatibilité
\definecolor{mcbleu}{RGB}{52, 152, 219}
\definecolor{mcvert}{RGB}{125, 194, 70}
\definecolor{mcmauve}{RGB}{154, 0, 215}
\definecolor{mcorange}{RGB}{255, 96, 0}
\definecolor{mcturquoise}{RGB}{0, 153, 153}
\definecolor{mcrouge}{RGB}{255, 0, 0}
\definecolor{mclightvert}{RGB}{205, 234, 190}

\definecolor{gris}{RGB}{220, 220, 220}
\definecolor{bleu}{RGB}{52, 152, 219}
\definecolor{vert}{RGB}{125, 194, 70}
\definecolor{mauve}{RGB}{154, 0, 215}
\definecolor{orange}{RGB}{255, 96, 0}
\definecolor{turquoise}{RGB}{0, 153, 153}
\definecolor{rouge}{RGB}{255, 0, 0}
\definecolor{lightvert}{RGB}{205, 234, 190}
\setitemize[0]{label=\color{lightvert}  $\bullet$}

\pagestyle{fancy}
\renewcommand{\headrulewidth}{0.2pt}
\fancyhead[L]{maths-cours.fr}
\fancyhead[R]{\thepage}
\renewcommand{\footrulewidth}{0.2pt}
\fancyfoot[C]{}

\newcolumntype{C}{>{\centering\arraybackslash}X}
\newcolumntype{s}{>{\hsize=.35\hsize\arraybackslash}X}

\setlength{\parindent}{0pt}		 
\setlength{\parskip}{3mm}
\setlength{\headheight}{1cm}

\def\ebook{ebook}
\def\book{book}
\def\web{web}
\def\type{web}

\newcommand{\vect}[1]{\overrightarrow{\,\mathstrut#1\,}}

\def\Oij{$\left(\text{O}~;~\vect{\imath},~\vect{\jmath}\right)$}
\def\Oijk{$\left(\text{O}~;~\vect{\imath},~\vect{\jmath},~\vect{k}\right)$}
\def\Ouv{$\left(\text{O}~;~\vect{u},~\vect{v}\right)$}

\hypersetup{breaklinks=true, colorlinks = true, linkcolor = OliveGreen, urlcolor = OliveGreen, citecolor = OliveGreen, pdfauthor={Didier BONNEL - https://www.maths-cours.fr} } % supprime les bordures autour des liens

\renewcommand{\arg}[0]{\text{arg}}

\everymath{\displaystyle}

%================================================================================================================================
%
% Macros - Commandes
%
%================================================================================================================================

\newcommand\meta[2]{    			% Utilisé pour créer le post HTML.
	\def\titre{titre}
	\def\url{url}
	\def\arg{#1}
	\ifx\titre\arg
		\newcommand\maintitle{#2}
		\fancyhead[L]{#2}
		{\Large\sffamily \MakeUppercase{#2}}
		\vspace{1mm}\textcolor{mcvert}{\hrule}
	\fi 
	\ifx\url\arg
		\fancyfoot[L]{\href{https://www.maths-cours.fr#2}{\black \footnotesize{https://www.maths-cours.fr#2}}}
	\fi 
}


\newcommand\TitreC[1]{    		% Titre centré
     \needspace{3\baselineskip}
     \begin{center}\textbf{#1}\end{center}
}

\newcommand\newpar{    		% paragraphe
     \par
}

\newcommand\nosp {    		% commande vide (pas d'espace)
}
\newcommand{\id}[1]{} %ignore

\newcommand\boite[2]{				% Boite simple sans titre
	\vspace{5mm}
	\setlength{\fboxrule}{0.2mm}
	\setlength{\fboxsep}{5mm}	
	\fcolorbox{#1}{#1!3}{\makebox[\linewidth-2\fboxrule-2\fboxsep]{
  		\begin{minipage}[t]{\linewidth-2\fboxrule-4\fboxsep}\setlength{\parskip}{3mm}
  			 #2
  		\end{minipage}
	}}
	\vspace{5mm}
}

\newcommand\CBox[4]{				% Boites
	\vspace{5mm}
	\setlength{\fboxrule}{0.2mm}
	\setlength{\fboxsep}{5mm}
	
	\fcolorbox{#1}{#1!3}{\makebox[\linewidth-2\fboxrule-2\fboxsep]{
		\begin{minipage}[t]{1cm}\setlength{\parskip}{3mm}
	  		\textcolor{#1}{\LARGE{#2}}    
 	 	\end{minipage}  
  		\begin{minipage}[t]{\linewidth-2\fboxrule-4\fboxsep}\setlength{\parskip}{3mm}
			\raisebox{1.2mm}{\normalsize\sffamily{\textcolor{#1}{#3}}}						
  			 #4
  		\end{minipage}
	}}
	\vspace{5mm}
}

\newcommand\cadre[3]{				% Boites convertible html
	\par
	\vspace{2mm}
	\setlength{\fboxrule}{0.1mm}
	\setlength{\fboxsep}{5mm}
	\fcolorbox{#1}{white}{\makebox[\linewidth-2\fboxrule-2\fboxsep]{
  		\begin{minipage}[t]{\linewidth-2\fboxrule-4\fboxsep}\setlength{\parskip}{3mm}
			\raisebox{-2.5mm}{\sffamily \small{\textcolor{#1}{\MakeUppercase{#2}}}}		
			\par		
  			 #3
 	 		\end{minipage}
	}}
		\vspace{2mm}
	\par
}

\newcommand\bloc[3]{				% Boites convertible html sans bordure
     \needspace{2\baselineskip}
     {\sffamily \small{\textcolor{#1}{\MakeUppercase{#2}}}}    
		\par		
  			 #3
		\par
}

\newcommand\CHelp[1]{
     \CBox{Plum}{\faInfoCircle}{À RETENIR}{#1}
}

\newcommand\CUp[1]{
     \CBox{NavyBlue}{\faThumbsOUp}{EN PRATIQUE}{#1}
}

\newcommand\CInfo[1]{
     \CBox{Sepia}{\faArrowCircleRight}{REMARQUE}{#1}
}

\newcommand\CRedac[1]{
     \CBox{PineGreen}{\faEdit}{BIEN R\'EDIGER}{#1}
}

\newcommand\CError[1]{
     \CBox{Red}{\faExclamationTriangle}{ATTENTION}{#1}
}

\newcommand\TitreExo[2]{
\needspace{4\baselineskip}
 {\sffamily\large EXERCICE #1\ (\emph{#2 points})}
\vspace{5mm}
}

\newcommand\img[2]{
          \includegraphics[width=#2\paperwidth]{\imgdir#1}
}

\newcommand\imgsvg[2]{
       \begin{center}   \includegraphics[width=#2\paperwidth]{\imgsvgdir#1} \end{center}
}


\newcommand\Lien[2]{
     \href{#1}{#2 \tiny \faExternalLink}
}
\newcommand\mcLien[2]{
     \href{https~://www.maths-cours.fr/#1}{#2 \tiny \faExternalLink}
}

\newcommand{\euro}{\eurologo{}}

%================================================================================================================================
%
% Macros - Environement
%
%================================================================================================================================

\newenvironment{tex}{ %
}
{%
}

\newenvironment{indente}{ %
	\setlength\parindent{10mm}
}

{
	\setlength\parindent{0mm}
}

\newenvironment{corrige}{%
     \needspace{3\baselineskip}
     \medskip
     \textbf{\textsc{Corrigé}}
     \medskip
}
{
}

\newenvironment{extern}{%
     \begin{center}
     }
     {
     \end{center}
}

\NewEnviron{code}{%
	\par
     \boite{gray}{\texttt{%
     \BODY
     }}
     \par
}

\newenvironment{vbloc}{% boite sans cadre empeche saut de page
     \begin{minipage}[t]{\linewidth}
     }
     {
     \end{minipage}
}
\NewEnviron{h2}{%
    \needspace{3\baselineskip}
    \vspace{0.6cm}
	\noindent \MakeUppercase{\sffamily \large \BODY}
	\vspace{1mm}\textcolor{mcgris}{\hrule}\vspace{0.4cm}
	\par
}{}

\NewEnviron{h3}{%
    \needspace{3\baselineskip}
	\vspace{5mm}
	\textsc{\BODY}
	\par
}

\NewEnviron{margeneg}{ %
\begin{addmargin}[-1cm]{0cm}
\BODY
\end{addmargin}
}

\NewEnviron{html}{%
}

\begin{document}
\meta{url}{/methode/determiner-le-centre-et-le-rayon-dun-cercle-a-partir-de-son-equation/}
\meta{pid}{11187}
\meta{titre}{Déterminer le centre et le rayon d'un cercle à partir de son équation}
\meta{type}{methode}
%
\cadre{rouge}{Méthode}{ % id=m010
     Dans cette fiche, on cherchera à déterminer si une équation du type~:
     \[
     x^2 +y^2 +ax+by+c=0
     \]
     correspond à l'équation d'un cercle et, si c'est le cas, à déterminer les coordonnées du centre et du rayon de ce cercle.
}% fin méthode
On utilisera, pour cela, le \mcLien{https://www.maths-cours.fr/cours/produit-scalaire\#t110}{résultat suivant}~:
\cadre{bleu}{Rappel}{ % id=r015
     Le plan est rapporté à un repère orthonormé $\left(O, \vec{i}, \vec{j}\right)$.
     \par
     Soit $ \Omega ( \alpha  ;  \beta) $ un point quelconque du plan et $r$ un réel positif.
     \par
     Une équation du cercle de centre $ \Omega $ et de rayon $r$ est :
     \[
     \left(x- \alpha \right)^{2}+\left(y- \beta \right)^{2}=r^{2}
     \]
}%
\textbf{L'objectif} sera donc de chercher si l'équation $x^2 +y^2 +ax+by+c=0 $ peut s'écrire sous la forme $\left(x- \alpha \right)^{2}+\left(y- \beta \right)^{2}=r^{2}$.
\medbreak
\textbf{En pratique}, pour déterminer si l'ensemble cherché est un cercle et déterminer les éléments caractéristiques de ce cercle, on procède de la manière suivante~:
\begin{enumerate}
     \item
     on regroupe les termes en  $x$ et les termes en $y$
     \item
     on fait apparaître des carrés de la forme $(x -  \alpha )^2 $ et $(y -  \beta )^2 $ en utilisant des identités remarquables
     \item
     On écrit l'équation sous la forme $(x- \alpha )^2 +(y -  \beta )^2 = \gamma $ ;\\
     À partir de cette équation~:
     \begin{itemize}
          \item
          si $ \gamma >0$ , l'ensemble cherché est un cercle de centre  $ \Omega ( \alpha ; \beta )$ et de rayon $r=\sqrt{ \gamma }$ .
          \item
          si $ \gamma =0$ , l'ensemble cherché est le point $ \Omega ( \alpha ; \beta )$
          \item
          si $ \gamma <0$ , l'ensemble cherché est l'ensemble vide.
     \end{itemize}
\end{enumerate}
\bloc{orange}{Exemple 1}{ % id=e020
     \cadre{vert}{Énoncé}{ % id=r020
          Déterminer l'ensemble des points  $M$ tels que~:
          \[
          x^2 +y^2 +2x - 4y=0
          \]
     }% fin énoncé
     \begin{enumerate}
          \item
          En regroupant les termes en $x$ et les termes en $y$ on obtient~:\\
          $x^2 +2x+ y^2  - 4y=0$
          \item
          $x^2 +2x$ est le début de l'identité remarquable~:  $x^2 +2x+1=(x+1)^2.$ \\
          Donc $x^2 +2x=(x+1)^2 - 1$
          \newpar
          De la même manière, $y^2  - 4y$ est le début de l'identité remarquable~: $y^2  - 4y+4=(y - 2)^2 $ \\
          Donc $y^2  - 4y=(y - 2)^2  - 4$
          \item
          En utilisant les résultats précédents, l'équation de départ peut s'écrire sous la forme~:\\
          $(x+1)^2 - 1+(y - 2)^2  - 4=0$
          \newpar
          C'est-à-dire~:\\
          $(x+1)^2+(y - 2)^2 =5$ \\
          $(x - ( - 1))^2+(y - 2)^2 =\left(\sqrt{5}\right)^2 $
          \newpar
          L'ensemble des points  $M$ cherché est donc le cercle de centre  $ \Omega ( - 1~;~2)$ et de rayon  $\sqrt{5}.$
          \begin{center}
               \begin{extern} %width="400" alt=""
                    \resizebox{8cm}{!}{
                         %
                         \begin{tikzpicture}[line cap=round,line join=round,>=triangle 45,x=1.0cm,y=1.0cm]
                              \begin{axis}[
                                   x=1.0cm,y=1.0cm,
                                   axis lines=middle,
                                   ymajorgrids=true,
                                   xmajorgrids=true,
                                   xmin=-5.0,
                                   xmax=3.0,
                                   ymin=-1.0,
                                   ymax=5.0,
                                   xtick={-5.0,-4.0,...,3.0},
                                   ytick={-1.0,0.0,...,5.0},]
                                   \clip(-5.,-1.) rectangle (3.,5.);
                                   \draw [line width=0.4pt,color=ffqqqq] (-1.,2.) circle (2.23606797749979cm);
                                   \draw [color=ffqqqq](-1.2,2.2) node{$\Omega$};
                                   \begin{scriptsize}
                                        \draw [fill=ffqqqq] (-1.,2.) circle (1.0pt);
                                        \draw [fill=tttttt] (0.,0.) circle (1.0pt);
                                        \draw[color=tttttt] (0.2,-0.3) node {$O$};
                                   \end{scriptsize}
                              \end{axis}
                         \end{tikzpicture}
                    }
               \end{extern}
          \end{center}
          \textbf{Remarque} : Ce cercle passe par l'origine du repère puisque l'équation est vérifiée pour $x=0$ et $y=0.$
     \end{enumerate}
}% fin exemple
\bloc{orange}{Exemple 2}{ % id=e030
     \cadre{vert}{Énoncé}{ % id=r030
          Quel est l'ensemble des points  $M( x ; y )$ dont les coordonnées vérifient~:
          \[
          x^2 +y^2 +x - 3y+7=0 \quad ?
          \]
     }% fin énoncé
     \begin{enumerate}
          \item
          On commence par regrouper les termes en $x$ et les termes en $y$~:\\
          $x^2 +x+ y^2  - 3y + 7=0$
          \item
          $x^2 +x$ est le début de l'identité remarquable~:  $x^2 +x+\frac{ 1 }{ 4 }=\left(x+\frac{ 1 }{ 2 }\right)^2$ \\
          Donc $x^2 +x=\left(x+\frac{ 1 }{ 2 }\right)^2 - \frac{ 1 }{ 4 } $
          \newpar
          $y^2  - 3y$ est le début de l'identité remarquable~: $y^2  - 3y+\frac{ 9 }{ 4 }=\left(y - \frac{ 3 }{ 2 }\right)^2 $ \\
          Ce qui donne : $y^2  - 3y=\left(y - \frac{ 3 }{ 2 }\right)^2  - \frac{ 9 }{ 4 } $
          \item
          Finalement, notre équation peut s'écrire~:\\
          $\left(x+\frac{ 1 }{ 2 }\right)^2 - \frac{ 1 }{ 4 } +\left(y - \frac{ 3 }{ 2 }\right)^2  - \frac{ 9 }{ 4 } + 7 =0$
          \newpar
          Ou encore~:\\
          $\left(x+\frac{ 1 }{ 2 }\right)^2  + \left(y - \frac{ 3 }{ 2 }\right)^2 =  - \frac{ 9 }{ 2 }$
          \newpar
          La somme de deux carrés ne peut jamais être strictement négative~; par conséquent, l'équation n'admet aucun couple solution.
          \newpar
          L'ensemble cherché est donc l'ensemble vide.
     \end{enumerate}
}% fin exemple
\bloc{orange}{Exemple 3}{ % id=e040
     \cadre{vert}{Énoncé}{ % id=r040
          L'équation~:
          \[
          x^2 +y^2 -6x +2y+10=0
          \]
          est-elle l'équation d'un cercle ?
     }% fin énoncé
     \begin{enumerate}
          \item
          Là encore, la première étape consiste à regrouper les termes en $x$ et les termes en $y$~:\\
          $x^2 -6x+ y^2  +2y + 10=0$
          \item
          Ensuite, on recherche le début d'identités remarquables~:
          $x^2 -6x$ est le début de~:  $x^2 -6x+9=\left(x - 3\right)^2$ \\
          Donc $x^2 -6x=\left(x - 3\right)^2-9$
          \newpar
          De même, pour les termes en $y$~:
          $y^2  +2y$ est le début de l'identité remarquable~: $y^2  +2y+1=\left(y+1\right)^2 $ \\
          Ce qui donne : $y^2  +y=\left(y +1\right)^2  - 1 $
          \item
          L'équation de départ peut alors s'écrire~:\\
          $\left(x - 3\right)^2-9+\left(y +1\right)^2  - 1  + 10 =0$
          \newpar
          Soit~:\\
          $\left(x - 3\right)^2+\left(y +1\right)^2 =0$
          \newpar
          La somme de deux carrés et nulle si et seulement si chacun de ces carrés est nul, c'est-à-dire si et seulement si~:\\
          $x - 3=0$ et $y+1=0.$
          \newpar
          L'équation proposée admet donc un unique couple solution qui est $\left( 3~;~ - 1 \right).$
          \newpar
          L'ensemble d'équation  $x^2 +y^2 -6x +2y+10=0$ est donc réduit au point de coordonnées $\left( 3~;~ - 1 \right).$
     \end{enumerate}
}% fin exemple

\end{document}
µ
\documentclass[a4paper]{article}

%================================================================================================================================
%
% Packages
%
%================================================================================================================================

\usepackage[T1]{fontenc} 	% pour caractères accentués
\usepackage[utf8]{inputenc}  % encodage utf8
\usepackage[french]{babel}	% langue : français
\usepackage{fourier}			% caractères plus lisibles
\usepackage[dvipsnames]{xcolor} % couleurs
\usepackage{fancyhdr}		% réglage header footer
\usepackage{needspace}		% empêcher sauts de page mal placés
\usepackage{graphicx}		% pour inclure des graphiques
\usepackage{enumitem,cprotect}		% personnalise les listes d'items (nécessaire pour ol, al ...)
\usepackage{hyperref}		% Liens hypertexte
\usepackage{pstricks,pst-all,pst-node,pstricks-add,pst-math,pst-plot,pst-tree,pst-eucl} % pstricks
\usepackage[a4paper,includeheadfoot,top=2cm,left=3cm, bottom=2cm,right=3cm]{geometry} % marges etc.
\usepackage{comment}			% commentaires multilignes
\usepackage{amsmath,environ} % maths (matrices, etc.)
\usepackage{amssymb,makeidx}
\usepackage{bm}				% bold maths
\usepackage{tabularx}		% tableaux
\usepackage{colortbl}		% tableaux en couleur
\usepackage{fontawesome}		% Fontawesome
\usepackage{environ}			% environment with command
\usepackage{fp}				% calculs pour ps-tricks
\usepackage{multido}			% pour ps tricks
\usepackage[np]{numprint}	% formattage nombre
\usepackage{tikz,tkz-tab} 			% package principal TikZ
\usepackage{pgfplots}   % axes
\usepackage{mathrsfs}    % cursives
\usepackage{calc}			% calcul taille boites
\usepackage[scaled=0.875]{helvet} % font sans serif
\usepackage{svg} % svg
\usepackage{scrextend} % local margin
\usepackage{scratch} %scratch
\usepackage{multicol} % colonnes
%\usepackage{infix-RPN,pst-func} % formule en notation polanaise inversée
\usepackage{listings}

%================================================================================================================================
%
% Réglages de base
%
%================================================================================================================================

\lstset{
language=Python,   % R code
literate=
{á}{{\'a}}1
{à}{{\`a}}1
{ã}{{\~a}}1
{é}{{\'e}}1
{è}{{\`e}}1
{ê}{{\^e}}1
{í}{{\'i}}1
{ó}{{\'o}}1
{õ}{{\~o}}1
{ú}{{\'u}}1
{ü}{{\"u}}1
{ç}{{\c{c}}}1
{~}{{ }}1
}


\definecolor{codegreen}{rgb}{0,0.6,0}
\definecolor{codegray}{rgb}{0.5,0.5,0.5}
\definecolor{codepurple}{rgb}{0.58,0,0.82}
\definecolor{backcolour}{rgb}{0.95,0.95,0.92}

\lstdefinestyle{mystyle}{
    backgroundcolor=\color{backcolour},   
    commentstyle=\color{codegreen},
    keywordstyle=\color{magenta},
    numberstyle=\tiny\color{codegray},
    stringstyle=\color{codepurple},
    basicstyle=\ttfamily\footnotesize,
    breakatwhitespace=false,         
    breaklines=true,                 
    captionpos=b,                    
    keepspaces=true,                 
    numbers=left,                    
xleftmargin=2em,
framexleftmargin=2em,            
    showspaces=false,                
    showstringspaces=false,
    showtabs=false,                  
    tabsize=2,
    upquote=true
}

\lstset{style=mystyle}


\lstset{style=mystyle}
\newcommand{\imgdir}{C:/laragon/www/newmc/assets/imgsvg/}
\newcommand{\imgsvgdir}{C:/laragon/www/newmc/assets/imgsvg/}

\definecolor{mcgris}{RGB}{220, 220, 220}% ancien~; pour compatibilité
\definecolor{mcbleu}{RGB}{52, 152, 219}
\definecolor{mcvert}{RGB}{125, 194, 70}
\definecolor{mcmauve}{RGB}{154, 0, 215}
\definecolor{mcorange}{RGB}{255, 96, 0}
\definecolor{mcturquoise}{RGB}{0, 153, 153}
\definecolor{mcrouge}{RGB}{255, 0, 0}
\definecolor{mclightvert}{RGB}{205, 234, 190}

\definecolor{gris}{RGB}{220, 220, 220}
\definecolor{bleu}{RGB}{52, 152, 219}
\definecolor{vert}{RGB}{125, 194, 70}
\definecolor{mauve}{RGB}{154, 0, 215}
\definecolor{orange}{RGB}{255, 96, 0}
\definecolor{turquoise}{RGB}{0, 153, 153}
\definecolor{rouge}{RGB}{255, 0, 0}
\definecolor{lightvert}{RGB}{205, 234, 190}
\setitemize[0]{label=\color{lightvert}  $\bullet$}

\pagestyle{fancy}
\renewcommand{\headrulewidth}{0.2pt}
\fancyhead[L]{maths-cours.fr}
\fancyhead[R]{\thepage}
\renewcommand{\footrulewidth}{0.2pt}
\fancyfoot[C]{}

\newcolumntype{C}{>{\centering\arraybackslash}X}
\newcolumntype{s}{>{\hsize=.35\hsize\arraybackslash}X}

\setlength{\parindent}{0pt}		 
\setlength{\parskip}{3mm}
\setlength{\headheight}{1cm}

\def\ebook{ebook}
\def\book{book}
\def\web{web}
\def\type{web}

\newcommand{\vect}[1]{\overrightarrow{\,\mathstrut#1\,}}

\def\Oij{$\left(\text{O}~;~\vect{\imath},~\vect{\jmath}\right)$}
\def\Oijk{$\left(\text{O}~;~\vect{\imath},~\vect{\jmath},~\vect{k}\right)$}
\def\Ouv{$\left(\text{O}~;~\vect{u},~\vect{v}\right)$}

\hypersetup{breaklinks=true, colorlinks = true, linkcolor = OliveGreen, urlcolor = OliveGreen, citecolor = OliveGreen, pdfauthor={Didier BONNEL - https://www.maths-cours.fr} } % supprime les bordures autour des liens

\renewcommand{\arg}[0]{\text{arg}}

\everymath{\displaystyle}

%================================================================================================================================
%
% Macros - Commandes
%
%================================================================================================================================

\newcommand\meta[2]{    			% Utilisé pour créer le post HTML.
	\def\titre{titre}
	\def\url{url}
	\def\arg{#1}
	\ifx\titre\arg
		\newcommand\maintitle{#2}
		\fancyhead[L]{#2}
		{\Large\sffamily \MakeUppercase{#2}}
		\vspace{1mm}\textcolor{mcvert}{\hrule}
	\fi 
	\ifx\url\arg
		\fancyfoot[L]{\href{https://www.maths-cours.fr#2}{\black \footnotesize{https://www.maths-cours.fr#2}}}
	\fi 
}


\newcommand\TitreC[1]{    		% Titre centré
     \needspace{3\baselineskip}
     \begin{center}\textbf{#1}\end{center}
}

\newcommand\newpar{    		% paragraphe
     \par
}

\newcommand\nosp {    		% commande vide (pas d'espace)
}
\newcommand{\id}[1]{} %ignore

\newcommand\boite[2]{				% Boite simple sans titre
	\vspace{5mm}
	\setlength{\fboxrule}{0.2mm}
	\setlength{\fboxsep}{5mm}	
	\fcolorbox{#1}{#1!3}{\makebox[\linewidth-2\fboxrule-2\fboxsep]{
  		\begin{minipage}[t]{\linewidth-2\fboxrule-4\fboxsep}\setlength{\parskip}{3mm}
  			 #2
  		\end{minipage}
	}}
	\vspace{5mm}
}

\newcommand\CBox[4]{				% Boites
	\vspace{5mm}
	\setlength{\fboxrule}{0.2mm}
	\setlength{\fboxsep}{5mm}
	
	\fcolorbox{#1}{#1!3}{\makebox[\linewidth-2\fboxrule-2\fboxsep]{
		\begin{minipage}[t]{1cm}\setlength{\parskip}{3mm}
	  		\textcolor{#1}{\LARGE{#2}}    
 	 	\end{minipage}  
  		\begin{minipage}[t]{\linewidth-2\fboxrule-4\fboxsep}\setlength{\parskip}{3mm}
			\raisebox{1.2mm}{\normalsize\sffamily{\textcolor{#1}{#3}}}						
  			 #4
  		\end{minipage}
	}}
	\vspace{5mm}
}

\newcommand\cadre[3]{				% Boites convertible html
	\par
	\vspace{2mm}
	\setlength{\fboxrule}{0.1mm}
	\setlength{\fboxsep}{5mm}
	\fcolorbox{#1}{white}{\makebox[\linewidth-2\fboxrule-2\fboxsep]{
  		\begin{minipage}[t]{\linewidth-2\fboxrule-4\fboxsep}\setlength{\parskip}{3mm}
			\raisebox{-2.5mm}{\sffamily \small{\textcolor{#1}{\MakeUppercase{#2}}}}		
			\par		
  			 #3
 	 		\end{minipage}
	}}
		\vspace{2mm}
	\par
}

\newcommand\bloc[3]{				% Boites convertible html sans bordure
     \needspace{2\baselineskip}
     {\sffamily \small{\textcolor{#1}{\MakeUppercase{#2}}}}    
		\par		
  			 #3
		\par
}

\newcommand\CHelp[1]{
     \CBox{Plum}{\faInfoCircle}{À RETENIR}{#1}
}

\newcommand\CUp[1]{
     \CBox{NavyBlue}{\faThumbsOUp}{EN PRATIQUE}{#1}
}

\newcommand\CInfo[1]{
     \CBox{Sepia}{\faArrowCircleRight}{REMARQUE}{#1}
}

\newcommand\CRedac[1]{
     \CBox{PineGreen}{\faEdit}{BIEN R\'EDIGER}{#1}
}

\newcommand\CError[1]{
     \CBox{Red}{\faExclamationTriangle}{ATTENTION}{#1}
}

\newcommand\TitreExo[2]{
\needspace{4\baselineskip}
 {\sffamily\large EXERCICE #1\ (\emph{#2 points})}
\vspace{5mm}
}

\newcommand\img[2]{
          \includegraphics[width=#2\paperwidth]{\imgdir#1}
}

\newcommand\imgsvg[2]{
       \begin{center}   \includegraphics[width=#2\paperwidth]{\imgsvgdir#1} \end{center}
}


\newcommand\Lien[2]{
     \href{#1}{#2 \tiny \faExternalLink}
}
\newcommand\mcLien[2]{
     \href{https~://www.maths-cours.fr/#1}{#2 \tiny \faExternalLink}
}

\newcommand{\euro}{\eurologo{}}

%================================================================================================================================
%
% Macros - Environement
%
%================================================================================================================================

\newenvironment{tex}{ %
}
{%
}

\newenvironment{indente}{ %
	\setlength\parindent{10mm}
}

{
	\setlength\parindent{0mm}
}

\newenvironment{corrige}{%
     \needspace{3\baselineskip}
     \medskip
     \textbf{\textsc{Corrigé}}
     \medskip
}
{
}

\newenvironment{extern}{%
     \begin{center}
     }
     {
     \end{center}
}

\NewEnviron{code}{%
	\par
     \boite{gray}{\texttt{%
     \BODY
     }}
     \par
}

\newenvironment{vbloc}{% boite sans cadre empeche saut de page
     \begin{minipage}[t]{\linewidth}
     }
     {
     \end{minipage}
}
\NewEnviron{h2}{%
    \needspace{3\baselineskip}
    \vspace{0.6cm}
	\noindent \MakeUppercase{\sffamily \large \BODY}
	\vspace{1mm}\textcolor{mcgris}{\hrule}\vspace{0.4cm}
	\par
}{}

\NewEnviron{h3}{%
    \needspace{3\baselineskip}
	\vspace{5mm}
	\textsc{\BODY}
	\par
}

\NewEnviron{margeneg}{ %
\begin{addmargin}[-1cm]{0cm}
\BODY
\end{addmargin}
}

\NewEnviron{html}{%
}

\begin{document}
\meta{url}{/exercices/qcm-controle-continu-1ere-2020-sujet-zero/}
\meta{pid}{11189}
\meta{titre}{QCM - Contrôle continu 1ère - 2020 - Sujet zéro}
\meta{type}{exercices}
%
Ce QCM comprend 5 questions.
\newpar
Pour chacune des questions, une seule des 4 réponses proposées est correcte.
\newpar
Les questions sont indépendantes.
\newpar
Pour chaque question, indiquer le numéro de la question et recopier sur la copie la lettre correspondant à la réponse choisie.
\newpar
Aucune justification n'est demandée mais il peut être nécessaire d'effectuer des recherches au brouillon pour aider à déterminer votre réponse.
\newpar
Chaque réponse correcte rapporte 1 point. Une réponse incorrecte ou une question sans réponse n'apporte ni ne retire de point.
\newpar
\begin{h2}Question 1\end{h2}
Pour tout réel $x$,  $\left( \text{e}^{ x } \right)^{ 3 }$ est égale à :
\newpar
\begin{enumerate}[label=\alph*.]
     \item
     $\text{e}^{ x } \times \text{e}^{ 3 }$
     \item
     $\text{e}^{ x+3 }$
     \item
     $\text{e}^{ 3x }$
     \item
     $\text{e}^{ x{}^{ 3 } }$
\end{enumerate}
\begin{h2}Question 2\end{h2}
Pour tout réel $x$,  $\cos{ ( x+ \pi  ) }$ est égale à :
\newpar
\begin{enumerate}[label=\alph*.]
     \item
     $\sin{ x }$
     \item
     $ - \cos{ x }$
     \item
     $\cos{ x }$
     \item
     $ - \sin{ x }$
\end{enumerate}
\begin{h2}Question 3\end{h2}
On souhaite modéliser le niveau de la mer par une suite $\left( U_{ n } \right)$ de façon que $U_{ 0 }$ représente le niveau de la mer, en millimètres, en 2003 et que $U_{ n }$ représente le niveau de la mer, en millimètres,  $n$  années après 2003.
\newpar
Selon le site \textit{notre-planete.info}, on constate une hausse assez rapide du niveau de la mer, qu'on estime à 3,3 mm par an depuis 2003.
\newpar
Pour traduire ce constat, la suite $\left( U_{ n } \right)$ doit être~:
\newpar
\begin{enumerate}[label=\alph*.]
     \item
     Une suite géométrique de raison 3,3.
     \item
     Une suite géométrique de raison 1,033.
     \item
     Une suite arithmétique de raison 1,033.
     \item
     Une suite arithmétique de raison 3,3.
\end{enumerate}
\begin{h2}Question 4\end{h2}
Les figures ci-dessous représentent 4 polynômes du second degré dans un repère orthonormé et le signe de leur discriminant  $ \Delta $ .
\newpar
Parmi ces propositions, laquelle est juste ?
\newpar
\begin{enumerate}[label=\alph*.]
     \item
     \begin{center}
          \begin{extern} %width="400" alt="Parabole 1"
               \resizebox{8cm}{!}{
                    %
                    \begin{tikzpicture}[line cap=round,line join=round,>=triangle 45,x=1.0cm,y=1.0cm]
                         \begin{axis}[
                              x=1.0cm,y=1.0cm,
                              axis lines=middle,
                              ymajorgrids=true,
                              xmajorgrids=true,
                              xmin=-3.0,
                              xmax=5.0,
                              ymin=-3.0,
                              ymax=3.0,
                              xtick={-3.0,-2.0,...,5.0},
                              ytick={-3.0,-2.0,...,3.0},]
                              \clip(-3.,-3.) rectangle (5.,3.);
                              \draw[line width=2.pt,color=qqqqff,smooth,samples=100,domain=-3.0:5.0] plot(\x,{0-0.45*((\x)-1.5)^(2.0)+2.0});
                         \end{axis}
                    \end{tikzpicture}
               }
          \end{extern}
     \end{center}
     \begin{center}
          $  \Delta >0$
     \end{center}
     \medbreak
     \item
     \begin{center}
          \begin{extern} %width="400" alt=""
               \resizebox{8cm}{!}{
                    %
                    \begin{tikzpicture}[line cap=round,line join=round,>=triangle 45,x=1.0cm,y=1.0cm]
                         \begin{axis}[
                              x=1.0cm,y=1.0cm,
                              axis lines=middle,
                              ymajorgrids=true,
                              xmajorgrids=true,
                              xmin=-5.0,
                              xmax=5.0,
                              ymin=-5.0,
                              ymax=2.0,
                              xtick={-5.0,-4.0,...,5.0},
                              ytick={-5.0,-4.0,...,2.0},]
                              \clip(-5.,-5.) rectangle (5.,2.);
                              \draw[line width=2.pt,color=qqqqff,smooth,samples=100,domain=-5.0:5.0] plot(\x,{(\x)^(2.0)+4.0*(\x)});
                         \end{axis}
                    \end{tikzpicture}
               }
          \end{extern}
     \end{center}
     \begin{center}
          $ \Delta =0$
     \end{center}
     \medbreak
     \item
     \begin{center}
          \begin{extern} %width="400" alt=""
               \resizebox{8cm}{!}{
                    %
                    \begin{tikzpicture}[line cap=round,line join=round,>=triangle 45,x=1.0cm,y=1.0cm]
                         \begin{axis}[
                              x=1.0cm,y=1.0cm,
                              axis lines=middle,
                              ymajorgrids=true,
                              xmajorgrids=true,
                              xmin=-4.0,
                              xmax=4.0,
                              ymin=-0.5,
                              ymax=6.0,
                              xtick={-4.0,-3.0,...,4.0},
                              ytick={-0.0,1.0,...,6.0},]
                              \clip(-4.,-0.5) rectangle (4.,6.);
                              \draw[line width=2.pt,color=qqqqff,smooth,samples=100,domain=-4.0:4.0] plot(\x,{(\x)^(2.0)+1.0});
                         \end{axis}
                    \end{tikzpicture}
               }
          \end{extern}
     \end{center}
     \begin{center}
          $ \Delta >0$
     \end{center}
     \medbreak
     \item
     \begin{center}
          \begin{extern} %width="400" alt=""
               \resizebox{8cm}{!}{
                    %
                    \begin{tikzpicture}[line cap=round,line join=round,>=triangle 45,x=1.0cm,y=0.1cm]
                         \begin{axis}[
                              x=1.0cm,y=0.1cm,
                              axis lines=middle,
                              ymajorgrids=true,
                              xmajorgrids=true,
                              xmin=-2.0,
                              xmax=6.0,
                              ymin=-40.0,
                              ymax=10.0,
                              xtick={-2.0,-1.0,...,6.0},
                              ytick={-40.0,-30.0,...,10.0},]
                              \clip(-2.,-40.) rectangle (6.,10.);
                              \draw[line width=2.pt,color=qqqqff,smooth,samples=100,domain=-2.0:6.0] plot(\x,{0-5.0*((\x)-2.0)^(2.0)});
                         \end{axis}
                    \end{tikzpicture}
               }
          \end{extern}
     \end{center}
     \begin{center}
          $ \Delta <0$
     \end{center}
     \newpar
\end{enumerate}
\begin{h2}Question 5\end{h2}
Le plan est rapporté à un repère orthonormé.
\newpar
$D$ est une droite dont une équation cartésienne est $2x-y+3=0$.
\newpar
Parmi ces propositions, laquelle est juste ?
\newpar
\begin{enumerate}[label=\alph*.]
     \item
     La droite $D$ passe par le point $A$ de coordonnées $( 2;1 )$
     \item
     La droite $D$ est dirigée par le vecteur de coordonnées $(  - 1;2 )$
     \item
     Le vecteur de coordonnées $( 2; - 1 )$ est normal à la droite $D$
     \item
     Le point d'intersection de la droite  $D$ avec l'axe des abscisses a comme coordonnées $( 0;3 ).$
\end{enumerate}
\begin{corrige}
     \cadre{vert}{Remarque}{ % id=r010
     Bien qu'aucun justificatif ne soit demandé, une explication est fournie pour aider les élèves.}
     % fin propriété
     \begin{h2}Question 1\end{h2}
     \textbf{Réponse correcte~: c.}
     \newpar
     En effet, on utilise la formule~:
     \[
     \left( \text{e}^{ a } \right){}^{ b }=\text{e}^{ ab }
     \]
     \begin{h2}Question 2\end{h2}
     \textbf{Réponse correcte~: b.}
     \newpar
     C'est une formule du cours que l'on peut retrouver à l'aide d'un cercle trigonométrique.
     \begin{h2}Question 3\end{h2}
     \textbf{Réponse correcte~: d.}
     \newpar
     Puisque l'on \textbf{ajoute} 3,3 mm par an, la suite $( U_{ n } )$ vérifie la relation de récurrence~:
     \newpar
     $U_{ n+1 }=U_{ n }+3,3$
     \newpar
     ce qui est la \mcLien{https://www.maths-cours.fr/cours/suites-geometriques/\#d10}{formule caractéristique d'une suite arithmétique} de raison $r=3,3.$
     \begin{h2}Question 4\end{h2}
     \textbf{Réponse correcte~: a.}
     \newpar
     \begin{enumerate}[label=\alph*.]
          \item
          La courbe possède deux points d'intersection avec l'axe des abscisses. Le polynôme admet donc  2 racines et son discriminant est donc bien strictement positif.
          \item
          Là encore, la courbe possède deux points d'intersection avec l'axe des abscisses~; le discriminant est strictement positif donc, il n'est pas nul.
          \item
          La courbe ne possède pas de point d'intersection avec l'axe des abscisses~; le discriminant du polynôme est donc strictement négatif.
          \item
          La courbe est tangente à l'axe des abscisses~; le polynôme admet donc une unique racine~; son discriminant est donc égal à zéro.
     \end{enumerate}
     \begin{h2}Question 5\end{h2}
     \textbf{Réponse correcte~: c.}
     \begin{enumerate}[label=\alph*.]
          \item
          Le couple $\left( 2~;~1 \right)$ ne vérifie pas l'équation  $2x - y+3=0$~; en effet~:
          \newpar
          $2 \times 2 - 1+3 \neq 0$
          \item
          La droite $D$ est dirigé par un vecteur de coordonnées $(  1~;~2 )$ qui n'est pas colinéaire au vecteur de coordonnées  $(  - 1~;~2 ).$
          \item
          Cette réponse est exacte. En effet, le vecteur de coordonnées  $( a~;~b )$ est normal à la droite d'équation $ax+by+c=0.$
          \item
          Le point de coordonnées $( 0~;~3 )$ appartient bien à la droite $D$~; toutefois, ce point n'est pas situé sur l'axe des abscisses mais sur l'axe des ordonnées.
     \end{enumerate}
\end{corrige}

\end{document}
µ
\documentclass[a4paper]{article}

%================================================================================================================================
%
% Packages
%
%================================================================================================================================

\usepackage[T1]{fontenc} 	% pour caractères accentués
\usepackage[utf8]{inputenc}  % encodage utf8
\usepackage[french]{babel}	% langue : français
\usepackage{fourier}			% caractères plus lisibles
\usepackage[dvipsnames]{xcolor} % couleurs
\usepackage{fancyhdr}		% réglage header footer
\usepackage{needspace}		% empêcher sauts de page mal placés
\usepackage{graphicx}		% pour inclure des graphiques
\usepackage{enumitem,cprotect}		% personnalise les listes d'items (nécessaire pour ol, al ...)
\usepackage{hyperref}		% Liens hypertexte
\usepackage{pstricks,pst-all,pst-node,pstricks-add,pst-math,pst-plot,pst-tree,pst-eucl} % pstricks
\usepackage[a4paper,includeheadfoot,top=2cm,left=3cm, bottom=2cm,right=3cm]{geometry} % marges etc.
\usepackage{comment}			% commentaires multilignes
\usepackage{amsmath,environ} % maths (matrices, etc.)
\usepackage{amssymb,makeidx}
\usepackage{bm}				% bold maths
\usepackage{tabularx}		% tableaux
\usepackage{colortbl}		% tableaux en couleur
\usepackage{fontawesome}		% Fontawesome
\usepackage{environ}			% environment with command
\usepackage{fp}				% calculs pour ps-tricks
\usepackage{multido}			% pour ps tricks
\usepackage[np]{numprint}	% formattage nombre
\usepackage{tikz,tkz-tab} 			% package principal TikZ
\usepackage{pgfplots}   % axes
\usepackage{mathrsfs}    % cursives
\usepackage{calc}			% calcul taille boites
\usepackage[scaled=0.875]{helvet} % font sans serif
\usepackage{svg} % svg
\usepackage{scrextend} % local margin
\usepackage{scratch} %scratch
\usepackage{multicol} % colonnes
%\usepackage{infix-RPN,pst-func} % formule en notation polanaise inversée
\usepackage{listings}

%================================================================================================================================
%
% Réglages de base
%
%================================================================================================================================

\lstset{
language=Python,   % R code
literate=
{á}{{\'a}}1
{à}{{\`a}}1
{ã}{{\~a}}1
{é}{{\'e}}1
{è}{{\`e}}1
{ê}{{\^e}}1
{í}{{\'i}}1
{ó}{{\'o}}1
{õ}{{\~o}}1
{ú}{{\'u}}1
{ü}{{\"u}}1
{ç}{{\c{c}}}1
{~}{{ }}1
}


\definecolor{codegreen}{rgb}{0,0.6,0}
\definecolor{codegray}{rgb}{0.5,0.5,0.5}
\definecolor{codepurple}{rgb}{0.58,0,0.82}
\definecolor{backcolour}{rgb}{0.95,0.95,0.92}

\lstdefinestyle{mystyle}{
    backgroundcolor=\color{backcolour},   
    commentstyle=\color{codegreen},
    keywordstyle=\color{magenta},
    numberstyle=\tiny\color{codegray},
    stringstyle=\color{codepurple},
    basicstyle=\ttfamily\footnotesize,
    breakatwhitespace=false,         
    breaklines=true,                 
    captionpos=b,                    
    keepspaces=true,                 
    numbers=left,                    
xleftmargin=2em,
framexleftmargin=2em,            
    showspaces=false,                
    showstringspaces=false,
    showtabs=false,                  
    tabsize=2,
    upquote=true
}

\lstset{style=mystyle}


\lstset{style=mystyle}
\newcommand{\imgdir}{C:/laragon/www/newmc/assets/imgsvg/}
\newcommand{\imgsvgdir}{C:/laragon/www/newmc/assets/imgsvg/}

\definecolor{mcgris}{RGB}{220, 220, 220}% ancien~; pour compatibilité
\definecolor{mcbleu}{RGB}{52, 152, 219}
\definecolor{mcvert}{RGB}{125, 194, 70}
\definecolor{mcmauve}{RGB}{154, 0, 215}
\definecolor{mcorange}{RGB}{255, 96, 0}
\definecolor{mcturquoise}{RGB}{0, 153, 153}
\definecolor{mcrouge}{RGB}{255, 0, 0}
\definecolor{mclightvert}{RGB}{205, 234, 190}

\definecolor{gris}{RGB}{220, 220, 220}
\definecolor{bleu}{RGB}{52, 152, 219}
\definecolor{vert}{RGB}{125, 194, 70}
\definecolor{mauve}{RGB}{154, 0, 215}
\definecolor{orange}{RGB}{255, 96, 0}
\definecolor{turquoise}{RGB}{0, 153, 153}
\definecolor{rouge}{RGB}{255, 0, 0}
\definecolor{lightvert}{RGB}{205, 234, 190}
\setitemize[0]{label=\color{lightvert}  $\bullet$}

\pagestyle{fancy}
\renewcommand{\headrulewidth}{0.2pt}
\fancyhead[L]{maths-cours.fr}
\fancyhead[R]{\thepage}
\renewcommand{\footrulewidth}{0.2pt}
\fancyfoot[C]{}

\newcolumntype{C}{>{\centering\arraybackslash}X}
\newcolumntype{s}{>{\hsize=.35\hsize\arraybackslash}X}

\setlength{\parindent}{0pt}		 
\setlength{\parskip}{3mm}
\setlength{\headheight}{1cm}

\def\ebook{ebook}
\def\book{book}
\def\web{web}
\def\type{web}

\newcommand{\vect}[1]{\overrightarrow{\,\mathstrut#1\,}}

\def\Oij{$\left(\text{O}~;~\vect{\imath},~\vect{\jmath}\right)$}
\def\Oijk{$\left(\text{O}~;~\vect{\imath},~\vect{\jmath},~\vect{k}\right)$}
\def\Ouv{$\left(\text{O}~;~\vect{u},~\vect{v}\right)$}

\hypersetup{breaklinks=true, colorlinks = true, linkcolor = OliveGreen, urlcolor = OliveGreen, citecolor = OliveGreen, pdfauthor={Didier BONNEL - https://www.maths-cours.fr} } % supprime les bordures autour des liens

\renewcommand{\arg}[0]{\text{arg}}

\everymath{\displaystyle}

%================================================================================================================================
%
% Macros - Commandes
%
%================================================================================================================================

\newcommand\meta[2]{    			% Utilisé pour créer le post HTML.
	\def\titre{titre}
	\def\url{url}
	\def\arg{#1}
	\ifx\titre\arg
		\newcommand\maintitle{#2}
		\fancyhead[L]{#2}
		{\Large\sffamily \MakeUppercase{#2}}
		\vspace{1mm}\textcolor{mcvert}{\hrule}
	\fi 
	\ifx\url\arg
		\fancyfoot[L]{\href{https://www.maths-cours.fr#2}{\black \footnotesize{https://www.maths-cours.fr#2}}}
	\fi 
}


\newcommand\TitreC[1]{    		% Titre centré
     \needspace{3\baselineskip}
     \begin{center}\textbf{#1}\end{center}
}

\newcommand\newpar{    		% paragraphe
     \par
}

\newcommand\nosp {    		% commande vide (pas d'espace)
}
\newcommand{\id}[1]{} %ignore

\newcommand\boite[2]{				% Boite simple sans titre
	\vspace{5mm}
	\setlength{\fboxrule}{0.2mm}
	\setlength{\fboxsep}{5mm}	
	\fcolorbox{#1}{#1!3}{\makebox[\linewidth-2\fboxrule-2\fboxsep]{
  		\begin{minipage}[t]{\linewidth-2\fboxrule-4\fboxsep}\setlength{\parskip}{3mm}
  			 #2
  		\end{minipage}
	}}
	\vspace{5mm}
}

\newcommand\CBox[4]{				% Boites
	\vspace{5mm}
	\setlength{\fboxrule}{0.2mm}
	\setlength{\fboxsep}{5mm}
	
	\fcolorbox{#1}{#1!3}{\makebox[\linewidth-2\fboxrule-2\fboxsep]{
		\begin{minipage}[t]{1cm}\setlength{\parskip}{3mm}
	  		\textcolor{#1}{\LARGE{#2}}    
 	 	\end{minipage}  
  		\begin{minipage}[t]{\linewidth-2\fboxrule-4\fboxsep}\setlength{\parskip}{3mm}
			\raisebox{1.2mm}{\normalsize\sffamily{\textcolor{#1}{#3}}}						
  			 #4
  		\end{minipage}
	}}
	\vspace{5mm}
}

\newcommand\cadre[3]{				% Boites convertible html
	\par
	\vspace{2mm}
	\setlength{\fboxrule}{0.1mm}
	\setlength{\fboxsep}{5mm}
	\fcolorbox{#1}{white}{\makebox[\linewidth-2\fboxrule-2\fboxsep]{
  		\begin{minipage}[t]{\linewidth-2\fboxrule-4\fboxsep}\setlength{\parskip}{3mm}
			\raisebox{-2.5mm}{\sffamily \small{\textcolor{#1}{\MakeUppercase{#2}}}}		
			\par		
  			 #3
 	 		\end{minipage}
	}}
		\vspace{2mm}
	\par
}

\newcommand\bloc[3]{				% Boites convertible html sans bordure
     \needspace{2\baselineskip}
     {\sffamily \small{\textcolor{#1}{\MakeUppercase{#2}}}}    
		\par		
  			 #3
		\par
}

\newcommand\CHelp[1]{
     \CBox{Plum}{\faInfoCircle}{À RETENIR}{#1}
}

\newcommand\CUp[1]{
     \CBox{NavyBlue}{\faThumbsOUp}{EN PRATIQUE}{#1}
}

\newcommand\CInfo[1]{
     \CBox{Sepia}{\faArrowCircleRight}{REMARQUE}{#1}
}

\newcommand\CRedac[1]{
     \CBox{PineGreen}{\faEdit}{BIEN R\'EDIGER}{#1}
}

\newcommand\CError[1]{
     \CBox{Red}{\faExclamationTriangle}{ATTENTION}{#1}
}

\newcommand\TitreExo[2]{
\needspace{4\baselineskip}
 {\sffamily\large EXERCICE #1\ (\emph{#2 points})}
\vspace{5mm}
}

\newcommand\img[2]{
          \includegraphics[width=#2\paperwidth]{\imgdir#1}
}

\newcommand\imgsvg[2]{
       \begin{center}   \includegraphics[width=#2\paperwidth]{\imgsvgdir#1} \end{center}
}


\newcommand\Lien[2]{
     \href{#1}{#2 \tiny \faExternalLink}
}
\newcommand\mcLien[2]{
     \href{https~://www.maths-cours.fr/#1}{#2 \tiny \faExternalLink}
}

\newcommand{\euro}{\eurologo{}}

%================================================================================================================================
%
% Macros - Environement
%
%================================================================================================================================

\newenvironment{tex}{ %
}
{%
}

\newenvironment{indente}{ %
	\setlength\parindent{10mm}
}

{
	\setlength\parindent{0mm}
}

\newenvironment{corrige}{%
     \needspace{3\baselineskip}
     \medskip
     \textbf{\textsc{Corrigé}}
     \medskip
}
{
}

\newenvironment{extern}{%
     \begin{center}
     }
     {
     \end{center}
}

\NewEnviron{code}{%
	\par
     \boite{gray}{\texttt{%
     \BODY
     }}
     \par
}

\newenvironment{vbloc}{% boite sans cadre empeche saut de page
     \begin{minipage}[t]{\linewidth}
     }
     {
     \end{minipage}
}
\NewEnviron{h2}{%
    \needspace{3\baselineskip}
    \vspace{0.6cm}
	\noindent \MakeUppercase{\sffamily \large \BODY}
	\vspace{1mm}\textcolor{mcgris}{\hrule}\vspace{0.4cm}
	\par
}{}

\NewEnviron{h3}{%
    \needspace{3\baselineskip}
	\vspace{5mm}
	\textsc{\BODY}
	\par
}

\NewEnviron{margeneg}{ %
\begin{addmargin}[-1cm]{0cm}
\BODY
\end{addmargin}
}

\NewEnviron{html}{%
}

\begin{document}
\meta{url}{/exercices/fonction-exponentielle-controle-continu-1ere-2020-sujet-zero/}
\meta{pid}{11198}
\meta{titre}{Fonction exponentielle - Contrôle continu 1ère - 2020 - Sujet zéro}
\meta{type}{exercices}
%
\begin{h2}Exercice  2 (5 points)\end{h2}
Une entreprise de menuiserie réalise des découpes dans des plaques rectangulaires de bois.
\newpar
Dans un repère orthonormé d'unité 30 cm ci-dessous, on modélise la forme de la découpe dans la plaque rectangulaire par la courbe $ \mathscr{C}_{ f }$ représentatif de la fonction  $f$ définie sur l'intervalle $[  - 1~;~2 ]$ par~:
\[
f( x )=(  - x+2 )\text{e}^{ x }.
\]
\begin{center}
     \begin{extern} %width="400" alt=""
          \resizebox{8cm}{!}{
               %
               \begin{tikzpicture}[line cap=round,line join=round,>=triangle 45,x=2.0cm,y=2.0cm]
                    \begin{axis}[
                         x=2.0cm,y=2.0cm,
                         axis lines=middle,
                         xmin=-1.8,
                         xmax=3.0,
                         ymin=-0.9,
                         ymax=3.3,
                         xtick={-1.0,0.0,...,3.0},
                         ytick={-0.0,1.0,...,3.0},]
                         \clip(-1.8,-0.9) rectangle (3.,3.3);
                         \fill[line width=2.pt,color=qqqqff,fill=qqqqff,fill opacity=0.05] (-1.,0.) -- (2.,0.) -- (2,2.717) -- (-1,2.717) -- (-1.,0.) -- cycle;
                         \draw[line width=2.pt,color=ffqqqq] (-0.9999984000000025,1.1036395007290078) -- (-0.9999984000000025,1.1036395007290078);
                         \draw[line width=2.pt,color=ffqqqq] (-0.9999984000000025,1.1036395007290078) -- (-0.9924984030000025,1.109168041106173);
                         \draw[line width=2.pt,color=ffqqqq] (-0.9924984030000025,1.109168041106173) -- (-0.9849984060000024,1.114717274001722);
                         \draw[line width=2.pt,color=ffqqqq] (-0.9849984060000024,1.114717274001722) -- (-0.9774984090000024,1.1202871976474402);
                         \draw[line width=2.pt,color=ffqqqq] (-0.9774984090000024,1.1202871976474402) -- (-0.9699984120000024,1.1258778090757695);
                         \draw[line width=2.pt,color=ffqqqq] (-0.9699984120000024,1.1258778090757695) -- (-0.9624984150000023,1.1314891041018496);
                         \draw[line width=2.pt,color=ffqqqq] (-0.9624984150000023,1.1314891041018496) -- (-0.9549984180000023,1.1371210773053586);
                         \draw[line width=2.pt,color=ffqqqq] (-0.9549984180000023,1.1371210773053586) -- (-0.9474984210000023,1.1427737220121488);
                         \draw[line width=2.pt,color=ffqqqq] (-0.9474984210000023,1.1427737220121488) -- (-0.9399984240000022,1.1484470302756753);
                         \draw[line width=2.pt,color=ffqqqq] (-0.9399984240000022,1.1484470302756753) -- (-0.9324984270000022,1.154140992858217);
                         \draw[line width=2.pt,color=ffqqqq] (-0.9324984270000022,1.154140992858217) -- (-0.9249984300000021,1.1598555992118857);
                         \draw[line width=2.pt,color=ffqqqq] (-0.9249984300000021,1.1598555992118857) -- (-0.9174984330000021,1.165590837459425);
                         \draw[line width=2.pt,color=ffqqqq] (-0.9174984330000021,1.165590837459425) -- (-0.9099984360000021,1.17134669437479);
                         \draw[line width=2.pt,color=ffqqqq] (-0.9099984360000021,1.17134669437479) -- (-0.902498439000002,1.1771231553635153);
                         \draw[line width=2.pt,color=ffqqqq] (-0.902498439000002,1.1771231553635153) -- (-0.894998442000002,1.182920204442859);
                         \draw[line width=2.pt,color=ffqqqq] (-0.894998442000002,1.182920204442859) -- (-0.887498445000002,1.1887378242217284);
                         \draw[line width=2.pt,color=ffqqqq] (-0.887498445000002,1.1887378242217284) -- (-0.8799984480000019,1.1945759958803808);
                         \draw[line width=2.pt,color=ffqqqq] (-0.8799984480000019,1.1945759958803808) -- (-0.8724984510000019,1.2004346991498995);
                         \draw[line width=2.pt,color=ffqqqq] (-0.8724984510000019,1.2004346991498995) -- (-0.8649984540000019,1.2063139122914401);
                         \draw[line width=2.pt,color=ffqqqq] (-0.8649984540000019,1.2063139122914401) -- (-0.8574984570000018,1.2122136120752478);
                         \draw[line width=2.pt,color=ffqqqq] (-0.8574984570000018,1.2122136120752478) -- (-0.8499984600000018,1.2181337737594415);
                         \draw[line width=2.pt,color=ffqqqq] (-0.8499984600000018,1.2181337737594415) -- (-0.8424984630000018,1.2240743710685642);
                         \draw[line width=2.pt,color=ffqqqq] (-0.8424984630000018,1.2240743710685642) -- (-0.8349984660000017,1.2300353761718932);
                         \draw[line width=2.pt,color=ffqqqq] (-0.8349984660000017,1.2300353761718932) -- (-0.8274984690000017,1.2360167596615146);
                         \draw[line width=2.pt,color=ffqqqq] (-0.8274984690000017,1.2360167596615146) -- (-0.8199984720000016,1.2420184905301521);
                         \draw[line width=2.pt,color=ffqqqq] (-0.8199984720000016,1.2420184905301521) -- (-0.8124984750000016,1.2480405361487545);
                         \draw[line width=2.pt,color=ffqqqq] (-0.8124984750000016,1.2480405361487545) -- (-0.8049984780000016,1.2540828622438331);
                         \draw[line width=2.pt,color=ffqqqq] (-0.8049984780000016,1.2540828622438331) -- (-0.7974984810000015,1.260145432874554);
                         \draw[line width=2.pt,color=ffqqqq] (-0.7974984810000015,1.260145432874554) -- (-0.7899984840000015,1.266228210409574);
                         \draw[line width=2.pt,color=ffqqqq] (-0.7899984840000015,1.266228210409574) -- (-0.7824984870000015,1.2723311555036256);
                         \draw[line width=2.pt,color=ffqqqq] (-0.7824984870000015,1.2723311555036256) -- (-0.7749984900000014,1.2784542270738453);
                         \draw[line width=2.pt,color=ffqqqq] (-0.7749984900000014,1.2784542270738453) -- (-0.7674984930000014,1.2845973822758385);
                         \draw[line width=2.pt,color=ffqqqq] (-0.7674984930000014,1.2845973822758385) -- (-0.7599984960000014,1.2907605764794894);
                         \draw[line width=2.pt,color=ffqqqq] (-0.7599984960000014,1.2907605764794894) -- (-0.7524984990000013,1.2969437632444991);
                         \draw[line width=2.pt,color=ffqqqq] (-0.7524984990000013,1.2969437632444991) -- (-0.7449985020000013,1.3031468942956628);
                         \draw[line width=2.pt,color=ffqqqq] (-0.7449985020000013,1.3031468942956628) -- (-0.7374985050000012,1.3093699194978727);
                         \draw[line width=2.pt,color=ffqqqq] (-0.7374985050000012,1.3093699194978727) -- (-0.7299985080000012,1.315612786830852);
                         \draw[line width=2.pt,color=ffqqqq] (-0.7299985080000012,1.315612786830852) -- (-0.7224985110000012,1.3218754423636117);
                         \draw[line width=2.pt,color=ffqqqq] (-0.7224985110000012,1.3218754423636117) -- (-0.7149985140000011,1.3281578302286303);
                         \draw[line width=2.pt,color=ffqqqq] (-0.7149985140000011,1.3281578302286303) -- (-0.7074985170000011,1.3344598925957536);
                         \draw[line width=2.pt,color=ffqqqq] (-0.7074985170000011,1.3344598925957536) -- (-0.6999985200000011,1.3407815696458099);
                         \draw[line width=2.pt,color=ffqqqq] (-0.6999985200000011,1.3407815696458099) -- (-0.692498523000001,1.3471227995439412);
                         \draw[line width=2.pt,color=ffqqqq] (-0.692498523000001,1.3471227995439412) -- (-0.684998526000001,1.3534835184126432);
                         \draw[line width=2.pt,color=ffqqqq] (-0.684998526000001,1.3534835184126432) -- (-0.677498529000001,1.3598636603045158);
                         \draw[line width=2.pt,color=ffqqqq] (-0.677498529000001,1.3598636603045158) -- (-0.6699985320000009,1.366263157174718);
                         \draw[line width=2.pt,color=ffqqqq] (-0.6699985320000009,1.366263157174718) -- (-0.6624985350000009,1.3726819388531248);
                         \draw[line width=2.pt,color=ffqqqq] (-0.6624985350000009,1.3726819388531248) -- (-0.6549985380000009,1.379119933016185);
                         \draw[line width=2.pt,color=ffqqqq] (-0.6549985380000009,1.379119933016185) -- (-0.6474985410000008,1.385577065158475);
                         \draw[line width=2.pt,color=ffqqqq] (-0.6474985410000008,1.385577065158475) -- (-0.6399985440000008,1.3920532585639471);
                         \draw[line width=2.pt,color=ffqqqq] (-0.6399985440000008,1.3920532585639471) -- (-0.6324985470000007,1.3985484342768673);
                         \draw[line width=2.pt,color=ffqqqq] (-0.6324985470000007,1.3985484342768673) -- (-0.6249985500000007,1.4050625110724415);
                         \draw[line width=2.pt,color=ffqqqq] (-0.6249985500000007,1.4050625110724415) -- (-0.6174985530000007,1.4115954054271267);
                         \draw[line width=2.pt,color=ffqqqq] (-0.6174985530000007,1.4115954054271267) -- (-0.6099985560000006,1.418147031488624);
                         \draw[line width=2.pt,color=ffqqqq] (-0.6099985560000006,1.418147031488624) -- (-0.6024985590000006,1.4247173010455478);
                         \draw[line width=2.pt,color=ffqqqq] (-0.6024985590000006,1.4247173010455478) -- (-0.5949985620000006,1.4313061234967739);
                         \draw[line width=2.pt,color=ffqqqq] (-0.5949985620000006,1.4313061234967739) -- (-0.5874985650000005,1.4379134058204568);
                         \draw[line width=2.pt,color=ffqqqq] (-0.5874985650000005,1.4379134058204568) -- (-0.5799985680000005,1.444539052542718);
                         \draw[line width=2.pt,color=ffqqqq] (-0.5799985680000005,1.444539052542718) -- (-0.5724985710000005,1.4511829657059985);
                         \draw[line width=2.pt,color=ffqqqq] (-0.5724985710000005,1.4511829657059985) -- (-0.5649985740000004,1.4578450448370734);
                         \draw[line width=2.pt,color=ffqqqq] (-0.5649985740000004,1.4578450448370734) -- (-0.5574985770000004,1.4645251869147295);
                         \draw[line width=2.pt,color=ffqqqq] (-0.5574985770000004,1.4645251869147295) -- (-0.5499985800000003,1.4712232863370935);
                         \draw[line width=2.pt,color=ffqqqq] (-0.5499985800000003,1.4712232863370935) -- (-0.5424985830000003,1.4779392348886158);
                         \draw[line width=2.pt,color=ffqqqq] (-0.5424985830000003,1.4779392348886158) -- (-0.5349985860000003,1.484672921706703);
                         \draw[line width=2.pt,color=ffqqqq] (-0.5349985860000003,1.484672921706703) -- (-0.5274985890000002,1.4914242332479966);
                         \draw[line width=2.pt,color=ffqqqq] (-0.5274985890000002,1.4914242332479966) -- (-0.5199985920000002,1.498193053254292);
                         \draw[line width=2.pt,color=ffqqqq] (-0.5199985920000002,1.498193053254292) -- (-0.5124985950000002,1.504979262718099);
                         \draw[line width=2.pt,color=ffqqqq] (-0.5124985950000002,1.504979262718099) -- (-0.5049985980000001,1.5117827398478354);
                         \draw[line width=2.pt,color=ffqqqq] (-0.5049985980000001,1.5117827398478354) -- (-0.49749860100000015,1.5186033600326527);
                         \draw[line width=2.pt,color=ffqqqq] (-0.49749860100000015,1.5186033600326527) -- (-0.48999860400000017,1.5254409958068933);
                         \draw[line width=2.pt,color=ffqqqq] (-0.48999860400000017,1.5254409958068933) -- (-0.4824986070000002,1.5322955168141654);
                         \draw[line width=2.pt,color=ffqqqq] (-0.4824986070000002,1.5322955168141654) -- (-0.4749986100000002,1.5391667897710466);
                         \draw[line width=2.pt,color=ffqqqq] (-0.4749986100000002,1.5391667897710466) -- (-0.46749861300000023,1.5460546784303992);
                         \draw[line width=2.pt,color=ffqqqq] (-0.46749861300000023,1.5460546784303992) -- (-0.45999861600000025,1.552959043544301);
                         \draw[line width=2.pt,color=ffqqqq] (-0.45999861600000025,1.552959043544301) -- (-0.45249861900000027,1.5598797428265876);
                         \draw[line width=2.pt,color=ffqqqq] (-0.45249861900000027,1.5598797428265876) -- (-0.4449986220000003,1.5668166309149965);
                         \draw[line width=2.pt,color=ffqqqq] (-0.4449986220000003,1.5668166309149965) -- (-0.4374986250000003,1.573769559332919);
                         \draw[line width=2.pt,color=ffqqqq] (-0.4374986250000003,1.573769559332919) -- (-0.4299986280000003,1.5807383764507454);
                         \draw[line width=2.pt,color=ffqqqq] (-0.4299986280000003,1.5807383764507454) -- (-0.42249863100000035,1.5877229274468085);
                         \draw[line width=2.pt,color=ffqqqq] (-0.42249863100000035,1.5877229274468085) -- (-0.41499863400000037,1.594723054267914);
                         \draw[line width=2.pt,color=ffqqqq] (-0.41499863400000037,1.594723054267914) -- (-0.4074986370000004,1.6017385955894625);
                         \draw[line width=2.pt,color=ffqqqq] (-0.4074986370000004,1.6017385955894625) -- (-0.3999986400000004,1.6087693867751498);
                         \draw[line width=2.pt,color=ffqqqq] (-0.3999986400000004,1.6087693867751498) -- (-0.3924986430000004,1.615815259836246);
                         \draw[line width=2.pt,color=ffqqqq] (-0.3924986430000004,1.615815259836246) -- (-0.38499864600000044,1.6228760433904548);
                         \draw[line width=2.pt,color=ffqqqq] (-0.38499864600000044,1.6228760433904548) -- (-0.37749864900000046,1.6299515626203351);
                         \draw[line width=2.pt,color=ffqqqq] (-0.37749864900000046,1.6299515626203351) -- (-0.3699986520000005,1.6370416392312948);
                         \draw[line width=2.pt,color=ffqqqq] (-0.3699986520000005,1.6370416392312948) -- (-0.3624986550000005,1.644146091409145);
                         \draw[line width=2.pt,color=ffqqqq] (-0.3624986550000005,1.644146091409145) -- (-0.3549986580000005,1.6512647337772124);
                         \draw[line width=2.pt,color=ffqqqq] (-0.3549986580000005,1.6512647337772124) -- (-0.34749866100000054,1.6583973773530043);
                         \draw[line width=2.pt,color=ffqqqq] (-0.34749866100000054,1.6583973773530043) -- (-0.33999866400000056,1.6655438295044247);
                         \draw[line width=2.pt,color=ffqqqq] (-0.33999866400000056,1.6655438295044247) -- (-0.3324986670000006,1.672703893905535);
                         \draw[line width=2.pt,color=ffqqqq] (-0.3324986670000006,1.672703893905535) -- (-0.3249986700000006,1.6798773704918541);
                         \draw[line width=2.pt,color=ffqqqq] (-0.3249986700000006,1.6798773704918541) -- (-0.3174986730000006,1.6870640554151974);
                         \draw[line width=2.pt,color=ffqqqq] (-0.3174986730000006,1.6870640554151974) -- (-0.30999867600000064,1.6942637409980457);
                         \draw[line width=2.pt,color=ffqqqq] (-0.30999867600000064,1.6942637409980457) -- (-0.30249867900000066,1.701476215687442);
                         \draw[line width=2.pt,color=ffqqqq] (-0.30249867900000066,1.701476215687442) -- (-0.2949986820000007,1.708701264008411);
                         \draw[line width=2.pt,color=ffqqqq] (-0.2949986820000007,1.708701264008411) -- (-0.2874986850000007,1.7159386665168979);
                         \draw[line width=2.pt,color=ffqqqq] (-0.2874986850000007,1.7159386665168979) -- (-0.2799986880000007,1.7231881997522198);
                         \draw[line width=2.pt,color=ffqqqq] (-0.2799986880000007,1.7231881997522198) -- (-0.27249869100000074,1.730449636189026);
                         \draw[line width=2.pt,color=ffqqqq] (-0.27249869100000074,1.730449636189026) -- (-0.26499869400000076,1.737722744188764);
                         \draw[line width=2.pt,color=ffqqqq] (-0.26499869400000076,1.737722744188764) -- (-0.2574986970000008,1.7450072879506435);
                         \draw[line width=2.pt,color=ffqqqq] (-0.2574986970000008,1.7450072879506435) -- (-0.2499987000000008,1.7523030274620972);
                         \draw[line width=2.pt,color=ffqqqq] (-0.2499987000000008,1.7523030274620972) -- (-0.24249870300000081,1.7596097184487285);
                         \draw[line width=2.pt,color=ffqqqq] (-0.24249870300000081,1.7596097184487285) -- (-0.23499870600000083,1.7669271123237476);
                         \draw[line width=2.pt,color=ffqqqq] (-0.23499870600000083,1.7669271123237476) -- (-0.22749870900000085,1.7742549561368859);
                         \draw[line width=2.pt,color=ffqqqq] (-0.22749870900000085,1.7742549561368859) -- (-0.21999871200000087,1.781592992522786);
                         \draw[line width=2.pt,color=ffqqqq] (-0.21999871200000087,1.781592992522786) -- (-0.2124987150000009,1.7889409596488626);
                         \draw[line width=2.pt,color=ffqqqq] (-0.2124987150000009,1.7889409596488626) -- (-0.2049987180000009,1.7962985911626264);
                         \draw[line width=2.pt,color=ffqqqq] (-0.2049987180000009,1.7962985911626264) -- (-0.19749872100000093,1.8036656161384723);
                         \draw[line width=2.pt,color=ffqqqq] (-0.19749872100000093,1.8036656161384723) -- (-0.18999872400000095,1.8110417590239178);
                         \draw[line width=2.pt,color=ffqqqq] (-0.18999872400000095,1.8110417590239178) -- (-0.18249872700000097,1.8184267395852947);
                         \draw[line width=2.pt,color=ffqqqq] (-0.18249872700000097,1.8184267395852947) -- (-0.174998730000001,1.8258202728528827);
                         \draw[line width=2.pt,color=ffqqqq] (-0.174998730000001,1.8258202728528827) -- (-0.167498733000001,1.8332220690654846);
                         \draw[line width=2.pt,color=ffqqqq] (-0.167498733000001,1.8332220690654846) -- (-0.15999873600000103,1.8406318336144336);
                         \draw[line width=2.pt,color=ffqqqq] (-0.15999873600000103,1.8406318336144336) -- (-0.15249873900000105,1.8480492669870296);
                         \draw[line width=2.pt,color=ffqqqq] (-0.15249873900000105,1.8480492669870296) -- (-0.14499874200000107,1.8554740647093995);
                         \draw[line width=2.pt,color=ffqqqq] (-0.14499874200000107,1.8554740647093995) -- (-0.1374987450000011,1.8629059172887739);
                         \draw[line width=2.pt,color=ffqqqq] (-0.1374987450000011,1.8629059172887739) -- (-0.1299987480000011,1.8703445101551759);
                         \draw[line width=2.pt,color=ffqqqq] (-0.1299987480000011,1.8703445101551759) -- (-0.12249875100000111,1.8777895236025177);
                         \draw[line width=2.pt,color=ffqqqq] (-0.12249875100000111,1.8777895236025177) -- (-0.11499875400000112,1.885240632729097);
                         \draw[line width=2.pt,color=ffqqqq] (-0.11499875400000112,1.885240632729097) -- (-0.10749875700000112,1.8926975073774872);
                         \draw[line width=2.pt,color=ffqqqq] (-0.10749875700000112,1.8926975073774872) -- (-0.09999876000000113,1.9001598120738217);
                         \draw[line width=2.pt,color=ffqqqq] (-0.09999876000000113,1.9001598120738217) -- (-0.09249876300000114,1.9076272059664552);
                         \draw[line width=2.pt,color=ffqqqq] (-0.09249876300000114,1.9076272059664552) -- (-0.08499876600000114,1.9150993427640095);
                         \draw[line width=2.pt,color=ffqqqq] (-0.08499876600000114,1.9150993427640095) -- (-0.07749876900000115,1.9225758706727853);
                         \draw[line width=2.pt,color=ffqqqq] (-0.07749876900000115,1.9225758706727853) -- (-0.06999877200000115,1.9300564323335447);
                         \draw[line width=2.pt,color=ffqqqq] (-0.06999877200000115,1.9300564323335447) -- (-0.06249877500000116,1.9375406647576514);
                         \draw[line width=2.pt,color=ffqqqq] (-0.06249877500000116,1.9375406647576514) -- (-0.054998778000001164,1.9450281992625646);
                         \draw[line width=2.pt,color=ffqqqq] (-0.054998778000001164,1.9450281992625646) -- (-0.04749878100000117,1.952518661406682);
                         \draw[line width=2.pt,color=ffqqqq] (-0.04749878100000117,1.952518661406682) -- (-0.039998784000001175,1.9600116709235227);
                         \draw[line width=2.pt,color=ffqqqq] (-0.039998784000001175,1.9600116709235227) -- (-0.03249878700000118,1.9675068416552481);
                         \draw[line width=2.pt,color=ffqqqq] (-0.03249878700000118,1.9675068416552481) -- (-0.024998790000001186,1.9750037814855084);
                         \draw[line width=2.pt,color=ffqqqq] (-0.024998790000001186,1.9750037814855084) -- (-0.017498793000001192,1.9825020922716166);
                         \draw[line width=2.pt,color=ffqqqq] (-0.017498793000001192,1.9825020922716166) -- (-0.0099987960000012,1.9900013697760337);
                         \draw[line width=2.pt,color=ffqqqq] (-0.0099987960000012,1.9900013697760337) -- (-0.002498799000001207,1.9975012035971678);
                         \draw[line width=2.pt,color=ffqqqq] (-0.002498799000001207,1.9975012035971678) -- (0.005001197999998786,2.005001177099475);
                         \draw[line width=2.pt,color=ffqqqq] (0.005001197999998786,2.005001177099475) -- (0.012501194999998778,2.0125008673428595);
                         \draw[line width=2.pt,color=ffqqqq] (0.012501194999998778,2.0125008673428595) -- (0.020001191999998773,2.019999845011358);
                         \draw[line width=2.pt,color=ffqqqq] (0.020001191999998773,2.019999845011358) -- (0.027501188999998767,2.027497674341116);
                         \draw[line width=2.pt,color=ffqqqq] (0.027501188999998767,2.027497674341116) -- (0.03500118599999876,2.034993913047633);
                         \draw[line width=2.pt,color=ffqqqq] (0.03500118599999876,2.034993913047633) -- (0.042501182999998756,2.0424881122522867);
                         \draw[line width=2.pt,color=ffqqqq] (0.042501182999998756,2.0424881122522867) -- (0.05000117999999875,2.0499798164081082);
                         \draw[line width=2.pt,color=ffqqqq] (0.05000117999999875,2.0499798164081082) -- (0.057501176999998745,2.0574685632248264);
                         \draw[line width=2.pt,color=ffqqqq] (0.057501176999998745,2.0574685632248264) -- (0.06500117399999873,2.0649538835931533);
                         \draw[line width=2.pt,color=ffqqqq] (0.06500117399999873,2.0649538835931533) -- (0.07250117099999873,2.0724353015083135);
                         \draw[line width=2.pt,color=ffqqqq] (0.07250117099999873,2.0724353015083135) -- (0.08000116799999872,2.0799123339928114);
                         \draw[line width=2.pt,color=ffqqqq] (0.08000116799999872,2.0799123339928114) -- (0.08750116499999872,2.087384491018421);
                         \draw[line width=2.pt,color=ffqqqq] (0.08750116499999872,2.087384491018421) -- (0.09500116199999871,2.094851275427403);
                         \draw[line width=2.pt,color=ffqqqq] (0.09500116199999871,2.094851275427403) -- (0.1025011589999987,2.102312182852928);
                         \draw[line width=2.pt,color=ffqqqq] (0.1025011589999987,2.102312182852928) -- (0.1100011559999987,2.1097667016387134);
                         \draw[line width=2.pt,color=ffqqqq] (0.1100011559999987,2.1097667016387134) -- (0.11750115299999869,2.1172143127578518);
                         \draw[line width=2.pt,color=ffqqqq] (0.11750115299999869,2.1172143127578518) -- (0.12500114999999867,2.124654489730835);
                         \draw[line width=2.pt,color=ffqqqq] (0.12500114999999867,2.124654489730835) -- (0.13250114699999865,2.1320866985427602);
                         \draw[line width=2.pt,color=ffqqqq] (0.13250114699999865,2.1320866985427602) -- (0.14000114399999863,2.13951039755971);
                         \draw[line width=2.pt,color=ffqqqq] (0.14000114399999863,2.13951039755971) -- (0.14750114099999861,2.1469250374443045);
                         \draw[line width=2.pt,color=ffqqqq] (0.14750114099999861,2.1469250374443045) -- (0.1550011379999986,2.1543300610704095);
                         \draw[line width=2.pt,color=ffqqqq] (0.1550011379999986,2.1543300610704095) -- (0.16250113499999858,2.161724903436999);
                         \draw[line width=2.pt,color=ffqqqq] (0.16250113499999858,2.161724903436999) -- (0.17000113199999856,2.1691089915811648);
                         \draw[line width=2.pt,color=ffqqqq] (0.17000113199999856,2.1691089915811648) -- (0.17750112899999854,2.176481744490256);
                         \draw[line width=2.pt,color=ffqqqq] (0.17750112899999854,2.176481744490256) -- (0.18500112599999852,2.1838425730131545);
                         \draw[line width=2.pt,color=ffqqqq] (0.18500112599999852,2.1838425730131545) -- (0.1925011229999985,2.191190879770666);
                         \draw[line width=2.pt,color=ffqqqq] (0.1925011229999985,2.191190879770666) -- (0.20000111999999848,2.1985260590650224);
                         \draw[line width=2.pt,color=ffqqqq] (0.20000111999999848,2.1985260590650224) -- (0.20750111699999846,2.205847496788492);
                         \draw[line width=2.pt,color=ffqqqq] (0.20750111699999846,2.205847496788492) -- (0.21500111399999844,2.213154570331084);
                         \draw[line width=2.pt,color=ffqqqq] (0.21500111399999844,2.213154570331084) -- (0.22250111099999842,2.220446648487335);
                         \draw[line width=2.pt,color=ffqqqq] (0.22250111099999842,2.220446648487335) -- (0.2300011079999984,2.227723091362181);
                         \draw[line width=2.pt,color=ffqqqq] (0.2300011079999984,2.227723091362181) -- (0.23750110499999838,2.234983250275897);
                         \draw[line width=2.pt,color=ffqqqq] (0.23750110499999838,2.234983250275897) -- (0.24500110199999836,2.2422264676680927);
                         \draw[line width=2.pt,color=ffqqqq] (0.24500110199999836,2.2422264676680927) -- (0.25250109899999834,2.249452077000771);
                         \draw[line width=2.pt,color=ffqqqq] (0.25250109899999834,2.249452077000771) -- (0.2600010959999983,2.256659402660416);
                         \draw[line width=2.pt,color=ffqqqq] (0.2600010959999983,2.256659402660416) -- (0.2675010929999983,2.2638477598591327);
                         \draw[line width=2.pt,color=ffqqqq] (0.2675010929999983,2.2638477598591327) -- (0.2750010899999983,2.2710164545347964);
                         \draw[line width=2.pt,color=ffqqqq] (0.2750010899999983,2.2710164545347964) -- (0.28250108699999826,2.2781647832502294);
                         \draw[line width=2.pt,color=ffqqqq] (0.28250108699999826,2.2781647832502294) -- (0.29000108399999824,2.28529203309138);
                         \draw[line width=2.pt,color=ffqqqq] (0.29000108399999824,2.28529203309138) -- (0.2975010809999982,2.2923974815645027);
                         \draw[line width=2.pt,color=ffqqqq] (0.2975010809999982,2.2923974815645027) -- (0.3050010779999982,2.299480396492326);
                         \draw[line width=2.pt,color=ffqqqq] (0.3050010779999982,2.299480396492326) -- (0.3125010749999982,2.3065400359091988);
                         \draw[line width=2.pt,color=ffqqqq] (0.3125010749999982,2.3065400359091988) -- (0.32000107199999817,2.313575647955208);
                         \draw[line width=2.pt,color=ffqqqq] (0.32000107199999817,2.313575647955208) -- (0.32750106899999815,2.3205864707692583);
                         \draw[line width=2.pt,color=ffqqqq] (0.32750106899999815,2.3205864707692583) -- (0.3350010659999981,2.3275717323810996);
                         \draw[line width=2.pt,color=ffqqqq] (0.3350010659999981,2.3275717323810996) -- (0.3425010629999981,2.3345306506022974);
                         \draw[line width=2.pt,color=ffqqqq] (0.3425010629999981,2.3345306506022974) -- (0.3500010599999981,2.341462432916135);
                         \draw[line width=2.pt,color=ffqqqq] (0.3500010599999981,2.341462432916135) -- (0.35750105699999807,2.348366276366435);
                         \draw[line width=2.pt,color=ffqqqq] (0.35750105699999807,2.348366276366435) -- (0.36500105399999805,2.3552413674452963);
                         \draw[line width=2.pt,color=ffqqqq] (0.36500105399999805,2.3552413674452963) -- (0.37250105099999803,2.362086881979727);
                         \draw[line width=2.pt,color=ffqqqq] (0.37250105099999803,2.362086881979727) -- (0.380001047999998,2.3689019850171715);
                         \draw[line width=2.pt,color=ffqqqq] (0.380001047999998,2.3689019850171715) -- (0.387501044999998,2.3756858307099202);
                         \draw[line width=2.pt,color=ffqqqq] (0.387501044999998,2.3756858307099202) -- (0.39500104199999797,2.3824375621983847);
                         \draw[line width=2.pt,color=ffqqqq] (0.39500104199999797,2.3824375621983847) -- (0.40250103899999795,2.3891563114932395);
                         \draw[line width=2.pt,color=ffqqqq] (0.40250103899999795,2.3891563114932395) -- (0.41000103599999793,2.395841199356407);
                         \draw[line width=2.pt,color=ffqqqq] (0.41000103599999793,2.395841199356407) -- (0.4175010329999979,2.4024913351808848);
                         \draw[line width=2.pt,color=ffqqqq] (0.4175010329999979,2.4024913351808848) -- (0.4250010299999979,2.409105816869401);
                         \draw[line width=2.pt,color=ffqqqq] (0.4250010299999979,2.409105816869401) -- (0.4325010269999979,2.4156837307118866);
                         \draw[line width=2.pt,color=ffqqqq] (0.4325010269999979,2.4156837307118866) -- (0.44000102399999785,2.4222241512617515);
                         \draw[line width=2.pt,color=ffqqqq] (0.44000102399999785,2.4222241512617515) -- (0.44750102099999783,2.4287261412109626);
                         \draw[line width=2.pt,color=ffqqqq] (0.44750102099999783,2.4287261412109626) -- (0.4550010179999978,2.4351887512638974);
                         \draw[line width=2.pt,color=ffqqqq] (0.4550010179999978,2.4351887512638974) -- (0.4625010149999978,2.4416110200099777);
                         \draw[line width=2.pt,color=ffqqqq] (0.4625010149999978,2.4416110200099777) -- (0.4700010119999978,2.4479919737950597);
                         \draw[line width=2.pt,color=ffqqqq] (0.4700010119999978,2.4479919737950597) -- (0.47750100899999776,2.4543306265915805);
                         \draw[line width=2.pt,color=ffqqqq] (0.47750100899999776,2.4543306265915805) -- (0.48500100599999774,2.4606259798674373);
                         \draw[line width=2.pt,color=ffqqqq] (0.48500100599999774,2.4606259798674373) -- (0.4925010029999977,2.4668770224535956);
                         \draw[line width=2.pt,color=ffqqqq] (0.4925010029999977,2.4668770224535956) -- (0.5000009999999977,2.4730827304104137);
                         \draw[line width=2.pt,color=ffqqqq] (0.5000009999999977,2.4730827304104137) -- (0.5075009969999977,2.4792420668926716);
                         \draw[line width=2.pt,color=ffqqqq] (0.5075009969999977,2.4792420668926716) -- (0.5150009939999978,2.485353982013289);
                         \draw[line width=2.pt,color=ffqqqq] (0.5150009939999978,2.485353982013289) -- (0.5225009909999978,2.4914174127057227);
                         \draw[line width=2.pt,color=ffqqqq] (0.5225009909999978,2.4914174127057227) -- (0.5300009879999978,2.497431282585035);
                         \draw[line width=2.pt,color=ffqqqq] (0.5300009879999978,2.497431282585035) -- (0.5375009849999979,2.5033945018076125);
                         \draw[line width=2.pt,color=ffqqqq] (0.5375009849999979,2.5033945018076125) -- (0.5450009819999979,2.50930596692953);
                         \draw[line width=2.pt,color=ffqqqq] (0.5450009819999979,2.50930596692953) -- (0.552500978999998,2.515164560763547);
                         \draw[line width=2.pt,color=ffqqqq] (0.552500978999998,2.515164560763547) -- (0.560000975999998,2.5209691522347155);
                         \draw[line width=2.pt,color=ffqqqq] (0.560000975999998,2.5209691522347155) -- (0.567500972999998,2.5267185962345997);
                         \draw[line width=2.pt,color=ffqqqq] (0.567500972999998,2.5267185962345997) -- (0.5750009699999981,2.5324117334740825);
                         \draw[line width=2.pt,color=ffqqqq] (0.5750009699999981,2.5324117334740825) -- (0.5825009669999981,2.538047390334753);
                         \draw[line width=2.pt,color=ffqqqq] (0.5825009669999981,2.538047390334753) -- (0.5900009639999981,2.5436243787188637);
                         \draw[line width=2.pt,color=ffqqqq] (0.5900009639999981,2.5436243787188637) -- (0.5975009609999982,2.5491414958978327);
                         \draw[line width=2.pt,color=ffqqqq] (0.5975009609999982,2.5491414958978327) -- (0.6050009579999982,2.554597524359292);
                         \draw[line width=2.pt,color=ffqqqq] (0.6050009579999982,2.554597524359292) -- (0.6125009549999982,2.5599912316526634);
                         \draw[line width=2.pt,color=ffqqqq] (0.6125009549999982,2.5599912316526634) -- (0.6200009519999983,2.565321370233237);
                         \draw[line width=2.pt,color=ffqqqq] (0.6200009519999983,2.565321370233237) -- (0.6275009489999983,2.570586677304762);
                         \draw[line width=2.pt,color=ffqqqq] (0.6275009489999983,2.570586677304762) -- (0.6350009459999983,2.575785874660515);
                         \draw[line width=2.pt,color=ffqqqq] (0.6350009459999983,2.575785874660515) -- (0.6425009429999984,2.580917668522838);
                         \draw[line width=2.pt,color=ffqqqq] (0.6425009429999984,2.580917668522838) -- (0.6500009399999984,2.5859807493811413);
                         \draw[line width=2.pt,color=ffqqqq] (0.6500009399999984,2.5859807493811413) -- (0.6575009369999985,2.5909737918283438);
                         \draw[line width=2.pt,color=ffqqqq] (0.6575009369999985,2.5909737918283438) -- (0.6650009339999985,2.5958954543957433);
                         \draw[line width=2.pt,color=ffqqqq] (0.6650009339999985,2.5958954543957433) -- (0.6725009309999985,2.6007443793863048);
                         \draw[line width=2.pt,color=ffqqqq] (0.6725009309999985,2.6007443793863048) -- (0.6800009279999986,2.605519192706343);
                         \draw[line width=2.pt,color=ffqqqq] (0.6800009279999986,2.605519192706343) -- (0.6875009249999986,2.610218503695597);
                         \draw[line width=2.pt,color=ffqqqq] (0.6875009249999986,2.610218503695597) -- (0.6950009219999986,2.6148409049556696);
                         \draw[line width=2.pt,color=ffqqqq] (0.6950009219999986,2.6148409049556696) -- (0.7025009189999987,2.6193849721768276);
                         \draw[line width=2.pt,color=ffqqqq] (0.7025009189999987,2.6193849721768276) -- (0.7100009159999987,2.6238492639631392);
                         \draw[line width=2.pt,color=ffqqqq] (0.7100009159999987,2.6238492639631392) -- (0.7175009129999987,2.628232321655941);
                         \draw[line width=2.pt,color=ffqqqq] (0.7175009129999987,2.628232321655941) -- (0.7250009099999988,2.6325326691556077);
                         \draw[line width=2.pt,color=ffqqqq] (0.7250009099999988,2.6325326691556077) -- (0.7325009069999988,2.6367488127416276);
                         \draw[line width=2.pt,color=ffqqqq] (0.7325009069999988,2.6367488127416276) -- (0.7400009039999988,2.640879240890948);
                         \draw[line width=2.pt,color=ffqqqq] (0.7400009039999988,2.640879240890948) -- (0.7475009009999989,2.6449224240945908);
                         \draw[line width=2.pt,color=ffqqqq] (0.7475009009999989,2.6449224240945908) -- (0.7550008979999989,2.6488768146725095);
                         \draw[line width=2.pt,color=ffqqqq] (0.7550008979999989,2.6488768146725095) -- (0.762500894999999,2.6527408465866835);
                         \draw[line width=2.pt,color=ffqqqq] (0.762500894999999,2.6527408465866835) -- (0.770000891999999,2.656512935252428);
                         \draw[line width=2.pt,color=ffqqqq] (0.770000891999999,2.656512935252428) -- (0.777500888999999,2.660191477347898);
                         \draw[line width=2.pt,color=ffqqqq] (0.777500888999999,2.660191477347898) -- (0.7850008859999991,2.6637748506217793);
                         \draw[line width=2.pt,color=ffqqqq] (0.7850008859999991,2.6637748506217793) -- (0.7925008829999991,2.6672614136991486);
                         \draw[line width=2.pt,color=ffqqqq] (0.7925008829999991,2.6672614136991486) -- (0.8000008799999991,2.670649505885475);
                         \draw[line width=2.pt,color=ffqqqq] (0.8000008799999991,2.670649505885475) -- (0.8075008769999992,2.6739374469687633);
                         \draw[line width=2.pt,color=ffqqqq] (0.8075008769999992,2.6739374469687633) -- (0.8150008739999992,2.677123537019807);
                         \draw[line width=2.pt,color=ffqqqq] (0.8150008739999992,2.677123537019807) -- (0.8225008709999992,2.6802060561905443);
                         \draw[line width=2.pt,color=ffqqqq] (0.8225008709999992,2.6802060561905443) -- (0.8300008679999993,2.6831832645104896);
                         \draw[line width=2.pt,color=ffqqqq] (0.8300008679999993,2.6831832645104896) -- (0.8375008649999993,2.686053401681238);
                         \draw[line width=2.pt,color=ffqqqq] (0.8375008649999993,2.686053401681238) -- (0.8450008619999994,2.6888146868690055);
                         \draw[line width=2.pt,color=ffqqqq] (0.8450008619999994,2.6888146868690055) -- (0.8525008589999994,2.6914653184952075);
                         \draw[line width=2.pt,color=ffqqqq] (0.8525008589999994,2.6914653184952075) -- (0.8600008559999994,2.6940034740250387);
                         \draw[line width=2.pt,color=ffqqqq] (0.8600008559999994,2.6940034740250387) -- (0.8675008529999995,2.6964273097540565);
                         \draw[line width=2.pt,color=ffqqqq] (0.8675008529999995,2.6964273097540565) -- (0.8750008499999995,2.6987349605927267);
                         \draw[line width=2.pt,color=ffqqqq] (0.8750008499999995,2.6987349605927267) -- (0.8825008469999995,2.700924539848934);
                         \draw[line width=2.pt,color=ffqqqq] (0.8825008469999995,2.700924539848934) -- (0.8900008439999996,2.7029941390084256);
                         \draw[line width=2.pt,color=ffqqqq] (0.8900008439999996,2.7029941390084256) -- (0.8975008409999996,2.7049418275131756);
                         \draw[line width=2.pt,color=ffqqqq] (0.8975008409999996,2.7049418275131756) -- (0.9050008379999996,2.7067656525376464);
                         \draw[line width=2.pt,color=ffqqqq] (0.9050008379999996,2.7067656525376464) -- (0.9125008349999997,2.7084636387629337);
                         \draw[line width=2.pt,color=ffqqqq] (0.9125008349999997,2.7084636387629337) -- (0.9200008319999997,2.710033788148771);
                         \draw[line width=2.pt,color=ffqqqq] (0.9200008319999997,2.710033788148771) -- (0.9275008289999997,2.711474079703376);
                         \draw[line width=2.pt,color=ffqqqq] (0.9275008289999997,2.711474079703376) -- (0.9350008259999998,2.712782469251121);
                         \draw[line width=2.pt,color=ffqqqq] (0.9350008259999998,2.712782469251121) -- (0.9425008229999998,2.7139568891980055);
                         \draw[line width=2.pt,color=ffqqqq] (0.9425008229999998,2.7139568891980055) -- (0.9500008199999999,2.7149952482949087);
                         \draw[line width=2.pt,color=ffqqqq] (0.9500008199999999,2.7149952482949087) -- (0.9575008169999999,2.71589543139861);
                         \draw[line width=2.pt,color=ffqqqq] (0.9575008169999999,2.71589543139861) -- (0.9650008139999999,2.716655299230547);
                         \draw[line width=2.pt,color=ffqqqq] (0.9650008139999999,2.716655299230547) -- (0.972500811,2.717272688133298);
                         \draw[line width=2.pt,color=ffqqqq] (0.972500811,2.717272688133298) -- (0.980000808,2.717745409824766);
                         \draw[line width=2.pt,color=ffqqqq] (0.980000808,2.717745409824766) -- (0.987500805,2.7180712511500422);
                         \draw[line width=2.pt,color=ffqqqq] (0.987500805,2.7180712511500422) -- (0.9950008020000001,2.718247973830928);
                         \draw[line width=2.pt,color=ffqqqq] (0.9950008020000001,2.718247973830928) -- (1.002500799,2.7182733142131004);
                         \draw[line width=2.pt,color=ffqqqq] (1.002500799,2.7182733142131004) -- (1.0100007960000001,2.718144983010885);
                         \draw[line width=2.pt,color=ffqqqq] (1.0100007960000001,2.718144983010885) -- (1.0175007930000002,2.717860665049635);
                         \draw[line width=2.pt,color=ffqqqq] (1.0175007930000002,2.717860665049635) -- (1.0250007900000002,2.717418019005678);
                         \draw[line width=2.pt,color=ffqqqq] (1.0250007900000002,2.717418019005678) -- (1.0325007870000003,2.716814677143814);
                         \draw[line width=2.pt,color=ffqqqq] (1.0325007870000003,2.716814677143814) -- (1.0400007840000003,2.716048245052347);
                         \draw[line width=2.pt,color=ffqqqq] (1.0400007840000003,2.716048245052347) -- (1.0475007810000003,2.7151163013756205);
                         \draw[line width=2.pt,color=ffqqqq] (1.0475007810000003,2.7151163013756205) -- (1.0550007780000004,2.714016397544038);
                         \draw[line width=2.pt,color=ffqqqq] (1.0550007780000004,2.714016397544038) -- (1.0625007750000004,2.712746057501549);
                         \draw[line width=2.pt,color=ffqqqq] (1.0625007750000004,2.712746057501549) -- (1.0700007720000004,2.711302777430569);
                         \draw[line width=2.pt,color=ffqqqq] (1.0700007720000004,2.711302777430569) -- (1.0775007690000005,2.709684025474321);
                         \draw[line width=2.pt,color=ffqqqq] (1.0775007690000005,2.709684025474321) -- (1.0850007660000005,2.707887241456566);
                         \draw[line width=2.pt,color=ffqqqq] (1.0850007660000005,2.707887241456566) -- (1.0925007630000005,2.705909836598703);
                         \draw[line width=2.pt,color=ffqqqq] (1.0925007630000005,2.705909836598703) -- (1.1000007600000006,2.7037491932342177);
                         \draw[line width=2.pt,color=ffqqqq] (1.1000007600000006,2.7037491932342177) -- (1.1075007570000006,2.701402664520448);
                         \draw[line width=2.pt,color=ffqqqq] (1.1075007570000006,2.701402664520448) -- (1.1150007540000006,2.698867574147651);
                         \draw[line width=2.pt,color=ffqqqq] (1.1150007540000006,2.698867574147651) -- (1.1225007510000007,2.6961412160453424);
                         \draw[line width=2.pt,color=ffqqqq] (1.1225007510000007,2.6961412160453424) -- (1.1300007480000007,2.693220854085882);
                         \draw[line width=2.pt,color=ffqqqq] (1.1300007480000007,2.693220854085882) -- (1.1375007450000008,2.6901037217852846);
                         \draw[line width=2.pt,color=ffqqqq] (1.1375007450000008,2.6901037217852846) -- (1.1450007420000008,2.6867870220012313);
                         \draw[line width=2.pt,color=ffqqqq] (1.1450007420000008,2.6867870220012313) -- (1.1525007390000008,2.68326792662825);
                         \draw[line width=2.pt,color=ffqqqq] (1.1525007390000008,2.68326792662825) -- (1.1600007360000009,2.67954357629005);
                         \draw[line width=2.pt,color=ffqqqq] (1.1600007360000009,2.67954357629005) -- (1.167500733000001,2.6756110800289723);
                         \draw[line width=2.pt,color=ffqqqq] (1.167500733000001,2.6756110800289723) -- (1.175000730000001,2.671467514992542);
                         \draw[line width=2.pt,color=ffqqqq] (1.175000730000001,2.671467514992542) -- (1.182500727000001,2.6671099261170887);
                         \draw[line width=2.pt,color=ffqqqq] (1.182500727000001,2.6671099261170887) -- (1.190000724000001,2.6625353258084092);
                         \draw[line width=2.pt,color=ffqqqq] (1.190000724000001,2.6625353258084092) -- (1.197500721000001,2.657740693619452);
                         \draw[line width=2.pt,color=ffqqqq] (1.197500721000001,2.657740693619452) -- (1.205000718000001,2.652722975924989);
                         \draw[line width=2.pt,color=ffqqqq] (1.205000718000001,2.652722975924989) -- (1.2125007150000011,2.647479085593252);
                         \draw[line width=2.pt,color=ffqqqq] (1.2125007150000011,2.647479085593252) -- (1.2200007120000012,2.6420059016545063);
                         \draw[line width=2.pt,color=ffqqqq] (1.2200007120000012,2.6420059016545063) -- (1.2275007090000012,2.6363002689665302);
                         \draw[line width=2.pt,color=ffqqqq] (1.2275007090000012,2.6363002689665302) -- (1.2350007060000012,2.630358997876981);
                         \draw[line width=2.pt,color=ffqqqq] (1.2350007060000012,2.630358997876981) -- (1.2425007030000013,2.6241788638826073);
                         \draw[line width=2.pt,color=ffqqqq] (1.2425007030000013,2.6241788638826073) -- (1.2500007000000013,2.6177566072852936);
                         \draw[line width=2.pt,color=ffqqqq] (1.2500007000000013,2.6177566072852936) -- (1.2575006970000013,2.611088932844893);
                         \draw[line width=2.pt,color=ffqqqq] (1.2575006970000013,2.611088932844893) -- (1.2650006940000014,2.604172509428836);
                         \draw[line width=2.pt,color=ffqqqq] (1.2650006940000014,2.604172509428836) -- (1.2725006910000014,2.59700396965847);
                         \draw[line width=2.pt,color=ffqqqq] (1.2725006910000014,2.59700396965847) -- (1.2800006880000014,2.589579909552115);
                         \draw[line width=2.pt,color=ffqqqq] (1.2800006880000014,2.589579909552115) -- (1.2875006850000015,2.5818968881647937);
                         \draw[line width=2.pt,color=ffqqqq] (1.2875006850000015,2.5818968881647937) -- (1.2950006820000015,2.5739514272246153);
                         \draw[line width=2.pt,color=ffqqqq] (1.2950006820000015,2.5739514272246153) -- (1.3025006790000015,2.565740010765779);
                         \draw[line width=2.pt,color=ffqqqq] (1.3025006790000015,2.565740010765779) -- (1.3100006760000016,2.557259084758163);
                         \draw[line width=2.pt,color=ffqqqq] (1.3100006760000016,2.557259084758163) -- (1.3175006730000016,2.5485050567334806);
                         \draw[line width=2.pt,color=ffqqqq] (1.3175006730000016,2.5485050567334806) -- (1.3250006700000017,2.5394742954079574);
                         \draw[line width=2.pt,color=ffqqqq] (1.3250006700000017,2.5394742954079574) -- (1.3325006670000017,2.5301631303015104);
                         \draw[line width=2.pt,color=ffqqqq] (1.3325006670000017,2.5301631303015104) -- (1.3400006640000017,2.5205678513533893);
                         \draw[line width=2.pt,color=ffqqqq] (1.3400006640000017,2.5205678513533893) -- (1.3475006610000018,2.5106847085342574);
                         \draw[line width=2.pt,color=ffqqqq] (1.3475006610000018,2.5106847085342574) -- (1.3550006580000018,2.500509911454669);
                         \draw[line width=2.pt,color=ffqqqq] (1.3550006580000018,2.500509911454669) -- (1.3625006550000018,2.4900396289699227);
                         \draw[line width=2.pt,color=ffqqqq] (1.3625006550000018,2.4900396289699227) -- (1.3700006520000019,2.4792699887812475);
                         \draw[line width=2.pt,color=ffqqqq] (1.3700006520000019,2.4792699887812475) -- (1.377500649000002,2.4681970770332984);
                         \draw[line width=2.pt,color=ffqqqq] (1.377500649000002,2.4681970770332984) -- (1.385000646000002,2.456816937907921);
                         \draw[line width=2.pt,color=ffqqqq] (1.385000646000002,2.456816937907921) -- (1.392500643000002,2.4451255732141552);
                         \draw[line width=2.pt,color=ffqqqq] (1.392500643000002,2.4451255732141552) -- (1.400000640000002,2.4331189419744472);
                         \draw[line width=2.pt,color=ffqqqq] (1.400000640000002,2.4331189419744472) -- (1.407500637000002,2.4207929600070273);
                         \draw[line width=2.pt,color=ffqqqq] (1.407500637000002,2.4207929600070273) -- (1.415000634000002,2.408143499504429);
                         \draw[line width=2.pt,color=ffqqqq] (1.415000634000002,2.408143499504429) -- (1.4225006310000021,2.395166388608109);
                         \draw[line width=2.pt,color=ffqqqq] (1.4225006310000021,2.395166388608109) -- (1.4300006280000022,2.381857410979138);
                         \draw[line width=2.pt,color=ffqqqq] (1.4300006280000022,2.381857410979138) -- (1.4375006250000022,2.368212305364917);
                         \draw[line width=2.pt,color=ffqqqq] (1.4375006250000022,2.368212305364917) -- (1.4450006220000022,2.3542267651618967);
                         \draw[line width=2.pt,color=ffqqqq] (1.4450006220000022,2.3542267651618967) -- (1.4525006190000023,2.3398964379742595);
                         \draw[line width=2.pt,color=ffqqqq] (1.4525006190000023,2.3398964379742595) -- (1.4600006160000023,2.325216925168522);
                         \draw[line width=2.pt,color=ffqqqq] (1.4600006160000023,2.325216925168522) -- (1.4675006130000023,2.3101837814240294);
                         \draw[line width=2.pt,color=ffqqqq] (1.4675006130000023,2.3101837814240294) -- (1.4750006100000024,2.2947925142793033);
                         \draw[line width=2.pt,color=ffqqqq] (1.4750006100000024,2.2947925142793033) -- (1.4825006070000024,2.2790385836742035);
                         \draw[line width=2.pt,color=ffqqqq] (1.4825006070000024,2.2790385836742035) -- (1.4900006040000024,2.2629174014878677);
                         \draw[line width=2.pt,color=ffqqqq] (1.4900006040000024,2.2629174014878677) -- (1.4975006010000025,2.246424331072395);
                         \draw[line width=2.pt,color=ffqqqq] (1.4975006010000025,2.246424331072395) -- (1.5050005980000025,2.229554686782227);
                         \draw[line width=2.pt,color=ffqqqq] (1.5050005980000025,2.229554686782227) -- (1.5125005950000026,2.2123037334991977);
                         \draw[line width=2.pt,color=ffqqqq] (1.5125005950000026,2.2123037334991977) -- (1.5200005920000026,2.1946666861532083);
                         \draw[line width=2.pt,color=ffqqqq] (1.5200005920000026,2.1946666861532083) -- (1.5275005890000026,2.1766387092384862);
                         \draw[line width=2.pt,color=ffqqqq] (1.5275005890000026,2.1766387092384862) -- (1.5350005860000027,2.158214916325395);
                         \draw[line width=2.pt,color=ffqqqq] (1.5350005860000027,2.158214916325395) -- (1.5425005830000027,2.139390369567752);
                         \draw[line width=2.pt,color=ffqqqq] (1.5425005830000027,2.139390369567752) -- (1.5500005800000027,2.12016007920561);
                         \draw[line width=2.pt,color=ffqqqq] (1.5500005800000027,2.12016007920561) -- (1.5575005770000028,2.100519003063473);
                         \draw[line width=2.pt,color=ffqqqq] (1.5575005770000028,2.100519003063473) -- (1.5650005740000028,2.080462046043893);
                         \draw[line width=2.pt,color=ffqqqq] (1.5650005740000028,2.080462046043893) -- (1.5725005710000028,2.059984059616416);
                         \draw[line width=2.pt,color=ffqqqq] (1.5725005710000028,2.059984059616416) -- (1.5800005680000029,2.039079841301834);
                         \draw[line width=2.pt,color=ffqqqq] (1.5800005680000029,2.039079841301834) -- (1.587500565000003,2.0177441341516946);
                         \draw[line width=2.pt,color=ffqqqq] (1.587500565000003,2.0177441341516946) -- (1.595000562000003,1.9959716262230374);
                         \draw[line width=2.pt,color=ffqqqq] (1.595000562000003,1.9959716262230374) -- (1.602500559000003,1.973756950048304);
                         \draw[line width=2.pt,color=ffqqqq] (1.602500559000003,1.973756950048304) -- (1.610000556000003,1.951094682100389);
                         \draw[line width=2.pt,color=ffqqqq] (1.610000556000003,1.951094682100389) -- (1.617500553000003,1.9279793422527758);
                         \draw[line width=2.pt,color=ffqqqq] (1.617500553000003,1.9279793422527758) -- (1.625000550000003,1.9044053932347285);
                         \draw[line width=2.pt,color=ffqqqq] (1.625000550000003,1.9044053932347285) -- (1.6325005470000031,1.880367240081484);
                         \draw[line width=2.pt,color=ffqqqq] (1.6325005470000031,1.880367240081484) -- (1.6400005440000032,1.8558592295794063);
                         \draw[line width=2.pt,color=ffqqqq] (1.6400005440000032,1.8558592295794063) -- (1.6475005410000032,1.8308756497060572);
                         \draw[line width=2.pt,color=ffqqqq] (1.6475005410000032,1.8308756497060572) -- (1.6550005380000032,1.8054107290651353);
                         \draw[line width=2.pt,color=ffqqqq] (1.6550005380000032,1.8054107290651353) -- (1.6625005350000033,1.7794586363162423);
                         \draw[line width=2.pt,color=ffqqqq] (1.6625005350000033,1.7794586363162423) -- (1.6700005320000033,1.7530134795994283);
                         \draw[line width=2.pt,color=ffqqqq] (1.6700005320000033,1.7530134795994283) -- (1.6775005290000033,1.7260693059544692);
                         \draw[line width=2.pt,color=ffqqqq] (1.6775005290000033,1.7260693059544692) -- (1.6850005260000034,1.6986201007348298);
                         \draw[line width=2.pt,color=ffqqqq] (1.6850005260000034,1.6986201007348298) -- (1.6925005230000034,1.6706597870162705);
                         \draw[line width=2.pt,color=ffqqqq] (1.6925005230000034,1.6706597870162705) -- (1.7000005200000035,1.642182225000038);
                         \draw[line width=2.pt,color=ffqqqq] (1.7000005200000035,1.642182225000038) -- (1.7075005170000035,1.613181211410607);
                         \draw[line width=2.pt,color=ffqqqq] (1.7075005170000035,1.613181211410607) -- (1.7150005140000035,1.5836504788879142);
                         \draw[line width=2.pt,color=ffqqqq] (1.7150005140000035,1.5836504788879142) -- (1.7225005110000036,1.5535836953740376);
                         \draw[line width=2.pt,color=ffqqqq] (1.7225005110000036,1.5535836953740376) -- (1.7300005080000036,1.5229744634942772);
                         \draw[line width=2.pt,color=ffqqqq] (1.7300005080000036,1.5229744634942772) -- (1.7375005050000036,1.4918163199325802);
                         \draw[line width=2.pt,color=ffqqqq] (1.7375005050000036,1.4918163199325802) -- (1.7450005020000037,1.4601027348012654);
                         \draw[line width=2.pt,color=ffqqqq] (1.7450005020000037,1.4601027348012654) -- (1.7525004990000037,1.4278271110049958);
                         \draw[line width=2.pt,color=ffqqqq] (1.7525004990000037,1.4278271110049958) -- (1.7600004960000037,1.3949827835989461);
                         \draw[line width=2.pt,color=ffqqqq] (1.7600004960000037,1.3949827835989461) -- (1.7675004930000038,1.3615630191411172);
                         \draw[line width=2.pt,color=ffqqqq] (1.7675004930000038,1.3615630191411172) -- (1.7750004900000038,1.3275610150387425);
                         \draw[line width=2.pt,color=ffqqqq] (1.7750004900000038,1.3275610150387425) -- (1.7825004870000039,1.2929698988887373);
                         \draw[line width=2.pt,color=ffqqqq] (1.7825004870000039,1.2929698988887373) -- (1.7900004840000039,1.2577827278121345);
                         \draw[line width=2.pt,color=ffqqqq] (1.7900004840000039,1.2577827278121345) -- (1.797500481000004,1.221992487782459);
                         \draw[line width=2.pt,color=ffqqqq] (1.797500481000004,1.221992487782459) -- (1.805000478000004,1.1855920929479793);
                         \draw[line width=2.pt,color=ffqqqq] (1.805000478000004,1.1855920929479793) -- (1.812500475000004,1.1485743849477914);
                         \draw[line width=2.pt,color=ffqqqq] (1.812500475000004,1.1485743849477914) -- (1.820000472000004,1.1109321322216734);
                         \draw[line width=2.pt,color=ffqqqq] (1.820000472000004,1.1109321322216734) -- (1.827500469000004,1.0726580293136596);
                         \draw[line width=2.pt,color=ffqqqq] (1.827500469000004,1.0726580293136596) -- (1.835000466000004,1.0337446961692758);
                         \draw[line width=2.pt,color=ffqqqq] (1.835000466000004,1.0337446961692758) -- (1.8425004630000041,0.9941846774263837);
                         \draw[line width=2.pt,color=ffqqqq] (1.8425004630000041,0.9941846774263837) -- (1.8500004600000042,0.953970441699574);
                         \draw[line width=2.pt,color=ffqqqq] (1.8500004600000042,0.953970441699574) -- (1.8575004570000042,0.9130943808580536);
                         \draw[line width=2.pt,color=ffqqqq] (1.8575004570000042,0.9130943808580536) -- (1.8650004540000042,0.871548809296969);
                         \draw[line width=2.pt,color=ffqqqq] (1.8650004540000042,0.871548809296969) -- (1.8725004510000043,0.8293259632021076);
                         \draw[line width=2.pt,color=ffqqqq] (1.8725004510000043,0.8293259632021076) -- (1.8800004480000043,0.7864179998079196);
                         \draw[line width=2.pt,color=ffqqqq] (1.8800004480000043,0.7864179998079196) -- (1.8875004450000044,0.7428169966487997);
                         \draw[line width=2.pt,color=ffqqqq] (1.8875004450000044,0.7428169966487997) -- (1.8950004420000044,0.6985149508035706);
                         \draw[line width=2.pt,color=ffqqqq] (1.8950004420000044,0.6985149508035706) -- (1.9025004390000044,0.6535037781331088);
                         \draw[line width=2.pt,color=ffqqqq] (1.9025004390000044,0.6535037781331088) -- (1.9100004360000045,0.6077753125110507);
                         \draw[line width=2.pt,color=ffqqqq] (1.9100004360000045,0.6077753125110507) -- (1.9175004330000045,0.561321305047519);
                         \draw[line width=2.pt,color=ffqqqq] (1.9175004330000045,0.561321305047519) -- (1.9250004300000045,0.5141334233058077);
                         \draw[line width=2.pt,color=ffqqqq] (1.9250004300000045,0.5141334233058077) -- (1.9325004270000046,0.46620325051196454);
                         \draw[line width=2.pt,color=ffqqqq] (1.9325004270000046,0.46620325051196454) -- (1.9400004240000046,0.4175222847572058);
                         \draw[line width=2.pt,color=ffqqqq] (1.9400004240000046,0.4175222847572058) -- (1.9475004210000046,0.36808193819310403);
                         \draw[line width=2.pt,color=ffqqqq] (1.9475004210000046,0.36808193819310403) -- (1.9550004180000047,0.31787353621948133);
                         \draw[line width=2.pt,color=ffqqqq] (1.9550004180000047,0.31787353621948133) -- (1.9625004150000047,0.26688831666494744);
                         \draw[line width=2.pt,color=ffqqqq] (1.9625004150000047,0.26688831666494744) -- (1.9700004120000048,0.21511742896001462);
                         \draw[line width=2.pt,color=ffqqqq] (1.9700004120000048,0.21511742896001462) -- (1.9775004090000048,0.16255193330272688);
                         \draw[line width=2.pt,color=ffqqqq] (1.9775004090000048,0.16255193330272688) -- (1.9850004060000048,0.10918279981673583);
                         \draw[line width=2.pt,color=ffqqqq] (1.9850004060000048,0.10918279981673583) -- (1.9925004030000049,0.05500090770175727);
                         \draw [->,line width=0.4pt,color=tttttt] (0.,-0.3986194995685967) -- (1.9989669421487641,-0.38826574633304883);
                         \draw [->,line width=0.4pt,color=tttttt] (0.,-0.3986194995685967) -- (-0.9969008264462793,-0.3986194995685967);
                         \draw [color=tttttt](0.3460743801652916,-0.4400345125107877) node[anchor=north west] {$L$};
                         \draw [->,line width=0.4pt,color=tttttt] (2.6,1.) -- (2.6,2.72);
                         \draw [->,line width=0.4pt,color=tttttt] (2.6,1.) -- (2.6,0.);
                         \draw [color=ffqqqq](1.6167355371900858,2.5314926660914567) node[anchor=north west] {$\mathscr{C}_f$};
                         \draw [color=tttttt](2.70144628099174,1.6617773943054341) node[anchor=north west] {$l$};
                         \draw [line width=2.pt,color=qqqqff] (-1.,0.)-- (2.,0.);
                         \draw [line width=2.pt,color=qqqqff] (2.,0.)-- (1.988636363636367,2.7178602243313184);
                         \draw [line width=2.pt,color=qqqqff] (1.988636363636367,2.7178602243313184)-- (-0.9969008264462796,2.7282139775668663);
                         \draw [line width=2.pt,color=qqqqff] (-0.9969008264462796,2.7282139775668663)-- (-1.,0.);
                         \draw [line width=2.pt,color=qqqqff] (-1.,0.)-- (-1.,0.);
                    \end{axis}
               \end{tikzpicture}
          }
     \end{extern}
\end{center}
Le bord supérieur de la plaque rectangulaire est tangent à la courbe $ \mathscr{C}_{ f }$. On nomme $L$ la longueur de la plaque rectangulaire et $ \mathscr{l}$ sa largeur.
\begin{enumerate}
     \item
     On note $f' $ la fonction dérivée de  $f$.
     \begin{enumerate}[label=\alph*.]
          \item
          Montrer que pour tout réel $x$ de l'intervalle $[  - 1~;~2 ]$ , $f' ( x )=(  - x+1 )\text{e}^{ x }. $
          \item
          En déduire le tableau de variations de la fonction $f$ sur $[  - 1~;~2 ].$
     \end{enumerate}
     \item
     La longueur $L$ de la plaque rectangulaire est de 90 cm. Trouver sa largeur $ \mathscr{l}$ exacte en centimètres.
\end{enumerate}
\begin{corrige}
     \begin{enumerate}
          \item
          \begin{enumerate}[label=\alph*.]
               \item
               Pour calculer la dérivée $f'$ de la fonction $f$ on utilise la \mcLien{https://www.maths-cours.fr/cours/fonction-derivee/\#p60}{formule}~:
               \[
               ( uv )' =u' v+uv'
               \]
               où  $u$ et $v$ sont les fonctions définies par~:\\
               \begin{itemize}
                    \item
                    $u( x )= - x+2$
                    \item
                    $v( x )=\text{e}^{ x }$
               \end{itemize}
               On a alors~:
               \begin{itemize}
                    \item
                    $u' ( x )= - 1$
                    \item
                    $v' ( x )=\text{e}^{ x }$
               \end{itemize}
               Par conséquent, pour tout réel $x$ de l'intervalle  $\left[  - 1~;~2\right]$~:
               \newpar
               $f' ( x )= - \text{e}^{ x }+(  - x+2 )\text{e}^{ x }$\\
               $\phantom{f' ( x )}=\text{e}^{ x }\left(  - 1 - x+2 \right)$\\
               $\phantom{f' ( x )}=\left(  - x+1 \right)\text{e}^{ x }.$
               \item
               Pour tout réel $x$,  $\text{e}^{ x }$ est strictement positif~; donc $f' $ est du signe de  $ - x+1$ c'est-à-dire~:
               \begin{itemize}
                    \item
                    $f' $ s'annule pour $x=1$
                    \item
                    $f' $ est strictement positive pour $x < 1$
                    \item
                    $f' $ est strictement négative pour $x > 1. $
               \end{itemize}
               \newpar
               On a par ailleurs :
               \begin{itemize}
                    \item
                    $f(  - 1 )=( 1+2 )\text{e}^{  - 1 }=3\text{e}^{  - 1 }=\frac{ 3 }{ \text{e} }$
                    \item
                    $f( 1 )=(  - 1+2 )\text{e}^{ 1 }=\text{e}$
                    \item
                    $f( 2)=(  - 2 +2)\text{e}^{ 2 }=0$
               \end{itemize}
               \newpar
               On obtient alors le tableau de variation ci-dessous~:\\
               \begin{center}
                    \begin{extern}%width="340" alt="Tableau de variation Contrôle continu"
                         \begin{tikzpicture}[scale=0.875]
                              % Styles
                              \tikzstyle{cadre}=[thin]
                              \tikzstyle{fleche}=[->,>=latex,thin]
                              \tikzstyle{nondefini}=[lightgray]
                              % Dimensions Modifiables
                              \def\Lrg{1.5}
                              \def\HtX{1}
                              \def\HtY{0.5}
                              % Dimensions Calculées
                              \def\lignex{-0.5*\HtX}
                              \def\lignef{-1.5*\HtX}
                              \def\separateur{-0.5*\Lrg}
                              % Largeur du tableau
                              \def\gauche{-1.5*\Lrg}
                              \def\droite{4.5*\Lrg}
                              % Hauteur du tableau
                              \def\haut{0.5*\HtX}
                              \def\bas{-2.5*\HtX-2*\HtY}
                              % Pointillés
                              % Ligne de l'abscisse : x
                              \node at (-1*\Lrg,0) {$x$};
                              \node at (0*\Lrg,0) {$-1$};
                              \node at (2*\Lrg,0) {$1$};
                              \node at (4*\Lrg,0) {$2$};
                              % Ligne de la dérivée : f'(x)
                              \node at (-1*\Lrg,-1*\HtX) {$f'(x)$};
                              \node at (0*\Lrg,-1*\HtX) {$$};
                              \node at (1*\Lrg,-1*\HtX) {$+$};
                              \node at (2*\Lrg,-1*\HtX) {$0$};
                              \draw[gray] (2*\Lrg,\lignex) -- (2*\Lrg,\lignef);
                              \node at (3*\Lrg,-1*\HtX) {$-$};
                              \node at (4*\Lrg,-1*\HtX) {$$};
                              % Ligne de la fonction : f(x)
                              \node  at (-1*\Lrg,{-2*\HtX+(-1)*\HtY}) {$f(x)$};
                              \node (f1) at (0*\Lrg,{-2*\HtX+(-2)*\HtY}) {$3\text{e}^{  - 1 }$};
                              \node (f2) at (2*\Lrg,{-2*\HtX+(-0)*\HtY}) { $\text{e}$ };
                              \draw[gray] (2*\Lrg,\lignef) -- (2*\Lrg,\bas);
                              \node (f3) at (4*\Lrg,{-2*\HtX+(-2)*\HtY}) { $0$ };
                              % Flèches
                              \draw[fleche] (f1) -- (f2);
                              \draw[fleche] (f2) -- (f3);
                              % Encadrement
                              \draw[cadre] (\separateur,\haut) -- (\separateur,\bas);
                              \draw[cadre] (\gauche,\haut) rectangle  (\droite,\bas);
                              \draw[cadre] (\gauche,\lignex) -- (\droite,\lignex);
                              \draw[cadre] (\gauche,\lignef) -- (\droite,\lignef);
                         \end{tikzpicture}
                    \end{extern}
               \end{center}
          \end{enumerate}
          \item
          Le maximum de la fonction  $f$ est $f( 1 )=\text{e}$~;  son minimum est  $f( 2 )=0$. La largeur de la plaque est donc $\text{e}$ unités. L'unité mesurant 30 cm, la largeur de la plaque est donc  $l=30\text{e}$ centimètres (soit environ 81,5 cm mais c'est la valeur exacte qui est demandée…).
     \end{enumerate}
\end{corrige}

\end{document}
µ
\documentclass[a4paper]{article}

%================================================================================================================================
%
% Packages
%
%================================================================================================================================

\usepackage[T1]{fontenc} 	% pour caractères accentués
\usepackage[utf8]{inputenc}  % encodage utf8
\usepackage[french]{babel}	% langue : français
\usepackage{fourier}			% caractères plus lisibles
\usepackage[dvipsnames]{xcolor} % couleurs
\usepackage{fancyhdr}		% réglage header footer
\usepackage{needspace}		% empêcher sauts de page mal placés
\usepackage{graphicx}		% pour inclure des graphiques
\usepackage{enumitem,cprotect}		% personnalise les listes d'items (nécessaire pour ol, al ...)
\usepackage{hyperref}		% Liens hypertexte
\usepackage{pstricks,pst-all,pst-node,pstricks-add,pst-math,pst-plot,pst-tree,pst-eucl} % pstricks
\usepackage[a4paper,includeheadfoot,top=2cm,left=3cm, bottom=2cm,right=3cm]{geometry} % marges etc.
\usepackage{comment}			% commentaires multilignes
\usepackage{amsmath,environ} % maths (matrices, etc.)
\usepackage{amssymb,makeidx}
\usepackage{bm}				% bold maths
\usepackage{tabularx}		% tableaux
\usepackage{colortbl}		% tableaux en couleur
\usepackage{fontawesome}		% Fontawesome
\usepackage{environ}			% environment with command
\usepackage{fp}				% calculs pour ps-tricks
\usepackage{multido}			% pour ps tricks
\usepackage[np]{numprint}	% formattage nombre
\usepackage{tikz,tkz-tab} 			% package principal TikZ
\usepackage{pgfplots}   % axes
\usepackage{mathrsfs}    % cursives
\usepackage{calc}			% calcul taille boites
\usepackage[scaled=0.875]{helvet} % font sans serif
\usepackage{svg} % svg
\usepackage{scrextend} % local margin
\usepackage{scratch} %scratch
\usepackage{multicol} % colonnes
%\usepackage{infix-RPN,pst-func} % formule en notation polanaise inversée
\usepackage{listings}

%================================================================================================================================
%
% Réglages de base
%
%================================================================================================================================

\lstset{
language=Python,   % R code
literate=
{á}{{\'a}}1
{à}{{\`a}}1
{ã}{{\~a}}1
{é}{{\'e}}1
{è}{{\`e}}1
{ê}{{\^e}}1
{í}{{\'i}}1
{ó}{{\'o}}1
{õ}{{\~o}}1
{ú}{{\'u}}1
{ü}{{\"u}}1
{ç}{{\c{c}}}1
{~}{{ }}1
}


\definecolor{codegreen}{rgb}{0,0.6,0}
\definecolor{codegray}{rgb}{0.5,0.5,0.5}
\definecolor{codepurple}{rgb}{0.58,0,0.82}
\definecolor{backcolour}{rgb}{0.95,0.95,0.92}

\lstdefinestyle{mystyle}{
    backgroundcolor=\color{backcolour},   
    commentstyle=\color{codegreen},
    keywordstyle=\color{magenta},
    numberstyle=\tiny\color{codegray},
    stringstyle=\color{codepurple},
    basicstyle=\ttfamily\footnotesize,
    breakatwhitespace=false,         
    breaklines=true,                 
    captionpos=b,                    
    keepspaces=true,                 
    numbers=left,                    
xleftmargin=2em,
framexleftmargin=2em,            
    showspaces=false,                
    showstringspaces=false,
    showtabs=false,                  
    tabsize=2,
    upquote=true
}

\lstset{style=mystyle}


\lstset{style=mystyle}
\newcommand{\imgdir}{C:/laragon/www/newmc/assets/imgsvg/}
\newcommand{\imgsvgdir}{C:/laragon/www/newmc/assets/imgsvg/}

\definecolor{mcgris}{RGB}{220, 220, 220}% ancien~; pour compatibilité
\definecolor{mcbleu}{RGB}{52, 152, 219}
\definecolor{mcvert}{RGB}{125, 194, 70}
\definecolor{mcmauve}{RGB}{154, 0, 215}
\definecolor{mcorange}{RGB}{255, 96, 0}
\definecolor{mcturquoise}{RGB}{0, 153, 153}
\definecolor{mcrouge}{RGB}{255, 0, 0}
\definecolor{mclightvert}{RGB}{205, 234, 190}

\definecolor{gris}{RGB}{220, 220, 220}
\definecolor{bleu}{RGB}{52, 152, 219}
\definecolor{vert}{RGB}{125, 194, 70}
\definecolor{mauve}{RGB}{154, 0, 215}
\definecolor{orange}{RGB}{255, 96, 0}
\definecolor{turquoise}{RGB}{0, 153, 153}
\definecolor{rouge}{RGB}{255, 0, 0}
\definecolor{lightvert}{RGB}{205, 234, 190}
\setitemize[0]{label=\color{lightvert}  $\bullet$}

\pagestyle{fancy}
\renewcommand{\headrulewidth}{0.2pt}
\fancyhead[L]{maths-cours.fr}
\fancyhead[R]{\thepage}
\renewcommand{\footrulewidth}{0.2pt}
\fancyfoot[C]{}

\newcolumntype{C}{>{\centering\arraybackslash}X}
\newcolumntype{s}{>{\hsize=.35\hsize\arraybackslash}X}

\setlength{\parindent}{0pt}		 
\setlength{\parskip}{3mm}
\setlength{\headheight}{1cm}

\def\ebook{ebook}
\def\book{book}
\def\web{web}
\def\type{web}

\newcommand{\vect}[1]{\overrightarrow{\,\mathstrut#1\,}}

\def\Oij{$\left(\text{O}~;~\vect{\imath},~\vect{\jmath}\right)$}
\def\Oijk{$\left(\text{O}~;~\vect{\imath},~\vect{\jmath},~\vect{k}\right)$}
\def\Ouv{$\left(\text{O}~;~\vect{u},~\vect{v}\right)$}

\hypersetup{breaklinks=true, colorlinks = true, linkcolor = OliveGreen, urlcolor = OliveGreen, citecolor = OliveGreen, pdfauthor={Didier BONNEL - https://www.maths-cours.fr} } % supprime les bordures autour des liens

\renewcommand{\arg}[0]{\text{arg}}

\everymath{\displaystyle}

%================================================================================================================================
%
% Macros - Commandes
%
%================================================================================================================================

\newcommand\meta[2]{    			% Utilisé pour créer le post HTML.
	\def\titre{titre}
	\def\url{url}
	\def\arg{#1}
	\ifx\titre\arg
		\newcommand\maintitle{#2}
		\fancyhead[L]{#2}
		{\Large\sffamily \MakeUppercase{#2}}
		\vspace{1mm}\textcolor{mcvert}{\hrule}
	\fi 
	\ifx\url\arg
		\fancyfoot[L]{\href{https://www.maths-cours.fr#2}{\black \footnotesize{https://www.maths-cours.fr#2}}}
	\fi 
}


\newcommand\TitreC[1]{    		% Titre centré
     \needspace{3\baselineskip}
     \begin{center}\textbf{#1}\end{center}
}

\newcommand\newpar{    		% paragraphe
     \par
}

\newcommand\nosp {    		% commande vide (pas d'espace)
}
\newcommand{\id}[1]{} %ignore

\newcommand\boite[2]{				% Boite simple sans titre
	\vspace{5mm}
	\setlength{\fboxrule}{0.2mm}
	\setlength{\fboxsep}{5mm}	
	\fcolorbox{#1}{#1!3}{\makebox[\linewidth-2\fboxrule-2\fboxsep]{
  		\begin{minipage}[t]{\linewidth-2\fboxrule-4\fboxsep}\setlength{\parskip}{3mm}
  			 #2
  		\end{minipage}
	}}
	\vspace{5mm}
}

\newcommand\CBox[4]{				% Boites
	\vspace{5mm}
	\setlength{\fboxrule}{0.2mm}
	\setlength{\fboxsep}{5mm}
	
	\fcolorbox{#1}{#1!3}{\makebox[\linewidth-2\fboxrule-2\fboxsep]{
		\begin{minipage}[t]{1cm}\setlength{\parskip}{3mm}
	  		\textcolor{#1}{\LARGE{#2}}    
 	 	\end{minipage}  
  		\begin{minipage}[t]{\linewidth-2\fboxrule-4\fboxsep}\setlength{\parskip}{3mm}
			\raisebox{1.2mm}{\normalsize\sffamily{\textcolor{#1}{#3}}}						
  			 #4
  		\end{minipage}
	}}
	\vspace{5mm}
}

\newcommand\cadre[3]{				% Boites convertible html
	\par
	\vspace{2mm}
	\setlength{\fboxrule}{0.1mm}
	\setlength{\fboxsep}{5mm}
	\fcolorbox{#1}{white}{\makebox[\linewidth-2\fboxrule-2\fboxsep]{
  		\begin{minipage}[t]{\linewidth-2\fboxrule-4\fboxsep}\setlength{\parskip}{3mm}
			\raisebox{-2.5mm}{\sffamily \small{\textcolor{#1}{\MakeUppercase{#2}}}}		
			\par		
  			 #3
 	 		\end{minipage}
	}}
		\vspace{2mm}
	\par
}

\newcommand\bloc[3]{				% Boites convertible html sans bordure
     \needspace{2\baselineskip}
     {\sffamily \small{\textcolor{#1}{\MakeUppercase{#2}}}}    
		\par		
  			 #3
		\par
}

\newcommand\CHelp[1]{
     \CBox{Plum}{\faInfoCircle}{À RETENIR}{#1}
}

\newcommand\CUp[1]{
     \CBox{NavyBlue}{\faThumbsOUp}{EN PRATIQUE}{#1}
}

\newcommand\CInfo[1]{
     \CBox{Sepia}{\faArrowCircleRight}{REMARQUE}{#1}
}

\newcommand\CRedac[1]{
     \CBox{PineGreen}{\faEdit}{BIEN R\'EDIGER}{#1}
}

\newcommand\CError[1]{
     \CBox{Red}{\faExclamationTriangle}{ATTENTION}{#1}
}

\newcommand\TitreExo[2]{
\needspace{4\baselineskip}
 {\sffamily\large EXERCICE #1\ (\emph{#2 points})}
\vspace{5mm}
}

\newcommand\img[2]{
          \includegraphics[width=#2\paperwidth]{\imgdir#1}
}

\newcommand\imgsvg[2]{
       \begin{center}   \includegraphics[width=#2\paperwidth]{\imgsvgdir#1} \end{center}
}


\newcommand\Lien[2]{
     \href{#1}{#2 \tiny \faExternalLink}
}
\newcommand\mcLien[2]{
     \href{https~://www.maths-cours.fr/#1}{#2 \tiny \faExternalLink}
}

\newcommand{\euro}{\eurologo{}}

%================================================================================================================================
%
% Macros - Environement
%
%================================================================================================================================

\newenvironment{tex}{ %
}
{%
}

\newenvironment{indente}{ %
	\setlength\parindent{10mm}
}

{
	\setlength\parindent{0mm}
}

\newenvironment{corrige}{%
     \needspace{3\baselineskip}
     \medskip
     \textbf{\textsc{Corrigé}}
     \medskip
}
{
}

\newenvironment{extern}{%
     \begin{center}
     }
     {
     \end{center}
}

\NewEnviron{code}{%
	\par
     \boite{gray}{\texttt{%
     \BODY
     }}
     \par
}

\newenvironment{vbloc}{% boite sans cadre empeche saut de page
     \begin{minipage}[t]{\linewidth}
     }
     {
     \end{minipage}
}
\NewEnviron{h2}{%
    \needspace{3\baselineskip}
    \vspace{0.6cm}
	\noindent \MakeUppercase{\sffamily \large \BODY}
	\vspace{1mm}\textcolor{mcgris}{\hrule}\vspace{0.4cm}
	\par
}{}

\NewEnviron{h3}{%
    \needspace{3\baselineskip}
	\vspace{5mm}
	\textsc{\BODY}
	\par
}

\NewEnviron{margeneg}{ %
\begin{addmargin}[-1cm]{0cm}
\BODY
\end{addmargin}
}

\NewEnviron{html}{%
}

\begin{document}
\meta{url}{/exercices/probabilites-controle-continu-1ere-2020-sujet-zero/}
\meta{pid}{11205}
\meta{titre}{Probabilités - Contrôle continu 1ère - 2020 - Sujet zéro}
\meta{type}{exercices}
%
\begin{h2}Exercice 3 (5 points)\end{h2}
Une compagnie d'assurance auto propose deux types de contrat ~:
\begin{itemize}[label=---]
     \item
     Un contrat \og  Tous risques \fg{} dont le montant annuel est de 500 €~;
     \item
     Un contrat \og  De base \fg{} dont le montant annuel est de 400 €.
\end{itemize}
En consultant le fichier clients de la compagnie, on recueille les données suivantes~:
\begin{itemize}[label=---]
     \item
     60 \% des clients possèdent un véhicule récent ( moins de 5 ans ). Les autres clients ont un véhicule ancien~;
     \item
     parmi les clients possédant un véhicule récent, 70 \% ont souscrit au contrat \og  Tous risques \fg{}~;
     \item
     parmi les clients possédant un véhicule ancien, 50 \% ont souscrit au contrat \og  Tous risques \fg{}.
\end{itemize}
On considère un client choisi au hasard. \\
D'une manière générale, la probabilité d'un événement $A$ est notée $P( A )$ et son événement contraire est noté $\overline{A}.$
\newpar
On note les événements suivants :
\begin{itemize}
     \item
     $R$ : \og  Le client possède un véhicule récent \fg{}~;
     \item
     $T$ : \og  Le client a souscrit au contrat Tous risques \fg{}.
\end{itemize}
On note $X$ la variable aléatoire qui donne le montant du contrat souscrit par un client.
\begin{enumerate}
     \item
     Recopier et compléter l'arbre pondéré de probabilité traduisant les données de l'exercice.
     \begin{center}
          % Racine à Gauche, développement vers la droite
          \begin{extern}%width="360" alt="Arbre probabilites controle continu 1ere 2020 sujet-zero"
               \begin{tikzpicture}[xscale=1,yscale=1]
                    % Styles (MODIFIABLES)
                    \tikzstyle{fleche}=[-,>=latex,thick]
                    \tikzstyle{noeud}=[fill=white,circle,draw]
                    \tikzstyle{feuille}=[fill=white,circle,draw]
                    \tikzstyle{etiquette}=[midway,fill=white]
                    % Dimensions (MODIFIABLES)
                    \def\DistanceInterNiveaux{3}
                    \def\DistanceInterFeuilles{2}
                    % Dimensions calculées (NON MODIFIABLES)
                    \def\NiveauA{(0)*\DistanceInterNiveaux}
                    \def\NiveauB{(1.5)*\DistanceInterNiveaux}
                    \def\NiveauC{(2.5)*\DistanceInterNiveaux}
                    \def\InterFeuilles{(-1)*\DistanceInterFeuilles}
                    % Noeuds (MODIFIABLES : Styles et Coefficients d'InterFeuilles)
                    \node[noeud] (R) at ({\NiveauA},{(1.5)*\InterFeuilles}) {$\ $};
                    \node[noeud] (Ra) at ({\NiveauB},{(0.5)*\InterFeuilles}) {$R$};
                    \node[feuille] (Raa) at ({\NiveauC},{(0)*\InterFeuilles}) {$T$};
                    \node[feuille] (Rab) at ({\NiveauC},{(1)*\InterFeuilles}) {$\overline{T}$};
                    \node[noeud] (Rb) at ({\NiveauB},{(2.5)*\InterFeuilles}) {$\overline{R}$};
                    \node[feuille] (Rba) at ({\NiveauC},{(2)*\InterFeuilles}) {$T$};
                    \node[feuille] (Rbb) at ({\NiveauC},{(3)*\InterFeuilles}) {$\overline{T}$};
                    % Arcs (MODIFIABLES : Styles)
                    \draw[fleche] (R)--(Ra) node[etiquette] {$0,6$};
                    \draw[fleche] (Ra)--(Raa) ;
                    \draw[fleche] (Ra)--(Rab);
                    \draw[fleche] (R)--(Rb) ;
                    \draw[fleche] (Rb)--(Rba) ;
                    \draw[fleche] (Rb)--(Rbb) ;
               \end{tikzpicture}
          \end{extern}
     \end{center}
     \item
     Calculer la probabilité qu'un client pris au hasard possède un véhicule récent et ait souscrit au contrat \og  Tous risques \fg{}, c'est-à-dire calculer $P( R \cap T ).$
     \item
     Montrer que $P( T )=0,62.$
     \item
     La variable aléatoire $X$ ne prend que deux valeurs $a$ et $b$ . Déterminer ces deux valeurs, les probabilités $P( X=a )$ et $P( X=b )$ , puis l'espérance de $X$.
\end{enumerate}
\begin{corrige}
     \begin{enumerate}
          \item
          À partir des données de l'énoncé, on peut compléter l'arbre pondéré de la manière suivante~:
          \begin{center}
               % Racine à Gauche, développement vers la droite
               \begin{extern}%width="360" alt="Arbre probabilites controle continu 1ere 2020 sujet-zero"
                    \begin{tikzpicture}[xscale=1,yscale=1]
                         % Styles (MODIFIABLES)
                         \tikzstyle{fleche}=[-,>=latex,thick]
                         \tikzstyle{noeud}=[fill=white,circle,draw]
                         \tikzstyle{feuille}=[fill=white,circle,draw]
                         \tikzstyle{etiquette}=[midway,fill=white]
                         % Dimensions (MODIFIABLES)
                         \def\DistanceInterNiveaux{3}
                         \def\DistanceInterFeuilles{2}
                         % Dimensions calculées (NON MODIFIABLES)
                         \def\NiveauA{(0)*\DistanceInterNiveaux}
                         \def\NiveauB{(1.5)*\DistanceInterNiveaux}
                         \def\NiveauC{(2.5)*\DistanceInterNiveaux}
                         \def\InterFeuilles{(-1)*\DistanceInterFeuilles}
                         % Noeuds (MODIFIABLES : Styles et Coefficients d'InterFeuilles)
                         \node[noeud] (R) at ({\NiveauA},{(1.5)*\InterFeuilles}) {$\ $};
                         \node[noeud] (Ra) at ({\NiveauB},{(0.5)*\InterFeuilles}) {$R$};
                         \node[feuille] (Raa) at ({\NiveauC},{(0)*\InterFeuilles}) {$T$};
                         \node[feuille] (Rab) at ({\NiveauC},{(1)*\InterFeuilles}) {$\overline{T}$};
                         \node[noeud] (Rb) at ({\NiveauB},{(2.5)*\InterFeuilles}) {$\overline{R}$};
                         \node[feuille] (Rba) at ({\NiveauC},{(2)*\InterFeuilles}) {$T$};
                         \node[feuille] (Rbb) at ({\NiveauC},{(3)*\InterFeuilles}) {$\overline{T}$};
                         % Arcs (MODIFIABLES : Styles)
                         \draw[fleche] (R)--(Ra) node[etiquette] {$0,6$};
                         \draw[fleche] (Ra)--(Raa) node[etiquette] {$0,7$};
                         \draw[fleche] (Ra)--(Rab) node[etiquette] {$0,3$};
                         \draw[fleche] (R)--(Rb) node[etiquette] {$0,4$};
                         \draw[fleche] (Rb)--(Rba) node[etiquette] {$0,5$};
                         \draw[fleche] (Rb)--(Rbb) node[etiquette] {$0,5$};
                    \end{tikzpicture}
               \end{extern}
          \end{center}
          \item
          D'après la formule des probabilités conditionnelles, la probabilité qu'un client pris au hasard possède un véhicule récent et ait souscrit au contrat \og  Tous risques \fg{} est~:
          \newpar
          $P( R  \cap T ) =P( R ) \times P_{ R }( T )$ \\
          $\phantom{P( R  \cap T ) }=0,6 \times 0,7=0,42$
          \item
          D'après la formule des probabilités totales, la probabilité que le client ait souscrit un contrat \og  Tous risques \fg{} est égale à~:
          \newpar
          $P( T )=P( R ) \times P_{ R }( T )+P(  \overline{ R } ) \times P_{  \overline{ R } }$\\
          $\phantom{P( T )}=0,6 \times 0,7+0,4 \times 0,5=0,62$
          \item
          La variable aléatoire $X$ peut prendre 2 valeurs~:
          \begin{itemize}
               \item
               $X=500$ si le client a choisi le contrat \og  Tous risques \fg{}~;
               \item
               $X=400$ si et seulement si le client n'a pas choisi ce contrat.
          \end{itemize}
          \newpar
          D'après la question précédente~:
          \\
          $P( X=500 )=P( T )=0,62$
          \newpar
          Et~:\\
          $P( X=400 )=P(  \overline{ T } )=1 - 0,62=0,38.$
          \newpar
          Enfin, l'espérance mathématique de  $X$ est~:\\
          $E( X )=500 \times 0,62+400 \times 0,38=462.$
          \newpar
          Ce résultat peut s'interpréter de la façon suivante~: La compagnie d'assurance touchera, en moyenne, 462 € par contrat souscrit.
     \end{enumerate}
\end{corrige}

\end{document}
µ
\documentclass[a4paper]{article}

%================================================================================================================================
%
% Packages
%
%================================================================================================================================

\usepackage[T1]{fontenc} 	% pour caractères accentués
\usepackage[utf8]{inputenc}  % encodage utf8
\usepackage[french]{babel}	% langue : français
\usepackage{fourier}			% caractères plus lisibles
\usepackage[dvipsnames]{xcolor} % couleurs
\usepackage{fancyhdr}		% réglage header footer
\usepackage{needspace}		% empêcher sauts de page mal placés
\usepackage{graphicx}		% pour inclure des graphiques
\usepackage{enumitem,cprotect}		% personnalise les listes d'items (nécessaire pour ol, al ...)
\usepackage{hyperref}		% Liens hypertexte
\usepackage{pstricks,pst-all,pst-node,pstricks-add,pst-math,pst-plot,pst-tree,pst-eucl} % pstricks
\usepackage[a4paper,includeheadfoot,top=2cm,left=3cm, bottom=2cm,right=3cm]{geometry} % marges etc.
\usepackage{comment}			% commentaires multilignes
\usepackage{amsmath,environ} % maths (matrices, etc.)
\usepackage{amssymb,makeidx}
\usepackage{bm}				% bold maths
\usepackage{tabularx}		% tableaux
\usepackage{colortbl}		% tableaux en couleur
\usepackage{fontawesome}		% Fontawesome
\usepackage{environ}			% environment with command
\usepackage{fp}				% calculs pour ps-tricks
\usepackage{multido}			% pour ps tricks
\usepackage[np]{numprint}	% formattage nombre
\usepackage{tikz,tkz-tab} 			% package principal TikZ
\usepackage{pgfplots}   % axes
\usepackage{mathrsfs}    % cursives
\usepackage{calc}			% calcul taille boites
\usepackage[scaled=0.875]{helvet} % font sans serif
\usepackage{svg} % svg
\usepackage{scrextend} % local margin
\usepackage{scratch} %scratch
\usepackage{multicol} % colonnes
%\usepackage{infix-RPN,pst-func} % formule en notation polanaise inversée
\usepackage{listings}

%================================================================================================================================
%
% Réglages de base
%
%================================================================================================================================

\lstset{
language=Python,   % R code
literate=
{á}{{\'a}}1
{à}{{\`a}}1
{ã}{{\~a}}1
{é}{{\'e}}1
{è}{{\`e}}1
{ê}{{\^e}}1
{í}{{\'i}}1
{ó}{{\'o}}1
{õ}{{\~o}}1
{ú}{{\'u}}1
{ü}{{\"u}}1
{ç}{{\c{c}}}1
{~}{{ }}1
}


\definecolor{codegreen}{rgb}{0,0.6,0}
\definecolor{codegray}{rgb}{0.5,0.5,0.5}
\definecolor{codepurple}{rgb}{0.58,0,0.82}
\definecolor{backcolour}{rgb}{0.95,0.95,0.92}

\lstdefinestyle{mystyle}{
    backgroundcolor=\color{backcolour},   
    commentstyle=\color{codegreen},
    keywordstyle=\color{magenta},
    numberstyle=\tiny\color{codegray},
    stringstyle=\color{codepurple},
    basicstyle=\ttfamily\footnotesize,
    breakatwhitespace=false,         
    breaklines=true,                 
    captionpos=b,                    
    keepspaces=true,                 
    numbers=left,                    
xleftmargin=2em,
framexleftmargin=2em,            
    showspaces=false,                
    showstringspaces=false,
    showtabs=false,                  
    tabsize=2,
    upquote=true
}

\lstset{style=mystyle}


\lstset{style=mystyle}
\newcommand{\imgdir}{C:/laragon/www/newmc/assets/imgsvg/}
\newcommand{\imgsvgdir}{C:/laragon/www/newmc/assets/imgsvg/}

\definecolor{mcgris}{RGB}{220, 220, 220}% ancien~; pour compatibilité
\definecolor{mcbleu}{RGB}{52, 152, 219}
\definecolor{mcvert}{RGB}{125, 194, 70}
\definecolor{mcmauve}{RGB}{154, 0, 215}
\definecolor{mcorange}{RGB}{255, 96, 0}
\definecolor{mcturquoise}{RGB}{0, 153, 153}
\definecolor{mcrouge}{RGB}{255, 0, 0}
\definecolor{mclightvert}{RGB}{205, 234, 190}

\definecolor{gris}{RGB}{220, 220, 220}
\definecolor{bleu}{RGB}{52, 152, 219}
\definecolor{vert}{RGB}{125, 194, 70}
\definecolor{mauve}{RGB}{154, 0, 215}
\definecolor{orange}{RGB}{255, 96, 0}
\definecolor{turquoise}{RGB}{0, 153, 153}
\definecolor{rouge}{RGB}{255, 0, 0}
\definecolor{lightvert}{RGB}{205, 234, 190}
\setitemize[0]{label=\color{lightvert}  $\bullet$}

\pagestyle{fancy}
\renewcommand{\headrulewidth}{0.2pt}
\fancyhead[L]{maths-cours.fr}
\fancyhead[R]{\thepage}
\renewcommand{\footrulewidth}{0.2pt}
\fancyfoot[C]{}

\newcolumntype{C}{>{\centering\arraybackslash}X}
\newcolumntype{s}{>{\hsize=.35\hsize\arraybackslash}X}

\setlength{\parindent}{0pt}		 
\setlength{\parskip}{3mm}
\setlength{\headheight}{1cm}

\def\ebook{ebook}
\def\book{book}
\def\web{web}
\def\type{web}

\newcommand{\vect}[1]{\overrightarrow{\,\mathstrut#1\,}}

\def\Oij{$\left(\text{O}~;~\vect{\imath},~\vect{\jmath}\right)$}
\def\Oijk{$\left(\text{O}~;~\vect{\imath},~\vect{\jmath},~\vect{k}\right)$}
\def\Ouv{$\left(\text{O}~;~\vect{u},~\vect{v}\right)$}

\hypersetup{breaklinks=true, colorlinks = true, linkcolor = OliveGreen, urlcolor = OliveGreen, citecolor = OliveGreen, pdfauthor={Didier BONNEL - https://www.maths-cours.fr} } % supprime les bordures autour des liens

\renewcommand{\arg}[0]{\text{arg}}

\everymath{\displaystyle}

%================================================================================================================================
%
% Macros - Commandes
%
%================================================================================================================================

\newcommand\meta[2]{    			% Utilisé pour créer le post HTML.
	\def\titre{titre}
	\def\url{url}
	\def\arg{#1}
	\ifx\titre\arg
		\newcommand\maintitle{#2}
		\fancyhead[L]{#2}
		{\Large\sffamily \MakeUppercase{#2}}
		\vspace{1mm}\textcolor{mcvert}{\hrule}
	\fi 
	\ifx\url\arg
		\fancyfoot[L]{\href{https://www.maths-cours.fr#2}{\black \footnotesize{https://www.maths-cours.fr#2}}}
	\fi 
}


\newcommand\TitreC[1]{    		% Titre centré
     \needspace{3\baselineskip}
     \begin{center}\textbf{#1}\end{center}
}

\newcommand\newpar{    		% paragraphe
     \par
}

\newcommand\nosp {    		% commande vide (pas d'espace)
}
\newcommand{\id}[1]{} %ignore

\newcommand\boite[2]{				% Boite simple sans titre
	\vspace{5mm}
	\setlength{\fboxrule}{0.2mm}
	\setlength{\fboxsep}{5mm}	
	\fcolorbox{#1}{#1!3}{\makebox[\linewidth-2\fboxrule-2\fboxsep]{
  		\begin{minipage}[t]{\linewidth-2\fboxrule-4\fboxsep}\setlength{\parskip}{3mm}
  			 #2
  		\end{minipage}
	}}
	\vspace{5mm}
}

\newcommand\CBox[4]{				% Boites
	\vspace{5mm}
	\setlength{\fboxrule}{0.2mm}
	\setlength{\fboxsep}{5mm}
	
	\fcolorbox{#1}{#1!3}{\makebox[\linewidth-2\fboxrule-2\fboxsep]{
		\begin{minipage}[t]{1cm}\setlength{\parskip}{3mm}
	  		\textcolor{#1}{\LARGE{#2}}    
 	 	\end{minipage}  
  		\begin{minipage}[t]{\linewidth-2\fboxrule-4\fboxsep}\setlength{\parskip}{3mm}
			\raisebox{1.2mm}{\normalsize\sffamily{\textcolor{#1}{#3}}}						
  			 #4
  		\end{minipage}
	}}
	\vspace{5mm}
}

\newcommand\cadre[3]{				% Boites convertible html
	\par
	\vspace{2mm}
	\setlength{\fboxrule}{0.1mm}
	\setlength{\fboxsep}{5mm}
	\fcolorbox{#1}{white}{\makebox[\linewidth-2\fboxrule-2\fboxsep]{
  		\begin{minipage}[t]{\linewidth-2\fboxrule-4\fboxsep}\setlength{\parskip}{3mm}
			\raisebox{-2.5mm}{\sffamily \small{\textcolor{#1}{\MakeUppercase{#2}}}}		
			\par		
  			 #3
 	 		\end{minipage}
	}}
		\vspace{2mm}
	\par
}

\newcommand\bloc[3]{				% Boites convertible html sans bordure
     \needspace{2\baselineskip}
     {\sffamily \small{\textcolor{#1}{\MakeUppercase{#2}}}}    
		\par		
  			 #3
		\par
}

\newcommand\CHelp[1]{
     \CBox{Plum}{\faInfoCircle}{À RETENIR}{#1}
}

\newcommand\CUp[1]{
     \CBox{NavyBlue}{\faThumbsOUp}{EN PRATIQUE}{#1}
}

\newcommand\CInfo[1]{
     \CBox{Sepia}{\faArrowCircleRight}{REMARQUE}{#1}
}

\newcommand\CRedac[1]{
     \CBox{PineGreen}{\faEdit}{BIEN R\'EDIGER}{#1}
}

\newcommand\CError[1]{
     \CBox{Red}{\faExclamationTriangle}{ATTENTION}{#1}
}

\newcommand\TitreExo[2]{
\needspace{4\baselineskip}
 {\sffamily\large EXERCICE #1\ (\emph{#2 points})}
\vspace{5mm}
}

\newcommand\img[2]{
          \includegraphics[width=#2\paperwidth]{\imgdir#1}
}

\newcommand\imgsvg[2]{
       \begin{center}   \includegraphics[width=#2\paperwidth]{\imgsvgdir#1} \end{center}
}


\newcommand\Lien[2]{
     \href{#1}{#2 \tiny \faExternalLink}
}
\newcommand\mcLien[2]{
     \href{https~://www.maths-cours.fr/#1}{#2 \tiny \faExternalLink}
}

\newcommand{\euro}{\eurologo{}}

%================================================================================================================================
%
% Macros - Environement
%
%================================================================================================================================

\newenvironment{tex}{ %
}
{%
}

\newenvironment{indente}{ %
	\setlength\parindent{10mm}
}

{
	\setlength\parindent{0mm}
}

\newenvironment{corrige}{%
     \needspace{3\baselineskip}
     \medskip
     \textbf{\textsc{Corrigé}}
     \medskip
}
{
}

\newenvironment{extern}{%
     \begin{center}
     }
     {
     \end{center}
}

\NewEnviron{code}{%
	\par
     \boite{gray}{\texttt{%
     \BODY
     }}
     \par
}

\newenvironment{vbloc}{% boite sans cadre empeche saut de page
     \begin{minipage}[t]{\linewidth}
     }
     {
     \end{minipage}
}
\NewEnviron{h2}{%
    \needspace{3\baselineskip}
    \vspace{0.6cm}
	\noindent \MakeUppercase{\sffamily \large \BODY}
	\vspace{1mm}\textcolor{mcgris}{\hrule}\vspace{0.4cm}
	\par
}{}

\NewEnviron{h3}{%
    \needspace{3\baselineskip}
	\vspace{5mm}
	\textsc{\BODY}
	\par
}

\NewEnviron{margeneg}{ %
\begin{addmargin}[-1cm]{0cm}
\BODY
\end{addmargin}
}

\NewEnviron{html}{%
}

\begin{document}
\meta{url}{/exercices/suites-controle-continu-1ere-2020-sujet-zero/}
\meta{pid}{11213}
\meta{titre}{Suites - Contrôle continu 1ère - 2020 - Sujet zéro}
\meta{type}{exercices}
%
\begin{h2}Exercice 4 (5 points)\end{h2}
En traversant une plaque de verre teintée, un rayon lumineux perd 20  \% de son intensité lumineuse. L'intensité lumineuse est exprimée en candela ( cd ).
\newpar
On utilise une lampe torche qui émet un rayon d'intensité lumineuse réglée à 400 cd.
\newpar
On superpose $n$ plaques de verre identiques (  $n$ étant un entier naturel ) et on désire mesurer l'intensité lumineuse $I_{ n }$ du rayon à la sortie de la $n$-ième plaque.
\newpar
On note $I_{ 0 }=400$ l'intensité lumineuse du rayon émis par la lampe torche avant de traverser les plaques ( intensité lumineuse initiale ). Ainsi, cette situation est modélisée par la suite $( I_{ n } ).$
\begin{enumerate}
     \item
     Montrer par un calcul que $I_{ 1 }=320.$
     \item
     \begin{enumerate}[label=\alph*.]
          \item
          Pour tout entier naturel $n$ , exprimer $I_{ n+1 }$ en fonction de $I_{ n }.$
          \item
          En déduire la nature de la suite $( I_{ n } ).$ Préciser sa raison et son premier terme.
          \item
          Pour tout entier naturel $n$ , exprimer $I_{ n }$ en fonction de $n. $
     \end{enumerate}
     \item
     On souhaite déterminer le nombre minimal $n$ de plaques à superposer afin que le rayon initial ait perdu au moins 70 \% de son intensité lumineuse initiale après sa traversée des plaques.
     \begin{enumerate}[label=\alph*.]
          \item
          Afin de déterminer le nombre de plaques à superposer, on considère la fonction Python suivante~:
\begin{lstlisting}[language=Python]
def nombrePlaques(J):
	I=400
	n=0
	while I > J:
		I = 0.8*I
		n = n+1
	return n
     \end{lstlisting}
     Préciser, en justifiant, le nombre \texttt{J} de sorte que l'appel \texttt{nombrePlaques(J)} renvoie le nombre de plaques à superposer.
     \item
     Le tableau suivant donne des valeurs de $I_{ n }.$ Combien de plaques doit-on superposer ?
     \begin{center}
          \begin{tabular}{|c|c|c|c|c|c|c|c|c|}%class="compact" width="600"
               \hline
               $n$ &  0 & 1 & 2 & 3 & 4 & 5 & 6 & 7 \\
               \hline
               $I_{ n }$  & 400 & 320 & 256 & 204,8 & 163,84 & 131,07 & 104,85 & 83,886 \\
               \hline
          \end{tabular}
     \end{center}
\end{enumerate}
\end{enumerate}
\begin{corrige}
     \begin{enumerate}
          \item
          Le coefficient multiplicateur correspondant à une baisse de 20 \% est~:
          \newpar
          $CM=1 - \frac{ 20 }{ 100 }=0,8$
          \newpar
          L'intensité lumineuse $I_{ 1 }$ à la sortie de la première plaque est donc~:
          \newpar
          $I_{ 1 }=0,8 \times I_{ 0 }=0,8 \times 400=320$
          \medbreak
          \textbf{Remarque~: }On aurait également pu calculer la diminution de l'intensité lumineuse qui est égale à $\frac{ 20 }{ 100 } \times 400=80$, puis la nouvelle intensité $I_{ 1 }=400 - 80=320$ . Mais il est préférable de s'habituer à utiliser le coefficient multiplicateur qui facilite les calculs lors d'augmentation ou de diminution en pourcentage.
          \item
          \begin{enumerate}[label=\alph*.]
               \item
               De même, le raisonnement précédent indique que, pour tout entier naturel $n$ ~: \\
               $I_{ n+1 }=0,8 \times I_{ n }.$
               \item
               La formule précédente prouve que la suite $( I_{ n } )$ est une suite géométrique de raison  $q=0,8$~; son premier terme est $I_{ 0 }=400.$
               \item
               D'après le cours, le n-ième terme d'une suite géométrique de premier terme $u_{ 0 }$ et de raison $q$ est donné par la \mcLien{https://www.maths-cours.fr/cours/suites-geometriques/\#p110}{formule}~:
               \newpar
               \[
               u_{ n }=u_{ 0 } \times q{}^{ n }
               \]
               On obtient ici~:
               \newpar
               $I_{ n }=I_{ 0 } \times q{}^{ n }=400 \times 0,8{}^{ n }$
          \end{enumerate}
          \item
          \begin{enumerate}[label=\alph*.]
               \item
               L'appel à la fonction Python \texttt{nombrePlaques( )} avec l'argument  \texttt{J} renvoie le nombre minimal de plaques à superposer afin que l'intensité lumineuse du rayon à la sortie de la n-ième plaque soit inférieure ou égale à  \texttt{J}.
               \newpar
               Puisque l'on veut que le rayon initial perde au moins 70 \% de son intensité lumineuse, il faut que l'intensité lumineuse à la sortie de la n-ième plaque soit inférieure à 30 \% de  $I_{ 0 }$ c'est-à-dire inférieure à $\frac{ 30 }{ 100 } \times 400=120.$
               \newpar
               Il faut donc choisir le nombre  \texttt{J} = 120, pour obtenir, en sortie de la fonction \texttt{nombrePlaques( )}, le nombre de plaques à superposer.
               \item
               Le tableau montre qu'il faut choisir 6 plaques pour obtenir une intensité lumineuse inférieure ou égale à 120 cd.
          \end{enumerate}
     \end{enumerate}
\end{corrige}

\end{document}
µ
\documentclass[a4paper]{article}

%================================================================================================================================
%
% Packages
%
%================================================================================================================================

\usepackage[T1]{fontenc} 	% pour caractères accentués
\usepackage[utf8]{inputenc}  % encodage utf8
\usepackage[french]{babel}	% langue : français
\usepackage{fourier}			% caractères plus lisibles
\usepackage[dvipsnames]{xcolor} % couleurs
\usepackage{fancyhdr}		% réglage header footer
\usepackage{needspace}		% empêcher sauts de page mal placés
\usepackage{graphicx}		% pour inclure des graphiques
\usepackage{enumitem,cprotect}		% personnalise les listes d'items (nécessaire pour ol, al ...)
\usepackage{hyperref}		% Liens hypertexte
\usepackage{pstricks,pst-all,pst-node,pstricks-add,pst-math,pst-plot,pst-tree,pst-eucl} % pstricks
\usepackage[a4paper,includeheadfoot,top=2cm,left=3cm, bottom=2cm,right=3cm]{geometry} % marges etc.
\usepackage{comment}			% commentaires multilignes
\usepackage{amsmath,environ} % maths (matrices, etc.)
\usepackage{amssymb,makeidx}
\usepackage{bm}				% bold maths
\usepackage{tabularx}		% tableaux
\usepackage{colortbl}		% tableaux en couleur
\usepackage{fontawesome}		% Fontawesome
\usepackage{environ}			% environment with command
\usepackage{fp}				% calculs pour ps-tricks
\usepackage{multido}			% pour ps tricks
\usepackage[np]{numprint}	% formattage nombre
\usepackage{tikz,tkz-tab} 			% package principal TikZ
\usepackage{pgfplots}   % axes
\usepackage{mathrsfs}    % cursives
\usepackage{calc}			% calcul taille boites
\usepackage[scaled=0.875]{helvet} % font sans serif
\usepackage{svg} % svg
\usepackage{scrextend} % local margin
\usepackage{scratch} %scratch
\usepackage{multicol} % colonnes
%\usepackage{infix-RPN,pst-func} % formule en notation polanaise inversée
\usepackage{listings}

%================================================================================================================================
%
% Réglages de base
%
%================================================================================================================================

\lstset{
language=Python,   % R code
literate=
{á}{{\'a}}1
{à}{{\`a}}1
{ã}{{\~a}}1
{é}{{\'e}}1
{è}{{\`e}}1
{ê}{{\^e}}1
{í}{{\'i}}1
{ó}{{\'o}}1
{õ}{{\~o}}1
{ú}{{\'u}}1
{ü}{{\"u}}1
{ç}{{\c{c}}}1
{~}{{ }}1
}


\definecolor{codegreen}{rgb}{0,0.6,0}
\definecolor{codegray}{rgb}{0.5,0.5,0.5}
\definecolor{codepurple}{rgb}{0.58,0,0.82}
\definecolor{backcolour}{rgb}{0.95,0.95,0.92}

\lstdefinestyle{mystyle}{
    backgroundcolor=\color{backcolour},   
    commentstyle=\color{codegreen},
    keywordstyle=\color{magenta},
    numberstyle=\tiny\color{codegray},
    stringstyle=\color{codepurple},
    basicstyle=\ttfamily\footnotesize,
    breakatwhitespace=false,         
    breaklines=true,                 
    captionpos=b,                    
    keepspaces=true,                 
    numbers=left,                    
xleftmargin=2em,
framexleftmargin=2em,            
    showspaces=false,                
    showstringspaces=false,
    showtabs=false,                  
    tabsize=2,
    upquote=true
}

\lstset{style=mystyle}


\lstset{style=mystyle}
\newcommand{\imgdir}{C:/laragon/www/newmc/assets/imgsvg/}
\newcommand{\imgsvgdir}{C:/laragon/www/newmc/assets/imgsvg/}

\definecolor{mcgris}{RGB}{220, 220, 220}% ancien~; pour compatibilité
\definecolor{mcbleu}{RGB}{52, 152, 219}
\definecolor{mcvert}{RGB}{125, 194, 70}
\definecolor{mcmauve}{RGB}{154, 0, 215}
\definecolor{mcorange}{RGB}{255, 96, 0}
\definecolor{mcturquoise}{RGB}{0, 153, 153}
\definecolor{mcrouge}{RGB}{255, 0, 0}
\definecolor{mclightvert}{RGB}{205, 234, 190}

\definecolor{gris}{RGB}{220, 220, 220}
\definecolor{bleu}{RGB}{52, 152, 219}
\definecolor{vert}{RGB}{125, 194, 70}
\definecolor{mauve}{RGB}{154, 0, 215}
\definecolor{orange}{RGB}{255, 96, 0}
\definecolor{turquoise}{RGB}{0, 153, 153}
\definecolor{rouge}{RGB}{255, 0, 0}
\definecolor{lightvert}{RGB}{205, 234, 190}
\setitemize[0]{label=\color{lightvert}  $\bullet$}

\pagestyle{fancy}
\renewcommand{\headrulewidth}{0.2pt}
\fancyhead[L]{maths-cours.fr}
\fancyhead[R]{\thepage}
\renewcommand{\footrulewidth}{0.2pt}
\fancyfoot[C]{}

\newcolumntype{C}{>{\centering\arraybackslash}X}
\newcolumntype{s}{>{\hsize=.35\hsize\arraybackslash}X}

\setlength{\parindent}{0pt}		 
\setlength{\parskip}{3mm}
\setlength{\headheight}{1cm}

\def\ebook{ebook}
\def\book{book}
\def\web{web}
\def\type{web}

\newcommand{\vect}[1]{\overrightarrow{\,\mathstrut#1\,}}

\def\Oij{$\left(\text{O}~;~\vect{\imath},~\vect{\jmath}\right)$}
\def\Oijk{$\left(\text{O}~;~\vect{\imath},~\vect{\jmath},~\vect{k}\right)$}
\def\Ouv{$\left(\text{O}~;~\vect{u},~\vect{v}\right)$}

\hypersetup{breaklinks=true, colorlinks = true, linkcolor = OliveGreen, urlcolor = OliveGreen, citecolor = OliveGreen, pdfauthor={Didier BONNEL - https://www.maths-cours.fr} } % supprime les bordures autour des liens

\renewcommand{\arg}[0]{\text{arg}}

\everymath{\displaystyle}

%================================================================================================================================
%
% Macros - Commandes
%
%================================================================================================================================

\newcommand\meta[2]{    			% Utilisé pour créer le post HTML.
	\def\titre{titre}
	\def\url{url}
	\def\arg{#1}
	\ifx\titre\arg
		\newcommand\maintitle{#2}
		\fancyhead[L]{#2}
		{\Large\sffamily \MakeUppercase{#2}}
		\vspace{1mm}\textcolor{mcvert}{\hrule}
	\fi 
	\ifx\url\arg
		\fancyfoot[L]{\href{https://www.maths-cours.fr#2}{\black \footnotesize{https://www.maths-cours.fr#2}}}
	\fi 
}


\newcommand\TitreC[1]{    		% Titre centré
     \needspace{3\baselineskip}
     \begin{center}\textbf{#1}\end{center}
}

\newcommand\newpar{    		% paragraphe
     \par
}

\newcommand\nosp {    		% commande vide (pas d'espace)
}
\newcommand{\id}[1]{} %ignore

\newcommand\boite[2]{				% Boite simple sans titre
	\vspace{5mm}
	\setlength{\fboxrule}{0.2mm}
	\setlength{\fboxsep}{5mm}	
	\fcolorbox{#1}{#1!3}{\makebox[\linewidth-2\fboxrule-2\fboxsep]{
  		\begin{minipage}[t]{\linewidth-2\fboxrule-4\fboxsep}\setlength{\parskip}{3mm}
  			 #2
  		\end{minipage}
	}}
	\vspace{5mm}
}

\newcommand\CBox[4]{				% Boites
	\vspace{5mm}
	\setlength{\fboxrule}{0.2mm}
	\setlength{\fboxsep}{5mm}
	
	\fcolorbox{#1}{#1!3}{\makebox[\linewidth-2\fboxrule-2\fboxsep]{
		\begin{minipage}[t]{1cm}\setlength{\parskip}{3mm}
	  		\textcolor{#1}{\LARGE{#2}}    
 	 	\end{minipage}  
  		\begin{minipage}[t]{\linewidth-2\fboxrule-4\fboxsep}\setlength{\parskip}{3mm}
			\raisebox{1.2mm}{\normalsize\sffamily{\textcolor{#1}{#3}}}						
  			 #4
  		\end{minipage}
	}}
	\vspace{5mm}
}

\newcommand\cadre[3]{				% Boites convertible html
	\par
	\vspace{2mm}
	\setlength{\fboxrule}{0.1mm}
	\setlength{\fboxsep}{5mm}
	\fcolorbox{#1}{white}{\makebox[\linewidth-2\fboxrule-2\fboxsep]{
  		\begin{minipage}[t]{\linewidth-2\fboxrule-4\fboxsep}\setlength{\parskip}{3mm}
			\raisebox{-2.5mm}{\sffamily \small{\textcolor{#1}{\MakeUppercase{#2}}}}		
			\par		
  			 #3
 	 		\end{minipage}
	}}
		\vspace{2mm}
	\par
}

\newcommand\bloc[3]{				% Boites convertible html sans bordure
     \needspace{2\baselineskip}
     {\sffamily \small{\textcolor{#1}{\MakeUppercase{#2}}}}    
		\par		
  			 #3
		\par
}

\newcommand\CHelp[1]{
     \CBox{Plum}{\faInfoCircle}{À RETENIR}{#1}
}

\newcommand\CUp[1]{
     \CBox{NavyBlue}{\faThumbsOUp}{EN PRATIQUE}{#1}
}

\newcommand\CInfo[1]{
     \CBox{Sepia}{\faArrowCircleRight}{REMARQUE}{#1}
}

\newcommand\CRedac[1]{
     \CBox{PineGreen}{\faEdit}{BIEN R\'EDIGER}{#1}
}

\newcommand\CError[1]{
     \CBox{Red}{\faExclamationTriangle}{ATTENTION}{#1}
}

\newcommand\TitreExo[2]{
\needspace{4\baselineskip}
 {\sffamily\large EXERCICE #1\ (\emph{#2 points})}
\vspace{5mm}
}

\newcommand\img[2]{
          \includegraphics[width=#2\paperwidth]{\imgdir#1}
}

\newcommand\imgsvg[2]{
       \begin{center}   \includegraphics[width=#2\paperwidth]{\imgsvgdir#1} \end{center}
}


\newcommand\Lien[2]{
     \href{#1}{#2 \tiny \faExternalLink}
}
\newcommand\mcLien[2]{
     \href{https~://www.maths-cours.fr/#1}{#2 \tiny \faExternalLink}
}

\newcommand{\euro}{\eurologo{}}

%================================================================================================================================
%
% Macros - Environement
%
%================================================================================================================================

\newenvironment{tex}{ %
}
{%
}

\newenvironment{indente}{ %
	\setlength\parindent{10mm}
}

{
	\setlength\parindent{0mm}
}

\newenvironment{corrige}{%
     \needspace{3\baselineskip}
     \medskip
     \textbf{\textsc{Corrigé}}
     \medskip
}
{
}

\newenvironment{extern}{%
     \begin{center}
     }
     {
     \end{center}
}

\NewEnviron{code}{%
	\par
     \boite{gray}{\texttt{%
     \BODY
     }}
     \par
}

\newenvironment{vbloc}{% boite sans cadre empeche saut de page
     \begin{minipage}[t]{\linewidth}
     }
     {
     \end{minipage}
}
\NewEnviron{h2}{%
    \needspace{3\baselineskip}
    \vspace{0.6cm}
	\noindent \MakeUppercase{\sffamily \large \BODY}
	\vspace{1mm}\textcolor{mcgris}{\hrule}\vspace{0.4cm}
	\par
}{}

\NewEnviron{h3}{%
    \needspace{3\baselineskip}
	\vspace{5mm}
	\textsc{\BODY}
	\par
}

\NewEnviron{margeneg}{ %
\begin{addmargin}[-1cm]{0cm}
\BODY
\end{addmargin}
}

\NewEnviron{html}{%
}

\begin{document}
\meta{url}{/cours/python-au-lycee-2/}
\meta{pid}{11294}
\meta{titre}{Python au lycée (2)~: Les instructions conditionnelles}
\meta{type}{cours}
\begin{h2}1. Type booléen et conditions\end{h2}
\begin{h3} Type booléen\end{h3}
Dans le chapitre précédent, nous avons vu trois types de variables~: entiers, décimaux, chaînes de caractères. Nous allons maintenant étudier un quatrième type de variable~: le type \textbf{booléen}.
\newpar
Les variables de type booléen ne peuvent prendre que l'une ou l'autre des 2 valeurs suivantes~: \texttt{True} (vrai) ou \texttt{False} (faux).
\cadre{bleu}{attention}{ % id=a010
     Les valeurs \texttt{True} et \texttt{False} doivent être écrites avec une \textbf{initiale en majuscule} pour être reconnues par Python. Par ailleurs, il ne faut pas écrire ces valeurs entre apostrophes car elles seraient alors interprétées comme des chaînes de caractères et non comme des booléens.
}% fin attention
Par exemple~:
\begin{lstlisting}[language=Python]
>>> type(True) # affiche <class 'bool'>
>>> type(true) # affiche une erreur
>>> type("True") # affiche <class 'str'>
\end{lstlisting}
\textbf{Rappel~:}\\
Les chevrons >>> indiquent que les commandes ont été saisies en mode interactif - sous IDLE par exemple.\\
Les \# indiquent le début d'un commentaire.
\begin{h3}Conditions\end{h3}
Le type booléen est généralement utilisé pour indiquer le résultat d'une condition, par exemple pour comparer deux valeurs~:
\begin{lstlisting}[language=Python]
>>> a = 3 # la valeur 3 est affectée à la variable a
>>> a > 5 # affiche False (car 3 n'est pas supérieur à 5)
>>> a < 5 # affiche True (car 3 est inférieur à 5)
\end{lstlisting}
Les principaux opérateurs de comparaison que l'on peut utiliser en Python sont les suivants~:
\begin{center}
     \begin{tabularx}{0.9\linewidth}{|*{2}{>{\centering \arraybackslash }X|}}%class="compact"
          \hline
          \textbf{Op.} & \textbf{Description}
          \\ \hline
          == & égal à
          \\ \hline~!= & différent de
          \\ \hline
          $<$ & strictement inférieur à
          \\ \hline
          $>$ & strictement supérieur à
          \\ \hline
          $<$= & inférieur ou égal à
          \\ \hline
          $>$= & supérieur ou égal à
          \\ \hline
     \end{tabularx}
\end{center}
\cadre{bleu}{Attention}{ % id=a020
     Pour tester l'égalité de deux valeurs, il faut utiliser l'opérateur == (double signe égal) et non =. En effet, le symbole = simple est l'opérateur d'affectation~; son utilisation à l'intérieur d'une condition provoquera une erreur de Python.
}% fin Attention
Ces opérateurs permettent de comparer deux variables de type numérique (entier décimal) mais également deux variables de type \og chaîne de caractères\fg{}. Dans ce cas, les variables seront classées par ordre alphabétique.
\newpar
Par exemple~:
 \begin{lstlisting}[language=Python]
>>> 1 <= 3 # affiche True
>>> 1~!= 3 # affiche True
>>> 1 == 3 # affiche False
>>> 1 = 1 # affiche une erreur (il faut utiliser ==)
>>> "a" < "b" # affiche True
\end{lstlisting}
Bien sûr, il est généralement plus intéressant d'utiliser ces opérateurs avec des variables~:
 \begin{lstlisting}[language=Python]
>>> x = 1 # affecte la valeur 1 à la variable x
>>> y = 6 # affecte la valeur 6 à la variable y
>>> x~!= y # affiche True
>>> x == y # affiche False
>>> x <= y # affiche True
>>> x+5 == y # affiche True
\end{lstlisting}
\begin{h3} Conditions composées\end{h3}
Il est possible de modifier ou de relier différentes conditions à l'aide des opérateurs suivants~: \texttt{not}, \texttt{or}, \texttt{and}.
\begin{itemize}
     \item
     \textbf{\texttt{not}}~: inverse le résultat de la condition~; \texttt{not} \textit{condition} est vraie si et seulement si \textit{condition} est fausse.
     \item
     \textbf{\texttt{or}}~: \textit{condition1} \texttt{or} \textit{condition2} est vraie si et seulement si \textit{condition1} est vraie ou si \textit{condition2} est vraie ou si les 2 conditions sont vraies simultanément.
     \item
     \textbf{\texttt{and}}~: \textit{condition1} \texttt{and} \textit{condition2} est vraie si et seulement si les 2 conditions sont vraies simultanément.
\end{itemize}
\newpar
Par exemple~:
\begin{lstlisting}[language=Python]
>>> x=1 # affecte la valeur 1 à la variable x
>>> y=2 # affecte la valeur 2 à la variable y
>>> not x==1 # affiche False
>>> not y==1 # affiche True
>>> x==1 or y==1 # affiche True (car la première condition est vérifiée)
>>> x==2 or y==3 # affiche False
>>> x==1 and y==1 # affiche False (car la seconde condition n'est pas vérifiée)
>>> x==1 and y==2 # affiche True
\end{lstlisting}
\begin{h3}Remarque\end{h3}
Contrairement à d'autres langages, en Python, il n'est pas nécessaire de placer les conditions entre parenthèses.
\begin{h2}2. Instructions conditionnelles\end{h2}
\begin{h3} Test \og if \fg{}\end{h3}
En Python, un test est introduit par le mot-clé \og \texttt{if} \fg{} (si) suivie d'une condition qui peut être vraie ou fausse. Le bloc d'instructions suivant ( appelé bloc d'\textit{instructions conditionnelles} ) sera exécuté si et seulement si la condition est vraie~:
\begin{lstlisting}[language=Python]
if condition :
   # les instructions placées ici
   # seront exécutées si et seulement si 
   # la condition est vraie 
# tandis que les instructions placées ici
# seront exécutées dans tous les cas.
\end{lstlisting}
\begin{h3}Remarque\end{h3}
Deux choses sont importantes à noter~:
\begin{itemize}
     \item
     La condition doit toujours être suivie du caractère \og~: \fg{} qui introduit le bloc d'instructions conditionnelles.
     \item
     c'est l'\textbf{indentation} ( décalage des instructions vers la droite ) qui définit le début et la fin du bloc d'instructions conditionnelles
\end{itemize}
\begin{h3}Exemple \end{h3}
\begin{lstlisting}[language=Python]
nom = input("Saisir votre nom : ")
if nom=="Henri" :
    print("Bonjour Henri")
    print("Comment allez-vous~?")
print("Bienvenue sur maths-cours.fr~!")
\end{lstlisting}
Le programme ci-dessus demandera un nom d'utilisateur.
\newpar
Puis il affichera~:\\
\texttt{    Bonjour Henri\\
         Comment allez-vous~?\\
    Bienvenue sur maths-cours.fr~!\\}
si le nom entré est Henri~;
\newpar
tandis qu'il affichera uniquement~:\\
\texttt{    Bienvenue sur maths-cours.fr~!\\}
si un autre nom est saisi.
\newpar
(pour les instructions \texttt{input} et \texttt{print} voir le \mcLien{/cours/python-au-lycee-1/}{chapitre précédent}).
\begin{h3} Test \og if - else \fg{} \end{h3}
Il est possible de faire suivre une condition \og if \fg{} par une condition \og else \fg{} (sinon)~; le bloc d'instructions suivantes sera alors exécuté si et seulement si la condition suivant le \og if \fg{} est fausse~:
\begin{lstlisting}[language=Python]
if condition :
   # les instructions placées ici
   # seront exécutées si et seulement si 
   # la condition est vraie 
else :
   # les instructions placées ici
   # seront exécutées si et seulement si 
   # la condition est fausse
# enfin les instructions placées ici
# seront exécutées dans tous les cas.
\end{lstlisting}
\begin{h3}Remarque\end{h3}
Notez, là aussi, la présence du symbole \og~: \fg{} après le mot \texttt{else} ainsi que l'\textbf{indentation}.
\begin{h3}Exemple\end{h3}
\begin{lstlisting}[language=Python]
mdp = input("Saisir votre mot de passe : ")
if mdp=="12345678" :
    print("Bienvenue~!")
else :
    print("Mot de passe incorrect~!")
\end{lstlisting}
Ce programme demandera un mot de passe.
\newpar
Puis il affichera~:\\
\texttt{    Bienvenue~!\\}
si le mot de passe entré est 12345678~;
\newpar
alors qu'il affichera ~:\\
\texttt{    Mot de passe incorrect~! \\}
si un autre mot de passe a été entré.
\begin{h3} Test \og if - elif - else \fg{} \end{h3}
Lorsque le test comporte plusieurs conditions, on peut utiliser l'instruction \og \texttt{elif} \fg{} qui est l'abréviation de \texttt{else if}. Cette instruction s'utilise de la manière suivante~:
\begin{lstlisting}[language=Python]
if condition1 :
   # les instructions placées ici
   # seront exécutées si et seulement si 
   # la condition1 est vraie 
elif condition2 :
   # les instructions placées ici
   # seront exécutées si et seulement si 
   # la condition2 est vraie 
elif condition3 :
   # ...
   # ...
   # ...
else :
   # les instructions placées ici
   # seront exécutées si et seulement si 
   # toutes les conditions précédentes sont fausses
# enfin les instructions placées ici
# seront exécutées dans tous les cas.
\end{lstlisting}
Dans ce cas, on peut placer autant de conditions que l'on veut.
\newpar
Par exemple~:
\begin{lstlisting}[language=Python]
erreurs = int(input(" Entrez le nombre d'erreurs :  "))
if erreurs == 0 :
    print(" Il n'y a aucune erreur~!")
elif erreurs == 1 :
    print(" Il y a une seule erreur~!")
else :
    print(" Il y a ", erreurs, " erreurs~!")
\end{lstlisting}
Ce programme demande puis affiche le nombre d'erreurs présentes dans un texte.
\par

\end{document}
µ
\documentclass[a4paper]{article}

%================================================================================================================================
%
% Packages
%
%================================================================================================================================

\usepackage[T1]{fontenc} 	% pour caractères accentués
\usepackage[utf8]{inputenc}  % encodage utf8
\usepackage[french]{babel}	% langue : français
\usepackage{fourier}			% caractères plus lisibles
\usepackage[dvipsnames]{xcolor} % couleurs
\usepackage{fancyhdr}		% réglage header footer
\usepackage{needspace}		% empêcher sauts de page mal placés
\usepackage{graphicx}		% pour inclure des graphiques
\usepackage{enumitem,cprotect}		% personnalise les listes d'items (nécessaire pour ol, al ...)
\usepackage{hyperref}		% Liens hypertexte
\usepackage{pstricks,pst-all,pst-node,pstricks-add,pst-math,pst-plot,pst-tree,pst-eucl} % pstricks
\usepackage[a4paper,includeheadfoot,top=2cm,left=3cm, bottom=2cm,right=3cm]{geometry} % marges etc.
\usepackage{comment}			% commentaires multilignes
\usepackage{amsmath,environ} % maths (matrices, etc.)
\usepackage{amssymb,makeidx}
\usepackage{bm}				% bold maths
\usepackage{tabularx}		% tableaux
\usepackage{colortbl}		% tableaux en couleur
\usepackage{fontawesome}		% Fontawesome
\usepackage{environ}			% environment with command
\usepackage{fp}				% calculs pour ps-tricks
\usepackage{multido}			% pour ps tricks
\usepackage[np]{numprint}	% formattage nombre
\usepackage{tikz,tkz-tab} 			% package principal TikZ
\usepackage{pgfplots}   % axes
\usepackage{mathrsfs}    % cursives
\usepackage{calc}			% calcul taille boites
\usepackage[scaled=0.875]{helvet} % font sans serif
\usepackage{svg} % svg
\usepackage{scrextend} % local margin
\usepackage{scratch} %scratch
\usepackage{multicol} % colonnes
%\usepackage{infix-RPN,pst-func} % formule en notation polanaise inversée
\usepackage{listings}

%================================================================================================================================
%
% Réglages de base
%
%================================================================================================================================

\lstset{
language=Python,   % R code
literate=
{á}{{\'a}}1
{à}{{\`a}}1
{ã}{{\~a}}1
{é}{{\'e}}1
{è}{{\`e}}1
{ê}{{\^e}}1
{í}{{\'i}}1
{ó}{{\'o}}1
{õ}{{\~o}}1
{ú}{{\'u}}1
{ü}{{\"u}}1
{ç}{{\c{c}}}1
{~}{{ }}1
}


\definecolor{codegreen}{rgb}{0,0.6,0}
\definecolor{codegray}{rgb}{0.5,0.5,0.5}
\definecolor{codepurple}{rgb}{0.58,0,0.82}
\definecolor{backcolour}{rgb}{0.95,0.95,0.92}

\lstdefinestyle{mystyle}{
    backgroundcolor=\color{backcolour},   
    commentstyle=\color{codegreen},
    keywordstyle=\color{magenta},
    numberstyle=\tiny\color{codegray},
    stringstyle=\color{codepurple},
    basicstyle=\ttfamily\footnotesize,
    breakatwhitespace=false,         
    breaklines=true,                 
    captionpos=b,                    
    keepspaces=true,                 
    numbers=left,                    
xleftmargin=2em,
framexleftmargin=2em,            
    showspaces=false,                
    showstringspaces=false,
    showtabs=false,                  
    tabsize=2,
    upquote=true
}

\lstset{style=mystyle}


\lstset{style=mystyle}
\newcommand{\imgdir}{C:/laragon/www/newmc/assets/imgsvg/}
\newcommand{\imgsvgdir}{C:/laragon/www/newmc/assets/imgsvg/}

\definecolor{mcgris}{RGB}{220, 220, 220}% ancien~; pour compatibilité
\definecolor{mcbleu}{RGB}{52, 152, 219}
\definecolor{mcvert}{RGB}{125, 194, 70}
\definecolor{mcmauve}{RGB}{154, 0, 215}
\definecolor{mcorange}{RGB}{255, 96, 0}
\definecolor{mcturquoise}{RGB}{0, 153, 153}
\definecolor{mcrouge}{RGB}{255, 0, 0}
\definecolor{mclightvert}{RGB}{205, 234, 190}

\definecolor{gris}{RGB}{220, 220, 220}
\definecolor{bleu}{RGB}{52, 152, 219}
\definecolor{vert}{RGB}{125, 194, 70}
\definecolor{mauve}{RGB}{154, 0, 215}
\definecolor{orange}{RGB}{255, 96, 0}
\definecolor{turquoise}{RGB}{0, 153, 153}
\definecolor{rouge}{RGB}{255, 0, 0}
\definecolor{lightvert}{RGB}{205, 234, 190}
\setitemize[0]{label=\color{lightvert}  $\bullet$}

\pagestyle{fancy}
\renewcommand{\headrulewidth}{0.2pt}
\fancyhead[L]{maths-cours.fr}
\fancyhead[R]{\thepage}
\renewcommand{\footrulewidth}{0.2pt}
\fancyfoot[C]{}

\newcolumntype{C}{>{\centering\arraybackslash}X}
\newcolumntype{s}{>{\hsize=.35\hsize\arraybackslash}X}

\setlength{\parindent}{0pt}		 
\setlength{\parskip}{3mm}
\setlength{\headheight}{1cm}

\def\ebook{ebook}
\def\book{book}
\def\web{web}
\def\type{web}

\newcommand{\vect}[1]{\overrightarrow{\,\mathstrut#1\,}}

\def\Oij{$\left(\text{O}~;~\vect{\imath},~\vect{\jmath}\right)$}
\def\Oijk{$\left(\text{O}~;~\vect{\imath},~\vect{\jmath},~\vect{k}\right)$}
\def\Ouv{$\left(\text{O}~;~\vect{u},~\vect{v}\right)$}

\hypersetup{breaklinks=true, colorlinks = true, linkcolor = OliveGreen, urlcolor = OliveGreen, citecolor = OliveGreen, pdfauthor={Didier BONNEL - https://www.maths-cours.fr} } % supprime les bordures autour des liens

\renewcommand{\arg}[0]{\text{arg}}

\everymath{\displaystyle}

%================================================================================================================================
%
% Macros - Commandes
%
%================================================================================================================================

\newcommand\meta[2]{    			% Utilisé pour créer le post HTML.
	\def\titre{titre}
	\def\url{url}
	\def\arg{#1}
	\ifx\titre\arg
		\newcommand\maintitle{#2}
		\fancyhead[L]{#2}
		{\Large\sffamily \MakeUppercase{#2}}
		\vspace{1mm}\textcolor{mcvert}{\hrule}
	\fi 
	\ifx\url\arg
		\fancyfoot[L]{\href{https://www.maths-cours.fr#2}{\black \footnotesize{https://www.maths-cours.fr#2}}}
	\fi 
}


\newcommand\TitreC[1]{    		% Titre centré
     \needspace{3\baselineskip}
     \begin{center}\textbf{#1}\end{center}
}

\newcommand\newpar{    		% paragraphe
     \par
}

\newcommand\nosp {    		% commande vide (pas d'espace)
}
\newcommand{\id}[1]{} %ignore

\newcommand\boite[2]{				% Boite simple sans titre
	\vspace{5mm}
	\setlength{\fboxrule}{0.2mm}
	\setlength{\fboxsep}{5mm}	
	\fcolorbox{#1}{#1!3}{\makebox[\linewidth-2\fboxrule-2\fboxsep]{
  		\begin{minipage}[t]{\linewidth-2\fboxrule-4\fboxsep}\setlength{\parskip}{3mm}
  			 #2
  		\end{minipage}
	}}
	\vspace{5mm}
}

\newcommand\CBox[4]{				% Boites
	\vspace{5mm}
	\setlength{\fboxrule}{0.2mm}
	\setlength{\fboxsep}{5mm}
	
	\fcolorbox{#1}{#1!3}{\makebox[\linewidth-2\fboxrule-2\fboxsep]{
		\begin{minipage}[t]{1cm}\setlength{\parskip}{3mm}
	  		\textcolor{#1}{\LARGE{#2}}    
 	 	\end{minipage}  
  		\begin{minipage}[t]{\linewidth-2\fboxrule-4\fboxsep}\setlength{\parskip}{3mm}
			\raisebox{1.2mm}{\normalsize\sffamily{\textcolor{#1}{#3}}}						
  			 #4
  		\end{minipage}
	}}
	\vspace{5mm}
}

\newcommand\cadre[3]{				% Boites convertible html
	\par
	\vspace{2mm}
	\setlength{\fboxrule}{0.1mm}
	\setlength{\fboxsep}{5mm}
	\fcolorbox{#1}{white}{\makebox[\linewidth-2\fboxrule-2\fboxsep]{
  		\begin{minipage}[t]{\linewidth-2\fboxrule-4\fboxsep}\setlength{\parskip}{3mm}
			\raisebox{-2.5mm}{\sffamily \small{\textcolor{#1}{\MakeUppercase{#2}}}}		
			\par		
  			 #3
 	 		\end{minipage}
	}}
		\vspace{2mm}
	\par
}

\newcommand\bloc[3]{				% Boites convertible html sans bordure
     \needspace{2\baselineskip}
     {\sffamily \small{\textcolor{#1}{\MakeUppercase{#2}}}}    
		\par		
  			 #3
		\par
}

\newcommand\CHelp[1]{
     \CBox{Plum}{\faInfoCircle}{À RETENIR}{#1}
}

\newcommand\CUp[1]{
     \CBox{NavyBlue}{\faThumbsOUp}{EN PRATIQUE}{#1}
}

\newcommand\CInfo[1]{
     \CBox{Sepia}{\faArrowCircleRight}{REMARQUE}{#1}
}

\newcommand\CRedac[1]{
     \CBox{PineGreen}{\faEdit}{BIEN R\'EDIGER}{#1}
}

\newcommand\CError[1]{
     \CBox{Red}{\faExclamationTriangle}{ATTENTION}{#1}
}

\newcommand\TitreExo[2]{
\needspace{4\baselineskip}
 {\sffamily\large EXERCICE #1\ (\emph{#2 points})}
\vspace{5mm}
}

\newcommand\img[2]{
          \includegraphics[width=#2\paperwidth]{\imgdir#1}
}

\newcommand\imgsvg[2]{
       \begin{center}   \includegraphics[width=#2\paperwidth]{\imgsvgdir#1} \end{center}
}


\newcommand\Lien[2]{
     \href{#1}{#2 \tiny \faExternalLink}
}
\newcommand\mcLien[2]{
     \href{https~://www.maths-cours.fr/#1}{#2 \tiny \faExternalLink}
}

\newcommand{\euro}{\eurologo{}}

%================================================================================================================================
%
% Macros - Environement
%
%================================================================================================================================

\newenvironment{tex}{ %
}
{%
}

\newenvironment{indente}{ %
	\setlength\parindent{10mm}
}

{
	\setlength\parindent{0mm}
}

\newenvironment{corrige}{%
     \needspace{3\baselineskip}
     \medskip
     \textbf{\textsc{Corrigé}}
     \medskip
}
{
}

\newenvironment{extern}{%
     \begin{center}
     }
     {
     \end{center}
}

\NewEnviron{code}{%
	\par
     \boite{gray}{\texttt{%
     \BODY
     }}
     \par
}

\newenvironment{vbloc}{% boite sans cadre empeche saut de page
     \begin{minipage}[t]{\linewidth}
     }
     {
     \end{minipage}
}
\NewEnviron{h2}{%
    \needspace{3\baselineskip}
    \vspace{0.6cm}
	\noindent \MakeUppercase{\sffamily \large \BODY}
	\vspace{1mm}\textcolor{mcgris}{\hrule}\vspace{0.4cm}
	\par
}{}

\NewEnviron{h3}{%
    \needspace{3\baselineskip}
	\vspace{5mm}
	\textsc{\BODY}
	\par
}

\NewEnviron{margeneg}{ %
\begin{addmargin}[-1cm]{0cm}
\BODY
\end{addmargin}
}

\NewEnviron{html}{%
}

\begin{document}
\meta{url}{/exercices/python-determiner-la-parite-dun-entier-naturel/}
\meta{pid}{11332}
\meta{titre}{Python : déterminer la parité d'un entier naturel}
\meta{type}{exercices}
%
Écrire un programme Python qui demande à l'utilisateur de saisir un nombre entier naturel et qui indique si le nombre entier saisi est pair ou impair.
\begin{corrige}
     \begin{itemize}
          \item
          il faut bien penser à convertir le nombre saisi par l'utilisateur en nombre entier (int)
          \item
          pour savoir si le nombre est pair ou impair, on utilisera l'opérateur \% (modulo) qui donne le résultat de la division euclidienne
          \item
          enfin, on utilisera une structure \texttt{if - else} pour afficher un message différent selon les cas.
     \end{itemize}
     Voici un programme possible~:
\begin{lstlisting}[language=Python]
nombre = int(input(" Entrez un nombre entier naturel :"))
if nombre % 2 == 0 :
    print("Vous avez entré un nombre pair")
else :
    print("Vous avez entré un nombre impair")
\end{lstlisting}
\end{corrige}

\end{document}
µ
\documentclass[a4paper]{article}

%================================================================================================================================
%
% Packages
%
%================================================================================================================================

\usepackage[T1]{fontenc} 	% pour caractères accentués
\usepackage[utf8]{inputenc}  % encodage utf8
\usepackage[french]{babel}	% langue : français
\usepackage{fourier}			% caractères plus lisibles
\usepackage[dvipsnames]{xcolor} % couleurs
\usepackage{fancyhdr}		% réglage header footer
\usepackage{needspace}		% empêcher sauts de page mal placés
\usepackage{graphicx}		% pour inclure des graphiques
\usepackage{enumitem,cprotect}		% personnalise les listes d'items (nécessaire pour ol, al ...)
\usepackage{hyperref}		% Liens hypertexte
\usepackage{pstricks,pst-all,pst-node,pstricks-add,pst-math,pst-plot,pst-tree,pst-eucl} % pstricks
\usepackage[a4paper,includeheadfoot,top=2cm,left=3cm, bottom=2cm,right=3cm]{geometry} % marges etc.
\usepackage{comment}			% commentaires multilignes
\usepackage{amsmath,environ} % maths (matrices, etc.)
\usepackage{amssymb,makeidx}
\usepackage{bm}				% bold maths
\usepackage{tabularx}		% tableaux
\usepackage{colortbl}		% tableaux en couleur
\usepackage{fontawesome}		% Fontawesome
\usepackage{environ}			% environment with command
\usepackage{fp}				% calculs pour ps-tricks
\usepackage{multido}			% pour ps tricks
\usepackage[np]{numprint}	% formattage nombre
\usepackage{tikz,tkz-tab} 			% package principal TikZ
\usepackage{pgfplots}   % axes
\usepackage{mathrsfs}    % cursives
\usepackage{calc}			% calcul taille boites
\usepackage[scaled=0.875]{helvet} % font sans serif
\usepackage{svg} % svg
\usepackage{scrextend} % local margin
\usepackage{scratch} %scratch
\usepackage{multicol} % colonnes
%\usepackage{infix-RPN,pst-func} % formule en notation polanaise inversée
\usepackage{listings}

%================================================================================================================================
%
% Réglages de base
%
%================================================================================================================================

\lstset{
language=Python,   % R code
literate=
{á}{{\'a}}1
{à}{{\`a}}1
{ã}{{\~a}}1
{é}{{\'e}}1
{è}{{\`e}}1
{ê}{{\^e}}1
{í}{{\'i}}1
{ó}{{\'o}}1
{õ}{{\~o}}1
{ú}{{\'u}}1
{ü}{{\"u}}1
{ç}{{\c{c}}}1
{~}{{ }}1
}


\definecolor{codegreen}{rgb}{0,0.6,0}
\definecolor{codegray}{rgb}{0.5,0.5,0.5}
\definecolor{codepurple}{rgb}{0.58,0,0.82}
\definecolor{backcolour}{rgb}{0.95,0.95,0.92}

\lstdefinestyle{mystyle}{
    backgroundcolor=\color{backcolour},   
    commentstyle=\color{codegreen},
    keywordstyle=\color{magenta},
    numberstyle=\tiny\color{codegray},
    stringstyle=\color{codepurple},
    basicstyle=\ttfamily\footnotesize,
    breakatwhitespace=false,         
    breaklines=true,                 
    captionpos=b,                    
    keepspaces=true,                 
    numbers=left,                    
xleftmargin=2em,
framexleftmargin=2em,            
    showspaces=false,                
    showstringspaces=false,
    showtabs=false,                  
    tabsize=2,
    upquote=true
}

\lstset{style=mystyle}


\lstset{style=mystyle}
\newcommand{\imgdir}{C:/laragon/www/newmc/assets/imgsvg/}
\newcommand{\imgsvgdir}{C:/laragon/www/newmc/assets/imgsvg/}

\definecolor{mcgris}{RGB}{220, 220, 220}% ancien~; pour compatibilité
\definecolor{mcbleu}{RGB}{52, 152, 219}
\definecolor{mcvert}{RGB}{125, 194, 70}
\definecolor{mcmauve}{RGB}{154, 0, 215}
\definecolor{mcorange}{RGB}{255, 96, 0}
\definecolor{mcturquoise}{RGB}{0, 153, 153}
\definecolor{mcrouge}{RGB}{255, 0, 0}
\definecolor{mclightvert}{RGB}{205, 234, 190}

\definecolor{gris}{RGB}{220, 220, 220}
\definecolor{bleu}{RGB}{52, 152, 219}
\definecolor{vert}{RGB}{125, 194, 70}
\definecolor{mauve}{RGB}{154, 0, 215}
\definecolor{orange}{RGB}{255, 96, 0}
\definecolor{turquoise}{RGB}{0, 153, 153}
\definecolor{rouge}{RGB}{255, 0, 0}
\definecolor{lightvert}{RGB}{205, 234, 190}
\setitemize[0]{label=\color{lightvert}  $\bullet$}

\pagestyle{fancy}
\renewcommand{\headrulewidth}{0.2pt}
\fancyhead[L]{maths-cours.fr}
\fancyhead[R]{\thepage}
\renewcommand{\footrulewidth}{0.2pt}
\fancyfoot[C]{}

\newcolumntype{C}{>{\centering\arraybackslash}X}
\newcolumntype{s}{>{\hsize=.35\hsize\arraybackslash}X}

\setlength{\parindent}{0pt}		 
\setlength{\parskip}{3mm}
\setlength{\headheight}{1cm}

\def\ebook{ebook}
\def\book{book}
\def\web{web}
\def\type{web}

\newcommand{\vect}[1]{\overrightarrow{\,\mathstrut#1\,}}

\def\Oij{$\left(\text{O}~;~\vect{\imath},~\vect{\jmath}\right)$}
\def\Oijk{$\left(\text{O}~;~\vect{\imath},~\vect{\jmath},~\vect{k}\right)$}
\def\Ouv{$\left(\text{O}~;~\vect{u},~\vect{v}\right)$}

\hypersetup{breaklinks=true, colorlinks = true, linkcolor = OliveGreen, urlcolor = OliveGreen, citecolor = OliveGreen, pdfauthor={Didier BONNEL - https://www.maths-cours.fr} } % supprime les bordures autour des liens

\renewcommand{\arg}[0]{\text{arg}}

\everymath{\displaystyle}

%================================================================================================================================
%
% Macros - Commandes
%
%================================================================================================================================

\newcommand\meta[2]{    			% Utilisé pour créer le post HTML.
	\def\titre{titre}
	\def\url{url}
	\def\arg{#1}
	\ifx\titre\arg
		\newcommand\maintitle{#2}
		\fancyhead[L]{#2}
		{\Large\sffamily \MakeUppercase{#2}}
		\vspace{1mm}\textcolor{mcvert}{\hrule}
	\fi 
	\ifx\url\arg
		\fancyfoot[L]{\href{https://www.maths-cours.fr#2}{\black \footnotesize{https://www.maths-cours.fr#2}}}
	\fi 
}


\newcommand\TitreC[1]{    		% Titre centré
     \needspace{3\baselineskip}
     \begin{center}\textbf{#1}\end{center}
}

\newcommand\newpar{    		% paragraphe
     \par
}

\newcommand\nosp {    		% commande vide (pas d'espace)
}
\newcommand{\id}[1]{} %ignore

\newcommand\boite[2]{				% Boite simple sans titre
	\vspace{5mm}
	\setlength{\fboxrule}{0.2mm}
	\setlength{\fboxsep}{5mm}	
	\fcolorbox{#1}{#1!3}{\makebox[\linewidth-2\fboxrule-2\fboxsep]{
  		\begin{minipage}[t]{\linewidth-2\fboxrule-4\fboxsep}\setlength{\parskip}{3mm}
  			 #2
  		\end{minipage}
	}}
	\vspace{5mm}
}

\newcommand\CBox[4]{				% Boites
	\vspace{5mm}
	\setlength{\fboxrule}{0.2mm}
	\setlength{\fboxsep}{5mm}
	
	\fcolorbox{#1}{#1!3}{\makebox[\linewidth-2\fboxrule-2\fboxsep]{
		\begin{minipage}[t]{1cm}\setlength{\parskip}{3mm}
	  		\textcolor{#1}{\LARGE{#2}}    
 	 	\end{minipage}  
  		\begin{minipage}[t]{\linewidth-2\fboxrule-4\fboxsep}\setlength{\parskip}{3mm}
			\raisebox{1.2mm}{\normalsize\sffamily{\textcolor{#1}{#3}}}						
  			 #4
  		\end{minipage}
	}}
	\vspace{5mm}
}

\newcommand\cadre[3]{				% Boites convertible html
	\par
	\vspace{2mm}
	\setlength{\fboxrule}{0.1mm}
	\setlength{\fboxsep}{5mm}
	\fcolorbox{#1}{white}{\makebox[\linewidth-2\fboxrule-2\fboxsep]{
  		\begin{minipage}[t]{\linewidth-2\fboxrule-4\fboxsep}\setlength{\parskip}{3mm}
			\raisebox{-2.5mm}{\sffamily \small{\textcolor{#1}{\MakeUppercase{#2}}}}		
			\par		
  			 #3
 	 		\end{minipage}
	}}
		\vspace{2mm}
	\par
}

\newcommand\bloc[3]{				% Boites convertible html sans bordure
     \needspace{2\baselineskip}
     {\sffamily \small{\textcolor{#1}{\MakeUppercase{#2}}}}    
		\par		
  			 #3
		\par
}

\newcommand\CHelp[1]{
     \CBox{Plum}{\faInfoCircle}{À RETENIR}{#1}
}

\newcommand\CUp[1]{
     \CBox{NavyBlue}{\faThumbsOUp}{EN PRATIQUE}{#1}
}

\newcommand\CInfo[1]{
     \CBox{Sepia}{\faArrowCircleRight}{REMARQUE}{#1}
}

\newcommand\CRedac[1]{
     \CBox{PineGreen}{\faEdit}{BIEN R\'EDIGER}{#1}
}

\newcommand\CError[1]{
     \CBox{Red}{\faExclamationTriangle}{ATTENTION}{#1}
}

\newcommand\TitreExo[2]{
\needspace{4\baselineskip}
 {\sffamily\large EXERCICE #1\ (\emph{#2 points})}
\vspace{5mm}
}

\newcommand\img[2]{
          \includegraphics[width=#2\paperwidth]{\imgdir#1}
}

\newcommand\imgsvg[2]{
       \begin{center}   \includegraphics[width=#2\paperwidth]{\imgsvgdir#1} \end{center}
}


\newcommand\Lien[2]{
     \href{#1}{#2 \tiny \faExternalLink}
}
\newcommand\mcLien[2]{
     \href{https~://www.maths-cours.fr/#1}{#2 \tiny \faExternalLink}
}

\newcommand{\euro}{\eurologo{}}

%================================================================================================================================
%
% Macros - Environement
%
%================================================================================================================================

\newenvironment{tex}{ %
}
{%
}

\newenvironment{indente}{ %
	\setlength\parindent{10mm}
}

{
	\setlength\parindent{0mm}
}

\newenvironment{corrige}{%
     \needspace{3\baselineskip}
     \medskip
     \textbf{\textsc{Corrigé}}
     \medskip
}
{
}

\newenvironment{extern}{%
     \begin{center}
     }
     {
     \end{center}
}

\NewEnviron{code}{%
	\par
     \boite{gray}{\texttt{%
     \BODY
     }}
     \par
}

\newenvironment{vbloc}{% boite sans cadre empeche saut de page
     \begin{minipage}[t]{\linewidth}
     }
     {
     \end{minipage}
}
\NewEnviron{h2}{%
    \needspace{3\baselineskip}
    \vspace{0.6cm}
	\noindent \MakeUppercase{\sffamily \large \BODY}
	\vspace{1mm}\textcolor{mcgris}{\hrule}\vspace{0.4cm}
	\par
}{}

\NewEnviron{h3}{%
    \needspace{3\baselineskip}
	\vspace{5mm}
	\textsc{\BODY}
	\par
}

\NewEnviron{margeneg}{ %
\begin{addmargin}[-1cm]{0cm}
\BODY
\end{addmargin}
}

\NewEnviron{html}{%
}

\begin{document}
\meta{url}{/exercices/python-tarifs-et-pourcentage/}
\meta{pid}{11336}
\meta{titre}{Python : tarifs et pourcentage}
\meta{type}{exercices}
%
Un cinéma propose des places à 9 € l'unité pour un adulte. Par ailleurs, il offre une réduction de 40 \% pour les moins de 13 ans et une réduction de 30 \% pour les seniors âgés de 65 ans ou plus.
\begin{enumerate}
     \item
     Écrire un programme en Python qui demande l'âge de la personne et affiche le prix de la place correspondante.
     \item
     Un groupe comprend plusieurs adultes, plusieurs enfants de moins de 13 ans et plusieurs seniors âgés de 65 ans ou plus.\\
     Écrire un programme en Python qui demande le nombre d'adultes, le nombre d'enfants et le nombre de seniors et qui affiche le prix total que devra régler le groupe.
\end{enumerate}
\begin{corrige}
     \begin{enumerate}
          \item
          Le prix d'une place pour un enfant de moins de 13 ans est~:\\
          $ P_{ 1}=9 - \frac{ 40}{ 100 }  \times 9=9 - 3,6=5,4 $ euros.
          \newpar
          Le prix d'une place pour un senior âgé de 65 ans ou plus est~:\\
          $ P_{ 2}=9 - \frac{ 30}{ 100 }  \times 9=9 - 2,7=6,3 $ euros.
          \newpar
          On peut écrire le programme python correspondant en utilisant la structure « \texttt{if - elif - else »~:
\begin{lstlisting}[language=Python]
age = int(input(" Entrez l'âge de la personne : "))
if age < 13 :
    print(" Le prix de la place est 5,4 euros")
elif age >= 65 :
    print(" Le prix de la place est 6,3 euros")
else : 
    print(" Le prix de la place est 9 euros")
          \end{lstlisting}
          \newpar
          \textbf{Remarque~:}\\
          On aurait également plus laisser le programme Python effectuer les calculs~; par exemple de la manière suivante~:
\begin{lstlisting}[language=Python]
age = int(input("Entrez l'âge de la personne : "))
if age < 13 :
    prix = 9 - 40 / 100 * 9
elif age >= 65 :
    prix = 9 - 30 / 100 * 9
else :
    prix = 9
print("Le prix de la place est" , prix, "euros")
     \end{lstlisting}
     \item
     Pour chacune des catégories d'âge, il suffit de multiplier les effectifs par le prix correspondant puis d'effectuer le total.
\begin{lstlisting}[language=Python]
adultes = int(input( "Entrez le nombre d''adultes ne bénéficiant pas de réduction : "))
enfants = int(input( "Entrez le nombre d'enfants de moins de 13 ans : "))
seniors = int(input( "Entrez le nombre de seniors de 65 ans ou plus : "))
prix = 9*adultes + 5.4*enfants + 6.3*seniors
print(" Le prix total à payer pour le groupe est :", prix)
\end{lstlisting}
\textbf{Attention} :\\
Pour les prix décimaux, utilisez le point et non la virgule comme séparateur~!
\end{enumerate}
\end{corrige}

\end{document}
µ
\documentclass[a4paper]{article}

%================================================================================================================================
%
% Packages
%
%================================================================================================================================

\usepackage[T1]{fontenc} 	% pour caractères accentués
\usepackage[utf8]{inputenc}  % encodage utf8
\usepackage[french]{babel}	% langue : français
\usepackage{fourier}			% caractères plus lisibles
\usepackage[dvipsnames]{xcolor} % couleurs
\usepackage{fancyhdr}		% réglage header footer
\usepackage{needspace}		% empêcher sauts de page mal placés
\usepackage{graphicx}		% pour inclure des graphiques
\usepackage{enumitem,cprotect}		% personnalise les listes d'items (nécessaire pour ol, al ...)
\usepackage{hyperref}		% Liens hypertexte
\usepackage{pstricks,pst-all,pst-node,pstricks-add,pst-math,pst-plot,pst-tree,pst-eucl} % pstricks
\usepackage[a4paper,includeheadfoot,top=2cm,left=3cm, bottom=2cm,right=3cm]{geometry} % marges etc.
\usepackage{comment}			% commentaires multilignes
\usepackage{amsmath,environ} % maths (matrices, etc.)
\usepackage{amssymb,makeidx}
\usepackage{bm}				% bold maths
\usepackage{tabularx}		% tableaux
\usepackage{colortbl}		% tableaux en couleur
\usepackage{fontawesome}		% Fontawesome
\usepackage{environ}			% environment with command
\usepackage{fp}				% calculs pour ps-tricks
\usepackage{multido}			% pour ps tricks
\usepackage[np]{numprint}	% formattage nombre
\usepackage{tikz,tkz-tab} 			% package principal TikZ
\usepackage{pgfplots}   % axes
\usepackage{mathrsfs}    % cursives
\usepackage{calc}			% calcul taille boites
\usepackage[scaled=0.875]{helvet} % font sans serif
\usepackage{svg} % svg
\usepackage{scrextend} % local margin
\usepackage{scratch} %scratch
\usepackage{multicol} % colonnes
%\usepackage{infix-RPN,pst-func} % formule en notation polanaise inversée
\usepackage{listings}

%================================================================================================================================
%
% Réglages de base
%
%================================================================================================================================

\lstset{
language=Python,   % R code
literate=
{á}{{\'a}}1
{à}{{\`a}}1
{ã}{{\~a}}1
{é}{{\'e}}1
{è}{{\`e}}1
{ê}{{\^e}}1
{í}{{\'i}}1
{ó}{{\'o}}1
{õ}{{\~o}}1
{ú}{{\'u}}1
{ü}{{\"u}}1
{ç}{{\c{c}}}1
{~}{{ }}1
}


\definecolor{codegreen}{rgb}{0,0.6,0}
\definecolor{codegray}{rgb}{0.5,0.5,0.5}
\definecolor{codepurple}{rgb}{0.58,0,0.82}
\definecolor{backcolour}{rgb}{0.95,0.95,0.92}

\lstdefinestyle{mystyle}{
    backgroundcolor=\color{backcolour},   
    commentstyle=\color{codegreen},
    keywordstyle=\color{magenta},
    numberstyle=\tiny\color{codegray},
    stringstyle=\color{codepurple},
    basicstyle=\ttfamily\footnotesize,
    breakatwhitespace=false,         
    breaklines=true,                 
    captionpos=b,                    
    keepspaces=true,                 
    numbers=left,                    
xleftmargin=2em,
framexleftmargin=2em,            
    showspaces=false,                
    showstringspaces=false,
    showtabs=false,                  
    tabsize=2,
    upquote=true
}

\lstset{style=mystyle}


\lstset{style=mystyle}
\newcommand{\imgdir}{C:/laragon/www/newmc/assets/imgsvg/}
\newcommand{\imgsvgdir}{C:/laragon/www/newmc/assets/imgsvg/}

\definecolor{mcgris}{RGB}{220, 220, 220}% ancien~; pour compatibilité
\definecolor{mcbleu}{RGB}{52, 152, 219}
\definecolor{mcvert}{RGB}{125, 194, 70}
\definecolor{mcmauve}{RGB}{154, 0, 215}
\definecolor{mcorange}{RGB}{255, 96, 0}
\definecolor{mcturquoise}{RGB}{0, 153, 153}
\definecolor{mcrouge}{RGB}{255, 0, 0}
\definecolor{mclightvert}{RGB}{205, 234, 190}

\definecolor{gris}{RGB}{220, 220, 220}
\definecolor{bleu}{RGB}{52, 152, 219}
\definecolor{vert}{RGB}{125, 194, 70}
\definecolor{mauve}{RGB}{154, 0, 215}
\definecolor{orange}{RGB}{255, 96, 0}
\definecolor{turquoise}{RGB}{0, 153, 153}
\definecolor{rouge}{RGB}{255, 0, 0}
\definecolor{lightvert}{RGB}{205, 234, 190}
\setitemize[0]{label=\color{lightvert}  $\bullet$}

\pagestyle{fancy}
\renewcommand{\headrulewidth}{0.2pt}
\fancyhead[L]{maths-cours.fr}
\fancyhead[R]{\thepage}
\renewcommand{\footrulewidth}{0.2pt}
\fancyfoot[C]{}

\newcolumntype{C}{>{\centering\arraybackslash}X}
\newcolumntype{s}{>{\hsize=.35\hsize\arraybackslash}X}

\setlength{\parindent}{0pt}		 
\setlength{\parskip}{3mm}
\setlength{\headheight}{1cm}

\def\ebook{ebook}
\def\book{book}
\def\web{web}
\def\type{web}

\newcommand{\vect}[1]{\overrightarrow{\,\mathstrut#1\,}}

\def\Oij{$\left(\text{O}~;~\vect{\imath},~\vect{\jmath}\right)$}
\def\Oijk{$\left(\text{O}~;~\vect{\imath},~\vect{\jmath},~\vect{k}\right)$}
\def\Ouv{$\left(\text{O}~;~\vect{u},~\vect{v}\right)$}

\hypersetup{breaklinks=true, colorlinks = true, linkcolor = OliveGreen, urlcolor = OliveGreen, citecolor = OliveGreen, pdfauthor={Didier BONNEL - https://www.maths-cours.fr} } % supprime les bordures autour des liens

\renewcommand{\arg}[0]{\text{arg}}

\everymath{\displaystyle}

%================================================================================================================================
%
% Macros - Commandes
%
%================================================================================================================================

\newcommand\meta[2]{    			% Utilisé pour créer le post HTML.
	\def\titre{titre}
	\def\url{url}
	\def\arg{#1}
	\ifx\titre\arg
		\newcommand\maintitle{#2}
		\fancyhead[L]{#2}
		{\Large\sffamily \MakeUppercase{#2}}
		\vspace{1mm}\textcolor{mcvert}{\hrule}
	\fi 
	\ifx\url\arg
		\fancyfoot[L]{\href{https://www.maths-cours.fr#2}{\black \footnotesize{https://www.maths-cours.fr#2}}}
	\fi 
}


\newcommand\TitreC[1]{    		% Titre centré
     \needspace{3\baselineskip}
     \begin{center}\textbf{#1}\end{center}
}

\newcommand\newpar{    		% paragraphe
     \par
}

\newcommand\nosp {    		% commande vide (pas d'espace)
}
\newcommand{\id}[1]{} %ignore

\newcommand\boite[2]{				% Boite simple sans titre
	\vspace{5mm}
	\setlength{\fboxrule}{0.2mm}
	\setlength{\fboxsep}{5mm}	
	\fcolorbox{#1}{#1!3}{\makebox[\linewidth-2\fboxrule-2\fboxsep]{
  		\begin{minipage}[t]{\linewidth-2\fboxrule-4\fboxsep}\setlength{\parskip}{3mm}
  			 #2
  		\end{minipage}
	}}
	\vspace{5mm}
}

\newcommand\CBox[4]{				% Boites
	\vspace{5mm}
	\setlength{\fboxrule}{0.2mm}
	\setlength{\fboxsep}{5mm}
	
	\fcolorbox{#1}{#1!3}{\makebox[\linewidth-2\fboxrule-2\fboxsep]{
		\begin{minipage}[t]{1cm}\setlength{\parskip}{3mm}
	  		\textcolor{#1}{\LARGE{#2}}    
 	 	\end{minipage}  
  		\begin{minipage}[t]{\linewidth-2\fboxrule-4\fboxsep}\setlength{\parskip}{3mm}
			\raisebox{1.2mm}{\normalsize\sffamily{\textcolor{#1}{#3}}}						
  			 #4
  		\end{minipage}
	}}
	\vspace{5mm}
}

\newcommand\cadre[3]{				% Boites convertible html
	\par
	\vspace{2mm}
	\setlength{\fboxrule}{0.1mm}
	\setlength{\fboxsep}{5mm}
	\fcolorbox{#1}{white}{\makebox[\linewidth-2\fboxrule-2\fboxsep]{
  		\begin{minipage}[t]{\linewidth-2\fboxrule-4\fboxsep}\setlength{\parskip}{3mm}
			\raisebox{-2.5mm}{\sffamily \small{\textcolor{#1}{\MakeUppercase{#2}}}}		
			\par		
  			 #3
 	 		\end{minipage}
	}}
		\vspace{2mm}
	\par
}

\newcommand\bloc[3]{				% Boites convertible html sans bordure
     \needspace{2\baselineskip}
     {\sffamily \small{\textcolor{#1}{\MakeUppercase{#2}}}}    
		\par		
  			 #3
		\par
}

\newcommand\CHelp[1]{
     \CBox{Plum}{\faInfoCircle}{À RETENIR}{#1}
}

\newcommand\CUp[1]{
     \CBox{NavyBlue}{\faThumbsOUp}{EN PRATIQUE}{#1}
}

\newcommand\CInfo[1]{
     \CBox{Sepia}{\faArrowCircleRight}{REMARQUE}{#1}
}

\newcommand\CRedac[1]{
     \CBox{PineGreen}{\faEdit}{BIEN R\'EDIGER}{#1}
}

\newcommand\CError[1]{
     \CBox{Red}{\faExclamationTriangle}{ATTENTION}{#1}
}

\newcommand\TitreExo[2]{
\needspace{4\baselineskip}
 {\sffamily\large EXERCICE #1\ (\emph{#2 points})}
\vspace{5mm}
}

\newcommand\img[2]{
          \includegraphics[width=#2\paperwidth]{\imgdir#1}
}

\newcommand\imgsvg[2]{
       \begin{center}   \includegraphics[width=#2\paperwidth]{\imgsvgdir#1} \end{center}
}


\newcommand\Lien[2]{
     \href{#1}{#2 \tiny \faExternalLink}
}
\newcommand\mcLien[2]{
     \href{https~://www.maths-cours.fr/#1}{#2 \tiny \faExternalLink}
}

\newcommand{\euro}{\eurologo{}}

%================================================================================================================================
%
% Macros - Environement
%
%================================================================================================================================

\newenvironment{tex}{ %
}
{%
}

\newenvironment{indente}{ %
	\setlength\parindent{10mm}
}

{
	\setlength\parindent{0mm}
}

\newenvironment{corrige}{%
     \needspace{3\baselineskip}
     \medskip
     \textbf{\textsc{Corrigé}}
     \medskip
}
{
}

\newenvironment{extern}{%
     \begin{center}
     }
     {
     \end{center}
}

\NewEnviron{code}{%
	\par
     \boite{gray}{\texttt{%
     \BODY
     }}
     \par
}

\newenvironment{vbloc}{% boite sans cadre empeche saut de page
     \begin{minipage}[t]{\linewidth}
     }
     {
     \end{minipage}
}
\NewEnviron{h2}{%
    \needspace{3\baselineskip}
    \vspace{0.6cm}
	\noindent \MakeUppercase{\sffamily \large \BODY}
	\vspace{1mm}\textcolor{mcgris}{\hrule}\vspace{0.4cm}
	\par
}{}

\NewEnviron{h3}{%
    \needspace{3\baselineskip}
	\vspace{5mm}
	\textsc{\BODY}
	\par
}

\NewEnviron{margeneg}{ %
\begin{addmargin}[-1cm]{0cm}
\BODY
\end{addmargin}
}

\NewEnviron{html}{%
}

\begin{document}
\meta{url}{/exercices/python-fonction-definie-par-morceaux/}
\meta{pid}{11341}
\meta{titre}{Python : Fonction définie par morceaux}
\meta{type}{exercices}
%
On considère la fonction  $ f $ définie sur $  \mathbb{R} $ par~:
\newpar
\[ f(x) =  \left\{ \begin{matrix}  x  & \texttt{si} & x < 0\\  x^2  - 1 &\texttt{si} & 0
\leqslant x<1 \\  x+5 & \texttt{si} & x \geqslant 1   \end{matrix} \right.\]
\begin{enumerate}
     \item
     Compléter le tableau de valeurs suivant~:\\
     \begin{center}
          \begin{tabular}{|c|c|c|c|c|c|c|c|}%class="compact"
               \hline
               $ x$ &  - 2  &  - 1  & 0  & 0,5  & 1  & 2  & 3  \\
               \hline
               $ f (x) $   &   &   &   &   &   &   &   \\
               \hline
          \end{tabular}
     \end{center}
     \item
     Écrire un programme Python qui demande à l'utilisateur d'entrer une valeur de $x$ et qui calcule l'image de  $x$ par la fonction  $f$.\\
     À l'aide de ce programme, vérifier les résultats de la question précédente.
     \begin{corrige}
          \begin{enumerate}
               \item
               \begin{itemize}
                    \item
                    comme $  - 2<0$, $ f (-2) =-2$
                    \item
                    comme $ -1 < 0 $, $ f ( - 1) = - 1$
                    \item
                    comme $ 0 \leqslant 0 < 1$, $ f (0) =0^2  - 1= - 1$
                    \item
                    comme $ 0 \leqslant 0,5 < 1$, $ f (0,5) =0,5^2  - 1= - 0,75$
                    \item
                    comme $ 1 \geqslant 1$,  $ f (1) =1+5=6 $
                    \item
                    comme $ 2 \geqslant 1$, $ f (2) =2+5=7$
                    \item
                    comme $ 3 \geqslant 1 $, $ f (3) =3+5=8 $
               \end{itemize}
               On obtient donc le tableau de valeurs suivant~:
               \begin{center}
                    \begin{tabular}{|c|c|c|c|c|c|c|c|}%class="compact"
                         \hline
                         $ x$ &  - 2  &  - 1  & 0  & 0,5  & 1  & 2  & 3  \\
                         \hline
                         $ f (x) $   &   - 2     &   - 1 &   - 1 &   - 0,75 & 6  & 7  & 8  \\
                         \hline
                    \end{tabular}
               \end{center}
               \item
               Le programme Python devra exécuter les tâches suivantes~:
               \begin{itemize}
                    \item
                    demander à l'utilisateur d'entrer en nombre décimal  (penser à convertir en \texttt{float})
                    \item
                    calculer l'image de $ x$ par la fonction  $ f$ en distinguant les différents cas à l'aide d'une instruction \texttt{if - elif - else}
                    \item
                    afficher le résultat trouvé pour $ f (x) $.
               \end{itemize}
               Voici un exemple possible~:
\begin{lstlisting}[language=Python]
x=float(input("Entrer une valeur de x :"))
if x<0 :
   resultat = x
elif x<1 :
    resultat = x**2-1
else :
    resultat = x+5
print(resultat)
          \end{lstlisting}
     \end{enumerate}
     \bloc{cyan}{Remarque}{ % id= r010
          En ligne 4., on aurait pu écrire également « \texttt{elif x>=0 and x<1} », toutefois comme la condition « \texttt{x<0} » a déjà été traité en ligne 2. on est sûr, lorsque l'on arrive en ligne 4, que « \texttt{x>=0} » et il n'y a donc pas besoin de faire figurer alors la condition « \texttt{ x>=0} ».
     }% fin remarque
     \newpar
     En saisissant ensuite les valeurs de $x$ données dans le tableau, on retrouve bien, grâce au programme ci-dessus, les images trouvées à la question 1.
\end{corrige}

\end{document}
µ
\documentclass[a4paper]{article}

%================================================================================================================================
%
% Packages
%
%================================================================================================================================

\usepackage[T1]{fontenc} 	% pour caractères accentués
\usepackage[utf8]{inputenc}  % encodage utf8
\usepackage[french]{babel}	% langue : français
\usepackage{fourier}			% caractères plus lisibles
\usepackage[dvipsnames]{xcolor} % couleurs
\usepackage{fancyhdr}		% réglage header footer
\usepackage{needspace}		% empêcher sauts de page mal placés
\usepackage{graphicx}		% pour inclure des graphiques
\usepackage{enumitem,cprotect}		% personnalise les listes d'items (nécessaire pour ol, al ...)
\usepackage{hyperref}		% Liens hypertexte
\usepackage{pstricks,pst-all,pst-node,pstricks-add,pst-math,pst-plot,pst-tree,pst-eucl} % pstricks
\usepackage[a4paper,includeheadfoot,top=2cm,left=3cm, bottom=2cm,right=3cm]{geometry} % marges etc.
\usepackage{comment}			% commentaires multilignes
\usepackage{amsmath,environ} % maths (matrices, etc.)
\usepackage{amssymb,makeidx}
\usepackage{bm}				% bold maths
\usepackage{tabularx}		% tableaux
\usepackage{colortbl}		% tableaux en couleur
\usepackage{fontawesome}		% Fontawesome
\usepackage{environ}			% environment with command
\usepackage{fp}				% calculs pour ps-tricks
\usepackage{multido}			% pour ps tricks
\usepackage[np]{numprint}	% formattage nombre
\usepackage{tikz,tkz-tab} 			% package principal TikZ
\usepackage{pgfplots}   % axes
\usepackage{mathrsfs}    % cursives
\usepackage{calc}			% calcul taille boites
\usepackage[scaled=0.875]{helvet} % font sans serif
\usepackage{svg} % svg
\usepackage{scrextend} % local margin
\usepackage{scratch} %scratch
\usepackage{multicol} % colonnes
%\usepackage{infix-RPN,pst-func} % formule en notation polanaise inversée
\usepackage{listings}

%================================================================================================================================
%
% Réglages de base
%
%================================================================================================================================

\lstset{
language=Python,   % R code
literate=
{á}{{\'a}}1
{à}{{\`a}}1
{ã}{{\~a}}1
{é}{{\'e}}1
{è}{{\`e}}1
{ê}{{\^e}}1
{í}{{\'i}}1
{ó}{{\'o}}1
{õ}{{\~o}}1
{ú}{{\'u}}1
{ü}{{\"u}}1
{ç}{{\c{c}}}1
{~}{{ }}1
}


\definecolor{codegreen}{rgb}{0,0.6,0}
\definecolor{codegray}{rgb}{0.5,0.5,0.5}
\definecolor{codepurple}{rgb}{0.58,0,0.82}
\definecolor{backcolour}{rgb}{0.95,0.95,0.92}

\lstdefinestyle{mystyle}{
    backgroundcolor=\color{backcolour},   
    commentstyle=\color{codegreen},
    keywordstyle=\color{magenta},
    numberstyle=\tiny\color{codegray},
    stringstyle=\color{codepurple},
    basicstyle=\ttfamily\footnotesize,
    breakatwhitespace=false,         
    breaklines=true,                 
    captionpos=b,                    
    keepspaces=true,                 
    numbers=left,                    
xleftmargin=2em,
framexleftmargin=2em,            
    showspaces=false,                
    showstringspaces=false,
    showtabs=false,                  
    tabsize=2,
    upquote=true
}

\lstset{style=mystyle}


\lstset{style=mystyle}
\newcommand{\imgdir}{C:/laragon/www/newmc/assets/imgsvg/}
\newcommand{\imgsvgdir}{C:/laragon/www/newmc/assets/imgsvg/}

\definecolor{mcgris}{RGB}{220, 220, 220}% ancien~; pour compatibilité
\definecolor{mcbleu}{RGB}{52, 152, 219}
\definecolor{mcvert}{RGB}{125, 194, 70}
\definecolor{mcmauve}{RGB}{154, 0, 215}
\definecolor{mcorange}{RGB}{255, 96, 0}
\definecolor{mcturquoise}{RGB}{0, 153, 153}
\definecolor{mcrouge}{RGB}{255, 0, 0}
\definecolor{mclightvert}{RGB}{205, 234, 190}

\definecolor{gris}{RGB}{220, 220, 220}
\definecolor{bleu}{RGB}{52, 152, 219}
\definecolor{vert}{RGB}{125, 194, 70}
\definecolor{mauve}{RGB}{154, 0, 215}
\definecolor{orange}{RGB}{255, 96, 0}
\definecolor{turquoise}{RGB}{0, 153, 153}
\definecolor{rouge}{RGB}{255, 0, 0}
\definecolor{lightvert}{RGB}{205, 234, 190}
\setitemize[0]{label=\color{lightvert}  $\bullet$}

\pagestyle{fancy}
\renewcommand{\headrulewidth}{0.2pt}
\fancyhead[L]{maths-cours.fr}
\fancyhead[R]{\thepage}
\renewcommand{\footrulewidth}{0.2pt}
\fancyfoot[C]{}

\newcolumntype{C}{>{\centering\arraybackslash}X}
\newcolumntype{s}{>{\hsize=.35\hsize\arraybackslash}X}

\setlength{\parindent}{0pt}		 
\setlength{\parskip}{3mm}
\setlength{\headheight}{1cm}

\def\ebook{ebook}
\def\book{book}
\def\web{web}
\def\type{web}

\newcommand{\vect}[1]{\overrightarrow{\,\mathstrut#1\,}}

\def\Oij{$\left(\text{O}~;~\vect{\imath},~\vect{\jmath}\right)$}
\def\Oijk{$\left(\text{O}~;~\vect{\imath},~\vect{\jmath},~\vect{k}\right)$}
\def\Ouv{$\left(\text{O}~;~\vect{u},~\vect{v}\right)$}

\hypersetup{breaklinks=true, colorlinks = true, linkcolor = OliveGreen, urlcolor = OliveGreen, citecolor = OliveGreen, pdfauthor={Didier BONNEL - https://www.maths-cours.fr} } % supprime les bordures autour des liens

\renewcommand{\arg}[0]{\text{arg}}

\everymath{\displaystyle}

%================================================================================================================================
%
% Macros - Commandes
%
%================================================================================================================================

\newcommand\meta[2]{    			% Utilisé pour créer le post HTML.
	\def\titre{titre}
	\def\url{url}
	\def\arg{#1}
	\ifx\titre\arg
		\newcommand\maintitle{#2}
		\fancyhead[L]{#2}
		{\Large\sffamily \MakeUppercase{#2}}
		\vspace{1mm}\textcolor{mcvert}{\hrule}
	\fi 
	\ifx\url\arg
		\fancyfoot[L]{\href{https://www.maths-cours.fr#2}{\black \footnotesize{https://www.maths-cours.fr#2}}}
	\fi 
}


\newcommand\TitreC[1]{    		% Titre centré
     \needspace{3\baselineskip}
     \begin{center}\textbf{#1}\end{center}
}

\newcommand\newpar{    		% paragraphe
     \par
}

\newcommand\nosp {    		% commande vide (pas d'espace)
}
\newcommand{\id}[1]{} %ignore

\newcommand\boite[2]{				% Boite simple sans titre
	\vspace{5mm}
	\setlength{\fboxrule}{0.2mm}
	\setlength{\fboxsep}{5mm}	
	\fcolorbox{#1}{#1!3}{\makebox[\linewidth-2\fboxrule-2\fboxsep]{
  		\begin{minipage}[t]{\linewidth-2\fboxrule-4\fboxsep}\setlength{\parskip}{3mm}
  			 #2
  		\end{minipage}
	}}
	\vspace{5mm}
}

\newcommand\CBox[4]{				% Boites
	\vspace{5mm}
	\setlength{\fboxrule}{0.2mm}
	\setlength{\fboxsep}{5mm}
	
	\fcolorbox{#1}{#1!3}{\makebox[\linewidth-2\fboxrule-2\fboxsep]{
		\begin{minipage}[t]{1cm}\setlength{\parskip}{3mm}
	  		\textcolor{#1}{\LARGE{#2}}    
 	 	\end{minipage}  
  		\begin{minipage}[t]{\linewidth-2\fboxrule-4\fboxsep}\setlength{\parskip}{3mm}
			\raisebox{1.2mm}{\normalsize\sffamily{\textcolor{#1}{#3}}}						
  			 #4
  		\end{minipage}
	}}
	\vspace{5mm}
}

\newcommand\cadre[3]{				% Boites convertible html
	\par
	\vspace{2mm}
	\setlength{\fboxrule}{0.1mm}
	\setlength{\fboxsep}{5mm}
	\fcolorbox{#1}{white}{\makebox[\linewidth-2\fboxrule-2\fboxsep]{
  		\begin{minipage}[t]{\linewidth-2\fboxrule-4\fboxsep}\setlength{\parskip}{3mm}
			\raisebox{-2.5mm}{\sffamily \small{\textcolor{#1}{\MakeUppercase{#2}}}}		
			\par		
  			 #3
 	 		\end{minipage}
	}}
		\vspace{2mm}
	\par
}

\newcommand\bloc[3]{				% Boites convertible html sans bordure
     \needspace{2\baselineskip}
     {\sffamily \small{\textcolor{#1}{\MakeUppercase{#2}}}}    
		\par		
  			 #3
		\par
}

\newcommand\CHelp[1]{
     \CBox{Plum}{\faInfoCircle}{À RETENIR}{#1}
}

\newcommand\CUp[1]{
     \CBox{NavyBlue}{\faThumbsOUp}{EN PRATIQUE}{#1}
}

\newcommand\CInfo[1]{
     \CBox{Sepia}{\faArrowCircleRight}{REMARQUE}{#1}
}

\newcommand\CRedac[1]{
     \CBox{PineGreen}{\faEdit}{BIEN R\'EDIGER}{#1}
}

\newcommand\CError[1]{
     \CBox{Red}{\faExclamationTriangle}{ATTENTION}{#1}
}

\newcommand\TitreExo[2]{
\needspace{4\baselineskip}
 {\sffamily\large EXERCICE #1\ (\emph{#2 points})}
\vspace{5mm}
}

\newcommand\img[2]{
          \includegraphics[width=#2\paperwidth]{\imgdir#1}
}

\newcommand\imgsvg[2]{
       \begin{center}   \includegraphics[width=#2\paperwidth]{\imgsvgdir#1} \end{center}
}


\newcommand\Lien[2]{
     \href{#1}{#2 \tiny \faExternalLink}
}
\newcommand\mcLien[2]{
     \href{https~://www.maths-cours.fr/#1}{#2 \tiny \faExternalLink}
}

\newcommand{\euro}{\eurologo{}}

%================================================================================================================================
%
% Macros - Environement
%
%================================================================================================================================

\newenvironment{tex}{ %
}
{%
}

\newenvironment{indente}{ %
	\setlength\parindent{10mm}
}

{
	\setlength\parindent{0mm}
}

\newenvironment{corrige}{%
     \needspace{3\baselineskip}
     \medskip
     \textbf{\textsc{Corrigé}}
     \medskip
}
{
}

\newenvironment{extern}{%
     \begin{center}
     }
     {
     \end{center}
}

\NewEnviron{code}{%
	\par
     \boite{gray}{\texttt{%
     \BODY
     }}
     \par
}

\newenvironment{vbloc}{% boite sans cadre empeche saut de page
     \begin{minipage}[t]{\linewidth}
     }
     {
     \end{minipage}
}
\NewEnviron{h2}{%
    \needspace{3\baselineskip}
    \vspace{0.6cm}
	\noindent \MakeUppercase{\sffamily \large \BODY}
	\vspace{1mm}\textcolor{mcgris}{\hrule}\vspace{0.4cm}
	\par
}{}

\NewEnviron{h3}{%
    \needspace{3\baselineskip}
	\vspace{5mm}
	\textsc{\BODY}
	\par
}

\NewEnviron{margeneg}{ %
\begin{addmargin}[-1cm]{0cm}
\BODY
\end{addmargin}
}

\NewEnviron{html}{%
}

\begin{document}
\meta{url}{/cours/python-au-lycee-3-les-boucles/}
\meta{pid}{11354}
\meta{titre}{Python au lycée (3)~: Les boucles}
\meta{type}{cours}
%
\\
L'un des intérêts de la programmation est de pouvoir faire exécuter facilement à une machine des tâches \textbf{répétitives}.
\newpar
Le langage Python propose deux instructions~: « for » et « while » qui permettent de répéter automatiquement l'exécution de certains blocs de code.
\id{h05}\begin{h2}1. Les boucles \og for \fg{} (Boucles bornées) \end{h2}
Une boucle \og for \fg{} (ou boucle \og pour \fg{} ou boucle bornée) est généralement utilisée lorsque l'on connaît le nombre de répétitions que l'on souhaite exécuter.
\newpar
La syntaxe de cette instruction est la suivante~:
\begin{lstlisting}[language=Python]
for variable in [liste de valeurs] :
   # bloc d'instructions à répéter
# instructions à exécuter une fois la boucle terminée
\end{lstlisting}
Ce programme se déroule de la manière suivante~:
\begin{itemize}
     \item
     les \og instructions à répéter \fg{} sont exécutées en donnant à la \og variable \fg{} chacune des valeurs de la \og liste de valeurs \fg{}~; c''est l'indentation (écriture décalée vers la droite) qui détermine la taille du bloc d'instructions à répéter
     \item
     une fois que tous les items de la liste ont été parcourus, le programme passe au \og instructions à exécuter une fois la boucle terminée \fg{}.
\end{itemize}
Par exemple, le programme Python ci-dessous ~:
\begin{lstlisting}[language=Python]
for i in [1, 2, 5, 11] :
   j = i**2 # calcul du carré de i
   print(j, end=' - ') # affichage~; on sépare les valeurs par des tirets 
print("fin") 
\end{lstlisting}
affichera~:\\
1 - 4 - 25 - 121 - fin\\
ce qui correspond aux carrés des nombres de la liste.
\newpar
\textbf{Remarque~: } on aurait pu faire l'économie de la variable j, en écrivant plus simplement \og print(i**2, end=' - ') \fg{} mais le but, ici, était de montrer qu'un bloc pouvait comporter plusieurs lignes.
\newpar
Avec l'instruction \texttt{for} on utilise fréquemment la fonction \texttt{range}~;
en effet, \texttt{range(a,b)} renvoie la liste des entiers compris (au sens large) entre \texttt{a} et \texttt{b - 1}.
\cadre{rouge}{Attention}{ % id=a010
     L'instruction \texttt{range(a,b)} créer une liste qui s'arrête à l'entier \texttt{b -1} et non à l'entier \texttt{b}~!
}% fin théorème
\newpar
Par exemple, le programme suivant affiche les doubles des nombres entiers compris entre 3 et 5~:\\
\begin{lstlisting}[language=Python]
for i in range(3, 6) :
   print(2*i, end=' - ') # affiche 6 - 8 - 10 -
\end{lstlisting}
\textbf{Remarque~:} si l'on utilise l'instruction \texttt{range} avec un seul paramètre \texttt{b}, celle-ci retournera la liste des entiers compris entre 0 et \texttt{b - 1}~:
\begin{lstlisting}[language=Python]
for i in range(4):
   print(i, end=' - ') # affiche 0 - 1 - 2 - 3 -
\end{lstlisting}
\id{h10}\begin{h2}2. Les boucles « while » (Boucles non bornées) \end{h2}
On utilise une boucle \texttt{while} (ou boucle \og Tant que \fg{} ou boucle non bornée) lorsque l'on doit répéter l'exécution d'un bloc d'instructions \textbf{tant qu'une condition est vérifiée} (mais en général, on ne sait pas au préalable le nombre de répétitions que l'on devra effectuer). Là encore, c'est l'indentation qui détermine la fin du bloc d'instructions à répéter.
\newpar
La syntaxe de l'instruction « while » est~:
\begin{lstlisting}[language=Python]
while condition :
   #bloc d'instructions à répéter 
#instructions à exécuter une fois la boucle terminée
\end{lstlisting}
Le programme se déroule alors de la façon suivante~:
\begin{itemize}
     \item
     tant que la \og condition \fg{} de la ligne 1. est vraie, les \og instructions à répéter \fg{} de la ligne 2. sont exécutées \item
     dès que la \og condition \fg{} de la ligne 1. devient fausse, le programme passe aux \og instructions à exécuter une fois la boucle terminée \fg{} (ligne 3.).
\end{itemize}
Par exemple, le programme ci-dessous affiche la plus petite puissance de 2 qui est supérieure ou égale à 1~000~:
\begin{lstlisting}[language=Python]
i = 1 # initialisation de i
while i < 1000 :
   i = i * 2
print(i) # affiche 1024
\end{lstlisting}
À chaque passage à la ligne 3., on multiplie la valeur précédente de de \texttt{i} par 2. On obtient ainsi les puissances successives de 2.
\newpar
Pour bien comprendre comment fonctionne ce programme, il peut être utile de représenter les différentes valeurs de \texttt{i} et de la condition \texttt{i < 1000} dans un tableau~:
\begin{center}
     \begin{tabular}{|c|c|c|c|c|c|c|c|c|c|c|c|c|c|c|}%class="compact" width="600"
          \hline
          $i$ & 1 & 2 & 4 & 8 & 16 & 32 & 64 & 128 & 256 & 512 & 1024
          \\ \hline
          $ i < 1000 $ & vrai & vrai & vrai & vrai & vrai & vrai & vrai & vrai & vrai & vrai & faux
          \\ \hline
     \end{tabular}
\end{center}
L'utilisation de l'instruction \texttt{while} peut être assez délicate~; si, suite à une erreur de programmation, la condition figurant dans le \texttt{while} est toujours vérifiée, le programme ne sortira jamais de la boucle et tournera indéfiniment (sauf si un intervenant extérieur met fin à son exécution...)
\newpar
En particulier, il faut faire attention aux points suivants~:
\begin{itemize}
     \item
     penser à initialiser les variables avant l'instruction \texttt{while}~: dans notre exemple, python provoquerait une erreur si la ligne 1. (initialisation de i) était manquante.
     \item
     la condition figurant après l'instruction \texttt{while} est celle qui permet de \textbf{rester} dans la boucle~; c'est donc le \textbf{contraire} de la condition de \textbf{sortie} de boucle. Dans l'exemple précédent, on souhaitait \textbf{sortir} de la boucle lorsque la valeur de \texttt{i} était \textbf{supérieure ou égale à 1000}~; il fallait donc coder \texttt{i < 1000} à l'intérieur de l'instruction \texttt{while}.
\end{itemize}

\end{document}
µ
\documentclass[a4paper]{article}

%================================================================================================================================
%
% Packages
%
%================================================================================================================================

\usepackage[T1]{fontenc} 	% pour caractères accentués
\usepackage[utf8]{inputenc}  % encodage utf8
\usepackage[french]{babel}	% langue : français
\usepackage{fourier}			% caractères plus lisibles
\usepackage[dvipsnames]{xcolor} % couleurs
\usepackage{fancyhdr}		% réglage header footer
\usepackage{needspace}		% empêcher sauts de page mal placés
\usepackage{graphicx}		% pour inclure des graphiques
\usepackage{enumitem,cprotect}		% personnalise les listes d'items (nécessaire pour ol, al ...)
\usepackage{hyperref}		% Liens hypertexte
\usepackage{pstricks,pst-all,pst-node,pstricks-add,pst-math,pst-plot,pst-tree,pst-eucl} % pstricks
\usepackage[a4paper,includeheadfoot,top=2cm,left=3cm, bottom=2cm,right=3cm]{geometry} % marges etc.
\usepackage{comment}			% commentaires multilignes
\usepackage{amsmath,environ} % maths (matrices, etc.)
\usepackage{amssymb,makeidx}
\usepackage{bm}				% bold maths
\usepackage{tabularx}		% tableaux
\usepackage{colortbl}		% tableaux en couleur
\usepackage{fontawesome}		% Fontawesome
\usepackage{environ}			% environment with command
\usepackage{fp}				% calculs pour ps-tricks
\usepackage{multido}			% pour ps tricks
\usepackage[np]{numprint}	% formattage nombre
\usepackage{tikz,tkz-tab} 			% package principal TikZ
\usepackage{pgfplots}   % axes
\usepackage{mathrsfs}    % cursives
\usepackage{calc}			% calcul taille boites
\usepackage[scaled=0.875]{helvet} % font sans serif
\usepackage{svg} % svg
\usepackage{scrextend} % local margin
\usepackage{scratch} %scratch
\usepackage{multicol} % colonnes
%\usepackage{infix-RPN,pst-func} % formule en notation polanaise inversée
\usepackage{listings}

%================================================================================================================================
%
% Réglages de base
%
%================================================================================================================================

\lstset{
language=Python,   % R code
literate=
{á}{{\'a}}1
{à}{{\`a}}1
{ã}{{\~a}}1
{é}{{\'e}}1
{è}{{\`e}}1
{ê}{{\^e}}1
{í}{{\'i}}1
{ó}{{\'o}}1
{õ}{{\~o}}1
{ú}{{\'u}}1
{ü}{{\"u}}1
{ç}{{\c{c}}}1
{~}{{ }}1
}


\definecolor{codegreen}{rgb}{0,0.6,0}
\definecolor{codegray}{rgb}{0.5,0.5,0.5}
\definecolor{codepurple}{rgb}{0.58,0,0.82}
\definecolor{backcolour}{rgb}{0.95,0.95,0.92}

\lstdefinestyle{mystyle}{
    backgroundcolor=\color{backcolour},   
    commentstyle=\color{codegreen},
    keywordstyle=\color{magenta},
    numberstyle=\tiny\color{codegray},
    stringstyle=\color{codepurple},
    basicstyle=\ttfamily\footnotesize,
    breakatwhitespace=false,         
    breaklines=true,                 
    captionpos=b,                    
    keepspaces=true,                 
    numbers=left,                    
xleftmargin=2em,
framexleftmargin=2em,            
    showspaces=false,                
    showstringspaces=false,
    showtabs=false,                  
    tabsize=2,
    upquote=true
}

\lstset{style=mystyle}


\lstset{style=mystyle}
\newcommand{\imgdir}{C:/laragon/www/newmc/assets/imgsvg/}
\newcommand{\imgsvgdir}{C:/laragon/www/newmc/assets/imgsvg/}

\definecolor{mcgris}{RGB}{220, 220, 220}% ancien~; pour compatibilité
\definecolor{mcbleu}{RGB}{52, 152, 219}
\definecolor{mcvert}{RGB}{125, 194, 70}
\definecolor{mcmauve}{RGB}{154, 0, 215}
\definecolor{mcorange}{RGB}{255, 96, 0}
\definecolor{mcturquoise}{RGB}{0, 153, 153}
\definecolor{mcrouge}{RGB}{255, 0, 0}
\definecolor{mclightvert}{RGB}{205, 234, 190}

\definecolor{gris}{RGB}{220, 220, 220}
\definecolor{bleu}{RGB}{52, 152, 219}
\definecolor{vert}{RGB}{125, 194, 70}
\definecolor{mauve}{RGB}{154, 0, 215}
\definecolor{orange}{RGB}{255, 96, 0}
\definecolor{turquoise}{RGB}{0, 153, 153}
\definecolor{rouge}{RGB}{255, 0, 0}
\definecolor{lightvert}{RGB}{205, 234, 190}
\setitemize[0]{label=\color{lightvert}  $\bullet$}

\pagestyle{fancy}
\renewcommand{\headrulewidth}{0.2pt}
\fancyhead[L]{maths-cours.fr}
\fancyhead[R]{\thepage}
\renewcommand{\footrulewidth}{0.2pt}
\fancyfoot[C]{}

\newcolumntype{C}{>{\centering\arraybackslash}X}
\newcolumntype{s}{>{\hsize=.35\hsize\arraybackslash}X}

\setlength{\parindent}{0pt}		 
\setlength{\parskip}{3mm}
\setlength{\headheight}{1cm}

\def\ebook{ebook}
\def\book{book}
\def\web{web}
\def\type{web}

\newcommand{\vect}[1]{\overrightarrow{\,\mathstrut#1\,}}

\def\Oij{$\left(\text{O}~;~\vect{\imath},~\vect{\jmath}\right)$}
\def\Oijk{$\left(\text{O}~;~\vect{\imath},~\vect{\jmath},~\vect{k}\right)$}
\def\Ouv{$\left(\text{O}~;~\vect{u},~\vect{v}\right)$}

\hypersetup{breaklinks=true, colorlinks = true, linkcolor = OliveGreen, urlcolor = OliveGreen, citecolor = OliveGreen, pdfauthor={Didier BONNEL - https://www.maths-cours.fr} } % supprime les bordures autour des liens

\renewcommand{\arg}[0]{\text{arg}}

\everymath{\displaystyle}

%================================================================================================================================
%
% Macros - Commandes
%
%================================================================================================================================

\newcommand\meta[2]{    			% Utilisé pour créer le post HTML.
	\def\titre{titre}
	\def\url{url}
	\def\arg{#1}
	\ifx\titre\arg
		\newcommand\maintitle{#2}
		\fancyhead[L]{#2}
		{\Large\sffamily \MakeUppercase{#2}}
		\vspace{1mm}\textcolor{mcvert}{\hrule}
	\fi 
	\ifx\url\arg
		\fancyfoot[L]{\href{https://www.maths-cours.fr#2}{\black \footnotesize{https://www.maths-cours.fr#2}}}
	\fi 
}


\newcommand\TitreC[1]{    		% Titre centré
     \needspace{3\baselineskip}
     \begin{center}\textbf{#1}\end{center}
}

\newcommand\newpar{    		% paragraphe
     \par
}

\newcommand\nosp {    		% commande vide (pas d'espace)
}
\newcommand{\id}[1]{} %ignore

\newcommand\boite[2]{				% Boite simple sans titre
	\vspace{5mm}
	\setlength{\fboxrule}{0.2mm}
	\setlength{\fboxsep}{5mm}	
	\fcolorbox{#1}{#1!3}{\makebox[\linewidth-2\fboxrule-2\fboxsep]{
  		\begin{minipage}[t]{\linewidth-2\fboxrule-4\fboxsep}\setlength{\parskip}{3mm}
  			 #2
  		\end{minipage}
	}}
	\vspace{5mm}
}

\newcommand\CBox[4]{				% Boites
	\vspace{5mm}
	\setlength{\fboxrule}{0.2mm}
	\setlength{\fboxsep}{5mm}
	
	\fcolorbox{#1}{#1!3}{\makebox[\linewidth-2\fboxrule-2\fboxsep]{
		\begin{minipage}[t]{1cm}\setlength{\parskip}{3mm}
	  		\textcolor{#1}{\LARGE{#2}}    
 	 	\end{minipage}  
  		\begin{minipage}[t]{\linewidth-2\fboxrule-4\fboxsep}\setlength{\parskip}{3mm}
			\raisebox{1.2mm}{\normalsize\sffamily{\textcolor{#1}{#3}}}						
  			 #4
  		\end{minipage}
	}}
	\vspace{5mm}
}

\newcommand\cadre[3]{				% Boites convertible html
	\par
	\vspace{2mm}
	\setlength{\fboxrule}{0.1mm}
	\setlength{\fboxsep}{5mm}
	\fcolorbox{#1}{white}{\makebox[\linewidth-2\fboxrule-2\fboxsep]{
  		\begin{minipage}[t]{\linewidth-2\fboxrule-4\fboxsep}\setlength{\parskip}{3mm}
			\raisebox{-2.5mm}{\sffamily \small{\textcolor{#1}{\MakeUppercase{#2}}}}		
			\par		
  			 #3
 	 		\end{minipage}
	}}
		\vspace{2mm}
	\par
}

\newcommand\bloc[3]{				% Boites convertible html sans bordure
     \needspace{2\baselineskip}
     {\sffamily \small{\textcolor{#1}{\MakeUppercase{#2}}}}    
		\par		
  			 #3
		\par
}

\newcommand\CHelp[1]{
     \CBox{Plum}{\faInfoCircle}{À RETENIR}{#1}
}

\newcommand\CUp[1]{
     \CBox{NavyBlue}{\faThumbsOUp}{EN PRATIQUE}{#1}
}

\newcommand\CInfo[1]{
     \CBox{Sepia}{\faArrowCircleRight}{REMARQUE}{#1}
}

\newcommand\CRedac[1]{
     \CBox{PineGreen}{\faEdit}{BIEN R\'EDIGER}{#1}
}

\newcommand\CError[1]{
     \CBox{Red}{\faExclamationTriangle}{ATTENTION}{#1}
}

\newcommand\TitreExo[2]{
\needspace{4\baselineskip}
 {\sffamily\large EXERCICE #1\ (\emph{#2 points})}
\vspace{5mm}
}

\newcommand\img[2]{
          \includegraphics[width=#2\paperwidth]{\imgdir#1}
}

\newcommand\imgsvg[2]{
       \begin{center}   \includegraphics[width=#2\paperwidth]{\imgsvgdir#1} \end{center}
}


\newcommand\Lien[2]{
     \href{#1}{#2 \tiny \faExternalLink}
}
\newcommand\mcLien[2]{
     \href{https~://www.maths-cours.fr/#1}{#2 \tiny \faExternalLink}
}

\newcommand{\euro}{\eurologo{}}

%================================================================================================================================
%
% Macros - Environement
%
%================================================================================================================================

\newenvironment{tex}{ %
}
{%
}

\newenvironment{indente}{ %
	\setlength\parindent{10mm}
}

{
	\setlength\parindent{0mm}
}

\newenvironment{corrige}{%
     \needspace{3\baselineskip}
     \medskip
     \textbf{\textsc{Corrigé}}
     \medskip
}
{
}

\newenvironment{extern}{%
     \begin{center}
     }
     {
     \end{center}
}

\NewEnviron{code}{%
	\par
     \boite{gray}{\texttt{%
     \BODY
     }}
     \par
}

\newenvironment{vbloc}{% boite sans cadre empeche saut de page
     \begin{minipage}[t]{\linewidth}
     }
     {
     \end{minipage}
}
\NewEnviron{h2}{%
    \needspace{3\baselineskip}
    \vspace{0.6cm}
	\noindent \MakeUppercase{\sffamily \large \BODY}
	\vspace{1mm}\textcolor{mcgris}{\hrule}\vspace{0.4cm}
	\par
}{}

\NewEnviron{h3}{%
    \needspace{3\baselineskip}
	\vspace{5mm}
	\textsc{\BODY}
	\par
}

\NewEnviron{margeneg}{ %
\begin{addmargin}[-1cm]{0cm}
\BODY
\end{addmargin}
}

\NewEnviron{html}{%
}

\begin{document}
\meta{url}{/exercices/famille-de-fonctions-tableaux-de-variations/}
\meta{pid}{11399}
\meta{titre}{Famille de fonctions - Tableaux de variations}
\meta{type}{exercices}
%
Soit $ m $ un nombre réel donné et $ f_{ m } $ la fonction définie sur  $  \mathbb{R}  $ par~:
\[
f_{ m }  (x) = \frac{ x^2 +m }{ x^2 +1 }
\]
\begin{enumerate}
     \item
     Justifier que la fonction $ f_{ m }  $ est bien définie sur $  \mathbb{R} . $
     \item
     Étudier la parité de la fonction $ f_{ m } . $
     \item
     Calculer $ f'_{ m }  (x)  $ pour tout réel $ x $.
     \item
     \begin{enumerate}[label=\alph*.]
          \item
          Dans cette question, on suppose $ m < 1. $\\
          Dresser le tableau de variations de la fonction $ f_{ m } . $
          \item
          Même question si $ m > 1. $
          \item
          Que peut-on dire de la fonction $ f_{ 1 }$  (obtenue pour $ m=1 $ ) ~?
     \end{enumerate}
\end{enumerate}
\begin{corrige}
     \begin{enumerate}
          \item
          Pour montrer que la fonction $ f_{ m }  $ est définie sur $ \mathbb{R}$, il suffit de montrer que son dénominateur ne s'annule pas sur $  \mathbb{R} . $
          \newpar
          Or, pour tout réel $x$, $ x^2  \geqslant 0 $ donc  $ x^2 +1 \geqslant 1. $ \\
          $ x^2 + 1 $ n'est  donc jamais nul sur $\mathbb{R}.$
          \item
          $ f_{ m }  ( - x) = \frac{  ( - x) ^2 +m }{  ( - x) ^2 +1 } = \frac{ x^2 +m }{ x^2 +1 } =f_{ m }  (x) .$
          \newpar
          La fonction $ f_{ m }  $ est donc paire quelle que soit la valeur de  $m.$
          \item
          Posons~:  $ u (x) =x^2 +m $ et $ v (x) =x^2 +1 $.
          \newpar
          Alors~:  $ u'  (x) =2x $ et $ v'  (x) =2x $
          \newpar
          Par conséquent~:
          \newpar
          $ f' _{ m }  (x) = \frac{ u'  (x) v (x)  - u (x) v'  (x)  }{ v (x) ^2  }  $
          \newpar
          $\phantom{ f' _{ m }  (x) } =  \frac{ 2x (x^2 +1)  - 2x (x^2 +m)  }{  \left( x^2 +1 \right) ^2  } $
          \newpar
          $\phantom{ f' _{ m }  (x) } =  \frac{ 2x ( 1- m)  }{  \left( x^2 +1 \right) ^2  } .$
          \item
          \begin{enumerate}[label=\alph*.]
               \item
               $ \left( x^2 +1 \right) ^2  $ est strictement positif quel que soit le réel  $ x.$
               \newpar
               Si $ m < 1$, alors $ 1 - m $ est strictement positif~;  $ f' _{ m }  (x)  $ est donc du signe de  $ x $.
               \newpar
               $ f_{ m }  $ admet donc un maximum pour $ x=0 $ ; ce maximum est égal à~:
               \newpar
               $ f_{ m }  (0) = \frac{ 0^2 +m }{ 0^2 +1 } =m $~
               \newpar
               D'où le tableau de variations~:
               \newpar
               \begin{extern}%width="400" alt="tableau de variations de la fonction f"
                    \begin{tikzpicture}[scale=0.875]
                         % Styles
                         \tikzstyle{cadre}=[thin]
                         \tikzstyle{fleche}=[->,>=latex,thin]
                         \tikzstyle{nondefini}=[lightgray]
                         % Dimensions Modifiables
                         \def\Lrg{1.5}
                         \def\HtX{1}
                         \def\HtY{0.5}
                         % Dimensions Calculées
                         \def\lignex{-0.5*\HtX}
                         \def\lignef{-1.5*\HtX}
                         \def\separateur{-0.5*\Lrg}
                         % Largeur du tableau
                         \def\gauche{-1.5*\Lrg}
                         \def\droite{5.5*\Lrg}
                         % Hauteur du tableau
                         \def\haut{0.5*\HtX}
                         \def\bas{-2.5*\HtX-2*\HtY}
                         % Ligne de l'abscisse : x
                         \node at (-1*\Lrg,0) {$x$};
                         \node at (0*\Lrg,0) {$ -  \infty $};
                         \node at (2.5*\Lrg,0) {$0$};
                         \node at (5*\Lrg,0) {$+ \infty $};
                         % Ligne de la dérivée : f'(x)
                         \node at (-1*\Lrg,-1*\HtX) {$f'_{ m }(x)$};
                         \node at (0*\Lrg,-1*\HtX) {$\ $};
                         \node at (1*\Lrg,-1*\HtX) {$-$};
                         \node at (2.5*\Lrg,-1*\HtX) {$0$};
                         \node at (4*\Lrg,-1*\HtX) {$+$};
                         \node at (5*\Lrg,-1*\HtX) {$$};
                         % Ligne de la fonction : f(x)
                         \node  at (-1*\Lrg,{-2*\HtX+(-1)*\HtY}) {$f_{ m }(x)$};
                         \node (f1) at (0*\Lrg,{-2*\HtX+(0)*\HtY}) {$$};
                         \node (f2) at (2.5*\Lrg,{-2*\HtX+(-2)*\HtY}) {$m$};
                         \node (f3) at (5*\Lrg,{-2*\HtX+(0)*\HtY}) {$$};
                         % Flèches
                         \draw[fleche] (f1) -- (f2);
                         \draw[fleche] (f2) -- (f3);
                         % Encadrement
                         \draw[cadre] (\separateur,\haut) -- (\separateur,\bas);
                         \draw[cadre] (\gauche,\haut) rectangle  (\droite,\bas);
                         \draw[cadre] (\gauche,\lignex) -- (\droite,\lignex);
                         \draw[cadre] (\gauche,\lignef) -- (\droite,\lignef);
                    \end{tikzpicture}
               \end{extern}
               \item
               Par contre, si $ m  > 1$, $ 1 - m $ est strictement négatif~;  $ f' _{ m }  (x)  $ est donc du signe opposé à $ x $.
               \newpar
               $ f_{ m }  $ admet donc un maximum pour $ x=0 $~; le tableau de variations est alors~:
               \newpar
               \begin{extern}%width="400" alt="tableau de variations de la fonction f"
                    \begin{tikzpicture}[scale=0.875]
                         % Styles
                         \tikzstyle{cadre}=[thin]
                         \tikzstyle{fleche}=[->,>=latex,thin]
                         \tikzstyle{nondefini}=[lightgray]
                         % Dimensions Modifiables
                         \def\Lrg{1.5}
                         \def\HtX{1}
                         \def\HtY{0.5}
                         % Dimensions Calculées
                         \def\lignex{-0.5*\HtX}
                         \def\lignef{-1.5*\HtX}
                         \def\separateur{-0.5*\Lrg}
                         % Largeur du tableau
                         \def\gauche{-1.5*\Lrg}
                         \def\droite{5.5*\Lrg}
                         % Hauteur du tableau
                         \def\haut{0.5*\HtX}
                         \def\bas{-2.5*\HtX-2*\HtY}
                         % Ligne de l'abscisse : x
                         \node at (-1*\Lrg,0) {$x$};
                         \node at (0*\Lrg,0) {$ -  \infty $};
                         \node at (2.5*\Lrg,0) {$0$};
                         \node at (5*\Lrg,0) {$+ \infty $};
                         % Ligne de la dérivée : f'(x)
                         \node at (-1*\Lrg,-1*\HtX) {$f'_{ m }(x)$};
                         \node at (0*\Lrg,-1*\HtX) {$\ $};
                         \node at (1*\Lrg,-1*\HtX) {$+$};
                         \node at (2.5*\Lrg,-1*\HtX) {$0$};
                         \node at (4*\Lrg,-1*\HtX) {$-$};
                         \node at (5*\Lrg,-1*\HtX) {$$};
                         % Ligne de la fonction : f(x)
                         \node  at (-1*\Lrg,{-2*\HtX+(-1)*\HtY}) {$f_{ m }(x)$};
                         \node (f1) at (0*\Lrg,{-2*\HtX+(-2)*\HtY}) {$$};
                         \node (f2) at (2.5*\Lrg,{-2*\HtX+(0)*\HtY}) {$m$};
                         \node (f3) at (5*\Lrg,{-2*\HtX+(-2)*\HtY}) {$$};
                         % Flèches
                         \draw[fleche] (f1) -- (f2);
                         \draw[fleche] (f2) -- (f3);
                         % Encadrement
                         \draw[cadre] (\separateur,\haut) -- (\separateur,\bas);
                         \draw[cadre] (\gauche,\haut) rectangle  (\droite,\bas);
                         \draw[cadre] (\gauche,\lignex) -- (\droite,\lignex);
                         \draw[cadre] (\gauche,\lignef) -- (\droite,\lignef);
                    \end{tikzpicture}
               \end{extern}
               \item
               Pour $ m=1 $~:
               \newpar
               $ f_{ 1 }  (x) = \frac{ x^2 +1 }{ x^2 +1 } =1 $~
               \newpar
               La fonction $ f_{ 1 }  $ est donc constante et égale à 1 sur $  \mathbb{R} . $
          \end{enumerate}
     \end{enumerate}
\end{corrige}

\end{document}
µ
\documentclass[a4paper]{article}

%================================================================================================================================
%
% Packages
%
%================================================================================================================================

\usepackage[T1]{fontenc} 	% pour caractères accentués
\usepackage[utf8]{inputenc}  % encodage utf8
\usepackage[french]{babel}	% langue : français
\usepackage{fourier}			% caractères plus lisibles
\usepackage[dvipsnames]{xcolor} % couleurs
\usepackage{fancyhdr}		% réglage header footer
\usepackage{needspace}		% empêcher sauts de page mal placés
\usepackage{graphicx}		% pour inclure des graphiques
\usepackage{enumitem,cprotect}		% personnalise les listes d'items (nécessaire pour ol, al ...)
\usepackage{hyperref}		% Liens hypertexte
\usepackage{pstricks,pst-all,pst-node,pstricks-add,pst-math,pst-plot,pst-tree,pst-eucl} % pstricks
\usepackage[a4paper,includeheadfoot,top=2cm,left=3cm, bottom=2cm,right=3cm]{geometry} % marges etc.
\usepackage{comment}			% commentaires multilignes
\usepackage{amsmath,environ} % maths (matrices, etc.)
\usepackage{amssymb,makeidx}
\usepackage{bm}				% bold maths
\usepackage{tabularx}		% tableaux
\usepackage{colortbl}		% tableaux en couleur
\usepackage{fontawesome}		% Fontawesome
\usepackage{environ}			% environment with command
\usepackage{fp}				% calculs pour ps-tricks
\usepackage{multido}			% pour ps tricks
\usepackage[np]{numprint}	% formattage nombre
\usepackage{tikz,tkz-tab} 			% package principal TikZ
\usepackage{pgfplots}   % axes
\usepackage{mathrsfs}    % cursives
\usepackage{calc}			% calcul taille boites
\usepackage[scaled=0.875]{helvet} % font sans serif
\usepackage{svg} % svg
\usepackage{scrextend} % local margin
\usepackage{scratch} %scratch
\usepackage{multicol} % colonnes
%\usepackage{infix-RPN,pst-func} % formule en notation polanaise inversée
\usepackage{listings}

%================================================================================================================================
%
% Réglages de base
%
%================================================================================================================================

\lstset{
language=Python,   % R code
literate=
{á}{{\'a}}1
{à}{{\`a}}1
{ã}{{\~a}}1
{é}{{\'e}}1
{è}{{\`e}}1
{ê}{{\^e}}1
{í}{{\'i}}1
{ó}{{\'o}}1
{õ}{{\~o}}1
{ú}{{\'u}}1
{ü}{{\"u}}1
{ç}{{\c{c}}}1
{~}{{ }}1
}


\definecolor{codegreen}{rgb}{0,0.6,0}
\definecolor{codegray}{rgb}{0.5,0.5,0.5}
\definecolor{codepurple}{rgb}{0.58,0,0.82}
\definecolor{backcolour}{rgb}{0.95,0.95,0.92}

\lstdefinestyle{mystyle}{
    backgroundcolor=\color{backcolour},   
    commentstyle=\color{codegreen},
    keywordstyle=\color{magenta},
    numberstyle=\tiny\color{codegray},
    stringstyle=\color{codepurple},
    basicstyle=\ttfamily\footnotesize,
    breakatwhitespace=false,         
    breaklines=true,                 
    captionpos=b,                    
    keepspaces=true,                 
    numbers=left,                    
xleftmargin=2em,
framexleftmargin=2em,            
    showspaces=false,                
    showstringspaces=false,
    showtabs=false,                  
    tabsize=2,
    upquote=true
}

\lstset{style=mystyle}


\lstset{style=mystyle}
\newcommand{\imgdir}{C:/laragon/www/newmc/assets/imgsvg/}
\newcommand{\imgsvgdir}{C:/laragon/www/newmc/assets/imgsvg/}

\definecolor{mcgris}{RGB}{220, 220, 220}% ancien~; pour compatibilité
\definecolor{mcbleu}{RGB}{52, 152, 219}
\definecolor{mcvert}{RGB}{125, 194, 70}
\definecolor{mcmauve}{RGB}{154, 0, 215}
\definecolor{mcorange}{RGB}{255, 96, 0}
\definecolor{mcturquoise}{RGB}{0, 153, 153}
\definecolor{mcrouge}{RGB}{255, 0, 0}
\definecolor{mclightvert}{RGB}{205, 234, 190}

\definecolor{gris}{RGB}{220, 220, 220}
\definecolor{bleu}{RGB}{52, 152, 219}
\definecolor{vert}{RGB}{125, 194, 70}
\definecolor{mauve}{RGB}{154, 0, 215}
\definecolor{orange}{RGB}{255, 96, 0}
\definecolor{turquoise}{RGB}{0, 153, 153}
\definecolor{rouge}{RGB}{255, 0, 0}
\definecolor{lightvert}{RGB}{205, 234, 190}
\setitemize[0]{label=\color{lightvert}  $\bullet$}

\pagestyle{fancy}
\renewcommand{\headrulewidth}{0.2pt}
\fancyhead[L]{maths-cours.fr}
\fancyhead[R]{\thepage}
\renewcommand{\footrulewidth}{0.2pt}
\fancyfoot[C]{}

\newcolumntype{C}{>{\centering\arraybackslash}X}
\newcolumntype{s}{>{\hsize=.35\hsize\arraybackslash}X}

\setlength{\parindent}{0pt}		 
\setlength{\parskip}{3mm}
\setlength{\headheight}{1cm}

\def\ebook{ebook}
\def\book{book}
\def\web{web}
\def\type{web}

\newcommand{\vect}[1]{\overrightarrow{\,\mathstrut#1\,}}

\def\Oij{$\left(\text{O}~;~\vect{\imath},~\vect{\jmath}\right)$}
\def\Oijk{$\left(\text{O}~;~\vect{\imath},~\vect{\jmath},~\vect{k}\right)$}
\def\Ouv{$\left(\text{O}~;~\vect{u},~\vect{v}\right)$}

\hypersetup{breaklinks=true, colorlinks = true, linkcolor = OliveGreen, urlcolor = OliveGreen, citecolor = OliveGreen, pdfauthor={Didier BONNEL - https://www.maths-cours.fr} } % supprime les bordures autour des liens

\renewcommand{\arg}[0]{\text{arg}}

\everymath{\displaystyle}

%================================================================================================================================
%
% Macros - Commandes
%
%================================================================================================================================

\newcommand\meta[2]{    			% Utilisé pour créer le post HTML.
	\def\titre{titre}
	\def\url{url}
	\def\arg{#1}
	\ifx\titre\arg
		\newcommand\maintitle{#2}
		\fancyhead[L]{#2}
		{\Large\sffamily \MakeUppercase{#2}}
		\vspace{1mm}\textcolor{mcvert}{\hrule}
	\fi 
	\ifx\url\arg
		\fancyfoot[L]{\href{https://www.maths-cours.fr#2}{\black \footnotesize{https://www.maths-cours.fr#2}}}
	\fi 
}


\newcommand\TitreC[1]{    		% Titre centré
     \needspace{3\baselineskip}
     \begin{center}\textbf{#1}\end{center}
}

\newcommand\newpar{    		% paragraphe
     \par
}

\newcommand\nosp {    		% commande vide (pas d'espace)
}
\newcommand{\id}[1]{} %ignore

\newcommand\boite[2]{				% Boite simple sans titre
	\vspace{5mm}
	\setlength{\fboxrule}{0.2mm}
	\setlength{\fboxsep}{5mm}	
	\fcolorbox{#1}{#1!3}{\makebox[\linewidth-2\fboxrule-2\fboxsep]{
  		\begin{minipage}[t]{\linewidth-2\fboxrule-4\fboxsep}\setlength{\parskip}{3mm}
  			 #2
  		\end{minipage}
	}}
	\vspace{5mm}
}

\newcommand\CBox[4]{				% Boites
	\vspace{5mm}
	\setlength{\fboxrule}{0.2mm}
	\setlength{\fboxsep}{5mm}
	
	\fcolorbox{#1}{#1!3}{\makebox[\linewidth-2\fboxrule-2\fboxsep]{
		\begin{minipage}[t]{1cm}\setlength{\parskip}{3mm}
	  		\textcolor{#1}{\LARGE{#2}}    
 	 	\end{minipage}  
  		\begin{minipage}[t]{\linewidth-2\fboxrule-4\fboxsep}\setlength{\parskip}{3mm}
			\raisebox{1.2mm}{\normalsize\sffamily{\textcolor{#1}{#3}}}						
  			 #4
  		\end{minipage}
	}}
	\vspace{5mm}
}

\newcommand\cadre[3]{				% Boites convertible html
	\par
	\vspace{2mm}
	\setlength{\fboxrule}{0.1mm}
	\setlength{\fboxsep}{5mm}
	\fcolorbox{#1}{white}{\makebox[\linewidth-2\fboxrule-2\fboxsep]{
  		\begin{minipage}[t]{\linewidth-2\fboxrule-4\fboxsep}\setlength{\parskip}{3mm}
			\raisebox{-2.5mm}{\sffamily \small{\textcolor{#1}{\MakeUppercase{#2}}}}		
			\par		
  			 #3
 	 		\end{minipage}
	}}
		\vspace{2mm}
	\par
}

\newcommand\bloc[3]{				% Boites convertible html sans bordure
     \needspace{2\baselineskip}
     {\sffamily \small{\textcolor{#1}{\MakeUppercase{#2}}}}    
		\par		
  			 #3
		\par
}

\newcommand\CHelp[1]{
     \CBox{Plum}{\faInfoCircle}{À RETENIR}{#1}
}

\newcommand\CUp[1]{
     \CBox{NavyBlue}{\faThumbsOUp}{EN PRATIQUE}{#1}
}

\newcommand\CInfo[1]{
     \CBox{Sepia}{\faArrowCircleRight}{REMARQUE}{#1}
}

\newcommand\CRedac[1]{
     \CBox{PineGreen}{\faEdit}{BIEN R\'EDIGER}{#1}
}

\newcommand\CError[1]{
     \CBox{Red}{\faExclamationTriangle}{ATTENTION}{#1}
}

\newcommand\TitreExo[2]{
\needspace{4\baselineskip}
 {\sffamily\large EXERCICE #1\ (\emph{#2 points})}
\vspace{5mm}
}

\newcommand\img[2]{
          \includegraphics[width=#2\paperwidth]{\imgdir#1}
}

\newcommand\imgsvg[2]{
       \begin{center}   \includegraphics[width=#2\paperwidth]{\imgsvgdir#1} \end{center}
}


\newcommand\Lien[2]{
     \href{#1}{#2 \tiny \faExternalLink}
}
\newcommand\mcLien[2]{
     \href{https~://www.maths-cours.fr/#1}{#2 \tiny \faExternalLink}
}

\newcommand{\euro}{\eurologo{}}

%================================================================================================================================
%
% Macros - Environement
%
%================================================================================================================================

\newenvironment{tex}{ %
}
{%
}

\newenvironment{indente}{ %
	\setlength\parindent{10mm}
}

{
	\setlength\parindent{0mm}
}

\newenvironment{corrige}{%
     \needspace{3\baselineskip}
     \medskip
     \textbf{\textsc{Corrigé}}
     \medskip
}
{
}

\newenvironment{extern}{%
     \begin{center}
     }
     {
     \end{center}
}

\NewEnviron{code}{%
	\par
     \boite{gray}{\texttt{%
     \BODY
     }}
     \par
}

\newenvironment{vbloc}{% boite sans cadre empeche saut de page
     \begin{minipage}[t]{\linewidth}
     }
     {
     \end{minipage}
}
\NewEnviron{h2}{%
    \needspace{3\baselineskip}
    \vspace{0.6cm}
	\noindent \MakeUppercase{\sffamily \large \BODY}
	\vspace{1mm}\textcolor{mcgris}{\hrule}\vspace{0.4cm}
	\par
}{}

\NewEnviron{h3}{%
    \needspace{3\baselineskip}
	\vspace{5mm}
	\textsc{\BODY}
	\par
}

\NewEnviron{margeneg}{ %
\begin{addmargin}[-1cm]{0cm}
\BODY
\end{addmargin}
}

\NewEnviron{html}{%
}

\begin{document}
\meta{url}{/exercices/etude-dune-fonction-a-laide-dune-fonction-annexe/}
\meta{pid}{11404}
\meta{titre}{Étude d'une fonction à l'aide d'une fonction annexe}
\meta{type}{exercices}
%
\begin{center}
     \begin{h3} Partie A \end{h3}
\end{center}
Soit la fonction $ g $ définie sur $  \mathbb{R}  $ par~:
\[
g (x) =2x^{ 3 } +2x^2  - 1.
\]
\begin{enumerate}
     \item
     Étudier les variations de la fonction $ g $ sur $  \mathbb{R} . $
     \item
     Calculer $ g (0)  $ et $ g (1).  $\\
     On admet que l'équation $ g (x) =0 $ admet une unique solution  $ x_{ 0 }  $ sur $  \mathbb{R}$.  \\
     Justifier que $ x_{ 0 }  \in  \left] 0 ; 1 \right[$.
     \item
     Déterminer le signe de $ g (x)  $ sur $  \mathbb{R}$.
     \item
     On considère le programme Python ci-dessous~:
\begin{lstlisting}[language=Python]
def g(x) :
    return 2*x**3 + 2*x**2 - 1
def solution() :
    x = 0
    y = g(x)
    while y < 0 :
        x = x + 0.01
        y = g(x)
    return x
\end{lstlisting}
L'appel de la fonction \texttt{solution() } définie ci-dessus retourne \texttt{0.57}.
\newpar
donner un encadrement d'amplitude 0,01 de  $ x_{ 0 }. $
\end{enumerate}
\begin{center}
     \begin{h3} Partie b \end{h3}
\end{center}
On considère la fonction $ f $ définie sur  $  \mathbb{R}  \backslash  \{ 0 \}  $ par~: \\
\[
f (x) = \frac{ x^{ 3 } +2x^2 +1 }{ x } .
\]
\begin{enumerate}
     \item
     Montrer que pour tout réel $ x $ non nul~:
     \[
     f'  (x)= \frac{ g (x)  }{ x^2  } .
     \]
     \item
     Dresser le tableau de variations de $ f $ sur $  \mathbb{R} . $  (On ne cherchera pas à déterminer la valeur de l’extremum de cette fonction.)
\end{enumerate}
%
\begin{corrige}
     \begin{center}
          \begin{h3} Partie A \end{h3}
     \end{center}
     \begin{enumerate}
          \item
          La fonction $ g $ est une fonction polynôme, donc, elle est dérivable sur  $  \mathbb{R}  $ et~:
          \newpar
          $ g'  (x) =2 \times 3x^2+2 \times 2x=6x^2 +4x$\nosp$=2x (3x+2).  $
          \newpar
          $ g'  $ possède donc 2 racines~:  $ x_{ 1 } =0 $ et $ x_{ 2 } =  - \frac{ 2 }{ 3 } . $
          \newpar
          Le coefficient du terme du second degré est positif donc $ g' $ est négative entre $   - \frac{ 2 }{ 3 }  $ et $ 0 $ et est positive à l'extérieur de ces racines.
          \newpar
          $ g (0) = - 1 $
          \newpar
          $ g  \left( -  \frac{ 2 }{ 3 }  \right)  =2 \times  \left(  -  \frac{ 2 }{ 3 }  \right) ^{3}$\nosp$ +2 \times  \left(  -  \frac{ 2 }{ 3 }  \right) ^2  - 1$\nosp$= -  \frac{ 19 }{ 27 } . $
          \newpar
          On peut alors dresser le tableau de variations  de  $ g $~:  \\
          \begin{center}
               \begin{extern}%width="450" alt="tableau de variations de la fonction"
                    \begin{tikzpicture}[scale=0.875]
                         % Styles
                         \tikzstyle{cadre}=[thin]
                         \tikzstyle{fleche}=[->,>=latex,thin]
                         \tikzstyle{nondefini}=[lightgray]
                         % Dimensions modifiables
                         \def\Lrg{1.5}
                         \def\HtX{1}
                         \def\HtY{0.5}
                         % Dimensions calculées
                         \def\lignex{-0.5*\HtX}
                         \def\lignef{-1.5*\HtX}
                         \def\separateur{-0.5*\Lrg}
                         % Largeur du tableau
                         \def\gauche{-1.5*\Lrg}
                         \def\droite{6.5*\Lrg}
                         % Hauteur du tableau
                         \def\haut{0.5*\HtX}
                         \def\bas{-2.5*\HtX-2*\HtY}
                         % Pointillés
                         \draw[dotted,black] (0*\Lrg,\lignex-0.1*\HtX) -- (0*\Lrg,\lignef+0.1*\HtX);
                         \draw[dotted,black] (0*\Lrg,\lignef-0.1*\HtX) -- (0*\Lrg,\bas+0.1*\HtX);
                         \draw[dotted,black] (2*\Lrg,\lignex-0.1*\HtX) -- (2*\Lrg,\lignef+0.1*\HtX);
                         \draw[dotted,black] (2*\Lrg,\lignef-0.1*\HtX) -- (2*\Lrg,\bas+0.1*\HtX);
                         \draw[dotted,black] (4*\Lrg,\lignex-0.1*\HtX) -- (4*\Lrg,\lignef+0.1*\HtX);
                         \draw[dotted,black] (4*\Lrg,\lignef-0.1*\HtX) -- (4*\Lrg,\bas+0.1*\HtX);
                         \draw[dotted,black] (6*\Lrg,\lignex-0.1*\HtX) -- (6*\Lrg,\lignef+0.1*\HtX);
                         \draw[dotted,black] (6*\Lrg,\lignef-0.1*\HtX) -- (6*\Lrg,\bas+0.1*\HtX);
                         % Ligne de l'abscisse~: x
                         \node at (-1*\Lrg,0) {$x$};
                         \node at (0*\Lrg,0) {$ -  \infty $};
                         \node at (2*\Lrg,0) {$ -  \frac{ 2 }{ 3 } $};
                         \node at (4*\Lrg,0) {$0$};
                         \node at (6*\Lrg,0) {$+ \infty $};
                         % Ligne de la dérivée~: f'(x)
                         \node at (-1*\Lrg,-1*\HtX) {$g'(x)$};
                         \node at (0*\Lrg,-1*\HtX) {$$};
                         \node at (1*\Lrg,-1*\HtX) {$+$};
                         \node at (2*\Lrg,-1*\HtX) {$0$};
                         \node at (3*\Lrg,-1*\HtX) {$-$};
                         \node at (4*\Lrg,-1*\HtX) {$0$};
                         \node at (5*\Lrg,-1*\HtX) {$+$};
                         \node at (6*\Lrg,-1*\HtX) {$$};
                         % Ligne de la fonction~: f(x)
                         \node  at (-1*\Lrg,{-2*\HtX+(-1)*\HtY}) {$g(x)$};
                         \node (f1) at (0*\Lrg,{-2*\HtX+(-2)*\HtY}) {$$};
                         \node (f2) at (2*\Lrg,{-2*\HtX+(0)*\HtY}) {$ -  \frac{ 19 }{ 27 } $};
                         \node (f3) at (4*\Lrg,{-2*\HtX+(-2)*\HtY}) {$-1$};
                         \node (f4) at (6*\Lrg,{-2*\HtX+(0)*\HtY}) {$$};
                         % Flèches
                         \draw[fleche] (f1) -- (f2);
                         \draw[fleche] (f2) -- (f3);
                         \draw[fleche] (f3) -- (f4);
                         % Encadrement
                         \draw[cadre] (\separateur,\haut) -- (\separateur,\bas);
                         \draw[cadre] (\gauche,\haut) rectangle  (\droite,\bas);
                         \draw[cadre] (\gauche,\lignex) -- (\droite,\lignex);
                         \draw[cadre] (\gauche,\lignef) -- (\droite,\lignef);
                    \end{tikzpicture}
               \end{extern}
          \end{center}
          \item
          On a déjà calculé $ g (0) = - 1 $ qui est strictement négatif. \\
          $ g (1) =3 $ est strictement positif.   \\
          $ g $ change de signe entre  $ 0 $  et  $ 1 $ donc s'annule pour un nombre $ x_{ 0 }  $ appartenant à l'intervalle  $  \left] 0 ; 1 \right[ . $
          \newpar
          \textbf{Remarque~: } Une démonstration plus rigoureuse nécessiterait l'emploi du théorème des valeurs intermédiaires qui n'est pas au programme de Première. Ici, seule une justification était demandée.
          \item
          D'après le tableau de variations, $ g $ est négative sur l'intervalle $  \left] -  \infty   ; 0 \right]  $. \\
          D'après la question précédente, $ g $ est également négative sur l'intervalle $  \left[ 0 ; x_{ 0 }  \right[  $ mais positive sur l'intervalle $  \left] x_{ 0 }  ; + \infty  \right[.$
          \newpar
          Le tableau de signes de $ g $ est donc le suivant~:  \\
          \begin{center}
               \begin{extern}%width="390" alt="Exemple tableau de signe 1"
                    \resizebox{8cm}{!}{
                         \begin{tikzpicture}[scale=0.875]
                              % Styles
                              \tikzstyle{cadre}=[thin]
                              \tikzstyle{fleche}=[->,>=latex,thin]
                              \tikzstyle{nondefini}=[lightgray]
                              % Dimensions modifiables
                              \def\Lrg{1.8}
                              \def\HtX{1.2}
                              \def\HtY{0.5}
                              % Dimensions calculées
                              \def\lignex{-0.5*\HtX}
                              \def\lignef{-1.5*\HtX}
                              \def\separateur{-0.5*\Lrg}
                              % Largeur du tableau
                              \def\gauche{-1.5*\Lrg}
                              \def\droite{4.5*\Lrg}
                              % Hauteur du tableau
                              \def\haut{0.5*\HtX}
                              \def\bas{-2.5*\HtX-2*\HtY}
                              % Pointillés
                              \draw[gray] (2*\Lrg,\lignex) -- (2*\Lrg,\lignef);
                              % Ligne de l'abscisse : x
                              \node at (-1*\Lrg,0) {$x$};
                              \node at (0*\Lrg,0) {$-\infty$};
                              \node at (2*\Lrg,0) {$x_{ 0 } $};
                              \node at (4*\Lrg,0) {$+\infty$};
                              % Ligne de la dérivée : f'(x)
                              \node at (-1*\Lrg,-1*\HtX) {$g (x) $};
                              \node at (0*\Lrg,-1*\HtX) {$ $};
                              \node at (1*\Lrg,-1*\HtX) {$-$};
                              \node at (2*\Lrg,-1*\HtX) {$0$};
                              \node at (3*\Lrg,-1*\HtX) {$+$};
                              \node at (4*\Lrg,-1*\HtX) {$ $};
                              % Ligne de la fonction : f(x)
                              % Encadrement
                              \draw[cadre] (\separateur,\haut) -- (\separateur, \lignef);
                              \draw[cadre] (\gauche,\haut) rectangle  (\droite, \lignef);
                              \draw[cadre] (\gauche,\lignex) -- (\droite,\lignex);
                         \end{tikzpicture}
                    }
               \end{extern}
          \end{center}
          \item
          La fonction  \og \texttt{solution () } \fg{}  calcule les valeurs de $ g (x)  $ pour $ x $ partant de  $ 0$ et augmentant par pas de $0,01.$
          \newpar
          La boucle \og  \texttt{while}  \fg{} s'arrête dès que $ g (x)   \geqslant 0$ et la fonction renvoie alors la valeur de la variable $ x. $  \\
          Ce nombre retourné est donc le plus petit nombre de deux décimales tel que  $ g (x)  \geqslant 0. $
          \newpar
          Ce nombre étant égal à $ 0,57 $ d'après l'énoncé, on a donc  $ g (0,57)  \geqslant 0 $ mais  $ g (0,56)  < 0. $
          \newpar
          Par conséquent~:  $ 0,56 < x_{ 0 }  \leqslant 0,57. $
     \end{enumerate}
     \begin{center}
          \begin{h3} Partie B \end{h3}
     \end{center}
     \begin{enumerate}
          \item
          Posons~:  $ u (x) =x^3 +2x^{ 2 } +1$ et $ v (x) =x $.
          \newpar
          Alors~:  $ u'  (x) =3x^2 +4x$ et $ v'  (x) =1 $
          \newpar
          Donc~:
          \newpar
          $ f'  (x) = \frac{ u'  (x) v (x)  - u (x) v'  (x)  }{ v (x) ^2  }  $
          \newpar
          $\phantom{ f'  (x) } =  \frac{ x (3x^2 +4x)  -  (x^{ 3 } +2x^2 +1)  }{  x^2  } $
          \newpar
          $\phantom{ f'  (x) } =  \frac{ 2x^{ 3 } +2x^2  - 1 }{  x^2  } $
          \newpar
          $\phantom{ f'  (x) } =  \frac{ g (x)  }{  x^2  } $
          \item
          $ x^2  $ est toujours strictement positif sur $  \mathbb{R} \backslash \{ 0 \}. $  \\
          $ f'  (x)  $ est donc du signe de  $ g (x)  $ qui est donné par le tableau de la question a.3. \\
          Toutefois, $ f $ n'est pas définie en $ 0. $
          \newpar
          On en déduit le tableau de variations de la fonction $ f $~:
          \begin{center}
               \begin{extern}%width="450" alt="tableau de variations de la fonction"
                    \begin{tikzpicture}[scale=0.875]
                         % Styles
                         \tikzstyle{cadre}=[thin]
                         \tikzstyle{fleche}=[->,>=latex,thin]
                         \tikzstyle{nondefini}=[lightgray]
                         % Dimensions modifiables
                         \def\Lrg{1.5}
                         \def\HtX{1}
                         \def\HtY{0.5}
                         % Dimensions calculées
                         \def\lignex{-0.5*\HtX}
                         \def\lignef{-1.5*\HtX}
                         \def\separateur{-0.5*\Lrg}
                         % Largeur du tableau
                         \def\gauche{-1.5*\Lrg}
                         \def\droite{6.5*\Lrg}
                         % Hauteur du tableau
                         \def\haut{0.5*\HtX}
                         \def\bas{-2.5*\HtX-2*\HtY}
                         % Pointillés
                         \draw[dotted,black] (4*\Lrg,\lignex-0.1*\HtX) -- (4*\Lrg,\lignef+0.1*\HtX);
                         \draw[dotted,black] (4*\Lrg,\lignef-0.1*\HtX) -- (4*\Lrg,\bas+0.1*\HtX);
                         % Ligne de l'abscisse~: x
                         \node at (-1*\Lrg,0) {$x$};
                         \node at (0*\Lrg,0) {$ -  \infty $};
                         \node at (2*\Lrg,0) {$0$};
                         \node at (4*\Lrg,0) {$x_0$};
                         \node at (6*\Lrg,0) {$+ \infty $};
                         % Ligne de la dérivée~: f'(x)
                         \node at (-1*\Lrg,-1*\HtX) {$f'(x)$};
                         \node at (0*\Lrg,-1*\HtX) {$$};
                         \node at (1*\Lrg,-1*\HtX) {$-$};
                         \node at (2*\Lrg,-1*\HtX) {$$};
                         \node at (3*\Lrg,-1*\HtX) {$-$};
                         \node at (4*\Lrg,-1*\HtX) {$0$};
                         \node at (5*\Lrg,-1*\HtX) {$+$};
                         \node at (6*\Lrg,-1*\HtX) {$$};
                         % Ligne de la fonction~: f(x)
                         \node  at (-1*\Lrg,{-2*\HtX+(-1)*\HtY}) {$f(x)$};
                         \node (f1) at (0*\Lrg,{-2*\HtX+(0)*\HtY}) {$$};
                         \node (f2) at (1.5*\Lrg,{-2*\HtX+(-2)*\HtY}) {$$};
                         \node (f5) at (2.5*\Lrg,{-2*\HtX+(0)*\HtY}) {$$};
                         \node (f3) at (4*\Lrg,{-2*\HtX+(-2)*\HtY}) {$f(x_0)$};
                         \node (f4) at (6*\Lrg,{-2*\HtX+(0)*\HtY}) {$$};
                         % Flèches
                         \draw[fleche] (f1) -- (f2);
                         \draw[fleche] (f5) -- (f3);
                         \draw[fleche] (f3) -- (f4);
                         % Doubles barres
                         \draw[double distance=2pt] (2*\Lrg,\lignex) -- (2*\Lrg,\bas);
                         % Encadrement
                         \draw[cadre] (\separateur,\haut) -- (\separateur,\bas);
                         \draw[cadre] (\gauche,\haut) rectangle  (\droite,\bas);
                         \draw[cadre] (\gauche,\lignex) -- (\droite,\lignex);
                         \draw[cadre] (\gauche,\lignef) -- (\droite,\lignef);
                    \end{tikzpicture}
               \end{extern}
          \end{center}
     \end{enumerate}
\end{corrige}

\end{document}
µ
\documentclass[a4paper]{article}

%================================================================================================================================
%
% Packages
%
%================================================================================================================================

\usepackage[T1]{fontenc} 	% pour caractères accentués
\usepackage[utf8]{inputenc}  % encodage utf8
\usepackage[french]{babel}	% langue : français
\usepackage{fourier}			% caractères plus lisibles
\usepackage[dvipsnames]{xcolor} % couleurs
\usepackage{fancyhdr}		% réglage header footer
\usepackage{needspace}		% empêcher sauts de page mal placés
\usepackage{graphicx}		% pour inclure des graphiques
\usepackage{enumitem,cprotect}		% personnalise les listes d'items (nécessaire pour ol, al ...)
\usepackage{hyperref}		% Liens hypertexte
\usepackage{pstricks,pst-all,pst-node,pstricks-add,pst-math,pst-plot,pst-tree,pst-eucl} % pstricks
\usepackage[a4paper,includeheadfoot,top=2cm,left=3cm, bottom=2cm,right=3cm]{geometry} % marges etc.
\usepackage{comment}			% commentaires multilignes
\usepackage{amsmath,environ} % maths (matrices, etc.)
\usepackage{amssymb,makeidx}
\usepackage{bm}				% bold maths
\usepackage{tabularx}		% tableaux
\usepackage{colortbl}		% tableaux en couleur
\usepackage{fontawesome}		% Fontawesome
\usepackage{environ}			% environment with command
\usepackage{fp}				% calculs pour ps-tricks
\usepackage{multido}			% pour ps tricks
\usepackage[np]{numprint}	% formattage nombre
\usepackage{tikz,tkz-tab} 			% package principal TikZ
\usepackage{pgfplots}   % axes
\usepackage{mathrsfs}    % cursives
\usepackage{calc}			% calcul taille boites
\usepackage[scaled=0.875]{helvet} % font sans serif
\usepackage{svg} % svg
\usepackage{scrextend} % local margin
\usepackage{scratch} %scratch
\usepackage{multicol} % colonnes
%\usepackage{infix-RPN,pst-func} % formule en notation polanaise inversée
\usepackage{listings}

%================================================================================================================================
%
% Réglages de base
%
%================================================================================================================================

\lstset{
language=Python,   % R code
literate=
{á}{{\'a}}1
{à}{{\`a}}1
{ã}{{\~a}}1
{é}{{\'e}}1
{è}{{\`e}}1
{ê}{{\^e}}1
{í}{{\'i}}1
{ó}{{\'o}}1
{õ}{{\~o}}1
{ú}{{\'u}}1
{ü}{{\"u}}1
{ç}{{\c{c}}}1
{~}{{ }}1
}


\definecolor{codegreen}{rgb}{0,0.6,0}
\definecolor{codegray}{rgb}{0.5,0.5,0.5}
\definecolor{codepurple}{rgb}{0.58,0,0.82}
\definecolor{backcolour}{rgb}{0.95,0.95,0.92}

\lstdefinestyle{mystyle}{
    backgroundcolor=\color{backcolour},   
    commentstyle=\color{codegreen},
    keywordstyle=\color{magenta},
    numberstyle=\tiny\color{codegray},
    stringstyle=\color{codepurple},
    basicstyle=\ttfamily\footnotesize,
    breakatwhitespace=false,         
    breaklines=true,                 
    captionpos=b,                    
    keepspaces=true,                 
    numbers=left,                    
xleftmargin=2em,
framexleftmargin=2em,            
    showspaces=false,                
    showstringspaces=false,
    showtabs=false,                  
    tabsize=2,
    upquote=true
}

\lstset{style=mystyle}


\lstset{style=mystyle}
\newcommand{\imgdir}{C:/laragon/www/newmc/assets/imgsvg/}
\newcommand{\imgsvgdir}{C:/laragon/www/newmc/assets/imgsvg/}

\definecolor{mcgris}{RGB}{220, 220, 220}% ancien~; pour compatibilité
\definecolor{mcbleu}{RGB}{52, 152, 219}
\definecolor{mcvert}{RGB}{125, 194, 70}
\definecolor{mcmauve}{RGB}{154, 0, 215}
\definecolor{mcorange}{RGB}{255, 96, 0}
\definecolor{mcturquoise}{RGB}{0, 153, 153}
\definecolor{mcrouge}{RGB}{255, 0, 0}
\definecolor{mclightvert}{RGB}{205, 234, 190}

\definecolor{gris}{RGB}{220, 220, 220}
\definecolor{bleu}{RGB}{52, 152, 219}
\definecolor{vert}{RGB}{125, 194, 70}
\definecolor{mauve}{RGB}{154, 0, 215}
\definecolor{orange}{RGB}{255, 96, 0}
\definecolor{turquoise}{RGB}{0, 153, 153}
\definecolor{rouge}{RGB}{255, 0, 0}
\definecolor{lightvert}{RGB}{205, 234, 190}
\setitemize[0]{label=\color{lightvert}  $\bullet$}

\pagestyle{fancy}
\renewcommand{\headrulewidth}{0.2pt}
\fancyhead[L]{maths-cours.fr}
\fancyhead[R]{\thepage}
\renewcommand{\footrulewidth}{0.2pt}
\fancyfoot[C]{}

\newcolumntype{C}{>{\centering\arraybackslash}X}
\newcolumntype{s}{>{\hsize=.35\hsize\arraybackslash}X}

\setlength{\parindent}{0pt}		 
\setlength{\parskip}{3mm}
\setlength{\headheight}{1cm}

\def\ebook{ebook}
\def\book{book}
\def\web{web}
\def\type{web}

\newcommand{\vect}[1]{\overrightarrow{\,\mathstrut#1\,}}

\def\Oij{$\left(\text{O}~;~\vect{\imath},~\vect{\jmath}\right)$}
\def\Oijk{$\left(\text{O}~;~\vect{\imath},~\vect{\jmath},~\vect{k}\right)$}
\def\Ouv{$\left(\text{O}~;~\vect{u},~\vect{v}\right)$}

\hypersetup{breaklinks=true, colorlinks = true, linkcolor = OliveGreen, urlcolor = OliveGreen, citecolor = OliveGreen, pdfauthor={Didier BONNEL - https://www.maths-cours.fr} } % supprime les bordures autour des liens

\renewcommand{\arg}[0]{\text{arg}}

\everymath{\displaystyle}

%================================================================================================================================
%
% Macros - Commandes
%
%================================================================================================================================

\newcommand\meta[2]{    			% Utilisé pour créer le post HTML.
	\def\titre{titre}
	\def\url{url}
	\def\arg{#1}
	\ifx\titre\arg
		\newcommand\maintitle{#2}
		\fancyhead[L]{#2}
		{\Large\sffamily \MakeUppercase{#2}}
		\vspace{1mm}\textcolor{mcvert}{\hrule}
	\fi 
	\ifx\url\arg
		\fancyfoot[L]{\href{https://www.maths-cours.fr#2}{\black \footnotesize{https://www.maths-cours.fr#2}}}
	\fi 
}


\newcommand\TitreC[1]{    		% Titre centré
     \needspace{3\baselineskip}
     \begin{center}\textbf{#1}\end{center}
}

\newcommand\newpar{    		% paragraphe
     \par
}

\newcommand\nosp {    		% commande vide (pas d'espace)
}
\newcommand{\id}[1]{} %ignore

\newcommand\boite[2]{				% Boite simple sans titre
	\vspace{5mm}
	\setlength{\fboxrule}{0.2mm}
	\setlength{\fboxsep}{5mm}	
	\fcolorbox{#1}{#1!3}{\makebox[\linewidth-2\fboxrule-2\fboxsep]{
  		\begin{minipage}[t]{\linewidth-2\fboxrule-4\fboxsep}\setlength{\parskip}{3mm}
  			 #2
  		\end{minipage}
	}}
	\vspace{5mm}
}

\newcommand\CBox[4]{				% Boites
	\vspace{5mm}
	\setlength{\fboxrule}{0.2mm}
	\setlength{\fboxsep}{5mm}
	
	\fcolorbox{#1}{#1!3}{\makebox[\linewidth-2\fboxrule-2\fboxsep]{
		\begin{minipage}[t]{1cm}\setlength{\parskip}{3mm}
	  		\textcolor{#1}{\LARGE{#2}}    
 	 	\end{minipage}  
  		\begin{minipage}[t]{\linewidth-2\fboxrule-4\fboxsep}\setlength{\parskip}{3mm}
			\raisebox{1.2mm}{\normalsize\sffamily{\textcolor{#1}{#3}}}						
  			 #4
  		\end{minipage}
	}}
	\vspace{5mm}
}

\newcommand\cadre[3]{				% Boites convertible html
	\par
	\vspace{2mm}
	\setlength{\fboxrule}{0.1mm}
	\setlength{\fboxsep}{5mm}
	\fcolorbox{#1}{white}{\makebox[\linewidth-2\fboxrule-2\fboxsep]{
  		\begin{minipage}[t]{\linewidth-2\fboxrule-4\fboxsep}\setlength{\parskip}{3mm}
			\raisebox{-2.5mm}{\sffamily \small{\textcolor{#1}{\MakeUppercase{#2}}}}		
			\par		
  			 #3
 	 		\end{minipage}
	}}
		\vspace{2mm}
	\par
}

\newcommand\bloc[3]{				% Boites convertible html sans bordure
     \needspace{2\baselineskip}
     {\sffamily \small{\textcolor{#1}{\MakeUppercase{#2}}}}    
		\par		
  			 #3
		\par
}

\newcommand\CHelp[1]{
     \CBox{Plum}{\faInfoCircle}{À RETENIR}{#1}
}

\newcommand\CUp[1]{
     \CBox{NavyBlue}{\faThumbsOUp}{EN PRATIQUE}{#1}
}

\newcommand\CInfo[1]{
     \CBox{Sepia}{\faArrowCircleRight}{REMARQUE}{#1}
}

\newcommand\CRedac[1]{
     \CBox{PineGreen}{\faEdit}{BIEN R\'EDIGER}{#1}
}

\newcommand\CError[1]{
     \CBox{Red}{\faExclamationTriangle}{ATTENTION}{#1}
}

\newcommand\TitreExo[2]{
\needspace{4\baselineskip}
 {\sffamily\large EXERCICE #1\ (\emph{#2 points})}
\vspace{5mm}
}

\newcommand\img[2]{
          \includegraphics[width=#2\paperwidth]{\imgdir#1}
}

\newcommand\imgsvg[2]{
       \begin{center}   \includegraphics[width=#2\paperwidth]{\imgsvgdir#1} \end{center}
}


\newcommand\Lien[2]{
     \href{#1}{#2 \tiny \faExternalLink}
}
\newcommand\mcLien[2]{
     \href{https~://www.maths-cours.fr/#1}{#2 \tiny \faExternalLink}
}

\newcommand{\euro}{\eurologo{}}

%================================================================================================================================
%
% Macros - Environement
%
%================================================================================================================================

\newenvironment{tex}{ %
}
{%
}

\newenvironment{indente}{ %
	\setlength\parindent{10mm}
}

{
	\setlength\parindent{0mm}
}

\newenvironment{corrige}{%
     \needspace{3\baselineskip}
     \medskip
     \textbf{\textsc{Corrigé}}
     \medskip
}
{
}

\newenvironment{extern}{%
     \begin{center}
     }
     {
     \end{center}
}

\NewEnviron{code}{%
	\par
     \boite{gray}{\texttt{%
     \BODY
     }}
     \par
}

\newenvironment{vbloc}{% boite sans cadre empeche saut de page
     \begin{minipage}[t]{\linewidth}
     }
     {
     \end{minipage}
}
\NewEnviron{h2}{%
    \needspace{3\baselineskip}
    \vspace{0.6cm}
	\noindent \MakeUppercase{\sffamily \large \BODY}
	\vspace{1mm}\textcolor{mcgris}{\hrule}\vspace{0.4cm}
	\par
}{}

\NewEnviron{h3}{%
    \needspace{3\baselineskip}
	\vspace{5mm}
	\textsc{\BODY}
	\par
}

\NewEnviron{margeneg}{ %
\begin{addmargin}[-1cm]{0cm}
\BODY
\end{addmargin}
}

\NewEnviron{html}{%
}

\begin{document}
\meta{url}{/exercices/points-dintersection-avec-une-tangente/}
\meta{pid}{11408}
\meta{titre}{Points d'intersection avec une tangente}
\meta{type}{exercices}
%
Soit la fonction $ f $ définie sur $  \mathbb{R}  $ par~:
\[
f (x) =x^3 - x^2  - x
\]
On note $  \mathscr{C}  $ sa courbe représentative dans un repère orthonormé.
\begin{enumerate}
     \item
     Étudier les variations de la fonction $ f $ sur $  \mathbb{R} . $
     \item
     Déterminer les coordonnées des points d'intersection de la courbe $  \mathscr{C}  $ avec l'axe des abscisses.
     \item
     Donnez l'équation réduite de la tangente $  (T_a)  $ à la courbe $  \mathscr{C}  $ au point d'abscisse $ a. $
     \item
     \begin{enumerate}[label=\alph*.]
          \item
          Développer $  (x - a) ^2  (x+2a - 1) . $
          \item
          Déterminer, en fonction de $ a $, le nombre et les abscisses des points d'intersection de la courbe $  \mathscr{C}  $ et de la tangente $  (T_a). $
     \end{enumerate}
\end{enumerate}
\begin{corrige}
     \begin{enumerate}
          \item
          La fonction $ f $ est une fonction polynôme donc elle est dérivable sur $  \mathbb{R} . $
          \newpar
          Sa dérivée est définie par~:  \\
          $ f'  (x) =3x^2  - 2x - 1 $
          \newpar
          Étudions le signe de $ f'  $ : \\
          $  \Delta = ( - 2) ^2  - 4 \times 3 \times  ( - 1)  = 16 > 0  $
          \newpar
          $ f'  $ admet donc 2 racines~: \\
          $ x_1= \frac{ 2 -  \sqrt { 16 }}{ 2 \times 3 } =  - \frac{ 2 }{ 6 } = -  \frac{ 1 }{ 3 } $  \\
          $ x_2= \frac{ 2 +\sqrt { 16 }}{ 2 \times 3 } =   \frac{ 6 }{ 6 } = 1 $
          \newpar
          Le coefficient de $ x^2  ~ (a=3) $, est strictement positif. On en déduit le tableau de signes de  $ f' $ et le tableau de variations de  $f$ , compte tenu du fait que~:
          \newpar
          $ f  \left(  -  \frac{ 1 }{ 3 }  \right) = -  \frac{ 1 }{ 27 } -  \frac{ 1 }{ 9 } + \frac{ 1 }{ 3 }  = \frac{ 5 }{ 27  }  $  \\
          $ f (1) =1 - 1 - 1= - 1 $  \\
          \begin{extern}%width="450" alt="tableau de variations de la fonction"
               \begin{tikzpicture}[scale=0.875]
                    % Styles
                    \tikzstyle{cadre}=[thin]
                    \tikzstyle{fleche}=[->,>=latex,thin]
                    \tikzstyle{nondefini}=[lightgray]
                    % Dimensions Modifiables
                    \def\Lrg{1.5}
                    \def\HtX{1}
                    \def\HtY{0.5}
                    % Dimensions Calculées
                    \def\lignex{-0.5*\HtX}
                    \def\lignef{-1.5*\HtX}
                    \def\separateur{-0.5*\Lrg}
                    % Largeur du tableau
                    \def\gauche{-1.5*\Lrg}
                    \def\droite{6.5*\Lrg}
                    % Hauteur du tableau
                    \def\haut{0.5*\HtX}
                    \def\bas{-2.5*\HtX-2*\HtY}
                    % Pointillés
                    \draw[dotted,black] (0*\Lrg,\lignex-0.1*\HtX) -- (0*\Lrg,\lignef+0.1*\HtX);
                    \draw[dotted,black] (0*\Lrg,\lignef-0.1*\HtX) -- (0*\Lrg,\bas+0.1*\HtX);
                    \draw[dotted,black] (2*\Lrg,\lignex-0.1*\HtX) -- (2*\Lrg,\lignef+0.1*\HtX);
                    \draw[dotted,black] (2*\Lrg,\lignef-0.1*\HtX) -- (2*\Lrg,\bas+0.1*\HtX);
                    \draw[dotted,black] (4*\Lrg,\lignex-0.1*\HtX) -- (4*\Lrg,\lignef+0.1*\HtX);
                    \draw[dotted,black] (4*\Lrg,\lignef-0.1*\HtX) -- (4*\Lrg,\bas+0.1*\HtX);
                    \draw[dotted,black] (6*\Lrg,\lignex-0.1*\HtX) -- (6*\Lrg,\lignef+0.1*\HtX);
                    \draw[dotted,black] (6*\Lrg,\lignef-0.1*\HtX) -- (6*\Lrg,\bas+0.1*\HtX);
                    % Ligne de l'abscisse~: x
                    \node at (-1*\Lrg,0) {$x$};
                    \node at (0*\Lrg,0) {$ -  \infty $};
                    \node at (2*\Lrg,0) {$ -  \frac{ 1 }{ 3 } $};
                    \node at (4*\Lrg,0) {$1$};
                    \node at (6*\Lrg,0) {$+ \infty $};
                    % Ligne de la dérivée~: f'(x)
                    \node at (-1*\Lrg,-1*\HtX) {$f'(x)$};
                    \node at (0*\Lrg,-1*\HtX) {$$};
                    \node at (1*\Lrg,-1*\HtX) {$+$};
                    \node at (2*\Lrg,-1*\HtX) {$0$};
                    \node at (3*\Lrg,-1*\HtX) {$-$};
                    \node at (4*\Lrg,-1*\HtX) {$0$};
                    \node at (5*\Lrg,-1*\HtX) {$+$};
                    \node at (6*\Lrg,-1*\HtX) {$$};
                    % Ligne de la fonction~: f(x)
                    \node  at (-1*\Lrg,{-2*\HtX+(-1)*\HtY}) {$f(x)$};
                    \node (f1) at (0*\Lrg,{-2*\HtX+(-2)*\HtY}) {$$};
                    \node (f2) at (2*\Lrg,{-2*\HtX+(0)*\HtY}) {$ \frac{ 5 }{ 27 } $};
                    \node (f3) at (4*\Lrg,{-2*\HtX+(-2)*\HtY}) {$-1$};
                    \node (f4) at (6*\Lrg,{-2*\HtX+(0)*\HtY}) {$$};
                    % Flèches
                    \draw[fleche] (f1) -- (f2);
                    \draw[fleche] (f2) -- (f3);
                    \draw[fleche] (f3) -- (f4);
                    % Encadrement
                    \draw[cadre] (\separateur,\haut) -- (\separateur,\bas);
                    \draw[cadre] (\gauche,\haut) rectangle  (\droite,\bas);
                    \draw[cadre] (\gauche,\lignex) -- (\droite,\lignex);
                    \draw[cadre] (\gauche,\lignef) -- (\droite,\lignef);
               \end{tikzpicture}
          \end{extern}
          \item
          Les abscisses des points d'intersection de la courbe $\mathscr{C}$ avec l'axe des abscisses sont les solutions de l'équation~:  \\
          $ x^{ 3 }  - x^2  - x=0 $
          \newpar
          Cette équation équivaut à~:  \\
          $ x \left( x^2  - x - 1 \right) =0 $
          \newpar
          soit $ x=0 $ ou  $ x^2  - x - 1=0 $
          \newpar
          Le discriminant de $ x^2  - x - 1 $ est~:  \\
          $  \Delta = ( - 1) ^2  - 4 \times 1 \times  ( - 1) =5 >0 $
          \newpar
          L'équation $ x^2  - x - 1 $ admet donc 2 solutions~:  \\
          $ x= \frac{ 1+ \sqrt { 5 }  }{ 2 }  $ ou $ x= \frac{ 1 -  \sqrt { 5 }  }{ 2 } . $
          \newpar
          En conclusion, la courbe $\mathscr{C}$ coupe l'axe des abscisses en trois points de coordonnées respectives~:  \\
          $  \left( 0 ; 0 \right) ,  \left( \frac{ 1 -  \sqrt { 5 }  }{ 2 } ; 0 \right) , \left( \frac{ 1 +  \sqrt { 5 }  }{ 2 } ; 0 \right)  $.
          \item
          L'équation de la tangente $  \left( T_{ a }  \right)  $ au point d'abscisse  $a$ est donnée par la formule~:  \\
          \begin{encadrer}
               $ y=f'  (a)  (x - a) +f (a)  $
          \end{encadrer}
          Ici, on obtient~:  \\
          $ y= (3a^2  - 2a - 1)  (x - a) +a^{ 3 }  - a^2  - a $  \\
          $ y= (3a^2  - 2a - 1) x - 3a^{ 3} +2a^2  +a+a^{ 3 } - a^2  - a  $  \\
          $ y = (3a^2  - 2a - 1) x - 2a^{ 3 } +a^2 $~
          \item
          \begin{enumerate}[label=\alph*.]
               \item
               $  (x - a) ^2  (x+2a - 1) = (x^2  - 2ax+a^2 )  (x+2a - 1)  $\\
               $\phantom{  (x - a) ^2  (x+2a - 1) } = x^{ 3} +2ax^2  - x^2 $\nosp$ - 2ax^2  - 4a^2 x+2ax$\nosp$+a^2 x+2a^{ 3 }  - a^2   $\\
               $\phantom{  (x - a) ^2  (x+2a - 1) } = x^{ 3}  - x^2  - 3a^2 x+2ax+2a^{ 3 }  - a^2 .   $
               \item
               Pour déterminer les abscisses des points d'intersection de $\mathscr{C}$ et de $  \left( T_{ a }  \right)  $, on résout l'équation~:
               \newpar
               $ x^{ 3} - x^2  - x= (3a^2  - 2a - 1) x - 2a^{ 3 } +a^2  $~ \\
               $ x^{ 3} - x^2  - x -  (3a^2  - 2a - 1) x+ 2a^{ 3 }  - a^2=0  $~ \\
               $ x^{ 3}  - x^2  - 3a^2 x+2ax+2a^{ 3 }  - a^2=0$
               \newpar
               d'après la question précédente, cette équation équivaut à~:  \\
               $  (x - a) ^2  (x+2a - 1) =0 $~
               \newpar
               par conséquent~:  \\
               $  (x - a) ^2 =0 $ ou $ x+2a - 1=0 $~ \\
               $ x=a $ ou $ x= - 2a+1 $~
               \newpar
               On a donc, en général, deux points d'intersection d'abscisses respectives  $a$ et $  - 2a+1. $
               \newpar
               Toutefois, ces points peuvent être confondus si  $ a= - 2a+1 $ c'est-à-dire si $ 3a=1 $ soit $ a= \frac{ 1 }{ 3 }  $~; on a alors un seul point d'intersection qui est le point d'abscisse $  \frac{ 1 }{ 3 } . $~
          \end{enumerate}
     \end{enumerate}
\end{corrige}

\end{document}
µ
\documentclass[a4paper]{article}

%================================================================================================================================
%
% Packages
%
%================================================================================================================================

\usepackage[T1]{fontenc} 	% pour caractères accentués
\usepackage[utf8]{inputenc}  % encodage utf8
\usepackage[french]{babel}	% langue : français
\usepackage{fourier}			% caractères plus lisibles
\usepackage[dvipsnames]{xcolor} % couleurs
\usepackage{fancyhdr}		% réglage header footer
\usepackage{needspace}		% empêcher sauts de page mal placés
\usepackage{graphicx}		% pour inclure des graphiques
\usepackage{enumitem,cprotect}		% personnalise les listes d'items (nécessaire pour ol, al ...)
\usepackage{hyperref}		% Liens hypertexte
\usepackage{pstricks,pst-all,pst-node,pstricks-add,pst-math,pst-plot,pst-tree,pst-eucl} % pstricks
\usepackage[a4paper,includeheadfoot,top=2cm,left=3cm, bottom=2cm,right=3cm]{geometry} % marges etc.
\usepackage{comment}			% commentaires multilignes
\usepackage{amsmath,environ} % maths (matrices, etc.)
\usepackage{amssymb,makeidx}
\usepackage{bm}				% bold maths
\usepackage{tabularx}		% tableaux
\usepackage{colortbl}		% tableaux en couleur
\usepackage{fontawesome}		% Fontawesome
\usepackage{environ}			% environment with command
\usepackage{fp}				% calculs pour ps-tricks
\usepackage{multido}			% pour ps tricks
\usepackage[np]{numprint}	% formattage nombre
\usepackage{tikz,tkz-tab} 			% package principal TikZ
\usepackage{pgfplots}   % axes
\usepackage{mathrsfs}    % cursives
\usepackage{calc}			% calcul taille boites
\usepackage[scaled=0.875]{helvet} % font sans serif
\usepackage{svg} % svg
\usepackage{scrextend} % local margin
\usepackage{scratch} %scratch
\usepackage{multicol} % colonnes
%\usepackage{infix-RPN,pst-func} % formule en notation polanaise inversée
\usepackage{listings}

%================================================================================================================================
%
% Réglages de base
%
%================================================================================================================================

\lstset{
language=Python,   % R code
literate=
{á}{{\'a}}1
{à}{{\`a}}1
{ã}{{\~a}}1
{é}{{\'e}}1
{è}{{\`e}}1
{ê}{{\^e}}1
{í}{{\'i}}1
{ó}{{\'o}}1
{õ}{{\~o}}1
{ú}{{\'u}}1
{ü}{{\"u}}1
{ç}{{\c{c}}}1
{~}{{ }}1
}


\definecolor{codegreen}{rgb}{0,0.6,0}
\definecolor{codegray}{rgb}{0.5,0.5,0.5}
\definecolor{codepurple}{rgb}{0.58,0,0.82}
\definecolor{backcolour}{rgb}{0.95,0.95,0.92}

\lstdefinestyle{mystyle}{
    backgroundcolor=\color{backcolour},   
    commentstyle=\color{codegreen},
    keywordstyle=\color{magenta},
    numberstyle=\tiny\color{codegray},
    stringstyle=\color{codepurple},
    basicstyle=\ttfamily\footnotesize,
    breakatwhitespace=false,         
    breaklines=true,                 
    captionpos=b,                    
    keepspaces=true,                 
    numbers=left,                    
xleftmargin=2em,
framexleftmargin=2em,            
    showspaces=false,                
    showstringspaces=false,
    showtabs=false,                  
    tabsize=2,
    upquote=true
}

\lstset{style=mystyle}


\lstset{style=mystyle}
\newcommand{\imgdir}{C:/laragon/www/newmc/assets/imgsvg/}
\newcommand{\imgsvgdir}{C:/laragon/www/newmc/assets/imgsvg/}

\definecolor{mcgris}{RGB}{220, 220, 220}% ancien~; pour compatibilité
\definecolor{mcbleu}{RGB}{52, 152, 219}
\definecolor{mcvert}{RGB}{125, 194, 70}
\definecolor{mcmauve}{RGB}{154, 0, 215}
\definecolor{mcorange}{RGB}{255, 96, 0}
\definecolor{mcturquoise}{RGB}{0, 153, 153}
\definecolor{mcrouge}{RGB}{255, 0, 0}
\definecolor{mclightvert}{RGB}{205, 234, 190}

\definecolor{gris}{RGB}{220, 220, 220}
\definecolor{bleu}{RGB}{52, 152, 219}
\definecolor{vert}{RGB}{125, 194, 70}
\definecolor{mauve}{RGB}{154, 0, 215}
\definecolor{orange}{RGB}{255, 96, 0}
\definecolor{turquoise}{RGB}{0, 153, 153}
\definecolor{rouge}{RGB}{255, 0, 0}
\definecolor{lightvert}{RGB}{205, 234, 190}
\setitemize[0]{label=\color{lightvert}  $\bullet$}

\pagestyle{fancy}
\renewcommand{\headrulewidth}{0.2pt}
\fancyhead[L]{maths-cours.fr}
\fancyhead[R]{\thepage}
\renewcommand{\footrulewidth}{0.2pt}
\fancyfoot[C]{}

\newcolumntype{C}{>{\centering\arraybackslash}X}
\newcolumntype{s}{>{\hsize=.35\hsize\arraybackslash}X}

\setlength{\parindent}{0pt}		 
\setlength{\parskip}{3mm}
\setlength{\headheight}{1cm}

\def\ebook{ebook}
\def\book{book}
\def\web{web}
\def\type{web}

\newcommand{\vect}[1]{\overrightarrow{\,\mathstrut#1\,}}

\def\Oij{$\left(\text{O}~;~\vect{\imath},~\vect{\jmath}\right)$}
\def\Oijk{$\left(\text{O}~;~\vect{\imath},~\vect{\jmath},~\vect{k}\right)$}
\def\Ouv{$\left(\text{O}~;~\vect{u},~\vect{v}\right)$}

\hypersetup{breaklinks=true, colorlinks = true, linkcolor = OliveGreen, urlcolor = OliveGreen, citecolor = OliveGreen, pdfauthor={Didier BONNEL - https://www.maths-cours.fr} } % supprime les bordures autour des liens

\renewcommand{\arg}[0]{\text{arg}}

\everymath{\displaystyle}

%================================================================================================================================
%
% Macros - Commandes
%
%================================================================================================================================

\newcommand\meta[2]{    			% Utilisé pour créer le post HTML.
	\def\titre{titre}
	\def\url{url}
	\def\arg{#1}
	\ifx\titre\arg
		\newcommand\maintitle{#2}
		\fancyhead[L]{#2}
		{\Large\sffamily \MakeUppercase{#2}}
		\vspace{1mm}\textcolor{mcvert}{\hrule}
	\fi 
	\ifx\url\arg
		\fancyfoot[L]{\href{https://www.maths-cours.fr#2}{\black \footnotesize{https://www.maths-cours.fr#2}}}
	\fi 
}


\newcommand\TitreC[1]{    		% Titre centré
     \needspace{3\baselineskip}
     \begin{center}\textbf{#1}\end{center}
}

\newcommand\newpar{    		% paragraphe
     \par
}

\newcommand\nosp {    		% commande vide (pas d'espace)
}
\newcommand{\id}[1]{} %ignore

\newcommand\boite[2]{				% Boite simple sans titre
	\vspace{5mm}
	\setlength{\fboxrule}{0.2mm}
	\setlength{\fboxsep}{5mm}	
	\fcolorbox{#1}{#1!3}{\makebox[\linewidth-2\fboxrule-2\fboxsep]{
  		\begin{minipage}[t]{\linewidth-2\fboxrule-4\fboxsep}\setlength{\parskip}{3mm}
  			 #2
  		\end{minipage}
	}}
	\vspace{5mm}
}

\newcommand\CBox[4]{				% Boites
	\vspace{5mm}
	\setlength{\fboxrule}{0.2mm}
	\setlength{\fboxsep}{5mm}
	
	\fcolorbox{#1}{#1!3}{\makebox[\linewidth-2\fboxrule-2\fboxsep]{
		\begin{minipage}[t]{1cm}\setlength{\parskip}{3mm}
	  		\textcolor{#1}{\LARGE{#2}}    
 	 	\end{minipage}  
  		\begin{minipage}[t]{\linewidth-2\fboxrule-4\fboxsep}\setlength{\parskip}{3mm}
			\raisebox{1.2mm}{\normalsize\sffamily{\textcolor{#1}{#3}}}						
  			 #4
  		\end{minipage}
	}}
	\vspace{5mm}
}

\newcommand\cadre[3]{				% Boites convertible html
	\par
	\vspace{2mm}
	\setlength{\fboxrule}{0.1mm}
	\setlength{\fboxsep}{5mm}
	\fcolorbox{#1}{white}{\makebox[\linewidth-2\fboxrule-2\fboxsep]{
  		\begin{minipage}[t]{\linewidth-2\fboxrule-4\fboxsep}\setlength{\parskip}{3mm}
			\raisebox{-2.5mm}{\sffamily \small{\textcolor{#1}{\MakeUppercase{#2}}}}		
			\par		
  			 #3
 	 		\end{minipage}
	}}
		\vspace{2mm}
	\par
}

\newcommand\bloc[3]{				% Boites convertible html sans bordure
     \needspace{2\baselineskip}
     {\sffamily \small{\textcolor{#1}{\MakeUppercase{#2}}}}    
		\par		
  			 #3
		\par
}

\newcommand\CHelp[1]{
     \CBox{Plum}{\faInfoCircle}{À RETENIR}{#1}
}

\newcommand\CUp[1]{
     \CBox{NavyBlue}{\faThumbsOUp}{EN PRATIQUE}{#1}
}

\newcommand\CInfo[1]{
     \CBox{Sepia}{\faArrowCircleRight}{REMARQUE}{#1}
}

\newcommand\CRedac[1]{
     \CBox{PineGreen}{\faEdit}{BIEN R\'EDIGER}{#1}
}

\newcommand\CError[1]{
     \CBox{Red}{\faExclamationTriangle}{ATTENTION}{#1}
}

\newcommand\TitreExo[2]{
\needspace{4\baselineskip}
 {\sffamily\large EXERCICE #1\ (\emph{#2 points})}
\vspace{5mm}
}

\newcommand\img[2]{
          \includegraphics[width=#2\paperwidth]{\imgdir#1}
}

\newcommand\imgsvg[2]{
       \begin{center}   \includegraphics[width=#2\paperwidth]{\imgsvgdir#1} \end{center}
}


\newcommand\Lien[2]{
     \href{#1}{#2 \tiny \faExternalLink}
}
\newcommand\mcLien[2]{
     \href{https~://www.maths-cours.fr/#1}{#2 \tiny \faExternalLink}
}

\newcommand{\euro}{\eurologo{}}

%================================================================================================================================
%
% Macros - Environement
%
%================================================================================================================================

\newenvironment{tex}{ %
}
{%
}

\newenvironment{indente}{ %
	\setlength\parindent{10mm}
}

{
	\setlength\parindent{0mm}
}

\newenvironment{corrige}{%
     \needspace{3\baselineskip}
     \medskip
     \textbf{\textsc{Corrigé}}
     \medskip
}
{
}

\newenvironment{extern}{%
     \begin{center}
     }
     {
     \end{center}
}

\NewEnviron{code}{%
	\par
     \boite{gray}{\texttt{%
     \BODY
     }}
     \par
}

\newenvironment{vbloc}{% boite sans cadre empeche saut de page
     \begin{minipage}[t]{\linewidth}
     }
     {
     \end{minipage}
}
\NewEnviron{h2}{%
    \needspace{3\baselineskip}
    \vspace{0.6cm}
	\noindent \MakeUppercase{\sffamily \large \BODY}
	\vspace{1mm}\textcolor{mcgris}{\hrule}\vspace{0.4cm}
	\par
}{}

\NewEnviron{h3}{%
    \needspace{3\baselineskip}
	\vspace{5mm}
	\textsc{\BODY}
	\par
}

\NewEnviron{margeneg}{ %
\begin{addmargin}[-1cm]{0cm}
\BODY
\end{addmargin}
}

\NewEnviron{html}{%
}

\begin{document}
\meta{url}{/exercices/demonstration-par-recurrence/}
\meta{pid}{8311}
\meta{pi_}{4557}
\meta{titre}{Privé : Démonstration par récurrence}
\meta{type}{exercices}
On considère la suite $(u_n)$ définie par $u_0=1$ et pour tout entier naturel $n$ : $ u_{n+1}=2u_n-n+1$
\par
Démontrer par récurrence que pour tout $n \in \mathbb{N}$ :  \\
\begin{center}
     $u_n=2^n+n. $
\end{center}
\par

\end{document}
µ
\documentclass[a4paper]{article}

%================================================================================================================================
%
% Packages
%
%================================================================================================================================

\usepackage[T1]{fontenc} 	% pour caractères accentués
\usepackage[utf8]{inputenc}  % encodage utf8
\usepackage[french]{babel}	% langue : français
\usepackage{fourier}			% caractères plus lisibles
\usepackage[dvipsnames]{xcolor} % couleurs
\usepackage{fancyhdr}		% réglage header footer
\usepackage{needspace}		% empêcher sauts de page mal placés
\usepackage{graphicx}		% pour inclure des graphiques
\usepackage{enumitem,cprotect}		% personnalise les listes d'items (nécessaire pour ol, al ...)
\usepackage{hyperref}		% Liens hypertexte
\usepackage{pstricks,pst-all,pst-node,pstricks-add,pst-math,pst-plot,pst-tree,pst-eucl} % pstricks
\usepackage[a4paper,includeheadfoot,top=2cm,left=3cm, bottom=2cm,right=3cm]{geometry} % marges etc.
\usepackage{comment}			% commentaires multilignes
\usepackage{amsmath,environ} % maths (matrices, etc.)
\usepackage{amssymb,makeidx}
\usepackage{bm}				% bold maths
\usepackage{tabularx}		% tableaux
\usepackage{colortbl}		% tableaux en couleur
\usepackage{fontawesome}		% Fontawesome
\usepackage{environ}			% environment with command
\usepackage{fp}				% calculs pour ps-tricks
\usepackage{multido}			% pour ps tricks
\usepackage[np]{numprint}	% formattage nombre
\usepackage{tikz,tkz-tab} 			% package principal TikZ
\usepackage{pgfplots}   % axes
\usepackage{mathrsfs}    % cursives
\usepackage{calc}			% calcul taille boites
\usepackage[scaled=0.875]{helvet} % font sans serif
\usepackage{svg} % svg
\usepackage{scrextend} % local margin
\usepackage{scratch} %scratch
\usepackage{multicol} % colonnes
%\usepackage{infix-RPN,pst-func} % formule en notation polanaise inversée
\usepackage{listings}

%================================================================================================================================
%
% Réglages de base
%
%================================================================================================================================

\lstset{
language=Python,   % R code
literate=
{á}{{\'a}}1
{à}{{\`a}}1
{ã}{{\~a}}1
{é}{{\'e}}1
{è}{{\`e}}1
{ê}{{\^e}}1
{í}{{\'i}}1
{ó}{{\'o}}1
{õ}{{\~o}}1
{ú}{{\'u}}1
{ü}{{\"u}}1
{ç}{{\c{c}}}1
{~}{{ }}1
}


\definecolor{codegreen}{rgb}{0,0.6,0}
\definecolor{codegray}{rgb}{0.5,0.5,0.5}
\definecolor{codepurple}{rgb}{0.58,0,0.82}
\definecolor{backcolour}{rgb}{0.95,0.95,0.92}

\lstdefinestyle{mystyle}{
    backgroundcolor=\color{backcolour},   
    commentstyle=\color{codegreen},
    keywordstyle=\color{magenta},
    numberstyle=\tiny\color{codegray},
    stringstyle=\color{codepurple},
    basicstyle=\ttfamily\footnotesize,
    breakatwhitespace=false,         
    breaklines=true,                 
    captionpos=b,                    
    keepspaces=true,                 
    numbers=left,                    
xleftmargin=2em,
framexleftmargin=2em,            
    showspaces=false,                
    showstringspaces=false,
    showtabs=false,                  
    tabsize=2,
    upquote=true
}

\lstset{style=mystyle}


\lstset{style=mystyle}
\newcommand{\imgdir}{C:/laragon/www/newmc/assets/imgsvg/}
\newcommand{\imgsvgdir}{C:/laragon/www/newmc/assets/imgsvg/}

\definecolor{mcgris}{RGB}{220, 220, 220}% ancien~; pour compatibilité
\definecolor{mcbleu}{RGB}{52, 152, 219}
\definecolor{mcvert}{RGB}{125, 194, 70}
\definecolor{mcmauve}{RGB}{154, 0, 215}
\definecolor{mcorange}{RGB}{255, 96, 0}
\definecolor{mcturquoise}{RGB}{0, 153, 153}
\definecolor{mcrouge}{RGB}{255, 0, 0}
\definecolor{mclightvert}{RGB}{205, 234, 190}

\definecolor{gris}{RGB}{220, 220, 220}
\definecolor{bleu}{RGB}{52, 152, 219}
\definecolor{vert}{RGB}{125, 194, 70}
\definecolor{mauve}{RGB}{154, 0, 215}
\definecolor{orange}{RGB}{255, 96, 0}
\definecolor{turquoise}{RGB}{0, 153, 153}
\definecolor{rouge}{RGB}{255, 0, 0}
\definecolor{lightvert}{RGB}{205, 234, 190}
\setitemize[0]{label=\color{lightvert}  $\bullet$}

\pagestyle{fancy}
\renewcommand{\headrulewidth}{0.2pt}
\fancyhead[L]{maths-cours.fr}
\fancyhead[R]{\thepage}
\renewcommand{\footrulewidth}{0.2pt}
\fancyfoot[C]{}

\newcolumntype{C}{>{\centering\arraybackslash}X}
\newcolumntype{s}{>{\hsize=.35\hsize\arraybackslash}X}

\setlength{\parindent}{0pt}		 
\setlength{\parskip}{3mm}
\setlength{\headheight}{1cm}

\def\ebook{ebook}
\def\book{book}
\def\web{web}
\def\type{web}

\newcommand{\vect}[1]{\overrightarrow{\,\mathstrut#1\,}}

\def\Oij{$\left(\text{O}~;~\vect{\imath},~\vect{\jmath}\right)$}
\def\Oijk{$\left(\text{O}~;~\vect{\imath},~\vect{\jmath},~\vect{k}\right)$}
\def\Ouv{$\left(\text{O}~;~\vect{u},~\vect{v}\right)$}

\hypersetup{breaklinks=true, colorlinks = true, linkcolor = OliveGreen, urlcolor = OliveGreen, citecolor = OliveGreen, pdfauthor={Didier BONNEL - https://www.maths-cours.fr} } % supprime les bordures autour des liens

\renewcommand{\arg}[0]{\text{arg}}

\everymath{\displaystyle}

%================================================================================================================================
%
% Macros - Commandes
%
%================================================================================================================================

\newcommand\meta[2]{    			% Utilisé pour créer le post HTML.
	\def\titre{titre}
	\def\url{url}
	\def\arg{#1}
	\ifx\titre\arg
		\newcommand\maintitle{#2}
		\fancyhead[L]{#2}
		{\Large\sffamily \MakeUppercase{#2}}
		\vspace{1mm}\textcolor{mcvert}{\hrule}
	\fi 
	\ifx\url\arg
		\fancyfoot[L]{\href{https://www.maths-cours.fr#2}{\black \footnotesize{https://www.maths-cours.fr#2}}}
	\fi 
}


\newcommand\TitreC[1]{    		% Titre centré
     \needspace{3\baselineskip}
     \begin{center}\textbf{#1}\end{center}
}

\newcommand\newpar{    		% paragraphe
     \par
}

\newcommand\nosp {    		% commande vide (pas d'espace)
}
\newcommand{\id}[1]{} %ignore

\newcommand\boite[2]{				% Boite simple sans titre
	\vspace{5mm}
	\setlength{\fboxrule}{0.2mm}
	\setlength{\fboxsep}{5mm}	
	\fcolorbox{#1}{#1!3}{\makebox[\linewidth-2\fboxrule-2\fboxsep]{
  		\begin{minipage}[t]{\linewidth-2\fboxrule-4\fboxsep}\setlength{\parskip}{3mm}
  			 #2
  		\end{minipage}
	}}
	\vspace{5mm}
}

\newcommand\CBox[4]{				% Boites
	\vspace{5mm}
	\setlength{\fboxrule}{0.2mm}
	\setlength{\fboxsep}{5mm}
	
	\fcolorbox{#1}{#1!3}{\makebox[\linewidth-2\fboxrule-2\fboxsep]{
		\begin{minipage}[t]{1cm}\setlength{\parskip}{3mm}
	  		\textcolor{#1}{\LARGE{#2}}    
 	 	\end{minipage}  
  		\begin{minipage}[t]{\linewidth-2\fboxrule-4\fboxsep}\setlength{\parskip}{3mm}
			\raisebox{1.2mm}{\normalsize\sffamily{\textcolor{#1}{#3}}}						
  			 #4
  		\end{minipage}
	}}
	\vspace{5mm}
}

\newcommand\cadre[3]{				% Boites convertible html
	\par
	\vspace{2mm}
	\setlength{\fboxrule}{0.1mm}
	\setlength{\fboxsep}{5mm}
	\fcolorbox{#1}{white}{\makebox[\linewidth-2\fboxrule-2\fboxsep]{
  		\begin{minipage}[t]{\linewidth-2\fboxrule-4\fboxsep}\setlength{\parskip}{3mm}
			\raisebox{-2.5mm}{\sffamily \small{\textcolor{#1}{\MakeUppercase{#2}}}}		
			\par		
  			 #3
 	 		\end{minipage}
	}}
		\vspace{2mm}
	\par
}

\newcommand\bloc[3]{				% Boites convertible html sans bordure
     \needspace{2\baselineskip}
     {\sffamily \small{\textcolor{#1}{\MakeUppercase{#2}}}}    
		\par		
  			 #3
		\par
}

\newcommand\CHelp[1]{
     \CBox{Plum}{\faInfoCircle}{À RETENIR}{#1}
}

\newcommand\CUp[1]{
     \CBox{NavyBlue}{\faThumbsOUp}{EN PRATIQUE}{#1}
}

\newcommand\CInfo[1]{
     \CBox{Sepia}{\faArrowCircleRight}{REMARQUE}{#1}
}

\newcommand\CRedac[1]{
     \CBox{PineGreen}{\faEdit}{BIEN R\'EDIGER}{#1}
}

\newcommand\CError[1]{
     \CBox{Red}{\faExclamationTriangle}{ATTENTION}{#1}
}

\newcommand\TitreExo[2]{
\needspace{4\baselineskip}
 {\sffamily\large EXERCICE #1\ (\emph{#2 points})}
\vspace{5mm}
}

\newcommand\img[2]{
          \includegraphics[width=#2\paperwidth]{\imgdir#1}
}

\newcommand\imgsvg[2]{
       \begin{center}   \includegraphics[width=#2\paperwidth]{\imgsvgdir#1} \end{center}
}


\newcommand\Lien[2]{
     \href{#1}{#2 \tiny \faExternalLink}
}
\newcommand\mcLien[2]{
     \href{https~://www.maths-cours.fr/#1}{#2 \tiny \faExternalLink}
}

\newcommand{\euro}{\eurologo{}}

%================================================================================================================================
%
% Macros - Environement
%
%================================================================================================================================

\newenvironment{tex}{ %
}
{%
}

\newenvironment{indente}{ %
	\setlength\parindent{10mm}
}

{
	\setlength\parindent{0mm}
}

\newenvironment{corrige}{%
     \needspace{3\baselineskip}
     \medskip
     \textbf{\textsc{Corrigé}}
     \medskip
}
{
}

\newenvironment{extern}{%
     \begin{center}
     }
     {
     \end{center}
}

\NewEnviron{code}{%
	\par
     \boite{gray}{\texttt{%
     \BODY
     }}
     \par
}

\newenvironment{vbloc}{% boite sans cadre empeche saut de page
     \begin{minipage}[t]{\linewidth}
     }
     {
     \end{minipage}
}
\NewEnviron{h2}{%
    \needspace{3\baselineskip}
    \vspace{0.6cm}
	\noindent \MakeUppercase{\sffamily \large \BODY}
	\vspace{1mm}\textcolor{mcgris}{\hrule}\vspace{0.4cm}
	\par
}{}

\NewEnviron{h3}{%
    \needspace{3\baselineskip}
	\vspace{5mm}
	\textsc{\BODY}
	\par
}

\NewEnviron{margeneg}{ %
\begin{addmargin}[-1cm]{0cm}
\BODY
\end{addmargin}
}

\NewEnviron{html}{%
}

\begin{document}
\meta{url}{/exercices/demonstration-par-recurrence/}
\meta{pid}{8311}
\meta{pi_}{4557}
\meta{titre}{Privé : Démonstration par récurrence}
\meta{type}{exercices}
On considère la suite $(u_n)$ définie par $u_0=1$ et pour tout entier naturel $n$ : $ u_{n+1}=2u_n-n+1$
\par
Démontrer par récurrence que pour tout $n \in \mathbb{N}$ :  \\ 
\begin{center}
$u_n=2^n+n. $
\end{center}
\par

\end{document}
µ
\documentclass[a4paper]{article}

%================================================================================================================================
%
% Packages
%
%================================================================================================================================

\usepackage[T1]{fontenc} 	% pour caractères accentués
\usepackage[utf8]{inputenc}  % encodage utf8
\usepackage[french]{babel}	% langue : français
\usepackage{fourier}			% caractères plus lisibles
\usepackage[dvipsnames]{xcolor} % couleurs
\usepackage{fancyhdr}		% réglage header footer
\usepackage{needspace}		% empêcher sauts de page mal placés
\usepackage{graphicx}		% pour inclure des graphiques
\usepackage{enumitem,cprotect}		% personnalise les listes d'items (nécessaire pour ol, al ...)
\usepackage{hyperref}		% Liens hypertexte
\usepackage{pstricks,pst-all,pst-node,pstricks-add,pst-math,pst-plot,pst-tree,pst-eucl} % pstricks
\usepackage[a4paper,includeheadfoot,top=2cm,left=3cm, bottom=2cm,right=3cm]{geometry} % marges etc.
\usepackage{comment}			% commentaires multilignes
\usepackage{amsmath,environ} % maths (matrices, etc.)
\usepackage{amssymb,makeidx}
\usepackage{bm}				% bold maths
\usepackage{tabularx}		% tableaux
\usepackage{colortbl}		% tableaux en couleur
\usepackage{fontawesome}		% Fontawesome
\usepackage{environ}			% environment with command
\usepackage{fp}				% calculs pour ps-tricks
\usepackage{multido}			% pour ps tricks
\usepackage[np]{numprint}	% formattage nombre
\usepackage{tikz,tkz-tab} 			% package principal TikZ
\usepackage{pgfplots}   % axes
\usepackage{mathrsfs}    % cursives
\usepackage{calc}			% calcul taille boites
\usepackage[scaled=0.875]{helvet} % font sans serif
\usepackage{svg} % svg
\usepackage{scrextend} % local margin
\usepackage{scratch} %scratch
\usepackage{multicol} % colonnes
%\usepackage{infix-RPN,pst-func} % formule en notation polanaise inversée
\usepackage{listings}

%================================================================================================================================
%
% Réglages de base
%
%================================================================================================================================

\lstset{
language=Python,   % R code
literate=
{á}{{\'a}}1
{à}{{\`a}}1
{ã}{{\~a}}1
{é}{{\'e}}1
{è}{{\`e}}1
{ê}{{\^e}}1
{í}{{\'i}}1
{ó}{{\'o}}1
{õ}{{\~o}}1
{ú}{{\'u}}1
{ü}{{\"u}}1
{ç}{{\c{c}}}1
{~}{{ }}1
}


\definecolor{codegreen}{rgb}{0,0.6,0}
\definecolor{codegray}{rgb}{0.5,0.5,0.5}
\definecolor{codepurple}{rgb}{0.58,0,0.82}
\definecolor{backcolour}{rgb}{0.95,0.95,0.92}

\lstdefinestyle{mystyle}{
    backgroundcolor=\color{backcolour},   
    commentstyle=\color{codegreen},
    keywordstyle=\color{magenta},
    numberstyle=\tiny\color{codegray},
    stringstyle=\color{codepurple},
    basicstyle=\ttfamily\footnotesize,
    breakatwhitespace=false,         
    breaklines=true,                 
    captionpos=b,                    
    keepspaces=true,                 
    numbers=left,                    
xleftmargin=2em,
framexleftmargin=2em,            
    showspaces=false,                
    showstringspaces=false,
    showtabs=false,                  
    tabsize=2,
    upquote=true
}

\lstset{style=mystyle}


\lstset{style=mystyle}
\newcommand{\imgdir}{C:/laragon/www/newmc/assets/imgsvg/}
\newcommand{\imgsvgdir}{C:/laragon/www/newmc/assets/imgsvg/}

\definecolor{mcgris}{RGB}{220, 220, 220}% ancien~; pour compatibilité
\definecolor{mcbleu}{RGB}{52, 152, 219}
\definecolor{mcvert}{RGB}{125, 194, 70}
\definecolor{mcmauve}{RGB}{154, 0, 215}
\definecolor{mcorange}{RGB}{255, 96, 0}
\definecolor{mcturquoise}{RGB}{0, 153, 153}
\definecolor{mcrouge}{RGB}{255, 0, 0}
\definecolor{mclightvert}{RGB}{205, 234, 190}

\definecolor{gris}{RGB}{220, 220, 220}
\definecolor{bleu}{RGB}{52, 152, 219}
\definecolor{vert}{RGB}{125, 194, 70}
\definecolor{mauve}{RGB}{154, 0, 215}
\definecolor{orange}{RGB}{255, 96, 0}
\definecolor{turquoise}{RGB}{0, 153, 153}
\definecolor{rouge}{RGB}{255, 0, 0}
\definecolor{lightvert}{RGB}{205, 234, 190}
\setitemize[0]{label=\color{lightvert}  $\bullet$}

\pagestyle{fancy}
\renewcommand{\headrulewidth}{0.2pt}
\fancyhead[L]{maths-cours.fr}
\fancyhead[R]{\thepage}
\renewcommand{\footrulewidth}{0.2pt}
\fancyfoot[C]{}

\newcolumntype{C}{>{\centering\arraybackslash}X}
\newcolumntype{s}{>{\hsize=.35\hsize\arraybackslash}X}

\setlength{\parindent}{0pt}		 
\setlength{\parskip}{3mm}
\setlength{\headheight}{1cm}

\def\ebook{ebook}
\def\book{book}
\def\web{web}
\def\type{web}

\newcommand{\vect}[1]{\overrightarrow{\,\mathstrut#1\,}}

\def\Oij{$\left(\text{O}~;~\vect{\imath},~\vect{\jmath}\right)$}
\def\Oijk{$\left(\text{O}~;~\vect{\imath},~\vect{\jmath},~\vect{k}\right)$}
\def\Ouv{$\left(\text{O}~;~\vect{u},~\vect{v}\right)$}

\hypersetup{breaklinks=true, colorlinks = true, linkcolor = OliveGreen, urlcolor = OliveGreen, citecolor = OliveGreen, pdfauthor={Didier BONNEL - https://www.maths-cours.fr} } % supprime les bordures autour des liens

\renewcommand{\arg}[0]{\text{arg}}

\everymath{\displaystyle}

%================================================================================================================================
%
% Macros - Commandes
%
%================================================================================================================================

\newcommand\meta[2]{    			% Utilisé pour créer le post HTML.
	\def\titre{titre}
	\def\url{url}
	\def\arg{#1}
	\ifx\titre\arg
		\newcommand\maintitle{#2}
		\fancyhead[L]{#2}
		{\Large\sffamily \MakeUppercase{#2}}
		\vspace{1mm}\textcolor{mcvert}{\hrule}
	\fi 
	\ifx\url\arg
		\fancyfoot[L]{\href{https://www.maths-cours.fr#2}{\black \footnotesize{https://www.maths-cours.fr#2}}}
	\fi 
}


\newcommand\TitreC[1]{    		% Titre centré
     \needspace{3\baselineskip}
     \begin{center}\textbf{#1}\end{center}
}

\newcommand\newpar{    		% paragraphe
     \par
}

\newcommand\nosp {    		% commande vide (pas d'espace)
}
\newcommand{\id}[1]{} %ignore

\newcommand\boite[2]{				% Boite simple sans titre
	\vspace{5mm}
	\setlength{\fboxrule}{0.2mm}
	\setlength{\fboxsep}{5mm}	
	\fcolorbox{#1}{#1!3}{\makebox[\linewidth-2\fboxrule-2\fboxsep]{
  		\begin{minipage}[t]{\linewidth-2\fboxrule-4\fboxsep}\setlength{\parskip}{3mm}
  			 #2
  		\end{minipage}
	}}
	\vspace{5mm}
}

\newcommand\CBox[4]{				% Boites
	\vspace{5mm}
	\setlength{\fboxrule}{0.2mm}
	\setlength{\fboxsep}{5mm}
	
	\fcolorbox{#1}{#1!3}{\makebox[\linewidth-2\fboxrule-2\fboxsep]{
		\begin{minipage}[t]{1cm}\setlength{\parskip}{3mm}
	  		\textcolor{#1}{\LARGE{#2}}    
 	 	\end{minipage}  
  		\begin{minipage}[t]{\linewidth-2\fboxrule-4\fboxsep}\setlength{\parskip}{3mm}
			\raisebox{1.2mm}{\normalsize\sffamily{\textcolor{#1}{#3}}}						
  			 #4
  		\end{minipage}
	}}
	\vspace{5mm}
}

\newcommand\cadre[3]{				% Boites convertible html
	\par
	\vspace{2mm}
	\setlength{\fboxrule}{0.1mm}
	\setlength{\fboxsep}{5mm}
	\fcolorbox{#1}{white}{\makebox[\linewidth-2\fboxrule-2\fboxsep]{
  		\begin{minipage}[t]{\linewidth-2\fboxrule-4\fboxsep}\setlength{\parskip}{3mm}
			\raisebox{-2.5mm}{\sffamily \small{\textcolor{#1}{\MakeUppercase{#2}}}}		
			\par		
  			 #3
 	 		\end{minipage}
	}}
		\vspace{2mm}
	\par
}

\newcommand\bloc[3]{				% Boites convertible html sans bordure
     \needspace{2\baselineskip}
     {\sffamily \small{\textcolor{#1}{\MakeUppercase{#2}}}}    
		\par		
  			 #3
		\par
}

\newcommand\CHelp[1]{
     \CBox{Plum}{\faInfoCircle}{À RETENIR}{#1}
}

\newcommand\CUp[1]{
     \CBox{NavyBlue}{\faThumbsOUp}{EN PRATIQUE}{#1}
}

\newcommand\CInfo[1]{
     \CBox{Sepia}{\faArrowCircleRight}{REMARQUE}{#1}
}

\newcommand\CRedac[1]{
     \CBox{PineGreen}{\faEdit}{BIEN R\'EDIGER}{#1}
}

\newcommand\CError[1]{
     \CBox{Red}{\faExclamationTriangle}{ATTENTION}{#1}
}

\newcommand\TitreExo[2]{
\needspace{4\baselineskip}
 {\sffamily\large EXERCICE #1\ (\emph{#2 points})}
\vspace{5mm}
}

\newcommand\img[2]{
          \includegraphics[width=#2\paperwidth]{\imgdir#1}
}

\newcommand\imgsvg[2]{
       \begin{center}   \includegraphics[width=#2\paperwidth]{\imgsvgdir#1} \end{center}
}


\newcommand\Lien[2]{
     \href{#1}{#2 \tiny \faExternalLink}
}
\newcommand\mcLien[2]{
     \href{https~://www.maths-cours.fr/#1}{#2 \tiny \faExternalLink}
}

\newcommand{\euro}{\eurologo{}}

%================================================================================================================================
%
% Macros - Environement
%
%================================================================================================================================

\newenvironment{tex}{ %
}
{%
}

\newenvironment{indente}{ %
	\setlength\parindent{10mm}
}

{
	\setlength\parindent{0mm}
}

\newenvironment{corrige}{%
     \needspace{3\baselineskip}
     \medskip
     \textbf{\textsc{Corrigé}}
     \medskip
}
{
}

\newenvironment{extern}{%
     \begin{center}
     }
     {
     \end{center}
}

\NewEnviron{code}{%
	\par
     \boite{gray}{\texttt{%
     \BODY
     }}
     \par
}

\newenvironment{vbloc}{% boite sans cadre empeche saut de page
     \begin{minipage}[t]{\linewidth}
     }
     {
     \end{minipage}
}
\NewEnviron{h2}{%
    \needspace{3\baselineskip}
    \vspace{0.6cm}
	\noindent \MakeUppercase{\sffamily \large \BODY}
	\vspace{1mm}\textcolor{mcgris}{\hrule}\vspace{0.4cm}
	\par
}{}

\NewEnviron{h3}{%
    \needspace{3\baselineskip}
	\vspace{5mm}
	\textsc{\BODY}
	\par
}

\NewEnviron{margeneg}{ %
\begin{addmargin}[-1cm]{0cm}
\BODY
\end{addmargin}
}

\NewEnviron{html}{%
}

\begin{document}
\meta{url}{/exercices/theoreme-de-thales-et-parallelogramme/}
\meta{pid}{11476}
\meta{titre}{Théorème de Thalès et parallélogramme}
\meta{type}{exercices}
%
Sur la figure ci-dessous, $ ABCD $ est un parallélogramme. $ N$  est un point du côté $  \left[ AD \right] . $
\par
La parallèle à la droite $  \left( AB \right)   $ passant par $N$ coupe la diagonale $  \left[ AC \right]   $ en $M$.
\par
Enfin, la droite $  \left( BM \right)   $ coupe la droite $  \left( AD \right)   $ en $I$.
\begin{center}
     \begin{extern}%width="500" alt="géométrie"
          \newrgbcolor{grey}{0.2 0.2 0.2}
          \psset{xunit=1cm,yunit=1cm,algebraic=true,dimen=middle,linecolor=grey,dotstyle=*,dotsize=5pt 0,linewidth=1.6pt,arrowsize=3pt 2,arrowinset=0.25}
          \begin{pspicture*}(0.,0.)(10.,8.)
               \psline[linewidth=0.4pt,linecolor=grey](4.,7.)(9.,7.)
               \par\psline[linewidth=0.4pt,linecolor=grey](9.,7.)(6.,1.)
               \psline[linewidth=0.4pt,linecolor=grey](6.,1.)(1.,1.)
               \psline[linewidth=0.4pt,linecolor=grey](1.,1.)(4.,7.)
               \psline[linewidth=0.4pt,linecolor=grey](4.,7.)(6.,1.)
               \psline[linewidth=0.4pt,linecolor=grey](3.,5.)(8.,5.)
               \psline[linewidth=0.4pt,linecolor=grey](9.,7.)(2.5,4.)
               \begin{scriptsize}
                    \psdots[dotsize=2pt 0,dotstyle=*,linecolor=grey](4.,7.)
                    \rput[bl](3.6845133506321366,7.094154302649859){\grey{$A$}}
                    \psdots[dotsize=2pt 0,dotstyle=*,linecolor=grey](9.,7.)
                    \rput[bl](9.049212521237859,7.04493687906632){\grey{$B$}}
                    \psdots[dotsize=2pt 0,dotstyle=*,linecolor=grey](6.,1.)
                    \rput[bl](6.046949682641996,0.6835848808939424){\grey{$C$}}
                    \psdots[dotsize=2pt 0,dotstyle=*,linecolor=grey](1.,1.)
                    \rput[bl](0.731467935619812,0.6589761691021732){\grey{$D$}}
                    \psdots[dotsize=2pt 0,dotstyle=*](3.,5.)
                    \rput[bl](2.7001648789613615,4.9778050885576945){$N$}
                    \psdots[dotsize=2pt 0,dotstyle=*,linecolor=grey](4.666666666666667,5.)
                    \rput[bl](4.791905381261758,4.6825005470564625){\grey{$M$}}
                    \psdots[dotsize=2pt 0,dotstyle=*,linecolor=grey](2.5,4.)
                    \rput[bl](2.306425490293052,3.9811522609910357){\grey{$I$}}
               \end{scriptsize}
          \end{pspicture*}
     \end{extern}
\end{center}
\begin{enumerate}
     \item
     Montrer que~:
     \begin{center}
          $ \frac{ NM }{ AB } = \frac{ NI }{ AI }   $
     \end{center}
     \item
     Montrer que~:
     \begin{center}
          $ \frac{ NM }{ DC } = \frac{ AN }{ AD }   $
     \end{center}
     \item
     En déduire que~:
     \begin{center}
          $ \frac{ NI }{ AI } = \frac{ AN }{ AD }.$
     \end{center}
     \item  \textbf{Application numérique~: }
     \\
     Calculer la longueur $ NI $ sachant que $ AN = 2 $ cm et $ AD = 6 $ cm.
     \item
     (Question subsidiaire) Que peut-on dire de la position du point $I$ lorsque l'on modifie la position du point $B$~?
\end{enumerate}
\begin{corrige}
     \begin{enumerate}
          \item
          Les triangles $ IAB $ et $ INM $ sont en situation de Thalès~; en effet~:
          \begin{itemize}
               \item
               les points $I$, $N$ et $A$ sont alignés
               \item
               les points $ I $, $M$ et $ B $ sont alignés
               \item
               les droites $  \left( MN \right)   $ et $  \left( AB \right)   $ sont parallèles.
          \end{itemize}
          D'après le \mcLien{https://www.maths-cours.fr/cours/theoreme-thales/\#d10}{théorème de Thalès}, on a donc~:
          \par
          $ \frac{ NM }{ AB } = \frac{ NI }{ AI }   = \frac{ MI }{ BI } .  $
          \medskip
          (Le troisième rapport $  \frac{ MI }{ BI }  $ sera inutile pour cet exercice.)
          \item
          De même, les triangles $ ANM $ et $ ACD $ sont en situation de Thalès car~:
          \begin{itemize}
               \item
               les points $D$, $N$ et $A$ sont alignés
               \item
               les points $ C$, $M$ et $ A $ sont alignés
               \item
               les droites $  \left( MN \right)   $ et $  \left( DC \right)   $ sont parallèles.
          \end{itemize}
          D'après le théorème de Thalès~:
          \par
          $ \frac{ NM }{ DC } = \frac{ AN }{ AD } = \frac{ AM }{ AC }    $
          \item
          Comme $ ABCD $ est un parallélogramme, $ AB=DC $ donc $ \frac{ NM }{ AB } = \frac{ NM }{ DC }    $ et donc les rapports des questions  \textbf{1.} et  \textbf{2.} sont tous égaux.
          \par
          En particulier~:
          \par
          $ \frac{ NI }{ AI } = \frac{ AN }{ AD }.$
          \item  \textbf{Application numérique~: }
          \\
          Posons $ NI = x. $
          \par
          Alors:
          \\
          $ AI=AN+NI=x+2 $
          \par
          D'après la question précédente, on a donc~:
          \par
          $ \frac{ x }{ x+2 } = \frac{ 2 }{ 6 }    $
          \par
          En effectuant le produit en croix cette équation s'écrit:
          \par
          $ 6x=2  \left( x+2 \right)   $
          \\
          $ 6x=2x+4 $
          \\
          $ 6x-2x=4 $
          \\
          $4x=4  $
          \\
          $ x=1 $
          \par
          La longueur $ NI $ mesure donc 1~cm.
          \item
          La position du point $ I$  \textbf{ reste inchangée} lorsque l'on modifie la position du point $ B $ (à partir du moment où $ ABCD $ est un parallélogramme et où l'on suit les consignes de l'énoncé. ).
          \\
          On peut d'ailleurs vérifier ce résultat à l'aide d'un logiciel de géométrie dynamique comme geogebra.
          \par
          En effet, le résultat de la question  \textbf{3.} montre que la distance $ NI $ ne dépend que de la position de $A$, $N$ et $D$ et ne dépend ni de la distance $ AB $ ni de la mesure de l'angle $  \widehat{ DAB } .  $
     \end{enumerate}
\end{corrige}

\end{document}
µ
\documentclass[a4paper]{article}

%================================================================================================================================
%
% Packages
%
%================================================================================================================================

\usepackage[T1]{fontenc} 	% pour caractères accentués
\usepackage[utf8]{inputenc}  % encodage utf8
\usepackage[french]{babel}	% langue : français
\usepackage{fourier}			% caractères plus lisibles
\usepackage[dvipsnames]{xcolor} % couleurs
\usepackage{fancyhdr}		% réglage header footer
\usepackage{needspace}		% empêcher sauts de page mal placés
\usepackage{graphicx}		% pour inclure des graphiques
\usepackage{enumitem,cprotect}		% personnalise les listes d'items (nécessaire pour ol, al ...)
\usepackage{hyperref}		% Liens hypertexte
\usepackage{pstricks,pst-all,pst-node,pstricks-add,pst-math,pst-plot,pst-tree,pst-eucl} % pstricks
\usepackage[a4paper,includeheadfoot,top=2cm,left=3cm, bottom=2cm,right=3cm]{geometry} % marges etc.
\usepackage{comment}			% commentaires multilignes
\usepackage{amsmath,environ} % maths (matrices, etc.)
\usepackage{amssymb,makeidx}
\usepackage{bm}				% bold maths
\usepackage{tabularx}		% tableaux
\usepackage{colortbl}		% tableaux en couleur
\usepackage{fontawesome}		% Fontawesome
\usepackage{environ}			% environment with command
\usepackage{fp}				% calculs pour ps-tricks
\usepackage{multido}			% pour ps tricks
\usepackage[np]{numprint}	% formattage nombre
\usepackage{tikz,tkz-tab} 			% package principal TikZ
\usepackage{pgfplots}   % axes
\usepackage{mathrsfs}    % cursives
\usepackage{calc}			% calcul taille boites
\usepackage[scaled=0.875]{helvet} % font sans serif
\usepackage{svg} % svg
\usepackage{scrextend} % local margin
\usepackage{scratch} %scratch
\usepackage{multicol} % colonnes
%\usepackage{infix-RPN,pst-func} % formule en notation polanaise inversée
\usepackage{listings}

%================================================================================================================================
%
% Réglages de base
%
%================================================================================================================================

\lstset{
language=Python,   % R code
literate=
{á}{{\'a}}1
{à}{{\`a}}1
{ã}{{\~a}}1
{é}{{\'e}}1
{è}{{\`e}}1
{ê}{{\^e}}1
{í}{{\'i}}1
{ó}{{\'o}}1
{õ}{{\~o}}1
{ú}{{\'u}}1
{ü}{{\"u}}1
{ç}{{\c{c}}}1
{~}{{ }}1
}


\definecolor{codegreen}{rgb}{0,0.6,0}
\definecolor{codegray}{rgb}{0.5,0.5,0.5}
\definecolor{codepurple}{rgb}{0.58,0,0.82}
\definecolor{backcolour}{rgb}{0.95,0.95,0.92}

\lstdefinestyle{mystyle}{
    backgroundcolor=\color{backcolour},   
    commentstyle=\color{codegreen},
    keywordstyle=\color{magenta},
    numberstyle=\tiny\color{codegray},
    stringstyle=\color{codepurple},
    basicstyle=\ttfamily\footnotesize,
    breakatwhitespace=false,         
    breaklines=true,                 
    captionpos=b,                    
    keepspaces=true,                 
    numbers=left,                    
xleftmargin=2em,
framexleftmargin=2em,            
    showspaces=false,                
    showstringspaces=false,
    showtabs=false,                  
    tabsize=2,
    upquote=true
}

\lstset{style=mystyle}


\lstset{style=mystyle}
\newcommand{\imgdir}{C:/laragon/www/newmc/assets/imgsvg/}
\newcommand{\imgsvgdir}{C:/laragon/www/newmc/assets/imgsvg/}

\definecolor{mcgris}{RGB}{220, 220, 220}% ancien~; pour compatibilité
\definecolor{mcbleu}{RGB}{52, 152, 219}
\definecolor{mcvert}{RGB}{125, 194, 70}
\definecolor{mcmauve}{RGB}{154, 0, 215}
\definecolor{mcorange}{RGB}{255, 96, 0}
\definecolor{mcturquoise}{RGB}{0, 153, 153}
\definecolor{mcrouge}{RGB}{255, 0, 0}
\definecolor{mclightvert}{RGB}{205, 234, 190}

\definecolor{gris}{RGB}{220, 220, 220}
\definecolor{bleu}{RGB}{52, 152, 219}
\definecolor{vert}{RGB}{125, 194, 70}
\definecolor{mauve}{RGB}{154, 0, 215}
\definecolor{orange}{RGB}{255, 96, 0}
\definecolor{turquoise}{RGB}{0, 153, 153}
\definecolor{rouge}{RGB}{255, 0, 0}
\definecolor{lightvert}{RGB}{205, 234, 190}
\setitemize[0]{label=\color{lightvert}  $\bullet$}

\pagestyle{fancy}
\renewcommand{\headrulewidth}{0.2pt}
\fancyhead[L]{maths-cours.fr}
\fancyhead[R]{\thepage}
\renewcommand{\footrulewidth}{0.2pt}
\fancyfoot[C]{}

\newcolumntype{C}{>{\centering\arraybackslash}X}
\newcolumntype{s}{>{\hsize=.35\hsize\arraybackslash}X}

\setlength{\parindent}{0pt}		 
\setlength{\parskip}{3mm}
\setlength{\headheight}{1cm}

\def\ebook{ebook}
\def\book{book}
\def\web{web}
\def\type{web}

\newcommand{\vect}[1]{\overrightarrow{\,\mathstrut#1\,}}

\def\Oij{$\left(\text{O}~;~\vect{\imath},~\vect{\jmath}\right)$}
\def\Oijk{$\left(\text{O}~;~\vect{\imath},~\vect{\jmath},~\vect{k}\right)$}
\def\Ouv{$\left(\text{O}~;~\vect{u},~\vect{v}\right)$}

\hypersetup{breaklinks=true, colorlinks = true, linkcolor = OliveGreen, urlcolor = OliveGreen, citecolor = OliveGreen, pdfauthor={Didier BONNEL - https://www.maths-cours.fr} } % supprime les bordures autour des liens

\renewcommand{\arg}[0]{\text{arg}}

\everymath{\displaystyle}

%================================================================================================================================
%
% Macros - Commandes
%
%================================================================================================================================

\newcommand\meta[2]{    			% Utilisé pour créer le post HTML.
	\def\titre{titre}
	\def\url{url}
	\def\arg{#1}
	\ifx\titre\arg
		\newcommand\maintitle{#2}
		\fancyhead[L]{#2}
		{\Large\sffamily \MakeUppercase{#2}}
		\vspace{1mm}\textcolor{mcvert}{\hrule}
	\fi 
	\ifx\url\arg
		\fancyfoot[L]{\href{https://www.maths-cours.fr#2}{\black \footnotesize{https://www.maths-cours.fr#2}}}
	\fi 
}


\newcommand\TitreC[1]{    		% Titre centré
     \needspace{3\baselineskip}
     \begin{center}\textbf{#1}\end{center}
}

\newcommand\newpar{    		% paragraphe
     \par
}

\newcommand\nosp {    		% commande vide (pas d'espace)
}
\newcommand{\id}[1]{} %ignore

\newcommand\boite[2]{				% Boite simple sans titre
	\vspace{5mm}
	\setlength{\fboxrule}{0.2mm}
	\setlength{\fboxsep}{5mm}	
	\fcolorbox{#1}{#1!3}{\makebox[\linewidth-2\fboxrule-2\fboxsep]{
  		\begin{minipage}[t]{\linewidth-2\fboxrule-4\fboxsep}\setlength{\parskip}{3mm}
  			 #2
  		\end{minipage}
	}}
	\vspace{5mm}
}

\newcommand\CBox[4]{				% Boites
	\vspace{5mm}
	\setlength{\fboxrule}{0.2mm}
	\setlength{\fboxsep}{5mm}
	
	\fcolorbox{#1}{#1!3}{\makebox[\linewidth-2\fboxrule-2\fboxsep]{
		\begin{minipage}[t]{1cm}\setlength{\parskip}{3mm}
	  		\textcolor{#1}{\LARGE{#2}}    
 	 	\end{minipage}  
  		\begin{minipage}[t]{\linewidth-2\fboxrule-4\fboxsep}\setlength{\parskip}{3mm}
			\raisebox{1.2mm}{\normalsize\sffamily{\textcolor{#1}{#3}}}						
  			 #4
  		\end{minipage}
	}}
	\vspace{5mm}
}

\newcommand\cadre[3]{				% Boites convertible html
	\par
	\vspace{2mm}
	\setlength{\fboxrule}{0.1mm}
	\setlength{\fboxsep}{5mm}
	\fcolorbox{#1}{white}{\makebox[\linewidth-2\fboxrule-2\fboxsep]{
  		\begin{minipage}[t]{\linewidth-2\fboxrule-4\fboxsep}\setlength{\parskip}{3mm}
			\raisebox{-2.5mm}{\sffamily \small{\textcolor{#1}{\MakeUppercase{#2}}}}		
			\par		
  			 #3
 	 		\end{minipage}
	}}
		\vspace{2mm}
	\par
}

\newcommand\bloc[3]{				% Boites convertible html sans bordure
     \needspace{2\baselineskip}
     {\sffamily \small{\textcolor{#1}{\MakeUppercase{#2}}}}    
		\par		
  			 #3
		\par
}

\newcommand\CHelp[1]{
     \CBox{Plum}{\faInfoCircle}{À RETENIR}{#1}
}

\newcommand\CUp[1]{
     \CBox{NavyBlue}{\faThumbsOUp}{EN PRATIQUE}{#1}
}

\newcommand\CInfo[1]{
     \CBox{Sepia}{\faArrowCircleRight}{REMARQUE}{#1}
}

\newcommand\CRedac[1]{
     \CBox{PineGreen}{\faEdit}{BIEN R\'EDIGER}{#1}
}

\newcommand\CError[1]{
     \CBox{Red}{\faExclamationTriangle}{ATTENTION}{#1}
}

\newcommand\TitreExo[2]{
\needspace{4\baselineskip}
 {\sffamily\large EXERCICE #1\ (\emph{#2 points})}
\vspace{5mm}
}

\newcommand\img[2]{
          \includegraphics[width=#2\paperwidth]{\imgdir#1}
}

\newcommand\imgsvg[2]{
       \begin{center}   \includegraphics[width=#2\paperwidth]{\imgsvgdir#1} \end{center}
}


\newcommand\Lien[2]{
     \href{#1}{#2 \tiny \faExternalLink}
}
\newcommand\mcLien[2]{
     \href{https~://www.maths-cours.fr/#1}{#2 \tiny \faExternalLink}
}

\newcommand{\euro}{\eurologo{}}

%================================================================================================================================
%
% Macros - Environement
%
%================================================================================================================================

\newenvironment{tex}{ %
}
{%
}

\newenvironment{indente}{ %
	\setlength\parindent{10mm}
}

{
	\setlength\parindent{0mm}
}

\newenvironment{corrige}{%
     \needspace{3\baselineskip}
     \medskip
     \textbf{\textsc{Corrigé}}
     \medskip
}
{
}

\newenvironment{extern}{%
     \begin{center}
     }
     {
     \end{center}
}

\NewEnviron{code}{%
	\par
     \boite{gray}{\texttt{%
     \BODY
     }}
     \par
}

\newenvironment{vbloc}{% boite sans cadre empeche saut de page
     \begin{minipage}[t]{\linewidth}
     }
     {
     \end{minipage}
}
\NewEnviron{h2}{%
    \needspace{3\baselineskip}
    \vspace{0.6cm}
	\noindent \MakeUppercase{\sffamily \large \BODY}
	\vspace{1mm}\textcolor{mcgris}{\hrule}\vspace{0.4cm}
	\par
}{}

\NewEnviron{h3}{%
    \needspace{3\baselineskip}
	\vspace{5mm}
	\textsc{\BODY}
	\par
}

\NewEnviron{margeneg}{ %
\begin{addmargin}[-1cm]{0cm}
\BODY
\end{addmargin}
}

\NewEnviron{html}{%
}

\begin{document}
\meta{url}{/exercices/thales-calcul-longueurs-141012/}
\meta{pid}{8311}
\meta{pi_}{1745}
\meta{titre}{Th. de Thalès (Brevet 2013)}
\meta{type}{exercices}
\textit{(D'après Brevet Centres étrangers 2013)}
\par
\textit{Dans cet exercice, toute trace de recherche, même incomplète, sera prise en compte dans l'évaluation.}
\medskip
On considère la figure ci-dessous, qui n'est pas en vraie grandeur.
\begin{center}
     \begin{extern}
          \psset{xunit=1.0cm,yunit=1.0cm,algebraic=true,dimen=middle,dotstyle=o,dotsize=5pt 0,linewidth=1.6pt,arrowsize=3pt 2,arrowinset=0.25}
          \newrgbcolor{tttttt}{0.2 0.2 0.2}
          \psset{xunit=1.0cm,yunit=1.0cm,algebraic=true,dimen=middle,dotstyle=o,dotsize=5pt 0,linewidth=1.6pt,arrowsize=3pt 2,arrowinset=0.25}
          \begin{pspicture*}(0.,0.)(11.,8.)
               \psline[linewidth=0.4pt,linecolor=tttttt](1.,7.)(10.,7.)
               \psline[linewidth=0.4pt,linecolor=tttttt](10.,7.)(10.,1.)
               \psline[linewidth=0.4pt,linecolor=tttttt](10.,1.)(4.,1.)
               \psline[linewidth=0.4pt,linecolor=tttttt](4.,1.)(4.,7.)
               \psline[linewidth=0.4pt,linecolor=tttttt](1.,7.)(10.,3.)
               \begin{scriptsize}
                    \psdots[dotsize=2pt 0,dotstyle=*,linecolor=tttttt](1.,7.)
                    \rput[bl](0.7560766474115809,7.1187630144416305){\tttttt{$A$}}
                    \psdots[dotsize=2pt 0,dotstyle=*,linecolor=tttttt](4.,7.)
                    \rput[bl](4.004426603925136,7.1187630144416305){\tttttt{$B$}}
                    \psdots[dotsize=2pt 0,dotstyle=*,linecolor=tttttt](10.,7.)
                    \rput[bl](10.045865348804515,7.044936879066323){\tttttt{$C$}}
                    \psdots[dotsize=2pt 0,dotstyle=*,linecolor=tttttt](10.,1.)
                    \rput[bl](10.070474060596284,0.9542807106034092){\tttttt{$D$}}
                    \psdots[dotsize=2pt 0,dotstyle=*,linecolor=tttttt](4.,1.)
                    \rput[bl](3.684513350632135,1.0281068459787173){\tttttt{$E$}}
                    \psdots[dotsize=2pt 0,dotstyle=*,linecolor=tttttt](10.,3.)
                    \rput[bl](10.095082772388054,2.9475863657367265){\tttttt{$F$}}
                    \psdots[dotsize=2pt 0,dotstyle=*,linecolor=darkgray](4.,5.666666666666667)
                    \rput[bl](4.053644027508676,5.716066442310778){\darkgray{$M$}}
               \end{scriptsize}
          \end{pspicture*}
     \end{extern}
\end{center}
\\
$BCDE$ est un carré de $6$ cm de côté.
\par
Les points $A$, $B$ et $C$ sont alignés et $AB=3$cm.
\par
$F$ est un point du segment $\left[CD\right]$.
\par
La droite $\left(AF\right)$ coupe le segment $\left[BE\right]$ en $M$.
\par
Déterminer la longueur $CF$ par calcul ou par construction pour que les longueurs $BM$ et $FD$ soient égales.
\begin{corrige}
     Notons $x=CF$.
     \par
     Comme les points $C, F$ et $D$ sont alignés :
     \par
     $FD=CD-CF=6-x     \qquad     $\textbf{(1)}
     \par
     Comme $BCDE$ est un carré, les droites $\left(BE\right)$ et $\left(CD\right)$ sont parallèles.
     \par
     D'après le théorème de Thalès :
     \par
     $\frac{AB}{AC}=\frac{AM}{AF}=\frac{BM}{CF}$
     \par
     $\frac{3}{9}=\frac{AM}{AF}=\frac{BM}{x}$
     \medskip
     De l'égalité $\frac{3}{9}=\frac{BM}{x}$, on déduit :
     \par
     $BM=\frac{3x}{9}=\frac{x}{3}  \qquad     $\textbf{(2)}
     \medskip
     En utilisant les égalités \textbf{(1)} et \textbf{(2)}, on peut dire que les longueurs $FD$ et $BM$ sont donc égales lorsque :
     \par
     $6-x=\frac{x}{3}$
     \par
     $6=\frac{x}{3}+x$
     \par
     $6=\frac{4}{3}x$
     \par
     $\frac{4}{3}x=6$
     \par
     $x=6\times \frac{3}{4}$
     \par
     $x=4,5$
     \medskip
     Les longueurs $BM$ et $FD$ sont donc égales lorsque  $CF$ vaut $4,5$cm.
\end{corrige}

\end{document}
µ
\documentclass[a4paper]{article}

%================================================================================================================================
%
% Packages
%
%================================================================================================================================

\usepackage[T1]{fontenc} 	% pour caractères accentués
\usepackage[utf8]{inputenc}  % encodage utf8
\usepackage[french]{babel}	% langue : français
\usepackage{fourier}			% caractères plus lisibles
\usepackage[dvipsnames]{xcolor} % couleurs
\usepackage{fancyhdr}		% réglage header footer
\usepackage{needspace}		% empêcher sauts de page mal placés
\usepackage{graphicx}		% pour inclure des graphiques
\usepackage{enumitem,cprotect}		% personnalise les listes d'items (nécessaire pour ol, al ...)
\usepackage{hyperref}		% Liens hypertexte
\usepackage{pstricks,pst-all,pst-node,pstricks-add,pst-math,pst-plot,pst-tree,pst-eucl} % pstricks
\usepackage[a4paper,includeheadfoot,top=2cm,left=3cm, bottom=2cm,right=3cm]{geometry} % marges etc.
\usepackage{comment}			% commentaires multilignes
\usepackage{amsmath,environ} % maths (matrices, etc.)
\usepackage{amssymb,makeidx}
\usepackage{bm}				% bold maths
\usepackage{tabularx}		% tableaux
\usepackage{colortbl}		% tableaux en couleur
\usepackage{fontawesome}		% Fontawesome
\usepackage{environ}			% environment with command
\usepackage{fp}				% calculs pour ps-tricks
\usepackage{multido}			% pour ps tricks
\usepackage[np]{numprint}	% formattage nombre
\usepackage{tikz,tkz-tab} 			% package principal TikZ
\usepackage{pgfplots}   % axes
\usepackage{mathrsfs}    % cursives
\usepackage{calc}			% calcul taille boites
\usepackage[scaled=0.875]{helvet} % font sans serif
\usepackage{svg} % svg
\usepackage{scrextend} % local margin
\usepackage{scratch} %scratch
\usepackage{multicol} % colonnes
%\usepackage{infix-RPN,pst-func} % formule en notation polanaise inversée
\usepackage{listings}

%================================================================================================================================
%
% Réglages de base
%
%================================================================================================================================

\lstset{
language=Python,   % R code
literate=
{á}{{\'a}}1
{à}{{\`a}}1
{ã}{{\~a}}1
{é}{{\'e}}1
{è}{{\`e}}1
{ê}{{\^e}}1
{í}{{\'i}}1
{ó}{{\'o}}1
{õ}{{\~o}}1
{ú}{{\'u}}1
{ü}{{\"u}}1
{ç}{{\c{c}}}1
{~}{{ }}1
}


\definecolor{codegreen}{rgb}{0,0.6,0}
\definecolor{codegray}{rgb}{0.5,0.5,0.5}
\definecolor{codepurple}{rgb}{0.58,0,0.82}
\definecolor{backcolour}{rgb}{0.95,0.95,0.92}

\lstdefinestyle{mystyle}{
    backgroundcolor=\color{backcolour},   
    commentstyle=\color{codegreen},
    keywordstyle=\color{magenta},
    numberstyle=\tiny\color{codegray},
    stringstyle=\color{codepurple},
    basicstyle=\ttfamily\footnotesize,
    breakatwhitespace=false,         
    breaklines=true,                 
    captionpos=b,                    
    keepspaces=true,                 
    numbers=left,                    
xleftmargin=2em,
framexleftmargin=2em,            
    showspaces=false,                
    showstringspaces=false,
    showtabs=false,                  
    tabsize=2,
    upquote=true
}

\lstset{style=mystyle}


\lstset{style=mystyle}
\newcommand{\imgdir}{C:/laragon/www/newmc/assets/imgsvg/}
\newcommand{\imgsvgdir}{C:/laragon/www/newmc/assets/imgsvg/}

\definecolor{mcgris}{RGB}{220, 220, 220}% ancien~; pour compatibilité
\definecolor{mcbleu}{RGB}{52, 152, 219}
\definecolor{mcvert}{RGB}{125, 194, 70}
\definecolor{mcmauve}{RGB}{154, 0, 215}
\definecolor{mcorange}{RGB}{255, 96, 0}
\definecolor{mcturquoise}{RGB}{0, 153, 153}
\definecolor{mcrouge}{RGB}{255, 0, 0}
\definecolor{mclightvert}{RGB}{205, 234, 190}

\definecolor{gris}{RGB}{220, 220, 220}
\definecolor{bleu}{RGB}{52, 152, 219}
\definecolor{vert}{RGB}{125, 194, 70}
\definecolor{mauve}{RGB}{154, 0, 215}
\definecolor{orange}{RGB}{255, 96, 0}
\definecolor{turquoise}{RGB}{0, 153, 153}
\definecolor{rouge}{RGB}{255, 0, 0}
\definecolor{lightvert}{RGB}{205, 234, 190}
\setitemize[0]{label=\color{lightvert}  $\bullet$}

\pagestyle{fancy}
\renewcommand{\headrulewidth}{0.2pt}
\fancyhead[L]{maths-cours.fr}
\fancyhead[R]{\thepage}
\renewcommand{\footrulewidth}{0.2pt}
\fancyfoot[C]{}

\newcolumntype{C}{>{\centering\arraybackslash}X}
\newcolumntype{s}{>{\hsize=.35\hsize\arraybackslash}X}

\setlength{\parindent}{0pt}		 
\setlength{\parskip}{3mm}
\setlength{\headheight}{1cm}

\def\ebook{ebook}
\def\book{book}
\def\web{web}
\def\type{web}

\newcommand{\vect}[1]{\overrightarrow{\,\mathstrut#1\,}}

\def\Oij{$\left(\text{O}~;~\vect{\imath},~\vect{\jmath}\right)$}
\def\Oijk{$\left(\text{O}~;~\vect{\imath},~\vect{\jmath},~\vect{k}\right)$}
\def\Ouv{$\left(\text{O}~;~\vect{u},~\vect{v}\right)$}

\hypersetup{breaklinks=true, colorlinks = true, linkcolor = OliveGreen, urlcolor = OliveGreen, citecolor = OliveGreen, pdfauthor={Didier BONNEL - https://www.maths-cours.fr} } % supprime les bordures autour des liens

\renewcommand{\arg}[0]{\text{arg}}

\everymath{\displaystyle}

%================================================================================================================================
%
% Macros - Commandes
%
%================================================================================================================================

\newcommand\meta[2]{    			% Utilisé pour créer le post HTML.
	\def\titre{titre}
	\def\url{url}
	\def\arg{#1}
	\ifx\titre\arg
		\newcommand\maintitle{#2}
		\fancyhead[L]{#2}
		{\Large\sffamily \MakeUppercase{#2}}
		\vspace{1mm}\textcolor{mcvert}{\hrule}
	\fi 
	\ifx\url\arg
		\fancyfoot[L]{\href{https://www.maths-cours.fr#2}{\black \footnotesize{https://www.maths-cours.fr#2}}}
	\fi 
}


\newcommand\TitreC[1]{    		% Titre centré
     \needspace{3\baselineskip}
     \begin{center}\textbf{#1}\end{center}
}

\newcommand\newpar{    		% paragraphe
     \par
}

\newcommand\nosp {    		% commande vide (pas d'espace)
}
\newcommand{\id}[1]{} %ignore

\newcommand\boite[2]{				% Boite simple sans titre
	\vspace{5mm}
	\setlength{\fboxrule}{0.2mm}
	\setlength{\fboxsep}{5mm}	
	\fcolorbox{#1}{#1!3}{\makebox[\linewidth-2\fboxrule-2\fboxsep]{
  		\begin{minipage}[t]{\linewidth-2\fboxrule-4\fboxsep}\setlength{\parskip}{3mm}
  			 #2
  		\end{minipage}
	}}
	\vspace{5mm}
}

\newcommand\CBox[4]{				% Boites
	\vspace{5mm}
	\setlength{\fboxrule}{0.2mm}
	\setlength{\fboxsep}{5mm}
	
	\fcolorbox{#1}{#1!3}{\makebox[\linewidth-2\fboxrule-2\fboxsep]{
		\begin{minipage}[t]{1cm}\setlength{\parskip}{3mm}
	  		\textcolor{#1}{\LARGE{#2}}    
 	 	\end{minipage}  
  		\begin{minipage}[t]{\linewidth-2\fboxrule-4\fboxsep}\setlength{\parskip}{3mm}
			\raisebox{1.2mm}{\normalsize\sffamily{\textcolor{#1}{#3}}}						
  			 #4
  		\end{minipage}
	}}
	\vspace{5mm}
}

\newcommand\cadre[3]{				% Boites convertible html
	\par
	\vspace{2mm}
	\setlength{\fboxrule}{0.1mm}
	\setlength{\fboxsep}{5mm}
	\fcolorbox{#1}{white}{\makebox[\linewidth-2\fboxrule-2\fboxsep]{
  		\begin{minipage}[t]{\linewidth-2\fboxrule-4\fboxsep}\setlength{\parskip}{3mm}
			\raisebox{-2.5mm}{\sffamily \small{\textcolor{#1}{\MakeUppercase{#2}}}}		
			\par		
  			 #3
 	 		\end{minipage}
	}}
		\vspace{2mm}
	\par
}

\newcommand\bloc[3]{				% Boites convertible html sans bordure
     \needspace{2\baselineskip}
     {\sffamily \small{\textcolor{#1}{\MakeUppercase{#2}}}}    
		\par		
  			 #3
		\par
}

\newcommand\CHelp[1]{
     \CBox{Plum}{\faInfoCircle}{À RETENIR}{#1}
}

\newcommand\CUp[1]{
     \CBox{NavyBlue}{\faThumbsOUp}{EN PRATIQUE}{#1}
}

\newcommand\CInfo[1]{
     \CBox{Sepia}{\faArrowCircleRight}{REMARQUE}{#1}
}

\newcommand\CRedac[1]{
     \CBox{PineGreen}{\faEdit}{BIEN R\'EDIGER}{#1}
}

\newcommand\CError[1]{
     \CBox{Red}{\faExclamationTriangle}{ATTENTION}{#1}
}

\newcommand\TitreExo[2]{
\needspace{4\baselineskip}
 {\sffamily\large EXERCICE #1\ (\emph{#2 points})}
\vspace{5mm}
}

\newcommand\img[2]{
          \includegraphics[width=#2\paperwidth]{\imgdir#1}
}

\newcommand\imgsvg[2]{
       \begin{center}   \includegraphics[width=#2\paperwidth]{\imgsvgdir#1} \end{center}
}


\newcommand\Lien[2]{
     \href{#1}{#2 \tiny \faExternalLink}
}
\newcommand\mcLien[2]{
     \href{https~://www.maths-cours.fr/#1}{#2 \tiny \faExternalLink}
}

\newcommand{\euro}{\eurologo{}}

%================================================================================================================================
%
% Macros - Environement
%
%================================================================================================================================

\newenvironment{tex}{ %
}
{%
}

\newenvironment{indente}{ %
	\setlength\parindent{10mm}
}

{
	\setlength\parindent{0mm}
}

\newenvironment{corrige}{%
     \needspace{3\baselineskip}
     \medskip
     \textbf{\textsc{Corrigé}}
     \medskip
}
{
}

\newenvironment{extern}{%
     \begin{center}
     }
     {
     \end{center}
}

\NewEnviron{code}{%
	\par
     \boite{gray}{\texttt{%
     \BODY
     }}
     \par
}

\newenvironment{vbloc}{% boite sans cadre empeche saut de page
     \begin{minipage}[t]{\linewidth}
     }
     {
     \end{minipage}
}
\NewEnviron{h2}{%
    \needspace{3\baselineskip}
    \vspace{0.6cm}
	\noindent \MakeUppercase{\sffamily \large \BODY}
	\vspace{1mm}\textcolor{mcgris}{\hrule}\vspace{0.4cm}
	\par
}{}

\NewEnviron{h3}{%
    \needspace{3\baselineskip}
	\vspace{5mm}
	\textsc{\BODY}
	\par
}

\NewEnviron{margeneg}{ %
\begin{addmargin}[-1cm]{0cm}
\BODY
\end{addmargin}
}

\NewEnviron{html}{%
}

\begin{document}
\meta{url}{/exercices/mesure-arbre-141012/}
\meta{pid}{8311}
\meta{pi_}{1743}
\meta{titre}{Mesure d'un arbre (Brevet 2013)}
\meta{type}{exercices}
\textit{(D'après Brevet Polynésie 2013)}
\medskip
Teiki se promène en montagne et aimerait connaître la hauteur d'un Pinus (ou Pin des Caraïbes) situé devant lui. Pour cela, il utilise un bâton et prend quelques mesures au sol.
\par
Il procède de la façon suivante :
\begin{itemize}
     \item Il pique le bâton en terre, verticalement, à 12 mètres du Pinus.
     \item La partie visible (hors du sol) du bâton mesure 2 m.
     \item Teiki se place derrière le bâton, de façon à ce que son œil, situé à 1,60 m au dessus du sol, voit en alignement le sommet de l'arbre et l'extrémité du bâton.
     \item Teiki marque sa position au sol, puis mesure la distance entre sa position et le bâton. Il trouve alors 1,2 m.
\end{itemize}
On peut représenter cette situation à l'aide du schéma ci-dessous :
\begin{center}
     \img{mc-0528}{0.1}%width="400" alt="exercice théorème de Thalès
     Brevet Métropole 2018"
\end{center}
Quelle est la hauteur du Pinus au-dessus du sol ?
\begin{corrige}
     \begin{center}
          \img{mc-0529}{0.1}%width="400" alt="corrigé théorème de Thalès
          Brevet Métropole 2018"
     \end{center}
     On peut modéliser la situation à l'aide de la figure ci-dessus où $\left[AE\right]$ représente l'arbre et $\left[FG\right]$ le bâton.
     \par
     D'après les données de l'énoncé on a :
     \begin{itemize}
          \item $EF=BH=12$m
          \item $GF=2$m
          \item $HF=CD=1,6$m
          \item $FD=HC=1,2$m
     \end{itemize}
     On cherche à calculer la hauteur de l'arbre c'est à dire la longueur $AE$.
     \par
     Les points $F, H$ et $G$ étant alignés :
     \par
     $GH=GF-HF=2-1,6=0,4$
     \par
     Les points $E, F$ et $D$ étant alignés :
     \par
     $ED=EF+FD=12+1,2=13,2$ et par conséquent $BC=13,2$
     \par
     Les droites $\left(AE\right)$ et $\left(GF\right)$, étant toutes deux verticales, sont parallèles ; donc d'après le théorème de Thalès :
     \par
     $\frac{GC}{AC}=\frac{HC}{BC}=\frac{GH}{AB}$
     \par
     $\frac{GC}{AC}=\frac{1,2}{13,2}=\frac{0,4}{AB}$
     \par
     De l'égalité des rapports $\frac{1,2}{13,2}=\frac{0,4}{AB}$ on déduit :
     \par
     $AB=\frac{0,4\times 13,2}{1,2}=4,4$
     \par
     La hauteur totale de l'arbre est donc :
     \par
     $AE=AB+BE=4,4+1,6=6$m
     \par
     La hauteur du Pinus au-dessus du sol est $6$ mètres.
\end{corrige}

\end{document}
µ
\documentclass[a4paper]{article}

%================================================================================================================================
%
% Packages
%
%================================================================================================================================

\usepackage[T1]{fontenc} 	% pour caractères accentués
\usepackage[utf8]{inputenc}  % encodage utf8
\usepackage[french]{babel}	% langue : français
\usepackage{fourier}			% caractères plus lisibles
\usepackage[dvipsnames]{xcolor} % couleurs
\usepackage{fancyhdr}		% réglage header footer
\usepackage{needspace}		% empêcher sauts de page mal placés
\usepackage{graphicx}		% pour inclure des graphiques
\usepackage{enumitem,cprotect}		% personnalise les listes d'items (nécessaire pour ol, al ...)
\usepackage{hyperref}		% Liens hypertexte
\usepackage{pstricks,pst-all,pst-node,pstricks-add,pst-math,pst-plot,pst-tree,pst-eucl} % pstricks
\usepackage[a4paper,includeheadfoot,top=2cm,left=3cm, bottom=2cm,right=3cm]{geometry} % marges etc.
\usepackage{comment}			% commentaires multilignes
\usepackage{amsmath,environ} % maths (matrices, etc.)
\usepackage{amssymb,makeidx}
\usepackage{bm}				% bold maths
\usepackage{tabularx}		% tableaux
\usepackage{colortbl}		% tableaux en couleur
\usepackage{fontawesome}		% Fontawesome
\usepackage{environ}			% environment with command
\usepackage{fp}				% calculs pour ps-tricks
\usepackage{multido}			% pour ps tricks
\usepackage[np]{numprint}	% formattage nombre
\usepackage{tikz,tkz-tab} 			% package principal TikZ
\usepackage{pgfplots}   % axes
\usepackage{mathrsfs}    % cursives
\usepackage{calc}			% calcul taille boites
\usepackage[scaled=0.875]{helvet} % font sans serif
\usepackage{svg} % svg
\usepackage{scrextend} % local margin
\usepackage{scratch} %scratch
\usepackage{multicol} % colonnes
%\usepackage{infix-RPN,pst-func} % formule en notation polanaise inversée
\usepackage{listings}

%================================================================================================================================
%
% Réglages de base
%
%================================================================================================================================

\lstset{
language=Python,   % R code
literate=
{á}{{\'a}}1
{à}{{\`a}}1
{ã}{{\~a}}1
{é}{{\'e}}1
{è}{{\`e}}1
{ê}{{\^e}}1
{í}{{\'i}}1
{ó}{{\'o}}1
{õ}{{\~o}}1
{ú}{{\'u}}1
{ü}{{\"u}}1
{ç}{{\c{c}}}1
{~}{{ }}1
}


\definecolor{codegreen}{rgb}{0,0.6,0}
\definecolor{codegray}{rgb}{0.5,0.5,0.5}
\definecolor{codepurple}{rgb}{0.58,0,0.82}
\definecolor{backcolour}{rgb}{0.95,0.95,0.92}

\lstdefinestyle{mystyle}{
    backgroundcolor=\color{backcolour},   
    commentstyle=\color{codegreen},
    keywordstyle=\color{magenta},
    numberstyle=\tiny\color{codegray},
    stringstyle=\color{codepurple},
    basicstyle=\ttfamily\footnotesize,
    breakatwhitespace=false,         
    breaklines=true,                 
    captionpos=b,                    
    keepspaces=true,                 
    numbers=left,                    
xleftmargin=2em,
framexleftmargin=2em,            
    showspaces=false,                
    showstringspaces=false,
    showtabs=false,                  
    tabsize=2,
    upquote=true
}

\lstset{style=mystyle}


\lstset{style=mystyle}
\newcommand{\imgdir}{C:/laragon/www/newmc/assets/imgsvg/}
\newcommand{\imgsvgdir}{C:/laragon/www/newmc/assets/imgsvg/}

\definecolor{mcgris}{RGB}{220, 220, 220}% ancien~; pour compatibilité
\definecolor{mcbleu}{RGB}{52, 152, 219}
\definecolor{mcvert}{RGB}{125, 194, 70}
\definecolor{mcmauve}{RGB}{154, 0, 215}
\definecolor{mcorange}{RGB}{255, 96, 0}
\definecolor{mcturquoise}{RGB}{0, 153, 153}
\definecolor{mcrouge}{RGB}{255, 0, 0}
\definecolor{mclightvert}{RGB}{205, 234, 190}

\definecolor{gris}{RGB}{220, 220, 220}
\definecolor{bleu}{RGB}{52, 152, 219}
\definecolor{vert}{RGB}{125, 194, 70}
\definecolor{mauve}{RGB}{154, 0, 215}
\definecolor{orange}{RGB}{255, 96, 0}
\definecolor{turquoise}{RGB}{0, 153, 153}
\definecolor{rouge}{RGB}{255, 0, 0}
\definecolor{lightvert}{RGB}{205, 234, 190}
\setitemize[0]{label=\color{lightvert}  $\bullet$}

\pagestyle{fancy}
\renewcommand{\headrulewidth}{0.2pt}
\fancyhead[L]{maths-cours.fr}
\fancyhead[R]{\thepage}
\renewcommand{\footrulewidth}{0.2pt}
\fancyfoot[C]{}

\newcolumntype{C}{>{\centering\arraybackslash}X}
\newcolumntype{s}{>{\hsize=.35\hsize\arraybackslash}X}

\setlength{\parindent}{0pt}		 
\setlength{\parskip}{3mm}
\setlength{\headheight}{1cm}

\def\ebook{ebook}
\def\book{book}
\def\web{web}
\def\type{web}

\newcommand{\vect}[1]{\overrightarrow{\,\mathstrut#1\,}}

\def\Oij{$\left(\text{O}~;~\vect{\imath},~\vect{\jmath}\right)$}
\def\Oijk{$\left(\text{O}~;~\vect{\imath},~\vect{\jmath},~\vect{k}\right)$}
\def\Ouv{$\left(\text{O}~;~\vect{u},~\vect{v}\right)$}

\hypersetup{breaklinks=true, colorlinks = true, linkcolor = OliveGreen, urlcolor = OliveGreen, citecolor = OliveGreen, pdfauthor={Didier BONNEL - https://www.maths-cours.fr} } % supprime les bordures autour des liens

\renewcommand{\arg}[0]{\text{arg}}

\everymath{\displaystyle}

%================================================================================================================================
%
% Macros - Commandes
%
%================================================================================================================================

\newcommand\meta[2]{    			% Utilisé pour créer le post HTML.
	\def\titre{titre}
	\def\url{url}
	\def\arg{#1}
	\ifx\titre\arg
		\newcommand\maintitle{#2}
		\fancyhead[L]{#2}
		{\Large\sffamily \MakeUppercase{#2}}
		\vspace{1mm}\textcolor{mcvert}{\hrule}
	\fi 
	\ifx\url\arg
		\fancyfoot[L]{\href{https://www.maths-cours.fr#2}{\black \footnotesize{https://www.maths-cours.fr#2}}}
	\fi 
}


\newcommand\TitreC[1]{    		% Titre centré
     \needspace{3\baselineskip}
     \begin{center}\textbf{#1}\end{center}
}

\newcommand\newpar{    		% paragraphe
     \par
}

\newcommand\nosp {    		% commande vide (pas d'espace)
}
\newcommand{\id}[1]{} %ignore

\newcommand\boite[2]{				% Boite simple sans titre
	\vspace{5mm}
	\setlength{\fboxrule}{0.2mm}
	\setlength{\fboxsep}{5mm}	
	\fcolorbox{#1}{#1!3}{\makebox[\linewidth-2\fboxrule-2\fboxsep]{
  		\begin{minipage}[t]{\linewidth-2\fboxrule-4\fboxsep}\setlength{\parskip}{3mm}
  			 #2
  		\end{minipage}
	}}
	\vspace{5mm}
}

\newcommand\CBox[4]{				% Boites
	\vspace{5mm}
	\setlength{\fboxrule}{0.2mm}
	\setlength{\fboxsep}{5mm}
	
	\fcolorbox{#1}{#1!3}{\makebox[\linewidth-2\fboxrule-2\fboxsep]{
		\begin{minipage}[t]{1cm}\setlength{\parskip}{3mm}
	  		\textcolor{#1}{\LARGE{#2}}    
 	 	\end{minipage}  
  		\begin{minipage}[t]{\linewidth-2\fboxrule-4\fboxsep}\setlength{\parskip}{3mm}
			\raisebox{1.2mm}{\normalsize\sffamily{\textcolor{#1}{#3}}}						
  			 #4
  		\end{minipage}
	}}
	\vspace{5mm}
}

\newcommand\cadre[3]{				% Boites convertible html
	\par
	\vspace{2mm}
	\setlength{\fboxrule}{0.1mm}
	\setlength{\fboxsep}{5mm}
	\fcolorbox{#1}{white}{\makebox[\linewidth-2\fboxrule-2\fboxsep]{
  		\begin{minipage}[t]{\linewidth-2\fboxrule-4\fboxsep}\setlength{\parskip}{3mm}
			\raisebox{-2.5mm}{\sffamily \small{\textcolor{#1}{\MakeUppercase{#2}}}}		
			\par		
  			 #3
 	 		\end{minipage}
	}}
		\vspace{2mm}
	\par
}

\newcommand\bloc[3]{				% Boites convertible html sans bordure
     \needspace{2\baselineskip}
     {\sffamily \small{\textcolor{#1}{\MakeUppercase{#2}}}}    
		\par		
  			 #3
		\par
}

\newcommand\CHelp[1]{
     \CBox{Plum}{\faInfoCircle}{À RETENIR}{#1}
}

\newcommand\CUp[1]{
     \CBox{NavyBlue}{\faThumbsOUp}{EN PRATIQUE}{#1}
}

\newcommand\CInfo[1]{
     \CBox{Sepia}{\faArrowCircleRight}{REMARQUE}{#1}
}

\newcommand\CRedac[1]{
     \CBox{PineGreen}{\faEdit}{BIEN R\'EDIGER}{#1}
}

\newcommand\CError[1]{
     \CBox{Red}{\faExclamationTriangle}{ATTENTION}{#1}
}

\newcommand\TitreExo[2]{
\needspace{4\baselineskip}
 {\sffamily\large EXERCICE #1\ (\emph{#2 points})}
\vspace{5mm}
}

\newcommand\img[2]{
          \includegraphics[width=#2\paperwidth]{\imgdir#1}
}

\newcommand\imgsvg[2]{
       \begin{center}   \includegraphics[width=#2\paperwidth]{\imgsvgdir#1} \end{center}
}


\newcommand\Lien[2]{
     \href{#1}{#2 \tiny \faExternalLink}
}
\newcommand\mcLien[2]{
     \href{https~://www.maths-cours.fr/#1}{#2 \tiny \faExternalLink}
}

\newcommand{\euro}{\eurologo{}}

%================================================================================================================================
%
% Macros - Environement
%
%================================================================================================================================

\newenvironment{tex}{ %
}
{%
}

\newenvironment{indente}{ %
	\setlength\parindent{10mm}
}

{
	\setlength\parindent{0mm}
}

\newenvironment{corrige}{%
     \needspace{3\baselineskip}
     \medskip
     \textbf{\textsc{Corrigé}}
     \medskip
}
{
}

\newenvironment{extern}{%
     \begin{center}
     }
     {
     \end{center}
}

\NewEnviron{code}{%
	\par
     \boite{gray}{\texttt{%
     \BODY
     }}
     \par
}

\newenvironment{vbloc}{% boite sans cadre empeche saut de page
     \begin{minipage}[t]{\linewidth}
     }
     {
     \end{minipage}
}
\NewEnviron{h2}{%
    \needspace{3\baselineskip}
    \vspace{0.6cm}
	\noindent \MakeUppercase{\sffamily \large \BODY}
	\vspace{1mm}\textcolor{mcgris}{\hrule}\vspace{0.4cm}
	\par
}{}

\NewEnviron{h3}{%
    \needspace{3\baselineskip}
	\vspace{5mm}
	\textsc{\BODY}
	\par
}

\NewEnviron{margeneg}{ %
\begin{addmargin}[-1cm]{0cm}
\BODY
\end{addmargin}
}

\NewEnviron{html}{%
}

\begin{document}
\meta{url}{/exercices/mesure-arbre-141012/}
\meta{pid}{8311}
\meta{pi_}{1743}
\meta{titre}{Mesure d'un arbre (Brevet 2013)}
\meta{type}{exercices}
\textit{(D'après Brevet Polynésie 2013)}
\medskip
Teiki se promène en montagne et aimerait connaître la hauteur d'un Pinus (ou Pin des Caraïbes) situé devant lui. Pour cela, il utilise un bâton et prend quelques mesures au sol.
\par
Il procède de la façon suivante :
\begin{itemize}
     \item Il pique le bâton en terre, verticalement, à 12 mètres du Pinus.
     \item La partie visible (hors du sol) du bâton mesure 2 m.
     \item Teiki se place derrière le bâton, de façon à ce que son œil, situé à 1,60 m au dessus du sol, voit en alignement le sommet de l'arbre et l'extrémité du bâton.
     \item Teiki marque sa position au sol, puis mesure la distance entre sa position et le bâton. Il trouve alors 1,2 m.
\end{itemize}
On peut représenter cette situation à l'aide du schéma ci-dessous :
\begin{center}
     \img{mc-0528}{0.1}%width="400" alt="exercice théorème de Thalès
     Brevet Métropole 2018"
\end{center}
Quelle est la hauteur du Pinus au-dessus du sol ?
\begin{corrige}
     \begin{center}
          \img{mc-0529}{0.1}%width="400" alt="corrigé théorème de Thalès
          Brevet Métropole 2018"
     \end{center}
     On peut modéliser la situation à l'aide de la figure ci-dessus où $\left[AE\right]$ représente l'arbre et $\left[FG\right]$ le bâton.
     \par
     D'après les données de l'énoncé on a :
     \begin{itemize}
          \item $EF=BH=12$m
          \item $GF=2$m
          \item $HF=CD=1,6$m
          \item $FD=HC=1,2$m
     \end{itemize}
     On cherche à calculer la hauteur de l'arbre c'est à dire la longueur $AE$.
     \par
     Les points $F, H$ et $G$ étant alignés :
     \par
     $GH=GF-HF=2-1,6=0,4$
     \par
     Les points $E, F$ et $D$ étant alignés :
     \par
     $ED=EF+FD=12+1,2=13,2$ et par conséquent $BC=13,2$
     \par
     Les droites $\left(AE\right)$ et $\left(GF\right)$, étant toutes deux verticales, sont parallèles ; donc d'après le théorème de Thalès :
     \par
     $\frac{GC}{AC}=\frac{HC}{BC}=\frac{GH}{AB}$
     \par
     $\frac{GC}{AC}=\frac{1,2}{13,2}=\frac{0,4}{AB}$
     \par
     De l'égalité des rapports $\frac{1,2}{13,2}=\frac{0,4}{AB}$ on déduit :
     \par
     $AB=\frac{0,4\times 13,2}{1,2}=4,4$
     \par
     La hauteur totale de l'arbre est donc :
     \par
     $AE=AB+BE=4,4+1,6=6$m
     \par
     La hauteur du Pinus au-dessus du sol est $6$ mètres.
\end{corrige}

\end{document}
µ
\documentclass[a4paper]{article}

%================================================================================================================================
%
% Packages
%
%================================================================================================================================

\usepackage[T1]{fontenc} 	% pour caractères accentués
\usepackage[utf8]{inputenc}  % encodage utf8
\usepackage[french]{babel}	% langue : français
\usepackage{fourier}			% caractères plus lisibles
\usepackage[dvipsnames]{xcolor} % couleurs
\usepackage{fancyhdr}		% réglage header footer
\usepackage{needspace}		% empêcher sauts de page mal placés
\usepackage{graphicx}		% pour inclure des graphiques
\usepackage{enumitem,cprotect}		% personnalise les listes d'items (nécessaire pour ol, al ...)
\usepackage{hyperref}		% Liens hypertexte
\usepackage{pstricks,pst-all,pst-node,pstricks-add,pst-math,pst-plot,pst-tree,pst-eucl} % pstricks
\usepackage[a4paper,includeheadfoot,top=2cm,left=3cm, bottom=2cm,right=3cm]{geometry} % marges etc.
\usepackage{comment}			% commentaires multilignes
\usepackage{amsmath,environ} % maths (matrices, etc.)
\usepackage{amssymb,makeidx}
\usepackage{bm}				% bold maths
\usepackage{tabularx}		% tableaux
\usepackage{colortbl}		% tableaux en couleur
\usepackage{fontawesome}		% Fontawesome
\usepackage{environ}			% environment with command
\usepackage{fp}				% calculs pour ps-tricks
\usepackage{multido}			% pour ps tricks
\usepackage[np]{numprint}	% formattage nombre
\usepackage{tikz,tkz-tab} 			% package principal TikZ
\usepackage{pgfplots}   % axes
\usepackage{mathrsfs}    % cursives
\usepackage{calc}			% calcul taille boites
\usepackage[scaled=0.875]{helvet} % font sans serif
\usepackage{svg} % svg
\usepackage{scrextend} % local margin
\usepackage{scratch} %scratch
\usepackage{multicol} % colonnes
%\usepackage{infix-RPN,pst-func} % formule en notation polanaise inversée
\usepackage{listings}

%================================================================================================================================
%
% Réglages de base
%
%================================================================================================================================

\lstset{
language=Python,   % R code
literate=
{á}{{\'a}}1
{à}{{\`a}}1
{ã}{{\~a}}1
{é}{{\'e}}1
{è}{{\`e}}1
{ê}{{\^e}}1
{í}{{\'i}}1
{ó}{{\'o}}1
{õ}{{\~o}}1
{ú}{{\'u}}1
{ü}{{\"u}}1
{ç}{{\c{c}}}1
{~}{{ }}1
}


\definecolor{codegreen}{rgb}{0,0.6,0}
\definecolor{codegray}{rgb}{0.5,0.5,0.5}
\definecolor{codepurple}{rgb}{0.58,0,0.82}
\definecolor{backcolour}{rgb}{0.95,0.95,0.92}

\lstdefinestyle{mystyle}{
    backgroundcolor=\color{backcolour},   
    commentstyle=\color{codegreen},
    keywordstyle=\color{magenta},
    numberstyle=\tiny\color{codegray},
    stringstyle=\color{codepurple},
    basicstyle=\ttfamily\footnotesize,
    breakatwhitespace=false,         
    breaklines=true,                 
    captionpos=b,                    
    keepspaces=true,                 
    numbers=left,                    
xleftmargin=2em,
framexleftmargin=2em,            
    showspaces=false,                
    showstringspaces=false,
    showtabs=false,                  
    tabsize=2,
    upquote=true
}

\lstset{style=mystyle}


\lstset{style=mystyle}
\newcommand{\imgdir}{C:/laragon/www/newmc/assets/imgsvg/}
\newcommand{\imgsvgdir}{C:/laragon/www/newmc/assets/imgsvg/}

\definecolor{mcgris}{RGB}{220, 220, 220}% ancien~; pour compatibilité
\definecolor{mcbleu}{RGB}{52, 152, 219}
\definecolor{mcvert}{RGB}{125, 194, 70}
\definecolor{mcmauve}{RGB}{154, 0, 215}
\definecolor{mcorange}{RGB}{255, 96, 0}
\definecolor{mcturquoise}{RGB}{0, 153, 153}
\definecolor{mcrouge}{RGB}{255, 0, 0}
\definecolor{mclightvert}{RGB}{205, 234, 190}

\definecolor{gris}{RGB}{220, 220, 220}
\definecolor{bleu}{RGB}{52, 152, 219}
\definecolor{vert}{RGB}{125, 194, 70}
\definecolor{mauve}{RGB}{154, 0, 215}
\definecolor{orange}{RGB}{255, 96, 0}
\definecolor{turquoise}{RGB}{0, 153, 153}
\definecolor{rouge}{RGB}{255, 0, 0}
\definecolor{lightvert}{RGB}{205, 234, 190}
\setitemize[0]{label=\color{lightvert}  $\bullet$}

\pagestyle{fancy}
\renewcommand{\headrulewidth}{0.2pt}
\fancyhead[L]{maths-cours.fr}
\fancyhead[R]{\thepage}
\renewcommand{\footrulewidth}{0.2pt}
\fancyfoot[C]{}

\newcolumntype{C}{>{\centering\arraybackslash}X}
\newcolumntype{s}{>{\hsize=.35\hsize\arraybackslash}X}

\setlength{\parindent}{0pt}		 
\setlength{\parskip}{3mm}
\setlength{\headheight}{1cm}

\def\ebook{ebook}
\def\book{book}
\def\web{web}
\def\type{web}

\newcommand{\vect}[1]{\overrightarrow{\,\mathstrut#1\,}}

\def\Oij{$\left(\text{O}~;~\vect{\imath},~\vect{\jmath}\right)$}
\def\Oijk{$\left(\text{O}~;~\vect{\imath},~\vect{\jmath},~\vect{k}\right)$}
\def\Ouv{$\left(\text{O}~;~\vect{u},~\vect{v}\right)$}

\hypersetup{breaklinks=true, colorlinks = true, linkcolor = OliveGreen, urlcolor = OliveGreen, citecolor = OliveGreen, pdfauthor={Didier BONNEL - https://www.maths-cours.fr} } % supprime les bordures autour des liens

\renewcommand{\arg}[0]{\text{arg}}

\everymath{\displaystyle}

%================================================================================================================================
%
% Macros - Commandes
%
%================================================================================================================================

\newcommand\meta[2]{    			% Utilisé pour créer le post HTML.
	\def\titre{titre}
	\def\url{url}
	\def\arg{#1}
	\ifx\titre\arg
		\newcommand\maintitle{#2}
		\fancyhead[L]{#2}
		{\Large\sffamily \MakeUppercase{#2}}
		\vspace{1mm}\textcolor{mcvert}{\hrule}
	\fi 
	\ifx\url\arg
		\fancyfoot[L]{\href{https://www.maths-cours.fr#2}{\black \footnotesize{https://www.maths-cours.fr#2}}}
	\fi 
}


\newcommand\TitreC[1]{    		% Titre centré
     \needspace{3\baselineskip}
     \begin{center}\textbf{#1}\end{center}
}

\newcommand\newpar{    		% paragraphe
     \par
}

\newcommand\nosp {    		% commande vide (pas d'espace)
}
\newcommand{\id}[1]{} %ignore

\newcommand\boite[2]{				% Boite simple sans titre
	\vspace{5mm}
	\setlength{\fboxrule}{0.2mm}
	\setlength{\fboxsep}{5mm}	
	\fcolorbox{#1}{#1!3}{\makebox[\linewidth-2\fboxrule-2\fboxsep]{
  		\begin{minipage}[t]{\linewidth-2\fboxrule-4\fboxsep}\setlength{\parskip}{3mm}
  			 #2
  		\end{minipage}
	}}
	\vspace{5mm}
}

\newcommand\CBox[4]{				% Boites
	\vspace{5mm}
	\setlength{\fboxrule}{0.2mm}
	\setlength{\fboxsep}{5mm}
	
	\fcolorbox{#1}{#1!3}{\makebox[\linewidth-2\fboxrule-2\fboxsep]{
		\begin{minipage}[t]{1cm}\setlength{\parskip}{3mm}
	  		\textcolor{#1}{\LARGE{#2}}    
 	 	\end{minipage}  
  		\begin{minipage}[t]{\linewidth-2\fboxrule-4\fboxsep}\setlength{\parskip}{3mm}
			\raisebox{1.2mm}{\normalsize\sffamily{\textcolor{#1}{#3}}}						
  			 #4
  		\end{minipage}
	}}
	\vspace{5mm}
}

\newcommand\cadre[3]{				% Boites convertible html
	\par
	\vspace{2mm}
	\setlength{\fboxrule}{0.1mm}
	\setlength{\fboxsep}{5mm}
	\fcolorbox{#1}{white}{\makebox[\linewidth-2\fboxrule-2\fboxsep]{
  		\begin{minipage}[t]{\linewidth-2\fboxrule-4\fboxsep}\setlength{\parskip}{3mm}
			\raisebox{-2.5mm}{\sffamily \small{\textcolor{#1}{\MakeUppercase{#2}}}}		
			\par		
  			 #3
 	 		\end{minipage}
	}}
		\vspace{2mm}
	\par
}

\newcommand\bloc[3]{				% Boites convertible html sans bordure
     \needspace{2\baselineskip}
     {\sffamily \small{\textcolor{#1}{\MakeUppercase{#2}}}}    
		\par		
  			 #3
		\par
}

\newcommand\CHelp[1]{
     \CBox{Plum}{\faInfoCircle}{À RETENIR}{#1}
}

\newcommand\CUp[1]{
     \CBox{NavyBlue}{\faThumbsOUp}{EN PRATIQUE}{#1}
}

\newcommand\CInfo[1]{
     \CBox{Sepia}{\faArrowCircleRight}{REMARQUE}{#1}
}

\newcommand\CRedac[1]{
     \CBox{PineGreen}{\faEdit}{BIEN R\'EDIGER}{#1}
}

\newcommand\CError[1]{
     \CBox{Red}{\faExclamationTriangle}{ATTENTION}{#1}
}

\newcommand\TitreExo[2]{
\needspace{4\baselineskip}
 {\sffamily\large EXERCICE #1\ (\emph{#2 points})}
\vspace{5mm}
}

\newcommand\img[2]{
          \includegraphics[width=#2\paperwidth]{\imgdir#1}
}

\newcommand\imgsvg[2]{
       \begin{center}   \includegraphics[width=#2\paperwidth]{\imgsvgdir#1} \end{center}
}


\newcommand\Lien[2]{
     \href{#1}{#2 \tiny \faExternalLink}
}
\newcommand\mcLien[2]{
     \href{https~://www.maths-cours.fr/#1}{#2 \tiny \faExternalLink}
}

\newcommand{\euro}{\eurologo{}}

%================================================================================================================================
%
% Macros - Environement
%
%================================================================================================================================

\newenvironment{tex}{ %
}
{%
}

\newenvironment{indente}{ %
	\setlength\parindent{10mm}
}

{
	\setlength\parindent{0mm}
}

\newenvironment{corrige}{%
     \needspace{3\baselineskip}
     \medskip
     \textbf{\textsc{Corrigé}}
     \medskip
}
{
}

\newenvironment{extern}{%
     \begin{center}
     }
     {
     \end{center}
}

\NewEnviron{code}{%
	\par
     \boite{gray}{\texttt{%
     \BODY
     }}
     \par
}

\newenvironment{vbloc}{% boite sans cadre empeche saut de page
     \begin{minipage}[t]{\linewidth}
     }
     {
     \end{minipage}
}
\NewEnviron{h2}{%
    \needspace{3\baselineskip}
    \vspace{0.6cm}
	\noindent \MakeUppercase{\sffamily \large \BODY}
	\vspace{1mm}\textcolor{mcgris}{\hrule}\vspace{0.4cm}
	\par
}{}

\NewEnviron{h3}{%
    \needspace{3\baselineskip}
	\vspace{5mm}
	\textsc{\BODY}
	\par
}

\NewEnviron{margeneg}{ %
\begin{addmargin}[-1cm]{0cm}
\BODY
\end{addmargin}
}

\NewEnviron{html}{%
}

\begin{document}
\meta{url}{/methode/calculer-des-longueurs-avec-le-theoreme-de-thales/}
\meta{pid}{11534}
\meta{titre}{Calculer des longueurs avec le théorème de Thalès}
\meta{type}{methode}
%
\begin{h2} Méthode~: \end{h2}
$ ABC $ et $ DCE $ sont deux triangles tels que~:
\begin{itemize}
     \item
     les points $ C, A, E $ et les points $ C, B, D $ sont alignés
     \item
     les droites $ \left( AB \right) $ et $ \left( ED \right) $ sont \textbf{ parallèles}.
\end{itemize}
\medskip
Deux cas de figure sont possibles~:
\begin{multicols}{2}
     \begin{center}
          \begin{extern}%width="300" alt="configuration de Thalès n°1"
               \psset{xunit=1.0cm,yunit=1.0cm,algebraic=true,dimen=middle,dotstyle=o,dotsize=5pt 0,linewidth=1.6pt,arrowsize=3pt 2,arrowinset=0.25}
               \newrgbcolor{tttttt}{0.2 0.2 0.2}
               \psset{xunit=1.0cm,yunit=1.0cm,algebraic=true,dimen=middle,dotstyle=o,dotsize=5pt 0,linewidth=1.6pt,arrowsize=3pt 2,arrowinset=0.25}
               \begin{pspicture*}(5.,1.)(11.,9.)
                    \psline[linewidth=0.4pt,linecolor=tttttt](6.,5.)(8.,4.)
                    \psline[linewidth=0.4pt,linecolor=tttttt](6.,3.)(10.,5.)
                    \psline[linewidth=0.4pt,linecolor=tttttt](10.,5.)(6.,7.)
                    \psline[linewidth=0.4pt,linecolor=tttttt](6.,7.)(6.,3.)
                    \begin{scriptsize}
                         \psdots[dotsize=2pt 0,dotstyle=*,linecolor=tttttt](6.,5.)
                         \rput[bl](5.677819005765453,5.027022512141235){\tttttt{$A$}}
                         \psdots[dotsize=2pt 0,dotstyle=*,linecolor=tttttt](8.,4.)
                         \rput[bl](8.150994540838274,3.9073261256157292){\tttttt{$B$}}
                         \psdots[dotsize=2pt 0,dotstyle=*,linecolor=tttttt](6.,7.)
                         \rput[bl](5.727036429348992,7.06954559085809){\tttttt{$E$}}
                         \psdots[dotsize=2pt 0,dotstyle=*](10.,5.)
                         \rput[bl](10.045865348804515,5.051631223933004){\tttttt{$D$}}
                         \psdots[dotsize=2pt 0,dotstyle=*,linecolor=tttttt](6.,3.)
                         \rput[bl](5.628601582181914,2.7999340949861087){\tttttt{$C$}}
                    \end{scriptsize}
               \end{pspicture*}
          \end{extern}
     \end{center}
     \columnbreak
     \begin{center}
          \begin{extern}%width="300" alt="configuration de Thalès n°2"
               \psset{xunit=1.0cm,yunit=1.0cm,algebraic=true,dimen=middle,dotstyle=o,dotsize=5pt 0,linewidth=1.6pt,arrowsize=3pt 2,arrowinset=0.25}
               \newrgbcolor{tttttt}{0.2 0.2 0.2}
               \psset{xunit=1.0cm,yunit=1.0cm,algebraic=true,dimen=middle,dotstyle=o,dotsize=5pt 0,linewidth=1.6pt,arrowsize=3pt 2,arrowinset=0.25}
               \begin{pspicture*}(4.,1.)(11.,8.)
                    \psline[linewidth=0.4pt,linecolor=tttttt](5.,3.)(7.,2.)
                    \psline[linewidth=0.4pt,linecolor=tttttt](6.,7.)(7.,2.)
                    \psline[linewidth=0.4pt,linecolor=tttttt](5.,3.)(8.673155137107534,5.663422431446233)
                    \psline[linewidth=0.4pt,linecolor=tttttt](8.673155137107534,5.663422431446233)(6.,7.)
                    \begin{scriptsize}
                         \psdots[dotsize=2pt 0,dotstyle=*,linecolor=tttttt](5.,3.)
                         \rput[bl](4.681166178198794,3.021412501112033){\tttttt{$A$}}
                         \psdots[dotsize=2pt 0,dotstyle=*,linecolor=tttttt](7.,2.)
                         \rput[bl](7.14203735737573,1.9017161145865278){\tttttt{$B$}}
                         \psdots[dotsize=2pt 0,dotstyle=*,linecolor=tttttt](6.,7.)
                         \rput[bl](5.727036429348992,7.06954559085809){\tttttt{$D$}}
                         \psdots[dotsize=2pt 0,dotstyle=*](8.673155137107534,5.663422431446233)
                         \rput[bl](8.864647182799585,5.543805459768391){\tttttt{$E$}}
                         \psdots[dotsize=2pt 0,dotstyle=*,linecolor=tttttt](6.572023624013923,4.139881879930384)
                         \rput[bl](6.723689256915651,3.9196304815116143){\tttttt{$C$}}
                    \end{scriptsize}
               \end{pspicture*}
          \end{extern}
     \end{center}
\end{multicols}
Le \mcLien{https://www.maths-cours.fr/cours/theoreme-thales/\#d10}{ \textbf{théorème de Thalès}} conclut que les longueurs des côtés du triangle $ ABC $ et les longueurs des côtés du triangle $ CDE $ sont proportionnelles, c'est à dire que~:
\begin{center}
     $ \frac{ CA }{ CE } = \frac{ CB }{ CD } = \frac{ AB }{ DE } $
\end{center}
\medskip
\par \textbf{Attention} à l'ordre des côtés - voir \mcLien{https://www.maths-cours.fr/cours/theoreme-thales/\#r10}{cours}~!
\medskip
En général, l'un des trois rapports sera inutile pour résoudre l'exercice.
\par
À partir des deux autres rapports, on peut calculer le côté recherché en effectuant, par exemple, un produit en croix.
\par
\begin{h2}Exemple 1\end{h2}
\begin{center}
     \begin{extern}%width="500" alt="Thalès exercice 1"
          \newrgbcolor{grey}{0.2 0.2 0.2}
          \psset{xunit=1.0cm,yunit=1.0cm,algebraic=true,dimen=middle,dotstyle=o,dotsize=5pt 0,linewidth=1.6pt,arrowsize=3pt 2,arrowinset=0.25}
          \begin{pspicture*}(1.,4.)(10.,9.)
               \psline[linewidth=0.4pt,linecolor=grey](2.,8.)(6.,8.)
               \psline[linewidth=0.4pt,linecolor=grey](6.,8.)(3.,5.)
               \psline[linewidth=0.4pt,linecolor=grey](3.,5.)(8.,5.)
               \psline[linewidth=0.4pt,linecolor=grey](8.,5.)(2.,8.)
               \rput[tl](7.338907051709888,7.241806573400475){$\grey{(MN)//(RS)}$}
               \begin{scriptsize}
                    \psdots[dotsize=2pt 0,dotstyle=*,linecolor=grey](2.,8.)
                    \rput[bl](1.7281207631864715,8.053894062528864){\grey{$R$}}
                    \psdots[dotsize=2pt 0,dotstyle=*,linecolor=grey](6.,8.)
                    \rput[bl](6.046949682641996,8.053894062528864){\grey{$S$}}
                    \psdots[dotsize=2pt 0,dotstyle=*,linecolor=grey](3.,5.)
                    \rput[bl](2.675556167169592,5.076239935724772){\grey{$M$}}
                    \psdots[dotsize=2pt 0,dotstyle=*,linecolor=grey](8.,5.)
                    \rput[bl](8.0525596936712,5.051631223933002){\grey{$N$}}
                    \psdots[dotsize=2pt 0,dotstyle=*,linecolor=grey](4.666666666666667,6.666666666666667)
                    \rput[bl](4.71807924588645,6.318979881209124){\grey{$I$}}
               \end{scriptsize}
          \end{pspicture*}
     \end{extern}
\end{center}
Calculer $ MI $ sachant que $ RS = 4 \texttt{cm} $ , $ MN = 5 \texttt{cm} $ et $ IS = 2 \texttt{cm} . $
\bloc{orange}{Solution~:}{ % id=s010
     \par
     On se place dans les triangles $ RSI $ et $ MNI $~;
     \par
     On précise les conditions d'alignement et de parallélisme~:
     \begin{itemize}
          \item
          les points $ R, I, N $ ainsi que les points $ M, I, S $ sont alignés
          \item
          les droites $ \left( MN \right) $ et $ \left( RS \right) $ sont parallèles.
     \end{itemize}
     Par conséquent, d'après le théorème de Thalès, on peut écrire~:
     \begin{center}
          $ \frac{ RI }{ IN} = \frac{ SI }{ IM } = \frac{ RS }{ MN } $
     \end{center}
     Ici, le rapport $ \frac{ RI }{ IN } $ ne nous intéresse pas car on ne connait ni la longueur $ RI $ ni la longueur $ IN $ .
     \par
     On utilise les deux autres rapports en remplaçant les longueurs connues par leurs valeurs~:
     \par
     $ \frac{ 2 }{ MI } $ = $\frac{4 }{ 5 } $
     \par
     En effectuant le produit en croix, on obtient~:
     \par
     $ 4MI = 2 \times 5 $
     \\
     $ 4MI=10 $
     \\
     $ MI= \frac{ 10}{ 4 } = 2,5 \texttt{cm} $
} % fin solution
\par
\begin{h2}Exemple 2 \end{h2}
\begin{center}
     \begin{extern}%width="500" alt="Thalès exercice 2"
          \newrgbcolor{grey}{0.2 0.2 0.2}
          \psset{xunit=1.0cm,yunit=1.0cm,algebraic=true,dimen=middle,dotstyle=o,dotsize=5pt 0,linewidth=1.6pt,arrowsize=3pt 2,arrowinset=0.25}
          \newrgbcolor{grey}{0.2 0.2 0.2}
          \psset{xunit=1.0cm,yunit=1.0cm,algebraic=true,dimen=middle,dotstyle=o,dotsize=5pt 0,linewidth=1.6pt,arrowsize=3pt 2,arrowinset=0.25}
          \begin{pspicture*}(0.,3.)(10.,8.)
               \rput[tl](7.77199779622993,6.47589412117782){$\grey{(BC)//(DE)}$}
               \psline[linewidth=.4pt,linecolor=grey](0.9365246496171764,3.624349766411451)(6.,7.)
               \psline[linewidth=.4pt,linecolor=grey](6.,7.)(9.37565023358855,3.624349766411451)
               \psline[linewidth=.4pt,linecolor=grey](9.37565023358855,3.624349766411451)(0.9365246496171764,3.624349766411451)
               \psline[linewidth=.4pt,linecolor=grey](3.,5.)(8.,5.)
               \begin{scriptsize}
                    \psdots[dotsize=2pt 0,dotstyle=*,linecolor=grey](6.,7.)
                    \rput[bl](5.858511062628407,7.186290393532876){\grey{$A$}}
                    \psdots[dotsize=2pt 0,dotstyle=*,linecolor=grey](3.,5.)
                    \rput[bl](2.684643845816301,5.0207275632892365){\grey{$B$}}
                    \psdots[dotsize=2pt 0,dotstyle=*,linecolor=grey](8.,5.)
                    \rput[bl](8.046989901657694,5.043643572074884){\grey{$C$}}
                    \psdots[dotsize=2pt 0,dotstyle=*](0.9365246496171764,3.624349766411451)
                    \rput[bl](0.6565770682865436,3.4166069482939485){$D$}
                    \psdots[dotsize=2pt 0,dotstyle=*,linecolor=grey](9.37565023358855,3.624349766411451)
                    \rput[bl](9.5136144639391,3.4395229570795953){\grey{$E$}}
               \end{scriptsize}
          \end{pspicture*}
     \end{extern}
\end{center}
On donne~: $ AC =3 \texttt{cm} $, $ AD = 5 \texttt{cm} $ et $ AE = 4 \texttt{cm} $.
\\
Calculer $ BD. $
\bloc{orange}{Solution~:}{ % id=s040
     \par
     Les triangles $ ABC $ et $ ADE$ sont en situation de Thalès car~:
     \begin{itemize}
          \item
          les points $ A, B, D $ ainsi que les points $ A, C, E $ sont alignés
          \item
          les droites $ \left( BC \right) $ et $ \left( DE \right) $ sont parallèles.
          \par
          En utilisant le théorème de Thalès, on obtient~:
          \begin{center}
               $ \frac{ AB }{ AD } = \frac{ AC }{ AE } = \frac{ BC }{ DE } $
          \end{center}
          \bloc{cyan}{Remarque}{ % id=r020
               Il ne faut pas chercher à tout prix à faire figurer la longueur recherchée (ici $ BD $) dans l'égalité des rapports.
               \\
               En effet, $ \left[ BD \right] $ n'est \textbf{pas} un côté de l'un des triangles $ ABC $ ou $ ADE $. Toutefois, il sera facile de calculer $ BD $ une fois la longueur $ AB$ connue.
          } % fin remarque
          \par
          L'égalité des deux premiers quotients donne~:
          \par
          $ \frac{ AB }{ 5 } = \frac{ 3 }{ 4 } $
          \par
          Avec le produit en croix~:
          \par
          $ 4AB = 3 \times 5 $
          \\
          $ 4 AB = 15 $
          \par
          $ AB = \frac{ 15 }{ 4 } = 3,75 $
          \par
          Pour calculer la distance $ BD $, il suffit maintenant de faire~:
          \\
          $ BD = AD - AB = 5 - 3,75 = 1,25.$
          \par
          Donc le segment $ \left[ BD \right] $ mesure $ 1,25 $cm.
     \end{itemize}
} % fin solution
\par
\begin{h2}Exemple 3 \end{h2}
\begin{center}
     \begin{extern}%width="500" alt="Thalès exercice 2"
          \newrgbcolor{grey}{0.2 0.2 0.2}
          \psset{xunit=1.0cm,yunit=1.0cm,algebraic=true,dimen=middle,dotstyle=o,dotsize=5pt 0,linewidth=1.6pt,arrowsize=3pt 2,arrowinset=0.25}
          \newrgbcolor{grey}{0.2 0.2 0.2}
          \psset{xunit=1.0cm,yunit=1.0cm,algebraic=true,dimen=middle,dotstyle=o,dotsize=5pt 0,linewidth=1.6pt,arrowsize=3pt 2,arrowinset=0.25}
          \begin{pspicture*}(0.,3.)(10.,8.)
               \rput[tl](7.77199779622993,6.47589412117782){$\grey{(BC)//(DE)}$}
               \psline[linewidth=.4pt,linecolor=grey](0.9365246496171764,3.624349766411451)(6.,7.)
               \psline[linewidth=.4pt,linecolor=grey](6.,7.)(9.37565023358855,3.624349766411451)
               \psline[linewidth=.4pt,linecolor=grey](9.37565023358855,3.624349766411451)(0.9365246496171764,3.624349766411451)
               \psline[linewidth=.4pt,linecolor=grey](3.,5.)(8.,5.)
               \begin{scriptsize}
                    \psdots[dotsize=2pt 0,dotstyle=*,linecolor=grey](6.,7.)
                    \rput[bl](5.858511062628407,7.186290393532876){\grey{$A$}}
                    \psdots[dotsize=2pt 0,dotstyle=*,linecolor=grey](3.,5.)
                    \rput[bl](2.684643845816301,5.0207275632892365){\grey{$B$}}
                    \psdots[dotsize=2pt 0,dotstyle=*,linecolor=grey](8.,5.)
                    \rput[bl](8.046989901657694,5.043643572074884){\grey{$C$}}
                    \psdots[dotsize=2pt 0,dotstyle=*](0.9365246496171764,3.624349766411451)
                    \rput[bl](0.6565770682865436,3.4166069482939485){$D$}
                    \psdots[dotsize=2pt 0,dotstyle=*,linecolor=grey](9.37565023358855,3.624349766411451)
                    \rput[bl](9.5136144639391,3.4395229570795953){\grey{$E$}}
               \end{scriptsize}
          \end{pspicture*}
     \end{extern}
\end{center}
La figure est similaire à celle de ll'exercice précédent mais, cette fois, on connait~: $ BD = 8\texttt{cm} $, $ AE = 9\texttt{cm} $ et $ AC = 4 \texttt{cm} $.
\\
On cherche à calculer la longueur $ AB. $
\bloc{orange}{Solution~:}{ %
     \par
     Le raisonnement précédent donne, là aussi~:
     \begin{center}
          $ \frac{ AB }{ AD } = \frac{ AC }{ AE } = \frac{ BC }{ DE } $
     \end{center}
     Cette fois, le calcul est légèrement plus compliqué car on ne connait ni $ AD $ ni $ AB $ (que l'on recherche).
     \par
     On pose alors~: $ AB = x $. Par conséquent~:
     \\
     $ AD = AB + BD = x + 8 $
     \par
     À partir de l'égalité~:
     \par
     $ \frac{ AB }{ AD } = \frac{ AC }{ AE } $
     \par
     on obtient l'équation~:
     \par
     $ \frac{ x }{ x+8 } = \frac{ 4 }{ 9 } $
     \par
     et en effectuant le produit en croix~:
     \par
     $ 9x = 4 \left( x+8 \right) $
     \\
     $ 9x = 4x + 32 $
     \\
     $ 9x - 4x = 32 $
     \\
     $ 5x = 32 $
     \\
     $ x = \frac{ 32 }{ 5 } = 6,4$
     \par
     La longueur $ AB $ est donc $ 6,4 $ cm.
} % fin solution

\end{document}
µ
\documentclass[a4paper]{article}

%================================================================================================================================
%
% Packages
%
%================================================================================================================================

\usepackage[T1]{fontenc} 	% pour caractères accentués
\usepackage[utf8]{inputenc}  % encodage utf8
\usepackage[french]{babel}	% langue : français
\usepackage{fourier}			% caractères plus lisibles
\usepackage[dvipsnames]{xcolor} % couleurs
\usepackage{fancyhdr}		% réglage header footer
\usepackage{needspace}		% empêcher sauts de page mal placés
\usepackage{graphicx}		% pour inclure des graphiques
\usepackage{enumitem,cprotect}		% personnalise les listes d'items (nécessaire pour ol, al ...)
\usepackage{hyperref}		% Liens hypertexte
\usepackage{pstricks,pst-all,pst-node,pstricks-add,pst-math,pst-plot,pst-tree,pst-eucl} % pstricks
\usepackage[a4paper,includeheadfoot,top=2cm,left=3cm, bottom=2cm,right=3cm]{geometry} % marges etc.
\usepackage{comment}			% commentaires multilignes
\usepackage{amsmath,environ} % maths (matrices, etc.)
\usepackage{amssymb,makeidx}
\usepackage{bm}				% bold maths
\usepackage{tabularx}		% tableaux
\usepackage{colortbl}		% tableaux en couleur
\usepackage{fontawesome}		% Fontawesome
\usepackage{environ}			% environment with command
\usepackage{fp}				% calculs pour ps-tricks
\usepackage{multido}			% pour ps tricks
\usepackage[np]{numprint}	% formattage nombre
\usepackage{tikz,tkz-tab} 			% package principal TikZ
\usepackage{pgfplots}   % axes
\usepackage{mathrsfs}    % cursives
\usepackage{calc}			% calcul taille boites
\usepackage[scaled=0.875]{helvet} % font sans serif
\usepackage{svg} % svg
\usepackage{scrextend} % local margin
\usepackage{scratch} %scratch
\usepackage{multicol} % colonnes
%\usepackage{infix-RPN,pst-func} % formule en notation polanaise inversée
\usepackage{listings}

%================================================================================================================================
%
% Réglages de base
%
%================================================================================================================================

\lstset{
language=Python,   % R code
literate=
{á}{{\'a}}1
{à}{{\`a}}1
{ã}{{\~a}}1
{é}{{\'e}}1
{è}{{\`e}}1
{ê}{{\^e}}1
{í}{{\'i}}1
{ó}{{\'o}}1
{õ}{{\~o}}1
{ú}{{\'u}}1
{ü}{{\"u}}1
{ç}{{\c{c}}}1
{~}{{ }}1
}


\definecolor{codegreen}{rgb}{0,0.6,0}
\definecolor{codegray}{rgb}{0.5,0.5,0.5}
\definecolor{codepurple}{rgb}{0.58,0,0.82}
\definecolor{backcolour}{rgb}{0.95,0.95,0.92}

\lstdefinestyle{mystyle}{
    backgroundcolor=\color{backcolour},   
    commentstyle=\color{codegreen},
    keywordstyle=\color{magenta},
    numberstyle=\tiny\color{codegray},
    stringstyle=\color{codepurple},
    basicstyle=\ttfamily\footnotesize,
    breakatwhitespace=false,         
    breaklines=true,                 
    captionpos=b,                    
    keepspaces=true,                 
    numbers=left,                    
xleftmargin=2em,
framexleftmargin=2em,            
    showspaces=false,                
    showstringspaces=false,
    showtabs=false,                  
    tabsize=2,
    upquote=true
}

\lstset{style=mystyle}


\lstset{style=mystyle}
\newcommand{\imgdir}{C:/laragon/www/newmc/assets/imgsvg/}
\newcommand{\imgsvgdir}{C:/laragon/www/newmc/assets/imgsvg/}

\definecolor{mcgris}{RGB}{220, 220, 220}% ancien~; pour compatibilité
\definecolor{mcbleu}{RGB}{52, 152, 219}
\definecolor{mcvert}{RGB}{125, 194, 70}
\definecolor{mcmauve}{RGB}{154, 0, 215}
\definecolor{mcorange}{RGB}{255, 96, 0}
\definecolor{mcturquoise}{RGB}{0, 153, 153}
\definecolor{mcrouge}{RGB}{255, 0, 0}
\definecolor{mclightvert}{RGB}{205, 234, 190}

\definecolor{gris}{RGB}{220, 220, 220}
\definecolor{bleu}{RGB}{52, 152, 219}
\definecolor{vert}{RGB}{125, 194, 70}
\definecolor{mauve}{RGB}{154, 0, 215}
\definecolor{orange}{RGB}{255, 96, 0}
\definecolor{turquoise}{RGB}{0, 153, 153}
\definecolor{rouge}{RGB}{255, 0, 0}
\definecolor{lightvert}{RGB}{205, 234, 190}
\setitemize[0]{label=\color{lightvert}  $\bullet$}

\pagestyle{fancy}
\renewcommand{\headrulewidth}{0.2pt}
\fancyhead[L]{maths-cours.fr}
\fancyhead[R]{\thepage}
\renewcommand{\footrulewidth}{0.2pt}
\fancyfoot[C]{}

\newcolumntype{C}{>{\centering\arraybackslash}X}
\newcolumntype{s}{>{\hsize=.35\hsize\arraybackslash}X}

\setlength{\parindent}{0pt}		 
\setlength{\parskip}{3mm}
\setlength{\headheight}{1cm}

\def\ebook{ebook}
\def\book{book}
\def\web{web}
\def\type{web}

\newcommand{\vect}[1]{\overrightarrow{\,\mathstrut#1\,}}

\def\Oij{$\left(\text{O}~;~\vect{\imath},~\vect{\jmath}\right)$}
\def\Oijk{$\left(\text{O}~;~\vect{\imath},~\vect{\jmath},~\vect{k}\right)$}
\def\Ouv{$\left(\text{O}~;~\vect{u},~\vect{v}\right)$}

\hypersetup{breaklinks=true, colorlinks = true, linkcolor = OliveGreen, urlcolor = OliveGreen, citecolor = OliveGreen, pdfauthor={Didier BONNEL - https://www.maths-cours.fr} } % supprime les bordures autour des liens

\renewcommand{\arg}[0]{\text{arg}}

\everymath{\displaystyle}

%================================================================================================================================
%
% Macros - Commandes
%
%================================================================================================================================

\newcommand\meta[2]{    			% Utilisé pour créer le post HTML.
	\def\titre{titre}
	\def\url{url}
	\def\arg{#1}
	\ifx\titre\arg
		\newcommand\maintitle{#2}
		\fancyhead[L]{#2}
		{\Large\sffamily \MakeUppercase{#2}}
		\vspace{1mm}\textcolor{mcvert}{\hrule}
	\fi 
	\ifx\url\arg
		\fancyfoot[L]{\href{https://www.maths-cours.fr#2}{\black \footnotesize{https://www.maths-cours.fr#2}}}
	\fi 
}


\newcommand\TitreC[1]{    		% Titre centré
     \needspace{3\baselineskip}
     \begin{center}\textbf{#1}\end{center}
}

\newcommand\newpar{    		% paragraphe
     \par
}

\newcommand\nosp {    		% commande vide (pas d'espace)
}
\newcommand{\id}[1]{} %ignore

\newcommand\boite[2]{				% Boite simple sans titre
	\vspace{5mm}
	\setlength{\fboxrule}{0.2mm}
	\setlength{\fboxsep}{5mm}	
	\fcolorbox{#1}{#1!3}{\makebox[\linewidth-2\fboxrule-2\fboxsep]{
  		\begin{minipage}[t]{\linewidth-2\fboxrule-4\fboxsep}\setlength{\parskip}{3mm}
  			 #2
  		\end{minipage}
	}}
	\vspace{5mm}
}

\newcommand\CBox[4]{				% Boites
	\vspace{5mm}
	\setlength{\fboxrule}{0.2mm}
	\setlength{\fboxsep}{5mm}
	
	\fcolorbox{#1}{#1!3}{\makebox[\linewidth-2\fboxrule-2\fboxsep]{
		\begin{minipage}[t]{1cm}\setlength{\parskip}{3mm}
	  		\textcolor{#1}{\LARGE{#2}}    
 	 	\end{minipage}  
  		\begin{minipage}[t]{\linewidth-2\fboxrule-4\fboxsep}\setlength{\parskip}{3mm}
			\raisebox{1.2mm}{\normalsize\sffamily{\textcolor{#1}{#3}}}						
  			 #4
  		\end{minipage}
	}}
	\vspace{5mm}
}

\newcommand\cadre[3]{				% Boites convertible html
	\par
	\vspace{2mm}
	\setlength{\fboxrule}{0.1mm}
	\setlength{\fboxsep}{5mm}
	\fcolorbox{#1}{white}{\makebox[\linewidth-2\fboxrule-2\fboxsep]{
  		\begin{minipage}[t]{\linewidth-2\fboxrule-4\fboxsep}\setlength{\parskip}{3mm}
			\raisebox{-2.5mm}{\sffamily \small{\textcolor{#1}{\MakeUppercase{#2}}}}		
			\par		
  			 #3
 	 		\end{minipage}
	}}
		\vspace{2mm}
	\par
}

\newcommand\bloc[3]{				% Boites convertible html sans bordure
     \needspace{2\baselineskip}
     {\sffamily \small{\textcolor{#1}{\MakeUppercase{#2}}}}    
		\par		
  			 #3
		\par
}

\newcommand\CHelp[1]{
     \CBox{Plum}{\faInfoCircle}{À RETENIR}{#1}
}

\newcommand\CUp[1]{
     \CBox{NavyBlue}{\faThumbsOUp}{EN PRATIQUE}{#1}
}

\newcommand\CInfo[1]{
     \CBox{Sepia}{\faArrowCircleRight}{REMARQUE}{#1}
}

\newcommand\CRedac[1]{
     \CBox{PineGreen}{\faEdit}{BIEN R\'EDIGER}{#1}
}

\newcommand\CError[1]{
     \CBox{Red}{\faExclamationTriangle}{ATTENTION}{#1}
}

\newcommand\TitreExo[2]{
\needspace{4\baselineskip}
 {\sffamily\large EXERCICE #1\ (\emph{#2 points})}
\vspace{5mm}
}

\newcommand\img[2]{
          \includegraphics[width=#2\paperwidth]{\imgdir#1}
}

\newcommand\imgsvg[2]{
       \begin{center}   \includegraphics[width=#2\paperwidth]{\imgsvgdir#1} \end{center}
}


\newcommand\Lien[2]{
     \href{#1}{#2 \tiny \faExternalLink}
}
\newcommand\mcLien[2]{
     \href{https~://www.maths-cours.fr/#1}{#2 \tiny \faExternalLink}
}

\newcommand{\euro}{\eurologo{}}

%================================================================================================================================
%
% Macros - Environement
%
%================================================================================================================================

\newenvironment{tex}{ %
}
{%
}

\newenvironment{indente}{ %
	\setlength\parindent{10mm}
}

{
	\setlength\parindent{0mm}
}

\newenvironment{corrige}{%
     \needspace{3\baselineskip}
     \medskip
     \textbf{\textsc{Corrigé}}
     \medskip
}
{
}

\newenvironment{extern}{%
     \begin{center}
     }
     {
     \end{center}
}

\NewEnviron{code}{%
	\par
     \boite{gray}{\texttt{%
     \BODY
     }}
     \par
}

\newenvironment{vbloc}{% boite sans cadre empeche saut de page
     \begin{minipage}[t]{\linewidth}
     }
     {
     \end{minipage}
}
\NewEnviron{h2}{%
    \needspace{3\baselineskip}
    \vspace{0.6cm}
	\noindent \MakeUppercase{\sffamily \large \BODY}
	\vspace{1mm}\textcolor{mcgris}{\hrule}\vspace{0.4cm}
	\par
}{}

\NewEnviron{h3}{%
    \needspace{3\baselineskip}
	\vspace{5mm}
	\textsc{\BODY}
	\par
}

\NewEnviron{margeneg}{ %
\begin{addmargin}[-1cm]{0cm}
\BODY
\end{addmargin}
}

\NewEnviron{html}{%
}

\begin{document}
\meta{url}{/methode/determiner-si-deux-droites-sont-paralleles-thales/}
\meta{pid}{11546}
\meta{titre}{Déterminer si deux droites sont parallèles (Thalès)}
\meta{type}{methode}
%
\begin{h2} Problème~: \end{h2}
$ A $ et $ B $ sont deux points d'une droite $\mathscr{D} $ et $ C $ et $ D $ deux points d'une droite $ \mathscr{D'} $.
\\
Les droites $ \left( AC \right) $ et $ \left( BD \right) $ sont sécantes en un point $ M $.
\\
Deux cas de figure sont possibles~:
\begin{multicols}{2}
     \begin{center}
          \begin{extern}%width="500" alt=""
               \newrgbcolor{tttttt}{0.2 0.2 0.2}
               \psset{xunit=1.0cm,yunit=1.0cm,algebraic=true,dimen=middle,dotstyle=o,dotsize=5pt 0,linewidth=1.6pt,arrowsize=3pt 2,arrowinset=0.25}
               \begin{pspicture*}(1.,1.)(7.,8.)
                    \rput[tl](5.3728597107438,7.1){$\tttttt{\mathscr{D}}$}
                    \rput[tl](6.487572210743799,4.1){$\tttttt{\mathscr{D'}}$}
                    \psline[linewidth=0.4pt,linecolor=tttttt](5.,7.)(2.,2.)
                    \psline[linewidth=0.4pt,linecolor=tttttt](6.,4.)(3.,6.)
                    \psplot[linewidth=0.4pt,linecolor=tttttt]{1.}{7.}{(--9.--1.*x)/2.}
                    \psplot[linewidth=0.4pt,linecolor=tttttt]{1.}{7.}{(--4.--2.*x)/4.}
                    \begin{scriptsize}
                         \psdots[dotsize=2pt 0,dotstyle=*,linecolor=tttttt](3.,6.)
                         \rput[bl](2.7690909607438003,6.065112062122526){\tttttt{$A$}}
                         \psdots[dotsize=2pt 0,dotstyle=*,linecolor=tttttt](5.,7.)
                         \rput[bl](5.0151534607438,7.115625918895607){\tttttt{$B$}}
                         \psdots[dotsize=2pt 0,dotstyle=*,linecolor=tttttt](6.,4.)
                         \rput[bl](6.079953460743799,3.780661294219158){\tttttt{$D$}}
                         \psdots[dotsize=2pt 0,dotstyle=*,linecolor=tttttt](2.,2.)
                         \rput[bl](1.8873034607438006,1.7630076962899062){\tttttt{$C$}}
                         \psdots[dotsize=2pt 0,dotstyle=*,linecolor=tttttt](4.,5.333333333333333)
                         \rput[bl](3.6,5.2){\tttttt{$M$}}
                    \end{scriptsize}
               \end{pspicture*}
          \end{extern}
     \end{center}
     \columnbreak
     \begin{center}
          \begin{extern}%width="500" alt=""
               \newrgbcolor{tttttt}{0.2 0.2 0.2}
               \psset{xunit=1.0cm,yunit=1.0cm,algebraic=true,dimen=middle,dotstyle=o,dotsize=5pt 0,linewidth=1.6pt,arrowsize=3pt 2,arrowinset=0.25}
               \begin{pspicture*}(1.,1.)(7.,8.)
                    \psplot[linewidth=0.4pt,linecolor=tttttt]{1.}{7.}{(--5.841086716091681--1.0004893874029346*x)/2.0131375}
                    \psplot[linewidth=0.4pt,linecolor=tttttt]{1.}{7.}{(--4.--2.*x)/4.}
                    \rput[tl](5.3728597107438,5.5){$\tttttt{\mathscr{D}}$}
                    \rput[tl](6.487572210743799,4.1){$\tttttt{\mathscr{D'}}$}
                    \psline[linewidth=0.4pt,linecolor=tttttt](3.997317828962791,6.802444984387303)(2.,2.)
                    \psline[linewidth=0.4pt,linecolor=tttttt](3.997317828962791,6.802444984387303)(6.,4.)
                    \begin{scriptsize}
                         \psdots[dotsize=2pt 0,dotstyle=*,linecolor=tttttt](2.9936972107438002,4.38929233822261)
                         \rput[bl](2.7607722107438004,4.455991630716139){\tttttt{$A$}}
                         \psdots[dotsize=2pt 0,dotstyle=*,linecolor=tttttt](5.0068347107438,5.389781725625545)
                         \rput[bl](5.0234722107438,5.50650548748922){\tttttt{$B$}}
                         \psdots[dotsize=2pt 0,dotstyle=*,linecolor=tttttt](6.,4.)
                         \rput[bl](6.079953460743799,3.780661294219158){\tttttt{$D$}}
                         \psdots[dotsize=2pt 0,dotstyle=*,linecolor=tttttt](2.,2.)
                         \rput[bl](1.8873034607438006,1.7630076962899062){\tttttt{$C$}}
                         \psdots[dotsize=2pt 0,dotstyle=*,linecolor=darkgray](3.997317828962791,6.802444984387303)
                         \rput[bl](4.0335409607438,6.832153925798109){\darkgray{$M$}}
                    \end{scriptsize}
               \end{pspicture*}
          \end{extern}
     \end{center}
\end{multicols}
On connait les longueurs $ AM, BM, CM, DM. $
\\
On se pose la question suivante~:
\begin{center}
     \textbf{ Les droites $\mathscr{D} $ et $ \mathscr{D'} $ sont-elles parallèles~?}
\end{center}
\begin{h2} Méthode \end{h2}
On calcule séparément chacun des deux rapports $ \frac{ MA }{ MD } $ et $ \frac{ MB }{ MC } . $
\begin{itemize}
     \item
     Si ces deux rapports sont égaux, les droites $\mathscr{D} $ et $\mathscr{D'} $ sont parallèles d'après la \mcLien{https://www.maths-cours.fr/cours/theoreme-thales/\#t50}{réciproque du théorème de Thalès}
     \item
     Sinon, les droites $\mathscr{D} $ et $\mathscr{D'} $ ne sont pas parallèles (en effet, si elles étaient parallèles, on aurait $ \frac{ MA }{ MD } = \frac{ MB }{ MC } $ d'après le théorème de Thalès.)
\end{itemize}
\bloc{cyan}{Attention à la rédaction~:}{ % id=r010
     \begin{itemize}
          \item
          N'écrivez pas $ \frac{ MA }{ MD } = \frac{ MB }{ MC } $ tant que vous n'avez pas prouvé que ces rapports étaient égaux~!
          \item
          La \textbf{ réciproque } du théorème de Thalès sert à prouver que les droites $\mathscr{D} $ et $\mathscr{D'} $ \textbf{sont} parallèles. Ne citez pas la réciproque du théorème de Thalès si $\mathscr{D} $ et $\mathscr{D'} $ \textbf{ne sont pas} parallèles~!
          \item
          Il faut faire attention à l'alignement et à l' \textbf{ordre} des points.
          //
          Dans le cas de la première figure ci-dessus, on signalera que $ A, M, D $ et $ B, M, C $ sont dans le même ordre~; dans le cas de la seconde figure, que $ M, A, D $ et $ M, B, C $ sont dans le même ordre.
     \end{itemize}
} % fin Attention à la rédaction
\bloc{cyan}{Remarque:}{ % id=r030
     N'oubliez pas qu'il existe d'autres méthodes pour démontrer que deux droites sont parallèles.
     \par
     En particulier, un théorème couramment utilisé au collège pour montrer que deux droites sont parallèles est le suivant~:
     \cadre{rouge}{Théorème}{ % id=t030
          Deux droites prependiculaires à une même troisième sont parallèles entre elles.
     } % fin théorème
     \begin{center}
          \begin{extern}%width="400" alt="perpendiculaires et parallèles "
               \newrgbcolor{grey}{0.2 0.2 0.2}
               \psset{xunit=1.0cm,yunit=1.0cm,algebraic=true,dimen=middle,dotstyle=o,dotsize=5pt 0,linewidth=1.6pt,arrowsize=3pt 2,arrowinset=0.25}
               \fontsize{9pt}{9pt}\selectfont
               \begin{pspicture*}(0.,0.)(5.,5.)
                    \pspolygon[linewidth=0.4pt,linecolor=grey](2.2980044729391027,3.6490022364695514)(2.2490022364695514,3.747006709408654)(2.150997763530449,3.6980044729391026)(2.2,3.6)
                    \pspolygon[linewidth=0.4pt,linecolor=grey](2.6980044729391026,2.849002236469551)(2.6490022364695514,2.9470067094086536)(2.550997763530449,2.8980044729391023)(2.6,2.8)
                    \psplot[linewidth=0.4pt,linecolor=red]{0.}{5.}{(--5.--1.*x)/2.}
                    \psplot[linewidth=0.4pt,linecolor=red]{0.}{5.}{(--3.--1.*x)/2.}
                    \psplot[linewidth=0.4pt,linecolor=blue]{0.}{3.5}{(--8.-2.*x)/1.}
                    \rput[tl](2.884090909090908,4.434685073339088){$\red{\mathscr{D}}$}
                    \rput[tl](3.1940082644628086,3.596031061259709){$\red{\mathscr{D'}}$}
                    \rput[tl](1.9369834710743792,4.6106988783434025){$\blue{(d)}$}
               \end{pspicture*}
          \end{extern}
     \end{center}
} % fin remarque
\begin{h2}Exemple 1 \end{h2}
Dans la figure ci-dessous, tracée à main levée, on a~: $ IM=5 $cm, $ IJ = 3 $cm, $ IL=4 $cm et $ IK=2,5 $cm.
\begin{center}
     \img{reciproque-thales}{0.1}%width="400" alt="thales et reciproque"
\end{center}
Les droites $(JK)$ et $ (LM) $ sont-elles parallèles~?
\bloc{orange}{Solution}{ % id=s040
     Les droites $ \left( JM \right) $ et $ \left( KL \right) $ sont sécantes en $I$.
     \\
     Les points $ J, I, M $ et $ K, I, L $ sont dans le même ordre.
     \par
     On calcule $ \frac{ IJ }{ IM } $ et $ \frac{ IK }{ IL } $~:
     \par
     $ \frac{ IJ }{ IM } = \frac{ 3 }{ 5 } $
     \par
     $ \frac{ IK }{ IL } = \frac{ 2,5 }{ 4 } = \frac{ 25 }{ 40 } = \frac{ 5 }{ 8 } $
     \par
     Les rapports $ \frac{ IJ }{ IM } $ et $ \frac{ IK }{ IL } $ ne sont pas égaux~; par conséquent, les droites $ \left( JK \right) $ et $ \left( LM \right) $ ne sont pas parallèles.
} % fin Solution
\begin{h2}Exemple 2 \end{h2}
\begin{center}
     \begin{extern}%width="500" alt="réciproque du théorème de Thalès"
          \newrgbcolor{tttttt}{0.2 0.2 0.2}
          \psset{xunit=1.0cm,yunit=1.0cm,algebraic=true,dimen=middle,dotstyle=o,dotsize=5pt 0,linewidth=1.6pt,arrowsize=3pt 2,arrowinset=0.25}
          \begin{pspicture*}(0.,0.)(8.,6.)
               \psplot[linewidth=0.4pt,linecolor=tttttt]{0.}{8.}{(--5.389132322680942--0.606992187729279*x)/5.996124510410222}
               \psline[linewidth=0.4pt,linecolor=tttttt](2.,5.)(1.,1.)
               \psplot[linewidth=0.4pt,linecolor=tttttt]{0.}{8.}{(--19.372549727742015--0.606992187729279*x)/5.996124510410222}
               \psline[linewidth=0.4pt,linecolor=tttttt](2.,5.)(6.996124510410222,1.606992187729279)
               \begin{scriptsize}
                    \psdots[dotsize=2pt 0,dotstyle=*,linecolor=tttttt](2.,5.)
                    \rput[bl](1.9064419707276834,5.085640095555049){\tttttt{$A$}}
                    \psdots[dotsize=2pt 0,dotstyle=*,linecolor=tttttt](1.,1.)
                    \rput[bl](0.9049680478251896,0.6757659957857199){\tttttt{$B$}}
                    \psdots[dotsize=2pt 0,dotstyle=*,linecolor=tttttt](6.996124510410222,1.606992187729279)
                    \rput[bl](7.050999793856933,1.3144978370986039){\tttttt{$C$}}
                    \psdots[dotsize=2pt 0,dotstyle=*](1.5981569416534365,3.3926277666137454)
                    \rput[bl](1.35,3.449438115194311){$D$}
                    \psdots[dotsize=2pt 0,dotstyle=*,linecolor=darkgray](4.007657953143472,3.63654336372335)
                    \rput[bl](4.07,3.710056053482655){\darkgray{$E$}}
               \end{scriptsize}
          \end{pspicture*}
     \end{extern}
\end{center}
Sur la figure ci-dessus, on sait que~:
\begin{itemize}
     \item
     $ AD = 2 $ cm
     \item
     $ AB = 5 $ cm
     \item
     $ AE = 3 $ cm
     \item
     $ EC = 4,5 $ cm
\end{itemize}
Les droites $ \left( BD \right) $ et $ \left( EC \right) $ sont-elles parallèles~?
\bloc{orange}{Solution}{ % id=e060
     Les droites $ \left(BD \right) $ et $ \left( EC \right) $ sont sécantes au point $A$.
     \\
     Les points $ A, D, B $ et les points $ A, E, C $ sont situés dans le même ordre.
     \par
     On va comparer les rapports $ \frac{AD}{ AB } $ et $ \frac{ AE }{ AC } $.
     \par
     Pour cela, on calcule d'abord $ AC $~:
     \\
     $ AC = AE + EC = 3 + 4,5 = 7,5 .$
     \par
     Alors~:
     \par
     $ \frac{ AD }{ AB } = \frac{ 2 }{ 5 } $
     \par
     $ \frac{ AE }{ AC } = \frac{ 3 }{ 7,5 } = \frac{ 2 }{ 5 } $
     \par
     Les rapports $ \frac{ AD }{ AB } $ et $ \frac{ AE }{ AC }$ sont égaux, donc, d'après \textbf{la réciproque du théorème de Thalès} les droites $ \left( DE \right) $ et $ \left( BC \right) $ sont parallèles.
} % fin Solution

\end{document}
µ
\documentclass[a4paper]{article}

%================================================================================================================================
%
% Packages
%
%================================================================================================================================

\usepackage[T1]{fontenc} 	% pour caractères accentués
\usepackage[utf8]{inputenc}  % encodage utf8
\usepackage[french]{babel}	% langue : français
\usepackage{fourier}			% caractères plus lisibles
\usepackage[dvipsnames]{xcolor} % couleurs
\usepackage{fancyhdr}		% réglage header footer
\usepackage{needspace}		% empêcher sauts de page mal placés
\usepackage{graphicx}		% pour inclure des graphiques
\usepackage{enumitem,cprotect}		% personnalise les listes d'items (nécessaire pour ol, al ...)
\usepackage{hyperref}		% Liens hypertexte
\usepackage{pstricks,pst-all,pst-node,pstricks-add,pst-math,pst-plot,pst-tree,pst-eucl} % pstricks
\usepackage[a4paper,includeheadfoot,top=2cm,left=3cm, bottom=2cm,right=3cm]{geometry} % marges etc.
\usepackage{comment}			% commentaires multilignes
\usepackage{amsmath,environ} % maths (matrices, etc.)
\usepackage{amssymb,makeidx}
\usepackage{bm}				% bold maths
\usepackage{tabularx}		% tableaux
\usepackage{colortbl}		% tableaux en couleur
\usepackage{fontawesome}		% Fontawesome
\usepackage{environ}			% environment with command
\usepackage{fp}				% calculs pour ps-tricks
\usepackage{multido}			% pour ps tricks
\usepackage[np]{numprint}	% formattage nombre
\usepackage{tikz,tkz-tab} 			% package principal TikZ
\usepackage{pgfplots}   % axes
\usepackage{mathrsfs}    % cursives
\usepackage{calc}			% calcul taille boites
\usepackage[scaled=0.875]{helvet} % font sans serif
\usepackage{svg} % svg
\usepackage{scrextend} % local margin
\usepackage{scratch} %scratch
\usepackage{multicol} % colonnes
%\usepackage{infix-RPN,pst-func} % formule en notation polanaise inversée
\usepackage{listings}

%================================================================================================================================
%
% Réglages de base
%
%================================================================================================================================

\lstset{
language=Python,   % R code
literate=
{á}{{\'a}}1
{à}{{\`a}}1
{ã}{{\~a}}1
{é}{{\'e}}1
{è}{{\`e}}1
{ê}{{\^e}}1
{í}{{\'i}}1
{ó}{{\'o}}1
{õ}{{\~o}}1
{ú}{{\'u}}1
{ü}{{\"u}}1
{ç}{{\c{c}}}1
{~}{{ }}1
}


\definecolor{codegreen}{rgb}{0,0.6,0}
\definecolor{codegray}{rgb}{0.5,0.5,0.5}
\definecolor{codepurple}{rgb}{0.58,0,0.82}
\definecolor{backcolour}{rgb}{0.95,0.95,0.92}

\lstdefinestyle{mystyle}{
    backgroundcolor=\color{backcolour},   
    commentstyle=\color{codegreen},
    keywordstyle=\color{magenta},
    numberstyle=\tiny\color{codegray},
    stringstyle=\color{codepurple},
    basicstyle=\ttfamily\footnotesize,
    breakatwhitespace=false,         
    breaklines=true,                 
    captionpos=b,                    
    keepspaces=true,                 
    numbers=left,                    
xleftmargin=2em,
framexleftmargin=2em,            
    showspaces=false,                
    showstringspaces=false,
    showtabs=false,                  
    tabsize=2,
    upquote=true
}

\lstset{style=mystyle}


\lstset{style=mystyle}
\newcommand{\imgdir}{C:/laragon/www/newmc/assets/imgsvg/}
\newcommand{\imgsvgdir}{C:/laragon/www/newmc/assets/imgsvg/}

\definecolor{mcgris}{RGB}{220, 220, 220}% ancien~; pour compatibilité
\definecolor{mcbleu}{RGB}{52, 152, 219}
\definecolor{mcvert}{RGB}{125, 194, 70}
\definecolor{mcmauve}{RGB}{154, 0, 215}
\definecolor{mcorange}{RGB}{255, 96, 0}
\definecolor{mcturquoise}{RGB}{0, 153, 153}
\definecolor{mcrouge}{RGB}{255, 0, 0}
\definecolor{mclightvert}{RGB}{205, 234, 190}

\definecolor{gris}{RGB}{220, 220, 220}
\definecolor{bleu}{RGB}{52, 152, 219}
\definecolor{vert}{RGB}{125, 194, 70}
\definecolor{mauve}{RGB}{154, 0, 215}
\definecolor{orange}{RGB}{255, 96, 0}
\definecolor{turquoise}{RGB}{0, 153, 153}
\definecolor{rouge}{RGB}{255, 0, 0}
\definecolor{lightvert}{RGB}{205, 234, 190}
\setitemize[0]{label=\color{lightvert}  $\bullet$}

\pagestyle{fancy}
\renewcommand{\headrulewidth}{0.2pt}
\fancyhead[L]{maths-cours.fr}
\fancyhead[R]{\thepage}
\renewcommand{\footrulewidth}{0.2pt}
\fancyfoot[C]{}

\newcolumntype{C}{>{\centering\arraybackslash}X}
\newcolumntype{s}{>{\hsize=.35\hsize\arraybackslash}X}

\setlength{\parindent}{0pt}		 
\setlength{\parskip}{3mm}
\setlength{\headheight}{1cm}

\def\ebook{ebook}
\def\book{book}
\def\web{web}
\def\type{web}

\newcommand{\vect}[1]{\overrightarrow{\,\mathstrut#1\,}}

\def\Oij{$\left(\text{O}~;~\vect{\imath},~\vect{\jmath}\right)$}
\def\Oijk{$\left(\text{O}~;~\vect{\imath},~\vect{\jmath},~\vect{k}\right)$}
\def\Ouv{$\left(\text{O}~;~\vect{u},~\vect{v}\right)$}

\hypersetup{breaklinks=true, colorlinks = true, linkcolor = OliveGreen, urlcolor = OliveGreen, citecolor = OliveGreen, pdfauthor={Didier BONNEL - https://www.maths-cours.fr} } % supprime les bordures autour des liens

\renewcommand{\arg}[0]{\text{arg}}

\everymath{\displaystyle}

%================================================================================================================================
%
% Macros - Commandes
%
%================================================================================================================================

\newcommand\meta[2]{    			% Utilisé pour créer le post HTML.
	\def\titre{titre}
	\def\url{url}
	\def\arg{#1}
	\ifx\titre\arg
		\newcommand\maintitle{#2}
		\fancyhead[L]{#2}
		{\Large\sffamily \MakeUppercase{#2}}
		\vspace{1mm}\textcolor{mcvert}{\hrule}
	\fi 
	\ifx\url\arg
		\fancyfoot[L]{\href{https://www.maths-cours.fr#2}{\black \footnotesize{https://www.maths-cours.fr#2}}}
	\fi 
}


\newcommand\TitreC[1]{    		% Titre centré
     \needspace{3\baselineskip}
     \begin{center}\textbf{#1}\end{center}
}

\newcommand\newpar{    		% paragraphe
     \par
}

\newcommand\nosp {    		% commande vide (pas d'espace)
}
\newcommand{\id}[1]{} %ignore

\newcommand\boite[2]{				% Boite simple sans titre
	\vspace{5mm}
	\setlength{\fboxrule}{0.2mm}
	\setlength{\fboxsep}{5mm}	
	\fcolorbox{#1}{#1!3}{\makebox[\linewidth-2\fboxrule-2\fboxsep]{
  		\begin{minipage}[t]{\linewidth-2\fboxrule-4\fboxsep}\setlength{\parskip}{3mm}
  			 #2
  		\end{minipage}
	}}
	\vspace{5mm}
}

\newcommand\CBox[4]{				% Boites
	\vspace{5mm}
	\setlength{\fboxrule}{0.2mm}
	\setlength{\fboxsep}{5mm}
	
	\fcolorbox{#1}{#1!3}{\makebox[\linewidth-2\fboxrule-2\fboxsep]{
		\begin{minipage}[t]{1cm}\setlength{\parskip}{3mm}
	  		\textcolor{#1}{\LARGE{#2}}    
 	 	\end{minipage}  
  		\begin{minipage}[t]{\linewidth-2\fboxrule-4\fboxsep}\setlength{\parskip}{3mm}
			\raisebox{1.2mm}{\normalsize\sffamily{\textcolor{#1}{#3}}}						
  			 #4
  		\end{minipage}
	}}
	\vspace{5mm}
}

\newcommand\cadre[3]{				% Boites convertible html
	\par
	\vspace{2mm}
	\setlength{\fboxrule}{0.1mm}
	\setlength{\fboxsep}{5mm}
	\fcolorbox{#1}{white}{\makebox[\linewidth-2\fboxrule-2\fboxsep]{
  		\begin{minipage}[t]{\linewidth-2\fboxrule-4\fboxsep}\setlength{\parskip}{3mm}
			\raisebox{-2.5mm}{\sffamily \small{\textcolor{#1}{\MakeUppercase{#2}}}}		
			\par		
  			 #3
 	 		\end{minipage}
	}}
		\vspace{2mm}
	\par
}

\newcommand\bloc[3]{				% Boites convertible html sans bordure
     \needspace{2\baselineskip}
     {\sffamily \small{\textcolor{#1}{\MakeUppercase{#2}}}}    
		\par		
  			 #3
		\par
}

\newcommand\CHelp[1]{
     \CBox{Plum}{\faInfoCircle}{À RETENIR}{#1}
}

\newcommand\CUp[1]{
     \CBox{NavyBlue}{\faThumbsOUp}{EN PRATIQUE}{#1}
}

\newcommand\CInfo[1]{
     \CBox{Sepia}{\faArrowCircleRight}{REMARQUE}{#1}
}

\newcommand\CRedac[1]{
     \CBox{PineGreen}{\faEdit}{BIEN R\'EDIGER}{#1}
}

\newcommand\CError[1]{
     \CBox{Red}{\faExclamationTriangle}{ATTENTION}{#1}
}

\newcommand\TitreExo[2]{
\needspace{4\baselineskip}
 {\sffamily\large EXERCICE #1\ (\emph{#2 points})}
\vspace{5mm}
}

\newcommand\img[2]{
          \includegraphics[width=#2\paperwidth]{\imgdir#1}
}

\newcommand\imgsvg[2]{
       \begin{center}   \includegraphics[width=#2\paperwidth]{\imgsvgdir#1} \end{center}
}


\newcommand\Lien[2]{
     \href{#1}{#2 \tiny \faExternalLink}
}
\newcommand\mcLien[2]{
     \href{https~://www.maths-cours.fr/#1}{#2 \tiny \faExternalLink}
}

\newcommand{\euro}{\eurologo{}}

%================================================================================================================================
%
% Macros - Environement
%
%================================================================================================================================

\newenvironment{tex}{ %
}
{%
}

\newenvironment{indente}{ %
	\setlength\parindent{10mm}
}

{
	\setlength\parindent{0mm}
}

\newenvironment{corrige}{%
     \needspace{3\baselineskip}
     \medskip
     \textbf{\textsc{Corrigé}}
     \medskip
}
{
}

\newenvironment{extern}{%
     \begin{center}
     }
     {
     \end{center}
}

\NewEnviron{code}{%
	\par
     \boite{gray}{\texttt{%
     \BODY
     }}
     \par
}

\newenvironment{vbloc}{% boite sans cadre empeche saut de page
     \begin{minipage}[t]{\linewidth}
     }
     {
     \end{minipage}
}
\NewEnviron{h2}{%
    \needspace{3\baselineskip}
    \vspace{0.6cm}
	\noindent \MakeUppercase{\sffamily \large \BODY}
	\vspace{1mm}\textcolor{mcgris}{\hrule}\vspace{0.4cm}
	\par
}{}

\NewEnviron{h3}{%
    \needspace{3\baselineskip}
	\vspace{5mm}
	\textsc{\BODY}
	\par
}

\NewEnviron{margeneg}{ %
\begin{addmargin}[-1cm]{0cm}
\BODY
\end{addmargin}
}

\NewEnviron{html}{%
}

\begin{document}
\meta{url}{/exercices/theoreme-de-thales-et-projections-orthogonales/}
\meta{pid}{11564}
\meta{titre}{Théorème de Thalès et projections orthogonales}
\meta{type}{exercices}
%
\begin{center}
     \begin{extern}%width="400" alt=""
          \newrgbcolor{grey}{0.2 0.2 0.2}
          \psset{xunit=1.0cm,yunit=1.0cm,algebraic=true,dimen=middle,dotstyle=o,dotsize=5pt 0,linewidth=1.6pt,arrowsize=3pt 2,arrowinset=0.25}
          \begin{pspicture*}(2.,0.)(9.,5.)
               \pspolygon[linewidth=0.4pt,linecolor=grey](5.738795805833042,3.3693979029165204)(5.8040177825740145,3.238953949434575)(5.93446173605596,3.3041759261755477)(5.869239759314987,3.434619879657493)
               \pspolygon[linewidth=0.4pt,linecolor=grey](7.142504831241246,1.101090032454455)(7.001129447156849,1.1369024887369967)(6.965316990874307,0.995527104652599)(7.106692374958705,0.9597146483700572)
               \pspolygon[linewidth=0.4pt,linecolor=grey](5.003467207433093,1.6429396046799178)(4.8620918233486945,1.6787520609624595)(4.826279367066153,1.5373766768780617)(4.967654751150551,1.50156422059552)
               \pspolygon[linewidth=0.4pt,linecolor=grey](5.267143958152722,3.133571979076361)(5.332365934893695,3.0031280255944157)(5.462809888375641,3.0683500023353885)(5.397587911634668,3.198793955817334)
               \psplot[linewidth=0.4pt,linecolor=grey]{2.}{9.}{(--1.--1.*x)/2.}
               \psplot[linewidth=0.4pt,linecolor=grey]{2.}{9.}{(--5.6380408873423296-0.5174736842105285*x)/2.042809917355372}
               \psline[linewidth=0.4pt,linecolor=blue](6.43162538544843,1.130719008189808)(5.397587911634668,3.198793955817334)
               \psline[linewidth=0.4pt,linecolor=blue](5.397587911634668,3.198793955817334)(4.967654751150551,1.50156422059552)
               \psline[linewidth=0.4pt,linecolor=blue](8.004005905632527,4.502002952816263)(7.106692374958705,0.9597146483700572)
               \psline[linewidth=0.4pt,linecolor=blue](7.106692374958705,0.9597146483700572)(5.869239759314987,3.434619879657493)
               \psplot[linewidth=0.4pt,linecolor=red]{2.}{9.}{(--8.248965338892633-1.9330556590619732*x)/-0.9015850081644361}
               \psplot[linewidth=0.4pt,linecolor=red]{2.}{9.}{(--19.904914857534692-3.3712839446264553*x)/-1.5723805201840966}
               \fontsize{9pt}{9pt}\selectfont
               \rput[tl](8.217190082644626,4.6){$\tiny\grey{\mathscr{D}}$}
               \rput[tl](8.065940082644625,0.5432786885245919){$\tiny\grey{\mathscr{D'}}$}
               \begin{scriptsize}
                    \psdots[dotsize=2pt 0,dotstyle=*,linecolor=grey](3.,2.)
                    \rput[bl](2.9440650826446273,2.1109387402933586){\grey{$O$}}
                    \psdots[dotsize=2pt 0,dotstyle=*](8.004005905632527,4.502002952816263)
                    \rput[bl](7.907815082644626,4.598381363244179){$I$}
                    \psdots[dotsize=2pt 0,dotstyle=*,linecolor=grey](7.106692374958705,0.9597146483700572)
                    \rput[bl](7.0759400826446255,0.6259637618636773){\grey{$K$}}
                    \psdots[dotsize=2pt 0,dotstyle=*](6.43162538544843,1.130719008189808)
                    \rput[bl](6.429690082644626,0.7844434857635911){$J$}
                    \psdots[dotsize=2pt 0,dotstyle=*,linecolor=grey](5.397587911634668,3.198793955817334)
                    \rput[bl](5.302190082644627,3.3787765314926688){\grey{$M$}}
                    \psdots[dotsize=2pt 0,dotstyle=*,linecolor=grey](4.967654751150551,1.50156422059552)
                    \rput[bl](4.9928150826446265,1.1703071613459898){\grey{$N$}}
                    \psdots[dotsize=2pt 0,dotstyle=*,linecolor=grey](5.869239759314987,3.434619879657493)
                    \rput[bl](5.797190082644626,3.5854892148403823){\grey{$L$}}
               \end{scriptsize}
          \end{pspicture*}
     \end{extern}
\end{center}
$\mathscr{D} $ et $\mathscr{D'} $ sont deux droites sécantes en $ O $.
\\
$ I $ est un point quelconque de $\mathscr{D} $ et $ J $ un point quelconque de $\mathscr{D'}. $
\par
$K$ est la projection orthogonale de $ I $ sur $\mathscr{D'} $
\\
(cela signifie que $ K \in \mathscr{D'} $ et que les droites $ \left( IJ \right) $ et $\mathscr{D'} $ sont perpendiculaires.)
\\
$L$ est la projection orthogonale de $ K $ sur $\mathscr{D} $
\\
$M$ est la projection orthogonale de $ J $ sur $\mathscr{D} $
\\
$N$ est la projection orthogonale de $M$ sur $\mathscr{D'}. $
\par
Démontrer que les droites $ \left( IJ \right) $ et $ \left( LN \right) $ sont parallèles.
\begin{corrige}
     Les droites $ \left( JM \right) $ et $ \left( KL \right) $ sont toutes les deux perpendiculaires à la droite $\mathscr{D} $, donc, elles sont parallèles entre elles.
     \par
     Par ailleurs, les points $ O, N, K $ sont alignés ainsi que les points $ O, L, M $~;
     \\
     par conséquent, d'après le théorème de Thalès~:
     \begin{center}
          $ \frac{ OJ }{ OK } = \frac{ OM }{ OL } = \frac{ JM }{ KL } $
     \end{center}
     L'égalité $ \frac{ OJ }{ OK } = \frac{ OM }{ OL } $ est équivalente à~:
     \begin{center}
          $ OM \times OK = OJ \times OL \quad \textbf{(1)} $
     \end{center}
     \medskip
     De même, les droites $ \left( IK \right) $ et $ \left( NM \right) $ sont parallèles puisqu'elles sont toutes les deux perpendiculaires à la droite $\mathscr{D'} $.
     \par
     Les points $ O, M, K $ et $ O, M, I $ sont alignés~;
     \par
     donc, d'après le théorème de Thalès~:
     \begin{center}
          $ \frac{ OK }{ ON } = \frac{ OI }{ OM } = \frac{ IK }{ NM } $
     \end{center}
     L'égalité $ \frac{ OK }{ ON } = \frac{ OI }{ OM } $ est équivalente à~:
     \begin{center}
          $ OM \times OK = OI \times ON \quad \textbf{(2)} $
     \end{center}
     \medskip
     Des égalités \textbf{(1) } et \textbf{(2)} on en déduit que~:
     \begin{center}
          $ OJ \times OL = OI \times ON $
     \end{center}
     En divisant chaque membre de l'égalité par $ OL \times ON $ on en déduit que~:
     \par
     $ \frac{ OJ \times OL }{ OL \times ON } = \frac{ OI \times ON }{ OL \times ON } $
     \par
     $ \frac{ OJ }{ ON } = \frac{ OI }{ OL } $
     \par
     Donc, d'après \mcLien{https://www.maths-cours.fr/cours/theoreme-thales/\#t50}{la réciproque du théorème de Thalès}, les droites $ \left( NL \right) $ et $ \left( IJ \right) $ sont parallèles.
\end{corrige}

\end{document}
µ
\documentclass[a4paper]{article}

%================================================================================================================================
%
% Packages
%
%================================================================================================================================

\usepackage[T1]{fontenc} 	% pour caractères accentués
\usepackage[utf8]{inputenc}  % encodage utf8
\usepackage[french]{babel}	% langue : français
\usepackage{fourier}			% caractères plus lisibles
\usepackage[dvipsnames]{xcolor} % couleurs
\usepackage{fancyhdr}		% réglage header footer
\usepackage{needspace}		% empêcher sauts de page mal placés
\usepackage{graphicx}		% pour inclure des graphiques
\usepackage{enumitem,cprotect}		% personnalise les listes d'items (nécessaire pour ol, al ...)
\usepackage{hyperref}		% Liens hypertexte
\usepackage{pstricks,pst-all,pst-node,pstricks-add,pst-math,pst-plot,pst-tree,pst-eucl} % pstricks
\usepackage[a4paper,includeheadfoot,top=2cm,left=3cm, bottom=2cm,right=3cm]{geometry} % marges etc.
\usepackage{comment}			% commentaires multilignes
\usepackage{amsmath,environ} % maths (matrices, etc.)
\usepackage{amssymb,makeidx}
\usepackage{bm}				% bold maths
\usepackage{tabularx}		% tableaux
\usepackage{colortbl}		% tableaux en couleur
\usepackage{fontawesome}		% Fontawesome
\usepackage{environ}			% environment with command
\usepackage{fp}				% calculs pour ps-tricks
\usepackage{multido}			% pour ps tricks
\usepackage[np]{numprint}	% formattage nombre
\usepackage{tikz,tkz-tab} 			% package principal TikZ
\usepackage{pgfplots}   % axes
\usepackage{mathrsfs}    % cursives
\usepackage{calc}			% calcul taille boites
\usepackage[scaled=0.875]{helvet} % font sans serif
\usepackage{svg} % svg
\usepackage{scrextend} % local margin
\usepackage{scratch} %scratch
\usepackage{multicol} % colonnes
%\usepackage{infix-RPN,pst-func} % formule en notation polanaise inversée
\usepackage{listings}

%================================================================================================================================
%
% Réglages de base
%
%================================================================================================================================

\lstset{
language=Python,   % R code
literate=
{á}{{\'a}}1
{à}{{\`a}}1
{ã}{{\~a}}1
{é}{{\'e}}1
{è}{{\`e}}1
{ê}{{\^e}}1
{í}{{\'i}}1
{ó}{{\'o}}1
{õ}{{\~o}}1
{ú}{{\'u}}1
{ü}{{\"u}}1
{ç}{{\c{c}}}1
{~}{{ }}1
}


\definecolor{codegreen}{rgb}{0,0.6,0}
\definecolor{codegray}{rgb}{0.5,0.5,0.5}
\definecolor{codepurple}{rgb}{0.58,0,0.82}
\definecolor{backcolour}{rgb}{0.95,0.95,0.92}

\lstdefinestyle{mystyle}{
    backgroundcolor=\color{backcolour},   
    commentstyle=\color{codegreen},
    keywordstyle=\color{magenta},
    numberstyle=\tiny\color{codegray},
    stringstyle=\color{codepurple},
    basicstyle=\ttfamily\footnotesize,
    breakatwhitespace=false,         
    breaklines=true,                 
    captionpos=b,                    
    keepspaces=true,                 
    numbers=left,                    
xleftmargin=2em,
framexleftmargin=2em,            
    showspaces=false,                
    showstringspaces=false,
    showtabs=false,                  
    tabsize=2,
    upquote=true
}

\lstset{style=mystyle}


\lstset{style=mystyle}
\newcommand{\imgdir}{C:/laragon/www/newmc/assets/imgsvg/}
\newcommand{\imgsvgdir}{C:/laragon/www/newmc/assets/imgsvg/}

\definecolor{mcgris}{RGB}{220, 220, 220}% ancien~; pour compatibilité
\definecolor{mcbleu}{RGB}{52, 152, 219}
\definecolor{mcvert}{RGB}{125, 194, 70}
\definecolor{mcmauve}{RGB}{154, 0, 215}
\definecolor{mcorange}{RGB}{255, 96, 0}
\definecolor{mcturquoise}{RGB}{0, 153, 153}
\definecolor{mcrouge}{RGB}{255, 0, 0}
\definecolor{mclightvert}{RGB}{205, 234, 190}

\definecolor{gris}{RGB}{220, 220, 220}
\definecolor{bleu}{RGB}{52, 152, 219}
\definecolor{vert}{RGB}{125, 194, 70}
\definecolor{mauve}{RGB}{154, 0, 215}
\definecolor{orange}{RGB}{255, 96, 0}
\definecolor{turquoise}{RGB}{0, 153, 153}
\definecolor{rouge}{RGB}{255, 0, 0}
\definecolor{lightvert}{RGB}{205, 234, 190}
\setitemize[0]{label=\color{lightvert}  $\bullet$}

\pagestyle{fancy}
\renewcommand{\headrulewidth}{0.2pt}
\fancyhead[L]{maths-cours.fr}
\fancyhead[R]{\thepage}
\renewcommand{\footrulewidth}{0.2pt}
\fancyfoot[C]{}

\newcolumntype{C}{>{\centering\arraybackslash}X}
\newcolumntype{s}{>{\hsize=.35\hsize\arraybackslash}X}

\setlength{\parindent}{0pt}		 
\setlength{\parskip}{3mm}
\setlength{\headheight}{1cm}

\def\ebook{ebook}
\def\book{book}
\def\web{web}
\def\type{web}

\newcommand{\vect}[1]{\overrightarrow{\,\mathstrut#1\,}}

\def\Oij{$\left(\text{O}~;~\vect{\imath},~\vect{\jmath}\right)$}
\def\Oijk{$\left(\text{O}~;~\vect{\imath},~\vect{\jmath},~\vect{k}\right)$}
\def\Ouv{$\left(\text{O}~;~\vect{u},~\vect{v}\right)$}

\hypersetup{breaklinks=true, colorlinks = true, linkcolor = OliveGreen, urlcolor = OliveGreen, citecolor = OliveGreen, pdfauthor={Didier BONNEL - https://www.maths-cours.fr} } % supprime les bordures autour des liens

\renewcommand{\arg}[0]{\text{arg}}

\everymath{\displaystyle}

%================================================================================================================================
%
% Macros - Commandes
%
%================================================================================================================================

\newcommand\meta[2]{    			% Utilisé pour créer le post HTML.
	\def\titre{titre}
	\def\url{url}
	\def\arg{#1}
	\ifx\titre\arg
		\newcommand\maintitle{#2}
		\fancyhead[L]{#2}
		{\Large\sffamily \MakeUppercase{#2}}
		\vspace{1mm}\textcolor{mcvert}{\hrule}
	\fi 
	\ifx\url\arg
		\fancyfoot[L]{\href{https://www.maths-cours.fr#2}{\black \footnotesize{https://www.maths-cours.fr#2}}}
	\fi 
}


\newcommand\TitreC[1]{    		% Titre centré
     \needspace{3\baselineskip}
     \begin{center}\textbf{#1}\end{center}
}

\newcommand\newpar{    		% paragraphe
     \par
}

\newcommand\nosp {    		% commande vide (pas d'espace)
}
\newcommand{\id}[1]{} %ignore

\newcommand\boite[2]{				% Boite simple sans titre
	\vspace{5mm}
	\setlength{\fboxrule}{0.2mm}
	\setlength{\fboxsep}{5mm}	
	\fcolorbox{#1}{#1!3}{\makebox[\linewidth-2\fboxrule-2\fboxsep]{
  		\begin{minipage}[t]{\linewidth-2\fboxrule-4\fboxsep}\setlength{\parskip}{3mm}
  			 #2
  		\end{minipage}
	}}
	\vspace{5mm}
}

\newcommand\CBox[4]{				% Boites
	\vspace{5mm}
	\setlength{\fboxrule}{0.2mm}
	\setlength{\fboxsep}{5mm}
	
	\fcolorbox{#1}{#1!3}{\makebox[\linewidth-2\fboxrule-2\fboxsep]{
		\begin{minipage}[t]{1cm}\setlength{\parskip}{3mm}
	  		\textcolor{#1}{\LARGE{#2}}    
 	 	\end{minipage}  
  		\begin{minipage}[t]{\linewidth-2\fboxrule-4\fboxsep}\setlength{\parskip}{3mm}
			\raisebox{1.2mm}{\normalsize\sffamily{\textcolor{#1}{#3}}}						
  			 #4
  		\end{minipage}
	}}
	\vspace{5mm}
}

\newcommand\cadre[3]{				% Boites convertible html
	\par
	\vspace{2mm}
	\setlength{\fboxrule}{0.1mm}
	\setlength{\fboxsep}{5mm}
	\fcolorbox{#1}{white}{\makebox[\linewidth-2\fboxrule-2\fboxsep]{
  		\begin{minipage}[t]{\linewidth-2\fboxrule-4\fboxsep}\setlength{\parskip}{3mm}
			\raisebox{-2.5mm}{\sffamily \small{\textcolor{#1}{\MakeUppercase{#2}}}}		
			\par		
  			 #3
 	 		\end{minipage}
	}}
		\vspace{2mm}
	\par
}

\newcommand\bloc[3]{				% Boites convertible html sans bordure
     \needspace{2\baselineskip}
     {\sffamily \small{\textcolor{#1}{\MakeUppercase{#2}}}}    
		\par		
  			 #3
		\par
}

\newcommand\CHelp[1]{
     \CBox{Plum}{\faInfoCircle}{À RETENIR}{#1}
}

\newcommand\CUp[1]{
     \CBox{NavyBlue}{\faThumbsOUp}{EN PRATIQUE}{#1}
}

\newcommand\CInfo[1]{
     \CBox{Sepia}{\faArrowCircleRight}{REMARQUE}{#1}
}

\newcommand\CRedac[1]{
     \CBox{PineGreen}{\faEdit}{BIEN R\'EDIGER}{#1}
}

\newcommand\CError[1]{
     \CBox{Red}{\faExclamationTriangle}{ATTENTION}{#1}
}

\newcommand\TitreExo[2]{
\needspace{4\baselineskip}
 {\sffamily\large EXERCICE #1\ (\emph{#2 points})}
\vspace{5mm}
}

\newcommand\img[2]{
          \includegraphics[width=#2\paperwidth]{\imgdir#1}
}

\newcommand\imgsvg[2]{
       \begin{center}   \includegraphics[width=#2\paperwidth]{\imgsvgdir#1} \end{center}
}


\newcommand\Lien[2]{
     \href{#1}{#2 \tiny \faExternalLink}
}
\newcommand\mcLien[2]{
     \href{https~://www.maths-cours.fr/#1}{#2 \tiny \faExternalLink}
}

\newcommand{\euro}{\eurologo{}}

%================================================================================================================================
%
% Macros - Environement
%
%================================================================================================================================

\newenvironment{tex}{ %
}
{%
}

\newenvironment{indente}{ %
	\setlength\parindent{10mm}
}

{
	\setlength\parindent{0mm}
}

\newenvironment{corrige}{%
     \needspace{3\baselineskip}
     \medskip
     \textbf{\textsc{Corrigé}}
     \medskip
}
{
}

\newenvironment{extern}{%
     \begin{center}
     }
     {
     \end{center}
}

\NewEnviron{code}{%
	\par
     \boite{gray}{\texttt{%
     \BODY
     }}
     \par
}

\newenvironment{vbloc}{% boite sans cadre empeche saut de page
     \begin{minipage}[t]{\linewidth}
     }
     {
     \end{minipage}
}
\NewEnviron{h2}{%
    \needspace{3\baselineskip}
    \vspace{0.6cm}
	\noindent \MakeUppercase{\sffamily \large \BODY}
	\vspace{1mm}\textcolor{mcgris}{\hrule}\vspace{0.4cm}
	\par
}{}

\NewEnviron{h3}{%
    \needspace{3\baselineskip}
	\vspace{5mm}
	\textsc{\BODY}
	\par
}

\NewEnviron{margeneg}{ %
\begin{addmargin}[-1cm]{0cm}
\BODY
\end{addmargin}
}

\NewEnviron{html}{%
}

\begin{document}
\meta{url}{/losange-dans-un-triangle}
\meta{pid}{11572}
\meta{titre}{Losange dans un triangle}
\meta{type}{exercices}
%
$ ABC $ est un triangle quelconque tel que $ AB = 7 $cm, $ AC= 5 $cm et $ BC = 4 $cm.
\\
$ M$ est un point du segment $ \left[ BC \right] . $
\\
La parallèle à la droite $ \left( AB \right) $ passant par $ M $ coupe le côté $ [AC] $ en $ N.$
\\
La parallèle à la droite $ \left( AC \right) $ passant par $ M $ coupe le côté $ [AB] $ en $ P.$
\begin{center}
     \begin{extern}%width="500" alt="Thalès et losange"
          \newrgbcolor{grey}{0.2 0.2 0.2}
          \psset{xunit=1.0cm,yunit=1.0cm,algebraic=true,dimen=middle,dotstyle=o,dotsize=5pt 0,linewidth=1.6pt,arrowsize=3pt 2,arrowinset=0.25}
          \begin{pspicture*}(0.,0.)(9.,5.)
               \psline[linewidth=0.4pt,linecolor=grey](1.,1.)(8.,1.)
               \psline[linewidth=0.4pt,linecolor=grey](8.,1.)(5.142857142857143,3.79941684889506)
               \psline[linewidth=0.4pt,linecolor=grey](5.142857142857143,3.79941684889506)(1.,1.)
               \psline[linewidth=0.4pt,linecolor=red](3.6425664347163558,2.785638448678848)(6.1775403898507895,2.7856384486788484)
               \psline[linewidth=0.4pt,linecolor=red](6.1775403898507895,2.7856384486788484)(3.5349739551344332,1.)
               \psline[linewidth=0.4pt,linecolor=red](3.5349739551344332,1.)(1.,1.)
               \psline[linewidth=0.4pt,linecolor=red](1.,1.)(3.6425664347163558,2.785638448678848)
               \begin{scriptsize}
                    \psdots[dotsize=2pt 0,dotstyle=*,linecolor=grey](1.,1.)
                    \rput[bl](0.855296988302761,0.7186720182506175){\grey{$A$}}
                    \psdots[dotsize=2pt 0,dotstyle=*,linecolor=grey](8.,1.)
                    \rput[bl](8.06583327519023,0.7186720182506175){\grey{$B$}}
                    \psdots[dotsize=2pt 0,dotstyle=*,linecolor=grey](5.142857142857143,3.79941684889506)
                    \rput[bl](5.1767631804440715,3.835885016525611){\grey{$C$}}
                    \psdots[dotsize=2pt 0,dotstyle=*](6.1775403898507895,2.7856384486788484)
                    \rput[bl](6.212620245227008,2.822036562804175){$M$}
                    \psdots[dotsize=2pt 0,dotstyle=*,linecolor=grey](3.6425664347163558,2.785638448678848)
                    \rput[bl](3.5096807168090343,2.854479713323261){\grey{$N$}}
                    \psdots[dotsize=2pt 0,dotstyle=*,linecolor=grey](3.5349739551344332,1.)
                    \rput[bl](3.533958616764884,0.7186720182506175){\grey{$P$}}
               \end{scriptsize}
          \end{pspicture*}
     \end{extern}
\end{center}
\begin{enumerate}
     \item
     À quelle distance du point $ C $ faut-il placer le point $ M $ pour que le quadrilatère $ APMN $ soit un parallélogramme~?
     \item
     Sans utiliser le résultat de la question précédente, construire le point $ M$ à l'aide d'une règle non graduée et d'un compas.
\end{enumerate}
\begin{corrige}
     \begin{enumerate}
          \item
          On sait déjà que $ APMN $ est un parallélogramme car les droites $ \left( MN \right) $ et $ \left( AP \right) $ sont parallèles ainsi que les droites $ \left( MP \right) et \left( AN \right).$
          \par
          Pour que ce soit un losange il faut et il suffit que $ MN = MP. $
          \par
          Calculons $ MP $ puis $ MN $ en utilisant le théorème de Thalès~:
          \begin{itemize}
               \item
               \textbf{Calcul de $ MP$ }
               \par
               Posons $ x = MC. $
               \par
               On a alors~:
               \\
               $ BM = BC - MC = 4 - x $
               \par
               Les droites $ \left( MP \right) $ et $ \left( AC \right) $ sont parallèles~; les points $ B, M, C $ et les points $ B, P, A $ sont alignés.
               \par
               Donc, d'après le théorème de Thalès~:
               \begin{center}
                    $ \frac{ BM }{ BC } = \frac{ MP }{ AC } = \frac{ BP }{ AB } $
               \end{center}
               La première égalité donne~:
               \par
               $ \frac{ 4-x }{ 4 } = \frac{ MP }{ 5 } $
               \par
               donc, avec un produit en croix~:
               \par
               $ 4 MP = 5 \left( 4-x \right) $
               \par
               $ MP = \frac{ 5 \left( 4-x \right) }{ 4 }. $
               \\
               \item
               \textbf{Calcul de MN}
               \par
               De même, les droites $ \left( MN \right) $ et $ \left( AB \right) $ sont parallèles et les points $ C, M, B$ et $ C, N, A $ sont alignés.
               \par
               Par conséquent~:
               \begin{center}
                    $ \frac{ CM }{ BC } = \frac{ MN }{ AB } = \frac{ CN }{ CA } $
               \end{center}
               L'égalité des deux premiers quotients équivaut à~:
               \par
               $ \frac{ x }{ 4 } = \frac{ MN }{ 7 } $
               \par
               soit~:
               $ 4 MN = 7x $
               \\
               $ MN = \frac{ 7x }{ 4 }. $
               \item
               \textbf{Conclusion }
               \par
               $ APMN $ est donc un losange si et seulement si~:
               \par
               $ MP = MN $
               \par
               $ \frac{ 5(4-x) }{ 4 } = \frac{ 7x }{ 4 } $
               \par
               $ 5 \left( 4 - x \right) = 7x $
               \\
               $ 20 - 5x = 7x $
               \\
               $ 20 = 7x + 5x $
               \\
               $ 12x = 20$
               \par
               $ x = \frac{ 5 }{ 3 } $
               \par
               Il faut placer le point $ M $ à $ \frac{ 5 }{ 3 } $cm ( $ \approx 1,67 $ cm) de $C$ pour que le quadrilatère $ APMN $ soit un parallélogramme.
          \end{itemize}
          \item
          Les diagonales d'un losange sont des axes de symétrie de ce losange.
          \par
          Donc, si $ APMN $ est un losange, la droite $ \left( AM \right) $ est un axe de symétrie donc une bissectrice de l'angle $ \widehat{ PAN } $ qui est aussi l'angle $ \widehat{ BAC }.$
          \par
          Pour placer le point $M$, il suffit donc de construire au compas la bissectrice de l'angle $ \widehat{ BAC }. $
          \par
          $M$ est alors le point d'intersection de cette bissectrice avec le côté $ \left[ BC \right] $~:
          \begin{center}
               \begin{extern}%width="500" alt=""
                    \newrgbcolor{grey}{0.2 0.2 0.2}
                    \psset{xunit=1.0cm,yunit=1.0cm,algebraic=true,dimen=middle,dotstyle=o,dotsize=5pt 0,linewidth=1.6pt,arrowsize=3pt 2,arrowinset=0.25}
                    \begin{pspicture*}(0.,0.)(9.,5.)
                         \psline[linewidth=0.4pt,linecolor=grey](1.,1.)(8.,1.)
                         \psline[linewidth=0.4pt,linecolor=grey](8.,1.)(5.142857142857143,3.79941684889506)
                         \psline[linewidth=0.4pt,linecolor=grey](5.142857142857143,3.79941684889506)(1.,1.)
                         \psline[linewidth=0.4pt,linecolor=red](3.382592586423977,2.609968579772903)(6.356832699017947,2.609968579772903)
                         \psline[linewidth=0.4pt,linecolor=red](6.356832699017947,2.609968579772903)(3.97424011259397,1.)
                         \psline[linewidth=0.4pt,linecolor=red](3.97424011259397,1.)(1.,1.)
                         \psline[linewidth=0.4pt,linecolor=red](1.,1.)(3.382592586423977,2.609968579772903)
                         \parametricplot[linewidth=0.4pt,linecolor=grey]{0.4430230091369528}{0.7809029873351472}{1.*1.500369888604726*cos(t)+0.*1.500369888604726*sin(t)+1.|0.*1.500369888604726*cos(t)+1.*1.500369888604726*sin(t)+1.}
                         \parametricplot[linewidth=0.4pt,linecolor=grey]{-0.18351779158694992}{0.18839081772518879}{1.*1.5003698886047259*cos(t)+0.*1.5003698886047259*sin(t)+1.|0.*1.5003698886047259*cos(t)+1.*1.5003698886047259*sin(t)+1.}
                         \parametricplot[linewidth=0.4pt,linecolor=grey]{-0.14893404425129297}{0.16320932683782904}{1.*1.5086156333068808*cos(t)+0.*1.5086156333068808*sin(t)+2.2499958104542728|0.*1.5086156333068808*cos(t)+1.*1.5086156333068808*sin(t)+1.8446488044771545}
                         \parametricplot[linewidth=0.4pt,linecolor=grey]{0.4243963995945869}{0.7436288249084765}{1.*1.506194185300565*cos(t)+0.*1.506194185300565*sin(t)+2.5061941853005654|0.*1.506194185300565*cos(t)+1.*1.506194185300565*sin(t)+1.}
                         \psplot[linewidth=0.4pt,linecolor=grey]{0.}{9.}{(--3.7468641192450445--1.6099685797729029*x)/5.356832699017947}
                         \begin{scriptsize}
                              \psdots[dotsize=2pt 0,dotstyle=*,linecolor=grey](1.,1.)
                              \rput[bl](0.855296988302761,0.7186720182506175){\grey{$A$}}
                              \psdots[dotsize=2pt 0,dotstyle=*,linecolor=grey](8.,1.)
                              \rput[bl](8.06583327519023,0.7186720182506175){\grey{$B$}}
                              \psdots[dotsize=2pt 0,dotstyle=*,linecolor=grey](5.142857142857143,3.79941684889506)
                              \rput[bl](5.1767631804440715,3.835885016525611){\grey{$C$}}
                              \psdots[dotsize=2pt 0,dotstyle=*](6.356832699017947,2.609968579772903)
                              \rput[bl](6.4,2.68){$M$}
                              \psdots[dotsize=2pt 0,dotstyle=*,linecolor=grey](3.382592586423977,2.609968579772903)
                              \rput[bl](3.2507164506133,2.676042385468288){\grey{$N$}}
                              \psdots[dotsize=2pt 0,dotstyle=*,linecolor=grey](3.97424011259397,1.)
                              \rput[bl](3.9709608159701855,0.8186720182506175){\grey{$P$}}
                         \end{scriptsize}
                    \end{pspicture*}
               \end{extern}
          \end{center}
     \end{enumerate}
\end{corrige}

\end{document}
µ
\documentclass[a4paper]{article}

%================================================================================================================================
%
% Packages
%
%================================================================================================================================

\usepackage[T1]{fontenc} 	% pour caractères accentués
\usepackage[utf8]{inputenc}  % encodage utf8
\usepackage[french]{babel}	% langue : français
\usepackage{fourier}			% caractères plus lisibles
\usepackage[dvipsnames]{xcolor} % couleurs
\usepackage{fancyhdr}		% réglage header footer
\usepackage{needspace}		% empêcher sauts de page mal placés
\usepackage{graphicx}		% pour inclure des graphiques
\usepackage{enumitem,cprotect}		% personnalise les listes d'items (nécessaire pour ol, al ...)
\usepackage{hyperref}		% Liens hypertexte
\usepackage{pstricks,pst-all,pst-node,pstricks-add,pst-math,pst-plot,pst-tree,pst-eucl} % pstricks
\usepackage[a4paper,includeheadfoot,top=2cm,left=3cm, bottom=2cm,right=3cm]{geometry} % marges etc.
\usepackage{comment}			% commentaires multilignes
\usepackage{amsmath,environ} % maths (matrices, etc.)
\usepackage{amssymb,makeidx}
\usepackage{bm}				% bold maths
\usepackage{tabularx}		% tableaux
\usepackage{colortbl}		% tableaux en couleur
\usepackage{fontawesome}		% Fontawesome
\usepackage{environ}			% environment with command
\usepackage{fp}				% calculs pour ps-tricks
\usepackage{multido}			% pour ps tricks
\usepackage[np]{numprint}	% formattage nombre
\usepackage{tikz,tkz-tab} 			% package principal TikZ
\usepackage{pgfplots}   % axes
\usepackage{mathrsfs}    % cursives
\usepackage{calc}			% calcul taille boites
\usepackage[scaled=0.875]{helvet} % font sans serif
\usepackage{svg} % svg
\usepackage{scrextend} % local margin
\usepackage{scratch} %scratch
\usepackage{multicol} % colonnes
%\usepackage{infix-RPN,pst-func} % formule en notation polanaise inversée
\usepackage{listings}

%================================================================================================================================
%
% Réglages de base
%
%================================================================================================================================

\lstset{
language=Python,   % R code
literate=
{á}{{\'a}}1
{à}{{\`a}}1
{ã}{{\~a}}1
{é}{{\'e}}1
{è}{{\`e}}1
{ê}{{\^e}}1
{í}{{\'i}}1
{ó}{{\'o}}1
{õ}{{\~o}}1
{ú}{{\'u}}1
{ü}{{\"u}}1
{ç}{{\c{c}}}1
{~}{{ }}1
}


\definecolor{codegreen}{rgb}{0,0.6,0}
\definecolor{codegray}{rgb}{0.5,0.5,0.5}
\definecolor{codepurple}{rgb}{0.58,0,0.82}
\definecolor{backcolour}{rgb}{0.95,0.95,0.92}

\lstdefinestyle{mystyle}{
    backgroundcolor=\color{backcolour},   
    commentstyle=\color{codegreen},
    keywordstyle=\color{magenta},
    numberstyle=\tiny\color{codegray},
    stringstyle=\color{codepurple},
    basicstyle=\ttfamily\footnotesize,
    breakatwhitespace=false,         
    breaklines=true,                 
    captionpos=b,                    
    keepspaces=true,                 
    numbers=left,                    
xleftmargin=2em,
framexleftmargin=2em,            
    showspaces=false,                
    showstringspaces=false,
    showtabs=false,                  
    tabsize=2,
    upquote=true
}

\lstset{style=mystyle}


\lstset{style=mystyle}
\newcommand{\imgdir}{C:/laragon/www/newmc/assets/imgsvg/}
\newcommand{\imgsvgdir}{C:/laragon/www/newmc/assets/imgsvg/}

\definecolor{mcgris}{RGB}{220, 220, 220}% ancien~; pour compatibilité
\definecolor{mcbleu}{RGB}{52, 152, 219}
\definecolor{mcvert}{RGB}{125, 194, 70}
\definecolor{mcmauve}{RGB}{154, 0, 215}
\definecolor{mcorange}{RGB}{255, 96, 0}
\definecolor{mcturquoise}{RGB}{0, 153, 153}
\definecolor{mcrouge}{RGB}{255, 0, 0}
\definecolor{mclightvert}{RGB}{205, 234, 190}

\definecolor{gris}{RGB}{220, 220, 220}
\definecolor{bleu}{RGB}{52, 152, 219}
\definecolor{vert}{RGB}{125, 194, 70}
\definecolor{mauve}{RGB}{154, 0, 215}
\definecolor{orange}{RGB}{255, 96, 0}
\definecolor{turquoise}{RGB}{0, 153, 153}
\definecolor{rouge}{RGB}{255, 0, 0}
\definecolor{lightvert}{RGB}{205, 234, 190}
\setitemize[0]{label=\color{lightvert}  $\bullet$}

\pagestyle{fancy}
\renewcommand{\headrulewidth}{0.2pt}
\fancyhead[L]{maths-cours.fr}
\fancyhead[R]{\thepage}
\renewcommand{\footrulewidth}{0.2pt}
\fancyfoot[C]{}

\newcolumntype{C}{>{\centering\arraybackslash}X}
\newcolumntype{s}{>{\hsize=.35\hsize\arraybackslash}X}

\setlength{\parindent}{0pt}		 
\setlength{\parskip}{3mm}
\setlength{\headheight}{1cm}

\def\ebook{ebook}
\def\book{book}
\def\web{web}
\def\type{web}

\newcommand{\vect}[1]{\overrightarrow{\,\mathstrut#1\,}}

\def\Oij{$\left(\text{O}~;~\vect{\imath},~\vect{\jmath}\right)$}
\def\Oijk{$\left(\text{O}~;~\vect{\imath},~\vect{\jmath},~\vect{k}\right)$}
\def\Ouv{$\left(\text{O}~;~\vect{u},~\vect{v}\right)$}

\hypersetup{breaklinks=true, colorlinks = true, linkcolor = OliveGreen, urlcolor = OliveGreen, citecolor = OliveGreen, pdfauthor={Didier BONNEL - https://www.maths-cours.fr} } % supprime les bordures autour des liens

\renewcommand{\arg}[0]{\text{arg}}

\everymath{\displaystyle}

%================================================================================================================================
%
% Macros - Commandes
%
%================================================================================================================================

\newcommand\meta[2]{    			% Utilisé pour créer le post HTML.
	\def\titre{titre}
	\def\url{url}
	\def\arg{#1}
	\ifx\titre\arg
		\newcommand\maintitle{#2}
		\fancyhead[L]{#2}
		{\Large\sffamily \MakeUppercase{#2}}
		\vspace{1mm}\textcolor{mcvert}{\hrule}
	\fi 
	\ifx\url\arg
		\fancyfoot[L]{\href{https://www.maths-cours.fr#2}{\black \footnotesize{https://www.maths-cours.fr#2}}}
	\fi 
}


\newcommand\TitreC[1]{    		% Titre centré
     \needspace{3\baselineskip}
     \begin{center}\textbf{#1}\end{center}
}

\newcommand\newpar{    		% paragraphe
     \par
}

\newcommand\nosp {    		% commande vide (pas d'espace)
}
\newcommand{\id}[1]{} %ignore

\newcommand\boite[2]{				% Boite simple sans titre
	\vspace{5mm}
	\setlength{\fboxrule}{0.2mm}
	\setlength{\fboxsep}{5mm}	
	\fcolorbox{#1}{#1!3}{\makebox[\linewidth-2\fboxrule-2\fboxsep]{
  		\begin{minipage}[t]{\linewidth-2\fboxrule-4\fboxsep}\setlength{\parskip}{3mm}
  			 #2
  		\end{minipage}
	}}
	\vspace{5mm}
}

\newcommand\CBox[4]{				% Boites
	\vspace{5mm}
	\setlength{\fboxrule}{0.2mm}
	\setlength{\fboxsep}{5mm}
	
	\fcolorbox{#1}{#1!3}{\makebox[\linewidth-2\fboxrule-2\fboxsep]{
		\begin{minipage}[t]{1cm}\setlength{\parskip}{3mm}
	  		\textcolor{#1}{\LARGE{#2}}    
 	 	\end{minipage}  
  		\begin{minipage}[t]{\linewidth-2\fboxrule-4\fboxsep}\setlength{\parskip}{3mm}
			\raisebox{1.2mm}{\normalsize\sffamily{\textcolor{#1}{#3}}}						
  			 #4
  		\end{minipage}
	}}
	\vspace{5mm}
}

\newcommand\cadre[3]{				% Boites convertible html
	\par
	\vspace{2mm}
	\setlength{\fboxrule}{0.1mm}
	\setlength{\fboxsep}{5mm}
	\fcolorbox{#1}{white}{\makebox[\linewidth-2\fboxrule-2\fboxsep]{
  		\begin{minipage}[t]{\linewidth-2\fboxrule-4\fboxsep}\setlength{\parskip}{3mm}
			\raisebox{-2.5mm}{\sffamily \small{\textcolor{#1}{\MakeUppercase{#2}}}}		
			\par		
  			 #3
 	 		\end{minipage}
	}}
		\vspace{2mm}
	\par
}

\newcommand\bloc[3]{				% Boites convertible html sans bordure
     \needspace{2\baselineskip}
     {\sffamily \small{\textcolor{#1}{\MakeUppercase{#2}}}}    
		\par		
  			 #3
		\par
}

\newcommand\CHelp[1]{
     \CBox{Plum}{\faInfoCircle}{À RETENIR}{#1}
}

\newcommand\CUp[1]{
     \CBox{NavyBlue}{\faThumbsOUp}{EN PRATIQUE}{#1}
}

\newcommand\CInfo[1]{
     \CBox{Sepia}{\faArrowCircleRight}{REMARQUE}{#1}
}

\newcommand\CRedac[1]{
     \CBox{PineGreen}{\faEdit}{BIEN R\'EDIGER}{#1}
}

\newcommand\CError[1]{
     \CBox{Red}{\faExclamationTriangle}{ATTENTION}{#1}
}

\newcommand\TitreExo[2]{
\needspace{4\baselineskip}
 {\sffamily\large EXERCICE #1\ (\emph{#2 points})}
\vspace{5mm}
}

\newcommand\img[2]{
          \includegraphics[width=#2\paperwidth]{\imgdir#1}
}

\newcommand\imgsvg[2]{
       \begin{center}   \includegraphics[width=#2\paperwidth]{\imgsvgdir#1} \end{center}
}


\newcommand\Lien[2]{
     \href{#1}{#2 \tiny \faExternalLink}
}
\newcommand\mcLien[2]{
     \href{https~://www.maths-cours.fr/#1}{#2 \tiny \faExternalLink}
}

\newcommand{\euro}{\eurologo{}}

%================================================================================================================================
%
% Macros - Environement
%
%================================================================================================================================

\newenvironment{tex}{ %
}
{%
}

\newenvironment{indente}{ %
	\setlength\parindent{10mm}
}

{
	\setlength\parindent{0mm}
}

\newenvironment{corrige}{%
     \needspace{3\baselineskip}
     \medskip
     \textbf{\textsc{Corrigé}}
     \medskip
}
{
}

\newenvironment{extern}{%
     \begin{center}
     }
     {
     \end{center}
}

\NewEnviron{code}{%
	\par
     \boite{gray}{\texttt{%
     \BODY
     }}
     \par
}

\newenvironment{vbloc}{% boite sans cadre empeche saut de page
     \begin{minipage}[t]{\linewidth}
     }
     {
     \end{minipage}
}
\NewEnviron{h2}{%
    \needspace{3\baselineskip}
    \vspace{0.6cm}
	\noindent \MakeUppercase{\sffamily \large \BODY}
	\vspace{1mm}\textcolor{mcgris}{\hrule}\vspace{0.4cm}
	\par
}{}

\NewEnviron{h3}{%
    \needspace{3\baselineskip}
	\vspace{5mm}
	\textsc{\BODY}
	\par
}

\NewEnviron{margeneg}{ %
\begin{addmargin}[-1cm]{0cm}
\BODY
\end{addmargin}
}

\NewEnviron{html}{%
}

\begin{document}
\meta{url}{/cours/python-au-lycee-4-les-fonctions/}
\meta{pid}{11610}
\meta{titre}{Python au lycée (4)~: Les fonctions}
\meta{type}{cours}
%
\begin{h2}Fonctions en programmation\end{h2}
Lorsqu'un groupe d'instructions se répète plusieurs fois dans un programme, il peut être utile de regrouper ces instructions à l'intérieur de \textit{fonctions.}
\par
Cela permet de diminuer la taille du programme, de faciliter sa maintenance et de le rendre plus lisible en fractionnant le code en blocs indépendants. Par ailleurs, un programmeur peut ainsi utiliser un code écrit par une autre personne.
\par
Dans les chapitres précédents, nous avons déjà utilisé des fonctions comme \texttt{print()} ou \texttt{len()} qui sont des fonctions internes à Python.
\par
L'appel à une fonction peut être schématisé de la manière suivante~:
\par
\begin{center}
     \begin{extern}%alt="Diagramme fonction python" model="modeles-diagramme" style="width:50rem"
          \newrgbcolor{qqzzqq}{0. 0.6 0.}
          \psset{xunit=1.0cm,yunit=1.0cm,algebraic=true,dimen=middle,dotstyle=o,dotsize=5pt 0,linewidth=0.6pt,arrowsize=3pt 2,arrowinset=0.25}
          \begin{pspicture*}(0.,1.)(8.,4.)
               \pspolygon[linecolor=blue,fillcolor=white,fillstyle=solid,opacity=0.05](1.,3.)(1.,2.)(3.,2.)(3.,3.)
               \pspolygon[linecolor=red,fillcolor=white,fillstyle=solid,opacity=0.05](5.,3.)(5.,2.)(7.,2.)(7.,3.)
               \psline[linecolor=blue](1.,3.)(1.,2.)
               \psline[linecolor=blue](1.,2.)(3.,2.)
               \psline[linecolor=blue](3.,2.)(3.,3.)
               \psline[linecolor=blue](3.,3.)(1.,3.)
               \psline[linecolor=red](5.,3.)(5.,2.)
               \psline[linecolor=red](5.,2.)(7.,2.)
               \psline[linecolor=red](7.,2.)(7.,3.)
               \psline[linecolor=red](7.,3.)(5.,3.)
               \psline[linecolor=qqzzqq]{->}(3.,2.8)(5.,2.8)
               \psline[linecolor=magenta]{->}(5.,2.2)(3.,2.2)
               \rput[t](4,3.05){\tiny\qqzzqq{Argument(s)}}
               \rput[t](2,2.6){\tiny\blue{Programme principal}}
               \rput[t](6,2.6){\tiny\red{Fonction}}
               \rput[t](4,2.1){\tiny\magenta{Valeur retournée}}
          \end{pspicture*}
     \end{extern}
\end{center}
\par
\begin{itemize}
     \item
     Le programme appelle la fonction en lui passant éventuellement certaines valeurs appelées \textit{arguments} ou \textit{paramètres}. Il peut y avoir zéro, un ou plusieurs paramètre(s) de n'importe quel type.
     \item
     La fonction effectue certaines opérations en utilisant, au besoin, les valeurs passées en paramètre.
     \item
     La fonction peut ensuite retourner un résultat au programme qui l'a appelé~; le retour d'une valeur n'est toutefois pas obligatoire.
\end{itemize}
\par
Considérons, par exemple, l'instruction~:
\\
\begin{lstlisting}[language=Python]
x=len("Bonjour") 
\end{lstlisting}
L'exécution de cette commande se déroule de la manière suivante~:
\par
\begin{center}
     \begin{extern}%alt="Diagramme exemple fct python" model="modeles-diagramme" style="width:50rem"
          \newrgbcolor{qqzzqq}{0. 0.6 0.}
          \psset{xunit=1.0cm,yunit=1.0cm,algebraic=true,dimen=middle,dotstyle=o,dotsize=5pt 0,linewidth=0.6pt,arrowsize=3pt 2,arrowinset=0.25}
          \begin{pspicture*}(0.,1.)(8.,4.)
               \pspolygon[linecolor=blue,fillcolor=white,fillstyle=solid,opacity=0.05](1.,3.)(1.,2.)(3.,2.)(3.,3.)
               \pspolygon[linecolor=red,fillcolor=white,fillstyle=solid,opacity=0.05](5.,3.)(5.,2.)(7.,2.)(7.,3.)
               \psline[linecolor=blue](1.,3.)(1.,2.)
               \psline[linecolor=blue](1.,2.)(3.,2.)
               \psline[linecolor=blue](3.,2.)(3.,3.)
               \psline[linecolor=blue](3.,3.)(1.,3.)
               \psline[linecolor=red](5.,3.)(5.,2.)
               \psline[linecolor=red](5.,2.)(7.,2.)
               \psline[linecolor=red](7.,2.)(7.,3.)
               \psline[linecolor=red](7.,3.)(5.,3.)
               \psline[linecolor=qqzzqq]{->}(3.,2.8)(5.,2.8)
               \psline[linecolor=magenta]{->}(5.,2.2)(3.,2.2)
               \rput[t](4,3.05){\tiny\qqzzqq{"Bonjour"}}
               \rput[t](2,2.6){\tiny\blue{Programme principal}}
               \rput[t](6,2.6){\tiny\red{Fonction len()}}
               \rput[t](4.,2.1){\tiny\magenta{7}}
          \end{pspicture*}
     \end{extern}
\end{center}
\par\begin{itemize}
     \item
     le programme appelle la fonction Python \texttt{len()} en lui passant le paramètre "Bonjour" de type chaîne de caractères~;
     \item
     la fonction \texttt{len()} calcule la longueur de cette chaîne~;
     \item
     la fonction renvoie l'entier 7 qui correspond à cette longueur~;
     \item
     le résultat est stocké dans la variable \texttt{x} qui pourra être utilisé par la suite.
\end{itemize}
\par
Dans les exemples précédents, les fonctions ont été appelées à partir du programme principal. Cependant, une fonction peut être appelée à partir de n'importe où~; en particulier, une fonction peut très bien être appelée par une autre fonction.
\begin{h2} Définition et appel d'une fonction en Python \end{h2}
En plus des fonctions prédéfinies par Python, le programmeur a la possibilité de définir ses propres fonctions.
\par
On utilise le mot-clé \texttt{def} pour définir une nouvelle fonction avec la syntaxe suivante~:
\begin{lstlisting}[language=Python]
def nom_de_la_fonction(liste_des_paramètres):
   # bloc d'instructions
   # calculant le résultat
   return resultat # facultatif
\end{lstlisting}
\par
Par exemple~:
 \begin{lstlisting}[language=Python]
def ajoute(x,y):
   somme=x+y
   return somme
\end{lstlisting}
\par
Les paramètres passés à la fonction sont placés entre parenthèses. Même si la fonction n'admet pas d'argument, il faut utiliser des parenthèses~; dans ce cas on ne place rien entre les parenthèses.
\par
Lorsque l'on définit une fonction en Python, aucune instruction n'est exécutée. L'exécution des instructions situées dans le corps de la fonction n'a lieu que lors de l' \textbf{appel} de la fonction.
\par
L'appel d'une fonction en Python obéit à la syntaxe suivante~:
\begin{lstlisting}[language=Python]
nom_de_la_fonction(valeurs_des_paramètres)
\end{lstlisting}
Le résultat est alors souvent stocké dans une variable~:
\begin{lstlisting}[language=Python]
resultat=nom_de_la_fonction(valeurs_des_paramètres)
\end{lstlisting}
ou affiché à l'écran~:
\begin{lstlisting}[language=Python]
print(nom_de_la_fonction(valeurs_des_paramètres))
\end{lstlisting}
Par exemple, pour utiliser la fonction \texttt{ajoute} définie précédemment~:
 \begin{lstlisting}[language=Python]
x=ajoute(4,3) # x contient la valeur 7
\end{lstlisting}
\begin{h2} Les modules Python \end{h2}
En Python, il est possible de regrouper des fonctions (ou d'autres objets) dans des fichiers appelés \textit{modules} afin de pouvoir les réutiliser par la suite dans d'autres programmes.
\par
Un certain nombre de modules sont intégrés à Python mais il est nécessaire de les charger dans son programme afin de pouvoir les utiliser. Pour charger un module on peut utiliser la syntaxe suivante~:
\begin{lstlisting}[language=Python]
import nom_du_module
\end{lstlisting}
Par la suite on peut utiliser une fonction de ce module en la préfixant par le nom du module de la manière suivante~:
\begin{lstlisting}[language=Python]
resultat = nom_du_module.nom_de la fonction # Noter la présence du point
\end{lstlisting}
Par exemple pour utiliser la fonction \texttt{sqrt()} (racine carrée) du module \texttt{math}~:
\begin{lstlisting}[language=Python]
import math
print(math.sqrt(9)) # affiche 3.0
\end{lstlisting}
Pour éviter de devoir préfixer la fonction à chaque fois qu'on l'utilise, on peut choisir de charger cette fonction lors de l'\texttt{import} en utilisant la syntaxe suivante~:
  \begin{lstlisting}[language=Python]
from math import sqrt
print(sqrt(9)) # affiche 3.0 
\end{lstlisting}
On peut également importer tous les objets d'un module en utilisant une *~:
  \begin{lstlisting}[language=Python]
from math import *
print(sqrt(9)) # affiche 3.0 
print(cos(pi)) # affiche -1.0 
\end{lstlisting}
Parmi les nombreux modules intégrés à Python, les modules \texttt{math} et \texttt{random} seront fréquemment utilisés au lycée~:
\begin{itemize}
     \item
     \textbf{Le module "math"}~:
     \\
     Ce module contient différentes fonctions et constantes mathématiques~: \texttt{ pi, sqrt(), exp(), ln(), sin(), cos(), ...}
     \item
     \textbf{Le module "random"}~:
     \\
     Ce module, utile en statistiques et en probabilités, permet de génères des nombres aléatoires.
     \begin{itemize}
          \item
          random() renvoie un nombre réel aléatoire compris entre 0 et 1.
          \item
          randint(min, max) renvoie un nombre entier aléatoire compris entre les entiers min et max (bornes incluses).
     \end{itemize}
\end{itemize}
Par exemple on peut simuler dix lancers d'un dé à six faces grâce au programme suivant~:
  \begin{lstlisting}[language=Python]
from random import randint
for i in range (10):
    print(randint(1,6))
\end{lstlisting}

\end{document}
µ
\documentclass[a4paper]{article}

%================================================================================================================================
%
% Packages
%
%================================================================================================================================

\usepackage[T1]{fontenc} 	% pour caractères accentués
\usepackage[utf8]{inputenc}  % encodage utf8
\usepackage[french]{babel}	% langue : français
\usepackage{fourier}			% caractères plus lisibles
\usepackage[dvipsnames]{xcolor} % couleurs
\usepackage{fancyhdr}		% réglage header footer
\usepackage{needspace}		% empêcher sauts de page mal placés
\usepackage{graphicx}		% pour inclure des graphiques
\usepackage{enumitem,cprotect}		% personnalise les listes d'items (nécessaire pour ol, al ...)
\usepackage{hyperref}		% Liens hypertexte
\usepackage{pstricks,pst-all,pst-node,pstricks-add,pst-math,pst-plot,pst-tree,pst-eucl} % pstricks
\usepackage[a4paper,includeheadfoot,top=2cm,left=3cm, bottom=2cm,right=3cm]{geometry} % marges etc.
\usepackage{comment}			% commentaires multilignes
\usepackage{amsmath,environ} % maths (matrices, etc.)
\usepackage{amssymb,makeidx}
\usepackage{bm}				% bold maths
\usepackage{tabularx}		% tableaux
\usepackage{colortbl}		% tableaux en couleur
\usepackage{fontawesome}		% Fontawesome
\usepackage{environ}			% environment with command
\usepackage{fp}				% calculs pour ps-tricks
\usepackage{multido}			% pour ps tricks
\usepackage[np]{numprint}	% formattage nombre
\usepackage{tikz,tkz-tab} 			% package principal TikZ
\usepackage{pgfplots}   % axes
\usepackage{mathrsfs}    % cursives
\usepackage{calc}			% calcul taille boites
\usepackage[scaled=0.875]{helvet} % font sans serif
\usepackage{svg} % svg
\usepackage{scrextend} % local margin
\usepackage{scratch} %scratch
\usepackage{multicol} % colonnes
%\usepackage{infix-RPN,pst-func} % formule en notation polanaise inversée
\usepackage{listings}

%================================================================================================================================
%
% Réglages de base
%
%================================================================================================================================

\lstset{
language=Python,   % R code
literate=
{á}{{\'a}}1
{à}{{\`a}}1
{ã}{{\~a}}1
{é}{{\'e}}1
{è}{{\`e}}1
{ê}{{\^e}}1
{í}{{\'i}}1
{ó}{{\'o}}1
{õ}{{\~o}}1
{ú}{{\'u}}1
{ü}{{\"u}}1
{ç}{{\c{c}}}1
{~}{{ }}1
}


\definecolor{codegreen}{rgb}{0,0.6,0}
\definecolor{codegray}{rgb}{0.5,0.5,0.5}
\definecolor{codepurple}{rgb}{0.58,0,0.82}
\definecolor{backcolour}{rgb}{0.95,0.95,0.92}

\lstdefinestyle{mystyle}{
    backgroundcolor=\color{backcolour},   
    commentstyle=\color{codegreen},
    keywordstyle=\color{magenta},
    numberstyle=\tiny\color{codegray},
    stringstyle=\color{codepurple},
    basicstyle=\ttfamily\footnotesize,
    breakatwhitespace=false,         
    breaklines=true,                 
    captionpos=b,                    
    keepspaces=true,                 
    numbers=left,                    
xleftmargin=2em,
framexleftmargin=2em,            
    showspaces=false,                
    showstringspaces=false,
    showtabs=false,                  
    tabsize=2,
    upquote=true
}

\lstset{style=mystyle}


\lstset{style=mystyle}
\newcommand{\imgdir}{C:/laragon/www/newmc/assets/imgsvg/}
\newcommand{\imgsvgdir}{C:/laragon/www/newmc/assets/imgsvg/}

\definecolor{mcgris}{RGB}{220, 220, 220}% ancien~; pour compatibilité
\definecolor{mcbleu}{RGB}{52, 152, 219}
\definecolor{mcvert}{RGB}{125, 194, 70}
\definecolor{mcmauve}{RGB}{154, 0, 215}
\definecolor{mcorange}{RGB}{255, 96, 0}
\definecolor{mcturquoise}{RGB}{0, 153, 153}
\definecolor{mcrouge}{RGB}{255, 0, 0}
\definecolor{mclightvert}{RGB}{205, 234, 190}

\definecolor{gris}{RGB}{220, 220, 220}
\definecolor{bleu}{RGB}{52, 152, 219}
\definecolor{vert}{RGB}{125, 194, 70}
\definecolor{mauve}{RGB}{154, 0, 215}
\definecolor{orange}{RGB}{255, 96, 0}
\definecolor{turquoise}{RGB}{0, 153, 153}
\definecolor{rouge}{RGB}{255, 0, 0}
\definecolor{lightvert}{RGB}{205, 234, 190}
\setitemize[0]{label=\color{lightvert}  $\bullet$}

\pagestyle{fancy}
\renewcommand{\headrulewidth}{0.2pt}
\fancyhead[L]{maths-cours.fr}
\fancyhead[R]{\thepage}
\renewcommand{\footrulewidth}{0.2pt}
\fancyfoot[C]{}

\newcolumntype{C}{>{\centering\arraybackslash}X}
\newcolumntype{s}{>{\hsize=.35\hsize\arraybackslash}X}

\setlength{\parindent}{0pt}		 
\setlength{\parskip}{3mm}
\setlength{\headheight}{1cm}

\def\ebook{ebook}
\def\book{book}
\def\web{web}
\def\type{web}

\newcommand{\vect}[1]{\overrightarrow{\,\mathstrut#1\,}}

\def\Oij{$\left(\text{O}~;~\vect{\imath},~\vect{\jmath}\right)$}
\def\Oijk{$\left(\text{O}~;~\vect{\imath},~\vect{\jmath},~\vect{k}\right)$}
\def\Ouv{$\left(\text{O}~;~\vect{u},~\vect{v}\right)$}

\hypersetup{breaklinks=true, colorlinks = true, linkcolor = OliveGreen, urlcolor = OliveGreen, citecolor = OliveGreen, pdfauthor={Didier BONNEL - https://www.maths-cours.fr} } % supprime les bordures autour des liens

\renewcommand{\arg}[0]{\text{arg}}

\everymath{\displaystyle}

%================================================================================================================================
%
% Macros - Commandes
%
%================================================================================================================================

\newcommand\meta[2]{    			% Utilisé pour créer le post HTML.
	\def\titre{titre}
	\def\url{url}
	\def\arg{#1}
	\ifx\titre\arg
		\newcommand\maintitle{#2}
		\fancyhead[L]{#2}
		{\Large\sffamily \MakeUppercase{#2}}
		\vspace{1mm}\textcolor{mcvert}{\hrule}
	\fi 
	\ifx\url\arg
		\fancyfoot[L]{\href{https://www.maths-cours.fr#2}{\black \footnotesize{https://www.maths-cours.fr#2}}}
	\fi 
}


\newcommand\TitreC[1]{    		% Titre centré
     \needspace{3\baselineskip}
     \begin{center}\textbf{#1}\end{center}
}

\newcommand\newpar{    		% paragraphe
     \par
}

\newcommand\nosp {    		% commande vide (pas d'espace)
}
\newcommand{\id}[1]{} %ignore

\newcommand\boite[2]{				% Boite simple sans titre
	\vspace{5mm}
	\setlength{\fboxrule}{0.2mm}
	\setlength{\fboxsep}{5mm}	
	\fcolorbox{#1}{#1!3}{\makebox[\linewidth-2\fboxrule-2\fboxsep]{
  		\begin{minipage}[t]{\linewidth-2\fboxrule-4\fboxsep}\setlength{\parskip}{3mm}
  			 #2
  		\end{minipage}
	}}
	\vspace{5mm}
}

\newcommand\CBox[4]{				% Boites
	\vspace{5mm}
	\setlength{\fboxrule}{0.2mm}
	\setlength{\fboxsep}{5mm}
	
	\fcolorbox{#1}{#1!3}{\makebox[\linewidth-2\fboxrule-2\fboxsep]{
		\begin{minipage}[t]{1cm}\setlength{\parskip}{3mm}
	  		\textcolor{#1}{\LARGE{#2}}    
 	 	\end{minipage}  
  		\begin{minipage}[t]{\linewidth-2\fboxrule-4\fboxsep}\setlength{\parskip}{3mm}
			\raisebox{1.2mm}{\normalsize\sffamily{\textcolor{#1}{#3}}}						
  			 #4
  		\end{minipage}
	}}
	\vspace{5mm}
}

\newcommand\cadre[3]{				% Boites convertible html
	\par
	\vspace{2mm}
	\setlength{\fboxrule}{0.1mm}
	\setlength{\fboxsep}{5mm}
	\fcolorbox{#1}{white}{\makebox[\linewidth-2\fboxrule-2\fboxsep]{
  		\begin{minipage}[t]{\linewidth-2\fboxrule-4\fboxsep}\setlength{\parskip}{3mm}
			\raisebox{-2.5mm}{\sffamily \small{\textcolor{#1}{\MakeUppercase{#2}}}}		
			\par		
  			 #3
 	 		\end{minipage}
	}}
		\vspace{2mm}
	\par
}

\newcommand\bloc[3]{				% Boites convertible html sans bordure
     \needspace{2\baselineskip}
     {\sffamily \small{\textcolor{#1}{\MakeUppercase{#2}}}}    
		\par		
  			 #3
		\par
}

\newcommand\CHelp[1]{
     \CBox{Plum}{\faInfoCircle}{À RETENIR}{#1}
}

\newcommand\CUp[1]{
     \CBox{NavyBlue}{\faThumbsOUp}{EN PRATIQUE}{#1}
}

\newcommand\CInfo[1]{
     \CBox{Sepia}{\faArrowCircleRight}{REMARQUE}{#1}
}

\newcommand\CRedac[1]{
     \CBox{PineGreen}{\faEdit}{BIEN R\'EDIGER}{#1}
}

\newcommand\CError[1]{
     \CBox{Red}{\faExclamationTriangle}{ATTENTION}{#1}
}

\newcommand\TitreExo[2]{
\needspace{4\baselineskip}
 {\sffamily\large EXERCICE #1\ (\emph{#2 points})}
\vspace{5mm}
}

\newcommand\img[2]{
          \includegraphics[width=#2\paperwidth]{\imgdir#1}
}

\newcommand\imgsvg[2]{
       \begin{center}   \includegraphics[width=#2\paperwidth]{\imgsvgdir#1} \end{center}
}


\newcommand\Lien[2]{
     \href{#1}{#2 \tiny \faExternalLink}
}
\newcommand\mcLien[2]{
     \href{https~://www.maths-cours.fr/#1}{#2 \tiny \faExternalLink}
}

\newcommand{\euro}{\eurologo{}}

%================================================================================================================================
%
% Macros - Environement
%
%================================================================================================================================

\newenvironment{tex}{ %
}
{%
}

\newenvironment{indente}{ %
	\setlength\parindent{10mm}
}

{
	\setlength\parindent{0mm}
}

\newenvironment{corrige}{%
     \needspace{3\baselineskip}
     \medskip
     \textbf{\textsc{Corrigé}}
     \medskip
}
{
}

\newenvironment{extern}{%
     \begin{center}
     }
     {
     \end{center}
}

\NewEnviron{code}{%
	\par
     \boite{gray}{\texttt{%
     \BODY
     }}
     \par
}

\newenvironment{vbloc}{% boite sans cadre empeche saut de page
     \begin{minipage}[t]{\linewidth}
     }
     {
     \end{minipage}
}
\NewEnviron{h2}{%
    \needspace{3\baselineskip}
    \vspace{0.6cm}
	\noindent \MakeUppercase{\sffamily \large \BODY}
	\vspace{1mm}\textcolor{mcgris}{\hrule}\vspace{0.4cm}
	\par
}{}

\NewEnviron{h3}{%
    \needspace{3\baselineskip}
	\vspace{5mm}
	\textsc{\BODY}
	\par
}

\NewEnviron{margeneg}{ %
\begin{addmargin}[-1cm]{0cm}
\BODY
\end{addmargin}
}

\NewEnviron{html}{%
}

\begin{document}
\meta{url}{/exercices/calcul-dinterets-en-python/}
\meta{pid}{11636}
\meta{titre}{Calcul d'intérêts en Python}
\meta{type}{exercices}
%
On place un capital de 10~000~euros sur un compte rémunéré à 1,5\% d'intérêts par an (à intérêts composés).
\par
\begin{enumerate}
     \item
     Compléter le programme Python ci-dessous afin qu'il affiche le capital disponible au bout de 5 ans.
\begin{lstlisting}[language=Python]
C=10000
for i in range (...) :
   C = ...
print(C)
\end{lstlisting}
\item
On souhaite savoir au bout de combien d'années le capital aura dépassé 12~000~euros.
\par
Compléter le programme Python ci-dessous afin qu'il affiche ce nombre d'années.
\begin{lstlisting}[language=Python]
C=10000
n=0
while ...:
   n = ...
   C = ...
print(...)
\end{lstlisting}
Répondre à la question posée en utilisant ce programme.
\end{enumerate}
\begin{corrige}
     \begin{enumerate}
          \item
          Le capital $ C' $ obtenu après un an en plaçant un montant $C$ à $ t = 1,5\% $ est~:
          \par
          $C' = C \left( 1 &plus; \frac{ t }{ 100 } \right) $\nosp$= C \left( 1 + \frac{ 1,5 }{ 100 } \right) =1,015\ C$.
          \par
          On doit effectuer cette opération cinq fois pour obtenir le capital au bout de 5 ans.
          \par
          On peut donc compléter le programme comme suit~:
\begin{lstlisting}[language=Python]
C=10000
for i in range (5) :
   C = 1.015*C
print(C)
     \end{lstlisting}
     On obtient comme résultat $ 10~772,84 $ (arrondi au centime).
     \item
     Comme on ne connait pas, au départ, le nombre d'itérations, on va utiliser une boucle non bornée \texttt{while}
     \par
     On reste dans la boucle tant que \texttt{C < 12000} et chaque passage on incrémente le nombre d'années \texttt{n} et on calcule le nouveau capital \texttt{C}.
     \par
     À la sortie de la boucle, on affiche le nombre d'années \texttt{n}.
\begin{lstlisting}[language=Python]
C=10000
n=0
while C < 12000 :
   n = n + 1
   C = 1.015*C
print(n)
\end{lstlisting}
\end{enumerate}
Ce programme affiche le résultat 13.
\par
Le capital dépassera donc 12~000~euros au bout de 13 ans.
\end{corrige}

\end{document}
µ
\documentclass[a4paper]{article}

%================================================================================================================================
%
% Packages
%
%================================================================================================================================

\usepackage[T1]{fontenc} 	% pour caractères accentués
\usepackage[utf8]{inputenc}  % encodage utf8
\usepackage[french]{babel}	% langue : français
\usepackage{fourier}			% caractères plus lisibles
\usepackage[dvipsnames]{xcolor} % couleurs
\usepackage{fancyhdr}		% réglage header footer
\usepackage{needspace}		% empêcher sauts de page mal placés
\usepackage{graphicx}		% pour inclure des graphiques
\usepackage{enumitem,cprotect}		% personnalise les listes d'items (nécessaire pour ol, al ...)
\usepackage{hyperref}		% Liens hypertexte
\usepackage{pstricks,pst-all,pst-node,pstricks-add,pst-math,pst-plot,pst-tree,pst-eucl} % pstricks
\usepackage[a4paper,includeheadfoot,top=2cm,left=3cm, bottom=2cm,right=3cm]{geometry} % marges etc.
\usepackage{comment}			% commentaires multilignes
\usepackage{amsmath,environ} % maths (matrices, etc.)
\usepackage{amssymb,makeidx}
\usepackage{bm}				% bold maths
\usepackage{tabularx}		% tableaux
\usepackage{colortbl}		% tableaux en couleur
\usepackage{fontawesome}		% Fontawesome
\usepackage{environ}			% environment with command
\usepackage{fp}				% calculs pour ps-tricks
\usepackage{multido}			% pour ps tricks
\usepackage[np]{numprint}	% formattage nombre
\usepackage{tikz,tkz-tab} 			% package principal TikZ
\usepackage{pgfplots}   % axes
\usepackage{mathrsfs}    % cursives
\usepackage{calc}			% calcul taille boites
\usepackage[scaled=0.875]{helvet} % font sans serif
\usepackage{svg} % svg
\usepackage{scrextend} % local margin
\usepackage{scratch} %scratch
\usepackage{multicol} % colonnes
%\usepackage{infix-RPN,pst-func} % formule en notation polanaise inversée
\usepackage{listings}

%================================================================================================================================
%
% Réglages de base
%
%================================================================================================================================

\lstset{
language=Python,   % R code
literate=
{á}{{\'a}}1
{à}{{\`a}}1
{ã}{{\~a}}1
{é}{{\'e}}1
{è}{{\`e}}1
{ê}{{\^e}}1
{í}{{\'i}}1
{ó}{{\'o}}1
{õ}{{\~o}}1
{ú}{{\'u}}1
{ü}{{\"u}}1
{ç}{{\c{c}}}1
{~}{{ }}1
}


\definecolor{codegreen}{rgb}{0,0.6,0}
\definecolor{codegray}{rgb}{0.5,0.5,0.5}
\definecolor{codepurple}{rgb}{0.58,0,0.82}
\definecolor{backcolour}{rgb}{0.95,0.95,0.92}

\lstdefinestyle{mystyle}{
    backgroundcolor=\color{backcolour},   
    commentstyle=\color{codegreen},
    keywordstyle=\color{magenta},
    numberstyle=\tiny\color{codegray},
    stringstyle=\color{codepurple},
    basicstyle=\ttfamily\footnotesize,
    breakatwhitespace=false,         
    breaklines=true,                 
    captionpos=b,                    
    keepspaces=true,                 
    numbers=left,                    
xleftmargin=2em,
framexleftmargin=2em,            
    showspaces=false,                
    showstringspaces=false,
    showtabs=false,                  
    tabsize=2,
    upquote=true
}

\lstset{style=mystyle}


\lstset{style=mystyle}
\newcommand{\imgdir}{C:/laragon/www/newmc/assets/imgsvg/}
\newcommand{\imgsvgdir}{C:/laragon/www/newmc/assets/imgsvg/}

\definecolor{mcgris}{RGB}{220, 220, 220}% ancien~; pour compatibilité
\definecolor{mcbleu}{RGB}{52, 152, 219}
\definecolor{mcvert}{RGB}{125, 194, 70}
\definecolor{mcmauve}{RGB}{154, 0, 215}
\definecolor{mcorange}{RGB}{255, 96, 0}
\definecolor{mcturquoise}{RGB}{0, 153, 153}
\definecolor{mcrouge}{RGB}{255, 0, 0}
\definecolor{mclightvert}{RGB}{205, 234, 190}

\definecolor{gris}{RGB}{220, 220, 220}
\definecolor{bleu}{RGB}{52, 152, 219}
\definecolor{vert}{RGB}{125, 194, 70}
\definecolor{mauve}{RGB}{154, 0, 215}
\definecolor{orange}{RGB}{255, 96, 0}
\definecolor{turquoise}{RGB}{0, 153, 153}
\definecolor{rouge}{RGB}{255, 0, 0}
\definecolor{lightvert}{RGB}{205, 234, 190}
\setitemize[0]{label=\color{lightvert}  $\bullet$}

\pagestyle{fancy}
\renewcommand{\headrulewidth}{0.2pt}
\fancyhead[L]{maths-cours.fr}
\fancyhead[R]{\thepage}
\renewcommand{\footrulewidth}{0.2pt}
\fancyfoot[C]{}

\newcolumntype{C}{>{\centering\arraybackslash}X}
\newcolumntype{s}{>{\hsize=.35\hsize\arraybackslash}X}

\setlength{\parindent}{0pt}		 
\setlength{\parskip}{3mm}
\setlength{\headheight}{1cm}

\def\ebook{ebook}
\def\book{book}
\def\web{web}
\def\type{web}

\newcommand{\vect}[1]{\overrightarrow{\,\mathstrut#1\,}}

\def\Oij{$\left(\text{O}~;~\vect{\imath},~\vect{\jmath}\right)$}
\def\Oijk{$\left(\text{O}~;~\vect{\imath},~\vect{\jmath},~\vect{k}\right)$}
\def\Ouv{$\left(\text{O}~;~\vect{u},~\vect{v}\right)$}

\hypersetup{breaklinks=true, colorlinks = true, linkcolor = OliveGreen, urlcolor = OliveGreen, citecolor = OliveGreen, pdfauthor={Didier BONNEL - https://www.maths-cours.fr} } % supprime les bordures autour des liens

\renewcommand{\arg}[0]{\text{arg}}

\everymath{\displaystyle}

%================================================================================================================================
%
% Macros - Commandes
%
%================================================================================================================================

\newcommand\meta[2]{    			% Utilisé pour créer le post HTML.
	\def\titre{titre}
	\def\url{url}
	\def\arg{#1}
	\ifx\titre\arg
		\newcommand\maintitle{#2}
		\fancyhead[L]{#2}
		{\Large\sffamily \MakeUppercase{#2}}
		\vspace{1mm}\textcolor{mcvert}{\hrule}
	\fi 
	\ifx\url\arg
		\fancyfoot[L]{\href{https://www.maths-cours.fr#2}{\black \footnotesize{https://www.maths-cours.fr#2}}}
	\fi 
}


\newcommand\TitreC[1]{    		% Titre centré
     \needspace{3\baselineskip}
     \begin{center}\textbf{#1}\end{center}
}

\newcommand\newpar{    		% paragraphe
     \par
}

\newcommand\nosp {    		% commande vide (pas d'espace)
}
\newcommand{\id}[1]{} %ignore

\newcommand\boite[2]{				% Boite simple sans titre
	\vspace{5mm}
	\setlength{\fboxrule}{0.2mm}
	\setlength{\fboxsep}{5mm}	
	\fcolorbox{#1}{#1!3}{\makebox[\linewidth-2\fboxrule-2\fboxsep]{
  		\begin{minipage}[t]{\linewidth-2\fboxrule-4\fboxsep}\setlength{\parskip}{3mm}
  			 #2
  		\end{minipage}
	}}
	\vspace{5mm}
}

\newcommand\CBox[4]{				% Boites
	\vspace{5mm}
	\setlength{\fboxrule}{0.2mm}
	\setlength{\fboxsep}{5mm}
	
	\fcolorbox{#1}{#1!3}{\makebox[\linewidth-2\fboxrule-2\fboxsep]{
		\begin{minipage}[t]{1cm}\setlength{\parskip}{3mm}
	  		\textcolor{#1}{\LARGE{#2}}    
 	 	\end{minipage}  
  		\begin{minipage}[t]{\linewidth-2\fboxrule-4\fboxsep}\setlength{\parskip}{3mm}
			\raisebox{1.2mm}{\normalsize\sffamily{\textcolor{#1}{#3}}}						
  			 #4
  		\end{minipage}
	}}
	\vspace{5mm}
}

\newcommand\cadre[3]{				% Boites convertible html
	\par
	\vspace{2mm}
	\setlength{\fboxrule}{0.1mm}
	\setlength{\fboxsep}{5mm}
	\fcolorbox{#1}{white}{\makebox[\linewidth-2\fboxrule-2\fboxsep]{
  		\begin{minipage}[t]{\linewidth-2\fboxrule-4\fboxsep}\setlength{\parskip}{3mm}
			\raisebox{-2.5mm}{\sffamily \small{\textcolor{#1}{\MakeUppercase{#2}}}}		
			\par		
  			 #3
 	 		\end{minipage}
	}}
		\vspace{2mm}
	\par
}

\newcommand\bloc[3]{				% Boites convertible html sans bordure
     \needspace{2\baselineskip}
     {\sffamily \small{\textcolor{#1}{\MakeUppercase{#2}}}}    
		\par		
  			 #3
		\par
}

\newcommand\CHelp[1]{
     \CBox{Plum}{\faInfoCircle}{À RETENIR}{#1}
}

\newcommand\CUp[1]{
     \CBox{NavyBlue}{\faThumbsOUp}{EN PRATIQUE}{#1}
}

\newcommand\CInfo[1]{
     \CBox{Sepia}{\faArrowCircleRight}{REMARQUE}{#1}
}

\newcommand\CRedac[1]{
     \CBox{PineGreen}{\faEdit}{BIEN R\'EDIGER}{#1}
}

\newcommand\CError[1]{
     \CBox{Red}{\faExclamationTriangle}{ATTENTION}{#1}
}

\newcommand\TitreExo[2]{
\needspace{4\baselineskip}
 {\sffamily\large EXERCICE #1\ (\emph{#2 points})}
\vspace{5mm}
}

\newcommand\img[2]{
          \includegraphics[width=#2\paperwidth]{\imgdir#1}
}

\newcommand\imgsvg[2]{
       \begin{center}   \includegraphics[width=#2\paperwidth]{\imgsvgdir#1} \end{center}
}


\newcommand\Lien[2]{
     \href{#1}{#2 \tiny \faExternalLink}
}
\newcommand\mcLien[2]{
     \href{https~://www.maths-cours.fr/#1}{#2 \tiny \faExternalLink}
}

\newcommand{\euro}{\eurologo{}}

%================================================================================================================================
%
% Macros - Environement
%
%================================================================================================================================

\newenvironment{tex}{ %
}
{%
}

\newenvironment{indente}{ %
	\setlength\parindent{10mm}
}

{
	\setlength\parindent{0mm}
}

\newenvironment{corrige}{%
     \needspace{3\baselineskip}
     \medskip
     \textbf{\textsc{Corrigé}}
     \medskip
}
{
}

\newenvironment{extern}{%
     \begin{center}
     }
     {
     \end{center}
}

\NewEnviron{code}{%
	\par
     \boite{gray}{\texttt{%
     \BODY
     }}
     \par
}

\newenvironment{vbloc}{% boite sans cadre empeche saut de page
     \begin{minipage}[t]{\linewidth}
     }
     {
     \end{minipage}
}
\NewEnviron{h2}{%
    \needspace{3\baselineskip}
    \vspace{0.6cm}
	\noindent \MakeUppercase{\sffamily \large \BODY}
	\vspace{1mm}\textcolor{mcgris}{\hrule}\vspace{0.4cm}
	\par
}{}

\NewEnviron{h3}{%
    \needspace{3\baselineskip}
	\vspace{5mm}
	\textsc{\BODY}
	\par
}

\NewEnviron{margeneg}{ %
\begin{addmargin}[-1cm]{0cm}
\BODY
\end{addmargin}
}

\NewEnviron{html}{%
}

\begin{document}
\meta{url}{/exercices/nombres-triangulaires-et-python/}
\meta{pid}{11639}
\meta{titre}{Nombres triangulaires et Python}
\meta{type}{exercices}
%
On appelle nombre triangulaire d'ordre $ n $ la somme des nombres entiers naturels compris entre $1$ et $n$~:
\par
On notera~:
\par
$ T_n = 1 + 2 + 3 + \cdots + n $
\par
Par exemple, le nombre triangulaire d'ordre $4$ est~:
\par
$ T_4 = 1 + 2 + 3 + 4 = 10. $
\par
C'est le nombre de points représentés sur la figure ci-dessous~:
\par
\begin{center}
     \begin{extern}%alt="Nombres triangulaires" style="width:15rem"
          \psset{xunit=1.0cm,yunit=1.0cm,algebraic=true,dimen=middle,dotstyle=o,dotsize=5pt 0,linewidth=1.6pt,arrowsize=3pt 2,arrowinset=0.25}
          \begin{pspicture*}(2.,8.)(10.,16.)
               \begin{scriptsize}
                    \psdots[dotsize=19pt 0,dotstyle=*,linecolor=blue](6.,15.)
                    \psdots[dotsize=19pt 0,dotstyle=*,linecolor=blue](5.,13.)
                    \psdots[dotsize=19pt 0,dotstyle=*,linecolor=blue](7.,13.)
                    \psdots[dotsize=19pt 0,dotstyle=*,linecolor=blue](4.,11.)
                    \psdots[dotsize=19pt 0,dotstyle=*,linecolor=blue](6.,11.)
                    \psdots[dotsize=19pt 0,dotstyle=*,linecolor=blue](8.,11.)
                    \psdots[dotsize=19pt 0,dotstyle=*,linecolor=blue](3.,9.)
                    \psdots[dotsize=19pt 0,dotstyle=*,linecolor=blue](5.,9.)
                    \psdots[dotsize=19pt 0,dotstyle=*,linecolor=blue](7.,9.)
                    \psdots[dotsize=19pt 0,dotstyle=*,linecolor=blue](9.,9.)
               \end{scriptsize}
          \end{pspicture*}
     \end{extern}
\end{center}
\begin{enumerate}
     \item
     Compléter le programme Python ci-dessous afin qu'il calcule et affiche le nombre triangulaire d'ordre $20 $ ~:
\begin{lstlisting}[language=Python]
T=0
for n in range(...) :
   T = ...
print(T)
\end{lstlisting}
\item
On souhaite déterminer pour quelle valeur de $ n $ le nombre triangulaire d'ordre $n$ est supérieur ou égal à $1~000$.
\par
Compléter le programme Python ci-dessous afin qu'il affiche ce nombre $n$.
\begin{lstlisting}[language=Python]
T=0
n=0
while ... :
   n = ...
   T = ...
print(n)
\end{lstlisting}
Saisir ce programme dans un éditeur Python. Quelle valeur de $ n $ obtient-on~?
\end{enumerate}
\begin{corrige}
     \begin{enumerate}
          \item
          \par
          Dans le programme proposé, la variable \texttt{T} représente le nombre triangulaire d'ordre \texttt{n}.
          \par
          Pour calculer le nombre triangulaire d'ordre 20, il faut effectuer la somme des entiers compris (au sens large) entre $1$ et $20$.
          \par
          Comme \texttt{ range(a,b) } renvoie la liste des valeurs comprises (au sens large) entre \texttt{a} et \texttt{b-1}~; il faut donc utiliser l'instruction \texttt{ range(1, \textbf{21} ) } pour créer la boucle (l'instruction \texttt{ range(0, 21) } ou \texttt{ range(21) } est aussi valable puisqu'elle ne fait qu'ajouter $0$ à cette somme).
          \par
          Ensuite, à chaque étape de la boucle, on ajoute \texttt{n} à \texttt{T}.
          \par
          Voici le programme complet~:
\begin{lstlisting}[language=Python]
T=0
for n in range(1,21) :
   T = T+n
print(T)
     \end{lstlisting}
     L'exécution de ce programme affiche la valeur 210.
     \item
     Cette fois, on ne connait pas, dès le départ, le nombre d'itérations. On doit donc utiliser une boucle \texttt{while} (boucle non bornée).
     \par
     Ici, la variable \texttt{T} représente le nombre triangulaire d'ordre \texttt{n}.
     \par
     On sort de la boucle lorsque \texttt{T >= 1000}, c'est à dire qu'on reste dans la boucle tant que \texttt{T < 1000}.
     \par
     À chaque passage dans la boucle, on incrémente \texttt{n} puis on l'ajoute au nombre \texttt{T}~:
\begin{lstlisting}[language=Python]
T=0
n=0
while T < 1000 :
   n = n+1
   T = T+n
print(n)
\end{lstlisting}
\end{enumerate}
Ce programme affiche le nombre $ 45 $ comme résultat.
\end{corrige}

\end{document}
µ
\documentclass[a4paper]{article}

%================================================================================================================================
%
% Packages
%
%================================================================================================================================

\usepackage[T1]{fontenc} 	% pour caractères accentués
\usepackage[utf8]{inputenc}  % encodage utf8
\usepackage[french]{babel}	% langue : français
\usepackage{fourier}			% caractères plus lisibles
\usepackage[dvipsnames]{xcolor} % couleurs
\usepackage{fancyhdr}		% réglage header footer
\usepackage{needspace}		% empêcher sauts de page mal placés
\usepackage{graphicx}		% pour inclure des graphiques
\usepackage{enumitem,cprotect}		% personnalise les listes d'items (nécessaire pour ol, al ...)
\usepackage{hyperref}		% Liens hypertexte
\usepackage{pstricks,pst-all,pst-node,pstricks-add,pst-math,pst-plot,pst-tree,pst-eucl} % pstricks
\usepackage[a4paper,includeheadfoot,top=2cm,left=3cm, bottom=2cm,right=3cm]{geometry} % marges etc.
\usepackage{comment}			% commentaires multilignes
\usepackage{amsmath,environ} % maths (matrices, etc.)
\usepackage{amssymb,makeidx}
\usepackage{bm}				% bold maths
\usepackage{tabularx}		% tableaux
\usepackage{colortbl}		% tableaux en couleur
\usepackage{fontawesome}		% Fontawesome
\usepackage{environ}			% environment with command
\usepackage{fp}				% calculs pour ps-tricks
\usepackage{multido}			% pour ps tricks
\usepackage[np]{numprint}	% formattage nombre
\usepackage{tikz,tkz-tab} 			% package principal TikZ
\usepackage{pgfplots}   % axes
\usepackage{mathrsfs}    % cursives
\usepackage{calc}			% calcul taille boites
\usepackage[scaled=0.875]{helvet} % font sans serif
\usepackage{svg} % svg
\usepackage{scrextend} % local margin
\usepackage{scratch} %scratch
\usepackage{multicol} % colonnes
%\usepackage{infix-RPN,pst-func} % formule en notation polanaise inversée
\usepackage{listings}

%================================================================================================================================
%
% Réglages de base
%
%================================================================================================================================

\lstset{
language=Python,   % R code
literate=
{á}{{\'a}}1
{à}{{\`a}}1
{ã}{{\~a}}1
{é}{{\'e}}1
{è}{{\`e}}1
{ê}{{\^e}}1
{í}{{\'i}}1
{ó}{{\'o}}1
{õ}{{\~o}}1
{ú}{{\'u}}1
{ü}{{\"u}}1
{ç}{{\c{c}}}1
{~}{{ }}1
}


\definecolor{codegreen}{rgb}{0,0.6,0}
\definecolor{codegray}{rgb}{0.5,0.5,0.5}
\definecolor{codepurple}{rgb}{0.58,0,0.82}
\definecolor{backcolour}{rgb}{0.95,0.95,0.92}

\lstdefinestyle{mystyle}{
    backgroundcolor=\color{backcolour},   
    commentstyle=\color{codegreen},
    keywordstyle=\color{magenta},
    numberstyle=\tiny\color{codegray},
    stringstyle=\color{codepurple},
    basicstyle=\ttfamily\footnotesize,
    breakatwhitespace=false,         
    breaklines=true,                 
    captionpos=b,                    
    keepspaces=true,                 
    numbers=left,                    
xleftmargin=2em,
framexleftmargin=2em,            
    showspaces=false,                
    showstringspaces=false,
    showtabs=false,                  
    tabsize=2,
    upquote=true
}

\lstset{style=mystyle}


\lstset{style=mystyle}
\newcommand{\imgdir}{C:/laragon/www/newmc/assets/imgsvg/}
\newcommand{\imgsvgdir}{C:/laragon/www/newmc/assets/imgsvg/}

\definecolor{mcgris}{RGB}{220, 220, 220}% ancien~; pour compatibilité
\definecolor{mcbleu}{RGB}{52, 152, 219}
\definecolor{mcvert}{RGB}{125, 194, 70}
\definecolor{mcmauve}{RGB}{154, 0, 215}
\definecolor{mcorange}{RGB}{255, 96, 0}
\definecolor{mcturquoise}{RGB}{0, 153, 153}
\definecolor{mcrouge}{RGB}{255, 0, 0}
\definecolor{mclightvert}{RGB}{205, 234, 190}

\definecolor{gris}{RGB}{220, 220, 220}
\definecolor{bleu}{RGB}{52, 152, 219}
\definecolor{vert}{RGB}{125, 194, 70}
\definecolor{mauve}{RGB}{154, 0, 215}
\definecolor{orange}{RGB}{255, 96, 0}
\definecolor{turquoise}{RGB}{0, 153, 153}
\definecolor{rouge}{RGB}{255, 0, 0}
\definecolor{lightvert}{RGB}{205, 234, 190}
\setitemize[0]{label=\color{lightvert}  $\bullet$}

\pagestyle{fancy}
\renewcommand{\headrulewidth}{0.2pt}
\fancyhead[L]{maths-cours.fr}
\fancyhead[R]{\thepage}
\renewcommand{\footrulewidth}{0.2pt}
\fancyfoot[C]{}

\newcolumntype{C}{>{\centering\arraybackslash}X}
\newcolumntype{s}{>{\hsize=.35\hsize\arraybackslash}X}

\setlength{\parindent}{0pt}		 
\setlength{\parskip}{3mm}
\setlength{\headheight}{1cm}

\def\ebook{ebook}
\def\book{book}
\def\web{web}
\def\type{web}

\newcommand{\vect}[1]{\overrightarrow{\,\mathstrut#1\,}}

\def\Oij{$\left(\text{O}~;~\vect{\imath},~\vect{\jmath}\right)$}
\def\Oijk{$\left(\text{O}~;~\vect{\imath},~\vect{\jmath},~\vect{k}\right)$}
\def\Ouv{$\left(\text{O}~;~\vect{u},~\vect{v}\right)$}

\hypersetup{breaklinks=true, colorlinks = true, linkcolor = OliveGreen, urlcolor = OliveGreen, citecolor = OliveGreen, pdfauthor={Didier BONNEL - https://www.maths-cours.fr} } % supprime les bordures autour des liens

\renewcommand{\arg}[0]{\text{arg}}

\everymath{\displaystyle}

%================================================================================================================================
%
% Macros - Commandes
%
%================================================================================================================================

\newcommand\meta[2]{    			% Utilisé pour créer le post HTML.
	\def\titre{titre}
	\def\url{url}
	\def\arg{#1}
	\ifx\titre\arg
		\newcommand\maintitle{#2}
		\fancyhead[L]{#2}
		{\Large\sffamily \MakeUppercase{#2}}
		\vspace{1mm}\textcolor{mcvert}{\hrule}
	\fi 
	\ifx\url\arg
		\fancyfoot[L]{\href{https://www.maths-cours.fr#2}{\black \footnotesize{https://www.maths-cours.fr#2}}}
	\fi 
}


\newcommand\TitreC[1]{    		% Titre centré
     \needspace{3\baselineskip}
     \begin{center}\textbf{#1}\end{center}
}

\newcommand\newpar{    		% paragraphe
     \par
}

\newcommand\nosp {    		% commande vide (pas d'espace)
}
\newcommand{\id}[1]{} %ignore

\newcommand\boite[2]{				% Boite simple sans titre
	\vspace{5mm}
	\setlength{\fboxrule}{0.2mm}
	\setlength{\fboxsep}{5mm}	
	\fcolorbox{#1}{#1!3}{\makebox[\linewidth-2\fboxrule-2\fboxsep]{
  		\begin{minipage}[t]{\linewidth-2\fboxrule-4\fboxsep}\setlength{\parskip}{3mm}
  			 #2
  		\end{minipage}
	}}
	\vspace{5mm}
}

\newcommand\CBox[4]{				% Boites
	\vspace{5mm}
	\setlength{\fboxrule}{0.2mm}
	\setlength{\fboxsep}{5mm}
	
	\fcolorbox{#1}{#1!3}{\makebox[\linewidth-2\fboxrule-2\fboxsep]{
		\begin{minipage}[t]{1cm}\setlength{\parskip}{3mm}
	  		\textcolor{#1}{\LARGE{#2}}    
 	 	\end{minipage}  
  		\begin{minipage}[t]{\linewidth-2\fboxrule-4\fboxsep}\setlength{\parskip}{3mm}
			\raisebox{1.2mm}{\normalsize\sffamily{\textcolor{#1}{#3}}}						
  			 #4
  		\end{minipage}
	}}
	\vspace{5mm}
}

\newcommand\cadre[3]{				% Boites convertible html
	\par
	\vspace{2mm}
	\setlength{\fboxrule}{0.1mm}
	\setlength{\fboxsep}{5mm}
	\fcolorbox{#1}{white}{\makebox[\linewidth-2\fboxrule-2\fboxsep]{
  		\begin{minipage}[t]{\linewidth-2\fboxrule-4\fboxsep}\setlength{\parskip}{3mm}
			\raisebox{-2.5mm}{\sffamily \small{\textcolor{#1}{\MakeUppercase{#2}}}}		
			\par		
  			 #3
 	 		\end{minipage}
	}}
		\vspace{2mm}
	\par
}

\newcommand\bloc[3]{				% Boites convertible html sans bordure
     \needspace{2\baselineskip}
     {\sffamily \small{\textcolor{#1}{\MakeUppercase{#2}}}}    
		\par		
  			 #3
		\par
}

\newcommand\CHelp[1]{
     \CBox{Plum}{\faInfoCircle}{À RETENIR}{#1}
}

\newcommand\CUp[1]{
     \CBox{NavyBlue}{\faThumbsOUp}{EN PRATIQUE}{#1}
}

\newcommand\CInfo[1]{
     \CBox{Sepia}{\faArrowCircleRight}{REMARQUE}{#1}
}

\newcommand\CRedac[1]{
     \CBox{PineGreen}{\faEdit}{BIEN R\'EDIGER}{#1}
}

\newcommand\CError[1]{
     \CBox{Red}{\faExclamationTriangle}{ATTENTION}{#1}
}

\newcommand\TitreExo[2]{
\needspace{4\baselineskip}
 {\sffamily\large EXERCICE #1\ (\emph{#2 points})}
\vspace{5mm}
}

\newcommand\img[2]{
          \includegraphics[width=#2\paperwidth]{\imgdir#1}
}

\newcommand\imgsvg[2]{
       \begin{center}   \includegraphics[width=#2\paperwidth]{\imgsvgdir#1} \end{center}
}


\newcommand\Lien[2]{
     \href{#1}{#2 \tiny \faExternalLink}
}
\newcommand\mcLien[2]{
     \href{https~://www.maths-cours.fr/#1}{#2 \tiny \faExternalLink}
}

\newcommand{\euro}{\eurologo{}}

%================================================================================================================================
%
% Macros - Environement
%
%================================================================================================================================

\newenvironment{tex}{ %
}
{%
}

\newenvironment{indente}{ %
	\setlength\parindent{10mm}
}

{
	\setlength\parindent{0mm}
}

\newenvironment{corrige}{%
     \needspace{3\baselineskip}
     \medskip
     \textbf{\textsc{Corrigé}}
     \medskip
}
{
}

\newenvironment{extern}{%
     \begin{center}
     }
     {
     \end{center}
}

\NewEnviron{code}{%
	\par
     \boite{gray}{\texttt{%
     \BODY
     }}
     \par
}

\newenvironment{vbloc}{% boite sans cadre empeche saut de page
     \begin{minipage}[t]{\linewidth}
     }
     {
     \end{minipage}
}
\NewEnviron{h2}{%
    \needspace{3\baselineskip}
    \vspace{0.6cm}
	\noindent \MakeUppercase{\sffamily \large \BODY}
	\vspace{1mm}\textcolor{mcgris}{\hrule}\vspace{0.4cm}
	\par
}{}

\NewEnviron{h3}{%
    \needspace{3\baselineskip}
	\vspace{5mm}
	\textsc{\BODY}
	\par
}

\NewEnviron{margeneg}{ %
\begin{addmargin}[-1cm]{0cm}
\BODY
\end{addmargin}
}

\NewEnviron{html}{%
}

\begin{document}
\meta{url}{/cours/python-au-lycee-5-les-listes/}
\meta{pid}{11667}
\meta{titre}{Python au lycée (5)~: Les listes}
\meta{type}{cours}
%
\begin{h2}1. Le type liste\end{h2}
\begin{h3} Définition d'une liste en Python\end{h3}
Une \textit{liste} est un type de valeur qui contient une suite ordonnée de valeurs.
\par
On dit qu'une liste est définie \textit{en extension} lorsque l'on énumère toutes les valeurs de la liste. Une liste est encadrée par des \textbf{crochets} et les différentes valeurs de la liste sont séparées par des virgules~; par exemple~:
\begin{lstlisting}[language=Python]
liste_de_nombres = [ 7, 12, 1.5, 9, 43]
liste_de_couleurs = [ "rouge", "vert", "bleu", "jaune", "orange" ]
\end{lstlisting}
Il est possible, en Python, de définir des listes comportant des valeurs de types différents (nombres, chaîne de caractères où même des sous-listes...)~:
 \begin{lstlisting}[language=Python]
liste_de_types_differents = [ 3, "Bonjour", 2.5, ["a", "b"], true ]
\end{lstlisting}
Il est également courant de définir une liste vide que l'on remplira par la suite (grâce à l'instruction \texttt{append} que l'on détaillera ultérieurement)~:
 \begin{lstlisting}[language=Python]
liste_vide = []
\end{lstlisting}
\begin{h3}Accès à un élément\end{h3}
Il est possible d'accéder à un élément d'une liste grâce à sa position appelée \textit{indice}.
\par
\textbf{Attention~: } Le premier élément d'une liste correspond à l'indice 0 et si la liste contient \texttt{n} éléments, l'indice du dernier élément est \texttt{n-1}~:
\begin{lstlisting}[language=Python]
liste = [ "un", "deux", "trois", "quatre"]
# indices : 0       1        2         3
\end{lstlisting}
Pour accéder à un élément d'une liste on utilise la syntaxe suivante~: \texttt{nom_de_la_liste[indice]}, par exemple~:
\begin{lstlisting}[language=Python]
liste = [ "rouge", "vert", "bleu", "jaune", "orange" ]
print(liste[0]) # affiche rouge
print(liste[4]) # affiche orange
liste[2] = "violet" # modifie le 3ème élément de la liste
print(liste) # affiche ['rouge', 'vert', 'violet', 'jaune', 'orange']
\end{lstlisting}
\begin{h3} Liste définie en compréhension \end{h3}
En mathématiques, il est possible de définir un ensemble en \textit{compréhension }. Par exemple, l'ensemble~:
\begin{center}
     $ E = \left\{ n \in \mathbb{N} ~|~ 5 \leqslant n < 10 \right\} $
\end{center}
représente l'ensemble des entiers naturels supérieurs ou égaux à $5$ et strictement inférieurs à $10 $, c'est à dire l'ensemble~:
\begin{center}
     $ E = \left\{ 5~;~ 6~;~ 7~;~ 8~;~ 9 \right\} $
\end{center}
En Python, il est également possible de définir une liste en \textit{compréhension } en utilisant une boucle \texttt{for} à l'intérieur de la définition de la liste~:
\begin{lstlisting}[language=Python]
liste = [ n for n in range(5, 10) ]
# équivaut à liste = [ 5, 6, 7, 8, 9 ]
\end{lstlisting}
Voici deux exemples plus avancés de listes définies en compréhension~:
\begin{lstlisting}[language=Python]
liste1 = [ n**2 for n in range(6) ]
# liste1 contient [ 0, 1, 4, 9, 16, 25 ]
liste2 = [ n for n in range(2, 7) if n!=4 ]
# liste2 contient [2, 3, 5, 6]
\end{lstlisting}
\begin{h2} 2. Opérations sur les listes \end{h2}
\begin{h3} La méthode \texttt{.append()} \end{h3}
En programmation, une \textit{ méthode} est une fonction qui agit sur un certain objet.
\par
La méthode \texttt{.append()} permet d'ajouter un élément à la fin d'une liste, par exemple~:
\begin{lstlisting}[language=Python]
mes_notes = [12, 14, 9, 16]
mes_notes.append(13)
print(mes_notes) # affiche [12, 14, 9, 16, 13]
\end{lstlisting}
Il est fréquent, lorsque l'on souhaite créer une liste pas à pas, de partir d'une liste vide et d'ajouter les éléments un par un. Par exemple, le programme suivant crée une liste en comptant de 3 en 3 en partant de 1 jusqu'à 16~:
\begin{lstlisting}[language=Python]
ma_liste = []
n = 1
while n <= 16 :
   ma_liste.append(n)
   n = n+3
print(ma_liste) # affiche [1, 4, 7, 10, 13, 16]
\end{lstlisting}
\begin{h3} Concaténation de listes \end{h3}
Comme pour les chaînes de caractères, l'opérateur \og + \fg{} permet de concaténer des listes~:
\begin{lstlisting}[language=Python]
couleurs1 = [ "rouge", "jaune", "vert"]
couleurs2 = ["orange", "bleu"]
couleurs = couleurs1 + couleurs2
print(couleurs) # affiche ['rouge', 'jaune', 'vert', 'orange', 'bleu']
\end{lstlisting}
\begin{h3} Longueur d'une liste \end{h3}
La fonction \texttt{len()} retourne la longueur de la liste passée en paramètre.
\par
Cette fonction peut être utilisée lorsque l'on souhaite parcourir les éléments d'une liste pour déterminer la fin de la boucle. Par exemple, le programme ci-dessous affiche les éléments de la liste séparés par des points-virgules~:
\begin{lstlisting}[language=Python]
liste = ["a", "b", "c", "d"]
for i in range(len(liste)) :
   print (liste[i], end=";") # affiche a;b;c;d;
\end{lstlisting}

\end{document}
µ
\documentclass[a4paper]{article}

%================================================================================================================================
%
% Packages
%
%================================================================================================================================

\usepackage[T1]{fontenc} 	% pour caractères accentués
\usepackage[utf8]{inputenc}  % encodage utf8
\usepackage[french]{babel}	% langue : français
\usepackage{fourier}			% caractères plus lisibles
\usepackage[dvipsnames]{xcolor} % couleurs
\usepackage{fancyhdr}		% réglage header footer
\usepackage{needspace}		% empêcher sauts de page mal placés
\usepackage{graphicx}		% pour inclure des graphiques
\usepackage{enumitem,cprotect}		% personnalise les listes d'items (nécessaire pour ol, al ...)
\usepackage{hyperref}		% Liens hypertexte
\usepackage{pstricks,pst-all,pst-node,pstricks-add,pst-math,pst-plot,pst-tree,pst-eucl} % pstricks
\usepackage[a4paper,includeheadfoot,top=2cm,left=3cm, bottom=2cm,right=3cm]{geometry} % marges etc.
\usepackage{comment}			% commentaires multilignes
\usepackage{amsmath,environ} % maths (matrices, etc.)
\usepackage{amssymb,makeidx}
\usepackage{bm}				% bold maths
\usepackage{tabularx}		% tableaux
\usepackage{colortbl}		% tableaux en couleur
\usepackage{fontawesome}		% Fontawesome
\usepackage{environ}			% environment with command
\usepackage{fp}				% calculs pour ps-tricks
\usepackage{multido}			% pour ps tricks
\usepackage[np]{numprint}	% formattage nombre
\usepackage{tikz,tkz-tab} 			% package principal TikZ
\usepackage{pgfplots}   % axes
\usepackage{mathrsfs}    % cursives
\usepackage{calc}			% calcul taille boites
\usepackage[scaled=0.875]{helvet} % font sans serif
\usepackage{svg} % svg
\usepackage{scrextend} % local margin
\usepackage{scratch} %scratch
\usepackage{multicol} % colonnes
%\usepackage{infix-RPN,pst-func} % formule en notation polanaise inversée
\usepackage{listings}

%================================================================================================================================
%
% Réglages de base
%
%================================================================================================================================

\lstset{
language=Python,   % R code
literate=
{á}{{\'a}}1
{à}{{\`a}}1
{ã}{{\~a}}1
{é}{{\'e}}1
{è}{{\`e}}1
{ê}{{\^e}}1
{í}{{\'i}}1
{ó}{{\'o}}1
{õ}{{\~o}}1
{ú}{{\'u}}1
{ü}{{\"u}}1
{ç}{{\c{c}}}1
{~}{{ }}1
}


\definecolor{codegreen}{rgb}{0,0.6,0}
\definecolor{codegray}{rgb}{0.5,0.5,0.5}
\definecolor{codepurple}{rgb}{0.58,0,0.82}
\definecolor{backcolour}{rgb}{0.95,0.95,0.92}

\lstdefinestyle{mystyle}{
    backgroundcolor=\color{backcolour},   
    commentstyle=\color{codegreen},
    keywordstyle=\color{magenta},
    numberstyle=\tiny\color{codegray},
    stringstyle=\color{codepurple},
    basicstyle=\ttfamily\footnotesize,
    breakatwhitespace=false,         
    breaklines=true,                 
    captionpos=b,                    
    keepspaces=true,                 
    numbers=left,                    
xleftmargin=2em,
framexleftmargin=2em,            
    showspaces=false,                
    showstringspaces=false,
    showtabs=false,                  
    tabsize=2,
    upquote=true
}

\lstset{style=mystyle}


\lstset{style=mystyle}
\newcommand{\imgdir}{C:/laragon/www/newmc/assets/imgsvg/}
\newcommand{\imgsvgdir}{C:/laragon/www/newmc/assets/imgsvg/}

\definecolor{mcgris}{RGB}{220, 220, 220}% ancien~; pour compatibilité
\definecolor{mcbleu}{RGB}{52, 152, 219}
\definecolor{mcvert}{RGB}{125, 194, 70}
\definecolor{mcmauve}{RGB}{154, 0, 215}
\definecolor{mcorange}{RGB}{255, 96, 0}
\definecolor{mcturquoise}{RGB}{0, 153, 153}
\definecolor{mcrouge}{RGB}{255, 0, 0}
\definecolor{mclightvert}{RGB}{205, 234, 190}

\definecolor{gris}{RGB}{220, 220, 220}
\definecolor{bleu}{RGB}{52, 152, 219}
\definecolor{vert}{RGB}{125, 194, 70}
\definecolor{mauve}{RGB}{154, 0, 215}
\definecolor{orange}{RGB}{255, 96, 0}
\definecolor{turquoise}{RGB}{0, 153, 153}
\definecolor{rouge}{RGB}{255, 0, 0}
\definecolor{lightvert}{RGB}{205, 234, 190}
\setitemize[0]{label=\color{lightvert}  $\bullet$}

\pagestyle{fancy}
\renewcommand{\headrulewidth}{0.2pt}
\fancyhead[L]{maths-cours.fr}
\fancyhead[R]{\thepage}
\renewcommand{\footrulewidth}{0.2pt}
\fancyfoot[C]{}

\newcolumntype{C}{>{\centering\arraybackslash}X}
\newcolumntype{s}{>{\hsize=.35\hsize\arraybackslash}X}

\setlength{\parindent}{0pt}		 
\setlength{\parskip}{3mm}
\setlength{\headheight}{1cm}

\def\ebook{ebook}
\def\book{book}
\def\web{web}
\def\type{web}

\newcommand{\vect}[1]{\overrightarrow{\,\mathstrut#1\,}}

\def\Oij{$\left(\text{O}~;~\vect{\imath},~\vect{\jmath}\right)$}
\def\Oijk{$\left(\text{O}~;~\vect{\imath},~\vect{\jmath},~\vect{k}\right)$}
\def\Ouv{$\left(\text{O}~;~\vect{u},~\vect{v}\right)$}

\hypersetup{breaklinks=true, colorlinks = true, linkcolor = OliveGreen, urlcolor = OliveGreen, citecolor = OliveGreen, pdfauthor={Didier BONNEL - https://www.maths-cours.fr} } % supprime les bordures autour des liens

\renewcommand{\arg}[0]{\text{arg}}

\everymath{\displaystyle}

%================================================================================================================================
%
% Macros - Commandes
%
%================================================================================================================================

\newcommand\meta[2]{    			% Utilisé pour créer le post HTML.
	\def\titre{titre}
	\def\url{url}
	\def\arg{#1}
	\ifx\titre\arg
		\newcommand\maintitle{#2}
		\fancyhead[L]{#2}
		{\Large\sffamily \MakeUppercase{#2}}
		\vspace{1mm}\textcolor{mcvert}{\hrule}
	\fi 
	\ifx\url\arg
		\fancyfoot[L]{\href{https://www.maths-cours.fr#2}{\black \footnotesize{https://www.maths-cours.fr#2}}}
	\fi 
}


\newcommand\TitreC[1]{    		% Titre centré
     \needspace{3\baselineskip}
     \begin{center}\textbf{#1}\end{center}
}

\newcommand\newpar{    		% paragraphe
     \par
}

\newcommand\nosp {    		% commande vide (pas d'espace)
}
\newcommand{\id}[1]{} %ignore

\newcommand\boite[2]{				% Boite simple sans titre
	\vspace{5mm}
	\setlength{\fboxrule}{0.2mm}
	\setlength{\fboxsep}{5mm}	
	\fcolorbox{#1}{#1!3}{\makebox[\linewidth-2\fboxrule-2\fboxsep]{
  		\begin{minipage}[t]{\linewidth-2\fboxrule-4\fboxsep}\setlength{\parskip}{3mm}
  			 #2
  		\end{minipage}
	}}
	\vspace{5mm}
}

\newcommand\CBox[4]{				% Boites
	\vspace{5mm}
	\setlength{\fboxrule}{0.2mm}
	\setlength{\fboxsep}{5mm}
	
	\fcolorbox{#1}{#1!3}{\makebox[\linewidth-2\fboxrule-2\fboxsep]{
		\begin{minipage}[t]{1cm}\setlength{\parskip}{3mm}
	  		\textcolor{#1}{\LARGE{#2}}    
 	 	\end{minipage}  
  		\begin{minipage}[t]{\linewidth-2\fboxrule-4\fboxsep}\setlength{\parskip}{3mm}
			\raisebox{1.2mm}{\normalsize\sffamily{\textcolor{#1}{#3}}}						
  			 #4
  		\end{minipage}
	}}
	\vspace{5mm}
}

\newcommand\cadre[3]{				% Boites convertible html
	\par
	\vspace{2mm}
	\setlength{\fboxrule}{0.1mm}
	\setlength{\fboxsep}{5mm}
	\fcolorbox{#1}{white}{\makebox[\linewidth-2\fboxrule-2\fboxsep]{
  		\begin{minipage}[t]{\linewidth-2\fboxrule-4\fboxsep}\setlength{\parskip}{3mm}
			\raisebox{-2.5mm}{\sffamily \small{\textcolor{#1}{\MakeUppercase{#2}}}}		
			\par		
  			 #3
 	 		\end{minipage}
	}}
		\vspace{2mm}
	\par
}

\newcommand\bloc[3]{				% Boites convertible html sans bordure
     \needspace{2\baselineskip}
     {\sffamily \small{\textcolor{#1}{\MakeUppercase{#2}}}}    
		\par		
  			 #3
		\par
}

\newcommand\CHelp[1]{
     \CBox{Plum}{\faInfoCircle}{À RETENIR}{#1}
}

\newcommand\CUp[1]{
     \CBox{NavyBlue}{\faThumbsOUp}{EN PRATIQUE}{#1}
}

\newcommand\CInfo[1]{
     \CBox{Sepia}{\faArrowCircleRight}{REMARQUE}{#1}
}

\newcommand\CRedac[1]{
     \CBox{PineGreen}{\faEdit}{BIEN R\'EDIGER}{#1}
}

\newcommand\CError[1]{
     \CBox{Red}{\faExclamationTriangle}{ATTENTION}{#1}
}

\newcommand\TitreExo[2]{
\needspace{4\baselineskip}
 {\sffamily\large EXERCICE #1\ (\emph{#2 points})}
\vspace{5mm}
}

\newcommand\img[2]{
          \includegraphics[width=#2\paperwidth]{\imgdir#1}
}

\newcommand\imgsvg[2]{
       \begin{center}   \includegraphics[width=#2\paperwidth]{\imgsvgdir#1} \end{center}
}


\newcommand\Lien[2]{
     \href{#1}{#2 \tiny \faExternalLink}
}
\newcommand\mcLien[2]{
     \href{https~://www.maths-cours.fr/#1}{#2 \tiny \faExternalLink}
}

\newcommand{\euro}{\eurologo{}}

%================================================================================================================================
%
% Macros - Environement
%
%================================================================================================================================

\newenvironment{tex}{ %
}
{%
}

\newenvironment{indente}{ %
	\setlength\parindent{10mm}
}

{
	\setlength\parindent{0mm}
}

\newenvironment{corrige}{%
     \needspace{3\baselineskip}
     \medskip
     \textbf{\textsc{Corrigé}}
     \medskip
}
{
}

\newenvironment{extern}{%
     \begin{center}
     }
     {
     \end{center}
}

\NewEnviron{code}{%
	\par
     \boite{gray}{\texttt{%
     \BODY
     }}
     \par
}

\newenvironment{vbloc}{% boite sans cadre empeche saut de page
     \begin{minipage}[t]{\linewidth}
     }
     {
     \end{minipage}
}
\NewEnviron{h2}{%
    \needspace{3\baselineskip}
    \vspace{0.6cm}
	\noindent \MakeUppercase{\sffamily \large \BODY}
	\vspace{1mm}\textcolor{mcgris}{\hrule}\vspace{0.4cm}
	\par
}{}

\NewEnviron{h3}{%
    \needspace{3\baselineskip}
	\vspace{5mm}
	\textsc{\BODY}
	\par
}

\NewEnviron{margeneg}{ %
\begin{addmargin}[-1cm]{0cm}
\BODY
\end{addmargin}
}

\NewEnviron{html}{%
}

\begin{document}
\meta{url}{/exercices/liste-definie-en-comprehension/}
\meta{pid}{11674}
\meta{titre}{Liste définie en compréhension}
\meta{type}{exercices}
%
On souhaite écrire une fonction nommée \texttt{fini_par_1} qui prend deux arguments entiers naturels \texttt{a} et \texttt{b} et qui retourne la liste des entiers compris, au sens large, entre \texttt{a} et \texttt{b} dont le chiffre des unités est égal à 1.
\par
Par exemple, on souhaite que \texttt{fini_par_1(0, 21)} retourne la liste \texttt{[1, 11, 21]}
\par
Il faut que cette fonction ne contienne qu'une unique commande (qui peut toutefois faire appel à plusieurs instructions) suivant le modèle ci-dessous~:
\begin{lstlisting}[language=Python]
def fini_par_1(a, b) :
   return ...
\end{lstlisting}
\medskip
\textit{Indications~: }
\begin{itemize}
     \item
     On utilisera une liste définie en compréhension.
     \item
     On rappelle que l'instruction \texttt{ x \% y } renvoie le reste de la division euclidienne de \texttt{x} par \texttt{y}.
\end{itemize}
\medskip
\textit{Test~: }
\par
Pour vérifier votre réponse, testez votre fonction avec les commandes~:
\begin{lstlisting}[language=Python]
>>> fini_par_1(0,1)
>>> fini_par_1(1,11)
>>> fini_par_1(10,50)
>>> fini_par_1(10,51)
>>> fini_par_1(100,50)
\end{lstlisting}
\begin{corrige}
     Voici une fonction possible~:
     \par
\begin{lstlisting}[language=Python]
def fini_par_1(a, b) :
   return [x for x in range(a,b+1) if x%10==1]
\end{lstlisting}
\medskip
\textit{Explications~: }
\begin{itemize}
     \item
     \texttt{ range(a,b+1) } retourne les entiers naturels compris entre \texttt{a} et \texttt{b}. On filtre ensuite cette liste à l'aide d'une instruction \texttt{if}.
     \item
     \texttt{ x \%10 } retourne le reste de la division euclidienne de \texttt{x} par \texttt{10}, c'est à dire le dernier chiffre de \texttt{x}. On teste alors si ce chiffre est égal à 1.
\end{itemize}
\medskip
\textit{Test~: }
\par
Voici les listes renvoyées par cette fonction pour les exemples donnés dans l'énoncé~:
\begin{lstlisting}[language=Python]
>>> fini_par_1(0,1)
[1]
>>> fini_par_1(1,11)
[1, 11]
>>> fini_par_1(10,50)
[11, 21, 31, 41]
>>> fini_par_1(10,51)
[11, 21, 31, 41, 51]
>>> fini_par_1(100,50)
[] # car 100 > 50
\end{lstlisting}
\end{corrige}

\end{document}
µ
\documentclass[a4paper]{article}

%================================================================================================================================
%
% Packages
%
%================================================================================================================================

\usepackage[T1]{fontenc} 	% pour caractères accentués
\usepackage[utf8]{inputenc}  % encodage utf8
\usepackage[french]{babel}	% langue : français
\usepackage{fourier}			% caractères plus lisibles
\usepackage[dvipsnames]{xcolor} % couleurs
\usepackage{fancyhdr}		% réglage header footer
\usepackage{needspace}		% empêcher sauts de page mal placés
\usepackage{graphicx}		% pour inclure des graphiques
\usepackage{enumitem,cprotect}		% personnalise les listes d'items (nécessaire pour ol, al ...)
\usepackage{hyperref}		% Liens hypertexte
\usepackage{pstricks,pst-all,pst-node,pstricks-add,pst-math,pst-plot,pst-tree,pst-eucl} % pstricks
\usepackage[a4paper,includeheadfoot,top=2cm,left=3cm, bottom=2cm,right=3cm]{geometry} % marges etc.
\usepackage{comment}			% commentaires multilignes
\usepackage{amsmath,environ} % maths (matrices, etc.)
\usepackage{amssymb,makeidx}
\usepackage{bm}				% bold maths
\usepackage{tabularx}		% tableaux
\usepackage{colortbl}		% tableaux en couleur
\usepackage{fontawesome}		% Fontawesome
\usepackage{environ}			% environment with command
\usepackage{fp}				% calculs pour ps-tricks
\usepackage{multido}			% pour ps tricks
\usepackage[np]{numprint}	% formattage nombre
\usepackage{tikz,tkz-tab} 			% package principal TikZ
\usepackage{pgfplots}   % axes
\usepackage{mathrsfs}    % cursives
\usepackage{calc}			% calcul taille boites
\usepackage[scaled=0.875]{helvet} % font sans serif
\usepackage{svg} % svg
\usepackage{scrextend} % local margin
\usepackage{scratch} %scratch
\usepackage{multicol} % colonnes
%\usepackage{infix-RPN,pst-func} % formule en notation polanaise inversée
\usepackage{listings}

%================================================================================================================================
%
% Réglages de base
%
%================================================================================================================================

\lstset{
language=Python,   % R code
literate=
{á}{{\'a}}1
{à}{{\`a}}1
{ã}{{\~a}}1
{é}{{\'e}}1
{è}{{\`e}}1
{ê}{{\^e}}1
{í}{{\'i}}1
{ó}{{\'o}}1
{õ}{{\~o}}1
{ú}{{\'u}}1
{ü}{{\"u}}1
{ç}{{\c{c}}}1
{~}{{ }}1
}


\definecolor{codegreen}{rgb}{0,0.6,0}
\definecolor{codegray}{rgb}{0.5,0.5,0.5}
\definecolor{codepurple}{rgb}{0.58,0,0.82}
\definecolor{backcolour}{rgb}{0.95,0.95,0.92}

\lstdefinestyle{mystyle}{
    backgroundcolor=\color{backcolour},   
    commentstyle=\color{codegreen},
    keywordstyle=\color{magenta},
    numberstyle=\tiny\color{codegray},
    stringstyle=\color{codepurple},
    basicstyle=\ttfamily\footnotesize,
    breakatwhitespace=false,         
    breaklines=true,                 
    captionpos=b,                    
    keepspaces=true,                 
    numbers=left,                    
xleftmargin=2em,
framexleftmargin=2em,            
    showspaces=false,                
    showstringspaces=false,
    showtabs=false,                  
    tabsize=2,
    upquote=true
}

\lstset{style=mystyle}


\lstset{style=mystyle}
\newcommand{\imgdir}{C:/laragon/www/newmc/assets/imgsvg/}
\newcommand{\imgsvgdir}{C:/laragon/www/newmc/assets/imgsvg/}

\definecolor{mcgris}{RGB}{220, 220, 220}% ancien~; pour compatibilité
\definecolor{mcbleu}{RGB}{52, 152, 219}
\definecolor{mcvert}{RGB}{125, 194, 70}
\definecolor{mcmauve}{RGB}{154, 0, 215}
\definecolor{mcorange}{RGB}{255, 96, 0}
\definecolor{mcturquoise}{RGB}{0, 153, 153}
\definecolor{mcrouge}{RGB}{255, 0, 0}
\definecolor{mclightvert}{RGB}{205, 234, 190}

\definecolor{gris}{RGB}{220, 220, 220}
\definecolor{bleu}{RGB}{52, 152, 219}
\definecolor{vert}{RGB}{125, 194, 70}
\definecolor{mauve}{RGB}{154, 0, 215}
\definecolor{orange}{RGB}{255, 96, 0}
\definecolor{turquoise}{RGB}{0, 153, 153}
\definecolor{rouge}{RGB}{255, 0, 0}
\definecolor{lightvert}{RGB}{205, 234, 190}
\setitemize[0]{label=\color{lightvert}  $\bullet$}

\pagestyle{fancy}
\renewcommand{\headrulewidth}{0.2pt}
\fancyhead[L]{maths-cours.fr}
\fancyhead[R]{\thepage}
\renewcommand{\footrulewidth}{0.2pt}
\fancyfoot[C]{}

\newcolumntype{C}{>{\centering\arraybackslash}X}
\newcolumntype{s}{>{\hsize=.35\hsize\arraybackslash}X}

\setlength{\parindent}{0pt}		 
\setlength{\parskip}{3mm}
\setlength{\headheight}{1cm}

\def\ebook{ebook}
\def\book{book}
\def\web{web}
\def\type{web}

\newcommand{\vect}[1]{\overrightarrow{\,\mathstrut#1\,}}

\def\Oij{$\left(\text{O}~;~\vect{\imath},~\vect{\jmath}\right)$}
\def\Oijk{$\left(\text{O}~;~\vect{\imath},~\vect{\jmath},~\vect{k}\right)$}
\def\Ouv{$\left(\text{O}~;~\vect{u},~\vect{v}\right)$}

\hypersetup{breaklinks=true, colorlinks = true, linkcolor = OliveGreen, urlcolor = OliveGreen, citecolor = OliveGreen, pdfauthor={Didier BONNEL - https://www.maths-cours.fr} } % supprime les bordures autour des liens

\renewcommand{\arg}[0]{\text{arg}}

\everymath{\displaystyle}

%================================================================================================================================
%
% Macros - Commandes
%
%================================================================================================================================

\newcommand\meta[2]{    			% Utilisé pour créer le post HTML.
	\def\titre{titre}
	\def\url{url}
	\def\arg{#1}
	\ifx\titre\arg
		\newcommand\maintitle{#2}
		\fancyhead[L]{#2}
		{\Large\sffamily \MakeUppercase{#2}}
		\vspace{1mm}\textcolor{mcvert}{\hrule}
	\fi 
	\ifx\url\arg
		\fancyfoot[L]{\href{https://www.maths-cours.fr#2}{\black \footnotesize{https://www.maths-cours.fr#2}}}
	\fi 
}


\newcommand\TitreC[1]{    		% Titre centré
     \needspace{3\baselineskip}
     \begin{center}\textbf{#1}\end{center}
}

\newcommand\newpar{    		% paragraphe
     \par
}

\newcommand\nosp {    		% commande vide (pas d'espace)
}
\newcommand{\id}[1]{} %ignore

\newcommand\boite[2]{				% Boite simple sans titre
	\vspace{5mm}
	\setlength{\fboxrule}{0.2mm}
	\setlength{\fboxsep}{5mm}	
	\fcolorbox{#1}{#1!3}{\makebox[\linewidth-2\fboxrule-2\fboxsep]{
  		\begin{minipage}[t]{\linewidth-2\fboxrule-4\fboxsep}\setlength{\parskip}{3mm}
  			 #2
  		\end{minipage}
	}}
	\vspace{5mm}
}

\newcommand\CBox[4]{				% Boites
	\vspace{5mm}
	\setlength{\fboxrule}{0.2mm}
	\setlength{\fboxsep}{5mm}
	
	\fcolorbox{#1}{#1!3}{\makebox[\linewidth-2\fboxrule-2\fboxsep]{
		\begin{minipage}[t]{1cm}\setlength{\parskip}{3mm}
	  		\textcolor{#1}{\LARGE{#2}}    
 	 	\end{minipage}  
  		\begin{minipage}[t]{\linewidth-2\fboxrule-4\fboxsep}\setlength{\parskip}{3mm}
			\raisebox{1.2mm}{\normalsize\sffamily{\textcolor{#1}{#3}}}						
  			 #4
  		\end{minipage}
	}}
	\vspace{5mm}
}

\newcommand\cadre[3]{				% Boites convertible html
	\par
	\vspace{2mm}
	\setlength{\fboxrule}{0.1mm}
	\setlength{\fboxsep}{5mm}
	\fcolorbox{#1}{white}{\makebox[\linewidth-2\fboxrule-2\fboxsep]{
  		\begin{minipage}[t]{\linewidth-2\fboxrule-4\fboxsep}\setlength{\parskip}{3mm}
			\raisebox{-2.5mm}{\sffamily \small{\textcolor{#1}{\MakeUppercase{#2}}}}		
			\par		
  			 #3
 	 		\end{minipage}
	}}
		\vspace{2mm}
	\par
}

\newcommand\bloc[3]{				% Boites convertible html sans bordure
     \needspace{2\baselineskip}
     {\sffamily \small{\textcolor{#1}{\MakeUppercase{#2}}}}    
		\par		
  			 #3
		\par
}

\newcommand\CHelp[1]{
     \CBox{Plum}{\faInfoCircle}{À RETENIR}{#1}
}

\newcommand\CUp[1]{
     \CBox{NavyBlue}{\faThumbsOUp}{EN PRATIQUE}{#1}
}

\newcommand\CInfo[1]{
     \CBox{Sepia}{\faArrowCircleRight}{REMARQUE}{#1}
}

\newcommand\CRedac[1]{
     \CBox{PineGreen}{\faEdit}{BIEN R\'EDIGER}{#1}
}

\newcommand\CError[1]{
     \CBox{Red}{\faExclamationTriangle}{ATTENTION}{#1}
}

\newcommand\TitreExo[2]{
\needspace{4\baselineskip}
 {\sffamily\large EXERCICE #1\ (\emph{#2 points})}
\vspace{5mm}
}

\newcommand\img[2]{
          \includegraphics[width=#2\paperwidth]{\imgdir#1}
}

\newcommand\imgsvg[2]{
       \begin{center}   \includegraphics[width=#2\paperwidth]{\imgsvgdir#1} \end{center}
}


\newcommand\Lien[2]{
     \href{#1}{#2 \tiny \faExternalLink}
}
\newcommand\mcLien[2]{
     \href{https~://www.maths-cours.fr/#1}{#2 \tiny \faExternalLink}
}

\newcommand{\euro}{\eurologo{}}

%================================================================================================================================
%
% Macros - Environement
%
%================================================================================================================================

\newenvironment{tex}{ %
}
{%
}

\newenvironment{indente}{ %
	\setlength\parindent{10mm}
}

{
	\setlength\parindent{0mm}
}

\newenvironment{corrige}{%
     \needspace{3\baselineskip}
     \medskip
     \textbf{\textsc{Corrigé}}
     \medskip
}
{
}

\newenvironment{extern}{%
     \begin{center}
     }
     {
     \end{center}
}

\NewEnviron{code}{%
	\par
     \boite{gray}{\texttt{%
     \BODY
     }}
     \par
}

\newenvironment{vbloc}{% boite sans cadre empeche saut de page
     \begin{minipage}[t]{\linewidth}
     }
     {
     \end{minipage}
}
\NewEnviron{h2}{%
    \needspace{3\baselineskip}
    \vspace{0.6cm}
	\noindent \MakeUppercase{\sffamily \large \BODY}
	\vspace{1mm}\textcolor{mcgris}{\hrule}\vspace{0.4cm}
	\par
}{}

\NewEnviron{h3}{%
    \needspace{3\baselineskip}
	\vspace{5mm}
	\textsc{\BODY}
	\par
}

\NewEnviron{margeneg}{ %
\begin{addmargin}[-1cm]{0cm}
\BODY
\end{addmargin}
}

\NewEnviron{html}{%
}

\begin{document}
\meta{url}{/exercices/calcul-du-pgcd-avec-python/}
\meta{pid}{11681}
\meta{titre}{Calcul du PGCD avec Python}
\meta{type}{exercices}
%
\begin{enumerate}
     \item
     Ecrire une fonction Python nommée \texttt{diviseurs} qui prend en argument un nombre entier naturel non nul \texttt{n} et qui retourne la liste de ses diviseurs triés par ordre croissant.
     \par
     Tester votre fonction à l'aide des instructions suivantes~:
\begin{lstlisting}[language=Python]
>>> diviseurs(1)
>>> diviseurs(2)
>>> diviseurs(100)
>>> diviseurs(101)
\end{lstlisting}
\par
\textit{Rappel}~:
\\
L'opérateur \texttt{\%} renvoie le reste de la division euclidienne de deux entiers.
\item
En utilisant la fonction \texttt{diviseurs} de la question précédente, écrire une fonction \texttt{est_premier} qui prend un argument entier et retourne un booléen valant \texttt{True} si l'argument est un nombre premier et \texttt{False} dans le cas contraire.
\par
Tester votre fonction à l'aide des instructions suivantes~:
\begin{lstlisting}[language=Python]
>>> est_premier(1)
>>> est_premier(2)
>>> est_premier(100)
>>> est_premier(101)
\end{lstlisting}
\item
L'opérateur Python \texttt{in} permet de savoir si un élément appartient ou non à une liste.
\\
On l'utilise de la manière suivante~:
\begin{lstlisting}[language=Python]
liste = ['rouge', 'vert', 'jaune']
print('rouge' in liste) # affiche True
print('bleu' in liste) # affiche False
\end{lstlisting}
\\
À l'aide de la fonction \texttt{diviseurs} de la première question et de l'opérateur \texttt{in} écrire une fonction nommée \texttt{inter} qui prend deux listes en arguments et qui renvoie une troisième liste contenant les éléments communs aux deux listes.
\par
Par exemple on souhaite que l'instruction \texttt{inter(['a', 'b', 'c'], ['b', 'c', 'd', 'e']) } retourne la liste \texttt{['b', 'c']}.
\par
Tester votre fonction à l'aide des instructions suivantes~:
\begin{lstlisting}[language=Python]
>>> inter(['a', 'b', 'c'], ['b', 'c', 'd', 'e'])
>>> inter([1,2,4], [1, 2, 3, 6])
>>> inter([1,2,3], [4,5,6])
\end{lstlisting}
\item
À l'aide des fonctions définies aux questions \textbf{1} et \textbf{3}, écrire une fonction nommée \texttt{pgcd} qui prend deux entiers naturels en arguments et qui retourne leur PGCD.
\par
Tester votre fonction à l'aide des instructions suivantes~:
\begin{lstlisting}[language=Python]
>>> pgcd(10, 11)
>>> pgcd(15,20)
>>> pgcd(501,666)
>>> pgcd(110,121)
\end{lstlisting}
\end{enumerate}
\begin{corrige}
     \begin{enumerate}
          \item
          On peut coder la fonction à l'aide d'une boucle \texttt{for}.
          \par
          L'instruction \texttt{range(1, n+1)} retourne les entiers naturels compris, au sens large, entre \texttt{1} et \texttt{n}. On teste ensuite si l'entier \texttt{i} divise \texttt{n} grâce à l'instruction \texttt{if n % i == 0~:} et, si c'est le cas, on ajoute \texttt{i} à la liste des diviseurs~:
          \\
\begin{lstlisting}[language=Python]
 
def diviseurs(n) :
    liste = []
    for i in range(1, n+1) :
        if n % i == 0 :
            liste.append(i)
    return liste

     \end{lstlisting}
     \\
     Le mode de construction de cette liste fait que celle-ci est automatiquement triée par ordre croissant.
     \par
     Il est également possible et plus concis d'utiliser une définition de la liste en \textit{compréhension}~:
\begin{lstlisting}[language=Python]
def diviseurs(n) :
    return [i for i in range(1,n+1) if n % i == 0]
\end{lstlisting}
\par
Les tests proposés dans l'énoncé donnent les résultats suivants~:
\\
\begin{lstlisting}[language=Python]
>>> diviseurs(12)
[1, 2, 3, 4, 6, 12]
>>> diviseurs(1)
[1]
>>> diviseurs(2)
[1, 2]
>>> diviseurs(100)
[1, 2, 4, 5, 10, 20, 25, 50, 100]
>>> diviseurs(101)
[1, 101]
\end{lstlisting}
\item
Un nombre entier naturel est premier si et seulement s'il possède exactement deux diviseurs~: 1 et lui-même.
\par
Il suffit donc de tester la longueur de la liste retournée par la fonction \texttt{diviseurs()} pour déterminer si l'argument est premier. La commande \texttt{len(diviseurs(n)) == 2} renvoie \texttt{True} si cette liste contient deux nombres et \texttt{False} sinon~:
\begin{lstlisting}[language=Python]
def est_premier(n) :
    return len(diviseurs(n)) == 2
\end{lstlisting}
\par
Cette fonction donne les résultats suivants~:
\begin{lstlisting}[language=Python]
>>> est_premier(1)
False
>>> est_premier(2)
True
>>> est_premier(100)
False
>>> est_premier(101)
True
\end{lstlisting}
\item
Voici un exemple de fonction qui donne le résultat souhaité (mais qui n'est pas le plus concis~:
\\
\begin{lstlisting}[language=Python]
def inter(liste1, liste2) :
    liste3=[]
    for i in range(len(liste1)) :
        a = liste1[i]
        if a in liste2 :
            liste3.append(a)
    return liste3
\end{lstlisting}
\\
Pour chaque élément de la première liste, on regarde s'il appartient à la seconde liste grâce à l'opérateur \texttt{in}~; si c'est le cas, on ajoute cet élément à la réponse.
\par
Voici le résultat des tests~:
\begin{lstlisting}[language=Python]
>>> inter(['a', 'b', 'c'], ['b', 'c', 'd', 'e'])
['b', 'c']
>>> inter([1,2,4], [1, 2, 3, 6])
[1, 2]
>>> inter([1,2,3], [4,5,6])
[]
\end{lstlisting}
\item
La liste des diviseurs étant ordonnées par ordre croissant, il suffit de retourner le dernier élément de la liste des diviseurs communs (obtenue en faisant l'intersection des deux listes de diviseurs)~:
\par
\textbf{Attention~:} les éléments d'une liste sont indexés de \texttt{0} à \texttt{len(liste)-1}. L'indice du dernier élément est donc \texttt{len(liste)-1}.
\par
\begin{lstlisting}[language=Python]
def pgcd(a, b) :
    diviseurs_communs = inter(diviseurs(a), diviseurs(b))
    return diviseurs_communs[len(diviseurs_communs)-1]
\end{lstlisting}
\par
Les tests demandés donnent les résultats ci-dessous~:
\\
\begin{lstlisting}[language=Python]
>>> pgcd(10, 11)
1
>>> pgcd(15,20)
5
>>> pgcd(501,666)
3
>>> pgcd(110,121)
11
\end{lstlisting}
\end{enumerate}
\end{corrige}

\end{document}
µ
\documentclass[a4paper]{article}

%================================================================================================================================
%
% Packages
%
%================================================================================================================================

\usepackage[T1]{fontenc} 	% pour caractères accentués
\usepackage[utf8]{inputenc}  % encodage utf8
\usepackage[french]{babel}	% langue : français
\usepackage{fourier}			% caractères plus lisibles
\usepackage[dvipsnames]{xcolor} % couleurs
\usepackage{fancyhdr}		% réglage header footer
\usepackage{needspace}		% empêcher sauts de page mal placés
\usepackage{graphicx}		% pour inclure des graphiques
\usepackage{enumitem,cprotect}		% personnalise les listes d'items (nécessaire pour ol, al ...)
\usepackage{hyperref}		% Liens hypertexte
\usepackage{pstricks,pst-all,pst-node,pstricks-add,pst-math,pst-plot,pst-tree,pst-eucl} % pstricks
\usepackage[a4paper,includeheadfoot,top=2cm,left=3cm, bottom=2cm,right=3cm]{geometry} % marges etc.
\usepackage{comment}			% commentaires multilignes
\usepackage{amsmath,environ} % maths (matrices, etc.)
\usepackage{amssymb,makeidx}
\usepackage{bm}				% bold maths
\usepackage{tabularx}		% tableaux
\usepackage{colortbl}		% tableaux en couleur
\usepackage{fontawesome}		% Fontawesome
\usepackage{environ}			% environment with command
\usepackage{fp}				% calculs pour ps-tricks
\usepackage{multido}			% pour ps tricks
\usepackage[np]{numprint}	% formattage nombre
\usepackage{tikz,tkz-tab} 			% package principal TikZ
\usepackage{pgfplots}   % axes
\usepackage{mathrsfs}    % cursives
\usepackage{calc}			% calcul taille boites
\usepackage[scaled=0.875]{helvet} % font sans serif
\usepackage{svg} % svg
\usepackage{scrextend} % local margin
\usepackage{scratch} %scratch
\usepackage{multicol} % colonnes
%\usepackage{infix-RPN,pst-func} % formule en notation polanaise inversée
\usepackage{listings}

%================================================================================================================================
%
% Réglages de base
%
%================================================================================================================================

\lstset{
language=Python,   % R code
literate=
{á}{{\'a}}1
{à}{{\`a}}1
{ã}{{\~a}}1
{é}{{\'e}}1
{è}{{\`e}}1
{ê}{{\^e}}1
{í}{{\'i}}1
{ó}{{\'o}}1
{õ}{{\~o}}1
{ú}{{\'u}}1
{ü}{{\"u}}1
{ç}{{\c{c}}}1
{~}{{ }}1
}


\definecolor{codegreen}{rgb}{0,0.6,0}
\definecolor{codegray}{rgb}{0.5,0.5,0.5}
\definecolor{codepurple}{rgb}{0.58,0,0.82}
\definecolor{backcolour}{rgb}{0.95,0.95,0.92}

\lstdefinestyle{mystyle}{
    backgroundcolor=\color{backcolour},   
    commentstyle=\color{codegreen},
    keywordstyle=\color{magenta},
    numberstyle=\tiny\color{codegray},
    stringstyle=\color{codepurple},
    basicstyle=\ttfamily\footnotesize,
    breakatwhitespace=false,         
    breaklines=true,                 
    captionpos=b,                    
    keepspaces=true,                 
    numbers=left,                    
xleftmargin=2em,
framexleftmargin=2em,            
    showspaces=false,                
    showstringspaces=false,
    showtabs=false,                  
    tabsize=2,
    upquote=true
}

\lstset{style=mystyle}


\lstset{style=mystyle}
\newcommand{\imgdir}{C:/laragon/www/newmc/assets/imgsvg/}
\newcommand{\imgsvgdir}{C:/laragon/www/newmc/assets/imgsvg/}

\definecolor{mcgris}{RGB}{220, 220, 220}% ancien~; pour compatibilité
\definecolor{mcbleu}{RGB}{52, 152, 219}
\definecolor{mcvert}{RGB}{125, 194, 70}
\definecolor{mcmauve}{RGB}{154, 0, 215}
\definecolor{mcorange}{RGB}{255, 96, 0}
\definecolor{mcturquoise}{RGB}{0, 153, 153}
\definecolor{mcrouge}{RGB}{255, 0, 0}
\definecolor{mclightvert}{RGB}{205, 234, 190}

\definecolor{gris}{RGB}{220, 220, 220}
\definecolor{bleu}{RGB}{52, 152, 219}
\definecolor{vert}{RGB}{125, 194, 70}
\definecolor{mauve}{RGB}{154, 0, 215}
\definecolor{orange}{RGB}{255, 96, 0}
\definecolor{turquoise}{RGB}{0, 153, 153}
\definecolor{rouge}{RGB}{255, 0, 0}
\definecolor{lightvert}{RGB}{205, 234, 190}
\setitemize[0]{label=\color{lightvert}  $\bullet$}

\pagestyle{fancy}
\renewcommand{\headrulewidth}{0.2pt}
\fancyhead[L]{maths-cours.fr}
\fancyhead[R]{\thepage}
\renewcommand{\footrulewidth}{0.2pt}
\fancyfoot[C]{}

\newcolumntype{C}{>{\centering\arraybackslash}X}
\newcolumntype{s}{>{\hsize=.35\hsize\arraybackslash}X}

\setlength{\parindent}{0pt}		 
\setlength{\parskip}{3mm}
\setlength{\headheight}{1cm}

\def\ebook{ebook}
\def\book{book}
\def\web{web}
\def\type{web}

\newcommand{\vect}[1]{\overrightarrow{\,\mathstrut#1\,}}

\def\Oij{$\left(\text{O}~;~\vect{\imath},~\vect{\jmath}\right)$}
\def\Oijk{$\left(\text{O}~;~\vect{\imath},~\vect{\jmath},~\vect{k}\right)$}
\def\Ouv{$\left(\text{O}~;~\vect{u},~\vect{v}\right)$}

\hypersetup{breaklinks=true, colorlinks = true, linkcolor = OliveGreen, urlcolor = OliveGreen, citecolor = OliveGreen, pdfauthor={Didier BONNEL - https://www.maths-cours.fr} } % supprime les bordures autour des liens

\renewcommand{\arg}[0]{\text{arg}}

\everymath{\displaystyle}

%================================================================================================================================
%
% Macros - Commandes
%
%================================================================================================================================

\newcommand\meta[2]{    			% Utilisé pour créer le post HTML.
	\def\titre{titre}
	\def\url{url}
	\def\arg{#1}
	\ifx\titre\arg
		\newcommand\maintitle{#2}
		\fancyhead[L]{#2}
		{\Large\sffamily \MakeUppercase{#2}}
		\vspace{1mm}\textcolor{mcvert}{\hrule}
	\fi 
	\ifx\url\arg
		\fancyfoot[L]{\href{https://www.maths-cours.fr#2}{\black \footnotesize{https://www.maths-cours.fr#2}}}
	\fi 
}


\newcommand\TitreC[1]{    		% Titre centré
     \needspace{3\baselineskip}
     \begin{center}\textbf{#1}\end{center}
}

\newcommand\newpar{    		% paragraphe
     \par
}

\newcommand\nosp {    		% commande vide (pas d'espace)
}
\newcommand{\id}[1]{} %ignore

\newcommand\boite[2]{				% Boite simple sans titre
	\vspace{5mm}
	\setlength{\fboxrule}{0.2mm}
	\setlength{\fboxsep}{5mm}	
	\fcolorbox{#1}{#1!3}{\makebox[\linewidth-2\fboxrule-2\fboxsep]{
  		\begin{minipage}[t]{\linewidth-2\fboxrule-4\fboxsep}\setlength{\parskip}{3mm}
  			 #2
  		\end{minipage}
	}}
	\vspace{5mm}
}

\newcommand\CBox[4]{				% Boites
	\vspace{5mm}
	\setlength{\fboxrule}{0.2mm}
	\setlength{\fboxsep}{5mm}
	
	\fcolorbox{#1}{#1!3}{\makebox[\linewidth-2\fboxrule-2\fboxsep]{
		\begin{minipage}[t]{1cm}\setlength{\parskip}{3mm}
	  		\textcolor{#1}{\LARGE{#2}}    
 	 	\end{minipage}  
  		\begin{minipage}[t]{\linewidth-2\fboxrule-4\fboxsep}\setlength{\parskip}{3mm}
			\raisebox{1.2mm}{\normalsize\sffamily{\textcolor{#1}{#3}}}						
  			 #4
  		\end{minipage}
	}}
	\vspace{5mm}
}

\newcommand\cadre[3]{				% Boites convertible html
	\par
	\vspace{2mm}
	\setlength{\fboxrule}{0.1mm}
	\setlength{\fboxsep}{5mm}
	\fcolorbox{#1}{white}{\makebox[\linewidth-2\fboxrule-2\fboxsep]{
  		\begin{minipage}[t]{\linewidth-2\fboxrule-4\fboxsep}\setlength{\parskip}{3mm}
			\raisebox{-2.5mm}{\sffamily \small{\textcolor{#1}{\MakeUppercase{#2}}}}		
			\par		
  			 #3
 	 		\end{minipage}
	}}
		\vspace{2mm}
	\par
}

\newcommand\bloc[3]{				% Boites convertible html sans bordure
     \needspace{2\baselineskip}
     {\sffamily \small{\textcolor{#1}{\MakeUppercase{#2}}}}    
		\par		
  			 #3
		\par
}

\newcommand\CHelp[1]{
     \CBox{Plum}{\faInfoCircle}{À RETENIR}{#1}
}

\newcommand\CUp[1]{
     \CBox{NavyBlue}{\faThumbsOUp}{EN PRATIQUE}{#1}
}

\newcommand\CInfo[1]{
     \CBox{Sepia}{\faArrowCircleRight}{REMARQUE}{#1}
}

\newcommand\CRedac[1]{
     \CBox{PineGreen}{\faEdit}{BIEN R\'EDIGER}{#1}
}

\newcommand\CError[1]{
     \CBox{Red}{\faExclamationTriangle}{ATTENTION}{#1}
}

\newcommand\TitreExo[2]{
\needspace{4\baselineskip}
 {\sffamily\large EXERCICE #1\ (\emph{#2 points})}
\vspace{5mm}
}

\newcommand\img[2]{
          \includegraphics[width=#2\paperwidth]{\imgdir#1}
}

\newcommand\imgsvg[2]{
       \begin{center}   \includegraphics[width=#2\paperwidth]{\imgsvgdir#1} \end{center}
}


\newcommand\Lien[2]{
     \href{#1}{#2 \tiny \faExternalLink}
}
\newcommand\mcLien[2]{
     \href{https~://www.maths-cours.fr/#1}{#2 \tiny \faExternalLink}
}

\newcommand{\euro}{\eurologo{}}

%================================================================================================================================
%
% Macros - Environement
%
%================================================================================================================================

\newenvironment{tex}{ %
}
{%
}

\newenvironment{indente}{ %
	\setlength\parindent{10mm}
}

{
	\setlength\parindent{0mm}
}

\newenvironment{corrige}{%
     \needspace{3\baselineskip}
     \medskip
     \textbf{\textsc{Corrigé}}
     \medskip
}
{
}

\newenvironment{extern}{%
     \begin{center}
     }
     {
     \end{center}
}

\NewEnviron{code}{%
	\par
     \boite{gray}{\texttt{%
     \BODY
     }}
     \par
}

\newenvironment{vbloc}{% boite sans cadre empeche saut de page
     \begin{minipage}[t]{\linewidth}
     }
     {
     \end{minipage}
}
\NewEnviron{h2}{%
    \needspace{3\baselineskip}
    \vspace{0.6cm}
	\noindent \MakeUppercase{\sffamily \large \BODY}
	\vspace{1mm}\textcolor{mcgris}{\hrule}\vspace{0.4cm}
	\par
}{}

\NewEnviron{h3}{%
    \needspace{3\baselineskip}
	\vspace{5mm}
	\textsc{\BODY}
	\par
}

\NewEnviron{margeneg}{ %
\begin{addmargin}[-1cm]{0cm}
\BODY
\end{addmargin}
}

\NewEnviron{html}{%
}

\begin{document}
\meta{url}{/exercices/modelisation-par-une-fonction-exponentielle/}
\meta{pid}{15434}
\meta{titre}{Modélisation par une fonction exponentielle}
\meta{type}{exercices}
%
\\
Le maire d'une ville française a effectué un recensement de la population de sa municipalité pendant 7 ans.
\\
Les données recueillies sont présentées dans le tableau ci-dessous~:
\par\begin{center}
     \begin{tabularx}{0.8linewidth}{|*{8}{>{centering arraybackslash }X|}}%class="compact" style="width:50rem"
          \hline
          Année & 2013 & 2014 & 2015 & 2016 & 2017 & 2018 & 2019\\ \hline
          Rang & 0 & 1 & 2 & 3 & 4 & 5 & 6 \\ \hline
          Habitants & 2~502 & 2~475 & 2~452 & 2~430 & 2~398 & 2~378 & 2~351 \\ \hline
     \end{tabularx}
\end{center}
\par
Dans la première partie de l'exercice, on modélisera le nombre d'habitants à l'aide d'une suite géométrique et dans la seconde partie, on utilisera une fonction exponentielle.
\par
\begin{h2} Partie 1~: Modélisation à l'aide d'une suite \end{h2}
\begin{enumerate}
     \item
     Calculer le pourcentage d'évolution de la population de la ville entre 2013 et 2014, entre 2014 et 2015, entre 2015 et 2016 et entre 2018 et 2019.
     \item
     Par la suite on estimera que la population diminue de 1\% par an.
     \par
     On note $ p_n $ le nombre d'habitants l'année 2013+$n$.
     \par
     Montrer que la suite $(p_n)$ est une suite géométrique dont on donnera le premier terme et la raison.
     \item
     À l'aide de la suite $ (p_n) $ estimer la population de la ville en 2030 en supposant que la diminution de la population s'effectue au même rythme pendant les années à venir.
\end{enumerate}
\begin{h2} Partie 2~: Modélisation à l'aide d'une fonction exponentielle\end{h2}
\begin{enumerate}
     \item
     On cherche à modéliser le nombre d'habitants à l'aide de la fonction $f$ définie sur $ \left[ 0~;~ +\infty \right[ $ par~:
     \begin{center}
          $f~: \ t \longmapsto 2500\ \text{e}^{ -0,01t } $
     \end{center}
     où $t$ désigne la durée écoulée, en année, depuis 2013.
     \par
     Montrer que la fonction $f$ est strictement décroissante sur l'intervalle $ \left[ 0~;~ +\infty \right[ $.
     \item
     Compléter la fonction Python ci-dessous afin qu'elle retourne les images de la variable $t$ par la fonction $f$~:
\begin{lstlisting}[language=Python]
def f(t) :
    return ...
\end{lstlisting}
À l'aide d'une boucle, écrire un script Python qui retourne les images par $f$ des entiers compris entre 0 et 6.
\\ Comparer aux données de l'énoncé.
\\ Cette modélisation vous semble-t-elle valable~?
\item
Le maire souhaite prévoir en quelle année le nombre d'habitants de sa ville passera sous la barre des 2~200 d'après ce modèle.
\par
En utilisant la fonction précédente, écrire un programme Python qui répond à cette question.
\end{enumerate}
\begin{corrige}
     \TitreC{ Partie 1 }
     \begin{enumerate}
          \item
          Le pourcentage d'évolution de la population entre 2013 et 2014 est (voir \mcLien{https://www.maths-cours.fr/cours/pourcentages/\#d40}{formule de calcul d'une évolution})~:
          \par
          $ t_1 = \frac{ p_1 - p_0 }{ p_0 } = \frac{ 2~475 - 2~502 }{ 2~502 } $\nosp$ \approx -0,0108 \approx \frac{ -1,08 }{ 100 } = -1,08\%$
          \medskip
          De même, le pourcentage d'évolution entre 2014 et 2015 est~:
          \par
          $ t_2 = \frac{ p_2 - p_1}{ p_1 } = \frac{ 2~452 - 2~475 }{ 2~475 } $\nosp$ \approx -0,0093 \approx \frac{ -0,93 }{ 100 } = -0,93\%$
          \medskip
          entre 2015 et 2016~:
          \par
          $ t_3 = \frac{ p_3 - p_2}{ p_2 } = \frac{ 2~430 - 2~452 }{ 2~452 } $\nosp$ \approx -0,0090 \approx \frac{ -0,90 }{ 100 } = -0,90\%$
          \medskip
          enfin, entre 2018 et 2019~:
          \par
          $ t_6 = \frac{ p_6 - p_5}{ p_5 } = \frac{ 2~351 - 2~378 }{ 2~378 } $\nosp$ \approx -0,0114 \approx \frac{ -1,14 }{ 100 } = -1,14\%$
          \medskip
          On remarque que, dans tous les cas, la diminution est proche de 1\%.
          \item
          Le coefficient multiplicateur qui fait passer de $p_{n+1}$ à $p_n$ correspondant à une baisse de 1\% est (voir \mcLien{https://www.maths-cours.fr/cours/pourcentages/#d30}{coefficient multiplicateur})~:
          \par
          $ CM=1- \frac{ 1 }{ 100 } =0,99 $
          \par
          On a donc, pour tout entier naturel $n$~:
          \par
          $p_{n+1} = 0,99p_n $
          \par
          La suite $ \left( p_n \right) $ est donc une suite géométrique de raison $ q = 0,99. $ Son premier terme est $ p_0=2502. $
          \item
          La population de la ville à l'année de rang $n$ est~:
          \par
          $ p_n=p_0\ q^n = 2502 \times 0,99^n$
          \medskip
          L'année 2030 correspond au rang 17. La population en 2030 peut donc, d'après ce modèle, être estimée à~:
          \par
          $ p_{ 17 } = 2502 \times 0,99^{ 17 } \approx 2109.$
     \end{enumerate}
     \TitreC{ Partie 2 }
     \begin{enumerate}
          \item
          $f$ est dérivable sur $ \left[ 0~;~ +\infty \right[ $. Pour déterminer le sens de variation de $f$, on calcule sa dérivée $ f' $.
          \\Sachant que la dérivée de la fonction $ t \longmapsto \text{e}^{ at } $ est la fonction $ t \longmapsto a\ \text{e}^{ at }$ on obtient~:
          \par
          $f'(t)=2500 \times -0,01 \text{e}^{ -0,01t } =-25 \ \text{e}^{ -0,01t } $
          \medskip
          $ -25 $ est strictement négatif tandis que $\text{e}^{ -0,01t } $ est strictement positif (car la fonction exponentielle ne prend que des valeurs strictement positives) donc $f'(t) < 0$ sur $ \left[ 0~;~ +\infty \right[ $.
          \par
          Par conséquent, la fonction $f$ est strictement décroissante sur l'intervalle $ \left[ 0~;~ +\infty \right[ $.
          \item
          La fonction Python se définit simplement comme suit~:
\begin{lstlisting}[language=Python]
def f(t) :
    return 2500 * exp(-0.01 * t)
     \end{lstlisting}
     \par
     On doit toutefois importer le module math qui contient la fonction exp~; par exemple~:
\begin{lstlisting}[language=Python]
from math import exp
def f(t) :
    return 2500 * exp(0.01 * t)
\end{lstlisting}
\par
Comme on connait le nombre d'itérations, on peut employer une boucle \texttt{for} pour afficher les images des 7 premières valeurs entières de $t$~:
\begin{lstlisting}[language=Python]
from math import exp
def f(t) :
    return 2500 * exp(-0.01 * t)

for t in range(7) : 
    print(f(t))
\end{lstlisting}
\par
\\
On obtient le résultat suivant~:
\begin{lstlisting}[language=Python]
2500.0
2475.1245843729203
2450.4966832668883
2426.1138338712703
2401.973597880808
2378.073561251785
2354.411333960622
\end{lstlisting}
\par
Ces valeurs sont suffisamment proches de celles du tableau donné dans l'énoncé pour considérer que cette modélisation est satisfaisante.
\item
On utilise une boucle \texttt{while} pour répondre à la question.
\\On reste dans la boucle tant que le nombre d'habitants est supérieur ou égal à 2~200 et on sort de la boucle dès que ce nombre devient strictement inférieur à 2~200.
\par
Il faut penser à initialiser la variable \texttt{t} avant la boucle et à l’incrémenter à l'intérieur de la boucle (voir~: \mcLien{https://www.maths-cours.fr/cours/python-au-lycee-3-les-boucles/#h10}{boucles while}). On peut ensuite afficher la valeur de \texttt{t} à la sortie de la boucle~:
\begin{lstlisting}[language=Python]
from math import exp
def f(t) :
    return 2500 * exp(-0.01 * t)

t=0
while f(t) >= 2200: 
    t=t+1
print(t)
\end{lstlisting}
\par
Ce programme affiche la valeur 13.
\par
D'après ce modèle, la population passera sous la barre des 2~200 l'année de rang 13 c'est à dire en 2013+13 = 2026.
\end{enumerate}
\end{corrige}

\end{document}
µ
\documentclass[a4paper]{article}

%================================================================================================================================
%
% Packages
%
%================================================================================================================================

\usepackage[T1]{fontenc} 	% pour caractères accentués
\usepackage[utf8]{inputenc}  % encodage utf8
\usepackage[french]{babel}	% langue : français
\usepackage{fourier}			% caractères plus lisibles
\usepackage[dvipsnames]{xcolor} % couleurs
\usepackage{fancyhdr}		% réglage header footer
\usepackage{needspace}		% empêcher sauts de page mal placés
\usepackage{graphicx}		% pour inclure des graphiques
\usepackage{enumitem,cprotect}		% personnalise les listes d'items (nécessaire pour ol, al ...)
\usepackage{hyperref}		% Liens hypertexte
\usepackage{pstricks,pst-all,pst-node,pstricks-add,pst-math,pst-plot,pst-tree,pst-eucl} % pstricks
\usepackage[a4paper,includeheadfoot,top=2cm,left=3cm, bottom=2cm,right=3cm]{geometry} % marges etc.
\usepackage{comment}			% commentaires multilignes
\usepackage{amsmath,environ} % maths (matrices, etc.)
\usepackage{amssymb,makeidx}
\usepackage{bm}				% bold maths
\usepackage{tabularx}		% tableaux
\usepackage{colortbl}		% tableaux en couleur
\usepackage{fontawesome}		% Fontawesome
\usepackage{environ}			% environment with command
\usepackage{fp}				% calculs pour ps-tricks
\usepackage{multido}			% pour ps tricks
\usepackage[np]{numprint}	% formattage nombre
\usepackage{tikz,tkz-tab} 			% package principal TikZ
\usepackage{pgfplots}   % axes
\usepackage{mathrsfs}    % cursives
\usepackage{calc}			% calcul taille boites
\usepackage[scaled=0.875]{helvet} % font sans serif
\usepackage{svg} % svg
\usepackage{scrextend} % local margin
\usepackage{scratch} %scratch
\usepackage{multicol} % colonnes
%\usepackage{infix-RPN,pst-func} % formule en notation polanaise inversée
\usepackage{listings}

%================================================================================================================================
%
% Réglages de base
%
%================================================================================================================================

\lstset{
language=Python,   % R code
literate=
{á}{{\'a}}1
{à}{{\`a}}1
{ã}{{\~a}}1
{é}{{\'e}}1
{è}{{\`e}}1
{ê}{{\^e}}1
{í}{{\'i}}1
{ó}{{\'o}}1
{õ}{{\~o}}1
{ú}{{\'u}}1
{ü}{{\"u}}1
{ç}{{\c{c}}}1
{~}{{ }}1
}


\definecolor{codegreen}{rgb}{0,0.6,0}
\definecolor{codegray}{rgb}{0.5,0.5,0.5}
\definecolor{codepurple}{rgb}{0.58,0,0.82}
\definecolor{backcolour}{rgb}{0.95,0.95,0.92}

\lstdefinestyle{mystyle}{
    backgroundcolor=\color{backcolour},   
    commentstyle=\color{codegreen},
    keywordstyle=\color{magenta},
    numberstyle=\tiny\color{codegray},
    stringstyle=\color{codepurple},
    basicstyle=\ttfamily\footnotesize,
    breakatwhitespace=false,         
    breaklines=true,                 
    captionpos=b,                    
    keepspaces=true,                 
    numbers=left,                    
xleftmargin=2em,
framexleftmargin=2em,            
    showspaces=false,                
    showstringspaces=false,
    showtabs=false,                  
    tabsize=2,
    upquote=true
}

\lstset{style=mystyle}


\lstset{style=mystyle}
\newcommand{\imgdir}{C:/laragon/www/newmc/assets/imgsvg/}
\newcommand{\imgsvgdir}{C:/laragon/www/newmc/assets/imgsvg/}

\definecolor{mcgris}{RGB}{220, 220, 220}% ancien~; pour compatibilité
\definecolor{mcbleu}{RGB}{52, 152, 219}
\definecolor{mcvert}{RGB}{125, 194, 70}
\definecolor{mcmauve}{RGB}{154, 0, 215}
\definecolor{mcorange}{RGB}{255, 96, 0}
\definecolor{mcturquoise}{RGB}{0, 153, 153}
\definecolor{mcrouge}{RGB}{255, 0, 0}
\definecolor{mclightvert}{RGB}{205, 234, 190}

\definecolor{gris}{RGB}{220, 220, 220}
\definecolor{bleu}{RGB}{52, 152, 219}
\definecolor{vert}{RGB}{125, 194, 70}
\definecolor{mauve}{RGB}{154, 0, 215}
\definecolor{orange}{RGB}{255, 96, 0}
\definecolor{turquoise}{RGB}{0, 153, 153}
\definecolor{rouge}{RGB}{255, 0, 0}
\definecolor{lightvert}{RGB}{205, 234, 190}
\setitemize[0]{label=\color{lightvert}  $\bullet$}

\pagestyle{fancy}
\renewcommand{\headrulewidth}{0.2pt}
\fancyhead[L]{maths-cours.fr}
\fancyhead[R]{\thepage}
\renewcommand{\footrulewidth}{0.2pt}
\fancyfoot[C]{}

\newcolumntype{C}{>{\centering\arraybackslash}X}
\newcolumntype{s}{>{\hsize=.35\hsize\arraybackslash}X}

\setlength{\parindent}{0pt}		 
\setlength{\parskip}{3mm}
\setlength{\headheight}{1cm}

\def\ebook{ebook}
\def\book{book}
\def\web{web}
\def\type{web}

\newcommand{\vect}[1]{\overrightarrow{\,\mathstrut#1\,}}

\def\Oij{$\left(\text{O}~;~\vect{\imath},~\vect{\jmath}\right)$}
\def\Oijk{$\left(\text{O}~;~\vect{\imath},~\vect{\jmath},~\vect{k}\right)$}
\def\Ouv{$\left(\text{O}~;~\vect{u},~\vect{v}\right)$}

\hypersetup{breaklinks=true, colorlinks = true, linkcolor = OliveGreen, urlcolor = OliveGreen, citecolor = OliveGreen, pdfauthor={Didier BONNEL - https://www.maths-cours.fr} } % supprime les bordures autour des liens

\renewcommand{\arg}[0]{\text{arg}}

\everymath{\displaystyle}

%================================================================================================================================
%
% Macros - Commandes
%
%================================================================================================================================

\newcommand\meta[2]{    			% Utilisé pour créer le post HTML.
	\def\titre{titre}
	\def\url{url}
	\def\arg{#1}
	\ifx\titre\arg
		\newcommand\maintitle{#2}
		\fancyhead[L]{#2}
		{\Large\sffamily \MakeUppercase{#2}}
		\vspace{1mm}\textcolor{mcvert}{\hrule}
	\fi 
	\ifx\url\arg
		\fancyfoot[L]{\href{https://www.maths-cours.fr#2}{\black \footnotesize{https://www.maths-cours.fr#2}}}
	\fi 
}


\newcommand\TitreC[1]{    		% Titre centré
     \needspace{3\baselineskip}
     \begin{center}\textbf{#1}\end{center}
}

\newcommand\newpar{    		% paragraphe
     \par
}

\newcommand\nosp {    		% commande vide (pas d'espace)
}
\newcommand{\id}[1]{} %ignore

\newcommand\boite[2]{				% Boite simple sans titre
	\vspace{5mm}
	\setlength{\fboxrule}{0.2mm}
	\setlength{\fboxsep}{5mm}	
	\fcolorbox{#1}{#1!3}{\makebox[\linewidth-2\fboxrule-2\fboxsep]{
  		\begin{minipage}[t]{\linewidth-2\fboxrule-4\fboxsep}\setlength{\parskip}{3mm}
  			 #2
  		\end{minipage}
	}}
	\vspace{5mm}
}

\newcommand\CBox[4]{				% Boites
	\vspace{5mm}
	\setlength{\fboxrule}{0.2mm}
	\setlength{\fboxsep}{5mm}
	
	\fcolorbox{#1}{#1!3}{\makebox[\linewidth-2\fboxrule-2\fboxsep]{
		\begin{minipage}[t]{1cm}\setlength{\parskip}{3mm}
	  		\textcolor{#1}{\LARGE{#2}}    
 	 	\end{minipage}  
  		\begin{minipage}[t]{\linewidth-2\fboxrule-4\fboxsep}\setlength{\parskip}{3mm}
			\raisebox{1.2mm}{\normalsize\sffamily{\textcolor{#1}{#3}}}						
  			 #4
  		\end{minipage}
	}}
	\vspace{5mm}
}

\newcommand\cadre[3]{				% Boites convertible html
	\par
	\vspace{2mm}
	\setlength{\fboxrule}{0.1mm}
	\setlength{\fboxsep}{5mm}
	\fcolorbox{#1}{white}{\makebox[\linewidth-2\fboxrule-2\fboxsep]{
  		\begin{minipage}[t]{\linewidth-2\fboxrule-4\fboxsep}\setlength{\parskip}{3mm}
			\raisebox{-2.5mm}{\sffamily \small{\textcolor{#1}{\MakeUppercase{#2}}}}		
			\par		
  			 #3
 	 		\end{minipage}
	}}
		\vspace{2mm}
	\par
}

\newcommand\bloc[3]{				% Boites convertible html sans bordure
     \needspace{2\baselineskip}
     {\sffamily \small{\textcolor{#1}{\MakeUppercase{#2}}}}    
		\par		
  			 #3
		\par
}

\newcommand\CHelp[1]{
     \CBox{Plum}{\faInfoCircle}{À RETENIR}{#1}
}

\newcommand\CUp[1]{
     \CBox{NavyBlue}{\faThumbsOUp}{EN PRATIQUE}{#1}
}

\newcommand\CInfo[1]{
     \CBox{Sepia}{\faArrowCircleRight}{REMARQUE}{#1}
}

\newcommand\CRedac[1]{
     \CBox{PineGreen}{\faEdit}{BIEN R\'EDIGER}{#1}
}

\newcommand\CError[1]{
     \CBox{Red}{\faExclamationTriangle}{ATTENTION}{#1}
}

\newcommand\TitreExo[2]{
\needspace{4\baselineskip}
 {\sffamily\large EXERCICE #1\ (\emph{#2 points})}
\vspace{5mm}
}

\newcommand\img[2]{
          \includegraphics[width=#2\paperwidth]{\imgdir#1}
}

\newcommand\imgsvg[2]{
       \begin{center}   \includegraphics[width=#2\paperwidth]{\imgsvgdir#1} \end{center}
}


\newcommand\Lien[2]{
     \href{#1}{#2 \tiny \faExternalLink}
}
\newcommand\mcLien[2]{
     \href{https~://www.maths-cours.fr/#1}{#2 \tiny \faExternalLink}
}

\newcommand{\euro}{\eurologo{}}

%================================================================================================================================
%
% Macros - Environement
%
%================================================================================================================================

\newenvironment{tex}{ %
}
{%
}

\newenvironment{indente}{ %
	\setlength\parindent{10mm}
}

{
	\setlength\parindent{0mm}
}

\newenvironment{corrige}{%
     \needspace{3\baselineskip}
     \medskip
     \textbf{\textsc{Corrigé}}
     \medskip
}
{
}

\newenvironment{extern}{%
     \begin{center}
     }
     {
     \end{center}
}

\NewEnviron{code}{%
	\par
     \boite{gray}{\texttt{%
     \BODY
     }}
     \par
}

\newenvironment{vbloc}{% boite sans cadre empeche saut de page
     \begin{minipage}[t]{\linewidth}
     }
     {
     \end{minipage}
}
\NewEnviron{h2}{%
    \needspace{3\baselineskip}
    \vspace{0.6cm}
	\noindent \MakeUppercase{\sffamily \large \BODY}
	\vspace{1mm}\textcolor{mcgris}{\hrule}\vspace{0.4cm}
	\par
}{}

\NewEnviron{h3}{%
    \needspace{3\baselineskip}
	\vspace{5mm}
	\textsc{\BODY}
	\par
}

\NewEnviron{margeneg}{ %
\begin{addmargin}[-1cm]{0cm}
\BODY
\end{addmargin}
}

\NewEnviron{html}{%
}

\begin{document}
\meta{url}{/exercices/mesure-arbre-141012/}
\meta{pid}{8311}
\meta{pi_}{1743}
\meta{titre}{Mesure d'un arbre (Brevet 2013)}
\meta{type}{exercices}
\textit{(D'après Brevet Polynésie 2013)}
\medskip
Teiki se promène en montagne et aimerait connaître la hauteur d'un Pinus (ou Pin des Caraïbes) situé devant lui. Pour cela, il utilise un bâton et prend quelques mesures au sol.
\par
Il procède de la façon suivante :
\begin{itemize}
     \item Il pique le bâton en terre, verticalement, à 12 mètres du Pinus.
     \item La partie visible (hors du sol) du bâton mesure 2 m.
     \item Teiki se place derrière le bâton, de façon à ce que son œil, situé à 1,60 m au dessus du sol, voit en alignement le sommet de l'arbre et l'extrémité du bâton.
     \item Teiki marque sa position au sol, puis mesure la distance entre sa position et le bâton. Il trouve alors 1,2 m.
\end{itemize}
On peut représenter cette situation à l'aide du schéma ci-dessous :
\begin{center}
     \img{mc-0528}{0.1}%width="400" alt="exercice théorème de Thalès
     Brevet Métropole 2018"
\end{center}
Quelle est la hauteur du Pinus au-dessus du sol ?
\begin{corrige}
     \begin{center}
          \img{mc-0529}{0.1}%width="400" alt="corrigé théorème de Thalès
          Brevet Métropole 2018"
     \end{center}
     On peut modéliser la situation à l'aide de la figure ci-dessus où $\left[AE\right]$ représente l'arbre et $\left[FG\right]$ le bâton.
     \par
     D'après les données de l'énoncé on a :
     \begin{itemize}
          \item $EF=BH=12$m
          \item $GF=2$m
          \item $HF=CD=1,6$m
          \item $FD=HC=1,2$m
     \end{itemize}
     On cherche à calculer la hauteur de l'arbre c'est à dire la longueur $AE$.
     \par
     Les points $F, H$ et $G$ étant alignés :
     \par
     $GH=GF-HF=2-1,6=0,4$
     \par
     Les points $E, F$ et $D$ étant alignés :
     \par
     $ED=EF+FD=12+1,2=13,2$ et par conséquent $BC=13,2$
     \par
     Les droites $\left(AE\right)$ et $\left(GF\right)$, étant toutes deux verticales, sont parallèles ; donc d'après le théorème de Thalès :
     \par
     $\frac{GC}{AC}=\frac{HC}{BC}=\frac{GH}{AB}$
     \par
     $\frac{GC}{AC}=\frac{1,2}{13,2}=\frac{0,4}{AB}$
     \par
     De l'égalité des rapports $\frac{1,2}{13,2}=\frac{0,4}{AB}$ on déduit :
     \par
     $AB=\frac{0,4\times 13,2}{1,2}=4,4$
     \par
     La hauteur totale de l'arbre est donc :
     \par
     $AE=AB+BE=4,4+1,6=6$m
     \par
     La hauteur du Pinus au-dessus du sol est $6$ mètres.
\end{corrige}

\end{document}
µ
\documentclass[a4paper]{article}

%================================================================================================================================
%
% Packages
%
%================================================================================================================================

\usepackage[T1]{fontenc} 	% pour caractères accentués
\usepackage[utf8]{inputenc}  % encodage utf8
\usepackage[french]{babel}	% langue : français
\usepackage{fourier}			% caractères plus lisibles
\usepackage[dvipsnames]{xcolor} % couleurs
\usepackage{fancyhdr}		% réglage header footer
\usepackage{needspace}		% empêcher sauts de page mal placés
\usepackage{graphicx}		% pour inclure des graphiques
\usepackage{enumitem,cprotect}		% personnalise les listes d'items (nécessaire pour ol, al ...)
\usepackage{hyperref}		% Liens hypertexte
\usepackage{pstricks,pst-all,pst-node,pstricks-add,pst-math,pst-plot,pst-tree,pst-eucl} % pstricks
\usepackage[a4paper,includeheadfoot,top=2cm,left=3cm, bottom=2cm,right=3cm]{geometry} % marges etc.
\usepackage{comment}			% commentaires multilignes
\usepackage{amsmath,environ} % maths (matrices, etc.)
\usepackage{amssymb,makeidx}
\usepackage{bm}				% bold maths
\usepackage{tabularx}		% tableaux
\usepackage{colortbl}		% tableaux en couleur
\usepackage{fontawesome}		% Fontawesome
\usepackage{environ}			% environment with command
\usepackage{fp}				% calculs pour ps-tricks
\usepackage{multido}			% pour ps tricks
\usepackage[np]{numprint}	% formattage nombre
\usepackage{tikz,tkz-tab} 			% package principal TikZ
\usepackage{pgfplots}   % axes
\usepackage{mathrsfs}    % cursives
\usepackage{calc}			% calcul taille boites
\usepackage[scaled=0.875]{helvet} % font sans serif
\usepackage{svg} % svg
\usepackage{scrextend} % local margin
\usepackage{scratch} %scratch
\usepackage{multicol} % colonnes
%\usepackage{infix-RPN,pst-func} % formule en notation polanaise inversée
\usepackage{listings}

%================================================================================================================================
%
% Réglages de base
%
%================================================================================================================================

\lstset{
language=Python,   % R code
literate=
{á}{{\'a}}1
{à}{{\`a}}1
{ã}{{\~a}}1
{é}{{\'e}}1
{è}{{\`e}}1
{ê}{{\^e}}1
{í}{{\'i}}1
{ó}{{\'o}}1
{õ}{{\~o}}1
{ú}{{\'u}}1
{ü}{{\"u}}1
{ç}{{\c{c}}}1
{~}{{ }}1
}


\definecolor{codegreen}{rgb}{0,0.6,0}
\definecolor{codegray}{rgb}{0.5,0.5,0.5}
\definecolor{codepurple}{rgb}{0.58,0,0.82}
\definecolor{backcolour}{rgb}{0.95,0.95,0.92}

\lstdefinestyle{mystyle}{
    backgroundcolor=\color{backcolour},   
    commentstyle=\color{codegreen},
    keywordstyle=\color{magenta},
    numberstyle=\tiny\color{codegray},
    stringstyle=\color{codepurple},
    basicstyle=\ttfamily\footnotesize,
    breakatwhitespace=false,         
    breaklines=true,                 
    captionpos=b,                    
    keepspaces=true,                 
    numbers=left,                    
xleftmargin=2em,
framexleftmargin=2em,            
    showspaces=false,                
    showstringspaces=false,
    showtabs=false,                  
    tabsize=2,
    upquote=true
}

\lstset{style=mystyle}


\lstset{style=mystyle}
\newcommand{\imgdir}{C:/laragon/www/newmc/assets/imgsvg/}
\newcommand{\imgsvgdir}{C:/laragon/www/newmc/assets/imgsvg/}

\definecolor{mcgris}{RGB}{220, 220, 220}% ancien~; pour compatibilité
\definecolor{mcbleu}{RGB}{52, 152, 219}
\definecolor{mcvert}{RGB}{125, 194, 70}
\definecolor{mcmauve}{RGB}{154, 0, 215}
\definecolor{mcorange}{RGB}{255, 96, 0}
\definecolor{mcturquoise}{RGB}{0, 153, 153}
\definecolor{mcrouge}{RGB}{255, 0, 0}
\definecolor{mclightvert}{RGB}{205, 234, 190}

\definecolor{gris}{RGB}{220, 220, 220}
\definecolor{bleu}{RGB}{52, 152, 219}
\definecolor{vert}{RGB}{125, 194, 70}
\definecolor{mauve}{RGB}{154, 0, 215}
\definecolor{orange}{RGB}{255, 96, 0}
\definecolor{turquoise}{RGB}{0, 153, 153}
\definecolor{rouge}{RGB}{255, 0, 0}
\definecolor{lightvert}{RGB}{205, 234, 190}
\setitemize[0]{label=\color{lightvert}  $\bullet$}

\pagestyle{fancy}
\renewcommand{\headrulewidth}{0.2pt}
\fancyhead[L]{maths-cours.fr}
\fancyhead[R]{\thepage}
\renewcommand{\footrulewidth}{0.2pt}
\fancyfoot[C]{}

\newcolumntype{C}{>{\centering\arraybackslash}X}
\newcolumntype{s}{>{\hsize=.35\hsize\arraybackslash}X}

\setlength{\parindent}{0pt}		 
\setlength{\parskip}{3mm}
\setlength{\headheight}{1cm}

\def\ebook{ebook}
\def\book{book}
\def\web{web}
\def\type{web}

\newcommand{\vect}[1]{\overrightarrow{\,\mathstrut#1\,}}

\def\Oij{$\left(\text{O}~;~\vect{\imath},~\vect{\jmath}\right)$}
\def\Oijk{$\left(\text{O}~;~\vect{\imath},~\vect{\jmath},~\vect{k}\right)$}
\def\Ouv{$\left(\text{O}~;~\vect{u},~\vect{v}\right)$}

\hypersetup{breaklinks=true, colorlinks = true, linkcolor = OliveGreen, urlcolor = OliveGreen, citecolor = OliveGreen, pdfauthor={Didier BONNEL - https://www.maths-cours.fr} } % supprime les bordures autour des liens

\renewcommand{\arg}[0]{\text{arg}}

\everymath{\displaystyle}

%================================================================================================================================
%
% Macros - Commandes
%
%================================================================================================================================

\newcommand\meta[2]{    			% Utilisé pour créer le post HTML.
	\def\titre{titre}
	\def\url{url}
	\def\arg{#1}
	\ifx\titre\arg
		\newcommand\maintitle{#2}
		\fancyhead[L]{#2}
		{\Large\sffamily \MakeUppercase{#2}}
		\vspace{1mm}\textcolor{mcvert}{\hrule}
	\fi 
	\ifx\url\arg
		\fancyfoot[L]{\href{https://www.maths-cours.fr#2}{\black \footnotesize{https://www.maths-cours.fr#2}}}
	\fi 
}


\newcommand\TitreC[1]{    		% Titre centré
     \needspace{3\baselineskip}
     \begin{center}\textbf{#1}\end{center}
}

\newcommand\newpar{    		% paragraphe
     \par
}

\newcommand\nosp {    		% commande vide (pas d'espace)
}
\newcommand{\id}[1]{} %ignore

\newcommand\boite[2]{				% Boite simple sans titre
	\vspace{5mm}
	\setlength{\fboxrule}{0.2mm}
	\setlength{\fboxsep}{5mm}	
	\fcolorbox{#1}{#1!3}{\makebox[\linewidth-2\fboxrule-2\fboxsep]{
  		\begin{minipage}[t]{\linewidth-2\fboxrule-4\fboxsep}\setlength{\parskip}{3mm}
  			 #2
  		\end{minipage}
	}}
	\vspace{5mm}
}

\newcommand\CBox[4]{				% Boites
	\vspace{5mm}
	\setlength{\fboxrule}{0.2mm}
	\setlength{\fboxsep}{5mm}
	
	\fcolorbox{#1}{#1!3}{\makebox[\linewidth-2\fboxrule-2\fboxsep]{
		\begin{minipage}[t]{1cm}\setlength{\parskip}{3mm}
	  		\textcolor{#1}{\LARGE{#2}}    
 	 	\end{minipage}  
  		\begin{minipage}[t]{\linewidth-2\fboxrule-4\fboxsep}\setlength{\parskip}{3mm}
			\raisebox{1.2mm}{\normalsize\sffamily{\textcolor{#1}{#3}}}						
  			 #4
  		\end{minipage}
	}}
	\vspace{5mm}
}

\newcommand\cadre[3]{				% Boites convertible html
	\par
	\vspace{2mm}
	\setlength{\fboxrule}{0.1mm}
	\setlength{\fboxsep}{5mm}
	\fcolorbox{#1}{white}{\makebox[\linewidth-2\fboxrule-2\fboxsep]{
  		\begin{minipage}[t]{\linewidth-2\fboxrule-4\fboxsep}\setlength{\parskip}{3mm}
			\raisebox{-2.5mm}{\sffamily \small{\textcolor{#1}{\MakeUppercase{#2}}}}		
			\par		
  			 #3
 	 		\end{minipage}
	}}
		\vspace{2mm}
	\par
}

\newcommand\bloc[3]{				% Boites convertible html sans bordure
     \needspace{2\baselineskip}
     {\sffamily \small{\textcolor{#1}{\MakeUppercase{#2}}}}    
		\par		
  			 #3
		\par
}

\newcommand\CHelp[1]{
     \CBox{Plum}{\faInfoCircle}{À RETENIR}{#1}
}

\newcommand\CUp[1]{
     \CBox{NavyBlue}{\faThumbsOUp}{EN PRATIQUE}{#1}
}

\newcommand\CInfo[1]{
     \CBox{Sepia}{\faArrowCircleRight}{REMARQUE}{#1}
}

\newcommand\CRedac[1]{
     \CBox{PineGreen}{\faEdit}{BIEN R\'EDIGER}{#1}
}

\newcommand\CError[1]{
     \CBox{Red}{\faExclamationTriangle}{ATTENTION}{#1}
}

\newcommand\TitreExo[2]{
\needspace{4\baselineskip}
 {\sffamily\large EXERCICE #1\ (\emph{#2 points})}
\vspace{5mm}
}

\newcommand\img[2]{
          \includegraphics[width=#2\paperwidth]{\imgdir#1}
}

\newcommand\imgsvg[2]{
       \begin{center}   \includegraphics[width=#2\paperwidth]{\imgsvgdir#1} \end{center}
}


\newcommand\Lien[2]{
     \href{#1}{#2 \tiny \faExternalLink}
}
\newcommand\mcLien[2]{
     \href{https~://www.maths-cours.fr/#1}{#2 \tiny \faExternalLink}
}

\newcommand{\euro}{\eurologo{}}

%================================================================================================================================
%
% Macros - Environement
%
%================================================================================================================================

\newenvironment{tex}{ %
}
{%
}

\newenvironment{indente}{ %
	\setlength\parindent{10mm}
}

{
	\setlength\parindent{0mm}
}

\newenvironment{corrige}{%
     \needspace{3\baselineskip}
     \medskip
     \textbf{\textsc{Corrigé}}
     \medskip
}
{
}

\newenvironment{extern}{%
     \begin{center}
     }
     {
     \end{center}
}

\NewEnviron{code}{%
	\par
     \boite{gray}{\texttt{%
     \BODY
     }}
     \par
}

\newenvironment{vbloc}{% boite sans cadre empeche saut de page
     \begin{minipage}[t]{\linewidth}
     }
     {
     \end{minipage}
}
\NewEnviron{h2}{%
    \needspace{3\baselineskip}
    \vspace{0.6cm}
	\noindent \MakeUppercase{\sffamily \large \BODY}
	\vspace{1mm}\textcolor{mcgris}{\hrule}\vspace{0.4cm}
	\par
}{}

\NewEnviron{h3}{%
    \needspace{3\baselineskip}
	\vspace{5mm}
	\textsc{\BODY}
	\par
}

\NewEnviron{margeneg}{ %
\begin{addmargin}[-1cm]{0cm}
\BODY
\end{addmargin}
}

\NewEnviron{html}{%
}

\begin{document}
\meta{url}{/cours/equations-inequations/}
\meta{pid}{19188}
\meta{titre}{Équations et inéquations}
\meta{type}{cours}
\begin{h2}I. Généralités sur les équations\end{h2}
\cadre{bleu}{Définition}{ % id="d010"
Une  \textbf{équation} à une inconnue est une égalité comprenant une lettre (souvent $ x $) appelée  \textbf{inconnue}.  \og  \textbf{Résoudre}  \fg{} cette équation ( ou  \og trouver les  \textbf{solutions} \fg{} de cette équation ) consiste à déterminer les valeurs de l'inconnue pour laquelle l'égalité est vraie.
} % fin définition

\bloc{orange}{Exemple}{ % id="e010"
Par exemple~:
\begin{center}
$ 3x-15 = 0 $
\end{center}
est une équation d'inconnue $ x $.
\par
L'égalité est vraie si l'on remplace $x$ par $5$ car $ 3  \times 5 - 15 = 0 $.
\\
Elle est fausse si l'on remplace $x$ par une valeur différente de $5$.
\par
On dira donc que l'équation $ 3x-15 = 0 $ admet 5 comme unique solution.
} % fin exemple

\bloc{cyan}{Remarque}{ % id="r010"
Une équation peut n'admettre aucune solution ou admettre une ou plusieurs solution(s). 
} % fin remarque
\begin{h3}Méthode de résolution d'une équation \end{h3}
Pour trouver les solutions d'une équation, on pourrait rechercher  \og au hasard \fg{} les valeurs de $x$ pour lesquelles l'égalité est vraie (et il arrive quelquefois que l'on procède ainsi pour trouver UNE solution).
\par
Toutefois cette façon de procéder possède de graves inconvénients~:

\begin{itemize}
\item
on peut mettre beaucoup de temps avant de trouver une solution...

\item
si l'équation n'a pas de solution, on cherchera longtemps en vain !

\item
ce n'est pas parce que l'on a trouvé une solution qu'il n'y en a pas d'autres...

\end{itemize}
Heureusement, il existe une méthode simple et rigoureuse qui permet de résoudre un grand nombre d'équations. L'idée est de remplacer l'équation par des équations équivalentes (c'est à dire qui ont les mêmes solutions) de plus en plus simple à résoudre. Pour cela, on utilisera les règles ci-dessous.

\cadre{vert}{Propriété}{ % id="p030" 
Si l'on ajoute ou si l'on soustrait un même nombre à chaque membre d'une équation, on obtient une équation équivalente (c'est à dire qui possède les mêmes solutions).
} % fin propriété

\bloc{orange}{Exemple}{ % id="e30"
Les équations~: 

\begin{itemize}
\item
$ 3x-15 = 0$ 
\item
$ 3x-15+1 = 1$ ou $ 3x-14 = 1$ (obtenue en ajoutant 1 à chaque membre de l'égalité précédente)
\item
$ 3x-15+15 = 15$ ou $ 3x = 15$ (obtenue en ajoutant 15 à chaque membre de l'égalité de départ)
\end{itemize}
sont toutes les trois équivalentes. 
} % fin exemple

\cadre{vert}{Propriété}{ % id="p040" 
 Si l'on multiplie ou si l'on divise chaque membre d'une équation par un même nombre \textbf{non nul}, on obtient une équation équivalente.
} % fin propriété

\bloc{orange}{Exemple}{ % id="e40"
Les équations~: 

\begin{itemize}
\item
$ 3x=15$ 
\item
$ 2  \times 3x = 2  \times 15$ ou $ 6x = 30$ (obtenue en multipliant chaque membre de l'égalité précédente par $ 2 $)
\item
$ \frac{ 3x }{ 3 } = \frac{ 15 }{ 3 } $ ou $ x = 5$ (obtenue en divisant chaque membre de l'égalité de départ par $ 3 $)
\end{itemize}
sont équivalentes. Mais la dernière, $ x=5 $, est évidente à résoudre~: sa solution unique est le nombre $5$. 
} % fin exemple


\end{document}

µ
\documentclass[a4paper]{article}

%================================================================================================================================
%
% Packages
%
%================================================================================================================================

\usepackage[T1]{fontenc} 	% pour caractères accentués
\usepackage[utf8]{inputenc}  % encodage utf8
\usepackage[french]{babel}	% langue : français
\usepackage{fourier}			% caractères plus lisibles
\usepackage[dvipsnames]{xcolor} % couleurs
\usepackage{fancyhdr}		% réglage header footer
\usepackage{needspace}		% empêcher sauts de page mal placés
\usepackage{graphicx}		% pour inclure des graphiques
\usepackage{enumitem,cprotect}		% personnalise les listes d'items (nécessaire pour ol, al ...)
\usepackage{hyperref}		% Liens hypertexte
\usepackage{pstricks,pst-all,pst-node,pstricks-add,pst-math,pst-plot,pst-tree,pst-eucl} % pstricks
\usepackage[a4paper,includeheadfoot,top=2cm,left=3cm, bottom=2cm,right=3cm]{geometry} % marges etc.
\usepackage{comment}			% commentaires multilignes
\usepackage{amsmath,environ} % maths (matrices, etc.)
\usepackage{amssymb,makeidx}
\usepackage{bm}				% bold maths
\usepackage{tabularx}		% tableaux
\usepackage{colortbl}		% tableaux en couleur
\usepackage{fontawesome}		% Fontawesome
\usepackage{environ}			% environment with command
\usepackage{fp}				% calculs pour ps-tricks
\usepackage{multido}			% pour ps tricks
\usepackage[np]{numprint}	% formattage nombre
\usepackage{tikz,tkz-tab} 			% package principal TikZ
\usepackage{pgfplots}   % axes
\usepackage{mathrsfs}    % cursives
\usepackage{calc}			% calcul taille boites
\usepackage[scaled=0.875]{helvet} % font sans serif
\usepackage{svg} % svg
\usepackage{scrextend} % local margin
\usepackage{scratch} %scratch
\usepackage{multicol} % colonnes
%\usepackage{infix-RPN,pst-func} % formule en notation polanaise inversée
\usepackage{listings}

%================================================================================================================================
%
% Réglages de base
%
%================================================================================================================================

\lstset{
language=Python,   % R code
literate=
{á}{{\'a}}1
{à}{{\`a}}1
{ã}{{\~a}}1
{é}{{\'e}}1
{è}{{\`e}}1
{ê}{{\^e}}1
{í}{{\'i}}1
{ó}{{\'o}}1
{õ}{{\~o}}1
{ú}{{\'u}}1
{ü}{{\"u}}1
{ç}{{\c{c}}}1
{~}{{ }}1
}


\definecolor{codegreen}{rgb}{0,0.6,0}
\definecolor{codegray}{rgb}{0.5,0.5,0.5}
\definecolor{codepurple}{rgb}{0.58,0,0.82}
\definecolor{backcolour}{rgb}{0.95,0.95,0.92}

\lstdefinestyle{mystyle}{
    backgroundcolor=\color{backcolour},   
    commentstyle=\color{codegreen},
    keywordstyle=\color{magenta},
    numberstyle=\tiny\color{codegray},
    stringstyle=\color{codepurple},
    basicstyle=\ttfamily\footnotesize,
    breakatwhitespace=false,         
    breaklines=true,                 
    captionpos=b,                    
    keepspaces=true,                 
    numbers=left,                    
xleftmargin=2em,
framexleftmargin=2em,            
    showspaces=false,                
    showstringspaces=false,
    showtabs=false,                  
    tabsize=2,
    upquote=true
}

\lstset{style=mystyle}


\lstset{style=mystyle}
\newcommand{\imgdir}{C:/laragon/www/newmc/assets/imgsvg/}
\newcommand{\imgsvgdir}{C:/laragon/www/newmc/assets/imgsvg/}

\definecolor{mcgris}{RGB}{220, 220, 220}% ancien~; pour compatibilité
\definecolor{mcbleu}{RGB}{52, 152, 219}
\definecolor{mcvert}{RGB}{125, 194, 70}
\definecolor{mcmauve}{RGB}{154, 0, 215}
\definecolor{mcorange}{RGB}{255, 96, 0}
\definecolor{mcturquoise}{RGB}{0, 153, 153}
\definecolor{mcrouge}{RGB}{255, 0, 0}
\definecolor{mclightvert}{RGB}{205, 234, 190}

\definecolor{gris}{RGB}{220, 220, 220}
\definecolor{bleu}{RGB}{52, 152, 219}
\definecolor{vert}{RGB}{125, 194, 70}
\definecolor{mauve}{RGB}{154, 0, 215}
\definecolor{orange}{RGB}{255, 96, 0}
\definecolor{turquoise}{RGB}{0, 153, 153}
\definecolor{rouge}{RGB}{255, 0, 0}
\definecolor{lightvert}{RGB}{205, 234, 190}
\setitemize[0]{label=\color{lightvert}  $\bullet$}

\pagestyle{fancy}
\renewcommand{\headrulewidth}{0.2pt}
\fancyhead[L]{maths-cours.fr}
\fancyhead[R]{\thepage}
\renewcommand{\footrulewidth}{0.2pt}
\fancyfoot[C]{}

\newcolumntype{C}{>{\centering\arraybackslash}X}
\newcolumntype{s}{>{\hsize=.35\hsize\arraybackslash}X}

\setlength{\parindent}{0pt}		 
\setlength{\parskip}{3mm}
\setlength{\headheight}{1cm}

\def\ebook{ebook}
\def\book{book}
\def\web{web}
\def\type{web}

\newcommand{\vect}[1]{\overrightarrow{\,\mathstrut#1\,}}

\def\Oij{$\left(\text{O}~;~\vect{\imath},~\vect{\jmath}\right)$}
\def\Oijk{$\left(\text{O}~;~\vect{\imath},~\vect{\jmath},~\vect{k}\right)$}
\def\Ouv{$\left(\text{O}~;~\vect{u},~\vect{v}\right)$}

\hypersetup{breaklinks=true, colorlinks = true, linkcolor = OliveGreen, urlcolor = OliveGreen, citecolor = OliveGreen, pdfauthor={Didier BONNEL - https://www.maths-cours.fr} } % supprime les bordures autour des liens

\renewcommand{\arg}[0]{\text{arg}}

\everymath{\displaystyle}

%================================================================================================================================
%
% Macros - Commandes
%
%================================================================================================================================

\newcommand\meta[2]{    			% Utilisé pour créer le post HTML.
	\def\titre{titre}
	\def\url{url}
	\def\arg{#1}
	\ifx\titre\arg
		\newcommand\maintitle{#2}
		\fancyhead[L]{#2}
		{\Large\sffamily \MakeUppercase{#2}}
		\vspace{1mm}\textcolor{mcvert}{\hrule}
	\fi 
	\ifx\url\arg
		\fancyfoot[L]{\href{https://www.maths-cours.fr#2}{\black \footnotesize{https://www.maths-cours.fr#2}}}
	\fi 
}


\newcommand\TitreC[1]{    		% Titre centré
     \needspace{3\baselineskip}
     \begin{center}\textbf{#1}\end{center}
}

\newcommand\newpar{    		% paragraphe
     \par
}

\newcommand\nosp {    		% commande vide (pas d'espace)
}
\newcommand{\id}[1]{} %ignore

\newcommand\boite[2]{				% Boite simple sans titre
	\vspace{5mm}
	\setlength{\fboxrule}{0.2mm}
	\setlength{\fboxsep}{5mm}	
	\fcolorbox{#1}{#1!3}{\makebox[\linewidth-2\fboxrule-2\fboxsep]{
  		\begin{minipage}[t]{\linewidth-2\fboxrule-4\fboxsep}\setlength{\parskip}{3mm}
  			 #2
  		\end{minipage}
	}}
	\vspace{5mm}
}

\newcommand\CBox[4]{				% Boites
	\vspace{5mm}
	\setlength{\fboxrule}{0.2mm}
	\setlength{\fboxsep}{5mm}
	
	\fcolorbox{#1}{#1!3}{\makebox[\linewidth-2\fboxrule-2\fboxsep]{
		\begin{minipage}[t]{1cm}\setlength{\parskip}{3mm}
	  		\textcolor{#1}{\LARGE{#2}}    
 	 	\end{minipage}  
  		\begin{minipage}[t]{\linewidth-2\fboxrule-4\fboxsep}\setlength{\parskip}{3mm}
			\raisebox{1.2mm}{\normalsize\sffamily{\textcolor{#1}{#3}}}						
  			 #4
  		\end{minipage}
	}}
	\vspace{5mm}
}

\newcommand\cadre[3]{				% Boites convertible html
	\par
	\vspace{2mm}
	\setlength{\fboxrule}{0.1mm}
	\setlength{\fboxsep}{5mm}
	\fcolorbox{#1}{white}{\makebox[\linewidth-2\fboxrule-2\fboxsep]{
  		\begin{minipage}[t]{\linewidth-2\fboxrule-4\fboxsep}\setlength{\parskip}{3mm}
			\raisebox{-2.5mm}{\sffamily \small{\textcolor{#1}{\MakeUppercase{#2}}}}		
			\par		
  			 #3
 	 		\end{minipage}
	}}
		\vspace{2mm}
	\par
}

\newcommand\bloc[3]{				% Boites convertible html sans bordure
     \needspace{2\baselineskip}
     {\sffamily \small{\textcolor{#1}{\MakeUppercase{#2}}}}    
		\par		
  			 #3
		\par
}

\newcommand\CHelp[1]{
     \CBox{Plum}{\faInfoCircle}{À RETENIR}{#1}
}

\newcommand\CUp[1]{
     \CBox{NavyBlue}{\faThumbsOUp}{EN PRATIQUE}{#1}
}

\newcommand\CInfo[1]{
     \CBox{Sepia}{\faArrowCircleRight}{REMARQUE}{#1}
}

\newcommand\CRedac[1]{
     \CBox{PineGreen}{\faEdit}{BIEN R\'EDIGER}{#1}
}

\newcommand\CError[1]{
     \CBox{Red}{\faExclamationTriangle}{ATTENTION}{#1}
}

\newcommand\TitreExo[2]{
\needspace{4\baselineskip}
 {\sffamily\large EXERCICE #1\ (\emph{#2 points})}
\vspace{5mm}
}

\newcommand\img[2]{
          \includegraphics[width=#2\paperwidth]{\imgdir#1}
}

\newcommand\imgsvg[2]{
       \begin{center}   \includegraphics[width=#2\paperwidth]{\imgsvgdir#1} \end{center}
}


\newcommand\Lien[2]{
     \href{#1}{#2 \tiny \faExternalLink}
}
\newcommand\mcLien[2]{
     \href{https~://www.maths-cours.fr/#1}{#2 \tiny \faExternalLink}
}

\newcommand{\euro}{\eurologo{}}

%================================================================================================================================
%
% Macros - Environement
%
%================================================================================================================================

\newenvironment{tex}{ %
}
{%
}

\newenvironment{indente}{ %
	\setlength\parindent{10mm}
}

{
	\setlength\parindent{0mm}
}

\newenvironment{corrige}{%
     \needspace{3\baselineskip}
     \medskip
     \textbf{\textsc{Corrigé}}
     \medskip
}
{
}

\newenvironment{extern}{%
     \begin{center}
     }
     {
     \end{center}
}

\NewEnviron{code}{%
	\par
     \boite{gray}{\texttt{%
     \BODY
     }}
     \par
}

\newenvironment{vbloc}{% boite sans cadre empeche saut de page
     \begin{minipage}[t]{\linewidth}
     }
     {
     \end{minipage}
}
\NewEnviron{h2}{%
    \needspace{3\baselineskip}
    \vspace{0.6cm}
	\noindent \MakeUppercase{\sffamily \large \BODY}
	\vspace{1mm}\textcolor{mcgris}{\hrule}\vspace{0.4cm}
	\par
}{}

\NewEnviron{h3}{%
    \needspace{3\baselineskip}
	\vspace{5mm}
	\textsc{\BODY}
	\par
}

\NewEnviron{margeneg}{ %
\begin{addmargin}[-1cm]{0cm}
\BODY
\end{addmargin}
}

\NewEnviron{html}{%
}

\begin{document}
\meta{url}{/supplement/deux-methodes-de-calcul-du-pgcd/}
\meta{pid}{19957}
\meta{titre}{Deux méthodes de calcul du PGCD}
\meta{type}{supplement}
%
\begin{h2} 1 - Méthode des soustractions successives \end{h2}
Cette méthode se base sur le résultat suivant~:
\cadre{rouge}{Théorème}{% id="t60"
     Soient $a$ et $b$ sont deux entiers naturels non nuls avec $a > b$ alors $PGCD\left(a~; b\right) = PGCD\left(b~; a-b\right)$.
}
\bloc{cyan}{Remarque}{% id="r60"
     L'intérêt de ce théorème est qu'il remplace le calcul du PGCD de deux nombres par le calcul du PGCD de deux nombres plus petits. On peut ainsi calculer un PGCD utilisant plusieurs fois de suite ce théorème (méthode des \textbf{soustractions successives}).
}
\bloc{orange}{Exemple}{% id="e60"
     Calculons le PGCD de $600$ et $315$ par la méthode des soustractions successives.
     \par
     $600-315 =285$ donc $PGCD\left(600~; 315\right)=PGCD\left(315~; 285\right)$
     \par
     $315-285 =30$ donc $PGCD\left(315~; 285\right)=PGCD\left(285~; 30\right)$
     \par
     $285-30 =255$ donc $PGCD\left(285~; 30\right)=PGCD\left(255~; 30\right)$
     \par
     $255-30 =225$ donc $PGCD\left(255~; 30\right)=PGCD\left(225~; 30\right)$
     \par
     $225-30 =195$ donc $PGCD\left(225~; 30\right)=PGCD\left(195~; 30\right)$
     \par
     $195-30 =165$ donc $PGCD\left(195~; 30\right)=PGCD\left(165~; 30\right)$
     \par
     $165-30 =135$ donc $PGCD\left(165~; 30\right)=PGCD\left(135~; 30\right)$
     \par
     $135-30 =105$ donc $PGCD\left(135~; 30\right)=PGCD\left(105~; 30\right)$
     \par
     $105-30 =75$ donc $PGCD\left(105~; 30\right)=PGCD\left(75~; 30\right)$
     \par
     $75-30 =45$ donc $PGCD\left(75~; 30\right)=PGCD\left(45~; 30\right)$
     \par
     $45-30 =15$ donc $PGCD\left(45~; 30\right)=PGCD\left(15~; 30\right)$
     \par
     $30-15 =15$ donc $PGCD\left(30~; 15\right)=PGCD\left(15~; 15\right)$
     \par
     Or évidemment $PGCD\left(15~; 15\right) = 15$ donc finalement~:
     \par
     $PGCD\left(600~; 315\right)=15$.
} % fin exemple
\bloc{cyan}{Remarque}{ % id=r65
     On voit que cette méthode est assez simple au niveau calcul (elle ne requiert que des soustractions) mais elle peut dans certains cas s'avérer assez longue. Une méthode plus rapide est \textbf{l'algorithme d'Euclide.}
} % fin remarque
\begin{h2}2 - Algorithme d'Euclide \end{h2}
L'algorithme d'Euclide utilise le théorème suivant~:
\cadre{rouge}{Théorème}{% id="t70"
     Soient $a$ et $b$ sont deux entiers naturels non nuls avec $a > b$ et $r$ le reste de la division euclidienne de $a$ par $b$.
     \par
     Si $r\neq 0$ alors $PGCD\left(a~; b\right) = PGCD\left(b~; r\right)$.
}
\bloc{cyan}{Remarques}{% id="r70"
     \begin{itemize}
          \item Si $r=0$, cela signifie que $b$ divise $a$ donc dans ce cas $PGCD\left(a;b\right)=b$
          \item Comme le précédent, ce théorème remplace le calcul d'un PGCD par le calcul du PGCD de deux nombres plus petits.
          \par
          On peut, là encore, calculer un PGCD utilisant plusieurs fois de suite ce théorème. cette méthode s'appelle \textbf{l'algorithme d'Euclide} (ou la méthode des \textbf{divisions successives}).
     \end{itemize}
}
\bloc{orange}{Exemple}{% id="e70"
     Calculons, à nouveau, le PGCD de $600$ et $315$ mais avec l'algorithme d'Euclide cette fois~:
     \par
     $600$ \img{touche-divise}{0.008}%width="8" alt="touche division euclidienne"
     $ 315$~: $ Q=1$ et $R=285$ donc $PGCD\left(600~; 315\right)=PGCD\left(315~; 285\right)$
     \par
     $315$ \img{touche-divise}{0.008}%width="8" alt="touche division euclidienne"
     $285$~: $ Q=1$ et $R=30$ donc $PGCD\left(315~; 285\right)=PGCD\left(285~; 30\right)$
     \par
     $285 $ \img{touche-divise}{0.008}%width="8" alt="touche division euclidienne"
     $30$~: $ Q=9$ et $R=15$ donc $PGCD\left(285~; 30\right)=PGCD\left(30~; 15\right)$
     \par
     $30 $ \img{touche-divise}{0.008}%width="8" alt="touche division euclidienne"
     $15$~: $ Q=2$ et $R=0$
     \par
     On s'arrête car le reste est nul.
     \par
     Le PGCD est \textbf{le dernier reste non nul} donc~:
     \par
     $PGCD\left(600~; 315\right)=15$.
} % fin exemple
\bloc{cyan}{Remarque}{ % id=r80
     Si vous souhaitez vous entraîner à cette méthode, vous pouvez vérifier votre calcul et votre résultat grâce à cet outil~: \mcLien{/solver/algorithme-euclide/}{PGCD Algorithme d'Euclide}
} % fin remarque@

\end{document}
µ
\documentclass[a4paper]{article}

%================================================================================================================================
%
% Packages
%
%================================================================================================================================

\usepackage[T1]{fontenc} 	% pour caractères accentués
\usepackage[utf8]{inputenc}  % encodage utf8
\usepackage[french]{babel}	% langue : français
\usepackage{fourier}			% caractères plus lisibles
\usepackage[dvipsnames]{xcolor} % couleurs
\usepackage{fancyhdr}		% réglage header footer
\usepackage{needspace}		% empêcher sauts de page mal placés
\usepackage{graphicx}		% pour inclure des graphiques
\usepackage{enumitem,cprotect}		% personnalise les listes d'items (nécessaire pour ol, al ...)
\usepackage{hyperref}		% Liens hypertexte
\usepackage{pstricks,pst-all,pst-node,pstricks-add,pst-math,pst-plot,pst-tree,pst-eucl} % pstricks
\usepackage[a4paper,includeheadfoot,top=2cm,left=3cm, bottom=2cm,right=3cm]{geometry} % marges etc.
\usepackage{comment}			% commentaires multilignes
\usepackage{amsmath,environ} % maths (matrices, etc.)
\usepackage{amssymb,makeidx}
\usepackage{bm}				% bold maths
\usepackage{tabularx}		% tableaux
\usepackage{colortbl}		% tableaux en couleur
\usepackage{fontawesome}		% Fontawesome
\usepackage{environ}			% environment with command
\usepackage{fp}				% calculs pour ps-tricks
\usepackage{multido}			% pour ps tricks
\usepackage[np]{numprint}	% formattage nombre
\usepackage{tikz,tkz-tab} 			% package principal TikZ
\usepackage{pgfplots}   % axes
\usepackage{mathrsfs}    % cursives
\usepackage{calc}			% calcul taille boites
\usepackage[scaled=0.875]{helvet} % font sans serif
\usepackage{svg} % svg
\usepackage{scrextend} % local margin
\usepackage{scratch} %scratch
\usepackage{multicol} % colonnes
%\usepackage{infix-RPN,pst-func} % formule en notation polanaise inversée
\usepackage{listings}

%================================================================================================================================
%
% Réglages de base
%
%================================================================================================================================

\lstset{
language=Python,   % R code
literate=
{á}{{\'a}}1
{à}{{\`a}}1
{ã}{{\~a}}1
{é}{{\'e}}1
{è}{{\`e}}1
{ê}{{\^e}}1
{í}{{\'i}}1
{ó}{{\'o}}1
{õ}{{\~o}}1
{ú}{{\'u}}1
{ü}{{\"u}}1
{ç}{{\c{c}}}1
{~}{{ }}1
}


\definecolor{codegreen}{rgb}{0,0.6,0}
\definecolor{codegray}{rgb}{0.5,0.5,0.5}
\definecolor{codepurple}{rgb}{0.58,0,0.82}
\definecolor{backcolour}{rgb}{0.95,0.95,0.92}

\lstdefinestyle{mystyle}{
    backgroundcolor=\color{backcolour},   
    commentstyle=\color{codegreen},
    keywordstyle=\color{magenta},
    numberstyle=\tiny\color{codegray},
    stringstyle=\color{codepurple},
    basicstyle=\ttfamily\footnotesize,
    breakatwhitespace=false,         
    breaklines=true,                 
    captionpos=b,                    
    keepspaces=true,                 
    numbers=left,                    
xleftmargin=2em,
framexleftmargin=2em,            
    showspaces=false,                
    showstringspaces=false,
    showtabs=false,                  
    tabsize=2,
    upquote=true
}

\lstset{style=mystyle}


\lstset{style=mystyle}
\newcommand{\imgdir}{C:/laragon/www/newmc/assets/imgsvg/}
\newcommand{\imgsvgdir}{C:/laragon/www/newmc/assets/imgsvg/}

\definecolor{mcgris}{RGB}{220, 220, 220}% ancien~; pour compatibilité
\definecolor{mcbleu}{RGB}{52, 152, 219}
\definecolor{mcvert}{RGB}{125, 194, 70}
\definecolor{mcmauve}{RGB}{154, 0, 215}
\definecolor{mcorange}{RGB}{255, 96, 0}
\definecolor{mcturquoise}{RGB}{0, 153, 153}
\definecolor{mcrouge}{RGB}{255, 0, 0}
\definecolor{mclightvert}{RGB}{205, 234, 190}

\definecolor{gris}{RGB}{220, 220, 220}
\definecolor{bleu}{RGB}{52, 152, 219}
\definecolor{vert}{RGB}{125, 194, 70}
\definecolor{mauve}{RGB}{154, 0, 215}
\definecolor{orange}{RGB}{255, 96, 0}
\definecolor{turquoise}{RGB}{0, 153, 153}
\definecolor{rouge}{RGB}{255, 0, 0}
\definecolor{lightvert}{RGB}{205, 234, 190}
\setitemize[0]{label=\color{lightvert}  $\bullet$}

\pagestyle{fancy}
\renewcommand{\headrulewidth}{0.2pt}
\fancyhead[L]{maths-cours.fr}
\fancyhead[R]{\thepage}
\renewcommand{\footrulewidth}{0.2pt}
\fancyfoot[C]{}

\newcolumntype{C}{>{\centering\arraybackslash}X}
\newcolumntype{s}{>{\hsize=.35\hsize\arraybackslash}X}

\setlength{\parindent}{0pt}		 
\setlength{\parskip}{3mm}
\setlength{\headheight}{1cm}

\def\ebook{ebook}
\def\book{book}
\def\web{web}
\def\type{web}

\newcommand{\vect}[1]{\overrightarrow{\,\mathstrut#1\,}}

\def\Oij{$\left(\text{O}~;~\vect{\imath},~\vect{\jmath}\right)$}
\def\Oijk{$\left(\text{O}~;~\vect{\imath},~\vect{\jmath},~\vect{k}\right)$}
\def\Ouv{$\left(\text{O}~;~\vect{u},~\vect{v}\right)$}

\hypersetup{breaklinks=true, colorlinks = true, linkcolor = OliveGreen, urlcolor = OliveGreen, citecolor = OliveGreen, pdfauthor={Didier BONNEL - https://www.maths-cours.fr} } % supprime les bordures autour des liens

\renewcommand{\arg}[0]{\text{arg}}

\everymath{\displaystyle}

%================================================================================================================================
%
% Macros - Commandes
%
%================================================================================================================================

\newcommand\meta[2]{    			% Utilisé pour créer le post HTML.
	\def\titre{titre}
	\def\url{url}
	\def\arg{#1}
	\ifx\titre\arg
		\newcommand\maintitle{#2}
		\fancyhead[L]{#2}
		{\Large\sffamily \MakeUppercase{#2}}
		\vspace{1mm}\textcolor{mcvert}{\hrule}
	\fi 
	\ifx\url\arg
		\fancyfoot[L]{\href{https://www.maths-cours.fr#2}{\black \footnotesize{https://www.maths-cours.fr#2}}}
	\fi 
}


\newcommand\TitreC[1]{    		% Titre centré
     \needspace{3\baselineskip}
     \begin{center}\textbf{#1}\end{center}
}

\newcommand\newpar{    		% paragraphe
     \par
}

\newcommand\nosp {    		% commande vide (pas d'espace)
}
\newcommand{\id}[1]{} %ignore

\newcommand\boite[2]{				% Boite simple sans titre
	\vspace{5mm}
	\setlength{\fboxrule}{0.2mm}
	\setlength{\fboxsep}{5mm}	
	\fcolorbox{#1}{#1!3}{\makebox[\linewidth-2\fboxrule-2\fboxsep]{
  		\begin{minipage}[t]{\linewidth-2\fboxrule-4\fboxsep}\setlength{\parskip}{3mm}
  			 #2
  		\end{minipage}
	}}
	\vspace{5mm}
}

\newcommand\CBox[4]{				% Boites
	\vspace{5mm}
	\setlength{\fboxrule}{0.2mm}
	\setlength{\fboxsep}{5mm}
	
	\fcolorbox{#1}{#1!3}{\makebox[\linewidth-2\fboxrule-2\fboxsep]{
		\begin{minipage}[t]{1cm}\setlength{\parskip}{3mm}
	  		\textcolor{#1}{\LARGE{#2}}    
 	 	\end{minipage}  
  		\begin{minipage}[t]{\linewidth-2\fboxrule-4\fboxsep}\setlength{\parskip}{3mm}
			\raisebox{1.2mm}{\normalsize\sffamily{\textcolor{#1}{#3}}}						
  			 #4
  		\end{minipage}
	}}
	\vspace{5mm}
}

\newcommand\cadre[3]{				% Boites convertible html
	\par
	\vspace{2mm}
	\setlength{\fboxrule}{0.1mm}
	\setlength{\fboxsep}{5mm}
	\fcolorbox{#1}{white}{\makebox[\linewidth-2\fboxrule-2\fboxsep]{
  		\begin{minipage}[t]{\linewidth-2\fboxrule-4\fboxsep}\setlength{\parskip}{3mm}
			\raisebox{-2.5mm}{\sffamily \small{\textcolor{#1}{\MakeUppercase{#2}}}}		
			\par		
  			 #3
 	 		\end{minipage}
	}}
		\vspace{2mm}
	\par
}

\newcommand\bloc[3]{				% Boites convertible html sans bordure
     \needspace{2\baselineskip}
     {\sffamily \small{\textcolor{#1}{\MakeUppercase{#2}}}}    
		\par		
  			 #3
		\par
}

\newcommand\CHelp[1]{
     \CBox{Plum}{\faInfoCircle}{À RETENIR}{#1}
}

\newcommand\CUp[1]{
     \CBox{NavyBlue}{\faThumbsOUp}{EN PRATIQUE}{#1}
}

\newcommand\CInfo[1]{
     \CBox{Sepia}{\faArrowCircleRight}{REMARQUE}{#1}
}

\newcommand\CRedac[1]{
     \CBox{PineGreen}{\faEdit}{BIEN R\'EDIGER}{#1}
}

\newcommand\CError[1]{
     \CBox{Red}{\faExclamationTriangle}{ATTENTION}{#1}
}

\newcommand\TitreExo[2]{
\needspace{4\baselineskip}
 {\sffamily\large EXERCICE #1\ (\emph{#2 points})}
\vspace{5mm}
}

\newcommand\img[2]{
          \includegraphics[width=#2\paperwidth]{\imgdir#1}
}

\newcommand\imgsvg[2]{
       \begin{center}   \includegraphics[width=#2\paperwidth]{\imgsvgdir#1} \end{center}
}


\newcommand\Lien[2]{
     \href{#1}{#2 \tiny \faExternalLink}
}
\newcommand\mcLien[2]{
     \href{https~://www.maths-cours.fr/#1}{#2 \tiny \faExternalLink}
}

\newcommand{\euro}{\eurologo{}}

%================================================================================================================================
%
% Macros - Environement
%
%================================================================================================================================

\newenvironment{tex}{ %
}
{%
}

\newenvironment{indente}{ %
	\setlength\parindent{10mm}
}

{
	\setlength\parindent{0mm}
}

\newenvironment{corrige}{%
     \needspace{3\baselineskip}
     \medskip
     \textbf{\textsc{Corrigé}}
     \medskip
}
{
}

\newenvironment{extern}{%
     \begin{center}
     }
     {
     \end{center}
}

\NewEnviron{code}{%
	\par
     \boite{gray}{\texttt{%
     \BODY
     }}
     \par
}

\newenvironment{vbloc}{% boite sans cadre empeche saut de page
     \begin{minipage}[t]{\linewidth}
     }
     {
     \end{minipage}
}
\NewEnviron{h2}{%
    \needspace{3\baselineskip}
    \vspace{0.6cm}
	\noindent \MakeUppercase{\sffamily \large \BODY}
	\vspace{1mm}\textcolor{mcgris}{\hrule}\vspace{0.4cm}
	\par
}{}

\NewEnviron{h3}{%
    \needspace{3\baselineskip}
	\vspace{5mm}
	\textsc{\BODY}
	\par
}

\NewEnviron{margeneg}{ %
\begin{addmargin}[-1cm]{0cm}
\BODY
\end{addmargin}
}

\NewEnviron{html}{%
}

\begin{document}
\meta{url}{/methode/decomposer-un-entier-en-produit-de-facteurs-premiers/}
\meta{pid}{19981}
\meta{titre}{Décomposer un entier en produit de facteurs premiers}
\meta{type}{methode}
%
\cadre{rouge}{Méthode}{ % id=m010
     \begin{itemize}
          \item
          Pour décomposer un entier naturel en produits de facteurs premiers, on essaie de le diviser par les nombres premiers en allant du plus petit au plus grand~: 2, 3, 5, 7, 11, etc.
          \item
          On présente souvent les calculs en deux colonnes~: la colonne de droite contient les nombres premiers et la colonne de gauche, les quotients successifs.
          \item
          Si pour un entier $n$ on n'a trouvé aucun diviseur premier inférieur ou égal à $ \sqrt{ n } $, on peut arrêter la recherche. Le nombre $n$ est alors premier~; son seul diviseur premier est alors $n$ lui-même.
     \end{itemize}
} % fin Méthode
\bloc{orange}{Exemple détaillé}{ % id=e20
     \begin{h3} Décomposition de 4440 en produit de facteurs premiers~: \end{h3}
     \begin{itemize}
          \item
          Première étape~:
          \par
          On trace un barre verticale pour former deux colonnes et on place le nombre à décomposer dans la colonne de gauche.
          \begin{center}
               \begin{extern}%alt="Décomposition facteurs premiers - étape 0" style="width:10rem"
                    \newrgbcolor{tttttt}{0.2 0.2 0.2}
                    \psset{xunit=1.0cm,yunit=1.0cm,algebraic=true,dimen=middle,dotstyle=o,dotsize=5pt 0,linewidth=.3pt,arrowsize=3pt 2,arrowinset=0.25}
                    \begin{pspicture*}(2.47,4.7)(3.48,5.00)
                         \fontsize{4}{4}
                         \psline[linewidth=0.3pt,linecolor=tttttt](3,4.7)(3.,5.)
                         \rput[tl](2.6,4.9){$\tttttt{4440}$}
                         \rput[tl](3.12,4.9){$\white{2} $ }
                    \end{pspicture*}
               \end{extern}
          \end{center}
          \item
          Deuxième étape~:
          \par
          On cherche si 4440 est divisible par 2. C'est le cas ici (4440 se termine par un chiffre pair).
          \par
          On inscrit donc le nombre 2 dans la colonne de droite et le quotient de 4440 par 2 (soit 2220) sous 4440 dans la colonne de gauche~:
          \begin{center}
               \begin{extern}%alt="Décomposition facteurs premiers - étape 1" style="width:10rem"
                    \newrgbcolor{tttttt}{0.2 0.2 0.2}
                    \psset{xunit=1.0cm,yunit=1.0cm,algebraic=true,dimen=middle,dotstyle=o,dotsize=5pt 0,linewidth=.3pt,arrowsize=3pt 2,arrowinset=0.25}
                    \begin{pspicture*}(2.47,4.5)(3.48,5.00)
                         \fontsize{4}{4}
                         \psline[linewidth=0.3pt,linecolor=tttttt](3,4.5)(3.,5.)
                         \rput[tl](2.6,4.9){$\tttttt{4440}$}
                         \rput[tl](2.6,4.7){$\tttttt{2220}$}
                         \rput[tl](3.12,4.9){$\tttttt{2}$}
                    \end{pspicture*}
               \end{extern}
          \end{center}
          \item
          Troisième étape~:
          \par
          On recommence le procédé pour 2220 qui est divisible par 2 et donne 1110 comme quotient puis pour 1110 qui est aussi divisible par 2 et donne le quotient 555~:
          \begin{center}
               \begin{extern}%alt="Décomposition facteurs premiers - étape 2" style="width:10rem"
                    \newrgbcolor{tttttt}{0.2 0.2 0.2}
                    \psset{xunit=1.0cm,yunit=1.0cm,algebraic=true,dimen=middle,dotstyle=o,dotsize=5pt 0,linewidth=.3pt,arrowsize=3pt 2,arrowinset=0.25}
                    \begin{pspicture*}(2.47,4.1)(3.48,5.00)
                         \fontsize{4}{4}
                         \psline[linewidth=0.3pt,linecolor=tttttt](3,4.1)(3.,5.)
                         \rput[tl](2.6,4.9){$\tttttt{4440}$}
                         \rput[tl](2.6,4.7){$\tttttt{2220}$}
                         \rput[tl](2.6,4.5){$\tttttt{1110}$}
                         \rput[tl](2.65,4.3){$\tttttt{555}$}
                         \rput[tl](3.12,4.5){$\tttttt{2}$}
                         \rput[tl](3.12,4.7){$\tttttt{2}$}
                         \rput[tl](3.12,4.9){$\tttttt{2}$}
                    \end{pspicture*}
               \end{extern}
          \end{center}
          \item
          Quatrième étape~:
          \par
          555 est impair donc n'est pas divisible par 2. On essaie alors de le diviser par le nombre premier qui suit 2 c'est à dire 3. 555 est divisible par 3 (la somme des chiffres vaut 15). Le quotient est égal à 185~:
          \begin{center}
               \begin{extern}%alt="Décomposition facteurs premiers - étape 3" style="width:10rem"
                    \newrgbcolor{tttttt}{0.2 0.2 0.2}
                    \psset{xunit=1.0cm,yunit=1.0cm,algebraic=true,dimen=middle,dotstyle=o,dotsize=5pt 0,linewidth=.3pt,arrowsize=3pt 2,arrowinset=0.25}
                    \begin{pspicture*}(2.47,3.9)(3.48,5.00)
                         \fontsize{4}{4}
                         \psline[linewidth=0.3pt,linecolor=tttttt](3.,3.9)(3,5)
                         \rput[tl](2.6,4.9){$\tttttt{4440}$}
                         \rput[tl](2.6,4.7){$\tttttt{2220}$}
                         \rput[tl](2.6,4.5){$\tttttt{1110}$}
                         \rput[tl](2.65,4.3){$\tttttt{555}$}
                         \rput[tl](2.65,4.1){$\tttttt{185}$}
                         \rput[tl](3.12,4.3){$\tttttt{3}$}
                         \rput[tl](3.12,4.5){$\tttttt{2}$}
                         \rput[tl](3.12,4.7){$\tttttt{2}$}
                         \rput[tl](3.12,4.9){$\tttttt{2}$}
                    \end{pspicture*}
               \end{extern}
          \end{center}
          \item
          Cinquième étape~:
          \par
          185 n'est pas divisible par 3 (1+8+5=14). Il est, par contre, divisible par 5 (le chiffre des unités est 5). Le quotient vaut alors 37~:
          \begin{center}
               \begin{extern}%alt="Décomposition facteurs premiers - étape 4" style="width:10rem"
                    \newrgbcolor{tttttt}{0.2 0.2 0.2}
                    \psset{xunit=1.0cm,yunit=1.0cm,algebraic=true,dimen=middle,dotstyle=o,dotsize=5pt 0,linewidth=.3pt,arrowsize=3pt 2,arrowinset=0.25}
                    \begin{pspicture*}(2.47,3.7)(3.48,5.00)
                         \fontsize{4}{4}
                         \psline[linewidth=0.3pt,linecolor=tttttt](3.,5.)(3,3.7)
                         \rput[tl](2.6,4.9){$\tttttt{4440}$}
                         \rput[tl](2.6,4.7){$\tttttt{2220}$}
                         \rput[tl](2.6,4.5){$\tttttt{1110}$}
                         \rput[tl](2.65,4.3){$\tttttt{555}$}
                         \rput[tl](2.65,4.1){$\tttttt{185}$}
                         \rput[tl](2.7,3.9){$\tttttt{37}$}
                         \rput[tl](3.12,4.1){$\tttttt{5}$}
                         \rput[tl](3.12,4.3){$\tttttt{3}$}
                         \rput[tl](3.12,4.5){$\tttttt{2}$}
                         \rput[tl](3.12,4.7){$\tttttt{2}$}
                         \rput[tl](3.12,4.9){$\tttttt{2}$}
                    \end{pspicture*}
               \end{extern}
          \end{center}
          \item
          Sixième étape~:
          \par
          37 n'est pas divisible par 5. Comme $ \sqrt{ 37 } \approx 6,08 $, ce n'est pas la peine d'essayer de diviser par 7 (qui est supérieur à 6,08) ou par des nombres supérieurs. Par conséquent, 37 est un nombre premier et le dernier facteur premier est donc 37.
          \par
          Le quotient est alors 1 et le calcul est terminé~:
          \begin{center}
               \begin{extern}%alt="Décomposition facteurs premiers - étape finale" style="width:10rem"
                    \newrgbcolor{tttttt}{0.2 0.2 0.2}
                    \psset{xunit=1.0cm,yunit=1.0cm,algebraic=true,dimen=middle,dotstyle=o,dotsize=5pt 0,linewidth=.3pt,arrowsize=3pt 2,arrowinset=0.25}
                    \begin{pspicture*}(2.47,3.5)(3.48,5.00)
                         \fontsize{4}{4}
                         \psline[linewidth=0.3pt,linecolor=tttttt](3.,5.)(3,3.5)
                         \rput[tl](2.6,4.9){$\tttttt{4440}$}
                         \rput[tl](2.6,4.7){$\tttttt{2220}$}
                         \rput[tl](2.6,4.5){$\tttttt{1110}$}
                         \rput[tl](2.65,4.3){$\tttttt{555}$}
                         \rput[tl](2.65,4.1){$\tttttt{185}$}
                         \rput[tl](2.7,3.9){$\tttttt{37}$}
                         \rput[tl](2.77,3.7){$\tttttt{1}$}
                         \rput[tl](3.07,3.9){$\tttttt{37}$}
                         \rput[tl](3.12,4.1){$\tttttt{5}$}
                         \rput[tl](3.12,4.3){$\tttttt{3}$}
                         \rput[tl](3.12,4.5){$\tttttt{2}$}
                         \rput[tl](3.12,4.7){$\tttttt{2}$}
                         \rput[tl](3.12,4.9){$\tttttt{2}$}
                    \end{pspicture*}
               \end{extern}
          \end{center}
          \item
          Conclusion~:
          On obtient la décomposition suivante~:
          \begin{center}
               $4440 = 2 \times 2 \times 2 \times 3 \times 5 \times 37 $\nosp$ = 2^3 \times 3 \times 5 \times 37$
          \end{center}
     \end{itemize}
} % fin exemple

\end{document}
µ
\documentclass[a4paper]{article}

%================================================================================================================================
%
% Packages
%
%================================================================================================================================

\usepackage[T1]{fontenc} 	% pour caractères accentués
\usepackage[utf8]{inputenc}  % encodage utf8
\usepackage[french]{babel}	% langue : français
\usepackage{fourier}			% caractères plus lisibles
\usepackage[dvipsnames]{xcolor} % couleurs
\usepackage{fancyhdr}		% réglage header footer
\usepackage{needspace}		% empêcher sauts de page mal placés
\usepackage{graphicx}		% pour inclure des graphiques
\usepackage{enumitem,cprotect}		% personnalise les listes d'items (nécessaire pour ol, al ...)
\usepackage{hyperref}		% Liens hypertexte
\usepackage{pstricks,pst-all,pst-node,pstricks-add,pst-math,pst-plot,pst-tree,pst-eucl} % pstricks
\usepackage[a4paper,includeheadfoot,top=2cm,left=3cm, bottom=2cm,right=3cm]{geometry} % marges etc.
\usepackage{comment}			% commentaires multilignes
\usepackage{amsmath,environ} % maths (matrices, etc.)
\usepackage{amssymb,makeidx}
\usepackage{bm}				% bold maths
\usepackage{tabularx}		% tableaux
\usepackage{colortbl}		% tableaux en couleur
\usepackage{fontawesome}		% Fontawesome
\usepackage{environ}			% environment with command
\usepackage{fp}				% calculs pour ps-tricks
\usepackage{multido}			% pour ps tricks
\usepackage[np]{numprint}	% formattage nombre
\usepackage{tikz,tkz-tab} 			% package principal TikZ
\usepackage{pgfplots}   % axes
\usepackage{mathrsfs}    % cursives
\usepackage{calc}			% calcul taille boites
\usepackage[scaled=0.875]{helvet} % font sans serif
\usepackage{svg} % svg
\usepackage{scrextend} % local margin
\usepackage{scratch} %scratch
\usepackage{multicol} % colonnes
%\usepackage{infix-RPN,pst-func} % formule en notation polanaise inversée
\usepackage{listings}

%================================================================================================================================
%
% Réglages de base
%
%================================================================================================================================

\lstset{
language=Python,   % R code
literate=
{á}{{\'a}}1
{à}{{\`a}}1
{ã}{{\~a}}1
{é}{{\'e}}1
{è}{{\`e}}1
{ê}{{\^e}}1
{í}{{\'i}}1
{ó}{{\'o}}1
{õ}{{\~o}}1
{ú}{{\'u}}1
{ü}{{\"u}}1
{ç}{{\c{c}}}1
{~}{{ }}1
}


\definecolor{codegreen}{rgb}{0,0.6,0}
\definecolor{codegray}{rgb}{0.5,0.5,0.5}
\definecolor{codepurple}{rgb}{0.58,0,0.82}
\definecolor{backcolour}{rgb}{0.95,0.95,0.92}

\lstdefinestyle{mystyle}{
    backgroundcolor=\color{backcolour},   
    commentstyle=\color{codegreen},
    keywordstyle=\color{magenta},
    numberstyle=\tiny\color{codegray},
    stringstyle=\color{codepurple},
    basicstyle=\ttfamily\footnotesize,
    breakatwhitespace=false,         
    breaklines=true,                 
    captionpos=b,                    
    keepspaces=true,                 
    numbers=left,                    
xleftmargin=2em,
framexleftmargin=2em,            
    showspaces=false,                
    showstringspaces=false,
    showtabs=false,                  
    tabsize=2,
    upquote=true
}

\lstset{style=mystyle}


\lstset{style=mystyle}
\newcommand{\imgdir}{C:/laragon/www/newmc/assets/imgsvg/}
\newcommand{\imgsvgdir}{C:/laragon/www/newmc/assets/imgsvg/}

\definecolor{mcgris}{RGB}{220, 220, 220}% ancien~; pour compatibilité
\definecolor{mcbleu}{RGB}{52, 152, 219}
\definecolor{mcvert}{RGB}{125, 194, 70}
\definecolor{mcmauve}{RGB}{154, 0, 215}
\definecolor{mcorange}{RGB}{255, 96, 0}
\definecolor{mcturquoise}{RGB}{0, 153, 153}
\definecolor{mcrouge}{RGB}{255, 0, 0}
\definecolor{mclightvert}{RGB}{205, 234, 190}

\definecolor{gris}{RGB}{220, 220, 220}
\definecolor{bleu}{RGB}{52, 152, 219}
\definecolor{vert}{RGB}{125, 194, 70}
\definecolor{mauve}{RGB}{154, 0, 215}
\definecolor{orange}{RGB}{255, 96, 0}
\definecolor{turquoise}{RGB}{0, 153, 153}
\definecolor{rouge}{RGB}{255, 0, 0}
\definecolor{lightvert}{RGB}{205, 234, 190}
\setitemize[0]{label=\color{lightvert}  $\bullet$}

\pagestyle{fancy}
\renewcommand{\headrulewidth}{0.2pt}
\fancyhead[L]{maths-cours.fr}
\fancyhead[R]{\thepage}
\renewcommand{\footrulewidth}{0.2pt}
\fancyfoot[C]{}

\newcolumntype{C}{>{\centering\arraybackslash}X}
\newcolumntype{s}{>{\hsize=.35\hsize\arraybackslash}X}

\setlength{\parindent}{0pt}		 
\setlength{\parskip}{3mm}
\setlength{\headheight}{1cm}

\def\ebook{ebook}
\def\book{book}
\def\web{web}
\def\type{web}

\newcommand{\vect}[1]{\overrightarrow{\,\mathstrut#1\,}}

\def\Oij{$\left(\text{O}~;~\vect{\imath},~\vect{\jmath}\right)$}
\def\Oijk{$\left(\text{O}~;~\vect{\imath},~\vect{\jmath},~\vect{k}\right)$}
\def\Ouv{$\left(\text{O}~;~\vect{u},~\vect{v}\right)$}

\hypersetup{breaklinks=true, colorlinks = true, linkcolor = OliveGreen, urlcolor = OliveGreen, citecolor = OliveGreen, pdfauthor={Didier BONNEL - https://www.maths-cours.fr} } % supprime les bordures autour des liens

\renewcommand{\arg}[0]{\text{arg}}

\everymath{\displaystyle}

%================================================================================================================================
%
% Macros - Commandes
%
%================================================================================================================================

\newcommand\meta[2]{    			% Utilisé pour créer le post HTML.
	\def\titre{titre}
	\def\url{url}
	\def\arg{#1}
	\ifx\titre\arg
		\newcommand\maintitle{#2}
		\fancyhead[L]{#2}
		{\Large\sffamily \MakeUppercase{#2}}
		\vspace{1mm}\textcolor{mcvert}{\hrule}
	\fi 
	\ifx\url\arg
		\fancyfoot[L]{\href{https://www.maths-cours.fr#2}{\black \footnotesize{https://www.maths-cours.fr#2}}}
	\fi 
}


\newcommand\TitreC[1]{    		% Titre centré
     \needspace{3\baselineskip}
     \begin{center}\textbf{#1}\end{center}
}

\newcommand\newpar{    		% paragraphe
     \par
}

\newcommand\nosp {    		% commande vide (pas d'espace)
}
\newcommand{\id}[1]{} %ignore

\newcommand\boite[2]{				% Boite simple sans titre
	\vspace{5mm}
	\setlength{\fboxrule}{0.2mm}
	\setlength{\fboxsep}{5mm}	
	\fcolorbox{#1}{#1!3}{\makebox[\linewidth-2\fboxrule-2\fboxsep]{
  		\begin{minipage}[t]{\linewidth-2\fboxrule-4\fboxsep}\setlength{\parskip}{3mm}
  			 #2
  		\end{minipage}
	}}
	\vspace{5mm}
}

\newcommand\CBox[4]{				% Boites
	\vspace{5mm}
	\setlength{\fboxrule}{0.2mm}
	\setlength{\fboxsep}{5mm}
	
	\fcolorbox{#1}{#1!3}{\makebox[\linewidth-2\fboxrule-2\fboxsep]{
		\begin{minipage}[t]{1cm}\setlength{\parskip}{3mm}
	  		\textcolor{#1}{\LARGE{#2}}    
 	 	\end{minipage}  
  		\begin{minipage}[t]{\linewidth-2\fboxrule-4\fboxsep}\setlength{\parskip}{3mm}
			\raisebox{1.2mm}{\normalsize\sffamily{\textcolor{#1}{#3}}}						
  			 #4
  		\end{minipage}
	}}
	\vspace{5mm}
}

\newcommand\cadre[3]{				% Boites convertible html
	\par
	\vspace{2mm}
	\setlength{\fboxrule}{0.1mm}
	\setlength{\fboxsep}{5mm}
	\fcolorbox{#1}{white}{\makebox[\linewidth-2\fboxrule-2\fboxsep]{
  		\begin{minipage}[t]{\linewidth-2\fboxrule-4\fboxsep}\setlength{\parskip}{3mm}
			\raisebox{-2.5mm}{\sffamily \small{\textcolor{#1}{\MakeUppercase{#2}}}}		
			\par		
  			 #3
 	 		\end{minipage}
	}}
		\vspace{2mm}
	\par
}

\newcommand\bloc[3]{				% Boites convertible html sans bordure
     \needspace{2\baselineskip}
     {\sffamily \small{\textcolor{#1}{\MakeUppercase{#2}}}}    
		\par		
  			 #3
		\par
}

\newcommand\CHelp[1]{
     \CBox{Plum}{\faInfoCircle}{À RETENIR}{#1}
}

\newcommand\CUp[1]{
     \CBox{NavyBlue}{\faThumbsOUp}{EN PRATIQUE}{#1}
}

\newcommand\CInfo[1]{
     \CBox{Sepia}{\faArrowCircleRight}{REMARQUE}{#1}
}

\newcommand\CRedac[1]{
     \CBox{PineGreen}{\faEdit}{BIEN R\'EDIGER}{#1}
}

\newcommand\CError[1]{
     \CBox{Red}{\faExclamationTriangle}{ATTENTION}{#1}
}

\newcommand\TitreExo[2]{
\needspace{4\baselineskip}
 {\sffamily\large EXERCICE #1\ (\emph{#2 points})}
\vspace{5mm}
}

\newcommand\img[2]{
          \includegraphics[width=#2\paperwidth]{\imgdir#1}
}

\newcommand\imgsvg[2]{
       \begin{center}   \includegraphics[width=#2\paperwidth]{\imgsvgdir#1} \end{center}
}


\newcommand\Lien[2]{
     \href{#1}{#2 \tiny \faExternalLink}
}
\newcommand\mcLien[2]{
     \href{https~://www.maths-cours.fr/#1}{#2 \tiny \faExternalLink}
}

\newcommand{\euro}{\eurologo{}}

%================================================================================================================================
%
% Macros - Environement
%
%================================================================================================================================

\newenvironment{tex}{ %
}
{%
}

\newenvironment{indente}{ %
	\setlength\parindent{10mm}
}

{
	\setlength\parindent{0mm}
}

\newenvironment{corrige}{%
     \needspace{3\baselineskip}
     \medskip
     \textbf{\textsc{Corrigé}}
     \medskip
}
{
}

\newenvironment{extern}{%
     \begin{center}
     }
     {
     \end{center}
}

\NewEnviron{code}{%
	\par
     \boite{gray}{\texttt{%
     \BODY
     }}
     \par
}

\newenvironment{vbloc}{% boite sans cadre empeche saut de page
     \begin{minipage}[t]{\linewidth}
     }
     {
     \end{minipage}
}
\NewEnviron{h2}{%
    \needspace{3\baselineskip}
    \vspace{0.6cm}
	\noindent \MakeUppercase{\sffamily \large \BODY}
	\vspace{1mm}\textcolor{mcgris}{\hrule}\vspace{0.4cm}
	\par
}{}

\NewEnviron{h3}{%
    \needspace{3\baselineskip}
	\vspace{5mm}
	\textsc{\BODY}
	\par
}

\NewEnviron{margeneg}{ %
\begin{addmargin}[-1cm]{0cm}
\BODY
\end{addmargin}
}

\NewEnviron{html}{%
}

\begin{document}
\meta{url}{/exercices/vecteurs-et-cercles-secants/}
\meta{pid}{20038}
\meta{titre}{Vecteurs et cercles sécants}
\meta{type}{exercices}
%
$ \mathscr{C} $ et $ \mathscr{C}' $ sont deux cercles de même rayon sécants en deux points distincts $A$ et $B$.
\par
$ [AP] $ et $ [BQ] $ sont des diamètres du cercle $ \mathscr{C} $ et $ [AR] $ et $ [BS] $ des diamètres du cercle $ \mathscr{C}'. $
\\
\begin{center}
\begin{extern}%alt="cercles sécants" style="width:60rem"
\newrgbcolor{tttttt}{0.2 0.2 0.2} 
\psset{xunit=1.0cm,yunit=1.0cm,algebraic=true,dimen=middle,dotstyle=o,dotsize=5pt 0,linewidth=2.pt,arrowsize=3pt 2,arrowinset=0.25}
\begin{pspicture*}(-0.5,-0.5)(11.5,6.5)
\pscircle[linewidth=0.5pt,linecolor=tttttt](3.,3.){3.}
\pscircle[linewidth=0.5pt,linecolor=tttttt](8.,3.){3.}
\psline[linewidth=0.5pt,linecolor=tttttt](0.5,1.3416876048223005)(5.5,4.658312395177699)
\psline[linewidth=0.5pt,linecolor=tttttt](0.5,4.658312395177701)(5.5,1.3416876048223)
\psline[linewidth=0.5pt,linecolor=tttttt](5.5,1.3416876048223)(10.5,4.658312395177702)
\psline[linewidth=0.5pt,linecolor=tttttt](5.5,4.658312395177699)(10.5,1.3416876048222974)
\rput[tl](0.9121563349924472,6.023806232132701){$\tttttt{\mathscr{C}}$}
\rput[tl](9.925359952528456,5.932559534595379){$\tttttt{\mathscr{C'}}$}
\begin{scriptsize}
\psdots[dotsize=2pt 0,dotstyle=*,linecolor=tttttt](3.,3.)
\rput[b](2.997028686927544,3.094787241184669){\tttttt{$I$}}
\psdots[dotsize=2pt 0,dotstyle=*,linecolor=tttttt](8.,3.)
\rput[b](7.986155537846467,3.094787241184669){\tttttt{$J$}}
\psdots[dotsize=2pt 0,dotstyle=*,linecolor=darkgray](5.5,4.658312395177699)
\rput[b](5.5,4.801100485132588){\darkgray{$A$}}
\psdots[dotsize=2pt 0,dotstyle=*,linecolor=darkgray](5.5,1.3416876048223)
\rput[bl](5.4,1.0508612163486593){\darkgray{$B$}}
\psdots[dotsize=2pt 0,dotstyle=*,linecolor=darkgray](10.5,4.658312395177702)
\rput[bl](10.535344439775843,4.691604448087801){\darkgray{$S$}}
\psdots[dotsize=2pt 0,dotstyle=*,linecolor=darkgray](0.5,4.658312395177701)
\rput[b](0.33858883205833745,4.746352466610195){\darkgray{$Q$}}
\psdots[dotsize=2pt 0,dotstyle=*,linecolor=darkgray](0.5,1.3416876048223005)
\rput[b](0.28,1.2151052719158386){\darkgray{$P$}}
\psdots[dotsize=2pt 0,dotstyle=*,linecolor=darkgray](10.5,1.3416876048222974)
\rput[b](10.65,1.2151052719158386){\darkgray{$R$}}
\end{scriptsize}
\end{pspicture*}
\end{extern}
\end{center}
\\

\begin{enumerate}
\item
Quelle est la nature du quadrilatère $ AJBI $~?
\\
Justifier votre réponse.
\item
Quelle est la nature du quadrilatère $ ABPQ $~?
\\
Justifier votre réponse.
\item
Montrer que  $  \overrightarrow{ AI } =  \overrightarrow{ IP }. $
\item
En déduire que $ IPBJ $ est un parallélogramme.
\\
De même, montrer que  $ IBRJ $ est un parallélogramme.

\item
Montrer que $B$ est le milieu de $ [PR]. $
\end{enumerate}

\begin{corrige}

\begin{enumerate}
\item
$ [AI] $ et $ [BI] $ sont deux rayons du cercle  $ \mathscr{C} $ et  $ [AJ] $ et $ [BJ] $ sont deux rayons du cercle  $ \mathscr{C}' $.
\par
Les deux cercles ayant le même rayon : $ AI=BI=AJ=BJ $. Par conséquent, $ AIBJ $ est un  \textbf{losange.} 

\item
$ [AP] $ et $ [BQ] $ , les diagonales du quadrilatère  $ ABPQ $, sont deux diamètres du cercle $ \mathscr{C}  $.
\par
Ces diagonales sont donc de même longueur et elles se coupent en leur milieu.
\par
Par conséquent,  $ ABPQ $ est un rectangle.

\item
$I$ est le milieu du segment $ [AP] $ donc $  \overrightarrow{ AI } =  \overrightarrow{ IP }. $


\item
$ AJBI $ est un losange donc c'est également un parallélogramme.
\par
Par conséquent~: $  \overrightarrow{ AI } =  \overrightarrow{ JB }. $
\par
Or, d'après la question précédente  $  \overrightarrow{ AI } =  \overrightarrow{ IP }. $ On en déduit que  $  \overrightarrow{ JB } =  \overrightarrow{ IP }. $ et donc que  $ IPBJ $ est un parallélogramme.
\par
La démonstration est analogue pour $ IBRJ $~:
\par
$  \overrightarrow{ IB } =  \overrightarrow{ AJ } =  \overrightarrow{ JR }, $
\par
donc   $ IBRJ $ est un parallélogramme.

\item
Puisque $ IPBJ $ est un parallélogramme :  $  \overrightarrow{ PB } =  \overrightarrow{ IJ }.  $
\\
Puisque $ IBRJ $ est un parallélogramme :  $  \overrightarrow{ BR } =  \overrightarrow{ IJ }.  $
\par
Par conséquent,  $  \overrightarrow{ PB } =  \overrightarrow{ BR }$   donc  $B$ est le milieu du segment $ [PR].$
\end{enumerate}
\end{corrige}

\end{document}

µ
\documentclass[a4paper]{article}

%================================================================================================================================
%
% Packages
%
%================================================================================================================================

\usepackage[T1]{fontenc} 	% pour caractères accentués
\usepackage[utf8]{inputenc}  % encodage utf8
\usepackage[french]{babel}	% langue : français
\usepackage{fourier}			% caractères plus lisibles
\usepackage[dvipsnames]{xcolor} % couleurs
\usepackage{fancyhdr}		% réglage header footer
\usepackage{needspace}		% empêcher sauts de page mal placés
\usepackage{graphicx}		% pour inclure des graphiques
\usepackage{enumitem,cprotect}		% personnalise les listes d'items (nécessaire pour ol, al ...)
\usepackage{hyperref}		% Liens hypertexte
\usepackage{pstricks,pst-all,pst-node,pstricks-add,pst-math,pst-plot,pst-tree,pst-eucl} % pstricks
\usepackage[a4paper,includeheadfoot,top=2cm,left=3cm, bottom=2cm,right=3cm]{geometry} % marges etc.
\usepackage{comment}			% commentaires multilignes
\usepackage{amsmath,environ} % maths (matrices, etc.)
\usepackage{amssymb,makeidx}
\usepackage{bm}				% bold maths
\usepackage{tabularx}		% tableaux
\usepackage{colortbl}		% tableaux en couleur
\usepackage{fontawesome}		% Fontawesome
\usepackage{environ}			% environment with command
\usepackage{fp}				% calculs pour ps-tricks
\usepackage{multido}			% pour ps tricks
\usepackage[np]{numprint}	% formattage nombre
\usepackage{tikz,tkz-tab} 			% package principal TikZ
\usepackage{pgfplots}   % axes
\usepackage{mathrsfs}    % cursives
\usepackage{calc}			% calcul taille boites
\usepackage[scaled=0.875]{helvet} % font sans serif
\usepackage{svg} % svg
\usepackage{scrextend} % local margin
\usepackage{scratch} %scratch
\usepackage{multicol} % colonnes
%\usepackage{infix-RPN,pst-func} % formule en notation polanaise inversée
\usepackage{listings}

%================================================================================================================================
%
% Réglages de base
%
%================================================================================================================================

\lstset{
language=Python,   % R code
literate=
{á}{{\'a}}1
{à}{{\`a}}1
{ã}{{\~a}}1
{é}{{\'e}}1
{è}{{\`e}}1
{ê}{{\^e}}1
{í}{{\'i}}1
{ó}{{\'o}}1
{õ}{{\~o}}1
{ú}{{\'u}}1
{ü}{{\"u}}1
{ç}{{\c{c}}}1
{~}{{ }}1
}


\definecolor{codegreen}{rgb}{0,0.6,0}
\definecolor{codegray}{rgb}{0.5,0.5,0.5}
\definecolor{codepurple}{rgb}{0.58,0,0.82}
\definecolor{backcolour}{rgb}{0.95,0.95,0.92}

\lstdefinestyle{mystyle}{
    backgroundcolor=\color{backcolour},   
    commentstyle=\color{codegreen},
    keywordstyle=\color{magenta},
    numberstyle=\tiny\color{codegray},
    stringstyle=\color{codepurple},
    basicstyle=\ttfamily\footnotesize,
    breakatwhitespace=false,         
    breaklines=true,                 
    captionpos=b,                    
    keepspaces=true,                 
    numbers=left,                    
xleftmargin=2em,
framexleftmargin=2em,            
    showspaces=false,                
    showstringspaces=false,
    showtabs=false,                  
    tabsize=2,
    upquote=true
}

\lstset{style=mystyle}


\lstset{style=mystyle}
\newcommand{\imgdir}{C:/laragon/www/newmc/assets/imgsvg/}
\newcommand{\imgsvgdir}{C:/laragon/www/newmc/assets/imgsvg/}

\definecolor{mcgris}{RGB}{220, 220, 220}% ancien~; pour compatibilité
\definecolor{mcbleu}{RGB}{52, 152, 219}
\definecolor{mcvert}{RGB}{125, 194, 70}
\definecolor{mcmauve}{RGB}{154, 0, 215}
\definecolor{mcorange}{RGB}{255, 96, 0}
\definecolor{mcturquoise}{RGB}{0, 153, 153}
\definecolor{mcrouge}{RGB}{255, 0, 0}
\definecolor{mclightvert}{RGB}{205, 234, 190}

\definecolor{gris}{RGB}{220, 220, 220}
\definecolor{bleu}{RGB}{52, 152, 219}
\definecolor{vert}{RGB}{125, 194, 70}
\definecolor{mauve}{RGB}{154, 0, 215}
\definecolor{orange}{RGB}{255, 96, 0}
\definecolor{turquoise}{RGB}{0, 153, 153}
\definecolor{rouge}{RGB}{255, 0, 0}
\definecolor{lightvert}{RGB}{205, 234, 190}
\setitemize[0]{label=\color{lightvert}  $\bullet$}

\pagestyle{fancy}
\renewcommand{\headrulewidth}{0.2pt}
\fancyhead[L]{maths-cours.fr}
\fancyhead[R]{\thepage}
\renewcommand{\footrulewidth}{0.2pt}
\fancyfoot[C]{}

\newcolumntype{C}{>{\centering\arraybackslash}X}
\newcolumntype{s}{>{\hsize=.35\hsize\arraybackslash}X}

\setlength{\parindent}{0pt}		 
\setlength{\parskip}{3mm}
\setlength{\headheight}{1cm}

\def\ebook{ebook}
\def\book{book}
\def\web{web}
\def\type{web}

\newcommand{\vect}[1]{\overrightarrow{\,\mathstrut#1\,}}

\def\Oij{$\left(\text{O}~;~\vect{\imath},~\vect{\jmath}\right)$}
\def\Oijk{$\left(\text{O}~;~\vect{\imath},~\vect{\jmath},~\vect{k}\right)$}
\def\Ouv{$\left(\text{O}~;~\vect{u},~\vect{v}\right)$}

\hypersetup{breaklinks=true, colorlinks = true, linkcolor = OliveGreen, urlcolor = OliveGreen, citecolor = OliveGreen, pdfauthor={Didier BONNEL - https://www.maths-cours.fr} } % supprime les bordures autour des liens

\renewcommand{\arg}[0]{\text{arg}}

\everymath{\displaystyle}

%================================================================================================================================
%
% Macros - Commandes
%
%================================================================================================================================

\newcommand\meta[2]{    			% Utilisé pour créer le post HTML.
	\def\titre{titre}
	\def\url{url}
	\def\arg{#1}
	\ifx\titre\arg
		\newcommand\maintitle{#2}
		\fancyhead[L]{#2}
		{\Large\sffamily \MakeUppercase{#2}}
		\vspace{1mm}\textcolor{mcvert}{\hrule}
	\fi 
	\ifx\url\arg
		\fancyfoot[L]{\href{https://www.maths-cours.fr#2}{\black \footnotesize{https://www.maths-cours.fr#2}}}
	\fi 
}


\newcommand\TitreC[1]{    		% Titre centré
     \needspace{3\baselineskip}
     \begin{center}\textbf{#1}\end{center}
}

\newcommand\newpar{    		% paragraphe
     \par
}

\newcommand\nosp {    		% commande vide (pas d'espace)
}
\newcommand{\id}[1]{} %ignore

\newcommand\boite[2]{				% Boite simple sans titre
	\vspace{5mm}
	\setlength{\fboxrule}{0.2mm}
	\setlength{\fboxsep}{5mm}	
	\fcolorbox{#1}{#1!3}{\makebox[\linewidth-2\fboxrule-2\fboxsep]{
  		\begin{minipage}[t]{\linewidth-2\fboxrule-4\fboxsep}\setlength{\parskip}{3mm}
  			 #2
  		\end{minipage}
	}}
	\vspace{5mm}
}

\newcommand\CBox[4]{				% Boites
	\vspace{5mm}
	\setlength{\fboxrule}{0.2mm}
	\setlength{\fboxsep}{5mm}
	
	\fcolorbox{#1}{#1!3}{\makebox[\linewidth-2\fboxrule-2\fboxsep]{
		\begin{minipage}[t]{1cm}\setlength{\parskip}{3mm}
	  		\textcolor{#1}{\LARGE{#2}}    
 	 	\end{minipage}  
  		\begin{minipage}[t]{\linewidth-2\fboxrule-4\fboxsep}\setlength{\parskip}{3mm}
			\raisebox{1.2mm}{\normalsize\sffamily{\textcolor{#1}{#3}}}						
  			 #4
  		\end{minipage}
	}}
	\vspace{5mm}
}

\newcommand\cadre[3]{				% Boites convertible html
	\par
	\vspace{2mm}
	\setlength{\fboxrule}{0.1mm}
	\setlength{\fboxsep}{5mm}
	\fcolorbox{#1}{white}{\makebox[\linewidth-2\fboxrule-2\fboxsep]{
  		\begin{minipage}[t]{\linewidth-2\fboxrule-4\fboxsep}\setlength{\parskip}{3mm}
			\raisebox{-2.5mm}{\sffamily \small{\textcolor{#1}{\MakeUppercase{#2}}}}		
			\par		
  			 #3
 	 		\end{minipage}
	}}
		\vspace{2mm}
	\par
}

\newcommand\bloc[3]{				% Boites convertible html sans bordure
     \needspace{2\baselineskip}
     {\sffamily \small{\textcolor{#1}{\MakeUppercase{#2}}}}    
		\par		
  			 #3
		\par
}

\newcommand\CHelp[1]{
     \CBox{Plum}{\faInfoCircle}{À RETENIR}{#1}
}

\newcommand\CUp[1]{
     \CBox{NavyBlue}{\faThumbsOUp}{EN PRATIQUE}{#1}
}

\newcommand\CInfo[1]{
     \CBox{Sepia}{\faArrowCircleRight}{REMARQUE}{#1}
}

\newcommand\CRedac[1]{
     \CBox{PineGreen}{\faEdit}{BIEN R\'EDIGER}{#1}
}

\newcommand\CError[1]{
     \CBox{Red}{\faExclamationTriangle}{ATTENTION}{#1}
}

\newcommand\TitreExo[2]{
\needspace{4\baselineskip}
 {\sffamily\large EXERCICE #1\ (\emph{#2 points})}
\vspace{5mm}
}

\newcommand\img[2]{
          \includegraphics[width=#2\paperwidth]{\imgdir#1}
}

\newcommand\imgsvg[2]{
       \begin{center}   \includegraphics[width=#2\paperwidth]{\imgsvgdir#1} \end{center}
}


\newcommand\Lien[2]{
     \href{#1}{#2 \tiny \faExternalLink}
}
\newcommand\mcLien[2]{
     \href{https~://www.maths-cours.fr/#1}{#2 \tiny \faExternalLink}
}

\newcommand{\euro}{\eurologo{}}

%================================================================================================================================
%
% Macros - Environement
%
%================================================================================================================================

\newenvironment{tex}{ %
}
{%
}

\newenvironment{indente}{ %
	\setlength\parindent{10mm}
}

{
	\setlength\parindent{0mm}
}

\newenvironment{corrige}{%
     \needspace{3\baselineskip}
     \medskip
     \textbf{\textsc{Corrigé}}
     \medskip
}
{
}

\newenvironment{extern}{%
     \begin{center}
     }
     {
     \end{center}
}

\NewEnviron{code}{%
	\par
     \boite{gray}{\texttt{%
     \BODY
     }}
     \par
}

\newenvironment{vbloc}{% boite sans cadre empeche saut de page
     \begin{minipage}[t]{\linewidth}
     }
     {
     \end{minipage}
}
\NewEnviron{h2}{%
    \needspace{3\baselineskip}
    \vspace{0.6cm}
	\noindent \MakeUppercase{\sffamily \large \BODY}
	\vspace{1mm}\textcolor{mcgris}{\hrule}\vspace{0.4cm}
	\par
}{}

\NewEnviron{h3}{%
    \needspace{3\baselineskip}
	\vspace{5mm}
	\textsc{\BODY}
	\par
}

\NewEnviron{margeneg}{ %
\begin{addmargin}[-1cm]{0cm}
\BODY
\end{addmargin}
}

\NewEnviron{html}{%
}

\begin{document}
\meta{url}{/tex-source/}
\meta{pid}{0}
\meta{titre}{Tex source}
\meta{type}{page}
%

\end{document}


µ"
"20051";"{{unknown}}"
"20099";
"20111";"Cette page a été supprimée !"
"20114";
