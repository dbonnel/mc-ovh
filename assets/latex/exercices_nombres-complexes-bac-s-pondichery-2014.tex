\documentclass[a4paper]{article}

%================================================================================================================================
%
% Packages
%
%================================================================================================================================

\usepackage[T1]{fontenc} 	% pour caractères accentués
\usepackage[utf8]{inputenc}  % encodage utf8
\usepackage[french]{babel}	% langue : français
\usepackage{fourier}			% caractères plus lisibles
\usepackage[dvipsnames]{xcolor} % couleurs
\usepackage{fancyhdr}		% réglage header footer
\usepackage{needspace}		% empêcher sauts de page mal placés
\usepackage{graphicx}		% pour inclure des graphiques
\usepackage{enumitem,cprotect}		% personnalise les listes d'items (nécessaire pour ol, al ...)
\usepackage{hyperref}		% Liens hypertexte
\usepackage{pstricks,pst-all,pst-node,pstricks-add,pst-math,pst-plot,pst-tree,pst-eucl} % pstricks
\usepackage[a4paper,includeheadfoot,top=2cm,left=3cm, bottom=2cm,right=3cm]{geometry} % marges etc.
\usepackage{comment}			% commentaires multilignes
\usepackage{amsmath,environ} % maths (matrices, etc.)
\usepackage{amssymb,makeidx}
\usepackage{bm}				% bold maths
\usepackage{tabularx}		% tableaux
\usepackage{colortbl}		% tableaux en couleur
\usepackage{fontawesome}		% Fontawesome
\usepackage{environ}			% environment with command
\usepackage{fp}				% calculs pour ps-tricks
\usepackage{multido}			% pour ps tricks
\usepackage[np]{numprint}	% formattage nombre
\usepackage{tikz,tkz-tab} 			% package principal TikZ
\usepackage{pgfplots}   % axes
\usepackage{mathrsfs}    % cursives
\usepackage{calc}			% calcul taille boites
\usepackage[scaled=0.875]{helvet} % font sans serif
\usepackage{svg} % svg
\usepackage{scrextend} % local margin
\usepackage{scratch} %scratch
\usepackage{multicol} % colonnes
%\usepackage{infix-RPN,pst-func} % formule en notation polanaise inversée
\usepackage{listings}

%================================================================================================================================
%
% Réglages de base
%
%================================================================================================================================

\lstset{
language=Python,   % R code
literate=
{á}{{\'a}}1
{à}{{\`a}}1
{ã}{{\~a}}1
{é}{{\'e}}1
{è}{{\`e}}1
{ê}{{\^e}}1
{í}{{\'i}}1
{ó}{{\'o}}1
{õ}{{\~o}}1
{ú}{{\'u}}1
{ü}{{\"u}}1
{ç}{{\c{c}}}1
{~}{{ }}1
}


\definecolor{codegreen}{rgb}{0,0.6,0}
\definecolor{codegray}{rgb}{0.5,0.5,0.5}
\definecolor{codepurple}{rgb}{0.58,0,0.82}
\definecolor{backcolour}{rgb}{0.95,0.95,0.92}

\lstdefinestyle{mystyle}{
    backgroundcolor=\color{backcolour},   
    commentstyle=\color{codegreen},
    keywordstyle=\color{magenta},
    numberstyle=\tiny\color{codegray},
    stringstyle=\color{codepurple},
    basicstyle=\ttfamily\footnotesize,
    breakatwhitespace=false,         
    breaklines=true,                 
    captionpos=b,                    
    keepspaces=true,                 
    numbers=left,                    
xleftmargin=2em,
framexleftmargin=2em,            
    showspaces=false,                
    showstringspaces=false,
    showtabs=false,                  
    tabsize=2,
    upquote=true
}

\lstset{style=mystyle}


\lstset{style=mystyle}
\newcommand{\imgdir}{C:/laragon/www/newmc/assets/imgsvg/}
\newcommand{\imgsvgdir}{C:/laragon/www/newmc/assets/imgsvg/}

\definecolor{mcgris}{RGB}{220, 220, 220}% ancien~; pour compatibilité
\definecolor{mcbleu}{RGB}{52, 152, 219}
\definecolor{mcvert}{RGB}{125, 194, 70}
\definecolor{mcmauve}{RGB}{154, 0, 215}
\definecolor{mcorange}{RGB}{255, 96, 0}
\definecolor{mcturquoise}{RGB}{0, 153, 153}
\definecolor{mcrouge}{RGB}{255, 0, 0}
\definecolor{mclightvert}{RGB}{205, 234, 190}

\definecolor{gris}{RGB}{220, 220, 220}
\definecolor{bleu}{RGB}{52, 152, 219}
\definecolor{vert}{RGB}{125, 194, 70}
\definecolor{mauve}{RGB}{154, 0, 215}
\definecolor{orange}{RGB}{255, 96, 0}
\definecolor{turquoise}{RGB}{0, 153, 153}
\definecolor{rouge}{RGB}{255, 0, 0}
\definecolor{lightvert}{RGB}{205, 234, 190}
\setitemize[0]{label=\color{lightvert}  $\bullet$}

\pagestyle{fancy}
\renewcommand{\headrulewidth}{0.2pt}
\fancyhead[L]{maths-cours.fr}
\fancyhead[R]{\thepage}
\renewcommand{\footrulewidth}{0.2pt}
\fancyfoot[C]{}

\newcolumntype{C}{>{\centering\arraybackslash}X}
\newcolumntype{s}{>{\hsize=.35\hsize\arraybackslash}X}

\setlength{\parindent}{0pt}		 
\setlength{\parskip}{3mm}
\setlength{\headheight}{1cm}

\def\ebook{ebook}
\def\book{book}
\def\web{web}
\def\type{web}

\newcommand{\vect}[1]{\overrightarrow{\,\mathstrut#1\,}}

\def\Oij{$\left(\text{O}~;~\vect{\imath},~\vect{\jmath}\right)$}
\def\Oijk{$\left(\text{O}~;~\vect{\imath},~\vect{\jmath},~\vect{k}\right)$}
\def\Ouv{$\left(\text{O}~;~\vect{u},~\vect{v}\right)$}

\hypersetup{breaklinks=true, colorlinks = true, linkcolor = OliveGreen, urlcolor = OliveGreen, citecolor = OliveGreen, pdfauthor={Didier BONNEL - https://www.maths-cours.fr} } % supprime les bordures autour des liens

\renewcommand{\arg}[0]{\text{arg}}

\everymath{\displaystyle}

%================================================================================================================================
%
% Macros - Commandes
%
%================================================================================================================================

\newcommand\meta[2]{    			% Utilisé pour créer le post HTML.
	\def\titre{titre}
	\def\url{url}
	\def\arg{#1}
	\ifx\titre\arg
		\newcommand\maintitle{#2}
		\fancyhead[L]{#2}
		{\Large\sffamily \MakeUppercase{#2}}
		\vspace{1mm}\textcolor{mcvert}{\hrule}
	\fi 
	\ifx\url\arg
		\fancyfoot[L]{\href{https://www.maths-cours.fr#2}{\black \footnotesize{https://www.maths-cours.fr#2}}}
	\fi 
}


\newcommand\TitreC[1]{    		% Titre centré
     \needspace{3\baselineskip}
     \begin{center}\textbf{#1}\end{center}
}

\newcommand\newpar{    		% paragraphe
     \par
}

\newcommand\nosp {    		% commande vide (pas d'espace)
}
\newcommand{\id}[1]{} %ignore

\newcommand\boite[2]{				% Boite simple sans titre
	\vspace{5mm}
	\setlength{\fboxrule}{0.2mm}
	\setlength{\fboxsep}{5mm}	
	\fcolorbox{#1}{#1!3}{\makebox[\linewidth-2\fboxrule-2\fboxsep]{
  		\begin{minipage}[t]{\linewidth-2\fboxrule-4\fboxsep}\setlength{\parskip}{3mm}
  			 #2
  		\end{minipage}
	}}
	\vspace{5mm}
}

\newcommand\CBox[4]{				% Boites
	\vspace{5mm}
	\setlength{\fboxrule}{0.2mm}
	\setlength{\fboxsep}{5mm}
	
	\fcolorbox{#1}{#1!3}{\makebox[\linewidth-2\fboxrule-2\fboxsep]{
		\begin{minipage}[t]{1cm}\setlength{\parskip}{3mm}
	  		\textcolor{#1}{\LARGE{#2}}    
 	 	\end{minipage}  
  		\begin{minipage}[t]{\linewidth-2\fboxrule-4\fboxsep}\setlength{\parskip}{3mm}
			\raisebox{1.2mm}{\normalsize\sffamily{\textcolor{#1}{#3}}}						
  			 #4
  		\end{minipage}
	}}
	\vspace{5mm}
}

\newcommand\cadre[3]{				% Boites convertible html
	\par
	\vspace{2mm}
	\setlength{\fboxrule}{0.1mm}
	\setlength{\fboxsep}{5mm}
	\fcolorbox{#1}{white}{\makebox[\linewidth-2\fboxrule-2\fboxsep]{
  		\begin{minipage}[t]{\linewidth-2\fboxrule-4\fboxsep}\setlength{\parskip}{3mm}
			\raisebox{-2.5mm}{\sffamily \small{\textcolor{#1}{\MakeUppercase{#2}}}}		
			\par		
  			 #3
 	 		\end{minipage}
	}}
		\vspace{2mm}
	\par
}

\newcommand\bloc[3]{				% Boites convertible html sans bordure
     \needspace{2\baselineskip}
     {\sffamily \small{\textcolor{#1}{\MakeUppercase{#2}}}}    
		\par		
  			 #3
		\par
}

\newcommand\CHelp[1]{
     \CBox{Plum}{\faInfoCircle}{À RETENIR}{#1}
}

\newcommand\CUp[1]{
     \CBox{NavyBlue}{\faThumbsOUp}{EN PRATIQUE}{#1}
}

\newcommand\CInfo[1]{
     \CBox{Sepia}{\faArrowCircleRight}{REMARQUE}{#1}
}

\newcommand\CRedac[1]{
     \CBox{PineGreen}{\faEdit}{BIEN R\'EDIGER}{#1}
}

\newcommand\CError[1]{
     \CBox{Red}{\faExclamationTriangle}{ATTENTION}{#1}
}

\newcommand\TitreExo[2]{
\needspace{4\baselineskip}
 {\sffamily\large EXERCICE #1\ (\emph{#2 points})}
\vspace{5mm}
}

\newcommand\img[2]{
          \includegraphics[width=#2\paperwidth]{\imgdir#1}
}

\newcommand\imgsvg[2]{
       \begin{center}   \includegraphics[width=#2\paperwidth]{\imgsvgdir#1} \end{center}
}


\newcommand\Lien[2]{
     \href{#1}{#2 \tiny \faExternalLink}
}
\newcommand\mcLien[2]{
     \href{https~://www.maths-cours.fr/#1}{#2 \tiny \faExternalLink}
}

\newcommand{\euro}{\eurologo{}}

%================================================================================================================================
%
% Macros - Environement
%
%================================================================================================================================

\newenvironment{tex}{ %
}
{%
}

\newenvironment{indente}{ %
	\setlength\parindent{10mm}
}

{
	\setlength\parindent{0mm}
}

\newenvironment{corrige}{%
     \needspace{3\baselineskip}
     \medskip
     \textbf{\textsc{Corrigé}}
     \medskip
}
{
}

\newenvironment{extern}{%
     \begin{center}
     }
     {
     \end{center}
}

\NewEnviron{code}{%
	\par
     \boite{gray}{\texttt{%
     \BODY
     }}
     \par
}

\newenvironment{vbloc}{% boite sans cadre empeche saut de page
     \begin{minipage}[t]{\linewidth}
     }
     {
     \end{minipage}
}
\NewEnviron{h2}{%
    \needspace{3\baselineskip}
    \vspace{0.6cm}
	\noindent \MakeUppercase{\sffamily \large \BODY}
	\vspace{1mm}\textcolor{mcgris}{\hrule}\vspace{0.4cm}
	\par
}{}

\NewEnviron{h3}{%
    \needspace{3\baselineskip}
	\vspace{5mm}
	\textsc{\BODY}
	\par
}

\NewEnviron{margeneg}{ %
\begin{addmargin}[-1cm]{0cm}
\BODY
\end{addmargin}
}

\NewEnviron{html}{%
}

\begin{document}
\meta{url}{/exercices/nombres-complexes-bac-s-pondichery-2014/}
\meta{pid}{1413}
\meta{titre}{Nombres complexes - Bac S Pondichéry 2014}
\meta{type}{exercices}
%
\begin{h2}Exercice 3   (5 points)\end{h2}
\textbf{Candidats n'ayant pas suivi l'enseignement de spécialité}
Le plan complexe est muni d'un repère orthonormé $\left(O; \vec{u}, \vec{v}\right)$.
\par
Pour tout entier naturel $n$, on note $A_{n}$ le point d'affixe $z_{n}$ défini par :
\par
$z_{0}=1$     et    $ z_{n+1}=\left(\frac{3}{4}+\frac{\sqrt{3}}{4}i\right)z_{n}$
\par
On définit la suite $\left(r_{n}\right)$ par $r_{n}=|z_{n}|$ pour tout entier naturel $n$.
\begin{enumerate}
     \item
     Donner la forme exponentielle du nombre complexe $\frac{3}{4}+\frac{\sqrt{3}}{4}i$.
     \item
     \begin{enumerate}[label=\alph*.]
          \item
          Montrer que la suite $\left(r_{n}\right)$ est géométrique de raison $\frac{\sqrt{3}}{2}$.
          \item
          En déduire l'expression de $r_{n}$ en fonction de $n$.
          \item
          Que dire de la longueur $OA_{n}$ lorsque $n$ tend vers $+ \infty $ ?
     \end{enumerate}
     \item
     On considère l'algorithme suivant :
     \begin{tabularx}{0.8\linewidth}{|*{3}{>{\centering \arraybackslash }X|}}%class="singleborder" width="600"
          \hline
          \textbf{Variables} &  $n$ entier naturel
          \\ \hline
          & $R$ réel
          \\ \hline
          & $P$ réel strictement positif
          \\ \hline
          \textbf{Entrée} &  Demander la valeur de $P$
          \\ \hline
          \textbf{Traitement}  & $R$ prend la valeur $1$
          \\ \hline
          & $n$ prend la valeur $0$
          \\ \hline
          & Tant que $R > P$
          \\ \hline
          & <span style="color:transparent">...</span><span style="color:transparent">...</span>$n$ prend la valeur $n+1$
          \\ \hline
          & <span style="color:transparent">...</span><span style="color:transparent">...</span>$R$ prend la valeur $\frac{\sqrt{3}}{2}R$
          \\ \hline
          & Fin tant que
          \\ \hline
          \textbf{Sortie}  & Afficher $n$
          \\ \hline
\end{tabularx}
\begin{enumerate}[label=\alph*.]
     \item
     Quelle est la valeur affichée par l'algorithme pour $P=0,5$ ?
     \item
     Pour $P=0,01$ on obtient $n=33$. Quel est le rôle de cet algorithme ?
\end{enumerate}
\item
\begin{enumerate}[label=\alph*.]
     \item
     Démontrer que le triangle $OA_{n}A_{n+1}$ est rectangle en $A_{n+1}$.
     \item
     On admet que $z_{n}=r_{n}e^{i\frac{n\pi }{6}}$.
     \par
     Déterminer les valeurs de $n$ pour lesquelles $A_{n}$ est un point de l'axe des ordonnées.
     \item
     Compléter la figure ci-dessous, à rendre avec la copie, en représentant les points $A_{6}, A_{7}, A_{8}$ et $A_{9}$.
     \par
     Les traits de construction seront apparents.
\begin{center}
\imgsvg{mc-0260}{0.3}% alt="Nombres complexes - Bac S Pondichéry 2014" style="width:35rem" 
\end{center}
\end{enumerate}
\end{enumerate}
\begin{corrige}
     \begin{enumerate}
          \item
          Soit $r$ le module de $\frac{3}{4}+\frac{\sqrt{3}}{4}i$ :
          \par
          $r^2=\left(\frac{3}{4}\right)^2+\left(\frac{\sqrt{3}}{4}\right)^2=\frac{12}{16}=\frac{3}{4}$
          \par
          Donc :
          \par
          $r=\frac{\sqrt{3}}{2}$
          \par
          $\frac{3}{4}+\frac{\sqrt{3}}{4}i=\frac{\sqrt{3}}{2}\left(\frac{\sqrt{3}}{2}+\frac{1}{2}i\right)$
          \par
          Si $\theta $ est un argument de  $\frac{3}{4}+\frac{\sqrt{3}}{4}i$ :
          \par
          $cos \theta  = \frac{\sqrt{3}}{2}$ et $\sin \theta  = \frac{1}{2}$ donc $\theta  = \frac{\pi }{6} + 2k\pi $.
          \par
          La forme exponentielle du nombre complexe $\frac{3}{4}+\frac{\sqrt{3}}{4}i$ est donc $\frac{\sqrt{3}}{2}e^{i\frac{\pi }{6}}$
          \item
          \begin{enumerate}[label=\alph*.]
               \item
               $ z_{n+1}=\left(\frac{3}{4}+\frac{\sqrt{3}}{4}i\right)z_{n}$ donc :
               \par
               $ |z_{n+1}|=\left|\frac{3}{4}+\frac{\sqrt{3}}{4}i\right|\times \left|z_{n}\right|$
               \par
               $r_{n+1}=\frac{\sqrt{3}}{2}r_{n}$
               \par
               La suite $\left(r_{n}\right)$ est donc une suite géométrique de raison $q=\frac{\sqrt{3}}{2}$ et de premier terme $r_{0}=|z_{0}|=1$.
               \item
               $r_{n}=r_{0}\times q^{n}=\left(\frac{\sqrt{3}}{2}\right)^{n}$
               \item
               $OA_{n}=r_{n}$.
               \par
               $\left(r_{n}\right)$ est une suite géométrique de raison $q=\frac{\sqrt{3}}{2}$. Comme $0 < q < 1$ la suite $\left(r_{n}\right)$ \textbf{converge vers 0} lorsque $n$ tend vers $+ \infty $ .
          \end{enumerate}
          \item
          \begin{enumerate}[label=\alph*.]
               \item
               Voici les valeurs prises par les variables lors de l'exécution pas à pas de l'algorithme pour $P=0,5$:
               \begin{tabularx}{0.8\linewidth}{|*{3}{>{\centering \arraybackslash }X|}}%class="compact" width="600"
                    \hline
                    $     n     $  &  $     R     $  &  $     P     $  &  condition $R > P$
                    \\ \hline
                    0  &  1  &  0,5  &  Vraie
                    \\ \hline
                    1  &  0,866  &  0,5  &  Vraie
                    \\ \hline
                    2  &  0,75  &  0,5  &  Vraie
                    \\ \hline
                    3  &  0,6495  &  0,5  &  Vraie
                    \\ \hline
                    4  &  0,5625  &  0,5  &  Vraie
                    \\ \hline
                    5  &  0,487  &  0,5  &  Fausse
                    \\ \hline
               \end{tabularx}
               A la fin,\textbf{ l'algorithme affiche la valeur $5$}.
               \item
               Cet algorithme affiche la plus petite valeur de $n$ telle que $OA_{n} \leqslant  P$.
          \end{enumerate}
          \item
          \begin{enumerate}[label=\alph*.]
               \item
               $OA_{n}=r_{n} ,  OA_{n+1}=r_{n+1}=\frac{\sqrt{3}}{2}r_{n}$ et :
               \par
               $A_{n}A_{n+1}= | z_{n+1}-z_{n} | = \left| \left(\frac{3}{4}+\frac{\sqrt{3}}{4}i\right)z_{n}-z_{n} \right|  = \left| \left(-\frac{1}{4} + \frac{\sqrt{3}}{4}i\right) z_{n} \right|$
               \par
               $A_{n}A_{n+1}= \left| \left(-\frac{1}{4} + \frac{\sqrt{3}}{4}i\right) \right| \times  r_{n} $
               \par
               Or :
               \par
               $\left| \left(-\frac{1}{4} + \frac{\sqrt{3}}{4}i\right) \right| ^{2} = \frac{1}{16}+\frac{3}{16}=\frac{1}{4}$
               \par
               donc $\left| \left(-\frac{1}{4} + \frac{\sqrt{3}}{4}i\right) \right| = \frac{1}{2}$ et $A_{n}A_{n+1}=\frac{1}{2}r_{n}$
               \par
               Finalement :
               \par
               $ OA_{n+1}^{2} + A_{n}A_{n+1}^{2} = \frac{3}{4}r_{n}^{2}+\frac{1}{4}r_{n}^{2} = r_{n}^{2} = OA_{n}^{2}$
               \par
               Donc, d'après la réciproque du théorème de Pythagore, le triangle $OA_{n}A_{n+1}$ est rectangle en $A_{n+1}$.
               \item
               $z_{n}=r_{n} \left(\cos\frac{n\pi }{6}+i \sin\frac{n\pi }{6}\right)$
               \par
               Le point $A_{n}$ appartient à l'axe des ordonnées si et seulement si $\cos\frac{n\pi }{6} = 0$, c'est à dire $\frac{n\pi }{6}=\frac{\pi }{2}+2k\pi $ ou $n\frac{\pi }{6}=\left(3\frac{\pi }{2}\right)+2k\pi $ ou encore $\frac{n\pi }{6}=\frac{\pi }{2} + k\pi $ avec $k \in  \mathbb{Z}$
               \par
               Or :
               \par
               $\frac{n\pi }{6}=\frac{\pi }{2} + k\pi   \Leftrightarrow  \left(n\pi \right)=3\pi  + 6k\pi    \Leftrightarrow  n= 3 + 6k$  (avec $k \in  \mathbb{Z}$)
               \par
               Comme $n\geqslant 0$, $k$ doit être positif ou nul (donc appartenir à $\mathbb{N}$).
               \par
               Les valeurs de $n$  pour lesquelles $A_{n}$ est un point de l'axe des ordonnées sont donc
               \par
               $n= 3 + 6k$  avec $k \in  \mathbb{N}$ (soit $n = 3, 9, 15, 21,$ etc.).
               \item

\begin{center}
\imgsvg{mc-0274}{0.3}% alt="Nombres complexes - Bac S Pondichéry 2014 corrigé" style="width:35rem" 
\end{center}
               Pour la construction (à l'équerre ou au compas) on utilise le fait que les triangles $OA_{n}A_{n+1}$ sont rectangles en $A_{n+1}$.
          \end{enumerate}
     \end{enumerate}
\end{corrige}

\end{document}