\documentclass[a4paper]{article}

%================================================================================================================================
%
% Packages
%
%================================================================================================================================

\usepackage[T1]{fontenc} 	% pour caractères accentués
\usepackage[utf8]{inputenc}  % encodage utf8
\usepackage[french]{babel}	% langue : français
\usepackage{fourier}			% caractères plus lisibles
\usepackage[dvipsnames]{xcolor} % couleurs
\usepackage{fancyhdr}		% réglage header footer
\usepackage{needspace}		% empêcher sauts de page mal placés
\usepackage{graphicx}		% pour inclure des graphiques
\usepackage{enumitem,cprotect}		% personnalise les listes d'items (nécessaire pour ol, al ...)
\usepackage{hyperref}		% Liens hypertexte
\usepackage{pstricks,pst-all,pst-node,pstricks-add,pst-math,pst-plot,pst-tree,pst-eucl} % pstricks
\usepackage[a4paper,includeheadfoot,top=2cm,left=3cm, bottom=2cm,right=3cm]{geometry} % marges etc.
\usepackage{comment}			% commentaires multilignes
\usepackage{amsmath,environ} % maths (matrices, etc.)
\usepackage{amssymb,makeidx}
\usepackage{bm}				% bold maths
\usepackage{tabularx}		% tableaux
\usepackage{colortbl}		% tableaux en couleur
\usepackage{fontawesome}		% Fontawesome
\usepackage{environ}			% environment with command
\usepackage{fp}				% calculs pour ps-tricks
\usepackage{multido}			% pour ps tricks
\usepackage[np]{numprint}	% formattage nombre
\usepackage{tikz,tkz-tab} 			% package principal TikZ
\usepackage{pgfplots}   % axes
\usepackage{mathrsfs}    % cursives
\usepackage{calc}			% calcul taille boites
\usepackage[scaled=0.875]{helvet} % font sans serif
\usepackage{svg} % svg
\usepackage{scrextend} % local margin
\usepackage{scratch} %scratch
\usepackage{multicol} % colonnes
%\usepackage{infix-RPN,pst-func} % formule en notation polanaise inversée
\usepackage{listings}

%================================================================================================================================
%
% Réglages de base
%
%================================================================================================================================

\lstset{
language=Python,   % R code
literate=
{á}{{\'a}}1
{à}{{\`a}}1
{ã}{{\~a}}1
{é}{{\'e}}1
{è}{{\`e}}1
{ê}{{\^e}}1
{í}{{\'i}}1
{ó}{{\'o}}1
{õ}{{\~o}}1
{ú}{{\'u}}1
{ü}{{\"u}}1
{ç}{{\c{c}}}1
{~}{{ }}1
}


\definecolor{codegreen}{rgb}{0,0.6,0}
\definecolor{codegray}{rgb}{0.5,0.5,0.5}
\definecolor{codepurple}{rgb}{0.58,0,0.82}
\definecolor{backcolour}{rgb}{0.95,0.95,0.92}

\lstdefinestyle{mystyle}{
    backgroundcolor=\color{backcolour},   
    commentstyle=\color{codegreen},
    keywordstyle=\color{magenta},
    numberstyle=\tiny\color{codegray},
    stringstyle=\color{codepurple},
    basicstyle=\ttfamily\footnotesize,
    breakatwhitespace=false,         
    breaklines=true,                 
    captionpos=b,                    
    keepspaces=true,                 
    numbers=left,                    
xleftmargin=2em,
framexleftmargin=2em,            
    showspaces=false,                
    showstringspaces=false,
    showtabs=false,                  
    tabsize=2,
    upquote=true
}

\lstset{style=mystyle}


\lstset{style=mystyle}
\newcommand{\imgdir}{C:/laragon/www/newmc/assets/imgsvg/}
\newcommand{\imgsvgdir}{C:/laragon/www/newmc/assets/imgsvg/}

\definecolor{mcgris}{RGB}{220, 220, 220}% ancien~; pour compatibilité
\definecolor{mcbleu}{RGB}{52, 152, 219}
\definecolor{mcvert}{RGB}{125, 194, 70}
\definecolor{mcmauve}{RGB}{154, 0, 215}
\definecolor{mcorange}{RGB}{255, 96, 0}
\definecolor{mcturquoise}{RGB}{0, 153, 153}
\definecolor{mcrouge}{RGB}{255, 0, 0}
\definecolor{mclightvert}{RGB}{205, 234, 190}

\definecolor{gris}{RGB}{220, 220, 220}
\definecolor{bleu}{RGB}{52, 152, 219}
\definecolor{vert}{RGB}{125, 194, 70}
\definecolor{mauve}{RGB}{154, 0, 215}
\definecolor{orange}{RGB}{255, 96, 0}
\definecolor{turquoise}{RGB}{0, 153, 153}
\definecolor{rouge}{RGB}{255, 0, 0}
\definecolor{lightvert}{RGB}{205, 234, 190}
\setitemize[0]{label=\color{lightvert}  $\bullet$}

\pagestyle{fancy}
\renewcommand{\headrulewidth}{0.2pt}
\fancyhead[L]{maths-cours.fr}
\fancyhead[R]{\thepage}
\renewcommand{\footrulewidth}{0.2pt}
\fancyfoot[C]{}

\newcolumntype{C}{>{\centering\arraybackslash}X}
\newcolumntype{s}{>{\hsize=.35\hsize\arraybackslash}X}

\setlength{\parindent}{0pt}		 
\setlength{\parskip}{3mm}
\setlength{\headheight}{1cm}

\def\ebook{ebook}
\def\book{book}
\def\web{web}
\def\type{web}

\newcommand{\vect}[1]{\overrightarrow{\,\mathstrut#1\,}}

\def\Oij{$\left(\text{O}~;~\vect{\imath},~\vect{\jmath}\right)$}
\def\Oijk{$\left(\text{O}~;~\vect{\imath},~\vect{\jmath},~\vect{k}\right)$}
\def\Ouv{$\left(\text{O}~;~\vect{u},~\vect{v}\right)$}

\hypersetup{breaklinks=true, colorlinks = true, linkcolor = OliveGreen, urlcolor = OliveGreen, citecolor = OliveGreen, pdfauthor={Didier BONNEL - https://www.maths-cours.fr} } % supprime les bordures autour des liens

\renewcommand{\arg}[0]{\text{arg}}

\everymath{\displaystyle}

%================================================================================================================================
%
% Macros - Commandes
%
%================================================================================================================================

\newcommand\meta[2]{    			% Utilisé pour créer le post HTML.
	\def\titre{titre}
	\def\url{url}
	\def\arg{#1}
	\ifx\titre\arg
		\newcommand\maintitle{#2}
		\fancyhead[L]{#2}
		{\Large\sffamily \MakeUppercase{#2}}
		\vspace{1mm}\textcolor{mcvert}{\hrule}
	\fi 
	\ifx\url\arg
		\fancyfoot[L]{\href{https://www.maths-cours.fr#2}{\black \footnotesize{https://www.maths-cours.fr#2}}}
	\fi 
}


\newcommand\TitreC[1]{    		% Titre centré
     \needspace{3\baselineskip}
     \begin{center}\textbf{#1}\end{center}
}

\newcommand\newpar{    		% paragraphe
     \par
}

\newcommand\nosp {    		% commande vide (pas d'espace)
}
\newcommand{\id}[1]{} %ignore

\newcommand\boite[2]{				% Boite simple sans titre
	\vspace{5mm}
	\setlength{\fboxrule}{0.2mm}
	\setlength{\fboxsep}{5mm}	
	\fcolorbox{#1}{#1!3}{\makebox[\linewidth-2\fboxrule-2\fboxsep]{
  		\begin{minipage}[t]{\linewidth-2\fboxrule-4\fboxsep}\setlength{\parskip}{3mm}
  			 #2
  		\end{minipage}
	}}
	\vspace{5mm}
}

\newcommand\CBox[4]{				% Boites
	\vspace{5mm}
	\setlength{\fboxrule}{0.2mm}
	\setlength{\fboxsep}{5mm}
	
	\fcolorbox{#1}{#1!3}{\makebox[\linewidth-2\fboxrule-2\fboxsep]{
		\begin{minipage}[t]{1cm}\setlength{\parskip}{3mm}
	  		\textcolor{#1}{\LARGE{#2}}    
 	 	\end{minipage}  
  		\begin{minipage}[t]{\linewidth-2\fboxrule-4\fboxsep}\setlength{\parskip}{3mm}
			\raisebox{1.2mm}{\normalsize\sffamily{\textcolor{#1}{#3}}}						
  			 #4
  		\end{minipage}
	}}
	\vspace{5mm}
}

\newcommand\cadre[3]{				% Boites convertible html
	\par
	\vspace{2mm}
	\setlength{\fboxrule}{0.1mm}
	\setlength{\fboxsep}{5mm}
	\fcolorbox{#1}{white}{\makebox[\linewidth-2\fboxrule-2\fboxsep]{
  		\begin{minipage}[t]{\linewidth-2\fboxrule-4\fboxsep}\setlength{\parskip}{3mm}
			\raisebox{-2.5mm}{\sffamily \small{\textcolor{#1}{\MakeUppercase{#2}}}}		
			\par		
  			 #3
 	 		\end{minipage}
	}}
		\vspace{2mm}
	\par
}

\newcommand\bloc[3]{				% Boites convertible html sans bordure
     \needspace{2\baselineskip}
     {\sffamily \small{\textcolor{#1}{\MakeUppercase{#2}}}}    
		\par		
  			 #3
		\par
}

\newcommand\CHelp[1]{
     \CBox{Plum}{\faInfoCircle}{À RETENIR}{#1}
}

\newcommand\CUp[1]{
     \CBox{NavyBlue}{\faThumbsOUp}{EN PRATIQUE}{#1}
}

\newcommand\CInfo[1]{
     \CBox{Sepia}{\faArrowCircleRight}{REMARQUE}{#1}
}

\newcommand\CRedac[1]{
     \CBox{PineGreen}{\faEdit}{BIEN R\'EDIGER}{#1}
}

\newcommand\CError[1]{
     \CBox{Red}{\faExclamationTriangle}{ATTENTION}{#1}
}

\newcommand\TitreExo[2]{
\needspace{4\baselineskip}
 {\sffamily\large EXERCICE #1\ (\emph{#2 points})}
\vspace{5mm}
}

\newcommand\img[2]{
          \includegraphics[width=#2\paperwidth]{\imgdir#1}
}

\newcommand\imgsvg[2]{
       \begin{center}   \includegraphics[width=#2\paperwidth]{\imgsvgdir#1} \end{center}
}


\newcommand\Lien[2]{
     \href{#1}{#2 \tiny \faExternalLink}
}
\newcommand\mcLien[2]{
     \href{https~://www.maths-cours.fr/#1}{#2 \tiny \faExternalLink}
}

\newcommand{\euro}{\eurologo{}}

%================================================================================================================================
%
% Macros - Environement
%
%================================================================================================================================

\newenvironment{tex}{ %
}
{%
}

\newenvironment{indente}{ %
	\setlength\parindent{10mm}
}

{
	\setlength\parindent{0mm}
}

\newenvironment{corrige}{%
     \needspace{3\baselineskip}
     \medskip
     \textbf{\textsc{Corrigé}}
     \medskip
}
{
}

\newenvironment{extern}{%
     \begin{center}
     }
     {
     \end{center}
}

\NewEnviron{code}{%
	\par
     \boite{gray}{\texttt{%
     \BODY
     }}
     \par
}

\newenvironment{vbloc}{% boite sans cadre empeche saut de page
     \begin{minipage}[t]{\linewidth}
     }
     {
     \end{minipage}
}
\NewEnviron{h2}{%
    \needspace{3\baselineskip}
    \vspace{0.6cm}
	\noindent \MakeUppercase{\sffamily \large \BODY}
	\vspace{1mm}\textcolor{mcgris}{\hrule}\vspace{0.4cm}
	\par
}{}

\NewEnviron{h3}{%
    \needspace{3\baselineskip}
	\vspace{5mm}
	\textsc{\BODY}
	\par
}

\NewEnviron{margeneg}{ %
\begin{addmargin}[-1cm]{0cm}
\BODY
\end{addmargin}
}

\NewEnviron{html}{%
}

\begin{document}
\meta{url}{/exercices/probabilites-bac-s-amerique-du-nord-2018/}
\meta{pid}{7916}
\meta{titre}{Probabilités - Bac S Amérique du Nord  2018}
\meta{type}{exercices}
%
\begin{h2}Exercice 1 (6 points)\end{h2}
\textbf{Commun à  tous les candidats}
\medskip
On étudie certaines caractéristiques d'un supermarché d'une petite ville.
\bigskip
\begin{center}\begin{h3}Partie A - Démonstration préliminaire \end{h3}\end{center}
\medskip
Soit $X$ une variable aléatoire qui suit la loi exponentielle de paramètre $0,2$.
\par
On rappelle que l'espérance de la variable aléatoire $X$, notée $E(X)$, est égale à~:
\par
\[\displaystyle\lim_{x \to + \infty}\displaystyle\int_{0}^{x}  0,2t\text{e}^{-0,2t}\:\text{d}t.\]
\par
Le but de cette partie est de démontrer que $E(X) = 5$.
\medskip
\begin{enumerate}
     \item On note $g$ la fonction définie sur l'intervalle $[0~:~+\infty[$ par $g(t) = 0,2t\text{e}^{-0,2t}$.
     \par
     On définit la fonction $G$ sur l'intervalle $[0~:~+\infty[$ par $G(t) = (- t - 5)\text{e}^{-0,2t}$.
     \par
     Vérifier que $G$ est une primitive de $g$ sur l'intervalle $[0~:~+\infty[$.
     \item  En déduire que la valeur exacte de $E(X)$ est 5.
     \par
     \emph{Indication~: on pourra utiliser, sans le démontrer, le résultat suivant }~:
     \par
     \[\displaystyle\lim_{x \to + \infty} x \text{e}^{- 0,2x} = 0.\]
\end{enumerate}
\bigskip
\begin{center}\begin{h3}Partie B - Étude de la durée de présence d'un client dans le supermarché \end{h3}\end{center}
\medskip
Une étude commandée par le gérant du supermarché permet de modéliser la durée, exprimée en
minutes, passée dans le supermarché par un client choisi au hasard par une variable aléatoire $T$.
\par
Cette variable $T$ suit une loi normale d'espérance $40$ minutes et d'écart type un réel positif noté $\sigma$.
\par
Grâce à cette étude, on estime que $P(T < 10) = 0,067$.
\medskip
\begin{enumerate}
     \item Déterminer une valeur arrondie du réel $\sigma$ à la seconde près.
     \item Dans cette question, on prend $\sigma = 20$~minutes. Quelle est alors la proportion de clients qui
     passent plus d'une heure dans le supermarché~?
\end{enumerate}
\bigskip
\begin{center}\begin{h3}Partie C - Durée d'attente pour le paiement \end{h3}\end{center}
\medskip
Ce supermarché laisse le choix au client d'utiliser seul des bornes automatiques de paiement ou
bien de passer par une caisse gérée par un opérateur.
\medskip
\begin{enumerate}
     \item La durée d'attente à une borne automatique, exprimée en minutes, est modélisée par une
     variable aléatoire qui suit la loi exponentielle de paramètre $0,2$~min$^{-1}$.
     \begin{enumerate}[label=\alph*.]
          \item Donner la durée moyenne d'attente d'un client à une borne automatique de paiement.
          \item Calculer la probabilité, arrondie à $10^{-3}$, que la durée d'attente d'un client à une borne automatique de paiement soit supérieure à $10$ minutes.
     \end{enumerate}
     \item L'étude commandée par le gérant conduit à la modélisation suivante~:
     \begin{indent}
          \begin{itemize}
               \item parmi les clients ayant choisi de passer à une borne automatique, 86\,\% attendent moins de $10$ minutes~:
               \item parmi les clients passant en caisse, 63\,\% attendent moins de $10$ minutes.
          \end{itemize}
     \end{indent}
     \medskip
     On choisit un client du magasin au hasard et on définit les événements suivants~:
     \par
     $B$~: \og le client paye à une borne automatique \fg{}~:
     \par
     $\overline{B}$~: \og le client paye à une caisse avec opérateur \fg{}~:
     \par
     $S$~: \og la durée d'attente du client lors du paiement est inférieure à $10$ minutes \fg.
     \par
     Une attente supérieure à dix minutes à une caisse avec opérateur ou à une borne automatique
     engendre chez le client une perception négative du magasin. Le gérant souhaite que
     plus de 75\,\% des clients attendent moins de $10$ minutes.
     \par
     Quelle est la proportion minimale de clients qui doivent choisir une borne automatique de
     paiement pour que cet objectif soit atteint~?
\end{enumerate}
\bigskip
\begin{center}\begin{h3}Partie D - Bons d'achat \end{h3}\end{center}
\medskip
Lors du paiement, des cartes à gratter, gagnantes ou perdantes, sont distribuées aux clients. Le
nombre de cartes distribuées dépend du montant des achats. Chaque client a droit à une carte à
gratter par tranche de $10$~\euro{} d'achats.
\par
Par exemple, si le montant des achats est 58,64~\euro, alors le client obtient $5$ cartes~: si le montant est
124,31~\euro, le client obtient $12$~cartes.
\par
Les cartes gagnantes représentent $0,5$\,\% de l'ensemble du stock de cartes. De plus, ce stock est
suffisamment grand pour assimiler la distribution d'une carte à un tirage avec remise.
\medskip
\begin{enumerate}
     \item Un client effectue des achats pour un montant de 158,02~\euro.
     \par
     Quelle est la probabilité, arrondie à $10^{-2}$, qu'il obtienne au moins une carte gagnante~?
     \item  À partir de quel montant d'achats, arrondi à 10~\euro, la probabilité d'obtenir au moins une carte
     gagnante est-elle supérieure à 50\,\%~?
\end{enumerate}

\end{document}