\documentclass[a4paper]{article}

%================================================================================================================================
%
% Packages
%
%================================================================================================================================

\usepackage[T1]{fontenc} 	% pour caractères accentués
\usepackage[utf8]{inputenc}  % encodage utf8
\usepackage[french]{babel}	% langue : français
\usepackage{fourier}			% caractères plus lisibles
\usepackage[dvipsnames]{xcolor} % couleurs
\usepackage{fancyhdr}		% réglage header footer
\usepackage{needspace}		% empêcher sauts de page mal placés
\usepackage{graphicx}		% pour inclure des graphiques
\usepackage{enumitem,cprotect}		% personnalise les listes d'items (nécessaire pour ol, al ...)
\usepackage{hyperref}		% Liens hypertexte
\usepackage{pstricks,pst-all,pst-node,pstricks-add,pst-math,pst-plot,pst-tree,pst-eucl} % pstricks
\usepackage[a4paper,includeheadfoot,top=2cm,left=3cm, bottom=2cm,right=3cm]{geometry} % marges etc.
\usepackage{comment}			% commentaires multilignes
\usepackage{amsmath,environ} % maths (matrices, etc.)
\usepackage{amssymb,makeidx}
\usepackage{bm}				% bold maths
\usepackage{tabularx}		% tableaux
\usepackage{colortbl}		% tableaux en couleur
\usepackage{fontawesome}		% Fontawesome
\usepackage{environ}			% environment with command
\usepackage{fp}				% calculs pour ps-tricks
\usepackage{multido}			% pour ps tricks
\usepackage[np]{numprint}	% formattage nombre
\usepackage{tikz,tkz-tab} 			% package principal TikZ
\usepackage{pgfplots}   % axes
\usepackage{mathrsfs}    % cursives
\usepackage{calc}			% calcul taille boites
\usepackage[scaled=0.875]{helvet} % font sans serif
\usepackage{svg} % svg
\usepackage{scrextend} % local margin
\usepackage{scratch} %scratch
\usepackage{multicol} % colonnes
%\usepackage{infix-RPN,pst-func} % formule en notation polanaise inversée
\usepackage{listings}

%================================================================================================================================
%
% Réglages de base
%
%================================================================================================================================

\lstset{
language=Python,   % R code
literate=
{á}{{\'a}}1
{à}{{\`a}}1
{ã}{{\~a}}1
{é}{{\'e}}1
{è}{{\`e}}1
{ê}{{\^e}}1
{í}{{\'i}}1
{ó}{{\'o}}1
{õ}{{\~o}}1
{ú}{{\'u}}1
{ü}{{\"u}}1
{ç}{{\c{c}}}1
{~}{{ }}1
}


\definecolor{codegreen}{rgb}{0,0.6,0}
\definecolor{codegray}{rgb}{0.5,0.5,0.5}
\definecolor{codepurple}{rgb}{0.58,0,0.82}
\definecolor{backcolour}{rgb}{0.95,0.95,0.92}

\lstdefinestyle{mystyle}{
    backgroundcolor=\color{backcolour},   
    commentstyle=\color{codegreen},
    keywordstyle=\color{magenta},
    numberstyle=\tiny\color{codegray},
    stringstyle=\color{codepurple},
    basicstyle=\ttfamily\footnotesize,
    breakatwhitespace=false,         
    breaklines=true,                 
    captionpos=b,                    
    keepspaces=true,                 
    numbers=left,                    
xleftmargin=2em,
framexleftmargin=2em,            
    showspaces=false,                
    showstringspaces=false,
    showtabs=false,                  
    tabsize=2,
    upquote=true
}

\lstset{style=mystyle}


\lstset{style=mystyle}
\newcommand{\imgdir}{C:/laragon/www/newmc/assets/imgsvg/}
\newcommand{\imgsvgdir}{C:/laragon/www/newmc/assets/imgsvg/}

\definecolor{mcgris}{RGB}{220, 220, 220}% ancien~; pour compatibilité
\definecolor{mcbleu}{RGB}{52, 152, 219}
\definecolor{mcvert}{RGB}{125, 194, 70}
\definecolor{mcmauve}{RGB}{154, 0, 215}
\definecolor{mcorange}{RGB}{255, 96, 0}
\definecolor{mcturquoise}{RGB}{0, 153, 153}
\definecolor{mcrouge}{RGB}{255, 0, 0}
\definecolor{mclightvert}{RGB}{205, 234, 190}

\definecolor{gris}{RGB}{220, 220, 220}
\definecolor{bleu}{RGB}{52, 152, 219}
\definecolor{vert}{RGB}{125, 194, 70}
\definecolor{mauve}{RGB}{154, 0, 215}
\definecolor{orange}{RGB}{255, 96, 0}
\definecolor{turquoise}{RGB}{0, 153, 153}
\definecolor{rouge}{RGB}{255, 0, 0}
\definecolor{lightvert}{RGB}{205, 234, 190}
\setitemize[0]{label=\color{lightvert}  $\bullet$}

\pagestyle{fancy}
\renewcommand{\headrulewidth}{0.2pt}
\fancyhead[L]{maths-cours.fr}
\fancyhead[R]{\thepage}
\renewcommand{\footrulewidth}{0.2pt}
\fancyfoot[C]{}

\newcolumntype{C}{>{\centering\arraybackslash}X}
\newcolumntype{s}{>{\hsize=.35\hsize\arraybackslash}X}

\setlength{\parindent}{0pt}		 
\setlength{\parskip}{3mm}
\setlength{\headheight}{1cm}

\def\ebook{ebook}
\def\book{book}
\def\web{web}
\def\type{web}

\newcommand{\vect}[1]{\overrightarrow{\,\mathstrut#1\,}}

\def\Oij{$\left(\text{O}~;~\vect{\imath},~\vect{\jmath}\right)$}
\def\Oijk{$\left(\text{O}~;~\vect{\imath},~\vect{\jmath},~\vect{k}\right)$}
\def\Ouv{$\left(\text{O}~;~\vect{u},~\vect{v}\right)$}

\hypersetup{breaklinks=true, colorlinks = true, linkcolor = OliveGreen, urlcolor = OliveGreen, citecolor = OliveGreen, pdfauthor={Didier BONNEL - https://www.maths-cours.fr} } % supprime les bordures autour des liens

\renewcommand{\arg}[0]{\text{arg}}

\everymath{\displaystyle}

%================================================================================================================================
%
% Macros - Commandes
%
%================================================================================================================================

\newcommand\meta[2]{    			% Utilisé pour créer le post HTML.
	\def\titre{titre}
	\def\url{url}
	\def\arg{#1}
	\ifx\titre\arg
		\newcommand\maintitle{#2}
		\fancyhead[L]{#2}
		{\Large\sffamily \MakeUppercase{#2}}
		\vspace{1mm}\textcolor{mcvert}{\hrule}
	\fi 
	\ifx\url\arg
		\fancyfoot[L]{\href{https://www.maths-cours.fr#2}{\black \footnotesize{https://www.maths-cours.fr#2}}}
	\fi 
}


\newcommand\TitreC[1]{    		% Titre centré
     \needspace{3\baselineskip}
     \begin{center}\textbf{#1}\end{center}
}

\newcommand\newpar{    		% paragraphe
     \par
}

\newcommand\nosp {    		% commande vide (pas d'espace)
}
\newcommand{\id}[1]{} %ignore

\newcommand\boite[2]{				% Boite simple sans titre
	\vspace{5mm}
	\setlength{\fboxrule}{0.2mm}
	\setlength{\fboxsep}{5mm}	
	\fcolorbox{#1}{#1!3}{\makebox[\linewidth-2\fboxrule-2\fboxsep]{
  		\begin{minipage}[t]{\linewidth-2\fboxrule-4\fboxsep}\setlength{\parskip}{3mm}
  			 #2
  		\end{minipage}
	}}
	\vspace{5mm}
}

\newcommand\CBox[4]{				% Boites
	\vspace{5mm}
	\setlength{\fboxrule}{0.2mm}
	\setlength{\fboxsep}{5mm}
	
	\fcolorbox{#1}{#1!3}{\makebox[\linewidth-2\fboxrule-2\fboxsep]{
		\begin{minipage}[t]{1cm}\setlength{\parskip}{3mm}
	  		\textcolor{#1}{\LARGE{#2}}    
 	 	\end{minipage}  
  		\begin{minipage}[t]{\linewidth-2\fboxrule-4\fboxsep}\setlength{\parskip}{3mm}
			\raisebox{1.2mm}{\normalsize\sffamily{\textcolor{#1}{#3}}}						
  			 #4
  		\end{minipage}
	}}
	\vspace{5mm}
}

\newcommand\cadre[3]{				% Boites convertible html
	\par
	\vspace{2mm}
	\setlength{\fboxrule}{0.1mm}
	\setlength{\fboxsep}{5mm}
	\fcolorbox{#1}{white}{\makebox[\linewidth-2\fboxrule-2\fboxsep]{
  		\begin{minipage}[t]{\linewidth-2\fboxrule-4\fboxsep}\setlength{\parskip}{3mm}
			\raisebox{-2.5mm}{\sffamily \small{\textcolor{#1}{\MakeUppercase{#2}}}}		
			\par		
  			 #3
 	 		\end{minipage}
	}}
		\vspace{2mm}
	\par
}

\newcommand\bloc[3]{				% Boites convertible html sans bordure
     \needspace{2\baselineskip}
     {\sffamily \small{\textcolor{#1}{\MakeUppercase{#2}}}}    
		\par		
  			 #3
		\par
}

\newcommand\CHelp[1]{
     \CBox{Plum}{\faInfoCircle}{À RETENIR}{#1}
}

\newcommand\CUp[1]{
     \CBox{NavyBlue}{\faThumbsOUp}{EN PRATIQUE}{#1}
}

\newcommand\CInfo[1]{
     \CBox{Sepia}{\faArrowCircleRight}{REMARQUE}{#1}
}

\newcommand\CRedac[1]{
     \CBox{PineGreen}{\faEdit}{BIEN R\'EDIGER}{#1}
}

\newcommand\CError[1]{
     \CBox{Red}{\faExclamationTriangle}{ATTENTION}{#1}
}

\newcommand\TitreExo[2]{
\needspace{4\baselineskip}
 {\sffamily\large EXERCICE #1\ (\emph{#2 points})}
\vspace{5mm}
}

\newcommand\img[2]{
          \includegraphics[width=#2\paperwidth]{\imgdir#1}
}

\newcommand\imgsvg[2]{
       \begin{center}   \includegraphics[width=#2\paperwidth]{\imgsvgdir#1} \end{center}
}


\newcommand\Lien[2]{
     \href{#1}{#2 \tiny \faExternalLink}
}
\newcommand\mcLien[2]{
     \href{https~://www.maths-cours.fr/#1}{#2 \tiny \faExternalLink}
}

\newcommand{\euro}{\eurologo{}}

%================================================================================================================================
%
% Macros - Environement
%
%================================================================================================================================

\newenvironment{tex}{ %
}
{%
}

\newenvironment{indente}{ %
	\setlength\parindent{10mm}
}

{
	\setlength\parindent{0mm}
}

\newenvironment{corrige}{%
     \needspace{3\baselineskip}
     \medskip
     \textbf{\textsc{Corrigé}}
     \medskip
}
{
}

\newenvironment{extern}{%
     \begin{center}
     }
     {
     \end{center}
}

\NewEnviron{code}{%
	\par
     \boite{gray}{\texttt{%
     \BODY
     }}
     \par
}

\newenvironment{vbloc}{% boite sans cadre empeche saut de page
     \begin{minipage}[t]{\linewidth}
     }
     {
     \end{minipage}
}
\NewEnviron{h2}{%
    \needspace{3\baselineskip}
    \vspace{0.6cm}
	\noindent \MakeUppercase{\sffamily \large \BODY}
	\vspace{1mm}\textcolor{mcgris}{\hrule}\vspace{0.4cm}
	\par
}{}

\NewEnviron{h3}{%
    \needspace{3\baselineskip}
	\vspace{5mm}
	\textsc{\BODY}
	\par
}

\NewEnviron{margeneg}{ %
\begin{addmargin}[-1cm]{0cm}
\BODY
\end{addmargin}
}

\NewEnviron{html}{%
}

\begin{document}
\meta{url}{/exercices/fonctions-loi-normale-bac-es-pondichery-2013/}
\meta{pid}{2044}
\meta{titre}{Fonctions - Loi normale  - Bac ES/L Pondichéry 2013}
\meta{type}{exercices}
%
\begin{h2}Exercice 4   (6 points)\end{h2}
\textbf{Commun à tous les candidats}
\par
\textit{La partie C peut être traitée indépendamment des parties A et B.}
\begin{h3}PARTIE A\end{h3}
On désigne part la fonction définie sur l'intervalle $\left[0 ; 6\right]$ par
\par
$f\left(x\right)=1-\left(x+1\right)e^{- x}$
\begin{enumerate}
     \item
     Montrer que $f^{\prime}\left(x\right)=xe^{- x}$ où $f^{\prime}$ désigne la fonction dérivée de la fonction $f$.
     \item
     Démontrer que l'équation $f\left(x\right)=0,5$ admet une solution unique $\alpha $ sur l'intervalle [0 ; 6].
     \par
     Déterminer une valeur arrondie de $\alpha $ à $0,01$.
     \item
     On admet que la fonction $F$ définie sur $\left[0 ; 6\right]$ par $F\left(x\right)=x+\left(x+2\right)e^{- x}$ est une primitive de $f$ sur $\left[0 ; 6\right]$. Donner la valeur exacte puis une valeur arrondie à $10^{-3}$ de $I=\int_{0}^{6} f\left(x\right)dx$.
\end{enumerate}
\begin{h3}PARTIE B\end{h3}
Une entreprise lance la production de batteries pour véhicules électriques.
\par
Une étude a modélisé le rythme de la production journalière sur les six premiers mois à l'aide de la fonction $f$ définie dans la partie \textbf{A} pour $x$ compris entre 0 et 6.
\par
$x$ représente le nombre de mois (de 30 jours) depuis le lancement du produit.
\par
$f\left(x\right)$ représente la production journalière de batteries en milliers.
\begin{enumerate}
     \item
     Exprimer en mois, puis en jours, le moment où la production atteindra $0,5$ millier soit $500$ unités.
     \item
     Déterminer une valeur arrondie à $10^{-3}$ de la valeur moyenne, exprimée en milliers, de la production sur les six premiers mois.
\end{enumerate}
\begin{h3}
PARTIE C\end{h3}
Il est prévu que l'autonomie permise par ce type de batteries, sous certaines conditions de conduite, soit de 200 km.
\par
Sur un parcours joignant une ville située à 160 km, on suppose que l'autonomie, exprimée en km, permise par ces batteries suit une loi normale d'espérance $\mu =200$ et d'écart-type $\sigma =40$.
\begin{enumerate}
     \item
     Quelle est la probabilité, arrondie au centième, de ne pas atteindre cette ville ?
     \item
     La probabilité de pouvoir faire l'aller-retour jusqu'à cette ville sans recharge des batteries est-elle supérieure à $0,01$ ? Justifier votre réponse.
\end{enumerate}
\begin{corrige}
     \begin{h3}PARTIE A\end{h3}
     On désigne part la fonction définie sur l'intervalle $\left[0 ; 6\right]$ par
     \par
     $f\left(x\right)=1-\left(x+1\right)e^{- x}$
     \begin{enumerate}
          \item
          $f$ est dérivable sur $\left[0;6\right]$ comme somme, produit et composée de fonctions dérivables.
          \par
          Pour dériver $\left(x+1\right)e^{- x}$ on emploie la formule $\left(uv\right)^{\prime}=u^{\prime}v+uv^{\prime}$ avec $u\left(x\right)=x+1$ et $v\left(x\right)=e^{-x}$.
          \par
          On a alors $u^{\prime}\left(x\right)=1$ et $v^{\prime}\left(x\right)=-e^{-x}$.
          \par
          Donc :
          \par
          $f^{\prime}\left(x\right)=0-\left[1\times e^{-x}+\left(x+1\right)\times \left(-e^{-x}\right)\right]=-\left(-xe^{-x}\right)=xe^{-x}$
          \item
          Sur $\left]0;6\right]$, $x$ est strictement positif ainsi que $e^{-x}$ (\textit{une exponentielle est toujours positive...}) donc $f^{\prime}\left(x\right) > 0$ sur $\left]0;6\right]$.
          \par
          La fonction $f$ est donc strictement croissante sur $\left[0;6\right]$.
          \par
          Par ailleurs : $f\left(0\right)=1-1=0$ et $f\left(6\right)=1-7e^{-6} \approx 0.96$
          \par
          $f$ est continue (car dérivable) et strictement croissante sur $\left[0;6\right]$. 0,5 est compris entre $f\left(0\right)=0$ et $f\left(6\right)\approx 0,98$. Donc l'équation $f\left(x\right)=0,5$ admet une et une seule solution $\alpha $ sur $\left[0;6\right]$.
          \par
          A la calculatrice on trouve $\alpha  \approx 1,68$ à $0,01$ près.
          \item
          $I=\int_{0}^{6} f\left(x\right)dx=\left[F\left(x\right)\right] _{0}^{6}=F\left(6\right)-F\left(0\right)=6+8e^{-6}-2=4+8e^{-6}$
          \par
          $I\approx 4,02$ à$10^{-3}$ près.
     \end{enumerate}
     \begin{h3}PARTIE B\end{h3}
     \begin{enumerate}
          \item
          Le moment (en mois) où la production atteindra 0,5 millier est la solution de l'équation $f\left(x\right)=0,5$.
          \par
          C'est donc le nombre $\alpha  \approx 1,68$ trouvé dans la partie A.
          \par
          Pour convertir $\alpha $ en jours, il suffit de multiplier par 30, ce qui donne 50,4.
          \par
          La production dépassera 0,5 millier à partir du 51ème jour.
          \item
          La valeur moyenne de la production au cours des 6 premiers mois est donnée par :
          \par
          $m=\frac{1}{6-0}\int_{0}^{6}f\left(x\right)dx=\frac{1}{6}\approx \frac{4,02}{6}\approx 0,67$ à $10^{-3}$ près.
     \end{enumerate}
     \begin{h3}
     PARTIE C\end{h3}
     \begin{enumerate}
          \item
          On ne peut pas atteindre cette ville si l'autonomie est inférieure à 160km. La probabilité cherchée est donc $p\left(X < 160\right)$.
          \par
          A la calculatrice (loi normale - espérance 200 - écart-type 40 - min 0  - max 160) on trouve $p\left(X < 160\right)\approx 0,16$ à $10^{-2}$ près.
          \textit{On peut aussi utiliser le fait que $p\left(160 < X < 240\right)=p\left(\mu -\sigma \leqslant X\leqslant \mu +\sigma \right)\approx 0,68$ (cf. cours)}. Donc l'évènement contraire a une probabilité :
          \par
          $p\left(X < 160 \cup X > 240\right)\approx 0,32$
          \par
          Par symétrie, $p\left(X < 160\right)=p\left(X > 240\right)$ donc : $p\left(X < 160\right)\approx \frac{1}{2}\times 0,32=0,16$
          \item
          On peut effectuer l'aller-retour jusqu'à cette ville sans recharge des batteries si l'autonomie est supérieure à 320km. La probabilité cherchée est donc $p\left(X\geqslant 320\right)$.
          \par
          A la calculatrice (loi normale - espérance 200 - écart-type 40 - min 320 - max 10^99) on trouve $p\left(X\geqslant 320\right)\approx 0,0013$ à $10^{-4}$ près.
          \par
          Cette probabilité est inférieure à 0,01.
\par
          \textit{On peut aussi utiliser le fait que $p\left(\mu -3\sigma \leqslant X \leqslant \mu +3 \sigma \right)\approx 0,0095$ (cf. cours) et un raisonnement analogue au \textbf{1.}
          }
     \end{enumerate}
\end{corrige}

\end{document}