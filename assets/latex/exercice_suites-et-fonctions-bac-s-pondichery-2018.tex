\documentclass[a4paper]{article}

%================================================================================================================================
%
% Packages
%
%================================================================================================================================

\usepackage[T1]{fontenc} 	% pour caractères accentués
\usepackage[utf8]{inputenc}  % encodage utf8
\usepackage[french]{babel}	% langue : français
\usepackage{fourier}			% caractères plus lisibles
\usepackage[dvipsnames]{xcolor} % couleurs
\usepackage{fancyhdr}		% réglage header footer
\usepackage{needspace}		% empêcher sauts de page mal placés
\usepackage{graphicx}		% pour inclure des graphiques
\usepackage{enumitem,cprotect}		% personnalise les listes d'items (nécessaire pour ol, al ...)
\usepackage{hyperref}		% Liens hypertexte
\usepackage{pstricks,pst-all,pst-node,pstricks-add,pst-math,pst-plot,pst-tree,pst-eucl} % pstricks
\usepackage[a4paper,includeheadfoot,top=2cm,left=3cm, bottom=2cm,right=3cm]{geometry} % marges etc.
\usepackage{comment}			% commentaires multilignes
\usepackage{amsmath,environ} % maths (matrices, etc.)
\usepackage{amssymb,makeidx}
\usepackage{bm}				% bold maths
\usepackage{tabularx}		% tableaux
\usepackage{colortbl}		% tableaux en couleur
\usepackage{fontawesome}		% Fontawesome
\usepackage{environ}			% environment with command
\usepackage{fp}				% calculs pour ps-tricks
\usepackage{multido}			% pour ps tricks
\usepackage[np]{numprint}	% formattage nombre
\usepackage{tikz,tkz-tab} 			% package principal TikZ
\usepackage{pgfplots}   % axes
\usepackage{mathrsfs}    % cursives
\usepackage{calc}			% calcul taille boites
\usepackage[scaled=0.875]{helvet} % font sans serif
\usepackage{svg} % svg
\usepackage{scrextend} % local margin
\usepackage{scratch} %scratch
\usepackage{multicol} % colonnes
%\usepackage{infix-RPN,pst-func} % formule en notation polanaise inversée
\usepackage{listings}

%================================================================================================================================
%
% Réglages de base
%
%================================================================================================================================

\lstset{
language=Python,   % R code
literate=
{á}{{\'a}}1
{à}{{\`a}}1
{ã}{{\~a}}1
{é}{{\'e}}1
{è}{{\`e}}1
{ê}{{\^e}}1
{í}{{\'i}}1
{ó}{{\'o}}1
{õ}{{\~o}}1
{ú}{{\'u}}1
{ü}{{\"u}}1
{ç}{{\c{c}}}1
{~}{{ }}1
}


\definecolor{codegreen}{rgb}{0,0.6,0}
\definecolor{codegray}{rgb}{0.5,0.5,0.5}
\definecolor{codepurple}{rgb}{0.58,0,0.82}
\definecolor{backcolour}{rgb}{0.95,0.95,0.92}

\lstdefinestyle{mystyle}{
    backgroundcolor=\color{backcolour},   
    commentstyle=\color{codegreen},
    keywordstyle=\color{magenta},
    numberstyle=\tiny\color{codegray},
    stringstyle=\color{codepurple},
    basicstyle=\ttfamily\footnotesize,
    breakatwhitespace=false,         
    breaklines=true,                 
    captionpos=b,                    
    keepspaces=true,                 
    numbers=left,                    
xleftmargin=2em,
framexleftmargin=2em,            
    showspaces=false,                
    showstringspaces=false,
    showtabs=false,                  
    tabsize=2,
    upquote=true
}

\lstset{style=mystyle}


\lstset{style=mystyle}
\newcommand{\imgdir}{C:/laragon/www/newmc/assets/imgsvg/}
\newcommand{\imgsvgdir}{C:/laragon/www/newmc/assets/imgsvg/}

\definecolor{mcgris}{RGB}{220, 220, 220}% ancien~; pour compatibilité
\definecolor{mcbleu}{RGB}{52, 152, 219}
\definecolor{mcvert}{RGB}{125, 194, 70}
\definecolor{mcmauve}{RGB}{154, 0, 215}
\definecolor{mcorange}{RGB}{255, 96, 0}
\definecolor{mcturquoise}{RGB}{0, 153, 153}
\definecolor{mcrouge}{RGB}{255, 0, 0}
\definecolor{mclightvert}{RGB}{205, 234, 190}

\definecolor{gris}{RGB}{220, 220, 220}
\definecolor{bleu}{RGB}{52, 152, 219}
\definecolor{vert}{RGB}{125, 194, 70}
\definecolor{mauve}{RGB}{154, 0, 215}
\definecolor{orange}{RGB}{255, 96, 0}
\definecolor{turquoise}{RGB}{0, 153, 153}
\definecolor{rouge}{RGB}{255, 0, 0}
\definecolor{lightvert}{RGB}{205, 234, 190}
\setitemize[0]{label=\color{lightvert}  $\bullet$}

\pagestyle{fancy}
\renewcommand{\headrulewidth}{0.2pt}
\fancyhead[L]{maths-cours.fr}
\fancyhead[R]{\thepage}
\renewcommand{\footrulewidth}{0.2pt}
\fancyfoot[C]{}

\newcolumntype{C}{>{\centering\arraybackslash}X}
\newcolumntype{s}{>{\hsize=.35\hsize\arraybackslash}X}

\setlength{\parindent}{0pt}		 
\setlength{\parskip}{3mm}
\setlength{\headheight}{1cm}

\def\ebook{ebook}
\def\book{book}
\def\web{web}
\def\type{web}

\newcommand{\vect}[1]{\overrightarrow{\,\mathstrut#1\,}}

\def\Oij{$\left(\text{O}~;~\vect{\imath},~\vect{\jmath}\right)$}
\def\Oijk{$\left(\text{O}~;~\vect{\imath},~\vect{\jmath},~\vect{k}\right)$}
\def\Ouv{$\left(\text{O}~;~\vect{u},~\vect{v}\right)$}

\hypersetup{breaklinks=true, colorlinks = true, linkcolor = OliveGreen, urlcolor = OliveGreen, citecolor = OliveGreen, pdfauthor={Didier BONNEL - https://www.maths-cours.fr} } % supprime les bordures autour des liens

\renewcommand{\arg}[0]{\text{arg}}

\everymath{\displaystyle}

%================================================================================================================================
%
% Macros - Commandes
%
%================================================================================================================================

\newcommand\meta[2]{    			% Utilisé pour créer le post HTML.
	\def\titre{titre}
	\def\url{url}
	\def\arg{#1}
	\ifx\titre\arg
		\newcommand\maintitle{#2}
		\fancyhead[L]{#2}
		{\Large\sffamily \MakeUppercase{#2}}
		\vspace{1mm}\textcolor{mcvert}{\hrule}
	\fi 
	\ifx\url\arg
		\fancyfoot[L]{\href{https://www.maths-cours.fr#2}{\black \footnotesize{https://www.maths-cours.fr#2}}}
	\fi 
}


\newcommand\TitreC[1]{    		% Titre centré
     \needspace{3\baselineskip}
     \begin{center}\textbf{#1}\end{center}
}

\newcommand\newpar{    		% paragraphe
     \par
}

\newcommand\nosp {    		% commande vide (pas d'espace)
}
\newcommand{\id}[1]{} %ignore

\newcommand\boite[2]{				% Boite simple sans titre
	\vspace{5mm}
	\setlength{\fboxrule}{0.2mm}
	\setlength{\fboxsep}{5mm}	
	\fcolorbox{#1}{#1!3}{\makebox[\linewidth-2\fboxrule-2\fboxsep]{
  		\begin{minipage}[t]{\linewidth-2\fboxrule-4\fboxsep}\setlength{\parskip}{3mm}
  			 #2
  		\end{minipage}
	}}
	\vspace{5mm}
}

\newcommand\CBox[4]{				% Boites
	\vspace{5mm}
	\setlength{\fboxrule}{0.2mm}
	\setlength{\fboxsep}{5mm}
	
	\fcolorbox{#1}{#1!3}{\makebox[\linewidth-2\fboxrule-2\fboxsep]{
		\begin{minipage}[t]{1cm}\setlength{\parskip}{3mm}
	  		\textcolor{#1}{\LARGE{#2}}    
 	 	\end{minipage}  
  		\begin{minipage}[t]{\linewidth-2\fboxrule-4\fboxsep}\setlength{\parskip}{3mm}
			\raisebox{1.2mm}{\normalsize\sffamily{\textcolor{#1}{#3}}}						
  			 #4
  		\end{minipage}
	}}
	\vspace{5mm}
}

\newcommand\cadre[3]{				% Boites convertible html
	\par
	\vspace{2mm}
	\setlength{\fboxrule}{0.1mm}
	\setlength{\fboxsep}{5mm}
	\fcolorbox{#1}{white}{\makebox[\linewidth-2\fboxrule-2\fboxsep]{
  		\begin{minipage}[t]{\linewidth-2\fboxrule-4\fboxsep}\setlength{\parskip}{3mm}
			\raisebox{-2.5mm}{\sffamily \small{\textcolor{#1}{\MakeUppercase{#2}}}}		
			\par		
  			 #3
 	 		\end{minipage}
	}}
		\vspace{2mm}
	\par
}

\newcommand\bloc[3]{				% Boites convertible html sans bordure
     \needspace{2\baselineskip}
     {\sffamily \small{\textcolor{#1}{\MakeUppercase{#2}}}}    
		\par		
  			 #3
		\par
}

\newcommand\CHelp[1]{
     \CBox{Plum}{\faInfoCircle}{À RETENIR}{#1}
}

\newcommand\CUp[1]{
     \CBox{NavyBlue}{\faThumbsOUp}{EN PRATIQUE}{#1}
}

\newcommand\CInfo[1]{
     \CBox{Sepia}{\faArrowCircleRight}{REMARQUE}{#1}
}

\newcommand\CRedac[1]{
     \CBox{PineGreen}{\faEdit}{BIEN R\'EDIGER}{#1}
}

\newcommand\CError[1]{
     \CBox{Red}{\faExclamationTriangle}{ATTENTION}{#1}
}

\newcommand\TitreExo[2]{
\needspace{4\baselineskip}
 {\sffamily\large EXERCICE #1\ (\emph{#2 points})}
\vspace{5mm}
}

\newcommand\img[2]{
          \includegraphics[width=#2\paperwidth]{\imgdir#1}
}

\newcommand\imgsvg[2]{
       \begin{center}   \includegraphics[width=#2\paperwidth]{\imgsvgdir#1} \end{center}
}


\newcommand\Lien[2]{
     \href{#1}{#2 \tiny \faExternalLink}
}
\newcommand\mcLien[2]{
     \href{https~://www.maths-cours.fr/#1}{#2 \tiny \faExternalLink}
}

\newcommand{\euro}{\eurologo{}}

%================================================================================================================================
%
% Macros - Environement
%
%================================================================================================================================

\newenvironment{tex}{ %
}
{%
}

\newenvironment{indente}{ %
	\setlength\parindent{10mm}
}

{
	\setlength\parindent{0mm}
}

\newenvironment{corrige}{%
     \needspace{3\baselineskip}
     \medskip
     \textbf{\textsc{Corrigé}}
     \medskip
}
{
}

\newenvironment{extern}{%
     \begin{center}
     }
     {
     \end{center}
}

\NewEnviron{code}{%
	\par
     \boite{gray}{\texttt{%
     \BODY
     }}
     \par
}

\newenvironment{vbloc}{% boite sans cadre empeche saut de page
     \begin{minipage}[t]{\linewidth}
     }
     {
     \end{minipage}
}
\NewEnviron{h2}{%
    \needspace{3\baselineskip}
    \vspace{0.6cm}
	\noindent \MakeUppercase{\sffamily \large \BODY}
	\vspace{1mm}\textcolor{mcgris}{\hrule}\vspace{0.4cm}
	\par
}{}

\NewEnviron{h3}{%
    \needspace{3\baselineskip}
	\vspace{5mm}
	\textsc{\BODY}
	\par
}

\NewEnviron{margeneg}{ %
\begin{addmargin}[-1cm]{0cm}
\BODY
\end{addmargin}
}

\NewEnviron{html}{%
}

\begin{document}
\meta{url}{/exercices/suites-et-fonctions-bac-s-pondichery-2018/}
\meta{pid}{7127}
\meta{titre}{Géométrie dans l'espace – Bac S Pondichéry 2018}
\meta{type}{exercice}
\begin{h2}Exercice 1 (6 points)\end{h2}
\textbf{Commun  à tous les candidats}
\medskip
\emph{Les parties } A \emph{et}   B \emph{peuvent être traitées de façon indépendante.}
\bigskip
Dans une usine, un four cuit des céramiques à la température de 1~000~\degres C. À la fin de la
cuisson, il est éteint et il refroidit.
\smallskip
On s'intéresse à la phase de refroidissement du four, qui débute dès l'instant où il est éteint.
\smallskip
La température du four est exprimée en degré Celsius (\degres~C).
\smallskip
La porte du four peut être ouverte sans risque pour les céramiques dès que sa température est
inférieure à $70$\degres~C. Sinon les céramiques peuvent se fissurer, voire se casser.
\begin{center}\begin{h3}Partie A \end{h3}\end{center}
Pour un nombre entier naturel $n$, on note $T_n$ la température en degré Celsius du four au bout
de $n$ heures écoulées à partir de l'instant où il a été éteint. On a donc $T_0 = 1~000$.
\par
La température $T_n$ est calculée par l'algorithme suivant~:
\begin{center}
     \begin{extern}%width="230px" alt="algorithme suite arithmético-géométrique"
          \begin{tabularx}{0.35\linewidth}{|X|}\hline
               $T \gets 1~000$\\
               Pour $i$ allant de $1$ à $n$\\
               \hspace{1cm}$T \gets 0,82 \times T + 3,6$\\
               Fin Pour\\\hline
          \end{tabularx}
     \end{extern}
\end{center}
\begin{enumerate}
     \item Déterminer la température du four, arrondie à l'unité, au bout de $4$ heures de
     refroidissement.
     \item  Démontrer que, pour tout nombre entier naturel $n$, on a~: $T_n = 980 \times 0,82^n + 20$.
     \item  Au bout de combien d'heures le four peut-il être ouvert sans risque pour les céramiques~?
\end{enumerate}
\begin{center}\begin{h3}Partie B \end{h3}\end{center}
Dans cette partie, on note $t$ le temps (en heure) écoulé depuis l'instant où le four a été éteint.
\par
La température du four (en degré Celsius) à l'instant $t$ est donnée par la fonction $f$ définie,
pour tout nombre réel $t$ positif, par~:
\[f(t) = a\text{e}^{- \frac{t}{5}} + b,\]
où $a$ et $b$ sont deux nombres réels.
\par
On admet que $f$ vérifie la relation suivante~: $f'(t) + \dfrac{1}{5}f(t) = 4$.
\medskip
\begin{enumerate}
     \item Déterminer les valeurs de $a$ et $b$ sachant qu'initialement, la température du four est de
     $1~000$~\degres C, c'est-à-dire que $f(0) = 1~000$.
     \item  Pour la suite, on admet que, pour tout nombre réel positif $t$~:
     \par
     \[f(t) = 980\text{e}^{- \frac{t}{5}} + 20.\]
     \begin{enumerate}[label=\alph*.]
          \item Déterminer la limite de $f$ lorsque $t$ tend vers $+ \infty$.
          \item Étudier les variations de $f$ sur $[0~;~+ \infty[$.
          \par
          En déduire son tableau de variations complet.
          \item Avec ce modèle, après combien de minutes le four peut-il être ouvert sans risque pour
          les céramiques~?
     \end{enumerate}
     \item  La température moyenne (en degré Celsius) du four entre deux instants $t_1$ et $t_2$ est donnée
     par~: $\dfrac{1}{t_2 - t_1}\displaystyle\int_{t_1}^{t_2} f(t)\:\text{d}t$.
     \begin{enumerate}[label=\alph*.]
          \item À l'aide de la représentation graphique de $f$ ci-dessous, donner une estimation de la
          température moyenne $\theta$ du four sur les $15$ premières heures de refroidissement.
          \par
          Expliquer votre démarche.
          \begin{center}
               \begin{extern}%alt="fonction température du four"
                    \definecolor{darkgreen}{rgb}{0.,0.4,0.}
                    \psset{xunit=0.6cm,yunit=0.01cm}
                    \begin{pspicture}(-1,-50)(19,1120)
                         \multido{\n=0+1}{20}{\psline[linecolor=lightgray,linewidth=0.5pt](\n,0)(\n,1100)}
                         \multido{\n=0+100}{12}{\psline[linecolor=lightgray,linewidth=0.5pt](0,\n)(19,\n)}
                         \psaxes[linewidth=0.5pt,Dy=200]{->}(0,0)(0,0)(19,1120)
                         \psaxes[linewidth=0.5pt,Dy=200](0,0)(0,0)(19,1120)
                         \uput[d](16.2,-60){temps écoulé (en heures)}
                         \uput[r](0,1050){température (en degrés Celsius)}
                         \psplot[plotpoints=3000,linewidth=0.75pt,linecolor=darkgreen]{0}{19}{980 2.71828 0.2 x mul exp div 20 add}
                    \end{pspicture}
               \end{extern}
          \end{center}
          \item  Calculer la valeur exacte de cette température moyenne $\theta$ et en donner la valeur
          arrondie au degré Celsius.
     \end{enumerate}
     \item  Dans cette question, on s'intéresse à l'abaissement de température (en degré Celsius) du
     four au cours d'une heure, soit entre deux instants $t$ et $(t + 1)$. Cet abaissement est donné
     par la fonction $d$ définie, pour tout nombre réel $t$ positif, par~: $d(t) = f(t) - f(t + 1)$.
     \begin{enumerate}[label=\alph*.]
          \item Vérifier que. pour tout nombre réel $t$ positif~: $d(t) = 980\left(1 - \text{e}^{- \frac{1}{5}}\right)\text{e}^{- \frac{t}{5}}$.
          \item Déterminer la limite de $d(t)$ lorsque $t$ tend vers $+ \infty$.
          \par
          Quelle interprétation peut-on en donner~?
     \end{enumerate}
\end{enumerate}
\begin{corrige}
     \begin{center}\begin{h3}Partie A \end{h3}\end{center}
     \begin{enumerate}
          \item La température du four au bout de 4 heures de
          refroidissement est égale à $T_4$.
          \par
          L'algorithme fourni par l'énoncé conduit à la relation de récurrence~:
          \begin{center}$T_{n+1}=0,82T_n+3,6 $\end{center}
          Pour trouver $T_4$ on peut~:
          \begin{itemize}[label=---]
               \item %
               programmer l'algorithme sur sa calculatrice
               \item %
               utiliser le menu suite de la calculatrice
               \item %
               calculer directement $T_1, T_2, T_3, T_4$.
          \end{itemize}
          $T_1=0,82 T_0 + 3,6 $\nosp$= 0,82 \times 1\ 000+3,6$\nosp$ \approx 824$\\
          $T_2=0,82 T_1 + 3,6 \approx 679$\\
          $T_3=0,82 T_2 + 3,6 \approx 560$\\
          $T_4=0,82 T_3 + 3,6 \approx 463$
          \par
          La température du four au bout de 4 heures de
          refroidissement est $T_4\approx 463$~\degres~C (arrondie à l'unité).
          \item
          Montrons par récurrence que pour tout entier naturel $n$~:
          \begin{center}$ T_n=980 \times 0,82^n + 20 $\end{center}
          \smallskip
          \textbf{\textit{Initialisation}}
          \smallskip
          $T_0=1~000$ et $980 \times 0,82^0+20=1~000$\\
          \par
          La propriété est donc vraie au rang 0.
          \smallskip
          \textbf{\textit{Hérédité}}
          \smallskip
          Supposons que pour un certain entier naturel $n$~: $T_n=980 \times 0,82^n + 20$ et montrons que $T_{n+1} = 980 \times 0,82^{n+1} + 20 $.\\
          \par
          $T_{n+1}=0,82T_n+3,6$\\
          $\phantom{T_{n+1}}=0,82\times \left( 980 \times 0,82^n + 20\right) +3,6$\\
          $\phantom{T_{n+1}}=980 \times 0,82^{n+1} + 16,4+3,6 $\\
          $\phantom{T_{n+1}}=980 \times 0,82^{n+1} + 20 $\\
          \par
          La propriété est donc héréditaire.
          \smallskip
          \textbf{\textit{Conclusion}}
          \smallskip
          Pour tout entier naturel $n$~:
          \begin{center}$ T_n=980 \times 0,82^n + 20 $\end{center}
          \item
          Le four peut être ouvert sans risque pour les céramiques si et seulement si $T_n \leqslant 70$.
          \par
          $T_n \leqslant 70 \Leftrightarrow 980 \times 0,82^n + 20 \leqslant 70$\\
          $\phantom{T_n \leqslant 70} \Leftrightarrow 0,82^n \leqslant \dfrac{50}{980}$
          \par
          La fonction $\ln$ étant strictement croissante sur $]0~;~+\infty[$~:
          \par
          $\phantom{T_n \leqslant 70} \Leftrightarrow n\ln(0,82) \leqslant \ln\left( \dfrac{5}{98}\right)$
          \par
          Comme $\ln(0,82)<0$ on divise par $\ln(0,82)$ en changeant le sens de l'inégalité~:
          \par
          $\phantom{T_n \leqslant 70} \Leftrightarrow  n \geqslant \dfrac{\ln\left( \dfrac{5}{98}\right)}{\ln(0,82)}\approx 14,99.$
          \par
          Le four peut être ouvert sans risque pour les céramiques au bout de 15 heures.
     \end{enumerate}
     \begin{center}\begin{h3}Partie B \end{h3}\end{center}
     \begin{enumerate}
          \item
          $f$ est dérivable sur $[0~;~+\infty[$ et~:
          \par
          $f'(t)=-\dfrac{a}{5}\text{e}^{-\frac{t}{5}}$.
          \par
          La condition $f'(t)+\dfrac{1}{5}f(t)=4$ s'écrit alors~:
          \par
          $-\dfrac{a}{5}\text{e}^{-\frac{t}{5}} + \dfrac{a}{5}\text{e}^{-\frac{t}{5}} + \dfrac{b}{5} = 4$
          \par
          Donc  $ \dfrac{b}{5} = 4$ et $b=20$.
          \par
          De plus $f(0) = 1~000$ donc $a\text{e}^0+20=1~000$ et $a=980$.
          \par
          $f$ est donc définie sur $[0~;~+\infty[$ par $f(t)=980\text{e}^{-\frac{t}{5}} + 20$.
          \item
          \begin{enumerate}[label=\alph*.]
               \item
               $\lim\limits_{t \rightarrow +\infty}-\dfrac{t}{5} = -\infty$\\
               \par
               $\lim\limits_{x \rightarrow -\infty}\text{e}^{x} = 0$\\
               \par
               Donc par composition~:\\
               \par
               $\lim\limits_{x \rightarrow -\infty}\text{e}^{-\frac{t}{5}} = 0$\\
               \par
               et\\
               \par
               $\lim\limits_{x \rightarrow -\infty}f(t) = \lim\limits_{x \rightarrow -\infty}980\text{e}^{-\frac{t}{5}} + 20 = 20.$\\
               \item
               Pour tout réel positif $t$, $f'(t)=-196\text{e}^{-\frac{t}{5}}$ est strictement négatif. La fonction $f$ est donc strictement décroissante sur $[0~;~+\infty[$.\\
               \par
               On obtient alors le tableau de variation suivant~:
               \begin{center}
                    \begin{extern}%width="260px" alt="tableau de variations"
                         \begin{tikzpicture}[scale=0.875]
                              % source : tikz
                              % Styles
                              \tikzstyle{cadre}=[thin]
                              \tikzstyle{fleche}=[->,>=latex,thin]
                              \tikzstyle{nondefini}=[lightgray]
                              % Dimensions Modifiables
                              \def\Lrg{1.5}
                              \def\HtX{1}
                              \def\HtY{0.5}
                              % Dimensions Calculées
                              \def\lignex{-0.5*\HtX}
                              \def\lignef{-1.5*\HtX}
                              \def\separateur{-0.5*\Lrg}
                              % Largeur du tableau
                              \def\gauche{-1.5*\Lrg}
                              \def\droite{2.5*\Lrg}
                              % Hauteur du tableau
                              \def\haut{0.5*\HtX}
                              \def\bas{-2.5*\HtX-2*\HtY}
                              % Ligne de l'abscisse : x
                              \node at (-1*\Lrg,0) {$t$};
                              \node at (0*\Lrg,0) {$0$};
                              \node at (2*\Lrg,0) {$+\infty$};
                              % Ligne de la dérivée : f'(x)
                              \node at (-1*\Lrg,-1*\HtX) {$f'(t)$};
                              \node at (0*\Lrg,-1*\HtX) {$$};
                              \node at (1*\Lrg,-1*\HtX) {$-$};
                              \node at (2*\Lrg,-1*\HtX) {$$};
                              % Ligne de la fonction : f(x)
                              \node  at (-1*\Lrg,{-2*\HtX+(-1)*\HtY}) {$f(t)$};
                              \node (f1) at (0*\Lrg,{-2*\HtX+(0)*\HtY}) {$1~000$};
                              \node (f2) at (2*\Lrg,{-2*\HtX+(-2)*\HtY}) {$20$};
                              % Flèches
                              \draw[fleche] (f1) -- (f2);
                              % Encadrement
                              \draw[cadre] (\separateur,\haut) -- (\separateur,\bas);
                              \draw[cadre] (\gauche,\haut) rectangle  (\droite,\bas);
                              \draw[cadre] (\gauche,\lignex) -- (\droite,\lignex);
                              \draw[cadre] (\gauche,\lignef) -- (\droite,\lignef);
                         \end{tikzpicture}
                    \end{extern}
               \end{center}
               \item
               Le four peut être ouvert sans risque pour
               les céramiques si et seulement si $f(t) \leqslant 70$.\\
               \par
               $f(t) \leqslant 70 \Leftrightarrow 980\text{e}^{-\frac{t}{5}} + 20  \leqslant 70$\\
               $\phantom{f(t) \leqslant 70} \Leftrightarrow \text{e}^{-\frac{t}{5}} \leqslant \dfrac{50}{980}$\\
               \par
               La fonction $\ln$ est strictement croissante sur $]0~;~+\infty[$, donc~:\\
               \par
               $\phantom{f(t) \leqslant 70} \Leftrightarrow \ln \text{e}^{-\frac{t}{5}} \leqslant \ln\dfrac{5}{98}$\\
               $\phantom{f(t) \leqslant 70} \Leftrightarrow t \geqslant -5\ln\dfrac{5}{98} \approx 14,88$\\
               \par
               On peut ouvrir le four sans risque au bout de 14,88 heures soit 14,88$\times$60$\approx$893 minutes. \\
          \end{enumerate}
          \item
          \begin{enumerate}[label=\alph*.]
               \item %
               L'intégrale $\displaystyle\int_{0}^{15}f(t)\text{d}t$ est égale à l'aire, en unités d'aire, du domaine délimité par la courbe $\mathcal{C}_f$, l'axe des abscisses, l'axe des ordonnées et la droite d'équation $x=15$. \\
               \begin{center}
                    \begin{extern}%alt="courbe et intégrale"
                         \definecolor{darkgreen}{rgb}{0.,0.4,0.}
                         \psset{xunit=0.6cm,yunit=0.01cm}
                         \begin{pspicture}(-1,-50)(19,1120)
                              \multido{\n=0+1}{20}{\psline[linecolor=lightgray,linewidth=0.5pt](\n,0)(\n,1100)}
                              \multido{\n=0+100}{12}{\psline[linecolor=lightgray,linewidth=0.5pt](0,\n)(19,\n)}
                              \psaxes[linewidth=0.5pt,Dy=200]{->}(0,0)(0,0)(19,1120)
                              \psaxes[linewidth=0.5pt,Dy=200](0,0)(0,0)(19,1120)
                              \uput[d](16.2,-60){temps écoulé (en heures)}
                              \uput[r](0,1050){température (en degrés Celsius)}
                              \pscustom[fillstyle=solid,fillcolor=darkgreen,opacity=0.1]{
                                   \psplot[plotpoints=3000,linewidth=0.75pt,linecolor=darkgreen]{0}{15}{980 2.71828 0.2 x mul exp div 20 add}
                              \psline[linewidth=0.75pt,linecolor=darkgreen](15,0)(0,0)(0,1000)}
                              \psplot[plotpoints=3000,linewidth=0.75pt,linecolor=darkgreen]{0}{19}{980 2.71828 0.2 x mul exp div 20 add}
                         \end{pspicture}
                    \end{extern}
               \end{center}
               En comptant les carreaux rectangulaires du graphique, on peut estimer cette aire a environ 50 carreaux soit 5~000 U.A.
               \par
               Une valeur approchée de la température moyenne $\theta$ est donc~:
               \par
               $\theta \approx \dfrac{1}{15} \times 500 \approx 333$\degres~C
               \item
               Une primitive de la fonction $f$ est la fonction $F$ définie sur $[0~;~+\infty[$ par $F(t)=-5 \times 980\text{e}^{-\frac{t}{5}} + 20t$ $=-4~900\text{e}^{-\frac{t}{5}} + 20t$.\\
               \par
               On a donc~:
               $\theta =\dfrac{1}{15}\displaystyle\int_{0}^{15}f(t)\text{d}t=\dfrac{1}{15}\left(F(15)-F(0)\right)$\\
               $\theta =  \dfrac{-980\text{e}^{-3}+1~040}{3}$\\
               $\theta \approx 330$\degres~C  (arrondi au degré).
          \end{enumerate}
          \item
          \begin{enumerate}[label=\alph*.]
               \item
               Pour tout réel $t$ positif ou nul~:\\
               \par
               $d(t) = f(t) -f(t + 1)$\\
               $\phantom{d(t)}=\left(980\text{e}^{-\frac{t}{5}} + 20 \right)$\nosp$ -\left(980\text{e}^{-\frac{t+1}{5}} + 20 \right)$\\
               $\phantom{d(t)} =980\left(\text{e}^{-\frac{t}{5}}-\text{e}^{-\frac{t+1}{5}}\right) $\\
               $\phantom{d(t)} = 980\left(1 -\text{e}^{-\frac{1}{5}}\right)\text{e}^{-\frac{t}{5}}$
               \item
               $\lim\limits_{t \rightarrow +\infty} -\dfrac{t}{5}=~-\infty$
               \par
               Donc~:
               \par
               $\lim\limits_{t \rightarrow +\infty} \text{e}^{-\frac{t}{5}} = 0$
               \par
               et $\lim\limits_{t \rightarrow +\infty} d(t)=0$.
               \par
               Il est difficile de trouver une interprétation correcte de ce résultat.
               \par
               On peut certes dire que la température se stabilise au cours du temps mais cela résulte du fait que $\lim\limits_{t \rightarrow +\infty} f(t)=20$ et non du fait que  $\lim\limits_{t \rightarrow +\infty} d(t)=0$.
          \end{enumerate}
     \end{enumerate}
\end{corrige}

\end{document}