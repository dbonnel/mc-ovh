\documentclass[a4paper]{article}

%================================================================================================================================
%
% Packages
%
%================================================================================================================================

\usepackage[T1]{fontenc} 	% pour caractères accentués
\usepackage[utf8]{inputenc}  % encodage utf8
\usepackage[french]{babel}	% langue : français
\usepackage{fourier}			% caractères plus lisibles
\usepackage[dvipsnames]{xcolor} % couleurs
\usepackage{fancyhdr}		% réglage header footer
\usepackage{needspace}		% empêcher sauts de page mal placés
\usepackage{graphicx}		% pour inclure des graphiques
\usepackage{enumitem,cprotect}		% personnalise les listes d'items (nécessaire pour ol, al ...)
\usepackage{hyperref}		% Liens hypertexte
\usepackage{pstricks,pst-all,pst-node,pstricks-add,pst-math,pst-plot,pst-tree,pst-eucl} % pstricks
\usepackage[a4paper,includeheadfoot,top=2cm,left=3cm, bottom=2cm,right=3cm]{geometry} % marges etc.
\usepackage{comment}			% commentaires multilignes
\usepackage{amsmath,environ} % maths (matrices, etc.)
\usepackage{amssymb,makeidx}
\usepackage{bm}				% bold maths
\usepackage{tabularx}		% tableaux
\usepackage{colortbl}		% tableaux en couleur
\usepackage{fontawesome}		% Fontawesome
\usepackage{environ}			% environment with command
\usepackage{fp}				% calculs pour ps-tricks
\usepackage{multido}			% pour ps tricks
\usepackage[np]{numprint}	% formattage nombre
\usepackage{tikz,tkz-tab} 			% package principal TikZ
\usepackage{pgfplots}   % axes
\usepackage{mathrsfs}    % cursives
\usepackage{calc}			% calcul taille boites
\usepackage[scaled=0.875]{helvet} % font sans serif
\usepackage{svg} % svg
\usepackage{scrextend} % local margin
\usepackage{scratch} %scratch
\usepackage{multicol} % colonnes
%\usepackage{infix-RPN,pst-func} % formule en notation polanaise inversée
\usepackage{listings}

%================================================================================================================================
%
% Réglages de base
%
%================================================================================================================================

\lstset{
language=Python,   % R code
literate=
{á}{{\'a}}1
{à}{{\`a}}1
{ã}{{\~a}}1
{é}{{\'e}}1
{è}{{\`e}}1
{ê}{{\^e}}1
{í}{{\'i}}1
{ó}{{\'o}}1
{õ}{{\~o}}1
{ú}{{\'u}}1
{ü}{{\"u}}1
{ç}{{\c{c}}}1
{~}{{ }}1
}


\definecolor{codegreen}{rgb}{0,0.6,0}
\definecolor{codegray}{rgb}{0.5,0.5,0.5}
\definecolor{codepurple}{rgb}{0.58,0,0.82}
\definecolor{backcolour}{rgb}{0.95,0.95,0.92}

\lstdefinestyle{mystyle}{
    backgroundcolor=\color{backcolour},   
    commentstyle=\color{codegreen},
    keywordstyle=\color{magenta},
    numberstyle=\tiny\color{codegray},
    stringstyle=\color{codepurple},
    basicstyle=\ttfamily\footnotesize,
    breakatwhitespace=false,         
    breaklines=true,                 
    captionpos=b,                    
    keepspaces=true,                 
    numbers=left,                    
xleftmargin=2em,
framexleftmargin=2em,            
    showspaces=false,                
    showstringspaces=false,
    showtabs=false,                  
    tabsize=2,
    upquote=true
}

\lstset{style=mystyle}


\lstset{style=mystyle}
\newcommand{\imgdir}{C:/laragon/www/newmc/assets/imgsvg/}
\newcommand{\imgsvgdir}{C:/laragon/www/newmc/assets/imgsvg/}

\definecolor{mcgris}{RGB}{220, 220, 220}% ancien~; pour compatibilité
\definecolor{mcbleu}{RGB}{52, 152, 219}
\definecolor{mcvert}{RGB}{125, 194, 70}
\definecolor{mcmauve}{RGB}{154, 0, 215}
\definecolor{mcorange}{RGB}{255, 96, 0}
\definecolor{mcturquoise}{RGB}{0, 153, 153}
\definecolor{mcrouge}{RGB}{255, 0, 0}
\definecolor{mclightvert}{RGB}{205, 234, 190}

\definecolor{gris}{RGB}{220, 220, 220}
\definecolor{bleu}{RGB}{52, 152, 219}
\definecolor{vert}{RGB}{125, 194, 70}
\definecolor{mauve}{RGB}{154, 0, 215}
\definecolor{orange}{RGB}{255, 96, 0}
\definecolor{turquoise}{RGB}{0, 153, 153}
\definecolor{rouge}{RGB}{255, 0, 0}
\definecolor{lightvert}{RGB}{205, 234, 190}
\setitemize[0]{label=\color{lightvert}  $\bullet$}

\pagestyle{fancy}
\renewcommand{\headrulewidth}{0.2pt}
\fancyhead[L]{maths-cours.fr}
\fancyhead[R]{\thepage}
\renewcommand{\footrulewidth}{0.2pt}
\fancyfoot[C]{}

\newcolumntype{C}{>{\centering\arraybackslash}X}
\newcolumntype{s}{>{\hsize=.35\hsize\arraybackslash}X}

\setlength{\parindent}{0pt}		 
\setlength{\parskip}{3mm}
\setlength{\headheight}{1cm}

\def\ebook{ebook}
\def\book{book}
\def\web{web}
\def\type{web}

\newcommand{\vect}[1]{\overrightarrow{\,\mathstrut#1\,}}

\def\Oij{$\left(\text{O}~;~\vect{\imath},~\vect{\jmath}\right)$}
\def\Oijk{$\left(\text{O}~;~\vect{\imath},~\vect{\jmath},~\vect{k}\right)$}
\def\Ouv{$\left(\text{O}~;~\vect{u},~\vect{v}\right)$}

\hypersetup{breaklinks=true, colorlinks = true, linkcolor = OliveGreen, urlcolor = OliveGreen, citecolor = OliveGreen, pdfauthor={Didier BONNEL - https://www.maths-cours.fr} } % supprime les bordures autour des liens

\renewcommand{\arg}[0]{\text{arg}}

\everymath{\displaystyle}

%================================================================================================================================
%
% Macros - Commandes
%
%================================================================================================================================

\newcommand\meta[2]{    			% Utilisé pour créer le post HTML.
	\def\titre{titre}
	\def\url{url}
	\def\arg{#1}
	\ifx\titre\arg
		\newcommand\maintitle{#2}
		\fancyhead[L]{#2}
		{\Large\sffamily \MakeUppercase{#2}}
		\vspace{1mm}\textcolor{mcvert}{\hrule}
	\fi 
	\ifx\url\arg
		\fancyfoot[L]{\href{https://www.maths-cours.fr#2}{\black \footnotesize{https://www.maths-cours.fr#2}}}
	\fi 
}


\newcommand\TitreC[1]{    		% Titre centré
     \needspace{3\baselineskip}
     \begin{center}\textbf{#1}\end{center}
}

\newcommand\newpar{    		% paragraphe
     \par
}

\newcommand\nosp {    		% commande vide (pas d'espace)
}
\newcommand{\id}[1]{} %ignore

\newcommand\boite[2]{				% Boite simple sans titre
	\vspace{5mm}
	\setlength{\fboxrule}{0.2mm}
	\setlength{\fboxsep}{5mm}	
	\fcolorbox{#1}{#1!3}{\makebox[\linewidth-2\fboxrule-2\fboxsep]{
  		\begin{minipage}[t]{\linewidth-2\fboxrule-4\fboxsep}\setlength{\parskip}{3mm}
  			 #2
  		\end{minipage}
	}}
	\vspace{5mm}
}

\newcommand\CBox[4]{				% Boites
	\vspace{5mm}
	\setlength{\fboxrule}{0.2mm}
	\setlength{\fboxsep}{5mm}
	
	\fcolorbox{#1}{#1!3}{\makebox[\linewidth-2\fboxrule-2\fboxsep]{
		\begin{minipage}[t]{1cm}\setlength{\parskip}{3mm}
	  		\textcolor{#1}{\LARGE{#2}}    
 	 	\end{minipage}  
  		\begin{minipage}[t]{\linewidth-2\fboxrule-4\fboxsep}\setlength{\parskip}{3mm}
			\raisebox{1.2mm}{\normalsize\sffamily{\textcolor{#1}{#3}}}						
  			 #4
  		\end{minipage}
	}}
	\vspace{5mm}
}

\newcommand\cadre[3]{				% Boites convertible html
	\par
	\vspace{2mm}
	\setlength{\fboxrule}{0.1mm}
	\setlength{\fboxsep}{5mm}
	\fcolorbox{#1}{white}{\makebox[\linewidth-2\fboxrule-2\fboxsep]{
  		\begin{minipage}[t]{\linewidth-2\fboxrule-4\fboxsep}\setlength{\parskip}{3mm}
			\raisebox{-2.5mm}{\sffamily \small{\textcolor{#1}{\MakeUppercase{#2}}}}		
			\par		
  			 #3
 	 		\end{minipage}
	}}
		\vspace{2mm}
	\par
}

\newcommand\bloc[3]{				% Boites convertible html sans bordure
     \needspace{2\baselineskip}
     {\sffamily \small{\textcolor{#1}{\MakeUppercase{#2}}}}    
		\par		
  			 #3
		\par
}

\newcommand\CHelp[1]{
     \CBox{Plum}{\faInfoCircle}{À RETENIR}{#1}
}

\newcommand\CUp[1]{
     \CBox{NavyBlue}{\faThumbsOUp}{EN PRATIQUE}{#1}
}

\newcommand\CInfo[1]{
     \CBox{Sepia}{\faArrowCircleRight}{REMARQUE}{#1}
}

\newcommand\CRedac[1]{
     \CBox{PineGreen}{\faEdit}{BIEN R\'EDIGER}{#1}
}

\newcommand\CError[1]{
     \CBox{Red}{\faExclamationTriangle}{ATTENTION}{#1}
}

\newcommand\TitreExo[2]{
\needspace{4\baselineskip}
 {\sffamily\large EXERCICE #1\ (\emph{#2 points})}
\vspace{5mm}
}

\newcommand\img[2]{
          \includegraphics[width=#2\paperwidth]{\imgdir#1}
}

\newcommand\imgsvg[2]{
       \begin{center}   \includegraphics[width=#2\paperwidth]{\imgsvgdir#1} \end{center}
}


\newcommand\Lien[2]{
     \href{#1}{#2 \tiny \faExternalLink}
}
\newcommand\mcLien[2]{
     \href{https~://www.maths-cours.fr/#1}{#2 \tiny \faExternalLink}
}

\newcommand{\euro}{\eurologo{}}

%================================================================================================================================
%
% Macros - Environement
%
%================================================================================================================================

\newenvironment{tex}{ %
}
{%
}

\newenvironment{indente}{ %
	\setlength\parindent{10mm}
}

{
	\setlength\parindent{0mm}
}

\newenvironment{corrige}{%
     \needspace{3\baselineskip}
     \medskip
     \textbf{\textsc{Corrigé}}
     \medskip
}
{
}

\newenvironment{extern}{%
     \begin{center}
     }
     {
     \end{center}
}

\NewEnviron{code}{%
	\par
     \boite{gray}{\texttt{%
     \BODY
     }}
     \par
}

\newenvironment{vbloc}{% boite sans cadre empeche saut de page
     \begin{minipage}[t]{\linewidth}
     }
     {
     \end{minipage}
}
\NewEnviron{h2}{%
    \needspace{3\baselineskip}
    \vspace{0.6cm}
	\noindent \MakeUppercase{\sffamily \large \BODY}
	\vspace{1mm}\textcolor{mcgris}{\hrule}\vspace{0.4cm}
	\par
}{}

\NewEnviron{h3}{%
    \needspace{3\baselineskip}
	\vspace{5mm}
	\textsc{\BODY}
	\par
}

\NewEnviron{margeneg}{ %
\begin{addmargin}[-1cm]{0cm}
\BODY
\end{addmargin}
}

\NewEnviron{html}{%
}

\begin{document}
\meta{url}{/exercices/probabilites-bac-s-pondichery-2016/}
\meta{pid}{4018}
\meta{titre}{Probabilités – Bac S Pondichéry 2016}
\meta{type}{exercices}
%
\begin{h2}Exercice 1 -  4 points\end{h2}

\textbf{Commun  à tous les candidats}
\\
\textit{Les deux parties A et B peuvent être traitées de façon indépendante}
\begin{h3}Partie A\end{h3}
Des études statistiques ont permis de modéliser le temps hebdomadaire, en heures, de connexion à internet des jeunes en France âgés de 16 à 24 ans par une variable aléatoire $T$suivant une loi normale de moyenne $\mu = 13,9$ et d'écart type $\sigma$.
\par
La fonction densité de probabilité de $T$ est représentée ci-dessous :
\begin{center}
\imgsvg{probabilites-bac-s-pondichery-2016-1}{0.3}% alt="Probabilités – Bac S Pondichéry 2016 - 1" style="width:40rem"
\end{center}
\begin{enumerate}
     \item
     On sait que $p(T \geqslant 22) =  0,023$.
     \par
     En exploitant cette information :
     \begin{enumerate}[label=\alph*.]
          \item
          hachurer sur le graphique donné un annexe, deux domaines distincts dont l'aire est égale à $0,023$ ;
          \item
          déterminer $P(5,8 \leqslant T \leqslant 22)$. Justifier le résultat.Montrer qu'une valeur approchée de $\sigma$ au dixième est $4,1$.
     \end{enumerate}
     \item
     On choisit un jeune en France au hasard.
     \par
     Déterminer la probabilité qu'il soit connecté à internet plus de 18 heures par semaine.
     \par
     Arrondir au centième.
\end{enumerate}
\begin{h3}Partie B\end{h3}
\textit{Dans cette partie, les valeurs seront arrondies au millième.}
La Hadopi (Haute Autorité pour la diffusion des Œuvres et la Protection des droits sur Internet) souhaite connaître la proportion en France de jeunes âgés de 16 à 24 ans pratiquant au moins une fois par semaine le téléchargement illégal sur internet. Pour cela, elle envisage de réaliser un sondage.
\par
Mais la Hadopi craint que les jeunes interrogés ne répondent pas tous de façon sincère. Aussi,elle propose le protocole (P) suivant :
\par
On choisit aléatoirement un échantillon de jeunes âgés de 16 à 24 ans.
\par
Pour chaque jeune de cet échantillon :
\par
- le jeune lance un dé équilibré à 6 faces; l'enquêteur ne connaît pas le résultat du lancer ;
\par
- l'enquêteur pose la question : « Effectuez-vous un téléchargement illégal au moins une fois par semaine? » ;
<div class="border">
\begin{itemize}
     \item
     si le résultat du lancer est pair alors le jeune doit répondre à la question par « Oui » ou « Non » de façon sincère;
     \item
     si le résultat du lancer est « 1 » alors le jeune doit répondre « Oui » ;
     \item
     si le résultat du lancer est « 3 ou 5 » alors le jeune doit répondre « Non ».
\end{itemize}
}
Grâce à ce protocole, l'enquêteur ne sait jamais si la réponse donnée porte sur la question posée ou résulte du lancer de dé, ce qui encourage les réponses sincères.
\par
On note $p$ la proportion inconnue de jeunes âgés de 16 à 24 ans qui pratiquent au moins une fois par semaine le téléchargement illégal sur internet.
\begin{enumerate}
     \item
     \textit{Calculs de probabilités}
     On choisit aléatoirement un jeune faisant parti du protocole (P).
     \par
     On note : $R$ l'évènement "le résultat du lancer est pair",
     \par
     $O$ l'évènement "le jeune a répondu Oui".
     \par
     Reproduire et compléter l'arbre pondéré ci-dessous :
\begin{center}
\imgsvg{probabilites-bac-s-pondichery-2016-2}{0.3}% alt="Probabilités – Bac S Pondichéry 2016 - 2" style="width:35rem"
\end{center}
     En déduire que la probabilité $q$ de l'évènement "le jeune a répondu Oui" est :
     \begin{center}$q = \dfrac{1}{2}p+\dfrac{1}{6}.$\end{center}
     \item
     \textit{Intervalle de confiance}
     \begin{enumerate}[label=\alph*.]
          \item
          À la demande de l'Hadopi, un institut de sondage réalise une enquête selon le protocole (P). Sur un échantillon de taille $1500$, il dénombre $625$ réponses "Oui".
          \par
          Donner un intervalle de confiance, au niveau de confiance de $95$\%, de la proportion $q$de jeunes qui répondent "Oui " à un tel sondage, parmi la population des jeunes français âgés de 16 à 24 ans.
          \item
          Que peut-on en conclure sur la proportion $p$ de jeunes qui pratiquent au moins une fois par semaine le téléchargement illégal sur internet ?
     \end{enumerate}
\end{enumerate}
\begin{corrige}
     \begin{h3}Partie A\end{h3}
     \begin{enumerate}
          \item
          \begin{enumerate}[label=\alph*.]
               \item
\begin{center}
\imgsvg{probabilites-bac-s-pondichery-2016-3}{0.3}% alt="Probabilités – Bac S Pondichéry 2016 - 3" style="width:40rem"
\end{center}
               Le domaine hachuré en bleu correspond à l'évènement $(T \geqslant 22)$. Son aire vaut donc $p(T \geqslant 22)=0,023$.
               \par
               Par symétrie, le domaine hachuré en rouge qui correspond à l'évènement $(T \leqslant 5,8)$ (car $13,9$ est la moyenne de $5,8$ et $22$) a la même aire : $p(T \leqslant 5,8) = p(T \geqslant 22)=0,023$.
               \item
               L'évènement $(5,8 \leqslant T \leqslant 22)$ est l'évènement contraire de $(T \leqslant 5,8) \cup(T \geqslant 22)$.
               \par
               On a donc :
               \par
               $p(5,8 \leqslant T \leqslant 22)= 1-(p(T \leqslant 5,8) + p(T \geqslant 22))$
               \par
               $\phantom{p(5,8 \leqslant T \leqslant 22)}= 1-2 \times 0,023=0.954$
               \par
               $p(T \leqslant 22)= 1-p(T \leqslant 5,8)$
               \par
               $\phantom{T \leqslant 22)} = 1- 0,023=0.977$
               \par
               Pour se ramener à une loi normale centrée réduite, on pose : $Z=\frac{T-13,9}{\sigma}$.
               \par
               Alors :
               \par
               $T \leqslant 22 \Leftrightarrow T-13,9\leqslant 8,1 $
               \par
               $\phantom{T \leqslant 22} \Leftrightarrow \frac{T-13,9}{\sigma}\leqslant \frac{8,1}{\sigma}  $
               \par
               $\phantom{T \leqslant 22} \Leftrightarrow Z\leqslant \frac{8,1}{\sigma}  $
               \par
               Par conséquent :
               \par
               $ p\left(Z\leqslant \frac{8,1}{\sigma}\right)=0,977$
               \par
               A la calculatrice on obtient INVNORM(0.977) $\approx $ 1,995 (ou FRACNORM(0.977)  ... ).
               \par
               On en déduit que
               \par
               $\frac{8,1}{\sigma}\approx 1,995$
               \par
               $\sigma\approx \frac{8,1}{1,995} \approx 4,1$ au dixième près.
               \item
               La probabilité cherchée est $p(T \geqslant 18)$.
               \par
               A la calculatrice (NORMCDF(18, 1E99, 13.9, 4.1) ou NORMALFREP ...) on trouve :
               \par
               $p(T \geqslant 18) \approx 0,16$ au centième près.
               \par
               NB : On pouvait aussi répondre sans utiliser la calculatrice en remarquant que $18=\mu+\sigma$ et en utilisant la formule $p(\mu-\sigma \leqslant T \leqslant \mu+\sigma)\approx 0,68$.
               \par
               Par conséquent $p(T \leqslant \mu-\sigma) + p(T \geqslant \mu+\sigma)\approx 1-0,68 = 0,32$
               \par
               $p(T \geqslant \mu+\sigma)\approx \frac{0,32}{2}=0,16$
          \end{enumerate}
     \end{enumerate}
     \begin{h3}Partie B\end{h3}
     \begin{enumerate}
          \item
\begin{center}
\imgsvg{probabilites-bac-s-pondichery-2016-4}{0.3}% alt="Probabilités – Bac S Pondichéry 2016 - 4" style="width:35rem"
\end{center}
          D'après la formule des probabilités totales :
          \par
          $p(O)=p(R)\times p_R(O)+p(\overline{R})\times p_{\overline{R}}(O)$
          \par
          $p(O)=\frac{1}{2}\times p+\frac{1}{2} \times \frac{1}{3}= \dfrac{1}{2}p+\dfrac{1}{6}$
          \item
          \begin{enumerate}
               \item
               La fréquence observée de réponses "Oui" dans l'échantillon de taille $1500$ est $f = \frac{625}{1500} = \frac{5}{12}$.
               \par
               Les conditions $n \geqslant 30$, $nf \geqslant 5$ et $n(1-f) \geqslant 5$ étant satisfaites, l'intervalle de confiance, au niveau de confiance de $95$\% est donné par :
               \par
               $I=\left[f-\dfrac{1}{\sqrt{n}}~;~ f+\dfrac{1}{\sqrt{n}}\right]$
               \par
               $I=\left[\dfrac{5}{12}-\dfrac{1}{\sqrt{1500}}~;~ \dfrac{5}{12}+\dfrac{1}{\sqrt{1500}}\right]$
               \par
               $I \approx [0,390~;~0,443]$
               \item
               Au seuil de confiance de $95$\%, $q$ est compris entre $0,390$ et $0,443$. D'après la question \textbf{1}, on a :
               \par
               $0,390 \leqslant q \leqslant 0,443 \Leftrightarrow 0,390 \leqslant \dfrac{1}{2}p+\dfrac{1}{6} \leqslant 0,443$
               \par
               $\phantom{0,390 \leqslant q \leqslant 0,443 }\Leftrightarrow   0,390 \leqslant \frac{3p+1}{6} \leqslant 0,443 $
               \par
               $\phantom{0,390 \leqslant q \leqslant 0,443 }\Leftrightarrow 2,340 \leqslant 3p+1 \leqslant 2,658 $
               \par
               $\phantom{0,390 \leqslant q \leqslant 0,443 }\Leftrightarrow 1,340 \leqslant 3p \leqslant 1,658$
               \par
               $\phantom{0,390 \leqslant q \leqslant 0,443 }\Leftrightarrow  0,446 \leqslant p \leqslant 0,553 $
               \par
               Au seuil de confiance de$95$\%, on peut conclure que la proportion de jeunes qui pratiquent au moins une fois par semaine le téléchargement illégal sur internet est comprise entre $44,4$\% et $55,3$\%
          \end{enumerate}
     \end{enumerate}
\end{corrige}

\end{document}