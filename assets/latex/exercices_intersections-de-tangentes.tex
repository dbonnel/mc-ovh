\documentclass[a4paper]{article}

%================================================================================================================================
%
% Packages
%
%================================================================================================================================

\usepackage[T1]{fontenc} 	% pour caractères accentués
\usepackage[utf8]{inputenc}  % encodage utf8
\usepackage[french]{babel}	% langue : français
\usepackage{fourier}			% caractères plus lisibles
\usepackage[dvipsnames]{xcolor} % couleurs
\usepackage{fancyhdr}		% réglage header footer
\usepackage{needspace}		% empêcher sauts de page mal placés
\usepackage{graphicx}		% pour inclure des graphiques
\usepackage{enumitem,cprotect}		% personnalise les listes d'items (nécessaire pour ol, al ...)
\usepackage{hyperref}		% Liens hypertexte
\usepackage{pstricks,pst-all,pst-node,pstricks-add,pst-math,pst-plot,pst-tree,pst-eucl} % pstricks
\usepackage[a4paper,includeheadfoot,top=2cm,left=3cm, bottom=2cm,right=3cm]{geometry} % marges etc.
\usepackage{comment}			% commentaires multilignes
\usepackage{amsmath,environ} % maths (matrices, etc.)
\usepackage{amssymb,makeidx}
\usepackage{bm}				% bold maths
\usepackage{tabularx}		% tableaux
\usepackage{colortbl}		% tableaux en couleur
\usepackage{fontawesome}		% Fontawesome
\usepackage{environ}			% environment with command
\usepackage{fp}				% calculs pour ps-tricks
\usepackage{multido}			% pour ps tricks
\usepackage[np]{numprint}	% formattage nombre
\usepackage{tikz,tkz-tab} 			% package principal TikZ
\usepackage{pgfplots}   % axes
\usepackage{mathrsfs}    % cursives
\usepackage{calc}			% calcul taille boites
\usepackage[scaled=0.875]{helvet} % font sans serif
\usepackage{svg} % svg
\usepackage{scrextend} % local margin
\usepackage{scratch} %scratch
\usepackage{multicol} % colonnes
%\usepackage{infix-RPN,pst-func} % formule en notation polanaise inversée
\usepackage{listings}

%================================================================================================================================
%
% Réglages de base
%
%================================================================================================================================

\lstset{
language=Python,   % R code
literate=
{á}{{\'a}}1
{à}{{\`a}}1
{ã}{{\~a}}1
{é}{{\'e}}1
{è}{{\`e}}1
{ê}{{\^e}}1
{í}{{\'i}}1
{ó}{{\'o}}1
{õ}{{\~o}}1
{ú}{{\'u}}1
{ü}{{\"u}}1
{ç}{{\c{c}}}1
{~}{{ }}1
}


\definecolor{codegreen}{rgb}{0,0.6,0}
\definecolor{codegray}{rgb}{0.5,0.5,0.5}
\definecolor{codepurple}{rgb}{0.58,0,0.82}
\definecolor{backcolour}{rgb}{0.95,0.95,0.92}

\lstdefinestyle{mystyle}{
    backgroundcolor=\color{backcolour},   
    commentstyle=\color{codegreen},
    keywordstyle=\color{magenta},
    numberstyle=\tiny\color{codegray},
    stringstyle=\color{codepurple},
    basicstyle=\ttfamily\footnotesize,
    breakatwhitespace=false,         
    breaklines=true,                 
    captionpos=b,                    
    keepspaces=true,                 
    numbers=left,                    
xleftmargin=2em,
framexleftmargin=2em,            
    showspaces=false,                
    showstringspaces=false,
    showtabs=false,                  
    tabsize=2,
    upquote=true
}

\lstset{style=mystyle}


\lstset{style=mystyle}
\newcommand{\imgdir}{C:/laragon/www/newmc/assets/imgsvg/}
\newcommand{\imgsvgdir}{C:/laragon/www/newmc/assets/imgsvg/}

\definecolor{mcgris}{RGB}{220, 220, 220}% ancien~; pour compatibilité
\definecolor{mcbleu}{RGB}{52, 152, 219}
\definecolor{mcvert}{RGB}{125, 194, 70}
\definecolor{mcmauve}{RGB}{154, 0, 215}
\definecolor{mcorange}{RGB}{255, 96, 0}
\definecolor{mcturquoise}{RGB}{0, 153, 153}
\definecolor{mcrouge}{RGB}{255, 0, 0}
\definecolor{mclightvert}{RGB}{205, 234, 190}

\definecolor{gris}{RGB}{220, 220, 220}
\definecolor{bleu}{RGB}{52, 152, 219}
\definecolor{vert}{RGB}{125, 194, 70}
\definecolor{mauve}{RGB}{154, 0, 215}
\definecolor{orange}{RGB}{255, 96, 0}
\definecolor{turquoise}{RGB}{0, 153, 153}
\definecolor{rouge}{RGB}{255, 0, 0}
\definecolor{lightvert}{RGB}{205, 234, 190}
\setitemize[0]{label=\color{lightvert}  $\bullet$}

\pagestyle{fancy}
\renewcommand{\headrulewidth}{0.2pt}
\fancyhead[L]{maths-cours.fr}
\fancyhead[R]{\thepage}
\renewcommand{\footrulewidth}{0.2pt}
\fancyfoot[C]{}

\newcolumntype{C}{>{\centering\arraybackslash}X}
\newcolumntype{s}{>{\hsize=.35\hsize\arraybackslash}X}

\setlength{\parindent}{0pt}		 
\setlength{\parskip}{3mm}
\setlength{\headheight}{1cm}

\def\ebook{ebook}
\def\book{book}
\def\web{web}
\def\type{web}

\newcommand{\vect}[1]{\overrightarrow{\,\mathstrut#1\,}}

\def\Oij{$\left(\text{O}~;~\vect{\imath},~\vect{\jmath}\right)$}
\def\Oijk{$\left(\text{O}~;~\vect{\imath},~\vect{\jmath},~\vect{k}\right)$}
\def\Ouv{$\left(\text{O}~;~\vect{u},~\vect{v}\right)$}

\hypersetup{breaklinks=true, colorlinks = true, linkcolor = OliveGreen, urlcolor = OliveGreen, citecolor = OliveGreen, pdfauthor={Didier BONNEL - https://www.maths-cours.fr} } % supprime les bordures autour des liens

\renewcommand{\arg}[0]{\text{arg}}

\everymath{\displaystyle}

%================================================================================================================================
%
% Macros - Commandes
%
%================================================================================================================================

\newcommand\meta[2]{    			% Utilisé pour créer le post HTML.
	\def\titre{titre}
	\def\url{url}
	\def\arg{#1}
	\ifx\titre\arg
		\newcommand\maintitle{#2}
		\fancyhead[L]{#2}
		{\Large\sffamily \MakeUppercase{#2}}
		\vspace{1mm}\textcolor{mcvert}{\hrule}
	\fi 
	\ifx\url\arg
		\fancyfoot[L]{\href{https://www.maths-cours.fr#2}{\black \footnotesize{https://www.maths-cours.fr#2}}}
	\fi 
}


\newcommand\TitreC[1]{    		% Titre centré
     \needspace{3\baselineskip}
     \begin{center}\textbf{#1}\end{center}
}

\newcommand\newpar{    		% paragraphe
     \par
}

\newcommand\nosp {    		% commande vide (pas d'espace)
}
\newcommand{\id}[1]{} %ignore

\newcommand\boite[2]{				% Boite simple sans titre
	\vspace{5mm}
	\setlength{\fboxrule}{0.2mm}
	\setlength{\fboxsep}{5mm}	
	\fcolorbox{#1}{#1!3}{\makebox[\linewidth-2\fboxrule-2\fboxsep]{
  		\begin{minipage}[t]{\linewidth-2\fboxrule-4\fboxsep}\setlength{\parskip}{3mm}
  			 #2
  		\end{minipage}
	}}
	\vspace{5mm}
}

\newcommand\CBox[4]{				% Boites
	\vspace{5mm}
	\setlength{\fboxrule}{0.2mm}
	\setlength{\fboxsep}{5mm}
	
	\fcolorbox{#1}{#1!3}{\makebox[\linewidth-2\fboxrule-2\fboxsep]{
		\begin{minipage}[t]{1cm}\setlength{\parskip}{3mm}
	  		\textcolor{#1}{\LARGE{#2}}    
 	 	\end{minipage}  
  		\begin{minipage}[t]{\linewidth-2\fboxrule-4\fboxsep}\setlength{\parskip}{3mm}
			\raisebox{1.2mm}{\normalsize\sffamily{\textcolor{#1}{#3}}}						
  			 #4
  		\end{minipage}
	}}
	\vspace{5mm}
}

\newcommand\cadre[3]{				% Boites convertible html
	\par
	\vspace{2mm}
	\setlength{\fboxrule}{0.1mm}
	\setlength{\fboxsep}{5mm}
	\fcolorbox{#1}{white}{\makebox[\linewidth-2\fboxrule-2\fboxsep]{
  		\begin{minipage}[t]{\linewidth-2\fboxrule-4\fboxsep}\setlength{\parskip}{3mm}
			\raisebox{-2.5mm}{\sffamily \small{\textcolor{#1}{\MakeUppercase{#2}}}}		
			\par		
  			 #3
 	 		\end{minipage}
	}}
		\vspace{2mm}
	\par
}

\newcommand\bloc[3]{				% Boites convertible html sans bordure
     \needspace{2\baselineskip}
     {\sffamily \small{\textcolor{#1}{\MakeUppercase{#2}}}}    
		\par		
  			 #3
		\par
}

\newcommand\CHelp[1]{
     \CBox{Plum}{\faInfoCircle}{À RETENIR}{#1}
}

\newcommand\CUp[1]{
     \CBox{NavyBlue}{\faThumbsOUp}{EN PRATIQUE}{#1}
}

\newcommand\CInfo[1]{
     \CBox{Sepia}{\faArrowCircleRight}{REMARQUE}{#1}
}

\newcommand\CRedac[1]{
     \CBox{PineGreen}{\faEdit}{BIEN R\'EDIGER}{#1}
}

\newcommand\CError[1]{
     \CBox{Red}{\faExclamationTriangle}{ATTENTION}{#1}
}

\newcommand\TitreExo[2]{
\needspace{4\baselineskip}
 {\sffamily\large EXERCICE #1\ (\emph{#2 points})}
\vspace{5mm}
}

\newcommand\img[2]{
          \includegraphics[width=#2\paperwidth]{\imgdir#1}
}

\newcommand\imgsvg[2]{
       \begin{center}   \includegraphics[width=#2\paperwidth]{\imgsvgdir#1} \end{center}
}


\newcommand\Lien[2]{
     \href{#1}{#2 \tiny \faExternalLink}
}
\newcommand\mcLien[2]{
     \href{https~://www.maths-cours.fr/#1}{#2 \tiny \faExternalLink}
}

\newcommand{\euro}{\eurologo{}}

%================================================================================================================================
%
% Macros - Environement
%
%================================================================================================================================

\newenvironment{tex}{ %
}
{%
}

\newenvironment{indente}{ %
	\setlength\parindent{10mm}
}

{
	\setlength\parindent{0mm}
}

\newenvironment{corrige}{%
     \needspace{3\baselineskip}
     \medskip
     \textbf{\textsc{Corrigé}}
     \medskip
}
{
}

\newenvironment{extern}{%
     \begin{center}
     }
     {
     \end{center}
}

\NewEnviron{code}{%
	\par
     \boite{gray}{\texttt{%
     \BODY
     }}
     \par
}

\newenvironment{vbloc}{% boite sans cadre empeche saut de page
     \begin{minipage}[t]{\linewidth}
     }
     {
     \end{minipage}
}
\NewEnviron{h2}{%
    \needspace{3\baselineskip}
    \vspace{0.6cm}
	\noindent \MakeUppercase{\sffamily \large \BODY}
	\vspace{1mm}\textcolor{mcgris}{\hrule}\vspace{0.4cm}
	\par
}{}

\NewEnviron{h3}{%
    \needspace{3\baselineskip}
	\vspace{5mm}
	\textsc{\BODY}
	\par
}

\NewEnviron{margeneg}{ %
\begin{addmargin}[-1cm]{0cm}
\BODY
\end{addmargin}
}

\NewEnviron{html}{%
}

\begin{document}
\meta{url}{/exercices/intersections-de-tangentes/}
\meta{pid}{4224}
\meta{titre}{Intersections de tangentes}
\meta{type}{exercices}
%
Le plan est rapporté à un repère orthonormé $\left(O; \vec{i}, \vec{j}\right)$.
\par
$P$ est la parabole d'équation $y=x^{2}$
\par
$D_{m}$ est la droite d'équation $8mx-4y+1=0$ où $m\in \mathbb{R}$
\begin{enumerate}
     \item
     Montrer que pour tout $m\in \mathbb{R}$, $P$ et $D_{m}$ se coupent en deux points distincts $A_{m}$ et $B_{m}$.
     \item
     \begin{enumerate}[label=\alph*.]
          \item
          Calculer les coordonnées du point d'intersection $I_{m}$ des tangentes à la courbe $P$ aux points $A_{m}$ et $B_{m}$.
          \item
          Quel est l'ensemble des points $I_{m}$ lorsque $m$ décrit $\mathbb{R}$ ?
     \end{enumerate}
\end{enumerate}
\begin{corrige}
     \begin{enumerate}
          \item
          $M\left(x;y\right)$ est un point d'intersection de $P$ et de $D_{m}$ si et seulement si :
          \par
          $\begin{cases} y=x^{2} \\8mx-4y+1=0  \end{cases}$
          \par
          On remplace $y$ par $x^2$ dans la seconde équation :
          \par
          $8mx-4x^{2}+1=0$
          \par
          $-4x^{2}+8mx+1=0$
          \par
          $\Delta =\left(8m\right)^{2}-4 \times (-4) \times 1=64m^{2}+16$
          \par
          $\Delta $ est strictement positif donc l'équation a \textbf{deux solutions distinctes} :
          \par
          $x_{1}=\frac{-8m+\sqrt{64m^{2}+16}}{-8}=\frac{-8m+4\sqrt{4m^{2}+1}}{-8}=m-\frac{\sqrt{4m^{2}+1}}{2}$
          \par
          $x_{2}=m+\frac{\sqrt{4m^{2}+1}}{2}$
          \par
          On a alors $y_{1}=x_{1}^{2}$ et $y_{2}=x_{2}^{2}$
          \par
          $P$ et $D_{m}$ se coupent donc en deux points distincts $A_m\left( m-\frac{\sqrt{4m^{2}+1}}{2} ; \left(m-\frac{\sqrt{4m^{2}+1}}{2}\right)^{2}  \right)$ et $B_m\left(m+\frac{\sqrt{4m^{2}+1}}{2} ; \left(m+\frac{\sqrt{4m^{2}+1}}{2}\right)^{2}\right)$
          \item
          \begin{enumerate}
               \item

\begin{center}
\imgsvg{intersections-de-tangentes}{0.3}% alt="Intersections de tangentes" style="width:45rem" class="aligncenter"
\end{center}
               \begin{center}Cas $m=1$\end{center}
               Comme $f\left(x\right)=x^{2}$,  $f^{\prime}\left(x\right)=2x$.
               \par
               L'équation de la tangente à la parabole en $A_{m}$ a pour équation:
               \par
               $y=f^{\prime}\left(x_{1}\right)\left(x-x_{1}\right)+f\left(x_{1}\right)$
               \par
               c'est à dire
               \par
               $y=2x_{1}\left(x-x_{1}\right)+x_{1}^{2}$
               \par
               $y=2x_{1}x-x_{1}^{2}$
               \par
               De même, l'équation de la tangente à la parabole en $B_{m}$ a pour équation:
               \par
               $y=2x_{2}x-x_{2}^{2}$
               \par
               Pour trouver les coordonnées de l'intersection $I_{m}$ on résout le système :
               \par
               $\left\{ \begin{matrix} y=2x_{1}x-x_{1}^{2} \\ y=2x_{2}x-x_{2}^{2} \end{matrix}\right.$
                    \par
                    Par substitution, il est équivalent à :
                    \par
                    $\left\{ \begin{matrix} y=2x_{1}x-x_{1}^{2} \\ 2x_{1}x+x_{1}^{2}=2x_{2}x-x_{2}^{2} \end{matrix}\right.$
                         \par
                         La deuxième équation donne successivement :
                         \par
                         $2x_{1}x-2x_{2}x=x_{1}^{2}-x_{2}^{2}$
                         \par
                         $2\left(x_{1}-x_{2}\right)x=\left(x_{1}-x_{2}\right)\left(x_{1}+x_{2}\right)$
                         \par
                         $2x=x_{1}+x_{2}$
                         \par
                         or $x_{1}+x_{2}=m+\frac{\sqrt{4m^{2}+1}}{2}+m-\frac{\sqrt{4m^{2}+1}}{2}=2m$
                         \par
                         donc l'équation devient:
                         \par
                         $2x=2m$ c'est à dire $x=m$.
                         \par
                         En remplaçant $x$ par $m$ dans la première équation du système on obtient :
                         \par
                         $y=2mx_{1}-x_{1}^{2}=x_{1}\left(2m-x_{1}\right)$
                         \par
                         $y=\left(m+\frac{\sqrt{4m^{2}+1}}{2}\right)\times \left(2m-m-\frac{\sqrt{4m^{2}+1}}{2}\right)$
                         \par
                         $y=\left(m+\frac{\sqrt{4m^{2}+1}}{2}\right)\times \left(m-\frac{\sqrt{4m^{2}+1}}{2}\right)$
                         \par
                         C'est une identité remarquable:
                         \par
                         $y=m^{2}-\left(\frac{\sqrt{4m^{2}+1}}{2}\right)^{2}=m^{2}-\frac{4m^{2}+1}{4}=\frac{4m^{2}-4m^{2}-1}{4}=-\frac{1}{4}$
                         \par
                         Les coordonnées de $I_{m}$ sont donc $\left(m;-\frac{1}{4}\right)$.
                         \item
                         Lorsque $m$ décrit $\mathbb{R}$ l'abscisse de $I_{m}$ décrit $\mathbb{R}$ tandis que son ordonnée est constante et égale à $-\frac{1}{4}$.
                         \par
                         L'ensemble des points $I_{m}$ lorsque $m$ décrit $\mathbb{R}$ est donc \textbf{la droite d'équation $y=-\frac{1}{4}$}
                    \end{enumerate}
               \end{enumerate}
          \end{corrige}

\end{document}