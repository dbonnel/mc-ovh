\documentclass[a4paper]{article}

%================================================================================================================================
%
% Packages
%
%================================================================================================================================

\usepackage[T1]{fontenc} 	% pour caractères accentués
\usepackage[utf8]{inputenc}  % encodage utf8
\usepackage[french]{babel}	% langue : français
\usepackage{fourier}			% caractères plus lisibles
\usepackage[dvipsnames]{xcolor} % couleurs
\usepackage{fancyhdr}		% réglage header footer
\usepackage{needspace}		% empêcher sauts de page mal placés
\usepackage{graphicx}		% pour inclure des graphiques
\usepackage{enumitem,cprotect}		% personnalise les listes d'items (nécessaire pour ol, al ...)
\usepackage{hyperref}		% Liens hypertexte
\usepackage{pstricks,pst-all,pst-node,pstricks-add,pst-math,pst-plot,pst-tree,pst-eucl} % pstricks
\usepackage[a4paper,includeheadfoot,top=2cm,left=3cm, bottom=2cm,right=3cm]{geometry} % marges etc.
\usepackage{comment}			% commentaires multilignes
\usepackage{amsmath,environ} % maths (matrices, etc.)
\usepackage{amssymb,makeidx}
\usepackage{bm}				% bold maths
\usepackage{tabularx}		% tableaux
\usepackage{colortbl}		% tableaux en couleur
\usepackage{fontawesome}		% Fontawesome
\usepackage{environ}			% environment with command
\usepackage{fp}				% calculs pour ps-tricks
\usepackage{multido}			% pour ps tricks
\usepackage[np]{numprint}	% formattage nombre
\usepackage{tikz,tkz-tab} 			% package principal TikZ
\usepackage{pgfplots}   % axes
\usepackage{mathrsfs}    % cursives
\usepackage{calc}			% calcul taille boites
\usepackage[scaled=0.875]{helvet} % font sans serif
\usepackage{svg} % svg
\usepackage{scrextend} % local margin
\usepackage{scratch} %scratch
\usepackage{multicol} % colonnes
%\usepackage{infix-RPN,pst-func} % formule en notation polanaise inversée
\usepackage{listings}

%================================================================================================================================
%
% Réglages de base
%
%================================================================================================================================

\lstset{
language=Python,   % R code
literate=
{á}{{\'a}}1
{à}{{\`a}}1
{ã}{{\~a}}1
{é}{{\'e}}1
{è}{{\`e}}1
{ê}{{\^e}}1
{í}{{\'i}}1
{ó}{{\'o}}1
{õ}{{\~o}}1
{ú}{{\'u}}1
{ü}{{\"u}}1
{ç}{{\c{c}}}1
{~}{{ }}1
}


\definecolor{codegreen}{rgb}{0,0.6,0}
\definecolor{codegray}{rgb}{0.5,0.5,0.5}
\definecolor{codepurple}{rgb}{0.58,0,0.82}
\definecolor{backcolour}{rgb}{0.95,0.95,0.92}

\lstdefinestyle{mystyle}{
    backgroundcolor=\color{backcolour},   
    commentstyle=\color{codegreen},
    keywordstyle=\color{magenta},
    numberstyle=\tiny\color{codegray},
    stringstyle=\color{codepurple},
    basicstyle=\ttfamily\footnotesize,
    breakatwhitespace=false,         
    breaklines=true,                 
    captionpos=b,                    
    keepspaces=true,                 
    numbers=left,                    
xleftmargin=2em,
framexleftmargin=2em,            
    showspaces=false,                
    showstringspaces=false,
    showtabs=false,                  
    tabsize=2,
    upquote=true
}

\lstset{style=mystyle}


\lstset{style=mystyle}
\newcommand{\imgdir}{C:/laragon/www/newmc/assets/imgsvg/}
\newcommand{\imgsvgdir}{C:/laragon/www/newmc/assets/imgsvg/}

\definecolor{mcgris}{RGB}{220, 220, 220}% ancien~; pour compatibilité
\definecolor{mcbleu}{RGB}{52, 152, 219}
\definecolor{mcvert}{RGB}{125, 194, 70}
\definecolor{mcmauve}{RGB}{154, 0, 215}
\definecolor{mcorange}{RGB}{255, 96, 0}
\definecolor{mcturquoise}{RGB}{0, 153, 153}
\definecolor{mcrouge}{RGB}{255, 0, 0}
\definecolor{mclightvert}{RGB}{205, 234, 190}

\definecolor{gris}{RGB}{220, 220, 220}
\definecolor{bleu}{RGB}{52, 152, 219}
\definecolor{vert}{RGB}{125, 194, 70}
\definecolor{mauve}{RGB}{154, 0, 215}
\definecolor{orange}{RGB}{255, 96, 0}
\definecolor{turquoise}{RGB}{0, 153, 153}
\definecolor{rouge}{RGB}{255, 0, 0}
\definecolor{lightvert}{RGB}{205, 234, 190}
\setitemize[0]{label=\color{lightvert}  $\bullet$}

\pagestyle{fancy}
\renewcommand{\headrulewidth}{0.2pt}
\fancyhead[L]{maths-cours.fr}
\fancyhead[R]{\thepage}
\renewcommand{\footrulewidth}{0.2pt}
\fancyfoot[C]{}

\newcolumntype{C}{>{\centering\arraybackslash}X}
\newcolumntype{s}{>{\hsize=.35\hsize\arraybackslash}X}

\setlength{\parindent}{0pt}		 
\setlength{\parskip}{3mm}
\setlength{\headheight}{1cm}

\def\ebook{ebook}
\def\book{book}
\def\web{web}
\def\type{web}

\newcommand{\vect}[1]{\overrightarrow{\,\mathstrut#1\,}}

\def\Oij{$\left(\text{O}~;~\vect{\imath},~\vect{\jmath}\right)$}
\def\Oijk{$\left(\text{O}~;~\vect{\imath},~\vect{\jmath},~\vect{k}\right)$}
\def\Ouv{$\left(\text{O}~;~\vect{u},~\vect{v}\right)$}

\hypersetup{breaklinks=true, colorlinks = true, linkcolor = OliveGreen, urlcolor = OliveGreen, citecolor = OliveGreen, pdfauthor={Didier BONNEL - https://www.maths-cours.fr} } % supprime les bordures autour des liens

\renewcommand{\arg}[0]{\text{arg}}

\everymath{\displaystyle}

%================================================================================================================================
%
% Macros - Commandes
%
%================================================================================================================================

\newcommand\meta[2]{    			% Utilisé pour créer le post HTML.
	\def\titre{titre}
	\def\url{url}
	\def\arg{#1}
	\ifx\titre\arg
		\newcommand\maintitle{#2}
		\fancyhead[L]{#2}
		{\Large\sffamily \MakeUppercase{#2}}
		\vspace{1mm}\textcolor{mcvert}{\hrule}
	\fi 
	\ifx\url\arg
		\fancyfoot[L]{\href{https://www.maths-cours.fr#2}{\black \footnotesize{https://www.maths-cours.fr#2}}}
	\fi 
}


\newcommand\TitreC[1]{    		% Titre centré
     \needspace{3\baselineskip}
     \begin{center}\textbf{#1}\end{center}
}

\newcommand\newpar{    		% paragraphe
     \par
}

\newcommand\nosp {    		% commande vide (pas d'espace)
}
\newcommand{\id}[1]{} %ignore

\newcommand\boite[2]{				% Boite simple sans titre
	\vspace{5mm}
	\setlength{\fboxrule}{0.2mm}
	\setlength{\fboxsep}{5mm}	
	\fcolorbox{#1}{#1!3}{\makebox[\linewidth-2\fboxrule-2\fboxsep]{
  		\begin{minipage}[t]{\linewidth-2\fboxrule-4\fboxsep}\setlength{\parskip}{3mm}
  			 #2
  		\end{minipage}
	}}
	\vspace{5mm}
}

\newcommand\CBox[4]{				% Boites
	\vspace{5mm}
	\setlength{\fboxrule}{0.2mm}
	\setlength{\fboxsep}{5mm}
	
	\fcolorbox{#1}{#1!3}{\makebox[\linewidth-2\fboxrule-2\fboxsep]{
		\begin{minipage}[t]{1cm}\setlength{\parskip}{3mm}
	  		\textcolor{#1}{\LARGE{#2}}    
 	 	\end{minipage}  
  		\begin{minipage}[t]{\linewidth-2\fboxrule-4\fboxsep}\setlength{\parskip}{3mm}
			\raisebox{1.2mm}{\normalsize\sffamily{\textcolor{#1}{#3}}}						
  			 #4
  		\end{minipage}
	}}
	\vspace{5mm}
}

\newcommand\cadre[3]{				% Boites convertible html
	\par
	\vspace{2mm}
	\setlength{\fboxrule}{0.1mm}
	\setlength{\fboxsep}{5mm}
	\fcolorbox{#1}{white}{\makebox[\linewidth-2\fboxrule-2\fboxsep]{
  		\begin{minipage}[t]{\linewidth-2\fboxrule-4\fboxsep}\setlength{\parskip}{3mm}
			\raisebox{-2.5mm}{\sffamily \small{\textcolor{#1}{\MakeUppercase{#2}}}}		
			\par		
  			 #3
 	 		\end{minipage}
	}}
		\vspace{2mm}
	\par
}

\newcommand\bloc[3]{				% Boites convertible html sans bordure
     \needspace{2\baselineskip}
     {\sffamily \small{\textcolor{#1}{\MakeUppercase{#2}}}}    
		\par		
  			 #3
		\par
}

\newcommand\CHelp[1]{
     \CBox{Plum}{\faInfoCircle}{À RETENIR}{#1}
}

\newcommand\CUp[1]{
     \CBox{NavyBlue}{\faThumbsOUp}{EN PRATIQUE}{#1}
}

\newcommand\CInfo[1]{
     \CBox{Sepia}{\faArrowCircleRight}{REMARQUE}{#1}
}

\newcommand\CRedac[1]{
     \CBox{PineGreen}{\faEdit}{BIEN R\'EDIGER}{#1}
}

\newcommand\CError[1]{
     \CBox{Red}{\faExclamationTriangle}{ATTENTION}{#1}
}

\newcommand\TitreExo[2]{
\needspace{4\baselineskip}
 {\sffamily\large EXERCICE #1\ (\emph{#2 points})}
\vspace{5mm}
}

\newcommand\img[2]{
          \includegraphics[width=#2\paperwidth]{\imgdir#1}
}

\newcommand\imgsvg[2]{
       \begin{center}   \includegraphics[width=#2\paperwidth]{\imgsvgdir#1} \end{center}
}


\newcommand\Lien[2]{
     \href{#1}{#2 \tiny \faExternalLink}
}
\newcommand\mcLien[2]{
     \href{https~://www.maths-cours.fr/#1}{#2 \tiny \faExternalLink}
}

\newcommand{\euro}{\eurologo{}}

%================================================================================================================================
%
% Macros - Environement
%
%================================================================================================================================

\newenvironment{tex}{ %
}
{%
}

\newenvironment{indente}{ %
	\setlength\parindent{10mm}
}

{
	\setlength\parindent{0mm}
}

\newenvironment{corrige}{%
     \needspace{3\baselineskip}
     \medskip
     \textbf{\textsc{Corrigé}}
     \medskip
}
{
}

\newenvironment{extern}{%
     \begin{center}
     }
     {
     \end{center}
}

\NewEnviron{code}{%
	\par
     \boite{gray}{\texttt{%
     \BODY
     }}
     \par
}

\newenvironment{vbloc}{% boite sans cadre empeche saut de page
     \begin{minipage}[t]{\linewidth}
     }
     {
     \end{minipage}
}
\NewEnviron{h2}{%
    \needspace{3\baselineskip}
    \vspace{0.6cm}
	\noindent \MakeUppercase{\sffamily \large \BODY}
	\vspace{1mm}\textcolor{mcgris}{\hrule}\vspace{0.4cm}
	\par
}{}

\NewEnviron{h3}{%
    \needspace{3\baselineskip}
	\vspace{5mm}
	\textsc{\BODY}
	\par
}

\NewEnviron{margeneg}{ %
\begin{addmargin}[-1cm]{0cm}
\BODY
\end{addmargin}
}

\NewEnviron{html}{%
}

\begin{document}
\meta{url}{/exercices/fonctions-bac-es-centres-etrangers-2014/}
\meta{pid}{2170}
\meta{titre}{Fonctions - Bac ES/L Centres étrangers 2014}
\meta{type}{exercices}
%
\begin{h2}Exercice 2   (6 points)\end{h2}
\textbf{Commun  à tous les candidats}
\begin{h3}Partie A : Étude d'une fonction\end{h3}
Soit $f$ la fonction définie sur $\mathbb{R}$ par

\begin{center}
$f\left(x\right) = x e^{x^{2}-1}$.
\end{center}

$\mathscr C_{f}$ est la courbe représentative de la fonction $f$ dans un repère orthonormé du plan. On note $f^{\prime}$ la fonction dérivée de $f$ et $f^{\prime\prime}$ la fonction dérivée seconde de $f$.
\begin{enumerate}
     \item
     \begin{enumerate}[label=\alph*.]
          \item
          Montrer que pour tout réel $x, f^{\prime}\left(x\right)= \left(2x^{2}+1\right) e^{x^{2}-1}$.
          \item
     En déduire le sens de variation de $f$ sur $\mathbb{R}$.\end{enumerate}
     \item
     On admet que pour tout réel $x, f^{\prime\prime}\left(x\right)=2x \left(2x^{2}+3\right) e^{x^{2}-1}$.
     \par
     Déterminer, en justifiant, l'intervalle sur lequel la fonction $f$ est convexe.
     \item
     Soit $h$ la fonction définie sur $\mathbb{R}$ par

\begin{center}
     $h\left(x\right)=x \left(1-e^{x^{2}-1}\right).$
\end{center}

     \begin{enumerate}[label=\alph*.]
          \item
          Justifier que l'inéquation $1-e^{x^{2}-1}\geqslant 0$ a pour ensemble de solutions l'intervalle $\left[-1 ; 1\right]$.
          \item
          Déterminer le signe de $h\left(x\right)$ sur 1'intervalle $\left[-1 ; 1\right]$.
          \item
     En remarquant que pour tout réel $x$, on a l'égalité $h\left(x\right)=x-f\left(x\right)$, déduire de la question précédente la position relative de la courbe $\mathscr C_{f}$ et de la droite $D$ d'équation $y=x$ sur l'intervalle [0 ;  1].\end{enumerate}
     \item
     Soit $H$ la fonction définie sur $\mathbb{R}$ par $H\left(x\right)=\frac{1}{2}x^{2}-\frac{1}{2}e^{x^{2}-1}$ et soit $I=\int_{0}^{1} h\left(x\right)dx$.
     \par
     On admet que $H$ est une primitive de la fonction $h$ sur $\mathbb{R}$.
     \par
     Calculer la valeur exacte de $I$.
\end{enumerate}
\begin{h3}Partie B : Applications\end{h3}
Sur le graphique suivant, sont tracées sur l'intervalle $\left[0 ; 1\right]$ :
\begin{itemize}
     \item
     la courbe $\mathscr C_{f}$ représentative de la fonction étudiée en partie A ;
     \item
     la courbe $\mathscr C_{g}$ représentative de la fonction définie par $g\left(x\right)=x^{3}$ ;
     \item
     la droite $D$ d'équation $y=x$.
\end{itemize}

\begin{center}
\imgsvg{mc-0293}{0.3}% alt="Courbe des salaires" style="width:40rem"
\end{center}
Les courbes $\mathscr C_{f}$ et $\mathscr C_{g}$ illustrent ici la répartition des salaires dans deux entreprises F et G :
\begin{itemize}
     \item
     sur l'axe des abscisses, $x$ représente la proportion des employés ayant les salaires les plus faibles par rapport à l'effectif total de l'entreprise ;
     \item
     sur l'axe des ordonnées, $f\left(x\right)$ et $g\left(x\right)$ représentent pour chaque entreprise la proportion de la  masse salariale (c'est-à-dire la somme de tous les salaires) correspondante.
\end{itemize}
\textit{\textbf{Par exemple} :
     \par
     Le point $M\left(0,5 ; 0,125\right)$ est un point appartenant à la courbe $\mathscr C_{g}$. Pour l'entreprise G cela se traduit de la façon suivante :
     \par
     Si on classe les employés par revenu croissant, le total des salaires de la première moitié (c'est-à-dire des 50\% aux revenus les plus faibles) représente 12,5\% de la masse salariale.)
}
\begin{enumerate}
     \item
     Calculer le pourcentage de la masse salariale détenue par 80\% des employés ayant les salaires les plus faibles dans l'entreprise F. On donnera une valeur du résultat arrondie à l'unité.
     \item
     On note $\mathscr A_{f} $ l'aire du domaine délimité par la droite $D$, la courbe $\mathscr C_{f}$ et les droites d'équations $x=0$ et $x=1$.
     \par
     On appelle indice de Gini associé à la fonction $f$, le nombre réel noté $I_{f}$ et défini par $I_{f} = 2\times \mathscr A_{f}$.
     \begin{enumerate}[label=\alph*.]
          \item
          Montrer que $I_{f}=\frac{1}{e}$.
          \item
          On admet que, plus l'indice de Gini est petit, plus la répartition des salaires dans l'entreprise est égalitaire.
\\
     Déterminer, en justifiant, l'entreprise pour laquelle la distribution des salaires est la plus égalitaire.\end{enumerate}
\end{enumerate}

\end{document}