\documentclass[a4paper]{article}

%================================================================================================================================
%
% Packages
%
%================================================================================================================================

\usepackage[T1]{fontenc} 	% pour caractères accentués
\usepackage[utf8]{inputenc}  % encodage utf8
\usepackage[french]{babel}	% langue : français
\usepackage{fourier}			% caractères plus lisibles
\usepackage[dvipsnames]{xcolor} % couleurs
\usepackage{fancyhdr}		% réglage header footer
\usepackage{needspace}		% empêcher sauts de page mal placés
\usepackage{graphicx}		% pour inclure des graphiques
\usepackage{enumitem,cprotect}		% personnalise les listes d'items (nécessaire pour ol, al ...)
\usepackage{hyperref}		% Liens hypertexte
\usepackage{pstricks,pst-all,pst-node,pstricks-add,pst-math,pst-plot,pst-tree,pst-eucl} % pstricks
\usepackage[a4paper,includeheadfoot,top=2cm,left=3cm, bottom=2cm,right=3cm]{geometry} % marges etc.
\usepackage{comment}			% commentaires multilignes
\usepackage{amsmath,environ} % maths (matrices, etc.)
\usepackage{amssymb,makeidx}
\usepackage{bm}				% bold maths
\usepackage{tabularx}		% tableaux
\usepackage{colortbl}		% tableaux en couleur
\usepackage{fontawesome}		% Fontawesome
\usepackage{environ}			% environment with command
\usepackage{fp}				% calculs pour ps-tricks
\usepackage{multido}			% pour ps tricks
\usepackage[np]{numprint}	% formattage nombre
\usepackage{tikz,tkz-tab} 			% package principal TikZ
\usepackage{pgfplots}   % axes
\usepackage{mathrsfs}    % cursives
\usepackage{calc}			% calcul taille boites
\usepackage[scaled=0.875]{helvet} % font sans serif
\usepackage{svg} % svg
\usepackage{scrextend} % local margin
\usepackage{scratch} %scratch
\usepackage{multicol} % colonnes
%\usepackage{infix-RPN,pst-func} % formule en notation polanaise inversée
\usepackage{listings}

%================================================================================================================================
%
% Réglages de base
%
%================================================================================================================================

\lstset{
language=Python,   % R code
literate=
{á}{{\'a}}1
{à}{{\`a}}1
{ã}{{\~a}}1
{é}{{\'e}}1
{è}{{\`e}}1
{ê}{{\^e}}1
{í}{{\'i}}1
{ó}{{\'o}}1
{õ}{{\~o}}1
{ú}{{\'u}}1
{ü}{{\"u}}1
{ç}{{\c{c}}}1
{~}{{ }}1
}


\definecolor{codegreen}{rgb}{0,0.6,0}
\definecolor{codegray}{rgb}{0.5,0.5,0.5}
\definecolor{codepurple}{rgb}{0.58,0,0.82}
\definecolor{backcolour}{rgb}{0.95,0.95,0.92}

\lstdefinestyle{mystyle}{
    backgroundcolor=\color{backcolour},   
    commentstyle=\color{codegreen},
    keywordstyle=\color{magenta},
    numberstyle=\tiny\color{codegray},
    stringstyle=\color{codepurple},
    basicstyle=\ttfamily\footnotesize,
    breakatwhitespace=false,         
    breaklines=true,                 
    captionpos=b,                    
    keepspaces=true,                 
    numbers=left,                    
xleftmargin=2em,
framexleftmargin=2em,            
    showspaces=false,                
    showstringspaces=false,
    showtabs=false,                  
    tabsize=2,
    upquote=true
}

\lstset{style=mystyle}


\lstset{style=mystyle}
\newcommand{\imgdir}{C:/laragon/www/newmc/assets/imgsvg/}
\newcommand{\imgsvgdir}{C:/laragon/www/newmc/assets/imgsvg/}

\definecolor{mcgris}{RGB}{220, 220, 220}% ancien~; pour compatibilité
\definecolor{mcbleu}{RGB}{52, 152, 219}
\definecolor{mcvert}{RGB}{125, 194, 70}
\definecolor{mcmauve}{RGB}{154, 0, 215}
\definecolor{mcorange}{RGB}{255, 96, 0}
\definecolor{mcturquoise}{RGB}{0, 153, 153}
\definecolor{mcrouge}{RGB}{255, 0, 0}
\definecolor{mclightvert}{RGB}{205, 234, 190}

\definecolor{gris}{RGB}{220, 220, 220}
\definecolor{bleu}{RGB}{52, 152, 219}
\definecolor{vert}{RGB}{125, 194, 70}
\definecolor{mauve}{RGB}{154, 0, 215}
\definecolor{orange}{RGB}{255, 96, 0}
\definecolor{turquoise}{RGB}{0, 153, 153}
\definecolor{rouge}{RGB}{255, 0, 0}
\definecolor{lightvert}{RGB}{205, 234, 190}
\setitemize[0]{label=\color{lightvert}  $\bullet$}

\pagestyle{fancy}
\renewcommand{\headrulewidth}{0.2pt}
\fancyhead[L]{maths-cours.fr}
\fancyhead[R]{\thepage}
\renewcommand{\footrulewidth}{0.2pt}
\fancyfoot[C]{}

\newcolumntype{C}{>{\centering\arraybackslash}X}
\newcolumntype{s}{>{\hsize=.35\hsize\arraybackslash}X}

\setlength{\parindent}{0pt}		 
\setlength{\parskip}{3mm}
\setlength{\headheight}{1cm}

\def\ebook{ebook}
\def\book{book}
\def\web{web}
\def\type{web}

\newcommand{\vect}[1]{\overrightarrow{\,\mathstrut#1\,}}

\def\Oij{$\left(\text{O}~;~\vect{\imath},~\vect{\jmath}\right)$}
\def\Oijk{$\left(\text{O}~;~\vect{\imath},~\vect{\jmath},~\vect{k}\right)$}
\def\Ouv{$\left(\text{O}~;~\vect{u},~\vect{v}\right)$}

\hypersetup{breaklinks=true, colorlinks = true, linkcolor = OliveGreen, urlcolor = OliveGreen, citecolor = OliveGreen, pdfauthor={Didier BONNEL - https://www.maths-cours.fr} } % supprime les bordures autour des liens

\renewcommand{\arg}[0]{\text{arg}}

\everymath{\displaystyle}

%================================================================================================================================
%
% Macros - Commandes
%
%================================================================================================================================

\newcommand\meta[2]{    			% Utilisé pour créer le post HTML.
	\def\titre{titre}
	\def\url{url}
	\def\arg{#1}
	\ifx\titre\arg
		\newcommand\maintitle{#2}
		\fancyhead[L]{#2}
		{\Large\sffamily \MakeUppercase{#2}}
		\vspace{1mm}\textcolor{mcvert}{\hrule}
	\fi 
	\ifx\url\arg
		\fancyfoot[L]{\href{https://www.maths-cours.fr#2}{\black \footnotesize{https://www.maths-cours.fr#2}}}
	\fi 
}


\newcommand\TitreC[1]{    		% Titre centré
     \needspace{3\baselineskip}
     \begin{center}\textbf{#1}\end{center}
}

\newcommand\newpar{    		% paragraphe
     \par
}

\newcommand\nosp {    		% commande vide (pas d'espace)
}
\newcommand{\id}[1]{} %ignore

\newcommand\boite[2]{				% Boite simple sans titre
	\vspace{5mm}
	\setlength{\fboxrule}{0.2mm}
	\setlength{\fboxsep}{5mm}	
	\fcolorbox{#1}{#1!3}{\makebox[\linewidth-2\fboxrule-2\fboxsep]{
  		\begin{minipage}[t]{\linewidth-2\fboxrule-4\fboxsep}\setlength{\parskip}{3mm}
  			 #2
  		\end{minipage}
	}}
	\vspace{5mm}
}

\newcommand\CBox[4]{				% Boites
	\vspace{5mm}
	\setlength{\fboxrule}{0.2mm}
	\setlength{\fboxsep}{5mm}
	
	\fcolorbox{#1}{#1!3}{\makebox[\linewidth-2\fboxrule-2\fboxsep]{
		\begin{minipage}[t]{1cm}\setlength{\parskip}{3mm}
	  		\textcolor{#1}{\LARGE{#2}}    
 	 	\end{minipage}  
  		\begin{minipage}[t]{\linewidth-2\fboxrule-4\fboxsep}\setlength{\parskip}{3mm}
			\raisebox{1.2mm}{\normalsize\sffamily{\textcolor{#1}{#3}}}						
  			 #4
  		\end{minipage}
	}}
	\vspace{5mm}
}

\newcommand\cadre[3]{				% Boites convertible html
	\par
	\vspace{2mm}
	\setlength{\fboxrule}{0.1mm}
	\setlength{\fboxsep}{5mm}
	\fcolorbox{#1}{white}{\makebox[\linewidth-2\fboxrule-2\fboxsep]{
  		\begin{minipage}[t]{\linewidth-2\fboxrule-4\fboxsep}\setlength{\parskip}{3mm}
			\raisebox{-2.5mm}{\sffamily \small{\textcolor{#1}{\MakeUppercase{#2}}}}		
			\par		
  			 #3
 	 		\end{minipage}
	}}
		\vspace{2mm}
	\par
}

\newcommand\bloc[3]{				% Boites convertible html sans bordure
     \needspace{2\baselineskip}
     {\sffamily \small{\textcolor{#1}{\MakeUppercase{#2}}}}    
		\par		
  			 #3
		\par
}

\newcommand\CHelp[1]{
     \CBox{Plum}{\faInfoCircle}{À RETENIR}{#1}
}

\newcommand\CUp[1]{
     \CBox{NavyBlue}{\faThumbsOUp}{EN PRATIQUE}{#1}
}

\newcommand\CInfo[1]{
     \CBox{Sepia}{\faArrowCircleRight}{REMARQUE}{#1}
}

\newcommand\CRedac[1]{
     \CBox{PineGreen}{\faEdit}{BIEN R\'EDIGER}{#1}
}

\newcommand\CError[1]{
     \CBox{Red}{\faExclamationTriangle}{ATTENTION}{#1}
}

\newcommand\TitreExo[2]{
\needspace{4\baselineskip}
 {\sffamily\large EXERCICE #1\ (\emph{#2 points})}
\vspace{5mm}
}

\newcommand\img[2]{
          \includegraphics[width=#2\paperwidth]{\imgdir#1}
}

\newcommand\imgsvg[2]{
       \begin{center}   \includegraphics[width=#2\paperwidth]{\imgsvgdir#1} \end{center}
}


\newcommand\Lien[2]{
     \href{#1}{#2 \tiny \faExternalLink}
}
\newcommand\mcLien[2]{
     \href{https~://www.maths-cours.fr/#1}{#2 \tiny \faExternalLink}
}

\newcommand{\euro}{\eurologo{}}

%================================================================================================================================
%
% Macros - Environement
%
%================================================================================================================================

\newenvironment{tex}{ %
}
{%
}

\newenvironment{indente}{ %
	\setlength\parindent{10mm}
}

{
	\setlength\parindent{0mm}
}

\newenvironment{corrige}{%
     \needspace{3\baselineskip}
     \medskip
     \textbf{\textsc{Corrigé}}
     \medskip
}
{
}

\newenvironment{extern}{%
     \begin{center}
     }
     {
     \end{center}
}

\NewEnviron{code}{%
	\par
     \boite{gray}{\texttt{%
     \BODY
     }}
     \par
}

\newenvironment{vbloc}{% boite sans cadre empeche saut de page
     \begin{minipage}[t]{\linewidth}
     }
     {
     \end{minipage}
}
\NewEnviron{h2}{%
    \needspace{3\baselineskip}
    \vspace{0.6cm}
	\noindent \MakeUppercase{\sffamily \large \BODY}
	\vspace{1mm}\textcolor{mcgris}{\hrule}\vspace{0.4cm}
	\par
}{}

\NewEnviron{h3}{%
    \needspace{3\baselineskip}
	\vspace{5mm}
	\textsc{\BODY}
	\par
}

\NewEnviron{margeneg}{ %
\begin{addmargin}[-1cm]{0cm}
\BODY
\end{addmargin}
}

\NewEnviron{html}{%
}

\begin{document}
\meta{url}{/exercices/integrales-suites-bac-s-pondichery-2009/}
\meta{pid}{2340}
\meta{titre}{Intégrales et suites - Bac S Pondichéry 2009}
\meta{type}{exercices}
%
\begin{h2}Exercice 1\end{h2}
\textit{7 points - Commun à tous candidats }
\par
Soit $f$ la fonction définie sur l'intervalle $\left[0, + \infty \right[$ par: $f\left(x\right) = xe^{-x^{2}}$
\par
On désigne par $C$ la courbe représentative de la fonction $f$ dans un repère orthonormal $\left(O; \vec{i}, \vec{j}\right)$ du plan. Cette courbe est représentée ci-dessous.

\begin{center}
\imgsvg{Bac_S_Pondichery_2009}{0.3}% alt="Bac S Pondichery 2009" style="width:50rem"
\end{center}
\begin{h3}Partie A\end{h3}
\begin{enumerate}
     \item 
     \begin{enumerate}[label=\alph*.]
          \item
          Déterminer la limite de la fonction $f$ en $+\infty $.
          \par
          (On pourra écrire, pour $x$ différent de 0 : $f\left(x\right) = \frac{1}{x}\times \frac{x^{2}}{e^{x^{2}}}$).
          \item
          Démontrer que $f$ admet un maximum en $\frac{\sqrt{2}}{2}$ et calculer ce maximum.
     \end{enumerate}
     \item
     Soit $a$ un nombre réel positif ou nul. Exprimer en unités d'aire et en fonction de $a$, l'aire $F\left(a\right)$ de la partie du plan limitée par la courbe $C$, l'axe des abscisses et les droites d'équations respectives $x=0$ et $x = a$ . Quelle est la limite de $F\left(a\right)$ quand $a$ tend vers $+\infty $?
\end{enumerate}
 
\begin{h3}Partie B\end{h3}
On considère la suite $\left(u_{n}\right)$ définie pour tout entier naturel $n$ par: $u_{n} = \int_{n}^{n+1}f\left(x\right)dx$. On ne cherchera pas à expliciter $u_{n}$.
\begin{enumerate}
     \item
     \begin{enumerate}[label=\alph*.]
          \item
          Démontrer que, pour tout entier naturel $n$ différent de 0 et de 1,
          \par
          $f\left(n+1\right) \leqslant  u_{n} \leqslant  f\left(n\right)$
          \item
          Quel est le sens de variation de la suite $\left(u_{n}\right)_{n\geqslant 2}$?
          \item
          Montrer que la suite $\left(u_{n}\right)$ converge. Quelle est sa limite?
     \end{enumerate}
     \item
     \begin{enumerate}[label=\alph*.]
          \item
          Vérifier que, pour tout entier naturel strictement positif $n$, $F\left(n\right) = \sum_{k=0}^{n-1}u_{k}$.
          \item
          \textit{Dans cette question, toute trace de recherche, même incomplète, ou d'initiative même non fructueuse, sera prise en compte dans l'évaluation. }
          On donne ci-dessous les valeurs de $F\left(n\right)$ obtenues à l'aide d'un tableur, pour $n$ entier compris entre 3 et 7.

          \begin{tabularx}{0.8\linewidth}{|*{3}{>{\centering \arraybackslash }X|}}%class="compact" width="600"
               \hline
               $n$  &  3  &  4  &  5  &  6  &  7
               \\ \hline
               $F\left(n\right)$  &  0,4999382951  &  0,4999999437  &  0,5  &  0,5  &  0,5
               \\ \hline
          \end{tabularx}
          Interpréter ces résultats.
     \end{enumerate}
\end{enumerate}
\begin{corrige}
     \begin{h3}Partie A\end{h3}
     \begin{enumerate}
          \item
          \begin{enumerate}[label=\alph*.]
               \item
               Pour $x$ différent de 0 : $f\left(x\right) = \frac{1}{x}\times \frac{x^{2}}{e^{x^{2}}}$.
               \par
               En posant $X=x^{2}$ on a :
               \par
               $\lim\limits_{x\rightarrow +\infty }\frac{x^{2}}{e^{x^{2}}}=\lim\limits_{X\rightarrow +\infty }\frac{X}{e^{X}}=\lim\limits_{X\rightarrow +\infty }\frac{1}{\frac{e^{X}}{X}}=0$
               \par
               Par ailleurs :
               \par
               $\lim\limits_{x\rightarrow +\infty }\frac{1}{x}=0$
               \par
               donc, par produit des limites:
               \par
               $\lim\limits_{x\rightarrow +\infty }f\left(x\right)=0$
               \item
               La dérivée de la fonction $x\mapsto e^{-x^{2}}$ est la fonction $x\mapsto -2xe^{-x^{2}}$ (dérivée de fonction composée $\left(e^{u}\right)^{\prime}=u^{\prime}e^{u}$)
               \par
               On dérive $f$ grâce à la formule de dérivation d'un produit $\left(uv\right)^{\prime}=u^{\prime}v+uv^{\prime}$
               \par
               $f^{\prime}\left(x\right)=e^{-x^{2}}-2x^{2}e^{-x^{2}}=\left(1-2x^{2}\right)e^{-x^{2}}=\left(1-\sqrt{2}x\right)\left(1+\sqrt{2}x\right)e^{-x^{2}}$
               \par
               Le facteur $1+\sqrt{2}x$ est positif pour $x > 0$.Le facteur $e^{-x^{2}}$ est toujours positif donc $f^{\prime}\left(x\right)$ est du signe de $1-\sqrt{2}x$ qui s'annule pour $x=\frac{1}{\sqrt{2}}=\frac{\sqrt{2}}{2}$ et dont le coefficient directeur est négatif.
               \par
               Le tableau de variation de $f$ est~:
%##
% type=table; width=25; l3=20; alt=tableau de variation
%--
% x|   0  ~    \frac{\sqrt{ 2 }}{ 2 }     ~   +\infty 
% f'(x)|  ~            +   :0   -     ~
% f(x)|  0                /   :\frac{\sqrt{ 2 }}{ 2 \sqrt{ \text{e}}}    \   0
%--
\begin{center}
 \begin{extern}%style="width:25rem" alt="Exercice"
    \resizebox{11cm}{!}{
       \definecolor{dark}{gray}{0.1}
       \definecolor{light}{gray}{0.8}
       \tikzstyle{fleche}=[->,>=latex]
       \begin{tikzpicture}[scale=.8, line width=.5pt, dark]
       \def\width{.15}
       \def\height{.10}
       \draw (0, -10*\height) -- (54*\width, -10*\height);
       \draw (10*\width, 0*\height) -- (10*\width, -10*\height);
       \node (l0c0) at (5*\width,-5*\height) {$ x $};
       \node (l0c1) at (14*\width,-5*\height) {$ 0 $};
       \node (l0c2) at (23*\width,-5*\height) {$ ~ $};
       \node (l0c3) at (32*\width,-5*\height) {$ \frac{\sqrt{ 2 }}{ 2 } $};
       \node (l0c4) at (41*\width,-5*\height) {$ ~ $};
       \node (l0c5) at (50*\width,-5*\height) {$ +\infty $};
       \draw (0, -20*\height) -- (54*\width, -20*\height);
       \draw (10*\width, -10*\height) -- (10*\width, -20*\height);
       \node (l1c0) at (5*\width,-15*\height) {$ f'(x) $};
       \node (l1c1) at (14*\width,-15*\height) {$ ~ $};
       \node (l1c2) at (23*\width,-15*\height) {$ + $};
       \draw[light] (32*\width, -10*\height) -- (32*\width, -20*\height);
       \node (l1c3) at (32*\width,-15*\height) {$ 0 $};
       \node (l1c4) at (41*\width,-15*\height) {$ - $};
       \node (l1c5) at (50*\width,-15*\height) {$ ~ $};
       \draw (0, -40*\height) -- (54*\width, -40*\height);
       \draw (10*\width, -20*\height) -- (10*\width, -40*\height);
       \node (l2c0) at (5*\width,-30*\height) {$ f(x) $};
       \node (l2c1) at (14*\width,-35*\height) {$ 0 $};
       \node (l2c2) at (23*\width,-30*\height) {$ ~ $};
       \draw[light] (32*\width, -20*\height) -- (32*\width, -40*\height);
       \node (l2c3) at (32*\width,-25*\height) {$ \frac{\sqrt{ 2 }}{ 2 \sqrt{ \text{e}}} $};
       \node (l2c4) at (41*\width,-30*\height) {$ ~ $};
       \node (l2c5) at (50*\width,-35*\height) {$ 0 $};
       \draw (0, 0) rectangle (54*\width, -40*\height);
       \draw[fleche] (l2c1) -- (l2c3);
       \draw[fleche] (l2c3) -- (l2c5);
       \end{tikzpicture}
      }
   \end{extern}
\end{center}
%##
$f$ admet donc un maximum pour $x=\frac{\sqrt{2}}{2}$.
          \end{enumerate}
          \item
          L'aire $F\left(a\right)$ de la partie du plan limitée par la courbe $C$, l'axe des abscisses et les droites d'équations respectives $x=0$ et $x = a$ est donnée par :
          \par
          $F\left(a\right)=\int_{0}^{a}f\left(x\right)dx$
          \par
          Or $f\left(x\right)=-\frac{1}{2}\left(-2xe^{-x^{2}}\right)$ et la fonction $x\mapsto -2xe^{-x^{2}}$ est la dérivée de la fonction $x\mapsto e^{-x^{2}}$.
          \par
          Une primitive de $f$ est donc $x\mapsto -\frac{1}{2}e^{-x^{2}}$.
          \par
          Donc:
          \par
          $F\left(a\right)=\int_{0}^{a}f\left(x\right)dx=\left[-\frac{1}{2}e^{-x^{2}}\right]_{0}^{a}=-\frac{1}{2}e^{-a^{2}}+\frac{1}{2}$
          \par
          (Remarque: $F$ est la primitive de $f$ qui s'annule pour $x=0$)
          \par
          On a immédiatement :
          \par
          $\lim\limits_{a\rightarrow +\infty }F\left(a\right)=\frac{1}{2}$
     \end{enumerate}
      
     \begin{h3}Partie B\end{h3}
     \begin{enumerate}
          \item
          \begin{enumerate}[label=\alph*.]
               \item
               D'après la partie A, $f$ est décroissante pour $x > 1$ donc si $1 < n \leqslant  x \leqslant  n+1$:
               \par
               $f\left(n+1\right) \leqslant  f\left(x\right) \leqslant  f\left(n\right)$
               \par
               D'après les propriétés de l'intégrale, on a donc  :
               \par
               $\int_{n}^{n+1}f\left(n+1\right)dx \leqslant  \int_{n}^{n+1}f\left(x\right)dx \leqslant  \int_{n}^{n+1}f\left(n\right)dx$
               \par
               soit, comme $f\left(n\right)$ et $f\left(n+1\right)$ sont des constantes :
               \par
               $f\left(n+1\right)\int_{n}^{n+1}1dx \leqslant  \int_{n}^{n+1}f\left(x\right)dx \leqslant  f\left(n\right)\int_{n}^{n+1}1dx$
               \par
               Or :
               \par
               $\int_{n}^{n+1}1dx=\left[x\right]_{n}^{n+1}=n+1-n=1$
               \par
               donc :
               \par
               $f\left(n+1\right) \leqslant  \int_{n}^{n+1}f\left(x\right)dx \leqslant  f\left(n\right)$
               \par
               c'est à dire :
               \par
               $f\left(n+1\right) \leqslant  u_{n} \leqslant  f\left(n\right)$
               \item
               Pour $n\geqslant 2$, d'après la question précédente : $f\left(n+1\right) \leqslant  u_{n}$ et $u_{n+1}\leqslant  f\left(n+1\right)$ donc $u_{n+1}\leqslant  u_{n}$.
               \par
               La suite $\left(u_{n}\right)_{n\geqslant 2}$ est décroissante.
               \item
               La fonction $f$ étant positive sur $\left[0; +\infty \right[$, $u_{n}=\int_{n}^{n+1}f\left(x\right)dx \geqslant  0$ pour tout entier $n$.
               \par
               La suite $\left(u_{n}\right)$ est décroissante à partir du rang 2 et minorée par 0. Elle est donc convergente.
               \par
               D'après le théorème des gendarmes :
               \par
               $\lim\limits_{n\rightarrow +\infty }f\left(n+1\right) \leqslant  \lim\limits_{n\rightarrow +\infty }u_{n} \leqslant  \lim\limits_{n\rightarrow +\infty }f\left(n\right)$
               \par
               donc, d'après la partie A :
               \par
               $0 \leqslant  \lim\limits_{n\rightarrow +\infty }u_{n} \leqslant  0$
               \par
               c'est à dire :
               \par
               $\lim\limits_{n\rightarrow +\infty }u_{n} = 0$
          \end{enumerate}
          \item
          \begin{enumerate}[label=\alph*.]
               \item
               Montrons par récurrence que pour tout entier naturel $n$ strictement positif, $F\left(n\right) = \sum_{k=0}^{n-1}u_{k}$.
               \textbf{Initialisation}
               Cette propriété est vraie pour $n = 1$; en effet :
               \par
               $F\left(1\right) = \int_{0}^{1}f\left(x\right)dx=u_{0}$ et $\sum_{k=0}^{0}u_{k}=u_{0}$
               \textbf{Hérédité}
               Supposons $F\left(n\right) = \sum_{k=0}^{n-1}u_{k}$ pour un certain entier $n$ fixé.
               \par
               $F\left(n+1\right) = \int_{0}^{n+1}f\left(x\right)dx = \int_{0}^{n}f\left(x\right)dx + \int_{n}^{n+1}f\left(x\right)dx$
               \par
               d'après la relation de Chasles pour les intégrales.
               \par
               Donc :
               \par
               $F\left(n+1\right) = F\left(n\right) + u_{n} = \sum_{k=0}^{n-1}u_{k} + u_{n} = \sum_{k=0}^{n}u_{k}$
               \par
               ce qui prouve bien l'hérédité.
               \par
               Donc, pour tout entier naturel $n$ strictement positif, $F\left(n\right) = \sum_{k=0}^{n-1}u_{k}$.
               \item
               Le tableau confirme que $\lim\limits_{n\rightarrow +\infty }F\left(n\right)=\frac{1}{2}$
               \par
               En réalité on n'a pas exactement $F\left(5\right)=\frac{1}{2}$. Ce résultat est dû à l'arrondi effectué par le tableur.
               \par
               En effet, d'après la partie A. $F\left(5\right)=-\frac{1}{2}e^{-5^{2}}+\frac{1}{2}=\frac{1}{2}-\frac{1}{2}e^{-25}$
               \par
               Or $e^{-25} < 1,4\times 10^{-11}$ n'est pas pris en compte par le tableur qui ne conserve apparemment que 10 décimales.
          \end{enumerate}
     \end{enumerate}
\end{corrige}

\end{document}