\documentclass[a4paper]{article}

%================================================================================================================================
%
% Packages
%
%================================================================================================================================

\usepackage[T1]{fontenc} 	% pour caractères accentués
\usepackage[utf8]{inputenc}  % encodage utf8
\usepackage[french]{babel}	% langue : français
\usepackage{fourier}			% caractères plus lisibles
\usepackage[dvipsnames]{xcolor} % couleurs
\usepackage{fancyhdr}		% réglage header footer
\usepackage{needspace}		% empêcher sauts de page mal placés
\usepackage{graphicx}		% pour inclure des graphiques
\usepackage{enumitem,cprotect}		% personnalise les listes d'items (nécessaire pour ol, al ...)
\usepackage{hyperref}		% Liens hypertexte
\usepackage{pstricks,pst-all,pst-node,pstricks-add,pst-math,pst-plot,pst-tree,pst-eucl} % pstricks
\usepackage[a4paper,includeheadfoot,top=2cm,left=3cm, bottom=2cm,right=3cm]{geometry} % marges etc.
\usepackage{comment}			% commentaires multilignes
\usepackage{amsmath,environ} % maths (matrices, etc.)
\usepackage{amssymb,makeidx}
\usepackage{bm}				% bold maths
\usepackage{tabularx}		% tableaux
\usepackage{colortbl}		% tableaux en couleur
\usepackage{fontawesome}		% Fontawesome
\usepackage{environ}			% environment with command
\usepackage{fp}				% calculs pour ps-tricks
\usepackage{multido}			% pour ps tricks
\usepackage[np]{numprint}	% formattage nombre
\usepackage{tikz,tkz-tab} 			% package principal TikZ
\usepackage{pgfplots}   % axes
\usepackage{mathrsfs}    % cursives
\usepackage{calc}			% calcul taille boites
\usepackage[scaled=0.875]{helvet} % font sans serif
\usepackage{svg} % svg
\usepackage{scrextend} % local margin
\usepackage{scratch} %scratch
\usepackage{multicol} % colonnes
%\usepackage{infix-RPN,pst-func} % formule en notation polanaise inversée
\usepackage{listings}

%================================================================================================================================
%
% Réglages de base
%
%================================================================================================================================

\lstset{
language=Python,   % R code
literate=
{á}{{\'a}}1
{à}{{\`a}}1
{ã}{{\~a}}1
{é}{{\'e}}1
{è}{{\`e}}1
{ê}{{\^e}}1
{í}{{\'i}}1
{ó}{{\'o}}1
{õ}{{\~o}}1
{ú}{{\'u}}1
{ü}{{\"u}}1
{ç}{{\c{c}}}1
{~}{{ }}1
}


\definecolor{codegreen}{rgb}{0,0.6,0}
\definecolor{codegray}{rgb}{0.5,0.5,0.5}
\definecolor{codepurple}{rgb}{0.58,0,0.82}
\definecolor{backcolour}{rgb}{0.95,0.95,0.92}

\lstdefinestyle{mystyle}{
    backgroundcolor=\color{backcolour},   
    commentstyle=\color{codegreen},
    keywordstyle=\color{magenta},
    numberstyle=\tiny\color{codegray},
    stringstyle=\color{codepurple},
    basicstyle=\ttfamily\footnotesize,
    breakatwhitespace=false,         
    breaklines=true,                 
    captionpos=b,                    
    keepspaces=true,                 
    numbers=left,                    
xleftmargin=2em,
framexleftmargin=2em,            
    showspaces=false,                
    showstringspaces=false,
    showtabs=false,                  
    tabsize=2,
    upquote=true
}

\lstset{style=mystyle}


\lstset{style=mystyle}
\newcommand{\imgdir}{C:/laragon/www/newmc/assets/imgsvg/}
\newcommand{\imgsvgdir}{C:/laragon/www/newmc/assets/imgsvg/}

\definecolor{mcgris}{RGB}{220, 220, 220}% ancien~; pour compatibilité
\definecolor{mcbleu}{RGB}{52, 152, 219}
\definecolor{mcvert}{RGB}{125, 194, 70}
\definecolor{mcmauve}{RGB}{154, 0, 215}
\definecolor{mcorange}{RGB}{255, 96, 0}
\definecolor{mcturquoise}{RGB}{0, 153, 153}
\definecolor{mcrouge}{RGB}{255, 0, 0}
\definecolor{mclightvert}{RGB}{205, 234, 190}

\definecolor{gris}{RGB}{220, 220, 220}
\definecolor{bleu}{RGB}{52, 152, 219}
\definecolor{vert}{RGB}{125, 194, 70}
\definecolor{mauve}{RGB}{154, 0, 215}
\definecolor{orange}{RGB}{255, 96, 0}
\definecolor{turquoise}{RGB}{0, 153, 153}
\definecolor{rouge}{RGB}{255, 0, 0}
\definecolor{lightvert}{RGB}{205, 234, 190}
\setitemize[0]{label=\color{lightvert}  $\bullet$}

\pagestyle{fancy}
\renewcommand{\headrulewidth}{0.2pt}
\fancyhead[L]{maths-cours.fr}
\fancyhead[R]{\thepage}
\renewcommand{\footrulewidth}{0.2pt}
\fancyfoot[C]{}

\newcolumntype{C}{>{\centering\arraybackslash}X}
\newcolumntype{s}{>{\hsize=.35\hsize\arraybackslash}X}

\setlength{\parindent}{0pt}		 
\setlength{\parskip}{3mm}
\setlength{\headheight}{1cm}

\def\ebook{ebook}
\def\book{book}
\def\web{web}
\def\type{web}

\newcommand{\vect}[1]{\overrightarrow{\,\mathstrut#1\,}}

\def\Oij{$\left(\text{O}~;~\vect{\imath},~\vect{\jmath}\right)$}
\def\Oijk{$\left(\text{O}~;~\vect{\imath},~\vect{\jmath},~\vect{k}\right)$}
\def\Ouv{$\left(\text{O}~;~\vect{u},~\vect{v}\right)$}

\hypersetup{breaklinks=true, colorlinks = true, linkcolor = OliveGreen, urlcolor = OliveGreen, citecolor = OliveGreen, pdfauthor={Didier BONNEL - https://www.maths-cours.fr} } % supprime les bordures autour des liens

\renewcommand{\arg}[0]{\text{arg}}

\everymath{\displaystyle}

%================================================================================================================================
%
% Macros - Commandes
%
%================================================================================================================================

\newcommand\meta[2]{    			% Utilisé pour créer le post HTML.
	\def\titre{titre}
	\def\url{url}
	\def\arg{#1}
	\ifx\titre\arg
		\newcommand\maintitle{#2}
		\fancyhead[L]{#2}
		{\Large\sffamily \MakeUppercase{#2}}
		\vspace{1mm}\textcolor{mcvert}{\hrule}
	\fi 
	\ifx\url\arg
		\fancyfoot[L]{\href{https://www.maths-cours.fr#2}{\black \footnotesize{https://www.maths-cours.fr#2}}}
	\fi 
}


\newcommand\TitreC[1]{    		% Titre centré
     \needspace{3\baselineskip}
     \begin{center}\textbf{#1}\end{center}
}

\newcommand\newpar{    		% paragraphe
     \par
}

\newcommand\nosp {    		% commande vide (pas d'espace)
}
\newcommand{\id}[1]{} %ignore

\newcommand\boite[2]{				% Boite simple sans titre
	\vspace{5mm}
	\setlength{\fboxrule}{0.2mm}
	\setlength{\fboxsep}{5mm}	
	\fcolorbox{#1}{#1!3}{\makebox[\linewidth-2\fboxrule-2\fboxsep]{
  		\begin{minipage}[t]{\linewidth-2\fboxrule-4\fboxsep}\setlength{\parskip}{3mm}
  			 #2
  		\end{minipage}
	}}
	\vspace{5mm}
}

\newcommand\CBox[4]{				% Boites
	\vspace{5mm}
	\setlength{\fboxrule}{0.2mm}
	\setlength{\fboxsep}{5mm}
	
	\fcolorbox{#1}{#1!3}{\makebox[\linewidth-2\fboxrule-2\fboxsep]{
		\begin{minipage}[t]{1cm}\setlength{\parskip}{3mm}
	  		\textcolor{#1}{\LARGE{#2}}    
 	 	\end{minipage}  
  		\begin{minipage}[t]{\linewidth-2\fboxrule-4\fboxsep}\setlength{\parskip}{3mm}
			\raisebox{1.2mm}{\normalsize\sffamily{\textcolor{#1}{#3}}}						
  			 #4
  		\end{minipage}
	}}
	\vspace{5mm}
}

\newcommand\cadre[3]{				% Boites convertible html
	\par
	\vspace{2mm}
	\setlength{\fboxrule}{0.1mm}
	\setlength{\fboxsep}{5mm}
	\fcolorbox{#1}{white}{\makebox[\linewidth-2\fboxrule-2\fboxsep]{
  		\begin{minipage}[t]{\linewidth-2\fboxrule-4\fboxsep}\setlength{\parskip}{3mm}
			\raisebox{-2.5mm}{\sffamily \small{\textcolor{#1}{\MakeUppercase{#2}}}}		
			\par		
  			 #3
 	 		\end{minipage}
	}}
		\vspace{2mm}
	\par
}

\newcommand\bloc[3]{				% Boites convertible html sans bordure
     \needspace{2\baselineskip}
     {\sffamily \small{\textcolor{#1}{\MakeUppercase{#2}}}}    
		\par		
  			 #3
		\par
}

\newcommand\CHelp[1]{
     \CBox{Plum}{\faInfoCircle}{À RETENIR}{#1}
}

\newcommand\CUp[1]{
     \CBox{NavyBlue}{\faThumbsOUp}{EN PRATIQUE}{#1}
}

\newcommand\CInfo[1]{
     \CBox{Sepia}{\faArrowCircleRight}{REMARQUE}{#1}
}

\newcommand\CRedac[1]{
     \CBox{PineGreen}{\faEdit}{BIEN R\'EDIGER}{#1}
}

\newcommand\CError[1]{
     \CBox{Red}{\faExclamationTriangle}{ATTENTION}{#1}
}

\newcommand\TitreExo[2]{
\needspace{4\baselineskip}
 {\sffamily\large EXERCICE #1\ (\emph{#2 points})}
\vspace{5mm}
}

\newcommand\img[2]{
          \includegraphics[width=#2\paperwidth]{\imgdir#1}
}

\newcommand\imgsvg[2]{
       \begin{center}   \includegraphics[width=#2\paperwidth]{\imgsvgdir#1} \end{center}
}


\newcommand\Lien[2]{
     \href{#1}{#2 \tiny \faExternalLink}
}
\newcommand\mcLien[2]{
     \href{https~://www.maths-cours.fr/#1}{#2 \tiny \faExternalLink}
}

\newcommand{\euro}{\eurologo{}}

%================================================================================================================================
%
% Macros - Environement
%
%================================================================================================================================

\newenvironment{tex}{ %
}
{%
}

\newenvironment{indente}{ %
	\setlength\parindent{10mm}
}

{
	\setlength\parindent{0mm}
}

\newenvironment{corrige}{%
     \needspace{3\baselineskip}
     \medskip
     \textbf{\textsc{Corrigé}}
     \medskip
}
{
}

\newenvironment{extern}{%
     \begin{center}
     }
     {
     \end{center}
}

\NewEnviron{code}{%
	\par
     \boite{gray}{\texttt{%
     \BODY
     }}
     \par
}

\newenvironment{vbloc}{% boite sans cadre empeche saut de page
     \begin{minipage}[t]{\linewidth}
     }
     {
     \end{minipage}
}
\NewEnviron{h2}{%
    \needspace{3\baselineskip}
    \vspace{0.6cm}
	\noindent \MakeUppercase{\sffamily \large \BODY}
	\vspace{1mm}\textcolor{mcgris}{\hrule}\vspace{0.4cm}
	\par
}{}

\NewEnviron{h3}{%
    \needspace{3\baselineskip}
	\vspace{5mm}
	\textsc{\BODY}
	\par
}

\NewEnviron{margeneg}{ %
\begin{addmargin}[-1cm]{0cm}
\BODY
\end{addmargin}
}

\NewEnviron{html}{%
}

\begin{document}
\meta{url}{/cours/variations-une-fonction/}
\meta{pid}{315}
\meta{titre}{Variations d'une fonction - Fonctions associées}
\meta{type}{cours}
\begin{h2}I - Rappels\end{h2}
\cadre{bleu}{Définitions}{%
     On dit qu'une fonction $f$ définie sur un intervalle $I$ est :
     \begin{itemize}
          \item \textbf{croissante} sur l'intervalle $I$: si pour tous réels $x_{1}$ et $x_{2}$  appartenant à $I$ tels que $x_{1}\leqslant  x_{2}$ on a $f\left(x_{1}\right)\leqslant f\left(x_{2}\right)$.
          \item \textbf{décroissante} sur l'intervalle $I$: si pour tous réels $x_{1}$ et $x_{2}$  appartenant à $I$ tels que $x_{1} \leqslant x_{2}$ on a $f\left(x_{1}\right) \geqslant f\left(x_{2}\right)$.
          \item \textbf{strictement croissante} sur l'intervalle $I$: si pour tous réels $x_{1}$ et $x_{2}$  appartenant à $I$ tels que $x_{1} <   x_{2}$ on a $f\left(x_{1}\right) <  f\left(x_{2}\right)$.
          \item \textbf{strictement décroissante} sur l'intervalle $I$: si pour tous réels $x_{1}$ et $x_{2}$  appartenant à $I$ tels que $x_{1} <   x_{2}$ on a $f\left(x_{1}\right) >  f\left(x_{2}\right)$.
     \end{itemize}
}
\begin{center}
     \begin{extern}%width="550" alt="Fonctions croissante et décroissante"
          \begin{tabular}{c c c}
               \resizebox{5.5cm}{!}{%
                    % -+-+-+ variables modifiables
                    \def\fonction{1+0.2*x*x }
                    \def\xmin{-1.2}
                    \def\xmax{5}
                    \def\ymin{-0.9}
                    \def\ymax{5}
                    \def\xunit{1}  % unités en cm
                    \def\yunit{1}
                    \psset{xunit=\xunit,yunit=\yunit,algebraic=true}
                    \fontsize{12pt}{12pt}\selectfont
                    \begin{pspicture*}[linewidth=1pt](\xmin,\ymin)(\xmax,\ymax)
                         %      \psgrid[gridcolor=mcgris, subgriddiv=5, gridlabels=0pt](\xmin,\ymin)(\xmax,\ymax)
                         \psaxes[linewidth=0.75pt,Dx=10,Dy=10]{->}(0,0)(\xmin,\ymin)(\xmax,\ymax)
                         \psplot[plotpoints=2000,linecolor=red]{0.2}{\xmax}{\fonction}
                         \psline[linewidth=0.75pt,linecolor=lightgray](1,0)(1,1.2)(0,1.2)
                         \psline[linewidth=0.75pt,linecolor=lightgray](4,0)(4,4.2)(0,4.2)
                         \rput[tr](-0.3,-0.3){$O$} \rput[t](1,-0.3){$x_1$} \rput[t](4,-0.3){$x_2$}
                         \rput[r](-0.1,1.2){$f(x_1)$} \rput[r](-0.1,4.2){$f(x_2)$}
                         \rput[tl](4.3,4.5){$\color{red} \mathcal{C}_f$}
                    \end{pspicture*}
               }
               & ~~~~ &%
               \resizebox{5.5cm}{!}{%
                    % -+-+-+ variables modifiables
                    \def\fonction{4-0.1*x*x }
                    \def\xmin{-1.2}
                    \def\xmax{5}
                    \def\ymin{-0.9}
                    \def\ymax{5}
                    \def\xunit{1}  % unités en cm
                    \def\yunit{1}
                    \psset{xunit=\xunit,yunit=\yunit,algebraic=true}
                    \fontsize{12pt}{12pt}\selectfont
                    \begin{pspicture*}[linewidth=1pt](\xmin,\ymin)(\xmax,\ymax)
                         %      \psgrid[gridcolor=mcgris, subgriddiv=5, gridlabels=0pt](\xmin,\ymin)(\xmax,\ymax)
                         \psaxes[linewidth=0.75pt,Dx=10,Dy=10]{->}(0,0)(\xmin,\ymin)(\xmax,\ymax)
                         \psplot[plotpoints=2000,linecolor=red]{0.2}{\xmax}{\fonction}
                         \psline[linewidth=0.75pt,linecolor=lightgray](1,0)(1,3.9)(0,3.9)
                         \psline[linewidth=0.75pt,linecolor=lightgray](4,0)(4,2.4)(0,2.4)
                         \rput[tr](-0.3,-0.3){$O$} \rput[t](1,-0.3){$x_1$} \rput[t](4,-0.3){$x_2$}
                         \rput[r](-0.1,3.9){$f(x_1)$} \rput[r](-0.1,2.4){$f(x_2)$}
                         \rput[tl](4.5,2.5){$\color{red} \mathcal{C}_f$}
                    \end{pspicture*}
               }
               \\
               Fonction croissante & ~~~~ & Fonction décroissante %
               \\
          \end{tabular}
     \end{extern}
\end{center}
\bloc{cyan}{Remarques}{%
     \begin{itemize}
          \item Une fonction qui dont le sens de variations ne change pas sur $I$ (c'est à dire qui est soit croissante sur $I$ soit décroissante sur $I$) est dite \textbf{monotone} sur $I$.
          \item Une fonction constante ($x\mapsto k$ où $k$ est un réel fixé) est à la fois croissante et décroissante mais n'est ni strictement croissante, ni strictement décroissante.
     \end{itemize}
}
\cadre{vert}{Propriété}{%
     Une fonction affine $f : x\mapsto ax+b$ est croissante si son coefficient directeur $a$ est \textbf{positif ou nul}, et \textbf{décroissante} si son coefficient directeur est \textbf{négatif ou nul}.
}
\bloc{cyan}{Remarque}{%
     Si le coefficient directeur d'une fonction affine est nul la fonction est \textbf{constante}.
}
\begin{h2}II - Fonction associées\end{h2}
\cadre{bleu}{Fonctions $u+k$}{%
     Soit $u$ une fonction définie sur une partie $\mathscr D$ de $\mathbb{R}$ et $k \in  \mathbb{R}$
     \par
     On note $u+k$ la fonction définie sur $\mathscr D$ par :
     \begin{center}$u+k :  x\mapsto u\left(x\right)+k$\end{center}
}
\cadre{vert}{Propriété}{%
     Quel que soit $k \in  \mathbb{R}$, $u+k$ a le même sens de variation que $u$ sur $\mathscr D$.
}
\bloc{orange}{Exemple}{%
     Soit $f$ définie sur $\mathbb{R}$ par $f\left(x\right)=x^{2}-1$.
     \par
     Si on note $u$ la fonction \textit{carrée} définie sur $\mathbb{R}$ par  $u : x \mapsto x^{2}$
     \par
     on a $f = u-1$
     \par
     Le sens de variation de $f$ est donc identique à celui de $u$ d'après la propriété précédente.
     \par
     Donc
     \begin{itemize}
          \item $f$ est \textbf{décroissante} sur l'intervalle $\left]-\infty  ; 0\right]$
          \item $f$ est \textbf{croissante} sur l'intervalle $\left[0 ; +\infty \right[$
     \end{itemize}
     \begin{center}
          \begin{extern}%width="350" alt="Tableau de variation x²-1"
               \begin{tikzpicture}[scale=0.875]
                    % Styles
                    \tikzstyle{cadre}=[thin]
                    \tikzstyle{fleche}=[->,>=latex,thin]
                    \tikzstyle{nondefini}=[lightgray]
                    % Dimensions Modifiables
                    \def\Lrg{1.5}
                    \def\HtX{1}
                    \def\HtY{0.5}
                    % Dimensions Calculées
                    \def\lignex{-0.5*\HtX}
                    \def\lignef{-1.5*\HtX}
                    \def\separateur{-0.5*\Lrg}
                    % Largeur du tableau
                    \def\gauche{-1.5*\Lrg}
                    \def\droite{4.5*\Lrg}
                    % Hauteur du tableau
                    \def\haut{0.5*\HtX}
                    \def\bas{-1.5*\HtX-2*\HtY}
                    % Pointillés
                    \draw[lightgray] (2*\Lrg,\lignex) -- (2*\Lrg,\bas);
                    % Ligne de l'abscisse : x
                    \node at (-1*\Lrg,0) {$x$};
                    \node at (0*\Lrg,0) {$-\infty$};
                    \node at (2*\Lrg,0) {$0$};
                    \node at (4*\Lrg,0) {$+\infty$};
                    % Ligne de la fonction : f(x)
                    \node  at (-1*\Lrg,{-1*\HtX+(-1)*\HtY}) {$x^2-1$};
                    \node (f1) at (0*\Lrg,{-1*\HtX+(0)*\HtY}) {$ $};
                    \node (f2) at (2*\Lrg,{-1*\HtX+(-2)*\HtY}) {$-1$};
                    \node (f3) at (4*\Lrg,{-1*\HtX+(0)*\HtY}) {$ $};
                    % Flèches
                    \draw[fleche] (f1) -- (f2);
                    \draw[fleche] (f2) -- (f3);
                    % Encadrement
                    \draw[cadre] (\separateur,\haut) -- (\separateur,\bas);
                    \draw[cadre] (\gauche,\haut) rectangle  (\droite,\bas);
                    \draw[cadre] (\gauche,\lignex) -- (\droite,\lignex);
               \end{tikzpicture}
          \end{extern}
     \end{center}
     %:-+-+-+-+- Fin
}
\cadre{bleu}{Fonctions $k\times u$}{%
     Soit $u$ une fonction définie sur une partie $\mathscr D$ de $\mathbb{R}$ et $k \in  \mathbb{R}$
     \par
     On note $ku$ la fonction définie sur $\mathscr D$ par :
     \begin{center}$ku :  x\mapsto k\times u\left(x\right)$\end{center}
}
\cadre{vert}{Propriété}{%
     \begin{itemize}
          \item  si $k > 0$, $ku$ a le même sens de variation que $u$ sur $\mathscr D$.
          \item  si $k < 0$, le sens de variation de $ku$ est le contraire de celui de $u$ sur $\mathscr D$.
     \end{itemize}
}
\bloc{orange}{Exemple}{%
     Soit $f$ définie sur $\left]-\infty  ; 0\right[ \cup  \left]0 ; +\infty \right[$ par $f\left(x\right)=-\frac{1}{x}$.
     \par
     Si on note $u$ la fonction \textit{inverse} définie sur  $\left]-\infty  ; 0\right[ \cup  \left]0 ; +\infty \right[$ par  $u : x \mapsto \frac{1}{x}$
     \par
     on a $f = -1\times u$
     \par
     Comme $-1$ est négatif, le sens de variation de $f$ est inverse de celui de $u$ sur chacun des intervalles $\left]-\infty  ; 0\right[$ et $\left]0 ; +\infty \right[$
     \par
     Donc $f$ est \textbf{croissante} sur l'intervalle $\left]-\infty  ; 0\right]$ et sur l'intervalle $\left]0 ; +\infty \right[$
}
\begin{center}
     \begin{extern}%width="350" alt="tableau de variation fonction inverse négative"
          \begin{tikzpicture}[scale=0.875]
               % Styles
               \tikzstyle{cadre}=[thin]
               \tikzstyle{fleche}=[->,>=latex,thin]
               \tikzstyle{nondefini}=[lightgray]
               % Dimensions Modifiables
               \def\Lrg{1.5}
               \def\HtX{1}
               \def\HtY{0.5}
               % Dimensions Calculées
               \def\lignex{-0.5*\HtX}
               \def\lignef{-1.5*\HtX}
               \def\separateur{-0.5*\Lrg}
               % Largeur du tableau
               \def\gauche{-1.5*\Lrg}
               \def\droite{4.5*\Lrg}
               % Hauteur du tableau
               \def\haut{0.5*\HtX}
               \def\bas{-1.5*\HtX-2*\HtY}
               % Ligne de l'abscisse : x
               \node at (-1*\Lrg,0) {$x$};
               \node at (0*\Lrg,0) {$-\infty$};
               \node at (2*\Lrg,0) {$0$};
               \node at (4*\Lrg,0) {$+\infty$};
               % Ligne de la fonction : f(x)
               \node  at (-1*\Lrg,{-1*\HtX+(-1)*\HtY}) {$f$};
               \node (f1) at (0*\Lrg,{-1*\HtX+(-2)*\HtY}) {$ $};
               \node[left] (f2) at (2*\Lrg,{-1*\HtX+(0)*\HtY}) {$ $};
               \node[right] (f3) at (2*\Lrg,{-1*\HtX+(-2)*\HtY}) {$ $};
               \node (f4) at (4*\Lrg,{-1*\HtX+(0)*\HtY}) {$ $};
               % Flèches
               \draw[fleche] (f1) -- (f2);
               \draw[fleche] (f3) -- (f4);
               % Doubles barres
               \draw[double distance=2pt] (2*\Lrg,\lignef+0.9*\HtX) -- (2*\Lrg,\bas+0.1*\HtX);
               % Encadrement
               \draw[cadre] (\separateur,\haut) -- (\separateur,\bas);
               \draw[cadre] (\gauche,\haut) rectangle  (\droite,\bas);
               \draw[cadre] (\gauche,\lignex) -- (\droite,\lignex);
          \end{tikzpicture}
     \end{extern}
\end{center}
\cadre{bleu}{Fonctions $\sqrt{u}$}{%
     Soit $u$ une fonction définie sur une partie $\mathscr D$ de $\mathbb{R}$.
     \par
     On note $\sqrt{u}$ la fonction définie, pour tout $x$ de $\mathscr D$  tel que $u\left(x\right) \geqslant  0$, par~:
     \begin{center}$\sqrt{u} :  x\mapsto \sqrt{u\left(x\right)}$\end{center}
}
\cadre{vert}{Propriété}{%
     $\sqrt{u}$  a le \textbf{même sens de variation} que $u$ sur tout intervalle où $u$ est positive.
}
\bloc{orange}{Exemple}{%
     Soit $f : x \mapsto  \sqrt{x-2}$
     \par
     $f$ est définie si et seulement si $x-2 \geqslant  0$, c'est à dire sur $\mathscr D=\left[2 ; +\infty \right[$
     \par
     Sur l'intervalle $\mathscr D$ la fonction $f$ est croissante car la fonction $x \mapsto  x-2$ l'est (fonction affine dont le coefficient directeur est positif).
}
\begin{center}
     \begin{extern}%width="250" alt="Tableau de variation sqrt(x-2)"
          \begin{tikzpicture}[scale=0.875]
               % Styles
               \tikzstyle{cadre}=[thin]
               \tikzstyle{fleche}=[->,>=latex,thin]
               \tikzstyle{nondefini}=[lightgray]
               % Dimensions Modifiables
               \def\Lrg{1.5}
               \def\HtX{1}
               \def\HtY{0.5}
               % Dimensions Calculées
               \def\lignex{-0.5*\HtX}
               \def\lignef{-1.5*\HtX}
               \def\separateur{-0.5*\Lrg}
               % Largeur du tableau
               \def\gauche{-1.5*\Lrg}
               \def\droite{2.5*\Lrg}
               % Hauteur du tableau
               \def\haut{0.5*\HtX}
               \def\bas{-1.5*\HtX-2*\HtY}
               % Ligne de l'abscisse : x
               \node at (-1*\Lrg,0) {$x$};
               \node at (0*\Lrg,0) {$2$};
               \node at (2*\Lrg,0) {$+\infty$};
               % Ligne de la fonction : f(x)
               \node  at (-1*\Lrg,{-1*\HtX+(-1)*\HtY}) {$\sqrt{x-2}$};
               \node (f1) at (0*\Lrg,{-1*\HtX+(-2)*\HtY}) {$0$};
               \node (f2) at (2*\Lrg,{-1*\HtX+(0)*\HtY}) {$ $};
               % Flèches
               \draw[fleche] (f1) -- (f2);
               % Encadrement
               \draw[cadre] (\separateur,\haut) -- (\separateur,\bas);
               \draw[cadre] (\gauche,\haut) rectangle  (\droite,\bas);
               \draw[cadre] (\gauche,\lignex) -- (\droite,\lignex);
          \end{tikzpicture}
     \end{extern}
\end{center}
\cadre{bleu}{Fonctions $\frac{1}{u}$}{%
     Soit $u$ une fonction définie sur une partie $\mathscr D$ de $\mathbb{R}$.
     \par
     On note $\frac{1}{u}$ la fonction définie pour tout $x$ de $\mathscr D$ \textbf{ tel que $u\left(x\right) \neq  0$} par :
     \begin{center}$\frac{1}{u} :  x\mapsto \frac{1}{u\left(x\right)}$\end{center}
}
\cadre{vert}{Propriété}{%
     $\frac{1}{u}$  a le \textbf{sens de variation contraire} de $u$ sur tout intervalle où $u$ ne s'annule pas et garde un \textbf{signe constant}.
}
\bloc{orange}{Exemple}{%
     Soit $ f : x \mapsto  \frac{1}{x+1}$
     \par
     $f$ est définie si et seulement si $x+1 \neq  0$, c'est à dire sur $\mathscr D=\left]-\infty  ; -1\right[ \cup  \left]-1 ; +\infty \right[$
     \par
     La fonction $x \mapsto  x+1$ est croissante sur $\mathbb{R}$
     \par
     Sur l'intervalle $\left]-\infty  ; -1\right[ $ la fonction $x \mapsto  x+1$ est strictement négative (donc a un signe constant).
     \par
     Sur l'intervalle $\left]-1 ; +\infty \right[$ la fonction $x \mapsto  x+1$ est strictement positive (donc a un signe constant).
     \par
     Donc $f$ est strictement décroissante sur chacun des intervalles $\left]-\infty  ; -1\right[ $ et $ \left]-1 ; +\infty \right[$
     \par
}
\begin{center}
     \begin{extern}%width="350" alt="Tableau de variation 1/(x+1)"
          \begin{tikzpicture}[scale=0.875]
               % Styles
               \tikzstyle{cadre}=[thin]
               \tikzstyle{fleche}=[->,>=latex,thin]
               \tikzstyle{nondefini}=[lightgray]
               % Dimensions Modifiables
               \def\Lrg{1.5}
               \def\HtX{1}
               \def\HtY{0.5}
               % Dimensions Calculées
               \def\lignex{-0.5*\HtX}
               \def\lignef{-1.5*\HtX}
               \def\separateur{-0.5*\Lrg}
               % Largeur du tableau
               \def\gauche{-1.5*\Lrg}
               \def\droite{4.5*\Lrg}
               % Hauteur du tableau
               \def\haut{0.5*\HtX}
               \def\bas{-1.5*\HtX-2*\HtY}
               % Ligne de l'abscisse : x
               \node at (-1*\Lrg,0) {$x$};
               \node at (0*\Lrg,0) {$-\infty$};
               \node at (2*\Lrg,0) {$-1$};
               \node at (4*\Lrg,0) {$+\infty$};
               % Ligne de la fonction : f(x)
               \node  at (-1*\Lrg,{-1*\HtX+(-1)*\HtY}) {$\dfrac{1}{x+1}$};
               \node (f1) at (0*\Lrg,{-1*\HtX+(0)*\HtY}) {$ $};
               \node[left] (f2) at (2*\Lrg,{-1*\HtX+(-2)*\HtY}) {$~$};
               \node[right] (f3) at (2*\Lrg,{-1*\HtX+(0)*\HtY}) {$~$};
               \node (f4) at (4*\Lrg,{-1*\HtX+(-2)*\HtY}) {$ $};
               % Flèches
               \draw[fleche] (f1) -- (f2);
               \draw[fleche] (f3) -- (f4);
               % Doubles barres
               \draw[double distance=2pt] (2*\Lrg,\lignef+\HtX) -- (2*\Lrg,\bas+0);
               % Encadrement
               \draw[cadre] (\separateur,\haut) -- (\separateur,\bas);
               \draw[cadre] (\gauche,\haut) rectangle  (\droite,\bas);
               \draw[cadre] (\gauche,\lignex) -- (\droite,\lignex);
          \end{tikzpicture}
     \end{extern}
\end{center}

\end{document}