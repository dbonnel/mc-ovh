\documentclass[a4paper]{article}

%================================================================================================================================
%
% Packages
%
%================================================================================================================================

\usepackage[T1]{fontenc} 	% pour caractères accentués
\usepackage[utf8]{inputenc}  % encodage utf8
\usepackage[french]{babel}	% langue : français
\usepackage{fourier}			% caractères plus lisibles
\usepackage[dvipsnames]{xcolor} % couleurs
\usepackage{fancyhdr}		% réglage header footer
\usepackage{needspace}		% empêcher sauts de page mal placés
\usepackage{graphicx}		% pour inclure des graphiques
\usepackage{enumitem,cprotect}		% personnalise les listes d'items (nécessaire pour ol, al ...)
\usepackage{hyperref}		% Liens hypertexte
\usepackage{pstricks,pst-all,pst-node,pstricks-add,pst-math,pst-plot,pst-tree,pst-eucl} % pstricks
\usepackage[a4paper,includeheadfoot,top=2cm,left=3cm, bottom=2cm,right=3cm]{geometry} % marges etc.
\usepackage{comment}			% commentaires multilignes
\usepackage{amsmath,environ} % maths (matrices, etc.)
\usepackage{amssymb,makeidx}
\usepackage{bm}				% bold maths
\usepackage{tabularx}		% tableaux
\usepackage{colortbl}		% tableaux en couleur
\usepackage{fontawesome}		% Fontawesome
\usepackage{environ}			% environment with command
\usepackage{fp}				% calculs pour ps-tricks
\usepackage{multido}			% pour ps tricks
\usepackage[np]{numprint}	% formattage nombre
\usepackage{tikz,tkz-tab} 			% package principal TikZ
\usepackage{pgfplots}   % axes
\usepackage{mathrsfs}    % cursives
\usepackage{calc}			% calcul taille boites
\usepackage[scaled=0.875]{helvet} % font sans serif
\usepackage{svg} % svg
\usepackage{scrextend} % local margin
\usepackage{scratch} %scratch
\usepackage{multicol} % colonnes
%\usepackage{infix-RPN,pst-func} % formule en notation polanaise inversée
\usepackage{listings}

%================================================================================================================================
%
% Réglages de base
%
%================================================================================================================================

\lstset{
language=Python,   % R code
literate=
{á}{{\'a}}1
{à}{{\`a}}1
{ã}{{\~a}}1
{é}{{\'e}}1
{è}{{\`e}}1
{ê}{{\^e}}1
{í}{{\'i}}1
{ó}{{\'o}}1
{õ}{{\~o}}1
{ú}{{\'u}}1
{ü}{{\"u}}1
{ç}{{\c{c}}}1
{~}{{ }}1
}


\definecolor{codegreen}{rgb}{0,0.6,0}
\definecolor{codegray}{rgb}{0.5,0.5,0.5}
\definecolor{codepurple}{rgb}{0.58,0,0.82}
\definecolor{backcolour}{rgb}{0.95,0.95,0.92}

\lstdefinestyle{mystyle}{
    backgroundcolor=\color{backcolour},   
    commentstyle=\color{codegreen},
    keywordstyle=\color{magenta},
    numberstyle=\tiny\color{codegray},
    stringstyle=\color{codepurple},
    basicstyle=\ttfamily\footnotesize,
    breakatwhitespace=false,         
    breaklines=true,                 
    captionpos=b,                    
    keepspaces=true,                 
    numbers=left,                    
xleftmargin=2em,
framexleftmargin=2em,            
    showspaces=false,                
    showstringspaces=false,
    showtabs=false,                  
    tabsize=2,
    upquote=true
}

\lstset{style=mystyle}


\lstset{style=mystyle}
\newcommand{\imgdir}{C:/laragon/www/newmc/assets/imgsvg/}
\newcommand{\imgsvgdir}{C:/laragon/www/newmc/assets/imgsvg/}

\definecolor{mcgris}{RGB}{220, 220, 220}% ancien~; pour compatibilité
\definecolor{mcbleu}{RGB}{52, 152, 219}
\definecolor{mcvert}{RGB}{125, 194, 70}
\definecolor{mcmauve}{RGB}{154, 0, 215}
\definecolor{mcorange}{RGB}{255, 96, 0}
\definecolor{mcturquoise}{RGB}{0, 153, 153}
\definecolor{mcrouge}{RGB}{255, 0, 0}
\definecolor{mclightvert}{RGB}{205, 234, 190}

\definecolor{gris}{RGB}{220, 220, 220}
\definecolor{bleu}{RGB}{52, 152, 219}
\definecolor{vert}{RGB}{125, 194, 70}
\definecolor{mauve}{RGB}{154, 0, 215}
\definecolor{orange}{RGB}{255, 96, 0}
\definecolor{turquoise}{RGB}{0, 153, 153}
\definecolor{rouge}{RGB}{255, 0, 0}
\definecolor{lightvert}{RGB}{205, 234, 190}
\setitemize[0]{label=\color{lightvert}  $\bullet$}

\pagestyle{fancy}
\renewcommand{\headrulewidth}{0.2pt}
\fancyhead[L]{maths-cours.fr}
\fancyhead[R]{\thepage}
\renewcommand{\footrulewidth}{0.2pt}
\fancyfoot[C]{}

\newcolumntype{C}{>{\centering\arraybackslash}X}
\newcolumntype{s}{>{\hsize=.35\hsize\arraybackslash}X}

\setlength{\parindent}{0pt}		 
\setlength{\parskip}{3mm}
\setlength{\headheight}{1cm}

\def\ebook{ebook}
\def\book{book}
\def\web{web}
\def\type{web}

\newcommand{\vect}[1]{\overrightarrow{\,\mathstrut#1\,}}

\def\Oij{$\left(\text{O}~;~\vect{\imath},~\vect{\jmath}\right)$}
\def\Oijk{$\left(\text{O}~;~\vect{\imath},~\vect{\jmath},~\vect{k}\right)$}
\def\Ouv{$\left(\text{O}~;~\vect{u},~\vect{v}\right)$}

\hypersetup{breaklinks=true, colorlinks = true, linkcolor = OliveGreen, urlcolor = OliveGreen, citecolor = OliveGreen, pdfauthor={Didier BONNEL - https://www.maths-cours.fr} } % supprime les bordures autour des liens

\renewcommand{\arg}[0]{\text{arg}}

\everymath{\displaystyle}

%================================================================================================================================
%
% Macros - Commandes
%
%================================================================================================================================

\newcommand\meta[2]{    			% Utilisé pour créer le post HTML.
	\def\titre{titre}
	\def\url{url}
	\def\arg{#1}
	\ifx\titre\arg
		\newcommand\maintitle{#2}
		\fancyhead[L]{#2}
		{\Large\sffamily \MakeUppercase{#2}}
		\vspace{1mm}\textcolor{mcvert}{\hrule}
	\fi 
	\ifx\url\arg
		\fancyfoot[L]{\href{https://www.maths-cours.fr#2}{\black \footnotesize{https://www.maths-cours.fr#2}}}
	\fi 
}


\newcommand\TitreC[1]{    		% Titre centré
     \needspace{3\baselineskip}
     \begin{center}\textbf{#1}\end{center}
}

\newcommand\newpar{    		% paragraphe
     \par
}

\newcommand\nosp {    		% commande vide (pas d'espace)
}
\newcommand{\id}[1]{} %ignore

\newcommand\boite[2]{				% Boite simple sans titre
	\vspace{5mm}
	\setlength{\fboxrule}{0.2mm}
	\setlength{\fboxsep}{5mm}	
	\fcolorbox{#1}{#1!3}{\makebox[\linewidth-2\fboxrule-2\fboxsep]{
  		\begin{minipage}[t]{\linewidth-2\fboxrule-4\fboxsep}\setlength{\parskip}{3mm}
  			 #2
  		\end{minipage}
	}}
	\vspace{5mm}
}

\newcommand\CBox[4]{				% Boites
	\vspace{5mm}
	\setlength{\fboxrule}{0.2mm}
	\setlength{\fboxsep}{5mm}
	
	\fcolorbox{#1}{#1!3}{\makebox[\linewidth-2\fboxrule-2\fboxsep]{
		\begin{minipage}[t]{1cm}\setlength{\parskip}{3mm}
	  		\textcolor{#1}{\LARGE{#2}}    
 	 	\end{minipage}  
  		\begin{minipage}[t]{\linewidth-2\fboxrule-4\fboxsep}\setlength{\parskip}{3mm}
			\raisebox{1.2mm}{\normalsize\sffamily{\textcolor{#1}{#3}}}						
  			 #4
  		\end{minipage}
	}}
	\vspace{5mm}
}

\newcommand\cadre[3]{				% Boites convertible html
	\par
	\vspace{2mm}
	\setlength{\fboxrule}{0.1mm}
	\setlength{\fboxsep}{5mm}
	\fcolorbox{#1}{white}{\makebox[\linewidth-2\fboxrule-2\fboxsep]{
  		\begin{minipage}[t]{\linewidth-2\fboxrule-4\fboxsep}\setlength{\parskip}{3mm}
			\raisebox{-2.5mm}{\sffamily \small{\textcolor{#1}{\MakeUppercase{#2}}}}		
			\par		
  			 #3
 	 		\end{minipage}
	}}
		\vspace{2mm}
	\par
}

\newcommand\bloc[3]{				% Boites convertible html sans bordure
     \needspace{2\baselineskip}
     {\sffamily \small{\textcolor{#1}{\MakeUppercase{#2}}}}    
		\par		
  			 #3
		\par
}

\newcommand\CHelp[1]{
     \CBox{Plum}{\faInfoCircle}{À RETENIR}{#1}
}

\newcommand\CUp[1]{
     \CBox{NavyBlue}{\faThumbsOUp}{EN PRATIQUE}{#1}
}

\newcommand\CInfo[1]{
     \CBox{Sepia}{\faArrowCircleRight}{REMARQUE}{#1}
}

\newcommand\CRedac[1]{
     \CBox{PineGreen}{\faEdit}{BIEN R\'EDIGER}{#1}
}

\newcommand\CError[1]{
     \CBox{Red}{\faExclamationTriangle}{ATTENTION}{#1}
}

\newcommand\TitreExo[2]{
\needspace{4\baselineskip}
 {\sffamily\large EXERCICE #1\ (\emph{#2 points})}
\vspace{5mm}
}

\newcommand\img[2]{
          \includegraphics[width=#2\paperwidth]{\imgdir#1}
}

\newcommand\imgsvg[2]{
       \begin{center}   \includegraphics[width=#2\paperwidth]{\imgsvgdir#1} \end{center}
}


\newcommand\Lien[2]{
     \href{#1}{#2 \tiny \faExternalLink}
}
\newcommand\mcLien[2]{
     \href{https~://www.maths-cours.fr/#1}{#2 \tiny \faExternalLink}
}

\newcommand{\euro}{\eurologo{}}

%================================================================================================================================
%
% Macros - Environement
%
%================================================================================================================================

\newenvironment{tex}{ %
}
{%
}

\newenvironment{indente}{ %
	\setlength\parindent{10mm}
}

{
	\setlength\parindent{0mm}
}

\newenvironment{corrige}{%
     \needspace{3\baselineskip}
     \medskip
     \textbf{\textsc{Corrigé}}
     \medskip
}
{
}

\newenvironment{extern}{%
     \begin{center}
     }
     {
     \end{center}
}

\NewEnviron{code}{%
	\par
     \boite{gray}{\texttt{%
     \BODY
     }}
     \par
}

\newenvironment{vbloc}{% boite sans cadre empeche saut de page
     \begin{minipage}[t]{\linewidth}
     }
     {
     \end{minipage}
}
\NewEnviron{h2}{%
    \needspace{3\baselineskip}
    \vspace{0.6cm}
	\noindent \MakeUppercase{\sffamily \large \BODY}
	\vspace{1mm}\textcolor{mcgris}{\hrule}\vspace{0.4cm}
	\par
}{}

\NewEnviron{h3}{%
    \needspace{3\baselineskip}
	\vspace{5mm}
	\textsc{\BODY}
	\par
}

\NewEnviron{margeneg}{ %
\begin{addmargin}[-1cm]{0cm}
\BODY
\end{addmargin}
}

\NewEnviron{html}{%
}

\begin{document}
\meta{url}{/exercices/exponentielle-bac-s-pondichery-2013/}
\meta{pid}{2498}
\meta{titre}{Exponentielle - Bac S Pondichéry 2013}
\meta{type}{exercices}
%
\begin{h2}Exercice 1   (5 points)\end{h2}
\textbf{Commun  à tous les candidats}
\begin{h3}Partie 1\end{h3}
On s'intéresse à l'évolution de la hauteur d'un plant de maïs en fonction du temps. Le graphique ci-dessous représente cette évolution. La hauteur est en mètres et le temps en jours.

\begin{center}
\imgsvg{mc-0036}{0.3}% alt="Exponentielle - Bac S Pondichéry 2013" style="width:50rem"
\end{center}

On décide de modéliser cette croissance par une fonction logistique du type :
\par
$h\left(t\right)=	\frac{a}{1+be^{-0,04t}}$
\par
où $a$ et $b$ sont des constantes réelles positives, $t$ est la variable temps exprimée en jours et $h\left(t\right)$ désigne la hauteur du plant, exprimée en mètres.
\par
On sait qu'initialement, pour $t=0$, le plant mesure 0,1 m et que sa hauteur tend vers une hauteur limite de 2 m.
\par
Déterminer les constantes $a$ et $b$ afin que la fonction $h$ corresponde à la croissance du plant de maïs étudié.
\begin{h2}Partie 2\end{h2}
On considère désormais que la croissance du plant de maïs est donnée par la fonction $f$ définie sur $\left[0 ; 250\right]$ par
\par
$f\left(t\right)=\frac{2}{1+19e^{-0,04t}}$
\begin{enumerate}
     \item
     Déterminer $f^{\prime}\left(t\right)$ en fonction de $t$ ($f^{\prime}$ désignant la fonction dérivée de la fonction $f$).
     \par
     En déduire les variations de la fonction $f$ sur l'intervalle $\left[0 ; 250\right]$.
     \item
     Calculer le temps nécessaire pour que le plant de maïs atteigne une hauteur supérieure à 1,5 m.
     \item
     \begin{enumerate}[label=\alph*.]
          \item
          Vérifier que pour tout réel $t$ appartenant à l'intervalle $\left[0 ; 250\right]$ on a $f\left(t\right)=\frac{2e^{0,04t}}{e^{0,04t}+19}$.
          \par
          Montrer que la fonction $F$ définie sur l'intervalle $\left[0 ; 250\right]$ par
          \par
          $F\left(t\right)=50\ln \left(e^{0,04t}+19\right)$ est une primitive de la fonction $f$.
          \item
          Déterminer la valeur moyenne de $f$ sur l'intervalle $\left[50 ; 100\right]$.
          \par
          En donner une valeur approchée à $10^{-2}$ près et interpréter ce résultat.
     \end{enumerate}
     \item
     On s'intéresse à la vitesse de croissance du plant de maïs  ; elle est donnée par la fonction dérivée de la fonction $f$.
     \par
     La vitesse de croissance est maximale pour une valeur de $t$.
     \par
     En utilisant le graphique, déterminer une valeur approchée de celle-ci. Estimer alors la hauteur du plant.
\end{enumerate}
\begin{corrige}
     \begin{h3}Partie 1\end{h3}
     D'après l'énoncé, la hauteur tend vers une hauteur limite de 2 m donc :
     \par
     $\lim\limits_{t\rightarrow +\infty }h\left(t\right)=2$
     \par
     Or $\lim\limits_{t\rightarrow +\infty }\frac{a}{1+be^{-0.04t}}=a$ (puisque $\lim\limits_{t\rightarrow +\infty }e^{-0.04t}=0$)
     \par
     Donc $a=2$.
     \par
     Par ailleurs, pour $t=0$, le plant mesure 0,1 m donc $h\left(0\right)=0,1$, c'est à dire:
     \par
     $\frac{a}{1+b}=0,1$
     \par
     $0,1b=a-0,1$
     \par
     $0,1b=1,9$
     \par
     $b=19$
     \par
     On a donc :
     \par
     $f\left(t\right)=\frac{2}{1+19e^{-0,04t}}$
     \begin{h3}Partie 2\end{h3}
     \begin{enumerate}
          \item
          La dérivée de $\frac{1}{u}$ est $-\frac{u^{\prime}}{u^{2}}$ donc :
          \par
          $f^{\prime}\left(t\right)=-\frac{2\times 19\times \left(-0,04e^{-0,04t}\right)}{\left(1+19e^{-0,04t}\right)^{2}}=\frac{1,52e^{-0,04t}}{\left(1+19e^{-0,04t}\right)^{2}}$
          \par
          Le numérateur et le dénominateur sont strictement positifs sur $\left[0 ; 250\right]$ donc $f$ est strictement croissante sur $\left[0 ; 250\right]$
          \item
          On cherche à résoudre l'inéquation $f\left(t\right)\geqslant 1,5$
          \par
          $f\left(t\right)\geqslant 1,5 \Leftrightarrow  \frac{2}{1+19e^{-0,04t}} \geqslant  1,5$
          \par
          $f\left(t\right)\geqslant 1,5 \Leftrightarrow  1,5\left(1+19e^{-0,04t}\right) \leqslant  2$
          \par
          $f\left(t\right)\geqslant 1,5 \Leftrightarrow  28,5e^{-0,04t} \leqslant  0,5$
          \par
          $f\left(t\right)\geqslant 1,5 \Leftrightarrow e^{-0,04t} \leqslant  \frac{1}{57}$ (car $\frac{0,5}{28,5}=\frac{0,5\times 2}{28,5\times 2}= \frac{1}{57}$)
          \par
          $f\left(t\right)\geqslant 1,5 \Leftrightarrow -0,04t \leqslant  \ln\left(\frac{1}{57}\right)$ (car la fonction $\ln$ est strictement croissante sur $\left]0 ; +\infty \right[$)
          \par
          $f\left(t\right)\geqslant 1,5 \Leftrightarrow t \geqslant  \frac{\ln\left(57\right)}{0,04}$ (car $\ln\left(\frac{1}{57}\right)=-\ln\left(57\right)$
          \par
          Comme $\frac{\ln\left(57\right)}{0,04} \approx  101,08$, le plant de maïs dépassera 1,5 m à compter du 102ème jour.
          \item
          \begin{enumerate}[label=\alph*.]
               \item
               En multipliant le numérateur et le dénominateur de $f\left(t\right)$ par $e^{0,04t}$ on obtient:
               \par
               $f\left(t\right)=\frac{2\times e^{0,04t}}{\left(1+19e^{-0,04t}\right)\times e^{0,04t}}=\frac{2e^{0,04t}}{e^{0,04t}+19}$.
               \par
               En utilisant la formule $\left(\ln\left(u\right)\right)^{\prime}=\frac{u^{\prime}}{u}$, on obtient :
               \par
               $F^{\prime}\left(t\right)=50\times \frac{0,04e^{0,04t}}{e^{0,04t}+19}= \frac{2e^{0,04t}}{e^{0,04t}+19}=f\left(t\right)$
               \par
               donc la fonction $F$ définie sur l'intervalle $\left[0 ; 250\right]$ par $F\left(t\right)=50\ln \left(e^{0,04t}+19\right)$ est une primitive de la fonction $f$.
               \item
               La valeur moyenne de $f$ sur l'intervalle $\left[50 ; 100\right]$ est donnée par :
               \par
               $m=\frac{1}{50}\int_{50}^{100}f\left(t\right)dt=\frac{1}{50}\left[F\left(t\right)\right]_{50}^{100}=\frac{1}{50}\left(F\left(100\right)-F\left(50\right)\right)\approx 1,03$ à $10^{-2}$ près.
               \par
               La croissance moyenne du plant de maïs entre le 50ème et le 100ème jour est d'environ 1m.
          \end{enumerate}
          \item
          La vitesse de croissance est maximale lorsque la pente de la tangente à la courbe est maximale. Sur le graphique, on voit que ceci est obtenu pour $t$ proche de 70 jours. La hauteur du plant est alors d'environ 1m.
     \end{enumerate}
\end{corrige}

\end{document}