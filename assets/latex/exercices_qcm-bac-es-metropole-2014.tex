\documentclass[a4paper]{article}

%================================================================================================================================
%
% Packages
%
%================================================================================================================================

\usepackage[T1]{fontenc} 	% pour caractères accentués
\usepackage[utf8]{inputenc}  % encodage utf8
\usepackage[french]{babel}	% langue : français
\usepackage{fourier}			% caractères plus lisibles
\usepackage[dvipsnames]{xcolor} % couleurs
\usepackage{fancyhdr}		% réglage header footer
\usepackage{needspace}		% empêcher sauts de page mal placés
\usepackage{graphicx}		% pour inclure des graphiques
\usepackage{enumitem,cprotect}		% personnalise les listes d'items (nécessaire pour ol, al ...)
\usepackage{hyperref}		% Liens hypertexte
\usepackage{pstricks,pst-all,pst-node,pstricks-add,pst-math,pst-plot,pst-tree,pst-eucl} % pstricks
\usepackage[a4paper,includeheadfoot,top=2cm,left=3cm, bottom=2cm,right=3cm]{geometry} % marges etc.
\usepackage{comment}			% commentaires multilignes
\usepackage{amsmath,environ} % maths (matrices, etc.)
\usepackage{amssymb,makeidx}
\usepackage{bm}				% bold maths
\usepackage{tabularx}		% tableaux
\usepackage{colortbl}		% tableaux en couleur
\usepackage{fontawesome}		% Fontawesome
\usepackage{environ}			% environment with command
\usepackage{fp}				% calculs pour ps-tricks
\usepackage{multido}			% pour ps tricks
\usepackage[np]{numprint}	% formattage nombre
\usepackage{tikz,tkz-tab} 			% package principal TikZ
\usepackage{pgfplots}   % axes
\usepackage{mathrsfs}    % cursives
\usepackage{calc}			% calcul taille boites
\usepackage[scaled=0.875]{helvet} % font sans serif
\usepackage{svg} % svg
\usepackage{scrextend} % local margin
\usepackage{scratch} %scratch
\usepackage{multicol} % colonnes
%\usepackage{infix-RPN,pst-func} % formule en notation polanaise inversée
\usepackage{listings}

%================================================================================================================================
%
% Réglages de base
%
%================================================================================================================================

\lstset{
language=Python,   % R code
literate=
{á}{{\'a}}1
{à}{{\`a}}1
{ã}{{\~a}}1
{é}{{\'e}}1
{è}{{\`e}}1
{ê}{{\^e}}1
{í}{{\'i}}1
{ó}{{\'o}}1
{õ}{{\~o}}1
{ú}{{\'u}}1
{ü}{{\"u}}1
{ç}{{\c{c}}}1
{~}{{ }}1
}


\definecolor{codegreen}{rgb}{0,0.6,0}
\definecolor{codegray}{rgb}{0.5,0.5,0.5}
\definecolor{codepurple}{rgb}{0.58,0,0.82}
\definecolor{backcolour}{rgb}{0.95,0.95,0.92}

\lstdefinestyle{mystyle}{
    backgroundcolor=\color{backcolour},   
    commentstyle=\color{codegreen},
    keywordstyle=\color{magenta},
    numberstyle=\tiny\color{codegray},
    stringstyle=\color{codepurple},
    basicstyle=\ttfamily\footnotesize,
    breakatwhitespace=false,         
    breaklines=true,                 
    captionpos=b,                    
    keepspaces=true,                 
    numbers=left,                    
xleftmargin=2em,
framexleftmargin=2em,            
    showspaces=false,                
    showstringspaces=false,
    showtabs=false,                  
    tabsize=2,
    upquote=true
}

\lstset{style=mystyle}


\lstset{style=mystyle}
\newcommand{\imgdir}{C:/laragon/www/newmc/assets/imgsvg/}
\newcommand{\imgsvgdir}{C:/laragon/www/newmc/assets/imgsvg/}

\definecolor{mcgris}{RGB}{220, 220, 220}% ancien~; pour compatibilité
\definecolor{mcbleu}{RGB}{52, 152, 219}
\definecolor{mcvert}{RGB}{125, 194, 70}
\definecolor{mcmauve}{RGB}{154, 0, 215}
\definecolor{mcorange}{RGB}{255, 96, 0}
\definecolor{mcturquoise}{RGB}{0, 153, 153}
\definecolor{mcrouge}{RGB}{255, 0, 0}
\definecolor{mclightvert}{RGB}{205, 234, 190}

\definecolor{gris}{RGB}{220, 220, 220}
\definecolor{bleu}{RGB}{52, 152, 219}
\definecolor{vert}{RGB}{125, 194, 70}
\definecolor{mauve}{RGB}{154, 0, 215}
\definecolor{orange}{RGB}{255, 96, 0}
\definecolor{turquoise}{RGB}{0, 153, 153}
\definecolor{rouge}{RGB}{255, 0, 0}
\definecolor{lightvert}{RGB}{205, 234, 190}
\setitemize[0]{label=\color{lightvert}  $\bullet$}

\pagestyle{fancy}
\renewcommand{\headrulewidth}{0.2pt}
\fancyhead[L]{maths-cours.fr}
\fancyhead[R]{\thepage}
\renewcommand{\footrulewidth}{0.2pt}
\fancyfoot[C]{}

\newcolumntype{C}{>{\centering\arraybackslash}X}
\newcolumntype{s}{>{\hsize=.35\hsize\arraybackslash}X}

\setlength{\parindent}{0pt}		 
\setlength{\parskip}{3mm}
\setlength{\headheight}{1cm}

\def\ebook{ebook}
\def\book{book}
\def\web{web}
\def\type{web}

\newcommand{\vect}[1]{\overrightarrow{\,\mathstrut#1\,}}

\def\Oij{$\left(\text{O}~;~\vect{\imath},~\vect{\jmath}\right)$}
\def\Oijk{$\left(\text{O}~;~\vect{\imath},~\vect{\jmath},~\vect{k}\right)$}
\def\Ouv{$\left(\text{O}~;~\vect{u},~\vect{v}\right)$}

\hypersetup{breaklinks=true, colorlinks = true, linkcolor = OliveGreen, urlcolor = OliveGreen, citecolor = OliveGreen, pdfauthor={Didier BONNEL - https://www.maths-cours.fr} } % supprime les bordures autour des liens

\renewcommand{\arg}[0]{\text{arg}}

\everymath{\displaystyle}

%================================================================================================================================
%
% Macros - Commandes
%
%================================================================================================================================

\newcommand\meta[2]{    			% Utilisé pour créer le post HTML.
	\def\titre{titre}
	\def\url{url}
	\def\arg{#1}
	\ifx\titre\arg
		\newcommand\maintitle{#2}
		\fancyhead[L]{#2}
		{\Large\sffamily \MakeUppercase{#2}}
		\vspace{1mm}\textcolor{mcvert}{\hrule}
	\fi 
	\ifx\url\arg
		\fancyfoot[L]{\href{https://www.maths-cours.fr#2}{\black \footnotesize{https://www.maths-cours.fr#2}}}
	\fi 
}


\newcommand\TitreC[1]{    		% Titre centré
     \needspace{3\baselineskip}
     \begin{center}\textbf{#1}\end{center}
}

\newcommand\newpar{    		% paragraphe
     \par
}

\newcommand\nosp {    		% commande vide (pas d'espace)
}
\newcommand{\id}[1]{} %ignore

\newcommand\boite[2]{				% Boite simple sans titre
	\vspace{5mm}
	\setlength{\fboxrule}{0.2mm}
	\setlength{\fboxsep}{5mm}	
	\fcolorbox{#1}{#1!3}{\makebox[\linewidth-2\fboxrule-2\fboxsep]{
  		\begin{minipage}[t]{\linewidth-2\fboxrule-4\fboxsep}\setlength{\parskip}{3mm}
  			 #2
  		\end{minipage}
	}}
	\vspace{5mm}
}

\newcommand\CBox[4]{				% Boites
	\vspace{5mm}
	\setlength{\fboxrule}{0.2mm}
	\setlength{\fboxsep}{5mm}
	
	\fcolorbox{#1}{#1!3}{\makebox[\linewidth-2\fboxrule-2\fboxsep]{
		\begin{minipage}[t]{1cm}\setlength{\parskip}{3mm}
	  		\textcolor{#1}{\LARGE{#2}}    
 	 	\end{minipage}  
  		\begin{minipage}[t]{\linewidth-2\fboxrule-4\fboxsep}\setlength{\parskip}{3mm}
			\raisebox{1.2mm}{\normalsize\sffamily{\textcolor{#1}{#3}}}						
  			 #4
  		\end{minipage}
	}}
	\vspace{5mm}
}

\newcommand\cadre[3]{				% Boites convertible html
	\par
	\vspace{2mm}
	\setlength{\fboxrule}{0.1mm}
	\setlength{\fboxsep}{5mm}
	\fcolorbox{#1}{white}{\makebox[\linewidth-2\fboxrule-2\fboxsep]{
  		\begin{minipage}[t]{\linewidth-2\fboxrule-4\fboxsep}\setlength{\parskip}{3mm}
			\raisebox{-2.5mm}{\sffamily \small{\textcolor{#1}{\MakeUppercase{#2}}}}		
			\par		
  			 #3
 	 		\end{minipage}
	}}
		\vspace{2mm}
	\par
}

\newcommand\bloc[3]{				% Boites convertible html sans bordure
     \needspace{2\baselineskip}
     {\sffamily \small{\textcolor{#1}{\MakeUppercase{#2}}}}    
		\par		
  			 #3
		\par
}

\newcommand\CHelp[1]{
     \CBox{Plum}{\faInfoCircle}{À RETENIR}{#1}
}

\newcommand\CUp[1]{
     \CBox{NavyBlue}{\faThumbsOUp}{EN PRATIQUE}{#1}
}

\newcommand\CInfo[1]{
     \CBox{Sepia}{\faArrowCircleRight}{REMARQUE}{#1}
}

\newcommand\CRedac[1]{
     \CBox{PineGreen}{\faEdit}{BIEN R\'EDIGER}{#1}
}

\newcommand\CError[1]{
     \CBox{Red}{\faExclamationTriangle}{ATTENTION}{#1}
}

\newcommand\TitreExo[2]{
\needspace{4\baselineskip}
 {\sffamily\large EXERCICE #1\ (\emph{#2 points})}
\vspace{5mm}
}

\newcommand\img[2]{
          \includegraphics[width=#2\paperwidth]{\imgdir#1}
}

\newcommand\imgsvg[2]{
       \begin{center}   \includegraphics[width=#2\paperwidth]{\imgsvgdir#1} \end{center}
}


\newcommand\Lien[2]{
     \href{#1}{#2 \tiny \faExternalLink}
}
\newcommand\mcLien[2]{
     \href{https~://www.maths-cours.fr/#1}{#2 \tiny \faExternalLink}
}

\newcommand{\euro}{\eurologo{}}

%================================================================================================================================
%
% Macros - Environement
%
%================================================================================================================================

\newenvironment{tex}{ %
}
{%
}

\newenvironment{indente}{ %
	\setlength\parindent{10mm}
}

{
	\setlength\parindent{0mm}
}

\newenvironment{corrige}{%
     \needspace{3\baselineskip}
     \medskip
     \textbf{\textsc{Corrigé}}
     \medskip
}
{
}

\newenvironment{extern}{%
     \begin{center}
     }
     {
     \end{center}
}

\NewEnviron{code}{%
	\par
     \boite{gray}{\texttt{%
     \BODY
     }}
     \par
}

\newenvironment{vbloc}{% boite sans cadre empeche saut de page
     \begin{minipage}[t]{\linewidth}
     }
     {
     \end{minipage}
}
\NewEnviron{h2}{%
    \needspace{3\baselineskip}
    \vspace{0.6cm}
	\noindent \MakeUppercase{\sffamily \large \BODY}
	\vspace{1mm}\textcolor{mcgris}{\hrule}\vspace{0.4cm}
	\par
}{}

\NewEnviron{h3}{%
    \needspace{3\baselineskip}
	\vspace{5mm}
	\textsc{\BODY}
	\par
}

\NewEnviron{margeneg}{ %
\begin{addmargin}[-1cm]{0cm}
\BODY
\end{addmargin}
}

\NewEnviron{html}{%
}

\begin{document}
\meta{url}{/exercices/qcm-bac-es-metropole-2014/}
\meta{pid}{2191}
\meta{titre}{QCM - Bac ES/L Métropole 2014}
\meta{type}{exercices}
%
\begin{h2}Exercice 1   (5 points)\end{h2}
\textbf{Commun à tous les candidats}
\par
\textit{Cet exercice est un questionnaire à choix multiples (QCM).
     \par
     Pour chacune des questions posées, une seule des quatre réponses est exacte.
     \par
     Indiquer sur la copie le numéro de la question et la lettre correspondant à la réponse choisie. Aucune justification n'est demandée.
     \par
Une réponse exacte rapporte $1$ point. Une réponse fausse, une réponse multiple ou l'absence de réponse ne rapporte ni n'enlève aucun point.}
\begin{enumerate}
     \item
     L'arbre de probabilités ci-dessous représente une situation où $A$ et $B$ sont deux évènements, dont les évènements contraires sont respectivement notés $\overline{A}$ et $\overline{B}$.
%##
% type=arbre; width=25; wcell=3.5; hcell=1.5
%--
% >A:0,6
% >>B:0,3
% >>\overline{B}:...
% >\overline{A}:...
% >>B:0,2
% >>\overline{B}:...
%--
\begin{center}
 \begin{extern}%style="width:25rem" alt="Arbre pondéré"
    \resizebox{11cm}{!}{
       \definecolor{dark}{gray}{0.1}
       \begin{tikzpicture}[scale=.8, line width=.5pt, dark]
       \def\width{3.5}
       \def\height{1.5}
       \tikzstyle{noeud}=[fill=white,circle,draw]
       \tikzstyle{poids}=[fill=white,font=\footnotesize,midway]
    \node[noeud] (r) at ({1*\width},{-1.5*\height}) {$$};
    \node[noeud] (ra) at ({2*\width},{-0.5*\height}) {$A$};
     \draw (r) -- (ra) node [poids] {$0,6$};
    \node[noeud] (raa) at ({3*\width},{0*\height}) {$B$};
     \draw (ra) -- (raa) node [poids] {$0,3$};
    \node[noeud] (rab) at ({3*\width},{-1*\height}) {$\overline{B}$};
     \draw (ra) -- (rab) node [poids] {$...$};
    \node[noeud] (rb) at ({2*\width},{-2.5*\height}) {$\overline{A}$};
     \draw (r) -- (rb) node [poids] {$...$};
    \node[noeud] (rba) at ({3*\width},{-2*\height}) {$B$};
     \draw (rb) -- (rba) node [poids] {$0,2$};
    \node[noeud] (rbb) at ({3*\width},{-3*\height}) {$\overline{B}$};
     \draw (rb) -- (rbb) node [poids] {$...$};
       \end{tikzpicture}
      }
   \end{extern}
\end{center}
%##
Alors :
     \begin{tabularx}{0.8\linewidth}{|*{3}{>{\centering \arraybackslash }X|}}%class="noborder" width="600"
          \hline
          \textbf{a}.   $P_{A}\left(B\right)=0,18$ & \textbf{b}.   $P\left(A \cap  B\right)=0,9$ & \textbf{c}.   $P_{A}\left(\overline{B}\right)=0,7$ & \textbf{d}.   $P\left(B\right)=0,5$
          \\ \hline
     \end{tabularx}
     \item
     Avec le même arbre, la probabilité de l'évènement $B$ est égale à :
     \begin{tabularx}{0.8\linewidth}{|*{3}{>{\centering \arraybackslash }X|}}%class="noborder" width="600"
          \hline
          \textbf{a}.   0,5 & \textbf{b}.   0,18 & \textbf{c}.   0,26 & \textbf{d}.   0,38
          \\ \hline
     \end{tabularx}
     \item
     On considère une fonction $f$ définie et continue sur l'intervalle [1 ; 15]. Son tableau de variation est indiqué ci-dessous.
\begin{center}
\begin{extern}%alt="tableau de variation" style="width:40rem"
\psset{unit=.9cm} 
\begin{pspicture}(12,3)
\psset{linewidth=.75pt}
\psframe(12,3) \psline(1.5,0)(1.5,3) \psline(0,2)(12,2)
\rput(.75,2.5){$x$}\rput(2,2.5){$1$}\rput(3.5,2.5){$3$} \rput(5,2.5){$4$} \rput(8,2.5){$12$} \rput(11.5,2.5){$15$} 
\rput(.75,1){$f(x)$} \rput(2,1.6){\rnode{A}{$3$}} \rput(5,.4){\rnode{B}{$-2$}} \rput(8,1.6){\rnode{C}{{$-1$}}} \rput(11.5,.4){\rnode{D}{$-3$}} 
\psset{nodesep=5pt,arrows=->}
\ncline{A}{B}\ncput*{0}  \ncline{B}{C}\ncline{C}{D}
\end{pspicture}
\end{extern}
\end{center}

     Soit $F$ une primitive de la fonction $f$ sur l'intervalle [1 ; 15]. On peut être certain que :
     \begin{enumerate}
          \item
          La fonction $F$ est négative sur l'intervalle [3 ; 4].
          \item
          La fonction $F$ est positive sur l'intervalle [4 ; 12].
          \item
          La fonction $F$ est décroissante sur l'intervalle [4 ; 12].
          \item
     La fonction $F$ est décroissante sur l'intervalle [1 ; 3].\end{enumerate}
     \item
     Pour tout réel $x$ de l'intervalle $\left]0 ; +\infty \right[$,l'équation $\ln x+\ln \left(x+3\right)=3 \ln 2$ est équivalente à l'équation :
     \begin{tabularx}{0.8\linewidth}{|*{3}{>{\centering \arraybackslash }X|}}%class="noborder" width="600"
          \hline
          \textbf{a}.   $2x+3=6$ & \textbf{b}.   $2x+3=8$  & \textbf{c}.   $x^{2}+3x=6$  & \textbf{d}.   $x^{2}+3x=8$
          \\ \hline
     \end{tabularx}
     \item
     $g$ est la fonction définie sur l'intervalle $\left]0 ; +\infty \right[$ par $g\left(x\right)=\frac{5}{x}$.
     \par
     On note $C$ sa courbe représentative.
     \par
     L'aire, exprimée en unités d'aire, du domaine délimité par la courbe $C$, l'axe des abscisses, et les droites d'équations $x=2$ et $x=6$, est égale à :
     \begin{tabularx}{0.8\linewidth}{|*{3}{>{\centering \arraybackslash }X|}}%class="noborder" width="600"
          \hline
          \textbf{a}.   $5 \left(\ln 6-\ln 2\right)$ & \textbf{b}.   $\frac{1}{6-2}\int_{2}^{6}g\left(x\right)dx$  & \textbf{c}.   $5 \ln 6+5 \ln 2$  & \textbf{d}.   $g\left(6\right)-g\left(2\right)$
          \\ \hline
     \end{tabularx}
\end{enumerate}
\begin{corrige}
     \begin{enumerate}
          \item
          Réponse exacte : \textbf{c.}
          \item
          Réponse exacte : \textbf{c.}
          \item
          Réponse exacte : \textbf{c.}
          \item
          Réponse exacte : \textbf{d.}
          \item
          Réponse exacte : \textbf{a.}
     \end{enumerate}
}\end{corrige}

\end{document}