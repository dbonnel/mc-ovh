\documentclass[a4paper]{article}

%================================================================================================================================
%
% Packages
%
%================================================================================================================================

\usepackage[T1]{fontenc} 	% pour caractères accentués
\usepackage[utf8]{inputenc}  % encodage utf8
\usepackage[french]{babel}	% langue : français
\usepackage{fourier}			% caractères plus lisibles
\usepackage[dvipsnames]{xcolor} % couleurs
\usepackage{fancyhdr}		% réglage header footer
\usepackage{needspace}		% empêcher sauts de page mal placés
\usepackage{graphicx}		% pour inclure des graphiques
\usepackage{enumitem,cprotect}		% personnalise les listes d'items (nécessaire pour ol, al ...)
\usepackage{hyperref}		% Liens hypertexte
\usepackage{pstricks,pst-all,pst-node,pstricks-add,pst-math,pst-plot,pst-tree,pst-eucl} % pstricks
\usepackage[a4paper,includeheadfoot,top=2cm,left=3cm, bottom=2cm,right=3cm]{geometry} % marges etc.
\usepackage{comment}			% commentaires multilignes
\usepackage{amsmath,environ} % maths (matrices, etc.)
\usepackage{amssymb,makeidx}
\usepackage{bm}				% bold maths
\usepackage{tabularx}		% tableaux
\usepackage{colortbl}		% tableaux en couleur
\usepackage{fontawesome}		% Fontawesome
\usepackage{environ}			% environment with command
\usepackage{fp}				% calculs pour ps-tricks
\usepackage{multido}			% pour ps tricks
\usepackage[np]{numprint}	% formattage nombre
\usepackage{tikz,tkz-tab} 			% package principal TikZ
\usepackage{pgfplots}   % axes
\usepackage{mathrsfs}    % cursives
\usepackage{calc}			% calcul taille boites
\usepackage[scaled=0.875]{helvet} % font sans serif
\usepackage{svg} % svg
\usepackage{scrextend} % local margin
\usepackage{scratch} %scratch
\usepackage{multicol} % colonnes
%\usepackage{infix-RPN,pst-func} % formule en notation polanaise inversée
\usepackage{listings}

%================================================================================================================================
%
% Réglages de base
%
%================================================================================================================================

\lstset{
language=Python,   % R code
literate=
{á}{{\'a}}1
{à}{{\`a}}1
{ã}{{\~a}}1
{é}{{\'e}}1
{è}{{\`e}}1
{ê}{{\^e}}1
{í}{{\'i}}1
{ó}{{\'o}}1
{õ}{{\~o}}1
{ú}{{\'u}}1
{ü}{{\"u}}1
{ç}{{\c{c}}}1
{~}{{ }}1
}


\definecolor{codegreen}{rgb}{0,0.6,0}
\definecolor{codegray}{rgb}{0.5,0.5,0.5}
\definecolor{codepurple}{rgb}{0.58,0,0.82}
\definecolor{backcolour}{rgb}{0.95,0.95,0.92}

\lstdefinestyle{mystyle}{
    backgroundcolor=\color{backcolour},   
    commentstyle=\color{codegreen},
    keywordstyle=\color{magenta},
    numberstyle=\tiny\color{codegray},
    stringstyle=\color{codepurple},
    basicstyle=\ttfamily\footnotesize,
    breakatwhitespace=false,         
    breaklines=true,                 
    captionpos=b,                    
    keepspaces=true,                 
    numbers=left,                    
xleftmargin=2em,
framexleftmargin=2em,            
    showspaces=false,                
    showstringspaces=false,
    showtabs=false,                  
    tabsize=2,
    upquote=true
}

\lstset{style=mystyle}


\lstset{style=mystyle}
\newcommand{\imgdir}{C:/laragon/www/newmc/assets/imgsvg/}
\newcommand{\imgsvgdir}{C:/laragon/www/newmc/assets/imgsvg/}

\definecolor{mcgris}{RGB}{220, 220, 220}% ancien~; pour compatibilité
\definecolor{mcbleu}{RGB}{52, 152, 219}
\definecolor{mcvert}{RGB}{125, 194, 70}
\definecolor{mcmauve}{RGB}{154, 0, 215}
\definecolor{mcorange}{RGB}{255, 96, 0}
\definecolor{mcturquoise}{RGB}{0, 153, 153}
\definecolor{mcrouge}{RGB}{255, 0, 0}
\definecolor{mclightvert}{RGB}{205, 234, 190}

\definecolor{gris}{RGB}{220, 220, 220}
\definecolor{bleu}{RGB}{52, 152, 219}
\definecolor{vert}{RGB}{125, 194, 70}
\definecolor{mauve}{RGB}{154, 0, 215}
\definecolor{orange}{RGB}{255, 96, 0}
\definecolor{turquoise}{RGB}{0, 153, 153}
\definecolor{rouge}{RGB}{255, 0, 0}
\definecolor{lightvert}{RGB}{205, 234, 190}
\setitemize[0]{label=\color{lightvert}  $\bullet$}

\pagestyle{fancy}
\renewcommand{\headrulewidth}{0.2pt}
\fancyhead[L]{maths-cours.fr}
\fancyhead[R]{\thepage}
\renewcommand{\footrulewidth}{0.2pt}
\fancyfoot[C]{}

\newcolumntype{C}{>{\centering\arraybackslash}X}
\newcolumntype{s}{>{\hsize=.35\hsize\arraybackslash}X}

\setlength{\parindent}{0pt}		 
\setlength{\parskip}{3mm}
\setlength{\headheight}{1cm}

\def\ebook{ebook}
\def\book{book}
\def\web{web}
\def\type{web}

\newcommand{\vect}[1]{\overrightarrow{\,\mathstrut#1\,}}

\def\Oij{$\left(\text{O}~;~\vect{\imath},~\vect{\jmath}\right)$}
\def\Oijk{$\left(\text{O}~;~\vect{\imath},~\vect{\jmath},~\vect{k}\right)$}
\def\Ouv{$\left(\text{O}~;~\vect{u},~\vect{v}\right)$}

\hypersetup{breaklinks=true, colorlinks = true, linkcolor = OliveGreen, urlcolor = OliveGreen, citecolor = OliveGreen, pdfauthor={Didier BONNEL - https://www.maths-cours.fr} } % supprime les bordures autour des liens

\renewcommand{\arg}[0]{\text{arg}}

\everymath{\displaystyle}

%================================================================================================================================
%
% Macros - Commandes
%
%================================================================================================================================

\newcommand\meta[2]{    			% Utilisé pour créer le post HTML.
	\def\titre{titre}
	\def\url{url}
	\def\arg{#1}
	\ifx\titre\arg
		\newcommand\maintitle{#2}
		\fancyhead[L]{#2}
		{\Large\sffamily \MakeUppercase{#2}}
		\vspace{1mm}\textcolor{mcvert}{\hrule}
	\fi 
	\ifx\url\arg
		\fancyfoot[L]{\href{https://www.maths-cours.fr#2}{\black \footnotesize{https://www.maths-cours.fr#2}}}
	\fi 
}


\newcommand\TitreC[1]{    		% Titre centré
     \needspace{3\baselineskip}
     \begin{center}\textbf{#1}\end{center}
}

\newcommand\newpar{    		% paragraphe
     \par
}

\newcommand\nosp {    		% commande vide (pas d'espace)
}
\newcommand{\id}[1]{} %ignore

\newcommand\boite[2]{				% Boite simple sans titre
	\vspace{5mm}
	\setlength{\fboxrule}{0.2mm}
	\setlength{\fboxsep}{5mm}	
	\fcolorbox{#1}{#1!3}{\makebox[\linewidth-2\fboxrule-2\fboxsep]{
  		\begin{minipage}[t]{\linewidth-2\fboxrule-4\fboxsep}\setlength{\parskip}{3mm}
  			 #2
  		\end{minipage}
	}}
	\vspace{5mm}
}

\newcommand\CBox[4]{				% Boites
	\vspace{5mm}
	\setlength{\fboxrule}{0.2mm}
	\setlength{\fboxsep}{5mm}
	
	\fcolorbox{#1}{#1!3}{\makebox[\linewidth-2\fboxrule-2\fboxsep]{
		\begin{minipage}[t]{1cm}\setlength{\parskip}{3mm}
	  		\textcolor{#1}{\LARGE{#2}}    
 	 	\end{minipage}  
  		\begin{minipage}[t]{\linewidth-2\fboxrule-4\fboxsep}\setlength{\parskip}{3mm}
			\raisebox{1.2mm}{\normalsize\sffamily{\textcolor{#1}{#3}}}						
  			 #4
  		\end{minipage}
	}}
	\vspace{5mm}
}

\newcommand\cadre[3]{				% Boites convertible html
	\par
	\vspace{2mm}
	\setlength{\fboxrule}{0.1mm}
	\setlength{\fboxsep}{5mm}
	\fcolorbox{#1}{white}{\makebox[\linewidth-2\fboxrule-2\fboxsep]{
  		\begin{minipage}[t]{\linewidth-2\fboxrule-4\fboxsep}\setlength{\parskip}{3mm}
			\raisebox{-2.5mm}{\sffamily \small{\textcolor{#1}{\MakeUppercase{#2}}}}		
			\par		
  			 #3
 	 		\end{minipage}
	}}
		\vspace{2mm}
	\par
}

\newcommand\bloc[3]{				% Boites convertible html sans bordure
     \needspace{2\baselineskip}
     {\sffamily \small{\textcolor{#1}{\MakeUppercase{#2}}}}    
		\par		
  			 #3
		\par
}

\newcommand\CHelp[1]{
     \CBox{Plum}{\faInfoCircle}{À RETENIR}{#1}
}

\newcommand\CUp[1]{
     \CBox{NavyBlue}{\faThumbsOUp}{EN PRATIQUE}{#1}
}

\newcommand\CInfo[1]{
     \CBox{Sepia}{\faArrowCircleRight}{REMARQUE}{#1}
}

\newcommand\CRedac[1]{
     \CBox{PineGreen}{\faEdit}{BIEN R\'EDIGER}{#1}
}

\newcommand\CError[1]{
     \CBox{Red}{\faExclamationTriangle}{ATTENTION}{#1}
}

\newcommand\TitreExo[2]{
\needspace{4\baselineskip}
 {\sffamily\large EXERCICE #1\ (\emph{#2 points})}
\vspace{5mm}
}

\newcommand\img[2]{
          \includegraphics[width=#2\paperwidth]{\imgdir#1}
}

\newcommand\imgsvg[2]{
       \begin{center}   \includegraphics[width=#2\paperwidth]{\imgsvgdir#1} \end{center}
}


\newcommand\Lien[2]{
     \href{#1}{#2 \tiny \faExternalLink}
}
\newcommand\mcLien[2]{
     \href{https~://www.maths-cours.fr/#1}{#2 \tiny \faExternalLink}
}

\newcommand{\euro}{\eurologo{}}

%================================================================================================================================
%
% Macros - Environement
%
%================================================================================================================================

\newenvironment{tex}{ %
}
{%
}

\newenvironment{indente}{ %
	\setlength\parindent{10mm}
}

{
	\setlength\parindent{0mm}
}

\newenvironment{corrige}{%
     \needspace{3\baselineskip}
     \medskip
     \textbf{\textsc{Corrigé}}
     \medskip
}
{
}

\newenvironment{extern}{%
     \begin{center}
     }
     {
     \end{center}
}

\NewEnviron{code}{%
	\par
     \boite{gray}{\texttt{%
     \BODY
     }}
     \par
}

\newenvironment{vbloc}{% boite sans cadre empeche saut de page
     \begin{minipage}[t]{\linewidth}
     }
     {
     \end{minipage}
}
\NewEnviron{h2}{%
    \needspace{3\baselineskip}
    \vspace{0.6cm}
	\noindent \MakeUppercase{\sffamily \large \BODY}
	\vspace{1mm}\textcolor{mcgris}{\hrule}\vspace{0.4cm}
	\par
}{}

\NewEnviron{h3}{%
    \needspace{3\baselineskip}
	\vspace{5mm}
	\textsc{\BODY}
	\par
}

\NewEnviron{margeneg}{ %
\begin{addmargin}[-1cm]{0cm}
\BODY
\end{addmargin}
}

\NewEnviron{html}{%
}

\begin{document}
\meta{url}{/cours/convexite/}
\meta{pid}{441}
\meta{titre}{Convexité}
\meta{type}{cours}
\begin{h2}I. Fonction convexe - Fonction concave\end{h2}
\cadre{bleu}{Définition}{% id="d10"
     Soient $f$ une fonction dérivable sur un intervalle $I$ et $\mathscr C_{f}$ sa courbe représentative.
     \begin{itemize}\item On dit que $f$ est \textbf{convexe} sur $I$ si la courbe $\mathscr C_{f}$ est \textbf{au-dessus} de toutes ses tangentes sur l'intervalle $I$.
          \item On dit que $f$ est \textbf{concave} sur $I$ si la courbe $\mathscr C_{f}$ est \textbf{au-dessous} de toutes ses tangentes sur l'intervalle $I$.
     \end{itemize}
}
\bloc{orange}{Exemples}{% id="e10"
     \begin{center}
          \begin{extern}%width="460" alt="fonction convexe"
               % -+-+-+ variables modifiables
               \resizebox{10cm}{!}{%
                    \def\xmin{-3.5}
                    \def\xmax{3.5}
                    \def\ymin{-2.8}
                    \def\ymax{3.5}
                    \def\xunit{2}  % unités en cm
                    \def\yunit{2}
                    \psset{xunit=\xunit,yunit=\yunit,algebraic=true}
                    \fontsize{15pt}{15pt}\selectfont
                    \begin{pspicture*}[linewidth=1pt](\xmin,\ymin)(\xmax,\ymax)
                         \psgrid[gridcolor=mcgris,subgriddiv=0](-4,-3)(4,4)
                         \psaxes[linewidth=0.75pt]{->}(0,0)(\xmin,\ymin)(\xmax,\ymax)
                         \rput[tr](-0.2,-0.2){$O$}
                         \rput[tl](2.1,3){$\color{blue} \mathscr{C}_f$}
                         \psplot[plotpoints=1000,linewidth=0.8pt,linecolor=blue]{\xmin}{\xmax}{2.71828^x-x-2}
                         \psplot[plotpoints=10,linewidth=0.8pt,linecolor=vert]{\xmin}{\xmax}{-1}
                         \psplot[plotpoints=10,linewidth=0.8pt,linecolor=vert]{\xmin}{\xmax}{1.71828*x-2}
                         \psplot[plotpoints=10,linewidth=0.8pt,linecolor=vert]{\xmin}{\xmax}{(1/2.71828-1)*(x+2)}
                    \end{pspicture*}
               }
          \end{extern}
     \end{center}
     \begin{center}
          \textit{Fonction convexe (et quelques tangentes...)}
     \end{center}
     \begin{center}
          \begin{extern}%width="460" alt="fonction concave"
               % -+-+-+ variables modifiables
               \resizebox{10cm}{!}{%
                    \def\xmin{-3.5}
                    \def\xmax{3.5}
                    \def\ymin{-2.8}
                    \def\ymax{3.5}
                    \def\xunit{2}  % unités en cm
                    \def\yunit{2}
                    \psset{xunit=\xunit,yunit=\yunit,algebraic=true}
                    \fontsize{15pt}{15pt}\selectfont
                    \begin{pspicture*}[linewidth=1pt](\xmin,\ymin)(\xmax,\ymax)
                         \psgrid[gridcolor=mcgris,subgriddiv=0](-4,-3)(4,4)
                         \psaxes[linewidth=0.75pt]{->}(0,0)(\xmin,\ymin)(\xmax,\ymax)
                         \rput[tr](-0.2,-0.2){$O$}
                         \rput[tl](2.3,-2){$\color{blue} \mathscr{C}_f$}
                         \psplot[plotpoints=1000,linewidth=0.8pt,linecolor=blue]{\xmin}{\xmax}{-2.71828^x+x+4}
                         \psplot[plotpoints=10,linewidth=0.8pt,linecolor=vert]{\xmin}{\xmax}{3}
                         \psplot[plotpoints=10,linewidth=0.8pt,linecolor=vert]{\xmin}{\xmax}{-1.71828*x+4}
                         \psplot[plotpoints=10,linewidth=0.8pt,linecolor=vert]{\xmin}{\xmax}{(-1/2.71828+1)*(x+2)+2}
                    \end{pspicture*}
               }
          \end{extern}
     \end{center}
     \begin{center}
          \textit{Fonction concave (et quelques tangentes...)}
     \end{center}
}
\cadre{rouge}{Théorème}{% id="t20"
     Si $f$ est dérivable sur $I$ :
     \begin{itemize}\item  $f$ est convexe sur $I$ si et seulement si \textbf{$f^{\prime}$} est \textbf{croissante} sur $I$
          \item  $f$ est concave sur $I$ si et seulement si \textbf{$f^{\prime}$} est \textbf{décroissante} sur $I$
     \end{itemize}
}
\bloc{cyan}{Remarque}{% id="r20"
     L'étude de la convexité se ramène donc à l'étude des variations de $f^{\prime}$. Si $f^{\prime}$ est dérivable, on donc est amené a étudier le signe la dérivée de $f^{\prime}$. Cette dérivée s'appelle la \textbf{dérivée seconde} de $f$ et se note $f^{\prime\prime}$.
}
\cadre{rouge}{Théorème}{% id="t30"
     Si $f$ est dérivable sur $I$ et si $f^{\prime}$ est dérivable sur $I$ (on dit aussi que $f$ est 2 fois dérivable
     \par
     sur $I$) :
     \begin{itemize}\item  $f$ est convexe sur $I$ si et seulement si \textbf{$f^{\prime\prime}$} est \textbf{positive ou nulle} sur $I$
          \item  $f$ est concave sur $I$ si et seulement si \textbf{$f^{\prime\prime}$} est \textbf{négative ou nulle} sur $I$
     \end{itemize}
}
\bloc{orange}{Exemples}{% id="e30"
     \begin{itemize}
          \item  La fonction $f :   x \mapsto  x^{2}$ est deux fois dérivable sur $\mathbb{R}$.
          \par
          $f^{\prime}\left(x\right)=2x$ et $f^{\prime\prime}\left(x\right)=2$.
          \par
          Comme $f^{\prime\prime}$ est positive sur $\mathbb{R}$, $f$ est convexe sur $\mathbb{R}$.
          \item  La fonction $f :   x \mapsto  x^{3}$ est deux fois dérivable sur $\mathbb{R}$.
          \par
          $f^{\prime}\left(x\right)=3x^{2}$ et $f^{\prime\prime}\left(x\right)=6x$.
          \par
          $f^{\prime\prime}\geqslant 0$  sur $\left[0; +\infty \right[$, donc $f$ est convexe sur $\left[0; +\infty \right[$.
          \par
          $f^{\prime\prime}\leqslant 0$  sur $\left]-\infty ; 0\right]$, donc $f$ est concave sur $\left]-\infty ; 0\right]$.
     \end{itemize}
}
\begin{h2}II. Point d'inflexion\end{h2}
\cadre{bleu}{Définition}{% id="d50"
     Soient $f$ une fonction dérivable sur un intervalle $I$, $\mathscr C_{f}$ sa courbe représentative et $A\left(a;f\left(a\right)\right)$ un point de la courbe  $\mathscr C_{f}$ .
     \par
     On dit que $A$ est un \textbf{point d'inflexion} de la courbe  $\mathscr C_{f}$, si et seulement si la courbe  $\mathscr C_{f}$  traverse sa tangente en $A$.
}
\bloc{orange}{Exemple}{% id="e50"
     \begin{center}
          \begin{extern}%width="460" alt="point d'inflexion"
               % -+-+-+ variables modifiables
               \resizebox{10cm}{!}{%
                    \def\xmin{-3.5}
                    \def\xmax{3.5}
                    \def\ymin{-2.8}
                    \def\ymax{3.5}
                    \def\xunit{2}  % unités en cm
                    \def\yunit{2}
                    \psset{xunit=\xunit,yunit=\yunit,algebraic=true}
                    \fontsize{15pt}{15pt}\selectfont
                    \begin{pspicture*}[linewidth=1pt](\xmin,\ymin)(\xmax,\ymax)
                         \psgrid[gridcolor=mcgris,subgriddiv=0](-4,-3)(4,4)
                         \psaxes[linewidth=0.75pt]{->}(0,0)(\xmin,\ymin)(\xmax,\ymax)
                         \rput[tr](-0.2,-0.2){$O$}
                         \rput[tl](2.9,2.9){$\color{blue} \mathscr{C}_f$}
                         \rput[bl](1.1,0.433){$\red A$}
                         \psdots[linecolor=red](1,0.333)
                         \psplot[plotpoints=1000,linewidth=0.8pt,linecolor=blue]{\xmin}{\xmax}{1/3*x^3-x^2+1}
                         \psplot[plotpoints=10,linewidth=0.8pt,linecolor=vert]{\xmin}{\xmax}{-x+4/3}
                    \end{pspicture*}
               }
          \end{extern}
     \end{center}
     \begin{center}
          \textit{Point d'inflexion en A}
     \end{center}
}
\cadre{vert}{Propriété}{% id="p60"
     Si $A$ est un point d'inflexion d'abscisse $a$, $ f $ passe de concave à convexe ou de convexe à concave en $a$.
}
\cadre{rouge}{Théorème}{% id="t70"
     Soit $ f $ une fonction deux fois dérivable sur un intervalle $I$ de courbe représentative $\mathscr C_{f}$. Le point $A$ d'abscisse $a$ est un point d'inflexion de $\mathscr C_{f} $ si et seulement si \textbf{$f^{\prime\prime}$ s'annule et change de signe en $a$}.
}
\bloc{orange}{Exemple}{% id="e70"
     Le graphique de l'exemple précédent correspond à la fonction définie par :
     \par
     $f\left(x\right)=\frac{1}{3}x^{3}-x^{2}+1$
     \par
     On a $f^{\prime}\left(x\right)=x^{2}-2x$ et $f^{\prime\prime}\left(x\right)=2x-2$.
     \par
     On vérifie bien que $f^{\prime\prime}$ change de signe en $1$. Donc le point $A$ d'abscisse $1$ et d'ordonnée $f\left(1\right)=\frac{1}{3}$ est bien un point d'inflexion.
}

\end{document}