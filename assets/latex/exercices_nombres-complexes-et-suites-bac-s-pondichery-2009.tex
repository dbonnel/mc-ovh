\documentclass[a4paper]{article}

%================================================================================================================================
%
% Packages
%
%================================================================================================================================

\usepackage[T1]{fontenc} 	% pour caractères accentués
\usepackage[utf8]{inputenc}  % encodage utf8
\usepackage[french]{babel}	% langue : français
\usepackage{fourier}			% caractères plus lisibles
\usepackage[dvipsnames]{xcolor} % couleurs
\usepackage{fancyhdr}		% réglage header footer
\usepackage{needspace}		% empêcher sauts de page mal placés
\usepackage{graphicx}		% pour inclure des graphiques
\usepackage{enumitem,cprotect}		% personnalise les listes d'items (nécessaire pour ol, al ...)
\usepackage{hyperref}		% Liens hypertexte
\usepackage{pstricks,pst-all,pst-node,pstricks-add,pst-math,pst-plot,pst-tree,pst-eucl} % pstricks
\usepackage[a4paper,includeheadfoot,top=2cm,left=3cm, bottom=2cm,right=3cm]{geometry} % marges etc.
\usepackage{comment}			% commentaires multilignes
\usepackage{amsmath,environ} % maths (matrices, etc.)
\usepackage{amssymb,makeidx}
\usepackage{bm}				% bold maths
\usepackage{tabularx}		% tableaux
\usepackage{colortbl}		% tableaux en couleur
\usepackage{fontawesome}		% Fontawesome
\usepackage{environ}			% environment with command
\usepackage{fp}				% calculs pour ps-tricks
\usepackage{multido}			% pour ps tricks
\usepackage[np]{numprint}	% formattage nombre
\usepackage{tikz,tkz-tab} 			% package principal TikZ
\usepackage{pgfplots}   % axes
\usepackage{mathrsfs}    % cursives
\usepackage{calc}			% calcul taille boites
\usepackage[scaled=0.875]{helvet} % font sans serif
\usepackage{svg} % svg
\usepackage{scrextend} % local margin
\usepackage{scratch} %scratch
\usepackage{multicol} % colonnes
%\usepackage{infix-RPN,pst-func} % formule en notation polanaise inversée
\usepackage{listings}

%================================================================================================================================
%
% Réglages de base
%
%================================================================================================================================

\lstset{
language=Python,   % R code
literate=
{á}{{\'a}}1
{à}{{\`a}}1
{ã}{{\~a}}1
{é}{{\'e}}1
{è}{{\`e}}1
{ê}{{\^e}}1
{í}{{\'i}}1
{ó}{{\'o}}1
{õ}{{\~o}}1
{ú}{{\'u}}1
{ü}{{\"u}}1
{ç}{{\c{c}}}1
{~}{{ }}1
}


\definecolor{codegreen}{rgb}{0,0.6,0}
\definecolor{codegray}{rgb}{0.5,0.5,0.5}
\definecolor{codepurple}{rgb}{0.58,0,0.82}
\definecolor{backcolour}{rgb}{0.95,0.95,0.92}

\lstdefinestyle{mystyle}{
    backgroundcolor=\color{backcolour},   
    commentstyle=\color{codegreen},
    keywordstyle=\color{magenta},
    numberstyle=\tiny\color{codegray},
    stringstyle=\color{codepurple},
    basicstyle=\ttfamily\footnotesize,
    breakatwhitespace=false,         
    breaklines=true,                 
    captionpos=b,                    
    keepspaces=true,                 
    numbers=left,                    
xleftmargin=2em,
framexleftmargin=2em,            
    showspaces=false,                
    showstringspaces=false,
    showtabs=false,                  
    tabsize=2,
    upquote=true
}

\lstset{style=mystyle}


\lstset{style=mystyle}
\newcommand{\imgdir}{C:/laragon/www/newmc/assets/imgsvg/}
\newcommand{\imgsvgdir}{C:/laragon/www/newmc/assets/imgsvg/}

\definecolor{mcgris}{RGB}{220, 220, 220}% ancien~; pour compatibilité
\definecolor{mcbleu}{RGB}{52, 152, 219}
\definecolor{mcvert}{RGB}{125, 194, 70}
\definecolor{mcmauve}{RGB}{154, 0, 215}
\definecolor{mcorange}{RGB}{255, 96, 0}
\definecolor{mcturquoise}{RGB}{0, 153, 153}
\definecolor{mcrouge}{RGB}{255, 0, 0}
\definecolor{mclightvert}{RGB}{205, 234, 190}

\definecolor{gris}{RGB}{220, 220, 220}
\definecolor{bleu}{RGB}{52, 152, 219}
\definecolor{vert}{RGB}{125, 194, 70}
\definecolor{mauve}{RGB}{154, 0, 215}
\definecolor{orange}{RGB}{255, 96, 0}
\definecolor{turquoise}{RGB}{0, 153, 153}
\definecolor{rouge}{RGB}{255, 0, 0}
\definecolor{lightvert}{RGB}{205, 234, 190}
\setitemize[0]{label=\color{lightvert}  $\bullet$}

\pagestyle{fancy}
\renewcommand{\headrulewidth}{0.2pt}
\fancyhead[L]{maths-cours.fr}
\fancyhead[R]{\thepage}
\renewcommand{\footrulewidth}{0.2pt}
\fancyfoot[C]{}

\newcolumntype{C}{>{\centering\arraybackslash}X}
\newcolumntype{s}{>{\hsize=.35\hsize\arraybackslash}X}

\setlength{\parindent}{0pt}		 
\setlength{\parskip}{3mm}
\setlength{\headheight}{1cm}

\def\ebook{ebook}
\def\book{book}
\def\web{web}
\def\type{web}

\newcommand{\vect}[1]{\overrightarrow{\,\mathstrut#1\,}}

\def\Oij{$\left(\text{O}~;~\vect{\imath},~\vect{\jmath}\right)$}
\def\Oijk{$\left(\text{O}~;~\vect{\imath},~\vect{\jmath},~\vect{k}\right)$}
\def\Ouv{$\left(\text{O}~;~\vect{u},~\vect{v}\right)$}

\hypersetup{breaklinks=true, colorlinks = true, linkcolor = OliveGreen, urlcolor = OliveGreen, citecolor = OliveGreen, pdfauthor={Didier BONNEL - https://www.maths-cours.fr} } % supprime les bordures autour des liens

\renewcommand{\arg}[0]{\text{arg}}

\everymath{\displaystyle}

%================================================================================================================================
%
% Macros - Commandes
%
%================================================================================================================================

\newcommand\meta[2]{    			% Utilisé pour créer le post HTML.
	\def\titre{titre}
	\def\url{url}
	\def\arg{#1}
	\ifx\titre\arg
		\newcommand\maintitle{#2}
		\fancyhead[L]{#2}
		{\Large\sffamily \MakeUppercase{#2}}
		\vspace{1mm}\textcolor{mcvert}{\hrule}
	\fi 
	\ifx\url\arg
		\fancyfoot[L]{\href{https://www.maths-cours.fr#2}{\black \footnotesize{https://www.maths-cours.fr#2}}}
	\fi 
}


\newcommand\TitreC[1]{    		% Titre centré
     \needspace{3\baselineskip}
     \begin{center}\textbf{#1}\end{center}
}

\newcommand\newpar{    		% paragraphe
     \par
}

\newcommand\nosp {    		% commande vide (pas d'espace)
}
\newcommand{\id}[1]{} %ignore

\newcommand\boite[2]{				% Boite simple sans titre
	\vspace{5mm}
	\setlength{\fboxrule}{0.2mm}
	\setlength{\fboxsep}{5mm}	
	\fcolorbox{#1}{#1!3}{\makebox[\linewidth-2\fboxrule-2\fboxsep]{
  		\begin{minipage}[t]{\linewidth-2\fboxrule-4\fboxsep}\setlength{\parskip}{3mm}
  			 #2
  		\end{minipage}
	}}
	\vspace{5mm}
}

\newcommand\CBox[4]{				% Boites
	\vspace{5mm}
	\setlength{\fboxrule}{0.2mm}
	\setlength{\fboxsep}{5mm}
	
	\fcolorbox{#1}{#1!3}{\makebox[\linewidth-2\fboxrule-2\fboxsep]{
		\begin{minipage}[t]{1cm}\setlength{\parskip}{3mm}
	  		\textcolor{#1}{\LARGE{#2}}    
 	 	\end{minipage}  
  		\begin{minipage}[t]{\linewidth-2\fboxrule-4\fboxsep}\setlength{\parskip}{3mm}
			\raisebox{1.2mm}{\normalsize\sffamily{\textcolor{#1}{#3}}}						
  			 #4
  		\end{minipage}
	}}
	\vspace{5mm}
}

\newcommand\cadre[3]{				% Boites convertible html
	\par
	\vspace{2mm}
	\setlength{\fboxrule}{0.1mm}
	\setlength{\fboxsep}{5mm}
	\fcolorbox{#1}{white}{\makebox[\linewidth-2\fboxrule-2\fboxsep]{
  		\begin{minipage}[t]{\linewidth-2\fboxrule-4\fboxsep}\setlength{\parskip}{3mm}
			\raisebox{-2.5mm}{\sffamily \small{\textcolor{#1}{\MakeUppercase{#2}}}}		
			\par		
  			 #3
 	 		\end{minipage}
	}}
		\vspace{2mm}
	\par
}

\newcommand\bloc[3]{				% Boites convertible html sans bordure
     \needspace{2\baselineskip}
     {\sffamily \small{\textcolor{#1}{\MakeUppercase{#2}}}}    
		\par		
  			 #3
		\par
}

\newcommand\CHelp[1]{
     \CBox{Plum}{\faInfoCircle}{À RETENIR}{#1}
}

\newcommand\CUp[1]{
     \CBox{NavyBlue}{\faThumbsOUp}{EN PRATIQUE}{#1}
}

\newcommand\CInfo[1]{
     \CBox{Sepia}{\faArrowCircleRight}{REMARQUE}{#1}
}

\newcommand\CRedac[1]{
     \CBox{PineGreen}{\faEdit}{BIEN R\'EDIGER}{#1}
}

\newcommand\CError[1]{
     \CBox{Red}{\faExclamationTriangle}{ATTENTION}{#1}
}

\newcommand\TitreExo[2]{
\needspace{4\baselineskip}
 {\sffamily\large EXERCICE #1\ (\emph{#2 points})}
\vspace{5mm}
}

\newcommand\img[2]{
          \includegraphics[width=#2\paperwidth]{\imgdir#1}
}

\newcommand\imgsvg[2]{
       \begin{center}   \includegraphics[width=#2\paperwidth]{\imgsvgdir#1} \end{center}
}


\newcommand\Lien[2]{
     \href{#1}{#2 \tiny \faExternalLink}
}
\newcommand\mcLien[2]{
     \href{https~://www.maths-cours.fr/#1}{#2 \tiny \faExternalLink}
}

\newcommand{\euro}{\eurologo{}}

%================================================================================================================================
%
% Macros - Environement
%
%================================================================================================================================

\newenvironment{tex}{ %
}
{%
}

\newenvironment{indente}{ %
	\setlength\parindent{10mm}
}

{
	\setlength\parindent{0mm}
}

\newenvironment{corrige}{%
     \needspace{3\baselineskip}
     \medskip
     \textbf{\textsc{Corrigé}}
     \medskip
}
{
}

\newenvironment{extern}{%
     \begin{center}
     }
     {
     \end{center}
}

\NewEnviron{code}{%
	\par
     \boite{gray}{\texttt{%
     \BODY
     }}
     \par
}

\newenvironment{vbloc}{% boite sans cadre empeche saut de page
     \begin{minipage}[t]{\linewidth}
     }
     {
     \end{minipage}
}
\NewEnviron{h2}{%
    \needspace{3\baselineskip}
    \vspace{0.6cm}
	\noindent \MakeUppercase{\sffamily \large \BODY}
	\vspace{1mm}\textcolor{mcgris}{\hrule}\vspace{0.4cm}
	\par
}{}

\NewEnviron{h3}{%
    \needspace{3\baselineskip}
	\vspace{5mm}
	\textsc{\BODY}
	\par
}

\NewEnviron{margeneg}{ %
\begin{addmargin}[-1cm]{0cm}
\BODY
\end{addmargin}
}

\NewEnviron{html}{%
}

\begin{document}
\meta{url}{/exercices/nombres-complexes-et-suites-bac-s-pondichery-2009/}
\meta{pid}{2350}
\meta{titre}{Nombres complexes et suites - Bac S Pondichéry 2009}
\meta{type}{exercices}
%
\begin{h2}Exercice 2\end{h2}
\textit{ 5 points - Candidats ayant suivi l'enseignement de spécialité}
\par
Le plan complexe est muni d'un repère orthonormal direct $\left(O; \vec{u}, \vec{v}\right)$. On prendra pour unité graphique 2 cm.
\par
Soit $A$ et $B$ les points d'affixes respectives $z_{A}=i$ et $z_{B}=1+2i$.
\begin{enumerate}
     \item
     Justifier qu'il existe une unique similitude directe $S$ telle que :
     \par
     $S\left(O\right)=A$ et $S\left(A\right)=B$.
     \item
     Montrer que l'écriture complexe de $S$ est:
     \par
     $z^{\prime}=\left(1-i\right)z+i$.
     \par
     Préciser les éléments caractéristiques de $S$ (on notera $\Omega $ le centre de $S$).
     \par
     On considère la suite de points $\left(A_{n}\right)$ telle que:
     \par
     •~~ $A_{0}$ est l'origine du repère et,
     \par
     •~~ pour tout entier naturel $n$, $A_{n+l}= S\left(A_{n}\right)$.
     \par
     On note $z_{n}$, l'affixe de $A_{n}$ (On a donc $A_{0}=O$, $A_{1}=A$ et $A_{2}=B$).
     \item
     \begin{enumerate}[label=\alph*.]
          \item
          Démontrer que, pour tout entier naturel $n$, $z_{n}=1-\left(1-i\right)^{n}$.
          \item
          Déterminer, en fonction de $n$, les affixes des vecteurs $\overrightarrow{\Omega  A_{n}}$ et $\overrightarrow{A_{n}A_{n+1}}$.
          \par
          Comparer les normes de ces vecteurs et calculer une mesure de l'angle $\left(\overrightarrow{\Omega  A_{n}},\overrightarrow{A_{n}A_{n+1}}\right)$.
          \item
          En déduire une construction du point $A_{n+1}$ connaissant le point $A_{n}$.
          \par
          Construire les points $A_{3}$ et $A_{4}$.
     \end{enumerate}
     \item
     Quels sont les points de la suite $\left(A_{n}\right)$ appartenant à la droite $\left(\Omega  B\right)$ ?
\end{enumerate}
\begin{corrige}
     \begin{enumerate}
          \item
          On sait que si $A\neq A^{\prime}$ et $B\neq B^{\prime}$, il existe une unique similitude directe $S$ transformant $A$ en $A^{\prime}$ et $B$ en $B^{\prime}$. Comme $O\neq A$ et $A\neq B$, il existe une unique similitude directe $S$ telle que :
          \par
          $S\left(O\right)=A$ et $S\left(A\right)=B$.
          \item
          L'écriture complexe de $S$ est de la forme :
          \par
          $z^{\prime}=az+b$
          \par
          Comme $S\left(O\right)=A$, $z_{A}=a\times 0+b$ donc $b=i$
          \par
          Comme $S\left(A\right)=B$, $z_{B}=az_{A}+i$ donc $1+2i=ai+i$ soit :
          \par
          $a=\frac{1+i}{i}=1-i$
          \par
          L'expression complexe de $S$ est donc :
          \par
          $z^{\prime}=\left(1-i\right)z+i$
          \par
          Le rapport de la similitude $S$ est :
          \par
          $R=|1-i|=\sqrt{2}$
          \par
          L'angle de la similitude $S$ est :
          \par
          $\theta =\text{arg}\left(1-i\right)=\text{arg}\left(\sqrt{2}\left[\frac{\sqrt{2}}{2}-\frac{\sqrt{2}}{2}i\right]\right)$
          \par
          $\theta =\text{arg}\left(\sqrt{2}\left[\cos\left(-\frac{\pi }{4}\right)+i\sin\left(-\frac{\pi }{4}\right)\right]\right)=-\frac{\pi }{4}\ \left[2\pi \right]$
          \par
          $\Omega \left(\omega \right)$ le centre de $S$ est le point invariant de $S$ donc :
          \par
          $\omega =\left(1-i\right)\omega +i$
          \par
          $i \omega =i$
          \par
          $\omega =1$
          \par
          $S$ est donc la similitude directe de centre $\Omega \left(1\right)$, de rapport $\sqrt{2}$ et d'angle $-\frac{\pi }{4}$
          \item
          \begin{enumerate}[label=\alph*.]
               \item
               Montrons par récurrence que pour tout entier naturel $n$, $z_{n}=1-\left(1-i\right)^{n}$.
\par
               \textbf{Initialisation}
\\
               Cette propriété est vraie pour $n=0$; en effet $z_{0}$ est l'affixe de $O$ dons $z_{0}=0$ et
               \par
               $1-\left(1-i\right)^{0}=1-1=0$
\par
               \textbf{Hérédité}
\\
               Supposons $z_{n}=1-\left(1-i\right)^{n}$ pour un certain entier $n$ fixé.
               \par
               $z_{n+1}=\left(1-i\right)z_{n}+i=\left(1-i\right)\left[1-\left(1-i\right)^{n}\right]+i$
               \par
               $z_{n+1}\equiv \left(1-i\right)-\left(1-i\right)\left(1-i\right)^{n}+i=1-\left(1-i\right)^{n+1}$
               \par
               ce qui prouve bien l'hérédité.
               \par
               Donc, pour tout entier naturel $n$, $z_{n}=1-\left(1-i\right)^{n}$.
               \item
               $z_{\overrightarrow{\Omega  A_{n}}}=z_{n}-\omega =1-\left(1-i\right)^{n}-1=-\left(1-i\right)^{n}$
               \par
               $z_{\overrightarrow{A_{n}A_{n+1}}}=1-\left(1-i\right)^{n+1}-\left[1-\left(1-i\right)^{n}\right]=\left(1-i\right)^{n}-\left(1-i\right)^{n+1}$
               \par
               $z_{\overrightarrow{A_{n}A_{n+1}}}=\left(1-i\right)^{n}\left[1-\left(1-i\right)\right]=i\left(1-i\right)^{n}$.
               \par
               $||\overrightarrow{\Omega  A_{n}}||=|-\left(1-i\right)^{n}|=|1-i|^{n}=\sqrt{2}^{n}$
               \par
               $||\overrightarrow{A_{n}A_{n+1}}||=|i\left(1-i\right)^{n}|=|1-i|^{n}=\sqrt{2}^{n}$
               \par
               donc $||\overrightarrow{\Omega  A_{n}}||=||\overrightarrow{A_{n}A_{n+1}|}$
               \par
               De plus
               \par
               $\left(\overrightarrow{\Omega  A_{n}}, \overrightarrow{A_{n}A_{n+1}}\right)=\text{arg}\left(\frac{i\left(1-i\right)^{n}}{-\left(1-i\right)^{n}}\right)=\text{arg}\left(-i\right)=-\frac{\pi }{2}\ \left[2\pi \right]$
               \item
               De la question précédente on déduit que :
               \par
               $\left(\overrightarrow{A_{n} \Omega }, \overrightarrow{A_{n}A_{n+1}}\right)=\frac{\pi }{2}\ \left[2\pi \right]$
               \par
               $A_{n+1}$ est l'image de $A_{n}$ par la rotation de centre $\Omega $ d'angle $\frac{\pi }{2}$ , c'est à dire que le triangle $A_{n} \Omega  A_{n+1}$ est rectangle isocèle en $A_{n}$ de sens direct.

\begin{center}
\imgsvg{Bac_S_Pondichery_spe_2009}{0.3}% alt="Nombres complexes et suites " style="width:40rem"
\end{center}
          \end{enumerate}
          \item
          $A_{n}$ appartient à la droite $\left(\Omega  B\right)$ si et seulement si :
          \par
          $\left(\overrightarrow{\Omega  A_{n}}, \overrightarrow{\Omega  B}\right)=0 \ \left[\pi \right]$
          \par
          c'est à dire comme $B=A_{2}$ si et seulement si :
          \par
          $\text{arg}\left(\frac{\left(1-i\right)^{n}}{\left(1-i\right)^{2}}\right)=0\ \left[\pi \right]$
          \par
          $\text{arg}\left(\left(1-i\right)^{n-2}\right)=0\ \left[\pi \right]$
          \par
          $\left(n-2\right)\times -\frac{\pi }{4}=0\ \left[\pi \right]$
          \par
          $\left(n-2\right)\times \frac{\pi }{4}=0\ \left[\pi \right]$
          \par
          $\left(n-2\right)\times \frac{\pi }{4}=k\pi $ $\left(k\in \mathbb{Z}\right)$
          \par
          $n-2=4k$
          \par
          $n=4k+2$
          \par
          Donc $\left(A_{n}\right)$ appartient à la droite $\left(\Omega  B\right)$ si et seulement si le reste de la division euclidienne de $n$ par 4 est 2 ( $ n\in \left\{2; 6; 10; 14; . . .\right\} $ )
     \end{enumerate}
\end{corrige}

\end{document}