\documentclass[a4paper]{article}

%================================================================================================================================
%
% Packages
%
%================================================================================================================================

\usepackage[T1]{fontenc} 	% pour caractères accentués
\usepackage[utf8]{inputenc}  % encodage utf8
\usepackage[french]{babel}	% langue : français
\usepackage{fourier}			% caractères plus lisibles
\usepackage[dvipsnames]{xcolor} % couleurs
\usepackage{fancyhdr}		% réglage header footer
\usepackage{needspace}		% empêcher sauts de page mal placés
\usepackage{graphicx}		% pour inclure des graphiques
\usepackage{enumitem,cprotect}		% personnalise les listes d'items (nécessaire pour ol, al ...)
\usepackage{hyperref}		% Liens hypertexte
\usepackage{pstricks,pst-all,pst-node,pstricks-add,pst-math,pst-plot,pst-tree,pst-eucl} % pstricks
\usepackage[a4paper,includeheadfoot,top=2cm,left=3cm, bottom=2cm,right=3cm]{geometry} % marges etc.
\usepackage{comment}			% commentaires multilignes
\usepackage{amsmath,environ} % maths (matrices, etc.)
\usepackage{amssymb,makeidx}
\usepackage{bm}				% bold maths
\usepackage{tabularx}		% tableaux
\usepackage{colortbl}		% tableaux en couleur
\usepackage{fontawesome}		% Fontawesome
\usepackage{environ}			% environment with command
\usepackage{fp}				% calculs pour ps-tricks
\usepackage{multido}			% pour ps tricks
\usepackage[np]{numprint}	% formattage nombre
\usepackage{tikz,tkz-tab} 			% package principal TikZ
\usepackage{pgfplots}   % axes
\usepackage{mathrsfs}    % cursives
\usepackage{calc}			% calcul taille boites
\usepackage[scaled=0.875]{helvet} % font sans serif
\usepackage{svg} % svg
\usepackage{scrextend} % local margin
\usepackage{scratch} %scratch
\usepackage{multicol} % colonnes
%\usepackage{infix-RPN,pst-func} % formule en notation polanaise inversée
\usepackage{listings}

%================================================================================================================================
%
% Réglages de base
%
%================================================================================================================================

\lstset{
language=Python,   % R code
literate=
{á}{{\'a}}1
{à}{{\`a}}1
{ã}{{\~a}}1
{é}{{\'e}}1
{è}{{\`e}}1
{ê}{{\^e}}1
{í}{{\'i}}1
{ó}{{\'o}}1
{õ}{{\~o}}1
{ú}{{\'u}}1
{ü}{{\"u}}1
{ç}{{\c{c}}}1
{~}{{ }}1
}


\definecolor{codegreen}{rgb}{0,0.6,0}
\definecolor{codegray}{rgb}{0.5,0.5,0.5}
\definecolor{codepurple}{rgb}{0.58,0,0.82}
\definecolor{backcolour}{rgb}{0.95,0.95,0.92}

\lstdefinestyle{mystyle}{
    backgroundcolor=\color{backcolour},   
    commentstyle=\color{codegreen},
    keywordstyle=\color{magenta},
    numberstyle=\tiny\color{codegray},
    stringstyle=\color{codepurple},
    basicstyle=\ttfamily\footnotesize,
    breakatwhitespace=false,         
    breaklines=true,                 
    captionpos=b,                    
    keepspaces=true,                 
    numbers=left,                    
xleftmargin=2em,
framexleftmargin=2em,            
    showspaces=false,                
    showstringspaces=false,
    showtabs=false,                  
    tabsize=2,
    upquote=true
}

\lstset{style=mystyle}


\lstset{style=mystyle}
\newcommand{\imgdir}{C:/laragon/www/newmc/assets/imgsvg/}
\newcommand{\imgsvgdir}{C:/laragon/www/newmc/assets/imgsvg/}

\definecolor{mcgris}{RGB}{220, 220, 220}% ancien~; pour compatibilité
\definecolor{mcbleu}{RGB}{52, 152, 219}
\definecolor{mcvert}{RGB}{125, 194, 70}
\definecolor{mcmauve}{RGB}{154, 0, 215}
\definecolor{mcorange}{RGB}{255, 96, 0}
\definecolor{mcturquoise}{RGB}{0, 153, 153}
\definecolor{mcrouge}{RGB}{255, 0, 0}
\definecolor{mclightvert}{RGB}{205, 234, 190}

\definecolor{gris}{RGB}{220, 220, 220}
\definecolor{bleu}{RGB}{52, 152, 219}
\definecolor{vert}{RGB}{125, 194, 70}
\definecolor{mauve}{RGB}{154, 0, 215}
\definecolor{orange}{RGB}{255, 96, 0}
\definecolor{turquoise}{RGB}{0, 153, 153}
\definecolor{rouge}{RGB}{255, 0, 0}
\definecolor{lightvert}{RGB}{205, 234, 190}
\setitemize[0]{label=\color{lightvert}  $\bullet$}

\pagestyle{fancy}
\renewcommand{\headrulewidth}{0.2pt}
\fancyhead[L]{maths-cours.fr}
\fancyhead[R]{\thepage}
\renewcommand{\footrulewidth}{0.2pt}
\fancyfoot[C]{}

\newcolumntype{C}{>{\centering\arraybackslash}X}
\newcolumntype{s}{>{\hsize=.35\hsize\arraybackslash}X}

\setlength{\parindent}{0pt}		 
\setlength{\parskip}{3mm}
\setlength{\headheight}{1cm}

\def\ebook{ebook}
\def\book{book}
\def\web{web}
\def\type{web}

\newcommand{\vect}[1]{\overrightarrow{\,\mathstrut#1\,}}

\def\Oij{$\left(\text{O}~;~\vect{\imath},~\vect{\jmath}\right)$}
\def\Oijk{$\left(\text{O}~;~\vect{\imath},~\vect{\jmath},~\vect{k}\right)$}
\def\Ouv{$\left(\text{O}~;~\vect{u},~\vect{v}\right)$}

\hypersetup{breaklinks=true, colorlinks = true, linkcolor = OliveGreen, urlcolor = OliveGreen, citecolor = OliveGreen, pdfauthor={Didier BONNEL - https://www.maths-cours.fr} } % supprime les bordures autour des liens

\renewcommand{\arg}[0]{\text{arg}}

\everymath{\displaystyle}

%================================================================================================================================
%
% Macros - Commandes
%
%================================================================================================================================

\newcommand\meta[2]{    			% Utilisé pour créer le post HTML.
	\def\titre{titre}
	\def\url{url}
	\def\arg{#1}
	\ifx\titre\arg
		\newcommand\maintitle{#2}
		\fancyhead[L]{#2}
		{\Large\sffamily \MakeUppercase{#2}}
		\vspace{1mm}\textcolor{mcvert}{\hrule}
	\fi 
	\ifx\url\arg
		\fancyfoot[L]{\href{https://www.maths-cours.fr#2}{\black \footnotesize{https://www.maths-cours.fr#2}}}
	\fi 
}


\newcommand\TitreC[1]{    		% Titre centré
     \needspace{3\baselineskip}
     \begin{center}\textbf{#1}\end{center}
}

\newcommand\newpar{    		% paragraphe
     \par
}

\newcommand\nosp {    		% commande vide (pas d'espace)
}
\newcommand{\id}[1]{} %ignore

\newcommand\boite[2]{				% Boite simple sans titre
	\vspace{5mm}
	\setlength{\fboxrule}{0.2mm}
	\setlength{\fboxsep}{5mm}	
	\fcolorbox{#1}{#1!3}{\makebox[\linewidth-2\fboxrule-2\fboxsep]{
  		\begin{minipage}[t]{\linewidth-2\fboxrule-4\fboxsep}\setlength{\parskip}{3mm}
  			 #2
  		\end{minipage}
	}}
	\vspace{5mm}
}

\newcommand\CBox[4]{				% Boites
	\vspace{5mm}
	\setlength{\fboxrule}{0.2mm}
	\setlength{\fboxsep}{5mm}
	
	\fcolorbox{#1}{#1!3}{\makebox[\linewidth-2\fboxrule-2\fboxsep]{
		\begin{minipage}[t]{1cm}\setlength{\parskip}{3mm}
	  		\textcolor{#1}{\LARGE{#2}}    
 	 	\end{minipage}  
  		\begin{minipage}[t]{\linewidth-2\fboxrule-4\fboxsep}\setlength{\parskip}{3mm}
			\raisebox{1.2mm}{\normalsize\sffamily{\textcolor{#1}{#3}}}						
  			 #4
  		\end{minipage}
	}}
	\vspace{5mm}
}

\newcommand\cadre[3]{				% Boites convertible html
	\par
	\vspace{2mm}
	\setlength{\fboxrule}{0.1mm}
	\setlength{\fboxsep}{5mm}
	\fcolorbox{#1}{white}{\makebox[\linewidth-2\fboxrule-2\fboxsep]{
  		\begin{minipage}[t]{\linewidth-2\fboxrule-4\fboxsep}\setlength{\parskip}{3mm}
			\raisebox{-2.5mm}{\sffamily \small{\textcolor{#1}{\MakeUppercase{#2}}}}		
			\par		
  			 #3
 	 		\end{minipage}
	}}
		\vspace{2mm}
	\par
}

\newcommand\bloc[3]{				% Boites convertible html sans bordure
     \needspace{2\baselineskip}
     {\sffamily \small{\textcolor{#1}{\MakeUppercase{#2}}}}    
		\par		
  			 #3
		\par
}

\newcommand\CHelp[1]{
     \CBox{Plum}{\faInfoCircle}{À RETENIR}{#1}
}

\newcommand\CUp[1]{
     \CBox{NavyBlue}{\faThumbsOUp}{EN PRATIQUE}{#1}
}

\newcommand\CInfo[1]{
     \CBox{Sepia}{\faArrowCircleRight}{REMARQUE}{#1}
}

\newcommand\CRedac[1]{
     \CBox{PineGreen}{\faEdit}{BIEN R\'EDIGER}{#1}
}

\newcommand\CError[1]{
     \CBox{Red}{\faExclamationTriangle}{ATTENTION}{#1}
}

\newcommand\TitreExo[2]{
\needspace{4\baselineskip}
 {\sffamily\large EXERCICE #1\ (\emph{#2 points})}
\vspace{5mm}
}

\newcommand\img[2]{
          \includegraphics[width=#2\paperwidth]{\imgdir#1}
}

\newcommand\imgsvg[2]{
       \begin{center}   \includegraphics[width=#2\paperwidth]{\imgsvgdir#1} \end{center}
}


\newcommand\Lien[2]{
     \href{#1}{#2 \tiny \faExternalLink}
}
\newcommand\mcLien[2]{
     \href{https~://www.maths-cours.fr/#1}{#2 \tiny \faExternalLink}
}

\newcommand{\euro}{\eurologo{}}

%================================================================================================================================
%
% Macros - Environement
%
%================================================================================================================================

\newenvironment{tex}{ %
}
{%
}

\newenvironment{indente}{ %
	\setlength\parindent{10mm}
}

{
	\setlength\parindent{0mm}
}

\newenvironment{corrige}{%
     \needspace{3\baselineskip}
     \medskip
     \textbf{\textsc{Corrigé}}
     \medskip
}
{
}

\newenvironment{extern}{%
     \begin{center}
     }
     {
     \end{center}
}

\NewEnviron{code}{%
	\par
     \boite{gray}{\texttt{%
     \BODY
     }}
     \par
}

\newenvironment{vbloc}{% boite sans cadre empeche saut de page
     \begin{minipage}[t]{\linewidth}
     }
     {
     \end{minipage}
}
\NewEnviron{h2}{%
    \needspace{3\baselineskip}
    \vspace{0.6cm}
	\noindent \MakeUppercase{\sffamily \large \BODY}
	\vspace{1mm}\textcolor{mcgris}{\hrule}\vspace{0.4cm}
	\par
}{}

\NewEnviron{h3}{%
    \needspace{3\baselineskip}
	\vspace{5mm}
	\textsc{\BODY}
	\par
}

\NewEnviron{margeneg}{ %
\begin{addmargin}[-1cm]{0cm}
\BODY
\end{addmargin}
}

\NewEnviron{html}{%
}

\begin{document}
\meta{url}{/cours/variables-aleatoires-continues/}
\meta{pid}{488}
\meta{titre}{Variables aléatoires continues}
\meta{type}{cours}
\cadre{vert}{Introduction}{%id="i10"
     Il arrive qu'une variable aléatoire puisse prendre n'importe quelle valeur sur $\mathbb{R}$ ou sur un intervalle $I$ de $\mathbb{R}$. On parle alors de \textbf{variable aléatoire continue}.
     \par
     Pour une telle variable, les événements qui vont nous intéresser ne sont plus $(X=5)$, $(X=20)$, etc... , mais $(X \leqslant 5)$, $(5 \leqslant X \leqslant 20)$, etc...
}
\begin{h2}1. Généralités\end{h2}
\cadre{bleu}{Définition}{%id="d10"
     Soit $f$ une fonction \textbf{continue} et \textbf{positive} sur un intervalle $I=\left[a;b\right]$ telle que
     \[ \int_{a}^{b}f\left(x\right)dx=1. \]
     On dit que $X$ est une \textbf{variable aléatoire réelle continue de densité} $f$ si et seulement si pour tout $x_{1} \in I$ et tout $x_{2} \in I $ ($x_{1}\leqslant x_{2}$) :
     \begin{center}$p\left(x_{1}\leqslant X\leqslant x_{2}\right)=\int_{x_{1}}^{x_{2}}f\left(x\right)dx$\end{center}
}
\bloc{orange}{Exemple}{%id="e10"
     La fonction $f$ définie sur $I=\left[0;2\right]$ par $f\left(x\right)=\frac{x}{2}$ est une fonction continue et positive sur $I$.
     \par
     La fonction $F : x \longmapsto \dfrac{x^2}{4}$ est une primitive de $f$ sur $I$, par conséquent :
     \par
     $\int_{0}^{2}f\left(x\right)dx=\left[\frac{x^{2}}{4}\right]_{0}^{2}=1$.
     \par
     $f$ est donc une \textbf{densité de probabilité}.
     \par
     Soit X une variable aléatoire réelle à valeurs dans $I$ de densité $f$, on a alors, par exemple :
     \par
     $P\left(1\leqslant X\leqslant 1,5\right)=\int_{1}^{1,5}f\left(x\right)dx$.
     \par
     $P\left(1\leqslant X\leqslant 1,5\right)$ est donc l'aire (en u.a.) colorée ci-dessous :
     \begin{center}
          \begin{extern} %width="250" alt="densité linéaire"
               \begin{pspicture}(0,-0.5)(5,3)
                    \psset{xunit=2 cm, yunit=2 cm, algebraic=true}
                    %\psgrid[gridcolor=mcgris, subgriddiv=0, gridlabels=0pt](0,0)(2,1)
                    \psaxes{->}(0,0)(0,0)(2.5,1.5)
                    \psplot[linecolor=red,linewidth=0.75pt]{0}{2}{x/2}
                    \pscustom[linecolor=mcvert,linewidth=0.75pt,fillstyle=solid,fillcolor=mcvert,opacity=0.1]{
                         \psplot{1}{1.5}{x/2}
                         \psline(1.5,0)(1,0)(1,0.5)
                    }
               \end{pspicture}
          \end{extern}
     \end{center}
     Un calcul simple montre que $P\left(1\leqslant X\leqslant 1,5\right)=\left[\frac{x^{2}}{4}\right]_{1}^{1,5}=0,3125$.
}
\bloc{mauve}{Remarques}{%id="r10"
     \begin{itemize}
          \item On peut étendre cette définition aux cas où l'une ou les deux bornes $a$ et $b$ sont infinies.
          \par
          Dans ce cas, on remplace la condition $\int_{a}^{ b}f\left(x\right)dx=1$ par une condition portant sur une limite; par exemple si $b$ vaut $+\infty $, la condition $\int_{a}^{ b}f\left(x\right)dx=1$ deviendra $\lim\limits_{y\rightarrow +\infty }\int_{a}^{ y}f\left(x\right)dx=1$
          \item Comme indiqué en introduction, les événements du type $\left(X=k\right)$ ne sont pas intéressants car pour tout $k$ appartenant à $I$, $p\left(X=k\right)=\int_{k}^{ k}f\left(x\right)dx=0$.
          \item On peut employer indifféremment des inégalités larges ou strictes :
          \begin{center}$p\left(x_{1} < X < x_{2}\right)=p\left(x_{1}\leqslant X\leqslant x_{2}\right)$.\end{center}
     \end{itemize}
}
\cadre{bleu}{Définition}{%id="d20"
     L'espérance mathématique d'une variable aléatoire $X$ qui suit une loi de densité $f$ sur $\left[a;b\right]$ est le réel noté $E\left(X\right)$ défini par :
     \begin{center}$E\left(X\right)=\int_{a}^{b}xf\left(x\right)dx$.\end{center}
}
\bloc{orange}{Exemple}{%id="e20"
     Si l'on reprend l'exemple de la fonction $f$ définie sur $I=\left[0;2\right]$ par $f\left(x\right)=\frac{x}{2}$, l'espérance mathématique est :
     \par
     $E\left(X\right)=\int_{0}^{2}xf\left(x\right)dx$\nosp$=\int_{0}^{2}\frac{x^{2}}{2}dx$\nosp$=\left[\frac{x^{3}}{6}\right]_{0}^{2}$\nosp$=\frac{8}{6}=\frac{4}{3}$.
}
\begin{h2}2. Loi uniforme sur un intervalle\end{h2}
\cadre{bleu}{Définition}{%id="d40"
     On dit qu'une variable aléatoire $X$ suit la \textbf{loi uniforme} sur l'intervalle $\left[a~;~b\right]$ si sa densité de probabilité $f$ est constante sur $\left[a~;~b\right]$.
     \par
     Cette densité vaut alors, pour tout réel $x \in [a~;~b]$ :
     \[ f\left(x\right)=\frac{1}{b-a}. \]
}
\bloc{orange}{Exemple}{%id="e40"
     La densité de la loi uniforme sur l'intervalle $\left[0, 2\right]$ est représentée ci-dessous~:
     \begin{center}
          \begin{extern} %width="250" alt="loi uniforme"
               \begin{pspicture}(0,-0.5)(5,3)
                    \psset{xunit=2 cm, yunit=2 cm, algebraic=true}
                    %\psgrid[gridcolor=mcgris, subgriddiv=0, gridlabels=0pt](0,0)(2,1)
                    \psaxes{->}(0,0)(0,0)(2.5,1.5)
                    \psplot[linecolor=red,linewidth=0.75pt]{0}{2}{1/2}
               \end{pspicture}
          \end{extern}
     \end{center}
     \begin{center}\textit{Densité de la loi uniforme sur l'intervalle $\left[0, 2\right]$}\end{center}
}
\bloc{mauve}{Remarque}{%id="r40"
     Une primitive de la fonction $x \longmapsto \dfrac{1}{b-a}$ sur $[a~;~b]$ est  $x \longmapsto \dfrac{x}{b-a}$.
     \par
     On vérifie alors que :
     $\int_{a}^{b} \frac{1}{b-a} dx=\left[\frac{x}{b-a}\right]_{a}^{b}=1$.
}
\cadre{vert}{Propriété}{%id="p50"
     Si $X$ suit une \textbf{loi uniforme} sur $\left[a;b\right]$, alors pour tous réels $c$ et $d$ compris entre $a$ et $b$ avec $c < d$ :
     \begin{center}$p\left(c\leqslant X\leqslant d\right) = \frac{d-c}{b-a}$.\end{center}
}
\bloc{orange}{Démonstration}{%id="r50"
     En effet, si $a\leqslant c < d \leqslant b$ alors :
     \par
     $p\left(c \leqslant X\leqslant d \right)=\int_{c}^{d}\frac{1}{b-a}dx$\nosp$=\frac{d-c}{b-a}$
}
\cadre{rouge}{Théorème}{%id="t60"
     L'espérance mathématique d'une variable aléatoire $X$ qui suit une \textbf{loi uniforme} sur $\left[a;b\right]$ est :
     \begin{center}$E\left(X\right)=\frac{a+b}{2}$.\end{center}
}
\bloc{orange}{Démonstration}{%"m65"
     La fonction $x \longmapsto \dfrac{x^2}{2(b-a)}$ est une primitive de la fonction $x \longmapsto \dfrac{x}{b-a}$ sur $[a~;~b]$ ; par conséquent :
     \par
     $E\left(X\right) =\int_{a}^{ b}\frac{x}{b-a}dx$\\
     $\phantom{E\left(X\right)} =\left[\frac{x^{2}}{2\left(b-a\right)}\right]_{a}^{b}$\\
     $\phantom{E\left(X\right)}=\frac{b^{2}-a^{2}}{2\left(b-a\right)}$\\
     $\phantom{E\left(X\right)}=\frac{\left(b-a\right)\left(b+a\right)}{2\left(b-a\right)}$\\
     $\phantom{E\left(X\right)}=\frac{a+b}{2}$.
}
\begin{h2}3. Loi exponentielle de paramètre lambda\end{h2}
\cadre{bleu}{Définition}{%id="d70"
     On dit qu'une variable aléatoire X suit une \textbf{loi exponentielle de paramètre $\lambda  > 0$} sur $\left[0;+\infty \right[$ si sa densité de probabilité $f$ est définie sur $\left[0;+\infty \right[$ par :
     \begin{center}
          $f\left(x\right)=\lambda \text{e}^{-\lambda x}$.
     \end{center}
}
\bloc{orange}{Exemple}{%id="e70"
     La densité de la loi exponentielle de paramètre $\lambda =1,5$ est la fonction $f$ définie sur $\left[0;+\infty \right[$ par $f\left(x\right)=1,5 \text{e}^{-1,5 x}$.
     \par
     Cette fonction est représentée ci-dessous :
     \begin{center}
          \begin{extern} %width="350" alt="loi exponentielle"
               % -+-+-+ variables modifiables
               \def\e{2.7182818}
               \def\fonction{1.5*\e^(-1.5*x)  }
               \def\xmin{0}
               \def\xmax{3.5}
               \def\ymin{0}
               \def\ymax{2}
               \def\xunit{2}  % unités en cm
               \def\yunit{2}
               \begin{pspicture}(0,-0.5)(7,4.5)
                    \psset{xunit=\xunit, yunit=\yunit, algebraic=true}
                    %\psgrid[gridcolor=mcgris, subgriddiv=0, gridlabels=0pt](0,0)(2,1)
                    \psaxes{->}(0,0)(\xmin,\ymin)(\xmax,\ymax)
                    \psclip{
                         \psframe[linestyle=none](\xmin,\ymin)(\xmax,\ymax)
                    }
                    \psplot[linecolor=red,linewidth=0.75pt]{\xmin}{\xmax}{\fonction}
                    \endpsclip
                    \uput[l](0,1.5){$\red \lambda$}
               \end{pspicture}
          \end{extern}
     \end{center}
}
\bloc{mauve}{Remarque}{%id="r70"
     La fonction $x \longmapsto -\text{e}^{-\lambda x}$ est une primitive de la fonction $x \longmapsto \lambda \text{e}^{-\lambda x}$.
     \par
     On vérifie alors que :
     \par
     $\int_{0}^{+\infty } \lambda \text{e}^{-\lambda x} dx=\lim\limits_{t\rightarrow +\infty }\int_{0}^{t} \lambda \text{e}^{-\lambda x} dx$\\
     $\phantom{\int_{0}^{+\infty } \lambda \text{e}^{-\lambda x} dx}=\lim\limits_{t\rightarrow +\infty }\left[-\text{e}^{-\lambda x}\right]_{0}^{t}$\\
     $\phantom{\int_{0}^{+\infty } \lambda \text{e}^{-\lambda x} dx}=\lim\limits_{t\rightarrow +\infty }-\text{e}^{-\lambda t}+1=1$.
}
\cadre{vert}{Propriété}{%id="p80"
     Si $X$ suit une exponentielle de paramètre $\lambda $ sur $\left[0;+\infty \right[$, alors pour tous réels positifs $x_{1}$ et $x_{2}$ :
     \begin{itemize}
          \item %
          $p\left(x_{1}\leqslant X\leqslant x_{2}\right) = \text{e}^{-\lambda x_{1}}-\text{e}^{-\lambda x_{2}}$
          \item %
          $p\left(X\geqslant x_{1}\right) = \text{e}^{-\lambda x_{1}}$.
     \end{itemize}
}
\bloc{orange}{Démonstration}{%id="d80"
     $p\left(x_{1}\leqslant X\leqslant x_{2}\right)=\int_{x_{1}}^{x_{2}}\lambda \text{e}^{-\lambda x} dx$\\
     $\phantom{p\left(x_{1}\leqslant X\leqslant x_{2}\right)}=\left[-\text{e}^{-\lambda x}\right]_{x_{1}}^{x_{2}}$\\
     $\phantom{p\left(x_{1}\leqslant X\leqslant x_{2}\right)}=\text{e}^{-\lambda x_{1}}- \text{e}^{-\lambda x_{2}}$
     \par
     La seconde égalité s'obtient alors en faisant tendre $x_{2}$ vers $+\infty $.
}
\cadre{rouge}{Théorème}{%id="t90"
     L'espérance mathématique d'une variable aléatoire $X$ qui suit une \textbf{loi exponentielle} de paramètre $\lambda $ est :
     \begin{center}$E\left(X\right)=\frac{1}{\lambda}$\end{center}
}
\bloc{orange}{Démonstration}{%id="r95"
     Voir exercice : \Lien{https://www.maths-cours.fr/exercices/roc-esperance-mathematique-dune-loi-exponentielle/}{[ROC] Espérance mathématique d'une loi exponentielle}.
}
\cadre{vert}{Propriété}{%id="p100"
     Soient $X$ une variable aléatoire qui suit une exponentielle de paramètre $\lambda $ et $x$ et $x_{0}$ deux réels, alors :
     \begin{center}$p\left(X > x\right) = p_{(X > x_{0})}\left(X > x+x_{0}\right)$\end{center}
     On dit qu'une loi exponentielle est \og \textit{sans vieillissement} \fg{}.
}
\bloc{mauve}{Commentaire}{%id="r110"
     Tout d'abord, rappelons que la notation $p_{(X > x_{0})}\left(X > x+x_{0}\right)$ indique la probabilité (conditionnelle) de l'événement $\left(X > x+x_{0}\right)$ \textbf{sachant que} l'événement $(X > x_{0})$ est réalisé.
     \par
     Supposons que $X$ modélise la durée de vie d'une machine.
     \par
     \begin{itemize}
          \item %
          $p\left(X > x\right)$ correspond à la probabilité qu'une machine \og neuve \fg{} fonctionne pendant une durée supérieure ou égale à $x$ ;
          \item %
          $p_{(X > x_{0})}\left(X > x+x_{0}\right)$ est la probabilité qu'une machine, qui a déjà fonctionné pendant une durée $x_0$, fonctionne encore pendant une durée supérieure ou égale à $x$.
     \end{itemize}
     Dans le cadre d'une loi exponentielle, ces probabilités sont égales ce qui explique l'expression \og \textit{sans vieillissement} \fg{}.
}
\bloc{orange}{Démonstration}{%id="m120"
     Voir exercice : \Lien{https://www.maths-cours.fr/exercices/loi-exponentielle-bac-s-metropole-2008/}{Loi exponentielle - Bac S Métropole 2008}.
}

\end{document}