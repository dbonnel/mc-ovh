\documentclass[a4paper]{article}

%================================================================================================================================
%
% Packages
%
%================================================================================================================================

\usepackage[T1]{fontenc} 	% pour caractères accentués
\usepackage[utf8]{inputenc}  % encodage utf8
\usepackage[french]{babel}	% langue : français
\usepackage{fourier}			% caractères plus lisibles
\usepackage[dvipsnames]{xcolor} % couleurs
\usepackage{fancyhdr}		% réglage header footer
\usepackage{needspace}		% empêcher sauts de page mal placés
\usepackage{graphicx}		% pour inclure des graphiques
\usepackage{enumitem,cprotect}		% personnalise les listes d'items (nécessaire pour ol, al ...)
\usepackage{hyperref}		% Liens hypertexte
\usepackage{pstricks,pst-all,pst-node,pstricks-add,pst-math,pst-plot,pst-tree,pst-eucl} % pstricks
\usepackage[a4paper,includeheadfoot,top=2cm,left=3cm, bottom=2cm,right=3cm]{geometry} % marges etc.
\usepackage{comment}			% commentaires multilignes
\usepackage{amsmath,environ} % maths (matrices, etc.)
\usepackage{amssymb,makeidx}
\usepackage{bm}				% bold maths
\usepackage{tabularx}		% tableaux
\usepackage{colortbl}		% tableaux en couleur
\usepackage{fontawesome}		% Fontawesome
\usepackage{environ}			% environment with command
\usepackage{fp}				% calculs pour ps-tricks
\usepackage{multido}			% pour ps tricks
\usepackage[np]{numprint}	% formattage nombre
\usepackage{tikz,tkz-tab} 			% package principal TikZ
\usepackage{pgfplots}   % axes
\usepackage{mathrsfs}    % cursives
\usepackage{calc}			% calcul taille boites
\usepackage[scaled=0.875]{helvet} % font sans serif
\usepackage{svg} % svg
\usepackage{scrextend} % local margin
\usepackage{scratch} %scratch
\usepackage{multicol} % colonnes
%\usepackage{infix-RPN,pst-func} % formule en notation polanaise inversée
\usepackage{listings}

%================================================================================================================================
%
% Réglages de base
%
%================================================================================================================================

\lstset{
language=Python,   % R code
literate=
{á}{{\'a}}1
{à}{{\`a}}1
{ã}{{\~a}}1
{é}{{\'e}}1
{è}{{\`e}}1
{ê}{{\^e}}1
{í}{{\'i}}1
{ó}{{\'o}}1
{õ}{{\~o}}1
{ú}{{\'u}}1
{ü}{{\"u}}1
{ç}{{\c{c}}}1
{~}{{ }}1
}


\definecolor{codegreen}{rgb}{0,0.6,0}
\definecolor{codegray}{rgb}{0.5,0.5,0.5}
\definecolor{codepurple}{rgb}{0.58,0,0.82}
\definecolor{backcolour}{rgb}{0.95,0.95,0.92}

\lstdefinestyle{mystyle}{
    backgroundcolor=\color{backcolour},   
    commentstyle=\color{codegreen},
    keywordstyle=\color{magenta},
    numberstyle=\tiny\color{codegray},
    stringstyle=\color{codepurple},
    basicstyle=\ttfamily\footnotesize,
    breakatwhitespace=false,         
    breaklines=true,                 
    captionpos=b,                    
    keepspaces=true,                 
    numbers=left,                    
xleftmargin=2em,
framexleftmargin=2em,            
    showspaces=false,                
    showstringspaces=false,
    showtabs=false,                  
    tabsize=2,
    upquote=true
}

\lstset{style=mystyle}


\lstset{style=mystyle}
\newcommand{\imgdir}{C:/laragon/www/newmc/assets/imgsvg/}
\newcommand{\imgsvgdir}{C:/laragon/www/newmc/assets/imgsvg/}

\definecolor{mcgris}{RGB}{220, 220, 220}% ancien~; pour compatibilité
\definecolor{mcbleu}{RGB}{52, 152, 219}
\definecolor{mcvert}{RGB}{125, 194, 70}
\definecolor{mcmauve}{RGB}{154, 0, 215}
\definecolor{mcorange}{RGB}{255, 96, 0}
\definecolor{mcturquoise}{RGB}{0, 153, 153}
\definecolor{mcrouge}{RGB}{255, 0, 0}
\definecolor{mclightvert}{RGB}{205, 234, 190}

\definecolor{gris}{RGB}{220, 220, 220}
\definecolor{bleu}{RGB}{52, 152, 219}
\definecolor{vert}{RGB}{125, 194, 70}
\definecolor{mauve}{RGB}{154, 0, 215}
\definecolor{orange}{RGB}{255, 96, 0}
\definecolor{turquoise}{RGB}{0, 153, 153}
\definecolor{rouge}{RGB}{255, 0, 0}
\definecolor{lightvert}{RGB}{205, 234, 190}
\setitemize[0]{label=\color{lightvert}  $\bullet$}

\pagestyle{fancy}
\renewcommand{\headrulewidth}{0.2pt}
\fancyhead[L]{maths-cours.fr}
\fancyhead[R]{\thepage}
\renewcommand{\footrulewidth}{0.2pt}
\fancyfoot[C]{}

\newcolumntype{C}{>{\centering\arraybackslash}X}
\newcolumntype{s}{>{\hsize=.35\hsize\arraybackslash}X}

\setlength{\parindent}{0pt}		 
\setlength{\parskip}{3mm}
\setlength{\headheight}{1cm}

\def\ebook{ebook}
\def\book{book}
\def\web{web}
\def\type{web}

\newcommand{\vect}[1]{\overrightarrow{\,\mathstrut#1\,}}

\def\Oij{$\left(\text{O}~;~\vect{\imath},~\vect{\jmath}\right)$}
\def\Oijk{$\left(\text{O}~;~\vect{\imath},~\vect{\jmath},~\vect{k}\right)$}
\def\Ouv{$\left(\text{O}~;~\vect{u},~\vect{v}\right)$}

\hypersetup{breaklinks=true, colorlinks = true, linkcolor = OliveGreen, urlcolor = OliveGreen, citecolor = OliveGreen, pdfauthor={Didier BONNEL - https://www.maths-cours.fr} } % supprime les bordures autour des liens

\renewcommand{\arg}[0]{\text{arg}}

\everymath{\displaystyle}

%================================================================================================================================
%
% Macros - Commandes
%
%================================================================================================================================

\newcommand\meta[2]{    			% Utilisé pour créer le post HTML.
	\def\titre{titre}
	\def\url{url}
	\def\arg{#1}
	\ifx\titre\arg
		\newcommand\maintitle{#2}
		\fancyhead[L]{#2}
		{\Large\sffamily \MakeUppercase{#2}}
		\vspace{1mm}\textcolor{mcvert}{\hrule}
	\fi 
	\ifx\url\arg
		\fancyfoot[L]{\href{https://www.maths-cours.fr#2}{\black \footnotesize{https://www.maths-cours.fr#2}}}
	\fi 
}


\newcommand\TitreC[1]{    		% Titre centré
     \needspace{3\baselineskip}
     \begin{center}\textbf{#1}\end{center}
}

\newcommand\newpar{    		% paragraphe
     \par
}

\newcommand\nosp {    		% commande vide (pas d'espace)
}
\newcommand{\id}[1]{} %ignore

\newcommand\boite[2]{				% Boite simple sans titre
	\vspace{5mm}
	\setlength{\fboxrule}{0.2mm}
	\setlength{\fboxsep}{5mm}	
	\fcolorbox{#1}{#1!3}{\makebox[\linewidth-2\fboxrule-2\fboxsep]{
  		\begin{minipage}[t]{\linewidth-2\fboxrule-4\fboxsep}\setlength{\parskip}{3mm}
  			 #2
  		\end{minipage}
	}}
	\vspace{5mm}
}

\newcommand\CBox[4]{				% Boites
	\vspace{5mm}
	\setlength{\fboxrule}{0.2mm}
	\setlength{\fboxsep}{5mm}
	
	\fcolorbox{#1}{#1!3}{\makebox[\linewidth-2\fboxrule-2\fboxsep]{
		\begin{minipage}[t]{1cm}\setlength{\parskip}{3mm}
	  		\textcolor{#1}{\LARGE{#2}}    
 	 	\end{minipage}  
  		\begin{minipage}[t]{\linewidth-2\fboxrule-4\fboxsep}\setlength{\parskip}{3mm}
			\raisebox{1.2mm}{\normalsize\sffamily{\textcolor{#1}{#3}}}						
  			 #4
  		\end{minipage}
	}}
	\vspace{5mm}
}

\newcommand\cadre[3]{				% Boites convertible html
	\par
	\vspace{2mm}
	\setlength{\fboxrule}{0.1mm}
	\setlength{\fboxsep}{5mm}
	\fcolorbox{#1}{white}{\makebox[\linewidth-2\fboxrule-2\fboxsep]{
  		\begin{minipage}[t]{\linewidth-2\fboxrule-4\fboxsep}\setlength{\parskip}{3mm}
			\raisebox{-2.5mm}{\sffamily \small{\textcolor{#1}{\MakeUppercase{#2}}}}		
			\par		
  			 #3
 	 		\end{minipage}
	}}
		\vspace{2mm}
	\par
}

\newcommand\bloc[3]{				% Boites convertible html sans bordure
     \needspace{2\baselineskip}
     {\sffamily \small{\textcolor{#1}{\MakeUppercase{#2}}}}    
		\par		
  			 #3
		\par
}

\newcommand\CHelp[1]{
     \CBox{Plum}{\faInfoCircle}{À RETENIR}{#1}
}

\newcommand\CUp[1]{
     \CBox{NavyBlue}{\faThumbsOUp}{EN PRATIQUE}{#1}
}

\newcommand\CInfo[1]{
     \CBox{Sepia}{\faArrowCircleRight}{REMARQUE}{#1}
}

\newcommand\CRedac[1]{
     \CBox{PineGreen}{\faEdit}{BIEN R\'EDIGER}{#1}
}

\newcommand\CError[1]{
     \CBox{Red}{\faExclamationTriangle}{ATTENTION}{#1}
}

\newcommand\TitreExo[2]{
\needspace{4\baselineskip}
 {\sffamily\large EXERCICE #1\ (\emph{#2 points})}
\vspace{5mm}
}

\newcommand\img[2]{
          \includegraphics[width=#2\paperwidth]{\imgdir#1}
}

\newcommand\imgsvg[2]{
       \begin{center}   \includegraphics[width=#2\paperwidth]{\imgsvgdir#1} \end{center}
}


\newcommand\Lien[2]{
     \href{#1}{#2 \tiny \faExternalLink}
}
\newcommand\mcLien[2]{
     \href{https~://www.maths-cours.fr/#1}{#2 \tiny \faExternalLink}
}

\newcommand{\euro}{\eurologo{}}

%================================================================================================================================
%
% Macros - Environement
%
%================================================================================================================================

\newenvironment{tex}{ %
}
{%
}

\newenvironment{indente}{ %
	\setlength\parindent{10mm}
}

{
	\setlength\parindent{0mm}
}

\newenvironment{corrige}{%
     \needspace{3\baselineskip}
     \medskip
     \textbf{\textsc{Corrigé}}
     \medskip
}
{
}

\newenvironment{extern}{%
     \begin{center}
     }
     {
     \end{center}
}

\NewEnviron{code}{%
	\par
     \boite{gray}{\texttt{%
     \BODY
     }}
     \par
}

\newenvironment{vbloc}{% boite sans cadre empeche saut de page
     \begin{minipage}[t]{\linewidth}
     }
     {
     \end{minipage}
}
\NewEnviron{h2}{%
    \needspace{3\baselineskip}
    \vspace{0.6cm}
	\noindent \MakeUppercase{\sffamily \large \BODY}
	\vspace{1mm}\textcolor{mcgris}{\hrule}\vspace{0.4cm}
	\par
}{}

\NewEnviron{h3}{%
    \needspace{3\baselineskip}
	\vspace{5mm}
	\textsc{\BODY}
	\par
}

\NewEnviron{margeneg}{ %
\begin{addmargin}[-1cm]{0cm}
\BODY
\end{addmargin}
}

\NewEnviron{html}{%
}

\begin{document}
\meta{url}{/exercices/suites-equations-differentielles-bac-s-pondichery-2008/}
\meta{pid}{2258}
\meta{titre}{Suites et équations différentielles - Bac S Pondichéry 2008}
\meta{type}{exercices}
%
\begin{h2}Exercice 4\end{h2}
\textit{(7 points) Commun à tous les candidats}
\par
On cherche à modéliser de deux façons différentes l'évolution du nombre, exprimé en millions, de foyers français possédant un téléviseur à écran plat, en fonction de l'année.
\par
Les parties A et B sont indépendantes.
\begin{h3}Partie A : un modèle discret\end{h3}
Soit un le nombre, exprimé en millions, de foyers possédant un téléviseur à écran plat l'année n.
\par
On pose $ n = 0  $ en 2005, $u_{0}=1$ et, pour tout $n > 0$, $u_{n+1}=\frac{1}{10} u_{n} \left(20-u_{n}\right)$.
\begin{enumerate}
     \item
     Soit $f$ la fonction définie sur [0 ; 20] par $f\left(x\right)=\frac{1}{10} x\left(20-x\right)$.
     \begin{enumerate}[label=\alph*.]
          \item
          Etudier les variations de $f$ sur [0 ; 20].
          \item
          En déduire que pour tout $x\in \left[0; 10\right]$,  $f\left(x\right)\in \left[0; 10\right]$.
          \item
          On donne ci-dessous la courbe représentative $\left(C\right)$ de la fonction $f$ dans un repère orthonormal $\left(O; \vec{i}, \vec{j}\right)$ du plan. Représenter, sur l'axe des abscisses, à l'aide de ce graphique, les cinq premiers termes de la suite $\left(u_{n}\right)_{n > 0}$.


\begin{center}
\imgsvg{copie-0002}{0.3}% alt="Courbe suite 1" style="width:50rem"
\end{center}
     \end{enumerate}
     \item
     Montrer par récurrence que pour tout $n\in \mathbb{N}$ , $0 < u_{n} < u_{n+1} < 10$.
     \item
     Montrer que la suite $\left(u_{n}\right)_{n > 0}$ est convergente et déterminer sa limite.
\end{enumerate}
\begin{h3}Partie B : un modèle continu\end{h3}
Soit $g\left(x\right)$ le nombre, exprimé en millions, de tels foyers l'année $x$. On pose $x=0$ en 2005, $g\left(0\right)=1$ et $g$ est une solution, qui ne s'annule pas sur $\left[0,+\infty \right[$, de l'équation différentielle
\par
(E) : $y^{\prime}=\frac{1}{20} y \left(10-y\right)$
\begin{enumerate}
     \item
     On considère une fonction $y$ qui ne s'annule pas sur $\left[0,+\infty \right[$ et on pose $z=\frac{1}{y}$.
     \begin{enumerate}[label=\alph*.]
          \item
          Montrer que y est solution de (E) si et seulement si z est solution de l'équation différentielle :
          \par
          (E1) : $z^{\prime}=-\frac{1}{2} z+\frac{1}{20}$.
          \item
          Résoudre l'équation (E1) et en déduire les solutions de l'équation (E).
     \end{enumerate}
     \item
     Montrer que $g$ est définie sur $\left[0,+\infty \right[$ par $g\left(x\right)=\frac{10}{9e^{-\frac{1}{2}x}+1}$.
     \item
     Etudier les variations de $g$ sur $\left[0,+\infty \right[$.
     \item
     Calculer la limite de $g$ en $+\infty $ et interpréter le résultat.
     \item
     En quelle année le nombre de foyers possédant un tel équipement dépassera-t-il 5 millions ?
\end{enumerate}
\begin{corrige}
     \begin{h3}Partie A \end{h3}
     \begin{enumerate}
          \item
          \begin{enumerate}[label=\alph*.]
               \item
               $f\left(x\right)=-\frac{1}{10}x^{2}+2x$
               \par
               $f$ est une fonction polynôme du second degré avec un coefficient de $x^{2}$ négatif.
               \par
               Sa courbe représentative est une parabole dont l'abscisse du sommet est :
               \par
               $x_{0}=-\frac{b}{2a}=10$
               \par
               Son tableau de variation est le suivant :
%##
% type=table; width=25; l2=20
%--
% x|   0   ~   10  ~   20 
% f(x)|  0              /   10   \   0
%--
\begin{center}
 \begin{extern}%style="width:25rem" alt="Exercice"
    \resizebox{11cm}{!}{
       \definecolor{dark}{gray}{0.1}
       \definecolor{light}{gray}{0.8}
       \tikzstyle{fleche}=[->,>=latex]
       \begin{tikzpicture}[scale=.8, line width=.5pt, dark]
       \def\width{.15}
       \def\height{.10}
       \draw (0, -10*\height) -- (54*\width, -10*\height);
       \draw (10*\width, 0*\height) -- (10*\width, -10*\height);
       \node (l0c0) at (5*\width,-5*\height) {$ x $};
       \node (l0c1) at (14*\width,-5*\height) {$ 0 $};
       \node (l0c2) at (23*\width,-5*\height) {$ ~ $};
       \node (l0c3) at (32*\width,-5*\height) {$ 10 $};
       \node (l0c4) at (41*\width,-5*\height) {$ ~ $};
       \node (l0c5) at (50*\width,-5*\height) {$ 20 $};
       \draw (0, -30*\height) -- (54*\width, -30*\height);
       \draw (10*\width, -10*\height) -- (10*\width, -30*\height);
       \node (l1c0) at (5*\width,-20*\height) {$ f(x) $};
       \node (l1c1) at (14*\width,-25*\height) {$ 0 $};
       \node (l1c2) at (23*\width,-20*\height) {$ ~ $};
       \node (l1c3) at (32*\width,-15*\height) {$ 10 $};
       \node (l1c4) at (41*\width,-20*\height) {$ ~ $};
       \node (l1c5) at (50*\width,-25*\height) {$ 0 $};
       \draw (0, 0) rectangle (54*\width, -30*\height);
       \draw[fleche] (l1c1) -- (l1c3);
       \draw[fleche] (l1c3) -- (l1c5);
       \end{tikzpicture}
      }
   \end{extern}
\end{center}
%##
\item
               D'après le tableau il est évident que pour tout $x\in \left[0 ; 10\right] , f\left(x\right)\in \left[0 ; 10\right]$.
               \item

\begin{center}
\imgsvg{copie-0010}{0.3}% alt="Courbe suite 2" style="width:50rem"
\end{center}
          \end{enumerate}
          \item
          $u_{0}=1$ et $u1=1,9$ donc la propriété est vraie au rang 0 ($0\leqslant 1\leqslant 1,9\leqslant 10$)
          \par
          Supposons $0\leqslant u_{n}\leqslant u_{n+1}\leqslant 10$.
          \par
          Comme f est  croissante sur [0;10]
          \par
          $f\left(0\right)\leqslant f\left(u_{n}\right)\leqslant f\left(u_{n+1}\right)\leqslant f\left(10\right)$
          \par
          c'est à dire
          \par
          $0\leqslant u_{n+1}\leqslant u_{n+2}\leqslant 10$.
          \par
          Donc la propriété est héréditaire.
          \par
          Donc pour tout $n \in \mathbb{N}, \cdots  0 \leqslant u_{n} \leqslant u_{n+1} \leqslant 10$.
          \item
          La suite $\left(u_{n}\right)_{n \geqslant 0}$ est croissante et majorée (par 10) donc elle est convergente. Sa limite vérifie $l=f\left(l\right)$
          \par
          Ce qui donne $l^{2}-10l=0$ donc $l=0$ ou $l=10$.
          \par
          La solution $l=0$ ne peut convenir car, la suite étant croissante $l\geqslant u_{0}\geqslant 1$
          \par
          Donc $l=10$.
     \end{enumerate}
     \begin{h3}Partie B\end{h3}
     \begin{enumerate}
          \item
          \begin{enumerate}[label=\alph*.]
               \item
               $z=\frac{1}{y}$ ne s'annule pas sur $\left[0,+\infty \right[$ donc $y=\frac{1}{z}$ et $y^{\prime}=-\frac{z^{\prime}}{z^{2}}$.
               \par
               $y^{\prime}=\frac{1}{2}y-\frac{1}{20}y^{2} \Leftrightarrow -\frac{z^{\prime}}{z^{2}}=\frac{1}{2z}-\frac{1}{20z^{2}} \Leftrightarrow  z^{\prime}=-\frac{1}{2}z+\frac{1}{20}$
               \item
               Les solutions de l'équation (E1) sont les fonctions définies par :
               \par
               $z\left(x\right)=Ce^{-\frac{1}{2}x}-\frac{\frac{1}{20}}{-\frac{1}{2}}=Ce^{-\frac{1}{2}x}+\frac{1}{10}$ où $C \in \mathbb{R}$
               \par
               Les solutions de (E) sont les fonctions définies par :
               \par
               $\frac{1}{y\left(x\right)}=Ce^{-\frac{1}{2}x}+\frac{1}{10}$
               \par
               $y\left(x\right)=\frac{1}{Ce^{-\frac{1}{2}x}+\frac{1}{10}}=\frac{10}{Ke^{-\frac{1}{2}x}+1}$ où $K \in \mathbb{R}$ (on a posé $K=10C$)
          \end{enumerate}
          \item
          $g$ vérifie $g\left(0\right)=1$ ce qui donne:
          \par
          $\frac{10}{K+1}=1$ soit $K=9$
          \par
          Donc :
          \par
          $g\left(x\right)=\frac{10}{9e^{-\frac{1}{2}x}+1}$.
          \item
          $g^{\prime}\left(x\right)=-10\times \frac{-\frac{9}{2}e^{-\frac{1}{2}x}}{\left(9e^{-\frac{1}{2}x}+1\right)^{2}}=45\times \frac{e^{-\frac{1}{2}x}}{\left(9e^{-\frac{1}{2}x}+1\right)^{2}} > 0$
          \par
          Donc la fonction $g$ est strictement croissante sur $\left[0,+\infty \right[$
          \item
          $\lim\limits_{x\rightarrow +\infty } -\frac{1}{2}x=-\infty $ et $\lim\limits_{x\rightarrow -\infty } e^{x}=0$ donc par composition $\lim\limits_{x\rightarrow +\infty }e^{-\frac{1}{2}x}=0$ et
          \par
          $\lim\limits_{x\rightarrow +\infty }\frac{10}{9e^{-\frac{1}{2}x}+1}=10$
          \par
          Le nombre de foyers possédant un tel équipement se rapprochera progressivement de 10 millions.
          \item
          Comme $9e^{-\frac{1}{2}x}+1 > 0$ :
          \par
          $g\left(x\right)\geqslant 5 \Leftrightarrow  10\geqslant 5\left(9e^{-\frac{1}{2}x}+1\right) \Leftrightarrow  1\geqslant 9e^{-\frac{1}{2}x} $
          \par
          La fonction $\ln$ étant strictement croissante sur $\left]0,+\infty \right[$ :
          \par
          $g\left(x\right)\geqslant 5 \Leftrightarrow  \ln\left(1\right)\geqslant \ln\left(9e^{-\frac{1}{2}x}\right) \Leftrightarrow  0\geqslant -\frac{1}{2}x+\ln9 \Leftrightarrow  \frac{1}{2}x\geqslant \ln9 \Leftrightarrow  x\geqslant 2\ln9$
          \par
          Comme $4\leqslant 2\ln9\leqslant 5$, le nombre de foyers possédant un tel équipement dépassera les 5 millions au cours de l'année 2009.
     \end{enumerate}
\end{corrige}

\end{document}