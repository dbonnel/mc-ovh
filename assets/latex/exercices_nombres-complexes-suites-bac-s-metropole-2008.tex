\documentclass[a4paper]{article}

%================================================================================================================================
%
% Packages
%
%================================================================================================================================

\usepackage[T1]{fontenc} 	% pour caractères accentués
\usepackage[utf8]{inputenc}  % encodage utf8
\usepackage[french]{babel}	% langue : français
\usepackage{fourier}			% caractères plus lisibles
\usepackage[dvipsnames]{xcolor} % couleurs
\usepackage{fancyhdr}		% réglage header footer
\usepackage{needspace}		% empêcher sauts de page mal placés
\usepackage{graphicx}		% pour inclure des graphiques
\usepackage{enumitem,cprotect}		% personnalise les listes d'items (nécessaire pour ol, al ...)
\usepackage{hyperref}		% Liens hypertexte
\usepackage{pstricks,pst-all,pst-node,pstricks-add,pst-math,pst-plot,pst-tree,pst-eucl} % pstricks
\usepackage[a4paper,includeheadfoot,top=2cm,left=3cm, bottom=2cm,right=3cm]{geometry} % marges etc.
\usepackage{comment}			% commentaires multilignes
\usepackage{amsmath,environ} % maths (matrices, etc.)
\usepackage{amssymb,makeidx}
\usepackage{bm}				% bold maths
\usepackage{tabularx}		% tableaux
\usepackage{colortbl}		% tableaux en couleur
\usepackage{fontawesome}		% Fontawesome
\usepackage{environ}			% environment with command
\usepackage{fp}				% calculs pour ps-tricks
\usepackage{multido}			% pour ps tricks
\usepackage[np]{numprint}	% formattage nombre
\usepackage{tikz,tkz-tab} 			% package principal TikZ
\usepackage{pgfplots}   % axes
\usepackage{mathrsfs}    % cursives
\usepackage{calc}			% calcul taille boites
\usepackage[scaled=0.875]{helvet} % font sans serif
\usepackage{svg} % svg
\usepackage{scrextend} % local margin
\usepackage{scratch} %scratch
\usepackage{multicol} % colonnes
%\usepackage{infix-RPN,pst-func} % formule en notation polanaise inversée
\usepackage{listings}

%================================================================================================================================
%
% Réglages de base
%
%================================================================================================================================

\lstset{
language=Python,   % R code
literate=
{á}{{\'a}}1
{à}{{\`a}}1
{ã}{{\~a}}1
{é}{{\'e}}1
{è}{{\`e}}1
{ê}{{\^e}}1
{í}{{\'i}}1
{ó}{{\'o}}1
{õ}{{\~o}}1
{ú}{{\'u}}1
{ü}{{\"u}}1
{ç}{{\c{c}}}1
{~}{{ }}1
}


\definecolor{codegreen}{rgb}{0,0.6,0}
\definecolor{codegray}{rgb}{0.5,0.5,0.5}
\definecolor{codepurple}{rgb}{0.58,0,0.82}
\definecolor{backcolour}{rgb}{0.95,0.95,0.92}

\lstdefinestyle{mystyle}{
    backgroundcolor=\color{backcolour},   
    commentstyle=\color{codegreen},
    keywordstyle=\color{magenta},
    numberstyle=\tiny\color{codegray},
    stringstyle=\color{codepurple},
    basicstyle=\ttfamily\footnotesize,
    breakatwhitespace=false,         
    breaklines=true,                 
    captionpos=b,                    
    keepspaces=true,                 
    numbers=left,                    
xleftmargin=2em,
framexleftmargin=2em,            
    showspaces=false,                
    showstringspaces=false,
    showtabs=false,                  
    tabsize=2,
    upquote=true
}

\lstset{style=mystyle}


\lstset{style=mystyle}
\newcommand{\imgdir}{C:/laragon/www/newmc/assets/imgsvg/}
\newcommand{\imgsvgdir}{C:/laragon/www/newmc/assets/imgsvg/}

\definecolor{mcgris}{RGB}{220, 220, 220}% ancien~; pour compatibilité
\definecolor{mcbleu}{RGB}{52, 152, 219}
\definecolor{mcvert}{RGB}{125, 194, 70}
\definecolor{mcmauve}{RGB}{154, 0, 215}
\definecolor{mcorange}{RGB}{255, 96, 0}
\definecolor{mcturquoise}{RGB}{0, 153, 153}
\definecolor{mcrouge}{RGB}{255, 0, 0}
\definecolor{mclightvert}{RGB}{205, 234, 190}

\definecolor{gris}{RGB}{220, 220, 220}
\definecolor{bleu}{RGB}{52, 152, 219}
\definecolor{vert}{RGB}{125, 194, 70}
\definecolor{mauve}{RGB}{154, 0, 215}
\definecolor{orange}{RGB}{255, 96, 0}
\definecolor{turquoise}{RGB}{0, 153, 153}
\definecolor{rouge}{RGB}{255, 0, 0}
\definecolor{lightvert}{RGB}{205, 234, 190}
\setitemize[0]{label=\color{lightvert}  $\bullet$}

\pagestyle{fancy}
\renewcommand{\headrulewidth}{0.2pt}
\fancyhead[L]{maths-cours.fr}
\fancyhead[R]{\thepage}
\renewcommand{\footrulewidth}{0.2pt}
\fancyfoot[C]{}

\newcolumntype{C}{>{\centering\arraybackslash}X}
\newcolumntype{s}{>{\hsize=.35\hsize\arraybackslash}X}

\setlength{\parindent}{0pt}		 
\setlength{\parskip}{3mm}
\setlength{\headheight}{1cm}

\def\ebook{ebook}
\def\book{book}
\def\web{web}
\def\type{web}

\newcommand{\vect}[1]{\overrightarrow{\,\mathstrut#1\,}}

\def\Oij{$\left(\text{O}~;~\vect{\imath},~\vect{\jmath}\right)$}
\def\Oijk{$\left(\text{O}~;~\vect{\imath},~\vect{\jmath},~\vect{k}\right)$}
\def\Ouv{$\left(\text{O}~;~\vect{u},~\vect{v}\right)$}

\hypersetup{breaklinks=true, colorlinks = true, linkcolor = OliveGreen, urlcolor = OliveGreen, citecolor = OliveGreen, pdfauthor={Didier BONNEL - https://www.maths-cours.fr} } % supprime les bordures autour des liens

\renewcommand{\arg}[0]{\text{arg}}

\everymath{\displaystyle}

%================================================================================================================================
%
% Macros - Commandes
%
%================================================================================================================================

\newcommand\meta[2]{    			% Utilisé pour créer le post HTML.
	\def\titre{titre}
	\def\url{url}
	\def\arg{#1}
	\ifx\titre\arg
		\newcommand\maintitle{#2}
		\fancyhead[L]{#2}
		{\Large\sffamily \MakeUppercase{#2}}
		\vspace{1mm}\textcolor{mcvert}{\hrule}
	\fi 
	\ifx\url\arg
		\fancyfoot[L]{\href{https://www.maths-cours.fr#2}{\black \footnotesize{https://www.maths-cours.fr#2}}}
	\fi 
}


\newcommand\TitreC[1]{    		% Titre centré
     \needspace{3\baselineskip}
     \begin{center}\textbf{#1}\end{center}
}

\newcommand\newpar{    		% paragraphe
     \par
}

\newcommand\nosp {    		% commande vide (pas d'espace)
}
\newcommand{\id}[1]{} %ignore

\newcommand\boite[2]{				% Boite simple sans titre
	\vspace{5mm}
	\setlength{\fboxrule}{0.2mm}
	\setlength{\fboxsep}{5mm}	
	\fcolorbox{#1}{#1!3}{\makebox[\linewidth-2\fboxrule-2\fboxsep]{
  		\begin{minipage}[t]{\linewidth-2\fboxrule-4\fboxsep}\setlength{\parskip}{3mm}
  			 #2
  		\end{minipage}
	}}
	\vspace{5mm}
}

\newcommand\CBox[4]{				% Boites
	\vspace{5mm}
	\setlength{\fboxrule}{0.2mm}
	\setlength{\fboxsep}{5mm}
	
	\fcolorbox{#1}{#1!3}{\makebox[\linewidth-2\fboxrule-2\fboxsep]{
		\begin{minipage}[t]{1cm}\setlength{\parskip}{3mm}
	  		\textcolor{#1}{\LARGE{#2}}    
 	 	\end{minipage}  
  		\begin{minipage}[t]{\linewidth-2\fboxrule-4\fboxsep}\setlength{\parskip}{3mm}
			\raisebox{1.2mm}{\normalsize\sffamily{\textcolor{#1}{#3}}}						
  			 #4
  		\end{minipage}
	}}
	\vspace{5mm}
}

\newcommand\cadre[3]{				% Boites convertible html
	\par
	\vspace{2mm}
	\setlength{\fboxrule}{0.1mm}
	\setlength{\fboxsep}{5mm}
	\fcolorbox{#1}{white}{\makebox[\linewidth-2\fboxrule-2\fboxsep]{
  		\begin{minipage}[t]{\linewidth-2\fboxrule-4\fboxsep}\setlength{\parskip}{3mm}
			\raisebox{-2.5mm}{\sffamily \small{\textcolor{#1}{\MakeUppercase{#2}}}}		
			\par		
  			 #3
 	 		\end{minipage}
	}}
		\vspace{2mm}
	\par
}

\newcommand\bloc[3]{				% Boites convertible html sans bordure
     \needspace{2\baselineskip}
     {\sffamily \small{\textcolor{#1}{\MakeUppercase{#2}}}}    
		\par		
  			 #3
		\par
}

\newcommand\CHelp[1]{
     \CBox{Plum}{\faInfoCircle}{À RETENIR}{#1}
}

\newcommand\CUp[1]{
     \CBox{NavyBlue}{\faThumbsOUp}{EN PRATIQUE}{#1}
}

\newcommand\CInfo[1]{
     \CBox{Sepia}{\faArrowCircleRight}{REMARQUE}{#1}
}

\newcommand\CRedac[1]{
     \CBox{PineGreen}{\faEdit}{BIEN R\'EDIGER}{#1}
}

\newcommand\CError[1]{
     \CBox{Red}{\faExclamationTriangle}{ATTENTION}{#1}
}

\newcommand\TitreExo[2]{
\needspace{4\baselineskip}
 {\sffamily\large EXERCICE #1\ (\emph{#2 points})}
\vspace{5mm}
}

\newcommand\img[2]{
          \includegraphics[width=#2\paperwidth]{\imgdir#1}
}

\newcommand\imgsvg[2]{
       \begin{center}   \includegraphics[width=#2\paperwidth]{\imgsvgdir#1} \end{center}
}


\newcommand\Lien[2]{
     \href{#1}{#2 \tiny \faExternalLink}
}
\newcommand\mcLien[2]{
     \href{https~://www.maths-cours.fr/#1}{#2 \tiny \faExternalLink}
}

\newcommand{\euro}{\eurologo{}}

%================================================================================================================================
%
% Macros - Environement
%
%================================================================================================================================

\newenvironment{tex}{ %
}
{%
}

\newenvironment{indente}{ %
	\setlength\parindent{10mm}
}

{
	\setlength\parindent{0mm}
}

\newenvironment{corrige}{%
     \needspace{3\baselineskip}
     \medskip
     \textbf{\textsc{Corrigé}}
     \medskip
}
{
}

\newenvironment{extern}{%
     \begin{center}
     }
     {
     \end{center}
}

\NewEnviron{code}{%
	\par
     \boite{gray}{\texttt{%
     \BODY
     }}
     \par
}

\newenvironment{vbloc}{% boite sans cadre empeche saut de page
     \begin{minipage}[t]{\linewidth}
     }
     {
     \end{minipage}
}
\NewEnviron{h2}{%
    \needspace{3\baselineskip}
    \vspace{0.6cm}
	\noindent \MakeUppercase{\sffamily \large \BODY}
	\vspace{1mm}\textcolor{mcgris}{\hrule}\vspace{0.4cm}
	\par
}{}

\NewEnviron{h3}{%
    \needspace{3\baselineskip}
	\vspace{5mm}
	\textsc{\BODY}
	\par
}

\NewEnviron{margeneg}{ %
\begin{addmargin}[-1cm]{0cm}
\BODY
\end{addmargin}
}

\NewEnviron{html}{%
}

\begin{document}
\meta{url}{/exercices/nombres-complexes-suites-bac-s-metropole-2008/}
\meta{pid}{2227}
\meta{titre}{Nombres complexes et suites - Bac S Métropole 2008}
\meta{type}{exercices}
%
\begin{h2}Exercice 4 (5 points)\end{h2}
\textit{Candidats \textbf{ayant suivi l'enseignement de spécialité}}
\par
Le plan est rapporté à un repère orthonormal direct $\left(O ; \vec{u} , \vec{v} \right)$.
\par
Soient A et B les points d'affixes respectives $z_{A}=1-\text{i}$ et $z_{B}=7+\frac{7}{2}\text{i}$.
\begin{enumerate}
     \item
     On considère la droite $\left(d\right)$ d'équation $4x+3y=1$.
     \par
     Démontrer que l'ensemble des points de $\left(d\right)$ dont les coordonnées sont entières est l'ensemble des points $M_{k}\left(3k+1 , -4k-1\right)$ lorsque k décrit l'ensemble des entiers relatifs.
     \item
     Déterminer l'angle et le rapport de la similitude directe de centre A qui transforme B en $M_{-1}$(-2 , 3).
     \item
     Soit $s$ la transformation du plan qui à tout point M d'affixe $z$ associe le point M' d'affixe $ z^{\prime}=\frac{2}{3}\text{i}z+\frac{1}{3}-\frac{5}{3}\text{i}$.
     \par
     Déterminer l'image de A par $s$, puis donner la nature et les éléments caractéristiques de $s$.
     \item
     On note $B_{1}$ l'image de B par $s$ et pour tout entier naturel $n$ non nul, $B_{n+1}$ l'image de $B_{n}$ par $s$.
     \begin{enumerate}
          \item
          Déterminer la longueur $AB_{n+1}$ en fontion de $AB_{n}$.
          \item
          A partir de quel entier $n$ le point $B_{n}$ appartient t-il au disque de centre A et de rayon $10^{-2}$ ?
          \item
          Déterminer l'ensemble des entiers $n$ pour lesquels A, $B_{1}$, $B_{n}$ sont alignés.
     \end{enumerate}
\end{enumerate}
\begin{corrige}
     \begin{enumerate}
          \item
          $4 \times  1+3 \times  \left(-1\right)=1$, donc le point de coordonnées (1 ; -1) appartient à la droite $\left(d\right)$
          \par
          Soit M(x,y) un point de $\left(d\right)$ à coordonnées entières.
          \par
          $4x+3y=1 \Leftrightarrow  4x+3y=4\times 1+3\times \left(-1\right) \Leftrightarrow  4\left(x-1\right)=-3\left(y+1\right)=0$
          \par
          4 et 3 étant premiers entre eux, on en déduit, d'après le théorème de Gauss,  que x-1 est un multiple de 3, c'est à dire qu'il existe $k \in \mathbb{Z}$ tel que $x-1=3k$ c'est à dire $x=3k+1$.
          \par
          On a alors $3y=1-4x=-12k-3$ soit $y=-4k-1$
          \par
          Réciproquement comme :
          \par
          $4\left(3k+1\right)+3\left(-4k-1\right)=1$
          \par
          tout point de coordonnées $\left(3k+1;-4k-1\right)$ avec $k \in \mathbb{Z}$ est un point de $\left(d\right)$ à coordonnées entières.
          \item
          Le rapport de la similitude directe de centre A qui transforme B en $M_{-1}$ est :
          \par
          $k=\frac{AM_{-1}}{AB}=|\frac{z_{\overrightarrow{AM}_{-1}}}{z_{\overrightarrow{AB}}}|$
          \par
          or
          \par
          $\frac{z_{\overrightarrow{AM}_{-1}}}{z_{\overrightarrow{AB}}}=\frac{-3+4i}{6+\frac{9}{2}i}=\frac{-6+8i}{12+9i}=\frac{\left(-6+8i\right)\left(12-9i\right)}{\left(12+9i\right)\left(12-9i\right)}=\frac{150i}{225}=\frac{2}{3}i$
          \par
          donc
          \par
          $k=|\frac{2}{3}i|=\frac{2}{3}$
          \par
          L'angle de cette similitude est :
          \par
          $\theta =\text{arg}\left(\frac{z_{\overrightarrow{AM}_{-1}}}{z_{\overrightarrow{AB}}}\right)=\text{arg}\left(\frac{2}{3}i\right)=\frac{\pi }{2}\ \left(2\pi \right)$
          \item
          L'image de A par $s$ est le point A' d'affixe
          \par
          $z_{A^{\prime}}=\frac{2}{3}i\left(1-i\right)+\frac{1}{3}-\frac{5}{3}i=1-i=z_{A}$
          \par
          Donc A est le centre de la similitude directe $s$.
          \par
          Le rapport de $s$ est $k=|\frac{2}{3}i|=\frac{2}{3}$
          \par
          L'angle de $s$ est $\theta =\text{arg}\left(\frac{2}{3}i\right)=\frac{\pi }{2}\ \left(2\pi \right)$
          \item
          \begin{enumerate}
               \item
               La similitude s transforme $B_{n}$ en $B_{n+1}$. Son centre est A et son rapport $\frac{2}{3}$ donc :
               \par
               $\frac{AB_{n+1}}{AB_{n}}=\frac{2}{3}$
               \par
               c'est à dire:
               \par
               $AB_{n+1}=\frac{2}{3}AB_{n}$
               \item
               La suite $\left(AB_{n}\right)$ est une suite géométrique de raison 2/3 et de premier terme
               \par
               $AB=\sqrt{6^{2}+\left(\frac{9}{2}\right)^{2}}=\frac{15}{2}$
               \par
               donc:
               \par
               $AB_{n}=\frac{15}{2}\times \left(\frac{2}{3}\right)^{n}$
               \par
               $AB_{n}$ est donc inférieur à $10^{-2}$ si et seulement si :
               \par
               $\frac{15}{2}\times \left(\frac{2}{3}\right)^{n} < \frac{1}{100}$
               \par
               c'est à dire :
               \par
               $\left(\frac{2}{3}\right)^{n} < \frac{1}{750}$
               \par
               La fonction ln étant croissante cette inégalité équivaut à :
               \par
               $\ln\left(\frac{2}{3}\right)^{n} < \ln\left(\frac{1}{750}\right)$
               \par
               $n\left(\ln2-\ln3\right) < -\ln750$
               \par
               soit comme $\ln2-\ln3$ est négatif :
               \par
               $n > \frac{-\ln750}{\ln2-\ln3}\approx 16,3$
               \par
               Le point $B_n$ appartient au disque de centre $A$ et de rayon $10^{-2}$ à partir de $n=17$.
               \item
               Pour tout entier $n > 0 $, $B_{n+1}$ est l'image de $B_n$ par la similitude $s$ de centre $A$ et d'angle $\frac{\pi }{2}$ donc :
               \par
               $\left(\overrightarrow{AB}_{n};\overrightarrow{AB}_{n+1}\right)=\frac{\pi }{2} \  \left(2\pi \right)$
               \par
               Donc :
               \par
               $\left(\overrightarrow{AB}_{1};\overrightarrow{AB}_{n}\right)=\left(\overrightarrow{AB}_{1};\overrightarrow{AB}_{2}\right)+\left(\overrightarrow{AB}_{2};\overrightarrow{AB}_{3}\right)+. . .+\left(\overrightarrow{AB}_{n-1};\overrightarrow{AB}_{n}\right)\ \left(2\pi \right)$
               \par
               $\left(\overrightarrow{AB}_{1};\overrightarrow{AB}_{n}\right)=\frac{\pi }{2}+\frac{\pi }{2} +. . .+\frac{\pi }{2}=\left(n-1\right)\frac{\pi }{2}\ \left(2\pi \right)$
               \par
               Les points $A, B_1$ et $B_n$ sont alignés si et seulement si cet angle est un multiple de $\pi $ c'est à dire si et seulement si $n-1$ est pair donc si et seulement si \textbf{$n$ est impair}.
          \end{enumerate}
     \end{enumerate}
\end{corrige}

\end{document}