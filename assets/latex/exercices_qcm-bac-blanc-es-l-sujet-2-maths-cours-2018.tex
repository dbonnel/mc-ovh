\documentclass[a4paper]{article}

%================================================================================================================================
%
% Packages
%
%================================================================================================================================

\usepackage[T1]{fontenc} 	% pour caractères accentués
\usepackage[utf8]{inputenc}  % encodage utf8
\usepackage[french]{babel}	% langue : français
\usepackage{fourier}			% caractères plus lisibles
\usepackage[dvipsnames]{xcolor} % couleurs
\usepackage{fancyhdr}		% réglage header footer
\usepackage{needspace}		% empêcher sauts de page mal placés
\usepackage{graphicx}		% pour inclure des graphiques
\usepackage{enumitem,cprotect}		% personnalise les listes d'items (nécessaire pour ol, al ...)
\usepackage{hyperref}		% Liens hypertexte
\usepackage{pstricks,pst-all,pst-node,pstricks-add,pst-math,pst-plot,pst-tree,pst-eucl} % pstricks
\usepackage[a4paper,includeheadfoot,top=2cm,left=3cm, bottom=2cm,right=3cm]{geometry} % marges etc.
\usepackage{comment}			% commentaires multilignes
\usepackage{amsmath,environ} % maths (matrices, etc.)
\usepackage{amssymb,makeidx}
\usepackage{bm}				% bold maths
\usepackage{tabularx}		% tableaux
\usepackage{colortbl}		% tableaux en couleur
\usepackage{fontawesome}		% Fontawesome
\usepackage{environ}			% environment with command
\usepackage{fp}				% calculs pour ps-tricks
\usepackage{multido}			% pour ps tricks
\usepackage[np]{numprint}	% formattage nombre
\usepackage{tikz,tkz-tab} 			% package principal TikZ
\usepackage{pgfplots}   % axes
\usepackage{mathrsfs}    % cursives
\usepackage{calc}			% calcul taille boites
\usepackage[scaled=0.875]{helvet} % font sans serif
\usepackage{svg} % svg
\usepackage{scrextend} % local margin
\usepackage{scratch} %scratch
\usepackage{multicol} % colonnes
%\usepackage{infix-RPN,pst-func} % formule en notation polanaise inversée
\usepackage{listings}

%================================================================================================================================
%
% Réglages de base
%
%================================================================================================================================

\lstset{
language=Python,   % R code
literate=
{á}{{\'a}}1
{à}{{\`a}}1
{ã}{{\~a}}1
{é}{{\'e}}1
{è}{{\`e}}1
{ê}{{\^e}}1
{í}{{\'i}}1
{ó}{{\'o}}1
{õ}{{\~o}}1
{ú}{{\'u}}1
{ü}{{\"u}}1
{ç}{{\c{c}}}1
{~}{{ }}1
}


\definecolor{codegreen}{rgb}{0,0.6,0}
\definecolor{codegray}{rgb}{0.5,0.5,0.5}
\definecolor{codepurple}{rgb}{0.58,0,0.82}
\definecolor{backcolour}{rgb}{0.95,0.95,0.92}

\lstdefinestyle{mystyle}{
    backgroundcolor=\color{backcolour},   
    commentstyle=\color{codegreen},
    keywordstyle=\color{magenta},
    numberstyle=\tiny\color{codegray},
    stringstyle=\color{codepurple},
    basicstyle=\ttfamily\footnotesize,
    breakatwhitespace=false,         
    breaklines=true,                 
    captionpos=b,                    
    keepspaces=true,                 
    numbers=left,                    
xleftmargin=2em,
framexleftmargin=2em,            
    showspaces=false,                
    showstringspaces=false,
    showtabs=false,                  
    tabsize=2,
    upquote=true
}

\lstset{style=mystyle}


\lstset{style=mystyle}
\newcommand{\imgdir}{C:/laragon/www/newmc/assets/imgsvg/}
\newcommand{\imgsvgdir}{C:/laragon/www/newmc/assets/imgsvg/}

\definecolor{mcgris}{RGB}{220, 220, 220}% ancien~; pour compatibilité
\definecolor{mcbleu}{RGB}{52, 152, 219}
\definecolor{mcvert}{RGB}{125, 194, 70}
\definecolor{mcmauve}{RGB}{154, 0, 215}
\definecolor{mcorange}{RGB}{255, 96, 0}
\definecolor{mcturquoise}{RGB}{0, 153, 153}
\definecolor{mcrouge}{RGB}{255, 0, 0}
\definecolor{mclightvert}{RGB}{205, 234, 190}

\definecolor{gris}{RGB}{220, 220, 220}
\definecolor{bleu}{RGB}{52, 152, 219}
\definecolor{vert}{RGB}{125, 194, 70}
\definecolor{mauve}{RGB}{154, 0, 215}
\definecolor{orange}{RGB}{255, 96, 0}
\definecolor{turquoise}{RGB}{0, 153, 153}
\definecolor{rouge}{RGB}{255, 0, 0}
\definecolor{lightvert}{RGB}{205, 234, 190}
\setitemize[0]{label=\color{lightvert}  $\bullet$}

\pagestyle{fancy}
\renewcommand{\headrulewidth}{0.2pt}
\fancyhead[L]{maths-cours.fr}
\fancyhead[R]{\thepage}
\renewcommand{\footrulewidth}{0.2pt}
\fancyfoot[C]{}

\newcolumntype{C}{>{\centering\arraybackslash}X}
\newcolumntype{s}{>{\hsize=.35\hsize\arraybackslash}X}

\setlength{\parindent}{0pt}		 
\setlength{\parskip}{3mm}
\setlength{\headheight}{1cm}

\def\ebook{ebook}
\def\book{book}
\def\web{web}
\def\type{web}

\newcommand{\vect}[1]{\overrightarrow{\,\mathstrut#1\,}}

\def\Oij{$\left(\text{O}~;~\vect{\imath},~\vect{\jmath}\right)$}
\def\Oijk{$\left(\text{O}~;~\vect{\imath},~\vect{\jmath},~\vect{k}\right)$}
\def\Ouv{$\left(\text{O}~;~\vect{u},~\vect{v}\right)$}

\hypersetup{breaklinks=true, colorlinks = true, linkcolor = OliveGreen, urlcolor = OliveGreen, citecolor = OliveGreen, pdfauthor={Didier BONNEL - https://www.maths-cours.fr} } % supprime les bordures autour des liens

\renewcommand{\arg}[0]{\text{arg}}

\everymath{\displaystyle}

%================================================================================================================================
%
% Macros - Commandes
%
%================================================================================================================================

\newcommand\meta[2]{    			% Utilisé pour créer le post HTML.
	\def\titre{titre}
	\def\url{url}
	\def\arg{#1}
	\ifx\titre\arg
		\newcommand\maintitle{#2}
		\fancyhead[L]{#2}
		{\Large\sffamily \MakeUppercase{#2}}
		\vspace{1mm}\textcolor{mcvert}{\hrule}
	\fi 
	\ifx\url\arg
		\fancyfoot[L]{\href{https://www.maths-cours.fr#2}{\black \footnotesize{https://www.maths-cours.fr#2}}}
	\fi 
}


\newcommand\TitreC[1]{    		% Titre centré
     \needspace{3\baselineskip}
     \begin{center}\textbf{#1}\end{center}
}

\newcommand\newpar{    		% paragraphe
     \par
}

\newcommand\nosp {    		% commande vide (pas d'espace)
}
\newcommand{\id}[1]{} %ignore

\newcommand\boite[2]{				% Boite simple sans titre
	\vspace{5mm}
	\setlength{\fboxrule}{0.2mm}
	\setlength{\fboxsep}{5mm}	
	\fcolorbox{#1}{#1!3}{\makebox[\linewidth-2\fboxrule-2\fboxsep]{
  		\begin{minipage}[t]{\linewidth-2\fboxrule-4\fboxsep}\setlength{\parskip}{3mm}
  			 #2
  		\end{minipage}
	}}
	\vspace{5mm}
}

\newcommand\CBox[4]{				% Boites
	\vspace{5mm}
	\setlength{\fboxrule}{0.2mm}
	\setlength{\fboxsep}{5mm}
	
	\fcolorbox{#1}{#1!3}{\makebox[\linewidth-2\fboxrule-2\fboxsep]{
		\begin{minipage}[t]{1cm}\setlength{\parskip}{3mm}
	  		\textcolor{#1}{\LARGE{#2}}    
 	 	\end{minipage}  
  		\begin{minipage}[t]{\linewidth-2\fboxrule-4\fboxsep}\setlength{\parskip}{3mm}
			\raisebox{1.2mm}{\normalsize\sffamily{\textcolor{#1}{#3}}}						
  			 #4
  		\end{minipage}
	}}
	\vspace{5mm}
}

\newcommand\cadre[3]{				% Boites convertible html
	\par
	\vspace{2mm}
	\setlength{\fboxrule}{0.1mm}
	\setlength{\fboxsep}{5mm}
	\fcolorbox{#1}{white}{\makebox[\linewidth-2\fboxrule-2\fboxsep]{
  		\begin{minipage}[t]{\linewidth-2\fboxrule-4\fboxsep}\setlength{\parskip}{3mm}
			\raisebox{-2.5mm}{\sffamily \small{\textcolor{#1}{\MakeUppercase{#2}}}}		
			\par		
  			 #3
 	 		\end{minipage}
	}}
		\vspace{2mm}
	\par
}

\newcommand\bloc[3]{				% Boites convertible html sans bordure
     \needspace{2\baselineskip}
     {\sffamily \small{\textcolor{#1}{\MakeUppercase{#2}}}}    
		\par		
  			 #3
		\par
}

\newcommand\CHelp[1]{
     \CBox{Plum}{\faInfoCircle}{À RETENIR}{#1}
}

\newcommand\CUp[1]{
     \CBox{NavyBlue}{\faThumbsOUp}{EN PRATIQUE}{#1}
}

\newcommand\CInfo[1]{
     \CBox{Sepia}{\faArrowCircleRight}{REMARQUE}{#1}
}

\newcommand\CRedac[1]{
     \CBox{PineGreen}{\faEdit}{BIEN R\'EDIGER}{#1}
}

\newcommand\CError[1]{
     \CBox{Red}{\faExclamationTriangle}{ATTENTION}{#1}
}

\newcommand\TitreExo[2]{
\needspace{4\baselineskip}
 {\sffamily\large EXERCICE #1\ (\emph{#2 points})}
\vspace{5mm}
}

\newcommand\img[2]{
          \includegraphics[width=#2\paperwidth]{\imgdir#1}
}

\newcommand\imgsvg[2]{
       \begin{center}   \includegraphics[width=#2\paperwidth]{\imgsvgdir#1} \end{center}
}


\newcommand\Lien[2]{
     \href{#1}{#2 \tiny \faExternalLink}
}
\newcommand\mcLien[2]{
     \href{https~://www.maths-cours.fr/#1}{#2 \tiny \faExternalLink}
}

\newcommand{\euro}{\eurologo{}}

%================================================================================================================================
%
% Macros - Environement
%
%================================================================================================================================

\newenvironment{tex}{ %
}
{%
}

\newenvironment{indente}{ %
	\setlength\parindent{10mm}
}

{
	\setlength\parindent{0mm}
}

\newenvironment{corrige}{%
     \needspace{3\baselineskip}
     \medskip
     \textbf{\textsc{Corrigé}}
     \medskip
}
{
}

\newenvironment{extern}{%
     \begin{center}
     }
     {
     \end{center}
}

\NewEnviron{code}{%
	\par
     \boite{gray}{\texttt{%
     \BODY
     }}
     \par
}

\newenvironment{vbloc}{% boite sans cadre empeche saut de page
     \begin{minipage}[t]{\linewidth}
     }
     {
     \end{minipage}
}
\NewEnviron{h2}{%
    \needspace{3\baselineskip}
    \vspace{0.6cm}
	\noindent \MakeUppercase{\sffamily \large \BODY}
	\vspace{1mm}\textcolor{mcgris}{\hrule}\vspace{0.4cm}
	\par
}{}

\NewEnviron{h3}{%
    \needspace{3\baselineskip}
	\vspace{5mm}
	\textsc{\BODY}
	\par
}

\NewEnviron{margeneg}{ %
\begin{addmargin}[-1cm]{0cm}
\BODY
\end{addmargin}
}

\NewEnviron{html}{%
}

\begin{document}
\meta{url}{/exercices/qcm-bac-blanc-es-l-sujet-2-maths-cours-2018/}
\meta{pid}{10437}
\meta{titre}{QCM - Bac blanc ES/L Sujet 2 - Maths-cours 2018}
\meta{type}{exercices}
%
\begin{h2}Exercice 1 (5 points)\end{h2}
\par
\emph{Cet exercice est un questionnaire à choix multiples (QCM). Les questions sont indépendantes les unes des autres. Pour chacune des questions suivantes, une seule des trois réponses proposées est exacte.  \\Indiquer sur la copie le numéro de la question et la réponse exacte \textbf{en justifiant le choix effectué}. }
\par
\emph{\textbf{Toute réponse non justifiée ne sera pas prise en compte.}}
\par
\begin{itemize}
     \item \textbf{Question 1 :}
     \par
     Soient A et B deux événements d'une expérience aléatoire tels que $p(A)=0,7$, $p(B)=0,5$ et $p(A \cap B)=0,4$.
     \par
     Alors :
     \par
     \textbf{a.~~} $p(A \cup B)=0,9$ \\
     \textbf{b.~~} $p_A(B)=0,7$ \\
     \textbf{c.~~} $p_B(A)=0,8$ \\
     \par
     \item \textbf{Question 2 :}
     \par
     Soit la fonction $f$ définie sur $\mathbb{R}$ par $f(x)=x^3+2x^2-x+1$.
     \par
     Une équation de la tangente à la courbe représentative de $f$ au point $A(0~;~1)$ est :
     \par
     \textbf{a.~~} $y=-x+1$ \\
     \textbf{b.~~} $y=x-1$ \\
     \textbf{c.~~} $y=3x^2+4x-1$ \\
     \par
     \item \textbf{Question 3 :}
     \par
     On lance trois dés équilibrés à six faces. La probabilité $p$ d'obtenir au moins un \og 6 \fg{} (arrondie à $10^{-2}$) est :
     \par
     \textbf{a.~~} $p \approx 0,17$ \\
     \textbf{b.~~} $p \approx 0,42$  \\
     \textbf{c.~~} $p \approx 0,84$ \\
     \par
     \item \textbf{Question 4 :}
     \par
     $f$ est une fonction définie sur l'intervalle $[0~;~10]$ dont le tableau de variations est donné ci-après :
     \begin{center}
          \begin{extern}%width="350" alt="Tableau de Variations"
               \begin{tikzpicture}[scale=0.875]
                    % Styles
                    \tikzstyle{cadre}=[thin]
                    \tikzstyle{fleche}=[->,>=latex,thin]
                    \tikzstyle{nondefini}=[lightgray]
                    % Dimensions Modifiables
                    \def\Lrg{1.5}
                    \def\HtX{1}
                    \def\HtY{0.5}
                    % Dimensions Calculées
                    \def\lignex{-0.5*\HtX}
                    \def\lignef{-1.5*\HtX}
                    \def\separateur{-0.5*\Lrg}
                    % Largeur du tableau
                    \def\gauche{-1.5*\Lrg}
                    \def\droite{4.5*\Lrg}
                    % Hauteur du tableau
                    \def\haut{0.5*\HtX}
                    \def\bas{-1.5*\HtX-2*\HtY}
                    % Ligne de l'abscisse : x
                    \node at (-1*\Lrg,0) {$x$};
                    \node at (0*\Lrg,0) {$0$};
                    \node at (2*\Lrg,0) {$3$};
                    \node at (4*\Lrg,0) {$10$};
                    % Ligne de la fonction : f(x)
                    \node  at (-1*\Lrg,{-1*\HtX+(-1)*\HtY}) {$f(x)$};
                    \node (f1) at (0*\Lrg,{-1*\HtX+(0)*\HtY}) {$5$};
                    \node (f2) at (2*\Lrg,{-1*\HtX+(-2)*\HtY}) {$-2$};
                    \node (f3) at (4*\Lrg,{-1*\HtX+(0)*\HtY}) {$1$};
                    % Flèches
                    \draw[fleche] (f1) -- (f2);
                    \draw[fleche] (f2) -- (f3);
                    % Encadrement
                    \draw[cadre] (\separateur,\haut) -- (\separateur,\bas);
                    \draw[cadre] (\gauche,\haut) rectangle  (\droite,\bas);
                    \draw[cadre] (\gauche,\lignex) -- (\droite,\lignex);
               \end{tikzpicture}
          \end{extern}
     \end{center}
     L'équation $f(x)=3$ :
     \par
     \textbf{a.~~} n'admet aucune solution sur l'intervalle $[0~;~10]$ \\
     \textbf{b.~~} admet une unique solution sur l'intervalle $[0~;~10]$  \\
     \textbf{c.~~} admet deux solutions sur l'intervalle $[0~;~10]$ \\
     \par
     \item \textbf{Question 5 :}
     \par
     On considère la suite $(u_n)$ définie par $u_0=1$ et pour tout entier naturel $n$ :
     \[u_{n+1}=2u_n.\]
     \par
     La somme $S=u_0+u_1+u_2+\ \cdots\ +u_{10}$ vaut :
     \par
     \textbf{a.~~} $S=1\ 023$ \\
     \textbf{b.~~} $S=2\ 047$  \\
     \textbf{c.~~} $S=4\ 095$ \\
     \par
\end{itemize}
\begin{corrige}
     \begin{itemize}
          % =============================================================================================================================
          \item \textbf{Question 1 :}
          \par
          Réponse correcte :\quad\textbf{ c.}
          \par
          $p_B(A)=\dfrac{p(A \cap B)}{p(B)}=\dfrac{0,4}{0,5}=0,8$.
          \cadre{bleu}{Remarque}{
               Les réponses \textbf{a.} et \textbf{b.} sont incorrectes. En effet :
               $p(A \cup B) = p(A) + p(B) - p(A \cap B)$\\
               $\phantom{p(A \cup B)}=  0,7 + 0,5 - 0,4 = 0,8.$
               \par
               $ p_A(B)=\dfrac{p(A \cap B)}{p(A)}=\dfrac{0,4}{0,7}=\dfrac{4}{7}$.
          }
          \cadre{rouge}{À retenir}{
               Quels que soient les événements $A$ et $B$ :
               \begin{itemize}
                    \item
                    $p(A \cup B) = p(A) + p(B) - p(A \cap B)$
                    \item
                    $p_A(B)=\dfrac{p(A \cap B)}{p(A)}$.
               \end{itemize}
          }
          \par
          % =============================================================================================================================
          \item \textbf{Question 2 :}
          \par
          Réponse correcte :\quad\textbf{ a.}
          \par
          $f$ est une fonction polynôme donc $f$ est dérivable sur $\mathbb{R}$ et :
          \par
          $f'(x)=3x^2-4x-1$.
          \par
          L'équation réduite de la tangente à la courbe représentative de $f$ au point $A$ d'abscisse $0$ est :
          \par
          $y=f'(0)(x-0)+f(0)$.
          \par
          Or:
          \par
          $f(0)=0^3+2 \times 0^2 - 0 + 1 =1$
          \par
          $f'(0)=3 \times 0^2 - 4 \times 0 - 1 = -1$.
          \par
          L'équation cherchée est donc :
          \par
          $y=- 1(x-0)+1$
          \par
          $y=- x+1$.
          \par
          \cadre{rouge}{À retenir}{
               L'équation réduite de la tangente à la courbe représentative de $f$ au point d'\textbf{abscisse} $\bm{a}$ est :
               \[ y=f'(a)(x-a)+f(a). \]
          }
          \par
          % =============================================================================================================================
          \item \textbf{Question 3 :}
          \par
          Réponse correcte :\quad\textbf{ b.}
          \par
          Soit $X$ la variable aléatoire comptabilisant le nombre de \og 6 \fg{} obtenus.
          \par
          $X$ suit une loi binomiale de paramètres $n=3$ (nombre de dés) et $p=\dfrac{1}{6}$ (probabilité d'obtenir un \og 6 \fg{})
          \par
          La probabilité demandée est la probabilité de l'événement ${(X \geqslant 1)}$. L'événement contraire de ${(X \geqslant 1)}$ est ${(X < 1)}$ qui équivaut à ${(X = 0)}$.
          \par
          Par conséquent :
          \par
          $p=p(X \geqslant 1)=1 - p(X=0)$.
          \par
          Or:
          \par
          $p(X=0) = \begin{pmatrix} 3 \\ 0 \end{pmatrix} \times \left(\dfrac{1}{6}\right)^0 \times \left(\dfrac{5}{6}\right)^3$\nosp$ = \left(\dfrac{5}{6}\right)^3  \approx  0,58 $ (à $10^{-2}$ près).
          \cadre{bleu}{Remarque}{
               On peut également utiliser la calculatrice pour calculer $p(X=0)$ (par exemple BinomFdP(3, 1/6, 0) sur TI ou BinomialPD(0, 3, 1/6) sur Casio).
          }
          Par conséquent :
          \par
          $p \approx 0,42$ (à $10^{-2}$ près)
          \par
          \cadre{rouge}{À retenir}{
               \par
               L'événement \textbf{contraire} de l'événement \og obtenir \textbf{au moins un} six \fg{} est \og n'obtenir \textbf{aucun} six \fg{}.
               \par
          }
          \par
          %=============================================================================================================================
          \item \textbf{Question 4 :}
          \par
          Réponse correcte :\quad\textbf{ b.}
          \par
          Sur l'intervalle $[0~;~3]$, $f$ est \textbf{continue} et \textbf{strictement décroissante}. 3 appartient à l'intervalle $[-2~;~5]$ donc l'équation ${f(x)=3}$ admet une unique solution sur l'intervalle $[0~;~3]$ (\textit{théorème de la bijection} aussi appelé \textit{corollaire du théorème des valeurs intermédiaires}).
          \cadre{rouge}{Bien rédiger}{
               Pour prouver l'\textbf{existence} et l'\textbf{unicité} d'une solution il est important de préciser que :
               \begin{itemize}
                    \item la fonction $f$ est \textbf{continue},
                    \item la fonction $f$ est \textbf{strictement monotone}.
               \end{itemize}
          }
          Sur l'intervalle $[3~;~10]$, le maximum de $f$ est 1 donc l'équation ${f(x)=3}$ n'a pas de solution sur cet intervalle.
          \cadre{rouge}{Bien rédiger}{
               Pour montrer que l'équation $f(x)=k$ \textbf{n'admet pas de solution} sur un intervalle $I$, il suffit d'indiquer que le maximum de $f$ sur $I$ est strictement inférieur à $k$ ou que le minimum de $f$ sur $I$ est strictement supérieur à $k$.
               \par
               On n'utilise pas, dans ce cas, le théorème des valeurs intermédiaires (que l'on emploie, au contraire, lorsque l'on souhaite prouver qu'il y a une ou plusieurs solution(s) sur un intervalle).
          }
          Par conséquent, l'équation $f(x)=3$ admet une unique solution sur l'intervalle $[0~;~10]$.
          \par
          %=============================================================================================================================
          \item \textbf{Question 5 :}
          \par
          Réponse correcte :\quad\textbf{ b.}
          \par
          La relation $u_{n+1}=2u_n$, pour tout entier naturel $n$, montre que la suite $(u_n)$ est une suite géométrique de raison $q=2$.
          \par
          On a donc, pour tout entier naturel $n$ :
          \[ u_n=u_0q^n=2^n \]
          La somme $S$ vaut alors :
          \par
          $S=1+2+2^2+\cdots+2^{10}=\dfrac{1-2^{11}}{1-2}$\nosp$=2^{11}-1=2\ 047$.
          \cadre{rouge}{À retenir}{
               La formule suivante permet de calculer la somme des premiers termes d'une suite géométrique :
               \[ 1+q+q^2+\cdots+q^{n}=\dfrac{1-q^{n+1}}{1-q}. \]
          }
     \end{itemize}
\end{corrige}

\end{document}