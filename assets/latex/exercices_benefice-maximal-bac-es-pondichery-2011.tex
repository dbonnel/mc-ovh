\documentclass[a4paper]{article}

%================================================================================================================================
%
% Packages
%
%================================================================================================================================

\usepackage[T1]{fontenc} 	% pour caractères accentués
\usepackage[utf8]{inputenc}  % encodage utf8
\usepackage[french]{babel}	% langue : français
\usepackage{fourier}			% caractères plus lisibles
\usepackage[dvipsnames]{xcolor} % couleurs
\usepackage{fancyhdr}		% réglage header footer
\usepackage{needspace}		% empêcher sauts de page mal placés
\usepackage{graphicx}		% pour inclure des graphiques
\usepackage{enumitem,cprotect}		% personnalise les listes d'items (nécessaire pour ol, al ...)
\usepackage{hyperref}		% Liens hypertexte
\usepackage{pstricks,pst-all,pst-node,pstricks-add,pst-math,pst-plot,pst-tree,pst-eucl} % pstricks
\usepackage[a4paper,includeheadfoot,top=2cm,left=3cm, bottom=2cm,right=3cm]{geometry} % marges etc.
\usepackage{comment}			% commentaires multilignes
\usepackage{amsmath,environ} % maths (matrices, etc.)
\usepackage{amssymb,makeidx}
\usepackage{bm}				% bold maths
\usepackage{tabularx}		% tableaux
\usepackage{colortbl}		% tableaux en couleur
\usepackage{fontawesome}		% Fontawesome
\usepackage{environ}			% environment with command
\usepackage{fp}				% calculs pour ps-tricks
\usepackage{multido}			% pour ps tricks
\usepackage[np]{numprint}	% formattage nombre
\usepackage{tikz,tkz-tab} 			% package principal TikZ
\usepackage{pgfplots}   % axes
\usepackage{mathrsfs}    % cursives
\usepackage{calc}			% calcul taille boites
\usepackage[scaled=0.875]{helvet} % font sans serif
\usepackage{svg} % svg
\usepackage{scrextend} % local margin
\usepackage{scratch} %scratch
\usepackage{multicol} % colonnes
%\usepackage{infix-RPN,pst-func} % formule en notation polanaise inversée
\usepackage{listings}

%================================================================================================================================
%
% Réglages de base
%
%================================================================================================================================

\lstset{
language=Python,   % R code
literate=
{á}{{\'a}}1
{à}{{\`a}}1
{ã}{{\~a}}1
{é}{{\'e}}1
{è}{{\`e}}1
{ê}{{\^e}}1
{í}{{\'i}}1
{ó}{{\'o}}1
{õ}{{\~o}}1
{ú}{{\'u}}1
{ü}{{\"u}}1
{ç}{{\c{c}}}1
{~}{{ }}1
}


\definecolor{codegreen}{rgb}{0,0.6,0}
\definecolor{codegray}{rgb}{0.5,0.5,0.5}
\definecolor{codepurple}{rgb}{0.58,0,0.82}
\definecolor{backcolour}{rgb}{0.95,0.95,0.92}

\lstdefinestyle{mystyle}{
    backgroundcolor=\color{backcolour},   
    commentstyle=\color{codegreen},
    keywordstyle=\color{magenta},
    numberstyle=\tiny\color{codegray},
    stringstyle=\color{codepurple},
    basicstyle=\ttfamily\footnotesize,
    breakatwhitespace=false,         
    breaklines=true,                 
    captionpos=b,                    
    keepspaces=true,                 
    numbers=left,                    
xleftmargin=2em,
framexleftmargin=2em,            
    showspaces=false,                
    showstringspaces=false,
    showtabs=false,                  
    tabsize=2,
    upquote=true
}

\lstset{style=mystyle}


\lstset{style=mystyle}
\newcommand{\imgdir}{C:/laragon/www/newmc/assets/imgsvg/}
\newcommand{\imgsvgdir}{C:/laragon/www/newmc/assets/imgsvg/}

\definecolor{mcgris}{RGB}{220, 220, 220}% ancien~; pour compatibilité
\definecolor{mcbleu}{RGB}{52, 152, 219}
\definecolor{mcvert}{RGB}{125, 194, 70}
\definecolor{mcmauve}{RGB}{154, 0, 215}
\definecolor{mcorange}{RGB}{255, 96, 0}
\definecolor{mcturquoise}{RGB}{0, 153, 153}
\definecolor{mcrouge}{RGB}{255, 0, 0}
\definecolor{mclightvert}{RGB}{205, 234, 190}

\definecolor{gris}{RGB}{220, 220, 220}
\definecolor{bleu}{RGB}{52, 152, 219}
\definecolor{vert}{RGB}{125, 194, 70}
\definecolor{mauve}{RGB}{154, 0, 215}
\definecolor{orange}{RGB}{255, 96, 0}
\definecolor{turquoise}{RGB}{0, 153, 153}
\definecolor{rouge}{RGB}{255, 0, 0}
\definecolor{lightvert}{RGB}{205, 234, 190}
\setitemize[0]{label=\color{lightvert}  $\bullet$}

\pagestyle{fancy}
\renewcommand{\headrulewidth}{0.2pt}
\fancyhead[L]{maths-cours.fr}
\fancyhead[R]{\thepage}
\renewcommand{\footrulewidth}{0.2pt}
\fancyfoot[C]{}

\newcolumntype{C}{>{\centering\arraybackslash}X}
\newcolumntype{s}{>{\hsize=.35\hsize\arraybackslash}X}

\setlength{\parindent}{0pt}		 
\setlength{\parskip}{3mm}
\setlength{\headheight}{1cm}

\def\ebook{ebook}
\def\book{book}
\def\web{web}
\def\type{web}

\newcommand{\vect}[1]{\overrightarrow{\,\mathstrut#1\,}}

\def\Oij{$\left(\text{O}~;~\vect{\imath},~\vect{\jmath}\right)$}
\def\Oijk{$\left(\text{O}~;~\vect{\imath},~\vect{\jmath},~\vect{k}\right)$}
\def\Ouv{$\left(\text{O}~;~\vect{u},~\vect{v}\right)$}

\hypersetup{breaklinks=true, colorlinks = true, linkcolor = OliveGreen, urlcolor = OliveGreen, citecolor = OliveGreen, pdfauthor={Didier BONNEL - https://www.maths-cours.fr} } % supprime les bordures autour des liens

\renewcommand{\arg}[0]{\text{arg}}

\everymath{\displaystyle}

%================================================================================================================================
%
% Macros - Commandes
%
%================================================================================================================================

\newcommand\meta[2]{    			% Utilisé pour créer le post HTML.
	\def\titre{titre}
	\def\url{url}
	\def\arg{#1}
	\ifx\titre\arg
		\newcommand\maintitle{#2}
		\fancyhead[L]{#2}
		{\Large\sffamily \MakeUppercase{#2}}
		\vspace{1mm}\textcolor{mcvert}{\hrule}
	\fi 
	\ifx\url\arg
		\fancyfoot[L]{\href{https://www.maths-cours.fr#2}{\black \footnotesize{https://www.maths-cours.fr#2}}}
	\fi 
}


\newcommand\TitreC[1]{    		% Titre centré
     \needspace{3\baselineskip}
     \begin{center}\textbf{#1}\end{center}
}

\newcommand\newpar{    		% paragraphe
     \par
}

\newcommand\nosp {    		% commande vide (pas d'espace)
}
\newcommand{\id}[1]{} %ignore

\newcommand\boite[2]{				% Boite simple sans titre
	\vspace{5mm}
	\setlength{\fboxrule}{0.2mm}
	\setlength{\fboxsep}{5mm}	
	\fcolorbox{#1}{#1!3}{\makebox[\linewidth-2\fboxrule-2\fboxsep]{
  		\begin{minipage}[t]{\linewidth-2\fboxrule-4\fboxsep}\setlength{\parskip}{3mm}
  			 #2
  		\end{minipage}
	}}
	\vspace{5mm}
}

\newcommand\CBox[4]{				% Boites
	\vspace{5mm}
	\setlength{\fboxrule}{0.2mm}
	\setlength{\fboxsep}{5mm}
	
	\fcolorbox{#1}{#1!3}{\makebox[\linewidth-2\fboxrule-2\fboxsep]{
		\begin{minipage}[t]{1cm}\setlength{\parskip}{3mm}
	  		\textcolor{#1}{\LARGE{#2}}    
 	 	\end{minipage}  
  		\begin{minipage}[t]{\linewidth-2\fboxrule-4\fboxsep}\setlength{\parskip}{3mm}
			\raisebox{1.2mm}{\normalsize\sffamily{\textcolor{#1}{#3}}}						
  			 #4
  		\end{minipage}
	}}
	\vspace{5mm}
}

\newcommand\cadre[3]{				% Boites convertible html
	\par
	\vspace{2mm}
	\setlength{\fboxrule}{0.1mm}
	\setlength{\fboxsep}{5mm}
	\fcolorbox{#1}{white}{\makebox[\linewidth-2\fboxrule-2\fboxsep]{
  		\begin{minipage}[t]{\linewidth-2\fboxrule-4\fboxsep}\setlength{\parskip}{3mm}
			\raisebox{-2.5mm}{\sffamily \small{\textcolor{#1}{\MakeUppercase{#2}}}}		
			\par		
  			 #3
 	 		\end{minipage}
	}}
		\vspace{2mm}
	\par
}

\newcommand\bloc[3]{				% Boites convertible html sans bordure
     \needspace{2\baselineskip}
     {\sffamily \small{\textcolor{#1}{\MakeUppercase{#2}}}}    
		\par		
  			 #3
		\par
}

\newcommand\CHelp[1]{
     \CBox{Plum}{\faInfoCircle}{À RETENIR}{#1}
}

\newcommand\CUp[1]{
     \CBox{NavyBlue}{\faThumbsOUp}{EN PRATIQUE}{#1}
}

\newcommand\CInfo[1]{
     \CBox{Sepia}{\faArrowCircleRight}{REMARQUE}{#1}
}

\newcommand\CRedac[1]{
     \CBox{PineGreen}{\faEdit}{BIEN R\'EDIGER}{#1}
}

\newcommand\CError[1]{
     \CBox{Red}{\faExclamationTriangle}{ATTENTION}{#1}
}

\newcommand\TitreExo[2]{
\needspace{4\baselineskip}
 {\sffamily\large EXERCICE #1\ (\emph{#2 points})}
\vspace{5mm}
}

\newcommand\img[2]{
          \includegraphics[width=#2\paperwidth]{\imgdir#1}
}

\newcommand\imgsvg[2]{
       \begin{center}   \includegraphics[width=#2\paperwidth]{\imgsvgdir#1} \end{center}
}


\newcommand\Lien[2]{
     \href{#1}{#2 \tiny \faExternalLink}
}
\newcommand\mcLien[2]{
     \href{https~://www.maths-cours.fr/#1}{#2 \tiny \faExternalLink}
}

\newcommand{\euro}{\eurologo{}}

%================================================================================================================================
%
% Macros - Environement
%
%================================================================================================================================

\newenvironment{tex}{ %
}
{%
}

\newenvironment{indente}{ %
	\setlength\parindent{10mm}
}

{
	\setlength\parindent{0mm}
}

\newenvironment{corrige}{%
     \needspace{3\baselineskip}
     \medskip
     \textbf{\textsc{Corrigé}}
     \medskip
}
{
}

\newenvironment{extern}{%
     \begin{center}
     }
     {
     \end{center}
}

\NewEnviron{code}{%
	\par
     \boite{gray}{\texttt{%
     \BODY
     }}
     \par
}

\newenvironment{vbloc}{% boite sans cadre empeche saut de page
     \begin{minipage}[t]{\linewidth}
     }
     {
     \end{minipage}
}
\NewEnviron{h2}{%
    \needspace{3\baselineskip}
    \vspace{0.6cm}
	\noindent \MakeUppercase{\sffamily \large \BODY}
	\vspace{1mm}\textcolor{mcgris}{\hrule}\vspace{0.4cm}
	\par
}{}

\NewEnviron{h3}{%
    \needspace{3\baselineskip}
	\vspace{5mm}
	\textsc{\BODY}
	\par
}

\NewEnviron{margeneg}{ %
\begin{addmargin}[-1cm]{0cm}
\BODY
\end{addmargin}
}

\NewEnviron{html}{%
}

\begin{document}
\meta{url}{/exercices/benefice-maximal-bac-es-pondichery-2011/}
\meta{pid}{2000}
\meta{titre}{Recherche du bénéfice maximal - Bac ES Pondichéry 2011}
\meta{type}{exercices}
%
\begin{h2}Exercice 4\end{h2}
\par
\textbf{Commun  à tous les candidats}
\par
Un laboratoire pharmaceutique fabrique un médicament qu'il commercialise sous forme liquide. Sa capacité journalière de production est comprise entre 25 et 500 litres, et on suppose que toute la production est commercialisée.
\par
Dans tout l'exercice, les coûts et recettes sont exprimés en milliers d'euros, les quantités en centaines de litres.
\par
Si $x$ désigne la quantité journalière produite, on appelle $C_{T}\left(x\right)$, pour $x$ variant de 0,25 à 5, le coût total de production correspondant.
\par
La courbe $\Gamma _{1}$ ci-dessous est la représentation graphique de la fonction $C_{T}$ sur l'intervalle [0,25 ; 5].
\par
La tangente à $\Gamma _{1}$ au point $A\left(1 ; 1\right)$ est horizontale.

\begin{center}
\imgsvg{Bac_ES_Pondichery_2011-4-1}{0.3}% alt="Recherche du bénéfice maximal - Bac ES Pondichéry 2011" style="width:50rem"
\end{center}

\begin{h3}Partie A\end{h3}
\begin{enumerate}
     \item
     \begin{enumerate}[label=\alph*.] 
          \item
          On admet que la recette $R\left(x\right)$ (en milliers d'euros) résultant de la vente de $x$ centaines de litres de médicament, est définie sur [0,25 ; 5] par $R\left(x\right)=1,5 x$.
          \par
          Quelle est la recette (en euros) pour 200 litres de médicament vendus ?
          \item
          Tracer, sur le graphique fourni ci-dessus, le segment représentant graphiquement la fonction $R$.
     \end{enumerate}
     \item
     Lectures graphiques
     \par
     Les questions a., b., c. suivantes seront résolues à l'aide de lectures graphiques seulement. On fera apparaître les traits de construction sur le graphique.
     \par
     Toute trace de recherche même non aboutie sera prise en compte.
     \begin{enumerate}[label=\alph*.] 
          \item
          Déterminer des valeurs approximatives des bornes de la « plage de rentabilité », c'est-à-dire de l'intervalle correspondant aux quantités commercialisées dégageant un bénéfice positif.
          \item
          Donner une valeur approximative du bénéfice en euros réalisé par le laboratoire lorsque 200 litres de médicament sont commercialisés.
          \item
          Pour quelle quantité de médicament commercialisée le bénéfice paraît-il maximal ?
          \par
          A combien peut-on évaluer le bénéfice maximal obtenu?
     \end{enumerate}
\end{enumerate}
\begin{h3}Partie B\end{h3}
Dans la suite de l'exercice, on admet que la fonction coût total $C_{T}$ est définie sur l'intervalle [0,25 ; 5] par
\par
$C_{T}\left(x\right)=x^{2}-2x \ln \left(x\right)$.
\begin{enumerate}
     \item
     Justifier que le bénéfice, en milliers d'euros, réalisé par le laboratoire pour x centaines de litres commercialisés, est donné par :
     \par
     $B\left(x\right)=1,5x-x^{2}+2x \ln \left(x\right)$.
     \par
     Calculer $B\left(2\right)$, et comparer au résultat obtenu à la question 2.b. de la Partie A.
     \item
     On suppose que la fonction B est dérivable sur l'intervalle [0,25 ; 5] et on note B' sa fonction dérivée. Montrer que $B^{\prime}\left(x\right)=2\ln \left(x\right)-2x+3,5$.
     \item
     On donne ci-dessous le tableau de variation de la fonction $B^{\prime}$, dérivée de la fonction $B$, sur l'intervalle [0,25 ; 5] :
%##
% type=table; width=25; l2=20
%--
% x|   0,25   ~   1  ~   5 
% B'(x)|  y_1              /   :1,5   \   y_2
%--
\begin{center}
 \begin{extern}%style="width:25rem" alt="Exercice"
    \resizebox{11cm}{!}{
       \definecolor{dark}{gray}{0.1}
       \definecolor{light}{gray}{0.8}
       \tikzstyle{fleche}=[->,>=latex]
       \begin{tikzpicture}[scale=.8, line width=.5pt, dark]
       \def\width{.15}
       \def\height{.10}
       \draw (0, -10*\height) -- (54*\width, -10*\height);
       \draw (10*\width, 0*\height) -- (10*\width, -10*\height);
       \node (l0c0) at (5*\width,-5*\height) {$ x $};
       \node (l0c1) at (14*\width,-5*\height) {$ 0,25 $};
       \node (l0c2) at (23*\width,-5*\height) {$ ~ $};
       \node (l0c3) at (32*\width,-5*\height) {$ 1 $};
       \node (l0c4) at (41*\width,-5*\height) {$ ~ $};
       \node (l0c5) at (50*\width,-5*\height) {$ 5 $};
       \draw (0, -30*\height) -- (54*\width, -30*\height);
       \draw (10*\width, -10*\height) -- (10*\width, -30*\height);
       \node (l1c0) at (5*\width,-20*\height) {$ B'(x) $};
       \node (l1c1) at (14*\width,-25*\height) {$ y_1 $};
       \node (l1c2) at (23*\width,-20*\height) {$ ~ $};
       \draw[light] (32*\width, -10*\height) -- (32*\width, -30*\height);
       \node (l1c3) at (32*\width,-15*\height) {$ 1,5 $};
       \node (l1c4) at (41*\width,-20*\height) {$ ~ $};
       \node (l1c5) at (50*\width,-25*\height) {$ y_2 $};
       \draw (0, 0) rectangle (54*\width, -30*\height);
       \draw[fleche] (l1c1) -- (l1c3);
       \draw[fleche] (l1c3) -- (l1c5);
       \end{tikzpicture}
      }
   \end{extern}
\end{center}
%##
On précise les encadrements : $0,22 < y_{1} < 0,23$ et $-3,29 < y_{2}  < -3,28$.
     \begin{enumerate}[label=\alph*.] 
          \item
          Démontrer que l'équation $B^{\prime}\left(x\right)=0$ admet une solution unique $\alpha $ dans l'intervalle [0,25 ; 5].
          \par
          Pour la suite de l'exercice, on prendra 2,77 pour valeur approchée de $\alpha $.
          \item
          Dresser le tableau précisant le signe de $B^{\prime}\left(x\right)$ pour $x$ appartenant à l'intervalle [0,25 ; 5].
          \par
          En déduire le tableau de variations de la fonction $B$ sur l'intervalle [0,25 ; 5].
     \end{enumerate}
     \item
     \begin{enumerate}[label=\alph*.] 
          \item
          Pour quelle quantité de médicament commercialisée, le bénéfice est-il maximal ? (On donnera une valeur approchée de cette quantité en litres).
          \par
          Donner alors une valeur approchée en euros de ce bénéfice maximal.
          \item
          Ces résultats sont-ils cohérents avec ceux obtenus graphiquement à la question 2.c. de la partie A ?
     \end{enumerate}
\end{enumerate}

\end{document}